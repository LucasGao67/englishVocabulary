\documentclass[twocolumn]{book}

\title{English book}
\author{lucas gao}
\date{\today}
\usepackage{lipsum}
\usepackage{color}
\usepackage{sectsty}
\sectionfont{\color{cyan} \fontsize{18}{20}\selectfont}

\begin{document}
\section*{aesthetic}
{\large \color{blue}  }
\subsection*{Explain}
\begin{enumerate}
\item adjective \\
\textbf{Aesthetic} is used to talk about beauty or art , and people's appreciation of beautiful things.
 \textbf{The aesthetic} of a work of art is its aesthetic quality.
 \textit{
	\begin{itemize}
	\item ...products chosen for their aesthetic appeal as well as their durability and quality.
	\item He responded very strongly to the aesthetic of this particular work.
	\end{itemize}
}
\end{enumerate}

\section*{access}
{\large \color{blue}  accesses  accessing  accessed  }
\subsection*{Explain}
\begin{enumerate}
\item uncountable noun \\
If you have \textbf{access}  \textbf{to} a building or other place, you are able or allowed to go into it.
 \textit{
	\begin{itemize}
	\item The facilities have been adapted to give access to wheelchair users.
	\item Scientists have only recently been able to gain access to the area.
	\item The Mortimer Hotel offers easy access to central London.
	\end{itemize}
}
\item uncountable noun \\
If you have \textbf{access}  \textbf{to} something such as information or equipment , you have the opportunity or right to see it or use it.
 \textit{
	\begin{itemize}
	\item ...a Code of Practice that would give patients right of access to their medical records.
	\item Consultant-led teams will have access to the latest equipment.
	\end{itemize}
}
\item uncountable noun \\
If you have \textbf{access}  \textbf{to} a person, you have the opportunity or right to see them or meet them.
 \textit{
	\begin{itemize}
	\item He was not allowed access to a lawyer.
	\item He had direct access to the Prime Minister.
	\end{itemize}
}
\item verb \\
If you \textbf{access} something, especially information held on a computer, you succeed in finding or obtaining it.
 \textit{
	\begin{itemize}
	\item You've illegally accessed and misused confidential security files.
	\end{itemize}
}
\end{enumerate}

\section*{amateur}
{\large \color{blue}  amateurs  }
\subsection*{Explain}
\begin{enumerate}
\item countable noun \\
An \textbf{amateur} is someone who does something as a hobby and not as a job .
 \textit{
	\begin{itemize}
	\item Jerry is an amateur who dances because he feels like it.
	\item Taylor began his playing career as an amateur goalkeeper.
	\end{itemize}
}
\item adjective \\
\textbf{Amateur} sports or activities are done by people as a hobby and not as a job.
 \textit{
	\begin{itemize}
	\item ...the local amateur dramatics society.
	\item At college he studied English and did amateur boxing.
	\end{itemize}
}
\end{enumerate}

\section*{adjective}
{\large \color{blue}  adjectives  }
\subsection*{Explain}
\begin{enumerate}
\item countable noun \\
An \textbf{adjective} is a word such as ' big ', ' dead ', or ' financial ' that describes a person or thing, or gives extra  information about them. Adjectives usually come before nouns or after link  verbs .
 \textit{
	\begin{itemize}
	\end{itemize}
}
\end{enumerate}

\section*{analytic}
{\large \color{blue}  }
\subsection*{Explain}
\begin{enumerate}
\item adjective \\
\textbf{Analytic} means the same as analytical .
 \textit{
	\begin{itemize}
	\end{itemize}
}
\end{enumerate}

\section*{april}
{\large \color{blue}  Aprils  }
\subsection*{Explain}
\begin{enumerate}
\item variable noun \\
\textbf{April} is the fourth month of the year in the Western  calendar .
 \textit{
	\begin{itemize}
	\item The changes will be introduced in April.
	\item They were married on 7 April 1927 at Paddington Register Office.
	\item He announced that he will retire next April.
	\end{itemize}
}
\end{enumerate}

\section*{approval}
{\large \color{blue}  approvals  }
\subsection*{Explain}
\begin{enumerate}
\item uncountable noun \\
If you win someone's \textbf{approval}  \textbf{for} something that you ask for or suggest , they agree to it.
 \textit{
	\begin{itemize}
	\item ...efforts to win congressional approval for an aid package for Moscow.
	\item The chairman has also given his approval for an investigation into the case.
	\item The proposed modifications met with widespread approval.
	\end{itemize}
}
\item variable noun \\
\textbf{Approval} is a formal or official statement that something is acceptable .
 \textit{
	\begin{itemize}
	\item The testing and approval of new drugs will be speeded up.
	\end{itemize}
}
\item uncountable noun \\
If someone or something has your \textbf{approval} , you like and admire them.
 \textit{
	\begin{itemize}
	\item His son had an obsessive drive to gain his father's approval.
	\item The president's approval rating had risen.
	\end{itemize}
}
\item  \\
 seal of approval \textit{
	\begin{itemize}
	\end{itemize}
}
\end{enumerate}

\section*{cautious}
{\large \color{blue}  }
\subsection*{Explain}
\begin{enumerate}
\item adjective \\
Someone who is \textbf{cautious} acts very carefully in order to avoid  possible  danger .
 \textit{
	\begin{itemize}
	\item The scientists are cautious about using enzyme therapy on humans.
	\item He is a very cautious man.
	\end{itemize}
}
\item adjective \\
If you describe someone's attitude or reaction as \textbf{cautious} , you mean that it is limited or careful .
 \textit{
	\begin{itemize}
	\item He has been seen as a champion of a more cautious approach to economic reform.
	\end{itemize}
}
\end{enumerate}

\section*{athlete}
{\large \color{blue}  athletes  }
\subsection*{Explain}
\begin{enumerate}
\item countable noun \\
An \textbf{athlete} is a person who does a sport, especially  athletics , or track and field events.
 \textit{
	\begin{itemize}
	\item Many top athletes find it hard, if not impossible to find real life again after retiring.
	\end{itemize}
}
\item countable noun \\
You can refer to someone who is fit and athletic as an \textbf{athlete} .
 \textit{
	\begin{itemize}
	\item I was no athlete.
	\end{itemize}
}
\end{enumerate}

\section*{breach}
{\large \color{blue}  breaches  breaching  breached  }
\subsection*{Explain}
\begin{enumerate}
\item verb \\
If you \textbf{breach} an agreement , a law, or a promise, you break it.
 \textit{
	\begin{itemize}
	\item The newspaper breached the code of conduct on privacy.
	\item The film breached the criminal libel laws.
	\end{itemize}
}
\item variable noun \\
A \textbf{breach}  \textbf{of} an agreement, a law, or a promise is an act of breaking it.
 \textit{
	\begin{itemize}
	\item The congressman was accused of a breach of secrecy rules.
	\item ...a $1 billion breach of contract suit.
	\end{itemize}
}
\item countable noun \\
A \textbf{breach}  \textbf{in} a relationship is a serious  disagreement which often results in the relationship ending .
 \textit{
	\begin{itemize}
	\item Their actions threatened a serious breach in relations between the two countries.
	\item Little happens to heal the breach between the two warring factions.
	\end{itemize}
}
\item verb \\
If someone or something \textbf{breaches} a barrier , they make an opening in it, usually leaving it weakened or destroyed .
 \textit{
	\begin{itemize}
	\item The limestone is sufficiently fissured for tree roots to have breached the roof of
the cave.
	\item Fire may have breached the cargo tanks and set the oil ablaze.
	\end{itemize}
}
\item verb \\
If you \textbf{breach} someone's security or their defences, you manage to get through and attack an area that is heavily guarded and protected .
 \textbf{Breach} is also a noun .
 \textit{
	\begin{itemize}
	\item The bomber had breached security by hurling his dynamite from a roof overlooking
the building.
	\item ...widespread breaches of security at Ministry of Defence bases.
	\end{itemize}
}
\item  \\
 step into the breach \textit{
	\begin{itemize}
	\end{itemize}
}
\end{enumerate}

\section*{classmate}
{\large \color{blue}  classmates  }
\subsection*{Explain}
\begin{enumerate}
\item countable noun \\
Your \textbf{classmates} are students who are in the same class as you at school or college.
 \textit{
	\begin{itemize}
	\end{itemize}
}
\end{enumerate}

\section*{canal}
{\large \color{blue}  canals  }
\subsection*{Explain}
\begin{enumerate}
\item countable noun \\
A \textbf{canal} is a long, narrow  stretch of water that has been made for boats to travel along or to bring water to a particular area.
 \textit{
	\begin{itemize}
	\item ...the Grand Union Canal.
	\item ...Venetian canals and bridges.
	\end{itemize}
}
\item countable noun \\
A \textbf{canal} is a narrow tube  inside your body for carrying food, air, or other substances.
 \textit{
	\begin{itemize}
	\item ...delaying the food's progress through the alimentary canal.
	\end{itemize}
}
\end{enumerate}

\section*{copy}
{\large \color{blue}  copies  copying  copied  }
\subsection*{Explain}
\begin{enumerate}
\item countable noun \\
If you make a \textbf{copy}  \textbf{of} something, you produce something that looks  like the original thing.
 \textit{
	\begin{itemize}
	\item The reporter apparently obtained a copy of Steve's resignation letter.
	\item Always keep a copy of everything in your own files.
	\end{itemize}
}
\item verb \\
If you \textbf{copy} something, you produce something that looks like the original thing.
 \textit{
	\begin{itemize}
	\item She never participated in obtaining or copying any classified documents for anyone.
	\item ...lawsuits against companies who have unlawfully copied computer programs.
	\item ...top designers, whose work has been widely copied.
	\item He copied the chart from a book.
	\end{itemize}
}
\item verb \\
If you \textbf{copy} a piece of writing , you write it again exactly .
 \textbf{Copy out}  means the same as copy .
 \textit{
	\begin{itemize}
	\item He would allow John to copy his answers to difficult algebra questions.
	\item He copied the data into a notebook.
	\item We're copying from textbooks because we don't have enough to go round.
	\item He wrote the title on the blackboard, then copied out the text sentence by sentence.
	\item 'Did he leave a phone number?'—'Oh, yes.' She copied it out for him.
	\end{itemize}
}
\item verb \\
If you \textbf{copy} a person or what they do, you try to do what they do or try to be like them, usually because you admire them or what they have done .
 \textit{
	\begin{itemize}
	\item Children can be seen to copy the behaviour of others whom they admire or identify
with.
	\item He can claim to have been defeated by opponents copying his own tactics.
	\item ...the techniques she had copied from her tennis heroes.
	\end{itemize}
}
\item countable noun \\
A \textbf{copy}  \textbf{of} a book, newspaper, or CD is one of many that are exactly the same.
 \textit{
	\begin{itemize}
	\item I bought a copy of 'U.S.A. Today' from a street-corner machine.
	\item You can obtain a copy for $2 from New York Central Art Supply.
	\end{itemize}
}
\item uncountable noun \\
In journalism , \textbf{copy} is written material that is ready to be printed or read in a broadcast .
 \textit{
	\begin{itemize}
	\item ...his ability to write the most lyrical copy in the history of sports television.
	\item ...advertising copy.
	\end{itemize}
}
\item uncountable noun \\
In journalism, \textbf{copy} is news or information that can be used in an article in a newspaper.
 \textit{
	\begin{itemize}
	\item ...journalists looking for good copy.
	\end{itemize}
}
\end{enumerate}

\section*{consistent}
{\large \color{blue}  }
\subsection*{Explain}
\begin{enumerate}
\item adjective \\
Someone who is \textbf{consistent}  always  behaves in the same way, has the same attitudes towards people or things, or achieves the same level of success in something.
 \textit{
	\begin{itemize}
	\item He has never been the most consistent of players anyway.
	\item ...his consistent support of free trade.
	\end{itemize}
}
\item adjective \\
If one fact or idea is \textbf{consistent}  \textbf{with} another, they do not contradict each other.
 \textit{
	\begin{itemize}
	\item This result is consistent with the findings of Garnett & Tobin.
	\item New goals are not always consistent with the existing policies.
	\end{itemize}
}
\item adjective \\
An argument or set of ideas that is \textbf{consistent} is one in which no part contradicts or conflicts with any other part.
 \textit{
	\begin{itemize}
	\item These are clear consistent policies which we are putting into place.
	\end{itemize}
}
\end{enumerate}

\section*{corridor}
{\large \color{blue}  corridors  }
\subsection*{Explain}
\begin{enumerate}
\item countable noun \\
A \textbf{corridor} is a long passage in a building, with doors and rooms on one or both sides.
 \textit{
	\begin{itemize}
	\end{itemize}
}
\item countable noun \\
A \textbf{corridor} is a strip of land that connects one country to another or gives it a route to the
sea through another country.
 \textit{
	\begin{itemize}
	\item The republic lay in a narrow corridor of disputed land.
	\end{itemize}
}
\end{enumerate}

\section*{eager}
{\large \color{blue}  }
\subsection*{Explain}
\begin{enumerate}
\item adjective \\
If you are \textbf{eager}  \textbf{to} do or have something, you want to do or have it very much.
 \textit{
	\begin{itemize}
	\item Robert was eager to talk about life in the Army.
	\item When my own son was five years old, I became eager for another baby.
	\item The low prices still pull in crowds of eager buyers.
	\end{itemize}
}
\item adjective \\
If you look or sound \textbf{eager} , you look or sound as if you expect something interesting or enjoyable to happen .
 \textit{
	\begin{itemize}
	\item Arty sneered at the crowd of eager faces around him.
	\item Her voice was girlish and eager.
	\end{itemize}
}
\end{enumerate}

\section*{critic}
{\large \color{blue}  critics  }
\subsection*{Explain}
\begin{enumerate}
\item countable noun \\
A \textbf{critic} is a person who writes about and expresses opinions about things such as books, films, music, or art.
 \textit{
	\begin{itemize}
	\item The New York critics had praised her performance.
	\end{itemize}
}
\item countable noun \\
Someone who is a \textbf{critic} of a person or system disapproves of them and criticizes them publicly .
 \textit{
	\begin{itemize}
	\item The newspaper has been the most consistent critic of the government.
	\item He became a fierce critic of the tobacco industry.
	\item Her critics accused her of caring only about success.
	\end{itemize}
}
\end{enumerate}

\section*{enough}
{\large \color{blue}  }
\subsection*{Explain}
\begin{enumerate}
\item determiner \\
\textbf{Enough} means as much as you need or as much as is necessary.
 \textbf{Enough} is also an adverb .
 \textbf{Enough} is also a pronoun.
 \textbf{Enough} is also a quantifier .
 \textbf{Enough} is also an adjective .
 \textit{
	\begin{itemize}
	\item They had enough cash for a one-way ticket.
	\item There aren't enough tents to shelter them all.
	\item I was old enough to work and earn money.
	\item Do you believe that sentences for criminals are tough enough at present?
	\item She graduated with high enough marks to apply for university.
	\item Although the U.K. says efforts are being made, they are not doing enough.
	\item All parents worry about whether their child is getting enough of the right foods.
	\item By autumn it hopes to have 200 new lines – proof enough of the growing market.
	\end{itemize}
}
\item pronoun \\
If you say that something is \textbf{enough} , you mean that you do not want it to continue any longer or get any worse .
 \textbf{Enough} is also a quantifier.
 \textbf{Enough} is also a determiner .
 \textbf{Enough} is also an adverb.
 \textit{
	\begin{itemize}
	\item I met him only the once, and that was enough.
	\item I think I have said enough.
	\item You've got enough to think about for the moment.
	\item Ann had heard enough of this.
	\item He had messed up enough of these occasions to give rise to some anxieties.
	\item I've had enough problems with the police, I don't need this.
	\item Would you shut up, please! I'm having enough trouble with these children!
	\item I'm serious, things are difficult enough as they are.
	\end{itemize}
}
\item adverb \\
You can use \textbf{enough} to say that something is the case to a moderate or fairly large degree .
 \textit{
	\begin{itemize}
	\item Winter is a common enough German surname.
	\item I got this phone call from a gentleman, who seemed sincere enough.
	\item The rest of the evening passed pleasantly enough.
	\end{itemize}
}
\item adverb \\
You use \textbf{enough} in expressions such as \textbf{strangely enough} and \textbf{interestingly enough} to indicate that you think a fact is strange or interesting .
 \textit{
	\begin{itemize}
	\item Strangely enough, the last thing he thought of was his beloved Tanya.
	\item Her latest conquest is an Italian who, interestingly enough, doesn't speak a word
of his native language.
	\end{itemize}
}
\item  \\
 enough is enough \textit{
	\begin{itemize}
	\end{itemize}
}
\item  \\
 have had enough \textit{
	\begin{itemize}
	\end{itemize}
}
\item  \\
 enough said \textit{
	\begin{itemize}
	\end{itemize}
}
\item  \\
 that's enough \textit{
	\begin{itemize}
	\end{itemize}
}
\end{enumerate}

\section*{discipline}
{\large \color{blue}  disciplines  disciplining  disciplined  }
\subsection*{Explain}
\begin{enumerate}
\item uncountable noun \\
\textbf{Discipline} is the practice of making people obey rules or standards of behaviour, and punishing them when they do not.
 \textit{
	\begin{itemize}
	\item Order and discipline have been placed in the hands of headmasters and governing bodies.
	\item ...discipline problems in the classroom.
	\end{itemize}
}
\item uncountable noun \\
\textbf{Discipline} is the quality of being able to behave and work in a controlled way which involves obeying particular rules or standards.
 \textit{
	\begin{itemize}
	\item It was that image of calm and discipline that appealed to voters.
	\end{itemize}
}
\item variable noun \\
If you refer to an activity or situation as a \textbf{discipline} , you mean that, in order to be successful in it, you need to behave in a strictly controlled way and obey particular rules or standards.
 \textit{
	\begin{itemize}
	\item ...inner disciplines like transcendental meditation.
	\item The discipline of studying music can help children develop good work habits.
	\end{itemize}
}
\item verb \\
If someone \textbf{is disciplined} for something that they have done wrong , they are punished for it.
 \textit{
	\begin{itemize}
	\item The workman was disciplined by his company but not dismissed.
	\item Her husband had at last taken a share in disciplining the boy.
	\end{itemize}
}
\item verb \\
If you \textbf{discipline}  \textbf{yourself} to do something, you train yourself to behave and work in a strictly controlled and
 regular way.
 \textit{
	\begin{itemize}
	\item Out on the course you must discipline yourself to let go of detailed theory.
	\item I'm very good at disciplining myself.
	\end{itemize}
}
\item countable noun \\
A \textbf{discipline} is a particular area of study, especially a subject of study in a college or university .
 \textit{
	\begin{itemize}
	\item You've got to make sure that people work together across disciplines.
	\item We're looking for people from a wide range of disciplines.
	\end{itemize}
}
\end{enumerate}

\section*{fantastic}
{\large \color{blue}  }
\subsection*{Explain}
\begin{enumerate}
\item adjective \\
If you say that something is \textbf{fantastic} , you are emphasizing that you think it is very good or that you like it a lot .
 \textit{
	\begin{itemize}
	\item I have a fantastic social life.
	\item I thought she was fantastic.
	\end{itemize}
}
\item adjective \\
A \textbf{fantastic} amount or quantity is an extremely large one.
 \textit{
	\begin{itemize}
	\item ...fantastic amounts of money.
	\end{itemize}
}
\item adjective \\
You describe something as \textbf{fantastic} or \textbf{fantastical} when it seems strange and wonderful or unlikely .
 \textit{
	\begin{itemize}
	\item Unlikely and fantastic legends grew up around a great many figures, both real and
fictitious.
	\item The book has many fantastical aspects.
	\end{itemize}
}
\end{enumerate}

\section*{dividend}
{\large \color{blue}  dividends  }
\subsection*{Explain}
\begin{enumerate}
\item countable noun \\
A \textbf{dividend} is the part of a company's profits which is paid to people who have shares in the company.
 \textit{
	\begin{itemize}
	\item The first quarter dividend has been increased by nearly 4 per cent.
	\end{itemize}
}
\item  \\
 to pay dividends \textit{
	\begin{itemize}
	\end{itemize}
}
\end{enumerate}

\section*{favorable}
{\large \color{blue}  }
\subsection*{Explain}
\begin{enumerate}
\item adjective \\
1.  2.  3.  \textit{
	\begin{itemize}
	\item a favorable impression
	\end{itemize}
}
\end{enumerate}

\section*{engagement}
{\large \color{blue}  engagements  }
\subsection*{Explain}
\begin{enumerate}
\item countable noun \\
An \textbf{engagement} is an arrangement that you have made to do something at a particular time.
 \textit{
	\begin{itemize}
	\item He had an engagement at a restaurant in Greek Street at eight.
	\item ...business-related social engagements.
	\end{itemize}
}
\item countable noun \\
An \textbf{engagement} is an agreement that two people have made with each other to get  married .
 \textit{
	\begin{itemize}
	\item I've broken off my engagement to Arthur.
	\item Announcing our engagement was a relief.
	\end{itemize}
}
\item countable noun \\
You can refer to the period of time during which two people are engaged as their \textbf{engagement} .
 \textit{
	\begin{itemize}
	\item I felt our engagement was quite an unhappy time.
	\end{itemize}
}
\item variable noun \\
A military  \textbf{engagement} is an armed  conflict between two enemies .
 \textit{
	\begin{itemize}
	\item The constitution prohibits them from military engagement on foreign soil.
	\end{itemize}
}
\end{enumerate}

\section*{fond}
{\large \color{blue}  fonder  fondest  }
\subsection*{Explain}
\begin{enumerate}
\item adjective \\
If you are \textbf{fond of} someone, you feel  affection for them.
 \textit{
	\begin{itemize}
	\item I am very fond of Michael.
	\item She was especially fond of a little girl named Betsy.
	\end{itemize}
}
\item adjective \\
You use \textbf{fond} to describe people or their behaviour when they show affection.
 \textit{
	\begin{itemize}
	\item ...a fond father.
	\item He gave him a fond smile.
	\end{itemize}
}
\item adjective \\
If you are \textbf{fond of} something, you like it or you like doing it very much.
 \textit{
	\begin{itemize}
	\item He was fond of marmalade.
	\item She is fond of collecting rare carpets.
	\end{itemize}
}
\item adjective \\
If you have \textbf{fond}  memories of someone or something, you remember them with pleasure .
 \textit{
	\begin{itemize}
	\item I have very fond memories of living in our village.
	\end{itemize}
}
\item adjective \\
You use \textbf{fond} to describe hopes, wishes, or beliefs which you think are foolish because they seem unlikely to be fulfilled.
 \textit{
	\begin{itemize}
	\item My fond hope is that we will be ready by Christmastime.
	\end{itemize}
}
\end{enumerate}

\section*{entry}
{\large \color{blue}  entries  }
\subsection*{Explain}
\begin{enumerate}
\item uncountable noun \\
If you gain  \textbf{entry}  \textbf{to} a particular place, you are able to go in.
 \textit{
	\begin{itemize}
	\item Bill was among the first to gain entry to Buckingham Palace when it opened to the
public recently.
	\item Non-residents were refused entry into the region without authority from their own
district.
	\item The point of entry into Zambia would be the Chirundu border post.
	\item Entry to the museum is free.
	\item ...entry fees to places of scientific interest.
	\end{itemize}
}
\item countable noun \\
You can refer to someone's arrival in a place as their \textbf{entry} , especially when you think that they are trying to be noticed and admired .
 \textit{
	\begin{itemize}
	\item He made his triumphal entry into Mexico City.
	\end{itemize}
}
\item uncountable noun \\
Someone's \textbf{entry}  \textbf{into} a particular society or group is their joining of it.
 \textit{
	\begin{itemize}
	\item ...the minimum age for entry into a Brownie troop.
	\item ...people who cannot gain entry to the owner-occupied housing sector.
	\end{itemize}
}
\item countable noun \\
An \textbf{entry} in a diary, account book, computer file , or reference book is a short piece of writing in it.
 \textit{
	\begin{itemize}
	\item Violet's diary entry for 20 April 1917 records Brigit admitting to the affair.
	\item Many entries relate to the two world wars.
	\end{itemize}
}
\item countable noun \\
An \textbf{entry} for a competition is a piece of work, for example a story or drawing , or the answers to a set of questions , which you complete in order to take part in the competition.
 \textit{
	\begin{itemize}
	\item The closing date for entries is 31st December.
	\end{itemize}
}
\item singular noun \\
Journalists sometimes use \textbf{entry} to refer to the total number of people taking part in an event or competition. For example, if a competition
has an \textbf{entry} of twenty people, twenty people take part in it.
 \textit{
	\begin{itemize}
	\item Prize-money of nearly £90,000 has attracted a record entry of 14 horses from Britain
and Ireland.
	\item Our competition has attracted a huge entry.
	\end{itemize}
}
\item uncountable noun \\
\textbf{Entry} in a competition is the act of taking part in it.
 \textit{
	\begin{itemize}
	\item Entry to this competition is by invitation only.
	\item ...an entry form.
	\end{itemize}
}
\item countable noun \\
The \textbf{entry}  \textbf{to} a place is the way into it, for example a door or gate.
 \textit{
	\begin{itemize}
	\end{itemize}
}
\end{enumerate}

\section*{full}
{\large \color{blue}  fuller  fullest  }
\subsection*{Explain}
\begin{enumerate}
\item adjective \\
If something is \textbf{full} , it contains as much of a substance or as many objects as it can.
 \textit{
	\begin{itemize}
	\item Once the container is full, it stays shut until you turn it clockwise.
	\item ...a full tank of petrol.
	\end{itemize}
}
\item adjective \\
If a place or thing \textbf{is full of} things or people, it contains a large number of them.
 \textit{
	\begin{itemize}
	\item The case was full of clothes.
	\item The streets are still full of debris from two nights of rioting.
	\item ...a useful recipe leaflet full of ideas for using the new cream.
	\end{itemize}
}
\item adjective \\
If someone or something \textbf{is full of} a particular feeling or quality, they have a lot of it.
 \textit{
	\begin{itemize}
	\item I feel full of confidence and so open to possibilities.
	\item Mom's face was full of pain.
	\item ...an exquisite mousse, incredibly rich and full of flavour.
	\end{itemize}
}
\item adjective \\
You say that a place or vehicle is \textbf{full} when there is no space left in it for any more people or things.
 \textit{
	\begin{itemize}
	\item The main car park was full when I left about 10.45.
	\item They stay here a few hours before being sent to refugee camps, which are now almost
full.
	\item The bus was completely full, and lots of people were standing.
	\end{itemize}
}
\item adjective \\
If your hands or arms are \textbf{full} , you are carrying or holding as much as you can carry.
 \textit{
	\begin{itemize}
	\item Sylvia entered, her arms full of packages.
	\item People would go into the store and come out with their arms full.
	\end{itemize}
}
\item adjective \\
If you feel  \textbf{full} , you have eaten or drunk so much that you do not want anything else.
 \textit{
	\begin{itemize}
	\item It's healthy to eat when I'm hungry and to stop when I'm full.
	\end{itemize}
}
\item adjective \\
You use \textbf{full} before a noun to indicate that you are referring to all the details , things, or people that it can possibly include.
 \textit{
	\begin{itemize}
	\item Full details will be sent to you once your application has been accepted.
	\item May I have your full name?
	\item Is full employment any longer achievable?
	\end{itemize}
}
\item adjective \\
\textbf{Full} is used to describe a sound, light, or physical force which is being produced with the greatest possible
power or intensity .
 \textbf{Full} is also an adverb .
 \textit{
	\begin{itemize}
	\item From his study came the sound of Mahler, playing at full volume.
	\item Officials say the operation will be carried out in full daylight.
	\item Then abruptly he revved the engine to full power.
	\item ...a two-seater Lotus, parked with its headlamps full on.
	\end{itemize}
}
\item adjective \\
You use \textbf{full} to emphasize the completeness, intensity, or extent of something.
 \textit{
	\begin{itemize}
	\item We should conserve oil and gas by making full use of other energy sources.
	\item Television cameras are carrying the full horror of this war into homes around the
world.
	\item The lane leading to the farm was in full view of the house windows.
	\item By the time the tests took place, the athletes had had a full 17 hours' notice.
	\end{itemize}
}
\item adjective \\
A \textbf{full} statement or report contains a lot of information and detail.
 \textit{
	\begin{itemize}
	\item He gave a full account of his meeting with the President.
	\item ...the enormous detail in this very full document.
	\end{itemize}
}
\item adjective \\
If you say that someone has or leads a \textbf{full} life, you approve of the fact that they are always  busy and do a lot of different things.
 \textit{
	\begin{itemize}
	\item You will be successful in whatever you do and you will have a very full and interesting
life.
	\end{itemize}
}
\item adverb \\
You use \textbf{full} to emphasize the force or directness with which someone or something is hit or looked at.
 \textit{
	\begin{itemize}
	\item The burning liquid hit him full in the right eye.
	\item She kissed him full on the mouth.
	\item She looked him full in the face as she spoke.
	\end{itemize}
}
\item adjective \\
You use \textbf{full} to refer to something which gives you all the rights, status , or importance for a particular position or activity, rather than just some of them.
 \textit{
	\begin{itemize}
	\item How did the meeting go, did you get your full membership?
	\item Only those who have had full licences for five years may hire cars.
	\end{itemize}
}
\item adjective \\
A \textbf{full}  flavour is strong and rich.
 \textit{
	\begin{itemize}
	\item Italian plum tomatoes have a full flavour, and are best for cooking.
	\end{itemize}
}
\item adjective \\
If you describe a part of someone's body as \textbf{full} , you mean that it is rounded and rather large.
 \textit{
	\begin{itemize}
	\item The Juno Collection specialises in large sizes for ladies with a fuller figure.
	\item ...his strong chin, his full lips, his appealing mustache.
	\end{itemize}
}
\item adjective \\
A \textbf{full} skirt or sleeve is wide and has been made from a lot of fabric.
 \textit{
	\begin{itemize}
	\item My wedding dress has a very full skirt.
	\end{itemize}
}
\item adjective \\
When there is a \textbf{full} moon, the moon appears as a bright , complete circle .
 \textit{
	\begin{itemize}
	\end{itemize}
}
\item  \\
 in full \textit{
	\begin{itemize}
	\end{itemize}
}
\item  \\
 to know full well \textit{
	\begin{itemize}
	\end{itemize}
}
\item  \\
 to the full \textit{
	\begin{itemize}
	\end{itemize}
}
\item  \\
 be full of oneself \textit{
	\begin{itemize}
	\end{itemize}
}
\end{enumerate}

\section*{evolution}
{\large \color{blue}  evolutions  }
\subsection*{Explain}
\begin{enumerate}
\item uncountable noun \\
\textbf{Evolution} is a process of gradual change that takes place over many generations, during which
species of animals, plants, or insects slowly change some of their physical characteristics.
 \textit{
	\begin{itemize}
	\item ...the evolution of plants and animals.
	\item ...the theory of evolution by natural selection.
	\item ...human evolution.
	\end{itemize}
}
\item variable noun \\
\textbf{Evolution} is a process of gradual development in a particular situation or thing over a period of time.
 \textit{
	\begin{itemize}
	\item ...a crucial period in the evolution of modern physics.
	\item ...an accurate account of his country's evolution.
	\item His long life comprised a series of evolutions.
	\end{itemize}
}
\end{enumerate}

\section*{guilty}
{\large \color{blue}  guiltier  guiltiest  }
\subsection*{Explain}
\begin{enumerate}
\item adjective \\
If you feel  \textbf{guilty} , you feel unhappy because you think that you have done something wrong or have failed to do something which you should have done.
 \textit{
	\begin{itemize}
	\item I feel so guilty, leaving all this to you.
	\item When she saw me she looked guilty.
	\end{itemize}
}
\item adjective \\
\textbf{Guilty} is used of an action or fact that you feel guilty about.
 \textit{
	\begin{itemize}
	\item Many may be keeping it a guilty secret.
	\item I leave with a guilty sense of relief.
	\end{itemize}
}
\item adjective \\
If someone is \textbf{guilty}  \textbf{of} a crime or offence, they have committed that crime or offence.
 \textit{
	\begin{itemize}
	\item They were found guilty of murder.
	\item He pleaded guilty to causing actual bodily harm.
	\end{itemize}
}
\item adjective \\
If someone is \textbf{guilty}  \textbf{of} doing something wrong, they have done that thing.
 \textit{
	\begin{itemize}
	\item He claimed Mr Brooke had been guilty of a 'gross error of judgment'.
	\item They will consider whether or not he has been guilty of serious professional misconduct.
	\end{itemize}
}
\end{enumerate}

\section*{expedition}
{\large \color{blue}  expeditions  }
\subsection*{Explain}
\begin{enumerate}
\item countable noun \\
An \textbf{expedition} is an organized journey that is made for a particular purpose such as exploration.
 \textit{
	\begin{itemize}
	\item ...Byrd's 1928 expedition to Antarctica.
	\end{itemize}
}
\item countable noun \\
You can refer to a group of people who are going on an expedition as an \textbf{expedition} .
 \textit{
	\begin{itemize}
	\item Forty-three members of the expedition were killed.
	\end{itemize}
}
\item countable noun \\
An \textbf{expedition} is a short journey or trip that you make for pleasure.
 \textit{
	\begin{itemize}
	\item Caroline joined them on the shopping expeditions.
	\item ...a fishing expedition.
	\end{itemize}
}
\end{enumerate}

\section*{honorable}
{\large \color{blue}  }
\subsection*{Explain}
\begin{enumerate}
\end{enumerate}

\section*{inclusive}
{\large \color{blue}  }
\subsection*{Explain}
\begin{enumerate}
\item adjective \\
If a price is \textbf{inclusive} , it includes all the charges  connected with the goods or services  offered . If a price is \textbf{inclusive}  \textbf{of}  postage and packing , it includes the charge for this.
 \textbf{Inclusive} is also an adverb .
 \textit{
	\begin{itemize}
	\item ...all prices are inclusive of delivery.
	\item ...an inclusive price of £32.90.
	\item ...a special introductory offer of £5,995 fully inclusive.
	\end{itemize}
}
\item adjective \\
After stating the first and last  item in a set of things, you can add  \textbf{inclusive} to make it clear that the items stated are included in the set.
 \textit{
	\begin{itemize}
	\item Training will commence on 5 October, running from Tuesday to Saturday inclusive.
	\item ...£10 for senior citizens and children (5 to 16 inclusive).
	\end{itemize}
}
\item adjective \\
If you describe a group or organization as \textbf{inclusive} , you mean that it allows all kinds of people to belong to it, rather than just one kind of person.
 \textit{
	\begin{itemize}
	\item The academy is far more inclusive now than it used to be.
	\end{itemize}
}
\end{enumerate}

\section*{formation}
{\large \color{blue}  formations  }
\subsection*{Explain}
\begin{enumerate}
\item uncountable noun \\
\textbf{The}  \textbf{formation}  \textbf{of} something is the starting or creation of it.
 \textit{
	\begin{itemize}
	\item ...the formation of a new government.
	\end{itemize}
}
\item uncountable noun \\
\textbf{The}  \textbf{formation}  \textbf{of} an idea , habit , relationship , or character is the process of developing and establishing it.
 \textit{
	\begin{itemize}
	\item My profession had an important influence in the formation of my character and temperament.
	\end{itemize}
}
\item countable noun \\
If people or things are \textbf{in}  \textbf{formation} , they are arranged in a particular pattern as they move.
 \textit{
	\begin{itemize}
	\item He was flying in formation with seven other jets.
	\item The dancers step into a formation which represents the human being.
	\end{itemize}
}
\item countable noun \\
A rock or cloud  \textbf{formation} is rock or cloud of a particular shape or structure.
 \textit{
	\begin{itemize}
	\item ...a vast rock formation shaped like a pillar.
	\item Enormous cloud formations formed a purple mass.
	\end{itemize}
}
\end{enumerate}

\section*{independent}
{\large \color{blue}  independents  }
\subsection*{Explain}
\begin{enumerate}
\item adjective \\
If one thing or person is \textbf{independent}  \textbf{of} another, they are separate and not connected, so the first one is not affected or
influenced by the second.
 \textit{
	\begin{itemize}
	\item Your questions should be independent of each other.
	\item We're going independent from the university and setting up our own group.
	\item Two independent studies have been carried out.
	\end{itemize}
}
\item adjective \\
If someone is \textbf{independent} , they do not need  help or money from anyone else.
 \textit{
	\begin{itemize}
	\item Phil was now much more independent of his parents.
	\item She would like to be financially independent.
	\item There were benefits to being a single independent woman.
	\end{itemize}
}
\item adjective \\
\textbf{Independent} countries and states are not ruled by other countries but have their own government.
 \textit{
	\begin{itemize}
	\item ...a fully independent state.
	\item Papua New Guinea became independent from Australia in 1975.
	\end{itemize}
}
\item adjective \\
An \textbf{independent} organization or other body is one that controls its own finances and operations , rather than being controlled by someone else.
 \textit{
	\begin{itemize}
	\item ...an independent television station.
	\item ...the Office of Government Ethics, an independent agency.
	\item ...a fully independent, not-for-profit organisation.
	\end{itemize}
}
\item adjective \\
An \textbf{independent} school does not receive money from the government or local  council , but from the fees paid by its students ' parents or from charities .
 \textit{
	\begin{itemize}
	\item He taught chemistry at a leading independent school.
	\end{itemize}
}
\item adjective \\
An \textbf{independent}  inquiry or opinion is one that involves people who are not connected with a particular situation, and
should therefore be fair .
 \textit{
	\begin{itemize}
	\item The government ordered an independent inquiry into the affair.
	\item An independent opinion poll published today shows growing discontent with the government.
	\end{itemize}
}
\item adjective \\
An \textbf{independent}  politician is one who does not represent any political party.
 An \textbf{independent} is an independent politician.
 \textit{
	\begin{itemize}
	\item There's been a late surge of support for an independent candidate.
	\item ...the most powerful independent politician in France.
	\item ...Mr Brown has not ruled out the possibility of standing as an independent.
	\end{itemize}
}
\end{enumerate}

\section*{fortune}
{\large \color{blue}  fortunes  }
\subsection*{Explain}
\begin{enumerate}
\item countable noun \\
You can refer to a large sum of money as \textbf{a}  \textbf{fortune} or \textbf{a} small \textbf{fortune} to emphasize how large it is.
 \textit{
	\begin{itemize}
	\item We had to eat out all the time. It ended up costing a fortune.
	\item He made a small fortune in the London property boom.
	\end{itemize}
}
\item countable noun \\
Someone who has a \textbf{fortune} has a very large amount of money.
 \textit{
	\begin{itemize}
	\item He made his fortune in car sales.
	\item He inherited a multi-million-dollar fortune from his inventor mother.
	\end{itemize}
}
\item uncountable noun \\
\textbf{Fortune} or good \textbf{fortune} is good luck. Ill  \textbf{fortune} is bad luck.
 \textit{
	\begin{itemize}
	\item Government ministers are starting to wonder how long their good fortune can last.
	\end{itemize}
}
\item plural noun \\
If you talk about someone's \textbf{fortunes} or the \textbf{fortunes} of something, you are talking about the extent to which they are doing well or being successful .
 \textit{
	\begin{itemize}
	\item The electoral fortunes of the party may decline.
	\item She kept up with the fortunes of the Reeves family.
	\item The company had to do something to reverse its sliding fortunes.
	\end{itemize}
}
\item uncountable noun \\
If you talk about the way someone or something is treated by \textbf{fortune} , you are referring to the good or bad luck that they have.
 \textit{
	\begin{itemize}
	\item He is certainly being smiled on by fortune.
	\end{itemize}
}
\item  \\
 tell your fortune \textit{
	\begin{itemize}
	\end{itemize}
}
\end{enumerate}

\section*{indicative}
{\large \color{blue}  }
\subsection*{Explain}
\begin{enumerate}
\item adjective \\
If one thing is \textbf{indicative}  \textbf{of} another, it suggests what the other thing is likely to be.
 \textit{
	\begin{itemize}
	\item The result was indicative of a strong retail market.
	\item Often physical appearance is indicative of how a person feels.
	\end{itemize}
}
\item singular noun \\
In grammar , a clause that is in \textbf{the indicative} , or in \textbf{the indicative mood} , has a subject followed by a verb group. Examples are 'I'm hungry ' and 'She was followed'. Clauses of this kind are typically used to make statements.
 \textit{
	\begin{itemize}
	\end{itemize}
}
\end{enumerate}

\section*{image}
{\large \color{blue}  images  }
\subsection*{Explain}
\begin{enumerate}
\item countable noun \\
If you have an \textbf{image} of something or someone, you have a picture or idea of them in your mind.
 \textit{
	\begin{itemize}
	\item The image of art theft as a gentleman's crime is outdated.
	\item The words 'Cote d'Azur' conjure up images of sunny days in Mediterranean cafes.
	\end{itemize}
}
\item countable noun \\
The \textbf{image} of a person, group, or organization is the way that they appear to other people.
 \textit{
	\begin{itemize}
	\item Wellington controlled his image as carefully as any modern politician.
	\item The tobacco industry has been trying to improve its image.
	\end{itemize}
}
\item countable noun \\
An \textbf{image} is a picture of someone or something.
 \textit{
	\begin{itemize}
	\item ...photographic images of young children.
	\item A computer in the machine creates an image on the screen.
	\end{itemize}
}
\item countable noun \\
An \textbf{image} is a poetic  description of something.
 \textit{
	\begin{itemize}
	\item The natural images in the poem are meant to be suggestive of realities beyond themselves.
	\end{itemize}
}
\item  \\
 be the image of sb \textit{
	\begin{itemize}
	\end{itemize}
}
\end{enumerate}

\section*{innocent}
{\large \color{blue}  innocents  }
\subsection*{Explain}
\begin{enumerate}
\item adjective \\
If someone is \textbf{innocent} , they did not commit a crime which they have been accused of.
 \textit{
	\begin{itemize}
	\item He was sure that the man was innocent of any crime.
	\item The police knew from day one that I was innocent.
	\end{itemize}
}
\item adjective \\
If someone is \textbf{innocent} , they have no experience or knowledge of the more complex or unpleasant aspects of life .
 An \textbf{innocent} is someone who is innocent.
 \textit{
	\begin{itemize}
	\item They seemed so young and innocent.
	\item He's curiously innocent about what this means to other people.
	\item Ian was a hopeless innocent where women were concerned.
	\end{itemize}
}
\item adjective \\
\textbf{Innocent} people are those who are not involved in a crime or conflict , but are injured or killed as a result of it.
 \textit{
	\begin{itemize}
	\item All those wounded were innocent victims.
	\item The war was killing innocent women and children.
	\end{itemize}
}
\item adjective \\
An \textbf{innocent}  question , remark , or comment is not intended to offend or upset people, even if it does so.
 \textit{
	\begin{itemize}
	\item It was a perfectly innocent question.
	\end{itemize}
}
\end{enumerate}

\section*{joy}
{\large \color{blue}  joys  }
\subsection*{Explain}
\begin{enumerate}
\item uncountable noun \\
\textbf{Joy} is a feeling of great happiness.
 \textit{
	\begin{itemize}
	\item Salter shouted with joy.
	\item ...tears of joy.
	\end{itemize}
}
\item countable noun \\
A \textbf{joy} is something or someone that makes you feel happy or gives you great pleasure.
 \textit{
	\begin{itemize}
	\item One of the joys of being alone is the freedom to do exactly as you choose.
	\item It was a joy to see her looking so well.
	\end{itemize}
}
\item uncountable noun \\
If you get no \textbf{joy} , you do not have success or luck in achieving what you are trying to do.
 \textit{
	\begin{itemize}
	\item They expect no joy from the vote itself.
	\item If you don't get any joy, get in touch with your local councillor.
	\end{itemize}
}
\item  \\
 to jump for joy \textit{
	\begin{itemize}
	\end{itemize}
}
\end{enumerate}

\section*{jealous}
{\large \color{blue}  }
\subsection*{Explain}
\begin{enumerate}
\item adjective \\
If someone is \textbf{jealous} , they feel  angry or bitter because they think that another person is trying to take a lover or friend , or a possession , away from them.
 \textit{
	\begin{itemize}
	\item She got insanely jealous and there was a terrible fight.
	\end{itemize}
}
\item adjective \\
If you are \textbf{jealous}  \textbf{of} another person's possessions or qualities, you feel angry or bitter because you do
not have them.
 \textit{
	\begin{itemize}
	\item She was jealous of his wealth.
	\item You're jealous because the record company rejected your idea.
	\end{itemize}
}
\end{enumerate}

\section*{luck}
{\large \color{blue}  lucks  lucking  lucked  }
\subsection*{Explain}
\begin{enumerate}
\item uncountable noun \\
\textbf{Luck} or \textbf{good luck} is success or good things that happen to you, that do not come from your own abilities or efforts .
 \textit{
	\begin{itemize}
	\item I knew I needed a bit of luck to win.
	\item The Sri Lankans have been having no luck with the weather.
	\item The goal, when it came, owed more to good luck than good planning.
	\end{itemize}
}
\item uncountable noun \\
\textbf{Bad luck} is lack of success or bad things that happen to you, that have not been caused by yourself
or other people.
 \textit{
	\begin{itemize}
	\item I had a lot of bad luck during the first half of this season.
	\item Randall's illness was only bad luck.
	\end{itemize}
}
\item  \\
 any luck \textit{
	\begin{itemize}
	\end{itemize}
}
\item  \\
 bad luck \textit{
	\begin{itemize}
	\end{itemize}
}
\item  \\
 bring someone luck \textit{
	\begin{itemize}
	\end{itemize}
}
\item  \\
 down on one's luck \textit{
	\begin{itemize}
	\end{itemize}
}
\item  \\
 the luck of the draw \textit{
	\begin{itemize}
	\end{itemize}
}
\item  \\
 good luck \textit{
	\begin{itemize}
	\end{itemize}
}
\item  \\
 be in luck \textit{
	\begin{itemize}
	\end{itemize}
}
\item  \\
 be just sb's luck \textit{
	\begin{itemize}
	\end{itemize}
}
\item  \\
 be out of luck \textit{
	\begin{itemize}
	\end{itemize}
}
\item  \\
 no such luck \textit{
	\begin{itemize}
	\end{itemize}
}
\item  \\
 to push your luck \textit{
	\begin{itemize}
	\end{itemize}
}
\item  \\
 luck was on sb's side \textit{
	\begin{itemize}
	\end{itemize}
}
\item  \\
 to try your luck \textit{
	\begin{itemize}
	\end{itemize}
}
\item  \\
 with any luck \textit{
	\begin{itemize}
	\end{itemize}
}
\end{enumerate}

\section*{marvelous}
{\large \color{blue}  }
\subsection*{Explain}
\begin{enumerate}
\item adjective \\
1.  2.  3.  \textit{
	\begin{itemize}
	\end{itemize}
}
\end{enumerate}

\section*{mechanic}
{\large \color{blue}  mechanics  }
\subsection*{Explain}
\begin{enumerate}
\item countable noun \\
A \textbf{mechanic} is someone whose job is to repair and maintain machines and engines , especially  car engines.
 \textit{
	\begin{itemize}
	\item If you smell gas fumes or burning, take the car to your mechanic.
	\item An elevator mechanic can work the machinery directly by turning this lever.
	\end{itemize}
}
\item plural noun \\
\textbf{The}  \textbf{mechanics}  \textbf{of} a process, system, or activity are the way in which it works or the way in which
it is done.
 \textit{
	\begin{itemize}
	\item What are the mechanics of this new process?
	\item The mechanics of the job, however, have changed little since then.
	\end{itemize}
}
\item uncountable noun \\
\textbf{Mechanics} is the part of physics that deals with the natural forces that act on moving or stationary objects.
 \textit{
	\begin{itemize}
	\item ...the other great theory of 20th-century physics, quantum mechanics.
	\item He has not studied mechanics or engineering.
	\end{itemize}
}
\end{enumerate}

\section*{okay}
{\large \color{blue}  okays  okaying  okayed  }
\subsection*{Explain}
\begin{enumerate}
\item adjective \\
If you say that something is \textbf{okay} , you find it satisfactory or acceptable .
 \textbf{Okay} is also an adverb .
 \textit{
	\begin{itemize}
	\item ...a shooting range where it's OK to use weapons.
	\item Is it okay if I come by myself?
	\item I guess for a fashionable restaurant like this the prices are OK.
	\item We seemed to manage okay for the first year or so after David was born.
	\end{itemize}
}
\item adjective \\
If you say that someone is \textbf{okay} , you mean that they are safe and well .
 \textit{
	\begin{itemize}
	\item Check that the baby's okay.
	\item 'Don't worry about me,' I said. 'I'll be okay.'
	\end{itemize}
}
\item convention \\
You can say ' \textbf{Okay} ' to show that you agree to something.
 \textit{
	\begin{itemize}
	\item 'Just tell him Sir Kenneth would like to talk to him.'—'OK.'
	\item 'Shall I give you a ring on Friday?'—'Yeah okay.'
	\end{itemize}
}
\item convention \\
You can say ' \textbf{Okay?} ' to check whether the person you are talking to understands what you have said and accepts it.
 \textit{
	\begin{itemize}
	\item Add them together, divide by five, and you've got the average. Okay?
	\item We'll get together next week, OK?
	\end{itemize}
}
\item convention \\
You can use \textbf{okay} to indicate that you want to start talking about something else or doing something else.
 \textit{
	\begin{itemize}
	\item OK. Now, let's talk some business.
	\item Tim jumped to his feet. 'Okay, let's go.'
	\end{itemize}
}
\item convention \\
You can use \textbf{okay} to stop someone arguing with you by showing that you accept the point they are making, though you do not necessarily  regard it as very important .
 \textit{
	\begin{itemize}
	\item Okay, there is a slight difference.
	\item Okay, so I'm forty-two.
	\end{itemize}
}
\item verb \\
If someone in authority  \textbf{okays} something, they officially agree to it or allow it to happen .
 \textbf{Okay} is also a noun .
 \textit{
	\begin{itemize}
	\item His doctor wouldn't OK the trip.
	\item We are all wondering why the government is suddenly okaying a brand new school on
the island.
	\item He gave the okay to issue a new press release.
	\item We are ready to start flying to Britain as soon as we get the okay.
	\end{itemize}
}
\end{enumerate}

\section*{monument}
{\large \color{blue}  monuments  }
\subsection*{Explain}
\begin{enumerate}
\item countable noun \\
A \textbf{monument} is a large structure, usually made of stone , which is built to remind people of an event in history or of a famous person.
 \textit{
	\begin{itemize}
	\end{itemize}
}
\item countable noun \\
A \textbf{monument} is something such as a castle or bridge which was built a very long time ago and is regarded as an important part of a country's history.
 \textit{
	\begin{itemize}
	\item ...the ancient monuments of England
	\end{itemize}
}
\item countable noun \\
If you describe something as a \textbf{monument}  \textbf{to} someone's qualities, you mean that it is a very good example of the results or effects
of those qualities.
 \textit{
	\begin{itemize}
	\item By his international achievements he leaves a fitting monument to his beliefs.
	\end{itemize}
}
\end{enumerate}

\section*{proud}
{\large \color{blue}  prouder  proudest  }
\subsection*{Explain}
\begin{enumerate}
\item adjective \\
If you feel  \textbf{proud} , you feel pleased about something good that you possess or have done , or about something good that a person close to you has done.
 \textit{
	\begin{itemize}
	\item I felt proud of his efforts.
	\item They are proud that she is doing well at school.
	\item I am proud to be a Canadian.
	\item Derek is now the proud father of a bouncing baby girl.
	\end{itemize}
}
\item adjective \\
Your \textbf{proudest}  moments or achievements are the ones that you are most proud of.
 \textit{
	\begin{itemize}
	\item This must have been one of the proudest moments of his busy and hard working life.
	\end{itemize}
}
\item adjective \\
Someone who is \textbf{proud} has respect for themselves and does not want to lose the respect that other people have for them.
 \textit{
	\begin{itemize}
	\item He was too proud to ask his family for help and support.
	\item We are a proud people. We are not used to begging or taking things.
	\end{itemize}
}
\item adjective \\
Someone who is \textbf{proud} feels that they are better or more important than other people.
 \textit{
	\begin{itemize}
	\item She was said to be proud and arrogant.
	\end{itemize}
}
\item adjective \\
If one object  \textbf{stands}  \textbf{proud}  \textbf{of} another object that it is attached to or next to, it extends beyond it.
 \textit{
	\begin{itemize}
	\item The handles stand proud of the doors of the car.
	\end{itemize}
}
\item  \\
 do sb proud \textit{
	\begin{itemize}
	\end{itemize}
}
\end{enumerate}

\section*{motion}
{\large \color{blue}  motions  motioning  motioned  }
\subsection*{Explain}
\begin{enumerate}
\item uncountable noun \\
\textbf{Motion} is the activity or process of continually changing position or moving from one place to another.
 \textit{
	\begin{itemize}
	\item ...the laws governing light, sound, and motion.
	\item One group of muscles sets the next group in motion.
	\item The wind from the car's motion whipped her hair around her head.
	\end{itemize}
}
\item countable noun \\
A \textbf{motion} is an action, gesture, or movement.
 \textit{
	\begin{itemize}
	\item Cover each part of the body with long sweeping strokes or circular motions.
	\item He made a neat chopping motion with his hand.
	\end{itemize}
}
\item countable noun \\
A \textbf{motion} is a formal proposal or statement in a meeting, debate, or trial , which is discussed and then voted on or decided on.
 \textit{
	\begin{itemize}
	\item The conference is now debating the motion and will vote on it shortly.
	\item Opposition parties are likely to bring a no-confidence motion against the government.
	\item He is eligible now to file a motion for a new trial.
	\end{itemize}
}
\item verb \\
If you \textbf{motion} to someone, you move your hand or head as a way of telling them to do something or telling them where to go .
 \textit{
	\begin{itemize}
	\item She motioned for the locked front doors to be opened.
	\item He stood aside and motioned Don to the door.
	\item I motioned him to join us.
	\item He motioned to her to go behind the screen.
	\end{itemize}
}
\item countable noun \\
Some people, especially  doctors or nurses , use \textbf{motion} as a polite way of referring to a person's act of defecation or the faeces produced.
 \textit{
	\begin{itemize}
	\item Try to make sure your bowel motions are regular and that you avoid any constipation.
	\end{itemize}
}
\item  \\
 go through the motions \textit{
	\begin{itemize}
	\end{itemize}
}
\item  \\
 go through the motions \textit{
	\begin{itemize}
	\end{itemize}
}
\item  \\
 in motion \textit{
	\begin{itemize}
	\end{itemize}
}
\item  \\
 set the wheels in motion \textit{
	\begin{itemize}
	\end{itemize}
}
\end{enumerate}

\section*{ready}
{\large \color{blue}  readier  readiest  readies  readying  readied  }
\subsection*{Explain}
\begin{enumerate}
\item adjective \\
If someone is \textbf{ready} , they are properly prepared for something. If something is \textbf{ready} , it has been properly prepared and is now  able to be used.
 \textit{
	\begin{itemize}
	\item It took her a long time to get ready for church.
	\item The parts are packed and ready for shipping.
	\item Are you ready to board, Mr Daly?
	\item In a few days time the sprouts will be ready to eat.
	\item Tomorrow he would tell his pilot to get the aircraft ready.
	\item It's eight-fifteen, dear, and your breakfast's ready.
	\end{itemize}
}
\item adjective \\
If you are \textbf{ready}  \textbf{for} something or \textbf{ready}  \textbf{to} do something, you have enough experience to do it or you are old enough and sensible enough to do it.
 \textit{
	\begin{itemize}
	\item She says she's not ready for motherhood.
	\item You'll have no trouble getting him into a nursery when you feel he's ready to go.
	\end{itemize}
}
\item adjective \\
If you are \textbf{ready}  \textbf{to} do something, you are willing to do it.
 \textit{
	\begin{itemize}
	\item They were ready to die for their beliefs.
	\item She was always ready to give interviews.
	\end{itemize}
}
\item adjective \\
If you are \textbf{ready for} something, you need it or want it.
 \textit{
	\begin{itemize}
	\item I don't know about you, but I'm ready for bed.
	\item After five days in the heat of Bangkok, we were ready for the beach.
	\end{itemize}
}
\item adjective \\
To be \textbf{ready}  \textbf{to} do something means to be about to do it or likely to do it.
 \textit{
	\begin{itemize}
	\item She looked ready to cry.
	\item Just as we were ready to sit down to dinner, a little boy came running in.
	\item He says it's like a volcano ready to erupt.
	\end{itemize}
}
\item adjective \\
You use \textbf{ready} to describe things that are able to be used very quickly and easily .
 \textit{
	\begin{itemize}
	\item I didn't have a ready answer for this dilemma.
	\item 'But not quite yet,' he says quickly, with that ready smile of his.
	\item ...a ready supply of well-trained and well-motivated workers.
	\end{itemize}
}
\item adjective \\
\textbf{Ready}  money is in the form of notes and coins  rather than cheques or credit  cards , and so it can be used immediately .
 Ready money is sometimes  referred to as \textbf{the readies} .
 \textit{
	\begin{itemize}
	\item I'm afraid I don't have enough ready cash.
	\item She was a bit short of the readies.
	\end{itemize}
}
\item verb \\
When you \textbf{ready} something, you prepare it for a particular  purpose .
 \textit{
	\begin{itemize}
	\item John's soldiers were readying themselves for the final assault.
	\item Cameramen readied tripods.
	\end{itemize}
}
\item combining form \\
\textbf{Ready}  combines with past participles to indicate that something has already been done , and that therefore you do not have to do it yourself.
 \textit{
	\begin{itemize}
	\item You can buy ready-printed forms for wills at stationery shops.
	\item If you buy the fish ready filleted, make sure the flesh is firm and springy.
	\end{itemize}
}
\item  \\
 at the ready \textit{
	\begin{itemize}
	\end{itemize}
}
\item  \\
 ready and waiting \textit{
	\begin{itemize}
	\end{itemize}
}
\item  \\
 ready when you are \textit{
	\begin{itemize}
	\end{itemize}
}
\end{enumerate}

\section*{movement}
{\large \color{blue}  movements  }
\subsection*{Explain}
\begin{enumerate}
\item countable noun \\
A \textbf{movement} is a group of people who share the same beliefs , ideas , or aims .
 \textit{
	\begin{itemize}
	\item It's part of a broader Hindu nationalist movement that's gaining strength throughout
the country.
	\item ...the women's movement.
	\end{itemize}
}
\item variable noun \\
\textbf{Movement} involves changing position or going from one place to another.
 \textit{
	\begin{itemize}
	\item They actually monitor the movement of the fish going up river.
	\item ...the plan for free movement of people, goods, capital and services across internal
Community borders.
	\item There was movement behind the window in the back door.
	\item A man was directing the movements of a large removal van.
	\item Her hand movements are becoming more animated.
	\end{itemize}
}
\item variable noun \\
A \textbf{movement} is a planned change in position that an army makes during a battle or military exercise .
 \textit{
	\begin{itemize}
	\item There are reports of fresh troop movements across the border.
	\end{itemize}
}
\item variable noun \\
\textbf{Movement} is a gradual  development or change of an attitude , opinion , or policy .
 \textit{
	\begin{itemize}
	\item ...the movement towards democracy in Latin America.
	\item The talks went well and participants believed movement forward was possible.
	\end{itemize}
}
\item plural noun \\
Your \textbf{movements} are everything which you do or plan to do during a period of time.
 \textit{
	\begin{itemize}
	\item I want a full account of your movements the night Mr Gower was killed.
	\end{itemize}
}
\item countable noun \\
A \textbf{movement} of a piece of classical music is one of its main sections.
 \textit{
	\begin{itemize}
	\item ...the first movement of Beethoven's 7th symphony.
	\end{itemize}
}
\end{enumerate}

\section*{responsible}
{\large \color{blue}  }
\subsection*{Explain}
\begin{enumerate}
\item adjective \\
If someone or something is \textbf{responsible}  \textbf{for} a particular event or situation , they are the cause of it or they can be blamed for it.
 \textit{
	\begin{itemize}
	\item He still felt responsible for her death.
	\item I want you to do everything you can to find out who's responsible.
	\end{itemize}
}
\item adjective \\
If you are \textbf{responsible}  \textbf{for} something, it is your job or duty to deal with it and make decisions relating to it.
 \textit{
	\begin{itemize}
	\item ...the minister responsible for the environment.
	\item ...the man responsible for finding the volunteers.
	\end{itemize}
}
\item adjective \\
If you are \textbf{responsible to} a person or group, they have authority over you and you have to report to them about what you do.
 \textit{
	\begin{itemize}
	\item I'm responsible to my board of directors.
	\item The government will be responsible to the President alone.
	\end{itemize}
}
\item adjective \\
\textbf{Responsible} people behave properly and sensibly, without needing to be supervised .
 \textit{
	\begin{itemize}
	\item He's a very responsible sort of person.
	\item He feels that the media should be more responsible in what they report.
	\end{itemize}
}
\item adjective \\
\textbf{Responsible}  jobs involve making important decisions or carrying out important tasks .
 \textit{
	\begin{itemize}
	\item I work in a government office. It's a responsible position, I suppose, but not very
exciting.
	\item They have been demoted to less responsible jobs.
	\end{itemize}
}
\end{enumerate}

\section*{observation}
{\large \color{blue}  observations  }
\subsection*{Explain}
\begin{enumerate}
\item uncountable noun \\
\textbf{Observation} is the action or process of carefully watching someone or something.
 \textit{
	\begin{itemize}
	\item ...careful observation of the movement of the planets.
	\item In hospital she'll be under observation all the time.
	\end{itemize}
}
\item countable noun \\
An \textbf{observation} is something that you have learned by seeing or watching something and thinking about it.
 \textit{
	\begin{itemize}
	\item This book contains observations about the causes of addictions.
	\end{itemize}
}
\item countable noun \\
If a person makes an \textbf{observation} , they make a comment about something or someone, usually as a result of watching
how they behave .
 \textit{
	\begin{itemize}
	\item 'You're an obstinate man,' she said. 'Is that a criticism,' I said, 'or just an observation?'.
	\end{itemize}
}
\item uncountable noun \\
\textbf{Observation} is the ability to pay a lot of attention to things and to notice more about them than most people do.
 \textit{
	\begin{itemize}
	\item She has good powers of observation.
	\end{itemize}
}
\end{enumerate}

\section*{secure}
{\large \color{blue}  secures  securing  secured  }
\subsection*{Explain}
\begin{enumerate}
\item verb \\
If you \textbf{secure} something that you want or need , you obtain it, often after a lot of effort .
 \textit{
	\begin{itemize}
	\item Federal leaders continued their efforts to secure a ceasefire.
	\item Graham's achievements helped secure him the job.
	\end{itemize}
}
\item verb \\
If you \textbf{secure} a place, you make it safe from harm or attack.
 \textit{
	\begin{itemize}
	\item Staff withdrew from the main part of the prison but secured the perimeter.
	\item The shed was secured by a hasp and staple fastener.
	\end{itemize}
}
\item adjective \\
A \textbf{secure} place is tightly locked or well  protected , so that people cannot enter it or leave it.
 \textit{
	\begin{itemize}
	\item We shall make sure our home is as secure as possible from now on.
	\end{itemize}
}
\item verb \\
If you \textbf{secure} an object , you fasten it firmly to another object.
 \textit{
	\begin{itemize}
	\item He helped her close the cases up, and then he secured the canvas straps.
	\item The frames are secured by horizontal rails to the back wall.
	\end{itemize}
}
\item adjective \\
If an object is \textbf{secure} , it is fixed firmly in position.
 \textit{
	\begin{itemize}
	\item Check joints are secure and the wood is sound.
	\item Shelves are only as secure as their fixings.
	\end{itemize}
}
\item adjective \\
If you describe something such as a job as \textbf{secure} , it is certain not to change or end .
 \textit{
	\begin{itemize}
	\item ...trade union demands for secure wages and employment.
	\item ...the failure of financial institutions once thought to be secure.
	\end{itemize}
}
\item adjective \\
A \textbf{secure}  base or foundation is strong and reliable .
 \textit{
	\begin{itemize}
	\item He was determined to give his family a secure and solid base.
	\end{itemize}
}
\item adjective \\
If you feel  \textbf{secure} , you feel safe and happy and are not worried about life .
 \textit{
	\begin{itemize}
	\item She felt secure and protected when she was with him.
	\item The government must feel secure before it makes the concessions needed for peace.
	\end{itemize}
}
\item verb \\
If a loan  \textbf{is secured} , the person who lends the money  may take property such as a house from the person who borrows the money if they fail to repay it.
 \textit{
	\begin{itemize}
	\item The loan is secured against your home.
	\item His main task is to raise enough finance to repay secured loans.
	\end{itemize}
}
\end{enumerate}

\section*{operation}
{\large \color{blue}  operations  }
\subsection*{Explain}
\begin{enumerate}
\item countable noun \\
An \textbf{operation} is a highly  organized activity that involves many people doing different things.
 \textit{
	\begin{itemize}
	\item The rescue operation began on Friday afternoon.
	\item The soldiers were engaged in a military operation close to the border.
	\item ...a police operation against organised crime.
	\end{itemize}
}
\item countable noun \\
A business or company can be referred to as an \textbf{operation} .
 \textit{
	\begin{itemize}
	\item Thorn's electronics operation employs around 5,000 people.
	\item The two parent groups now run their business as a single combined operation.
	\end{itemize}
}
\item countable noun \\
When a patient has an \textbf{operation} , a surgeon  cuts open their body in order to remove, replace , or repair a diseased or damaged part.
 \textit{
	\begin{itemize}
	\item Charles was at the clinic recovering from an operation on his arm.
	\end{itemize}
}
\item uncountable noun \\
If a system is \textbf{in}  \textbf{operation} , it is being used.
 \textit{
	\begin{itemize}
	\item The scheme is expected to be in operation by the end of March.
	\item ...the free banking system that has been in operation since the early eighties.
	\end{itemize}
}
\item uncountable noun \\
If a machine or device is \textbf{in}  \textbf{operation} , it is working .
 \textit{
	\begin{itemize}
	\item There are three ski lifts in operation.
	\end{itemize}
}
\item  \\
 come into operation/put sth into operation \textit{
	\begin{itemize}
	\end{itemize}
}
\end{enumerate}

\section*{separate}
{\large \color{blue}  separates  separating  separated  }
\subsection*{Explain}
\begin{enumerate}
\item adjective \\
If one thing is \textbf{separate}  \textbf{from} another, there is a barrier, space , or division between them, so that they are clearly two things.
 \textit{
	\begin{itemize}
	\item Each villa has a separate sitting-room.
	\item They are now making plans to form their own separate party.
	\item Business bank accounts were kept separate from personal ones.
	\end{itemize}
}
\item adjective \\
If you refer to \textbf{separate} things, you mean several different things, rather than just one thing.
 \textit{
	\begin{itemize}
	\item Use separate chopping boards for raw meats, cooked meats, vegetables and salads.
	\item Men and women have separate exercise rooms.
	\item The authorities say six civilians have been killed in two separate attacks.
	\end{itemize}
}
\item verb \\
If you \textbf{separate} people or things that are together, or if they \textbf{separate} , they move apart.
 \textit{
	\begin{itemize}
	\item Police moved in to separate the two groups.
	\item The pans were held in both hands and swirled around to separate gold particles from
the dirt.
	\item The front end of the car separated from the rest of the vehicle.
	\item They separated. Stephen returned to the square.
	\item They're separated from the adult inmates.
	\end{itemize}
}
\item verb \\
If you \textbf{separate} people or things that have been connected , or if one \textbf{separates}  \textbf{from} another, the connection between them is ended .
 \textit{
	\begin{itemize}
	\item They want to separate teaching from research.
	\item It's very possible that we may see a movement to separate the two parts of the country.
	\item ...Catalan parties vowing to separate from Spain.
	\end{itemize}
}
\item verb \\
If a couple who are married or living together \textbf{separate} , they decide to live apart.
 \textit{
	\begin{itemize}
	\item Her parents separated when she was very young.
	\item Since I separated from my husband I have gone a long way.
	\end{itemize}
}
\item verb \\
An object , obstacle , distance , or period of time which \textbf{separates} two people, groups, or things exists between them.
 \textit{
	\begin{itemize}
	\item ...the white-railed fence that separated the yard from the paddock.
	\item They had undoubtedly made progress in the six years that separated the two periods.
	\item Rural communities are widely separated and often small.
	\item But a group of six women and 23 children got separated from the others.
	\end{itemize}
}
\item verb \\
If you \textbf{separate} one idea or fact  \textbf{from} another, you clearly see or show the difference between them.
 \textbf{Separate out} means the same as separate .
 \textit{
	\begin{itemize}
	\item It is difficult to separate legend from truth.
	\item ...learning how to separate real problems from imaginary illnesses.
	\item It is difficult to separate the two aims.
	\item How can one ever separate out the act from the attitudes that surround it?
	\end{itemize}
}
\item verb \\
A quality or factor that \textbf{separates} one thing \textbf{from} another is the reason why the two things are different from each other.
 \textit{
	\begin{itemize}
	\item The single most important factor that separates ordinary photographs from good photographs
is the lighting.
	\item The question of what separates man from animals has fascinated scientists for centuries.
	\end{itemize}
}
\item verb \\
If a particular number of points \textbf{separate} two teams or competitors , one of them is winning or has won by that number of points.
 \textit{
	\begin{itemize}
	\item In the end only three points separated the two teams.
	\end{itemize}
}
\item verb \\
If you \textbf{separate} a group of people or things \textbf{into} smaller elements , or if a group \textbf{separates} , it is divided into smaller elements.
 \textbf{Separate out} means the same as separate .
 \textit{
	\begin{itemize}
	\item The police wanted to separate them into smaller groups.
	\item Wallerstein's work can be separated into three main component themes.
	\item Let's separate into smaller groups.
	\item So all the colours that make up white light are sent in different directions and
they separate.
	\item If prepared many hours ahead, the mixture may separate out.
	\end{itemize}
}
\item plural noun \\
\textbf{Separates} are clothes such as skirts , trousers , and shirts which cover just the top  half or the bottom half of your body.
 \textit{
	\begin{itemize}
	\end{itemize}
}
\item  \\
 go their separate ways \textit{
	\begin{itemize}
	\end{itemize}
}
\end{enumerate}

\section*{playground}
{\large \color{blue}  playgrounds  }
\subsection*{Explain}
\begin{enumerate}
\item countable noun \\
A \textbf{playground} is a piece of land, at school or in a public area, where children can play.
 \textit{
	\begin{itemize}
	\end{itemize}
}
\item countable noun \\
If you describe a place as a \textbf{playground} for a certain group of people, you mean that those people like to enjoy themselves there or go on holiday there.
 \textit{
	\begin{itemize}
	\item ...St Tropez, playground of the rich and famous.
	\end{itemize}
}
\end{enumerate}

\section*{skilled}
{\large \color{blue}  }
\subsection*{Explain}
\begin{enumerate}
\item adjective \\
Someone who is \textbf{skilled} has the knowledge and ability to do something well .
 \textit{
	\begin{itemize}
	\item Not all doctors are skilled in helping their patients make choices.
	\item ...a network of amateur but highly skilled observers of wildlife.
	\end{itemize}
}
\item adjective \\
\textbf{Skilled} work can only be done by people who have had some training.
 \textit{
	\begin{itemize}
	\item New industries demanded skilled labour not available locally.
	\item ...skilled workers, such as plumbers and electricians.
	\end{itemize}
}
\end{enumerate}

\section*{refusal}
{\large \color{blue}  refusals  }
\subsection*{Explain}
\begin{enumerate}
\item variable noun \\
Someone's \textbf{refusal}  \textbf{to} do something is the fact of them showing or saying that they will not do it, allow it, or accept it.
 \textit{
	\begin{itemize}
	\item ...her refusal to accept change.
	\item His letter in response to her request had contained a firm refusal.
	\item ...the Council's refusal of planning permission for a major shopping centre.
	\item We would appreciate confirmation of your refusal of our invitation to take part.
	\end{itemize}
}
\item  \\
 first refusal \textit{
	\begin{itemize}
	\end{itemize}
}
\end{enumerate}

\section*{sophomore}
{\large \color{blue}  sophomores  }
\subsection*{Explain}
\begin{enumerate}
\item countable noun \\
A \textbf{sophomore} is a student in the second year of college or high school.
 \textit{
	\begin{itemize}
	\end{itemize}
}
\end{enumerate}

\section*{ring}
{\large \color{blue}  rings  ringing  rang  rung  }
\subsection*{Explain}
\begin{enumerate}
\item verb \\
When you \textbf{ring} someone, you phone them.
 \textbf{Ring up} means the same as ring1 .
 \textit{
	\begin{itemize}
	\item He rang me at my mother's.
	\item If you'd like more information, ring the Hotline on 414 3929.
	\item I would ring when I got back to the hotel.
	\item She has rung home just once.
	\item Could someone ring for a taxi?
	\item You can ring us up anytime.
	\item John rang up and invited himself over for dinner.
	\item A few months ago I rang up about some housing problems.
	\item Nobody rings up a doctor in the middle of the night for no reason.
	\end{itemize}
}
\item verb \\
When a phone \textbf{rings} , it makes a sound to let you know that someone is phoning you.
 \textbf{Ring} is also a noun.
 \textit{
	\begin{itemize}
	\item As soon as he got home, the phone rang.
	\item The phone never stopped ringing.
	\item After at least eight rings, an ancient-sounding maid answered the phone.
	\end{itemize}
}
\item verb \\
When you \textbf{ring} a bell or when a bell \textbf{rings} , it makes a sound.
 \textbf{Ring} is also a noun.
 \textit{
	\begin{itemize}
	\item He heard the school bell ring.
	\item The door was opened before she could ring the bell.
	\item There was a ring at the bell.
	\end{itemize}
}
\item verb \\
If you \textbf{ring}  \textbf{for} something, you ring a bell to call someone to bring it to you. If you \textbf{ring} for someone, you ring a bell so that they will come to you.
 \textit{
	\begin{itemize}
	\item Shall I ring for a fresh pot of tea?
	\item He rang for the guard to let him out.
	\end{itemize}
}
\item verb \\
If you say that a place \textbf{is ringing}  \textbf{with} sound, usually pleasant sound, you mean that the place is completely filled with that sound.
 \textit{
	\begin{itemize}
	\item The whole place was ringing with music.
	\end{itemize}
}
\item singular noun \\
You can use \textbf{ring} to describe a quality that something such as a statement, discussion , or argument seems to have. For example, if an argument \textbf{has a familiar ring} , it seems familiar.
 \textit{
	\begin{itemize}
	\item His proud boast of leading 'the party of low taxation' has a hollow ring.
	\end{itemize}
}
\item  \\
 to ring the changes \textit{
	\begin{itemize}
	\end{itemize}
}
\item  \\
 ring in one's ears/ring in one's head \textit{
	\begin{itemize}
	\end{itemize}
}
\item  \\
 give sb a ring \textit{
	\begin{itemize}
	\end{itemize}
}
\item  \\
 to ring true \textit{
	\begin{itemize}
	\end{itemize}
}
\end{enumerate}

\section*{sufficient}
{\large \color{blue}  }
\subsection*{Explain}
\begin{enumerate}
\item adjective \\
If something is \textbf{sufficient}  \textbf{for} a particular purpose, there is enough of it for the purpose.
 \textit{
	\begin{itemize}
	\item One metre of fabric is sufficient to cover the exterior of an 18-in-diameter hatbox.
	\item Lighting levels should be sufficient for photography without flash.
	\item There was not sufficient evidence to secure a conviction.
	\end{itemize}
}
\item adjective \\
If something is a \textbf{sufficient} cause or condition for something to happen , it can happen.
 \textit{
	\begin{itemize}
	\item Discipline is a necessary, but certainly not a sufficient condition for learning
to take place.
	\end{itemize}
}
\end{enumerate}

\section*{ruby}
{\large \color{blue}  rubies  }
\subsection*{Explain}
\begin{enumerate}
\item countable noun \\
A \textbf{ruby} is a dark red jewel .
 \textit{
	\begin{itemize}
	\item ...a ruby and diamond ring.
	\end{itemize}
}
\item colour \\
Something that is \textbf{ruby} is dark red in colour.
 \textit{
	\begin{itemize}
	\item ...a glass of ruby-red Cabernet Sauvignon.
	\end{itemize}
}
\end{enumerate}

\section*{suitable}
{\large \color{blue}  }
\subsection*{Explain}
\begin{enumerate}
\item adjective \\
Someone or something that is \textbf{suitable}  \textbf{for} a particular purpose or occasion is right or acceptable for it.
 \textit{
	\begin{itemize}
	\item Employers usually decide within five minutes whether someone is suitable for the
job.
	\item She had no other dress suitable for the occasion.
	\item The authority must make suitable accommodation available to the family.
	\end{itemize}
}
\end{enumerate}

\section*{shape}
{\large \color{blue}  shapes  shaping  shaped  }
\subsection*{Explain}
\begin{enumerate}
\item countable noun \\
The \textbf{shape}  \textbf{of} an object, a person, or an area is the appearance of their outside edges or surfaces, for example whether they are round , square , curved , or fat .
 \textit{
	\begin{itemize}
	\item Each mirror is made to order and can be designed to almost any shape or size.
	\item ...little pens in the shape of baseball bats.
	\item The glass bottle is the shape of a woman's torso.
	\item ...sofas and chairs of contrasting shapes and colours.
	\item The buds are conical or pyramidal in shape.
	\item These bras should be handwashed to help them keep their shape.
	\item Walking is extremely beneficial to your body shape.
	\end{itemize}
}
\item countable noun \\
You can refer to something that you can see as a \textbf{shape} if you cannot see it clearly , or if its outline is the clearest or most striking  aspect of it.
 \textit{
	\begin{itemize}
	\item The great grey shape of a tank rolled out of the village.
	\item Lying in bed we often see dark shapes of herons silhouetted against the moon.
	\end{itemize}
}
\item countable noun \\
A \textbf{shape} is a space  enclosed by an outline, for example a circle , a square, or a triangle .
 \textit{
	\begin{itemize}
	\item ...if you imagine a sort of a kidney shape.
	\item He suggested that the shapes represented a map of Britain and Ireland.
	\end{itemize}
}
\item singular noun \\
The \textbf{shape}  \textbf{of} something that is planned or organized is its structure and character .
 \textit{
	\begin{itemize}
	\item The last two weeks have seen a lot of talk about the future shape of Europe.
	\item Ultimately, we can change the shape of people's lives.
	\end{itemize}
}
\item verb \\
Someone or something that \textbf{shapes} a situation or an activity has a very great  influence on the way it develops .
 \textit{
	\begin{itemize}
	\item Their views provide an insight into how environmental issues are shaping the future
of business.
	\item Like it or not, our families shape our lives and make us what we are.
	\end{itemize}
}
\item verb \\
If you \textbf{shape} an object, you give it a particular shape, using your hands or a tool .
 \textit{
	\begin{itemize}
	\item Cut the dough in half and shape each half into a loaf.
	\item ...machinery for shaping the plutonium core of nuclear weapons.
	\end{itemize}
}
\item  \\
 the shape of things to come \textit{
	\begin{itemize}
	\end{itemize}
}
\item  \\
 in any shape or form \textit{
	\begin{itemize}
	\end{itemize}
}
\item  \\
 in good shape \textit{
	\begin{itemize}
	\end{itemize}
}
\item  \\
 in the shape of \textit{
	\begin{itemize}
	\end{itemize}
}
\item  \\
 to lick into shape \textit{
	\begin{itemize}
	\end{itemize}
}
\item  \\
 out of shape \textit{
	\begin{itemize}
	\end{itemize}
}
\item  \\
 out of shape \textit{
	\begin{itemize}
	\end{itemize}
}
\item  \\
 comes in all shapes and sizes \textit{
	\begin{itemize}
	\end{itemize}
}
\item  \\
 take shape \textit{
	\begin{itemize}
	\end{itemize}
}
\end{enumerate}

\section*{suspicious}
{\large \color{blue}  }
\subsection*{Explain}
\begin{enumerate}
\item adjective \\
If you are \textbf{suspicious}  \textbf{of} someone or something, you do not trust them, and are careful when dealing with them.
 \textit{
	\begin{itemize}
	\item He was rightly suspicious of meeting me until I reassured him I was not writing about
him.
	\item He has his father's suspicious nature.
	\end{itemize}
}
\item adjective \\
If you are \textbf{suspicious}  \textbf{of} someone or something, you believe that they are probably involved in a crime or some dishonest activity.
 \textit{
	\begin{itemize}
	\item Two officers on patrol became suspicious of two men in a car.
	\item A woman kept prisoner in a basement was rescued after suspicious neighbours tipped
off police.
	\end{itemize}
}
\item adjective \\
If you describe someone or something as \textbf{suspicious} , you mean that there is some aspect of them which makes you think that they are involved in a crime or a dishonest activity.
 \textit{
	\begin{itemize}
	\item He reported that two suspicious-looking characters had approached Callendar.
	\item Nottingham police last night found what they described as a suspicious package.
	\end{itemize}
}
\end{enumerate}

\section*{situation}
{\large \color{blue}  situations  }
\subsection*{Explain}
\begin{enumerate}
\item countable noun \\
You use \textbf{situation} to refer  generally to what is happening in a particular place at a particular time, or to refer to what is happening to you.
 \textit{
	\begin{itemize}
	\item Army officers said the situation was under control.
	\item And now for a look at the travel situation in the rest of the country.
	\item She's in a hopeless situation.
	\item If you want to improve your situation you must adopt a positive mental attitude.
	\end{itemize}
}
\item countable noun \\
The \textbf{situation} of a building or town is the kind of surroundings that it has.
 \textit{
	\begin{itemize}
	\item The garden is in a beautiful situation.
	\end{itemize}
}
\item  \\
 situations vacant \textit{
	\begin{itemize}
	\end{itemize}
}
\end{enumerate}

\section*{typical}
{\large \color{blue}  }
\subsection*{Explain}
\begin{enumerate}
\item adjective \\
You use \textbf{typical} to describe someone or something that shows the most usual characteristics of a particular type of person or thing, and is therefore a good example of that type.
 \textit{
	\begin{itemize}
	\item Cheney is everyone's image of a typical cop: a big white guy, six foot, 220 pounds.
	\item Carole goes in for such typical schoolgirl pastimes as horse-riding and watching
old films.
	\item Horrigan was typical of the new-generation executive Sticht had brought into the
company.
	\end{itemize}
}
\item adjective \\
If a particular action or feature is \textbf{typical}  \textbf{of} someone or something, it shows their usual qualities or characteristics.
 \textit{
	\begin{itemize}
	\item This reluctance to move towards a democratic state is typical of totalitarian regimes.
	\item This is not typical of Chinese, but is a feature of the Thai language.
	\item With typical energy he found new journalistic outlets.
	\end{itemize}
}
\item adjective \\
If you say that something is \textbf{typical}  \textbf{of} a person, situation , or thing, you are criticizing them or complaining about them and saying that they are just as bad or disappointing as you expected them to be.
 \textit{
	\begin{itemize}
	\item She threw her hands into the air. 'That is just typical of you, isn't it?'
	\item 'Typical!' Hattie slammed down the receiver. 'Absolutely typical!'
	\end{itemize}
}
\end{enumerate}

\section*{skill}
{\large \color{blue}  skills  }
\subsection*{Explain}
\begin{enumerate}
\item countable noun \\
A \textbf{skill} is a type of work or activity which requires special training and knowledge .
 \textit{
	\begin{itemize}
	\item Most of us will know someone who is always learning new skills, or studying new fields.
	\end{itemize}
}
\item uncountable noun \\
\textbf{Skill} is the knowledge and ability that enables you to do something well .
 \textit{
	\begin{itemize}
	\item The cut of a diamond depends on the skill of its craftsman.
	\end{itemize}
}
\end{enumerate}

\section*{technician}
{\large \color{blue}  technicians  }
\subsection*{Explain}
\begin{enumerate}
\item countable noun \\
A \textbf{technician} is someone whose job involves skilled practical work with scientific equipment , for example in a laboratory.
 \textit{
	\begin{itemize}
	\item ...a laboratory technician.
	\end{itemize}
}
\item countable noun \\
A \textbf{technician} is someone who is very good at the detailed technical aspects of an activity.
 \textit{
	\begin{itemize}
	\item ...a versatile, veteran player, a superb technician.
	\end{itemize}
}
\end{enumerate}

\section*{predominant}
{\large \color{blue}  }
\subsection*{Explain}
\begin{enumerate}
\item adjective \\
If something is \textbf{predominant} , it is more important or noticeable than anything else in a set of people or things.
 \textit{
	\begin{itemize}
	\item Amanda's predominant emotion was that of confusion.
	\end{itemize}
}
\end{enumerate}

\section*{technique}
{\large \color{blue}  techniques  }
\subsection*{Explain}
\begin{enumerate}
\item countable noun \\
A \textbf{technique} is a particular method of doing an activity , usually a method that involves practical skills.
 \textit{
	\begin{itemize}
	\item ...tests performed using a new technique.
	\end{itemize}
}
\item uncountable noun \\
\textbf{Technique} is skill and ability in an artistic , sporting , or other practical activity that you develop through training and practice .
 \textit{
	\begin{itemize}
	\item He went off to the Amsterdam Academy to improve his technique.
	\end{itemize}
}
\end{enumerate}

\section*{skillful}
{\large \color{blue}  }
\subsection*{Explain}
\begin{enumerate}
\end{enumerate}

\section*{viewpoint}
{\large \color{blue}  viewpoints  }
\subsection*{Explain}
\begin{enumerate}
\item countable noun \\
Someone's \textbf{viewpoint} is the way that they think about things in general , or the way they think about a particular thing.
 \textit{
	\begin{itemize}
	\item The novel is shown from the girl's viewpoint.
	\item To include as many viewpoints as possible, the editor reserves the right to shorten
letters.
	\end{itemize}
}
\item countable noun \\
A \textbf{viewpoint} is a place from which you can get a good view of something.
 \textit{
	\begin{itemize}
	\item You have to know where to stand for a good viewpoint.
	\end{itemize}
}
\end{enumerate}

\section*{static}
{\large \color{blue}  }
\subsection*{Explain}
\begin{enumerate}
\item adjective \\
Something that is \textbf{static} does not move or change.
 \textit{
	\begin{itemize}
	\item The number of young people obtaining qualifications has remained static or decreased.
	\item Both your pictures are of static subjects.
	\end{itemize}
}
\item uncountable noun \\
\textbf{Static} or \textbf{static electricity} is electricity which can be caused by things rubbing against each other and which collects on things such as your body or metal objects.
 \textit{
	\begin{itemize}
	\end{itemize}
}
\item uncountable noun \\
If there is \textbf{static} on the radio or television, you hear a series of loud  noises which spoils the sound.
 \textit{
	\begin{itemize}
	\end{itemize}
}
\end{enumerate}

\section*{wagon}
{\large \color{blue}  wagons  }
\subsection*{Explain}
\begin{enumerate}
\item countable noun \\
A \textbf{wagon} is a strong vehicle with four wheels, usually pulled by horses or oxen and used for carrying heavy loads.
 \textit{
	\begin{itemize}
	\end{itemize}
}
\item countable noun \\
A \textbf{wagon} is a large container on wheels which is pulled by a train .
 \textit{
	\begin{itemize}
	\end{itemize}
}
\item  \\
 on the wagon \textit{
	\begin{itemize}
	\end{itemize}
}
\end{enumerate}

\section*{adjacent}
{\large \color{blue}  }
\subsection*{Explain}
\begin{enumerate}
\item adjective \\
If one thing is \textbf{adjacent}  \textbf{to} another, the two things are next to each other.
 \textit{
	\begin{itemize}
	\item He sat in an adjacent room and waited.
	\item The schools were adjacent but there were separate doors.
	\item ...offices adjacent to the museum.
	\end{itemize}
}
\end{enumerate}

\section*{ballet}
{\large \color{blue}  ballets  }
\subsection*{Explain}
\begin{enumerate}
\item uncountable noun \\
\textbf{Ballet} is a type of very skilled and artistic dancing with carefully planned movements.
 \textit{
	\begin{itemize}
	\item I trained as a ballet dancer.
	\item She is also keen on the ballet.
	\end{itemize}
}
\item countable noun \\
A \textbf{ballet} is an artistic work that is performed by ballet dancers.
 \textit{
	\begin{itemize}
	\item The performance will include the premiere of three new ballets.
	\end{itemize}
}
\end{enumerate}

\section*{aerial}
{\large \color{blue}  aerials  }
\subsection*{Explain}
\begin{enumerate}
\item adjective \\
You talk about \textbf{aerial}  attacks and \textbf{aerial}  photographs to indicate that people or things on the ground are attacked or photographed by people in aeroplanes .
 \textit{
	\begin{itemize}
	\item Weeks of aerial bombardment had destroyed factories and highways.
	\item He planned to take aerial photographs of the Baru volcano.
	\item The film begins with an aerial view of the Great Basin of Nevada.
	\end{itemize}
}
\item adjective \\
You can use \textbf{aerial} to describe things that exist or happen above the ground or in the air.
 \textit{
	\begin{itemize}
	\item The seagulls swirled in aerial combat over the barges.
	\end{itemize}
}
\item countable noun \\
An \textbf{aerial} is a device or a piece of wire that receives television or radio signals and is usually attached to a radio, television, car , or building.
 \textit{
	\begin{itemize}
	\item ...a saucer-shaped satellite television aerial.
	\item ...the radio aerials of taxis and cars.
	\end{itemize}
}
\end{enumerate}

\section*{bloom}
{\large \color{blue}  blooms  blooming  bloomed  }
\subsection*{Explain}
\begin{enumerate}
\item countable noun \\
A \textbf{bloom} is the flower on a plant.
 \textit{
	\begin{itemize}
	\item ...the sweet fragrance of the white blooms.
	\item Harry carefully picked the bloom.
	\end{itemize}
}
\item  \\
 in bloom \textit{
	\begin{itemize}
	\end{itemize}
}
\item verb \\
When a plant or tree \textbf{blooms} , it produces flowers. When a flower \textbf{blooms} , it opens.
 \textit{
	\begin{itemize}
	\item This plant blooms between May and June.
	\end{itemize}
}
\item verb \\
If someone or something \textbf{blooms} , they develop good, attractive , or successful qualities.
 \textit{
	\begin{itemize}
	\item Not many economies bloomed that year, least of all gold exporters like Australia.
	\item She bloomed into an utterly beautiful creature.
	\end{itemize}
}
\item uncountable noun \\
If something such as someone's skin has a \textbf{bloom} , it has a fresh and healthy appearance.
 \textit{
	\begin{itemize}
	\item The skin loses its youthful bloom.
	\end{itemize}
}
\end{enumerate}

\section*{any}
{\large \color{blue}  }
\subsection*{Explain}
\begin{enumerate}
\item determiner \\
You use \textbf{any} in statements with negative  meaning to indicate that no thing or person of a particular type exists , is present , or is involved in a situation .
 \textbf{Any} is also a quantifier .
 \textbf{Any} is also a pronoun.
 \textit{
	\begin{itemize}
	\item I never make any big decisions.
	\item I'm not making any promises.
	\item We are doing this all without any support from the hospital.
	\item Earlier reports were unable to confirm that there were any survivors.
	\item It is too early to say what effect, if any, there will be on the workforce.
	\item You don't know any of my friends.
	\item There was nothing you could do, nothing any of us could do.
	\item The children needed new school clothes and Kim couldn't afford any.
	\end{itemize}
}
\item determiner \\
You use \textbf{any} in questions and conditional  clauses to ask whether there is some of a particular thing or some of a particular group of people,
or to suggest that there might be.
 \textbf{Any} is also a quantifier.
 \textbf{Any} is also a pronoun.
 \textit{
	\begin{itemize}
	\item Do you speak any foreign languages?
	\item Have you got any cheese I can have with this bread?
	\item Introduce foods one at a time and notice if you feel uncomfortable with any of them.
	\item Have you ever used a homeopathic remedy for any of the following reasons?
	\item If any bright thoughts occur to you pass them straight to me. Have you got any?
	\item I'll keep an eye out for books and if I find any, I'll send them to you.
	\end{itemize}
}
\item determiner \\
You use \textbf{any} in positive statements when you are referring to someone or something of a particular kind that might exist, occur, or be involved
in a situation, when their exact  identity or nature is not important .
 \textbf{Any} is also a quantifier.
 \textbf{Any} is also a pronoun.
 \textit{
	\begin{itemize}
	\item Any actor will tell you that it is easier to perform than to be themselves.
	\item I'm prepared to take any advice.
	\item I would overcome any weakness, any despair, any fear.
	\item Nealy disappeared two days ago, several miles away from any of the fighting.
	\item It had been the biggest mistake any of them could remember.
	\item Clean the mussels and discard any that do not close.
	\item ...mangoes, bananas, pineapples, pears, and grapes as delicious as any you have ever
eaten.
	\end{itemize}
}
\item adverb \\
You can also use \textbf{any} to emphasize a comparative  adjective or adverb in a negative statement.
 \textit{
	\begin{itemize}
	\item I can't see things getting any easier for graduates.
	\item Anne's not getting any younger.
	\end{itemize}
}
\item  \\
 not just any \textit{
	\begin{itemize}
	\end{itemize}
}
\item  \\
 any more \textit{
	\begin{itemize}
	\end{itemize}
}
\end{enumerate}

\section*{blossom}
{\large \color{blue}  blossoms  blossoming  blossomed  }
\subsection*{Explain}
\begin{enumerate}
\item variable noun \\
\textbf{Blossom} is the flowers that appear on a tree before the fruit.
 \textit{
	\begin{itemize}
	\item The cherry blossom came out early in Washington this year.
	\item ...the blossoms of plants, shrubs and trees.
	\end{itemize}
}
\item verb \\
If someone or something \textbf{blossoms} , they develop good, attractive , or successful qualities.
 \textit{
	\begin{itemize}
	\item Why do some people take longer than others to blossom?
	\item What began as a local festival has blossomed into an international event.
	\item The pair have tried to keep their blossoming relationship under wraps.
	\end{itemize}
}
\item verb \\
When a tree \textbf{blossoms} , it produces blossom.
 \textit{
	\begin{itemize}
	\item Rain begins to fall and peach trees blossom.
	\end{itemize}
}
\end{enumerate}

\section*{available}
{\large \color{blue}  }
\subsection*{Explain}
\begin{enumerate}
\item adjective \\
If something you want or need is \textbf{available} , you can  find it or obtain it.
 \textit{
	\begin{itemize}
	\item The amount of money available to buy books has fallen by 17%.
	\item There are three small boats available for hire.
	\item According to the best available information, the facts are these.
	\end{itemize}
}
\item adjective \\
Someone who is \textbf{available} is not busy and is therefore free to talk to you or to do a particular  task .
 \textit{
	\begin{itemize}
	\item Mr Leach is on holiday and was not available for comment.
	\end{itemize}
}
\end{enumerate}

\section*{bud}
{\large \color{blue}  buds  budding  budded  }
\subsection*{Explain}
\begin{enumerate}
\item countable noun \\
A \textbf{bud} is a small pointed lump that appears on a tree or plant and develops into a leaf or flower.
 \textit{
	\begin{itemize}
	\item Rosanna's favourite time is early summer, just before the buds open.
	\end{itemize}
}
\item verb \\
When a tree or plant \textbf{is budding} , buds are appearing on it or are beginning to open.
 \textit{
	\begin{itemize}
	\item The leaves were budding on the trees below.
	\end{itemize}
}
\item vocative noun \\
Some men use \textbf{bud} as a way of addressing other men.
 \textit{
	\begin{itemize}
	\item You heard what the boss said, bud.
	\end{itemize}
}
\item  \\
 in/into bud \textit{
	\begin{itemize}
	\end{itemize}
}
\item  \\
 to nip something in the bud \textit{
	\begin{itemize}
	\end{itemize}
}
\end{enumerate}

\section*{beneficial}
{\large \color{blue}  }
\subsection*{Explain}
\begin{enumerate}
\item adjective \\
Something that is \textbf{beneficial}  helps people or improves their lives.
 \textit{
	\begin{itemize}
	\item ...vitamins which are beneficial to our health.
	\item Using computers has a beneficial effect on children's learning.
	\end{itemize}
}
\end{enumerate}

\section*{cemetery}
{\large \color{blue}  cemeteries  }
\subsection*{Explain}
\begin{enumerate}
\item countable noun \\
A \textbf{cemetery} is a place where dead people's bodies or their ashes are buried.
 \textit{
	\begin{itemize}
	\end{itemize}
}
\end{enumerate}

\section*{benign}
{\large \color{blue}  }
\subsection*{Explain}
\begin{enumerate}
\item adjective \\
You use \textbf{benign} to describe someone who is kind , gentle, and harmless .
 \textit{
	\begin{itemize}
	\item They are normally a more benign audience.
	\item Critics of the scheme take a less benign view.
	\end{itemize}
}
\item adjective \\
A \textbf{benign} substance or process does not have any harmful  effects .
 \textit{
	\begin{itemize}
	\item We're taking relatively benign medicines and we're turning them into poisons.
	\end{itemize}
}
\item adjective \\
A \textbf{benign} tumour will not cause death or serious  harm .
 \textit{
	\begin{itemize}
	\item It wasn't cancer, only a benign tumour.
	\end{itemize}
}
\item adjective \\
\textbf{Benign} conditions are pleasant or make it easy for something to happen .
 \textit{
	\begin{itemize}
	\item They enjoyed an especially benign climate.
	\item This plunge came in a time of relatively benign economic conditions.
	\end{itemize}
}
\item  \\
 benign neglect \textit{
	\begin{itemize}
	\end{itemize}
}
\end{enumerate}

\section*{contribution}
{\large \color{blue}  contributions  }
\subsection*{Explain}
\begin{enumerate}
\item countable noun \\
If you make a \textbf{contribution}  \textbf{to} something, you do something to help make it successful or to produce it.
 \textit{
	\begin{itemize}
	\item The study made important contributions to the field of corporate economics.
	\item He was awarded a prize for his contribution to world peace.
	\end{itemize}
}
\item countable noun \\
A \textbf{contribution} is a sum of money that you give in order to help pay for something.
 \textit{
	\begin{itemize}
	\item ...charitable contributions of a half million dollars or more.
	\end{itemize}
}
\item countable noun \\
A \textbf{contribution}  \textbf{to} a magazine , newspaper, or book is something that you write to be published in it.
 \textit{
	\begin{itemize}
	\end{itemize}
}
\end{enumerate}

\section*{bitter}
{\large \color{blue}  bitterest  bitters  }
\subsection*{Explain}
\begin{enumerate}
\item adjective \\
In a \textbf{bitter}  argument or conflict , people argue very angrily or fight very fiercely.
 \textit{
	\begin{itemize}
	\item ...the scene of bitter fighting during the Second World War.
	\item ...a bitter attack on the Government's failure to support manufacturing.
	\item On the eve of the poll, campaigning was bitter.
	\end{itemize}
}
\item adjective \\
If someone is \textbf{bitter} after a disappointing  experience or after being treated unfairly, they continue to feel  angry about it.
 \textit{
	\begin{itemize}
	\item She is said to be very bitter about the way she was sacked.
	\item His long life was marked by bitter personal and political memories.
	\end{itemize}
}
\item adjective \\
A \textbf{bitter} experience makes you feel very disappointed. You can also use \textbf{bitter} to emphasize  feelings of disappointment .
 \textit{
	\begin{itemize}
	\item I think the decision was a bitter blow from which he never quite recovered.
	\item A great deal of bitter experience had taught him how to lose gracefully.
	\item The statement was greeted with bitter disappointment by many of the other delegates.
	\end{itemize}
}
\item adjective \\
\textbf{Bitter}  weather , or a \textbf{bitter}  wind , is extremely cold.
 \textit{
	\begin{itemize}
	\item Outside, a bitter east wind was accompanied by flurries of snow.
	\item ...after spending a night in the bitter cold.
	\end{itemize}
}
\item adjective \\
A \textbf{bitter} taste is sharp , not sweet , and often slightly unpleasant.
 \textit{
	\begin{itemize}
	\item The leaves taste rather bitter.
	\item ...as the wine ages, losing its bitter harshness, and becoming softer and smoother.
	\end{itemize}
}
\item variable noun \\
\textbf{Bitter} is a kind of beer that is light brown in colour .
 \textit{
	\begin{itemize}
	\item ...a pint of bitter.
	\end{itemize}
}
\item  \\
 to the bitter end \textit{
	\begin{itemize}
	\end{itemize}
}
\end{enumerate}

\section*{deficit}
{\large \color{blue}  deficits  }
\subsection*{Explain}
\begin{enumerate}
\item countable noun \\
A \textbf{deficit} is the amount by which something is less than what is required or expected, especially the amount by which the total  money  received is less than the total money spent .
 \textit{
	\begin{itemize}
	\item They're ready to cut the federal budget deficit for the next fiscal year.
	\item ...a deficit of 3.275 billion francs.
	\end{itemize}
}
\end{enumerate}

\section*{calm}
{\large \color{blue}  calmer  calmest  calms  calming  calmed  }
\subsection*{Explain}
\begin{enumerate}
\item adjective \\
A \textbf{calm} person does not show or feel any worry , anger , or excitement .
 \textbf{Calm} is also a noun .
 \textit{
	\begin{itemize}
	\item She is usually a calm and diplomatic woman.
	\item Try to keep calm and just tell me what happened.
	\item She sighed, then continued in a soft, calm voice.
	\item Diane felt very calm and unafraid as she saw him off the next morning.
	\item He felt a sudden sense of calm, of contentment.
	\end{itemize}
}
\item verb \\
If you \textbf{calm} someone, you do something to make them feel less angry , worried, or excited.
 \textit{
	\begin{itemize}
	\item The ruling party's veterans know how to calm their critics.
	\item She was breathing quickly and tried to calm herself.
	\item Some people say smoking calms your nerves.
	\end{itemize}
}
\item uncountable noun \\
\textbf{Calm} is used to refer to a quiet , still, or peaceful  atmosphere in a place.
 \textit{
	\begin{itemize}
	\item The house projects an atmosphere of calm and order.
	\item ...the rural calm of Grand Rapids, Michigan.
	\end{itemize}
}
\item adjective \\
If someone says that a place is \textbf{calm} , they mean that it is free from fighting or public  disorder , when trouble has recently occurred there or had been expected .
 \textbf{Calm} is also a noun.
 \textit{
	\begin{itemize}
	\item The city of Sarajevo appears relatively calm today.
	\item Community and church leaders have appealed for calm and no retaliation.
	\item An uneasy calm is reported to be prevailing in the area.
	\end{itemize}
}
\item verb \\
To \textbf{calm} a situation means to reduce the amount of trouble, violence , or panic there is.
 \textit{
	\begin{itemize}
	\item Mr Beazer tried to calm the protests.
	\end{itemize}
}
\item adjective \\
If the sea or a lake is \textbf{calm} , the water is not moving very much and there are no big  waves .
 \textit{
	\begin{itemize}
	\item ...as we slid into the calm waters of Cowes Harbour.
	\end{itemize}
}
\item adjective \\
\textbf{Calm}  weather is pleasant weather with little or no wind.
 \textit{
	\begin{itemize}
	\item Tuesday was a fine, clear and calm day.
	\end{itemize}
}
\item countable noun \\
In sailing , a flat  \textbf{calm} or a dead  \textbf{calm} is a condition of the sea or the weather in which there is very little wind or movement of the water.
 \textit{
	\begin{itemize}
	\item ...during flat calms when the water is crystal clear.
	\item We had the whole gamut of wind from a dead calm to a force 10 gale.
	\end{itemize}
}
\item verb \\
When the sea \textbf{calms} , it becomes still because the wind stops  blowing strongly. When the wind \textbf{calms} , it stops blowing strongly.
 \textit{
	\begin{itemize}
	\item Dawn came, the sea calmed but the cold was as bitter as ever.
	\end{itemize}
}
\item verb \\
To \textbf{calm} a pain or an itch means to reduce it or get  rid of it.
 \textit{
	\begin{itemize}
	\item ...more traditional methods of soothing the skin and calming the itch.
	\end{itemize}
}
\item  \\
 the calm before the storm \textit{
	\begin{itemize}
	\end{itemize}
}
\end{enumerate}

\section*{discount}
{\large \color{blue}  discounts  discounting  discounted  }
\subsection*{Explain}
\begin{enumerate}
\item countable noun \\
A \textbf{discount} is a reduction in the usual price of something.
 \textit{
	\begin{itemize}
	\item They are often available at a discount.
	\item Full-time staff get a 20 per cent discount.
	\item ...a discount store specializing in household goods.
	\item ...discontinued ranges of tiles at discount prices.
	\end{itemize}
}
\item verb \\
If a shop or company \textbf{discounts} an amount or percentage from something that they are selling, they take the amount
or percentage off the usual price.
 \textit{
	\begin{itemize}
	\item This has forced airlines to discount fares heavily in order to spur demand.
	\item Tour prices are being discounted as much as 33%.
	\end{itemize}
}
\item verb \\
If you \textbf{discount} an idea , fact , or theory, you consider that it is not true , not important , or not relevant .
 \textit{
	\begin{itemize}
	\item However, traders tended to discount the rumor.
	\item This theory has now been discounted.
	\end{itemize}
}
\end{enumerate}

\section*{chronic}
{\large \color{blue}  }
\subsection*{Explain}
\begin{enumerate}
\item adjective \\
A \textbf{chronic}  illness or disability  lasts for a very long time. Compare  acute .
 \textit{
	\begin{itemize}
	\item ...chronic back pain.
	\end{itemize}
}
\item adjective \\
You can describe someone's bad habits or behaviour as \textbf{chronic} when they have behaved like that for a long time and do not seem to be able to stop themselves.
 \textit{
	\begin{itemize}
	\item Anyone who does not believe that smoking is an addiction has never been a chronic
smoker.
	\item ...a chronic worrier.
	\end{itemize}
}
\item adjective \\
A \textbf{chronic}  situation or problem is very severe and unpleasant .
 \textit{
	\begin{itemize}
	\item One cause of the artist's suicide seems to have been chronic poverty.
	\item There is a chronic shortage of patrol cars in this police district.
	\end{itemize}
}
\end{enumerate}

\section*{equator}
{\large \color{blue}  }
\subsection*{Explain}
\begin{enumerate}
\item singular noun \\
\textbf{The equator} is an imaginary  line around the middle of the Earth at an equal distance from the North Pole and the South Pole.
 \textit{
	\begin{itemize}
	\end{itemize}
}
\end{enumerate}

\section*{comparable}
{\large \color{blue}  }
\subsection*{Explain}
\begin{enumerate}
\item adjective \\
Something that is \textbf{comparable} to something else is roughly similar, for example in amount or importance .
 \textit{
	\begin{itemize}
	\item ...paying the same wages to men and women for work of comparable value.
	\item Farmers were meant to get an income comparable to that of townspeople.
	\item The risk it poses is comparable with smoking just one cigarette every year.
	\end{itemize}
}
\item adjective \\
If two or more things are \textbf{comparable} , they are of the same kind or are in the same situation , and so they can reasonably be compared.
 \textit{
	\begin{itemize}
	\item In other comparable countries real wages increased much more rapidly.
	\item By contrast, the comparable figure for the Netherlands is 16 per cent.
	\item Published rates are not always directly comparable.
	\end{itemize}
}
\end{enumerate}

\section*{expenditure}
{\large \color{blue}  expenditures  }
\subsection*{Explain}
\begin{enumerate}
\item variable noun \\
\textbf{Expenditure} is the spending of money on something, or the money that is spent on something.
 \textit{
	\begin{itemize}
	\item Policies of tax reduction must lead to reduced public expenditure.
	\item They should cut their expenditure on defence.
	\item An expenditure for clothing will qualify as a trade or business expense.
	\end{itemize}
}
\item uncountable noun \\
\textbf{Expenditure of} something such as time or energy is the using of that thing for a particular purpose.
 \textit{
	\begin{itemize}
	\item The financial rewards justified the expenditure of effort.
	\end{itemize}
}
\end{enumerate}

\section*{contrary}
{\large \color{blue}  }
\subsection*{Explain}
\begin{enumerate}
\item adjective \\
Ideas , attitudes , or reactions that are \textbf{contrary}  \textbf{to} each other are completely different from each other.
 \textit{
	\begin{itemize}
	\item This view is contrary to the aims of critical social research for a number of reasons.
	\item Several of those present, including Weinberger, had contrary information.
	\end{itemize}
}
\item  \\
 contrary to \textit{
	\begin{itemize}
	\end{itemize}
}
\item  \\
 on the contrary \textit{
	\begin{itemize}
	\end{itemize}
}
\item  \\
 on the contrary \textit{
	\begin{itemize}
	\end{itemize}
}
\item  \\
 quite the contrary \textit{
	\begin{itemize}
	\end{itemize}
}
\item  \\
 to the contrary \textit{
	\begin{itemize}
	\end{itemize}
}
\end{enumerate}

\section*{expense}
{\large \color{blue}  expenses  }
\subsection*{Explain}
\begin{enumerate}
\item variable noun \\
\textbf{Expense} is the money that something costs you or that you need to spend in order to do something.
 \textit{
	\begin{itemize}
	\item Most of the marble had been imported at vast expense from Italy.
	\item Taking holidays with your dog can often involve extra expense.
	\item It was not a fortune but would help to cover household expenses.
	\end{itemize}
}
\item plural noun \\
\textbf{Expenses} are amounts of money that you spend while doing something in the course of your work, which will be paid back to you afterwards .
 \textit{
	\begin{itemize}
	\item As a politician, her salary and expenses were paid by the taxpayer.
	\item Can you claim this back on expenses?
	\end{itemize}
}
\item  \\
 at someone's expense \textit{
	\begin{itemize}
	\end{itemize}
}
\item  \\
 at someone's expense \textit{
	\begin{itemize}
	\end{itemize}
}
\item  \\
 at the expense of \textit{
	\begin{itemize}
	\end{itemize}
}
\item  \\
 at the expense of \textit{
	\begin{itemize}
	\end{itemize}
}
\item  \\
 go to ... expense \textit{
	\begin{itemize}
	\end{itemize}
}
\end{enumerate}

\section*{convenient}
{\large \color{blue}  }
\subsection*{Explain}
\begin{enumerate}
\item adjective \\
If a way of doing something is \textbf{convenient} , it is easy, or very useful or suitable for a particular purpose.
 \textit{
	\begin{itemize}
	\item ...a flexible and convenient way of paying for business expenses.
	\item The family thought it was more convenient to eat in the kitchen.
	\end{itemize}
}
\item adjective \\
If you describe a place as \textbf{convenient} , you are pleased because it is near to where you are, or because you can reach another place from there quickly and easily.
 \textit{
	\begin{itemize}
	\item The town is well placed for easy access to London and convenient for Heathrow Airport.
	\item Martin drove along until he found a convenient parking place.
	\end{itemize}
}
\item adjective \\
A \textbf{convenient} time to do something, for example to meet someone, is a time when you are free to do it or would like to do it.
 \textit{
	\begin{itemize}
	\item She will try to arrange a mutually convenient time and place for an interview.
	\item Would this evening be convenient for you?
	\end{itemize}
}
\item adjective \\
If you describe someone's attitudes or actions as \textbf{convenient} , you think they are only adopting those attitudes or performing those actions in order to avoid something difficult or unpleasant .
 \textit{
	\begin{itemize}
	\item We cannot make this minority a convenient excuse to turn our backs.
	\item ...a convenient scapegoat.
	\item It does seem a bit convenient, doesn't it?
	\end{itemize}
}
\end{enumerate}

\section*{flower}
{\large \color{blue}  flowers  flowering  flowered  }
\subsection*{Explain}
\begin{enumerate}
\item countable noun \\
A \textbf{flower} is the part of a plant which is often brightly coloured, grows at the end of a stem , and only survives for a short time.
 \textit{
	\begin{itemize}
	\item Each individual flower is tiny.
	\item ...large, purplish-blue flowers.
	\end{itemize}
}
\item countable noun \\
A \textbf{flower} is a stem of a plant that has one or more flowers on it and has been picked , usually with others, for example to give as a present or to put in a vase .
 \textit{
	\begin{itemize}
	\item ...a bunch of flowers sent by a new admirer.
	\end{itemize}
}
\item countable noun \\
\textbf{Flowers} are small plants that are grown for their flowers as opposed to trees, shrubs , and vegetables .
 \textit{
	\begin{itemize}
	\item ...a lawned area surrounded by plants and flowers.
	\item The flower garden will be ablaze with colour every day.
	\end{itemize}
}
\item verb \\
When a plant or tree \textbf{flowers} , its flowers appear and open.
 \textit{
	\begin{itemize}
	\item Several of these rhododendrons will flower this year for the first time.
	\end{itemize}
}
\item verb \\
When something \textbf{flowers} , for example a political movement or a relationship , it gets stronger and more successful .
 \textit{
	\begin{itemize}
	\item Their relationship flowered.
	\end{itemize}
}
\item singular noun \\
A person or thing that is described as \textbf{the flower} of something is the best part or example of it.
 \textit{
	\begin{itemize}
	\item Those killed have been described as the flower of Polish manhood.
	\end{itemize}
}
\item  \\
 in flower \textit{
	\begin{itemize}
	\end{itemize}
}
\end{enumerate}

\section*{discreet}
{\large \color{blue}  }
\subsection*{Explain}
\begin{enumerate}
\item adjective \\
If you are \textbf{discreet} , you are polite and careful in what you do or say , because you want to avoid embarrassing or offending someone.
 \textit{
	\begin{itemize}
	\item They were gossipy and not always discreet.
	\item He followed at a discreet distance.
	\end{itemize}
}
\item adjective \\
If you are \textbf{discreet}  \textbf{about} something you are doing, you do not tell other people about it, in order to avoid being embarrassed or to gain an advantage .
 \textit{
	\begin{itemize}
	\item We were very discreet about the romance.
	\item She's making a few discreet inquiries with her mother's friends.
	\end{itemize}
}
\item adjective \\
If you describe something as \textbf{discreet} , you approve of it because it is small in size or degree , or not easily  noticed .
 \textit{
	\begin{itemize}
	\item She wore discreet jewellery.
	\end{itemize}
}
\end{enumerate}

\section*{grave}
{\large \color{blue}  graves  graver  gravest  }
\subsection*{Explain}
\begin{enumerate}
\item countable noun \\
A \textbf{grave} is a place where a dead person is buried .
 \textit{
	\begin{itemize}
	\item They used to visit her grave twice a year.
	\end{itemize}
}
\item countable noun \\
You can refer to someone's death as their \textbf{grave} or to death as \textbf{the}  \textbf{grave} .
 \textit{
	\begin{itemize}
	\item ...drinking yourself to an early grave.
	\item Most men would rather go to the grave than own up to feelings of dependency.
	\end{itemize}
}
\item adjective \\
A \textbf{grave} event or situation is very serious, important, and worrying .
 \textit{
	\begin{itemize}
	\item He said that the situation in his country is very grave.
	\item I have grave doubts that the documents tell the whole story.
	\end{itemize}
}
\item adjective \\
A \textbf{grave} person is quiet and serious in their appearance or behaviour .
 \textit{
	\begin{itemize}
	\item William was up on the roof for some time and when he came down he looked grave.
	\item Anxiously, she examined his unusually grave face.
	\end{itemize}
}
\item adjective \\
In some languages such as French, a \textbf{grave} accent is a symbol that is placed over a vowel in a word to show how the vowel is pronounced . For example , the word 'mè re ' has a grave accent over the first 'e'.
 \textit{
	\begin{itemize}
	\end{itemize}
}
\item  \\
 dig one's own grave \textit{
	\begin{itemize}
	\end{itemize}
}
\item  \\
 turn in their grave \textit{
	\begin{itemize}
	\end{itemize}
}
\end{enumerate}

\section*{due}
{\large \color{blue}  dues  }
\subsection*{Explain}
\begin{enumerate}
\item phrase \\
If an event is \textbf{due to} something, it happens or exists as a direct result of that thing.
 \textit{
	\begin{itemize}
	\item The country's economic problems are largely due to the weakness of the recovery.
	\item If the trip is a success, a lot of this will be due to Mr Green's efforts.
	\end{itemize}
}
\item phrase \\
You can say  \textbf{due to} to introduce the reason for something happening . Some speakers of English believe that it is not correct to use \textbf{due to} in this way.
 \textit{
	\begin{itemize}
	\item Due to the large volume of letters he receives Dave regrets he is unable to answer
queries personally.
	\item Jobs could be lost in the defence industry due to political changes sweeping Europe.
	\end{itemize}
}
\item adjective \\
If something is \textbf{due} at a particular time, it is expected to happen, be done, or arrive at that time.
 \textit{
	\begin{itemize}
	\item The results are due at the end of the month.
	\item The first price increases are due to come into force in July.
	\item Her first novel is due out in May.
	\item Mr Carter is due in London on Monday.
	\item ...customers who paid later than twenty days after the due date.
	\end{itemize}
}
\item adjective \\
\textbf{Due}  attention or consideration is the proper, reasonable , or deserved amount of it under the circumstances .
 \textit{
	\begin{itemize}
	\item After due consideration it was decided to send him away to live with foster parents.
	\item I hope people will use the footpaths and treat them with due attention.
	\end{itemize}
}
\item adjective \\
Something that is \textbf{due} , or that is \textbf{due}  \textbf{to} someone, is owed to them, either as a debt or because they have a right to it.
 \textbf{Due} is also a preposition .
 \textit{
	\begin{itemize}
	\item I was sent a cheque for £1,525 and advised that no further pension was due.
	\item I've got some leave due to me and I was going to Tasmania for a fortnight.
	\item He had not taken a summer holiday that year but had accumulated the leave due him.
	\end{itemize}
}
\item adjective \\
If someone is \textbf{due for} something, that thing is planned to happen or be given to them now, or very soon , often after they have been waiting for it for a long time.
 \textbf{Due} is also a preposition.
 \textit{
	\begin{itemize}
	\item She was due for a follow-up appointment.
	\item He is not due for release until 2020.
	\item I reckon I'm due one of my travels.
	\end{itemize}
}
\item plural noun \\
\textbf{Dues} are sums of money that you give regularly to an organization that you belong to, for example a social club or trade union , in order to pay for being a member.
 \textit{
	\begin{itemize}
	\item Only 18 of the U.N.'s 180 members had paid their dues by the January deadline.
	\end{itemize}
}
\item adverb \\
\textbf{Due} is used before the words 'north', 'south', 'east', or 'west' to indicate that something
is in exactly the direction mentioned .
 \textit{
	\begin{itemize}
	\item They headed due north.
	\item The Thames flows due south from Oxford, through the market town of Abingdon.
	\item ...a mining town 40 miles due east of Los Angeles.
	\end{itemize}
}
\item  \\
 in due course \textit{
	\begin{itemize}
	\end{itemize}
}
\item  \\
 to give sb their due \textit{
	\begin{itemize}
	\end{itemize}
}
\item  \\
 with due respect \textit{
	\begin{itemize}
	\end{itemize}
}
\end{enumerate}

\section*{investment}
{\large \color{blue}  investments  }
\subsection*{Explain}
\begin{enumerate}
\item uncountable noun \\
\textbf{Investment} is the activity of investing money.
 \textit{
	\begin{itemize}
	\item He said the government must introduce tax incentives to encourage investment.
	\item One of the most important changes concerns the investment of pension contributions.
	\item ...investment bankers.
	\end{itemize}
}
\item variable noun \\
An \textbf{investment} is an amount of money that you invest, or the thing that you invest it in.
 \textit{
	\begin{itemize}
	\item ...an investment of twenty-eight million pounds.
	\item You'll be able to earn an average rate of return of 8% on your investments.
	\item ...people's desire to buy a house as an investment.
	\item Total foreign investment in America still constitutes only about 5% of U.S. assets.
	\end{itemize}
}
\item countable noun \\
If you describe something you buy as an \textbf{investment} , you mean that it will be useful , especially because it will help you to do a task more cheaply or efficiently.
 \textit{
	\begin{itemize}
	\item When selecting boots, fine, quality leather will be a wise investment.
	\item Theatre membership can be a good investment.
	\end{itemize}
}
\item uncountable noun \\
\textbf{Investment} of time or effort is the spending of time or effort on something in order to make it a success .
 \textit{
	\begin{itemize}
	\item I worry about this big investment of time and effort.
	\end{itemize}
}
\end{enumerate}

\section*{equivalent}
{\large \color{blue}  equivalents  }
\subsection*{Explain}
\begin{enumerate}
\item singular noun \\
If one amount or value is \textbf{the}  \textbf{equivalent}  \textbf{of} another, they are the same.
 \textbf{Equivalent} is also an adjective .
 \textit{
	\begin{itemize}
	\item The equivalent of two tablespoons of polyunsaturated oils is ample each day.
	\item Even the cheapest car costs the equivalent of 70 years' salary for a government worker.
	\item A unit is equivalent to a glass of wine or a single measure of spirits.
	\item Calls for equivalent wage increases are bound to be heard.
	\end{itemize}
}
\item countable noun \\
The \textbf{equivalent} of someone or something is a person or thing that has the same function in a different place, time, or system.
 \textbf{Equivalent} is also an adjective.
 \textit{
	\begin{itemize}
	\item ...the civil administrator of the West Bank and his equivalent in Gaza.
	\item ...the Red Cross emblem, and its equivalent in Muslim countries, the Red Crescent.
	\item ...a decrease of 10% in property investment compared with the equivalent period last
year.
	\end{itemize}
}
\item singular noun \\
You can use \textbf{equivalent} to emphasize the great or severe effect of something.
 \textit{
	\begin{itemize}
	\item His party has just suffered the equivalent of a near-fatal heart attack.
	\end{itemize}
}
\end{enumerate}

\section*{lace}
{\large \color{blue}  laces  lacing  laced  }
\subsection*{Explain}
\begin{enumerate}
\item uncountable noun \\
\textbf{Lace} is a very delicate cloth which is made with a lot of holes in it. It is made by twisting together very fine threads of cotton to form patterns.
 \textit{
	\begin{itemize}
	\item She finally found the perfect gown, a beautiful creation trimmed with lace.
	\item ...a plain white lace bedspread.
	\end{itemize}
}
\item countable noun \\
\textbf{Laces} are thin pieces of material that are put through special holes in some types of clothing, especially shoes. The laces are tied together in order to tighten the clothing.
 \textit{
	\begin{itemize}
	\item Barry was sitting on the bed, tying the laces of an old pair of running shoes.
	\end{itemize}
}
\item verb \\
If you \textbf{lace} something such as a pair of shoes, you tighten the shoes by pulling the laces through the holes, and usually tying them together.
 \textbf{Lace up} means the same as lace .
 \textit{
	\begin{itemize}
	\item I have a good pair of skates, but no matter how tightly I lace them, my ankles wobble.
	\item He sat on the steps, and laced up his boots.
	\item Nancy was lacing her shoe up when the doorbell rang.
	\end{itemize}
}
\item verb \\
To \textbf{lace} food or drink with a substance such as alcohol or a drug means to put a small amount
of the substance into the food or drink.
 \textit{
	\begin{itemize}
	\item She laced his food with sleeping pills.
	\end{itemize}
}
\item verb \\
If you \textbf{lace} your speech or writing with words of a particular kind , you include a lot of those words in what you say or write.
 \textit{
	\begin{itemize}
	\item Fred liked to lace his conversation with military terms.
	\item ...a speech laced with wry humour.
	\end{itemize}
}
\item verb \\
If you \textbf{lace} your fingers together, you put the palms of your hands together and fold your fingers over, fitting the fingers of one hand between the fingers of the other.
 \textit{
	\begin{itemize}
	\item He took to lacing his fingers together in an attempt to keep his hands still.
	\end{itemize}
}
\end{enumerate}

\section*{far}
{\large \color{blue}  }
\subsection*{Explain}
\begin{enumerate}
\item adverb \\
If one place, thing, or person is \textbf{far}  away from another, there is a great distance between them.
 \textit{
	\begin{itemize}
	\item I know a nice little Italian restaurant not far from here.
	\item They came from as far away as Florida.
	\item Both of my sisters moved even farther away from home.
	\item They lay in the cliff top grass with the sea stretching out far below.
	\item Is it far?
	\end{itemize}
}
\item adverb \\
If you ask  \textbf{how}  \textbf{far} a place is, you are asking what distance it is from you or from another place. If
you ask \textbf{how}  \textbf{far} someone went , you are asking what distance they travelled , or what place they reached .
 \textit{
	\begin{itemize}
	\item How far is Pawtucket from Providence?
	\item How far is it to Malcy?
	\item How far can you throw?
	\item You can only judge how high something is when you know how far away it is.
	\item She followed the tracks as far as the road.
	\end{itemize}
}
\item adjective \\
When there are two things of the same kind in a place, \textbf{the}  \textbf{far} one is the one that is a greater distance from you.
 \textit{
	\begin{itemize}
	\item He had wandered to the far end of the room.
	\item A narrow steep path leads down into a valley and up the far side.
	\end{itemize}
}
\item adjective \\
You can use \textbf{far} to refer to the part of an area or object that is the greatest distance from the centre in a particular  direction . For example , \textbf{the}  \textbf{far}  north  \textbf{of} a country is the part of it that is the greatest distance to the north.
 \textit{
	\begin{itemize}
	\item I've spent a lot of time walking around Britain from the far north of Scotland down
to Cornwall.
	\item I wrote the date at the far left of the blackboard.
	\end{itemize}
}
\item adverb \\
A time or event that is \textbf{far} away in the future or the past is a long time from the present or from a particular point in time.
 \textit{
	\begin{itemize}
	\item ...hidden conflicts whose roots lie far back in time.
	\item I can't see any farther than the next six months.
	\item The first day of term, which seemed so far away at the start of the summer holidays,
is looming.
	\end{itemize}
}
\item adverb \\
You can use \textbf{far} to talk about the extent or degree to which something happens or is true .
 \textit{
	\begin{itemize}
	\item How far did the film tell the truth about the inventor?
	\item But it is not clear how far they could help with the work on a power plant.
	\end{itemize}
}
\item adverb \\
You can talk about how \textbf{far} someone or something gets to describe the progress that they make.
 \textit{
	\begin{itemize}
	\item Discussions never progressed very far.
	\item Think of how far we have come in a little time.
	\item I don't think Mr Cavanagh would get far with that trick.
	\end{itemize}
}
\item adverb \\
You can talk about how \textbf{far} a person or action  goes to describe the degree to which someone's behaviour or actions are extreme .
 \textit{
	\begin{itemize}
	\item It's still not clear how far the government will go to implement its own plans.
	\item Competition can be healthy, but if it is pushed too far it can result in bullying.
	\item This time he's gone too far.
	\end{itemize}
}
\item graded adverb \\
You can use \textbf{far} in expressions  like ' \textbf{I wouldn't go that far} ' and ' \textbf{I would go so far} ' to indicate to what extent you agree with something.
 \textit{
	\begin{itemize}
	\item 'Does it sound like music?'—'I wouldn't go that far.'.
	\end{itemize}
}
\item adverb \\
You can use \textbf{far} to mean 'very much' when you are comparing two things and emphasizing the difference between them. For example, you can say that something is \textbf{far better} or \textbf{far worse} than something else to indicate that it is very much better or worse. You can also say that something is, for example, \textbf{far too big} to indicate that it is very much too big.
 \textit{
	\begin{itemize}
	\item Women who eat plenty of fresh vegetables are far less likely to suffer anxiety or
depression.
	\item The police say the response has been far better than expected.
	\item These trials are simply taking far too long.
	\item It now has debts reported to be far in excess of one thousand million pounds.
	\end{itemize}
}
\item adjective \\
You can describe people with extreme left-wing or right-wing  political  views as the \textbf{far}  left or the \textbf{far}  right .
 \textit{
	\begin{itemize}
	\item Anti-racist campaigners are urging the Government to ban all far-Right groups.
	\end{itemize}
}
\item  \\
 as far as I know \textit{
	\begin{itemize}
	\end{itemize}
}
\item  \\
 far and away \textit{
	\begin{itemize}
	\end{itemize}
}
\item  \\
 by far \textit{
	\begin{itemize}
	\end{itemize}
}
\item  \\
 far from \textit{
	\begin{itemize}
	\end{itemize}
}
\item  \\
 far from it \textit{
	\begin{itemize}
	\end{itemize}
}
\item  \\
 far be it from me to do sth \textit{
	\begin{itemize}
	\end{itemize}
}
\item  \\
 as far as it goes \textit{
	\begin{itemize}
	\end{itemize}
}
\item  \\
 sb will go far \textit{
	\begin{itemize}
	\end{itemize}
}
\item  \\
 far gone \textit{
	\begin{itemize}
	\end{itemize}
}
\item  \\
 not far wrong \textit{
	\begin{itemize}
	\end{itemize}
}
\item  \\
 as far as I can see \textit{
	\begin{itemize}
	\end{itemize}
}
\item  \\
 so far \textit{
	\begin{itemize}
	\end{itemize}
}
\item  \\
 so far \textit{
	\begin{itemize}
	\end{itemize}
}
\item  \\
 so far so good \textit{
	\begin{itemize}
	\end{itemize}
}
\item  \\
 thus far \textit{
	\begin{itemize}
	\end{itemize}
}
\item  \\
 far and wide \textit{
	\begin{itemize}
	\end{itemize}
}
\item  \\
 sb will/can not go far wrong \textit{
	\begin{itemize}
	\end{itemize}
}
\end{enumerate}

\section*{lever}
{\large \color{blue}  levers  levering  levered  }
\subsection*{Explain}
\begin{enumerate}
\item countable noun \\
A \textbf{lever} is a handle or bar that is attached to a piece of machinery and which you push or pull in order to operate the machinery.
 \textit{
	\begin{itemize}
	\item Push the tiny lever on the lock.
	\item The taps have a lever to control the mix of hot and cold water.
	\end{itemize}
}
\item countable noun \\
A \textbf{lever} is a long bar, one end of which is placed under a heavy object so that when you press down on the other end you can move the object.
 \textit{
	\begin{itemize}
	\end{itemize}
}
\item verb \\
If you \textbf{lever} something in a particular  direction , you move it there, especially by using a lot of effort .
 \textit{
	\begin{itemize}
	\item Neighbours eventually levered open the door with a crowbar.
	\item Insert the fork about 6in. from the root and simultaneously lever it backwards.
	\item Alex levered himself up from the sofa.
	\end{itemize}
}
\item countable noun \\
A \textbf{lever} is an idea or action that you can use to make people do what you want them to do, rather than what they want to do.
 \textit{
	\begin{itemize}
	\item He may use money as a lever to control and manipulate her.
	\end{itemize}
}
\end{enumerate}

\section*{fit}
{\large \color{blue}  fits  fitting  fitted  }
\subsection*{Explain}
\begin{enumerate}
\item verb \\
If something \textbf{fits} , it is the right size and shape to go onto a person's body or onto a particular object.
 \textit{
	\begin{itemize}
	\item The sash, kimono, and other garments were made to fit a child.
	\item She has to go to the men's department to find trousers that fit at the waist.
	\item Line a tin with lightly-greased greaseproof paper, making sure the corners fit well.
	\end{itemize}
}
\item singular noun \\
If something is a good \textbf{fit} , it fits well .
 \textit{
	\begin{itemize}
	\item Eventually he was happy that the sills and doors were a reasonably good fit.
	\end{itemize}
}
\item verb \\
If you \textbf{are fitted for} a particular piece of clothing, you try it on so that the person who is making it
can see where it needs to be altered .
 \textit{
	\begin{itemize}
	\item She was being fitted for her wedding dress.
	\end{itemize}
}
\item verb \\
If something \textbf{fits}  somewhere , it can be put there or is designed to be put there.
 \textit{
	\begin{itemize}
	\item ...a computer which is small enough to fit into your pocket.
	\item He folded his long legs to fit under the table.
	\item The crowd was too large to fit inside the hall.
	\item ...filters are available that fit over the lens of suitable cameras.
	\end{itemize}
}
\item verb \\
If you \textbf{fit} something into a particular space or place, you put it there.
 \textit{
	\begin{itemize}
	\item She fitted her key in the lock.
	\item Who could cut the millions of stone blocks and fit them together?
	\item When the crown has been made you go back and the dentist will fit it into place.
	\end{itemize}
}
\item verb \\
If you \textbf{fit} something somewhere, you attach it there, or put it there carefully and securely.
 \textit{
	\begin{itemize}
	\item Fit hinge bolts to give extra support to the door lock.
	\item Peter had built the overhead ladders, and the next day he fitted them to the wall.
	\item Home spas or mini whirlpools massage and relax, and can be fitted into the bath.
	\end{itemize}
}
\item verb \\
If something \textbf{fits} something else or \textbf{fits} into it, it goes together well with that thing or is able to be part of it.
 \textit{
	\begin{itemize}
	\item Her daughter doesn't fit the current feminine ideal.
	\item Fostering is a full-time job and you should consider how it will fit into your career.
	\item There's something about the way he talks of her that doesn't fit.
	\end{itemize}
}
\item verb \\
You can say that something \textbf{fits} a particular person or thing when it is appropriate or suitable for them or it.
 \textit{
	\begin{itemize}
	\item The punishment must always fit the crime.
	\end{itemize}
}
\item adjective \\
If something is \textbf{fit} for a particular purpose, it is suitable for that purpose.
 \textit{
	\begin{itemize}
	\item Of the seven bicycles we had, only two were fit for the road.
	\item ...safety measures intended to reassure consumers that the meat is fit to eat.
	\item Follow our guide to making your home a fit place to work, rest and play.
	\end{itemize}
}
\item adjective \\
If someone is \textbf{fit} to do something, they have the appropriate qualities or skills that will allow them
to do it.
 \textit{
	\begin{itemize}
	\item You're not fit to be a mother!
	\item In a word, this government isn't fit to rule.
	\item He was not a fit companion for their skipper that particular morning.
	\end{itemize}
}
\item verb \\
If something \textbf{fits} someone for a particular task or role , it makes them good enough or suitable for it.
 \textit{
	\begin{itemize}
	\item ...a man whose past experience fits him for the top job in education.
	\item His personality may not have fitted him to be Prime Minister.
	\end{itemize}
}
\item adjective \\
If you say that something or someone is \textbf{fit}  \textbf{to} produce some extreme result, you are emphasizing the extreme nature of that thing or that person's activity.
 \textbf{Fit} is also an adverb .
 \textit{
	\begin{itemize}
	\item The stink was fit to knock you down.
	\item ...hour after hour, the same exercises until you're fit to drop!
	\item Wally was laughing fit to burst.
	\item You're shivering fit to die, Gracie.
	\end{itemize}
}
\item  \\
 to see fit \textit{
	\begin{itemize}
	\end{itemize}
}
\end{enumerate}

\section*{newspaper}
{\large \color{blue}  newspapers  }
\subsection*{Explain}
\begin{enumerate}
\item countable noun \\
A \textbf{newspaper} is a publication consisting of a number of large sheets of folded paper , on which news, advertisements, and other information is printed .
 \textit{
	\begin{itemize}
	\item He was carrying a newspaper.
	\item They read their daughter's allegations in the newspaper.
	\item She writes a regular Sunday newspaper feature.
	\end{itemize}
}
\item countable noun \\
A \textbf{newspaper} is an organization that produces a newspaper.
 \textit{
	\begin{itemize}
	\item It is Britain's fastest growing national daily newspaper.
	\item He is a food critic for the newspaper.
	\end{itemize}
}
\item uncountable noun \\
\textbf{Newspaper} consists of pieces of old newspapers, especially when they are being used for another purpose such as wrapping things up.
 \textit{
	\begin{itemize}
	\item He found two pots, each wrapped in newspaper.
	\end{itemize}
}
\end{enumerate}

\section*{grateful}
{\large \color{blue}  }
\subsection*{Explain}
\begin{enumerate}
\item adjective \\
If you are \textbf{grateful}  \textbf{for} something that someone has given you or done for you, you have warm , friendly feelings towards them and wish to thank them.
 \textit{
	\begin{itemize}
	\item She was grateful to him for being so good to her.
	\item I should like to extend my grateful thanks to all the volunteers.
	\end{itemize}
}
\end{enumerate}

\section*{nurse}
{\large \color{blue}  nurses  nursing  nursed  }
\subsection*{Explain}
\begin{enumerate}
\item countable noun \\
A \textbf{nurse} is a person whose job is to care for people who are ill .
 \textit{
	\begin{itemize}
	\item She had spent 29 years as a nurse.
	\item I rang for the nurse and asked for some water.
	\end{itemize}
}
\item verb \\
If you \textbf{nurse} someone, you care for them when they are ill.
 \textit{
	\begin{itemize}
	\item All the years he was sick, my mother had nursed him.
	\item She rushed home to nurse her daughter back to health.
	\end{itemize}
}
\item verb \\
If you \textbf{nurse} an illness or injury , you allow it to get  better by resting as much as possible .
 \textit{
	\begin{itemize}
	\item We're going to go home and nurse our colds.
	\end{itemize}
}
\item verb \\
If you \textbf{nurse} an emotion or desire , you feel it strongly for a long time.
 \textit{
	\begin{itemize}
	\item Jane still nurses the pain of rejection.
	\item He had nursed an ambition to lead his own big orchestra.
	\end{itemize}
}
\item countable noun \\
A \textbf{nurse} is a person who is trained to look after young children.
 \textit{
	\begin{itemize}
	\item Every morning she got up early with the children and the nurse.
	\end{itemize}
}
\item verb \\
When a baby \textbf{nurses} or when its mother  \textbf{nurses} it, it feeds by sucking  milk from its mother's breast.
 \textit{
	\begin{itemize}
	\item Most authorities recommend letting the baby nurse whenever it wants.
	\item ...young women nursing babies.
	\item Young people and nursing mothers are exempted from charges.
	\end{itemize}
}
\end{enumerate}

\section*{helpful}
{\large \color{blue}  }
\subsection*{Explain}
\begin{enumerate}
\item adjective \\
If you describe someone as \textbf{helpful} , you mean that they help you in some way, such as doing part of your job for you or by giving you advice or information .
 \textit{
	\begin{itemize}
	\item The staff in the London office are helpful but only have limited information.
	\item James is a very helpful and cooperative lad.
	\item Thank you, you've been most helpful.
	\end{itemize}
}
\item adjective \\
If you describe information or advice as \textbf{helpful} , you mean that it is useful for you.
 \textit{
	\begin{itemize}
	\item The catalog includes helpful information on the different bike models available.
	\item The following information may be helpful to readers.
	\end{itemize}
}
\item adjective \\
Something that is \textbf{helpful} makes a situation more pleasant or more easy to tolerate .
 \textit{
	\begin{itemize}
	\item A predominantly liquid diet for a day or two may be helpful.
	\item It is often helpful to have your spouse in the room when major news is expected.
	\item Recognising this is can be very helpful in enabling one to cope.
	\end{itemize}
}
\end{enumerate}

\section*{passport}
{\large \color{blue}  passports  }
\subsection*{Explain}
\begin{enumerate}
\item countable noun \\
Your \textbf{passport} is an official document containing your name, photograph , and personal  details , which you need to show when you enter or leave a country.
 \textit{
	\begin{itemize}
	\item You should take your passport with you when changing money.
	\item ...a South African businessman travelling on a British passport.
	\end{itemize}
}
\item countable noun \\
If you say that a thing is a \textbf{passport to}  success or happiness, you mean that this thing makes success or happiness possible .
 \textit{
	\begin{itemize}
	\item Victory would give him a passport to the riches he craves.
	\item If the interview goes well it could be the passport to an exciting new career.
	\end{itemize}
}
\end{enumerate}

\section*{hot}
{\large \color{blue}  hotter  hottest  hots  hotting  hotted  }
\subsection*{Explain}
\begin{enumerate}
\item adjective \\
Something that is \textbf{hot} has a high temperature.
 \textit{
	\begin{itemize}
	\item When the oil is hot, add the sliced onion.
	\item What he needed was a hot bath and a good sleep.
	\item Metal-handled pans can get really hot and burn you.
	\end{itemize}
}
\item adjective \\
\textbf{Hot} is used to describe the weather or the air in a room or building when the temperature is high.
 \textit{
	\begin{itemize}
	\item It was too hot even for a gentle stroll.
	\item It was a hot, humid summer day.
	\item My small greenhouse gets very hot when the sun is shining.
	\end{itemize}
}
\item adjective \\
If you are \textbf{hot} , you feel as if your body is at an unpleasantly high temperature.
 \textit{
	\begin{itemize}
	\item I was too hot and tired to eat more than a few mouthfuls.
	\item My head was reeling. I felt hot all over.
	\end{itemize}
}
\item graded adjective \\
You use \textbf{hot} to talk or ask about how high the temperature of something is.
 \textit{
	\begin{itemize}
	\item They are called incandescent lights, and their colour depends on how hot they are.
	\item Remember that the top of the oven will be hotter than the bottom.
	\end{itemize}
}
\item adjective \\
\textbf{Hot} food is intended to be eaten as soon as it is cooked, as opposed to food that you eat when it has cooled or that you do not cook at all.
 \textit{
	\begin{itemize}
	\item If you live alone, you might not want to cook a hot meal every day.
	\end{itemize}
}
\item adjective \\
You can say that food is \textbf{hot} when it has a strong, burning taste caused by chillies , pepper , or ginger .
 \textit{
	\begin{itemize}
	\item ...hot curries.
	\item ...a dish that's spicy but not too hot.
	\end{itemize}
}
\item adjective \\
A \textbf{hot} issue or topic is one that is very important at the present time and is receiving a lot of publicity .
 \textit{
	\begin{itemize}
	\item The role of women in war is a hot topic of debate.
	\end{itemize}
}
\item adjective \\
\textbf{Hot}  news is new, recent, and fresh.
 \textit{
	\begin{itemize}
	\item ...eight pages of the latest movies and the hot news from Tinseltown.
	\end{itemize}
}
\item adjective \\
You can use \textbf{hot} to describe something that is very exciting and that many people want to see, use, obtain, or become involved with.
 \textit{
	\begin{itemize}
	\item The hottest show in town was the Monet Exhibition at the Art Institute.
	\item When I was last there, the hot place was the Royal Bachelors' Club.
	\end{itemize}
}
\item adjective \\
You can describe someone as \textbf{hot} if you think they are sexually attractive.
 \textit{
	\begin{itemize}
	\item This girl is incredibly hot.
	\end{itemize}
}
\item adjective \\
You can use \textbf{hot} to describe something that no one wants to deal with, often because it has been illegally
obtained and is very valuable or famous.
 \textit{
	\begin{itemize}
	\item If too much publicity is given to the theft, the works will become too hot to handle
and be destroyed.
	\end{itemize}
}
\item adjective \\
You can describe a situation that is created by a person's behaviour or attitude as
 \textbf{hot} when it is unpleasant and difficult to deal with.
 \textit{
	\begin{itemize}
	\item When the streets get too hot for them, they head south in one stolen car after another.
	\end{itemize}
}
\item adjective \\
A \textbf{hot} contest is one that is intense and involves a great deal of activity and determination .
 \textit{
	\begin{itemize}
	\item It took hot competition from abroad, however, to show us just how good Scottish cashmere
really is.
	\end{itemize}
}
\item adjective \\
If a person or team is the \textbf{hot}  favourite , people think that they are the one most likely to win a race or competition.
 \textit{
	\begin{itemize}
	\item Atlantic City is the hot favourite to stage the fight.
	\end{itemize}
}
\item adjective \\
Someone who has a \textbf{hot}  temper gets angry very quickly and easily.
 \textit{
	\begin{itemize}
	\item His hot temper was making it increasingly difficult for others to work with him.
	\end{itemize}
}
\item  \\
 to blow hot and cold \textit{
	\begin{itemize}
	\end{itemize}
}
\item  \\
 hot and bothered \textit{
	\begin{itemize}
	\end{itemize}
}
\item  \\
 get/have the hots for \textit{
	\begin{itemize}
	\end{itemize}
}
\end{enumerate}

\section*{peanut}
{\large \color{blue}  peanuts  }
\subsection*{Explain}
\begin{enumerate}
\item countable noun \\
\textbf{Peanuts} are small nuts that grow under the ground . Peanuts are often eaten as a snack , especially  roasted and salted .
 \textit{
	\begin{itemize}
	\item ...a packet of peanuts.
	\item Add 2 tablespoons of peanut oil.
	\end{itemize}
}
\item plural noun \\
If you say that a sum of money is \textbf{peanuts} , you mean that it is very small.
 \textit{
	\begin{itemize}
	\item The cost was peanuts compared to a new kitchen.
	\item The jobs they offer pay peanuts.
	\end{itemize}
}
\end{enumerate}

\section*{identical}
{\large \color{blue}  }
\subsection*{Explain}
\begin{enumerate}
\item adjective \\
Things that are \textbf{identical} are exactly the same.
 \textit{
	\begin{itemize}
	\item Nearly all the houses were identical.
	\item The two parties fought the last election on almost identical manifestos.
	\end{itemize}
}
\end{enumerate}

\section*{pit}
{\large \color{blue}  pits  pitting  pitted  }
\subsection*{Explain}
\begin{enumerate}
\item countable noun \\
A \textbf{pit} is a coal mine.
 \textit{
	\begin{itemize}
	\item It was a better community then when all the pits were working.
	\end{itemize}
}
\item countable noun \\
A \textbf{pit} is a large hole that is dug in the ground.
 \textit{
	\begin{itemize}
	\item Eric lost his footing and began to slide into the pit.
	\end{itemize}
}
\item countable noun \\
A \textbf{gravel pit} or \textbf{clay pit} is a very large hole that is left where gravel or clay has been dug from the ground.
 \textit{
	\begin{itemize}
	\item This area of former farmland was worked as a gravel pit until 1964.
	\end{itemize}
}
\item verb \\
If two opposing things or people \textbf{are pitted against} one another, they are in conflict .
 \textit{
	\begin{itemize}
	\item You will be pitted against people who are every bit as good as you are.
	\item This was one man pitted against the universe.
	\end{itemize}
}
\item plural noun \\
In motor  racing , \textbf{the}  \textbf{pits} are the areas at the side of the track where drivers  stop to get more fuel and to repair their cars during races.
 \textit{
	\begin{itemize}
	\item He moved quickly into the pits and climbed rapidly out of the car.
	\end{itemize}
}
\item plural noun \\
If you describe something as \textbf{the pits} , you mean that it is extremely  bad .
 \textit{
	\begin{itemize}
	\item Mary Ann asked him how dinner had been. 'The pits,' he replied.
	\end{itemize}
}
\item countable noun \\
A \textbf{pit} is the stone of a fruit or vegetable .
 \textit{
	\begin{itemize}
	\end{itemize}
}
\item  \\
 pit one's wits against sb \textit{
	\begin{itemize}
	\end{itemize}
}
\item  \\
 in the pit of one's stomach \textit{
	\begin{itemize}
	\end{itemize}
}
\end{enumerate}

\section*{immigrant}
{\large \color{blue}  immigrants  }
\subsection*{Explain}
\begin{enumerate}
\item countable noun \\
An \textbf{immigrant} is a person who has come to live in a country from some other country. Compare  emigrant .
 \textit{
	\begin{itemize}
	\item ...illegal immigrants.
	\item ...immigrant visas.
	\end{itemize}
}
\end{enumerate}

\section*{prestige}
{\large \color{blue}  }
\subsection*{Explain}
\begin{enumerate}
\item uncountable noun \\
If a person, a country, or an organization has \textbf{prestige} , they are admired and respected because of the position they hold or the things they have achieved.
 \textit{
	\begin{itemize}
	\item ...efforts to build up the prestige of the United Nations.
	\item It was his responsibility for foreign affairs that gained him international prestige.
	\item ...high prestige jobs.
	\end{itemize}
}
\item adjective \\
\textbf{Prestige} is used to describe  products , places, or activities which people admire because they are associated with being rich or having a high social position.
 \textit{
	\begin{itemize}
	\item ...such prestige cars as Cadillac, Mercedes, Porsche and Jaguar.
	\end{itemize}
}
\end{enumerate}

\section*{reed}
{\large \color{blue}  reeds  }
\subsection*{Explain}
\begin{enumerate}
\item countable noun \\
\textbf{Reeds} are tall plants that grow in large groups in shallow water or on ground that is always  wet and soft . They have strong , hollow stems that can be used for making things such as mats or baskets .
 \textit{
	\begin{itemize}
	\end{itemize}
}
\item countable noun \\
A \textbf{reed} is a small piece of cane or metal inserted into the mouthpiece of a woodwind instrument. The reed vibrates when you blow through it and makes a sound.
 \textit{
	\begin{itemize}
	\end{itemize}
}
\end{enumerate}

\section*{intrinsic}
{\large \color{blue}  }
\subsection*{Explain}
\begin{enumerate}
\item adjective \\
If something has \textbf{intrinsic} value or \textbf{intrinsic}  interest , it is valuable or interesting because of its basic nature or character , and not because of its connection with other things.
 \textit{
	\begin{itemize}
	\item The paintings have no intrinsic value except as curiosities.
	\item The rate is determined by intrinsic qualities such as the land's slope.
	\end{itemize}
}
\end{enumerate}

\section*{reporter}
{\large \color{blue}  reporters  }
\subsection*{Explain}
\begin{enumerate}
\item countable noun \\
A \textbf{reporter} is someone who writes news articles or who broadcasts news reports.
 \textit{
	\begin{itemize}
	\item ...a TV reporter.
	\item ...a trainee sports reporter.
	\item Our reporter Chris Loosemore sums up the findings.
	\end{itemize}
}
\end{enumerate}

\section*{restraint}
{\large \color{blue}  restraints  }
\subsection*{Explain}
\begin{enumerate}
\item variable noun \\
\textbf{Restraints} are rules or conditions that limit or restrict someone or something.
 \textit{
	\begin{itemize}
	\item The Prime Minister is calling for new restraints on trade unions.
	\item With open frontiers, criminals could cross into the country without restraint.
	\end{itemize}
}
\item uncountable noun \\
\textbf{Restraint} is calm , controlled, and unemotional  behaviour .
 \textit{
	\begin{itemize}
	\item They behaved with more restraint than I'd expected.
	\item I'll speak to the staff and ask them to exercise restraint and common sense.
	\end{itemize}
}
\item uncountable noun \\
\textbf{Restraint}  \textbf{of} something is the act of preventing it from increasing too much or from being done
freely.
 \textit{
	\begin{itemize}
	\item For a year and a half, wage restraint on a voluntary basis worked.
	\item He sued them for restraint of trade and won.
	\end{itemize}
}
\end{enumerate}

\section*{loose}
{\large \color{blue}  looser  loosest  looses  loosing  loosed  }
\subsection*{Explain}
\begin{enumerate}
\item adjective \\
Something that is \textbf{loose} is not firmly held or fixed in place.
 \textit{
	\begin{itemize}
	\item If a tooth feels very loose, your dentist may recommend that it's taken out.
	\item Two wooden beams had come loose from the ceiling.
	\item His tie was pulled loose and his collar hung open.
	\item She idly pulled at a loose thread on her skirt.
	\end{itemize}
}
\item adjective \\
Something that is \textbf{loose} is not attached to anything, or held or contained in anything.
 \textit{
	\begin{itemize}
	\item Two young men were racing motorcycles on the loose gravel.
	\item Frank emptied a handful of loose change on the table.
	\item A page came loose and floated onto the tiles.
	\end{itemize}
}
\item adjective \\
If people or animals break \textbf{loose} or are set \textbf{loose} , they are no longer held, tied , or kept somewhere and can move around freely.
 \textit{
	\begin{itemize}
	\item She broke loose from his embrace and crossed to the window.
	\item Why didn't you tell me she'd been set loose?
	\item Jack was chased by a loose dog.
	\end{itemize}
}
\item adjective \\
Clothes that are \textbf{loose} are rather large and do not fit closely.
 \textit{
	\begin{itemize}
	\item A pistol wasn't that hard to hide under a loose shirt.
	\item Wear loose clothes as they're more comfortable.
	\end{itemize}
}
\item adjective \\
If your hair is \textbf{loose} , it hangs freely round your shoulders and is not tied back.
 \textit{
	\begin{itemize}
	\item She was still in her nightdress, with her hair hanging loose over her shoulders.
	\end{itemize}
}
\item adjective \\
If something is \textbf{loose} in texture , there is space between the different particles or threads it consists of.
 \textit{
	\begin{itemize}
	\item She gathered loose soil and let it filter slowly through her fingers.
	\end{itemize}
}
\item adjective \\
A \textbf{loose}  grouping , arrangement, or organization is flexible rather than strictly controlled or organized .
 \textit{
	\begin{itemize}
	\item Murray and Alison came to some sort of loose arrangement before he went home.
	\item He wants a loose coalition of left wing forces.
	\end{itemize}
}
\item graded adjective \\
\textbf{Loose} words or expressions are not exact but rather vague .
 \textit{
	\begin{itemize}
	\item ...a loose translation.
	\item He despised loose thinking.
	\end{itemize}
}
\item graded adjective \\
If someone describes a woman or someone's behaviour as \textbf{loose} , they disapprove of that person because they think she or he has sexual relationships with too many people.
 \textit{
	\begin{itemize}
	\end{itemize}
}
\item verb \\
To \textbf{loose} a shot, arrow, or missile means to fire it.
 \textbf{Loose off} means the same as loose .
 \textit{
	\begin{itemize}
	\item He trained his gun down and loosed a brief burst.
	\item He loosed off two shots at the oncoming car.
	\end{itemize}
}
\item verb \\
If you \textbf{loose} something, you hold it less tightly or untie it slightly or completely.
 \textit{
	\begin{itemize}
	\item He gave a grunt and loosed his grip on the rifle.
	\item The guards loosed his arms.
	\end{itemize}
}
\item  \\
 on the loose \textit{
	\begin{itemize}
	\end{itemize}
}
\end{enumerate}

\section*{revenge}
{\large \color{blue}  revenges  revenging  revenged  }
\subsection*{Explain}
\begin{enumerate}
\item uncountable noun \\
\textbf{Revenge} involves hurting or punishing someone who has hurt or harmed you.
 \textit{
	\begin{itemize}
	\item The prisoners took revenge on their captors, eventually overcoming them.
	\item The killings were said to have been in revenge for the murder of her lover.
	\end{itemize}
}
\item verb \\
If you \textbf{revenge} yourself on someone who has hurt you, you hurt them in return.
 \textit{
	\begin{itemize}
	\item The Sunday Mercury accused her of trying to revenge herself on her former lover.
	\item ...the relatives of murdered villagers wanting to revenge the dead.
	\end{itemize}
}
\end{enumerate}

\section*{loyal}
{\large \color{blue}  }
\subsection*{Explain}
\begin{enumerate}
\item adjective \\
Someone who is \textbf{loyal}  remains  firm in their friendship or support for a person or thing.
 \textit{
	\begin{itemize}
	\item They had remained loyal to the president.
	\item He'd always been such a loyal friend to us all.
	\end{itemize}
}
\end{enumerate}

\section*{rod}
{\large \color{blue}  rods  }
\subsection*{Explain}
\begin{enumerate}
\item countable noun \\
A \textbf{rod} is a long, thin metal or wooden  bar .
 \textit{
	\begin{itemize}
	\item ...a 15-foot thick roof that was reinforced with steel rods.
	\end{itemize}
}
\end{enumerate}

\section*{own}
{\large \color{blue}  owns  owning  owned  }
\subsection*{Explain}
\begin{enumerate}
\item adjective \\
You use \textbf{own} to indicate that something belongs to a particular person or thing.
 \textbf{Own} is also a pronoun.
 \textit{
	\begin{itemize}
	\item Helen decided I should have my own shop.
	\item ...another group of patients who were taught to change their own dressings.
	\item Why can't I live a normal life in my own country?
	\item He could no longer trust his own judgement.
	\item His office had its own private entrance.
	\item He saw the Major's face a few inches from his own.
	\end{itemize}
}
\item adjective \\
You use \textbf{own} to indicate that something is used by, or is characteristic of, only one person,
thing, or group.
 \textbf{Own} is also a pronoun.
 \textit{
	\begin{itemize}
	\item Jennifer insisted on her own room.
	\item I let her tell me about it in her own way.
	\item Each nation has its own peculiarities when it comes to doing business.
	\item This young lady has a sense of style that is very much her own.
	\end{itemize}
}
\item adjective \\
You use \textbf{own} to indicate that someone does something without any help from other people.
 \textbf{Own} is also a pronoun.
 \textit{
	\begin{itemize}
	\item They enjoy making their own decisions.
	\item Tony also built his own house from his own plans.
	\item He'll have to make his own arrangements.
	\item There's no career structure, you have to create your own.
	\end{itemize}
}
\item verb \\
If you \textbf{own} something, it is your property.
 \textit{
	\begin{itemize}
	\item His father owns a local pub.
	\item Some of these companies are now owned by overseas corporations.
	\end{itemize}
}
\item verb \\
If you \textbf{own} someone, you completely defeat them in a game, competition , or argument .
 \textit{
	\begin{itemize}
	\item I just totally owned you.
	\end{itemize}
}
\item  \\
 to call something your own \textit{
	\begin{itemize}
	\end{itemize}
}
\item  \\
 come into one's/its own \textit{
	\begin{itemize}
	\end{itemize}
}
\item  \\
 to get your own back \textit{
	\begin{itemize}
	\end{itemize}
}
\item  \\
 make sth one's own \textit{
	\begin{itemize}
	\end{itemize}
}
\item  \\
 of one's own \textit{
	\begin{itemize}
	\end{itemize}
}
\item  \\
 of one's own/all of one's own \textit{
	\begin{itemize}
	\end{itemize}
}
\item  \\
 on one's own \textit{
	\begin{itemize}
	\end{itemize}
}
\item  \\
 on one's own \textit{
	\begin{itemize}
	\end{itemize}
}
\item  \\
 as if/like one owns the place \textit{
	\begin{itemize}
	\end{itemize}
}
\end{enumerate}

\section*{shell}
{\large \color{blue}  shells  shelling  shelled  }
\subsection*{Explain}
\begin{enumerate}
\item countable noun \\
The \textbf{shell} of a nut or egg is the hard covering which surrounds it.
 \textbf{Shell} is the substance that a shell is made of.
 \textit{
	\begin{itemize}
	\item They cracked the nuts and removed their shells.
	\item Once the eggs have hatched the shells are left behind.
	\item ...beads made from ostrich egg shell.
	\end{itemize}
}
\item countable noun \\
The \textbf{shell} of an animal such as a tortoise , snail , or crab is the hard protective covering that it has around its body or on its back.
 \textit{
	\begin{itemize}
	\end{itemize}
}
\item countable noun \\
\textbf{Shells} are hard objects found on beaches . They are usually pink , white, or brown and are the coverings which used to surround small sea creatures .
 \textit{
	\begin{itemize}
	\item I collect shells and interesting seaside items.
	\item ...sea shells.
	\end{itemize}
}
\item verb \\
If you \textbf{shell} nuts, peas , prawns , or other food, you remove their natural outer covering.
 \textit{
	\begin{itemize}
	\item She shelled and ate a few nuts.
	\item ...shelled prawns.
	\end{itemize}
}
\item countable noun \\
If someone comes out of their \textbf{shell} , they become more friendly and interested in other people and less quiet , shy , and reserved .
 \textit{
	\begin{itemize}
	\item Her normally shy son had come out of his shell.
	\item ...a lonely boy struggling to emerge from his shell.
	\end{itemize}
}
\item countable noun \\
The \textbf{shell} of a building, boat, car, or other structure is the outside frame of it.
 \textit{
	\begin{itemize}
	\item ...the shells of burned buildings.
	\item The solid feel of the car's shell is impressive.
	\end{itemize}
}
\item countable noun \\
A \textbf{shell} is a weapon consisting of a metal container filled with explosives that can be fired
from a large gun over long distances.
 \textit{
	\begin{itemize}
	\end{itemize}
}
\item verb \\
To \textbf{shell} a place means to fire explosive shells at it.
 \textit{
	\begin{itemize}
	\item The rebels shelled the densely-populated suburbs near the port.
	\end{itemize}
}
\end{enumerate}

\section*{preferable}
{\large \color{blue}  }
\subsection*{Explain}
\begin{enumerate}
\item adjective \\
If you say that one thing is \textbf{preferable}  \textbf{to} another, you mean that it is more desirable or suitable .
 \textit{
	\begin{itemize}
	\item A big earthquake a long way off is preferable to a smaller one nearby.
	\item The hazards of the theatre seemed preferable to joining the family paint business.
	\item Eating little and often may be preferable to large regular meals.
	\end{itemize}
}
\end{enumerate}

\section*{soda}
{\large \color{blue}  sodas  }
\subsection*{Explain}
\begin{enumerate}
\item variable noun \\
\textbf{Soda} is the same as soda water .
 \textit{
	\begin{itemize}
	\end{itemize}
}
\item variable noun \\
\textbf{Soda} is a sweet fizzy drink.
 A \textbf{soda} is a bottle of soda.
 \textit{
	\begin{itemize}
	\item ...a glass of diet soda.
	\item They had liquor for the adults and sodas for the children.
	\end{itemize}
}
\item countable noun \\
A \textbf{soda} is an ice-cream soda .
 \textit{
	\begin{itemize}
	\item ...self-service fountain sodas and small bags of potato chips.
	\end{itemize}
}
\end{enumerate}

\section*{regardless}
{\large \color{blue}  }
\subsection*{Explain}
\begin{enumerate}
\item  \\
 regardless of \textit{
	\begin{itemize}
	\end{itemize}
}
\item adverb \\
If you say that someone did something \textbf{regardless} , you mean that they did it even though there were problems or factors that could have stopped them, or perhaps should have stopped them.
 \textit{
	\begin{itemize}
	\item Despite her recent surgery she has been carrying on regardless.
	\end{itemize}
}
\end{enumerate}

\section*{sound}
{\large \color{blue}  sounds  sounding  sounded  }
\subsection*{Explain}
\begin{enumerate}
\item countable noun \\
A \textbf{sound} is something that you hear.
 \textit{
	\begin{itemize}
	\item Peter heard the sound of gunfire.
	\item Liza was so frightened she couldn't make a sound.
	\item There was a splintering sound as the railing gave way.
	\item ...the sounds of children playing.
	\end{itemize}
}
\item uncountable noun \\
\textbf{Sound} is energy that travels in waves through air, water, or other substances, and can
be heard.
 \textit{
	\begin{itemize}
	\item The aeroplane will travel at twice the speed of sound.
	\end{itemize}
}
\item singular noun \\
\textbf{The}  \textbf{sound} on a television, radio, or CD player is what you hear coming from the machine. Its loudness can be controlled.
 \textit{
	\begin{itemize}
	\item She went and turned the sound down.
	\item Compact discs have brought about a vast improvement in recorded sound quality.
	\end{itemize}
}
\item countable noun \\
A singer's or band's \textbf{sound} is the distinctive quality of their music.
 \textit{
	\begin{itemize}
	\item They have started showing a strong soul element in their sound.
	\item He's got a unique sound and a unique style.
	\end{itemize}
}
\item verb \\
If something such as a horn or a bell  \textbf{sounds} or if you \textbf{sound} it, it makes a noise.
 \textit{
	\begin{itemize}
	\item The buzzer sounded in Daniel's office.
	\item A young man sounds the bell to start the Sunday service.
	\end{itemize}
}
\item verb \\
If you \textbf{sound} a warning , you publicly give it. If you \textbf{sound} a note of caution or optimism , you say publicly that you are cautious or optimistic.
 \textit{
	\begin{itemize}
	\item The markets sounded a warning over a slowdown in the global economy.
	\item Others consider the move premature and have sounded a note of caution.
	\item Sir Patrick attempted to sound a positive note, describing the meeting as serving
a useful purpose.
	\end{itemize}
}
\item link verb \\
When you are describing a noise, you can talk about the way it \textbf{sounds} .
 \textit{
	\begin{itemize}
	\item They heard what sounded like a huge explosion.
	\item The creaking of the hinges sounded very loud in that silence.
	\item It sounded as if he were trying to say something.
	\end{itemize}
}
\item link verb \\
When you talk about the way someone \textbf{sounds} , you are describing the impression you have of them when they speak.
 \textit{
	\begin{itemize}
	\item She sounded a bit worried.
	\item Murphy sounds like a child.
	\item She sounded as if she really cared.
	\item I thought she sounded a genuinely caring and helpful person.
	\end{itemize}
}
\item link verb \\
When you are describing your impression or opinion of something you have heard about
or read about, you can talk about the way it \textbf{sounds} .
 \textit{
	\begin{itemize}
	\item It sounds like a wonderful idea to me, does it really work?
	\item It sounds as if they might have made a dreadful mistake.
	\item She decided that her doctor's advice sounded pretty good.
	\item The book is not as morbid as it sounds.
	\item I know this sounds a crazy thing for me to ask you.
	\end{itemize}
}
\item singular noun \\
You can describe your impression of something you have heard about or read about by
talking about \textbf{the sound of} it.
 \textit{
	\begin{itemize}
	\item Here's a new idea we liked the sound of.
	\item I don't like the sound of Toby Osborne.
	\item From the sound of things, he might well be the same man.
	\item He was being paid danger money from the sound of it.
	\end{itemize}
}
\end{enumerate}

\section*{statement}
{\large \color{blue}  statements  statementing  statemented  }
\subsection*{Explain}
\begin{enumerate}
\item countable noun \\
A \textbf{statement} is something that you say or write which gives information in a formal or definite way.
 \textit{
	\begin{itemize}
	\item Andrew now disowns that statement, saying he was depressed when he made it.
	\item 'Things are moving ahead.'– I found that statement vague and unclear.
	\end{itemize}
}
\item countable noun \\
A \textbf{statement} is an official or formal announcement that is issued on a particular occasion .
 \textit{
	\begin{itemize}
	\item The statement by the military denied any involvement in last night's attack.
	\end{itemize}
}
\item countable noun \\
You can refer to the official account of events which a suspect or a witness gives to the police as a \textbf{statement} .
 \textit{
	\begin{itemize}
	\item The 350-page report was based on statements from witnesses to the events.
	\end{itemize}
}
\item countable noun \\
If you describe an action or thing as a \textbf{statement} , you mean that it clearly  expresses a particular opinion or idea that you have.
 \textit{
	\begin{itemize}
	\item The following recipe is a statement of another kind–food is fun!
	\end{itemize}
}
\item countable noun \\
A printed  document  showing how much money has been paid into and taken out of a bank or building society account is called a \textbf{statement} .
 \textit{
	\begin{itemize}
	\end{itemize}
}
\item verb \\
If a child  \textbf{is statemented} , social  services  staff write a document stating that the child has special educational needs, and the local  education authority has to make sure that everything that is necessary is provided for that child.
 \textit{
	\begin{itemize}
	\item Nearly a year later, it was agreed that Tom would be statemented.
	\item I had eight statemented children in my class.
	\end{itemize}
}
\end{enumerate}

\section*{sensitive}
{\large \color{blue}  }
\subsection*{Explain}
\begin{enumerate}
\item adjective \\
If you are \textbf{sensitive}  \textbf{to} other people's needs , problems , or feelings, you show  understanding and awareness of them.
 \textit{
	\begin{itemize}
	\item The classroom teacher must be sensitive to a child's needs.
	\item He was always so sensitive and caring.
	\end{itemize}
}
\item adjective \\
If you are \textbf{sensitive}  \textbf{about} something, you are easily worried and offended when people talk about it.
 \textit{
	\begin{itemize}
	\item Some young people are very sensitive about their appearance.
	\item Take it easy. Don't be so sensitive.
	\end{itemize}
}
\item adjective \\
A \textbf{sensitive}  subject or issue needs to be dealt with carefully because it is likely to cause disagreement or make people angry or upset .
 \textit{
	\begin{itemize}
	\item Employment is a very sensitive issue.
	\item ...politically sensitive matters.
	\end{itemize}
}
\item adjective \\
\textbf{Sensitive}  documents or reports contain information that needs to be kept  secret and dealt with carefully.
 \textit{
	\begin{itemize}
	\item He instructed staff to shred sensitive documents.
	\item ...sensitive information which, in the wrong hands, could jeopardise the safety of
British troops.
	\end{itemize}
}
\item adjective \\
Something that is \textbf{sensitive}  \textbf{to} a physical force, substance, or treatment is easily affected by it and often harmed by it.
 \textit{
	\begin{itemize}
	\item ...a chemical which is sensitive to light.
	\item ...gentle cosmetics for sensitive skin.
	\end{itemize}
}
\item adjective \\
A \textbf{sensitive} piece of scientific  equipment is capable of measuring or recording very small changes.
 \textit{
	\begin{itemize}
	\item ...an extremely sensitive microscope.
	\end{itemize}
}
\end{enumerate}

\section*{tomb}
{\large \color{blue}  tombs  }
\subsection*{Explain}
\begin{enumerate}
\item countable noun \\
A \textbf{tomb} is a large grave that is above ground and that usually has a sculpture or other decoration on it.
 \textit{
	\begin{itemize}
	\end{itemize}
}
\end{enumerate}

\section*{similar}
{\large \color{blue}  }
\subsection*{Explain}
\begin{enumerate}
\item adjective \\
If one thing is \textbf{similar to} another, or if two things are \textbf{similar} , they have features that are the same.
 \textit{
	\begin{itemize}
	\item ...a savoury cake with a texture similar to that of carrot cake.
	\item The accident was similar to one that happened in 1973.
	\item ...a group of similar pictures.
	\end{itemize}
}
\end{enumerate}

\section*{torment}
{\large \color{blue}  torments  tormenting  tormented  }
\subsection*{Explain}
\begin{enumerate}
\item uncountable noun \\
\textbf{Torment} is extreme suffering, usually mental suffering.
 \textit{
	\begin{itemize}
	\item The torment of having her baby kidnapped is written all over her face.
	\item He spent days in torment while the police searched for his stolen car.
	\end{itemize}
}
\item countable noun \\
A \textbf{torment} is something that causes extreme suffering, usually mental suffering.
 \textit{
	\begin{itemize}
	\item Sooner or later most writers end up making books about the torments of being a writer.
	\item Outdoors, mosquitoes and midges were a perpetual torment.
	\end{itemize}
}
\item verb \\
If something \textbf{torments} you, it causes you extreme mental suffering.
 \textit{
	\begin{itemize}
	\item At times the memories returned to torment her.
	\item He had lain awake all night, tormented by jealousy.
	\end{itemize}
}
\item verb \\
If you \textbf{torment} a person or animal, you annoy them in a playful , rather  cruel way for your own amusement .
 \textit{
	\begin{itemize}
	\item My older brother and sister used to torment me by singing it to me.
	\end{itemize}
}
\end{enumerate}

\section*{called}
{\large \color{blue}  }
\subsection*{Explain}
\begin{enumerate}
\item adjective \\
having the name \textit{
	\begin{itemize}
	\item There are two men called Buckley at the Home Office.
	\item a device called an optical amplifier
	\end{itemize}
}
\end{enumerate}

\section*{tribute}
{\large \color{blue}  tributes  }
\subsection*{Explain}
\begin{enumerate}
\item variable noun \\
A \textbf{tribute} is something that you say , do, or make to show your admiration and respect for someone.
 \textit{
	\begin{itemize}
	\item The song is a tribute to Roy Orbison.
	\item He paid tribute to the organising committee.
	\item Over nine-thousand ex-servicemen and women marched past in tribute to their fallen
comrades.
	\end{itemize}
}
\item singular noun \\
If one thing is \textbf{a}  \textbf{tribute}  \textbf{to} another, the first thing is the result of the second and shows how good it is.
 \textit{
	\begin{itemize}
	\item His success has been a tribute to hard work, to professionalism.
	\item It is a tribute to Mr Chandler's skill that he has fashioned a fascinating book out
of such unpromising material.
	\end{itemize}
}
\end{enumerate}

\section*{solitary}
{\large \color{blue}  }
\subsection*{Explain}
\begin{enumerate}
\item adjective \\
A person or animal that is \textbf{solitary}  spends a lot of time alone.
 \textit{
	\begin{itemize}
	\item Paul was a shy, pleasant, solitary man.
	\item They often have a lonely and solitary life to lead.
	\end{itemize}
}
\item adjective \\
A \textbf{solitary} activity is one that you do alone.
 \textit{
	\begin{itemize}
	\item His evenings were spent in solitary drinking.
	\end{itemize}
}
\item adjective \\
A \textbf{solitary} person or object is alone, with no others near them.
 \textit{
	\begin{itemize}
	\item ...the occasional solitary figure making a study of wildflowers or grasses.
	\end{itemize}
}
\item uncountable noun \\
\textbf{Solitary} is the same as solitary confinement .
 \textit{
	\begin{itemize}
	\item Tom was in solitary across the way from me.
	\end{itemize}
}
\end{enumerate}

\section*{trigger}
{\large \color{blue}  triggers  triggering  triggered  }
\subsection*{Explain}
\begin{enumerate}
\item countable noun \\
The \textbf{trigger} of a gun is a small lever which you pull to fire it.
 \textit{
	\begin{itemize}
	\item A man pointed a gun at them and pulled the trigger.
	\end{itemize}
}
\item countable noun \\
The \textbf{trigger} of a bomb is the device which causes it to explode .
 \textit{
	\begin{itemize}
	\item ...trigger devices for nuclear weapons.
	\end{itemize}
}
\item verb \\
To \textbf{trigger} a bomb or system means to cause it to work.
 \textit{
	\begin{itemize}
	\item The thieves must have deliberately triggered the alarm and hidden inside the house.
	\item The one thousand pound bomb was triggered by a wire.
	\item ...nuclear triggering devices.
	\end{itemize}
}
\item verb \\
If something \textbf{triggers} an event or situation , it causes it to begin to happen or exist .
 \textbf{Trigger off} means the same as trigger .
 \textit{
	\begin{itemize}
	\item ...the incident which triggered the outbreak of the First World War.
	\item The current recession was triggered by a slump in consumer spending.
	\item Even a problem as simple as a bad back often has an underlying triggering factor.
	\item It is still not clear what events triggered off the demonstrations.
	\end{itemize}
}
\item countable noun \\
If something acts as a \textbf{trigger}  \textbf{for} another thing such as an illness , event, or situation, the first thing causes the second thing to begin to happen or exist.
 \textit{
	\begin{itemize}
	\item Stress may act as a trigger for these illnesses.
	\end{itemize}
}
\end{enumerate}

\section*{strict}
{\large \color{blue}  stricter  strictest  }
\subsection*{Explain}
\begin{enumerate}
\item adjective \\
A \textbf{strict} rule or order is very clear and precise or severe and must  always be obeyed completely.
 \textit{
	\begin{itemize}
	\item The officials had issued strict instructions that we were not to get out of the jeep.
	\item French privacy laws are very strict.
	\item All your replies will be treated in the strictest confidence.
	\item Even if you are on a fairly strict diet you can still go out for a good meal.
	\end{itemize}
}
\item adjective \\
If a parent or other person in authority is \textbf{strict} , they regard many actions as unacceptable and do not allow them.
 \textit{
	\begin{itemize}
	\item My parents were very strict.
	\item ...a few schools selected for their high standards and their strict discipline.
	\end{itemize}
}
\item adjective \\
If you talk about the \textbf{strict}  meaning of something, you mean the precise meaning of it.
 \textit{
	\begin{itemize}
	\item It's not quite peace in the strictest sense of the word, rather the absence of war.
	\end{itemize}
}
\item adjective \\
You use \textbf{strict} to describe someone who never does things that are against their beliefs .
 \textit{
	\begin{itemize}
	\item Many people in the country are now strict vegetarians.
	\item He was a strict, old-school Freudian.
	\end{itemize}
}
\end{enumerate}

\section*{vase}
{\large \color{blue}  vases  }
\subsection*{Explain}
\begin{enumerate}
\item countable noun \\
A \textbf{vase} is a jar , usually made of glass or pottery , used for holding cut flowers or as an ornament.
 \textit{
	\begin{itemize}
	\item ...a vase of red roses.
	\item ...lead crystal vases.
	\end{itemize}
}
\end{enumerate}

\section*{subordinate}
{\large \color{blue}  subordinates  subordinating  subordinated  }
\subsection*{Explain}
\begin{enumerate}
\item countable noun \\
If someone is your \textbf{subordinate} , they have a less important position than you in the organization that you both work for.
 \textit{
	\begin{itemize}
	\item Haig tended not to seek guidance from subordinates.
	\item Nearly all her subordinates adored her.
	\end{itemize}
}
\item adjective \\
Someone who is \textbf{subordinate}  \textbf{to} you has a less important position than you and has to obey you.
 \textit{
	\begin{itemize}
	\item Sixty of his subordinate officers followed his example.
	\item Some people still regard women as subordinate to men.
	\end{itemize}
}
\item adjective \\
Something that is \textbf{subordinate}  \textbf{to} something else is less important than the other thing.
 \textit{
	\begin{itemize}
	\item It was an art in which words were subordinate to images.
	\end{itemize}
}
\item verb \\
If you \textbf{subordinate} something \textbf{to} another thing, you regard it or treat it as less important than the other thing.
 \textit{
	\begin{itemize}
	\item He was both willing and able to subordinate all else to this aim.
	\end{itemize}
}
\end{enumerate}

\section*{voice}
{\large \color{blue}  voices  voicing  voiced  }
\subsection*{Explain}
\begin{enumerate}
\item countable noun \\
When someone speaks or sings, you hear their \textbf{voice} .
 \textit{
	\begin{itemize}
	\item Miriam's voice was strangely calm.
	\item 'The police are here,' she said in a low voice.
	\item There was a sound of loud voices from the kitchen.
	\item I ended up with bronchitis and no voice.
	\end{itemize}
}
\item countable noun \\
Someone's \textbf{voice} is their opinion on a particular topic and what they say about it.
 \textit{
	\begin{itemize}
	\item What does one do when a government simply refuses to listen to the voice of the opposition?
	\item There was no disagreement, there were no dissenting voices.
	\end{itemize}
}
\item singular noun \\
If you have a \textbf{voice in} something, you have the right to express an opinion on it.
 \textit{
	\begin{itemize}
	\item The people themselves must be an important voice in the debate.
	\item But your partners will have no voice in how you operate your company.
	\end{itemize}
}
\item verb \\
If you \textbf{voice} something such as an opinion or an emotion , you say what you think or feel .
 \textit{
	\begin{itemize}
	\item Some scientists have voiced concern that the disease could be passed on to humans.
	\item This is a criticism frequently voiced by opponents.
	\end{itemize}
}
\item singular noun \\
In grammar , if a verb is in \textbf{the active voice} , the person who performs the action is the subject of the verb. If a verb is in \textbf{the passive voice} , the thing or person affected by the action is the subject of the verb.
 \textit{
	\begin{itemize}
	\end{itemize}
}
\item  \\
 find one's voice \textit{
	\begin{itemize}
	\end{itemize}
}
\item  \\
 find one's voice \textit{
	\begin{itemize}
	\end{itemize}
}
\item  \\
 give voice to \textit{
	\begin{itemize}
	\end{itemize}
}
\item  \\
 keep one's voice down \textit{
	\begin{itemize}
	\end{itemize}
}
\item  \\
 lose one's voice \textit{
	\begin{itemize}
	\end{itemize}
}
\item  \\
 raise one's voice/lower one's voice \textit{
	\begin{itemize}
	\end{itemize}
}
\item  \\
 at the top of one's voice \textit{
	\begin{itemize}
	\end{itemize}
}
\item  \\
 with one voice \textit{
	\begin{itemize}
	\end{itemize}
}
\end{enumerate}

\section*{systematic}
{\large \color{blue}  }
\subsection*{Explain}
\begin{enumerate}
\item adjective \\
Something that is done in a \textbf{systematic} way is done according to a fixed plan, in a thorough and efficient way.
 \textit{
	\begin{itemize}
	\item They went about their business in a systematic way.
	\item They had not found any evidence of a systematic attempt to rig the ballot.
	\end{itemize}
}
\end{enumerate}

\section*{vote}
{\large \color{blue}  votes  voting  voted  }
\subsection*{Explain}
\begin{enumerate}
\item countable noun \\
A \textbf{vote} is a choice made by a particular person or group in a meeting or an election .
 \textit{
	\begin{itemize}
	\item He walked to the local polling centre to cast his vote.
	\item The government got a massive majority–well over 400 votes.
	\item Mr Reynolds was re-elected by 102 votes to 60.
	\end{itemize}
}
\item countable noun \\
\textbf{A}  \textbf{vote} is an occasion when a group of people make a decision by each person indicating his or her choice. The choice that most people support is accepted by the group.
 \textit{
	\begin{itemize}
	\item Why do you think we should have a vote on that?
	\item They took a vote and decided not to do it.
	\end{itemize}
}
\item singular noun \\
\textbf{The}  \textbf{vote} is the total number of votes or voters in an election, or the number of votes received or cast by a particular group.
 \textit{
	\begin{itemize}
	\item Opposition parties won about fifty-five per cent of the vote.
	\item The vote was overwhelmingly in favour of the Democratic Party.
	\item ...a huge majority of the white male vote.
	\end{itemize}
}
\item singular noun \\
If you have \textbf{the}  \textbf{vote} in an election, or have \textbf{a}  \textbf{vote} in a meeting, you have the legal right to indicate your choice.
 \textit{
	\begin{itemize}
	\item And of course we didn't even have the vote, did we?
	\item Before that, women did not have a vote at all.
	\item People with disabilities have got a vote as well, you know.
	\end{itemize}
}
\item verb \\
When you \textbf{vote} , you indicate your choice officially at a meeting or in an election, for example by raising your hand or writing on a piece of paper .
 \textit{
	\begin{itemize}
	\item Two-thirds of the electorate had the chance to vote in these elections.
	\item It seems many people would vote for the government, if there was a new leader.
	\item Both chambers plan to vote on that policy before January 15th.
	\item The residents of Leningrad voted to restore the city's original name of St Petersburg.
	\item The board of trustees voted by majority vote to remove the director.
	\item The council voted 9:8 for a five percent tax increase.
	\end{itemize}
}
\item verb \\
If you \textbf{vote} a particular political  party or leader , or \textbf{vote}  \textbf{yes} or \textbf{no} , you make that choice with the vote that you have.
 \textit{
	\begin{itemize}
	\item 52.5% of those questioned said they'd vote Labour.
	\item I probably would have voted that way anyway.
	\item A single candidate is put forward and the people vote yes or no.
	\end{itemize}
}
\item verb \\
If a government or other organization  \textbf{votes}  money  \textbf{for} something or \textbf{to} do something, they decide to spend the money in that way.
 \textit{
	\begin{itemize}
	\item The General Court had voted $250 for a monument to be erected to his memory.
	\item The Parliament voted more funds to help maintain American forces.
	\end{itemize}
}
\item verb \\
If people \textbf{vote} someone a particular title , they choose that person to have that title.
 \textit{
	\begin{itemize}
	\item His class voted him the man 'who had done the most for Yale.'.
	\item Michael has been voted Player of the Year.
	\end{itemize}
}
\item  \\
 to vote with your feet \textit{
	\begin{itemize}
	\end{itemize}
}
\item  \\
 I vote \textit{
	\begin{itemize}
	\end{itemize}
}
\item  \\
 one man one vote \textit{
	\begin{itemize}
	\end{itemize}
}
\end{enumerate}

\section*{tall}
{\large \color{blue}  taller  tallest  }
\subsection*{Explain}
\begin{enumerate}
\item adjective \\
Someone or something that is \textbf{tall} has a greater height than is normal or average.
 \textit{
	\begin{itemize}
	\item Being tall can make you feel incredibly self-confident.
	\item She was a young woman, fairly tall and fairly slim.
	\item The windows overlooked a lawn of tall waving grass.
	\end{itemize}
}
\item adjective \\
You use \textbf{tall} to ask or talk about the height of someone or something.
 \textit{
	\begin{itemize}
	\item How tall are you?
	\item I'm only 5ft tall, and I look younger than my age.
	\item I am already as tall as she is.
	\item Tony, my oldest, is already taller than me, and he's only eleven.
	\end{itemize}
}
\item  \\
 a tall order \textit{
	\begin{itemize}
	\end{itemize}
}
\item  \\
 to walk tall \textit{
	\begin{itemize}
	\end{itemize}
}
\end{enumerate}

\section*{wreath}
{\large \color{blue}  wreaths  }
\subsection*{Explain}
\begin{enumerate}
\item countable noun \\
A \textbf{wreath} is an arrangement of flowers and leaves, usually in the shape of a circle , which you put on a grave or by a statue to show that you remember a person who has died or people who have died.
 \textit{
	\begin{itemize}
	\item The coffin lying before the altar was bare, except for a single wreath of white roses.
	\item The British, Australian and Turkish Prime Ministers laid wreaths at the war memorial.
	\end{itemize}
}
\item countable noun \\
A \textbf{wreath} is a circle of leaves or flowers which someone wears around their head.
 \textit{
	\begin{itemize}
	\end{itemize}
}
\item countable noun \\
A \textbf{wreath} is a circle of leaves which some people hang on the front  door of their house at Christmas .
 \textit{
	\begin{itemize}
	\end{itemize}
}
\end{enumerate}

\section*{above}
{\large \color{blue}  }
\subsection*{Explain}
\begin{enumerate}
\item preposition \\
If one thing is \textbf{above} another one, it is directly over it or higher than it.
 \textbf{Above} is also an adverb .
 \textit{
	\begin{itemize}
	\item He lifted his hands above his head.
	\item Apartment 46 was a quiet apartment, unlike the one above it.
	\item He was staring into the mirror above him.
	\item A long scream sounded from somewhere above.
	\item ...a picture of the new plane as seen from above.
	\item There are five bedrooms, a large attic above, and wine cellars below.
	\end{itemize}
}
\item adverb \\
In writing , you use \textbf{above} to refer to something that has already been mentioned or discussed .
 \textbf{Above} is also a noun .
 \textbf{Above} is also an adjective .
 \textit{
	\begin{itemize}
	\item Several conclusions could be drawn from the results described above.
	\item Full details are in the table above.
	\item For additional information, contact any of the above.
	\item For a copy of their brochure, write to the above address.
	\end{itemize}
}
\item preposition \\
If an amount or measurement is \textbf{above} a particular level , it is greater than that level.
 \textbf{Above} is also an adverb.
 \textit{
	\begin{itemize}
	\item The temperature crept up to just above 40 degrees.
	\item Victoria Falls has had above average levels of rainfall this year.
	\item These plants must be stored in the light at above freezing temperature.
	\item Government spending is planned to rise 3 per cent above inflation.
	\item Banks have been charging 25 percent and above for unsecured loans.
	\end{itemize}
}
\item preposition \\
If you hear one sound \textbf{above} another, it is louder or clearer than the second one.
 \textit{
	\begin{itemize}
	\item ...trying to talk above the noise.
	\item Then there was a woman's voice, rising shrilly above the barking.
	\end{itemize}
}
\item preposition \\
If someone is \textbf{above} you, they are in a higher social position than you or in a position of authority over you.
 \textbf{Above} is also an adverb.
 \textit{
	\begin{itemize}
	\item I know you're above me socially, but I must say I find your attitude offensive.
	\item I married above myself–rich county people.
	\item Look at the people above you in the positions of power.
	\item The police officers admitted beating the student, but said they were acting on orders
from above.
	\end{itemize}
}
\item  \\
 get above oneself \textit{
	\begin{itemize}
	\end{itemize}
}
\item preposition \\
If you say that someone thinks they are \textbf{above} something, you mean that they act as if they are too good or important for it.
 \textit{
	\begin{itemize}
	\item This was clearly a failure by someone who thought he was above failure.
	\item I'm not above doing my own cleaning.
	\end{itemize}
}
\item preposition \\
If someone is \textbf{above}  criticism or suspicion , they cannot be criticized or suspected because of their good qualities or their position.
 \textit{
	\begin{itemize}
	\item Science should not be above criticism.
	\item He was a respected academic and above suspicion.
	\end{itemize}
}
\item preposition \\
If you value one person or thing \textbf{above} any other, you value them more or consider that they are more important.
 \textit{
	\begin{itemize}
	\item ...his tendency to put the team above everything.
	\item I want to be honest, honest above everything else.
	\end{itemize}
}
\end{enumerate}

\section*{actor}
{\large \color{blue}  actors  }
\subsection*{Explain}
\begin{enumerate}
\item countable noun \\
An \textbf{actor} is someone whose job is acting in plays or films. 'Actor' in the singular usually refers to a man, but some women who act prefer to be called 'actors' rather than ' actresses '.
 \textit{
	\begin{itemize}
	\item His father was an actor in the Cantonese Opera Company.
	\item You have to be a very good actor to play that part.
	\end{itemize}
}
\end{enumerate}

\section*{abundant}
{\large \color{blue}  }
\subsection*{Explain}
\begin{enumerate}
\item adjective \\
Something that is \textbf{abundant} is present in large quantities.
 \textit{
	\begin{itemize}
	\item There is an abundant supply of cheap labour.
	\item Birds are abundant in the tall vegetation.
	\end{itemize}
}
\end{enumerate}

\section*{ankle}
{\large \color{blue}  ankles  }
\subsection*{Explain}
\begin{enumerate}
\item countable noun \\
Your \textbf{ankle} is the joint where your foot joins your leg.
 \textit{
	\begin{itemize}
	\item John twisted his ankle badly.
	\end{itemize}
}
\end{enumerate}

\section*{beloved}
{\large \color{blue}  }
\subsection*{Explain}
\begin{enumerate}
\item adjective \\
A \textbf{beloved} person, thing, or place is one that you feel great affection for.
 \textit{
	\begin{itemize}
	\item He lost his beloved wife last year.
	\item The rose is the most romantic of flowers, beloved of poets, singers, and artists.
	\end{itemize}
}
\item singular noun \\
Your \textbf{beloved} is the person that you love.
 \textit{
	\begin{itemize}
	\item He takes his beloved into his arms.
	\end{itemize}
}
\end{enumerate}

\section*{boy}
{\large \color{blue}  boys  }
\subsection*{Explain}
\begin{enumerate}
\item countable noun \\
A \textbf{boy} is a child who will  grow up to be a man.
 \textit{
	\begin{itemize}
	\item I knew him when he was a little boy.
	\item He was still just a boy.
	\end{itemize}
}
\item countable noun \\
You can  refer to a young man as a \textbf{boy} , especially when talking about relationships between boys and girls .
 \textit{
	\begin{itemize}
	\item ...the age when girls get interested in boys.
	\end{itemize}
}
\item countable noun \\
Someone's \textbf{boy} is their son .
 \textit{
	\begin{itemize}
	\item Eric was my cousin Edward's boy.
	\item I have two boys.
	\end{itemize}
}
\item countable noun \\
You can refer to a man as a \textbf{boy} , especially when you are talking about him in an affectionate  way .
 \textit{
	\begin{itemize}
	\item ...the local boy who made President.
	\item 'Come on boys', he shouted to the sailors.
	\end{itemize}
}
\item vocative noun \\
You can use \textbf{boy} when giving  instructions to a horse or dog .
 \textit{
	\begin{itemize}
	\item Down, boy, down!
	\end{itemize}
}
\item  \\
 the boys in blue \textit{
	\begin{itemize}
	\end{itemize}
}
\item  \\
 boy/oh boy \textit{
	\begin{itemize}
	\end{itemize}
}
\item  \\
 boys will be boys \textit{
	\begin{itemize}
	\end{itemize}
}
\item  \\
 one of the boys \textit{
	\begin{itemize}
	\end{itemize}
}
\end{enumerate}

\section*{big}
{\large \color{blue}  bigger  biggest  }
\subsection*{Explain}
\begin{enumerate}
\item adjective \\
A \textbf{big} person or thing is large in physical size.
 \textit{
	\begin{itemize}
	\item Australia's a big country.
	\item Her husband was a big man.
	\item The car was too big to fit into our garage.
	\end{itemize}
}
\item adjective \\
Something that is \textbf{big} consists of many people or things.
 \textit{
	\begin{itemize}
	\item The crowd included a big contingent from Ipswich.
	\item ...the big backlog of applications.
	\end{itemize}
}
\item adjective \\
If you describe something such as a problem , increase, or change as a \textbf{big} one, you mean it is great in degree, extent, or importance .
 \textit{
	\begin{itemize}
	\item The problem was just too big for her to tackle on her own.
	\item There could soon be a big increase in unemployment.
	\end{itemize}
}
\item adjective \\
A \textbf{big} organization employs many people and has many customers .
 \textit{
	\begin{itemize}
	\item Exchange is largely controlled by big banks.
	\item ...one of the biggest companies in Italy.
	\end{itemize}
}
\item adjective \\
If you say that someone is \textbf{big}  \textbf{in} a particular organization, activity, or place, you mean that they have a lot of influence or authority in it.
 \textit{
	\begin{itemize}
	\item Their father was very big in the army.
	\item I'm sure all the big names will come to the club.
	\end{itemize}
}
\item adjective \\
If you call someone a \textbf{big}  bully or a \textbf{big}  coward , you are emphasizing your disapproval of them.
 \textit{
	\begin{itemize}
	\end{itemize}
}
\item adjective \\
Children often refer to their older brother or sister as their \textbf{big} brother or sister.
 \textit{
	\begin{itemize}
	\end{itemize}
}
\item adjective \\
Capital letters are sometimes referred to as \textbf{big} letters.
 \textit{
	\begin{itemize}
	\item ...a big letter J.
	\end{itemize}
}
\item adjective \\
\textbf{Big} words are long or rare words which have meanings that are difficult to understand .
 \textit{
	\begin{itemize}
	\item They use a lot of big words.
	\end{itemize}
}
\item  \\
 make it big \textit{
	\begin{itemize}
	\end{itemize}
}
\item  \\
 to think big \textit{
	\begin{itemize}
	\end{itemize}
}
\item  \\
 in a big way \textit{
	\begin{itemize}
	\end{itemize}
}
\end{enumerate}

\section*{cottage}
{\large \color{blue}  cottages  }
\subsection*{Explain}
\begin{enumerate}
\item countable noun \\
A \textbf{cottage} is a small house, usually in the country.
 \textit{
	\begin{itemize}
	\item They used to have a cottage in N.W. Scotland.
	\item My sister Yvonne also came to live at Ockenden Cottage with me.
	\end{itemize}
}
\end{enumerate}

\section*{bold}
{\large \color{blue}  bolder  boldest  }
\subsection*{Explain}
\begin{enumerate}
\item adjective \\
Someone who is \textbf{bold} is not afraid to do things which involve risk or danger .
 \textit{
	\begin{itemize}
	\item Amrita becomes a bold, daring rebel.
	\item In 1960 this was a bold move.
	\item Poland was already making bold economic reforms.
	\end{itemize}
}
\item adjective \\
Someone who is \textbf{bold} is not shy or embarrassed in the company of other people.
 \textit{
	\begin{itemize}
	\item I don't feel I'm being bold, because it's always been natural for me to just speak
out.
	\end{itemize}
}
\item adjective \\
A \textbf{bold} colour or pattern is very bright and noticeable .
 \textit{
	\begin{itemize}
	\item ...bold flowers in various shades of red, blue or white.
	\item ...bold, dramatic colours.
	\end{itemize}
}
\item adjective \\
\textbf{Bold} lines or designs are drawn in a clear , strong way.
 \textit{
	\begin{itemize}
	\item Each picture is shown in colour on one page and as a bold outline on the opposite
page.
	\end{itemize}
}
\item uncountable noun \\
\textbf{Bold} is print which is thicker and looks blacker than ordinary printed letters .
 \textit{
	\begin{itemize}
	\end{itemize}
}
\end{enumerate}

\section*{denial}
{\large \color{blue}  denials  }
\subsection*{Explain}
\begin{enumerate}
\item variable noun \\
A \textbf{denial} of something is a statement that it is not true , does not exist, or did not happen .
 \textit{
	\begin{itemize}
	\item Despite official denials, the rumours still persist.
	\item Denial of the Mafia's existence is nothing new.
	\end{itemize}
}
\item uncountable noun \\
The \textbf{denial}  \textbf{of} something to someone is the act of refusing to let them have it.
 \textit{
	\begin{itemize}
	\item ...the denial of visas to international relief workers.
	\item This does not justify the denial of constitutional protection.
	\end{itemize}
}
\item uncountable noun \\
In psychology , \textbf{denial} is when a person cannot or will not accept an unpleasant truth.
 \textit{
	\begin{itemize}
	\item With traumas like losing a loved one, the mind's first reaction is denial.
	\item ...an addict who is in denial about his addiction.
	\end{itemize}
}
\end{enumerate}

\section*{careful}
{\large \color{blue}  }
\subsection*{Explain}
\begin{enumerate}
\item adjective \\
If you are \textbf{careful} , you give serious  attention to what you are doing, in order to avoid  harm , damage , or mistakes . If you are \textbf{careful}  \textbf{to} do something, you make sure that you do it.
 \textit{
	\begin{itemize}
	\item Be very careful with this stuff, it can be dangerous if it isn't handled properly.
	\item Careful on those stairs!
	\item We had to be very careful not to be seen.
	\item Pupils will need careful guidance on their choice of options.
	\end{itemize}
}
\item adjective \\
\textbf{Careful} work, thought , or examination is thorough and shows a concern for details .
 \textit{
	\begin{itemize}
	\item He has decided to prosecute her after careful consideration of all the relevant facts.
	\item What we now know about the disease was learned by careful study of diseased organs.
	\end{itemize}
}
\item adjective \\
If you tell someone to be \textbf{careful about} doing something, you think that what they intend to do is probably  wrong , and that they should think seriously before they do it.
 \textit{
	\begin{itemize}
	\item I think you should be careful about talking of the rebels as heroes.
	\item It is important, I think, for everyone to be careful about claiming victory.
	\end{itemize}
}
\item adjective \\
If you are \textbf{careful}  \textbf{with} something such as money or resources , you use or spend only what is necessary .
 \textit{
	\begin{itemize}
	\item You will have to make a special effort to train your child to be careful with her
pocket-money.
	\item It would force industries to be more careful with natural resources.
	\end{itemize}
}
\item  \\
 you can't be too careful \textit{
	\begin{itemize}
	\end{itemize}
}
\end{enumerate}

\section*{doctor}
{\large \color{blue}  doctors  doctoring  doctored  }
\subsection*{Explain}
\begin{enumerate}
\item countable noun \\
A \textbf{doctor} is someone who is qualified in medicine and treats people who are ill .
 \textit{
	\begin{itemize}
	\item Do not discontinue the treatment without consulting your doctor.
	\item Doctor Paige will be here right after lunch to see her.
	\end{itemize}
}
\item countable noun \\
A dentist or , veterinarian can also be called \textbf{doctor} .
 \textit{
	\begin{itemize}
	\end{itemize}
}
\item countable noun \\
\textbf{The}  \textbf{doctor's} is used to refer to the surgery or office where a doctor works.
 \textit{
	\begin{itemize}
	\item I have an appointment at the doctor's.
	\end{itemize}
}
\item countable noun \\
A \textbf{doctor} is someone who has been awarded the highest academic or honorary degree by a university.
 \textit{
	\begin{itemize}
	\item He is a doctor of philosophy.
	\end{itemize}
}
\item verb \\
If someone \textbf{doctors} something, they change it in order to deceive people.
 \textit{
	\begin{itemize}
	\item They doctored the prints to make her look as awful as possible.
	\item ...a cleverly doctored photograph.
	\end{itemize}
}
\item verb \\
If someone \textbf{doctors} food or drink, they add a poison or drug to it.
 \textit{
	\begin{itemize}
	\item She had no doubt that it was he who had doctored her milk.
	\item ...a doctored cup of tea.
	\end{itemize}
}
\end{enumerate}

\section*{clear}
{\large \color{blue}  clearer  clearest  clears  clearing  cleared  }
\subsection*{Explain}
\begin{enumerate}
\item adjective \\
Something that is \textbf{clear} is easy to understand, see, or hear.
 \textit{
	\begin{itemize}
	\item The book is clear, readable and adequately illustrated.
	\item The space telescope has taken the clearest pictures ever of Pluto.
	\item He repeated his answer, this time in a clear, firm tone of voice.
	\end{itemize}
}
\item adjective \\
Something that is \textbf{clear} is obvious and impossible to be mistaken about.
 \textit{
	\begin{itemize}
	\item It was a clear case of homicide.
	\item The clear message of the scientific reports is that there should be a drastic cut
in car use.
	\item A spokesman said the British government's position is perfectly clear.
	\item It became clear that I hadn't been able to convince Mike.
	\item It's not clear whether the incident was an accident or deliberate.
	\end{itemize}
}
\item adjective \\
If you are \textbf{clear}  \textbf{about} something, you understand it completely.
 \textit{
	\begin{itemize}
	\item It is important to be clear about what Chomsky is doing here.
	\item He is not entirely clear on how he will go about it.
	\item People use scientific terms with no clear idea of their meaning.
	\end{itemize}
}
\item adjective \\
If your mind or your way of thinking is \textbf{clear} , you are able to think sensibly and reasonably, and you are not affected by confusion
or by a drug such as alcohol.
 \textit{
	\begin{itemize}
	\item She needed a clear head to carry out her instructions.
	\end{itemize}
}
\item verb \\
To \textbf{clear} your mind or your head means to free it from confused thoughts or from the effects
of a drug such as alcohol.
 \textit{
	\begin{itemize}
	\item He walked up Fifth Avenue to clear his head.
	\item Our therapists will show you how to clear your mind of worries.
	\end{itemize}
}
\item adjective \\
A \textbf{clear} substance is one which you can see through and which has no colour, like clean water.
 \textit{
	\begin{itemize}
	\item ...a clear glass panel.
	\item ...a clear gel.
	\item The water is clear and plenty of fish are visible.
	\end{itemize}
}
\item graded adjective \\
A \textbf{clear} colour is bright and strong.
 \textit{
	\begin{itemize}
	\item He has clear blue eyes and a dazzling smile.
	\end{itemize}
}
\item adjective \\
If a surface, place, or view is \textbf{clear} , it is free of unwanted objects or obstacles .
 \textit{
	\begin{itemize}
	\item The runway is clear–go ahead and land.
	\item All exits must be kept clear in case of fire or a bomb scare.
	\item Caroline prefers her worktops to be clear of clutter.
	\item The windows will allow a clear view of the beach.
	\end{itemize}
}
\item verb \\
When you \textbf{clear} an area or place or \textbf{clear} something \textbf{from} it, you remove things from it that you do not want to be there.
 \textit{
	\begin{itemize}
	\item To clear the land and harvest the bananas they decided they needed to hire specialist
machinery.
	\item Stewart was trying to clear a path for the stretcher.
	\item Workers could not clear the tunnels of smoke.
	\item Firefighters were still clearing rubble from apartments damaged at the scene of the
attack.
	\end{itemize}
}
\item verb \\
If something or someone \textbf{clears} the way or the path \textbf{for} something to happen , they make it possible.
 \textit{
	\begin{itemize}
	\item The Prime Minister resigned today, clearing the way for the formation of a new government.
	\item A court in Berlin has dropped the charges against him, clearing the way for him to
leave Germany.
	\end{itemize}
}
\item adjective \\
If it is a \textbf{clear} day or if the sky is \textbf{clear} , there is no mist, rain, or cloud.
 \textit{
	\begin{itemize}
	\item On a clear day you can see the French coast.
	\item The winter sky was clear.
	\end{itemize}
}
\item verb \\
When fog or mist \textbf{clears} , it gradually disappears.
 \textit{
	\begin{itemize}
	\item The early morning mist had cleared.
	\end{itemize}
}
\item adjective \\
\textbf{Clear} eyes look healthy , attractive, and shining .
 \textit{
	\begin{itemize}
	\item ...clear blue eyes.
	\item Her eyes were clear and steady.
	\end{itemize}
}
\item adjective \\
If your skin is \textbf{clear} , it is healthy and free from spots.
 \textit{
	\begin{itemize}
	\end{itemize}
}
\item adjective \\
If you say that your conscience is \textbf{clear} , you mean you do not think you have done anything wrong.
 \textit{
	\begin{itemize}
	\item Mr Garcia said his conscience was clear over the jail incidents.
	\item I can look back on things with a clear conscience. I did everything I could.
	\end{itemize}
}
\item adjective \\
If something or someone is \textbf{clear}  \textbf{of} something else, it is not touching it or is a safe distance away from it.
 \textit{
	\begin{itemize}
	\item As soon as he was clear of the terminal building he looked round.
	\item She placed a towel on a cluster of rocks just clear of the tidemark.
	\item He lifted him clear of the deck with one arm.
	\end{itemize}
}
\item adverb \\
If you drive \textbf{clear} to a place, especially a place that is far away, you go all the way there without delays .
 \textit{
	\begin{itemize}
	\item After that they drove clear over to St Paul.
	\end{itemize}
}
\item verb \\
If an animal or person \textbf{clears} an object or \textbf{clears} a certain height, they jump over the object, or over something that height, without
touching it.
 \textit{
	\begin{itemize}
	\item He was the first vaulter to clear 6.00 metres.
	\end{itemize}
}
\item verb \\
When a bank \textbf{clears} a cheque or when a cheque \textbf{clears} , the bank agrees to pay the sum of money mentioned on it.
 \textit{
	\begin{itemize}
	\item Polish banks can still take two or three weeks to clear a cheque.
	\item Allow time for the cheque to clear.
	\end{itemize}
}
\item verb \\
If a course of action \textbf{is cleared} , people in authority give permission for it to happen.
 \textit{
	\begin{itemize}
	\item Linda Gradstein has this report from Jerusalem, which was cleared by an Israeli censor.
	\item Within an hour, the helicopter was cleared for take-off.
	\item Some of the pesticides found were not cleared for use in Britain.
	\end{itemize}
}
\item verb \\
If someone \textbf{is cleared} , they are proved to be not guilty of a crime or mistake.
 \textit{
	\begin{itemize}
	\item She was cleared of murder and jailed for just five years for manslaughter.
	\item In a final effort to clear her name, Eunice has written a book.
	\end{itemize}
}
\item  \\
 is that/do I make myself clear? \textit{
	\begin{itemize}
	\end{itemize}
}
\item  \\
 in the clear \textit{
	\begin{itemize}
	\end{itemize}
}
\item  \\
 to make sth clear \textit{
	\begin{itemize}
	\end{itemize}
}
\item  \\
 clear of \textit{
	\begin{itemize}
	\end{itemize}
}
\item  \\
 to steer/stay clear \textit{
	\begin{itemize}
	\end{itemize}
}
\end{enumerate}

\section*{drought}
{\large \color{blue}  droughts  }
\subsection*{Explain}
\begin{enumerate}
\item variable noun \\
A \textbf{drought} is a long period of time during which no rain  falls .
 \textit{
	\begin{itemize}
	\item Drought and famines have killed up to two million people here.
	\item ...one of the worst droughts of the century.
	\end{itemize}
}
\end{enumerate}

\section*{coherent}
{\large \color{blue}  }
\subsection*{Explain}
\begin{enumerate}
\item adjective \\
If something is \textbf{coherent} , it is well  planned , so that it is clear and sensible and all its parts go well with each other.
 \textit{
	\begin{itemize}
	\item He has failed to work out a coherent strategy for modernising the service.
	\item The President's policy is perfectly coherent.
	\end{itemize}
}
\item adjective \\
If someone is \textbf{coherent} , they express their thoughts in a clear and calm way, so that other people can understand what they are saying .
 \textit{
	\begin{itemize}
	\item He's so calm when he answers questions in interviews. I wish I could be that coherent.
	\end{itemize}
}
\end{enumerate}

\section*{enclosure}
{\large \color{blue}  enclosures  }
\subsection*{Explain}
\begin{enumerate}
\item countable noun \\
An \textbf{enclosure} is an area of land that is surrounded by a wall or fence and that is used for a particular purpose .
 \textit{
	\begin{itemize}
	\item This enclosure was so vast that the outermost wall could hardly be seen.
	\end{itemize}
}
\end{enumerate}

\section*{cunning}
{\large \color{blue}  }
\subsection*{Explain}
\begin{enumerate}
\item adjective \\
Someone who is \textbf{cunning} has the ability to achieve things in a clever way, often by deceiving other people.
 \textit{
	\begin{itemize}
	\item These disturbed kids can be cunning.
	\item The clever folk in management came up with a cunning plan.
	\end{itemize}
}
\item uncountable noun \\
\textbf{Cunning} is the ability to achieve things in a clever way, often by deceiving other people.
 \textit{
	\begin{itemize}
	\item ...one more example of the cunning of today's art thieves.
	\item He tackled the job with a great deal of imagination, skill and cunning.
	\end{itemize}
}
\end{enumerate}

\section*{era}
{\large \color{blue}  eras  }
\subsection*{Explain}
\begin{enumerate}
\item countable noun \\
You can refer to a period of history or a long period of time as an \textbf{era} when you want to draw  attention to a particular feature or quality that it has.
 \textit{
	\begin{itemize}
	\item ...the nuclear era.
	\item ...the Victorian era.
	\item It was an era of austerity.
	\end{itemize}
}
\end{enumerate}

\section*{decimal}
{\large \color{blue}  decimals  }
\subsection*{Explain}
\begin{enumerate}
\item adjective \\
A \textbf{decimal} system involves counting in units of ten.
 \textit{
	\begin{itemize}
	\item ...the decimal system of metric weights and measures.
	\item In 1971, the 1p and 2p decimal coins were introduced in Britain.
	\end{itemize}
}
\item countable noun \\
A \textbf{decimal} is a fraction that is written in the form of a dot  followed by one or more numbers which represent tenths, hundredths , and so on: for example .5, .51, .517.
 \textit{
	\begin{itemize}
	\item ...simple math concepts, such as decimals and fractions.
	\end{itemize}
}
\end{enumerate}

\section*{football}
{\large \color{blue}  footballs  }
\subsection*{Explain}
\begin{enumerate}
\item uncountable noun \\
\textbf{Football} is a game played by two teams of eleven  players using a round ball. Players kick the ball to each other and try to score goals by kicking the ball into a large net .
 \textit{
	\begin{itemize}
	\item Several boys were still playing football on the waste ground.
	\item ...Arsenal Football Club.
	\item ...Italian football fans.
	\end{itemize}
}
\item uncountable noun \\
\textbf{Football} is a game played by two teams of eleven players using an oval ball. Players carry
the ball in their hands or throw it to each other as they try to score goals that are called touchdowns.
 \textit{
	\begin{itemize}
	\item Two blocks beyond our school was a field where boys played football.
	\item ...this year's national college football championship.
	\end{itemize}
}
\item countable noun \\
A \textbf{football} is a ball that is used for playing football.
 \textit{
	\begin{itemize}
	\end{itemize}
}
\end{enumerate}

\section*{direct}
{\large \color{blue}  directs  directing  directed  }
\subsection*{Explain}
\begin{enumerate}
\item adjective \\
\textbf{Direct} means moving towards a place or object, without changing direction and without stopping , for example in a journey .
 \textbf{Direct} is also an adverb .
 \textit{
	\begin{itemize}
	\item They'd come on a direct flight from Athens.
	\item ...the direct route from Amman to Bombay.
	\item You can fly direct to Amsterdam from most British airports.
	\end{itemize}
}
\item adjective \\
If something is in \textbf{direct} heat or light, it is strongly affected by the heat or light, because there is nothing
between it and the source of heat or light to protect it.
 \textit{
	\begin{itemize}
	\item Medicines should be stored away from direct sunlight.
	\item Direct illumination is harsh and unflattering.
	\end{itemize}
}
\item adjective \\
You use \textbf{direct} to describe an experience, activity, or system which only involves the people, actions, or things
that are necessary to make it happen .
 \textbf{Direct} is also an adverb.
 \textit{
	\begin{itemize}
	\item He has direct experience of the process of privatisation.
	\item He seemed to be in direct contact with the Boss.
	\item He is expected to extend direct rule by the central government for another six months.
	\item I can deal direct with your Inspector Kimble.
	\item Write to us direct with details of your clubs.
	\end{itemize}
}
\item adjective \\
You use \textbf{direct} to emphasize the closeness of a connection between two things.
 \textit{
	\begin{itemize}
	\item They were unable to prove that she died as a direct result of his injection.
	\item His visit is direct evidence of the improvement in their relationship.
	\item The minister denied there was a direct connection between the two issues.
	\end{itemize}
}
\item adjective \\
If you describe a person or their behaviour as \textbf{direct} , you mean that they are honest and open, and say  exactly what they mean.
 \textit{
	\begin{itemize}
	\item He avoided giving a direct answer.
	\item The new songs are more direct.
	\item No direct reference was made to the call by the Foreign Office minister.
	\end{itemize}
}
\item verb \\
If you \textbf{direct} something \textbf{at} a particular thing, you aim or point it at that thing.
 \textit{
	\begin{itemize}
	\item I directed the extinguisher at the fire without effect.
	\item He directed the tiny beam of light at the roof.
	\end{itemize}
}
\item verb \\
If your attention, emotions , or actions \textbf{are directed}  \textbf{at} a particular person or thing, you are focusing them on that person or thing.
 \textit{
	\begin{itemize}
	\item The learner's attention needs to be directed to the significant features.
	\item Do not be surprised if, initially, she directs her anger at you.
	\item One assassination attempt was directed against the country's top three government
leaders.
	\end{itemize}
}
\item verb \\
If a remark or look \textbf{is directed}  \textbf{at} you, someone says something to you or looks at you.
 \textit{
	\begin{itemize}
	\item She could hardly believe the question was directed towards her.
	\item The abuse was directed at the TV crews.
	\item Arnold directed a meaningful look at Irma.
	\end{itemize}
}
\item verb \\
If you \textbf{direct} someone somewhere , you tell them how to get there.
 \textit{
	\begin{itemize}
	\item Could you direct them to Dr Lamont's office, please?
	\item Inside, a guard directed them to the right.
	\end{itemize}
}
\item verb \\
When someone \textbf{directs} a project or a group of people, they are responsible for organizing the people and
activities that are involved.
 \textit{
	\begin{itemize}
	\item Christopher will direct day-to-day operations.
	\item ...his coolness in directing the rescue of nine hostages.
	\end{itemize}
}
\item verb \\
When someone \textbf{directs} a film, play, or television programme, they are responsible for the way in which
it is performed and for telling the actors and assistants what to do.
 \textit{
	\begin{itemize}
	\item He directed various TV shows.
	\item ...her long-held ambition to direct as well as act.
	\end{itemize}
}
\item verb \\
If you \textbf{are directed}  \textbf{to} do something, someone in authority tells you to do it.
 \textit{
	\begin{itemize}
	\item They have been directed to give special attention to the problem of poverty.
	\item The Bishop directed the faithful to stay at home.
	\end{itemize}
}
\item adjective \\
If you are a \textbf{direct}  descendant of someone, you are related to them through your parents and your grandparents and so on.
 \textit{
	\begin{itemize}
	\item She is a direct descendant of Queen Victoria.
	\end{itemize}
}
\end{enumerate}

\section*{fortnight}
{\large \color{blue}  fortnights  }
\subsection*{Explain}
\begin{enumerate}
\item countable noun \\
A \textbf{fortnight} is a period of two weeks.
 \textit{
	\begin{itemize}
	\item I hope to be back in a fortnight.
	\end{itemize}
}
\end{enumerate}

\section*{dry}
{\large \color{blue}  drier  dryer  driest  dryest  dries  drying  dried  }
\subsection*{Explain}
\begin{enumerate}
\item adjective \\
If something is \textbf{dry} , there is no water or moisture on it or in it.
 \textit{
	\begin{itemize}
	\item Clean the metal with a soft dry cloth.
	\item Pat it dry with a soft towel.
	\item Once the paint is dry, apply a coat of the red ochre emulsion paint.
	\item The path was dry and slithery from the drought.
	\end{itemize}
}
\item verb \\
When something \textbf{dries} or when you \textbf{dry} it, it becomes dry.
 \textit{
	\begin{itemize}
	\item The washing might dry outside today, the sun's shining.
	\item Leave your hair to dry naturally whenever possible.
	\item Wash and dry the lettuce.
	\item Liz laughed again, got up from the water and began to dry herself.
	\end{itemize}
}
\item verb \\
When you \textbf{dry} the dishes after a meal , you wipe the water off the plates, cups , knives , pans , and other things when they have been washed , using a cloth.
 \textbf{Dry up} means the same as dry .
 \textit{
	\begin{itemize}
	\item Mrs. Madrigal began drying dishes.
	\item He got up and stood beside Julie, drying up the dishes while she washed.
	\end{itemize}
}
\item adjective \\
If you say that your skin or hair is \textbf{dry} , you mean that it is less oily than, or not as soft as, normal.
 \textit{
	\begin{itemize}
	\item Nothing looks worse than dry, cracked lips.
	\item Dry hair can be damaged by washing it too frequently.
	\item My skin's been getting a little dry recently.
	\end{itemize}
}
\item adjective \\
If the weather or a period of time is \textbf{dry} , there is no rain or there is much less rain than average .
 \textit{
	\begin{itemize}
	\item Exceptionally dry weather over the past year had cut agricultural production.
	\item The spring has been unusually dry, with hardly any rain in May.
	\end{itemize}
}
\item adjective \\
A \textbf{dry} place or climate is one that gets very little rainfall.
 \textit{
	\begin{itemize}
	\item It was one of the driest and dustiest places in Africa.
	\item ...a hot, dry climate where the sun is shining all the time.
	\end{itemize}
}
\item singular noun \\
In \textbf{the dry} means in a place or at a time that is not damp, wet, or rainy .
 \textit{
	\begin{itemize}
	\item Such cars, however, do grip the road well, even in the dry.
	\end{itemize}
}
\item adjective \\
If a river, lake, or well is \textbf{dry} , it is empty of water, usually because of hot weather and lack of rain.
 \textit{
	\begin{itemize}
	\item The aquifer which had once fed the wells was pronounced dry.
	\item The single-engine plane landed at a dry lake in western Arizona.
	\item In the end the Volga's waters will run dry.
	\end{itemize}
}
\item adjective \\
If an oil well is \textbf{dry} , it is no longer producing any oil.
 \textit{
	\begin{itemize}
	\item To harvest oil and gas profitably from the North Sea, it must focus on the exploitation
of small reserves as the big wells run dry.
	\end{itemize}
}
\item graded adjective \\
If you are \textbf{dry} , you need to drink something.
 \textit{
	\begin{itemize}
	\item She was suddenly thirsty and dry.
	\end{itemize}
}
\item adjective \\
If your mouth or throat is \textbf{dry} , it has little or no saliva in it, and so feels very unpleasant , perhaps because you are tense or ill .
 \textit{
	\begin{itemize}
	\item His mouth was dry, he needed a drink.
	\item My throat was dry. I was at a loss for words.
	\end{itemize}
}
\item adjective \\
A \textbf{dry}  cough is one that does not produce any mucus .
 \textit{
	\begin{itemize}
	\end{itemize}
}
\item adjective \\
If someone has \textbf{dry} eyes, there are no tears in their eyes; often used with negatives or in contexts where you are expressing surprise that they are not crying .
 \textit{
	\begin{itemize}
	\item There were few dry eyes in the house when I finished.
	\item She didn't wince and her eyes were dry. Talk about brave. She was unbelievable.
	\end{itemize}
}
\item adjective \\
If a country, state, or city is \textbf{dry} , it has laws or rules which forbid anyone to drink, sell, or buy alcoholic drink.
 \textit{
	\begin{itemize}
	\item Gujurat has been a totally dry state for the past thirty years.
	\end{itemize}
}
\item  \\
 to suck someone dry \textit{
	\begin{itemize}
	\end{itemize}
}
\item adjective \\
\textbf{Dry} humour is very amusing , but in a subtle and clever way.
 \textit{
	\begin{itemize}
	\item Fulton has retained his dry humour.
	\item Mr Brooke is renowned for his dry wit.
	\end{itemize}
}
\item graded adjective \\
If you describe a voice as \textbf{dry} , you mean that it is cold or dull , and does not express any emotions.
 \textit{
	\begin{itemize}
	\item When he crept back to his desk, he heard the dry voice of Father Laurence.
	\end{itemize}
}
\item adjective \\
If you describe something such as a book, play, or activity as \textbf{dry} , you mean that it is dull and uninteresting .
 \textit{
	\begin{itemize}
	\item ...dry, academic phrases.
	\item A lot of the work was very dry and boring in Westminster.
	\end{itemize}
}
\item adjective \\
\textbf{Dry}  bread or toast is plain and not covered with butter or jam.
 \textit{
	\begin{itemize}
	\item For breakfast, they had dry bread and tea.
	\end{itemize}
}
\item adjective \\
\textbf{Dry}  sherry or wine does not have a sweet taste .
 \textit{
	\begin{itemize}
	\item ...a glass of chilled, dry white wine.
	\end{itemize}
}
\end{enumerate}

\section*{hospital}
{\large \color{blue}  hospitals  }
\subsection*{Explain}
\begin{enumerate}
\item variable noun \\
A \textbf{hospital} is a place where people who are ill are looked after by nurses and doctors .
 \textit{
	\begin{itemize}
	\item ...a children's hospital with 120 beds.
	\item A couple of weeks later my mother went into hospital.
	\item He may be able to leave hospital early next week.
	\end{itemize}
}
\end{enumerate}

\section*{each}
{\large \color{blue}  }
\subsection*{Explain}
\begin{enumerate}
\item determiner \\
If you refer to \textbf{each} thing or \textbf{each} person in a group, you are referring to every member of the group and considering
them as individuals.
 \textbf{Each} is also a pronoun.
 \textbf{Each} is also an emphasizing pronoun.
 \textbf{Each} is also an adverb .
 \textbf{Each} is also a quantifier .
 \textit{
	\begin{itemize}
	\item Each book is beautifully illustrated.
	\item Each year, hundreds of animals are killed in this way.
	\item Blend in the eggs, one at a time, beating well after each one.
	\item ...two bedrooms, each with three beds.
	\item She began to consult doctors, and each had a different diagnosis.
	\item We each have different needs and interests.
	\item The children were given one each, handed to them or placed on their plates.
	\item They were selling tickets at six pounds each.
	\item He handed each of them a page of photos.
	\item Each of these exercises takes one or two minutes to do.
	\item There are three main types of cloud, each of which has many variations.
	\end{itemize}
}
\item quantifier \\
If you refer to \textbf{each one}  \textbf{of} the members of a group, you are emphasizing that something applies to every one of
them.
 \textit{
	\begin{itemize}
	\item He picked up forty of these publications and read each one of them.
	\end{itemize}
}
\item  \\
 each and every \textit{
	\begin{itemize}
	\end{itemize}
}
\item  \\
 each other \textit{
	\begin{itemize}
	\end{itemize}
}
\end{enumerate}

\section*{junction}
{\large \color{blue}  junctions  }
\subsection*{Explain}
\begin{enumerate}
\item countable noun \\
A \textbf{junction} is a place where roads or railway lines join.
 \textit{
	\begin{itemize}
	\item Follow the road to a junction and turn left.
	\item Leave the M1 at junction 25.
	\item There's a good British Rail link at Clapham Junction.
	\end{itemize}
}
\end{enumerate}

\section*{few}
{\large \color{blue}  fewer  fewest  }
\subsection*{Explain}
\begin{enumerate}
\item determiner \\
You use \textbf{a few} to indicate that you are talking about a small number of people or things. You can also  say  \textbf{a very few} .
 \textbf{Few} is also a pronoun.
 \textbf{Few} is also a quantifier .
 \textit{
	\begin{itemize}
	\item I gave a dinner party for a few close friends.
	\item We had a few drinks afterwards.
	\item Here are a few more ideas to consider.
	\item She was silent for a few seconds.
	\item Doctors work an average of 90 hours a week, while a few are on call for up to 120
hours.
	\item A strict diet is appropriate for only a few.
	\item There are many ways eggs can be prepared; here are a few of them.
	\item ...a little tea-party I'm giving for a few of the teachers.
	\end{itemize}
}
\item adjective \\
You use \textbf{few} after adjectives and determiners to indicate that you are talking about a small number of things or people.
 \textit{
	\begin{itemize}
	\item The past few weeks of her life had been the most pleasant she could remember.
	\item The leaders are expected to seal the agreement in the next few days.
	\item ...in the last few chapters.
	\item A train would pass through there every few minutes at that time of day.
	\end{itemize}
}
\item determiner \\
You use \textbf{few} to indicate that you are talking about a small number of people or things. You can
use 'so', 'too', and 'very' in front of \textbf{few} .
 \textbf{Few} is also a pronoun.
 \textbf{Few} is also a quantifier.
 \textbf{Few} is also an adjective.
 \textit{
	\begin{itemize}
	\item She had few friends, and was generally not very happy.
	\item Few members planned to vote for him.
	\item Very few firms collect the tax, even when they're required to do so by law.
	\item The trouble is that few want to buy, despite the knockdown prices on offer.
	\item ...a true singing and songwriting talent that few suspected.
	\item Few of the beach houses still had lights on.
	\item Few of the volunteers had military experience.
	\item ...spending her few waking hours in front of the TV.
	\item His memories of his father are few.
	\end{itemize}
}
\item singular noun \\
\textbf{The few} means a small set of people considered as separate from the majority , especially because they share a particular opportunity or quality that the others do not have.
 \textit{
	\begin{itemize}
	\item This should not be an experience for the few.
	\item ...a system built on academic excellence for the few.
	\end{itemize}
}
\item  \\
 as few as \textit{
	\begin{itemize}
	\end{itemize}
}
\item  \\
 few and far between \textit{
	\begin{itemize}
	\end{itemize}
}
\item  \\
 a good few \textit{
	\begin{itemize}
	\end{itemize}
}
\item  \\
 have a few too many \textit{
	\begin{itemize}
	\end{itemize}
}
\item  \\
 no fewer than \textit{
	\begin{itemize}
	\end{itemize}
}
\end{enumerate}

\section*{lad}
{\large \color{blue}  lads  }
\subsection*{Explain}
\begin{enumerate}
\item countable noun \\
A \textbf{lad} is a young man or boy.
 \textit{
	\begin{itemize}
	\item When I was a lad his age I would laugh at the strangest things.
	\item Come along, lad. Time for you to get home.
	\end{itemize}
}
\item plural noun \\
Some men refer to their male friends or colleagues as \textbf{the lads} .
 \textit{
	\begin{itemize}
	\item ...a drink with the lads.
	\item The lads don't join the union because they're frightened of being victimized.
	\end{itemize}
}
\end{enumerate}

\section*{locker}
{\large \color{blue}  lockers  }
\subsection*{Explain}
\begin{enumerate}
\item countable noun \\
A \textbf{locker} is a small metal or wooden  cupboard with a lock, where you can put your personal  possessions , for example in a school, place of work, or sports club .
 \textit{
	\begin{itemize}
	\end{itemize}
}
\end{enumerate}

\section*{following}
{\large \color{blue}  followings  }
\subsection*{Explain}
\begin{enumerate}
\item preposition \\
\textbf{Following} a particular event means after that event.
 \textit{
	\begin{itemize}
	\item In the centuries following Christ's death, Christians genuinely believed the world
was about to end.
	\item Following a day of medical research, the conference focused on educational practices.
	\end{itemize}
}
\item adjective \\
The \textbf{following}  day , week , or year is the day, week, or year after the one you have just mentioned.
 \textit{
	\begin{itemize}
	\item The following day the picture appeared on the front pages of every newspaper in the
world.
	\item We went to dinner the following Monday evening.
	\item The following year she joined the Royal Opera House.
	\end{itemize}
}
\item adjective \\
You use \textbf{following} to refer to something that you are about to mention.
 \textbf{The following} refers to the thing or things that you are about to mention.
 \textit{
	\begin{itemize}
	\item Write down the following information: name of product, type, date purchased and price.
	\item The method of helping such patients is explained in the following chapters.
	\item The following is a paraphrase of what was said.
	\item Check with your doctor if you have any of the following: chest pains, high blood
pressure, or heart disease.
	\end{itemize}
}
\item countable noun \\
A person or organization that has a \textbf{following} has a group of people who support or admire their beliefs or actions.
 \textit{
	\begin{itemize}
	\item Australian rugby league enjoys a huge following in New Zealand.
	\end{itemize}
}
\item adjective \\
If a boat or vehicle has a \textbf{following}  wind , the wind is moving in the same direction as the boat or vehicle.
 \textit{
	\begin{itemize}
	\item The following wind and eastward running tide had given us a very pleasant, lazy sail.
	\end{itemize}
}
\end{enumerate}

\section*{mail}
{\large \color{blue}  mails  mailing  mailed  }
\subsection*{Explain}
\begin{enumerate}
\item singular noun \\
\textbf{The mail} is the public service or system by which letters and parcels are collected and delivered.
 \textit{
	\begin{itemize}
	\item Your check is in the mail.
	\item People had to renew their motor vehicle registrations through the mail.
	\item The firm has offices in several large cities, but does most of its business by mail.
	\end{itemize}
}
\item uncountable noun \\
You can refer to letters and parcels that are delivered to you as \textbf{mail} .
 \textit{
	\begin{itemize}
	\item There was no mail except the usual junk addressed to the occupier.
	\item Nora looked through the mail.
	\end{itemize}
}
\item verb \\
If you \textbf{mail} a letter or parcel to someone, you send it to them by putting it in a post box or taking it to a post office.
 \textit{
	\begin{itemize}
	\item Last year, he mailed the documents to French journalists.
	\item He mailed me the contract.
	\item The Government has already mailed some 18 million households with details of the
public offer.
	\end{itemize}
}
\item verb \\
To \textbf{mail} a message to someone means to send it to them by means of email or a computer network .
 \textbf{Mail} is also a noun .
 \textit{
	\begin{itemize}
	\item ...if a report must be electronically mailed to an office by 9 am the next day.
	\item If you have any problems then send me some mail.
	\end{itemize}
}
\end{enumerate}

\section*{general}
{\large \color{blue}  generals  }
\subsection*{Explain}
\begin{enumerate}
\item countable noun \\
A \textbf{general} is a senior officer in the armed forces, usually in the army .
 \textit{
	\begin{itemize}
	\item He rose through the ranks to become a general.
	\end{itemize}
}
\item adjective \\
If you talk about the \textbf{general}  situation  somewhere or talk about something in \textbf{general}  terms , you are describing the situation as a whole rather than considering its details or exceptions .
 \textit{
	\begin{itemize}
	\item The figures represent a general decline in employment.
	\item ...the general deterioration of English society.
	\end{itemize}
}
\item adjective \\
You use \textbf{general} to describe several items or activities when there are too many of them or when they
are not important enough to mention separately.
 \textit{
	\begin{itemize}
	\item £2,500 for software is soon swallowed up in general costs.
	\item His firm took over the planting and general maintenance of the park last March.
	\end{itemize}
}
\item adjective \\
You use \textbf{general} to describe something that involves or affects most people, or most people in a particular group.
 \textit{
	\begin{itemize}
	\item The project should raise general awareness about bullying.
	\end{itemize}
}
\item adjective \\
If you describe something as \textbf{general} , you mean that it is not restricted to any one thing or area.
 \textit{
	\begin{itemize}
	\item ...a general ache radiating from the back of the neck.
	\item ...a general sense of well-being.
	\item ...raising the level of general physical fitness.
	\end{itemize}
}
\item adjective \\
A \textbf{general} business offers a variety of services or goods rather than just one particular kind .
 \textit{
	\begin{itemize}
	\item They ran the general store and the farm dairy.
	\end{itemize}
}
\item adjective \\
\textbf{General} is used to describe a person's job , usually as part of their title, to indicate that they have complete  responsibility for the administration of an organization or business.
 \textit{
	\begin{itemize}
	\item He joined Sanders Roe, moving on later to become General Manager.
	\end{itemize}
}
\item adjective \\
\textbf{General}  workers do a variety of jobs which require no special  skill or training .
 \textit{
	\begin{itemize}
	\item The farm employed a tractor driver and two general labourers.
	\end{itemize}
}
\item graded adjective \\
\textbf{General} is used to describe a person who has an average amount of knowledge or interest in a particular subject.
 \textit{
	\begin{itemize}
	\item This book is intended for the general reader rather than the student.
	\end{itemize}
}
\item  \\
 in general \textit{
	\begin{itemize}
	\end{itemize}
}
\item  \\
 in general \textit{
	\begin{itemize}
	\end{itemize}
}
\item  \\
 in general \textit{
	\begin{itemize}
	\end{itemize}
}
\end{enumerate}

\section*{master}
{\large \color{blue}  masters  mastering  mastered  }
\subsection*{Explain}
\begin{enumerate}
\item countable noun \\
A servant's \textbf{master} is the man that he or she works for.
 \textit{
	\begin{itemize}
	\item My master ordered me not to deliver the message except in private.
	\end{itemize}
}
\item countable noun \\
A dog's \textbf{master} is the man or boy who owns it.
 \textit{
	\begin{itemize}
	\item The dog yelped excitedly when his master opened a desk drawer and produced his leash.
	\end{itemize}
}
\item countable noun \\
If you say that someone is a \textbf{master} of a particular activity, you mean that they are extremely skilled at it.
 \textbf{Master} is also an adjective .
 \textit{
	\begin{itemize}
	\item She was a master of the English language.
	\item He is a master at blocking progress.
	\item They appear masters in the art of making regulations work their way.
	\item ...a master craftsman.
	\item ...a master criminal.
	\end{itemize}
}
\item  \\
 See also  past master \textit{
	\begin{itemize}
	\end{itemize}
}
\item variable noun \\
If you are \textbf{master} of a situation, you have complete control over it.
 \textit{
	\begin{itemize}
	\item Jackson remained calm and always master of his passions.
	\item He was under no illusions as to who was master in his house.
	\end{itemize}
}
\item verb \\
If you \textbf{master} something, you learn how to do it properly or you succeed in understanding it completely.
 \textit{
	\begin{itemize}
	\item Duff soon mastered the skills of radio production.
	\item Students are expected to master a second language.
	\end{itemize}
}
\item verb \\
If you \textbf{master} a difficult situation, you succeed in controlling it.
 \textit{
	\begin{itemize}
	\item When you have mastered one situation you have to go on to the next.
	\item His genius alone has mastered every crisis.
	\end{itemize}
}
\item countable noun \\
A \textbf{master} is a male teacher.
 \textit{
	\begin{itemize}
	\item Mr Palmer was a retired maths master.
	\end{itemize}
}
\item countable noun \\
A famous male painter of the past is often called a \textbf{master} .
 \textit{
	\begin{itemize}
	\item ...a portrait by the Dutch master, Vincent Van Gogh.
	\end{itemize}
}
\item adjective \\
A \textbf{master} copy of something such as a film or a tape recording is an original copy that can be used to produce other copies.
 \textit{
	\begin{itemize}
	\item Keep one as a master copy for your own reference and circulate the others.
	\end{itemize}
}
\item countable noun \\
The \textbf{master} of a ship that carries passengers or goods is its captain .
 \textit{
	\begin{itemize}
	\item ...the Royal Pacific's master.
	\end{itemize}
}
\item singular noun \\
A \textbf{master's degree} can be referred to as a \textbf{master's} .
 \textit{
	\begin{itemize}
	\item I've a master's in economics.
	\end{itemize}
}
\item countable noun \\
\textbf{Master} is sometimes used by the followers of a male religious teacher or leader as a way
of referring to him or addressing him.
 \textit{
	\begin{itemize}
	\item She believed that she had been selected by the Master to reveal forgotten wisdom.
	\end{itemize}
}
\item title noun \\
In the past, \textbf{Master} was used before a boy's name as a polite way of referring to him or addressing him. Nowadays , \textbf{Master} can be written before a boy's name when addressing a letter to him.
 \textit{
	\begin{itemize}
	\item Nice to see you, Master Simon.
	\end{itemize}
}
\item  \\
 be one's own master \textit{
	\begin{itemize}
	\end{itemize}
}
\end{enumerate}

\section*{human}
{\large \color{blue}  humans  }
\subsection*{Explain}
\begin{enumerate}
\item adjective \\
\textbf{Human} means relating to or concerning people.
 \textit{
	\begin{itemize}
	\item ...the human body.
	\item ...human history.
	\end{itemize}
}
\item countable noun \\
You can refer to people as \textbf{humans} , especially when you are comparing them with animals or machines.
 \textit{
	\begin{itemize}
	\item Its rate of growth was fast–much more like that of an ape than that of a human.
	\end{itemize}
}
\item adjective \\
\textbf{Human}  feelings , weaknesses , or errors are ones that are typical of humans rather than machines.
 \textit{
	\begin{itemize}
	\item ...an ever-growing risk of human error.
	\item We're not perfect. We're only human.
	\end{itemize}
}
\end{enumerate}

\section*{material}
{\large \color{blue}  materials  }
\subsection*{Explain}
\begin{enumerate}
\item variable noun \\
A \textbf{material} is a solid substance.
 \textit{
	\begin{itemize}
	\item ...electrons in a conducting material such as a metal.
	\item ...the design of new absorbent materials.
	\item ...recycling of all materials.
	\end{itemize}
}
\item variable noun \\
\textbf{Material} is cloth.
 \textit{
	\begin{itemize}
	\item ...the thick material of her skirt.
	\item The materials are soft and comfortable to wear.
	\end{itemize}
}
\item plural noun \\
\textbf{Materials} are the things that you need for a particular activity .
 \textit{
	\begin{itemize}
	\item The builders ran out of materials.
	\item ...sewing materials.
	\end{itemize}
}
\item uncountable noun \\
Ideas or information that are used as a basis for a book , play, or film can be referred to as \textbf{material} .
 \textit{
	\begin{itemize}
	\item In my version of the story, I added some new material.
	\item ...the film producer's debt to the author of original screen material.
	\end{itemize}
}
\item adjective \\
\textbf{Material} things are related to possessions or money , rather than to more abstract things such as ideas or values .
 \textit{
	\begin{itemize}
	\item Every room must have been stuffed with material things.
	\item ...the material world.
	\item ...his descriptions of their poor material conditions.
	\end{itemize}
}
\item uncountable noun \\
If you say that someone is a particular kind of \textbf{material} , you mean that they have the qualities or abilities to do a particular job or task .
 \textit{
	\begin{itemize}
	\item She was not university material.
	\item His message has changed little since he became presidential material.
	\end{itemize}
}
\item adjective \\
\textbf{Material}  evidence or information is directly relevant and important in a legal or academic  argument .
 \textit{
	\begin{itemize}
	\item The nature and availability of material evidence was not to be discussed.
	\item They contend that the company failed to disclose material information.
	\end{itemize}
}
\end{enumerate}

\section*{industrial}
{\large \color{blue}  }
\subsection*{Explain}
\begin{enumerate}
\item adjective \\
You use \textbf{industrial} to describe things which relate to or are used in industry.
 \textit{
	\begin{itemize}
	\item ...industrial machinery and equipment.
	\item ...a link between industrial chemicals and cancer.
	\end{itemize}
}
\item adjective \\
An \textbf{industrial} city or country is one in which industry is important or highly  developed .
 \textit{
	\begin{itemize}
	\item ...ministers from leading western industrial countries.
	\end{itemize}
}
\end{enumerate}

\section*{nut}
{\large \color{blue}  nuts  }
\subsection*{Explain}
\begin{enumerate}
\item countable noun \\
The firm shelled fruit of some trees and bushes are called  \textbf{nuts} . Some nuts can be eaten .
 \textit{
	\begin{itemize}
	\item Nuts and seeds are good sources of vitamin E.
	\end{itemize}
}
\item countable noun \\
A \textbf{nut} is a thick metal ring which you screw onto a metal rod called a bolt. Nuts and bolts are used to hold things such as pieces of machinery  together .
 \textit{
	\begin{itemize}
	\item If you want to repair the wheels, you just undo the four nuts.
	\item ...nuts and bolts that haven't been tightened up.
	\end{itemize}
}
\item countable noun \\
If you describe someone as, for example , a football  \textbf{nut} or a health  \textbf{nut} , you mean that they are extremely enthusiastic about the thing mentioned .
 \textit{
	\begin{itemize}
	\item ...a football nut who spends thousands of pounds travelling to watch games.
	\end{itemize}
}
\item adjective \\
If you are \textbf{nuts}  \textbf{about} something or someone, you like them very much.
 \textit{
	\begin{itemize}
	\item They're nuts about the car.
	\item She's nuts about you.
	\end{itemize}
}
\item countable noun \\
If you refer to someone as a \textbf{nut} , you mean that they are mad .
 \textit{
	\begin{itemize}
	\item There's some nut out there with a gun.
	\end{itemize}
}
\item adjective \\
If you say that someone goes  \textbf{nuts} or is \textbf{nuts} , you mean that they go  crazy or are very foolish.
 \textit{
	\begin{itemize}
	\item You guys are nuts.
	\item A number of the French players went nuts, completely out of control.
	\end{itemize}
}
\item plural noun \\
A man's testicles can be referred to as his \textbf{nuts} .
 \textit{
	\begin{itemize}
	\end{itemize}
}
\item countable noun \\
Your head can be referred to as your \textbf{nut} .
 \textit{
	\begin{itemize}
	\end{itemize}
}
\item  \\
 do one's nut/go nuts \textit{
	\begin{itemize}
	\end{itemize}
}
\item  \\
 nuts and bolts \textit{
	\begin{itemize}
	\end{itemize}
}
\item  \\
 a tough nut \textit{
	\begin{itemize}
	\end{itemize}
}
\item  \\
 a hard nut to crack/ a tough nut to crack \textit{
	\begin{itemize}
	\end{itemize}
}
\end{enumerate}

\section*{inferior}
{\large \color{blue}  inferiors  }
\subsection*{Explain}
\begin{enumerate}
\item adjective \\
Something that is \textbf{inferior} is not as good as something else.
 \textit{
	\begin{itemize}
	\item Much of the imported coffee is of inferior quality.
	\item This resulted in overpriced and often inferior products.
	\item If children were made to feel inferior to other children their confidence declined.
	\end{itemize}
}
\item adjective \\
If one person is regarded as \textbf{inferior}  \textbf{to} another, they are regarded as less important because they have less status or ability .
 \textbf{Inferior} is also a noun .
 \textit{
	\begin{itemize}
	\item He preferred the company of those who were intellectually inferior to himself.
	\item A gentleman should always be civil, even to his inferiors.
	\end{itemize}
}
\end{enumerate}

\section*{pace}
{\large \color{blue}  paces  pacing  paced  }
\subsection*{Explain}
\begin{enumerate}
\item singular noun \\
The \textbf{pace} of something is the speed at which it happens or is done.
 \textit{
	\begin{itemize}
	\item Many people were not satisfied with the pace of change.
	\item ...people who prefer to live at a slower pace.
	\item They could not stand the pace or the workload.
	\item Interest rates would come down as the recovery gathered pace.
	\end{itemize}
}
\item singular noun \\
Your \textbf{pace} is the speed at which you walk.
 \textit{
	\begin{itemize}
	\item He moved at a brisk pace down the rue St Antoine.
	\item Their pace quickened as they approached their cars.
	\end{itemize}
}
\item countable noun \\
A \textbf{pace} is the distance that you move when you take one step.
 \textit{
	\begin{itemize}
	\item He'd only gone a few paces before he stopped again.
	\item I took a pace backwards.
	\end{itemize}
}
\item verb \\
If you \textbf{pace} a small area, you keep walking up and down it, because you are anxious or impatient .
 \textit{
	\begin{itemize}
	\item As they waited, Kravis paced the room nervously.
	\item He found John pacing around the flat, unable to sleep.
	\item She stared as he paced and yelled.
	\end{itemize}
}
\item verb \\
If you \textbf{pace}  \textbf{yourself} when doing something, you do it at a steady rate.
 \textit{
	\begin{itemize}
	\item It was a tough race and I had to pace myself.
	\end{itemize}
}
\item  \\
 to keep pace \textit{
	\begin{itemize}
	\end{itemize}
}
\item  \\
 to keep pace \textit{
	\begin{itemize}
	\end{itemize}
}
\item  \\
 at one's own pace \textit{
	\begin{itemize}
	\end{itemize}
}
\item  \\
 put sb/go through their paces \textit{
	\begin{itemize}
	\end{itemize}
}
\end{enumerate}

\section*{infrared}
{\large \color{blue}  }
\subsection*{Explain}
\begin{enumerate}
\item adjective \\
\textbf{Infrared} radiation is similar to light but has a longer wavelength, so we cannot see it without special  equipment .
 \textit{
	\begin{itemize}
	\end{itemize}
}
\item adjective \\
\textbf{Infrared} equipment detects infrared radiation.
 \textit{
	\begin{itemize}
	\item ...searching with infra-red scanners for weapons and artillery.
	\end{itemize}
}
\end{enumerate}

\section*{pedestrian}
{\large \color{blue}  pedestrians  }
\subsection*{Explain}
\begin{enumerate}
\item countable noun \\
A \textbf{pedestrian} is a person who is walking , especially in a town or city, rather than travelling in a vehicle.
 \textit{
	\begin{itemize}
	\item In Los Angeles a pedestrian is a rare spectacle.
	\item More than a third of all pedestrian injuries are to children.
	\end{itemize}
}
\item adjective \\
If you describe something as \textbf{pedestrian} , you mean that it is ordinary and not at all interesting.
 \textit{
	\begin{itemize}
	\item His style is so pedestrian that the book becomes a real bore.
	\item I drove home contemplating my own more pedestrian lifestyle.
	\end{itemize}
}
\end{enumerate}

\section*{inland}
{\large \color{blue}  }
\subsection*{Explain}
\begin{enumerate}
\item adverb \\
If something is situated  \textbf{inland} , it is away from the coast , towards or near the middle of a country. If you go  \textbf{inland} , you go away from the coast, towards the middle of a country.
 \textit{
	\begin{itemize}
	\item The vast majority live further inland.
	\item It's about 15 minutes' drive inland from Cannes.
	\item The car turned away from the coast and headed inland.
	\end{itemize}
}
\item adjective \\
\textbf{Inland} areas, lakes , and places are not on the coast, but in or near the middle of a country.
 \textit{
	\begin{itemize}
	\item ...a rather quiet inland town.
	\end{itemize}
}
\end{enumerate}

\section*{period}
{\large \color{blue}  periods  }
\subsection*{Explain}
\begin{enumerate}
\item countable noun \\
A \textbf{period} is a length of time.
 \textit{
	\begin{itemize}
	\item This crisis might last for a long period of time.
	\item ...a period of a few months.
	\item ...for a limited period only.
	\end{itemize}
}
\item countable noun \\
A \textbf{period} in the life of a person, organization, or society is a length of time which is remembered for a particular situation or activity.
 \textit{
	\begin{itemize}
	\item ...a period of economic good health and expansion.
	\item He went through a period of wanting to be accepted.
	\item The South African years were his most creative period.
	\end{itemize}
}
\item countable noun \\
A particular length of time in history is sometimes called a \textbf{period} . For example, you can talk about \textbf{the Victorian period} or \textbf{the Elizabethan period} in Britain.
 \textit{
	\begin{itemize}
	\item ...the Roman period.
	\item No reference to their existence appears in any literature of the period.
	\item ...the most difficult periods of history.
	\end{itemize}
}
\item adjective \\
\textbf{Period}  costumes , furniture , and instruments were made at an earlier time in history, or look as if they were
made then.
 \textit{
	\begin{itemize}
	\item ...dressed in full period costume.
	\item ...replicas of period instruments.
	\end{itemize}
}
\item countable noun \\
Exercise , training, or study \textbf{periods} are lengths of time that are set aside for exercise, training, or study.
 \textit{
	\begin{itemize}
	\item They accompanied him during his exercise periods.
	\end{itemize}
}
\item countable noun \\
At a school or college , a \textbf{period} is one of the parts that the day is divided into during which lessons or other activities take place.
 \textit{
	\begin{itemize}
	\item ...periods of private study.
	\item ...taking his scripts to school in order to learn the lines in free periods.
	\end{itemize}
}
\item countable noun \\
When a woman has a \textbf{period} , she bleeds from her womb . This usually happens once a month , unless she is pregnant .
 \textit{
	\begin{itemize}
	\end{itemize}
}
\item adverb \\
Some people say  \textbf{period} after stating a fact or opinion when they want to emphasize that they are definite about something and do not want to discuss it further.
 \textit{
	\begin{itemize}
	\item I don't want to do it, period.
	\end{itemize}
}
\item countable noun \\
A \textbf{period} is the punctuation mark (.) which you use at the end of a sentence when it is not
a question or an exclamation.
 \textit{
	\begin{itemize}
	\end{itemize}
}
\end{enumerate}

\section*{junior}
{\large \color{blue}  juniors  }
\subsection*{Explain}
\begin{enumerate}
\item adjective \\
A \textbf{junior}  official or employee  holds a low-ranking position in an organization or profession .
 \textbf{Junior} is also a noun .
 \textit{
	\begin{itemize}
	\item Junior and middle-ranking civil servants have pledged to join the indefinite strike.
	\item ...a junior minister attached to the prime minister's office.
	\item The Lord Chancellor has said legal aid work is for juniors when they start out in
the law.
	\end{itemize}
}
\item singular noun \\
If you are someone's \textbf{junior} , you are younger than they are.
 \textit{
	\begin{itemize}
	\item She now lives with actor Denis Lawson, 10 years her junior.
	\end{itemize}
}
\item countable noun \\
\textbf{Junior} is sometimes used after the name of the younger of two men in a family who have the same name, sometimes in order to prevent  confusion . The abbreviation  Jr is also used.
 \textit{
	\begin{itemize}
	\item His son, Arthur Ochs Junior, is expected to succeed him as publisher.
	\end{itemize}
}
\item countable noun \\
In the United  States , a student in the third year of a high school or university course is called a \textbf{junior} .
 \textit{
	\begin{itemize}
	\item Their youngest daughter Amy's a junior at the University of Evansville in Indiana.
	\item It was the summer before his junior year in high school.
	\end{itemize}
}
\end{enumerate}

\section*{portrait}
{\large \color{blue}  portraits  }
\subsection*{Explain}
\begin{enumerate}
\item countable noun \\
A \textbf{portrait} is a painting, drawing, or photograph of a particular person.
 \textit{
	\begin{itemize}
	\item The artist was asked to paint a portrait of the Queen.
	\item ...the English portrait painter Augustus John.
	\end{itemize}
}
\item countable noun \\
A \textbf{portrait} of a person, place, or thing is a verbal description of them.
 \textit{
	\begin{itemize}
	\item ...this gripping, funny portrait of Jewish life in 1950s London.
	\end{itemize}
}
\end{enumerate}

\section*{large}
{\large \color{blue}  larger  largest  }
\subsection*{Explain}
\begin{enumerate}
\item adjective \\
A \textbf{large} thing or person is greater in size than usual or average .
 \textit{
	\begin{itemize}
	\item The Pike lives mainly in large rivers and lakes.
	\item In the largest room about a dozen children and seven adults are sitting on the carpet.
	\item He was a large man with thick dark hair.
	\end{itemize}
}
\item adjective \\
A \textbf{large} amount or number of people or things is more than the average amount or number.
 \textit{
	\begin{itemize}
	\item The gang finally fled with a large amount of cash and jewellery.
	\item There are a large number of centres where you can take full-time courses.
	\item The figures involved are truly very large.
	\end{itemize}
}
\item adjective \\
A \textbf{large} organization or business does a lot of work or commercial activity and employs a lot of people.
 \textit{
	\begin{itemize}
	\item ...a large company in Chicago.
	\item Many large organizations run courses for their employees.
	\end{itemize}
}
\item adjective \\
\textbf{Large} is used to indicate that a problem or issue which is being discussed is very important or serious .
 \textit{
	\begin{itemize}
	\item ...the already large problem of under-age drinking.
	\item There's a very large question about the viability of the newspaper.
	\end{itemize}
}
\item  \\
 at large \textit{
	\begin{itemize}
	\end{itemize}
}
\item  \\
 at large \textit{
	\begin{itemize}
	\end{itemize}
}
\item  \\
 by and large \textit{
	\begin{itemize}
	\end{itemize}
}
\end{enumerate}

\section*{postage}
{\large \color{blue}  }
\subsection*{Explain}
\begin{enumerate}
\item uncountable noun \\
\textbf{Postage} is the money that you pay for sending  letters and packages by post .
 \textit{
	\begin{itemize}
	\end{itemize}
}
\end{enumerate}

\section*{little}
{\large \color{blue}  }
\subsection*{Explain}
\begin{enumerate}
\item determiner \\
You use \textbf{little} to indicate that there is only a very small amount of something. You can use 'so', 'too', and
'very' in front of \textbf{little} .
 \textbf{Little} is also a quantifier .
 \textbf{Little} is also a pronoun.
 \textit{
	\begin{itemize}
	\item I had little money and little free time.
	\item I find that I need very little sleep these days.
	\item There is little doubt that a diet high in fibre is more satisfying.
	\item So far little progress has been made towards ending the fighting.
	\item The pudding is quick and easy and needs little attention once in the oven.
	\item Little of the existing housing is of good enough quality.
	\item Little of the money gets through to the children who need it.
	\item He ate little, and drank less.
	\item In general, employers do little to help the single working mother.
	\item Little is known about his childhood.
	\end{itemize}
}
\item adverb \\
\textbf{Little} means not very often or to only a small extent.
 \textit{
	\begin{itemize}
	\item On their way back to Marseille they spoke very little.
	\item The animals were covered in dust, but otherwise little affected.
	\end{itemize}
}
\item determiner \\
\textbf{A little} of something is a small amount of it, but not very much. You can also say  \textbf{a very little} .
 \textbf{Little} is also a pronoun.
 \textbf{Little} is also a quantifier.
 \textit{
	\begin{itemize}
	\item Mrs Caan needs a little help getting her groceries home.
	\item A little food would do us all some good.
	\item ...a little light reading.
	\item I shall be only a very little time.
	\item They get paid for it. Not much. Just a little.
	\item Pour a little of the sauce over the chicken.
	\item I'm sure she won't mind sparing us a little of her time.
	\end{itemize}
}
\item adverb \\
If you do something \textbf{a little} , you do it for a short time.
 \textit{
	\begin{itemize}
	\item He walked a little by himself in the garden.
	\end{itemize}
}
\item adverb \\
\textbf{A little} or \textbf{a little bit} means to a small extent or degree.
 \textit{
	\begin{itemize}
	\item He complained a little of a nagging pain between his shoulder blades.
	\item He was a little bit afraid of his father's reaction.
	\item If you have to drive when you are tired, go a little more slowly than you would normally.
	\item He wanted to have someone to whom he could talk a little about himself.
	\end{itemize}
}
\item  \\
 little by little \textit{
	\begin{itemize}
	\end{itemize}
}
\end{enumerate}

\section*{postman}
{\large \color{blue}  postmen  }
\subsection*{Explain}
\begin{enumerate}
\item countable noun \\
A \textbf{postman} is a man whose job is to collect and deliver letters and packages that are sent by post .
 \textit{
	\begin{itemize}
	\end{itemize}
}
\end{enumerate}

\section*{loud}
{\large \color{blue}  louder  loudest  }
\subsection*{Explain}
\begin{enumerate}
\item adjective \\
If a noise is \textbf{loud} , the level of sound is very high and it can be easily  heard . Someone or something that is \textbf{loud} produces a lot of noise.
 \textbf{Loud} is also an adverb .
 \textit{
	\begin{itemize}
	\item Suddenly there was a loud bang.
	\item His voice became harsh and loud.
	\item The band was starting to play a fast, loud number.
	\item ...amazingly loud discos.
	\item She wonders whether Paul's hearing is OK because he turns the television up very
loud.
	\end{itemize}
}
\item adjective \\
If someone is \textbf{loud} in their support for or criticism of something, they express their opinion very often and in a very strong way.
 \textit{
	\begin{itemize}
	\item Mr Adams' speech yesterday was very loud in condemnation of the media.
	\item Mr Jones received loud support from his local community.
	\end{itemize}
}
\item adjective \\
If you describe something, especially a piece of clothing , as \textbf{loud} , you dislike it because it has very bright colours or very large, bold  patterns which look unpleasant .
 \textit{
	\begin{itemize}
	\item He liked to shock with his gold chains and loud clothes.
	\item I once paid £120 for an extremely loud shirt which I've yet to wear.
	\end{itemize}
}
\item  \\
 loud and clear \textit{
	\begin{itemize}
	\end{itemize}
}
\item  \\
 out loud \textit{
	\begin{itemize}
	\end{itemize}
}
\end{enumerate}

\section*{resolution}
{\large \color{blue}  resolutions  }
\subsection*{Explain}
\begin{enumerate}
\item countable noun \\
A \textbf{resolution} is a formal decision taken at a meeting by means of a vote.
 \textit{
	\begin{itemize}
	\item He replied that the U.N. had passed two major resolutions calling for a complete
withdrawal.
	\item ...a draft resolution on the occupied territories.
	\end{itemize}
}
\item countable noun \\
If you make a \textbf{resolution} , you decide to try very hard to do something.
 \textit{
	\begin{itemize}
	\item They made a resolution to lose all the weight gained during the Christmas period.
	\end{itemize}
}
\item uncountable noun \\
\textbf{Resolution} is determination to do something or not do something.
 \textit{
	\begin{itemize}
	\item 'I think I'll try a hypnotist,' I said with sudden resolution.
	\end{itemize}
}
\item singular noun \\
The \textbf{resolution} of a problem or difficulty is the final  solving of it.
 \textit{
	\begin{itemize}
	\item ...the successful resolution of a dispute.
	\item ...in order to find a peaceful resolution to the crisis.
	\end{itemize}
}
\item uncountable noun \\
The \textbf{resolution} of an image is how clear the image is.
 \textit{
	\begin{itemize}
	\item Now this machine gives us such high resolution that we can see very small specks
of calcium.
	\end{itemize}
}
\end{enumerate}

\section*{massive}
{\large \color{blue}  }
\subsection*{Explain}
\begin{enumerate}
\item adjective \\
Something that is \textbf{massive} is very large in size , quantity , or extent .
 \textit{
	\begin{itemize}
	\item There was evidence of massive fraud.
	\item ...massive air attacks.
	\item The scale of the problem is massive.
	\item ...a massive steam boat.
	\end{itemize}
}
\item adjective \\
If you describe a medical condition as \textbf{massive} , you mean that it is extremely  serious .
 \textit{
	\begin{itemize}
	\item He died six weeks later of a massive heart attack.
	\end{itemize}
}
\end{enumerate}

\section*{rifle}
{\large \color{blue}  rifles  rifling  rifled  }
\subsection*{Explain}
\begin{enumerate}
\item countable noun \\
A \textbf{rifle} is a gun with a long barrel.
 \textit{
	\begin{itemize}
	\item They shot him at point blank range with an automatic rifle.
	\item Neighbours heard the sound of rifle fire and alerted the police.
	\end{itemize}
}
\item verb \\
If you \textbf{rifle}  \textbf{through} things or \textbf{rifle} them, you make a quick search among them in order to find something or steal something.
 \textit{
	\begin{itemize}
	\item I discovered him rifling through the filing cabinet.
	\item The men rifled through his clothing and snatched the wallet.
	\item There were lockers by each seat and I quickly rifled the contents.
	\end{itemize}
}
\end{enumerate}

\section*{petty}
{\large \color{blue}  pettier  pettiest  }
\subsection*{Explain}
\begin{enumerate}
\item adjective \\
You can use \textbf{petty} to describe things such as problems , rules, or arguments which you think are unimportant or relate to unimportant things.
 \textit{
	\begin{itemize}
	\item He was miserable all the time and rows would start over petty things.
	\item ...endless rules and petty regulations.
	\item The meeting degenerated into petty squabbling.
	\end{itemize}
}
\item adjective \\
If you describe someone's behaviour as \textbf{petty} , you mean that they care too much about small, unimportant things and perhaps that they are unnecessarily unkind .
 \textit{
	\begin{itemize}
	\item He was petty-minded and obsessed with detail.
	\item I think that attitude is a bit petty.
	\end{itemize}
}
\item adjective \\
\textbf{Petty} is used of people or actions that are less important , serious , or great than others.
 \textit{
	\begin{itemize}
	\item Wilson was not a man who dealt with petty officials.
	\item ...petty crime, such as handbag-snatching and minor break-ins.
	\end{itemize}
}
\end{enumerate}

\section*{scarf}
{\large \color{blue}  scarfs  scarves  }
\subsection*{Explain}
\begin{enumerate}
\item countable noun \\
A \textbf{scarf} is a piece of cloth that you wear round your neck or head, usually to keep yourself warm .
 \textit{
	\begin{itemize}
	\item He reached up to loosen the scarf around his neck.
	\end{itemize}
}
\end{enumerate}

\section*{several}
{\large \color{blue}  }
\subsection*{Explain}
\begin{enumerate}
\item determiner \\
\textbf{Several} is used to refer to an imprecise number of people or things that is not large but is greater than two.
 \textbf{Several} is also a quantifier .
 \textbf{Several} is also a pronoun.
 \textit{
	\begin{itemize}
	\item I had lived two doors away from this family for several years.
	\item Several blue plastic boxes under the window were filled with record albums.
	\item Several hundred students gathered on campus.
	\item Several of the delays were caused by the new high-tech baggage system.
	\item According to several of their friends, their 25-year marriage has suffered some difficulties.
	\item No one drug will suit or work for everyone and sometimes several may have to be tried.
	\item Ben's case is not unique but one of several I have come up against during the past
few years.
	\end{itemize}
}
\end{enumerate}

\section*{schedule}
{\large \color{blue}  schedules  scheduling  scheduled  }
\subsection*{Explain}
\begin{enumerate}
\item countable noun \\
A \textbf{schedule} is a plan that gives a list of events or tasks and the times at which each one should happen or be done.
 \textit{
	\begin{itemize}
	\item He has been forced to adjust his schedule.
	\item We both have such hectic schedules.
	\end{itemize}
}
\item uncountable noun \\
You can use \textbf{schedule} to refer to the time or way something is planned to be done. For example , if something is completed  \textbf{on schedule} , it is completed at the time planned.
 \textit{
	\begin{itemize}
	\item The jet arrived in Johannesburg two minutes ahead of schedule.
	\item Everything went according to schedule.
	\item It will be completed several weeks behind schedule.
	\end{itemize}
}
\item verb \\
If something \textbf{is scheduled} to happen at a particular time, arrangements are made for it to happen at that time.
 \textit{
	\begin{itemize}
	\item The space shuttle had been scheduled to blast off at 04:38.
	\item A presidential election was scheduled for last December.
	\item No new talks are scheduled.
	\end{itemize}
}
\item countable noun \\
A \textbf{schedule} is a written list of things, for example a list of prices , details , or conditions .
 \textit{
	\begin{itemize}
	\end{itemize}
}
\item countable noun \\
A \textbf{schedule} is a list of all the times when trains , boats , buses , or aircraft are supposed to arrive at or leave a particular place.
 \textit{
	\begin{itemize}
	\item ...a bus schedule.
	\end{itemize}
}
\item countable noun \\
In a school or college , a \textbf{schedule} is a diagram that shows the times in the week at which particular subjects are taught .
 \textit{
	\begin{itemize}
	\end{itemize}
}
\end{enumerate}

\section*{small}
{\large \color{blue}  smaller  smallest  }
\subsection*{Explain}
\begin{enumerate}
\item adjective \\
A \textbf{small} person, thing, or amount of something is not large in physical size.
 \textit{
	\begin{itemize}
	\item She is small for her age.
	\item The window was far too small for him to get through.
	\item Next door to the garage is a small orchard area.
	\item Stick them on using a small amount of glue.
	\end{itemize}
}
\item adjective \\
A \textbf{small} group or quantity consists of only a few people or things.
 \textit{
	\begin{itemize}
	\item A small group of students meets regularly to learn Japanese.
	\item Guns continued to be produced in small numbers.
	\end{itemize}
}
\item adjective \\
A \textbf{small} child is a very young child.
 \textit{
	\begin{itemize}
	\item I have a wife and two small children.
	\item What were you like when you were small?
	\end{itemize}
}
\item adjective \\
You use \textbf{small} to describe something that is not significant or great in degree .
 \textit{
	\begin{itemize}
	\item It's quite easy to make quite small changes to the way that you work.
	\item No detail was too small to escape her attention.
	\item He believes this to be a relatively small problem.
	\end{itemize}
}
\item adjective \\
\textbf{Small} businesses or companies employ a small number of people and do business with a small number of clients .
 \textit{
	\begin{itemize}
	\item ...shops, restaurants and other small businesses.
	\item Tool companies here are generally small.
	\end{itemize}
}
\item graded adjective \\
If someone speaks in a \textbf{small}  voice , they speak in a quiet , high voice because they are frightened or ashamed .
 \textit{
	\begin{itemize}
	\item 'I'm scared,' she said in a very small voice.
	\end{itemize}
}
\item adjective \\
If someone makes you look or feel  \textbf{small} , they make you look or feel stupid or ashamed.
 \textit{
	\begin{itemize}
	\item This may just be another of her schemes to make me look small.
	\item When your children misbehave tell them without making them feel small.
	\end{itemize}
}
\item singular noun \\
\textbf{The small of} your \textbf{back} is the bottom part of your back that curves in slightly .
 \textit{
	\begin{itemize}
	\item Place your hands on the small of your back and breathe in.
	\end{itemize}
}
\end{enumerate}

\section*{serial}
{\large \color{blue}  serials  }
\subsection*{Explain}
\begin{enumerate}
\item countable noun \\
A \textbf{serial} is a story which is broadcast on television or radio or is published in a magazine or newspaper in a number of parts over a period of time.
 \textit{
	\begin{itemize}
	\item ...one of BBC television's most popular serials, Eastenders.
	\item Maupin's novels have all appeared originally as serials.
	\end{itemize}
}
\item adjective \\
\textbf{Serial}  killings or attacks are a series of killings or attacks committed by the same person. This person is known as a \textbf{serial}  killer or attacker.
 \textit{
	\begin{itemize}
	\item ...serial murders.
	\item The serial killer claimed to have killed 400 people.
	\end{itemize}
}
\end{enumerate}

\section*{soccer}
{\large \color{blue}  }
\subsection*{Explain}
\begin{enumerate}
\item uncountable noun \\
\textbf{Soccer} is a game played by two teams of eleven players using a round ball. Players kick the ball to each other and try to score goals by kicking the ball into a large net . Outside the USA, this game is also  referred to as football .
 \textit{
	\begin{itemize}
	\end{itemize}
}
\end{enumerate}

\section*{swift}
{\large \color{blue}  swifter  swiftest  swifts  }
\subsection*{Explain}
\begin{enumerate}
\item adjective \\
A \textbf{swift} event or process happens very quickly or without delay .
 \textit{
	\begin{itemize}
	\item Our task is to challenge the U.N. to make a swift decision.
	\item The police were swift to act.
	\end{itemize}
}
\item adjective \\
Something that is \textbf{swift} moves very quickly.
 \textit{
	\begin{itemize}
	\item With a swift movement, Matthew Jerrold sat upright.
	\end{itemize}
}
\item countable noun \\
A \textbf{swift} is a small bird with long curved wings.
 \textit{
	\begin{itemize}
	\end{itemize}
}
\end{enumerate}

\section*{stamp}
{\large \color{blue}  stamps  stamping  stamped  }
\subsection*{Explain}
\begin{enumerate}
\item countable noun \\
A \textbf{stamp} or a \textbf{postage stamp} is a small piece of paper which you stick on an envelope or package before you post it to pay for the cost of the postage.
 \textit{
	\begin{itemize}
	\item ...a book of stamps.
	\item It's FREEPOST, so there's no need for a stamp.
	\item ...two first class stamps.
	\end{itemize}
}
\item countable noun \\
A \textbf{stamp} is a small block of wood or metal which has a pattern or a group of letters on one side. You press it onto an pad of ink and then onto a piece of paper in order to produce a mark on the paper. The mark
that you produce is also  called a \textbf{stamp} .
 \textit{
	\begin{itemize}
	\item ...a date stamp and an ink pad.
	\item You may live only where the stamp in your passport says you may.
	\end{itemize}
}
\item verb \\
If you \textbf{stamp} a mark or word on an object, you press the mark or word onto the object using a stamp
or other device.
 \textit{
	\begin{itemize}
	\item Car manufacturers stamp a vehicle identification number in several places.
	\item When a gift voucher is exchanged it's stamped with the details of the store.
	\item 'Eat before JULY 14' was stamped on the label.
	\end{itemize}
}
\item verb \\
If you \textbf{stamp} or \textbf{stamp} your \textbf{foot} , you lift your foot and put it down very hard on the ground, for example because you are angry or because your feet are cold .
 \textbf{Stamp} is also a noun .
 \textit{
	\begin{itemize}
	\item Often he teased me till my temper went and I stamped and screamed.
	\item His foot stamped down on the accelerator.
	\item She stamped her feet on the pavement to keep out the cold.
	\item ...hearing the creak of a door and the stamp of cold feet.
	\end{itemize}
}
\item verb \\
If you \textbf{stamp}  somewhere , you walk there putting your feet down very hard on the ground because you are angry.
 \textit{
	\begin{itemize}
	\item 'I'm going before things get any worse!' he shouted as he stamped out of the bedroom.
	\item Overweight and sweating in the humid weather, she stamped from room to room.
	\end{itemize}
}
\item verb \\
If you \textbf{stamp on} something, you put your foot down on it very hard.
 \textit{
	\begin{itemize}
	\item He received the original ban last week after stamping on the referee's foot during
the supercup final.
	\end{itemize}
}
\item singular noun \\
If something bears \textbf{the}  \textbf{stamp}  \textbf{of} a particular quality or person, it clearly has that quality or was done by that person.
 \textit{
	\begin{itemize}
	\item ...lawns and flowerbeds that bore the stamp of years of confident care.
	\item Most of us want to put the stamp of our personality on our home.
	\end{itemize}
}
\item verb \\
A quality, feature, or action that \textbf{stamps} someone or something \textbf{as} a particular thing shows clearly that they are this thing.
 \textit{
	\begin{itemize}
	\item Chris Boardman stamped himself as the 4,000m favourite by setting the world's fastest
outdoor time in Barcelona last night.
	\end{itemize}
}
\end{enumerate}

\section*{that}
{\large \color{blue}  }
\subsection*{Explain}
\begin{enumerate}
\item pronoun \\
You use \textbf{that} to refer back to an idea or situation expressed in a previous sentence or sentences.
 \textbf{That} is also a determiner .
 \textit{
	\begin{itemize}
	\item They said you particularly wanted to talk to me. Why was that?
	\item 'Hey, is there anything the matter with my sisters?'—'Is that why you're phoning?'.
	\item Some feared Germany might raise its interest rates. That could have set the scene
for a confrontation with the U.S.
	\item Their main aim is to support you when making a claim for medical treatment. For that
reason the claims procedure is as simple and helpful as possible.
	\end{itemize}
}
\item determiner \\
You use \textbf{that} to refer to someone or something already mentioned.
 \textit{
	\begin{itemize}
	\item The Commissioners get between £50,000 and £60,000 a year in various allowances. But
that amount can soar to £90,000 a year.
	\item The biggest increase was on the cheapest model, the CRX-HF. That car had a 1990 base
price of $9,145.
	\end{itemize}
}
\item determiner \\
When you have been talking about a particular period of time, you use \textbf{that} to indicate that you are still referring to the same period. You use expressions such as \textbf{that morning} or \textbf{that afternoon} to indicate that you are referring to an earlier period of the same day.
 \textit{
	\begin{itemize}
	\item The story was published in a Sunday newspaper later that week.
	\item That morning I had put on a pair of black slacks and a long-sleeved black blouse.
	\end{itemize}
}
\item pronoun \\
You use \textbf{that} in expressions such as \textbf{that of} and \textbf{that which} to introduce more information about something already mentioned, instead of repeating the noun which refers to it.
 \textit{
	\begin{itemize}
	\item The cool air and green light made the atmosphere curiously like that of an aquarium.
	\item Pollution falls into two categories, that which we can see or smell, and that which
is invisible and odourless.
	\end{itemize}
}
\item pronoun \\
You use \textbf{that} in front of words or expressions which express agreement , responses , or reactions to what has just been said .
 \textit{
	\begin{itemize}
	\item 'She said she'd met you in England.'—'That's true.'
	\item 'I've never been to Paris.'—'That's a pity. You should go one day.'
	\end{itemize}
}
\item determiner \\
You use \textbf{that} to introduce a person or thing that you are going to give details or information about.
 \textbf{That which} is used to introduce a subject in very general terms.
 \textit{
	\begin{itemize}
	\item In my case I chose that course which I considered right.
	\item That person who violates the law and discriminates should suffer in his career.
	\item Too much time is spent worrying over that which one can't change.
	\end{itemize}
}
\item determiner \\
You use \textbf{that} when you are referring to someone or something which is a distance away from you in position or time, especially when you indicate or point to them. When there are two or more things near you, \textbf{that} refers to the more distant one.
 \textbf{That} is also a pronoun.
 \textit{
	\begin{itemize}
	\item Look at that guy. He's got red socks.
	\item Where did you get that hat?
	\item You see that man over there, that man who has just walked into the room?
	\item Leo, what's that you're writing?
	\item That looks heavy. May I carry it for you?
	\end{itemize}
}
\item pronoun \\
You use \textbf{that} when you are identifying someone or asking about their identity .
 \textit{
	\begin{itemize}
	\item That's my wife you were talking to.
	\item That's John Martin, operations chief for the company.
	\item 'Who's that with you?'—'A friend of mine.'.
	\item I answered the phone and this voice went, 'Hello? Is that Alison?'
	\end{itemize}
}
\item determiner \\
You can use \textbf{that} when you expect the person you are talking to to know what or who you are referring to, without needing to identify the particular person or thing fully .
 \textbf{That} is also a pronoun.
 \textit{
	\begin{itemize}
	\item I really thought I was something when I wore that hat and my patent leather shoes.
	\item Did you get that cheque I sent?
	\item That idiot porter again knocked on my door!
	\item That was a terrible case of blackmail in the paper today.
	\item That was a good year, wasn't it?
	\end{itemize}
}
\item adverb \\
If something is \textbf{not}  \textbf{that}  bad , funny , or expensive for example , it is not as bad, funny, or expensive as it might be or as has been suggested .
 \textit{
	\begin{itemize}
	\item Not even Gary, he said, was that stupid.
	\item It isn't that funny.
	\item He didn't look that bad.
	\item Kids don't change that fast.
	\end{itemize}
}
\item adverb \\
You can use \textbf{that} to emphasize the degree of a feeling or quality.
 \textit{
	\begin{itemize}
	\item I would have walked out, I was that angry.
	\item Do I look that stupid?
	\item They actually moved down from upstairs because the rent's that expensive.
	\end{itemize}
}
\item  \\
 and that/and all that \textit{
	\begin{itemize}
	\end{itemize}
}
\item  \\
 at that \textit{
	\begin{itemize}
	\end{itemize}
}
\item  \\
 that is/that is to say \textit{
	\begin{itemize}
	\end{itemize}
}
\item  \\
 that is it \textit{
	\begin{itemize}
	\end{itemize}
}
\item  \\
 that's it \textit{
	\begin{itemize}
	\end{itemize}
}
\item  \\
 just like that \textit{
	\begin{itemize}
	\end{itemize}
}
\item  \\
 that is that \textit{
	\begin{itemize}
	\end{itemize}
}
\end{enumerate}

\section*{ultraviolet}
{\large \color{blue}  }
\subsection*{Explain}
\begin{enumerate}
\item adjective \\
\textbf{Ultraviolet} light or radiation is what causes your skin to become darker in colour after you have been in sunlight . In large amounts ultraviolet light is harmful .
 \textit{
	\begin{itemize}
	\item The sun's ultraviolet rays are responsible for both tanning and burning.
	\end{itemize}
}
\end{enumerate}

\section*{succession}
{\large \color{blue}  successions  }
\subsection*{Explain}
\begin{enumerate}
\item singular noun \\
A \textbf{succession}  \textbf{of} things of the same kind is a number of them that exist or happen one after the other.
 \textit{
	\begin{itemize}
	\item Adams took a succession of jobs which have stood him in good stead.
	\item Scoring three goals in quick succession, he made it 10-8.
	\item She has won the championship for the third year in succession.
	\end{itemize}
}
\item uncountable noun \\
\textbf{Succession} is the fact or right of being the next person to have an important  job or position.
 \textit{
	\begin{itemize}
	\item She is now seventh in line of succession to the throne.
	\end{itemize}
}
\end{enumerate}

\section*{upper}
{\large \color{blue}  uppers  }
\subsection*{Explain}
\begin{enumerate}
\item adjective \\
You use \textbf{upper} to describe something that is above something else.
 \textit{
	\begin{itemize}
	\item There is a smart restaurant on the upper floor.
	\item Students travel on the cheap lower deck and tourists on the upper.
	\end{itemize}
}
\item adjective \\
You use \textbf{upper} to describe the higher part of something.
 \textit{
	\begin{itemize}
	\item ...the upper part of the foot.
	\item ...the muscles of the upper back and chest.
	\item ...the upper rungs of the ladder.
	\end{itemize}
}
\item  \\
 the upper hand \textit{
	\begin{itemize}
	\end{itemize}
}
\item countable noun \\
The \textbf{upper} of a shoe is the top part of it, which is attached to the sole and the heel .
 \textit{
	\begin{itemize}
	\item Wear well-fitting, lace-up shoes with soft uppers.
	\item Leather uppers allow the feet to breathe.
	\end{itemize}
}
\item countable noun \\
\textbf{Uppers} are drugs that make you feel very happy , excited , and full of energy .
 \textit{
	\begin{itemize}
	\item ...people crazy on uppers and downers.
	\item I'd taken a handful of uppers.
	\end{itemize}
}
\end{enumerate}

\section*{tense}
{\large \color{blue}  tenser  tensest  tenses  tensing  tensed  }
\subsection*{Explain}
\begin{enumerate}
\item adjective \\
A \textbf{tense}  situation or period of time is one that makes people anxious , because they do not know what is going to happen  next .
 \textit{
	\begin{itemize}
	\item This gesture of goodwill did little to improve the tense atmosphere at the talks.
	\item After three very tense weeks he phoned again.
	\item There was a tense silence.
	\end{itemize}
}
\item adjective \\
If you are \textbf{tense} , you are anxious and nervous and cannot relax .
 \textit{
	\begin{itemize}
	\item Dart, who had at first been very tense, at last relaxed.
	\end{itemize}
}
\item adjective \\
If your body is \textbf{tense} , your muscles are tight and not relaxed.
 \textit{
	\begin{itemize}
	\item She lay, eyes shut, body tense.
	\item A bath can relax tense muscles.
	\end{itemize}
}
\item verb \\
If your muscles \textbf{tense} , if you \textbf{tense} , or if you \textbf{tense} your muscles, your muscles become tight and stiff , often because you are anxious or frightened .
 \textbf{Tense up}  means the same as tense .
 \textit{
	\begin{itemize}
	\item Newman's stomach muscles tensed.
	\item He tensed as the big guy gripped his shoulder.
	\item Jane tensed her muscles to stop them from shaking.
	\item When we are under stress our bodies tend to tense up.
	\item I tried not to tense up, or become obviously wary.
	\item Tense up the muscles in both of your legs.
	\end{itemize}
}
\item countable noun \\
The \textbf{tense} of a verb group is its form, which usually shows whether you are referring to past, present, or future time. Compare  aspect [sense 4] .
 \textit{
	\begin{itemize}
	\item It was as though Corinne was already dead: they were speaking of her in the past
tense.
	\end{itemize}
}
\end{enumerate}

\section*{worse}
{\large \color{blue}  }
\subsection*{Explain}
\begin{enumerate}
\item  \\
\textbf{Worse} is the comparative of bad .
 \textit{
	\begin{itemize}
	\end{itemize}
}
\item  \\
\textbf{Worse} is the comparative of badly .
 \textit{
	\begin{itemize}
	\end{itemize}
}
\item  \\
\textbf{Worse} is used to form the comparative of compound  adjectives  beginning with ' bad ' and 'badly.' For example , the comparative of 'badly off' is 'worse off'.
 \textit{
	\begin{itemize}
	\end{itemize}
}
\item  \\
 to go from bad to worse \textit{
	\begin{itemize}
	\end{itemize}
}
\item  \\
 sb could do worse \textit{
	\begin{itemize}
	\end{itemize}
}
\item  \\
 to change for the worse \textit{
	\begin{itemize}
	\end{itemize}
}
\item  \\
 the worse/none the worse for sth \textit{
	\begin{itemize}
	\end{itemize}
}
\end{enumerate}

\section*{ton}
{\large \color{blue}  tons  }
\subsection*{Explain}
\begin{enumerate}
\item countable noun \\
A \textbf{ton} is a unit of weight that is equal to 2240 pounds in Britain and to 2000 pounds in the United  States .
 \textit{
	\begin{itemize}
	\item Hundreds of tons of oil spilled into the sea.
	\item Getting rid of rubbish can cost $100 a ton.
	\end{itemize}
}
\item countable noun \\
A \textbf{ton} is the same as a tonne .
 \textit{
	\begin{itemize}
	\end{itemize}
}
\item  \\
 to come down on somebody like a ton of bricks \textit{
	\begin{itemize}
	\end{itemize}
}
\item  \\
 weigh a ton \textit{
	\begin{itemize}
	\end{itemize}
}
\end{enumerate}

\section*{worst}
{\large \color{blue}  }
\subsection*{Explain}
\begin{enumerate}
\item  \\
\textbf{Worst} is the superlative of bad .
 \textit{
	\begin{itemize}
	\end{itemize}
}
\item  \\
\textbf{Worst} is the superlative of badly .
 \textit{
	\begin{itemize}
	\end{itemize}
}
\item singular noun \\
\textbf{The worst} is the most unpleasant or unfavourable thing that could happen or does happen.
 \textit{
	\begin{itemize}
	\item Though mine safety has much improved, miners' families still fear the worst.
	\item The country had come through the worst of the recession.
	\end{itemize}
}
\item  \\
\textbf{Worst} is used to form the superlative of compound  adjectives  beginning with 'bad' and 'badly'. For example , the superlative of 'badly-affected' is 'worst-affected'.
 \textit{
	\begin{itemize}
	\end{itemize}
}
\item  \\
 worst of all \textit{
	\begin{itemize}
	\end{itemize}
}
\item  \\
 at (the) worst \textit{
	\begin{itemize}
	\end{itemize}
}
\item  \\
 at one's worst \textit{
	\begin{itemize}
	\end{itemize}
}
\item  \\
 if the worst comes to the worst \textit{
	\begin{itemize}
	\end{itemize}
}
\item  \\
 to do one's worst \textit{
	\begin{itemize}
	\end{itemize}
}
\end{enumerate}

\section*{village}
{\large \color{blue}  villages  }
\subsection*{Explain}
\begin{enumerate}
\item countable noun \\
A \textbf{village} consists of a group of houses, together with other buildings such as a church and a school , in a country area.
 \textit{
	\begin{itemize}
	\item He lives quietly in the country in a village near Lahti.
	\item ...the village school.
	\end{itemize}
}
\end{enumerate}

\section*{artistic}
{\large \color{blue}  }
\subsection*{Explain}
\begin{enumerate}
\item adjective \\
Someone who is \textbf{artistic} is good at drawing or painting , or arranging things in a beautiful way.
 \textit{
	\begin{itemize}
	\item They encourage boys to be sensitive and artistic.
	\item Mary's got it all so nice–you remember how artistic she always was with colors.
	\end{itemize}
}
\item adjective \\
\textbf{Artistic} means relating to art or artists.
 \textit{
	\begin{itemize}
	\item ...the campaign for artistic freedom.
	\item ...their 1,300 year old artistic traditions.
	\end{itemize}
}
\item adjective \\
An \textbf{artistic} design or arrangement is beautiful.
 \textit{
	\begin{itemize}
	\item ...an artistic arrangement of stone paving.
	\end{itemize}
}
\end{enumerate}

\section*{address}
{\large \color{blue}  addresses  addressing  addressed  }
\subsection*{Explain}
\begin{enumerate}
\item countable noun \\
Your \textbf{address} is the number of the house, flat, or apartment and the name of the street and the town where you live or work.
 \textit{
	\begin{itemize}
	\item The address is 2025 M Street, Northwest, Washington, DC, 20036.
	\item We require details of your name and address.
	\end{itemize}
}
\item verb \\
If a letter, envelope , or parcel \textbf{is addressed to} you, your name and address have been written on it.
 \textit{
	\begin{itemize}
	\item Applications should be addressed to: The business affairs editor.
	\end{itemize}
}
\item countable noun \\
The \textbf{address} of a website is its location on the internet , for example http://www.collinsdictionary. com .
 \textit{
	\begin{itemize}
	\item Internet addresses are also known as URLs.
	\end{itemize}
}
\item countable noun \\
The \textbf{address} of a cell on a computer spreadsheet refers to the particular row and column number where it is located , for example 'C2' or 'N63'.
 \textit{
	\begin{itemize}
	\end{itemize}
}
\item verb \\
If you \textbf{address} a group of people, you give a speech to them.
 \textbf{Address} is also a noun .
 \textit{
	\begin{itemize}
	\item He is due to address a conference on human rights next week.
	\item The President gave an address to the American people.
	\end{itemize}
}
\item verb \\
If you \textbf{address} someone or \textbf{address} a remark  \textbf{to} them, you say something to them.
 \textit{
	\begin{itemize}
	\item The two foreign ministers did not address each other directly when they last met.
	\item He addressed his remarks to Eleanor, ignoring Maria.
	\end{itemize}
}
\item verb \\
If you \textbf{address} someone by a name or a title such as 'sir', you call them that name or title when
you talk or write to them.
 \textit{
	\begin{itemize}
	\item I heard him address her as darling.
	\item The Duchess should be addressed as Your Grace.
	\end{itemize}
}
\item verb \\
If you \textbf{address} a problem or task or if you \textbf{address}  \textbf{yourself to} it, you try to understand it or deal with it.
 \textit{
	\begin{itemize}
	\item Mr King sought to address those fears when he spoke at the meeting.
	\item Throughout the book we have addressed ourselves to the problem of ethics.
	\end{itemize}
}
\end{enumerate}

\section*{catholic}
{\large \color{blue}  Catholics  }
\subsection*{Explain}
\begin{enumerate}
\item adjective \\
The \textbf{Catholic} Church is the branch of the Christian Church that accepts the Pope as its leader and is based in the Vatican in Rome .
 \textit{
	\begin{itemize}
	\item ...the Catholic Church.
	\item ...Catholic priests.
	\item ...the Catholic faith.
	\end{itemize}
}
\item countable noun \\
A \textbf{Catholic} is a member of the Catholic Church.
 \textit{
	\begin{itemize}
	\item At least nine out of ten Mexicans are baptised Catholics.
	\end{itemize}
}
\item adjective \\
If you describe a collection of things or people as \textbf{catholic} , you are emphasizing that they are very varied .
 \textit{
	\begin{itemize}
	\item He was a man of catholic tastes, a lover of grand opera, history and the fine arts.
	\end{itemize}
}
\end{enumerate}

\section*{audience}
{\large \color{blue}  audiences  }
\subsection*{Explain}
\begin{enumerate}
\item countable noun \\
The \textbf{audience} at a play, concert, film, or public meeting is the group of people watching or listening to it.
 \textit{
	\begin{itemize}
	\item The entire audience broke into loud applause.
	\item He was speaking to an audience of students at the Institute for International Affairs.
	\end{itemize}
}
\item countable noun \\
The \textbf{audience} for a television or radio programme consists of all the people who watch or listen
to it.
 \textit{
	\begin{itemize}
	\item The concert will be relayed to a worldwide television audience.
	\end{itemize}
}
\item countable noun \\
The \textbf{audience} of a writer or artist is the people who read their books or look at their work.
 \textit{
	\begin{itemize}
	\item Merle's writings reached a wide audience during his lifetime.
	\item She began to find a receptive audience for her work.
	\end{itemize}
}
\item countable noun \\
If you have an \textbf{audience}  \textbf{with} someone important , you have a formal meeting with them.
 \textit{
	\begin{itemize}
	\item The Prime Minister will seek an audience with the Queen later this morning.
	\end{itemize}
}
\end{enumerate}

\section*{crude}
{\large \color{blue}  cruder  crudest  crudes  }
\subsection*{Explain}
\begin{enumerate}
\item adjective \\
A \textbf{crude} method or measurement is not exact or detailed , but may be useful or correct in a rough , general way.
 \textit{
	\begin{itemize}
	\item Measurements of blood pressure are a crude way of assessing the risk of heart disease.
	\item Birthplace data are only the crudest indicator of actual migration paths.
	\end{itemize}
}
\item adjective \\
If you describe an object that someone has made as \textbf{crude} , you mean that it has been made in a very simple way or from very simple parts.
 \textit{
	\begin{itemize}
	\item ...crude wooden boxes.
	\end{itemize}
}
\item adjective \\
If you describe someone as \textbf{crude} , you disapprove of them because they speak or behave in a rude , offensive , or unsophisticated way.
 \textit{
	\begin{itemize}
	\item Nev! Must you be quite so crude?
	\item ...crude language.
	\item ...crude sexual jokes.
	\end{itemize}
}
\item adjective \\
\textbf{Crude} substances are in a natural or unrefined state, and have not yet been used in manufacturing
processes.
 \textit{
	\begin{itemize}
	\item ...8.5 million tonnes of crude steel.
	\end{itemize}
}
\item variable noun \\
\textbf{Crude} is the same as crude oil .
 \textit{
	\begin{itemize}
	\end{itemize}
}
\end{enumerate}

\section*{author}
{\large \color{blue}  authors  authoring  authored  }
\subsection*{Explain}
\begin{enumerate}
\item countable noun \\
The \textbf{author}  \textbf{of} a piece of writing is the person who wrote it.
 \textit{
	\begin{itemize}
	\item She is the joint author of a booklet on Integrated Education.
	\end{itemize}
}
\item countable noun \\
An \textbf{author} is a person whose job is writing books.
 \textit{
	\begin{itemize}
	\item ...Philip Pullman, the best-selling author of children's fiction.
	\end{itemize}
}
\item countable noun \\
The \textbf{author of} a plan or proposal is the person who thinks of it and works out the details .
 \textit{
	\begin{itemize}
	\item The authors of the plan believe they can reach this point within about two years.
	\end{itemize}
}
\item verb \\
To \textbf{author} something means to be the author of it.
 \textit{
	\begin{itemize}
	\item Then he opened a restaurant, authored a book, and landed his own radio show.
	\end{itemize}
}
\end{enumerate}

\section*{customary}
{\large \color{blue}  }
\subsection*{Explain}
\begin{enumerate}
\item adjective \\
\textbf{Customary} is used to describe things that people usually do in a particular society or in particular circumstances .
 \textit{
	\begin{itemize}
	\item It is customary to offer a drink or a snack to guests.
	\item At Christmas it was customary for the children to perform bits of poetry.
	\item They interrupted the customary one minute's silence with jeers and shouts.
	\end{itemize}
}
\item adjective \\
\textbf{Customary} is used to describe something that a particular person usually does or has.
 \textit{
	\begin{itemize}
	\item The king carried himself with his customary elegance.
	\item Yvonne took her customary seat behind her desk.
	\end{itemize}
}
\end{enumerate}

\section*{barn}
{\large \color{blue}  barns  }
\subsection*{Explain}
\begin{enumerate}
\item countable noun \\
A \textbf{barn} is a building on a farm in which crops or animal food can be kept .
 \textit{
	\begin{itemize}
	\end{itemize}
}
\end{enumerate}

\section*{dominant}
{\large \color{blue}  }
\subsection*{Explain}
\begin{enumerate}
\item adjective \\
Someone or something that is \textbf{dominant} is more powerful , successful , influential , or noticeable than other people or things.
 \textit{
	\begin{itemize}
	\item ...a change which would maintain his party's dominant position in Scotland.
	\item She was a dominant figure in the French film industry.
	\end{itemize}
}
\item adjective \\
A \textbf{dominant} gene is one that produces a particular characteristic, whether a person has only
one of these genes from one parent , or two genes, one from each parent. Compare  recessive .
 \textit{
	\begin{itemize}
	\item Dimples are quite rare and caused by one dominant gene
	\end{itemize}
}
\end{enumerate}

\section*{benefit}
{\large \color{blue}  benefits  benefiting  benefitting  benefited  benefitted  }
\subsection*{Explain}
\begin{enumerate}
\item variable noun \\
The \textbf{benefit}  \textbf{of} something is the help that you get from it or the advantage that results from it.
 \textit{
	\begin{itemize}
	\item Each family farms individually and reaps the benefit of its labor.
	\item I'm a great believer in the benefits of this form of therapy.
	\item For maximum benefit, use your treatment every day.
	\end{itemize}
}
\item uncountable noun \\
If something is \textbf{to} your \textbf{benefit} or is \textbf{of}  \textbf{benefit}  \textbf{to} you, it helps you or improves your life.
 \textit{
	\begin{itemize}
	\item This could now work to Albania's benefit.
	\item I hope what I have written will be of benefit to someone else who may feel the same
way.
	\end{itemize}
}
\item verb \\
If you \textbf{benefit}  \textbf{from} something or if it \textbf{benefits} you, it helps you or improves your life.
 \textit{
	\begin{itemize}
	\item Both sides have benefited from the talks.
	\item ...a variety of government programs benefiting children.
	\end{itemize}
}
\item uncountable noun \\
If you have the \textbf{benefit of} some information, knowledge , or equipment , you are able to use it so that you can achieve something.
 \textit{
	\begin{itemize}
	\item Steve didn't have the benefit of a formal college education.
	\item With the benefit of hindsight, it is clear we ought to have done more.
	\item This remarkable achievement took place without the benefit of modern technology.
	\end{itemize}
}
\item variable noun \\
\textbf{Benefit} is money that is given by the government to people who are poor , ill, or unemployed.
 \textit{
	\begin{itemize}
	\item ...the removal of benefit from school-leavers.
	\item I was told that in order to get benefit payments I would have to answer some questions.
	\end{itemize}
}
\item countable noun \\
A \textbf{benefit} , or a \textbf{benefit}  concert or dinner , is an event that is held in order to raise money for a particular charity or person.
 \textit{
	\begin{itemize}
	\item I am organising a benefit gig in Bristol to raise these funds.
	\end{itemize}
}
\item  \\
 the benefit of the doubt \textit{
	\begin{itemize}
	\end{itemize}
}
\item  \\
 for the benefit of sb \textit{
	\begin{itemize}
	\end{itemize}
}
\end{enumerate}

\section*{extensive}
{\large \color{blue}  }
\subsection*{Explain}
\begin{enumerate}
\item adjective \\
Something that is \textbf{extensive} covers or includes a large physical area.
 \textit{
	\begin{itemize}
	\item ...an extensive tour of Latin America.
	\item When built, the palace and its grounds were more extensive than the city itself.
	\end{itemize}
}
\item adjective \\
Something that is \textbf{extensive} covers a wide  range of details , ideas , or items .
 \textit{
	\begin{itemize}
	\item There was extensive coverage of World Book Day on the BBC.
	\item ...the extensive research into public attitudes to science.
	\item The facilities available are very extensive.
	\end{itemize}
}
\item adjective \\
If something is \textbf{extensive} , it is very great.
 \textit{
	\begin{itemize}
	\item The blast caused extensive damage, shattering the ground-floor windows.
	\item The security forces have extensive powers of search and arrest.
	\item Mr Marr makes extensive use of exclusively Scottish words.
	\end{itemize}
}
\end{enumerate}

\section*{body}
{\large \color{blue}  bodies  }
\subsection*{Explain}
\begin{enumerate}
\item countable noun \\
Your \textbf{body} is all your physical parts, including your head, arms, and legs.
 \textit{
	\begin{itemize}
	\item The largest organ in the body is the liver.
	\end{itemize}
}
\item countable noun \\
You can also  refer to the main part of your body, except for your arms, head, and legs, as your \textbf{body} .
 \textit{
	\begin{itemize}
	\item Lying flat on the floor, twist your body on to one hip and cross your upper leg over
your body.
	\end{itemize}
}
\item countable noun \\
You can refer to a person's dead body as a \textbf{body} .
 \textit{
	\begin{itemize}
	\item Officials said they had found no traces of violence on the body of the politician.
	\end{itemize}
}
\item countable noun \\
A \textbf{body} is an organized group of people who deal with something officially .
 \textit{
	\begin{itemize}
	\item ...the chairperson of the police representative body, the Police Federation.
	\item ...the main trade union body, COSATU, Congress of South African Trade Unions.
	\end{itemize}
}
\item countable noun \\
A \textbf{body of} people is a group of people who are together or who are connected in some way.
 \textit{
	\begin{itemize}
	\item ...that large body of people which teaches other people how to teach.
	\end{itemize}
}
\item singular noun \\
\textbf{The body}  \textbf{of} something such as a building or a document is the main part of it or the largest part of it.
 \textit{
	\begin{itemize}
	\item The main body of the church had been turned into a massive television studio.
	\item Give an introduction, followed by the body of the material, then a brief summary.
	\end{itemize}
}
\item countable noun \\
The \textbf{body} of a car or aeroplane is the main part of it, not including its engine, wheels , or wings.
 \textit{
	\begin{itemize}
	\item The only shade was under the body of the plane.
	\end{itemize}
}
\item countable noun \\
A \textbf{body of} water is a large area of water, such as a lake or a sea.
 \textit{
	\begin{itemize}
	\item It is probably the most polluted body of water in the world.
	\end{itemize}
}
\item countable noun \\
A \textbf{body of} information is a large amount of it.
 \textit{
	\begin{itemize}
	\item An increasing body of evidence suggests that all of us have cancer cells in our bodies
at times during our lives.
	\end{itemize}
}
\item uncountable noun \\
If you say that an alcoholic drink has \textbf{body} , you mean that it has a full and strong flavour .
 \textit{
	\begin{itemize}
	\item ...a dry wine with good body.
	\end{itemize}
}
\item countable noun \\
A \textbf{body} is the same as a bodysuit .
 \textit{
	\begin{itemize}
	\end{itemize}
}
\item  \\
 over sb's dead body \textit{
	\begin{itemize}
	\end{itemize}
}
\item  \\
 body and soul \textit{
	\begin{itemize}
	\end{itemize}
}
\item  \\
 to keep body and soul together \textit{
	\begin{itemize}
	\end{itemize}
}
\end{enumerate}

\section*{feminine}
{\large \color{blue}  }
\subsection*{Explain}
\begin{enumerate}
\item adjective \\
\textbf{Feminine} qualities and things relate to or are considered typical of women, in contrast to men.
 \textit{
	\begin{itemize}
	\item Women all over the world are abandoning traditional feminine roles.
	\item ...a manufactured ideal of feminine beauty.
	\end{itemize}
}
\item adjective \\
Someone or something that is \textbf{feminine} has qualities that are considered typical of women, especially being pretty or gentle .
 \textit{
	\begin{itemize}
	\item I've always been attracted to very feminine, delicate women.
	\item The bedroom has a light, feminine look.
	\end{itemize}
}
\item adjective \\
In some languages, a \textbf{feminine} noun, pronoun , or adjective has a different form from a masculine or neuter one, or behaves in a different way.
 \textit{
	\begin{itemize}
	\end{itemize}
}
\end{enumerate}

\section*{bulk}
{\large \color{blue}  bulks  bulking  bulked  }
\subsection*{Explain}
\begin{enumerate}
\item singular noun \\
You can refer to something's \textbf{bulk} when you want to emphasize that it is very large.
 \textit{
	\begin{itemize}
	\item The truck pulled out of the lot, its bulk unnerving against the dawn.
	\item ...the shadowy bulk of an ancient barn.
	\end{itemize}
}
\item singular noun \\
You can refer to a large person's body or to their weight or size as their \textbf{bulk} .
 \textit{
	\begin{itemize}
	\item Bannol lowered his bulk carefully into the chair.
	\item Despite his bulk, he moved lightly on his feet.
	\end{itemize}
}
\item quantifier \\
The \textbf{bulk}  \textbf{of} something is most of it.
 \textbf{Bulk} is also a pronoun .
 \textit{
	\begin{itemize}
	\item The bulk of the text is essentially a review of these original documents.
	\item The vast bulk of imports and exports are carried by sea.
	\item They come from all over the world, though the bulk is from the Indian subcontinent.
	\end{itemize}
}
\item uncountable noun \\
If you buy or sell something \textbf{in}  \textbf{bulk} , you buy or sell it in large quantities .
 \textit{
	\begin{itemize}
	\item Buying in bulk is more economical than shopping for small quantities.
	\item ...bulk purchasing.
	\end{itemize}
}
\end{enumerate}

\section*{futile}
{\large \color{blue}  }
\subsection*{Explain}
\begin{enumerate}
\item adjective \\
If you say that something is \textbf{futile} , you mean there is no point in doing it, usually because it has no chance of succeeding .
 \textit{
	\begin{itemize}
	\item He brought his arm up in a futile attempt to ward off the blow.
	\item It would be futile to sustain his life when there is no chance of any improvement.
	\end{itemize}
}
\end{enumerate}

\section*{composition}
{\large \color{blue}  compositions  }
\subsection*{Explain}
\begin{enumerate}
\item uncountable noun \\
When you talk about the \textbf{composition} of something, you are referring to the way in which its various parts are put together and arranged .
 \textit{
	\begin{itemize}
	\item Television has transformed the size and composition of audiences.
	\item Forests vary greatly in composition from one part of the country to another.
	\end{itemize}
}
\item countable noun \\
The \textbf{compositions} of a composer , painter , or other artist are the works of art that they have produced.
 \textit{
	\begin{itemize}
	\item Mozart's compositions are undoubtedly amongst the world's greatest.
	\end{itemize}
}
\item countable noun \\
A \textbf{composition} is a piece of written work that children write at school.
 \textit{
	\begin{itemize}
	\end{itemize}
}
\item uncountable noun \\
\textbf{Composition} is the technique or skill involved in creating a work of art.
 \textit{
	\begin{itemize}
	\item He taught the piano, organ and composition.
	\item The course is designed to help students with colour and composition.
	\end{itemize}
}
\item uncountable noun \\
\textbf{Composition} is the act of composing something such as a piece of music or a poem .
 \textit{
	\begin{itemize}
	\item These plays are arranged in their order of composition.
	\end{itemize}
}
\end{enumerate}

\section*{ignorant}
{\large \color{blue}  }
\subsection*{Explain}
\begin{enumerate}
\item adjective \\
If you describe someone as \textbf{ignorant} , you mean that they do not know things they should know. If someone is \textbf{ignorant}  \textbf{of} a fact , they do not know it.
 \textit{
	\begin{itemize}
	\item People don't like to ask questions for fear of appearing ignorant.
	\item Many people are worryingly ignorant of the facts about global warming.
	\item ...ignorant peasants.
	\end{itemize}
}
\item adjective \\
People are sometimes described as \textbf{ignorant} when they do something that is not polite or kind . Some people think that it is not correct to use \textbf{ignorant} with this meaning .
 \textit{
	\begin{itemize}
	\item I met some very ignorant people who called me all kinds of names.
	\end{itemize}
}
\end{enumerate}

\section*{compromise}
{\large \color{blue}  compromises  compromising  compromised  }
\subsection*{Explain}
\begin{enumerate}
\item variable noun \\
A \textbf{compromise} is a situation in which people accept something slightly different from what they really  want , because of circumstances or because they are considering the wishes of other people.
 \textit{
	\begin{itemize}
	\item Encourage your child to reach a compromise between what he wants and what you want.
	\item Be ready and willing to make compromises between your needs and those of your partner.
	\item The government's policy of compromise is not universally popular.
	\end{itemize}
}
\item verb \\
If you \textbf{compromise}  \textbf{with} someone, you reach an agreement with them in which you both give up something that you originally wanted. You can also  say that two people or groups \textbf{compromise} .
 \textit{
	\begin{itemize}
	\item The government has compromised with its critics over monetary policies.
	\item 'Nine,' said I. 'Nine thirty,' tried he. We compromised on 9.15.
	\item Israel had originally wanted $1 billion in aid, but compromised on the $650 million.
	\end{itemize}
}
\item verb \\
If someone \textbf{compromises} themselves or \textbf{compromises} their beliefs , they do something which damages their reputation for honesty , loyalty , or high  moral  principles .
 \textit{
	\begin{itemize}
	\item He compromised himself by accepting a bribe.
	\item He would rather shoot himself than compromise his principles.
	\end{itemize}
}
\end{enumerate}

\section*{illegal}
{\large \color{blue}  illegals  }
\subsection*{Explain}
\begin{enumerate}
\item adjective \\
If something is \textbf{illegal} , the law says that it is not allowed .
 \textit{
	\begin{itemize}
	\item It is illegal to intercept radio messages.
	\item Birth control was illegal there until 1978.
	\item He has been charged with membership of an illegal organisation.
	\item ...illegal drugs.
	\item ...an illegal action.
	\end{itemize}
}
\item adjective \\
\textbf{Illegal}  immigrants or workers have travelled into a country or are working without official permission .
 Illegal immigrants or workers are sometimes  referred to as \textbf{illegals} .
 \textit{
	\begin{itemize}
	\item ...a clothing factory where many illegals worked.
	\end{itemize}
}
\end{enumerate}

\section*{confusion}
{\large \color{blue}  confusions  }
\subsection*{Explain}
\begin{enumerate}
\item variable noun \\
If there is \textbf{confusion} about something, it is not clear what the true  situation is, especially because people believe different things.
 \textit{
	\begin{itemize}
	\item There's still confusion about the number of casualties.
	\item Omissions in my recent article must have caused confusion.
	\end{itemize}
}
\item uncountable noun \\
\textbf{Confusion} is a situation in which everything is in disorder, especially because there are lots of things happening at the same time.
 \textit{
	\begin{itemize}
	\item There was confusion when a man fired shots.
	\item The rebel leader appears to have escaped in the confusion.
	\end{itemize}
}
\item variable noun \\
If your mind is in a state of \textbf{confusion} , you do not know what to believe or what you should do.
 \textit{
	\begin{itemize}
	\item We always left his office in a state of confusion.
	\item ...the pressures and confusions of puberty.
	\end{itemize}
}
\end{enumerate}

\section*{immense}
{\large \color{blue}  }
\subsection*{Explain}
\begin{enumerate}
\item adjective \\
If you describe something as \textbf{immense} , you mean that it is extremely large or great .
 \textit{
	\begin{itemize}
	\item ...an immense cloud of smoke.
	\item With immense relief I stopped running.
	\end{itemize}
}
\end{enumerate}

\section*{corn}
{\large \color{blue}  corns  }
\subsection*{Explain}
\begin{enumerate}
\item uncountable noun \\
\textbf{Corn} is used to refer to crops such as wheat and barley . It can also be used to refer to the seeds from these plants.
 \textit{
	\begin{itemize}
	\item ...fields of corn.
	\item He filled the barn to the roof with corn.
	\end{itemize}
}
\item uncountable noun \\
\textbf{Corn} is the same as maize .
 \textit{
	\begin{itemize}
	\item ...rows of corn in an Iowa field.
	\end{itemize}
}
\item countable noun \\
\textbf{Corns} are small, painful areas of hard skin which can form on your foot, especially near your toes .
 \textit{
	\begin{itemize}
	\end{itemize}
}
\end{enumerate}

\section*{impatient}
{\large \color{blue}  }
\subsection*{Explain}
\begin{enumerate}
\item adjective \\
If you are \textbf{impatient} , you are annoyed because you have to wait too long for something.
 \textit{
	\begin{itemize}
	\item He is impatient as the first hour passes and then another.
	\item The big clubs are becoming increasingly impatient at the rate of progress.
	\end{itemize}
}
\item adjective \\
If you are \textbf{impatient} , you are easily irritated by things.
 \textit{
	\begin{itemize}
	\item Beware of being too impatient with others.
	\item He threw it aside with an impatient gesture and another oath and walked off.
	\end{itemize}
}
\item adjective \\
If you are \textbf{impatient}  \textbf{to} do something or \textbf{impatient for} something to happen , you are eager to do it or for it to happen and do not want to wait.
 \textit{
	\begin{itemize}
	\item He was impatient to get home.
	\item They are impatient for jobs and security.
	\end{itemize}
}
\end{enumerate}

\section*{crop}
{\large \color{blue}  crops  cropping  cropped  }
\subsection*{Explain}
\begin{enumerate}
\item countable noun \\
\textbf{Crops} are plants such as wheat and potatoes that are grown in large quantities for food.
 \textit{
	\begin{itemize}
	\item Rice farmers here still plant and harvest their crops by hand.
	\item The main crop is wheat and this is grown even on the very steep slopes.
	\end{itemize}
}
\item countable noun \\
The plants or fruits that are collected at harvest time are referred to as a \textbf{crop} .
 \textit{
	\begin{itemize}
	\item Each year it produces a fine crop of fruit.
	\item This year's corn crop should be about 8 percent more than last year.
	\item In the Middle Ages, years of crop failure were always followed by terrible disease.
	\end{itemize}
}
\item singular noun \\
You can refer to a group of people or things that have appeared together as a \textbf{crop of} people or things.
 \textit{
	\begin{itemize}
	\item The present crop of books and documentaries about Marilyn Monroe exploit the thirtieth
anniversary of her death.
	\item Some of this year's crop of graduates are on a fast-track recruitment scheme.
	\end{itemize}
}
\item verb \\
When a plant \textbf{crops} , it produces fruits or parts which people want .
 \textit{
	\begin{itemize}
	\item Although these vegetables adapt well to our temperate climate, they tend to crop
poorly.
	\end{itemize}
}
\item verb \\
When you \textbf{crop} something that you have planted, you collect the fruits or parts that you want from
it.
 \textit{
	\begin{itemize}
	\item I started cropping my beans in July.
	\end{itemize}
}
\item verb \\
When an animal such as a cow or horse \textbf{crops} leaves or plants, it eats them.
 \textit{
	\begin{itemize}
	\item I let the horse drop his head to crop the spring grass.
	\end{itemize}
}
\item verb \\
To \textbf{crop} someone's hair means to cut it short.
 \textit{
	\begin{itemize}
	\item She cropped her hair and dyed it blonde.
	\end{itemize}
}
\item countable noun \\
A \textbf{crop} is a short hairstyle.
 \textit{
	\begin{itemize}
	\item She had her long hair cut into a boyish crop.
	\end{itemize}
}
\item verb \\
If you \textbf{crop} a photograph , you cut part of it off, in order to get  rid of part of the picture or to be able to frame it.
 \textit{
	\begin{itemize}
	\item I decided to crop the picture just above the water line.
	\item Her husband was cropped from the photograph.
	\end{itemize}
}
\end{enumerate}

\section*{impossible}
{\large \color{blue}  }
\subsection*{Explain}
\begin{enumerate}
\item adjective \\
Something that is \textbf{impossible} cannot be done or cannot happen .
 \textbf{The impossible} is something which is impossible.
 \textit{
	\begin{itemize}
	\item It was impossible for anyone to get in because no one knew the password.
	\item He thinks the tax is impossible to administer.
	\item You shouldn't promise what's impossible.
	\item Keller is good at describing music–an almost impossible task to do well.
	\item They were expected to do the impossible.
	\item No one can achieve the impossible.
	\end{itemize}
}
\item adjective \\
An \textbf{impossible}  situation or an \textbf{impossible} position is one that is very difficult to deal with.
 \textit{
	\begin{itemize}
	\item The Government was now in an almost impossible position.
	\end{itemize}
}
\item adjective \\
If you describe someone as \textbf{impossible} , you are annoyed that their bad  behaviour or strong  views make them difficult to deal with.
 \textit{
	\begin{itemize}
	\item The woman is impossible, thought Frannie.
	\end{itemize}
}
\end{enumerate}

\section*{dictation}
{\large \color{blue}  }
\subsection*{Explain}
\begin{enumerate}
\item uncountable noun \\
\textbf{Dictation} is the speaking or reading  aloud of words for someone else to write down.
 \textit{
	\begin{itemize}
	\item ...taking dictation from the dean of the Faculty.
	\item He had had an arm amputated and relied on her to take down his books from dictation.
	\end{itemize}
}
\item uncountable noun \\
\textbf{Dictation} is the giving of orders in a forceful and commanding way .
 \textit{
	\begin{itemize}
	\item She was unwilling to accept dictation from anyone.
	\end{itemize}
}
\end{enumerate}

\section*{incredible}
{\large \color{blue}  }
\subsection*{Explain}
\begin{enumerate}
\item adjective \\
If you describe something or someone as \textbf{incredible} , you like them very much or are impressed by them, because they are extremely or unusually good .
 \textit{
	\begin{itemize}
	\item The wildflowers will be incredible after this rain.
	\item Thanks for taking me, I had an incredible time.
	\item You're always an incredible help on these cases.
	\end{itemize}
}
\item adjective \\
If you say that something is \textbf{incredible} , you mean that it is very unusual or surprising , and you cannot believe it is really  true , although it may be.
 \textit{
	\begin{itemize}
	\item It seemed incredible that people would still want to play football during a war.
	\item We should not dismiss as lies the incredible stories that children may tell us.
	\end{itemize}
}
\item adjective \\
You use \textbf{incredible} to emphasize the degree , amount , or intensity of something.
 \textit{
	\begin{itemize}
	\item We import an incredible amount of cheese from the Continent.
	\item There was an incredible din.
	\item It's incredible how much Francesca wants her father's approval.
	\item His panic was incredible.
	\end{itemize}
}
\end{enumerate}

\section*{difficulty}
{\large \color{blue}  difficulties  }
\subsection*{Explain}
\begin{enumerate}
\item countable noun \\
A \textbf{difficulty} is a problem.
 \textit{
	\begin{itemize}
	\item ...the difficulty of getting accurate information.
	\item The country is facing great economic difficulties.
	\end{itemize}
}
\item uncountable noun \\
If you have \textbf{difficulty} doing something, you are not able to do it easily .
 \textit{
	\begin{itemize}
	\item Do you have difficulty getting up?
	\item The injured man mounted his horse with difficulty.
	\end{itemize}
}
\item  \\
 in difficulty \textit{
	\begin{itemize}
	\end{itemize}
}
\end{enumerate}

\section*{individual}
{\large \color{blue}  individuals  }
\subsection*{Explain}
\begin{enumerate}
\item adjective \\
\textbf{Individual} means relating to one person or thing, rather than to a large group.
 \textit{
	\begin{itemize}
	\item They wait for the group to decide rather than making individual decisions.
	\item Aid to individual countries would be linked to progress towards democracy.
	\item Divide the vegetables among four individual dishes.
	\end{itemize}
}
\item countable noun \\
An \textbf{individual} is a person.
 \textit{
	\begin{itemize}
	\item ...anonymous individuals who are doing good things within our community.
	\item ...the rights and responsibilities of the individual.
	\item A child's awareness of being an individual grows in stages during the pre-school
years.
	\end{itemize}
}
\item adjective \\
If you describe someone or something as \textbf{individual} , you mean that you admire them because they are very unusual and do not try to imitate other people or things.
 \textit{
	\begin{itemize}
	\item It was really all part of her very individual personality.
	\item The language is highly individual.
	\end{itemize}
}
\end{enumerate}

\section*{dwelling}
{\large \color{blue}  dwellings  }
\subsection*{Explain}
\begin{enumerate}
\item countable noun \\
A \textbf{dwelling} or a \textbf{dwelling place} is a place where someone lives.
 \textit{
	\begin{itemize}
	\item Some 3,500 new dwellings are planned for the area.
	\end{itemize}
}
\end{enumerate}

\section*{inevitable}
{\large \color{blue}  }
\subsection*{Explain}
\begin{enumerate}
\item adjective \\
If something is \textbf{inevitable} , it is certain to happen and cannot be prevented or avoided .
 \textbf{The inevitable} is something which is inevitable.
 \textit{
	\begin{itemize}
	\item If the case succeeds, it is inevitable that other trials will follow.
	\item The defeat had inevitable consequences for British policy.
	\item 'It's just delaying the inevitable,' he said.
	\end{itemize}
}
\end{enumerate}

\section*{infinite}
{\large \color{blue}  }
\subsection*{Explain}
\begin{enumerate}
\item adjective \\
If you describe something as \textbf{infinite} , you are emphasizing that it is extremely great in amount or degree .
 \textit{
	\begin{itemize}
	\item ...an infinite variety of landscapes.
	\item With infinite care, John shifted position.
	\item The choice is infinite.
	\end{itemize}
}
\item adjective \\
Something that is \textbf{infinite} has no limit, end, or edge .
 \textbf{The infinite} is something which is infinite.
 \textit{
	\begin{itemize}
	\item ...an infinite number of atoms.
	\item Obviously, no company has infinite resources.
	\item ...God's infinite mercy.
	\item ...pondering on the infinite.
	\end{itemize}
}
\end{enumerate}

\section*{finance}
{\large \color{blue}  finances  financing  financed  }
\subsection*{Explain}
\begin{enumerate}
\item verb \\
When someone \textbf{finances} something such as a project or a purchase , they provide the money that is needed to pay for them.
 \textbf{Finance} is also a noun .
 \textit{
	\begin{itemize}
	\item The fund has been used largely to finance the construction of federal prisons.
	\item Government expenditure is financed by taxation and by borrowing.
	\item They are seeking finance for a major scientific project.
	\end{itemize}
}
\item uncountable noun \\
\textbf{Finance} is the commercial or government activity of managing money, debt , credit, and investment .
 \textit{
	\begin{itemize}
	\item ...a major player in the world of high finance.
	\item The report recommends an overhaul of public finances.
	\item A former Finance Minister and five senior civil servants are accused of fraud.
	\end{itemize}
}
\item variable noun \\
You can  refer to the amount of money that you have and how well it is organized as your \textbf{finances} .
 \textit{
	\begin{itemize}
	\item Be prepared for unexpected news concerning your finances.
	\item Finance is usually the biggest problem for students.
	\end{itemize}
}
\end{enumerate}

\section*{innumerable}
{\large \color{blue}  }
\subsection*{Explain}
\begin{enumerate}
\item adjective \\
\textbf{Innumerable} means very many, or too many to be counted .
 \textit{
	\begin{itemize}
	\item He has invented innumerable excuses, told endless lies.
	\end{itemize}
}
\end{enumerate}

\section*{grain}
{\large \color{blue}  grains  }
\subsection*{Explain}
\begin{enumerate}
\item countable noun \\
A \textbf{grain}  \textbf{of} wheat, rice , or other cereal crop is a seed from it.
 \textit{
	\begin{itemize}
	\item ...a grain of wheat.
	\item ...rice grains.
	\end{itemize}
}
\item variable noun \\
\textbf{Grain} is a cereal crop, especially wheat or corn , that has been harvested and is used for food or in trade.
 \textit{
	\begin{itemize}
	\item ...a bag of grain.
	\item ...the best grains.
	\end{itemize}
}
\item countable noun \\
A \textbf{grain of} something such as sand or salt is a tiny hard piece of it.
 \textit{
	\begin{itemize}
	\item ...a grain of sand.
	\end{itemize}
}
\item singular noun \\
A \textbf{grain of} a quality is a very small amount of it.
 \textit{
	\begin{itemize}
	\item There's more than a grain of truth in that.
	\end{itemize}
}
\item singular noun \\
\textbf{The grain} of a piece of wood is the direction of its fibres . You can also refer to the pattern of lines on the surface of the wood as \textbf{the grain} .
 \textit{
	\begin{itemize}
	\item Brush the paint generously over the wood in the direction of the grain.
	\end{itemize}
}
\item  \\
 go against the grain \textit{
	\begin{itemize}
	\end{itemize}
}
\end{enumerate}

\section*{irrespective}
{\large \color{blue}  }
\subsection*{Explain}
\begin{enumerate}
\item  \\
 irrespective of \textit{
	\begin{itemize}
	\end{itemize}
}
\end{enumerate}

\section*{hearing}
{\large \color{blue}  hearings  }
\subsection*{Explain}
\begin{enumerate}
\item uncountable noun \\
A person's or animal's \textbf{hearing} is the sense which makes it possible for them to be aware of sounds.
 \textit{
	\begin{itemize}
	\item His mind still seemed clear and his hearing was excellent.
	\end{itemize}
}
\item countable noun \\
A \textbf{hearing} is an official meeting which is held in order to collect  facts about an incident or problem .
 \textit{
	\begin{itemize}
	\item The judge adjourned the hearing until next Tuesday.
	\end{itemize}
}
\item  \\
 a fair hearing \textit{
	\begin{itemize}
	\end{itemize}
}
\item  \\
 in/within sb's hearing \textit{
	\begin{itemize}
	\end{itemize}
}
\end{enumerate}

\section*{married}
{\large \color{blue}  }
\subsection*{Explain}
\begin{enumerate}
\item adjective \\
If you are \textbf{married} , you have a husband or wife.
 \textit{
	\begin{itemize}
	\item We have been married for 14 years.
	\item She is married to an Englishman.
	\item ...a married man with two children.
	\end{itemize}
}
\item adjective \\
\textbf{Married}  means relating to marriage or to people who are married.
 \textit{
	\begin{itemize}
	\item For the first ten years of our married life we lived in a farmhouse.
	\end{itemize}
}
\item adjective \\
If you say that someone is \textbf{married to} their work or another activity , you mean that they are very involved with it and have little interest in anything else.
 \textit{
	\begin{itemize}
	\item She was a very strict Christian who was married to her job.
	\item I'm married to my cricket.
	\end{itemize}
}
\end{enumerate}

\section*{hope}
{\large \color{blue}  hopes  hoping  hoped  }
\subsection*{Explain}
\begin{enumerate}
\item verb \\
If you \textbf{hope} that something is true , or if you \textbf{hope} for something, you want it to be true or to happen , and you usually believe that it is possible or likely .
 \textit{
	\begin{itemize}
	\item She had decided she must go on as usual, follow her normal routine, and hope and
pray.
	\item He hesitates before leaving, almost as though he had been hoping for conversation.
	\item I hope to get a job within the next two weeks.
	\item The researchers hope that such a vaccine could be available in about ten years' time.
	\item 'We'll speak again.'—'I hope so.'
	\item 'Will it happen again?'—'I hope not, but you never know.'
	\end{itemize}
}
\item verb \\
If you say that you cannot \textbf{hope}  \textbf{for} something, or if you talk about the only thing that you can \textbf{hope}  \textbf{to}  get , you mean that you are in a bad situation, and there is very little chance of improving it.
 \textbf{Hope} is also a noun .
 \textit{
	\begin{itemize}
	\item Things aren't ideal, but that's the best you can hope for.
	\item I always knew it was too much to hope for.
	\item ...these mountains, which no one can hope to penetrate.
	\item The only hope for underdeveloped countries is to become, as far as possible, self-reliant.
	\item The car was smashed beyond any hope of repair.
	\end{itemize}
}
\item uncountable noun \\
\textbf{Hope} is a feeling of desire and expectation that things will  go  well in the future.
 \textit{
	\begin{itemize}
	\item Now that he has become President, many people once again have hope for genuine changes
in the system.
	\item But Kevin hasn't given up hope of being fit.
	\item Consumer groups still hold out hope that the president will change his mind.
	\item Thousands of childless couples are to be given new hope by the government.
	\end{itemize}
}
\item countable noun \\
If someone wants something to happen, and considers it likely or possible, you can refer to their \textbf{hopes}  \textbf{of} that thing, or to their \textbf{hope}  \textbf{that} it will happen.
 \textit{
	\begin{itemize}
	\item They have hopes of increasing trade between the two regions.
	\item The delay in the programme has dashed Japan's hopes of commercial success in space.
	\item My hope is that, in the future, I will go over there and marry her.
	\end{itemize}
}
\item countable noun \\
If you think that the help or success of a particular person or thing will cause you to be successful or to get what you want, you can refer to them as your \textbf{hope} .
 \textit{
	\begin{itemize}
	\item ...England's last hope in the English Open Table Tennis Championships.
	\item He was one of our best hopes for a gold at the Commonwealth Games.
	\end{itemize}
}
\item  \\
 to hope for the best \textit{
	\begin{itemize}
	\end{itemize}
}
\item  \\
 get/build your hopes up \textit{
	\begin{itemize}
	\end{itemize}
}
\item  \\
 not a hope in hell \textit{
	\begin{itemize}
	\end{itemize}
}
\item  \\
 high/great hopes \textit{
	\begin{itemize}
	\end{itemize}
}
\item  \\
 hope against hope \textit{
	\begin{itemize}
	\end{itemize}
}
\item  \\
 I hope \textit{
	\begin{itemize}
	\end{itemize}
}
\item  \\
 I hope \textit{
	\begin{itemize}
	\end{itemize}
}
\item  \\
 I hope \textit{
	\begin{itemize}
	\end{itemize}
}
\item  \\
 in the hope of/that \textit{
	\begin{itemize}
	\end{itemize}
}
\item  \\
 live in hope \textit{
	\begin{itemize}
	\end{itemize}
}
\item  \\
 some hope/not a hope \textit{
	\begin{itemize}
	\end{itemize}
}
\end{enumerate}

\section*{marxist}
{\large \color{blue}  Marxists  }
\subsection*{Explain}
\begin{enumerate}
\item adjective \\
\textbf{Marxist}  means  based on Marxism or relating to Marxism.
 \textit{
	\begin{itemize}
	\item ...a Marxist state.
	\item ...Marxist ideology.
	\end{itemize}
}
\item countable noun \\
A \textbf{Marxist} is a person who believes in Marxism or who is a member of a Marxist party .
 \textit{
	\begin{itemize}
	\end{itemize}
}
\end{enumerate}

\section*{housing}
{\large \color{blue}  housings  }
\subsection*{Explain}
\begin{enumerate}
\item uncountable noun \\
You refer to the buildings in which people live as \textbf{housing} when you are talking about their standard, price , or availability.
 \textit{
	\begin{itemize}
	\item ...a shortage of affordable housing.
	\item Poor housing and family stress can affect both physical and mental health.
	\end{itemize}
}
\item uncountable noun \\
\textbf{Housing} is the job of providing houses for people to live in.
 \textit{
	\begin{itemize}
	\item ...graduate courses in housing and public administration.
	\item If you are a council tenant call the housing department about it.
	\end{itemize}
}
\item countable noun \\
A \textbf{housing} is a case or covering which protects parts of a machine.
 \textit{
	\begin{itemize}
	\item Both housings are waterproof to a depth of two metres.
	\end{itemize}
}
\end{enumerate}

\section*{naive}
{\large \color{blue}  }
\subsection*{Explain}
\begin{enumerate}
\item adjective \\
If you describe someone as \textbf{naive} , you think they lack experience and so expect things to be easy or people to be honest or kind .
 \textit{
	\begin{itemize}
	\item It's naive to think that teachers are always tolerant.
	\item I must have been naive to think we would get my parents' blessing.
	\item ...naive idealists.
	\item Their view was that he had been politically naive.
	\end{itemize}
}
\end{enumerate}

\section*{identity}
{\large \color{blue}  identities  }
\subsection*{Explain}
\begin{enumerate}
\item countable noun \\
Your \textbf{identity} is who you are.
 \textit{
	\begin{itemize}
	\item Abu is not his real name, but it's one he uses to disguise his identity.
	\item The police soon established his true identity and he was quickly found.
	\end{itemize}
}
\item variable noun \\
The \textbf{identity} of a person or place is the characteristics they have that distinguish them from others.
 \textit{
	\begin{itemize}
	\item I wanted a sense of my own identity.
	\item ...the distinct cultural, religious and national identity of many Italians.
	\end{itemize}
}
\end{enumerate}

\section*{obsolete}
{\large \color{blue}  }
\subsection*{Explain}
\begin{enumerate}
\item adjective \\
Something that is \textbf{obsolete} is no longer needed because something better has been invented .
 \textit{
	\begin{itemize}
	\item So much equipment becomes obsolete almost as soon as it's made.
	\end{itemize}
}
\end{enumerate}

\section*{implication}
{\large \color{blue}  implications  }
\subsection*{Explain}
\begin{enumerate}
\item countable noun \\
\textbf{The}  \textbf{implications}  \textbf{of} something are the things that are likely to happen as a result .
 \textit{
	\begin{itemize}
	\item The Attorney General was aware of the political implications of his decision to prosecute.
	\item The low level of current investment has serious implications for future economic
growth.
	\end{itemize}
}
\item countable noun \\
\textbf{The}  \textbf{implication} of a statement , event , or situation is what it implies or suggests is the case .
 \textit{
	\begin{itemize}
	\item The implication was obvious: vote for us or it will be very embarrassing for you.
	\end{itemize}
}
\end{enumerate}

\section*{opaque}
{\large \color{blue}  }
\subsection*{Explain}
\begin{enumerate}
\item adjective \\
If an object or substance is \textbf{opaque} , you cannot see through it.
 \textit{
	\begin{itemize}
	\item You can always use opaque glass if you need to block a street view.
	\end{itemize}
}
\item adjective \\
If you say that something is \textbf{opaque} , you mean that it is difficult to understand.
 \textit{
	\begin{itemize}
	\item ...the opaque language of the inspector's reports.
	\end{itemize}
}
\end{enumerate}

\section*{locality}
{\large \color{blue}  localities  }
\subsection*{Explain}
\begin{enumerate}
\item countable noun \\
A \textbf{locality} is a small area of a country or city.
 \textit{
	\begin{itemize}
	\item Following the discovery of the explosives the president cancelled his visit to the
locality.
	\item To find out what is available in your locality, see the website.
	\end{itemize}
}
\end{enumerate}

\section*{oral}
{\large \color{blue}  orals  }
\subsection*{Explain}
\begin{enumerate}
\item adjective \\
\textbf{Oral}  communication is spoken rather than written.
 \textit{
	\begin{itemize}
	\item ...the written and oral traditions of ancient cultures.
	\item ...an oral agreement.
	\end{itemize}
}
\item countable noun \\
An \textbf{oral} is an examination, especially in a foreign language, that is spoken rather than written.
 \textit{
	\begin{itemize}
	\item I spoke privately to the candidate after the oral.
	\end{itemize}
}
\item adjective \\
You use \textbf{oral} to indicate that something is done with a person's mouth or relates to a person's mouth.
 \textit{
	\begin{itemize}
	\item ...good oral hygiene.
	\end{itemize}
}
\item adjective \\
\textbf{Oral}  medicines are taken by mouth.
 \textit{
	\begin{itemize}
	\item ...oral contraceptives.
	\end{itemize}
}
\end{enumerate}

\section*{location}
{\large \color{blue}  locations  }
\subsection*{Explain}
\begin{enumerate}
\item countable noun \\
A \textbf{location} is the place where something happens or is situated .
 \textit{
	\begin{itemize}
	\item The first thing he looked at was his office's location.
	\item Macau's newest small luxury hotel has a beautiful location.
	\end{itemize}
}
\item countable noun \\
The \textbf{location} of someone or something is their exact position.
 \textit{
	\begin{itemize}
	\item She knew the exact location of The Eagle's headquarters.
	\end{itemize}
}
\item variable noun \\
A \textbf{location} is a place away from a studio where a film or part of a film is made.
 \textit{
	\begin{itemize}
	\item ...an art movie with dozens of exotic locations.
	\item We're shooting on location.
	\end{itemize}
}
\end{enumerate}

\section*{patent}
{\large \color{blue}  patents  patenting  patented  }
\subsection*{Explain}
\begin{enumerate}
\item countable noun \\
A \textbf{patent} is an official right to be the only person or company allowed to make or sell a new  product for a certain period of time.
 \textit{
	\begin{itemize}
	\item P&G applied for a patent on its cookies.
	\item He held a number of patents for his many innovations.
	\item It sued Centrocorp for patent infringement.
	\end{itemize}
}
\item verb \\
If you \textbf{patent} something, you obtain a patent for it.
 \textit{
	\begin{itemize}
	\item He patented the idea that the atom could be split.
	\item The invention has been patented by the university.
	\item ...a patented process for disinfecting liquids.
	\end{itemize}
}
\item adjective \\
You use \textbf{patent} to describe something, especially something bad , in order to indicate in an emphatic  way that you think its nature or existence is clear and obvious.
 \textit{
	\begin{itemize}
	\item This was patent nonsense.
	\item ...a patent lie.
	\end{itemize}
}
\end{enumerate}

\section*{needle}
{\large \color{blue}  needles  needling  needled  }
\subsection*{Explain}
\begin{enumerate}
\item countable noun \\
A \textbf{needle} is a small, very thin piece of polished metal which is used for sewing. It has a sharp point at one end and a hole in the
other for a thread to go through.
 \textit{
	\begin{itemize}
	\end{itemize}
}
\item countable noun \\
Knitting \textbf{needles} are thin sticks that are used for knitting. They are usually made of plastic or metal and have a point at one end.
 \textit{
	\begin{itemize}
	\end{itemize}
}
\item countable noun \\
A \textbf{needle} is a thin hollow metal rod with a sharp point, which is part of a medical instrument called a syringe. It is used to put a drug into someone's body, or to
take blood out.
 \textit{
	\begin{itemize}
	\end{itemize}
}
\item countable noun \\
A \textbf{needle} is a thin metal rod with a point which is put into a patient's body during acupuncture .
 \textit{
	\begin{itemize}
	\end{itemize}
}
\item countable noun \\
On a record player, the \textbf{needle} is the small pointed device that touches the record and picks up the sound signals .
 \textit{
	\begin{itemize}
	\item She took the needle off the record and turned the lights out.
	\end{itemize}
}
\item countable noun \\
On an instrument which measures something such as speed or weight , the \textbf{needle} is the long strip of metal or plastic on the dial that moves backwards and forwards , showing the measurement .
 \textit{
	\begin{itemize}
	\item She kept looking at the dial on the boiler. The needle had reached 250 degrees.
	\end{itemize}
}
\item countable noun \\
The \textbf{needles} of a fir or pine tree are its thin, hard, pointed leaves.
 \textit{
	\begin{itemize}
	\item The carpet of pine needles was soft underfoot.
	\end{itemize}
}
\item verb \\
If someone \textbf{needles} you, they annoy you continually, especially by criticizing you.
 \textit{
	\begin{itemize}
	\item Blake could see he had needled Jerrold, which might be unwise.
	\end{itemize}
}
\end{enumerate}

\section*{personal}
{\large \color{blue}  }
\subsection*{Explain}
\begin{enumerate}
\item adjective \\
A \textbf{personal}  opinion , quality, or thing belongs or relates to one particular person rather than to other
people.
 \textit{
	\begin{itemize}
	\item He learned this lesson the hard way–from his own personal experience.
	\item That's my personal opinion.
	\item ...books, furniture, and other personal belongings.
	\item The President arrived, followed by his personal bodyguard.
	\item ...an estimated personal fortune of almost seventy million dollars.
	\end{itemize}
}
\item adjective \\
If you give something your \textbf{personal} care or attention , you deal with it yourself rather than letting someone else deal with it.
 \textit{
	\begin{itemize}
	\item ...a business that requires a great deal of personal contact.
	\item ...a personal letter from the President's secretary.
	\item People do not mind paying a bit extra for the personal touch.
	\end{itemize}
}
\item adjective \\
\textbf{Personal} matters relate to your feelings , relationships , and health .
 \textit{
	\begin{itemize}
	\item ...teaching young people about marriage and personal relationships.
	\item You never allow personal problems to affect your performance.
	\item We sacrifice our personal lives to our work.
	\item Mr Knight said that he had resigned for personal reasons.
	\end{itemize}
}
\item adjective \\
\textbf{Personal}  comments refer to someone's appearance or character in an offensive way.
 \textit{
	\begin{itemize}
	\item Newspapers resorted to personal abuse.
	\item Myra was attacking something I'd written, and her attack got a little personal.
	\end{itemize}
}
\item adjective \\
\textbf{Personal} care involves looking after your body and appearance.
 \textit{
	\begin{itemize}
	\item ...people who take time and care over personal hygiene.
	\end{itemize}
}
\item adjective \\
A \textbf{personal} relationship is one that is not connected with your job or public life.
 \textit{
	\begin{itemize}
	\item He was a personal friend whom I've known for many years.
	\item What began as a professional relationship became a personal one pretty quickly.
	\end{itemize}
}
\end{enumerate}

\section*{pin}
{\large \color{blue}  pins  pinning  pinned  }
\subsection*{Explain}
\begin{enumerate}
\item countable noun \\
\textbf{Pins} are very small thin pointed pieces of metal. They are used in sewing to fasten pieces of material together until they have been sewn.
 \textit{
	\begin{itemize}
	\item ...needles and pins.
	\item Use pins to keep the braid in place as you work.
	\end{itemize}
}
\item verb \\
If you \textbf{pin} something \textbf{on} or \textbf{to} something, you attach it with a pin, a drawing pin, or a safety pin.
 \textit{
	\begin{itemize}
	\item They pinned a notice to the door.
	\item Everyone was supposed to dance with the bride and pin money on her dress.
	\item He had pinned up a map of Finland.
	\end{itemize}
}
\item verb \\
If someone \textbf{pins} you to something, they press you against a surface so that you cannot move.
 \textit{
	\begin{itemize}
	\item I pinned him against the wall.
	\item He fought at the bulk that pinned him.
	\end{itemize}
}
\item countable noun \\
A \textbf{pin} is any long narrow piece of metal or wood that is not sharp, especially one that is used to fasten two things together.
 \textit{
	\begin{itemize}
	\item ...the 18-inch steel pin holding his left leg together.
	\item ...a two-pin continental adaptor.
	\end{itemize}
}
\item verb \\
If someone tries to \textbf{pin} something \textbf{on} you or to \textbf{pin the blame on} you, they say , often unfairly, that you were responsible for something bad or illegal .
 \textit{
	\begin{itemize}
	\item They're trying to pin it on us.
	\item The trade unions are pinning the blame for the violence on the government.
	\end{itemize}
}
\item verb \\
If you \textbf{pin} your hopes  \textbf{on} something or \textbf{pin} your faith  \textbf{on} something, you hope very much that it will produce the result you want .
 \textit{
	\begin{itemize}
	\item The Democrats are pinning their hopes on the next election.
	\end{itemize}
}
\item verb \\
If someone \textbf{pins} their \textbf{hair} up or \textbf{pins} their \textbf{hair} back, they arrange their hair away from their face using hair pins.
 \textit{
	\begin{itemize}
	\item Cleanse your face thoroughly and pin back your hair.
	\item In an effort to look older she has pinned her hair into a bun.
	\end{itemize}
}
\item countable noun \\
A \textbf{pin} is something worn on your clothing, for example as jewellery , which is fastened with a pointed piece of metal.
 \textit{
	\begin{itemize}
	\item ...necklaces, bracelets, and pins.
	\end{itemize}
}
\item countable noun \\
A \textbf{pin} is the part of a hand grenade that is pulled out in order to make the grenade explode .
 \textit{
	\begin{itemize}
	\end{itemize}
}
\item  \\
 you could have heard a pin drop \textit{
	\begin{itemize}
	\end{itemize}
}
\end{enumerate}

\section*{reckless}
{\large \color{blue}  }
\subsection*{Explain}
\begin{enumerate}
\item adjective \\
If you say that someone is \textbf{reckless} , you mean that they act in a way which shows that they do not care about danger or the effect their behaviour  will have on other people.
 \textit{
	\begin{itemize}
	\item She loved to ride; on horseback, she was reckless and utterly without fear.
	\item He is charged with causing death by reckless driving.
	\end{itemize}
}
\end{enumerate}

\section*{position}
{\large \color{blue}  positions  positioning  positioned  }
\subsection*{Explain}
\begin{enumerate}
\item countable noun \\
The \textbf{position} of someone or something is the place where they are in relation to other things.
 \textit{
	\begin{itemize}
	\item The ship was identified, and its name and position were reported to the coastguard.
	\item This conservatory enjoys an enviable position overlooking a leafy expanse.
	\end{itemize}
}
\item countable noun \\
When someone or something is in a particular \textbf{position} , they are sitting , lying, or arranged in that way.
 \textit{
	\begin{itemize}
	\item Hold the upper back and neck in an erect position to give support for the head.
	\item Ensure the patient is turned into the recovery position.
	\item Mr. Dambar had raised himself to a sitting position.
	\end{itemize}
}
\item verb \\
If you \textbf{position} something somewhere , you put it there carefully, so that it is in the right place or position.
 \textit{
	\begin{itemize}
	\item Position trailing plants near the edges and in the sides of the basket to hang down.
	\item Place the pastry circles on to a baking sheet and position one apple on each circle.
	\end{itemize}
}
\item countable noun \\
Your \textbf{position} in society is the role and the importance that you have in it.
 \textit{
	\begin{itemize}
	\item ...the position of older people in society.
	\end{itemize}
}
\item countable noun \\
A \textbf{position} in a company or organization is a job.
 \textit{
	\begin{itemize}
	\item He left a career in teaching to take up a position with the Arts Council.
	\item Hyundai said this week it is scaling back its U.S. operations by eliminating 50 positions.
	\end{itemize}
}
\item countable noun \\
Your \textbf{position} in a race or competition is how well you did in relation to the other competitors or how well you are doing.
 \textit{
	\begin{itemize}
	\item Agassi and Sampras resumed their battle for the world's No. 1 position, both winning
their opening matches.
	\item By the ninth hour the car was running in eighth position.
	\end{itemize}
}
\item countable noun \\
You can describe your situation at a particular time by saying that you are in a particular \textbf{position} .
 \textit{
	\begin{itemize}
	\item He's going to be in a very difficult position indeed if things go badly for him.
	\item Companies should be made to reveal more about their financial position.
	\item It was not the only time he found himself in this position.
	\end{itemize}
}
\item countable noun \\
Your \textbf{position}  \textbf{on} a particular matter is your attitude towards it or your opinion of it.
 \textit{
	\begin{itemize}
	\item He could be depended on to take a moderate position on most of the key issues.
	\item Mr Howard is afraid to state his true position on the republic, which is that he
is opposed to it.
	\end{itemize}
}
\item singular noun \\
If you are \textbf{in a position}  \textbf{to} do something, you are able to do it. If you are \textbf{in no position}  \textbf{to} do something, you are unable to do it.
 \textit{
	\begin{itemize}
	\item The U.N. system will be in a position to support the extensive relief efforts needed.
	\item I am not in a position to comment.
	\end{itemize}
}
\item  \\
 in position \textit{
	\begin{itemize}
	\end{itemize}
}
\end{enumerate}

\section*{royal}
{\large \color{blue}  royals  }
\subsection*{Explain}
\begin{enumerate}
\item adjective \\
\textbf{Royal} is used to indicate that something is connected with a king, queen, or emperor , or their family. A \textbf{royal} person is a king, queen, or emperor, or a member of their family.
 \textit{
	\begin{itemize}
	\item ...an invitation to a royal garden party.
	\item The Spanish royal couple were to attend a celebration of Shakespeare and Cervantes.
	\end{itemize}
}
\item adjective \\
\textbf{Royal} is used in the names of institutions or organizations that are officially  appointed or supported by a member of a royal family.
 \textit{
	\begin{itemize}
	\item ...the Royal Academy of Music.
	\item ...several pilots of the Royal Navy's 846 Squadron.
	\end{itemize}
}
\item countable noun \\
Members of the royal family are sometimes  referred to as \textbf{royals} .
 \textit{
	\begin{itemize}
	\item The royals have always been patrons of charities pulling in large donations.
	\end{itemize}
}
\end{enumerate}

\section*{profit}
{\large \color{blue}  profits  profiting  profited  }
\subsection*{Explain}
\begin{enumerate}
\item variable noun \\
A \textbf{profit} is an amount of money that you gain when you are paid more for something than it
 cost you to make, get , or do it.
 \textit{
	\begin{itemize}
	\item The bank made pre-tax profits of £3.5 million.
	\item You can improve your chances of profit by sensible planning.
	\item The profit motive is inherently at odds with principles of fairness and equity.
	\end{itemize}
}
\item verb \\
If you \textbf{profit}  \textbf{from} something, you earn a profit from it.
 \textit{
	\begin{itemize}
	\item Footballers are accustomed to profiting handsomely from bonuses.
	\item He has profited by selling his holdings to other investors.
	\item The dealers profited shamefully at the expense of my family.
	\end{itemize}
}
\item verb \\
If you \textbf{profit}  \textbf{from} something, or it \textbf{profits} you, you gain some advantage or benefit from it.
 \textbf{Profit} is also a noun .
 \textit{
	\begin{itemize}
	\item Jennifer wasn't yet totally convinced that she'd profit from a more relaxed lifestyle.
	\item So far the French alliance had profited the rebels little.
	\item Whom would it profit to terrify or to kill this man?
	\item The artist found much to his profit in the Louvre.
	\end{itemize}
}
\end{enumerate}

\section*{rural}
{\large \color{blue}  }
\subsection*{Explain}
\begin{enumerate}
\item adjective \\
\textbf{Rural} places are far  away from large towns or cities.
 \textit{
	\begin{itemize}
	\item These plants have a tendency to grow in the more rural areas.
	\item ...the closure of rural schools.
	\end{itemize}
}
\item adjective \\
\textbf{Rural} means having features which are typical of areas that are far away from large towns or cities.
 \textit{
	\begin{itemize}
	\item ...the old rural way of life.
	\item He spoke with a heavy rural accent.
	\end{itemize}
}
\end{enumerate}

\section*{property}
{\large \color{blue}  properties  }
\subsection*{Explain}
\begin{enumerate}
\item uncountable noun \\
Someone's \textbf{property} is all the things that belong to them or something that belongs to them.
 \textit{
	\begin{itemize}
	\item Richard could easily destroy her personal property to punish her for walking out
on him.
	\item Security forces searched thousands of homes, confiscating weapons and stolen property.
	\end{itemize}
}
\item variable noun \\
A \textbf{property} is a building and the land belonging to it.
 \textit{
	\begin{itemize}
	\item Cecil inherited a family property near Stamford.
	\item This vehicle has been parked on private property.
	\end{itemize}
}
\item countable noun \\
The \textbf{properties} of a substance or object are the ways in which it behaves in particular conditions.
 \textit{
	\begin{itemize}
	\item A radio signal has both electrical and magnetic properties.
	\end{itemize}
}
\end{enumerate}

\section*{thirsty}
{\large \color{blue}  thirstier  thirstiest  }
\subsection*{Explain}
\begin{enumerate}
\item adjective \\
If you are \textbf{thirsty} , you feel a need to drink something.
 \textit{
	\begin{itemize}
	\item If a baby is thirsty, it feeds more often.
	\item Drink whenever you feel thirsty during exercise.
	\end{itemize}
}
\item adjective \\
If you are \textbf{thirsty for} something, you have a strong desire for it.
 \textit{
	\begin{itemize}
	\item People should understand how thirsty for revenge they are.
	\end{itemize}
}
\end{enumerate}

\section*{residence}
{\large \color{blue}  residences  }
\subsection*{Explain}
\begin{enumerate}
\item countable noun \\
A \textbf{residence} is a house where people live .
 \textit{
	\begin{itemize}
	\item ...hotels and private residences.
	\item She travels constantly, moving among her several residences around the world.
	\end{itemize}
}
\item uncountable noun \\
Your place of \textbf{residence} is the place where you live.
 \textit{
	\begin{itemize}
	\item ...differences among women based on age, place of residence and educational levels.
	\end{itemize}
}
\item uncountable noun \\
Someone's \textbf{residence} in a particular place is the fact that they live there or that they are officially
 allowed to live there.
 \textit{
	\begin{itemize}
	\item They had entered the country and had applied for permanent residence.
	\item He arrived discreetly on 13 November, and took up residence in Carisbrooke Castle.
	\end{itemize}
}
\item  \\
 in residence \textit{
	\begin{itemize}
	\end{itemize}
}
\item  \\
 in residence \textit{
	\begin{itemize}
	\end{itemize}
}
\end{enumerate}

\section*{timely}
{\large \color{blue}  }
\subsection*{Explain}
\begin{enumerate}
\item adjective \\
A \textbf{timely} event happens at a moment when it is useful , effective , or relevant .
 \textit{
	\begin{itemize}
	\item The recent outbreak is a timely reminder that this disease is a serious health hazard.
	\item The exhibition is timely, since 'self-taught' art is catching on in a big way.
	\end{itemize}
}
\end{enumerate}

\section*{revenue}
{\large \color{blue}  revenues  }
\subsection*{Explain}
\begin{enumerate}
\item uncountable noun \\
\textbf{Revenue} is money that a company , organization , or government receives from people.
 \textit{
	\begin{itemize}
	\item ...a boom year at the cinema, with record advertising revenue and the highest ticket
sales since 1980.
	\item One study said the government would gain about $12 billion in tax revenues over five
years.
	\end{itemize}
}
\end{enumerate}

\section*{uneasy}
{\large \color{blue}  }
\subsection*{Explain}
\begin{enumerate}
\item adjective \\
If you are \textbf{uneasy} , you feel anxious, afraid , or embarrassed , because you think that something is wrong or that there is danger .
 \textit{
	\begin{itemize}
	\item He said nothing but gave me a sly grin that made me feel terribly uneasy.
	\item He looked uneasy and refused to answer questions.
	\item I had an uneasy feeling that he was going to spoil it.
	\end{itemize}
}
\item adjective \\
If you are \textbf{uneasy}  \textbf{about} doing something, you are not sure that it is correct or wise .
 \textit{
	\begin{itemize}
	\item Richard was uneasy about how best to approach his elderly mother.
	\item Scientists feel uneasy about giving a positive answer.
	\end{itemize}
}
\item adjective \\
If you describe a situation or relationship as \textbf{uneasy} , you mean that the situation is not settled and may not last .
 \textit{
	\begin{itemize}
	\item An uneasy calm has settled over the city.
	\item The uneasy alliance between these two men offered a glimmer of hope.
	\item There is an uneasy relationship between us and the politicians.
	\end{itemize}
}
\item adjective \\
If you describe a book or music as \textbf{uneasy} , you are critical of it because it is difficult to read or listen to.
 \textit{
	\begin{itemize}
	\item These ballads are deeply uneasy listening, and compelling for that reason.
	\item This is an uneasy travel book.
	\item ...an uneasy mix of thudding bass, drums and screaming guitar.
	\end{itemize}
}
\end{enumerate}

\section*{site}
{\large \color{blue}  sites  siting  sited  }
\subsection*{Explain}
\begin{enumerate}
\item countable noun \\
A \textbf{site} is a piece of ground that is used for a particular purpose or where a particular thing happens .
 \textit{
	\begin{itemize}
	\item He became a hod carrier on a building site.
	\item ...a bat sanctuary with special nesting sites.
	\end{itemize}
}
\item countable noun \\
\textbf{The}  \textbf{site}  \textbf{of} an important  event is the place where it happened.
 \textit{
	\begin{itemize}
	\item ...the site of the worst ecological disaster on Earth.
	\item Plymouth Hoe is renowned as the site where Drake played bowls before tackling the
Spanish Armada.
	\end{itemize}
}
\item countable noun \\
A \textbf{site} is a piece of ground where something such as a statue or building stands or used to stand.
 \textit{
	\begin{itemize}
	\item ...the site of Moses' tomb.
	\item ...the Church of the Holy Sepulchre in Jerusalem, which is regarded by some as Christianity's
holiest site.
	\end{itemize}
}
\item countable noun \\
A \textbf{site} is the same as a website .
 \textit{
	\begin{itemize}
	\end{itemize}
}
\item verb \\
If something \textbf{is sited} in a particular place or position, it is put there or built there.
 \textit{
	\begin{itemize}
	\item He said chemical weapons had never been sited in this country.
	\item ...a damp, old castle, romantically sited on a river estuary.
	\end{itemize}
}
\item  \\
 on site \textit{
	\begin{itemize}
	\end{itemize}
}
\item  \\
 off site \textit{
	\begin{itemize}
	\end{itemize}
}
\end{enumerate}

\section*{unlike}
{\large \color{blue}  }
\subsection*{Explain}
\begin{enumerate}
\item preposition \\
If one thing is \textbf{unlike} another thing, the two things have different qualities or characteristics from each
other.
 \textit{
	\begin{itemize}
	\item This was a foreign country, so unlike San Jose.
	\item She was unlike him in every way except for her coal black eyes.
	\end{itemize}
}
\item preposition \\
You can use \textbf{unlike} to contrast two people, things, or situations , and show how they are different.
 \textit{
	\begin{itemize}
	\item Unlike aerobics, walking entails no expensive fees for classes or clubs.
	\end{itemize}
}
\item preposition \\
If you describe something that a particular person has done as being \textbf{unlike} them, you mean that you are surprised by it because it is not typical of their character or normal  behaviour .
 \textit{
	\begin{itemize}
	\item It was so unlike him to say something like that, with such intensity, that I was
astonished.
	\item 'We'll all be arrested!' Thomas yelled, which was most unlike him.
	\end{itemize}
}
\end{enumerate}

\section*{stadium}
{\large \color{blue}  stadiums  stadia  }
\subsection*{Explain}
\begin{enumerate}
\item countable noun \\
A \textbf{stadium} is a large sports ground with rows of seats all round it.
 \textit{
	\begin{itemize}
	\item ...a baseball stadium.
	\item ...Wembley Stadium.
	\end{itemize}
}
\end{enumerate}

\section*{unusual}
{\large \color{blue}  }
\subsection*{Explain}
\begin{enumerate}
\item adjective \\
If something is \textbf{unusual} , it does not happen very often or you do not see it or hear it very often.
 \textit{
	\begin{itemize}
	\item They have replanted many areas with rare and unusual plants.
	\item To be appreciated as a parent is quite unusual.
	\end{itemize}
}
\item adjective \\
If you describe someone as \textbf{unusual} , you think that they are interesting and different from other people.
 \textit{
	\begin{itemize}
	\item He was an unusual man with great business talents.
	\end{itemize}
}
\end{enumerate}

\section*{tent}
{\large \color{blue}  tents  }
\subsection*{Explain}
\begin{enumerate}
\item countable noun \\
A \textbf{tent} is a shelter made of canvas or nylon which is held up by poles and ropes, and is used mainly by people who are camping.
 \textit{
	\begin{itemize}
	\end{itemize}
}
\end{enumerate}

\section*{vague}
{\large \color{blue}  vaguer  vaguest  }
\subsection*{Explain}
\begin{enumerate}
\item adjective \\
If something written or spoken is \textbf{vague} , it does not explain or express things clearly.
 \textit{
	\begin{itemize}
	\item A lot of the talk was apparently vague and general.
	\item The description was pretty vague.
	\item ...vague information.
	\end{itemize}
}
\item adjective \\
If you have a \textbf{vague}  memory or idea of something, the memory or idea is not clear.
 \textit{
	\begin{itemize}
	\item They have only a vague idea of the amount of water available.
	\item Waite's memory of that first meeting was vague.
	\end{itemize}
}
\item adjective \\
If you are \textbf{vague} about something, you deliberately do not tell people much about it.
 \textit{
	\begin{itemize}
	\item He was vague, however, about just what U.S. forces might actually do.
	\item Democratic leaders under election pressure tend to respond with vague promises of
action.
	\item Christopher's answer was deliberately vague.
	\end{itemize}
}
\item adjective \\
If you describe someone as \textbf{vague} , you mean that they do not seem to be thinking clearly.
 \textit{
	\begin{itemize}
	\item She had married a charming but rather vague Englishman.
	\item His eyes were always so vague when he looked at her.
	\end{itemize}
}
\item adjective \\
If something such as a feeling is \textbf{vague} , you experience it only slightly .
 \textit{
	\begin{itemize}
	\item He was conscious of that vague feeling of irritation again.
	\item He had a vague impression of rain pounding on the packed earth.
	\end{itemize}
}
\item adjective \\
A \textbf{vague} shape or outline is not clear and is therefore not easy to see .
 \textit{
	\begin{itemize}
	\item He looked at her vague shape through the frosted glass.
	\item The bus was a vague shape in the distance.
	\end{itemize}
}
\end{enumerate}

\section*{treasure}
{\large \color{blue}  treasures  treasuring  treasured  }
\subsection*{Explain}
\begin{enumerate}
\item uncountable noun \\
\textbf{Treasure} is a collection of valuable old objects such as gold  coins and jewels that has been hidden or lost .
 \textit{
	\begin{itemize}
	\item It was here, the buried treasure, she knew it was.
	\end{itemize}
}
\item countable noun \\
\textbf{Treasures} are valuable objects, especially works of art and items of historical value.
 \textit{
	\begin{itemize}
	\item The house was large and full of art treasures.
	\end{itemize}
}
\item verb \\
If you \textbf{treasure} something that you have, you keep it or care for it carefully because it gives you great  pleasure and you think it is very special .
 \textbf{Treasure} is also a noun .
 \textit{
	\begin{itemize}
	\item She treasures her memories of those joyous days.
	\item His greatest treasure is his collection of rock records.
	\end{itemize}
}
\item countable noun \\
If you say that someone is a \textbf{treasure} , you mean that they are very helpful and useful to you.
 \textit{
	\begin{itemize}
	\item Charlie? Oh, he's a treasure, loves children.
	\end{itemize}
}
\end{enumerate}

\section*{wholesome}
{\large \color{blue}  }
\subsection*{Explain}
\begin{enumerate}
\item adjective \\
If you describe something as \textbf{wholesome} , you approve of it because you think it is likely to have a positive  influence on people's behaviour or mental state, especially because it does not involve anything sexually immoral .
 \textit{
	\begin{itemize}
	\item ...good, wholesome fun.
	\item ...a very decent and wholesome bunch of lads.
	\end{itemize}
}
\item adjective \\
If you describe food as \textbf{wholesome} , you approve of it because you think it is good for your health.
 \textit{
	\begin{itemize}
	\item ...fresh, wholesome ingredients.
	\item The food is filling and wholesome.
	\end{itemize}
}
\end{enumerate}

\section*{uncle}
{\large \color{blue}  uncles  }
\subsection*{Explain}
\begin{enumerate}
\item countable noun \\
Someone's \textbf{uncle} is the brother of their mother or father, or the husband of their aunt.
 \textit{
	\begin{itemize}
	\item My uncle was the mayor of Memphis.
	\item A text from Uncle Fred arrived.
	\item Uncle, pa wants to see you.
	\end{itemize}
}
\end{enumerate}

\section*{wooden}
{\large \color{blue}  }
\subsection*{Explain}
\begin{enumerate}
\item adjective \\
\textbf{Wooden} objects are made of wood.
 \textit{
	\begin{itemize}
	\item ...the shop's bare brick walls and faded wooden floorboards.
	\end{itemize}
}
\item adjective \\
If you describe an actor as \textbf{wooden} , you are critical of them because their performance is not at all lively or natural.
 \textit{
	\begin{itemize}
	\end{itemize}
}
\end{enumerate}

\section*{wealth}
{\large \color{blue}  }
\subsection*{Explain}
\begin{enumerate}
\item uncountable noun \\
\textbf{Wealth} is the possession of a large amount of money, property, or other valuable things.
You can also  refer to a particular person's money or property as their \textbf{wealth} .
 \textit{
	\begin{itemize}
	\item Economic reform has brought relative wealth to peasant farmers.
	\item His own wealth grew.
	\end{itemize}
}
\item singular noun \\
If you say that someone or something has \textbf{a wealth of} good qualities or things, you are emphasizing that they have a very large number or amount of them.
 \textit{
	\begin{itemize}
	\item ...such a wealth of creative expertise.
	\item The city boasts a wealth of beautiful churches.
	\end{itemize}
}
\end{enumerate}

\section*{abnormal}
{\large \color{blue}  }
\subsection*{Explain}
\begin{enumerate}
\item adjective \\
Someone or something that is \textbf{abnormal} is unusual , especially in a way that is worrying .
 \textit{
	\begin{itemize}
	\item ...abnormal heart rhythms and high anxiety levels.
	\item ...a child with an abnormal fear of strangers.
	\end{itemize}
}
\end{enumerate}

\section*{angle}
{\large \color{blue}  angles  angling  angled  }
\subsection*{Explain}
\begin{enumerate}
\item countable noun \\
An \textbf{angle} is the difference in direction between two lines or surfaces. Angles are measured in degrees.
 \textit{
	\begin{itemize}
	\item The boat is now leaning at a 30 degree angle.
	\end{itemize}
}
\item countable noun \\
An \textbf{angle} is the shape that is created where two lines or surfaces join together.
 \textit{
	\begin{itemize}
	\item ...the angle of the blade.
	\item ...brackets to adjust the steering wheel's angle.
	\end{itemize}
}
\item countable noun \\
An \textbf{angle} is the direction from which you look at something.
 \textit{
	\begin{itemize}
	\item Thanks to the angle at which he stood, he could just see the sunset.
	\item His face will be discreetly concealed by camera angles.
	\end{itemize}
}
\item countable noun \\
You can refer to a way of presenting something or thinking about it as a particular \textbf{angle} .
 \textit{
	\begin{itemize}
	\item We had to do the scene over and over again, from different angles.
	\item He was considering the idea from all angles.
	\end{itemize}
}
\item verb \\
If someone \textbf{is angling for} something, they are trying to get it without asking for it directly .
 \textit{
	\begin{itemize}
	\item It sounds as if he's just angling for sympathy.
	\end{itemize}
}
\item ergative verb \\
If you \textbf{angle} something or if it \textbf{angles} in a particular direction, it faces or points in that direction.
 \textit{
	\begin{itemize}
	\item You can open the slats for a bright light or angle them for more shade.
	\item The path angled downhill and northwards.
	\item He drove down the long, steeply angled driveway.
	\end{itemize}
}
\item  \\
 at an angle \textit{
	\begin{itemize}
	\end{itemize}
}
\end{enumerate}

\section*{central}
{\large \color{blue}  }
\subsection*{Explain}
\begin{enumerate}
\item adjective \\
Something that is \textbf{central} is in the middle of a place or area.
 \textit{
	\begin{itemize}
	\item ...Central America's Caribbean coast.
	\item Wild camels are a problem in central Australia.
	\item ...a rich woman living in central London.
	\end{itemize}
}
\item adjective \\
A place that is \textbf{central} is easy to reach because it is in the centre of a city, town, or particular area.
 \textit{
	\begin{itemize}
	\item ...a central location in the capital.
	\end{itemize}
}
\item adjective \\
A \textbf{central} group or organization makes all the important decisions that are followed throughout a larger organization or a country.
 \textit{
	\begin{itemize}
	\item There is a lack of trust towards the central government in Rome.
	\item ...the central committee of the Cuban communist party.
	\end{itemize}
}
\item adjective \\
The \textbf{central} person or thing in a particular situation is the most important one.
 \textit{
	\begin{itemize}
	\item Black dance music has been central to mainstream pop since the early '60s.
	\item ...a central part of their culture.
	\end{itemize}
}
\end{enumerate}

\section*{band}
{\large \color{blue}  bands  banding  banded  }
\subsection*{Explain}
\begin{enumerate}
\item countable noun \\
A \textbf{band} is a small group of musicians who play popular music such as jazz, rock, or pop.
 \textit{
	\begin{itemize}
	\item He was a drummer in a rock band.
	\item Local bands provide music for dancing.
	\end{itemize}
}
\item countable noun \\
A \textbf{band} is a group of musicians who play brass and percussion instruments.
 \textit{
	\begin{itemize}
	\item Bands played German marches.
	\end{itemize}
}
\item countable noun \\
A \textbf{band}  \textbf{of} people is a group of people who have joined together because they share an interest
or belief.
 \textit{
	\begin{itemize}
	\item Bands of criminals have been roaming some neighborhoods.
	\item ...a small but growing band of Japanese companies taking their first steps into American
publishing.
	\end{itemize}
}
\item countable noun \\
A \textbf{band} is a flat, narrow strip of cloth which you wear round your head or wrists , or which forms part of a piece of clothing.
 \textit{
	\begin{itemize}
	\item Almost all hospitals use a wrist-band of some kind with your name and details on
it.
	\end{itemize}
}
\item countable noun \\
A \textbf{band} is a strip of something such as colour, light, land, or cloth which contrasts with
the areas on either side of it.
 \textit{
	\begin{itemize}
	\item ...bands of natural vegetation between strips of crops.
	\item A band of light glowed in the space between floor and door.
	\end{itemize}
}
\item countable noun \\
A \textbf{band} is a strip or loop of metal or other strong material which strengthens something, or which holds several things together.
 \textit{
	\begin{itemize}
	\item Surgeons placed a metal band around the knee cap to help it knit back together.
	\item ...a strong band of flat muscle tissue.
	\end{itemize}
}
\item countable noun \\
A \textbf{band} is a range of numbers or values within a system of measurement .
 \textit{
	\begin{itemize}
	\item For an initial service, a 10 megahertz-wide band of frequencies will be needed.
	\item ...a tax band of 20p in the pound on the first £2,000 of taxable income.
	\end{itemize}
}
\item verb \\
If something such as a tax \textbf{is banded} , it is divided into bands according to the value of the thing being taxed.
 \textit{
	\begin{itemize}
	\item They appear to have ruled out banding the tax so higher earners would pay more.
	\item ...a banding system based on property values.
	\item The choice will be between a flat-rate or a banded charge.
	\end{itemize}
}
\end{enumerate}

\section*{chemical}
{\large \color{blue}  chemicals  }
\subsection*{Explain}
\begin{enumerate}
\item adjective \\
\textbf{Chemical} means involving or resulting from a reaction between two or more substances, or relating
to the substances that something consists of.
 \textit{
	\begin{itemize}
	\item ...chemical reactions that cause ozone destruction.
	\item ...the chemical composition of the ocean.
	\item ...chemical weapons.
	\end{itemize}
}
\item countable noun \\
\textbf{Chemicals} are substances that are used in a chemical process or made by a chemical process.
 \textit{
	\begin{itemize}
	\item The whole food chain is affected by the over-use of chemicals in agriculture.
	\item ...a chemicals company.
	\item ...the chemical industry.
	\end{itemize}
}
\end{enumerate}

\section*{civil}
{\large \color{blue}  }
\subsection*{Explain}
\begin{enumerate}
\item adjective \\
You use \textbf{civil} to describe  events that happen within a country and that involve the different groups of people in it.
 \textit{
	\begin{itemize}
	\item ...civil unrest.
	\end{itemize}
}
\item adjective \\
You use \textbf{civil} to describe people or things in a country that are not connected with its armed forces.
 \textit{
	\begin{itemize}
	\item ...the U.S. civil aviation industry.
	\end{itemize}
}
\item adjective \\
You use \textbf{civil} to describe things that are connected with the state rather than with a religion .
 \textit{
	\begin{itemize}
	\item They were married on August 9 in a civil ceremony in Venice.
	\item ...Jewish civil and religious law.
	\end{itemize}
}
\item adjective \\
You use \textbf{civil} to describe the rights that people have within a society .
 \textit{
	\begin{itemize}
	\item ...a United Nations covenant on civil and political rights.
	\end{itemize}
}
\item adjective \\
Someone who is \textbf{civil} is polite in a formal way, but not particularly  friendly .
 \textit{
	\begin{itemize}
	\item As visitors, the least we can do is be civil to the people in their own land.
	\end{itemize}
}
\end{enumerate}

\section*{bath}
{\large \color{blue}  baths  bathing  bathed  }
\subsection*{Explain}
\begin{enumerate}
\item countable noun \\
A \textbf{bath} is a container, usually a long rectangular one, which you fill with water and sit in while you wash your body.
 \textit{
	\begin{itemize}
	\item In those days, only quite wealthy families had baths of their own.
	\end{itemize}
}
\item countable noun \\
When you have or take a \textbf{bath} , or when you are in the \textbf{bath} , you sit or lie in a bath filled with water in order to wash your body.
 \textit{
	\begin{itemize}
	\item ...if you have a bath every morning.
	\item Take a shower instead of a bath.
	\item ...a bath and shower gel.
	\end{itemize}
}
\item verb \\
If you \textbf{bath} someone, especially a child, you wash them in a bath.
 \textbf{Bath} is also a noun .
 \textit{
	\begin{itemize}
	\item Don't feel you have to bath your child every day.
	\item The midwife gave him a warm bath.
	\end{itemize}
}
\item verb \\
When you \textbf{bath} , you have a bath.
 \textit{
	\begin{itemize}
	\item The three children all bath in the same bath water.
	\end{itemize}
}
\item countable noun \\
A \textbf{bath} or a \textbf{baths} is a public building containing a swimming pool, and sometimes other facilities that people can use to have a wash or a bath.
 \textit{
	\begin{itemize}
	\item One of the most important buildings in this ruined city is a public bath.
	\item As well as a Roman amphitheatre and baths, the town has two superb museums.
	\end{itemize}
}
\item countable noun \\
A \textbf{bath} is a container filled with a particular liquid, such as a dye or an acid, in which particular objects are placed, usually as part of a manufacturing or chemical process.
 \textit{
	\begin{itemize}
	\item ...a developing photograph placed in a bath of fixer.
	\end{itemize}
}
\end{enumerate}

\section*{cognitive}
{\large \color{blue}  }
\subsection*{Explain}
\begin{enumerate}
\item adjective \\
\textbf{Cognitive} means relating to the mental process involved in knowing , learning , and understanding things.
 \textit{
	\begin{itemize}
	\item As children grow older, their cognitive processes become sharper.
	\item ...Vygotsky's theory of cognitive development.
	\end{itemize}
}
\end{enumerate}

\section*{bed}
{\large \color{blue}  beds  bedding  bedded  }
\subsection*{Explain}
\begin{enumerate}
\item countable noun \\
A \textbf{bed} is a piece of furniture that you lie on when you sleep.
 \textit{
	\begin{itemize}
	\item She went into her bedroom and lay down on the bed.
	\item We finally went to bed at about 4am.
	\item By the time we got back from dinner, Nona was already in bed.
	\item When she had gone, Sam and Robina put the children to bed.
	\end{itemize}
}
\item countable noun \\
If a place such as a hospital or a hotel has a particular number of \textbf{beds} , it is able to hold that number of patients or guests .
 \textit{
	\begin{itemize}
	\end{itemize}
}
\item countable noun \\
A \textbf{bed} in a garden or park is an area of ground that has been specially prepared so that plants can be grown
in it.
 \textit{
	\begin{itemize}
	\item The geraniums in the flower bed looked bedraggled from the heavy rain.
	\item ...beds of strawberries and rhubarb.
	\end{itemize}
}
\item countable noun \\
A \textbf{bed} of shellfish or plants is an area in the sea or in a lake where a particular type of shellfish
or plant is found in large quantities.
 \textit{
	\begin{itemize}
	\item Fishermen fear valuable oyster and mussel beds could be decimated.
	\item The whole lake was rimmed with thick beds of reeds.
	\end{itemize}
}
\item countable noun \\
The sea \textbf{bed} or a river \textbf{bed} is the ground at the bottom of the sea or of a river.
 \textit{
	\begin{itemize}
	\item For three weeks a big operation went on to recover the wreckage from the sea bed.
	\item ...the bare bed of a dry stream.
	\end{itemize}
}
\item countable noun \\
A \textbf{bed} of rock is a layer of rock that is found within a larger area of rock.
 \textit{
	\begin{itemize}
	\item Between the white limestone and the greyish pink limestone is a thin bed of clay.
	\item ...a sandstone bed.
	\end{itemize}
}
\item countable noun \\
If a recipe or a menu  says that something is served on a \textbf{bed}  \textbf{of} a food such as rice or vegetables , it means it is served on a layer of that food.
 \textit{
	\begin{itemize}
	\item Heat the curry thoroughly and serve it on a bed of rice.
	\end{itemize}
}
\item  \\
 get sb into bed \textit{
	\begin{itemize}
	\end{itemize}
}
\item  \\
 go to bed \textit{
	\begin{itemize}
	\end{itemize}
}
\item  \\
 in bed \textit{
	\begin{itemize}
	\end{itemize}
}
\item  \\
 in bed \textit{
	\begin{itemize}
	\end{itemize}
}
\item  \\
 made one's bed...lie in it \textit{
	\begin{itemize}
	\end{itemize}
}
\item  \\
 make the bed/make sb's bed/make a bed \textit{
	\begin{itemize}
	\end{itemize}
}
\item  \\
 got out of bed on the wrong side \textit{
	\begin{itemize}
	\end{itemize}
}
\end{enumerate}

\section*{comparative}
{\large \color{blue}  comparatives  }
\subsection*{Explain}
\begin{enumerate}
\item adjective \\
You use \textbf{comparative} to show that you are judging something against a previous or different  situation . For example , \textbf{comparative}  calm is a situation which is calmer than before or calmer than the situation in other
places.
 \textit{
	\begin{itemize}
	\item ...those who manage to reach the comparative safety of Fendel.
	\item The task was accomplished with comparative ease.
	\end{itemize}
}
\item adjective \\
A \textbf{comparative}  study is a study that involves the comparison of two or more things of the same kind .
 \textit{
	\begin{itemize}
	\item ...a comparative study of the dietary practices of people from various regions.
	\item ...a professor of English and comparative literature.
	\end{itemize}
}
\item adjective \\
In grammar , the \textbf{comparative} form of an adjective or adverb shows that something has more of a quality than something else has. For example,
' bigger ' is the comparative form of 'big', and 'more quickly' is the comparative form of
'quickly'. Compare  superlative .
 \textbf{Comparative} is also a noun .
 \textit{
	\begin{itemize}
	\item The comparative of 'pretty' is 'prettier'.
	\end{itemize}
}
\end{enumerate}

\section*{belly}
{\large \color{blue}  bellies  }
\subsection*{Explain}
\begin{enumerate}
\item countable noun \\
The \textbf{belly} of a person or animal is their stomach or abdomen. In British English, this is an
 informal or literary use.
 \textit{
	\begin{itemize}
	\item She laid her hands on her swollen belly.
	\item You'll eat so much your belly'll be like a barrel.
	\end{itemize}
}
\item  \\
 belly up \textit{
	\begin{itemize}
	\end{itemize}
}
\end{enumerate}

\section*{car}
{\large \color{blue}  cars  }
\subsection*{Explain}
\begin{enumerate}
\item countable noun \\
A \textbf{car} is a motor vehicle with room for a small number of passengers.
 \textit{
	\begin{itemize}
	\item He had left his tickets in his car.
	\item They arrived by car.
	\end{itemize}
}
\item countable noun \\
A \textbf{car} is one of the separate sections of a train .
 \textit{
	\begin{itemize}
	\item Tour buses have replaced railway cars.
	\end{itemize}
}
\item countable noun \\
Railway carriages are called  \textbf{cars} when they are used for a particular purpose .
 \textit{
	\begin{itemize}
	\item He made his way into the dining car for breakfast.
	\end{itemize}
}
\end{enumerate}

\section*{dental}
{\large \color{blue}  }
\subsection*{Explain}
\begin{enumerate}
\item adjective \\
\textbf{Dental} is used to describe things that relate to teeth or to the care and treatment of teeth.
 \textit{
	\begin{itemize}
	\item You can get free dental treatment.
	\item ...the dental profession.
	\end{itemize}
}
\end{enumerate}

\section*{condition}
{\large \color{blue}  conditions  conditioning  conditioned  }
\subsection*{Explain}
\begin{enumerate}
\item singular noun \\
If you talk about the \textbf{condition} of a person or thing, you are talking about the state that they are in, especially how good or bad their physical state is.
 \textit{
	\begin{itemize}
	\item He remains in a critical condition in a California hospital.
	\item I received several compliments on the condition of my skin.
	\item The two-bedroom chalet is in good condition.
	\item You can't drive in that condition.
	\end{itemize}
}
\item plural noun \\
The \textbf{conditions} under which something is done or happens are all the factors or circumstances which directly affect it.
 \textit{
	\begin{itemize}
	\item This change has been timed under laboratory conditions.
	\item The mild winter has created the ideal conditions for an ant population explosion.
	\item The conditions are ripe for the spread of disease.
	\end{itemize}
}
\item plural noun \\
The \textbf{conditions} in which people live or work are the factors which affect their comfort , safety , or health.
 \textit{
	\begin{itemize}
	\item People are living in appalling conditions.
	\item He could not work in these conditions any longer.
	\item The conditions in the camp are just awful.
	\end{itemize}
}
\item singular noun \\
The \textbf{condition} of a group of people is their situation in life, especially with regard to the difficulties they have.
 \textit{
	\begin{itemize}
	\item The condition of the people could be elevated by a programme of social reform.
	\item The government has encouraged its people to better their condition.
	\item ...the human condition.
	\end{itemize}
}
\item countable noun \\
A \textbf{condition} is something which must happen or be done in order for something else to be possible , especially when this is written into a contract or law.
 \textit{
	\begin{itemize}
	\item ...economic targets set as a condition for loan payments.
	\item ...terms and conditions of employment.
	\item Egypt had agreed to a summit subject to certain conditions.
	\end{itemize}
}
\item countable noun \\
If someone has a particular \textbf{condition} , they have an illness or other medical problem.
 \textit{
	\begin{itemize}
	\item Doctors suspect he may have a heart condition.
	\item ...a rare condition that causes degeneration of the brain tissue.
	\end{itemize}
}
\item verb \\
If someone \textbf{is conditioned} by their experiences or environment , they are influenced by them over a period of time so that they do certain things
or think in a particular way.
 \textit{
	\begin{itemize}
	\item We are all conditioned by early impressions and experiences.
	\item You have been conditioned to believe that it is weak to be scared.
	\item People are conditioned into believing that they have no power over their situation.
	\item ...a conditioned response.
	\end{itemize}
}
\item verb \\
To \textbf{condition} your hair or skin means to put something on it which will keep it in good condition.
 \textit{
	\begin{itemize}
	\item ...a protein which is excellent for conditioning dry and damaged hair.
	\end{itemize}
}
\item  \\
 in no condition \textit{
	\begin{itemize}
	\end{itemize}
}
\item  \\
 on condition that \textit{
	\begin{itemize}
	\end{itemize}
}
\item  \\
 out of condition \textit{
	\begin{itemize}
	\end{itemize}
}
\end{enumerate}

\section*{conscience}
{\large \color{blue}  consciences  }
\subsection*{Explain}
\begin{enumerate}
\item  \\
 guilty conscience \textit{
	\begin{itemize}
	\end{itemize}
}
\item uncountable noun \\
\textbf{Conscience} is doing what you believe is right even though it might be unpopular , difficult , or dangerous .
 \textit{
	\begin{itemize}
	\item He refused for reasons of conscience to sign a new law legalising abortion.
	\item ...the law on freedom of conscience and religious organizations.
	\end{itemize}
}
\item uncountable noun \\
\textbf{Conscience} is a feeling of guilt because you know you have done something that is wrong.
 \textit{
	\begin{itemize}
	\item I'm so glad he had a pang of conscience.
	\item They have shown a ruthless lack of conscience.
	\end{itemize}
}
\item  \\
 in good conscience/in all conscience \textit{
	\begin{itemize}
	\end{itemize}
}
\item  \\
 on your conscience \textit{
	\begin{itemize}
	\end{itemize}
}
\end{enumerate}

\section*{enormous}
{\large \color{blue}  }
\subsection*{Explain}
\begin{enumerate}
\item adjective \\
Something that is \textbf{enormous} is extremely large in size or amount.
 \textit{
	\begin{itemize}
	\item The main bedroom is enormous.
	\item There is, of course, an enormous amount to see.
	\end{itemize}
}
\item adjective \\
You can use \textbf{enormous} to emphasize the great degree or extent of something.
 \textit{
	\begin{itemize}
	\item It was an enormous disappointment.
	\end{itemize}
}
\end{enumerate}

\section*{corner}
{\large \color{blue}  corners  cornering  cornered  }
\subsection*{Explain}
\begin{enumerate}
\item countable noun \\
A \textbf{corner} is a point or an area where two or more edges, sides, or surfaces of something join.
 \textit{
	\begin{itemize}
	\item He saw the corner of a magazine sticking out from under the blanket.
	\item Write 'By Airmail' in the top left-hand corner.
	\end{itemize}
}
\item countable noun \\
The \textbf{corner} of a room, box, or similar space is the area inside it where its edges or walls meet.
 \textit{
	\begin{itemize}
	\item ...a card table in the corner of the living room.
	\item The ball hurtled into the far corner of the net.
	\item Finally I spotted it, in a dark corner over by the piano.
	\end{itemize}
}
\item countable noun \\
The \textbf{corner}  \textbf{of} your mouth or eye is the side of it.
 \textit{
	\begin{itemize}
	\item She flicked a crumb off the corner of her mouth.
	\item Out of the corner of her eye she saw that a car had stopped.
	\end{itemize}
}
\item countable noun \\
The \textbf{corner} of a street is the place where one of its sides ends as it joins another street.
 \textit{
	\begin{itemize}
	\item She would spend the day hanging round street corners.
	\item We can't have police officers on every corner.
	\item He waited until the man had turned a corner.
	\end{itemize}
}
\item countable noun \\
A \textbf{corner} is a bend in a road.
 \textit{
	\begin{itemize}
	\item ...a sharp corner.
	\item The road is a succession of hairpin bends, hills, and blind corners.
	\end{itemize}
}
\item countable noun \\
If you talk about the \textbf{corners}  \textbf{of} the world, a country, or some other place, you are referring to places that are far away or difficult to get to.
 \textit{
	\begin{itemize}
	\item Buyers came from all corners of the world.
	\item The group has been living in a remote corner of the Cambodian jungle.
	\end{itemize}
}
\item countable noun \\
In football , hockey , and some other sports, a \textbf{corner} is a free shot or kick taken from the corner of the pitch .
 \textit{
	\begin{itemize}
	\end{itemize}
}
\item verb \\
If you \textbf{corner} a person or animal, you force them into a place they cannot escape from.
 \textit{
	\begin{itemize}
	\item A police motor-cycle chased his car twelve miles, and cornered him near Rome.
	\item He was still sitting huddled like a cornered animal.
	\end{itemize}
}
\item verb \\
If you \textbf{corner} someone, you force them to speak to you when they have been trying to avoid you.
 \textit{
	\begin{itemize}
	\item Golan managed to corner the young producer-director for an interview.
	\end{itemize}
}
\item verb \\
If a company or place \textbf{corners} an area of trade, they gain control over it so that no one else can have any success in that area.
 \textit{
	\begin{itemize}
	\item This restaurant has cornered the Madrid market for specialist paellas.
	\item Zurich's affluence came initially from cornering a sizeable chunk of the 14th Century
silk trade.
	\end{itemize}
}
\item verb \\
If a car, or the person driving it, \textbf{corners} in a particular way, the car goes round bends in roads in this way.
 \textit{
	\begin{itemize}
	\item Peter drove jerkily, cornering too fast and fumbling the gears.
	\end{itemize}
}
\item  \\
 around the corner \textit{
	\begin{itemize}
	\end{itemize}
}
\item  \\
 around the corner/round the corner \textit{
	\begin{itemize}
	\end{itemize}
}
\item  \\
 to cut corners \textit{
	\begin{itemize}
	\end{itemize}
}
\item  \\
 the four corners of \textit{
	\begin{itemize}
	\end{itemize}
}
\item  \\
 in a corner/in a tight corner \textit{
	\begin{itemize}
	\end{itemize}
}
\end{enumerate}

\section*{fair}
{\large \color{blue}  fairer  fairest  fairs  }
\subsection*{Explain}
\begin{enumerate}
\item adjective \\
Something or someone that is \textbf{fair} is reasonable , right, and just.
 \textit{
	\begin{itemize}
	\item It didn't seem fair to leave out her father.
	\item Do you feel they're paying their fair share?
	\item Independent observers say the campaign's been very much fairer than expected.
	\item I wanted them to get a fair deal.
	\item He claims that he would not get a fair trial.
	\end{itemize}
}
\item adjective \\
A \textbf{fair} amount, degree, size, or distance is quite a large amount, degree, size, or distance.
 \textit{
	\begin{itemize}
	\item My neighbours across the street travel a fair amount.
	\item My mother's brother lives a fair distance away so we don't see him and his family
very often.
	\end{itemize}
}
\item adjective \\
A \textbf{fair}  guess or idea about something is one that is likely to be correct .
 \textit{
	\begin{itemize}
	\item It's a fair guess to say that the damage will be extensive.
	\item I have a fair idea of how difficult things can be.
	\end{itemize}
}
\item adjective \\
If you describe someone or something as \textbf{fair} , you mean that they are average in standard or quality, neither very good nor very bad .
 \textit{
	\begin{itemize}
	\item Reimar had a fair command of English.
	\end{itemize}
}
\item adjective \\
Someone who is \textbf{fair} , or who has \textbf{fair} hair, has light-coloured hair.
 \textbf{Fair} is also a combining form.
 \textit{
	\begin{itemize}
	\item Both children were very like Robina, but were much fairer than she was.
	\item ...a tall, fair-haired Englishman.
	\end{itemize}
}
\item adjective \\
\textbf{Fair} skin is very pale and usually burns  easily .
 \textbf{Fair} is also a combining form.
 \textit{
	\begin{itemize}
	\item It's important to protect my fair skin from the sun.
	\item Fair-skinned people shouldn't spend a great deal of time in the sun.
	\end{itemize}
}
\item adjective \\
When the weather is \textbf{fair} , it is quite sunny and not raining .
 \textit{
	\begin{itemize}
	\item Weather conditions were fair.
	\end{itemize}
}
\item countable noun \\
A county, state, or country \textbf{fair} is an event where there are, for example, displays of goods and animals, and amusements,
games, and competitions .
 \textit{
	\begin{itemize}
	\end{itemize}
}
\item countable noun \\
A \textbf{fair} is the same as a funfair .
 \textit{
	\begin{itemize}
	\end{itemize}
}
\item countable noun \\
A \textbf{fair} is an event at which people display and sell goods, especially goods of a particular type.
 \textit{
	\begin{itemize}
	\item ...an antiques fair.
	\end{itemize}
}
\item  \\
 (to) be fair \textit{
	\begin{itemize}
	\end{itemize}
}
\item  \\
 fair enough \textit{
	\begin{itemize}
	\end{itemize}
}
\item  \\
 fair enough \textit{
	\begin{itemize}
	\end{itemize}
}
\item  \\
 to play fair \textit{
	\begin{itemize}
	\end{itemize}
}
\item  \\
 it's fair to say \textit{
	\begin{itemize}
	\end{itemize}
}
\item  \\
 fair and square \textit{
	\begin{itemize}
	\end{itemize}
}
\end{enumerate}

\section*{delight}
{\large \color{blue}  delights  delighting  delighted  }
\subsection*{Explain}
\begin{enumerate}
\item uncountable noun \\
\textbf{Delight} is a feeling of very great pleasure.
 \textit{
	\begin{itemize}
	\item Throughout the house, the views are a constant source of surprise and delight.
	\item Andrew roared with delight when he heard Rachel's nickname for the baby.
	\item To my great delight, it worked perfectly.
	\end{itemize}
}
\item  \\
 take delight in/take a delight in \textit{
	\begin{itemize}
	\end{itemize}
}
\item countable noun \\
You can  refer to someone or something that gives you great pleasure or enjoyment as a \textbf{delight} .
 \textit{
	\begin{itemize}
	\item Isn't she a delight?
	\item The aircraft was a delight to fly.
	\item Sampling the local cuisine is one of the delights of a holiday abroad.
	\end{itemize}
}
\item verb \\
If something \textbf{delights} you, it gives you a lot of pleasure.
 \textit{
	\begin{itemize}
	\item She has created a style of music that has delighted audiences all over the world.
	\item The report has delighted environmentalists.
	\end{itemize}
}
\item verb \\
If you \textbf{delight}  \textbf{in} something, you get a lot of pleasure from it.
 \textit{
	\begin{itemize}
	\item Generations of adults and children have delighted in the story.
	\item He delighted in sharing his love of birds with children.
	\end{itemize}
}
\end{enumerate}

\section*{fierce}
{\large \color{blue}  fiercer  fiercest  }
\subsection*{Explain}
\begin{enumerate}
\item adjective \\
A \textbf{fierce} animal or person is very aggressive or angry .
 \textit{
	\begin{itemize}
	\item They look like the teeth of some fierce animal.
	\end{itemize}
}
\item adjective \\
\textbf{Fierce}  feelings or actions are very intense or enthusiastic , or involve great activity.
 \textit{
	\begin{itemize}
	\item Standards are high and competition is fierce.
	\item The town was captured after a fierce battle with rebels at the weekend.
	\item He inspires fierce loyalty in his friends.
	\end{itemize}
}
\item adjective \\
\textbf{Fierce} conditions are very intense, great, or strong.
 \textit{
	\begin{itemize}
	\item The climbers were trapped by a fierce storm which went on for days.
	\end{itemize}
}
\end{enumerate}

\section*{desert}
{\large \color{blue}  deserts  deserting  deserted  }
\subsection*{Explain}
\begin{enumerate}
\item variable noun \\
A \textbf{desert} is a large area of land, usually in a hot region, where there is almost no water, rain , trees, or plants.
 \textit{
	\begin{itemize}
	\item ...the Sahara Desert.
	\item ...the burning desert sun.
	\item The vehicles have been modified to suit conditions in the desert.
	\end{itemize}
}
\item countable noun \\
If you refer to a place or situation as \textbf{a}  \textbf{desert} , you think it is bad for people because it is not interesting, exciting , or useful in any way.
 \textit{
	\begin{itemize}
	\item They live in 12 high-rise apartment buildings that sit in a desert of concrete.
	\end{itemize}
}
\item verb \\
If people or animals \textbf{desert} a place, they leave it and it becomes empty .
 \textit{
	\begin{itemize}
	\item Farmers are deserting their fields and coming here looking for jobs.
	\item After the show, the audience deserts the Blackpool streets.
	\end{itemize}
}
\item verb \\
If someone \textbf{deserts} you, they go away and leave you, and no longer help or support you.
 \textit{
	\begin{itemize}
	\item Mrs Roding's husband deserted her years ago.
	\item He has been deserted by most of his advisers.
	\end{itemize}
}
\item verb \\
If you \textbf{desert} something that you support, use, or are involved with, you stop supporting it, using it, or being involved with it.
 \textit{
	\begin{itemize}
	\item The paper's price rise will encourage readers to desert in even greater numbers.
	\item He was pained to see many youngsters deserting kibbutz life.
	\item Discerning shoppers are deserting supermarkets for artisan bakers.
	\end{itemize}
}
\item verb \\
If a quality or skill that you normally have \textbf{deserts} you, you suddenly  find that you do not have it when you need it or want it.
 \textit{
	\begin{itemize}
	\item Even when he appeared to be depressed, a dry sense of humour never deserted him.
	\item She lost the next five games, and the set, as her confidence abruptly deserted her.
	\end{itemize}
}
\item verb \\
If someone \textbf{deserts} , or \textbf{deserts} a job , especially a job in the armed forces, they leave that job without permission .
 \textit{
	\begin{itemize}
	\item He was a second-lieutenant in the army until he deserted.
	\item He deserted from army intelligence last month.
	\end{itemize}
}
\item  \\
 to get your just deserts \textit{
	\begin{itemize}
	\end{itemize}
}
\end{enumerate}

\section*{diet}
{\large \color{blue}  diets  dieting  dieted  }
\subsection*{Explain}
\begin{enumerate}
\item variable noun \\
Your \textbf{diet} is the type and range of food that you regularly eat .
 \textit{
	\begin{itemize}
	\item It's never too late to improve your diet.
	\item ...a healthy diet rich in fruit and vegetables.
	\item Poor diet and excess smoking will seriously damage the health of your hair.
	\end{itemize}
}
\item countable noun \\
If a doctor  puts someone on a \textbf{diet} , he or she makes them eat a special type or range of foods in order to improve their health .
 \textit{
	\begin{itemize}
	\item He was put on a diet of milky food.
	\item ...a special diet for children with high cholesterol.
	\end{itemize}
}
\item variable noun \\
If you are on a \textbf{diet} , you eat special kinds of food or you eat less food than usual because you are trying to lose weight.
 \textit{
	\begin{itemize}
	\item Have you been on a diet? You've lost a lot of weight.
	\item Diet and exercise will alter your shape.
	\item I've only lost sixteen pounds since I started this diet.
	\end{itemize}
}
\item verb \\
If you \textbf{are dieting} , you eat special kinds of food or you eat less food than usual because you are trying
to lose weight.
 \textit{
	\begin{itemize}
	\item I've been dieting ever since the birth of my fourth child.
	\item Most of us have dieted at some time in our lives.
	\end{itemize}
}
\item adjective \\
\textbf{Diet} drinks or foods have been specially produced so that they do not contain many calories .
 \textit{
	\begin{itemize}
	\item ...sugar-free diet drinks.
	\item ...diet margarine.
	\end{itemize}
}
\item countable noun \\
If you are fed on a \textbf{diet}  \textbf{of} something, especially something unpleasant or of poor quality, you receive or experience a very large amount of it.
 \textit{
	\begin{itemize}
	\item The radio had fed him a diet of pop songs.
	\item People are rejecting this constant diet of despair.
	\end{itemize}
}
\end{enumerate}

\section*{gigantic}
{\large \color{blue}  }
\subsection*{Explain}
\begin{enumerate}
\item adjective \\
If you describe something as \textbf{gigantic} , you are emphasizing that it is extremely large in size, amount, or degree.
 \textit{
	\begin{itemize}
	\item ...gigantic rocks.
	\item A gigantic task of national reconstruction awaits us.
	\end{itemize}
}
\end{enumerate}

\section*{disaster}
{\large \color{blue}  disasters  }
\subsection*{Explain}
\begin{enumerate}
\item countable noun \\
A \textbf{disaster} is a very bad  accident such as an earthquake or a plane  crash , especially one in which a lot of people are killed .
 \textit{
	\begin{itemize}
	\item It was the second air disaster in the region in less than two months.
	\item Many had lost all in the disaster and were destitute.
	\end{itemize}
}
\item countable noun \\
If you refer to something as a \textbf{disaster} , you are emphasizing that you think it is extremely bad or unacceptable .
 \textit{
	\begin{itemize}
	\item The whole production was just a disaster!
	\item It would be a disaster for them not to reach the semi-finals.
	\item 'This tax is a disaster waiting to happen,' said an angry Tory backbencher.
	\end{itemize}
}
\item uncountable noun \\
\textbf{Disaster} is something which has very bad consequences for you.
 \textit{
	\begin{itemize}
	\item The government brought itself to the brink of fiscal disaster.
	\item 'The potential for disaster is enormous,' he says.
	\end{itemize}
}
\item  \\
 a recipe for disaster \textit{
	\begin{itemize}
	\end{itemize}
}
\end{enumerate}

\section*{handy}
{\large \color{blue}  handier  handiest  }
\subsection*{Explain}
\begin{enumerate}
\item adjective \\
Something that is \textbf{handy} is useful .
 \textit{
	\begin{itemize}
	\item The book gives handy hints on looking after indoor plants.
	\item Credit cards can be handy–they mean you do not have to carry large sums of cash.
	\end{itemize}
}
\item  \\
 come in handy \textit{
	\begin{itemize}
	\end{itemize}
}
\item adjective \\
A thing or place that is \textbf{handy} is nearby and therefore easy to get or reach.
 \textit{
	\begin{itemize}
	\item It would be good to have a pencil and paper handy.
	\item Keep handy a lightweight sweater or cardigan.
	\item This lively town is handy for Londoners.
	\end{itemize}
}
\item adjective \\
Someone who is \textbf{handy with} a particular tool is skilful at using it.
 \textit{
	\begin{itemize}
	\item If you're handy with a needle you could brighten up your sweater with giant daisies.
	\end{itemize}
}
\end{enumerate}

\section*{ditch}
{\large \color{blue}  ditches  ditching  ditched  }
\subsection*{Explain}
\begin{enumerate}
\item countable noun \\
A \textbf{ditch} is a long narrow channel cut into the ground at the side of a road or field .
 \textit{
	\begin{itemize}
	\end{itemize}
}
\item verb \\
If you \textbf{ditch} something that you have or are responsible for, you abandon it or get  rid of it, because you no longer want it.
 \textit{
	\begin{itemize}
	\item I decided to ditch the sofa bed.
	\end{itemize}
}
\item verb \\
If someone \textbf{ditches} someone, they end a relationship with that person.
 \textit{
	\begin{itemize}
	\item I can't bring myself to ditch him and start again.
	\end{itemize}
}
\item verb \\
If a pilot  \textbf{ditches} an aircraft or if it \textbf{ditches} , the pilot makes an emergency landing .
 \textit{
	\begin{itemize}
	\item One American pilot was forced to ditch his jet in the Gulf.
	\item A survivor was knocked unconscious when the helicopter ditched.
	\end{itemize}
}
\end{enumerate}

\section*{historical}
{\large \color{blue}  }
\subsection*{Explain}
\begin{enumerate}
\item adjective \\
\textbf{Historical} people, situations , or things existed in the past and are considered to be a part of history.
 \textit{
	\begin{itemize}
	\item ...an important historical figure.
	\item ...the historical impact of Western capitalism on the world.
	\item In Buda, several historical monuments can be seen.
	\end{itemize}
}
\item adjective \\
\textbf{Historical} books, films, or pictures  describe or represent people, situations, or things that existed in the past.
 \textit{
	\begin{itemize}
	\item He is writing a historical novel about nineteenth-century France.
	\item ...another great Eisenstein historical film.
	\end{itemize}
}
\item adjective \\
\textbf{Historical}  information , research , and discussion is related to the study of history.
 \textit{
	\begin{itemize}
	\item ...historical records.
	\item ...modern historical research.
	\end{itemize}
}
\item adjective \\
If you look at an event within a \textbf{historical}  context , you look at what was happening at that time and what had happened previously, in order to judge the event and its importance .
 \textit{
	\begin{itemize}
	\item It was this kind of historical context that Morris brought to his work.
	\item The Telegraph puts the union in a historical perspective.
	\end{itemize}
}
\end{enumerate}

\section*{elbow}
{\large \color{blue}  elbows  elbowing  elbowed  }
\subsection*{Explain}
\begin{enumerate}
\item countable noun \\
Your \textbf{elbow} is the part of your arm where the upper and lower  halves of the arm are joined .
 \textit{
	\begin{itemize}
	\item He slipped and fell, badly bruising an elbow.
	\end{itemize}
}
\item verb \\
If you \textbf{elbow} people \textbf{aside} or \textbf{elbow} your \textbf{way}  somewhere , you push people with your elbows in order to move somewhere.
 \textit{
	\begin{itemize}
	\item They also claim that the security team elbowed aside a steward.
	\item Mr Smith elbowed me in the face.
	\item Brand elbowed his way to the centre of the group of bystanders.
	\end{itemize}
}
\item verb \\
If someone or something \textbf{elbows} their \textbf{way} somewhere, or \textbf{elbows} other people or things \textbf{out of the way} , they achieve  success by being aggressive and determined .
 \textit{
	\begin{itemize}
	\item Non-state firms gradually elbow aside the inefficient state-owned ones.
	\item Environmental concerns will elbow their way right to the top of the agenda.
	\end{itemize}
}
\end{enumerate}

\section*{huge}
{\large \color{blue}  huger  hugest  }
\subsection*{Explain}
\begin{enumerate}
\item adjective \\
Something or someone that is \textbf{huge} is extremely large in size.
 \textit{
	\begin{itemize}
	\item ...a tiny little woman with huge black glasses.
	\item Several painters were working on a huge piece of canvas which would serve as the
scenery.
	\item Our driver strolled up, huge and swarthy.
	\end{itemize}
}
\item adjective \\
Something that is \textbf{huge} is extremely large in amount or degree .
 \textit{
	\begin{itemize}
	\item I have a huge number of ties because I never throw them away.
	\end{itemize}
}
\item adjective \\
Something that is \textbf{huge}  exists or happens on a very large scale , and involves a lot of different people or things.
 \textit{
	\begin{itemize}
	\item Another team is looking at the huge problem of debts between companies.
	\item The result was human suffering on a huge scale.
	\end{itemize}
}
\end{enumerate}

\section*{finish}
{\large \color{blue}  finishes  finishing  finished  }
\subsection*{Explain}
\begin{enumerate}
\item verb \\
When you \textbf{finish} doing or dealing with something, you do or deal with the last part of it, so that there is no more for you to do or deal with.
 \textbf{Finish up} means the same as finish .
 \textit{
	\begin{itemize}
	\item As soon as he'd finished eating, he excused himself.
	\item Mr Gould was given a standing ovation and loud cheers when he finished his speech.
	\item I've practically finished the ironing.
	\item We waited a few minutes outside his office while he finished up his meeting.
	\end{itemize}
}
\item verb \\
When you \textbf{finish} something that you are making or producing, you reach the end of making or producing it, so that it is complete.
 \textbf{Finish off} and, in American English, \textbf{finish up} mean the same as finish .
 \textit{
	\begin{itemize}
	\item The consultants had been working to finish a report this week.
	\item Now she is busy finishing off a biography of Queen Caroline.
	\item ...the amount of stuff required to finish up a movie.
	\end{itemize}
}
\item verb \\
When something such as a course, film, or sale  \textbf{finishes} , especially at a planned time, it ends.
 \textit{
	\begin{itemize}
	\item The teaching day finishes at around 4pm.
	\item When a play finishes its run, the costumes are hired out to amateur dramatics companies.
	\end{itemize}
}
\item verb \\
You say that someone or something \textbf{finishes} a period of time or an event in a particular way to indicate what the final situation was like. You can also say that a period of time or an event \textbf{finishes} in a particular way.
 \textit{
	\begin{itemize}
	\item The two of them finished by kissing each other goodbye.
	\item The evening finished with the welcoming of three new members.
	\item They finished the meal with a fresh fruit salad.
	\item He finished the day two holes up.
	\item The last track finishes this compilation beautifully.
	\end{itemize}
}
\item verb \\
If someone \textbf{finishes} second, for example , in a race or competition , they are in second place at the end of the race or competition.
 \textit{
	\begin{itemize}
	\item He finished second in the championship four years in a row.
	\end{itemize}
}
\item verb \\
To \textbf{finish} means to reach the end of saying something.
 \textit{
	\begin{itemize}
	\item Her eyes flashed, but he held up a hand. 'Let me finish.'
	\end{itemize}
}
\item singular noun \\
\textbf{The finish} of something is the end of it or the last part of it.
 \textit{
	\begin{itemize}
	\item I intend to continue it and see the job through to the finish.
	\item From start to finish he believed in me, often more than I did myself.
	\end{itemize}
}
\item countable noun \\
The \textbf{finish} of a race is the end of it.
 \textit{
	\begin{itemize}
	\item Win a trip to see the finish of the Tour de France!
	\item It was a close finish but I won.
	\end{itemize}
}
\item countable noun \\
If the surface of something that has been made has a particular kind of \textbf{finish} , it has the appearance or texture mentioned .
 \textit{
	\begin{itemize}
	\item The finish and workmanship of the woodwork was excellent.
	\end{itemize}
}
\item  \\
 fight to the finish \textit{
	\begin{itemize}
	\end{itemize}
}
\item  \\
 the finishing touch \textit{
	\begin{itemize}
	\end{itemize}
}
\end{enumerate}

\section*{illiterate}
{\large \color{blue}  illiterates  }
\subsection*{Explain}
\begin{enumerate}
\item adjective \\
Someone who is \textbf{illiterate} does not know how to read or write.
 An \textbf{illiterate} is someone who is illiterate.
 \textit{
	\begin{itemize}
	\item A large percentage of the population is illiterate.
	\item ...an educational centre for illiterates.
	\end{itemize}
}
\item adjective \\
If you describe someone as musically, technologically, or economically  \textbf{illiterate} , you mean that they do not know much about music, technology , or economics .
 \textit{
	\begin{itemize}
	\end{itemize}
}
\end{enumerate}

\section*{interval}
{\large \color{blue}  intervals  }
\subsection*{Explain}
\begin{enumerate}
\item countable noun \\
An \textbf{interval} between two events or dates is the period of time between them.
 \textit{
	\begin{itemize}
	\item The ferry service has restarted after an interval of 12 years.
	\item There was a long interval of silence.
	\end{itemize}
}
\item countable noun \\
An \textbf{interval} during a film, concert, show , or game is a short break between two of the parts.
 \textit{
	\begin{itemize}
	\item During the interval, wine was served.
	\item England were two goals behind at the interval.
	\end{itemize}
}
\item countable noun \\
In music , an \textbf{interval} is the difference in pitch between two musical notes.
 \textit{
	\begin{itemize}
	\end{itemize}
}
\item  \\
 at intervals \textit{
	\begin{itemize}
	\end{itemize}
}
\item  \\
 at intervals \textit{
	\begin{itemize}
	\end{itemize}
}
\end{enumerate}

\section*{inner}
{\large \color{blue}  }
\subsection*{Explain}
\begin{enumerate}
\item adjective \\
The \textbf{inner} parts of something are the parts which are contained or are enclosed inside the other parts, and which are closest to the centre.
 \textit{
	\begin{itemize}
	\item She got up and went into an inner office.
	\item Wade stepped inside and closed the inner door behind him.
	\end{itemize}
}
\item adjective \\
Your \textbf{inner}  feelings are feelings which you have but do not show to other people.
 \textit{
	\begin{itemize}
	\item Loving relationships will give a child an inner sense of security.
	\item Michael needed to express his inner tensions.
	\end{itemize}
}
\end{enumerate}

\section*{island}
{\large \color{blue}  islands  }
\subsection*{Explain}
\begin{enumerate}
\item countable noun \\
An \textbf{island} is a piece of land that is completely surrounded by water.
 \textit{
	\begin{itemize}
	\item ...a wonderful day trip to the picturesque island of Gozo.
	\item We spent a day on Caldey Island.
	\item ...the Canary Islands.
	\end{itemize}
}
\end{enumerate}

\section*{interim}
{\large \color{blue}  }
\subsection*{Explain}
\begin{enumerate}
\item adjective \\
\textbf{Interim} is used to describe something that is intended to be used until something permanent is done or established .
 \textit{
	\begin{itemize}
	\item She was sworn in as head of an interim government in March.
	\item ...an interim report.
	\end{itemize}
}
\item  \\
 in the interim \textit{
	\begin{itemize}
	\end{itemize}
}
\end{enumerate}

\section*{item}
{\large \color{blue}  items  }
\subsection*{Explain}
\begin{enumerate}
\item countable noun \\
An \textbf{item} is one of a collection or list of objects.
 \textit{
	\begin{itemize}
	\item The most valuable item on show will be a Picasso drawing.
	\item The menu includes the occasional off-beat item.
	\end{itemize}
}
\item countable noun \\
An \textbf{item} is one of a list of things for someone to do, deal with, or talk about.
 \textit{
	\begin{itemize}
	\item The other item on the agenda is the tour.
	\end{itemize}
}
\item countable noun \\
An \textbf{item} is a report or article in a newspaper or magazine , or on television or radio .
 \textit{
	\begin{itemize}
	\item There was an item in the paper about him.
	\end{itemize}
}
\item singular noun \\
If you say that two people are an \textbf{item} , you mean that they are having a romantic or sexual relationship.
 \textit{
	\begin{itemize}
	\item She and Gino were an item.
	\end{itemize}
}
\end{enumerate}

\section*{interior}
{\large \color{blue}  interiors  }
\subsection*{Explain}
\begin{enumerate}
\item countable noun \\
The \textbf{interior} of something is the inside part of it.
 \textit{
	\begin{itemize}
	\item The interior of the house was furnished with heavy, old-fashioned pieces.
	\item The boat's interior badly needed painting.
	\end{itemize}
}
\item adjective \\
You use \textbf{interior} to describe something that is inside a building or vehicle .
 \textit{
	\begin{itemize}
	\item The interior walls were painted green.
	\item There is more interior space than in some rival cars.
	\item He pulled the car over, turned on the interior light, examined the map.
	\end{itemize}
}
\item singular noun \\
The \textbf{interior} of a country or continent is the central area of it.
 \textit{
	\begin{itemize}
	\item ...a 5-day hike into the interior.
	\item ...the unknown interior of South America.
	\item The Yangzi river would give access to much of China's interior.
	\end{itemize}
}
\item adjective \\
A country's \textbf{interior}  minister , ministry , or department  deals with affairs within that country, such as law and order.
 \textit{
	\begin{itemize}
	\item The French Interior Minister has intervened in a scandal over the role of a secret
police force.
	\end{itemize}
}
\item singular noun \\
A country's minister or ministry of \textbf{the interior} deals with affairs within that country, such as law and order.
 \textit{
	\begin{itemize}
	\item An official from the Ministry of the Interior said six people had died.
	\end{itemize}
}
\item adjective \\
\textbf{Interior}  thoughts or processes go on inside someone's head and are not expressed  aloud .
 \textit{
	\begin{itemize}
	\item We need some sort of interior life if we are to be happy.
	\item ...the mind's interior space.
	\end{itemize}
}
\end{enumerate}

\section*{liver}
{\large \color{blue}  livers  }
\subsection*{Explain}
\begin{enumerate}
\item countable noun \\
Your \textbf{liver} is a large organ in your body which processes your blood and helps to clean  unwanted substances out of it.
 \textit{
	\begin{itemize}
	\end{itemize}
}
\item variable noun \\
\textbf{Liver} is the liver of some animals, especially  lambs , pigs , and cows , which is cooked and eaten .
 \textit{
	\begin{itemize}
	\item ...grilled calves' liver.
	\end{itemize}
}
\end{enumerate}

\section*{intermediate}
{\large \color{blue}  intermediates  }
\subsection*{Explain}
\begin{enumerate}
\item adjective \\
An \textbf{intermediate} stage, level , or position is one that occurs between two other stages, levels, or positions.
 \textit{
	\begin{itemize}
	\item You should consider breaking the journey with intermediate stopovers at airport hotels.
	\item ...hourly trains to Perugia, Assisi, and intermediate stations.
	\end{itemize}
}
\item adjective \\
\textbf{Intermediate} learners of something have some knowledge or skill but are not yet advanced .
 An \textbf{intermediate} is an intermediate learner.
 \textit{
	\begin{itemize}
	\item The Badminton Club holds coaching sessions for beginners and intermediate players
on Friday evenings.
	\item The ski school coaches beginners, intermediates, and advanced skiers.
	\end{itemize}
}
\end{enumerate}

\section*{nothing}
{\large \color{blue}  nothings  }
\subsection*{Explain}
\begin{enumerate}
\item pronoun \\
\textbf{Nothing}  means not a single thing, or not a single part of something.
 \textit{
	\begin{itemize}
	\item I've done nothing much since coffee time.
	\item The man knows nothing of history or sociology.
	\item He was dressed in jeans and nothing else.
	\item There is nothing wrong with the car.
	\end{itemize}
}
\item pronoun \\
You use \textbf{nothing} to indicate that something or someone is not important or significant .
 \textbf{Nothing} is also a noun .
 \textit{
	\begin{itemize}
	\item Because he had always had money, it meant nothing to him.
	\item This cold snap is nothing compared to a real winter.
	\item She kept bursting into tears over nothing at work.
	\item Do our years together mean nothing?
	\item It is the picture itself that is the problem; so small, so dull. It's a nothing,
really.
	\item All it took was a word here, a word there, to convince him that he was a nothing.
	\end{itemize}
}
\item pronoun \\
If you say that something cost  \textbf{nothing} or is worth \textbf{nothing} , you are indicating that it cost or is worth a surprisingly small amount of money .
 \textit{
	\begin{itemize}
	\item The furniture was threadbare; he'd obviously picked it up for nothing.
	\item His net UK estate was worth nothing after debts were taken into account.
	\end{itemize}
}
\item pronoun \\
You use \textbf{nothing} before an adjective or 'to'- infinitive to say that something or someone does not have the quality indicated.
 \textit{
	\begin{itemize}
	\item Around the lake the countryside generally is nothing special.
	\item There was nothing remarkable about him.
	\item All kids her age do silly things; it's nothing to worry about.
	\end{itemize}
}
\item pronoun \\
You can use \textbf{nothing} before 'so' and an adjective or adverb , or before a comparative , to emphasize how strong or great a particular quality is.
 \textit{
	\begin{itemize}
	\item Youngsters learn nothing so fast as how to beat the system.
	\item I consider nothing more important in my life than songwriting.
	\item There's nothing better than a good cup of hot coffee.
	\end{itemize}
}
\item  \\
 all or nothing \textit{
	\begin{itemize}
	\end{itemize}
}
\item  \\
 to be better than nothing \textit{
	\begin{itemize}
	\end{itemize}
}
\item  \\
 nothing but \textit{
	\begin{itemize}
	\end{itemize}
}
\item  \\
 nothing doing \textit{
	\begin{itemize}
	\end{itemize}
}
\item  \\
 there is nothing for it \textit{
	\begin{itemize}
	\end{itemize}
}
\item  \\
 nothing if not \textit{
	\begin{itemize}
	\end{itemize}
}
\item  \\
 it's nothing \textit{
	\begin{itemize}
	\end{itemize}
}
\item  \\
 nothing in it/nothing to it \textit{
	\begin{itemize}
	\end{itemize}
}
\item  \\
 nothing to it/nothing in it \textit{
	\begin{itemize}
	\end{itemize}
}
\item  \\
 nothing in it \textit{
	\begin{itemize}
	\end{itemize}
}
\item  \\
 nothing less than \textit{
	\begin{itemize}
	\end{itemize}
}
\item  \\
 not for nothing \textit{
	\begin{itemize}
	\end{itemize}
}
\item  \\
 something for nothing \textit{
	\begin{itemize}
	\end{itemize}
}
\item  \\
 nothing of the sort \textit{
	\begin{itemize}
	\end{itemize}
}
\end{enumerate}

\section*{internal}
{\large \color{blue}  }
\subsection*{Explain}
\begin{enumerate}
\item adjective \\
\textbf{Internal} is used to describe things that exist or happen inside a country or organization.
 \textit{
	\begin{itemize}
	\item The country stepped up internal security.
	\item We now have a Europe without internal borders.
	\end{itemize}
}
\item adjective \\
\textbf{Internal} is used to describe things that exist or happen inside a particular person, object,
or place.
 \textit{
	\begin{itemize}
	\item ...massive internal bleeding.
	\item Some of the internal walls of my house are made of plasterboard.
	\end{itemize}
}
\end{enumerate}

\section*{pebble}
{\large \color{blue}  pebbles  }
\subsection*{Explain}
\begin{enumerate}
\item countable noun \\
A \textbf{pebble} is a small, smooth, round stone which is found on beaches and at the bottom of rivers .
 \textit{
	\begin{itemize}
	\end{itemize}
}
\end{enumerate}

\section*{kind}
{\large \color{blue}  kinds  }
\subsection*{Explain}
\begin{enumerate}
\item countable noun \\
If you talk about a particular \textbf{kind}  \textbf{of} thing, you are talking about one of the types or sorts of that thing.
 \textit{
	\begin{itemize}
	\item The party needs a different kind of leadership.
	\item Had Jamie ever been in any kind of trouble?
	\item I'm not the kind of person to get married.
	\item This book prize is the biggest of its kind in the world.
	\item Ear pain of any kind must never be ignored.
	\end{itemize}
}
\item countable noun \\
If you refer to someone's \textbf{kind} , you are referring to all the other people that are like them or that belong to the same class or set.
 \textit{
	\begin{itemize}
	\item I hate Lewis and his kind just as much as you do.
	\item I can take care of your kind.
	\end{itemize}
}
\item  \\
 all kinds of \textit{
	\begin{itemize}
	\end{itemize}
}
\item  \\
 kind of \textit{
	\begin{itemize}
	\end{itemize}
}
\item  \\
 of a kind \textit{
	\begin{itemize}
	\end{itemize}
}
\item  \\
 one of a kind \textit{
	\begin{itemize}
	\end{itemize}
}
\item  \\
 two/three/four of a kind \textit{
	\begin{itemize}
	\end{itemize}
}
\item  \\
 in kind \textit{
	\begin{itemize}
	\end{itemize}
}
\item  \\
 in kind \textit{
	\begin{itemize}
	\end{itemize}
}
\end{enumerate}

\section*{petrol}
{\large \color{blue}  }
\subsection*{Explain}
\begin{enumerate}
\item uncountable noun \\
\textbf{Petrol} is a liquid which is used as a fuel for motor vehicles.
 \textit{
	\begin{itemize}
	\end{itemize}
}
\end{enumerate}

\section*{literary}
{\large \color{blue}  }
\subsection*{Explain}
\begin{enumerate}
\item adjective \\
\textbf{Literary} means concerned with or connected with the writing, study, or appreciation of literature.
 \textit{
	\begin{itemize}
	\item Her literary criticism focuses on the way great literature suggests ideas.
	\item She's the literary editor of the 'Sunday Review'.
	\item ...a literary masterpiece.
	\end{itemize}
}
\item adjective \\
\textbf{Literary} words and expressions are often unusual in some way and are used to create a special  effect in a piece of writing such as a poem , speech , or novel .
 \textit{
	\begin{itemize}
	\end{itemize}
}
\end{enumerate}

\section*{pitch}
{\large \color{blue}  pitches  pitching  pitched  }
\subsection*{Explain}
\begin{enumerate}
\item countable noun \\
A \textbf{pitch} is an area of ground that is marked out and used for playing a game such as football , cricket, or hockey .
 \textit{
	\begin{itemize}
	\item There was a swimming-pool, cricket pitches, and playing fields.
	\item Their conduct both on and off the pitch was excellent.
	\end{itemize}
}
\item verb \\
If you \textbf{pitch} something somewhere , you throw it with quite a lot of force, usually aiming it carefully.
 \textit{
	\begin{itemize}
	\item Simon pitched the empty bottle into the lake.
	\end{itemize}
}
\item verb \\
To \textbf{pitch} somewhere means to fall forwards suddenly and with a lot of force.
 \textit{
	\begin{itemize}
	\item The movement took him by surprise, and he pitched forward.
	\item Alan staggered sideways, pitched head-first over the low wall and fell into the lake.
	\item I was pitched into the water and swam ashore.
	\end{itemize}
}
\item verb \\
If someone \textbf{is pitched into} a new situation, they are suddenly forced into it.
 \textit{
	\begin{itemize}
	\item They were being pitched into a new adventure.
	\item This could pitch the government into confrontation with the work-force.
	\end{itemize}
}
\item verb \\
In the game of baseball or rounders , when you \textbf{pitch} the ball, you throw it to the batter for them to hit it.
 \textit{
	\begin{itemize}
	\item We passed long, hot afternoons pitching a baseball.
	\end{itemize}
}
\item uncountable noun \\
The \textbf{pitch} of a sound is how high or low it is.
 \textit{
	\begin{itemize}
	\item He raised his voice to an even higher pitch.
	\end{itemize}
}
\item verb \\
If a sound \textbf{is pitched at} a particular level, it is produced at the level indicated.
 \textit{
	\begin{itemize}
	\item His cry is pitched at a level that makes it impossible to ignore.
	\item His voice was pitched high, the words muffled by his crying.
	\item Her voice was well pitched and brisk.
	\end{itemize}
}
\item verb \\
If something \textbf{is pitched}  \textbf{at} a particular level or degree of difficulty , it is set at that level.
 \textit{
	\begin{itemize}
	\item I think the material is pitched at too high a level for our purposes.
	\item The government has pitched High Street interest rates at a new level.
	\end{itemize}
}
\item singular noun \\
If something such as a feeling or a situation rises to a high \textbf{pitch} , it rises to a high level.
 \textit{
	\begin{itemize}
	\item No other emotion is able to keep the body at a high pitch for such long periods.
	\item The hysteria reached such a pitch that police were deployed to reassure parents at
the school gates.
	\end{itemize}
}
\item verb \\
If you \textbf{pitch} your \textbf{tent} , or \textbf{pitch camp} , you put up your tent in a place where you are going to stay .
 \textit{
	\begin{itemize}
	\item He had pitched his tent in the yard.
	\item At dusk we pitched camp in the middle of nowhere.
	\end{itemize}
}
\item verb \\
If a boat \textbf{pitches} , it moves violently up and down with the movement of the waves when the sea is rough .
 \textit{
	\begin{itemize}
	\item The ship is pitching and rolling in what looks like about fifteen-foot seas.
	\end{itemize}
}
\item uncountable noun \\
\textbf{Pitch} is a black substance that is sticky when it is hot and very hard when it is dry. Pitch is used on the bottoms of boats and on the roofs of houses to prevent water getting in.
 \textit{
	\begin{itemize}
	\item The timbers of similar houses were painted with pitch.
	\end{itemize}
}
\item  \\
 make a pitch/make one's pitch \textit{
	\begin{itemize}
	\end{itemize}
}
\end{enumerate}

\section*{preface}
{\large \color{blue}  prefaces  prefacing  prefaced  }
\subsection*{Explain}
\begin{enumerate}
\item countable noun \\
A \textbf{preface} is an introduction at the beginning of a book, which explains what the book is about or why it was written.
 \textit{
	\begin{itemize}
	\end{itemize}
}
\item verb \\
If you \textbf{preface} an action or speech \textbf{with} something else, you do or say this other thing first.
 \textit{
	\begin{itemize}
	\item I will preface what I am going to say with a few lines from Shakespeare.
	\item The president prefaced his remarks by saying he has supported unemployment benefits
all along.
	\end{itemize}
}
\end{enumerate}

\section*{lunar}
{\large \color{blue}  }
\subsection*{Explain}
\begin{enumerate}
\item adjective \\
\textbf{Lunar} means relating to the moon.
 \textit{
	\begin{itemize}
	\item The vast volcanic slope was eerily reminiscent of a lunar landscape.
	\item ...a magazine article celebrating the anniversary of man's first lunar landing.
	\end{itemize}
}
\end{enumerate}

\section*{role}
{\large \color{blue}  roles  }
\subsection*{Explain}
\begin{enumerate}
\item countable noun \\
If you have a \textbf{role} in a situation or in society , you have a particular position and function in it.
 \textit{
	\begin{itemize}
	\item ...the drug's role in preventing more serious effects of infection.
	\item ...bitter disagreements about the role of the monarchy.
	\item Both sides have roles to play.
	\end{itemize}
}
\item countable noun \\
A \textbf{role} is one of the characters that an actor or singer can play in a film, play, or opera .
 \textit{
	\begin{itemize}
	\item She has just landed the lead role in The Young Vic's latest production.
	\item Shakespearean women's roles were originally written to be played by men.
	\end{itemize}
}
\end{enumerate}

\section*{manual}
{\large \color{blue}  manuals  }
\subsection*{Explain}
\begin{enumerate}
\item adjective \\
\textbf{Manual} work is work in which you use your hands or your physical strength rather than your mind .
 \textit{
	\begin{itemize}
	\item ...skilled manual workers.
	\item They have no reservations about taking factory or manual jobs.
	\end{itemize}
}
\item adjective \\
\textbf{Manual} is used to talk about movements which are made by someone's hands.
 \textit{
	\begin{itemize}
	\item ...toys designed to help develop manual dexterity.
	\end{itemize}
}
\item adjective \\
\textbf{Manual} means operated by hand, rather than by electricity or a motor .
 \textit{
	\begin{itemize}
	\item There is a manual pump to get rid of the water.
	\end{itemize}
}
\item countable noun \\
A \textbf{manual} is a book which tells you how to do something or how a piece of machinery works.
 \textit{
	\begin{itemize}
	\item ...the instruction manual.
	\end{itemize}
}
\end{enumerate}

\section*{sand}
{\large \color{blue}  sands  sanding  sanded  }
\subsection*{Explain}
\begin{enumerate}
\item uncountable noun \\
\textbf{Sand} is a substance that looks  like  powder , and consists of extremely small pieces of stone . Some deserts and many beaches are made up of sand.
 \textit{
	\begin{itemize}
	\item They all walked barefoot across the damp sand to the water's edge.
	\item ...grains of sand.
	\end{itemize}
}
\item plural noun \\
\textbf{Sands} are a large area of sand, for example a beach.
 \textit{
	\begin{itemize}
	\item ...miles of golden sands.
	\end{itemize}
}
\item verb \\
If you \textbf{sand} a wood or metal surface, you rub sandpaper over it in order to make it smooth or clean .
 \textbf{Sand down}  means the same as sand .
 \textit{
	\begin{itemize}
	\item Sand the surface softly and carefully.
	\item I was going to sand down the chairs and repaint them.
	\item Simply sand them down with a fine grade of sandpaper.
	\end{itemize}
}
\item  \\
 shifting sands \textit{
	\begin{itemize}
	\end{itemize}
}
\end{enumerate}

\section*{medieval}
{\large \color{blue}  }
\subsection*{Explain}
\begin{enumerate}
\item adjective \\
Something that is \textbf{medieval} relates to or was made in the period of European history between the end of the Roman Empire in 476 AD and about 1500 AD.
 \textit{
	\begin{itemize}
	\item ...a medieval castle.
	\item ...the medieval chroniclers.
	\end{itemize}
}
\end{enumerate}

\section*{snack}
{\large \color{blue}  snacks  snacking  snacked  }
\subsection*{Explain}
\begin{enumerate}
\item countable noun \\
A \textbf{snack} is a simple meal that is quick to cook and to eat.
 \textit{
	\begin{itemize}
	\item Lunch was a snack in the fields.
	\end{itemize}
}
\item countable noun \\
A \textbf{snack} is something such as a chocolate  bar that you eat between meals.
 \textit{
	\begin{itemize}
	\item Do you eat sweets, cakes or sugary snacks?
	\end{itemize}
}
\item verb \\
If you \textbf{snack} , you eat snacks between meals.
 \textit{
	\begin{itemize}
	\item Instead of snacking on crisps and chocolate, nibble on celery or carrot.
	\item She would improve her diet if she ate less fried food and snacked less.
	\end{itemize}
}
\end{enumerate}

\section*{neutral}
{\large \color{blue}  neutrals  }
\subsection*{Explain}
\begin{enumerate}
\item adjective \\
If a person or country adopts a \textbf{neutral} position or remains  \textbf{neutral} , they do not support anyone in a disagreement , war, or contest .
 A \textbf{neutral} is someone who is neutral.
 \textit{
	\begin{itemize}
	\item Let's meet on neutral territory.
	\item Those who had tried to remain neutral now found themselves required to take sides.
	\item It was a good game to watch for the neutrals.
	\end{itemize}
}
\item adjective \\
If someone speaks in a \textbf{neutral}  voice or if the expression on their face is \textbf{neutral} , they do not show what they are thinking or feeling.
 \textit{
	\begin{itemize}
	\item Isabel put her magazine down and said in a neutral voice, 'You're very late, darling.'
	\item He told her about the death, describing the events in as neutral a manner as he could.
	\end{itemize}
}
\item adjective \\
If you say that something is \textbf{neutral} , you mean that it does not have any effect on other things because it lacks any significant qualities of its own, or it is an equal balance of two or more different qualities, amounts, or ideas .
 \textit{
	\begin{itemize}
	\item Three in every five interviewed felt that the Budget was neutral and they would be
no better off.
	\end{itemize}
}
\item graded adjective \\
If someone uses \textbf{neutral} language, they choose words which do not indicate that they approve or disapprove of something.
 \textit{
	\begin{itemize}
	\item Both sides had agreed to use neutral terms in their references to each other.
	\item He had departed from his prepared testimony, which was considered to be neutral.
	\end{itemize}
}
\item uncountable noun \\
\textbf{Neutral} is the position between the gears of a vehicle such as a car, in which the gears are not connected to the engine.
 \textit{
	\begin{itemize}
	\item Graham put the van in neutral and jumped out into the road.
	\end{itemize}
}
\item adjective \\
In an electrical device or system, the \textbf{neutral}  wire is one of the three wires needed to complete the circuit so that the current can flow. The other two wires are called the earth wire and the live or positive wire.
 \textit{
	\begin{itemize}
	\end{itemize}
}
\item colour \\
\textbf{Neutral} is used to describe things that have a pale colour such as cream or grey , or that have no colour at all.
 \textit{
	\begin{itemize}
	\item At the horizon the land mass becomes a continuous pale neutral grey.
	\item Mary suggests using a neutral lip pencil.
	\end{itemize}
}
\item adjective \\
In physics , \textbf{neutral} is used to describe things such as atomic  particles that have neither a positive nor a negative charge.
 \textit{
	\begin{itemize}
	\item A neutron is simply a neutral particle in the nucleus of an atom.
	\end{itemize}
}
\item adjective \\
In chemistry , \textbf{neutral} is used to describe things that are neither acid nor alkaline.
 \textit{
	\begin{itemize}
	\item Pure water is neutral with a pH of 7.
	\end{itemize}
}
\end{enumerate}

\section*{space}
{\large \color{blue}  spaces  spacing  spaced  }
\subsection*{Explain}
\begin{enumerate}
\item variable noun \\
You use \textbf{space} to refer to an area that is empty or available . The area can be any size. For example , you can refer to a large area outside as a large open \textbf{space} or to a small area between two objects as a small \textbf{space} .
 \textit{
	\begin{itemize}
	\item ...bits of open space such as fields and small parks.
	\item ...cutting down yet more trees to make space for houses.
	\item I had plenty of space to write and sew.
	\item The space underneath could be used as a storage area.
	\item He looked cautiously through a half-inch space between the curtains and saw an empty
bedroom.
	\item The bird was enclosed in such a small space that it could not turn without bending
its tail.
	\item List in the spaces below the specific changes you have made.
	\end{itemize}
}
\item variable noun \\
A particular kind of \textbf{space} is the area that is available for a particular activity or for putting a particular kind of thing in.
 \textit{
	\begin{itemize}
	\item ...the high cost of office space.
	\item You don't want your living space to look like a bedroom.
	\item Finding a parking space in the summer months is still a virtual impossibility.
	\item There is a communal space for people to gather.
	\end{itemize}
}
\item uncountable noun \\
If a place gives a feeling of \textbf{space} , it gives an impression of being large and open.
 \textit{
	\begin{itemize}
	\item Large paintings can enhance the feeling of space in small rooms.
	\item The sense of space and emptiness is overwhelming.
	\end{itemize}
}
\item uncountable noun \\
If you give someone \textbf{space} to think about something or to develop as a person, you allow them the time and freedom to
do this.
 \textit{
	\begin{itemize}
	\item You need space to think everything over.
	\item We will give each other space to develop.
	\end{itemize}
}
\item uncountable noun \\
The amount of \textbf{space} for a topic to be discussed in a document is the number of pages available to discuss the topic.
 \textit{
	\begin{itemize}
	\item We can't promise to publish a reply as space is limited.
	\item ...some work which we couldn't include because of lack of space in this issue.
	\end{itemize}
}
\item singular noun \\
A \textbf{space of} time is a period of time.
 \textit{
	\begin{itemize}
	\item They've come a long way in a short space of time.
	\item I have known dramatic changes occur in the space of a few minutes with this method.
	\end{itemize}
}
\item uncountable noun \\
\textbf{Space} is the area beyond the Earth's atmosphere, where the stars and planets are.
 \textit{
	\begin{itemize}
	\item The six astronauts on board will spend ten days in space.
	\item ...launching satellites into space.
	\item ...developments in space technology.
	\item ...outer space.
	\end{itemize}
}
\item uncountable noun \\
\textbf{Space} is the whole area within which everything exists.
 \textit{
	\begin{itemize}
	\item She felt herself transcending time and space.
	\item The physical universe is finite in space and time.
	\end{itemize}
}
\item verb \\
If you \textbf{space} a series of things, you arrange them so that they are not all together but have gaps
or intervals of time between them.
 \textbf{Space out} means the same as space .
 \textit{
	\begin{itemize}
	\item Women once again are having fewer children and spacing them further apart.
	\item His voice was angry and he spaced the words for emphasis.
	\item He talks quite slowly and spaces his words out.
	\item I was spacing out the seedlings into divided trays.
	\item Their last four games are spaced out over three weeks.
	\end{itemize}
}
\item  \\
 (staring) into space \textit{
	\begin{itemize}
	\end{itemize}
}
\item  \\
 waste of space \textit{
	\begin{itemize}
	\end{itemize}
}
\item  \\
 watch this space \textit{
	\begin{itemize}
	\end{itemize}
}
\end{enumerate}

\section*{owing}
{\large \color{blue}  }
\subsection*{Explain}
\begin{enumerate}
\item adjective \\
1.  2.  \textit{
	\begin{itemize}
	\end{itemize}
}
\end{enumerate}

\section*{spy}
{\large \color{blue}  spies  spying  spied  }
\subsection*{Explain}
\begin{enumerate}
\item countable noun \\
A \textbf{spy} is a person whose job is to find out secret information about another country or organization.
 \textit{
	\begin{itemize}
	\item He was jailed for five years as an alleged British spy.
	\item The spy ring passed secrets to the enemy.
	\end{itemize}
}
\item adjective \\
A \textbf{spy}  satellite or \textbf{spy}  plane obtains secret information about another country by taking  photographs from the sky .
 \textit{
	\begin{itemize}
	\end{itemize}
}
\item verb \\
Someone who \textbf{spies}  \textbf{for} a country or organization tries to find out secret information about another country or organization.
 \textit{
	\begin{itemize}
	\item The agent spied for the government for more than twenty years.
	\item East and West are still spying on one another.
	\item I never agreed to spy against the United States.
	\end{itemize}
}
\item verb \\
If you \textbf{spy on} someone, you watch them secretly.
 \textit{
	\begin{itemize}
	\item That day he spied on her while pretending to work on the shrubs.
	\item If you were invisible, who would you spy on?
	\end{itemize}
}
\item verb \\
If you \textbf{spy} someone or something, you notice them.
 \textit{
	\begin{itemize}
	\item He was walking down the street when he spied an old friend.
	\end{itemize}
}
\end{enumerate}

\section*{public}
{\large \color{blue}  }
\subsection*{Explain}
\begin{enumerate}
\item singular noun \\
You can refer to people in general, or to all the people in a particular country or community,
as \textbf{the public} .
 \textit{
	\begin{itemize}
	\item Lauderdale House is now open to the public.
	\item Tickets go on sale to the general public on July 1st.
	\item Trade unions are regarding the poll as a test of the public's confidence in the government.
	\end{itemize}
}
\item singular noun \\
You can refer to a set of people in a country who share a common interest, activity, or characteristic as a particular kind of \textbf{public} .
 \textit{
	\begin{itemize}
	\item Market research showed that 93% of the viewing public wanted a hit film channel.
	\item ...the American voting public.
	\end{itemize}
}
\item adjective \\
\textbf{Public} means relating to all the people in a country or community.
 \textit{
	\begin{itemize}
	\item The President is attempting to drum up public support for his economic program.
	\end{itemize}
}
\item adjective \\
\textbf{Public} means relating to the government or state, or things that are done for the people
by the state.
 \textit{
	\begin{itemize}
	\item The social services account for a substantial part of public spending.
	\end{itemize}
}
\item adjective \\
\textbf{Public} buildings and services are provided for everyone to use.
 \textit{
	\begin{itemize}
	\item ...the New York Public Library.
	\item The new museum must be accessible by public transport.
	\item ...a public health service available to all.
	\end{itemize}
}
\item adjective \\
A \textbf{public} place is one where people can go about freely and where you can easily be seen and heard .
 \textit{
	\begin{itemize}
	\item ...the heavily congested public areas of international airports.
	\item I avoid working in places which are too public.
	\end{itemize}
}
\item adjective \\
If someone is a \textbf{public figure} or in \textbf{public life} , many people know who they are because they are often mentioned in newspapers and on television .
 \textit{
	\begin{itemize}
	\item The archbishop hit out at public figures who commit adultery.
	\item I'd like to see more women in public life, especially Parliament.
	\end{itemize}
}
\item adjective \\
\textbf{Public} is used to describe  statements , actions, and events that are made or done in such a way that any member of the public
can see them or be aware of them.
 \textit{
	\begin{itemize}
	\item The National Heritage Committee has conducted a public inquiry to find the answer.
	\item The comments were the ministry's first detailed public statement on the subject.
	\item Marilyn made her last public appearance at Madison Square Garden.
	\end{itemize}
}
\item adjective \\
If a fact is made \textbf{public} or becomes \textbf{public} , it becomes known to everyone rather than being kept secret .
 \textit{
	\begin{itemize}
	\item His will, made public yesterday, showed that he had amassed an estate with a net
worth of £1,980,133.
	\item The facts could cause embarrassment if they ever became public.
	\end{itemize}
}
\item  \\
 the public eye \textit{
	\begin{itemize}
	\end{itemize}
}
\item  \\
 go public \textit{
	\begin{itemize}
	\end{itemize}
}
\item  \\
 in public \textit{
	\begin{itemize}
	\end{itemize}
}
\end{enumerate}

\section*{state}
{\large \color{blue}  states  stating  stated  }
\subsection*{Explain}
\begin{enumerate}
\item countable noun \\
You can refer to countries as \textbf{states} , particularly when you are discussing  politics .
 \textit{
	\begin{itemize}
	\item A successful secular state is built on liberal democratic foundations.
	\item Some weeks ago I recommended to E.U. member states that we should have discussions
with the Americans.
	\end{itemize}
}
\item countable noun \\
Some large countries such as the USA are divided into smaller areas called  \textbf{states} .
 \textit{
	\begin{itemize}
	\item Leaders of the Southern states are meeting in Louisville.
	\end{itemize}
}
\item proper noun \\
The USA is sometimes referred to as \textbf{the States} .
 \textit{
	\begin{itemize}
	\end{itemize}
}
\item singular noun \\
You can refer to the government of a country as \textbf{the state} .
 \textit{
	\begin{itemize}
	\item The state does not collect enough revenue to cover its expenditure.
	\item ...the sale of major state-owned corporations.
	\end{itemize}
}
\item adjective \\
\textbf{State}  industries or organizations are financed and organized by the government rather than private  companies .
 \textit{
	\begin{itemize}
	\item ...reform of the state social-security system.
	\end{itemize}
}
\item adjective \\
A \textbf{state} occasion is a formal one involving the head of a country.
 \textit{
	\begin{itemize}
	\item The president arrived in Britain last night for his official state visit.
	\end{itemize}
}
\item countable noun \\
When you talk about the \textbf{state}  \textbf{of} someone or something, you are referring to the condition they are in or what they
are like at a particular time.
 \textit{
	\begin{itemize}
	\item For the first few months after Daniel died, I was in a state of clinical depression.
	\item When we moved here the walls and ceiling were in an awful state.
	\item Look at the state of my car!
	\end{itemize}
}
\item verb \\
If you \textbf{state} something, you say or write it in a formal or definite  way .
 \textit{
	\begin{itemize}
	\item Clearly state your address and telephone number.
	\item The police report stated that he was arrested for allegedly assaulting an officer.
	\item 'Our relationship is totally platonic,' she stated.
	\item Buyers who do not apply within the stated period can lose their deposits.
	\end{itemize}
}
\item  \\
 not in a fit state \textit{
	\begin{itemize}
	\end{itemize}
}
\item  \\
 in a state/into a state \textit{
	\begin{itemize}
	\end{itemize}
}
\item  \\
 to lie in state \textit{
	\begin{itemize}
	\end{itemize}
}
\end{enumerate}

\section*{reluctant}
{\large \color{blue}  }
\subsection*{Explain}
\begin{enumerate}
\item adjective \\
If you are \textbf{reluctant}  \textbf{to} do something, you are unwilling to do it and hesitate before doing it, or do it slowly and without enthusiasm .
 \textit{
	\begin{itemize}
	\item Mr Spero was reluctant to ask for help.
	\item The police are very reluctant to get involved in this sort of thing.
	\end{itemize}
}
\end{enumerate}

\section*{steamer}
{\large \color{blue}  steamers  }
\subsection*{Explain}
\begin{enumerate}
\item countable noun \\
A \textbf{steamer} is a ship that has an engine powered by steam.
 \textit{
	\begin{itemize}
	\end{itemize}
}
\item countable noun \\
A \textbf{steamer} is a special  container used for steaming food such as vegetables and fish.
 \textit{
	\begin{itemize}
	\end{itemize}
}
\end{enumerate}

\section*{solar}
{\large \color{blue}  }
\subsection*{Explain}
\begin{enumerate}
\item adjective \\
\textbf{Solar} is used to describe things relating to the sun.
 \textit{
	\begin{itemize}
	\item A total solar eclipse is due to take place some time tomorrow.
	\end{itemize}
}
\item adjective \\
\textbf{Solar}  power is obtained from the sun's light and heat .
 \textit{
	\begin{itemize}
	\end{itemize}
}
\end{enumerate}

\section*{suspicion}
{\large \color{blue}  suspicions  }
\subsection*{Explain}
\begin{enumerate}
\item variable noun \\
\textbf{Suspicion} or a \textbf{suspicion} is a belief or feeling that someone has committed a crime or done something wrong.
 \textit{
	\begin{itemize}
	\item There was a suspicion that this runner attempted to avoid the procedures for dope
testing.
	\item The police said their suspicions were aroused because Mr Owens had other marks on
his body.
	\item Scotland Yard had assured him he was not under suspicion.
	\item ... police arrested nineteen people on suspicion of burglary.
	\end{itemize}
}
\item variable noun \\
If there is \textbf{suspicion}  \textbf{of} someone or something, people do not trust them or consider them to be reliable .
 \textit{
	\begin{itemize}
	\item ...the common British suspicion of psychotherapy.
	\item He may have had some suspicions of Michael Foster, the editor of the journal.
	\item I was always regarded in the Army with a certain amount of suspicion because of my
left-wing tendencies.
	\end{itemize}
}
\item countable noun \\
A \textbf{suspicion} is a feeling that something is probably  true or is likely to happen .
 \textit{
	\begin{itemize}
	\item I have a sneaking suspicion that they are going to succeed.
	\item Astronomers will have to collect more spectra from these stars to confirm their suspicions.
	\end{itemize}
}
\item singular noun \\
A \textbf{suspicion}  \textbf{of} something is a very small amount of it.
 \textit{
	\begin{itemize}
	\item ...large blooms of white with a suspicion of pale pink.
	\end{itemize}
}
\end{enumerate}

\section*{tedious}
{\large \color{blue}  }
\subsection*{Explain}
\begin{enumerate}
\item adjective \\
If you describe something such as a job , task , or situation as \textbf{tedious} , you mean it is boring and rather frustrating .
 \textit{
	\begin{itemize}
	\item Such lists are long and tedious to read.
	\item ...the tedious business of line-by-line programming.
	\end{itemize}
}
\end{enumerate}

\section*{system}
{\large \color{blue}  systems  }
\subsection*{Explain}
\begin{enumerate}
\item countable noun \\
A \textbf{system} is a way of working, organizing , or doing something which follows a fixed plan or set of rules. You can use \textbf{system} to refer to an organization or institution that is organized in this way.
 \textit{
	\begin{itemize}
	\item The present system of funding for higher education is unsatisfactory.
	\item ...a flexible and relatively efficient filing system.
	\item ...a multi-party system of government.
	\item The Court of Appeal has a pivotal role in the English legal system.
	\end{itemize}
}
\item countable noun \\
A \textbf{system} is a set of devices powered by electricity , for example a computer or an alarm .
 \textit{
	\begin{itemize}
	\item Viruses tend to be good at surviving when a computer system crashes.
	\end{itemize}
}
\item countable noun \\
A \textbf{system} is a set of equipment or parts such as water pipes or electrical wiring , which is used to supply water, heat, or electricity.
 \textit{
	\begin{itemize}
	\item ...a central heating system.
	\end{itemize}
}
\item countable noun \\
A \textbf{system} is a network of things that are linked together so that people or things can travel
from one place to another or communicate .
 \textit{
	\begin{itemize}
	\item ...Australia's road and rail system.
	\item ...a news channel on a local cable system.
	\end{itemize}
}
\item countable noun \\
Your \textbf{system} is your body's organs and other parts that together perform particular functions.
 \textit{
	\begin{itemize}
	\item He had slept for over fourteen hours, and his system seemed to have recuperated admirably.
	\item These gases would seriously damage the patient's respiratory system.
	\item ...the reproductive system.
	\end{itemize}
}
\item countable noun \\
A \textbf{system} is a particular set of rules, especially in mathematics or science, which is used to count or measure things.
 \textit{
	\begin{itemize}
	\item ...the decimal system of metric weights and measures.
	\item ...Trachtenberg's system of simplified mathematics.
	\end{itemize}
}
\item singular noun \\
People sometimes refer to the government or administration of a country as \textbf{the system} .
 \textit{
	\begin{itemize}
	\item These feelings are likely to make people attempt to overthrow the system.
	\item He wants to be the tough rebel who bucks the system.
	\end{itemize}
}
\item  \\
 get something out of one's system \textit{
	\begin{itemize}
	\end{itemize}
}
\end{enumerate}

\section*{tremendous}
{\large \color{blue}  }
\subsection*{Explain}
\begin{enumerate}
\item adjective \\
You use \textbf{tremendous} to emphasize how strong a feeling or quality is, or how large an amount is.
 \textit{
	\begin{itemize}
	\item I felt a tremendous pressure on my chest.
	\item That's a tremendous amount of information.
	\end{itemize}
}
\item adjective \\
You can describe someone or something as \textbf{tremendous} when you think they are very good or very impressive .
 \textit{
	\begin{itemize}
	\item He was a tremendous person.
	\item I thought it was absolutely tremendous.
	\end{itemize}
}
\end{enumerate}

\section*{treaty}
{\large \color{blue}  treaties  }
\subsection*{Explain}
\begin{enumerate}
\item countable noun \\
A \textbf{treaty} is a written agreement between countries in which they agree to do a particular thing
or to help each other.
 \textit{
	\begin{itemize}
	\item A formal peace treaty was signed later that year.
	\item ...a global treaty on cutting emissions.
	\end{itemize}
}
\end{enumerate}

\section*{vast}
{\large \color{blue}  vaster  vastest  }
\subsection*{Explain}
\begin{enumerate}
\item adjective \\
Something that is \textbf{vast} is extremely large.
 \textit{
	\begin{itemize}
	\item ...Afrikaner farmers who own vast stretches of land.
	\item The vast majority of the eggs would be cracked.
	\end{itemize}
}
\end{enumerate}

\section*{turbine}
{\large \color{blue}  turbines  }
\subsection*{Explain}
\begin{enumerate}
\item countable noun \\
A \textbf{turbine} is a machine or engine which uses a stream of air, gas , water, or steam to turn a wheel and produce power .
 \textit{
	\begin{itemize}
	\end{itemize}
}
\end{enumerate}

\section*{widespread}
{\large \color{blue}  }
\subsection*{Explain}
\begin{enumerate}
\item adjective \\
Something that is \textbf{widespread} exists or happens over a large area, or to a great extent .
 \textit{
	\begin{itemize}
	\item There is widespread support for the new proposals.
	\item Food shortages are widespread.
	\end{itemize}
}
\end{enumerate}

\section*{vicinity}
{\large \color{blue}  }
\subsection*{Explain}
\begin{enumerate}
\item singular noun \\
If something is \textbf{in}  \textbf{the vicinity}  \textbf{of} a particular place, it is near it.
 \textit{
	\begin{itemize}
	\item There were a hundred or so hotels in the vicinity of the railway station.
	\item The immediate vicinity of the house remains cordoned off.
	\end{itemize}
}
\end{enumerate}

\section*{accurate}
{\large \color{blue}  }
\subsection*{Explain}
\begin{enumerate}
\item adjective \\
\textbf{Accurate}  information , measurements , and statistics are correct to a very detailed  level . An \textbf{accurate}  instrument is able to give you information of this kind .
 \textit{
	\begin{itemize}
	\item Police have stressed that this is the most accurate description of the killer to
date.
	\item ... a quick and accurate way of monitoring the amount of carbon dioxide in the air.
	\item Quartz timepieces are very accurate, to a minute or two per year.
	\end{itemize}
}
\item adjective \\
An \textbf{accurate}  statement or account gives a true or fair  judgment of something.
 \textit{
	\begin{itemize}
	\item It is too early to give an accurate assessment of his condition.
	\item They were accurate in their prediction that he would change her life drastically.
	\end{itemize}
}
\item adjective \\
You can use \textbf{accurate} to describe the results of someone's actions when they do or copy something correctly or exactly .
 \textit{
	\begin{itemize}
	\item Marks were given for accurate spelling and punctuation.
	\item ...his maliciously accurate imitation of the Prime Minister.
	\end{itemize}
}
\item adjective \\
An \textbf{accurate}  weapon or throw  reaches the exact point or target that it was intended to reach. You can also describe a person as \textbf{accurate} if they fire a weapon or throw something in this way .
 \textit{
	\begin{itemize}
	\item The rifle was extremely accurate.
	\item The pilots, however, were not as accurate as they should be.
	\end{itemize}
}
\end{enumerate}

\section*{accessory}
{\large \color{blue}  accessories  }
\subsection*{Explain}
\begin{enumerate}
\item countable noun \\
\textbf{Accessories} are items of equipment that are not usually essential , but which can be used with or added to something else in order to make it more efficient , useful , or decorative .
 \textit{
	\begin{itemize}
	\item ...an exclusive range of hand-made bedroom and bathroom accessories.
	\end{itemize}
}
\item countable noun \\
\textbf{Accessories} are articles such as belts and scarves which you wear or carry but which are not part of your main  clothing .
 \textit{
	\begin{itemize}
	\end{itemize}
}
\item countable noun \\
If someone is guilty of being an \textbf{accessory}  \textbf{to} a crime, they helped the person who committed it, or knew it was being committed but did not tell the police .
 \textit{
	\begin{itemize}
	\item She was charged with being an accessory to the embezzlement of funds.
	\end{itemize}
}
\item adjective \\
You can use \textbf{accessory} to describe something which is part of an activity or process, but not the most essential or important part of it.
 \textit{
	\begin{itemize}
	\item Forster established that minerals are accessory food factors required for maintaining
life.
	\end{itemize}
}
\end{enumerate}

\section*{advisable}
{\large \color{blue}  }
\subsection*{Explain}
\begin{enumerate}
\item adjective \\
If you tell someone that \textbf{it} is \textbf{advisable}  \textbf{to} do something, you are suggesting that they should do it, because it is sensible or is likely to achieve the result they want .
 \textit{
	\begin{itemize}
	\item Because of the popularity of the region, it is advisable to book hotels in advance.
	\item It's not advisable to swim immediately after eating.
	\end{itemize}
}
\end{enumerate}

\section*{allowance}
{\large \color{blue}  allowances  }
\subsection*{Explain}
\begin{enumerate}
\item countable noun \\
An \textbf{allowance} is money that is given to someone, usually on a regular basis , in order to help them pay for the things that they need .
 \textit{
	\begin{itemize}
	\item He lives on a single parent's allowance of £70 a week.
	\item She gets an allowance for looking after Lillian.
	\end{itemize}
}
\item countable noun \\
A child's \textbf{allowance} is money that is given to him or her every week or every month by his or her parents .
 \textit{
	\begin{itemize}
	\end{itemize}
}
\item countable noun \\
Your tax \textbf{allowance} is the amount of money that you are allowed to earn before you have to start paying income tax.
 \textit{
	\begin{itemize}
	\item ...those earning less than the basic tax allowance.
	\end{itemize}
}
\item countable noun \\
A particular type of \textbf{allowance} is an amount of something that you are allowed in particular circumstances .
 \textit{
	\begin{itemize}
	\item Most of our flights have a baggage allowance of 44lbs per passenger.
	\end{itemize}
}
\item  \\
 make allowances for sth \textit{
	\begin{itemize}
	\end{itemize}
}
\item  \\
 make allowances for sb \textit{
	\begin{itemize}
	\end{itemize}
}
\end{enumerate}

\section*{ambiguous}
{\large \color{blue}  }
\subsection*{Explain}
\begin{enumerate}
\item adjective \\
If you describe something as \textbf{ambiguous} , you mean that it is unclear or confusing because it can be understood in more than one way.
 \textit{
	\begin{itemize}
	\item This agreement is very ambiguous and open to various interpretations.
	\item They may not be fully aware of what they are voting for because of ambiguous language
on the ballot paper.
	\end{itemize}
}
\item adjective \\
If you describe something as \textbf{ambiguous} , you mean that it contains several different ideas or attitudes that do not fit  well together.
 \textit{
	\begin{itemize}
	\item Students have ambiguous feelings about their role in the world.
	\end{itemize}
}
\end{enumerate}

\section*{appendix}
{\large \color{blue}  appendixes  }
\subsection*{Explain}
\begin{enumerate}
\item countable noun \\
Your \textbf{appendix} is a small closed  tube  inside your body which is attached to your digestive system.
 \textit{
	\begin{itemize}
	\item ...a burst appendix.
	\end{itemize}
}
\item countable noun \\
An \textbf{appendix}  \textbf{to} a book is extra  information that is placed after the end of the main  text .
 \textit{
	\begin{itemize}
	\end{itemize}
}
\end{enumerate}

\section*{ancient}
{\large \color{blue}  ancients  }
\subsection*{Explain}
\begin{enumerate}
\item adjective \\
\textbf{Ancient} means belonging to the distant past, especially to the period in history before the end of the Roman Empire.
 \textit{
	\begin{itemize}
	\item They believed ancient Greece and Rome were vital sources of learning.
	\end{itemize}
}
\item adjective \\
\textbf{Ancient} means very old, or having existed for a long time.
 \textit{
	\begin{itemize}
	\item ...ancient Jewish tradition.
	\item ...ancient fishing rights.
	\item ...a few acres of ancient woodland.
	\end{itemize}
}
\item plural noun \\
\textbf{The ancients} are the people of an old civilization , especially classical Greece or Rome .
 \textit{
	\begin{itemize}
	\item The ancients knew more than we do about the heavens.
	\end{itemize}
}
\end{enumerate}

\section*{certificate}
{\large \color{blue}  certificates  }
\subsection*{Explain}
\begin{enumerate}
\item countable noun \\
A \textbf{certificate} is an official document stating that particular facts are true .
 \textit{
	\begin{itemize}
	\item ...birth certificates.
	\item ...share certificates.
	\end{itemize}
}
\item countable noun \\
A \textbf{certificate} is an official document that you receive when you have completed a course of study or training . The qualification that you receive is sometimes  also  called a \textbf{certificate} .
 \textit{
	\begin{itemize}
	\item To the right of the fireplace are various framed certificates.
	\item ...the Post-Graduate Certificate of Education.
	\end{itemize}
}
\end{enumerate}

\section*{antique}
{\large \color{blue}  antiques  }
\subsection*{Explain}
\begin{enumerate}
\item countable noun \\
An \textbf{antique} is an old object such as a piece of china or furniture which is valuable because of its beauty or rarity .
 \textit{
	\begin{itemize}
	\item ...a genuine antique.
	\item ...antique silver jewellery.
	\item He finds material at auctions, antique shops and flea markets.
	\end{itemize}
}
\end{enumerate}

\section*{change}
{\large \color{blue}  changes  changing  changed  }
\subsection*{Explain}
\begin{enumerate}
\item variable noun \\
If there is a \textbf{change}  \textbf{in} something, it becomes different.
 \textit{
	\begin{itemize}
	\item The ambassador appealed for a change in U.S. policy.
	\item What is needed is a change of attitude on the part of architects.
	\item There are going to have to be some drastic changes.
	\item ...a passionate, eloquent campaigner for political change in her home country.
	\item This is a time of change for the corporation.
	\end{itemize}
}
\item singular noun \\
If you say that something is a \textbf{change} or \textbf{makes} a \textbf{change} , you mean that it is enjoyable because it is different from what you are used to.
 \textit{
	\begin{itemize}
	\item It is a complex system, but it certainly makes a change.
	\item Do you feel like you could do with a change?
	\end{itemize}
}
\item verb \\
If you \textbf{change}  \textbf{from} one thing \textbf{to} another, you stop using or doing the first one and start using or doing the second.
 \textit{
	\begin{itemize}
	\item His doctor increased the dosage but did not change to a different medication.
	\item He changed from voting against to abstaining.
	\end{itemize}
}
\item verb \\
When something \textbf{changes} or when you \textbf{change} it, it becomes different.
 \textit{
	\begin{itemize}
	\item We are trying to detect and understand how the climates change.
	\item In the union office, the mood gradually changed from resignation to rage.
	\item She has now changed into a happy, self-confident woman.
	\item They should change the law to make it illegal to own replica weapons.
	\item Trees are changing colour earlier than last year.
	\item He is a changed man since you left.
	\item A changing world has put pressures on the corporation.
	\end{itemize}
}
\item verb \\
To \textbf{change} something means to replace it with something new or different.
 \textbf{Change} is also a noun .
 \textit{
	\begin{itemize}
	\item I paid £80 to have my car radio fixed and I bet all they did was change a fuse.
	\item If you want to change your doctor there are two ways of doing it.
	\item A change of leadership alone will not be enough.
	\end{itemize}
}
\item verb \\
When you \textbf{change} your clothes or \textbf{change} , you take some or all of your clothes off and put on different ones.
 \textit{
	\begin{itemize}
	\item Ben had merely changed his shirt.
	\item They had allowed her to shower and change.
	\item I changed into a tracksuit.
	\item I've got to get changed first. I've got to put my uniform on.
	\end{itemize}
}
\item countable noun \\
A \textbf{change of} clothes is an extra set of clothes that you take with you when you go to stay  somewhere or to take part in an activity.
 \textit{
	\begin{itemize}
	\item He stuffed a bag with a few changes of clothing.
	\end{itemize}
}
\item verb \\
When you \textbf{change} a bed or \textbf{change} the sheets , you take off the dirty sheets and put on clean ones.
 \textit{
	\begin{itemize}
	\item After changing the bed, I would fall asleep quickly.
	\item I changed the sheets on your bed today.
	\end{itemize}
}
\item verb \\
When you \textbf{change} a baby or \textbf{change} its nappy or diaper , you take off the dirty one and put on a clean one.
 \textit{
	\begin{itemize}
	\item She criticizes me for the way I feed or change him.
	\item He needs his nappy changed.
	\end{itemize}
}
\item verb \\
When you \textbf{change} buses, trains, or planes or \textbf{change} , you get off one bus, train, or plane and get on to another in order to continue your journey .
 \textit{
	\begin{itemize}
	\item At Glasgow I changed trains for Greenock.
	\item We were turned off the train at Hanover, where we had to change.
	\end{itemize}
}
\item verb \\
When you \textbf{change} gear or \textbf{change} into another gear, you move the gear lever on a car, bicycle , or other vehicle in order to use a different gear.
 \textit{
	\begin{itemize}
	\item The driver tried to change gear, then swerved.
	\item He looked up into the mirror as he changed through his gears.
	\end{itemize}
}
\item uncountable noun \\
Your \textbf{change} is the money that you receive when you pay for something with more money than it
 costs because you do not have exactly the right amount of money.
 \textit{
	\begin{itemize}
	\item 'There's your change.'—'Thanks very much.'.
	\item They told the shopkeeper to keep the change.
	\end{itemize}
}
\item uncountable noun \\
\textbf{Change} is coins, rather than paper money.
 \textit{
	\begin{itemize}
	\item Thieves ransacked the office, taking a sack of loose change.
	\item The man in the store won't give him change for the phone unless he buys something.
	\end{itemize}
}
\item uncountable noun \\
If you have \textbf{change}  \textbf{for} larger notes, bills , or coins, you have the same value in smaller notes, bills, or coins, which you can
give to someone in exchange.
 \textit{
	\begin{itemize}
	\item The courier had change for a £10 note.
	\end{itemize}
}
\item verb \\
When you \textbf{change} money, you exchange it for the same amount of money in a different currency, or in
smaller notes, bills, or coins.
 \textit{
	\begin{itemize}
	\item You can expect to pay the bank a fee of around 1% to 2% every time you change money.
	\item Find an agency that will change one foreign currency directly into another.
	\end{itemize}
}
\item  \\
 for a change \textit{
	\begin{itemize}
	\end{itemize}
}
\end{enumerate}

\section*{applicable}
{\large \color{blue}  }
\subsection*{Explain}
\begin{enumerate}
\item adjective \\
Something that is \textbf{applicable}  \textbf{to} a particular situation is relevant to it or can be applied to it.
 \textit{
	\begin{itemize}
	\item What is a reasonable standard for one family is not applicable for another.
	\item Appraisal was seen as most applicable to those in management jobs.
	\item It should include a review of energy usage and, where applicable, the production
and disposal of waste.
	\end{itemize}
}
\end{enumerate}

\section*{chicken}
{\large \color{blue}  chickens  chickening  chickened  }
\subsection*{Explain}
\begin{enumerate}
\item countable noun \\
\textbf{Chickens} are birds which are kept on a farm for their eggs and for their meat .
 \textbf{Chicken} is the flesh of this bird eaten as food.
 \textit{
	\begin{itemize}
	\item Lionel built a coop so that they could raise chickens and have a supply of fresh
eggs.
	\item ...free-range chickens.
	\item ...roast chicken with wild mushrooms.
	\item ...chicken soup.
	\end{itemize}
}
\item countable noun \\
If someone calls you a \textbf{chicken} , they mean that you are afraid to do something.
 \textbf{Chicken} is also an adjective .
 \textit{
	\begin{itemize}
	\item I'm scared of the dark. I'm a big chicken.
	\item Why are you so chicken, Gregory?
	\end{itemize}
}
\item  \\
 count one's chickens \textit{
	\begin{itemize}
	\end{itemize}
}
\item  \\
 a chicken and egg situation \textit{
	\begin{itemize}
	\end{itemize}
}
\item  \\
 run around like a headless chicken/rush around like a headless chicken \textit{
	\begin{itemize}
	\end{itemize}
}
\end{enumerate}

\section*{controversial}
{\large \color{blue}  }
\subsection*{Explain}
\begin{enumerate}
\item adjective \\
If you describe something or someone as \textbf{controversial} , you mean that they are the subject of intense public argument , disagreement , or disapproval .
 \textit{
	\begin{itemize}
	\item Immigration is a controversial issue in many countries.
	\item When it was first suggested that passive smoking was harmful, the idea was controversial
and the evidence thin.
	\item The changes are bound to be controversial.
	\item ...the controversial 19th century politician Charles Parnell.
	\end{itemize}
}
\end{enumerate}

\section*{clinic}
{\large \color{blue}  clinics  }
\subsection*{Explain}
\begin{enumerate}
\item countable noun \\
A \textbf{clinic} is a building where people go to receive medical advice or treatment.
 \textit{
	\begin{itemize}
	\item ...a family planning clinic.
	\end{itemize}
}
\end{enumerate}

\section*{economical}
{\large \color{blue}  }
\subsection*{Explain}
\begin{enumerate}
\item adjective \\
Something that is \textbf{economical} does not require a lot of money to operate . For example a car that only uses a small amount of petrol is \textbf{economical} .
 \textit{
	\begin{itemize}
	\item ...plans to trade in their car for something smaller and more economical.
	\item It is more economical to wash a full load.
	\end{itemize}
}
\item adjective \\
Someone who is \textbf{economical}  spends money sensibly and does not want to waste it on things that are unnecessary . A way of life that is \textbf{economical} does not need a lot of money.
 \textit{
	\begin{itemize}
	\item ...ideas for economical housekeeping.
	\end{itemize}
}
\item adjective \\
\textbf{Economical}  means using the minimum amount of time, effort, or language that is necessary .
 \textit{
	\begin{itemize}
	\item His gestures were economical, his words generally mild.
	\end{itemize}
}
\end{enumerate}

\section*{coupon}
{\large \color{blue}  coupons  }
\subsection*{Explain}
\begin{enumerate}
\item countable noun \\
A \textbf{coupon} is a piece of printed  paper which allows you to pay less money than usual for a product , or to get it free.
 \textit{
	\begin{itemize}
	\item Bring the coupon below to any Tecno store and pay just £10.99.
	\item ...a 50p money-off coupon.
	\end{itemize}
}
\item countable noun \\
A \textbf{coupon} is a small form, for example in a newspaper or magazine , which you send off to ask for information , to order something, or to enter a competition.
 \textit{
	\begin{itemize}
	\item Send the coupon with a cheque for £18.50, made payable to 'Good Housekeeping'.
	\item He was filling in his pools coupon.
	\end{itemize}
}
\end{enumerate}

\section*{edible}
{\large \color{blue}  }
\subsection*{Explain}
\begin{enumerate}
\item adjective \\
If something is \textbf{edible} , it is safe to eat and not poisonous .
 \textit{
	\begin{itemize}
	\item ...edible fungi.
	\end{itemize}
}
\end{enumerate}

\section*{criticism}
{\large \color{blue}  criticisms  }
\subsection*{Explain}
\begin{enumerate}
\item variable noun \\
\textbf{Criticism} is the action of expressing disapproval of something or someone. A \textbf{criticism} is a statement that expresses disapproval.
 \textit{
	\begin{itemize}
	\item This policy had repeatedly come under strong criticism on Capitol Hill.
	\item ...unfair criticism of his tactics.
	\item The criticism that the English do not truly care about their children was often voiced.
	\end{itemize}
}
\item uncountable noun \\
\textbf{Criticism} is a serious  examination and judgment of something such as a book or play.
 \textit{
	\begin{itemize}
	\item She has published more than 20 books including novels, poetry and literary criticism.
	\end{itemize}
}
\end{enumerate}

\section*{essential}
{\large \color{blue}  essentials  }
\subsection*{Explain}
\begin{enumerate}
\item adjective \\
Something that is \textbf{essential} is extremely important or absolutely necessary to a particular subject, situation , or activity.
 \textit{
	\begin{itemize}
	\item It was absolutely essential to separate crops from the areas that animals used as
pasture.
	\item As they must also sprint over short distances, speed is essential.
	\item Jordan promised to trim the city budget without cutting essential services.
	\end{itemize}
}
\item countable noun \\
The \textbf{essentials} are the things that are absolutely necessary for the situation you are in or for
the task you are doing.
 \textit{
	\begin{itemize}
	\item The flat contained the basic essentials for bachelor life.
	\end{itemize}
}
\item adjective \\
The \textbf{essential}  aspects of something are its most basic or important aspects.
 \textit{
	\begin{itemize}
	\item Most authorities agree that play is an essential part of a child's development.
	\item In this trial two essential elements must be proven: motive and opportunity.
	\end{itemize}
}
\item plural noun \\
The \textbf{essentials} are the most important principles , ideas , or facts of a particular subject.
 \textit{
	\begin{itemize}
	\item ...the essentials of everyday life, such as eating and exercise.
	\item This has stripped the contest down to its essentials.
	\item ...the bare essentials.
	\end{itemize}
}
\end{enumerate}

\section*{curse}
{\large \color{blue}  curses  cursing  cursed  }
\subsection*{Explain}
\begin{enumerate}
\item verb \\
If you \textbf{curse} , you use rude or offensive language, usually because you are angry about something.
 \textbf{Curse} is also a noun .
 \textit{
	\begin{itemize}
	\item I cursed and hobbled to my feet.
	\item He shot her an angry look and a curse.
	\end{itemize}
}
\item verb \\
If you \textbf{curse} someone, you say  insulting things to them because you are angry with them.
 \textit{
	\begin{itemize}
	\item Grandma protested, but he cursed her and rudely pushed her aside.
	\item He cursed himself for having been so careless.
	\end{itemize}
}
\item verb \\
If you \textbf{curse} something, you complain angrily about it, especially using rude language.
 \textit{
	\begin{itemize}
	\item So we set off again, cursing the delay, towards the west.
	\item She silently cursed her own stupidity.
	\end{itemize}
}
\item countable noun \\
If you say that there is a \textbf{curse}  \textbf{on} someone, you mean that there seems to be a supernatural power causing unpleasant things to happen to them.
 \textit{
	\begin{itemize}
	\item Maybe there is a curse on my family.
	\item He's been the object of a voodoo curse.
	\end{itemize}
}
\item countable noun \\
You can refer to something that causes a great deal of trouble or harm as a \textbf{curse} .
 \textit{
	\begin{itemize}
	\item Apathy is the long-standing curse of British local democracy.
	\item Summer colds are a terrible curse.
	\end{itemize}
}
\end{enumerate}

\section*{extreme}
{\large \color{blue}  extremes  }
\subsection*{Explain}
\begin{enumerate}
\item adjective \\
\textbf{Extreme} means very great in degree or intensity.
 \textit{
	\begin{itemize}
	\item The girls were afraid of snakes and picked their way along with extreme caution.
	\item ...people living in extreme poverty.
	\item ...the author's extreme reluctance to generalise.
	\end{itemize}
}
\item adjective \\
You use \textbf{extreme} to describe  situations and behaviour which are much more severe or unusual than you would expect , especially when you disapprove of them because of this.
 \textit{
	\begin{itemize}
	\item The extreme case was Poland, where 29 parties won seats.
	\item His punishment seemed a little extreme.
	\item The scheme has been condemned as extreme.
	\end{itemize}
}
\item adjective \\
You use \textbf{extreme} to describe opinions , beliefs , or political movements which you disapprove of because they are very different from
those that most people would accept as reasonable or normal .
 \textit{
	\begin{itemize}
	\item This extreme view hasn't captured popular opinion.
	\item ...the racist politics of the extreme right.
	\end{itemize}
}
\item countable noun \\
You can use \textbf{extremes} to refer to situations or types of behaviour that have opposite qualities to each other, especially when each situation or type of behaviour has
such a quality to the greatest degree possible .
 \textit{
	\begin{itemize}
	\item ...a 'middle way' between the extremes of success and failure.
	\item They can withstand extremes of temperature and weather without fading or cracking.
	\end{itemize}
}
\item adjective \\
The \textbf{extreme} end or edge of something is its furthest end or edge.
 \textit{
	\begin{itemize}
	\item ...the room at the extreme end of the corridor.
	\item ...winds from the extreme north.
	\end{itemize}
}
\item  \\
 go/take/carry (sthg) to extremes \textit{
	\begin{itemize}
	\end{itemize}
}
\item  \\
 in the extreme \textit{
	\begin{itemize}
	\end{itemize}
}
\end{enumerate}

\section*{dictionary}
{\large \color{blue}  dictionaries  }
\subsection*{Explain}
\begin{enumerate}
\item countable noun \\
A \textbf{dictionary} is a book in which the words and phrases of a language are listed alphabetically, together with their meanings or their translations in another language.
 \textit{
	\begin{itemize}
	\item ...a Welsh-English dictionary.
	\end{itemize}
}
\item countable noun \\
A \textbf{dictionary} is an alphabetically ordered reference book on one particular subject or limited group of subjects.
 \textit{
	\begin{itemize}
	\item ...the Dictionary of National Biography.
	\end{itemize}
}
\end{enumerate}

\section*{fairy}
{\large \color{blue}  fairies  }
\subsection*{Explain}
\begin{enumerate}
\item countable noun \\
A \textbf{fairy} is an imaginary creature with magical powers. Fairies are often represented as small people with wings .
 \textit{
	\begin{itemize}
	\end{itemize}
}
\item countable noun \\
If someone describes a man as a \textbf{fairy} , they mean that he is a homosexual and they disapprove of this.
 \textit{
	\begin{itemize}
	\end{itemize}
}
\end{enumerate}

\section*{editorial}
{\large \color{blue}  editorials  }
\subsection*{Explain}
\begin{enumerate}
\item adjective \\
\textbf{Editorial} means involved in preparing a newspaper, magazine , or book for publication.
 \textit{
	\begin{itemize}
	\item He is on the editorial staff of the magazine.
	\item I went to the editorial board meetings when I had the time.
	\end{itemize}
}
\item adjective \\
\textbf{Editorial} means involving the attitudes , opinions, and contents of something such as a newspaper, magazine, or television  programme .
 \textit{
	\begin{itemize}
	\item We are not about to change our editorial policy.
	\end{itemize}
}
\item countable noun \\
An \textbf{editorial} is an article in a newspaper which gives the opinion of the editor or owner on a topic or item of news .
 \textit{
	\begin{itemize}
	\item An editorial in the London Evening Standard argued the police reaction was disproportionate
to the threat.
	\end{itemize}
}
\item countable noun \\
An \textbf{editorial} on television or radio is an item which gives the opinion of the network or radio station .
 \textit{
	\begin{itemize}
	\end{itemize}
}
\end{enumerate}

\section*{fearful}
{\large \color{blue}  }
\subsection*{Explain}
\begin{enumerate}
\item adjective \\
If you are \textbf{fearful}  \textbf{of} something, you are afraid of it.
 \textit{
	\begin{itemize}
	\item Bankers were fearful of a world banking crisis.
	\item I had often been very fearful, very angry, and very isolated.
	\end{itemize}
}
\item adjective \\
You use \textbf{fearful} to emphasize how serious or bad a situation is.
 \textit{
	\begin{itemize}
	\item The region is in a fearful recession.
	\item ...the fearful consequences which might flow from unilateral military moves.
	\end{itemize}
}
\item adjective \\
\textbf{Fearful} is used to emphasize that something is very bad.
 \textit{
	\begin{itemize}
	\item You gave me a fearful shock!
	\item 'It sounds the most fearful hard work,' Sybil said later.
	\end{itemize}
}
\end{enumerate}

\section*{effort}
{\large \color{blue}  efforts  }
\subsection*{Explain}
\begin{enumerate}
\item variable noun \\
If you make an \textbf{effort}  \textbf{to} do something, you try very hard to do it.
 \textit{
	\begin{itemize}
	\item He made no effort to hide his disappointment.
	\item Finding a cure requires considerable time and effort.
	\item ...his efforts to reform Italian research.
	\item Despite the efforts of the United Nations, the problem of drug traffic continues
to grow.
	\item But a concerted effort has begun to improve the quality of the urban air.
	\end{itemize}
}
\item uncountable noun \\
If you say that someone did something \textbf{with}  \textbf{effort} or \textbf{with}  \textbf{an effort} , you mean it was difficult for them to do.
 \textit{
	\begin{itemize}
	\item She took a deep breath and sat up slowly and with great effort.
	\item With an effort she contained her irritation.
	\end{itemize}
}
\item countable noun \\
An \textbf{effort} is a particular series of activities that is organized by a group of people in order to achieve something.
 \textit{
	\begin{itemize}
	\item ...a famine relief effort in Angola.
	\end{itemize}
}
\item singular noun \\
If you say that something is \textbf{an effort} , you mean that an unusual amount of physical or mental energy is needed to do it.
 \textit{
	\begin{itemize}
	\item He's very stooped and it's an effort to lift his head.
	\end{itemize}
}
\item  \\
 make the effort \textit{
	\begin{itemize}
	\end{itemize}
}
\item  \\
 an effort of will \textit{
	\begin{itemize}
	\end{itemize}
}
\item  \\
 worth the effort \textit{
	\begin{itemize}
	\end{itemize}
}
\end{enumerate}

\section*{feasible}
{\large \color{blue}  }
\subsection*{Explain}
\begin{enumerate}
\item adjective \\
If something is \textbf{feasible} , it can be done, made, or achieved .
 \textit{
	\begin{itemize}
	\item She questioned whether it was feasible to stimulate investment in these regions.
	\item That may be fine for the U.S., but it's not feasible for a mass European market.
	\end{itemize}
}
\end{enumerate}

\section*{endurance}
{\large \color{blue}  }
\subsection*{Explain}
\begin{enumerate}
\item uncountable noun \\
\textbf{Endurance} is the ability to continue with an unpleasant or difficult  situation , experience , or activity over a long period of time.
 \textit{
	\begin{itemize}
	\item The exercise obviously will improve strength and endurance.
	\item ...his powers of endurance.
	\end{itemize}
}
\end{enumerate}

\section*{formal}
{\large \color{blue}  formals  }
\subsection*{Explain}
\begin{enumerate}
\item adjective \\
\textbf{Formal} speech or behaviour is very correct and serious rather than relaxed and friendly , and is used especially in official situations.
 \textit{
	\begin{itemize}
	\item He wrote a very formal letter of apology to Douglas.
	\item Business relationships are necessarily a bit more formal.
	\end{itemize}
}
\item adjective \\
A \textbf{formal} action, statement , or request is an official one.
 \textit{
	\begin{itemize}
	\item U.N. officials said a formal request was passed to American authorities.
	\item No formal announcement had been made.
	\item ...a formal application.
	\end{itemize}
}
\item adjective \\
\textbf{Formal} occasions are special occasions at which people wear smart clothes and behave according to a set of accepted rules.
 \textbf{Formal} is also a noun .
 \textit{
	\begin{itemize}
	\item One evening the film company arranged a formal dinner after the play.
	\item ...a wide array of events, including school formals and speech nights, weddings,
and balls.
	\end{itemize}
}
\item adjective \\
\textbf{Formal} clothes are very smart clothes that are suitable for formal occasions.
 \textit{
	\begin{itemize}
	\item They wore ordinary ties instead of the more formal high collar and cravat.
	\end{itemize}
}
\item graded adjective \\
Something that is done, written, or studied in a \textbf{formal} way has a very ordered, organized method or style.
 \textit{
	\begin{itemize}
	\item This does not encourage the child to analyse the environment in a formal way.
	\item Classic Greek drama was written in verse, usually in an elevated and formal style.
	\item ...a formal methodology.
	\end{itemize}
}
\item adjective \\
\textbf{Formal}  education or training is given officially , usually in a school, college , or university.
 \textit{
	\begin{itemize}
	\item Although his formal education stopped after primary school, he was an avid reader.
	\item Leroy didn't have any formal dance training.
	\end{itemize}
}
\item adjective \\
A \textbf{formal}  garden or room is arranged in a very regular and controlled way, especially according to certain rules of design.
 \textit{
	\begin{itemize}
	\item ...a formal herb garden.
	\item The Coronata wallpaper lends a formal air to the dining room.
	\end{itemize}
}
\end{enumerate}

\section*{facility}
{\large \color{blue}  facilities  }
\subsection*{Explain}
\begin{enumerate}
\item countable noun \\
\textbf{Facilities} are buildings, pieces of equipment, or services that are provided for a particular
 purpose .
 \textit{
	\begin{itemize}
	\item What recreational facilities are now available?
	\item The problem lies in getting patients to a medical facility as soon as possible.
	\end{itemize}
}
\item countable noun \\
A \textbf{facility} is something such as an additional service provided by an organization or an extra  feature on a machine which is useful but not essential .
 \textit{
	\begin{itemize}
	\item It is very useful to have an overdraft facility.
	\item One of the new models has the facility to reproduce speech as well as text.
	\end{itemize}
}
\item countable noun \\
If you have a \textbf{facility}  \textbf{for} something, for example  learning a language, you find it easy to do.
 \textit{
	\begin{itemize}
	\item He and Marcia shared a facility for languages.
	\item Smell is a very basic sense but humans have lost much of the facility to use it properly.
	\end{itemize}
}
\end{enumerate}

\section*{former}
{\large \color{blue}  }
\subsection*{Explain}
\begin{enumerate}
\item adjective \\
\textbf{Former} is used to describe someone who used to have a particular job , position, or role , but no longer has it.
 \textit{
	\begin{itemize}
	\item The unemployed executives include former sales managers, directors and accountants.
	\item ...former President Richard Nixon.
	\item He pleaded not guilty to murdering his former wife.
	\end{itemize}
}
\item adjective \\
\textbf{Former} is used to refer to countries which no longer exist or whose boundaries have changed.
 \textit{
	\begin{itemize}
	\item ...the former Soviet Union.
	\item ...the former Yugoslavia.
	\end{itemize}
}
\item adjective \\
\textbf{Former} is used to describe something which used to belong to someone or which used to be
a particular thing.
 \textit{
	\begin{itemize}
	\item ...the former home of Sir Christopher Wren.
	\item ...a former monastery.
	\end{itemize}
}
\item adjective \\
\textbf{Former} is used to describe a situation or period of time which came before the present one.
 \textit{
	\begin{itemize}
	\item He would want you to remember him as he was in former years.
	\end{itemize}
}
\item pronoun \\
When two people, things, or groups have just been mentioned, you can refer to the
first of them as \textbf{the former} .
 \textit{
	\begin{itemize}
	\item If there is a choice between using fresh vegetables and canned foods, always choose
the former.
	\item Voters want personal prosperity and public spending. They will not sacrifice the
former to the latter.
	\end{itemize}
}
\end{enumerate}

\section*{intelligible}
{\large \color{blue}  }
\subsection*{Explain}
\begin{enumerate}
\item adjective \\
Something that is \textbf{intelligible} can be understood.
 \textit{
	\begin{itemize}
	\item The language of Darwin was intelligible to experts and non-experts alike.
	\item The woman moaned faintly but made no intelligible response.
	\end{itemize}
}
\end{enumerate}

\section*{lovely}
{\large \color{blue}  lovelier  loveliest  }
\subsection*{Explain}
\begin{enumerate}
\item adjective \\
If you describe someone or something as \textbf{lovely} , you mean that they are very beautiful and therefore pleasing to look at or listen to.
 \textit{
	\begin{itemize}
	\item You look lovely, Marcia.
	\item He had a lovely voice.
	\item It was just one of those lovely old English gardens.
	\end{itemize}
}
\item adjective \\
If you describe something as \textbf{lovely} , you mean that it gives you pleasure .
 \textit{
	\begin{itemize}
	\item Mary! How lovely to see you!
	\item It's a lovely day.
	\item What a lovely surprise!
	\end{itemize}
}
\item adjective \\
If you describe someone as \textbf{lovely} , you mean that they are friendly , kind , or generous .
 \textit{
	\begin{itemize}
	\item Laura is a lovely young woman.
	\item She's a lovely child.
	\end{itemize}
}
\end{enumerate}

\section*{innovation}
{\large \color{blue}  innovations  }
\subsection*{Explain}
\begin{enumerate}
\item countable noun \\
An \textbf{innovation} is a new thing or a new method of doing something.
 \textit{
	\begin{itemize}
	\item The vegetarian burger was an innovation which was rapidly exported to Britain.
	\item ...the transformation wrought by the technological innovations of the industrial
age.
	\end{itemize}
}
\item uncountable noun \\
\textbf{Innovation} is the introduction of new ideas , methods, or things.
 \textit{
	\begin{itemize}
	\item We must promote originality and encourage innovation.
	\end{itemize}
}
\end{enumerate}

\section*{inspiration}
{\large \color{blue}  inspirations  }
\subsection*{Explain}
\begin{enumerate}
\item uncountable noun \\
\textbf{Inspiration} is a feeling of enthusiasm you get from someone or something, which gives you new and creative ideas.
 \textit{
	\begin{itemize}
	\item My inspiration comes from poets like Baudelaire and Jacques Prévert.
	\item What better way of finding inspiration for your own garden than by visiting others.
	\end{itemize}
}
\item singular noun \\
If you describe someone or something good as \textbf{an}  \textbf{inspiration} , you mean that they make you or other people want to do or achieve something.
 \textit{
	\begin{itemize}
	\item Powell's unusual journey to high office is an inspiration to millions.
	\item My father was a great inspiration.
	\end{itemize}
}
\item singular noun \\
If something or someone is the \textbf{inspiration} for a particular  book , work of art , or action, they are the source of the ideas in it or act as a model for it.
 \textit{
	\begin{itemize}
	\item India's myths and songs are the inspiration for her books.
	\end{itemize}
}
\item countable noun \\
If you suddenly have an \textbf{inspiration} , you suddenly think of an idea of what to do or say .
 \textit{
	\begin{itemize}
	\item Alison had an inspiration.
	\end{itemize}
}
\end{enumerate}

\section*{native}
{\large \color{blue}  natives  }
\subsection*{Explain}
\begin{enumerate}
\item adjective \\
Your \textbf{native} country or area is the country or area where you were born and brought up.
 \textit{
	\begin{itemize}
	\item It was his first visit to his native country since 1948.
	\item Mother Teresa visited her native Albania.
	\end{itemize}
}
\item countable noun \\
A \textbf{native of} a particular country or region is someone who was born in that country or region.
 \textbf{Native} is also an adjective .
 \textit{
	\begin{itemize}
	\item Dr Aubin is a native of St Blaise.
	\item ...two Dutch volunteer workmen, natives of Tilburg.
	\item Joshua Halpern is a native Northern Californian.
	\item ...men and women native to countries such as Japan.
	\end{itemize}
}
\item countable noun \\
Some European people use \textbf{native} to refer to a person living in a non-Western country who belongs to the race or tribe that the majority of people there belong to. This use could cause offence .
 \textbf{Native} is also an adjective.
 \textit{
	\begin{itemize}
	\item They used force to banish the natives from the more fertile land.
	\item Native people were allowed to retain some sense of their traditional culture and
religion.
	\end{itemize}
}
\item adjective \\
Your \textbf{native} language or tongue is the first language that you learned to speak when you were a child .
 \textit{
	\begin{itemize}
	\item She spoke not only her native language, Swedish, but also English and French.
	\item French is not my native tongue.
	\end{itemize}
}
\item adjective \\
Plants or animals that are \textbf{native to} a particular region live or grow there naturally and were not brought there.
 \textbf{Native} is also a noun .
 \textit{
	\begin{itemize}
	\item ...a project to create a 50 acre forest of native Caledonian pines.
	\item Many of the plants are native to Brazil.
	\item The coconut palm is a native of Malaysia.
	\end{itemize}
}
\item adjective \\
A \textbf{native}  ability or quality is one that you possess naturally without having to learn it.
 \textit{
	\begin{itemize}
	\item We have our native inborn talent, yet we hardly use it.
	\end{itemize}
}
\item  \\
 go native \textit{
	\begin{itemize}
	\end{itemize}
}
\end{enumerate}

\section*{jam}
{\large \color{blue}  jams  jamming  jammed  }
\subsection*{Explain}
\begin{enumerate}
\item variable noun \\
\textbf{Jam} is a thick  sweet food that is made by cooking fruit with a large amount of sugar, and that is usually
 spread on bread .
 \textit{
	\begin{itemize}
	\item ...home-made jam.
	\end{itemize}
}
\item verb \\
If you \textbf{jam} something somewhere , you push or put it there roughly .
 \textit{
	\begin{itemize}
	\item He picked his cap up off the ground and jammed it on his head.
	\item Pete jammed his hands into his pockets.
	\end{itemize}
}
\item verb \\
If something such as a part of a machine \textbf{jams} , or if something \textbf{jams} it, the part becomes fixed in position and is unable to move freely or work properly.
 \textit{
	\begin{itemize}
	\item The second time he fired his gun jammed.
	\item A rope jammed the boat's propeller.
	\item Cracks appeared in the wall and a door jammed shut.
	\item The intake valve was jammed open.
	\item Every few minutes the motor cut out as the machinery became jammed.
	\end{itemize}
}
\item verb \\
If vehicles \textbf{jam} a road, there are so many of them that they cannot move.
 \textbf{Jam} is also a noun .
 \textit{
	\begin{itemize}
	\item Hundreds of departing motorists jammed the roads.
	\item Trucks sat in a jam for ten hours waiting to cross the bridge.
	\end{itemize}
}
\item verb \\
If a lot of people \textbf{jam} a place, or \textbf{jam}  \textbf{into} a place, they are pressed tightly together so that they can hardly move.
 \textit{
	\begin{itemize}
	\item Hundreds of people jammed the boardwalk to watch.
	\item They jammed into buses provided by the Red Cross and headed for safety.
	\end{itemize}
}
\item verb \\
To \textbf{jam} a radio or electronic signal means to interfere with it and prevent it from being received or heard  clearly .
 \textit{
	\begin{itemize}
	\item They will try to jam the transmissions electronically.
	\end{itemize}
}
\item verb \\
If callers  \textbf{are jamming}  phone lines, there are so many callers that the people answering the phones find it difficult to deal with them all.
 \textit{
	\begin{itemize}
	\item Hundreds of callers jammed the BBC switchboard for more than an hour.
	\item The telephone exchange has been jammed all day with people wanting to buy season
tickets.
	\end{itemize}
}
\item singular noun \\
If someone is \textbf{in} a \textbf{jam} , they are in a very difficult situation.
 \textit{
	\begin{itemize}
	\item They were in a real jam, Bob thought glumly.
	\end{itemize}
}
\item verb \\
When jazz or rock musicians  \textbf{are jamming} , they are informally playing music that has not been written down or planned in advance.
 \textbf{Jam} is also a noun.
 \textit{
	\begin{itemize}
	\item He was jamming with his saxophone.
	\item ...a free-form jazz jam.
	\item ...a jam session.
	\end{itemize}
}
\end{enumerate}

\section*{natural}
{\large \color{blue}  naturals  }
\subsection*{Explain}
\begin{enumerate}
\item adjective \\
If you say that it is \textbf{natural} for someone to act in a particular way or for something to happen in that way, you mean that it is reasonable in the circumstances .
 \textit{
	\begin{itemize}
	\item It is only natural for youngsters to crave the excitement of driving a fast car.
	\item It is only natural that he should resent you.
	\item A period of depression can be a perfectly natural response to certain aspects of
life.
	\end{itemize}
}
\item adjective \\
\textbf{Natural} behaviour is shared by all people or all animals of a particular type and has not
been learned.
 \textit{
	\begin{itemize}
	\item ...the insect's natural instinct to feed.
	\item Anger is the natural reaction we experience when we feel threatened or frustrated.
	\end{itemize}
}
\item adjective \\
Someone with a \textbf{natural} ability or skill was born with that ability and did not have to learn it.
 \textit{
	\begin{itemize}
	\item She has a natural ability to understand the motives of others.
	\item He had a natural flair for business.
	\end{itemize}
}
\item countable noun \\
If you say that someone is \textbf{a}  \textbf{natural} , you mean that they do something very well and very easily.
 \textit{
	\begin{itemize}
	\item He's a natural with any kind of engine.
	\item She proved to be a natural on camera.
	\end{itemize}
}
\item adjective \\
If someone's behaviour is \textbf{natural} , they appear to be relaxed and are not trying to hide anything.
 \textit{
	\begin{itemize}
	\item Bethan's sister was as friendly and natural as the rest of the family.
	\item Hannah's natural manner reassured her, and she relaxed.
	\end{itemize}
}
\item adjective \\
\textbf{Natural} things exist or occur in nature and are not made or caused by people.
 \textit{
	\begin{itemize}
	\item It has called the typhoon the worst natural disaster in South Korea in many years.
	\item The gigantic natural harbour of Poole is a haven for boats.
	\end{itemize}
}
\item adjective \\
Someone's \textbf{natural}  parent is their biological  father or mother , as opposed to an adult who is looking after them or has adopted them. Someone's \textbf{natural} child is their biological son or daughter , as opposed to a child they are looking after or have adopted.
 \textit{
	\begin{itemize}
	\item She has been reunited with her natural mother.
	\item His commitments to the stepchildren will not reduce his obligation to his natural
children.
	\end{itemize}
}
\item adjective \\
In music, a \textbf{natural} note is the ordinary note, not its sharp or flat form.
 \textbf{Natural} is also a noun .
 \textit{
	\begin{itemize}
	\item ...B natural.
	\item Is that F a natural or a sharp?
	\end{itemize}
}
\item  \\
 natural causes \textit{
	\begin{itemize}
	\end{itemize}
}
\end{enumerate}

\section*{king}
{\large \color{blue}  kings  }
\subsection*{Explain}
\begin{enumerate}
\item title noun \\
A \textbf{king} is a man who is the most important member of the royal family of his country, and who is considered to be the Head of State of that country.
 \textit{
	\begin{itemize}
	\item ...the king and queen of Spain.
	\item In 1154, Henry II became King of England.
	\item ...King Albert.
	\end{itemize}
}
\item countable noun \\
If you describe a man as \textbf{the king of} something, you mean that he is the most important person doing that thing or he is
the best at doing it.
 \textit{
	\begin{itemize}
	\item He's the king of unlicensed boxing.
	\item He was the king of the big love song.
	\end{itemize}
}
\item countable noun \\
A \textbf{king} is a playing card with a picture of a king on it.
 \textit{
	\begin{itemize}
	\item ...the king of diamonds.
	\end{itemize}
}
\item countable noun \\
In chess, the \textbf{king} is the most important piece. When you are in a position to capture your opponent's king, you win the game.
 \textit{
	\begin{itemize}
	\end{itemize}
}
\item  \\
 live like a king \textit{
	\begin{itemize}
	\end{itemize}
}
\end{enumerate}

\section*{negligible}
{\large \color{blue}  }
\subsection*{Explain}
\begin{enumerate}
\item adjective \\
An amount or effect that is \textbf{negligible} is so small that it is not worth considering or worrying about.
 \textit{
	\begin{itemize}
	\item The pay that the soldiers received was negligible.
	\item Senior managers are convinced that the strike will have a negligible impact.
	\end{itemize}
}
\end{enumerate}

\section*{normal}
{\large \color{blue}  }
\subsection*{Explain}
\begin{enumerate}
\item adjective \\
Something that is \textbf{normal} is usual and ordinary , and is what people expect .
 \textit{
	\begin{itemize}
	\item He has occasional injections to maintain his good health but otherwise he lives a
normal life.
	\item The two countries resumed normal diplomatic relations.
	\item Some of the shops were closed but that's quite normal for a Thursday afternoon.
	\item In November, Clean's bakery produced 50 percent more bread than normal.
	\item Life here will continue as normal.
	\end{itemize}
}
\item adjective \\
A \textbf{normal} person has no serious  physical or mental  health  problems .
 \textit{
	\begin{itemize}
	\item Normal people just don't react like that.
	\item Will the baby be normal?
	\end{itemize}
}
\end{enumerate}

\section*{layer}
{\large \color{blue}  layers  layering  layered  }
\subsection*{Explain}
\begin{enumerate}
\item countable noun \\
A \textbf{layer} of a material or substance is a quantity or piece of it that covers a surface or that is between two other things.
 \textit{
	\begin{itemize}
	\item A fresh layer of snow covered the street.
	\item ...the depletion of the ozone layer.
	\item Arrange all the vegetables except the potatoes in layers.
	\end{itemize}
}
\item countable noun \\
If something such as a system or an idea has many \textbf{layers} , it has many different levels or parts.
 \textit{
	\begin{itemize}
	\item ...an astounding ten layers of staff between the factory worker and the chief executive.
	\item Critics and the public puzzle out the layers of meaning in his photos.
	\end{itemize}
}
\item verb \\
If you \textbf{layer} something, you arrange it in layers.
 \textit{
	\begin{itemize}
	\item Layer the potatoes, asparagus and salmon in the tin.
	\item By lifting and layering her hair, Michael created a lighter frame for her face.
	\end{itemize}
}
\end{enumerate}

\section*{optional}
{\large \color{blue}  }
\subsection*{Explain}
\begin{enumerate}
\item adjective \\
If something is \textbf{optional} , you can  choose whether or not you do it or have it.
 \textit{
	\begin{itemize}
	\item A holiday isn't an optional extra. In this stressful, frantic world it's a must.
	\item Sex education is a sensitive area for some parents, and thus it should remain optional.
	\end{itemize}
}
\end{enumerate}

\section*{obstruction}
{\large \color{blue}  obstructions  }
\subsection*{Explain}
\begin{enumerate}
\item countable noun \\
An \textbf{obstruction} is something that blocks a road or path .
 \textit{
	\begin{itemize}
	\item John was irritated by drivers parking near his house and causing an obstruction.
	\end{itemize}
}
\item variable noun \\
An \textbf{obstruction} is something that blocks a passage in your body.
 \textit{
	\begin{itemize}
	\item The boy was suffering from a bowel obstruction and he died.
	\end{itemize}
}
\item uncountable noun \\
\textbf{Obstruction} is the act of deliberately delaying something or preventing something from happening , usually in business, law , or government.
 \textit{
	\begin{itemize}
	\item Mr Guest refused to let them in and now faces a criminal charge of obstruction.
	\end{itemize}
}
\end{enumerate}

\section*{pathetic}
{\large \color{blue}  }
\subsection*{Explain}
\begin{enumerate}
\item adjective \\
If you describe a person or animal as \textbf{pathetic} , you mean that they are sad and weak or helpless , and they make you feel very sorry for them.
 \textit{
	\begin{itemize}
	\item ...a pathetic little dog with a curly tail.
	\item The small group of onlookers presented a pathetic sight.
	\item She now looked small, shrunken and pathetic.
	\end{itemize}
}
\item adjective \\
If you describe someone or something as \textbf{pathetic} , you mean that they make you feel impatient or angry , often because they are weak or not very good .
 \textit{
	\begin{itemize}
	\item What pathetic excuses.
	\item 'This area is pathetic,' he says. 'It has so few hotels.'
	\item ...the pathetic attempts at public speaking made by members of all parties.
	\item Don't be so pathetic.
	\end{itemize}
}
\end{enumerate}

\section*{patch}
{\large \color{blue}  patches  patching  patched  }
\subsection*{Explain}
\begin{enumerate}
\item countable noun \\
A \textbf{patch} on a surface is a part of it which is different in appearance from the area around
it.
 \textit{
	\begin{itemize}
	\item ...the bald patch on the top of his head.
	\item There was a small patch of blue in the grey clouds.
	\item ...two big damp patches on the carpet.
	\end{itemize}
}
\item countable noun \\
A \textbf{patch}  \textbf{of} land is a small area of land where a particular plant or crop grows.
 \textit{
	\begin{itemize}
	\item ...a patch of land covered in forest.
	\item ...the little vegetable patch in her backyard.
	\item ...a patch of wild cornflowers.
	\end{itemize}
}
\item countable noun \\
A \textbf{patch} is a piece of material which you use to cover a hole in something.
 \textit{
	\begin{itemize}
	\item ...jackets with patches on the elbows.
	\item ...trying to fix the flat tire by putting a patch on it.
	\end{itemize}
}
\item countable noun \\
A \textbf{patch} is a small piece of material which you wear to cover an injured eye.
 \textit{
	\begin{itemize}
	\item She went to the hospital and found him lying down with a patch over his eye.
	\end{itemize}
}
\item verb \\
If you \textbf{patch} something that has a hole in it, you mend it by fastening a patch over the hole.
 \textit{
	\begin{itemize}
	\item He and Walker patched the barn roof.
	\item One of the mechanics took off the damaged tyre, and took it back to the station to
be patched.
	\item ...their patched clothes.
	\end{itemize}
}
\item countable noun \\
A \textbf{patch} is a piece of computer program code written as a temporary solution for dealing with a virus in computer software and distributed by the makers of the original program.
 \textit{
	\begin{itemize}
	\item Older machines will need a software patch to be loaded to correct the date.
	\end{itemize}
}
\item  \\
 a bad patch/a rough patch \textit{
	\begin{itemize}
	\end{itemize}
}
\item  \\
 not a patch on sb/sth \textit{
	\begin{itemize}
	\end{itemize}
}
\end{enumerate}

\section*{peculiar}
{\large \color{blue}  }
\subsection*{Explain}
\begin{enumerate}
\item adjective \\
If you describe someone or something as \textbf{peculiar} , you think that they are strange or unusual, sometimes in an unpleasant way.
 \textit{
	\begin{itemize}
	\item Mr Kennet has a rather peculiar sense of humour.
	\item Rachel thought it tasted peculiar.
	\end{itemize}
}
\item adjective \\
If something is \textbf{peculiar}  \textbf{to} a particular thing, person, or situation , it belongs or relates only to that thing, person, or situation.
 \textit{
	\begin{itemize}
	\item The problem is by no means peculiar to America.
	\end{itemize}
}
\item graded adjective \\
If you say that you \textbf{feel peculiar} , you mean that you feel slightly  ill or unsteady .
 \textit{
	\begin{itemize}
	\item All this has made me feel quite peculiar.
	\end{itemize}
}
\end{enumerate}

\section*{possible}
{\large \color{blue}  possibles  }
\subsection*{Explain}
\begin{enumerate}
\item adjective \\
If it is \textbf{possible}  \textbf{to} do something, it can be done.
 \textit{
	\begin{itemize}
	\item If it is possible to find out where your brother is, we shall.
	\item Everything is possible if we want it enough.
	\item This morning he had tried every way possible to contact her.
	\item Live as you like, leave home if you want–that was never possible when I was young.
	\item It's been a beautiful evening and you have made it all possible.
	\end{itemize}
}
\item adjective \\
A \textbf{possible}  event is one that might happen.
 \textit{
	\begin{itemize}
	\item The families are meeting lawyers to discuss possible action against the police.
	\item Her family is discussing a possible move to America.
	\item One possible solution, if all else fails, is to take legal action.
	\item Is this not a possible outcome of the development of genetically modified food?
	\end{itemize}
}
\item adjective \\
If you say that it is \textbf{possible}  \textbf{that} something is true or correct , you mean that although you do not know whether it is true or correct, you accept that it might be.
 \textit{
	\begin{itemize}
	\item It is possible that there's an explanation for all this.
	\item Of course it's possible that a severe shake-up would make your husband realize how
much you really mean to him.
	\end{itemize}
}
\item adjective \\
If you do something \textbf{as}  soon  \textbf{as possible} , you do it as soon as you can. If you get  \textbf{as} much \textbf{as possible} of something, you get as much of it as you can.
 \textit{
	\begin{itemize}
	\item Please make your decision as soon as possible.
	\item I want to learn as much as possible about the industry so that I'm better prepared.
	\item Michael sat down as far away from her as possible.
	\item Buy fresh produce as often as possible.
	\end{itemize}
}
\item adjective \\
You use \textbf{possible} with superlative  adjectives to emphasize that something has more or less of a quality than anything else of its kind .
 \textit{
	\begin{itemize}
	\item They have joined the job market at the worst possible time.
	\item We expressed in the clearest possible way our disappointment, hurt and anger.
	\item He is doing the best job possible.
	\end{itemize}
}
\item adjective \\
You use \textbf{possible} in expressions such as ' \textbf{if possible} ' and ' \textbf{if at all possible} ' when stating a wish or intention , to show that although this is what you really  want , you may have to accept something different .
 \textit{
	\begin{itemize}
	\item I need to see you, right away if possible.
	\item ...the moral duty to uphold peace if at all possible.
	\end{itemize}
}
\item adjective \\
If you describe someone as, for example , a \textbf{possible}  Prime  Minister , you mean that they may become Prime Minister.
 \textbf{Possible} is also a noun .
 \textit{
	\begin{itemize}
	\item It seems that the studio saw her as a possible successor to Bette Davis.
	\item Bradley has been considered a possible presidential contender himself.
	\item Kennedy was tipped as a presidential possible.
	\item He had been on the Nobel Prize committee's list of possibles.
	\end{itemize}
}
\item singular noun \\
\textbf{The possible} is everything that can be done in a situation .
 \textit{
	\begin{itemize}
	\item He is a democrat with the skill, nerve, and ingenuity to push the limits of the possible.
	\end{itemize}
}
\end{enumerate}

\section*{socialism}
{\large \color{blue}  }
\subsection*{Explain}
\begin{enumerate}
\item uncountable noun \\
\textbf{Socialism} is a set of left-wing political principles whose general  aim is to create a system in which everyone has an equal opportunity to benefit from a country's wealth. Under socialism, the country's main  industries are usually owned by the state.
 \textit{
	\begin{itemize}
	\end{itemize}
}
\end{enumerate}

\section*{raw}
{\large \color{blue}  rawer  rawest  }
\subsection*{Explain}
\begin{enumerate}
\item adjective \\
\textbf{Raw} materials or substances are in their natural state before being processed or used
in manufacturing.
 \textit{
	\begin{itemize}
	\item We import raw materials and energy and export mainly industrial products.
	\item ...two ships carrying raw sugar from Cuba.
	\end{itemize}
}
\item adjective \\
\textbf{Raw} food is food that is eaten  uncooked , that has not yet been cooked, or that has not been cooked enough.
 \textit{
	\begin{itemize}
	\item ...a popular dish made of raw fish.
	\item This versatile vegetable can be eaten raw or cooked.
	\item Half of it is burned and half of it is raw.
	\end{itemize}
}
\item adjective \\
If a part of your body is \textbf{raw} , it is red and painful , perhaps because the skin has come off or has been burnt .
 \textit{
	\begin{itemize}
	\item ...the drag of the rope against the raw flesh of my shoulders.
	\item Her feet hurt and her hands were rubbed raw from unaccustomed work.
	\end{itemize}
}
\item adjective \\
\textbf{Raw}  emotions are strong basic feelings or responses which are not weakened by other influences.
 \textit{
	\begin{itemize}
	\item ...the raw passions of nationalism.
	\item Her grief was still raw and he did not know how to help her.
	\end{itemize}
}
\item adjective \\
If you describe something as \textbf{raw} , you mean that it is simple , powerful , and real .
 \textit{
	\begin{itemize}
	\item ...the raw power of instinct.
	\item ...the raw vitality of his earlier painting.
	\end{itemize}
}
\item adjective \\
\textbf{Raw}  data is facts or information that has not yet been sorted , analysed , or prepared for use.
 \textit{
	\begin{itemize}
	\item Analyses were conducted on the raw data.
	\item ...a statistical model that fully adjusts the census's raw figures.
	\end{itemize}
}
\item adjective \\
If you describe someone in a new job as \textbf{raw} , or as a \textbf{raw}  recruit , you mean that they lack experience in that job.
 \textit{
	\begin{itemize}
	\item ...replacing experienced men with raw recruits.
	\item Davies is still raw but his potential shows.
	\end{itemize}
}
\item adjective \\
\textbf{Raw} weather feels unpleasantly cold.
 \textit{
	\begin{itemize}
	\item Once they cleared the housetops, the wind was raw and biting.
	\item ...a raw December morning.
	\end{itemize}
}
\item adjective \\
\textbf{Raw}  sewage is sewage that has not been treated to make it cleaner .
 \textit{
	\begin{itemize}
	\item ...contamination of bathing water by raw sewage.
	\end{itemize}
}
\item  \\
 a raw deal \textit{
	\begin{itemize}
	\end{itemize}
}
\item  \\
 in the raw \textit{
	\begin{itemize}
	\end{itemize}
}
\end{enumerate}

\section*{society}
{\large \color{blue}  societies  }
\subsection*{Explain}
\begin{enumerate}
\item uncountable noun \\
\textbf{Society} is people in general , thought of as a large organized group.
 \textit{
	\begin{itemize}
	\item This reflects attitudes and values prevailing in society.
	\item He maintains Islam must adapt to modern society.
	\end{itemize}
}
\item variable noun \\
A \textbf{society} is the people who live in a country or region, their organizations, and their way of life.
 \textit{
	\begin{itemize}
	\item We live in a capitalist society.
	\item ...the impact of advertising on the moral fabric of our society.
	\item ...the complexities of South African society.
	\end{itemize}
}
\item countable noun \\
A \textbf{society} is an organization for people who have the same interest or aim .
 \textit{
	\begin{itemize}
	\item ...the North of England Horticultural Society.
	\item ...the historical society.
	\end{itemize}
}
\item uncountable noun \\
\textbf{Society} is the rich , fashionable people in a particular place who meet on social occasions .
 \textit{
	\begin{itemize}
	\item The couple quickly became a fixture of society pages.
	\item ...the high season for society weddings.
	\end{itemize}
}
\end{enumerate}

\section*{sane}
{\large \color{blue}  saner  sanest  }
\subsection*{Explain}
\begin{enumerate}
\item adjective \\
Someone who is \textbf{sane} is able to think and behave  normally and reasonably, and is not mentally ill .
 \textit{
	\begin{itemize}
	\item He seemed perfectly sane.
	\item It wasn't the act of a sane person.
	\end{itemize}
}
\item adjective \\
If you refer to a \textbf{sane} person, action, or system, you mean one that you think is reasonable and sensible .
 \textit{
	\begin{itemize}
	\item No sane person wishes to see conflict or casualties.
	\item ...a sane and safe energy policy.
	\end{itemize}
}
\end{enumerate}

\section*{sociology}
{\large \color{blue}  }
\subsection*{Explain}
\begin{enumerate}
\item uncountable noun \\
\textbf{Sociology} is the study of society or of the way society is organized .
 \textit{
	\begin{itemize}
	\end{itemize}
}
\end{enumerate}

\section*{stereotype}
{\large \color{blue}  stereotypes  stereotyping  stereotyped  }
\subsection*{Explain}
\begin{enumerate}
\item countable noun \\
A \textbf{stereotype} is a fixed general  image or set of characteristics that a lot of people believe  represent a particular type of person or thing.
 \textit{
	\begin{itemize}
	\item There's always been a stereotype about successful businessmen.
	\item Many men feel their body shape doesn't live up to the stereotype of the ideal man.
	\end{itemize}
}
\item verb \\
If someone \textbf{is stereotyped} as something, people form a fixed general idea or image of them, so that it is assumed that they will  behave in a particular way.
 \textit{
	\begin{itemize}
	\item He was stereotyped by some as a rebel.
	\item I get very worked up about the way women are stereotyped in a lot of mainstream films.
	\item You are likely to find many people who have stereotyped ideas about women.
	\end{itemize}
}
\end{enumerate}

\section*{thrift}
{\large \color{blue}  thrifts  }
\subsection*{Explain}
\begin{enumerate}
\item uncountable noun \\
\textbf{Thrift} is the quality and practice of being careful with money and not wasting things.
 \textit{
	\begin{itemize}
	\item They were rightly praised for their thrift and enterprise.
	\end{itemize}
}
\item countable noun \\
A \textbf{thrift} or a \textbf{thrift institution} is a kind of savings bank.
 \textit{
	\begin{itemize}
	\end{itemize}
}
\end{enumerate}

\section*{subsidy}
{\large \color{blue}  subsidies  }
\subsection*{Explain}
\begin{enumerate}
\item countable noun \\
A \textbf{subsidy} is money that is paid by a government or other authority in order to help an industry or business , or to pay for a public service.
 \textit{
	\begin{itemize}
	\item European farmers are planning a massive demonstration against farm subsidy cuts.
	\item They've also slashed state subsidies to utilities and transportation.
	\end{itemize}
}
\end{enumerate}

\section*{twin}
{\large \color{blue}  twins  twinning  twinned  }
\subsection*{Explain}
\begin{enumerate}
\item countable noun \\
If two people are \textbf{twins} , they have the same mother and were born on the same day .
 \textit{
	\begin{itemize}
	\item Sarah was looking after the twins.
	\item I think there are many positive aspects to being a twin.
	\item She had a twin brother and a younger brother.
	\end{itemize}
}
\item adjective \\
\textbf{Twin} is used to describe a pair of things that look the same and are close together.
 \textit{
	\begin{itemize}
	\item ...the twin spires of the cathedral.
	\item ...the world's largest twin-engined aircraft.
	\end{itemize}
}
\item adjective \\
\textbf{Twin} is used to describe two things or ideas that are similar or connected in some way .
 \textit{
	\begin{itemize}
	\item ...the twin concepts of liberty and equality.
	\item Nothing was done to save these women from the twin evils of begging or the workhouse.
	\end{itemize}
}
\item verb \\
When a place or organization in one country \textbf{is twinned}  \textbf{with} a place or organization in another country, a special  relationship is formally established between them.
 \textit{
	\begin{itemize}
	\item My son's state primary school is twinned with a school near Johannesburg.
	\item The busy commercial town is twinned with Truro.
	\end{itemize}
}
\item adjective \\
\textbf{Twin} towns or cities are twinned with each other.
 \textit{
	\begin{itemize}
	\item The thirty Germans were visiting their UK twin town.
	\end{itemize}
}
\end{enumerate}

\section*{supplement}
{\large \color{blue}  supplements  supplementing  supplemented  }
\subsection*{Explain}
\begin{enumerate}
\item verb \\
If you \textbf{supplement} something, you add something to it in order to improve it.
 \textbf{Supplement} is also a noun .
 \textit{
	\begin{itemize}
	\item ...people doing extra jobs outside their regular jobs to supplement their incomes.
	\item I suggest supplementing your diet with vitamins E and A.
	\item Business sponsorship must be a supplement to, not a substitute for, public funding.
	\end{itemize}
}
\item countable noun \\
A \textbf{supplement} is a pill that you take or a special  kind of food that you eat in order to improve your health .
 \textit{
	\begin{itemize}
	\item ...a multiple vitamin and mineral supplement.
	\item I took regular supplements and exercised every day.
	\end{itemize}
}
\item countable noun \\
A \textbf{supplement} is a separate part of a magazine or newspaper, often dealing with a particular topic .
 \textit{
	\begin{itemize}
	\item ...a special supplement to a monthly financial magazine.
	\end{itemize}
}
\item countable noun \\
A \textbf{supplement}  \textbf{to} a book is an additional section, written some time after the main  text and published either at the end of the book or separately.
 \textit{
	\begin{itemize}
	\item ...the supplement to the Encyclopedia Britannica.
	\end{itemize}
}
\item countable noun \\
A \textbf{supplement} is an extra amount of money that you pay in order to obtain special facilities or services, for example when you are travelling or staying at a hotel .
 \textit{
	\begin{itemize}
	\item If you are travelling alone, the single room supplement is £11 a night.
	\end{itemize}
}
\item countable noun \\
A \textbf{supplement} is an extra amount of money that is paid to someone, in addition to their normal  pension or income .
 \textit{
	\begin{itemize}
	\item Some people may be entitled to a housing benefit supplement.
	\item ...people who need a supplement to their basic pension.
	\end{itemize}
}
\end{enumerate}

\section*{ugly}
{\large \color{blue}  uglier  ugliest  }
\subsection*{Explain}
\begin{enumerate}
\item adjective \\
If you say that someone or something is \textbf{ugly} , you mean that they are very unattractive and unpleasant to look at.
 \textit{
	\begin{itemize}
	\item ...an ugly little hat.
	\item She makes me feel dowdy and ugly.
	\end{itemize}
}
\item adjective \\
If you refer to an event or situation as \textbf{ugly} , you mean that it is very unpleasant, usually because it involves violent or aggressive  behaviour .
 \textit{
	\begin{itemize}
	\item There have been some ugly scenes.
	\item The confrontation turned ugly.
	\item ...an ugly publicity stunt.
	\end{itemize}
}
\end{enumerate}

\section*{tail}
{\large \color{blue}  tails  tailing  tailed  }
\subsection*{Explain}
\begin{enumerate}
\item countable noun \\
The \textbf{tail} of an animal, bird , or fish is the part extending beyond the end of its body.
 \textit{
	\begin{itemize}
	\item The cattle were swinging their tails to disperse the flies.
	\item ...a black dog with a long tail.
	\end{itemize}
}
\item countable noun \\
You can use \textbf{tail} to refer to the end or back of something, especially something long and thin .
 \textit{
	\begin{itemize}
	\item ...the horizontal stabilizer bar on the plane's tail.
	\item Elsie tugged her father's coat tail.
	\item ...a comet tail.
	\end{itemize}
}
\item plural noun \\
If a man is wearing  \textbf{tails} , he is wearing a formal  jacket which has two long pieces hanging down at the back.
 \textit{
	\begin{itemize}
	\end{itemize}
}
\item verb \\
To \textbf{tail} someone means to follow close behind them and watch where they go and what they do.
 \textit{
	\begin{itemize}
	\item Officers had tailed the gang from London during a major undercover inquiry.
	\item He trusted her so little that he had her tailed.
	\end{itemize}
}
\item countable noun \\
A \textbf{tail} is someone who is paid to watch and to follow another person.
 \textit{
	\begin{itemize}
	\item He checked behind. No tail.
	\end{itemize}
}
\item adverb \\
If you toss a coin and it comes down \textbf{tails} , you can see the side of it that does not have a picture of a head on it.
 \textit{
	\begin{itemize}
	\end{itemize}
}
\item  \\
 the tail is wagging the dog \textit{
	\begin{itemize}
	\end{itemize}
}
\item  \\
 with your tail between your legs \textit{
	\begin{itemize}
	\end{itemize}
}
\item  \\
 turn tail \textit{
	\begin{itemize}
	\end{itemize}
}
\end{enumerate}

\section*{unlikely}
{\large \color{blue}  unlikelier  unlikeliest  }
\subsection*{Explain}
\begin{enumerate}
\item adjective \\
If you say that something is \textbf{unlikely} to happen or \textbf{unlikely} to be true , you believe that it will not happen or that it is not true, although you are not completely sure .
 \textit{
	\begin{itemize}
	\item A military coup seems unlikely.
	\item As with many technological revolutions, you are unlikely to be aware of it.
	\item It's now unlikely that future parliaments will bring back the death penalty.
	\item In the unlikely event of anybody phoning, could you just scribble a message down?
	\end{itemize}
}
\end{enumerate}

\section*{testimony}
{\large \color{blue}  testimonies  }
\subsection*{Explain}
\begin{enumerate}
\item variable noun \\
In a court of law, someone's \textbf{testimony} is a formal  statement that they make about what they saw someone do or what they know of a situation , after having promised to tell the truth.
 \textit{
	\begin{itemize}
	\item His testimony was an important element of the Prosecution case.
	\item Prosecutors may try to determine if Robb gave false testimony when he appeared before
the grand jury.
	\end{itemize}
}
\item uncountable noun \\
If you say that one thing is \textbf{testimony}  \textbf{to} another, you mean that it shows  clearly that the second thing has a particular quality.
 \textit{
	\begin{itemize}
	\item This book is testimony to a very individual kind of courage.
	\item Her living room is filled with books and papers, a testimony to her dedication to
her work.
	\end{itemize}
}
\end{enumerate}

\section*{weird}
{\large \color{blue}  weirder  weirdest  }
\subsection*{Explain}
\begin{enumerate}
\item adjective \\
If you describe something or someone as \textbf{weird} , you mean that they are strange.
 \textit{
	\begin{itemize}
	\item That first day was weird.
	\item He's different. He's weird.
	\item In the 70s, we did a lot of creative things but also some weird things.
	\item It must be really weird to be rich.
	\item It felt weird going back to Liverpool.
	\end{itemize}
}
\end{enumerate}

\section*{vocabulary}
{\large \color{blue}  vocabularies  }
\subsection*{Explain}
\begin{enumerate}
\item variable noun \\
Your \textbf{vocabulary} is the total number of words you know in a particular language.
 \textit{
	\begin{itemize}
	\item His speech is immature, his vocabulary limited.
	\item We read to improve our vocabularies.
	\end{itemize}
}
\item singular noun \\
The \textbf{vocabulary} of a language is all the words in it.
 \textit{
	\begin{itemize}
	\item ...a new word in the German vocabulary.
	\end{itemize}
}
\item variable noun \\
The \textbf{vocabulary} of a subject is the group of words that are typically used when discussing it.
 \textit{
	\begin{itemize}
	\item ...the vocabulary of natural science.
	\end{itemize}
}
\end{enumerate}

\section*{wretched}
{\large \color{blue}  }
\subsection*{Explain}
\begin{enumerate}
\item adjective \\
You describe someone as \textbf{wretched} when you feel  sorry for them because they are in an unpleasant  situation or have suffered unpleasant experiences .
 \textit{
	\begin{itemize}
	\item These wretched people had seen their homes going up in flames.
	\end{itemize}
}
\item adjective \\
You use \textbf{wretched} to describe someone or something that you dislike or feel angry with.
 \textit{
	\begin{itemize}
	\item Of course this wretched woman was unforgivably irresponsible.
	\item Reality started to hit about four months after we had bought the wretched place.

	\end{itemize}
}
\item adjective \\
Someone who feels \textbf{wretched} feels very unhappy .
 \textit{
	\begin{itemize}
	\item I feel really confused and wretched.
	\item The wretched look on the little girl's face made him sorry.
	\end{itemize}
}
\item graded adjective \\
If you describe something as \textbf{wretched} , you are emphasizing that it is very bad or of very poor quality.
 \textit{
	\begin{itemize}
	\item What a wretched excuse.
	\item The pay has always been wretched.
	\end{itemize}
}
\item graded adjective \\
You describe someone as \textbf{wretched} when you feel sorry for them because they are in an unpleasant situation or have
suffered unpleasant experiences.
 \textit{
	\begin{itemize}
	\item You have built up a huge property empire by buying from wretched people who had to
sell or starve.
	\end{itemize}
}
\end{enumerate}

\section*{angry}
{\large \color{blue}  angrier  angriest  }
\subsection*{Explain}
\begin{enumerate}
\item adjective \\
When you are \textbf{angry} , you feel  strong  dislike or impatience about something.
 \textit{
	\begin{itemize}
	\item She had been very angry at the person who stole her new bike.
	\item Are you angry with me for some reason?
	\item I was angry about the rumours.
	\item He's angry that people have called him a racist.
	\item An angry mob gathered outside the courthouse.
	\end{itemize}
}
\item graded adjective \\
An \textbf{angry}  wound or rash is red and painful .
 \textit{
	\begin{itemize}
	\item He was badly concussed, the glass leaving two angry cuts across his forehead.
	\end{itemize}
}
\item graded adjective \\
If you describe the sky or sea as \textbf{angry} , you mean that it is dark and stormy .
 \textit{
	\begin{itemize}
	\item Under the angry red sky he ran, into the thickening darkness.
	\end{itemize}
}
\end{enumerate}

\section*{adolescent}
{\large \color{blue}  adolescents  }
\subsection*{Explain}
\begin{enumerate}
\item adjective \\
\textbf{Adolescent} is used to describe  young people who are no longer children but who have not yet become adults . It also  refers to their behaviour .
 An \textbf{adolescent} is an adolescent boy or girl .
 \textit{
	\begin{itemize}
	\item It is important that an adolescent boy should have an adult in whom he can confide.
	\item He spent his adolescent years playing guitar in the church band.
	\item ...adolescent rebellion.
	\item Young adolescents are happiest with small groups of close friends.
	\end{itemize}
}
\end{enumerate}

\section*{another}
{\large \color{blue}  }
\subsection*{Explain}
\begin{enumerate}
\item determiner \\
\textbf{Another} thing or person means an additional thing or person of the same type as one that already  exists .
 \textbf{Another} is also a pronoun.
 \textit{
	\begin{itemize}
	\item Mrs. Madrigal buttered another piece of toast.
	\item We're going to have another baby.
	\item The demand generated by one factory required the construction of another.
	\end{itemize}
}
\item determiner \\
You use \textbf{another} when you want to emphasize that an additional thing or person is different to one that already exists.
 \textbf{Another} is also a pronoun.
 \textit{
	\begin{itemize}
	\item I think he's just going to deal with this problem another day.
	\item The counsellor referred her to another therapist.
	\item It appeared to mean one thing but in fact meant quite another.
	\item He didn't really believe that any human being could read another's mind.
	\end{itemize}
}
\item determiner \\
You use \textbf{another} at the beginning of a statement to link it to a previous statement.
 \textit{
	\begin{itemize}
	\item Another time of great excitement for us boys was when war broke out.
	\item Another change that Sue made was to install central heating.
	\end{itemize}
}
\item determiner \\
You use \textbf{another} before a word referring to a distance , length of time, or other amount, to indicate an additional amount.
 \textit{
	\begin{itemize}
	\item Continue down the same road for another 2 kilometres.
	\item He believes prices will not rise by more than another 4 per cent.
	\end{itemize}
}
\item determiner \\
You use \textbf{another} in front of the name of a well-known person, place, or event to indicate that you think someone or something is just like that person, place, or event.
 \textit{
	\begin{itemize}
	\item You may never be another Hemingway, but you can learn to write well.
	\end{itemize}
}
\item  \\
 one another \textit{
	\begin{itemize}
	\end{itemize}
}
\item  \\
 one thing after another \textit{
	\begin{itemize}
	\end{itemize}
}
\item  \\
 or another \textit{
	\begin{itemize}
	\end{itemize}
}
\end{enumerate}

\section*{bronze}
{\large \color{blue}  bronzes  }
\subsection*{Explain}
\begin{enumerate}
\item uncountable noun \\
\textbf{Bronze} is a yellowish-brown metal which is a mixture of copper and tin.
 \textit{
	\begin{itemize}
	\item The bronze statue of Mars is a copy of a famous statue found just outside Todi in
1837.
	\end{itemize}
}
\item countable noun \\
A \textbf{bronze} is a statue or sculpture made of bronze.
 \textit{
	\begin{itemize}
	\item ...a bronze of Napoleon on horseback.
	\end{itemize}
}
\item countable noun \\
A \textbf{bronze} is a bronze medal .
 \textit{
	\begin{itemize}
	\end{itemize}
}
\item colour \\
Something that is \textbf{bronze} is yellowish-brown in colour.
 \textit{
	\begin{itemize}
	\item Her hair shone bronze and gold.
	\item ...huge bronze chrysanthemums.
	\end{itemize}
}
\end{enumerate}

\section*{chief}
{\large \color{blue}  chiefs  }
\subsection*{Explain}
\begin{enumerate}
\item countable noun \\
The \textbf{chief} of an organization is the person who is in charge of it.
 \textit{
	\begin{itemize}
	\item ...a commission appointed by the police chief.
	\item ...Putin's chief of security.
	\end{itemize}
}
\item countable noun \\
The \textbf{chief} of a tribe is its leader.
 \textit{
	\begin{itemize}
	\item ...Sitting Bull, chief of the Sioux tribes of the Great Plains.
	\end{itemize}
}
\item adjective \\
\textbf{Chief} is used in the job  titles of the most senior  worker or workers of a particular kind in an organization.
 \textit{
	\begin{itemize}
	\item ...the chief test pilot.
	\end{itemize}
}
\item adjective \\
The \textbf{chief} cause, part, or member of something is the most important one.
 \textit{
	\begin{itemize}
	\item Financial stress is well established as a chief reason for divorce.
	\item The job went to one of his chief rivals.
	\end{itemize}
}
\end{enumerate}

\section*{bruise}
{\large \color{blue}  bruises  bruising  bruised  }
\subsection*{Explain}
\begin{enumerate}
\item countable noun \\
A \textbf{bruise} is an injury which appears as a purple  mark on your body, although the skin is not broken.
 \textit{
	\begin{itemize}
	\item How did you get that bruise on your cheek?
	\item She was treated for cuts and bruises.
	\end{itemize}
}
\item verb \\
If you \textbf{bruise} a part of your body, a bruise appears on it, for example because something hits you. If you \textbf{bruise}  easily , bruises appear when something hits you only slightly .
 \textit{
	\begin{itemize}
	\item I had only bruised my knee.
	\item Some people bruise more easily than others.
	\end{itemize}
}
\item verb \\
If a fruit , vegetable , or plant \textbf{bruises} or \textbf{is bruised} , it is damaged by being handled  roughly , making a mark on the skin.
 \textbf{Bruise} is also a noun .
 \textit{
	\begin{itemize}
	\item Choose a warm, dry day to cut them off the plants, being careful not to bruise them.
	\item ...bruised tomatoes and cucumbers.
	\item Be sure to store them carefully as they bruise easily.
	\item ...bruises on the fruit's skin.
	\end{itemize}
}
\item verb \\
If you \textbf{are bruised} by an unpleasant  experience , it makes you feel  unhappy or upset .
 \textit{
	\begin{itemize}
	\item The government will be severely bruised by yesterday's events.
	\item Their egos are so easily bruised.
	\end{itemize}
}
\end{enumerate}

\section*{diplomatic}
{\large \color{blue}  }
\subsection*{Explain}
\begin{enumerate}
\item adjective \\
\textbf{Diplomatic} means relating to diplomacy and diplomats.
 \textit{
	\begin{itemize}
	\item ...before the two countries resume full diplomatic relations.
	\item Efforts are being made to avert war and find a diplomatic solution.
	\item These diplomatic skills led to her appointment as the President of the United Nations
General Assembly.
	\end{itemize}
}
\item adjective \\
Someone who is \textbf{diplomatic} is able to be careful to say or do things without offending people.
 \textit{
	\begin{itemize}
	\item She is very direct. I tend to be more diplomatic, I suppose.
	\end{itemize}
}
\end{enumerate}

\section*{burden}
{\large \color{blue}  burdens  burdening  burdened  }
\subsection*{Explain}
\begin{enumerate}
\item countable noun \\
If you describe a problem or a responsibility as a \textbf{burden} , you mean that it causes someone a lot of difficulty , worry , or hard work.
 \textit{
	\begin{itemize}
	\item The developing countries bear the burden of an enormous external debt.
	\item They don't go around with the burdens of the world on their shoulders the whole time.
	\item Her death will be an impossible burden on Paul.
	\item The financial burden will be more evenly shared.
	\end{itemize}
}
\item countable noun \\
A \textbf{burden} is a heavy load that is difficult to carry.
 \textit{
	\begin{itemize}
	\end{itemize}
}
\item verb \\
If someone \textbf{burdens} you \textbf{with} something that is likely to worry you, for example a problem or a difficult decision , they tell you about it.
 \textit{
	\begin{itemize}
	\item We decided not to burden him with the news.
	\end{itemize}
}
\item  \\
 burden of proof \textit{
	\begin{itemize}
	\end{itemize}
}
\end{enumerate}

\section*{eastern}
{\large \color{blue}  }
\subsection*{Explain}
\begin{enumerate}
\item adjective \\
\textbf{Eastern} means in or from the east of a region, state, or country.
 \textit{
	\begin{itemize}
	\item ...Eastern Europe.
	\item ...Pakistan's eastern city of Lahore.
	\item ...France's eastern border with Germany.
	\end{itemize}
}
\item adjective \\
\textbf{Eastern} means coming from or associated with the people or countries of the East, such as India , China , or Japan .
 \textit{
	\begin{itemize}
	\item In many Eastern countries massage was and is a part of everyday life.
	\end{itemize}
}
\end{enumerate}

\section*{button}
{\large \color{blue}  buttons  buttoning  buttoned  }
\subsection*{Explain}
\begin{enumerate}
\item countable noun \\
\textbf{Buttons} are small hard objects sewn on to shirts , coats, or other pieces of clothing. You fasten the clothing by pushing the buttons
through holes called buttonholes.
 \textit{
	\begin{itemize}
	\item ...a coat with brass buttons.
	\end{itemize}
}
\item verb \\
If you \textbf{button} a shirt, coat, or other piece of clothing, you fasten it by pushing its buttons through
the buttonholes.
 \textbf{Button up} means the same as button .
 \textit{
	\begin{itemize}
	\item Ferguson stood up and buttoned his coat.
	\item I buttoned up my coat; it was chilly.
	\item The young man slipped on the shirt and buttoned it up.
	\item It was freezing out there even in his buttoned-up overcoat.
	\end{itemize}
}
\item countable noun \\
A \textbf{button} is a small object on a machine or electrical device that you press in order to operate it.
 \textit{
	\begin{itemize}
	\item He reached for the remote control and pressed the 'play' button.
	\end{itemize}
}
\item countable noun \\
A \textbf{button} is a small piece of metal or plastic which you wear in order to show that you support
a particular movement, organization, or person. You fasten a button to your clothes
with a pin .
 \textit{
	\begin{itemize}
	\end{itemize}
}
\item  \\
 press/push the right button \textit{
	\begin{itemize}
	\end{itemize}
}
\end{enumerate}

\section*{electric}
{\large \color{blue}  }
\subsection*{Explain}
\begin{enumerate}
\item adjective \\
An \textbf{electric} device or machine works by means of electricity, rather than using some other source of power.
 \textit{
	\begin{itemize}
	\item ...her electric guitar.
	\end{itemize}
}
\item adjective \\
An \textbf{electric}  current , voltage , or charge is one that is produced by electricity.
 \textit{
	\begin{itemize}
	\end{itemize}
}
\item adjective \\
\textbf{Electric}  plugs , sockets , or power lines are designed to carry electricity.
 \textit{
	\begin{itemize}
	\end{itemize}
}
\item adjective \\
\textbf{Electric} is used to refer to the supply of electricity.
 \textit{
	\begin{itemize}
	\item An average electric bill might go up $2 or $3 per month.
	\end{itemize}
}
\item adjective \\
If you describe the atmosphere of a place or event as \textbf{electric} , you mean that people are in a state of great excitement .
 \textit{
	\begin{itemize}
	\item The mood in the hall was electric.
	\end{itemize}
}
\end{enumerate}

\section*{cash}
{\large \color{blue}  cashes  cashing  cashed  }
\subsection*{Explain}
\begin{enumerate}
\item uncountable noun \\
\textbf{Cash} is money in the form of notes and coins rather than cheques .
 \textit{
	\begin{itemize}
	\item ...two thousand pounds in cash.
	\end{itemize}
}
\item uncountable noun \\
\textbf{Cash} means the same as money, especially money which is immediately available.
 \textit{
	\begin{itemize}
	\item ...a state-owned financial-services group with plenty of cash.
	\end{itemize}
}
\end{enumerate}

\section*{electrical}
{\large \color{blue}  }
\subsection*{Explain}
\begin{enumerate}
\item adjective \\
\textbf{Electrical} goods, equipment , or appliances work by means of electricity.
 \textit{
	\begin{itemize}
	\item ...shipments of electrical equipment.
	\item ...electrical appliances.
	\end{itemize}
}
\item adjective \\
\textbf{Electrical} systems or parts supply or use electricity.
 \textit{
	\begin{itemize}
	\end{itemize}
}
\item adjective \\
\textbf{Electrical}  energy is energy in the form of electricity.
 \textit{
	\begin{itemize}
	\end{itemize}
}
\item adjective \\
\textbf{Electrical}  industries , engineers , or workers are involved in the production and supply of electricity or electrical goods.
 \textit{
	\begin{itemize}
	\end{itemize}
}
\end{enumerate}

\section*{circuit}
{\large \color{blue}  circuits  }
\subsection*{Explain}
\begin{enumerate}
\item countable noun \\
An electrical  \textbf{circuit} is a complete route which an electric current can flow around.
 \textit{
	\begin{itemize}
	\item Any attempts to cut through the cabling will break the electrical circuit.
	\end{itemize}
}
\item countable noun \\
A \textbf{circuit} is a series of places that are visited regularly by a person or group, especially as a part of their job .
 \textit{
	\begin{itemize}
	\item He joined the professional circuit.
	\item It's a common problem, the one I'm asked about most when I'm on the lecture circuit.
	\end{itemize}
}
\item countable noun \\
A racing \textbf{circuit} is a track on which cars , motorbikes , or cycles race.
 \textit{
	\begin{itemize}
	\end{itemize}
}
\item countable noun \\
A \textbf{circuit}  \textbf{of} a place or area is a journey all the way round it.
 \textit{
	\begin{itemize}
	\item She made a slow circuit of the room.
	\end{itemize}
}
\end{enumerate}

\section*{electronic}
{\large \color{blue}  }
\subsection*{Explain}
\begin{enumerate}
\item adjective \\
An \textbf{electronic} device has transistors or silicon  chips which control and change the electric  current  passing through the device.
 \textit{
	\begin{itemize}
	\item ...expensive electronic equipment.
	\end{itemize}
}
\item adjective \\
An \textbf{electronic} process or activity involves the use of electronic devices.
 \textit{
	\begin{itemize}
	\item ...electronic surveillance.
	\item ...electronic music.
	\end{itemize}
}
\end{enumerate}

\section*{climax}
{\large \color{blue}  climaxes  climaxing  climaxed  }
\subsection*{Explain}
\begin{enumerate}
\item countable noun \\
The \textbf{climax}  \textbf{of} something is the most exciting or important moment in it, usually near the end .
 \textit{
	\begin{itemize}
	\item For Pritchard, getting a medal was the climax of her career.
	\item It was the climax to 24 hours of growing anxiety.
	\item The last golf tournament of the European season is building up to a dramatic climax.
	\end{itemize}
}
\item verb \\
The event that \textbf{climaxes} a sequence of events is an exciting or important event that comes at the end. You can also  say that a sequence of events \textbf{climaxes}  \textbf{with} a particular event.
 \textit{
	\begin{itemize}
	\item The demonstration climaxed two weeks of strikes.
	\item They've just finished a sell-out U.K. tour that climaxed with a three-night stint
at Brixton Academy.
	\end{itemize}
}
\item variable noun \\
A \textbf{climax} is an orgasm .
 \textit{
	\begin{itemize}
	\end{itemize}
}
\item verb \\
When someone \textbf{climaxes} , they have an orgasm .
 \textit{
	\begin{itemize}
	\item Often, a man can enjoy making love but may not be sufficiently aroused to climax.
	\end{itemize}
}
\end{enumerate}

\section*{exciting}
{\large \color{blue}  }
\subsection*{Explain}
\begin{enumerate}
\item adjective \\
If something is \textbf{exciting} , it makes you feel very happy or enthusiastic .
 \textit{
	\begin{itemize}
	\item The race itself is very exciting.
	\item This voyage was the most exciting adventure of their lives.
	\item Jackie was an exciting player to watch.
	\end{itemize}
}
\end{enumerate}

\section*{complaint}
{\large \color{blue}  complaints  }
\subsection*{Explain}
\begin{enumerate}
\item variable noun \\
A \textbf{complaint} is a statement in which you express your dissatisfaction with a particular situation .
 \textit{
	\begin{itemize}
	\item There's been a record number of complaints about the standard of service on Britain's
railways.
	\item People have been reluctant to make formal complaints to the police.
	\item If you feel you have any cause for complaint about the service you should write to
the Hospital Administrator.
	\end{itemize}
}
\item countable noun \\
A \textbf{complaint} is a reason for complaining.
 \textit{
	\begin{itemize}
	\item If you have a complaint about shoes bought from a shop covered by the Footwear Code,
there are several ways of putting the matter right.
	\item I've got no complaints about them.
	\item My main complaint is that we can't go out on the racecourse anymore.
	\end{itemize}
}
\item countable noun \\
You can refer to an illness as a \textbf{complaint} , especially if it is not very serious .
 \textit{
	\begin{itemize}
	\item Eczema is a common skin complaint which often runs in families.
	\end{itemize}
}
\end{enumerate}

\section*{exterior}
{\large \color{blue}  exteriors  }
\subsection*{Explain}
\begin{enumerate}
\item countable noun \\
The \textbf{exterior} of something is its outside surface.
 \textit{
	\begin{itemize}
	\item In one ad the viewer scarcely sees the car's exterior.
	\item The exterior of the building was elegant and graceful.
	\end{itemize}
}
\item countable noun \\
You can refer to someone's usual appearance or behaviour as their \textbf{exterior} , especially when it is very different from their real character.
 \textit{
	\begin{itemize}
	\item According to Mandy, Pat's tough exterior hides a shy and sensitive soul.
	\end{itemize}
}
\item adjective \\
You use \textbf{exterior} to refer to the outside parts of something or things that are outside something.
 \textit{
	\begin{itemize}
	\item The exterior walls were made of pre-formed concrete.
	\item ...the oven's exterior surfaces.
	\end{itemize}
}
\end{enumerate}

\section*{doll}
{\large \color{blue}  dolls  dolling  dolled  }
\subsection*{Explain}
\begin{enumerate}
\item countable noun \\
A \textbf{doll} is a child's toy which looks like a small person or baby .
 \textit{
	\begin{itemize}
	\end{itemize}
}
\end{enumerate}

\section*{external}
{\large \color{blue}  }
\subsection*{Explain}
\begin{enumerate}
\item adjective \\
\textbf{External} is used to indicate that something is on the outside of a surface or body, or that it exists, happens , or comes from outside.
 \textit{
	\begin{itemize}
	\item ...a much reduced heat loss through external walls.
	\item ...internal and external allergic reactions.
	\end{itemize}
}
\item adjective \\
\textbf{External} means involving or intended for foreign countries.
 \textit{
	\begin{itemize}
	\item ...the commissioner for external affairs.
	\item ...Jamaica's external debt.
	\item ...the republic's external borders.
	\end{itemize}
}
\item adjective \\
\textbf{External} means happening or existing in the world in general and affecting you in some way.
 \textit{
	\begin{itemize}
	\item ...a reaction to external events.
	\item Such events occur only when the external conditions are favorable.
	\end{itemize}
}
\item  \\
 external examiner \textit{
	\begin{itemize}
	\end{itemize}
}
\item  \\
 for external use \textit{
	\begin{itemize}
	\end{itemize}
}
\end{enumerate}

\section*{donkey}
{\large \color{blue}  donkeys  }
\subsection*{Explain}
\begin{enumerate}
\item countable noun \\
A \textbf{donkey} is an animal which is like a horse but which is smaller and has longer ears .
 \textit{
	\begin{itemize}
	\end{itemize}
}
\item  \\
 donkey's years \textit{
	\begin{itemize}
	\end{itemize}
}
\end{enumerate}

\section*{extinct}
{\large \color{blue}  }
\subsection*{Explain}
\begin{enumerate}
\item adjective \\
A species of animal or plant that is \textbf{extinct} no longer has any living members, either in the world or in a particular place.
 \textit{
	\begin{itemize}
	\item It is 250 years since the wolf became extinct in Britain.
	\item ...the bones of extinct animals.
	\end{itemize}
}
\item adjective \\
If a particular kind of worker , way of life, or type of activity is \textbf{extinct} , it no longer exists , because of changes in society .
 \textit{
	\begin{itemize}
	\item Herbalism had become an all but extinct skill in the Western world.
	\end{itemize}
}
\item adjective \\
An \textbf{extinct} volcano is one that does not erupt or is not expected to erupt any more.
 \textit{
	\begin{itemize}
	\item Its tallest volcano, long extinct, is Olympus Mons.
	\end{itemize}
}
\end{enumerate}

\section*{drawer}
{\large \color{blue}  drawers  }
\subsection*{Explain}
\begin{enumerate}
\item countable noun \\
A \textbf{drawer} is part of a desk , chest, or other piece of furniture that is shaped like a box and is designed for putting things in. You pull it towards you to open it.
 \textit{
	\begin{itemize}
	\item He opened a drawer in his writing-table.
	\end{itemize}
}
\item plural noun \\
\textbf{Drawers} are underpants, especially ones worn by women.
 \textit{
	\begin{itemize}
	\end{itemize}
}
\end{enumerate}

\section*{flat}
{\large \color{blue}  flats  flatter  flattest  }
\subsection*{Explain}
\begin{enumerate}
\item countable noun \\
A \textbf{flat} is a set of rooms for living in, usually on one floor and part of a larger building.
A flat usually includes a kitchen and bathroom .
 \textit{
	\begin{itemize}
	\item Sara lives with her partner and children in a flat in central London.
	\item …a block of flats
	\item Later on, Victor from flat 10 called.
	\end{itemize}
}
\item adjective \\
Something that is \textbf{flat} is level, smooth, or even, rather than sloping, curved, or uneven .
 \textit{
	\begin{itemize}
	\item Tiles can be fixed to any surface as long as it's flat, firm and dry.
	\item After a moment his right hand moved across the cloth, smoothing it flat.
	\item ...windows which a thief can reach from a drainpipe or flat roof.
	\item The sea was calm, perfectly flat.
	\end{itemize}
}
\item adjective \\
\textbf{Flat} means horizontal and not upright .
 \textit{
	\begin{itemize}
	\item Two men near him threw themselves flat.
	\item As heartburn is usually worse when you're lying down, you should avoid lying flat.
	\end{itemize}
}
\item adjective \\
A \textbf{flat} object is not very tall or deep in relation to its length and width .
 \textit{
	\begin{itemize}
	\item Ellen is walking down the drive with a square flat box balanced on one hand.
	\end{itemize}
}
\item adjective \\
\textbf{Flat} land is level, with no high hills or other raised parts.
 \textit{
	\begin{itemize}
	\item To the north lie the flat and fertile farmlands of the Solway plain.
	\item The landscape became wider, flatter and very scenic.
	\item The highway stretched out flat and straight ahead.
	\end{itemize}
}
\item countable noun \\
A low flat area of uncultivated land, especially an area where the ground is soft and wet , can be referred to as \textbf{flats} or a \textbf{flat} .
 \textit{
	\begin{itemize}
	\item The salt marshes and mud flats attract large numbers of waterfowl.
	\end{itemize}
}
\item countable noun \\
You can refer to one of the broad flat surfaces of an object as \textbf{the flat of} that object.
 \textit{
	\begin{itemize}
	\item He slammed the counter with the flat of his hand.
	\item ...eight cloves of garlic crushed with the flat of a knife.
	\end{itemize}
}
\item adjective \\
\textbf{Flat} shoes have no heels or very low heels.
 \textbf{Flats} are flat shoes.
 \textit{
	\begin{itemize}
	\item People wear slacks, sweaters, flat shoes, and all manner of casual attire for travel.
	\item His mother looked ten years younger in jeans and flats.
	\end{itemize}
}
\item adjective \\
A \textbf{flat} tyre, ball, or balloon does not have enough air in it.
 \textit{
	\begin{itemize}
	\end{itemize}
}
\item countable noun \\
A \textbf{flat} is a tyre that does not have enough air in it.
 \textit{
	\begin{itemize}
	\item Then, after I finally got back on the highway, I developed a flat.
	\end{itemize}
}
\item adjective \\
A drink that is \textbf{flat} is no longer fizzy.
 \textit{
	\begin{itemize}
	\item Could this really stop the champagne from going flat?
	\end{itemize}
}
\item adjective \\
A \textbf{flat} battery has lost some or all of its electrical charge.
 \textit{
	\begin{itemize}
	\item His car alarm had been going off for two days and, as a result, the battery was flat.
	\end{itemize}
}
\item adjective \\
If you have \textbf{flat} feet, the arches of your feet are too low.
 \textit{
	\begin{itemize}
	\item The condition of flat feet runs in families.
	\end{itemize}
}
\item adjective \\
A \textbf{flat}  denial or refusal is definite and firm, and is unlikely to be changed.
 \textit{
	\begin{itemize}
	\item The Foreign Ministry has issued a flat denial of any involvement.
	\item She is likely to give you a flat refusal.
	\end{itemize}
}
\item adjective \\
If you say that something happened , for example, in ten seconds \textbf{flat} or ten minutes \textbf{flat} , you are emphasizing that it happened surprisingly quickly and only took ten seconds or ten minutes.
 \textit{
	\begin{itemize}
	\item You're sitting behind an engine that'll move you from 0 to 60mph in six seconds flat.
	\item I had it all explained to me in two minutes flat.
	\end{itemize}
}
\item adjective \\
A \textbf{flat} rate, price, or percentage is one that is fixed and which applies in every situation.
 \textit{
	\begin{itemize}
	\item Fees are charged at a flat rate, rather than on a percentage basis.
	\item Sometimes there's a flat fee for carrying out a particular task.
	\end{itemize}
}
\item adjective \\
If trade or business is \textbf{flat} , it is slow and inactive, rather than busy and improving or increasing.
 \textit{
	\begin{itemize}
	\item Sales of big pickups were up 14% while car sales stayed flat.
	\item For the country overall, house prices have remained flat.
	\end{itemize}
}
\item adjective \\
If you describe something as \textbf{flat} , you mean that it is dull and not exciting or interesting.
 \textit{
	\begin{itemize}
	\item The past few days have seemed comparatively flat and empty.
	\item The party leader delivered a dreadfully flat speech.
	\end{itemize}
}
\item adjective \\
You use \textbf{flat} to describe someone's voice when they are saying something without expressing any emotion.
 \textit{
	\begin{itemize}
	\item 'Whatever you say,' he said in a deadly flat voice. 'I'll sit here and wait.'
	\item Her voice was flat, with no question or hope in it.
	\end{itemize}
}
\item adjective \\
\textbf{Flat} is used after a letter representing a musical note to show that the note should be
played or sung half a tone lower than the note which otherwise matches that letter.
 \textbf{Flat} is often represented by the symbol ♭ after the letter.
 \textit{
	\begin{itemize}
	\item ...Schubert's B flat Piano Trio (Opus 99).
	\end{itemize}
}
\item adverb \\
If someone sings \textbf{flat} or if a musical instrument is \textbf{flat} , their singing or the instrument is slightly lower in pitch than it should be.
 \textbf{Flat} is also an adjective.
 \textit{
	\begin{itemize}
	\item Her vocal range was limited, and she had a tendency to sing flat.
	\item He had been fired because his singing was flat.
	\end{itemize}
}
\item  \\
 flat as a pancake \textit{
	\begin{itemize}
	\end{itemize}
}
\item  \\
 to fall flat \textit{
	\begin{itemize}
	\end{itemize}
}
\item  \\
 to fall flat \textit{
	\begin{itemize}
	\end{itemize}
}
\item  \\
 flat broke \textit{
	\begin{itemize}
	\end{itemize}
}
\item  \\
 flat out \textit{
	\begin{itemize}
	\end{itemize}
}
\item  \\
 flat out \textit{
	\begin{itemize}
	\end{itemize}
}
\item  \\
 on the flat \textit{
	\begin{itemize}
	\end{itemize}
}
\end{enumerate}

\section*{environment}
{\large \color{blue}  environments  }
\subsection*{Explain}
\begin{enumerate}
\item variable noun \\
Someone's \textbf{environment} is all the circumstances , people, things, and events around them that influence their life.
 \textit{
	\begin{itemize}
	\item Pupils in our schools are taught in a safe, secure environment.
	\item The moral characters of men are formed not by heredity but by environment.
	\item The twins were separated at birth and brought up in entirely different environments.
	\end{itemize}
}
\item countable noun \\
Your \textbf{environment} consists of the particular natural surroundings in which you live or exist , considered in relation to their physical characteristics or weather conditions.
 \textit{
	\begin{itemize}
	\item If our environment cools, then messages from the skin alert the body's thermostat.
	\item ...the maintenance of a safe environment for marine mammals.
	\end{itemize}
}
\item singular noun \\
\textbf{The environment} is the natural world of land, sea, air , plants, and animals.
 \textit{
	\begin{itemize}
	\item ...persuading people to respect the environment.
	\end{itemize}
}
\end{enumerate}

\section*{foreign}
{\large \color{blue}  }
\subsection*{Explain}
\begin{enumerate}
\item adjective \\
Something or someone that is \textbf{foreign}  comes from or relates to a country that is not your own.
 \textit{
	\begin{itemize}
	\item ...a huge attraction for foreign visitors.
	\item She was on her first foreign holiday without her parents.
	\item ...a foreign language.
	\item It is the largest ever private foreign investment in the Bolivian mining sector.
	\end{itemize}
}
\item adjective \\
In politics and journalism , \textbf{foreign} is used to describe people, jobs , and activities relating to countries that are not the country of the person or government
concerned.
 \textit{
	\begin{itemize}
	\item ...the German foreign minister.
	\item I am the foreign correspondent in Washington of La Tribuna newspaper of Honduras.
	\item ...the effects of U.S. foreign policy in the 'free world'.
	\end{itemize}
}
\item adjective \\
A \textbf{foreign} object is something that has got into something else, usually by accident , and should not be there.
 \textit{
	\begin{itemize}
	\item The patient's immune system would reject the transplanted organ as a foreign object.
	\end{itemize}
}
\item adjective \\
Something that is \textbf{foreign}  \textbf{to} a particular person or thing is not typical of them or is unknown to them.
 \textit{
	\begin{itemize}
	\item The very notion of price competition is foreign to many schools.
	\item The whole thing is foreign to us.
	\end{itemize}
}
\end{enumerate}

\section*{expression}
{\large \color{blue}  expressions  }
\subsection*{Explain}
\begin{enumerate}
\item variable noun \\
The \textbf{expression}  \textbf{of} ideas or feelings is the showing of them through words, actions , or artistic  activities .
 \textit{
	\begin{itemize}
	\item Laughter is one of the most infectious expressions of emotion.
	\item From Cairo came expressions of regret at the attack.
	\item ...the rights of the individual to freedom of expression.
	\item Her concern has now found expression in the new environmental protection act.
	\end{itemize}
}
\item variable noun \\
Your \textbf{expression} is the way that your face looks at a particular moment . It shows what you are thinking or feeling.
 \textit{
	\begin{itemize}
	\item The civil servant's expression, however, did not change, not so much as by a flicker.
	\item Levin sat there, an expression of sadness on his face.
	\item The face is entirely devoid of expression.
	\end{itemize}
}
\item uncountable noun \\
\textbf{Expression} is the showing of feeling when you are acting, singing , or playing a musical  instrument .
 \textit{
	\begin{itemize}
	\item I put more expression into my lyrics than a lot of other singers do.
	\end{itemize}
}
\item countable noun \\
An \textbf{expression} is a word or phrase.
 \textit{
	\begin{itemize}
	\item She spoke in a quiet voice but used remarkably coarse expressions.
	\end{itemize}
}
\item countable noun \\
In mathematics , an \textbf{expression} is a symbol or equation which represents a quantity or problem .
 \textit{
	\begin{itemize}
	\item This forms the basis for our mathematical expression for the electric field.
	\end{itemize}
}
\end{enumerate}

\section*{hasty}
{\large \color{blue}  hastier  hastiest  }
\subsection*{Explain}
\begin{enumerate}
\item adjective \\
A \textbf{hasty}  movement , action, or statement is sudden , and often done in reaction to something that has just happened .
 \textit{
	\begin{itemize}
	\item He started screaming insults so I made a hasty escape.
	\end{itemize}
}
\item adjective \\
A \textbf{hasty}  event or action is one that is completed more quickly than normal .
 \textit{
	\begin{itemize}
	\item After the hasty meal, the men had moved forward to take up their positions.
	\end{itemize}
}
\item adjective \\
If you describe a person or their behaviour as \textbf{hasty} , you mean that they are acting too quickly, without thinking carefully, for example because they are angry .
 \textit{
	\begin{itemize}
	\item So let's not be hasty. After all, he can't run away.
	\item A number of the United States' allies had urged him not to take a hasty decision.
	\end{itemize}
}
\end{enumerate}

\section*{fun}
{\large \color{blue}  }
\subsection*{Explain}
\begin{enumerate}
\item uncountable noun \\
You refer to an activity or situation as \textbf{fun} if you think it is pleasant and enjoyable and it causes you to feel  happy .
 \textit{
	\begin{itemize}
	\item This year promises to be terrifically good fun.
	\item It was such a success and we had so much fun doing it.
	\item It could be fun to watch them.
	\item You still have time to join in the fun.
	\end{itemize}
}
\item uncountable noun \\
If you say that someone is \textbf{fun} , you mean that you enjoy being with them because they say and do interesting or amusing things.
 \textit{
	\begin{itemize}
	\item Liz was wonderful fun to be with.
	\end{itemize}
}
\item adjective \\
If you describe something as a \textbf{fun} thing, you mean that you think it is enjoyable. If you describe someone as a \textbf{fun} person, you mean that you enjoy being with them.
 \textit{
	\begin{itemize}
	\item It was a fun evening.
	\item What a fun person he is!
	\end{itemize}
}
\item  \\
 figure of fun \textit{
	\begin{itemize}
	\end{itemize}
}
\item  \\
 for fun \textit{
	\begin{itemize}
	\end{itemize}
}
\item  \\
 fun and games \textit{
	\begin{itemize}
	\end{itemize}
}
\item  \\
 in fun \textit{
	\begin{itemize}
	\end{itemize}
}
\item  \\
 make fun of \textit{
	\begin{itemize}
	\end{itemize}
}
\end{enumerate}

\section*{horrible}
{\large \color{blue}  }
\subsection*{Explain}
\begin{enumerate}
\item adjective \\
If you describe something or someone as \textbf{horrible} , you do not like them at all.
 \textit{
	\begin{itemize}
	\item The record sounds horrible.
	\item ...a horrible small boy.
	\end{itemize}
}
\item adjective \\
You can call something \textbf{horrible} when it causes you to feel  great  shock , fear , and disgust .
 \textit{
	\begin{itemize}
	\item Still the horrible shrieking came out of his mouth.
	\end{itemize}
}
\item adjective \\
\textbf{Horrible} is used to emphasize how bad something is.
 \textit{
	\begin{itemize}
	\item That seems like a horrible mess that will drag on for years.
	\item Unless you respect other people's religions, horrible mistakes and conflict will
occur.
	\end{itemize}
}
\end{enumerate}

\section*{garbage}
{\large \color{blue}  }
\subsection*{Explain}
\begin{enumerate}
\item uncountable noun \\
\textbf{Garbage} is rubbish , especially waste from a kitchen .
 \textit{
	\begin{itemize}
	\item ...a garbage bag.
	\item ...rotting piles of garbage.
	\end{itemize}
}
\item uncountable noun \\
If someone says that an idea or opinion is \textbf{garbage} , they are emphasizing that they believe it is untrue or unimportant .
 \textit{
	\begin{itemize}
	\item I personally think this is complete garbage.
	\item Furious government officials branded her story 'garbage'.
	\end{itemize}
}
\end{enumerate}

\section*{hungry}
{\large \color{blue}  hungrier  hungriest  }
\subsection*{Explain}
\begin{enumerate}
\item adjective \\
When you are \textbf{hungry} , you want some food because you have not eaten for some time and have an uncomfortable or painful  feeling in your stomach .
 \textit{
	\begin{itemize}
	\item My friend was hungry, so we drove to a shopping mall to get some food.
	\end{itemize}
}
\item  \\
 go hungry \textit{
	\begin{itemize}
	\end{itemize}
}
\item adjective \\
If you say that someone is \textbf{hungry} for something, you are emphasizing that they want it very much.
 \textbf{Hungry} is also a combining form.
 \textit{
	\begin{itemize}
	\item Susan was certainly hungry for a life different from the one she had made for herself.
	\item I left Oxford in 1961 hungry to be a critic.
	\item ...power-hungry politicians.
	\end{itemize}
}
\end{enumerate}

\section*{juvenile}
{\large \color{blue}  juveniles  }
\subsection*{Explain}
\begin{enumerate}
\item countable noun \\
A \textbf{juvenile} is a child or young person who is not yet old enough to be regarded as an adult.
 \textit{
	\begin{itemize}
	\item The number of juveniles in the general population has fallen by a fifth in the past
10 years.
	\end{itemize}
}
\item adjective \\
\textbf{Juvenile} activity or behaviour involves young people who are not yet adults.
 \textit{
	\begin{itemize}
	\item Juvenile crime is increasing at a terrifying rate.
	\item ...a scheme to lock up persistent juvenile offenders.
	\end{itemize}
}
\item adjective \\
If you describe someone's behaviour as \textbf{juvenile} , you are critical of it because you think that it is silly or childish .
 \textit{
	\begin{itemize}
	\item Don't be so juvenile!
	\end{itemize}
}
\item countable noun \\
Young animals are sometimes  referred to as \textbf{juveniles} .
 \textit{
	\begin{itemize}
	\end{itemize}
}
\end{enumerate}

\section*{instrumental}
{\large \color{blue}  instrumentals  }
\subsection*{Explain}
\begin{enumerate}
\item adjective \\
Someone or something that is \textbf{instrumental}  \textbf{in} a process or event  helps to make it happen .
 \textit{
	\begin{itemize}
	\item In his first years as chairman he was instrumental in raising the company's wider
profile.
	\item The Senator was instrumental in the release of some of the hostages.
	\end{itemize}
}
\item adjective \\
\textbf{Instrumental} music is performed by instruments and not by voices.
 \textbf{Instrumentals} are pieces of instrumental music.
 \textit{
	\begin{itemize}
	\item ...a recording of vocal and instrumental music.
	\item The last track on the abum is an instrumental.
	\end{itemize}
}
\end{enumerate}

\section*{latitude}
{\large \color{blue}  latitudes  }
\subsection*{Explain}
\begin{enumerate}
\item variable noun \\
The \textbf{latitude} of a place is its distance from the equator. Compare  longitude .
 \textbf{Latitude} is also an adjective .
 \textit{
	\begin{itemize}
	\item In the middle to high latitudes rainfall has risen steadily over the last 20–30 years.
	\item The army must cease military operations above 36 degrees latitude north.
	\end{itemize}
}
\item uncountable noun \\
\textbf{Latitude} is freedom to choose the way in which you do something.
 \textit{
	\begin{itemize}
	\item He would be given every latitude in forming a new government.
	\item His status at the studio afforded him all the artistic latitude he could ask for.
	\end{itemize}
}
\end{enumerate}

\section*{northern}
{\large \color{blue}  }
\subsection*{Explain}
\begin{enumerate}
\item adjective \\
\textbf{Northern} means in or from the north of a region, state, or country.
 \textit{
	\begin{itemize}
	\item Their two children were immigrants to Northern Ireland from Pennsylvania.
	\item Prices at three-star hotels fell furthest in several northern cities.
	\end{itemize}
}
\end{enumerate}

\section*{liability}
{\large \color{blue}  liabilities  }
\subsection*{Explain}
\begin{enumerate}
\item countable noun \\
If you say that someone or something is \textbf{a}  \textbf{liability} , you mean that they cause a lot of problems or embarrassment .
 \textit{
	\begin{itemize}
	\item Team-mates and coach began to see him as a liability.
	\item What was once a vote-catching policy is now a political liability.
	\end{itemize}
}
\item countable noun \\
A company's or organization's \textbf{liabilities} are the sums of money which it owes .
 \textit{
	\begin{itemize}
	\item The company had assets of $138 million and liabilities of $120.5 million.
	\end{itemize}
}
\end{enumerate}

\section*{optimistic}
{\large \color{blue}  }
\subsection*{Explain}
\begin{enumerate}
\item adjective \\
Someone who is \textbf{optimistic} is hopeful about the future or the success of something in particular.
 \textit{
	\begin{itemize}
	\item The President says she is optimistic that an agreement can be worked out soon.
	\item Michael was in a jovial and optimistic mood.
	\end{itemize}
}
\end{enumerate}

\section*{list}
{\large \color{blue}  lists  listing  listed  }
\subsection*{Explain}
\begin{enumerate}
\item countable noun \\
A \textbf{list} of things such as names or addresses is a set of them which all belong to a particular category , written down one below the other.
 \textit{
	\begin{itemize}
	\item Make a list of the top 10 tasks that you can delegate.
	\item There were six names on the list.
	\item ...fine wine from the hotel's exhaustive wine list.
	\end{itemize}
}
\item countable noun \\
A \textbf{list} of things is a set of them that you think of as being in a particular order.
 \textit{
	\begin{itemize}
	\item High on the list of public demands is to end military control of broadcasting.
	\item I would have thought if they were looking for redundancies I would be last on the
list.
	\item The company joined a long list of failed banks.
	\end{itemize}
}
\item verb \\
To \textbf{list} several things such as reasons or names means to write or say them one after another, usually in a particular order.
 \textit{
	\begin{itemize}
	\item The pupils were asked to list the sports they loved most and hated most.
	\item Manufacturers must list ingredients in order of the amount used.
	\end{itemize}
}
\item verb \\
To \textbf{list} something in a particular way means to include it in that way in a list or report .
 \textit{
	\begin{itemize}
	\item A medical examiner has listed the deaths as homicides.
	\item He was not listed under his real name on the residents panel.
	\end{itemize}
}
\item verb \\
If a company \textbf{is listed} , or if it \textbf{lists} , on a stock  exchange , it obtains an official quotation for its shares so that people can buy and sell them.
 \textit{
	\begin{itemize}
	\item It will list on the London Stock Exchange next week with a value of 130 million pounds.
	\end{itemize}
}
\item verb \\
In sailing , if something, especially a ship, \textbf{lists} , it leans over to one side.
 \textbf{List} is also a noun .
 \textit{
	\begin{itemize}
	\item The ship listed again, and she was thrown back across the bunk.
	\item The ship's list was so strong now that almost at once she stumbled.
	\end{itemize}
}
\end{enumerate}

\section*{oriental}
{\large \color{blue}  orientals  }
\subsection*{Explain}
\begin{enumerate}
\item adjective \\
\textbf{Oriental} means coming from or associated with eastern Asia, especially  China and Japan .
 \textit{
	\begin{itemize}
	\item There were Oriental carpets on the floors.
	\item ...oriental food.
	\end{itemize}
}
\item countable noun \\
Some people refer to people from eastern Asia, especially China or Japan as \textbf{Orientals} . This use could cause offence .
 \textit{
	\begin{itemize}
	\item I have always considered tea the beverage of Orientals and hippies.
	\end{itemize}
}
\end{enumerate}

\section*{litter}
{\large \color{blue}  litters  littering  littered  }
\subsection*{Explain}
\begin{enumerate}
\item uncountable noun \\
\textbf{Litter} is rubbish that is left lying around outside.
 \textit{
	\begin{itemize}
	\item If you see litter in the corridor, pick it up.
	\item On Wednesday we cleared a beach and woodland of litter.
	\end{itemize}
}
\item uncountable noun \\
A \textbf{litter}  \textbf{of} things is a quantity of them that are lying around in a disorganized way.
 \textit{
	\begin{itemize}
	\item He pushed aside the litter of books and papers and laid two places at the table.
	\end{itemize}
}
\item verb \\
If a number of things \textbf{litter} a place, they are scattered untidily around it or over it.
 \textit{
	\begin{itemize}
	\item Glass from broken bottles litters the pavement.
	\end{itemize}
}
\item adjective \\
If something is \textbf{littered with} things, it contains many examples of it.
 \textit{
	\begin{itemize}
	\item History is littered with men and women spurred into achievement by a father's disregard.
	\item Charles' speech is littered with lots of marketing buzzwords like 'package' and 'product'.
	\end{itemize}
}
\item countable noun \\
A \textbf{litter} is a group of animals born to the same mother at the same time.
 \textit{
	\begin{itemize}
	\item ...a litter of pups.
	\end{itemize}
}
\item uncountable noun \\
\textbf{Litter} is a dry substance that you put in the container where you want your cat to go to the toilet .
 \textit{
	\begin{itemize}
	\end{itemize}
}
\end{enumerate}

\section*{other}
{\large \color{blue}  others  }
\subsection*{Explain}
\begin{enumerate}
\item adjective \\
You use \textbf{other} to refer to an additional thing or person of the same type as one that has been mentioned or is known about.
 \textbf{Other} is also a pronoun .
 \textit{
	\begin{itemize}
	\item They were just like any other young couple.
	\item The communique gave no other details.
	\item Four crewmen were killed, one other was injured.
	\item In 1914 he (like so many others) lied about his age so that he could join the war
effort.
	\end{itemize}
}
\item adjective \\
You use \textbf{other} to indicate that a thing or person is not the one already mentioned, but a different one.
 \textbf{Other} is also a pronoun.
 \textit{
	\begin{itemize}
	\item The authorities insist that the discussions must not be linked to any other issue.
	\item Calls cost 36p per minute cheap rate and 48p per minute at all other times.
	\item He would have to accept it; there was no other way.
	\item They will then have more money to spend on other things.
	\item This issue, more than any other, has divided her cabinet.
	\item Some of these methods will work. Others will not.
	\end{itemize}
}
\item adjective \\
You use \textbf{the}  \textbf{other} to refer to the second of two things or people when the identity of the first is already known or understood, or has already been mentioned.
 \textbf{The}  \textbf{other} is also a pronoun.
 \textit{
	\begin{itemize}
	\item The Captain was at the other end of the room.
	\item You deliberately went in the other direction.
	\item Half of PML's scientists have first degrees, the other half have PhDs.
	\item Henry was holding a duster in one hand and a kitchen pail in the other.
	\item While one of them tried to put his hand in my pocket, the other held me from behind.
	\end{itemize}
}
\item adjective \\
You use \textbf{other} at the end of a list or a group of examples , to refer generally to people or things like the ones just mentioned.
 \textbf{Other} is also a pronoun.
 \textit{
	\begin{itemize}
	\item The quay will incorporate shops, restaurants and other amenities.
	\item Place them in a jam jar, porcelain bowl, or other similar container.
	\item Descartes received his stimulus from the new physics and astronomy of Copernicus,
Galileo, and others.
	\end{itemize}
}
\item adjective \\
You use \textbf{the}  \textbf{other} to refer to the rest of the people or things in a group, when you are talking about one particular person or thing.
 \textbf{The}  \textbf{others} is also a pronoun.
 \textit{
	\begin{itemize}
	\item When the other pupils were taken to an exhibition, he was left behind.
	\item Aubrey's on his way here, with the others.
	\end{itemize}
}
\item adjective \\
\textbf{Other} people are people in general , as opposed to yourself or a person you have already mentioned.
 \textbf{Others}  means the same as \textbf{other people} .
 \textit{
	\begin{itemize}
	\item The suffering of other people appals me.
	\item She likes to be with other people.
	\item His humour depended on contempt for others.
	\end{itemize}
}
\item adjective \\
You use \textbf{other} in informal  expressions of time such as \textbf{the other day} , \textbf{the other evening} , or \textbf{the other week} to refer to a day, evening, or week in the recent  past .
 \textit{
	\begin{itemize}
	\item I rang her the other day and she said she'd like to come round.
	\item The other evening we had a party.
	\end{itemize}
}
\item  \\
 among other(s) \textit{
	\begin{itemize}
	\end{itemize}
}
\item  \\
 every other day/week/month \textit{
	\begin{itemize}
	\end{itemize}
}
\item  \\
 every other \textit{
	\begin{itemize}
	\end{itemize}
}
\item  \\
 none other than \textit{
	\begin{itemize}
	\end{itemize}
}
\item  \\
 no/nothing other than \textit{
	\begin{itemize}
	\end{itemize}
}
\item  \\
 or other \textit{
	\begin{itemize}
	\end{itemize}
}
\item  \\
 other than \textit{
	\begin{itemize}
	\end{itemize}
}
\end{enumerate}

\section*{lottery}
{\large \color{blue}  lotteries  }
\subsection*{Explain}
\begin{enumerate}
\item countable noun \\
A \textbf{lottery} is a type of gambling  game in which people buy numbered tickets. Several numbers are then chosen , and the people who have those numbers on their tickets win a prize.
 \textit{
	\begin{itemize}
	\item ...the national lottery.
	\end{itemize}
}
\item singular noun \\
If you describe something as \textbf{a lottery} , you mean that what happens  depends  entirely on luck or chance .
 \textit{
	\begin{itemize}
	\item The stockmarket is a lottery.
	\item Which judges are assigned to a case is always a bit of a lottery.
	\end{itemize}
}
\end{enumerate}

\section*{outer}
{\large \color{blue}  }
\subsection*{Explain}
\begin{enumerate}
\item adjective \\
The \textbf{outer} parts of something are the parts which contain or enclose the other parts, and which are furthest from the centre.
 \textit{
	\begin{itemize}
	\item He heard a voice in the outer room.
	\item ...the outer suburbs of the city.
	\end{itemize}
}
\end{enumerate}

\section*{paper}
{\large \color{blue}  papers  papering  papered  }
\subsection*{Explain}
\begin{enumerate}
\item uncountable noun \\
\textbf{Paper} is a material that you write on or wrap things with. The pages of this book are made of paper.
 \textit{
	\begin{itemize}
	\item He wrote his name down on a piece of paper for me.
	\item She sat at the table with pen and paper.
	\item ...a sheet of pretty wrapping paper.
	\item ...a paper bag.
	\end{itemize}
}
\item countable noun \\
A \textbf{paper} is a newspaper .
 \textit{
	\begin{itemize}
	\item I might get a paper in the village.
	\item I'll cook and you read the paper.
	\end{itemize}
}
\item countable noun \\
You can  refer to newspapers in general as \textbf{the paper} or \textbf{the papers} .
 \textit{
	\begin{itemize}
	\item You can't believe everything you read in the paper.
	\item There's been a lot in the papers about the problems facing stepchildren.
	\end{itemize}
}
\item plural noun \\
Your \textbf{papers} are sheets of paper with writing or information on them, which you might  keep in a safe place at home .
 \textit{
	\begin{itemize}
	\item Her papers included unpublished articles and correspondence.
	\end{itemize}
}
\item plural noun \\
Your \textbf{papers} are official documents, for example your passport or identity card , which prove who you are or which give you official permission to do something.
 \textit{
	\begin{itemize}
	\item They have arrested four people who were trying to leave the country with forged papers.
	\end{itemize}
}
\item countable noun \\
A \textbf{paper} is a long, formal piece of writing about an academic subject.
 \textit{
	\begin{itemize}
	\item He just published a paper in the journal Nature, analyzing the fires.
	\end{itemize}
}
\item countable noun \\
A \textbf{paper} is an essay written by a student.
 \textit{
	\begin{itemize}
	\item ...the ten common errors that appear most frequently in student papers.
	\end{itemize}
}
\item countable noun \\
A \textbf{paper} is a part of a written examination in which you answer a number of questions in a particular  period of time.
 \textit{
	\begin{itemize}
	\item We sat each paper in the Hall.
	\item She finished the exam paper.
	\item ...the applied mathematics paper.
	\end{itemize}
}
\item countable noun \\
A \textbf{paper}  prepared by a government or a committee is a report on a question they have been considering or a set of proposals for changes in the law .
 \textit{
	\begin{itemize}
	\item ...a new government paper on climate change.
	\end{itemize}
}
\item adjective \\
\textbf{Paper}  agreements , qualifications , or profits are ones that are stated by official documents to exist, although they may not really be effective or useful .
 \textit{
	\begin{itemize}
	\item They expressed deep mistrust of the paper promises.
	\item We're looking for people who have experience rather than paper qualifications.
	\end{itemize}
}
\item verb \\
If you \textbf{paper} a wall, you put wallpaper on it.
 \textit{
	\begin{itemize}
	\item We papered all four bedrooms.
	\item We have papered this bedroom in softest grey.
	\item The room was strange, the walls half papered, half painted.
	\end{itemize}
}
\item  \\
 on paper \textit{
	\begin{itemize}
	\end{itemize}
}
\item  \\
 on paper \textit{
	\begin{itemize}
	\end{itemize}
}
\item  \\
 not be worth the paper it is written on \textit{
	\begin{itemize}
	\end{itemize}
}
\end{enumerate}

\section*{outward}
{\large \color{blue}  }
\subsection*{Explain}
\begin{enumerate}
\item adjective \\
An \textbf{outward}  journey is a journey that you make away from a place that you are intending to return to later .
 \textit{
	\begin{itemize}
	\item Tickets must be bought seven days in advance, with outward and return journey dates
specified.
	\end{itemize}
}
\item adjective \\
The \textbf{outward}  feelings , qualities, or attitudes of someone or something are the ones they appear to have rather than the ones that they actually have.
 \textit{
	\begin{itemize}
	\item In spite of my outward calm, I was very shaken.
	\item What the military rulers have done is to restore the outward appearance of order.
	\end{itemize}
}
\item adjective \\
The \textbf{outward}  features of something are the ones that you can see from the outside.
 \textit{
	\begin{itemize}
	\item Mark was lying unconscious but with no outward sign of injury.
	\end{itemize}
}
\end{enumerate}

\section*{overseas}
{\large \color{blue}  }
\subsection*{Explain}
\begin{enumerate}
\item adjective \\
You use \textbf{overseas} to describe things that involve or are in foreign countries, usually across a sea or an ocean .
 \textbf{Overseas} is also an adverb .
 \textit{
	\begin{itemize}
	\item He has returned to South Africa from his long overseas trip.
	\item ...overseas trade figures.
	\item People think that living and working overseas is glamorous. It's not.
	\item Much of the investment was overseas.
	\end{itemize}
}
\item adjective \\
An \textbf{overseas}  student or visitor  comes from a foreign country, usually across a sea or an ocean.
 \textit{
	\begin{itemize}
	\item Every year millions of overseas visitors come to London.
	\end{itemize}
}
\end{enumerate}

\section*{rag}
{\large \color{blue}  rags  ragging  ragged  }
\subsection*{Explain}
\begin{enumerate}
\item variable noun \\
A \textbf{rag} is a piece of old cloth which you can use to clean or wipe things.
 \textit{
	\begin{itemize}
	\item He was wiping his hands on an oily rag.
	\item ...a bundle of old rags.
	\item It looked like a piece of rag.
	\end{itemize}
}
\item plural noun \\
\textbf{Rags} are old torn clothes.
 \textit{
	\begin{itemize}
	\item There were men, women and small children, some dressed in rags.
	\end{itemize}
}
\item countable noun \\
People refer to a newspaper as a \textbf{rag} when they have a poor  opinion of it.
 \textit{
	\begin{itemize}
	\item 'This man Tom works for a local rag,' he said.
	\end{itemize}
}
\item verb \\
To \textbf{rag} someone means to make fun of them in an unkind way.
 \textit{
	\begin{itemize}
	\item She was about thirty, ten years older than the youngsters ragging her.
	\end{itemize}
}
\item  \\
 lose one's rag \textit{
	\begin{itemize}
	\end{itemize}
}
\item  \\
 rags to riches \textit{
	\begin{itemize}
	\end{itemize}
}
\item  \\
 like a red rag to a bull \textit{
	\begin{itemize}
	\end{itemize}
}
\end{enumerate}

\section*{parallel}
{\large \color{blue}  parallels  parallelling  parallelled  }
\subsection*{Explain}
\begin{enumerate}
\item countable noun \\
If something has a \textbf{parallel} , it is similar to something else, but exists or happens in a different place or at a different time. If it has \textbf{no parallel} or is \textbf{without parallel} , it is not similar to anything else.
 \textit{
	\begin{itemize}
	\item Readers familiar with English history will find a vague parallel to the suppression
of the monasteries.
	\item It's an ecological disaster with no parallel anywhere else in the world.
	\item ...an achievement without parallel in the modern era.
	\end{itemize}
}
\item countable noun \\
If there are \textbf{parallels} between two things, they are similar in some ways.
 \textit{
	\begin{itemize}
	\item Detailed study of folk music from a variety of countries reveals many close parallels.
	\item There are significant parallels with the 1980s.
	\item Friends of the dead lawyer were quick to draw a parallel between the two murders.
	\end{itemize}
}
\item verb \\
If one thing \textbf{parallels} another, they happen at the same time or are similar, and often seem to be connected.
 \textit{
	\begin{itemize}
	\item Often there are emotional reasons paralleling the financial ones.
	\item His remarks paralleled those of the president.
	\end{itemize}
}
\item adjective \\
\textbf{Parallel} events or situations happen at the same time as one another, or are similar to one another.
 \textit{
	\begin{itemize}
	\item ...parallel talks between the two countries' Foreign Ministers.
	\item Their instincts do not always run parallel with ours.
	\item This is a real world, running parallel to our own.
	\end{itemize}
}
\item adjective \\
If two lines, two objects, or two lines of movement are \textbf{parallel} , they are the same distance apart along their whole length.
 \textit{
	\begin{itemize}
	\item ...seventy-two ships, drawn up in two parallel lines.
	\item Farthing Lane's just above the High Street and parallel with it.
	\item This trail was roughly parallel to the border.
	\end{itemize}
}
\item countable noun \\
A \textbf{parallel} is an imaginary line round the Earth that is parallel to the equator. Parallels are
 shown on maps .
 \textit{
	\begin{itemize}
	\item ...the area south of the 38th parallel.
	\end{itemize}
}
\item  \\
 in parallel \textit{
	\begin{itemize}
	\end{itemize}
}
\end{enumerate}

\section*{reality}
{\large \color{blue}  realities  }
\subsection*{Explain}
\begin{enumerate}
\item uncountable noun \\
You use \textbf{reality} to refer to real things or the real nature of things rather than imagined , invented , or theoretical  ideas .
 \textit{
	\begin{itemize}
	\item Fiction and reality were increasingly blurred.
	\end{itemize}
}
\item countable noun \\
\textbf{The}  \textbf{reality}  \textbf{of} a situation is the truth about it, especially when it is unpleasant or difficult to deal with.
 \textit{
	\begin{itemize}
	\item ...the harsh reality of top international competition.
	\end{itemize}
}
\item singular noun \\
You say that something has become a \textbf{reality} when it actually exists or is actually happening .
 \textit{
	\begin{itemize}
	\item ...the whole procedure that made this book become a reality.
	\item The reality is that they are poor.
	\end{itemize}
}
\item  \\
 in reality \textit{
	\begin{itemize}
	\end{itemize}
}
\end{enumerate}

\section*{regulation}
{\large \color{blue}  regulations  }
\subsection*{Explain}
\begin{enumerate}
\item countable noun \\
\textbf{Regulations} are rules made by a government or other authority in order to control the way something
is done or the way people behave .
 \textbf{Regulation} is also an adjective .
 \textit{
	\begin{itemize}
	\item Employers are using the new regulations to force out people over 65.
	\item Under pressure from the American government, Fiat and other manufacturers obeyed
the new safety regulations.
	\item ...a noisy cheerful group of people in regulation black parade tunics.
	\end{itemize}
}
\item uncountable noun \\
\textbf{Regulation} is the controlling of an activity or process, usually by means of rules.
 \textit{
	\begin{itemize}
	\item Social services also have responsibility for the regulation of nurseries.
	\item Some in the market now want government regulation in order to reduce costs.
	\end{itemize}
}
\end{enumerate}

\section*{pleasant}
{\large \color{blue}  pleasanter  pleasantest  }
\subsection*{Explain}
\begin{enumerate}
\item adjective \\
Something that is \textbf{pleasant} is nice , enjoyable, or attractive .
 \textit{
	\begin{itemize}
	\item I've got a pleasant little apartment.
	\item It's always pleasant to do what you're good at doing.
	\end{itemize}
}
\item adjective \\
Someone who is \textbf{pleasant} is friendly and likeable.
 \textit{
	\begin{itemize}
	\item The woman had a pleasant face.
	\item Lloyd George was most anxious to be agreeable and pleasant.
	\end{itemize}
}
\end{enumerate}

\section*{productive}
{\large \color{blue}  }
\subsection*{Explain}
\begin{enumerate}
\item adjective \\
Someone or something that is \textbf{productive} produces or does a lot for the amount of resources used.
 \textit{
	\begin{itemize}
	\item Training makes workers highly productive.
	\item More productive farmers have been able to provide cheaper food.
	\item ...fertile and productive soils.
	\end{itemize}
}
\item adjective \\
If you say that a relationship between people is \textbf{productive} , you mean that a lot of good or useful things happen as a result of it.
 \textit{
	\begin{itemize}
	\item He was hopeful that the next round of talks would also be productive.
	\item ...the chairman's role in fostering productive relationships between his senior colleagues.

	\end{itemize}
}
\item graded adjective \\
Something that is \textbf{productive of} a situation or feeling  creates it.
 \textit{
	\begin{itemize}
	\item Land, labor and capital are all productive of wealth.
	\end{itemize}
}
\end{enumerate}

\section*{rose}
{\large \color{blue}  roses  }
\subsection*{Explain}
\begin{enumerate}
\item  \\
\textbf{Rose} is the past  tense of rise .
 \textit{
	\begin{itemize}
	\end{itemize}
}
\item countable noun \\
A \textbf{rose} is a flower, often with a pleasant  smell , which grows on a bush with stems that have sharp points called thorns on them.
 \textit{
	\begin{itemize}
	\item She bent to pick a red rose.
	\item ...a bunch of yellow roses.
	\end{itemize}
}
\item countable noun \\
A \textbf{rose} is bush that roses grow on.
 \textit{
	\begin{itemize}
	\item Prune rambling roses when the flowers have faded.
	\item ...fragrant rose bushes.
	\end{itemize}
}
\item colour \\
Something that is \textbf{rose} is reddish-pink in colour.
 \textit{
	\begin{itemize}
	\item ...the rose and violet hues of a twilight sky.
	\end{itemize}
}
\item countable noun \\
A \textbf{rose} is a device with very small holes in it that fits onto the end of a hosepipe or watering
can. The water comes out of the rose in a fine spray so that you can water plants.
 \textit{
	\begin{itemize}
	\end{itemize}
}
\item  \\
 bed of roses \textit{
	\begin{itemize}
	\end{itemize}
}
\item  \\
 come up roses \textit{
	\begin{itemize}
	\end{itemize}
}
\end{enumerate}

\section*{reliable}
{\large \color{blue}  }
\subsection*{Explain}
\begin{enumerate}
\item adjective \\
People or things that are \textbf{reliable} can be trusted to work well or to behave in the way that you want them to.
 \textit{
	\begin{itemize}
	\item She was efficient and reliable.
	\item Japanese cars are so reliable.
	\end{itemize}
}
\item adjective \\
Information that is \textbf{reliable} or that is from a \textbf{reliable}  source is very likely to be correct .
 \textit{
	\begin{itemize}
	\item There is no reliable information about civilian casualties.
	\item It's very difficult to give a reliable estimate.
	\item We have reliable sources.
	\end{itemize}
}
\end{enumerate}

\section*{rubbish}
{\large \color{blue}  rubbishes  rubbishing  rubbished  }
\subsection*{Explain}
\begin{enumerate}
\item uncountable noun \\
\textbf{Rubbish} consists of unwanted things or waste material such as used paper , empty  tins and bottles , and waste food.
 \textit{
	\begin{itemize}
	\item ...unwanted household rubbish.
	\item They had piled most of their rubbish into yellow skips.
	\end{itemize}
}
\item uncountable noun \\
If you think that something is of very poor quality, you can say that it is \textbf{rubbish} .
 \textit{
	\begin{itemize}
	\item He described her book as absolute rubbish.
	\end{itemize}
}
\item uncountable noun \\
If you think that an idea or a statement is foolish or wrong , you can say that it is \textbf{rubbish} .
 \textit{
	\begin{itemize}
	\item He's talking rubbish.
	\item These reports are total and utter rubbish.
	\end{itemize}
}
\item adjective \\
If you think that someone is not very good at something, you can say that they are
 \textbf{rubbish}  \textbf{at} it.
 \textit{
	\begin{itemize}
	\item He was rubbish at his job.
	\item I tried playing golf, but I was rubbish.
	\end{itemize}
}
\item verb \\
If you \textbf{rubbish} a person, their ideas, or their work, you say they are of little value.
 \textit{
	\begin{itemize}
	\item Five whole pages of script were devoted to rubbishing her political opponents.
	\item Officials have simply rubbished all positive ideas.
	\end{itemize}
}
\end{enumerate}

\section*{right}
{\large \color{blue}  rights  righting  righted  }
\subsection*{Explain}
\begin{enumerate}
\item adjective \\
If something is \textbf{right} , it is correct and agrees with the facts.
 \textbf{Right} is also an adverb .
 \textit{
	\begin{itemize}
	\item That's absolutely right.
	\item Clocks never told the right time.
	\item You chip away at the problem until somebody comes up with the right answer.
	\item The barman tells me you saw Ann on Tuesday morning. Is that right?
	\item He guessed right about some things.
	\end{itemize}
}
\item adjective \\
If you do something in the \textbf{right} way or in the \textbf{right} place, you do it as or where it should be done or was planned to be done.
 \textbf{Right} is also an adverb.
 \textit{
	\begin{itemize}
	\item Walking, done in the right way, is a form of aerobic exercise.
	\item They have computerized systems to ensure delivery of the right pizza to the right
place.
	\item The chocolate is then melted down to exactly the right temperature.
	\item To make sure I did everything right, I bought a fat instruction book.
	\end{itemize}
}
\item adjective \\
If you say that someone is seen in \textbf{all the}  \textbf{right} places or knows  \textbf{all the}  \textbf{right} people, you mean that they go to places which are socially acceptable or know people who are socially acceptable.
 \textit{
	\begin{itemize}
	\item He was always to be seen in the right places.
	\item Through his father, he had met all the right people.
	\end{itemize}
}
\item adjective \\
If someone is \textbf{right}  \textbf{about} something, they are correct in what they say or think about it.
 \textit{
	\begin{itemize}
	\item Ron has been right about the result of every General Election but one.
	\item Is that true? Was she right?
	\item Am I right in thinking you're the only person in the club who's actually played in
the Cup Final?
	\end{itemize}
}
\item adjective \\
If something such as a choice , action, or decision is the \textbf{right} one, it is the best or most suitable one.
 \textit{
	\begin{itemize}
	\item She'd made the right choice in leaving New York.
	\item The right decision was made, but probably for the wrong reasons.
	\item They decided the time was right for their escape.
	\end{itemize}
}
\item adjective \\
If something is \textbf{not}  \textbf{right} , there is something unsatisfactory about the situation or thing that you are talking about.
 \textit{
	\begin{itemize}
	\item Ratatouille doesn't taste right with any other oil.
	\item The name Sue Anne never seemed quite right to Molly.
	\item He went into hospital and came out after a week. But he still wasn't right.
	\end{itemize}
}
\item adjective \\
If you think that someone was \textbf{right}  \textbf{to} do something, you think that there were good moral reasons why they did it.
 \textit{
	\begin{itemize}
	\item You were right to do what you did, under the circumstances.
	\item The president was absolutely right in ordering the bombing raid.
	\end{itemize}
}
\item adjective \\
\textbf{Right} is used to refer to activities or actions that are considered to be morally good
and acceptable.
 \textbf{Right} is also a noun.
 \textit{
	\begin{itemize}
	\item It's not right, leaving her like this.
	\item Fox hunting is popular among some people in this country. It doesn't make it right
though.
	\item The BBC thought it was right and proper not to show the film.
	\item At least he knew right from wrong.
	\end{itemize}
}
\item verb \\
If you \textbf{right} something or if it \textbf{rights}  \textbf{itself} , it returns to its normal or correct state, after being in an undesirable state.
 \textit{
	\begin{itemize}
	\item They recognise the urgency of righting the economy.
	\item Your eyesight rights itself very quickly.
	\end{itemize}
}
\item verb \\
If you \textbf{right} a wrong, you do something to make up for a mistake or something bad that you did in the past.
 \textit{
	\begin{itemize}
	\item We've made progress in righting the wrongs of the past.
	\item To right their mistakes, the company will compensate customers.
	\end{itemize}
}
\item verb \\
If you \textbf{right} something that has fallen or rolled over, or if it \textbf{rights}  \textbf{itself} , it returns to its normal upright position.
 \textit{
	\begin{itemize}
	\item He righted the yacht and continued the race.
	\item The helicopter turned at an awful angle before righting itself.
	\end{itemize}
}
\item adjective \\
The \textbf{right} side of a material is the side that is intended to be seen and that faces outwards
when it is made into something.
 \textit{
	\begin{itemize}
	\end{itemize}
}
\item  \\
 to go right \textit{
	\begin{itemize}
	\end{itemize}
}
\item  \\
 in the right \textit{
	\begin{itemize}
	\end{itemize}
}
\item  \\
 to put sth right \textit{
	\begin{itemize}
	\end{itemize}
}
\item  \\
 Mr Right \textit{
	\begin{itemize}
	\end{itemize}
}
\end{enumerate}

\section*{summit}
{\large \color{blue}  summits  }
\subsection*{Explain}
\begin{enumerate}
\item countable noun \\
A \textbf{summit} is a meeting at which the leaders of two or more countries discuss  important matters.
 \textit{
	\begin{itemize}
	\item ...next week's Washington summit.
	\item ...the NATO summit meeting in Rome.
	\end{itemize}
}
\item countable noun \\
The \textbf{summit} of a mountain is the top of it.
 \textit{
	\begin{itemize}
	\item ...the first man to reach the summit of Mount Everest.
	\end{itemize}
}
\end{enumerate}

\section*{satisfactory}
{\large \color{blue}  }
\subsection*{Explain}
\begin{enumerate}
\item adjective \\
Something that is \textbf{satisfactory} is acceptable to you or fulfils a particular need or purpose.
 \textit{
	\begin{itemize}
	\item I never got a satisfactory answer.
	\item It seemed a very satisfactory arrangement.
	\item Neither solution seemed satisfactory.
	\end{itemize}
}
\end{enumerate}

\section*{surface}
{\large \color{blue}  surfaces  surfacing  surfaced  }
\subsection*{Explain}
\begin{enumerate}
\item countable noun \\
The \textbf{surface} of something is the flat  top part of it or the outside of it.
 \textit{
	\begin{itemize}
	\item 97% of all the water on the Earth's surface is salt.
	\item ...tiny little waves on the surface of the water.
	\item The road surface has started breaking up.
	\item Its total surface area was seven thousand square feet.
	\end{itemize}
}
\item countable noun \\
A work \textbf{surface} is a flat area, for example the top of a table , desk , or kitchen  cupboard , on which you can work.
 \textit{
	\begin{itemize}
	\item It can simply be left on the work surface.
	\item Place the fish on a flat surface and sprinkle the flesh with lemon juice and pepper.
	\end{itemize}
}
\item singular noun \\
When you refer to \textbf{the}  \textbf{surface} of a situation , you are talking about what can be seen  easily rather than what is hidden or not immediately  obvious .
 \textit{
	\begin{itemize}
	\item Back in Britain, things appear, on the surface, simpler.
	\item Social unrest is never far below the surface in the capital.
	\item It's brought to the surface a much wider controversy.
	\end{itemize}
}
\item adjective \\
\textbf{Surface} is used to describe the parts of the armed forces which travel by ship or by land rather than underwater or in the air.
 \textit{
	\begin{itemize}
	\item In contrast with its surface fleet, Britain's submarine force was relatively small.
	\item ...Nato surface forces.
	\end{itemize}
}
\item verb \\
If someone or something under water \textbf{surfaces} , they come up to the surface of the water.
 \textit{
	\begin{itemize}
	\item He surfaced, gasping for air.
	\end{itemize}
}
\item verb \\
When something such as a piece of news , a feeling , or a problem  \textbf{surfaces} , it becomes known or becomes obvious.
 \textit{
	\begin{itemize}
	\item The paper says the evidence, when it surfaces, is certain to cause uproar.
	\item The emotions will surface at some point in life.
	\item The same old problems would surface again.
	\end{itemize}
}
\item verb \\
When someone \textbf{surfaces} , they appear after not being seen for some time, for example because they have been asleep .
 \textit{
	\begin{itemize}
	\item There's no chance that he'll surface because he's bound to have heard by now.
	\item What time do you surface?
	\end{itemize}
}
\end{enumerate}

\section*{soluble}
{\large \color{blue}  }
\subsection*{Explain}
\begin{enumerate}
\item adjective \\
A substance that is \textbf{soluble}  will dissolve in a liquid .
 \textit{
	\begin{itemize}
	\item Uranium is soluble in sea water.
	\end{itemize}
}
\item combining form \\
If something is \textbf{water-soluble} or \textbf{fat-soluble} , it will dissolve in water or in fat .
 \textit{
	\begin{itemize}
	\item The red dye on the leather is water-soluble.
	\item ...fat-soluble vitamins.
	\end{itemize}
}
\end{enumerate}

\section*{textile}
{\large \color{blue}  textiles  }
\subsection*{Explain}
\begin{enumerate}
\item countable noun \\
\textbf{Textiles} are types of cloth or fabric, especially ones that have been woven.
 \textit{
	\begin{itemize}
	\item ...decorative textiles for the home.
	\item ...the Scottish textile industry.
	\end{itemize}
}
\item plural noun \\
\textbf{Textiles} are the industries concerned with the manufacture of cloth.
 \textit{
	\begin{itemize}
	\item Another 75,000 jobs will be lost in textiles and clothing.
	\end{itemize}
}
\end{enumerate}

\section*{subjective}
{\large \color{blue}  }
\subsection*{Explain}
\begin{enumerate}
\item adjective \\
Something that is \textbf{subjective} is based on personal  opinions and feelings rather than on facts .
 \textit{
	\begin{itemize}
	\item We know that taste in art is a subjective matter.
	\item The way they interpreted their past was highly subjective.
	\end{itemize}
}
\end{enumerate}

\section*{token}
{\large \color{blue}  tokens  }
\subsection*{Explain}
\begin{enumerate}
\item adjective \\
You use \textbf{token} to describe things or actions which are small or unimportant , but are meant to show particular intentions or feelings which may not be sincere .
 \textit{
	\begin{itemize}
	\item The announcement was widely seen as a token gesture.
	\item Miners have staged a two-hour token stoppage to demand better pay and conditions.
	\end{itemize}
}
\item countable noun \\
A \textbf{token} is a piece of paper or card that can be exchanged for goods, either in a particular shop or as part of a special  offer .
 \textit{
	\begin{itemize}
	\item ...£10 book tokens.
	\item Here is the fifth token towards our offer. You need six of these tokens.
	\end{itemize}
}
\item countable noun \\
A \textbf{token} is a round  flat piece of metal or plastic that is sometimes used instead of money.
 \textit{
	\begin{itemize}
	\item Some of the older telephones still only accept tokens.
	\end{itemize}
}
\item countable noun \\
If you give something to a person or do something for them as a \textbf{token}  \textbf{of} your feelings, you give it or do it as a way of expressing those feelings.
 \textit{
	\begin{itemize}
	\item He kept sending gifts and assured her that they were merely small tokens of his appreciation.
	\item As a token of goodwill, I'm going to write another letter.
	\item ...the custom of exchanging love tokens to celebrate February 14.
	\end{itemize}
}
\item  \\
 by the same token \textit{
	\begin{itemize}
	\end{itemize}
}
\end{enumerate}

\section*{troublesome}
{\large \color{blue}  }
\subsection*{Explain}
\begin{enumerate}
\item adjective \\
You use \textbf{troublesome} to describe something or someone that causes annoying problems or difficulties .
 \textit{
	\begin{itemize}
	\item He needed surgery to cure a troublesome back injury.
	\item Parents may find that a troublesome teenager becomes unmanageable.
	\end{itemize}
}
\item adjective \\
A \textbf{troublesome} situation or issue is full of complicated problems or difficulties.
 \textit{
	\begin{itemize}
	\item The economy has become a troublesome issue for the Government.
	\end{itemize}
}
\end{enumerate}

\section*{used}
{\large \color{blue}  }
\subsection*{Explain}
\begin{enumerate}
\item phrase \\
If something \textbf{used to} be done or \textbf{used to} be the case , it was done regularly in the past or was the case in the past.
 \textit{
	\begin{itemize}
	\item People used to come and visit him every day.
	\item He used to be one of the professors at the School of Education.
	\item I feel more compassion and less anger than I used to.
	\end{itemize}
}
\item phrase \\
If something \textbf{used not to} be done or \textbf{used not to} be the case, it was not done in the past or was not the case in the past. The forms
 \textbf{did not use to} and \textbf{did not used to} are also found, especially in spoken English.
 \textit{
	\begin{itemize}
	\item Borrowing used not to be recommended.
	\item At some point kids start doing things they didn't use to do. They get more independent.
	\item He didn't used to like anyone walking on the lawns in the back garden.
	\end{itemize}
}
\item  \\
 be used to \textit{
	\begin{itemize}
	\end{itemize}
}
\item  \\
 get used to \textit{
	\begin{itemize}
	\end{itemize}
}
\end{enumerate}

\section*{toy}
{\large \color{blue}  toys  toying  toyed  }
\subsection*{Explain}
\begin{enumerate}
\item countable noun \\
A \textbf{toy} is an object that children play with, for example a doll or a model  car .
 \textit{
	\begin{itemize}
	\item He was really too old for children's toys.
	\item ...a toy telephone.
	\end{itemize}
}
\item countable noun \\
You can refer to objects that adults use for fun  rather than for a serious  purpose as \textbf{toys} .
 \textit{
	\begin{itemize}
	\item Computers have become household toys.
	\end{itemize}
}
\end{enumerate}

\section*{verbal}
{\large \color{blue}  }
\subsection*{Explain}
\begin{enumerate}
\item adjective \\
You use \textbf{verbal} to indicate that something is expressed in speech rather than in writing or action.
 \textit{
	\begin{itemize}
	\item They were jostled and subjected to a torrent of verbal abuse.
	\item We have a verbal agreement with her.
	\item The West must back up its verbal support with substantial economic aid.
	\end{itemize}
}
\item adjective \\
You use \textbf{verbal} to indicate that something is connected with words and the use of words.
 \textit{
	\begin{itemize}
	\item The test has scores for verbal skills, mathematical skills, and abstract reasoning
skills.
	\item Wayne has great verbal dexterity.
	\end{itemize}
}
\item adjective \\
In grammar , \textbf{verbal} means relating to a verb.
 \textit{
	\begin{itemize}
	\item ...a verbal noun.
	\end{itemize}
}
\end{enumerate}

\section*{tractor}
{\large \color{blue}  tractors  }
\subsection*{Explain}
\begin{enumerate}
\item countable noun \\
A \textbf{tractor} is a farm vehicle that is used to pull farm machinery and to provide the energy  needed for the machinery to work.
 \textit{
	\begin{itemize}
	\end{itemize}
}
\end{enumerate}

\section*{vital}
{\large \color{blue}  }
\subsection*{Explain}
\begin{enumerate}
\item adjective \\
If you say that something is \textbf{vital} , you mean that it is necessary or very important .
 \textit{
	\begin{itemize}
	\item The port is vital to supply relief to millions of drought victims.
	\item Nick Wileman is a school caretaker so it is vital that he gets on well with young
people.
	\item After her release, she was able to give vital information about her kidnapper.
	\end{itemize}
}
\item adjective \\
If you describe someone or something as \textbf{vital} , you mean that they are very energetic and full of life.
 \textit{
	\begin{itemize}
	\item They are both very vital people and a good match.
	\item They have something important to say and vital and radical ways of saying it.
	\end{itemize}
}
\end{enumerate}

\section*{weapon}
{\large \color{blue}  weapons  }
\subsection*{Explain}
\begin{enumerate}
\item countable noun \\
A \textbf{weapon} is an object such as a gun , a knife , or a missile , which is used to kill or hurt people in a fight or a war.
 \textit{
	\begin{itemize}
	\item ...nuclear weapons.
	\end{itemize}
}
\item countable noun \\
A \textbf{weapon} is something such as knowledge about a particular subject, which you can use to protect yourself or to get what you want in a difficult  situation .
 \textit{
	\begin{itemize}
	\item I attack politicians with the one weapon they don't have, a sense of humor.
	\end{itemize}
}
\end{enumerate}

\section*{white}
{\large \color{blue}  whiter  whitest  whites  }
\subsection*{Explain}
\begin{enumerate}
\item colour \\
Something that is \textbf{white} is the colour of snow or milk.
 \textit{
	\begin{itemize}
	\item He had nice square white teeth.
	\item Issa's white beach hat gleamed in the harsh lights.
	\item He was dressed in white from head to toe.
	\end{itemize}
}
\item adjective \\
A \textbf{white} person has a pale skin and belongs to a race which is of European origin.
 \textbf{Whites} are white people.
 \textit{
	\begin{itemize}
	\item Working with white people hasn't been a problem for me or for them.
	\item He was white, with brown shoulder-length hair and a moustache.
	\item It's a school that's brought blacks and whites and Hispanics together.
	\end{itemize}
}
\item adjective \\
If someone goes  \textbf{white} , the skin on their face becomes very pale, for example because of fear, shock , anger, or illness.
 \textit{
	\begin{itemize}
	\item Richard had gone very white, but he stood his ground.
	\item He turned white and began to stammer.
	\item His face was white with fury.
	\end{itemize}
}
\item adjective \\
\textbf{White} wine is pale yellow in colour.
 You can refer to white wine as \textbf{white} .
 \textit{
	\begin{itemize}
	\item Gregory poured another glass of white wine and went back to his bedroom.
	\item I bought a bottle of Californian white.
	\end{itemize}
}
\item adjective \\
\textbf{White} coffee has had milk or cream added to it.
 \textit{
	\begin{itemize}
	\item Wayne has a large white coffee in front of him.
	\end{itemize}
}
\item adjective \\
\textbf{White} blood cells are the cells in your blood which your body uses to fight infection.
 \textit{
	\begin{itemize}
	\end{itemize}
}
\item adjective \\
People who believe in \textbf{white}  magic believe it is possible to use magic to do good things.
 \textit{
	\begin{itemize}
	\item ...practitioners of white magic.
	\item They claim to be white witches.
	\end{itemize}
}
\item variable noun \\
The \textbf{white} of an egg is the transparent liquid that surrounds the yellow part called the yolk .
 \textit{
	\begin{itemize}
	\end{itemize}
}
\item countable noun \\
The \textbf{white} of someone's eye is the white part that surrounds the coloured part called the iris .
 \textit{
	\begin{itemize}
	\end{itemize}
}
\item plural noun \\
\textbf{Whites} are white clothes that you wear for playing some sports, for example tennis or cricket .
 \textit{
	\begin{itemize}
	\item There was a Frenchman sitting at the next table, immaculate in tennis whites.
	\end{itemize}
}
\end{enumerate}

\section*{youth}
{\large \color{blue}  youths  }
\subsection*{Explain}
\begin{enumerate}
\item uncountable noun \\
Someone's \textbf{youth} is the period of their life during which they are a child, before they are a fully  mature  adult .
 \textit{
	\begin{itemize}
	\item In my youth my ambition had been to be an inventor.
	\item ...the comic books of my youth.
	\end{itemize}
}
\item uncountable noun \\
\textbf{Youth} is the quality or state of being young.
 \textit{
	\begin{itemize}
	\item Gregory was still enchanted with Shannon's youth and joy and beauty.
	\item The team is now a good mixture of experience and youth.
	\end{itemize}
}
\item countable noun \\
Journalists often refer to young men as \textbf{youths} , especially when they are reporting that the young men have caused trouble .
 \textit{
	\begin{itemize}
	\item ...gangs of youths who broke windows and looted shops.
	\item A 17-year-old youth was remanded in custody yesterday.
	\end{itemize}
}
\item plural noun \\
\textbf{The}  \textbf{youth} are young people considered as a group.
 \textit{
	\begin{itemize}
	\item He represents the opinions of the youth of today.
	\item She's not a very good influence on the youth of this country.
	\end{itemize}
}
\end{enumerate}

\section*{alert}
{\large \color{blue}  alerts  alerting  alerted  }
\subsection*{Explain}
\begin{enumerate}
\item adjective \\
If you are \textbf{alert} , you are paying  full  attention to things around you and are able to deal with anything that might  happen .
 \textit{
	\begin{itemize}
	\item We all have to stay alert.
	\item He had been spotted by an alert neighbour.
	\end{itemize}
}
\item adjective \\
If you are \textbf{alert}  \textbf{to} something, you are fully  aware of it.
 \textit{
	\begin{itemize}
	\item The bank is alert to the danger.
	\end{itemize}
}
\item countable noun \\
An \textbf{alert} is a situation in which people prepare themselves for something dangerous that might happen soon .
 \textit{
	\begin{itemize}
	\item Due to a security alert, this train will not be stopping at Oxford Circus.
	\end{itemize}
}
\item verb \\
If you \textbf{alert} someone \textbf{to} a situation, especially a dangerous or unpleasant situation, you tell them about it.
 \textit{
	\begin{itemize}
	\item He wanted to alert people to the activities of the group.
	\item I was hoping he'd alert the police.
	\end{itemize}
}
\item  \\
 on alert \textit{
	\begin{itemize}
	\end{itemize}
}
\item  \\
 on the alert \textit{
	\begin{itemize}
	\end{itemize}
}
\end{enumerate}

\section*{acquisition}
{\large \color{blue}  acquisitions  }
\subsection*{Explain}
\begin{enumerate}
\item variable noun \\
If a company or business person makes an \textbf{acquisition} , they buy another company or part of a company.
 \textit{
	\begin{itemize}
	\item ...the acquisition of a profitable paper recycling company.
	\item ...the number of mergers and acquisitions made by Europe's 1,000 leading firms.
	\end{itemize}
}
\item countable noun \\
If you make an \textbf{acquisition} , you buy or obtain something, often to add to things that you already have.
 \textit{
	\begin{itemize}
	\item How did you go about making this marvellous acquisition then?
	\item ...the President's recent acquisition of a helicopter.
	\end{itemize}
}
\item countable noun \\
You can use \textbf{acquisition} to refer to an object that you buy or obtain, often to add to things that you already have.
 \textit{
	\begin{itemize}
	\item From her wardrobe Laura took her latest acquisition, a bright red dress.
	\end{itemize}
}
\item uncountable noun \\
The \textbf{acquisition} of a skill or a particular type of knowledge is the process of learning it or developing it.
 \textit{
	\begin{itemize}
	\item ...language acquisition.
	\end{itemize}
}
\end{enumerate}

\section*{acre}
{\large \color{blue}  acres  }
\subsection*{Explain}
\begin{enumerate}
\item countable noun \\
An \textbf{acre} is an area of land measuring 4840 square yards or 4047 square metres.
 \textit{
	\begin{itemize}
	\item The property is set in two acres of land.
	\end{itemize}
}
\end{enumerate}

\section*{compulsory}
{\large \color{blue}  }
\subsection*{Explain}
\begin{enumerate}
\item adjective \\
If something is \textbf{compulsory} , you must do it or accept it, because it is the law or because someone in a position of authority  says you must.
 \textit{
	\begin{itemize}
	\item In East Germany, learning Russian was compulsory.
	\item Many young men are trying to get away from compulsory military conscription.
	\end{itemize}
}
\end{enumerate}

\section*{apple}
{\large \color{blue}  apples  }
\subsection*{Explain}
\begin{enumerate}
\item variable noun \\
An \textbf{apple} is a round fruit with smooth green, yellow, or red skin and firm white flesh.
 \textit{
	\begin{itemize}
	\item I want an apple.
	\item ...2kg cooking apples.
	\item ...his ongoing search for the finest varieties of apple.
	\item ...a large garden with apple trees in it.
	\end{itemize}
}
\item  \\
 the apple of your eye \textit{
	\begin{itemize}
	\end{itemize}
}
\end{enumerate}

\section*{considerate}
{\large \color{blue}  }
\subsection*{Explain}
\begin{enumerate}
\item adjective \\
Someone who is \textbf{considerate} pays attention to the needs , wishes , or feelings of other people.
 \textit{
	\begin{itemize}
	\item I think he's the most charming, most considerate man I've ever known.
	\item I've always understood one should try and be considerate of other people.
	\end{itemize}
}
\end{enumerate}

\section*{avenue}
{\large \color{blue}  avenues  }
\subsection*{Explain}
\begin{enumerate}
\item countable noun \\
\textbf{Avenue} is sometimes used in the names of streets. The written abbreviation  Ave. is also used.
 \textit{
	\begin{itemize}
	\item ...the most expensive stores on Park Avenue.
	\end{itemize}
}
\item countable noun \\
An \textbf{avenue} is a wide , straight road, especially one with trees on either side.
 \textit{
	\begin{itemize}
	\end{itemize}
}
\item countable noun \\
An \textbf{avenue} is a way of getting something done.
 \textit{
	\begin{itemize}
	\item Talbot was presented with 80 potential avenues of investigation.
	\item There is another avenue to pursue–it involves further negotiations.
	\end{itemize}
}
\end{enumerate}

\section*{democratic}
{\large \color{blue}  }
\subsection*{Explain}
\begin{enumerate}
\item adjective \\
A \textbf{democratic} country, government, or political system is governed by representatives who are elected by the people.
 \textit{
	\begin{itemize}
	\item The country returned to democratic rule after a series of military governments.
	\item ...the country's first democratic elections.
	\end{itemize}
}
\item adjective \\
Something that is \textbf{democratic} is based on the idea that everyone should have equal rights and should be involved in making important  decisions .
 \textit{
	\begin{itemize}
	\item Education is the basis of a democratic society.
	\item He called for widespread changes to make the armed forces more democratic and less
expensive.
	\end{itemize}
}
\item adjective \\
\textbf{Democratic} is used in the titles of some political parties.
 \textit{
	\begin{itemize}
	\item ...the Social Democratic Party.
	\item ...the People's Democratic Party of Afghanistan.
	\end{itemize}
}
\end{enumerate}

\section*{board}
{\large \color{blue}  boards  boarding  boarded  }
\subsection*{Explain}
\begin{enumerate}
\item countable noun \\
A \textbf{board} is a flat, thin, rectangular piece of wood or plastic which is used for a particular
purpose.
 \textit{
	\begin{itemize}
	\item ...a chopping board.
	\end{itemize}
}
\item countable noun \\
A \textbf{board} is a square piece of wood or stiff cardboard that you use for playing games such as chess.
 \textit{
	\begin{itemize}
	\item ...a draughts board.
	\item Dr Tinsley had five pieces on the board against Chinook's four.
	\end{itemize}
}
\item countable noun \\
You can refer to a blackboard or a noticeboard as a \textbf{board} .
 \textit{
	\begin{itemize}
	\item He wrote a few more notes on the board.
	\end{itemize}
}
\item countable noun \\
\textbf{Boards} are long flat pieces of wood which are used, for example, to make floors or walls.
 \textit{
	\begin{itemize}
	\item The floor was draughty bare boards.
	\end{itemize}
}
\item countable noun \\
\textbf{The}  \textbf{board} of a company or organization is the group of people who control it and direct it.
 \textit{
	\begin{itemize}
	\item Arthur wants to put his recommendation before the board at a meeting tomorrow.
	\item ...the agenda for the September 12 board meeting.
	\end{itemize}
}
\item countable noun \\
\textbf{Board} is used in the names of various organizations which are involved in dealing with a particular kind of activity.
 \textit{
	\begin{itemize}
	\item A booklet listing all types of accommodation is published each year by the Spanish
Tourist Board.
	\item ...the U.S. National Transportation Safety Board.
	\end{itemize}
}
\item verb \\
When you \textbf{board} a train, ship, or aircraft, you get on it in order to travel  somewhere .
 \textit{
	\begin{itemize}
	\item I boarded the plane bound for England.
	\end{itemize}
}
\item uncountable noun \\
\textbf{Board} is the food which is provided when you stay somewhere, for example in a hotel .
 \textit{
	\begin{itemize}
	\item Free room and board are provided for all hotel staff.
	\end{itemize}
}
\item  \\
 above board \textit{
	\begin{itemize}
	\end{itemize}
}
\item  \\
 across the board \textit{
	\begin{itemize}
	\end{itemize}
}
\item  \\
 go by the board \textit{
	\begin{itemize}
	\end{itemize}
}
\item  \\
 on board \textit{
	\begin{itemize}
	\end{itemize}
}
\item  \\
 to sweep the board \textit{
	\begin{itemize}
	\end{itemize}
}
\item  \\
 take on board \textit{
	\begin{itemize}
	\end{itemize}
}
\end{enumerate}

\section*{dizzy}
{\large \color{blue}  dizzier  dizziest  dizzies  dizzying  dizzied  }
\subsection*{Explain}
\begin{enumerate}
\item adjective \\
If you feel  \textbf{dizzy} , you feel that you are losing your balance and are about to fall .
 \textit{
	\begin{itemize}
	\item Her head still hurt, and she felt slightly dizzy and disoriented.
	\item He began to get dizzy spells.
	\end{itemize}
}
\item adjective \\
You can use \textbf{dizzy} to describe a woman who is careless and forgets things, but is easy to like .
 \textit{
	\begin{itemize}
	\item She is famed for playing dizzy blondes.
	\item ...a charmingly dizzy great-grandmother.
	\end{itemize}
}
\item verb \\
If something \textbf{dizzies} you, it causes you to feel unsteady or confused.
 \textit{
	\begin{itemize}
	\item The sudden height dizzied her and she clung tightly.
	\end{itemize}
}
\item  \\
 dizzy heights \textit{
	\begin{itemize}
	\end{itemize}
}
\end{enumerate}

\section*{dynamic}
{\large \color{blue}  dynamics  }
\subsection*{Explain}
\begin{enumerate}
\item adjective \\
If you describe someone as \textbf{dynamic} , you approve of them because they are full of energy or full of new and exciting ideas.
 \textit{
	\begin{itemize}
	\item He seemed a dynamic and energetic leader.
	\item Marcus was handsome, dynamic and ambitious.
	\end{itemize}
}
\item adjective \\
If you describe something as \textbf{dynamic} , you approve of it because it is very active and energetic .
 \textit{
	\begin{itemize}
	\item South Asia continues to be the most dynamic economic region in the world.
	\item ...90 minutes of dynamic Indian folk dance.
	\end{itemize}
}
\item adjective \\
A \textbf{dynamic} process is one that constantly changes and progresses .
 \textit{
	\begin{itemize}
	\item ...a dynamic, evolving worldwide epidemic.
	\item Political debate is dynamic.
	\end{itemize}
}
\item countable noun \\
The \textbf{dynamic} of a system or process is the force that causes it to change or progress.
 \textit{
	\begin{itemize}
	\item The dynamic of the market demands constant change and adjustment.
	\item Politics has its own dynamic.
	\end{itemize}
}
\item plural noun \\
The \textbf{dynamics} of a situation or group of people are the opposing forces within it that cause it to change.
 \textit{
	\begin{itemize}
	\item ...the dynamics of the social system.
	\item The interchange of ideas aids an understanding of family dynamics.
	\end{itemize}
}
\item uncountable noun \\
\textbf{Dynamics} are forces which produce power or movement.
 \textit{
	\begin{itemize}
	\item Scientists observe the same dynamics in fluids.
	\end{itemize}
}
\item uncountable noun \\
\textbf{Dynamics} is the scientific study of motion, energy, and forces.
 \textit{
	\begin{itemize}
	\item His idea was to apply geometry to dynamics.
	\end{itemize}
}
\end{enumerate}

\section*{constitution}
{\large \color{blue}  constitutions  }
\subsection*{Explain}
\begin{enumerate}
\item countable noun \\
The \textbf{constitution} of a country or organization is the system of laws which formally states people's rights and duties .
 \textit{
	\begin{itemize}
	\item The king was forced to adopt a new constitution which reduced his powers.
	\item ...the American Constitution.
	\item At one time, the club's constitution prevented women from becoming full members.
	\end{itemize}
}
\item countable noun \\
Your \textbf{constitution} is your health.
 \textit{
	\begin{itemize}
	\item He must have an extremely strong constitution.
	\item I've always had the constitution of an ox.
	\end{itemize}
}
\end{enumerate}

\section*{excess}
{\large \color{blue}  excesses  }
\subsection*{Explain}
\begin{enumerate}
\item variable noun \\
An \textbf{excess}  \textbf{of} something is a larger amount than is needed , allowed, or usual .
 \textit{
	\begin{itemize}
	\item An excess of house plants in a small flat can be oppressive.
	\item Polyunsaturated oils are essential for health. Excess is harmful, however.
	\end{itemize}
}
\item adjective \\
\textbf{Excess} is used to describe amounts that are greater than what is needed, allowed, or usual.
 \textit{
	\begin{itemize}
	\item After cooking the fish, pour off any excess fat.
	\end{itemize}
}
\item uncountable noun \\
\textbf{Excess} is behaviour that is unacceptable because it is considered too extreme or immoral .
 \textit{
	\begin{itemize}
	\item She said she was sick of her life of excess.
	\item ...adolescent excess.
	\item ...the bloody excesses of warfare and empire-building.
	\end{itemize}
}
\item adjective \\
\textbf{Excess} is used to refer to additional amounts of money that need to be paid for services and activities that were not originally  planned or taken into account .
 \textit{
	\begin{itemize}
	\item ...a letter demanding an excess fare of £20.
	\item Staff who have to travel farther can claim excess travel expenses.
	\end{itemize}
}
\item countable noun \\
The \textbf{excess} on an insurance policy is a sum of money which the insured person has to pay towards the cost of a claim. The insurance company pays the rest .
 \textit{
	\begin{itemize}
	\item The company wanted £1,800 for a policy with a £400 excess for under-21s.
	\end{itemize}
}
\item  \\
 in excess of \textit{
	\begin{itemize}
	\end{itemize}
}
\item  \\
 to excess \textit{
	\begin{itemize}
	\end{itemize}
}
\end{enumerate}

\section*{excessive}
{\large \color{blue}  }
\subsection*{Explain}
\begin{enumerate}
\item adjective \\
If you describe the amount or level of something as \textbf{excessive} , you disapprove of it because it is more or higher than is necessary or reasonable .
 \textit{
	\begin{itemize}
	\item ...the alleged use of excessive force by police.
	\item The government says that local authority spending is excessive.
	\end{itemize}
}
\end{enumerate}

\section*{cricket}
{\large \color{blue}  crickets  }
\subsection*{Explain}
\begin{enumerate}
\item uncountable noun \\
\textbf{Cricket} is an outdoor game played between two teams. Players try to score points, called runs, by hitting a ball with a wooden bat.
 \textit{
	\begin{itemize}
	\item During the summer term we would play cricket at the village ground.
	\item ...the Yorkshire County Cricket Club.
	\end{itemize}
}
\item  \\
 not cricket \textit{
	\begin{itemize}
	\end{itemize}
}
\item countable noun \\
A \textbf{cricket} is a small jumping insect that produces short , loud sounds by rubbing its wings together.
 \textit{
	\begin{itemize}
	\end{itemize}
}
\end{enumerate}

\section*{horizontal}
{\large \color{blue}  horizontals  }
\subsection*{Explain}
\begin{enumerate}
\item adjective \\
Something that is \textbf{horizontal} is flat and level with the ground , rather than at an angle to it.
 \textbf{Horizontal} is also a noun .
 \textit{
	\begin{itemize}
	\item The board consists of vertical and horizontal lines.
	\item Swing the club back until it is horizontal.
	\item Do not raise your left arm above the horizontal.
	\end{itemize}
}
\item countable noun \\
A \textbf{horizontal} is a line or structure that is horizontal.
 \textit{
	\begin{itemize}
	\item Horizontals play a large part in all garden design.
	\end{itemize}
}
\end{enumerate}

\section*{cup}
{\large \color{blue}  cups  cupping  cupped  }
\subsection*{Explain}
\begin{enumerate}
\item countable noun \\
A \textbf{cup} is a small round container that you drink from. Cups usually have handles and are made from china or plastic .
 A \textbf{cup}  \textbf{of} something is the amount of something contained in a cup.
 \textit{
	\begin{itemize}
	\item ...cups and saucers.
	\item Mix about four cups of white flour with a pinch of salt.
	\end{itemize}
}
\item countable noun \\
Things, or parts of things, that are small, round, and hollow in shape can be referred to as \textbf{cups} .
 \textit{
	\begin{itemize}
	\item ...the brass cups of the small chandelier.
	\end{itemize}
}
\item countable noun \\
A \textbf{cup} is a large metal cup with two handles that is given to the winner of a game or competition .
 \textit{
	\begin{itemize}
	\end{itemize}
}
\item countable noun \\
\textbf{Cup} is used in the names of some sports competitions in which the prize is a cup.
 \textit{
	\begin{itemize}
	\item Sri Lanka's cricket team will play India in the final of the Asia Cup.
	\item ...after his fateful injury in the FA Cup final.
	\end{itemize}
}
\item verb \\
If you \textbf{cup} your \textbf{hands} , you make them into a curved shape like a cup.
 \textit{
	\begin{itemize}
	\item He cupped his hands around his mouth and called out for Diane.
	\item David knelt, cupped his hands and splashed river water on to his face.
	\item She held it in her cupped hands for us to see.
	\end{itemize}
}
\item verb \\
If you \textbf{cup} something in your hands, you make your hands into a curved dish-like shape and support
it or hold it gently.
 \textit{
	\begin{itemize}
	\item He cupped her chin in the palm of his hand.
	\item He cradled the baby in his arms, his hands cupping her tiny skull.
	\end{itemize}
}
\item  \\
 be in one's cups \textit{
	\begin{itemize}
	\end{itemize}
}
\end{enumerate}

\section*{immediate}
{\large \color{blue}  }
\subsection*{Explain}
\begin{enumerate}
\item adjective \\
An \textbf{immediate} result, action, or reaction  happens or is done without any delay.
 \textit{
	\begin{itemize}
	\item These tragic incidents have had an immediate effect.
	\item My immediate reaction was just disgust.
	\end{itemize}
}
\item adjective \\
\textbf{Immediate}  needs and concerns exist at the present time and must be dealt with quickly.
 \textit{
	\begin{itemize}
	\item Relief agencies say the immediate problem is not a lack of food, but transportation.
	\end{itemize}
}
\item adjective \\
The \textbf{immediate} person or thing comes just before or just after another person or thing in a sequence .
 \textit{
	\begin{itemize}
	\item In the immediate aftermath of the riots, a mood of hope and reconciliation sprang
up.
	\item His immediate superior, General Geichenko, had singled him out for special mention.
	\end{itemize}
}
\item adjective \\
You use \textbf{immediate} to describe an area or position that is next to or very near a particular place or person.
 \textit{
	\begin{itemize}
	\item Only a handful had returned to work in the immediate vicinity.
	\item I was seated at Sauter's immediate left.
	\end{itemize}
}
\item adjective \\
Your \textbf{immediate} family are the members of your family who are most closely related to you, for example your parents , children, brothers , and sisters .
 \textit{
	\begin{itemize}
	\item The presence of his immediate family is obviously having a calming effect on him.
	\end{itemize}
}
\end{enumerate}

\section*{diameter}
{\large \color{blue}  diameters  }
\subsection*{Explain}
\begin{enumerate}
\item variable noun \\
The \textbf{diameter} of a round object is the length of a straight line that can be drawn across it, passing through the middle of it.
 \textit{
	\begin{itemize}
	\item ...a tube less than a fifth of the diameter of a human hair.
	\item ...a length of 22-mm diameter steel pipe.
	\item ...a tiny capsule, between 1 and 3 millimetres in diameter.
	\end{itemize}
}
\end{enumerate}

\section*{ingenious}
{\large \color{blue}  }
\subsection*{Explain}
\begin{enumerate}
\item adjective \\
Something that is \textbf{ingenious} is very clever and involves new ideas, methods, or equipment .
 \textit{
	\begin{itemize}
	\item ...a truly ingenious invention.
	\item Gautier's solution to the puzzle is ingenious.
	\end{itemize}
}
\end{enumerate}

\section*{distress}
{\large \color{blue}  distresses  distressing  distressed  }
\subsection*{Explain}
\begin{enumerate}
\item uncountable noun \\
\textbf{Distress} is a state of extreme  sorrow , suffering , or pain.
 \textit{
	\begin{itemize}
	\item Jealousy causes distress and painful emotions.
	\item Her mouth grew stiff with pain and distress.
	\end{itemize}
}
\item uncountable noun \\
\textbf{Distress} is the state of being in extreme danger and needing  urgent  help .
 \textit{
	\begin{itemize}
	\item He expressed concern that the ship might be in distress.
	\item The constable received a distress call, and saw two youths attacking his colleague.
	\end{itemize}
}
\item verb \\
If someone or something \textbf{distresses} you, they cause you to be upset or worried .
 \textit{
	\begin{itemize}
	\item The idea of Toni being in danger distresses him enormously.
	\item I did not want to frighten or distress the horse.
	\end{itemize}
}
\end{enumerate}

\section*{instant}
{\large \color{blue}  instants  }
\subsection*{Explain}
\begin{enumerate}
\item countable noun \\
An \textbf{instant} is an extremely  short period of time.
 \textit{
	\begin{itemize}
	\item For an instant, Catherine was tempted to flee.
	\item The pain disappeared in an instant.
	\end{itemize}
}
\item singular noun \\
If you say that something happens  \textbf{at} a particular \textbf{instant} , you mean that it happens at exactly the time you have been referring to, and you are usually suggesting that it happens quickly or immediately .
 \textit{
	\begin{itemize}
	\item At that instant the museum was plunged into total darkness.
	\item In the same instant he flung open the car door.
	\end{itemize}
}
\item  \\
 the instant \textit{
	\begin{itemize}
	\end{itemize}
}
\item adjective \\
You use \textbf{instant} to describe something that happens immediately.
 \textit{
	\begin{itemize}
	\item Mr Porter's book was an instant hit.
	\item He had taken an instant dislike to Mortlake.
	\end{itemize}
}
\item adjective \\
\textbf{Instant} food is food that you can prepare very quickly, for example by just adding water.
 \textit{
	\begin{itemize}
	\item ...instant coffee.
	\end{itemize}
}
\end{enumerate}

\section*{enterprise}
{\large \color{blue}  enterprises  }
\subsection*{Explain}
\begin{enumerate}
\item countable noun \\
An \textbf{enterprise} is a company or business, often a small one.
 \textit{
	\begin{itemize}
	\item There are plenty of small industrial enterprises.
	\item ...the integration of farming enterprises.
	\end{itemize}
}
\item countable noun \\
An \textbf{enterprise} is something new, difficult , or important that you do or try to do.
 \textit{
	\begin{itemize}
	\item ...the first Director of such a novel enterprise.
	\item Horse breeding is indeed a risky enterprise.
	\end{itemize}
}
\item uncountable noun \\
\textbf{Enterprise} is the activity of managing companies and businesses and starting new ones.
 \textit{
	\begin{itemize}
	\item He is still involved in voluntary work promoting local enterprise.
	\item ...a national program of subsidies to private enterprise.
	\end{itemize}
}
\item uncountable noun \\
\textbf{Enterprise} is the ability to think of new and effective things to do, together with an eagerness to do them.
 \textit{
	\begin{itemize}
	\item ...the spirit of enterprise worthy of a free and industrious people.
	\item ...the group's lack of enterprise.
	\end{itemize}
}
\end{enumerate}

\section*{local}
{\large \color{blue}  locals  }
\subsection*{Explain}
\begin{enumerate}
\item adjective \\
\textbf{Local} means existing in or belonging to the area where you live , or to the area that you are talking about.
 The \textbf{locals} are local people.
 \textit{
	\begin{itemize}
	\item We'd better check on the match in the local paper.
	\item Some local residents joined the students' protest.
	\item I was going to pop up to the local library.
	\item That's what the locals call the place.
	\end{itemize}
}
\item adjective \\
\textbf{Local} government is elected by people in one area of a country and controls aspects such as education , housing , and transport within that area.
 \textit{
	\begin{itemize}
	\item Education comprises two-thirds of all local council spending.
	\item ...the controversial system of local taxation known as the poll tax.
	\end{itemize}
}
\item countable noun \\
Your \textbf{local} is a pub which is near where you live and where you often go for a drink.
 \textit{
	\begin{itemize}
	\item The Black Horse is my local.
	\end{itemize}
}
\item adjective \\
A \textbf{local}  anaesthetic or condition affects only a small area of your body.
 \textit{
	\begin{itemize}
	\item An injection of local anaesthetic is usually given first to numb the area.
	\end{itemize}
}
\end{enumerate}

\section*{funeral}
{\large \color{blue}  funerals  }
\subsection*{Explain}
\begin{enumerate}
\item countable noun \\
A \textbf{funeral} is the ceremony that is held when the body of someone who has died is buried or cremated.
 \textit{
	\begin{itemize}
	\item His funeral will be on Thursday at Blackburn Cathedral.
	\item He was given a state funeral.
	\end{itemize}
}
\item  \\
 it's your funeral \textit{
	\begin{itemize}
	\end{itemize}
}
\end{enumerate}

\section*{mad}
{\large \color{blue}  madder  maddest  }
\subsection*{Explain}
\begin{enumerate}
\item adjective \\
Someone who is \textbf{mad} has a mind that does not work in a normal way, with the result that their behaviour is very strange .
 \textit{
	\begin{itemize}
	\item She was afraid of going mad.
	\end{itemize}
}
\item adjective \\
You use \textbf{mad} to describe people or things that you think are very foolish.
 \textit{
	\begin{itemize}
	\item You'd be mad to work with him again.
	\item Isn't that a rather mad idea?
	\end{itemize}
}
\item adjective \\
If you say that someone is \textbf{mad} , you mean that they are very angry.
 \textit{
	\begin{itemize}
	\item You're just mad at me because I don't want to go.
	\item I'm pretty mad about it, I can tell you.
	\end{itemize}
}
\item adjective \\
If you are \textbf{mad about} or \textbf{mad on} something or someone, you like them very much indeed.
 \textbf{Mad} is also a combining form.
 \textit{
	\begin{itemize}
	\item She's not as mad about sport as I am.
	\item He's mad about you.
	\item He's mad on trains.
	\item ...his football-mad son.
	\item He's not power-mad.
	\end{itemize}
}
\item adjective \\
\textbf{Mad} behaviour is wild and uncontrolled .
 \textit{
	\begin{itemize}
	\item You only have an hour to complete the game so it's a mad dash against the clock.
	\item The audience went mad.
	\end{itemize}
}
\item  \\
 drive sb mad \textit{
	\begin{itemize}
	\end{itemize}
}
\item  \\
 like mad \textit{
	\begin{itemize}
	\end{itemize}
}
\end{enumerate}

\section*{gun}
{\large \color{blue}  guns  gunning  gunned  }
\subsection*{Explain}
\begin{enumerate}
\item countable noun \\
A \textbf{gun} is a weapon from which bullets or other things are fired .
 \textit{
	\begin{itemize}
	\item He produced a gun and he came into the house.
	\item The inner-city has guns and crime and drugs and deprivation.
	\item ...gun control laws.
	\end{itemize}
}
\item countable noun \\
A \textbf{gun} or a \textbf{starting gun} is an object  like a gun that is used to make a noise to signal the start of a race .
 \textit{
	\begin{itemize}
	\item The starting gun blasted and they were off.
	\end{itemize}
}
\item verb \\
To \textbf{gun} an engine or a vehicle  means to make it start or go faster by pressing on the accelerator pedal .
 \textit{
	\begin{itemize}
	\item He gunned his engine and drove off.
	\end{itemize}
}
\item  \\
 with guns blazing \textit{
	\begin{itemize}
	\end{itemize}
}
\item  \\
 to jump the gun \textit{
	\begin{itemize}
	\end{itemize}
}
\item  \\
 to stick to your guns \textit{
	\begin{itemize}
	\end{itemize}
}
\end{enumerate}

\section*{main}
{\large \color{blue}  mains  }
\subsection*{Explain}
\begin{enumerate}
\item adjective \\
The \textbf{main} thing is the most important one of several similar things in a particular situation .
 \textit{
	\begin{itemize}
	\item ...one of the main tourist areas of Amsterdam.
	\item My main concern now is to protect the children.
	\item What are the main differences and similarities between them?
	\end{itemize}
}
\item  \\
 in the main \textit{
	\begin{itemize}
	\end{itemize}
}
\item countable noun \\
The \textbf{mains} are the pipes which supply gas or water to buildings, or which take sewage  away from them.
 \textit{
	\begin{itemize}
	\item ...the water supply from the mains.
	\item The capital has been without mains water since Wednesday night.
	\end{itemize}
}
\item plural noun \\
\textbf{The mains} are the wires which supply electricity to buildings, or the place where the wires end inside the building.
 \textit{
	\begin{itemize}
	\item ...amplifiers which plug into the mains.
	\item Make sure plugs are disconnected from the mains.
	\item It is mains or battery powered.
	\end{itemize}
}
\end{enumerate}

\section*{helicopter}
{\large \color{blue}  helicopters  }
\subsection*{Explain}
\begin{enumerate}
\item countable noun \\
A \textbf{helicopter} is an aircraft with long blades on top that go round very fast . It is able to stay  still in the air and to move straight  upwards or downwards .
 \textit{
	\begin{itemize}
	\end{itemize}
}
\end{enumerate}

\section*{major}
{\large \color{blue}  majors  majoring  majored  }
\subsection*{Explain}
\begin{enumerate}
\item adjective \\
You use \textbf{major} when you want to describe something that is more important, serious, or significant than other things in a
group or situation.
 \textit{
	\begin{itemize}
	\item The major factor in the decision to stay or to leave was usually professional.
	\item Studies show that stress can also be a major problem.
	\item Exercise has a major part to play in preventing and combating disease.
	\end{itemize}
}
\item countable noun \\
A \textbf{major} is an officer of middle rank in the British army or the United States army, air force, or marines .
 \textit{
	\begin{itemize}
	\item I was a major in the war, you know.
	\item ...Major Alan Bulman.
	\end{itemize}
}
\item countable noun \\
At a university or college in the United States, a student's \textbf{major} is the main subject that they are studying.
 \textit{
	\begin{itemize}
	\item English majors would be asked to explore the roots of language.
	\end{itemize}
}
\item countable noun \\
At a university or college in the United States, if a student is, for example, a geology  \textbf{major} , geology is the main subject they are studying.
 \textit{
	\begin{itemize}
	\item She was named the outstanding undergraduate history major at the University of Oklahoma.
	\end{itemize}
}
\item verb \\
If a student at a university or college in the United States \textbf{majors}  \textbf{in} a particular subject, that subject is the main one they study.
 \textit{
	\begin{itemize}
	\item He majored in finance at Claremont Men's College in California.
	\end{itemize}
}
\item adjective \\
In music, a \textbf{major} scale is one in which the third note is two tones higher than the first.
 \textit{
	\begin{itemize}
	\item ...Mozart's Symphony No 35 in D Major.
	\end{itemize}
}
\item countable noun \\
A \textbf{major} is a large or important company.
 \textit{
	\begin{itemize}
	\item Oil majors need not fear being unable to sell their crude.
	\end{itemize}
}
\item plural noun \\
\textbf{The majors} are groups of professional sports teams that compete against each other, especially in American baseball .
 \textit{
	\begin{itemize}
	\item I knew what I could do in the minor leagues, I just wanted a chance to prove myself
in the majors.
	\end{itemize}
}
\item countable noun \\
A \textbf{major} is an important sporting competition , especially in golf or tennis .
 \textit{
	\begin{itemize}
	\item Sarazen became the first golfer to win all four majors.
	\end{itemize}
}
\end{enumerate}

\section*{hero}
{\large \color{blue}  heroes  }
\subsection*{Explain}
\begin{enumerate}
\item countable noun \\
The \textbf{hero} of a book, play, film, or story is the main male character, who usually has good qualities.
 \textit{
	\begin{itemize}
	\item The hero of Doctor Zhivago dies in 1929.
	\item ...the author's decision to make his hero a photographer.
	\end{itemize}
}
\item countable noun \\
A \textbf{hero} is someone, especially a man, who has done something brave , new, or good, and who is therefore greatly admired by a lot of people.
 \textit{
	\begin{itemize}
	\item He called Mr Mandela a hero who had inspired millions.
	\item ...the goalscoring hero of the British hockey team.
	\item They think you're some sort of hero.
	\end{itemize}
}
\item countable noun \\
If you describe someone as your \textbf{hero} , you mean that you admire them a great  deal , usually because of a particular quality or skill that they have.
 \textit{
	\begin{itemize}
	\item He was the boyhood hero for every kid of my generation who knew anything about tennis.
	\item No matter, he remained the hero of the crowds.
	\end{itemize}
}
\end{enumerate}

\section*{marginal}
{\large \color{blue}  marginals  }
\subsection*{Explain}
\begin{enumerate}
\item adjective \\
If you describe something as \textbf{marginal} , you mean that it is small or not very important.
 \textit{
	\begin{itemize}
	\item This is a marginal improvement on October.
	\item The role of the opposition party proved marginal.
	\end{itemize}
}
\item adjective \\
If you describe people as \textbf{marginal} , you mean that they are not involved in the main  events or developments in society because they are poor or have no power.
 \textit{
	\begin{itemize}
	\item Andy Warhol made glamorous icons out of the most marginal people.
	\item I don't want to call him marginal, but he's not a major character.
	\end{itemize}
}
\item adjective \\
In political elections, a \textbf{marginal}  seat or constituency is one which is usually won or lost by only a few votes , and is therefore of great  interest to politicians and journalists .
 A \textbf{marginal} is a marginal seat.
 \textit{
	\begin{itemize}
	\item ...the views of voters in five marginal seats.
	\item The votes in the marginals are those that really count.
	\item The coalition won a majority of the vote but failed to win enough of the key marginals.
	\end{itemize}
}
\item adjective \\
\textbf{Marginal}  activities , costs, or taxes are not the main part of a business or an economic system, but often make the difference between its success or failure , and are therefore important to control .
 \textit{
	\begin{itemize}
	\item Consumer electronics has become a marginal business for the group.
	\item For low-paid workers, the marginal tax rate is at least 75%.
	\end{itemize}
}
\item graded adjective \\
\textbf{Marginal} land is not very good for growing  crops or grass for animals.
 \textit{
	\begin{itemize}
	\item ...helping farmers, so they do not have to exploit marginal lands.
	\end{itemize}
}
\end{enumerate}

\section*{inch}
{\large \color{blue}  inches  inching  inched  }
\subsection*{Explain}
\begin{enumerate}
\item countable noun \\
An \textbf{inch} is an imperial unit of length, approximately equal to 2.54 centimetres . There are twelve inches in a foot.
 \textit{
	\begin{itemize}
	\item ...a candy tin 6 inches high and 8 inches in diameter.
	\item ...18 inches below the surface.
	\end{itemize}
}
\item verb \\
To \textbf{inch}  somewhere or to \textbf{inch} something somewhere means to move there very slowly and carefully, or to make something do this.
 \textit{
	\begin{itemize}
	\item ...a climber inching up a vertical wall of rock.
	\item He inched the van forward.
	\item An ambulance inched its way through the crowd.
	\end{itemize}
}
\item  \\
 every inch \textit{
	\begin{itemize}
	\end{itemize}
}
\item  \\
 every inch \textit{
	\begin{itemize}
	\end{itemize}
}
\item  \\
 inch by inch \textit{
	\begin{itemize}
	\end{itemize}
}
\end{enumerate}

\section*{mechanical}
{\large \color{blue}  }
\subsection*{Explain}
\begin{enumerate}
\item adjective \\
A \textbf{mechanical} device has parts that move when it is working , often using power from an engine or from electricity .
 \textit{
	\begin{itemize}
	\item ...a small mechanical device that taps out the numbers.
	\item ...the oldest working mechanical clock in the world.
	\item Most mechanical devices require oil as a lubricant.
	\end{itemize}
}
\item adjective \\
\textbf{Mechanical} means relating to machines and engines and the way they work.
 \textit{
	\begin{itemize}
	\item ...mechanical engineering.
	\item The company undertakes mechanical work on all types of cars.
	\item The train had stopped due to a mechanical problem.
	\end{itemize}
}
\item adjective \\
If you describe a person as \textbf{mechanical} , you mean they are naturally good at understanding how machines work.
 \textit{
	\begin{itemize}
	\item He was a very mechanical person, who knew a lot about sound.
	\item I'm not mechanical like my father; I have to follow the instructions.
	\end{itemize}
}
\item adjective \\
If you describe someone's action as \textbf{mechanical} , you mean that they do it automatically, without thinking about it.
 \textit{
	\begin{itemize}
	\item It is real prayer, and not mechanical repetition.
	\item Her retort was mechanical.
	\end{itemize}
}
\end{enumerate}

\section*{incident}
{\large \color{blue}  incidents  }
\subsection*{Explain}
\begin{enumerate}
\item countable noun \\
An \textbf{incident} is something that happens , often something that is unpleasant .
 \textit{
	\begin{itemize}
	\item These incidents were the latest in a series of disputes between the two nations.
	\item 26 people have been killed in a dramatic shooting incident.
	\item The voting went ahead without incident.
	\end{itemize}
}
\end{enumerate}

\section*{memorial}
{\large \color{blue}  memorials  }
\subsection*{Explain}
\begin{enumerate}
\item countable noun \\
A \textbf{memorial} is a structure built in order to remind people of a famous person or event.
 \textit{
	\begin{itemize}
	\item Building a memorial to Columbus has been his lifelong dream.
	\item Every village had its war memorial.
	\end{itemize}
}
\item adjective \\
A \textbf{memorial} event, object, or prize is in honour of someone who has died , so that they will be remembered .
 \textit{
	\begin{itemize}
	\item A memorial service is being held for her at St Paul's Church.
	\item ...memorial plaques to local regiments.
	\item He went on to win the James E. Sullivan Memorial Trophy as the outstanding amateur
athlete of 1962.
	\end{itemize}
}
\item countable noun \\
If you say that something will be a \textbf{memorial to} someone who has died, you mean that it will continue to exist and remind people of them.
 \textit{
	\begin{itemize}
	\item The museum will serve as a memorial to the millions who passed through Ellis Island.
	\item The public park is an impressive memorial to a good man.
	\end{itemize}
}
\end{enumerate}

\section*{intuition}
{\large \color{blue}  intuitions  }
\subsection*{Explain}
\begin{enumerate}
\item variable noun \\
Your \textbf{intuition} or your \textbf{intuitions} are unexplained  feelings you have that something is true  even when you have no evidence or proof of it.
 \textit{
	\begin{itemize}
	\item Her intuition was telling her that something was wrong.
	\item You can't make a case on your intuitions, Phil.
	\end{itemize}
}
\end{enumerate}

\section*{municipal}
{\large \color{blue}  }
\subsection*{Explain}
\begin{enumerate}
\item adjective \\
\textbf{Municipal} means associated with or belonging to a city or town that has its own local government.
 \textit{
	\begin{itemize}
	\item The municipal authorities gave the go-ahead for the march.
	\item ...next month's municipal elections.
	\item ...the municipal library.
	\end{itemize}
}
\end{enumerate}

\section*{loss}
{\large \color{blue}  losses  }
\subsection*{Explain}
\begin{enumerate}
\item variable noun \\
\textbf{Loss} is the fact of no longer having something or having less of it than before.
 \textit{
	\begin{itemize}
	\item ...loss of sight.
	\item The loss of income for the government is about $250 million a month.
	\item ...hair loss.
	\item The job losses will reduce the total workforce to 7,000.
	\end{itemize}
}
\item variable noun \\
\textbf{Loss} of life occurs when people die .
 \textit{
	\begin{itemize}
	\item ...a terrible loss of human life.
	\item The allies suffered less than 20 casualties while enemy losses were said to be high.
	\end{itemize}
}
\item uncountable noun \\
The \textbf{loss} of a relative or friend is their death.
 \textit{
	\begin{itemize}
	\item They took the time to talk about the loss of Thomas and how their grief was affecting
them.
	\item ...the loss of his mother.
	\end{itemize}
}
\item variable noun \\
If a business makes a \textbf{loss} , it earns less than it spends .
 \textit{
	\begin{itemize}
	\item That year the company made a loss of nine hundred million pounds.
	\item The company stopped producing fertilizer because of continued losses.
	\item ...profit and loss.
	\end{itemize}
}
\item uncountable noun \\
\textbf{Loss} is the feeling of sadness you experience when someone or something you like is taken away from you.
 \textit{
	\begin{itemize}
	\item Talk to others about your feelings of loss and grief.
	\item He always woke with a sense of deep sorrow and depressing loss.
	\end{itemize}
}
\item countable noun \\
A \textbf{loss} is the disadvantage you suffer when a valuable and useful person or thing leaves or is taken away.
 \textit{
	\begin{itemize}
	\item She said his death was a great loss to herself.
	\end{itemize}
}
\item uncountable noun \\
The \textbf{loss} of something such as heat, blood, or fluid is the gradual  reduction of it or of its level in a system or in someone's body.
 \textit{
	\begin{itemize}
	\item ...blood loss.
	\item ...weight loss.
	\item ...a rapid loss of heat from the body.
	\end{itemize}
}
\item  \\
 at a loss \textit{
	\begin{itemize}
	\end{itemize}
}
\item  \\
 be at a loss \textit{
	\begin{itemize}
	\end{itemize}
}
\item  \\
 cut your losses \textit{
	\begin{itemize}
	\end{itemize}
}
\item  \\
 a dead loss \textit{
	\begin{itemize}
	\end{itemize}
}
\end{enumerate}

\section*{national}
{\large \color{blue}  nationals  }
\subsection*{Explain}
\begin{enumerate}
\item adjective \\
\textbf{National} means relating to the whole of a country or nation rather than to part of it or to other nations.
 \textit{
	\begin{itemize}
	\item Ruling parties have lost ground in national and local elections.
	\item ...major national and international issues.
	\end{itemize}
}
\item adjective \\
\textbf{National} means typical of the people or customs of a particular country or nation.
 \textit{
	\begin{itemize}
	\item ...the national characteristics and history of the country.
	\item Baseball is the national pastime.
	\end{itemize}
}
\item countable noun \\
You can refer to someone who is legally a citizen of a country as a \textbf{national} of that country.
 \textit{
	\begin{itemize}
	\item ...a Sri-Lankan-born British national.
	\end{itemize}
}
\end{enumerate}

\section*{masterpiece}
{\large \color{blue}  masterpieces  }
\subsection*{Explain}
\begin{enumerate}
\item countable noun \\
A \textbf{masterpiece} is an extremely good painting , novel , film, or other work of art.
 \textit{
	\begin{itemize}
	\item His book, I must add, is a masterpiece.
	\item ...masterpieces by artists like Rembrandt, Raphael and Ingres.
	\end{itemize}
}
\item countable noun \\
An artist's, writer's, or composer's \textbf{masterpiece} is the best work that they have ever produced.
 \textit{
	\begin{itemize}
	\item 'Man's Fate,' translated into sixteen languages, is probably his masterpiece.
	\end{itemize}
}
\item countable noun \\
A \textbf{masterpiece} is an extremely clever or skilful  example of something.
 \textit{
	\begin{itemize}
	\item The whole thing was a masterpiece of crowd management.
	\end{itemize}
}
\end{enumerate}

\section*{necessary}
{\large \color{blue}  necessaries  }
\subsection*{Explain}
\begin{enumerate}
\item adjective \\
Something that is \textbf{necessary} is needed in order for something else to happen .
 \textit{
	\begin{itemize}
	\item I kept the engine running because it might be necessary to leave fast.
	\item We will do whatever is necessary to stop them.
	\item Is that really necessary?
	\item Make the necessary arrangements.
	\end{itemize}
}
\item adjective \\
A \textbf{necessary}  consequence or connection  must happen or exist , because of the nature of the things or events involved.
 \textit{
	\begin{itemize}
	\item Wastage was no doubt a necessary consequence of war.
	\item Scientific work has a necessary connection with the idea of progress.
	\end{itemize}
}
\item plural noun \\
\textbf{Necessaries} are things, such as food or clothing, that you need to have in order to live .
 \textit{
	\begin{itemize}
	\item ...a small parcel of necessaries tied up in a handkerchief and carried on a stick.
	\end{itemize}
}
\item  \\
 if necessary/when necessary/where necessary \textit{
	\begin{itemize}
	\end{itemize}
}
\end{enumerate}

\section*{meanwhile}
{\large \color{blue}  }
\subsection*{Explain}
\begin{enumerate}
\item adverb \\
\textbf{Meanwhile} means while a particular thing is happening .
 \textit{
	\begin{itemize}
	\item Bake the aubergines till soft. Meanwhile, heat the oil in a heavy pan.
	\item Kate turned to beckon Peter across from the car, but Bill waved him back, meanwhile
pushing Kate inside.
	\end{itemize}
}
\item adverb \\
\textbf{Meanwhile} means in the period of time between two events.
 \textit{
	\begin{itemize}
	\item You needn't worry; I'll be ready to greet them. Meanwhile I'm off to discuss the
Fowler's party with Felix.
	\end{itemize}
}
\item adverb \\
You use \textbf{meanwhile} to introduce a different aspect of a particular situation , especially one that is completely opposite to the one previously mentioned .
 \textit{
	\begin{itemize}
	\item Almost four million households are in debt to their energy company. Meanwhile, suppliers
profits have doubled.
	\end{itemize}
}
\end{enumerate}

\section*{opposite}
{\large \color{blue}  opposites  }
\subsection*{Explain}
\begin{enumerate}
\item preposition \\
If one thing is \textbf{opposite} another, it is on the other side of a space from it.
 \textbf{Opposite} is also an adverb .
 \textit{
	\begin{itemize}
	\item Jennie had sat opposite her at breakfast.
	\item He looked up at the buildings opposite, but could see no open window.
	\item He sits in one chair, I sit opposite.
	\end{itemize}
}
\item adjective \\
The \textbf{opposite} side or part of something is the side or part that is furthest  away from you.
 \textit{
	\begin{itemize}
	\item ...the opposite corner of the room.
	\end{itemize}
}
\item adjective \\
\textbf{Opposite} is used to describe things of the same kind which are completely different in a particular  way . For example , north and south are opposite directions, and winning and losing are opposite results in a game .
 \textit{
	\begin{itemize}
	\item All the cars driving in the opposite direction had their headlights on.
	\item I should have written the notes in the opposite order.
	\item In fact everything he does is opposite to what is considered normal behaviour.
	\end{itemize}
}
\item countable noun \\
\textbf{The}  \textbf{opposite}  \textbf{of} someone or something is the person or thing that is most different from them.
 \textit{
	\begin{itemize}
	\item Ritter was a very complex man but Marius was the opposite, a simple farmer.
	\item Well, whatever he says, you can bet he's thinking the opposite.
	\end{itemize}
}
\end{enumerate}

\section*{mile}
{\large \color{blue}  miles  }
\subsection*{Explain}
\begin{enumerate}
\item countable noun \\
A \textbf{mile} is a unit of distance equal to 1760 yards or approximately 1.6 kilometres.
 \textit{
	\begin{itemize}
	\item They drove 600 miles across the desert.
	\item The hurricane is moving to the west at about 18 miles per hour.
	\item She lives just half a mile away.
	\item There's a lake up there, about ten miles long.
	\item ...a 50-mile bike ride.
	\end{itemize}
}
\item plural noun \\
\textbf{Miles} is used, especially in the expression  \textbf{miles away} , to refer to a long distance.
 \textit{
	\begin{itemize}
	\item If you enrol at a gym that's miles away, you won't be visiting it as often as you
should.
	\item I was miles and miles from anywhere.
	\item 'Shall I come to see you?'—'Are you kidding? It's miles.'
	\end{itemize}
}
\item countable noun \\
\textbf{Miles} or \textbf{a mile} is used with the meaning 'very much' in order to emphasize the difference between two things or qualities, or the difference between what you aimed to do and what you actually  achieved .
 \textit{
	\begin{itemize}
	\item You're miles better than most of the performers we see nowadays.
	\item With a Labour candidate in place they won by a mile.
	\item The rehearsals were miles too slow and no work was getting done.
	\end{itemize}
}
\item  \\
 miles away \textit{
	\begin{itemize}
	\end{itemize}
}
\item  \\
 to go the extra mile \textit{
	\begin{itemize}
	\end{itemize}
}
\item  \\
 a mile off \textit{
	\begin{itemize}
	\end{itemize}
}
\item  \\
 run a mile \textit{
	\begin{itemize}
	\end{itemize}
}
\item  \\
 to stick out a mile \textit{
	\begin{itemize}
	\end{itemize}
}
\end{enumerate}

\section*{past}
{\large \color{blue}  pasts  }
\subsection*{Explain}
\begin{enumerate}
\item singular noun \\
\textbf{The past} is the time before the present, and the things that have happened .
 \textit{
	\begin{itemize}
	\item In the past, about a third of the babies born to women with diabetes were lost.
	\item He should learn from the mistakes of the past. We have been here before.
	\item We would like to put the past behind us.
	\end{itemize}
}
\item countable noun \\
Your \textbf{past} consists of all the things that you have done or that have happened to you.
 \textit{
	\begin{itemize}
	\item ...revelations about his past.
	\item ...Germany's recent past.
	\end{itemize}
}
\item adjective \\
\textbf{Past} events and things happened or existed before the present time.
 \textbf{Past} is also used after periods of time.
 \textit{
	\begin{itemize}
	\item I knew from past experience that alternative therapies could help.
	\item ...a return to the turbulence of past centuries.
	\item The list of past champions includes many British internationals.
	\item A South Korean newspaper said today the event will be smaller than in years past.
	\end{itemize}
}
\item adjective \\
You use \textbf{past} to talk about a period of time that has just finished. For example , if you talk about the \textbf{past five years} , you mean the period of five years that has just finished.
 \textit{
	\begin{itemize}
	\item Most shops have remained closed for the past three days.
	\item ...the momentous events of the past few days.
	\end{itemize}
}
\item adjective \\
If a situation is \textbf{past} , it has ended and no longer exists.
 \textit{
	\begin{itemize}
	\item Many economists believe the worst of the economic downturn is past.
	\item ...images from years long past.
	\item The time for loyalty is past.
	\end{itemize}
}
\item adjective \\
In grammar , the \textbf{past tenses} of a verb are the ones used to talk about things that happened at some time before
the present. The simple past tense uses the past form of a verb, which for regular verbs ends in '-ed', as in 'They walked back to the car '.
 \textit{
	\begin{itemize}
	\end{itemize}
}
\item preposition \\
You use \textbf{past} when you are stating a time which is thirty  minutes or less after a particular hour . For example, if it is \textbf{twenty past}  six , it is twenty minutes after six o'clock.
 \textbf{Past} is also an adverb .
 \textit{
	\begin{itemize}
	\item It's ten past eleven.
	\item I arrived at half past ten.
	\item I have my lunch at half past.
	\end{itemize}
}
\item preposition \\
If it is \textbf{past} a particular time, it is later than that time.
 \textit{
	\begin{itemize}
	\item It was past midnight.
	\item It's past your bedtime.
	\end{itemize}
}
\item preposition \\
If you go  \textbf{past} someone or something, you go near them and keep moving, so that they are then behind
you.
 \textbf{Past} is also an adverb.
 \textit{
	\begin{itemize}
	\item I dashed past him and out of the door.
	\item A steady procession of people filed past the coffin.
	\item He was never able to get past the border guards.
	\item An ambulance drove past.
	\end{itemize}
}
\item preposition \\
If you look or point \textbf{past} a person or thing, you look or point at something behind them.
 \textit{
	\begin{itemize}
	\item She stared past Christine at the bed.
	\end{itemize}
}
\item preposition \\
If something is \textbf{past} a place, it is on the other side of it.
 \textit{
	\begin{itemize}
	\item Go north on I-15 to the exit just past Barstow.
	\item Just past the Barlby roundabout there's temporary traffic lights.
	\end{itemize}
}
\item preposition \\
If someone or something is \textbf{past} a particular point or stage , they are no longer at that point or stage.
 \textit{
	\begin{itemize}
	\item He was well past retirement age.
	\item ...a piece of cheese four weeks past its sell-by date.
	\item The situation is long past the stage when anyone's advice would help.
	\end{itemize}
}
\item preposition \\
If you are \textbf{past} doing something, you are no longer able to do it. For example, if you are \textbf{past caring} , you do not care about something any more because so many bad things have happened to you.
 \textit{
	\begin{itemize}
	\item She was past caring about anything by then and just wanted the pain to end.
	\item Often by the time they do accept the truth they are past being able to put words
to feelings.
	\end{itemize}
}
\item  \\
 would not put it past sb/would not put anything past sb \textit{
	\begin{itemize}
	\end{itemize}
}
\end{enumerate}

\section*{moss}
{\large \color{blue}  mosses  }
\subsection*{Explain}
\begin{enumerate}
\item variable noun \\
\textbf{Moss} is a very small soft  green plant which grows on damp  soil , or on wood or stone .
 \textit{
	\begin{itemize}
	\item ...ground covered over with moss.
	\end{itemize}
}
\end{enumerate}

\section*{permanent}
{\large \color{blue}  permanents  }
\subsection*{Explain}
\begin{enumerate}
\item adjective \\
Something that is \textbf{permanent}  lasts for ever .
 \textit{
	\begin{itemize}
	\item Heavy drinking can cause permanent damage to the brain.
	\item ...a permanent solution to the problem.
	\item The ban is intended to be permanent.
	\end{itemize}
}
\item adjective \\
You use \textbf{permanent} to describe  situations or states that keep occurring or which seem to exist all the time; used especially to describe problems or difficulties .
 \textit{
	\begin{itemize}
	\item ...a permanent state of tension.
	\item They feel under permanent threat.
	\item There was a permanent 20-yard queue for the portable toilets.
	\end{itemize}
}
\item adjective \\
A \textbf{permanent}  employee is one who is employed for an unlimited length of time.
 \textit{
	\begin{itemize}
	\item At the end of the probationary period you will become a permanent employee.
	\item ...a permanent job.
	\end{itemize}
}
\item adjective \\
Your \textbf{permanent}  home or your \textbf{permanent}  address is the one at which you spend most of your time or the one that you return to after having stayed in other places.
 \textit{
	\begin{itemize}
	\item York Cottage was as near to a permanent home as the children knew.
	\item They had no permanent address.
	\end{itemize}
}
\item countable noun \\
A \textbf{permanent} is a treatment where a hair stylist  curls your hair and treats it with a chemical so that it stays curly for several months .
 \textit{
	\begin{itemize}
	\end{itemize}
}
\end{enumerate}

\section*{picture}
{\large \color{blue}  pictures  picturing  pictured  }
\subsection*{Explain}
\begin{enumerate}
\item countable noun \\
A \textbf{picture} consists of lines and shapes which are drawn , painted , or printed on a surface and show a person, thing, or scene.
 \textit{
	\begin{itemize}
	\item A picture of Rory O'Moore hangs in the dining room at Kildangan.
	\item ...drawing a small picture with coloured chalks.
	\end{itemize}
}
\item countable noun \\
A \textbf{picture} is a photograph.
 \textit{
	\begin{itemize}
	\item The tourists have nothing to do but take pictures of each other.
	\item The Observer carries a big front-page picture of rioters in a litter-strewn street.
	\end{itemize}
}
\item countable noun \\
Television \textbf{pictures} are the scenes which you see on a television screen.
 \textit{
	\begin{itemize}
	\item ...heartrending television pictures of human suffering.
	\end{itemize}
}
\item verb \\
To \textbf{be pictured}  somewhere , for example in a newspaper or magazine , means to appear in a photograph or picture.
 \textit{
	\begin{itemize}
	\item The golfer is pictured on many of the front pages, kissing his trophy as he holds
it aloft.
	\item ...a woman who claimed she had been pictured dancing with a celebrity in Stringfellows
nightclub.
	\item The rattan and wrought-iron chair pictured here costs £125.
	\end{itemize}
}
\item countable noun \\
You can refer to a film as a \textbf{picture} .
 \textit{
	\begin{itemize}
	\item Warner Communications Inc. has refused to distribute the picture in the United States.
	\item ...a director of epic action pictures.
	\end{itemize}
}
\item plural noun \\
If you go to \textbf{the pictures} , you go to a cinema to see a film.
 \textit{
	\begin{itemize}
	\item We're going to the pictures tonight.
	\item I'd rather see it at the pictures than on TV anyway.
	\end{itemize}
}
\item countable noun \\
If you have a \textbf{picture} of something in your mind , you have a clear  idea or memory of it in your mind as if you were actually seeing it.
 \textit{
	\begin{itemize}
	\item They have in their mind a picture of what a police officer should look like.
	\item We are just trying to get our picture of the whole afternoon straight.
	\item I tried to put the picture from my mind.
	\end{itemize}
}
\item verb \\
If you \textbf{picture} something in your mind, you think of it and have such a clear memory or idea of it that you seem to be able to see it.
 \textit{
	\begin{itemize}
	\item He pictured her with long black braided hair.
	\item I never would have pictured this as her home.
	\item He pictured Claire sitting out in the car, waiting for him.
	\item She pictured herself working with animals.
	\item I tried to picture the place, but could not.
	\end{itemize}
}
\item countable noun \\
A \textbf{picture} of something is a description of it or an indication of what it is like.
 \textit{
	\begin{itemize}
	\item I'll try and give you a better picture of what the boys do.
	\item Her book paints a bleak picture of the problems women now face.
	\item From the files that have now been released, a truer picture emerges.
	\end{itemize}
}
\item singular noun \\
When you refer to the \textbf{picture} in a particular place, you are referring to the situation there.
 \textit{
	\begin{itemize}
	\item But as with other charitable bodies, these figures mask the true picture.
	\item It's a similar picture across the border in Ethiopia.
	\end{itemize}
}
\item  \\
 get the picture \textit{
	\begin{itemize}
	\end{itemize}
}
\item  \\
 in the picture/out of the picture \textit{
	\begin{itemize}
	\end{itemize}
}
\item  \\
 a picture of sth/the picture of sth \textit{
	\begin{itemize}
	\end{itemize}
}
\item  \\
 put sb in the picture \textit{
	\begin{itemize}
	\end{itemize}
}
\end{enumerate}

\section*{perpetual}
{\large \color{blue}  }
\subsection*{Explain}
\begin{enumerate}
\item adjective \\
A \textbf{perpetual}  feeling , state, or quality is one that never ends or changes.
 \textit{
	\begin{itemize}
	\item ...the creation of a perpetual union.
	\end{itemize}
}
\item adjective \\
A \textbf{perpetual} act, situation , or state is one that happens again and again and so seems never to end.
 \textit{
	\begin{itemize}
	\item I thought her perpetual complaints were going to prove too much for me.
	\end{itemize}
}
\end{enumerate}

\section*{pillow}
{\large \color{blue}  pillows  }
\subsection*{Explain}
\begin{enumerate}
\item countable noun \\
A \textbf{pillow} is a rectangular cushion which you rest your head on when you are in bed .
 \textit{
	\begin{itemize}
	\end{itemize}
}
\end{enumerate}

\section*{present}
{\large \color{blue}  }
\subsection*{Explain}
\begin{enumerate}
\item adjective \\
You use \textbf{present} to describe things and people that exist now, rather than those that existed in the
past or those that may exist in the future.
 \textit{
	\begin{itemize}
	\item He has brought much of the present crisis on himself.
	\item ...the government's present economic difficulties.
	\item It has been skilfully renovated by the present owners.
	\item No statement can be made at the present time.
	\end{itemize}
}
\item singular noun \\
\textbf{The present} is the period of time that we are in now and the things that are happening now.
 \textit{
	\begin{itemize}
	\item ...his struggle to reconcile the past with the present.
	\item ...continuing right up to the present.
	\item Then her thoughts would switch to the present.
	\end{itemize}
}
\item adjective \\
In grammar , the \textbf{present} tenses of a verb are the ones that are used to talk about things that happen regularly or situations that exist at this time. The simple present tense uses the
base form or the 's' form of a verb, as in 'I play tennis  twice a week ' and 'He works in a bank'.
 \textit{
	\begin{itemize}
	\end{itemize}
}
\item  \\
 at present \textit{
	\begin{itemize}
	\end{itemize}
}
\item  \\
 the present day \textit{
	\begin{itemize}
	\end{itemize}
}
\item  \\
 for the present \textit{
	\begin{itemize}
	\end{itemize}
}
\item  \\
 there's no time like the present \textit{
	\begin{itemize}
	\end{itemize}
}
\end{enumerate}

\section*{pine}
{\large \color{blue}  pines  pining  pined  }
\subsection*{Explain}
\begin{enumerate}
\item variable noun \\
A \textbf{pine tree} or a \textbf{pine} is a tall tree which has very thin, sharp leaves and a fresh  smell . Pine trees have leaves all year round.
 \textbf{Pine} is the wood of this tree.
 \textit{
	\begin{itemize}
	\item ...high mountains covered in pine trees.
	\item ...a big pine table.
	\end{itemize}
}
\item verb \\
If you \textbf{pine for} someone who has died or gone away, you want them to be with you very much and feel sad because they are not there.
 \textit{
	\begin{itemize}
	\item When the family moved away, Polly pined for them.
	\item Make sure your pet won't pine while you're away.
	\end{itemize}
}
\item verb \\
If you \textbf{pine for} something, you want it very much, especially when it is unlikely that you will be able to have it.
 \textit{
	\begin{itemize}
	\item I pine for the countryside.
	\item ...the democracy they have pined for since 1939.
	\end{itemize}
}
\end{enumerate}

\section*{radiant}
{\large \color{blue}  }
\subsection*{Explain}
\begin{enumerate}
\item adjective \\
Someone who is \textbf{radiant} is so happy that their happiness shows in their face .
 \textit{
	\begin{itemize}
	\item Kathy smiled at her daughter's radiant face.
	\item On her wedding day the bride looked truly radiant.
	\end{itemize}
}
\item adjective \\
Something that is \textbf{radiant}  glows brightly.
 \textit{
	\begin{itemize}
	\item The evening sun warms the old red brick wall to a radiant glow.
	\item Out on the bay the morning is radiant.
	\end{itemize}
}
\item adjective \\
\textbf{Radiant} heat or energy is sent out in the form of rays.
 \textit{
	\begin{itemize}
	\item The earth would be a frozen ball if it were not for the radiant heat of the sun.
	\end{itemize}
}
\end{enumerate}

\section*{range}
{\large \color{blue}  ranges  ranging  ranged  }
\subsection*{Explain}
\begin{enumerate}
\item countable noun \\
A \textbf{range}  \textbf{of} things is a number of different things of the same general kind.
 \textit{
	\begin{itemize}
	\item A wide range of colours and patterns are available.
	\item The two men discussed a range of issues.
	\item The range includes chests of drawers, tables and wardrobes.
	\end{itemize}
}
\item countable noun \\
A \textbf{range} is the complete group that is included between two points on a scale of measurement
or quality.
 \textit{
	\begin{itemize}
	\item The average age range is between 35 and 55.
	\item ...properties available in the price range they are looking for.
	\item ...top-of-the-range products for which people are prepared to pay a little bit more.
	\end{itemize}
}
\item countable noun \\
The \textbf{range}  \textbf{of} something is the maximum area in which it can reach things or detect things.
 \textit{
	\begin{itemize}
	\item The 120mm mortar has a range of 18,000 yards.
	\item The trees on the mountains within my range of vision had all been felled.
	\item Tactical nuclear weapons have shorter ranges.
	\end{itemize}
}
\item verb \\
If things \textbf{range}  \textbf{between} two points or \textbf{range}  \textbf{from} one point \textbf{to} another, they vary within these points on a scale of measurement or quality.
 \textit{
	\begin{itemize}
	\item They range in price from $3 to $15.
	\item ...offering merchandise ranging from the everyday to the esoteric.
	\item ...temperatures ranging between 5°C and 20°C.
	\end{itemize}
}
\item verb \\
If a piece of writing or speech \textbf{ranges over} a group of topics , it includes all those topics.
 \textit{
	\begin{itemize}
	\item The discussion in this chapter has ranged over a number of matters.
	\end{itemize}
}
\item verb \\
If people or things \textbf{are ranged}  somewhere , they are arranged in a row or in lines.
 \textit{
	\begin{itemize}
	\item Some 300 trees have been ranged along the perimeter hedge.
	\item More than 1,500 police and troops are ranged against them.
	\end{itemize}
}
\item verb \\
If animals or people \textbf{range} somewhere, they move around in a place without having a particular destination in mind.
 \textit{
	\begin{itemize}
	\item Feeding is not a problem because the birds range over such a large area.
	\item They range widely in search of carrion.
	\end{itemize}
}
\item countable noun \\
A \textbf{range} of mountains or hills is a line of them.
 \textit{
	\begin{itemize}
	\item ...the massive mountain ranges to the north.
	\item ...an impressive range of hills topped with trees.
	\end{itemize}
}
\item countable noun \\
A \textbf{range} is a large area of open land, especially land in the United States where cattle are kept.
 \textit{
	\begin{itemize}
	\end{itemize}
}
\item countable noun \\
A rifle  \textbf{range} or a shooting \textbf{range} is a place where people can practise shooting at targets.
 \textit{
	\begin{itemize}
	\item It reminds me of my days on the rifle range preparing for duty in Vietnam.
	\item ...an Army firing range.
	\end{itemize}
}
\item countable noun \\
A \textbf{range} or \textbf{kitchen range} is an old-fashioned metal cooker .
 \textit{
	\begin{itemize}
	\end{itemize}
}
\item countable noun \\
A \textbf{range} or \textbf{kitchen range} is a large metal device for cooking food using gas or electricity . A range consists of a grill , an oven, and some gas or electric rings.
 \textit{
	\begin{itemize}
	\end{itemize}
}
\item  \\
 within range, out of range \textit{
	\begin{itemize}
	\end{itemize}
}
\item  \\
 at close range \textit{
	\begin{itemize}
	\end{itemize}
}
\end{enumerate}

\section*{republican}
{\large \color{blue}  republicans  }
\subsection*{Explain}
\begin{enumerate}
\item adjective \\
\textbf{Republican}  means relating to a republic. In \textbf{republican} systems of government , power is held by the people or the representatives that they elect .
 \textit{
	\begin{itemize}
	\item ...the nations that had adopted the republican form of government.
	\end{itemize}
}
\item adjective \\
In the United  States , if someone is \textbf{Republican} , they belong to or support the Republican Party.
 A \textbf{Republican} is someone who supports or belongs to the Republican Party.
 \textit{
	\begin{itemize}
	\item ...Republican voters.
	\item Some families have been Republican for generations.
	\item What made you decide to become a Republican?
	\end{itemize}
}
\item adjective \\
In Northern  Ireland , if someone is \textbf{Republican} , they believe that Northern Ireland should not be ruled by Britain but should become part of the Republic of Ireland.
 A \textbf{Republican} is someone who has Republican views .
 \textit{
	\begin{itemize}
	\item ...a Republican paramilitary group.
	\item ...a Northern Ireland Republican.
	\end{itemize}
}
\end{enumerate}

\section*{replacement}
{\large \color{blue}  replacements  }
\subsection*{Explain}
\begin{enumerate}
\item uncountable noun \\
If you refer to the \textbf{replacement} of one thing by another, you mean that the second thing takes the place of the first.
 \textit{
	\begin{itemize}
	\item ...the replacement of damaged or lost books.
	\end{itemize}
}
\item countable noun \\
Someone who takes someone else's place in an organization , government, or team can be referred to as their \textbf{replacement} .
 \textit{
	\begin{itemize}
	\item Taylor has nominated Adams as his replacement.
	\item The 18-year-old made his debut last week as a replacement for the injured striker.
	\end{itemize}
}
\end{enumerate}

\section*{roundabout}
{\large \color{blue}  roundabouts  }
\subsection*{Explain}
\begin{enumerate}
\item countable noun \\
A \textbf{roundabout} is a circular structure in the road at a place where several roads meet . You drive  round it until you come to the road that you want .
 \textit{
	\begin{itemize}
	\end{itemize}
}
\item countable noun \\
A \textbf{roundabout} at a fair is a large, circular mechanical device with seats, often in the shape of animals or cars , on which children  sit and go round and round.
 \textit{
	\begin{itemize}
	\end{itemize}
}
\item countable noun \\
A \textbf{roundabout} in a park or school play area is a circular platform that children sit or stand on. People push the platform to make it spin round.
 \textit{
	\begin{itemize}
	\end{itemize}
}
\item adjective \\
If you go somewhere by a \textbf{roundabout}  route , you do not go there by the shortest and quickest route.
 \textit{
	\begin{itemize}
	\item The party took a roundabout route overland.
	\end{itemize}
}
\item adjective \\
If you do or say something in a \textbf{roundabout}  way , you do not do or say it in a simple , clear , and direct way.
 \textit{
	\begin{itemize}
	\item We made a bit of a fuss in a roundabout way.
	\item ...using indirect or roundabout language in place of a precise noun.
	\end{itemize}
}
\end{enumerate}

\section*{squirrel}
{\large \color{blue}  squirrels  squirrelling  squirrelled  }
\subsection*{Explain}
\begin{enumerate}
\item countable noun \\
A \textbf{squirrel} is a small animal with a long furry tail. Squirrels live  mainly in trees.
 \textit{
	\begin{itemize}
	\end{itemize}
}
\end{enumerate}

\section*{shrewd}
{\large \color{blue}  shrewder  shrewdest  }
\subsection*{Explain}
\begin{enumerate}
\item adjective \\
A \textbf{shrewd} person is able to understand and judge a situation quickly and to use this understanding to their own advantage .
 \textit{
	\begin{itemize}
	\item She's a shrewd businesswoman.
	\item His grey eyes were shrewd but kindly.
	\item It should prove a shrewd investment.
	\end{itemize}
}
\end{enumerate}

\section*{stalk}
{\large \color{blue}  stalks  stalking  stalked  }
\subsection*{Explain}
\begin{enumerate}
\item countable noun \\
The \textbf{stalk} of a flower, leaf , or fruit is the thin part that joins it to the plant or tree.
 \textit{
	\begin{itemize}
	\item A single pale blue flower grows up from each joint on a long stalk.
	\item ...corn stalks.
	\end{itemize}
}
\item verb \\
If you \textbf{stalk} a person or a wild animal, you follow them quietly in order to kill them, catch them, or observe them carefully.
 \textit{
	\begin{itemize}
	\item The hunter stalked the stag for days.
	\end{itemize}
}
\item verb \\
If someone \textbf{stalks} someone else, especially a famous person or a person they used to have a relationship with, they keep following them or contacting them in an annoying and frightening way.
 \textit{
	\begin{itemize}
	\item Even after their divorce he continued to stalk and threaten her.
	\end{itemize}
}
\item verb \\
If you \textbf{stalk}  somewhere , you walk there in a stiff, proud , or angry way.
 \textit{
	\begin{itemize}
	\item If his patience is tried at meetings he has been known to stalk out.
	\end{itemize}
}
\item verb \\
If you say that something bad such as death , fear , or evil  \textbf{stalks} a place, you mean it is there.
 \textit{
	\begin{itemize}
	\item ...tales of famine stalking the streets of the city.
	\end{itemize}
}
\end{enumerate}

\section*{spicy}
{\large \color{blue}  spicier  spiciest  }
\subsection*{Explain}
\begin{enumerate}
\item adjective \\
\textbf{Spicy} food is strongly flavoured with spices.
 \textit{
	\begin{itemize}
	\item Thai food is hot and spicy.
	\item ...a spicy tomato and coriander sauce.
	\end{itemize}
}
\end{enumerate}

\section*{studio}
{\large \color{blue}  studios  }
\subsection*{Explain}
\begin{enumerate}
\item countable noun \\
A \textbf{studio} is a room where a painter , photographer, or designer works.
 \textit{
	\begin{itemize}
	\item She was in her studio again, painting onto a large canvas.
	\end{itemize}
}
\item countable noun \\
A \textbf{studio} is a room where radio or television programmes are recorded, CDs are produced, or
films are made.
 \textit{
	\begin{itemize}
	\item She's much happier performing live than in a recording studio.
	\end{itemize}
}
\item countable noun \\
You can  also  refer to film-making or recording companies as \textbf{studios} .
 \textit{
	\begin{itemize}
	\item She wrote to Paramount Studios and asked if they would audition her.
	\end{itemize}
}
\item countable noun \\
A \textbf{studio} is a small flat with one room for living and sleeping in, a kitchen , and a bathroom . You can also talk about a \textbf{studio flat} in British  English or a \textbf{studio apartment} in American English.
 \textit{
	\begin{itemize}
	\item I live on my own in a studio flat.
	\end{itemize}
}
\end{enumerate}

\section*{sudden}
{\large \color{blue}  }
\subsection*{Explain}
\begin{enumerate}
\item adjective \\
\textbf{Sudden} means happening quickly and unexpectedly.
 \textit{
	\begin{itemize}
	\item He had been deeply affected by the sudden death of his father-in-law.
	\item 'I hope,' the stranger said, 'that the sudden change of venue did not inconvenience
you.'.
	\item She started to thank him, but a sudden movement behind him caught her attention.
	\item It was all very sudden.
	\end{itemize}
}
\item  \\
 all of a sudden \textit{
	\begin{itemize}
	\end{itemize}
}
\end{enumerate}

\section*{survival}
{\large \color{blue}  }
\subsection*{Explain}
\begin{enumerate}
\item uncountable noun \\
If you refer to the \textbf{survival} of something or someone, you mean that they manage to continue or exist in spite of difficult  circumstances .
 \textit{
	\begin{itemize}
	\item ...companies which have been struggling for survival in the advancing recession.
	\item Ask for the free booklet 'Debt: a Survival Guide'.
	\end{itemize}
}
\item uncountable noun \\
If you refer to the \textbf{survival} of a person or living thing, you mean that they live through a dangerous  situation in which it was possible that they might  die .
 \textit{
	\begin{itemize}
	\item If cancers are spotted early there's a high chance of survival.
	\item An animal's sense of smell is still crucial to its survival.
	\end{itemize}
}
\item  \\
 survival of the fittest \textit{
	\begin{itemize}
	\end{itemize}
}
\end{enumerate}

\section*{thumb}
{\large \color{blue}  thumbs  thumbing  thumbed  }
\subsection*{Explain}
\begin{enumerate}
\item countable noun \\
Your \textbf{thumb} is the short thick part on the side of your hand next to your four  fingers .
 \textit{
	\begin{itemize}
	\item She bit the tip of her left thumb, not looking at me.
	\end{itemize}
}
\item countable noun \\
The \textbf{thumb} of a glove is the part which a person's thumb fits into.
 \textit{
	\begin{itemize}
	\end{itemize}
}
\item verb \\
If you \textbf{thumb} a lift or \textbf{thumb} a ride, you stand by the side of the road  holding out your thumb until a driver  stops and gives you a lift.
 \textit{
	\begin{itemize}
	\item It may interest you to know that a boy answering Rory's description thumbed a ride
to Howth.
	\item Thumbing a lift had once a carefree, easy-going image.
	\end{itemize}
}
\item  \\
 to stick out like a sore thumb \textit{
	\begin{itemize}
	\end{itemize}
}
\item  \\
 to twiddle your thumbs \textit{
	\begin{itemize}
	\end{itemize}
}
\item  \\
 under sb's thumb \textit{
	\begin{itemize}
	\end{itemize}
}
\end{enumerate}

\section*{western}
{\large \color{blue}  westerns  }
\subsection*{Explain}
\begin{enumerate}
\item adjective \\
\textbf{Western} means in or from the west of a region, state, or country.
 \textit{
	\begin{itemize}
	\item ...hand-made rugs from Western and Central Asia.
	\item ...Moi University, in western Kenya.
	\end{itemize}
}
\item adjective \\
\textbf{Western} is used to describe things, people, ideas , or ways of life that come from or are associated with the United States, Canada , and the countries of Western, Northern , and Southern Europe.
 \textit{
	\begin{itemize}
	\item Mexico had the support of the big western governments.
	\item Those statements have never been reported in the Western media.
	\end{itemize}
}
\item countable noun \\
A \textbf{western} is a book or film about life in the west of America in the nineteenth century, especially the lives of cowboys .
 \textit{
	\begin{itemize}
	\end{itemize}
}
\end{enumerate}

\section*{trash}
{\large \color{blue}  trashes  trashing  trashed  }
\subsection*{Explain}
\begin{enumerate}
\item uncountable noun \\
\textbf{Trash} consists of unwanted things or waste material such as used paper , empty  containers and bottles , and waste food.
 \textit{
	\begin{itemize}
	\item The yards are overgrown and cluttered with trash.
	\item Mowing lawns and taking out the trash are jobs for the tenant.
	\end{itemize}
}
\item uncountable noun \\
If you say that something such as a book , painting , or film is \textbf{trash} , you mean that it is of very poor quality.
 \textit{
	\begin{itemize}
	\item Pop music doesn't have to be trash; it can be art.
	\item Don't read that awful trash.
	\end{itemize}
}
\item verb \\
If someone \textbf{trashes} a place or vehicle , they deliberately destroy it or make it very dirty .
 \textit{
	\begin{itemize}
	\item Would they trash the place when the party was over?
	\item The building had been trashed and its electricity supply cut.
	\end{itemize}
}
\item verb \\
If you \textbf{trash} people or their ideas, you criticize them very strongly and say that they are worthless.
 \textit{
	\begin{itemize}
	\item People asked why the candidates spent so much time trashing each other.
	\end{itemize}
}
\end{enumerate}

\section*{arrival}
{\large \color{blue}  arrivals  }
\subsection*{Explain}
\begin{enumerate}
\item variable noun \\
When a person or vehicle arrives at a place, you can refer to their \textbf{arrival} .
 \textit{
	\begin{itemize}
	\item ...the day after his arrival in England.
	\item He was dead on arrival at the nearby hospital.
	\item ...the airport arrivals hall.
	\item ...the arrival gate at Penn Station.
	\end{itemize}
}
\item variable noun \\
When someone starts a new job , you can refer to their \textbf{arrival} in that job.
 \textit{
	\begin{itemize}
	\item ...the power vacuum created by the arrival of a new president.
	\item The company had eight departures and 11 new arrivals on its management board in 1980-89.
	\end{itemize}
}
\item singular noun \\
When something is brought to you or becomes available , you can refer to its \textbf{arrival} .
 \textit{
	\begin{itemize}
	\item I was flicking idly through a newspaper while awaiting the arrival of orange juice
and coffee.
	\item The coronation broadcast marked the arrival of television.
	\end{itemize}
}
\item singular noun \\
When a particular time comes or a particular event  happens , you can refer to its \textbf{arrival} .
 \textit{
	\begin{itemize}
	\item He celebrated the arrival of the New Year with a party for his friends.
	\end{itemize}
}
\item countable noun \\
You can refer to someone who has just arrived at a place as a new \textbf{arrival} .
 \textit{
	\begin{itemize}
	\item A high proportion of the new arrivals are skilled professionals.
	\item He was the most junior and most recent arrival at the embassy.
	\end{itemize}
}
\item singular noun \\
When a baby is born, you can refer to its \textbf{arrival} .
 \textit{
	\begin{itemize}
	\item ...a couple anticipating the arrival of a new child.
	\end{itemize}
}
\item countable noun \\
You can refer to a baby who has just been born as a new \textbf{arrival} .
 \textit{
	\begin{itemize}
	\item Her father was besotted with the new arrival.
	\end{itemize}
}
\end{enumerate}

\section*{ambitious}
{\large \color{blue}  }
\subsection*{Explain}
\begin{enumerate}
\item adjective \\
Someone who is \textbf{ambitious} has a strong desire to be successful , rich , or powerful .
 \textit{
	\begin{itemize}
	\item Chris is so ambitious, so determined to do it all.
	\item He's a very ambitious lad and he wants to play at the highest level.
	\end{itemize}
}
\item adjective \\
An \textbf{ambitious}  idea or plan is on a large scale and needs a lot of work to be carried out successfully.
 \textit{
	\begin{itemize}
	\item The ambitious project was completed in only nine months.
	\item Their goal was extraordinarily ambitious.
	\end{itemize}
}
\end{enumerate}

\section*{attitude}
{\large \color{blue}  attitudes  }
\subsection*{Explain}
\begin{enumerate}
\item variable noun \\
Your \textbf{attitude}  \textbf{to} something is the way that you think and feel about it, especially when this shows in the way you behave.
 \textit{
	\begin{itemize}
	\item ...the general change in attitude towards people with disabilities.
	\item His attitude made me angry.
	\item I don't think it's fair to accuse me of having an attitude problem.
	\end{itemize}
}
\item uncountable noun \\
If you refer to someone as a person \textbf{with}  \textbf{attitude} , you mean that they have a striking and individual  style of behaviour , especially a forceful or aggressive one.
 \textit{
	\begin{itemize}
	\end{itemize}
}
\item  \\
 attitude of mind \textit{
	\begin{itemize}
	\end{itemize}
}
\end{enumerate}

\section*{capable}
{\large \color{blue}  }
\subsection*{Explain}
\begin{enumerate}
\item adjective \\
If a person or thing is \textbf{capable of} doing something, they have the ability to do it.
 \textit{
	\begin{itemize}
	\item He appeared hardly capable of conducting a coherent conversation.
	\item The kitchen is capable of catering for several hundred people.
	\item I had no hesitation in calling the police because I realised he was capable of murder.
	\end{itemize}
}
\item adjective \\
Someone who is \textbf{capable} has the skill or qualities necessary to do a particular thing well , or is able to do most things well.
 \textit{
	\begin{itemize}
	\item She's a very capable speaker.
	\item Sam was a highly capable manager.
	\end{itemize}
}
\end{enumerate}

\section*{ceremony}
{\large \color{blue}  ceremonies  }
\subsection*{Explain}
\begin{enumerate}
\item countable noun \\
A \textbf{ceremony} is a formal event such as a wedding .
 \textit{
	\begin{itemize}
	\item ...his grandmother's funeral, a private ceremony attended only by the family.
	\item Today's award ceremony took place at the British Embassy in Tokyo.
	\end{itemize}
}
\item uncountable noun \\
\textbf{Ceremony} consists of the special things that are said and done on very formal occasions .
 \textit{
	\begin{itemize}
	\item The Republic was proclaimed in public with great ceremony.
	\item ...the pomp and ceremony of the Pope's visit.
	\end{itemize}
}
\item uncountable noun \\
If you do something \textbf{without ceremony} , you do it quickly and in a casual way.
 \textit{
	\begin{itemize}
	\item 'Is Hilton here?' she asked without ceremony.
	\end{itemize}
}
\end{enumerate}

\section*{competent}
{\large \color{blue}  }
\subsection*{Explain}
\begin{enumerate}
\item adjective \\
Someone who is \textbf{competent} is efficient and effective .
 \textit{
	\begin{itemize}
	\item He was a loyal, distinguished and very competent civil servant.
	\item ...a competent performance.
	\end{itemize}
}
\item adjective \\
If you are \textbf{competent}  \textbf{to} do something, you have the skills, abilities , or experience  necessary to do it well .
 \textit{
	\begin{itemize}
	\item Most adults do not feel competent to deal with a medical emergency involving a child.
	\end{itemize}
}
\end{enumerate}

\section*{conversion}
{\large \color{blue}  conversions  }
\subsection*{Explain}
\begin{enumerate}
\item variable noun \\
\textbf{Conversion} is the act or process of changing something into a different state or form.
 \textit{
	\begin{itemize}
	\item ...the conversion of disused rail lines into cycle routes.
	\item A loft conversion can add considerably to the value of a house.
	\end{itemize}
}
\item variable noun \\
If someone changes their religion or beliefs, you can refer to their \textbf{conversion}  \textbf{to} their new religion or beliefs.
 \textit{
	\begin{itemize}
	\item ...his conversion to Christianity.
	\item It's hard to trust the President's conversion.
	\end{itemize}
}
\item countable noun \\
In rugby , if a player makes or kicks a \textbf{conversion} , he scores points by kicking the ball over the goal after a try has been scored.
 \textit{
	\begin{itemize}
	\end{itemize}
}
\end{enumerate}

\section*{creative}
{\large \color{blue}  creatives  }
\subsection*{Explain}
\begin{enumerate}
\item adjective \\
A \textbf{creative} person has the ability to invent and develop  original  ideas , especially in the arts .
 \textit{
	\begin{itemize}
	\item Like so many creative people, he was never satisfied.
	\item ...her obvious creative talents.
	\end{itemize}
}
\item adjective \\
\textbf{Creative} activities involve the inventing and making of new kinds of things.
 \textit{
	\begin{itemize}
	\item ...creative writing.
	\item ...creative arts.
	\item Cooking is creative.
	\end{itemize}
}
\item adjective \\
If you use something in a \textbf{creative} way, you use it in a new way that produces interesting and unusual results.
 \textit{
	\begin{itemize}
	\item ...his creative use of words.
	\end{itemize}
}
\item countable noun \\
A \textbf{creative} is someone whose job is to be creative, especially someone who creates advertisements .
 \textit{
	\begin{itemize}
	\item Along with scores of other advertising creatives, he will be taking part in an exhibition
at the Saatchi gallery.
	\end{itemize}
}
\end{enumerate}

\section*{crucial}
{\large \color{blue}  }
\subsection*{Explain}
\begin{enumerate}
\item adjective \\
If you describe something as \textbf{crucial} , you mean it is extremely important.
 \textit{
	\begin{itemize}
	\item He had administrators under him but took the crucial decisions himself.
	\item ...the most crucial election campaign for years.
	\item Improved consumer confidence is crucial to an economic recovery.
	\end{itemize}
}
\end{enumerate}

\section*{country}
{\large \color{blue}  countries  }
\subsection*{Explain}
\begin{enumerate}
\item countable noun \\
A \textbf{country} is one of the political units which the world is divided into, covering a particular area of land.
 \textit{
	\begin{itemize}
	\item Indonesia is the fifth most populous country in the world.
	\item ...that disputed boundary between the two countries.
	\item Young people do move around the country quite a bit these days.
	\end{itemize}
}
\item singular noun \\
The people who live in a particular country can be referred to as \textbf{the}  \textbf{country} .
 \textit{
	\begin{itemize}
	\item The country goes to the polls today to elect a new government.
	\item Seventy per cent of this country is opposed to blood sports.
	\end{itemize}
}
\item singular noun \\
\textbf{The country} consists of places such as farms , open  fields , and villages which are away from towns and cities.
 \textit{
	\begin{itemize}
	\item ...a healthy life in the country.
	\item She was cycling along a country road near Compiègne.
	\item I was a simple country boy from Norfolk.
	\end{itemize}
}
\item uncountable noun \\
A particular kind of \textbf{country} is an area of land which has particular characteristics or is connected with a particular well-known person.
 \textit{
	\begin{itemize}
	\item Varese Ligure is a small town in mountainous country east of Genoa.
	\item ...some of the best walking country in the Sierras.
	\item The Japanese visitors set off in search of Brontë country.
	\end{itemize}
}
\item uncountable noun \\
\textbf{Country}  music is popular music from the southern  United  States .
 \textit{
	\begin{itemize}
	\item For a long time I just wanted to play country music.
	\item The brilliant young country singer tours her songs about lost love and cigarettes,
from March 7.
	\end{itemize}
}
\item  \\
 across country \textit{
	\begin{itemize}
	\end{itemize}
}
\item  \\
 across country \textit{
	\begin{itemize}
	\end{itemize}
}
\item  \\
 go to the country \textit{
	\begin{itemize}
	\end{itemize}
}
\end{enumerate}

\section*{dead}
{\large \color{blue}  }
\subsection*{Explain}
\begin{enumerate}
\item adjective \\
A person, animal, or plant that is \textbf{dead} is no longer living.
 \textbf{The dead} are people who are dead.
 \textit{
	\begin{itemize}
	\item Her husband's been dead a year now.
	\item The group had shot dead another hostage.
	\item ...their dead brother.
	\item ...old newspapers and dead flowers.
	\item The dead included six people attending a religious ceremony.
	\item ...the annual festival when Chinese traditionally honour the dead.
	\end{itemize}
}
\item adjective \\
Land or water that is \textbf{dead} contains no living things.
 \textit{
	\begin{itemize}
	\item ...charred land, mountainsides of dead earth and stumps of trees.
	\item But this water seems dead: it's polluted and horribly stagnant.
	\end{itemize}
}
\item adjective \\
If you describe a place or a period of time as \textbf{dead} , you do not like it because there is very little activity taking place in it.
 \textit{
	\begin{itemize}
	\item ...some dead little town where the liveliest thing is the flies.
	\item This made that holiday week a particularly dead period.
	\end{itemize}
}
\item adjective \\
Something that is \textbf{dead} is no longer being used or is finished .
 \textit{
	\begin{itemize}
	\item The dead cigarette was still between his fingers.
	\item This bottle's dead. But we've got another one.
	\end{itemize}
}
\item adjective \\
If you say that an idea, plan, or subject is \textbf{dead} , you mean that people are no longer interested in it or willing to develop it any
further.
 \textit{
	\begin{itemize}
	\item It's a dead issue, Baxter.
	\item But that doesn't mean this brand of politics is dead or dying.
	\item The deal with Chelsea may not, however, be dead.
	\end{itemize}
}
\item adjective \\
A \textbf{dead} language is no longer spoken or written as a means of communication , although it may still be studied.
 \textit{
	\begin{itemize}
	\item We used to grumble that we were wasting time learning a dead language.
	\end{itemize}
}
\item adjective \\
A telephone or piece of electrical equipment that is \textbf{dead} is no longer functioning, for example because it no longer has any electrical power.
 \textit{
	\begin{itemize}
	\item On another occasion I answered the phone and the line went dead.
	\end{itemize}
}
\item adjective \\
In sport, when a ball is \textbf{dead} , it has gone outside the playing area, or a situation has occurred in which the game
has to be temporarily stopped, and none of the players can score points or gain an advantage .
 \textit{
	\begin{itemize}
	\end{itemize}
}
\item graded adjective \\
A \textbf{dead} sound or colour is dull rather than lively or bright.
 \textbf{Dead} is also a combining form.
 \textit{
	\begin{itemize}
	\item 'That is correct, Meg,' he answered in his cold, dead voice.
	\item Then he heard a piercing scream echoing down the deep well, ending in a dull, dead
thud.
	\item The blood drained from his face, leaving the skin dead white.
	\end{itemize}
}
\item adjective \\
\textbf{Dead} is used to mean 'complete' or ' absolute ', especially before the words 'centre', ' silence ', and 'stop'.
 \textit{
	\begin{itemize}
	\item He adjusted each chesspiece so that it stood dead centre in its square.
	\item They hurried about in dead silence, with anxious faces.
	\item Lila's boat came to a dead stop.
	\end{itemize}
}
\item adverb \\
\textbf{Dead} means ' precisely ' or ' exactly '.
 \textit{
	\begin{itemize}
	\item Mars was visible, dead in the centre of the telescope.
	\item Their arrows are dead on target.
	\item A fishing boat came out of nowhere, dead ahead.
	\end{itemize}
}
\item adverb \\
\textbf{Dead} is sometimes used to mean 'very'.
 \textit{
	\begin{itemize}
	\item Meadowhall is also dead easy for people to get to.
	\item His poems sound dead boring, actually.
	\item I am dead against the legalisation of drugs.
	\end{itemize}
}
\item  \\
 over my dead body \textit{
	\begin{itemize}
	\end{itemize}
}
\item  \\
 dead and buried \textit{
	\begin{itemize}
	\end{itemize}
}
\item  \\
 drop (down) dead \textit{
	\begin{itemize}
	\end{itemize}
}
\item  \\
 to drop dead \textit{
	\begin{itemize}
	\end{itemize}
}
\item  \\
 dead and gone \textit{
	\begin{itemize}
	\end{itemize}
}
\item  \\
 be/feel/look (half) dead \textit{
	\begin{itemize}
	\end{itemize}
}
\item  \\
 at/in (the) dead of (the) night/winter \textit{
	\begin{itemize}
	\end{itemize}
}
\item  \\
 rise/raise sb from the dead \textit{
	\begin{itemize}
	\end{itemize}
}
\item  \\
 rise/come back from the dead \textit{
	\begin{itemize}
	\end{itemize}
}
\item  \\
 wouldn't be seen/caught dead \textit{
	\begin{itemize}
	\end{itemize}
}
\item  \\
 to stop dead \textit{
	\begin{itemize}
	\end{itemize}
}
\item  \\
 dead in the water \textit{
	\begin{itemize}
	\end{itemize}
}
\end{enumerate}

\section*{diversion}
{\large \color{blue}  diversions  }
\subsection*{Explain}
\begin{enumerate}
\item countable noun \\
A \textbf{diversion} is an action or event that attracts your attention away from what you are doing or concentrating on.
 \textit{
	\begin{itemize}
	\item The robbers threw smoke bombs to create a diversion.
	\item The whole argument is a diversion.
	\end{itemize}
}
\item countable noun \\
A \textbf{diversion} is an activity that you do for pleasure .
 \textit{
	\begin{itemize}
	\item Finger painting is very messy but an excellent diversion.
	\end{itemize}
}
\item countable noun \\
A \textbf{diversion} is a special route arranged for traffic to follow when the normal route cannot be used.
 \textit{
	\begin{itemize}
	\item They turned back because of traffic diversions.
	\end{itemize}
}
\item uncountable noun \\
\textbf{The diversion of} something involves changing its course or destination .
 \textit{
	\begin{itemize}
	\item ...the illegal diversion of profits from secret arms sales.
	\item ...the diversion of a ship to Lebanon with $8m worth of aluminium on board.
	\end{itemize}
}
\end{enumerate}

\section*{effective}
{\large \color{blue}  }
\subsection*{Explain}
\begin{enumerate}
\item adjective \\
Something that is \textbf{effective} works well and produces the results that were intended .
 \textit{
	\begin{itemize}
	\item Homoeopathic treatment can be effective in treating virtually any illness.
	\item Simple antibiotics are effective against this organism.
	\item ...an effective public transport system.
	\end{itemize}
}
\item adjective \\
\textbf{Effective} means having a particular role or result in practice, though not officially or in theory .
 \textit{
	\begin{itemize}
	\item They have had effective control of the area since the security forces left.
	\item The restructuring resulted in an effective increase on fares.
	\end{itemize}
}
\item adjective \\
When something such as a law or an agreement becomes \textbf{effective} , it begins officially to apply or be valid .
 \textit{
	\begin{itemize}
	\item The new rules will become effective in the next few days.
	\end{itemize}
}
\end{enumerate}

\section*{event}
{\large \color{blue}  events  }
\subsection*{Explain}
\begin{enumerate}
\item countable noun \\
An \textbf{event} is something that happens, especially when it is unusual or important. You can use \textbf{events} to describe all the things that are happening in a particular  situation .
 \textit{
	\begin{itemize}
	\item ...the events of Black Wednesday.
	\item ...in the wake of recent events in Europe.
	\item A new book by J K Rowling is always an event.
	\end{itemize}
}
\item countable noun \\
An \textbf{event} is a planned and organized  occasion , for example a social  gathering or a sports match .
 \textit{
	\begin{itemize}
	\item The cross-country section of the three-day event was held here yesterday.
	\item ...major sporting events.
	\item ...our programme of lectures and social events.
	\end{itemize}
}
\item countable noun \\
An \textbf{event} is one of the races or competitions that are part of an organized occasion such as a sports meeting .
 \textit{
	\begin{itemize}
	\item A solo piper opens Aberdeen Highland Games at 10am and the main events start at 1pm.
	\end{itemize}
}
\item  \\
 in the event of/that \textit{
	\begin{itemize}
	\end{itemize}
}
\item  \\
 in any event \textit{
	\begin{itemize}
	\end{itemize}
}
\item  \\
 in the event \textit{
	\begin{itemize}
	\end{itemize}
}
\end{enumerate}

\section*{evidence}
{\large \color{blue}  evidences  evidencing  evidenced  }
\subsection*{Explain}
\begin{enumerate}
\item uncountable noun \\
\textbf{Evidence} is anything that you see , experience , read , or are told that causes you to believe that something is true or has really  happened .
 \textit{
	\begin{itemize}
	\item Ganley said he'd seen no evidence of widespread fraud.
	\item There is a lot of evidence that stress is partly responsible for disease.
	\item To date there is no evidence to support this theory.
	\end{itemize}
}
\item uncountable noun \\
\textbf{Evidence} is the information which is used in a court of law to try to prove something. Evidence is obtained from documents, objects, or witnesses.
 \textit{
	\begin{itemize}
	\item The evidence against him was purely circumstantial.
	\item ...enough evidence for a successful prosecution.
	\end{itemize}
}
\item  \\
 give evidence \textit{
	\begin{itemize}
	\end{itemize}
}
\item verb \\
If a particular feeling , ability , or attitude  \textbf{is evidenced}  \textbf{by} something or someone, it is seen or felt .
 \textit{
	\begin{itemize}
	\item He's wise in other ways too, as evidenced by his reason for switching from tennis
to golf.
	\item She was not calculating and evidenced no specific interest in money.
	\end{itemize}
}
\item  \\
 be in evidence \textit{
	\begin{itemize}
	\end{itemize}
}
\end{enumerate}

\section*{evil}
{\large \color{blue}  evils  }
\subsection*{Explain}
\begin{enumerate}
\item uncountable noun \\
\textbf{Evil} is a powerful force that some people believe to exist , and which causes wicked and bad things to happen .
 \textit{
	\begin{itemize}
	\item We are still being attacked by the forces of evil.
	\item There's always a conflict between good and evil in his plays.
	\end{itemize}
}
\item uncountable noun \\
\textbf{Evil} is used to refer to all the wicked and bad things that happen in the world.
 \textit{
	\begin{itemize}
	\item He could not, after all, stop all the evil in the world.
	\item ...those who see television as the root of all evil.
	\end{itemize}
}
\item countable noun \\
If you refer to an \textbf{evil} , you mean a very unpleasant or harmful situation or activity.
 \textit{
	\begin{itemize}
	\item Higher taxes may be a necessary evil.
	\item ...a lecture on the evils of alcohol.
	\end{itemize}
}
\item adjective \\
If you describe someone as \textbf{evil} , you mean that they are very wicked by nature and take pleasure in doing things that harm other people.
 \textit{
	\begin{itemize}
	\item ...the country's most evil terrorists.
	\item She's an evil woman.
	\end{itemize}
}
\item adjective \\
If you describe something as \textbf{evil} , you mean that you think it causes a great  deal of harm to people and is morally bad.
 \textit{
	\begin{itemize}
	\item After 1760 few Americans refrained from condemning slavery as evil.
	\item They are setting an evil example for their children.
	\end{itemize}
}
\item adjective \\
If you describe something as \textbf{evil} , you mean that you think it is influenced by the devil .
 \textit{
	\begin{itemize}
	\item I think this is an evil spirit at work.
	\item According to local folklore it is an evil place.
	\end{itemize}
}
\item adjective \\
You can describe a very unpleasant smell as \textbf{evil} .
 \textit{
	\begin{itemize}
	\item Both men were smoking evil-smelling pipes.
	\end{itemize}
}
\item  \\
 the evil day/hour \textit{
	\begin{itemize}
	\end{itemize}
}
\item  \\
 the lesser of two evils \textit{
	\begin{itemize}
	\end{itemize}
}
\end{enumerate}

\section*{excitement}
{\large \color{blue}  excitements  }
\subsection*{Explain}
\begin{enumerate}
\item variable noun \\
You use \textbf{excitement} to refer to the state of being excited, or to something that excites you.
 \textit{
	\begin{itemize}
	\item Everyone is in a state of great excitement.
	\item ...the excitement of a thunderstorm.
	\item This game had its challenges, excitements and rewards.
	\end{itemize}
}
\end{enumerate}

\section*{faulty}
{\large \color{blue}  }
\subsection*{Explain}
\begin{enumerate}
\item adjective \\
A \textbf{faulty} piece of equipment has something wrong with it and is not working properly.
 \textit{
	\begin{itemize}
	\item The money will be used to repair faulty equipment.
	\end{itemize}
}
\item adjective \\
If you describe someone's argument or reasoning as \textbf{faulty} , you mean that it is wrong or contains mistakes , usually because they have not been thinking in a logical way.
 \textit{
	\begin{itemize}
	\item Their interpretation was faulty–they had misinterpreted things.
	\end{itemize}
}
\end{enumerate}

\section*{fact}
{\large \color{blue}  facts  }
\subsection*{Explain}
\begin{enumerate}
\item  \\
 the fact that \textit{
	\begin{itemize}
	\end{itemize}
}
\item  \\
 the fact that \textit{
	\begin{itemize}
	\end{itemize}
}
\item  \\
 in actual fact \textit{
	\begin{itemize}
	\end{itemize}
}
\item  \\
 in actual fact \textit{
	\begin{itemize}
	\end{itemize}
}
\item variable noun \\
When you refer to something as a \textbf{fact} or as \textbf{fact} , you mean that you think it is true or correct .
 \textit{
	\begin{itemize}
	\item ...a statement of verifiable historical fact.
	\item How much was fact and how much fancy no one knew.
	\end{itemize}
}
\item countable noun \\
\textbf{Facts} are pieces of information that can be discovered .
 \textit{
	\begin{itemize}
	\item There is so much information you can almost effortlessly find the facts for yourself.
	\item His opponent swamped him with facts and figures.
	\item The lorries always left at night when there were few witnesses around to record the
fact.
	\end{itemize}
}
\item  \\
 as a matter of fact \textit{
	\begin{itemize}
	\end{itemize}
}
\item  \\
 to know something for a fact \textit{
	\begin{itemize}
	\end{itemize}
}
\item  \\
 the fact is \textit{
	\begin{itemize}
	\end{itemize}
}
\item  \\
 the fact remains \textit{
	\begin{itemize}
	\end{itemize}
}
\item  \\
 and that's a fact \textit{
	\begin{itemize}
	\end{itemize}
}
\item  \\
 is that a fact? \textit{
	\begin{itemize}
	\end{itemize}
}
\end{enumerate}

\section*{finite}
{\large \color{blue}  }
\subsection*{Explain}
\begin{enumerate}
\item adjective \\
Something that is \textbf{finite} has a definite  fixed  size or extent.
 \textit{
	\begin{itemize}
	\item ...a finite set of elements.
	\item Only a finite number of situations can arise.
	\item The fossil fuels (coal and oil) are finite resources.
	\end{itemize}
}
\item adjective \\
A \textbf{finite}  clause is a clause based on a verb group which indicates tense, such as ' went ', 'is waiting ', or ' will be found', rather than on an infinitive or a participle . Compare  non-finite .
 \textit{
	\begin{itemize}
	\end{itemize}
}
\end{enumerate}

\section*{fixture}
{\large \color{blue}  fixtures  }
\subsection*{Explain}
\begin{enumerate}
\item countable noun \\
\textbf{Fixtures} are pieces of furniture or equipment , for example  baths and sinks , which are fixed inside a house or other building and which stay there if you move.
 \textit{
	\begin{itemize}
	\item ...a detailed list of what fixtures and fittings are included in the purchase price.
	\end{itemize}
}
\item countable noun \\
A \textbf{fixture} is a sports event which takes place on a particular date.
 \textit{
	\begin{itemize}
	\item City won this fixture 3-0 last season.
	\end{itemize}
}
\item countable noun \\
If you describe someone or something as \textbf{a}  \textbf{fixture}  \textbf{in} a particular place or occasion, you mean that they always  seem to be there.
 \textit{
	\begin{itemize}
	\item She was a fixture in New York's nightclubs.
	\item The cordless kettle may now be a fixture in most kitchens.
	\end{itemize}
}
\end{enumerate}

\section*{frontier}
{\large \color{blue}  frontiers  }
\subsection*{Explain}
\begin{enumerate}
\item countable noun \\
A \textbf{frontier} is a border between two countries.
 \textit{
	\begin{itemize}
	\item It wasn't difficult then to cross the frontier.
	\end{itemize}
}
\item countable noun \\
When you are talking about the western part of America before the twentieth  century , you use \textbf{frontier} to refer to the area beyond the part settled by Europeans .
 \textit{
	\begin{itemize}
	\item ...a far-flung outpost on the frontier.
	\end{itemize}
}
\item countable noun \\
\textbf{The}  \textbf{frontiers} of something, especially knowledge, are the limits to which it extends .
 \textit{
	\begin{itemize}
	\item ...pushing back the frontiers of science.
	\item ...technological frontiers.
	\end{itemize}
}
\end{enumerate}

\section*{historic}
{\large \color{blue}  }
\subsection*{Explain}
\begin{enumerate}
\item adjective \\
Something that is \textbf{historic} is important in history, or likely to be considered important at some time in the future .
 \textit{
	\begin{itemize}
	\item King gave this historic speech the night before he was assassinated.
	\item ...a fourth historic election victory.
	\end{itemize}
}
\end{enumerate}

\section*{gear}
{\large \color{blue}  gears  gearing  geared  }
\subsection*{Explain}
\begin{enumerate}
\item countable noun \\
The \textbf{gears} on a machine or vehicle are a device for changing the rate at which energy is changed into motion.
 \textit{
	\begin{itemize}
	\item On hills, he must use low gears.
	\item The car was in fourth gear.
	\item He put the truck in gear and drove on.
	\end{itemize}
}
\item uncountable noun \\
The \textbf{gear} involved in a particular activity is the equipment or special clothing that you use.
 \textit{
	\begin{itemize}
	\item About 100 officers in riot gear were needed to break up the fight.
	\item ...fishing gear.
	\item They helped us put our gear back into the van.
	\end{itemize}
}
\item uncountable noun \\
\textbf{Gear} means clothing.
 \textit{
	\begin{itemize}
	\item I used to wear trendy gear but it just looked ridiculous.
	\end{itemize}
}
\item uncountable noun \\
Some people refer to illegal drugs, especially  heroin , as \textbf{gear} .
 \textit{
	\begin{itemize}
	\item Are these people using gear and amphetamines at the same time?
	\end{itemize}
}
\item passive verb \\
If someone or something \textbf{is geared to} or \textbf{towards} a particular purpose, they are organized or designed in order to achieve that purpose.
 \textit{
	\begin{itemize}
	\item Colleges are not always geared to the needs of mature students.
	\item My training was geared towards winning gold in Munich.
	\end{itemize}
}
\end{enumerate}

\section*{incentive}
{\large \color{blue}  incentives  }
\subsection*{Explain}
\begin{enumerate}
\item variable noun \\
If something is an \textbf{incentive}  \textbf{to} do something, it encourages you to do it.
 \textit{
	\begin{itemize}
	\item There is little or no incentive to adopt such measures.
	\item Many companies in Britain are keen on the idea of tax incentives for R&D.
	\end{itemize}
}
\end{enumerate}

\section*{influential}
{\large \color{blue}  }
\subsection*{Explain}
\begin{enumerate}
\item adjective \\
Someone or something that is \textbf{influential} has a lot of influence over people or events.
 \textit{
	\begin{itemize}
	\item It helps to have influential friends.
	\item ...the influential position of president of the chamber.
	\item He had been influential in shaping economic policy.
	\item ...one of the most influential books ever written.
	\end{itemize}
}
\end{enumerate}

\section*{insect}
{\large \color{blue}  insects  }
\subsection*{Explain}
\begin{enumerate}
\item countable noun \\
An \textbf{insect} is a small animal that has six legs. Most insects have wings. Ants , flies , butterflies , and beetles are all insects.
 \textit{
	\begin{itemize}
	\end{itemize}
}
\end{enumerate}

\section*{interesting}
{\large \color{blue}  }
\subsection*{Explain}
\begin{enumerate}
\item adjective \\
If you find something \textbf{interesting} , it attracts your attention , for example because you think it is exciting or unusual .
 \textit{
	\begin{itemize}
	\item It was interesting to be in a different environment.
	\item The two halves of the town face each other, and both have interesting churches.
	\item His third album is by far his most interesting.
	\end{itemize}
}
\end{enumerate}

\section*{kidney}
{\large \color{blue}  kidneys  }
\subsection*{Explain}
\begin{enumerate}
\item countable noun \\
Your \textbf{kidneys} are the organs in your body that take waste matter from your blood and send it out of your body as urine.
 \textit{
	\begin{itemize}
	\end{itemize}
}
\item variable noun \\
\textbf{Kidneys} are the kidneys of an animal, for example a lamb , calf , or pig , that are eaten as meat .
 \textit{
	\begin{itemize}
	\item ...lambs' kidneys.
	\item ...steak and kidney pie.
	\end{itemize}
}
\end{enumerate}

\section*{liable}
{\large \color{blue}  }
\subsection*{Explain}
\begin{enumerate}
\item  \\
 be liable to do sth \textit{
	\begin{itemize}
	\end{itemize}
}
\item adjective \\
If people or things are \textbf{liable to} something unpleasant , they are likely to experience it or do it.
 \textit{
	\begin{itemize}
	\item She will grow into a woman particularly liable to depression.
	\item This volcanic rock is less liable to shatter than limestone.
	\end{itemize}
}
\item adjective \\
If you are \textbf{liable}  \textbf{for} something such as a debt , you are legally responsible for it.
 \textit{
	\begin{itemize}
	\item The airline's insurer is liable for damages to the victims' families.
	\item As the killings took place outside British jurisdiction, the Ministry of Defence
could not be held liable.
	\end{itemize}
}
\end{enumerate}

\section*{lavatory}
{\large \color{blue}  lavatories  }
\subsection*{Explain}
\begin{enumerate}
\item countable noun \\
A \textbf{lavatory} is the same as a toilet .
 \textit{
	\begin{itemize}
	\item ...the ladies' lavatory at the University of London.
	\item ...a public lavatory.
	\end{itemize}
}
\end{enumerate}

\section*{limited}
{\large \color{blue}  }
\subsection*{Explain}
\begin{enumerate}
\item adjective \\
Something that is \textbf{limited} is not very great in amount, range , or degree .
 \textit{
	\begin{itemize}
	\item They may only have a limited amount of time to get their points across.
	\item Shops have a very limited selection.
	\end{itemize}
}
\item adjective \\
A \textbf{limited} company is one whose owners are legally responsible for only a part of any money that it may  owe if it goes  bankrupt .
 \textit{
	\begin{itemize}
	\item They had plans to turn the club into a limited company.
	\item He is the founder of International Sports Management Limited.
	\end{itemize}
}
\end{enumerate}

\section*{loyalty}
{\large \color{blue}  loyalties  }
\subsection*{Explain}
\begin{enumerate}
\item uncountable noun \\
\textbf{Loyalty} is the quality of staying  firm in your friendship or support for someone or something.
 \textit{
	\begin{itemize}
	\item I have sworn an oath of loyalty to the monarchy.
	\item This is seen as a reward for the army's loyalty during a barracks revolt earlier
this month.
	\end{itemize}
}
\item countable noun \\
\textbf{Loyalties} are feelings of friendship, support, or duty towards someone or something.
 \textit{
	\begin{itemize}
	\item She had developed strong loyalties to the Manet family.
	\end{itemize}
}
\end{enumerate}

\section*{mature}
{\large \color{blue}  matures  maturing  matured  maturer  maturest  }
\subsection*{Explain}
\begin{enumerate}
\item verb \\
When a child or young animal \textbf{matures} , it becomes an adult.
 \textit{
	\begin{itemize}
	\item You will learn what to expect as your child matures physically.
	\item The eggs hatched and the chicks matured.
	\item ...young girls who'd not yet matured.
	\end{itemize}
}
\item verb \\
When something \textbf{matures} , it reaches a state of complete  development .
 \textit{
	\begin{itemize}
	\item When the trees matured they were cut in certain areas.
	\item Their songwriting has matured.
	\end{itemize}
}
\item verb \\
If someone \textbf{matures} , they become more fully developed in their personality and emotional  behaviour .
 \textit{
	\begin{itemize}
	\item Hopefully after three years at university I will have matured.
	\item I thought you had matured enough not to be giggly and silly about serious art.
	\end{itemize}
}
\item adjective \\
If you describe someone as \textbf{mature} , you think that they are fully developed and balanced in their personality and emotional behaviour.
 \textit{
	\begin{itemize}
	\item They are emotionally mature and should behave responsibly.
	\item You and I are mature, freethinking adults.
	\end{itemize}
}
\item graded adjective \\
If you describe the work of an artist , writer , or musician as \textbf{mature} , you mean that it is thoughtful and skilful and shows that their abilities have fully developed.
 \textit{
	\begin{itemize}
	\item It is his most mature comedy yet.
	\end{itemize}
}
\item verb \\
If something such as wine or cheese \textbf{matures} or \textbf{is matured} , it is left for a time to allow its full  flavour or strength to develop.
 \textit{
	\begin{itemize}
	\item Unlike wine, brandy matures only in wood, not glass.
	\item ...the cellars where the cheeses are matured.
	\item ...our best selling matured cheddar.
	\end{itemize}
}
\item adjective \\
\textbf{Mature} cheese or wine has been left for a time to allow its full flavour or strength to
develop.
 \textit{
	\begin{itemize}
	\item Grate some mature cheddar cheese.
	\item ...the best place to enjoy fine, mature wines.
	\end{itemize}
}
\item verb \\
When an investment such as a savings  policy or pension plan \textbf{matures} , it reaches the stage when you stop  paying  money and the company pays you back the money you have saved , and the interest your money has earned .
 \textit{
	\begin{itemize}
	\item These bonuses will be paid when your savings plan matures in ten years' time.
	\item ...an endowment policy that matured on September 1.
	\end{itemize}
}
\item adjective \\
If you say that someone is \textbf{mature} or of \textbf{mature}  years , you are saying politely that they are middle-aged or old .
 \textit{
	\begin{itemize}
	\item ...a man of mature years.
	\end{itemize}
}
\end{enumerate}

\section*{mineral}
{\large \color{blue}  minerals  }
\subsection*{Explain}
\begin{enumerate}
\item countable noun \\
A \textbf{mineral} is a substance such as tin , salt , or sulphur that is formed naturally in rocks and in the earth . Minerals are also found in small quantities in food and drink.
 \textit{
	\begin{itemize}
	\end{itemize}
}
\end{enumerate}

\section*{moderate}
{\large \color{blue}  moderates  moderating  moderated  }
\subsection*{Explain}
\begin{enumerate}
\item adjective \\
\textbf{Moderate}  political  opinions or policies are not extreme.
 \textit{
	\begin{itemize}
	\item He was an easygoing man of very moderate views.
	\item Both countries have called for a moderate approach to the use of force.
	\end{itemize}
}
\item adjective \\
You use \textbf{moderate} to describe people or groups who have moderate political opinions or policies.
 A \textbf{moderate} is someone with moderate political opinions.
 \textit{
	\begin{itemize}
	\item ...a moderate Democrat.
	\item ...the moderate wing of the army.
	\item If he presents himself as a radical, he risks scaring off the moderates.
	\end{itemize}
}
\item adjective \\
You use \textbf{moderate} to describe something that is neither large nor small in amount or degree .
 \textit{
	\begin{itemize}
	\item While a moderate amount of stress can be beneficial, too much stress can exhaust
you.
	\item ...moderate exercise.
	\end{itemize}
}
\item adjective \\
A \textbf{moderate}  change in something is a change that is not great .
 \textit{
	\begin{itemize}
	\item Most drugs offer either no real improvement or, at best, only moderate improvements.
	\end{itemize}
}
\item verb \\
If you \textbf{moderate} something or if it \textbf{moderates} , it becomes less extreme or violent and easier to deal with or accept .
 \textit{
	\begin{itemize}
	\item They are hoping that once in office he can be persuaded to moderate his views.
	\item Amongst relief workers, the immediate sense of crisis has moderated somewhat.
	\item Without Westcott's moderating influence, Mathers's autocratic manner became unbearable.
	\end{itemize}
}
\end{enumerate}

\section*{miracle}
{\large \color{blue}  miracles  }
\subsection*{Explain}
\begin{enumerate}
\item countable noun \\
If you say that a good event is a \textbf{miracle} , you mean that it is very surprising and unexpected .
 \textit{
	\begin{itemize}
	\item It is a miracle no one was killed.
	\item The Italian economic miracle has always been a mystery.
	\end{itemize}
}
\item adjective \\
A \textbf{miracle}  drug or product does something that was thought  almost  impossible .
 \textit{
	\begin{itemize}
	\item ...a miracle drug that is said to be a cure for Aids and cancer.
	\end{itemize}
}
\item countable noun \\
A \textbf{miracle} is a wonderful and surprising event that is believed to be caused by God .
 \textit{
	\begin{itemize}
	\item ...the miracle of the Virgin Birth.
	\item ...Jesus's ability to perform miracles.
	\end{itemize}
}
\end{enumerate}

\section*{overhead}
{\large \color{blue}  }
\subsection*{Explain}
\begin{enumerate}
\item adjective \\
You use \textbf{overhead} to indicate that something is above you or above the place that you are talking about.
 \textbf{Overhead} is also an adverb .
 \textit{
	\begin{itemize}
	\item She turned on the overhead light and looked around the little room.
	\item ...people who live under or near overhead cables.
	\item ...planes passing overhead.
	\item Now there were only the stars overhead.
	\end{itemize}
}
\end{enumerate}

\section*{monopoly}
{\large \color{blue}  monopolies  }
\subsection*{Explain}
\begin{enumerate}
\item variable noun \\
If a company, person, or state has a \textbf{monopoly}  \textbf{on} something such as an industry , they have complete control over it, so that it is impossible for others to become involved in it.
 \textit{
	\begin{itemize}
	\item ...Russian moves to end a state monopoly on land ownership.
	\item ...the governing party's monopoly over the media.
	\item ...an inquiry by the Monopolies Commission.
	\end{itemize}
}
\item countable noun \\
A \textbf{monopoly} is a company which is the only one providing a particular product or service.
 \textit{
	\begin{itemize}
	\item ...a state-owned monopoly.
	\end{itemize}
}
\item singular noun \\
If you say that someone does not have a \textbf{monopoly}  \textbf{on} something, you mean that they are not the only person who has that thing.
 \textit{
	\begin{itemize}
	\item Women do not have a monopoly on feelings of betrayal.
	\end{itemize}
}
\end{enumerate}

\section*{patient}
{\large \color{blue}  patients  }
\subsection*{Explain}
\begin{enumerate}
\item countable noun \\
A \textbf{patient} is a person who is receiving medical treatment from a doctor or hospital . A \textbf{patient} is also someone who is registered with a particular doctor.
 \textit{
	\begin{itemize}
	\item The earlier the treatment is given, the better the patient's chances.
	\item She was tough but wonderful with her patients.
	\item He specialized in treatment of cancer patients.
	\end{itemize}
}
\item adjective \\
If you are \textbf{patient} , you stay  calm and do not get  annoyed , for example when something takes a long time, or when someone is not doing what you want them to do.
 \textit{
	\begin{itemize}
	\item Please be patient–your cheque will arrive.
	\item He was endlessly kind and patient with children.
	\end{itemize}
}
\end{enumerate}

\section*{nationality}
{\large \color{blue}  nationalities  }
\subsection*{Explain}
\begin{enumerate}
\item variable noun \\
If you have the \textbf{nationality} of a particular country, you were born there or have the legal  right to be a citizen.
 \textit{
	\begin{itemize}
	\item Asked his nationality, he said British.
	\item The crew are of different nationalities and have no common language.
	\end{itemize}
}
\item countable noun \\
You can refer to people who have the same racial  origins as a \textbf{nationality} , especially when they do not have their own independent country.
 \textit{
	\begin{itemize}
	\item The poor of many nationalities struggle for survival.
	\end{itemize}
}
\end{enumerate}

\section*{poisonous}
{\large \color{blue}  }
\subsection*{Explain}
\begin{enumerate}
\item adjective \\
Something that is \textbf{poisonous}  will  kill you or make you ill if you swallow or absorb it.
 \textit{
	\begin{itemize}
	\item All parts of the yew tree are poisonous, including the berries.
	\item ...a large cloud of poisonous gas.
	\end{itemize}
}
\item adjective \\
An animal that is \textbf{poisonous} produces a poison that will kill you or make you ill if the animal bites you.
 \textit{
	\begin{itemize}
	\item There are hundreds of poisonous spiders and snakes.
	\end{itemize}
}
\item adjective \\
If you describe something as \textbf{poisonous} , you mean that it is extremely  unpleasant and likely to spoil or destroy a good relationship or situation .
 \textit{
	\begin{itemize}
	\item ...poisonous comments.
	\item ...lying awake half the night tormented by poisonous suspicions.
	\item ...poisonous attacks on the state-run church.
	\end{itemize}
}
\end{enumerate}

\section*{orchard}
{\large \color{blue}  orchards  }
\subsection*{Explain}
\begin{enumerate}
\item countable noun \\
An \textbf{orchard} is an area of land on which fruit trees are grown .
 \textit{
	\begin{itemize}
	\end{itemize}
}
\end{enumerate}

\section*{polite}
{\large \color{blue}  politer  politest  }
\subsection*{Explain}
\begin{enumerate}
\item adjective \\
Someone who is \textbf{polite} has good manners and behaves in a way that is socially correct and not rude to other people.
 \textit{
	\begin{itemize}
	\item Everyone around him was trying to be polite, but you could tell they were all bored.
	\item It's not polite to point or talk about strangers in public.
	\item Gately, a quiet and very polite young man, made a favourable impression.
	\item I hate having to make polite conversation.
	\end{itemize}
}
\item adjective \\
You can refer to people who consider themselves to be socially superior and to set standards of behaviour for everyone else as \textbf{polite society} or \textbf{polite company} .
 \textit{
	\begin{itemize}
	\item Certain words are vulgar and not acceptable in polite society.
	\end{itemize}
}
\end{enumerate}

\section*{ore}
{\large \color{blue}  ores  }
\subsection*{Explain}
\begin{enumerate}
\item variable noun \\
\textbf{Ore} is rock or earth from which metal can be obtained.
 \textit{
	\begin{itemize}
	\item ...a huge iron ore mine.
	\end{itemize}
}
\end{enumerate}

\section*{preceding}
{\large \color{blue}  }
\subsection*{Explain}
\begin{enumerate}
\item adjective \\
(prenominal) \textit{
	\begin{itemize}
	\end{itemize}
}
\end{enumerate}

\section*{outline}
{\large \color{blue}  outlines  outlining  outlined  }
\subsection*{Explain}
\begin{enumerate}
\item verb \\
If you \textbf{outline} an idea or a plan, you explain it in a general way.
 \textit{
	\begin{itemize}
	\item The mayor outlined his plan to clean up the town's image.
	\item The methods outlined in this book are only suggestions.
	\end{itemize}
}
\item variable noun \\
An \textbf{outline} is a general explanation or description of something.
 \textit{
	\begin{itemize}
	\item Following is an outline of the survey findings.
	\item The proposals were given in outline by the Secretary of State.
	\end{itemize}
}
\item passive verb \\
You say that an object \textbf{is outlined} when you can see its general shape because there is light behind it.
 \textit{
	\begin{itemize}
	\item The Ritz was outlined against the lights up there.
	\item It was a beautiful sight outlined above the starry sky.
	\end{itemize}
}
\item countable noun \\
The \textbf{outline} of something is its general shape, especially when it cannot be clearly seen.
 \textit{
	\begin{itemize}
	\item He could see only the hazy outline of the goalposts.
	\end{itemize}
}
\end{enumerate}

\section*{profitable}
{\large \color{blue}  }
\subsection*{Explain}
\begin{enumerate}
\item adjective \\
A \textbf{profitable} organization or practice makes a profit.
 \textit{
	\begin{itemize}
	\item Improved transport turned agriculture into a highly profitable business.
	\item It was profitable for them to produce large amounts of food.
	\end{itemize}
}
\item adjective \\
Something that is \textbf{profitable} results in some benefit for you.
 \textit{
	\begin{itemize}
	\item ...collaboration which leads to a profitable exchange of personnel and ideas.
	\end{itemize}
}
\end{enumerate}

\section*{paradigm}
{\large \color{blue}  paradigms  }
\subsection*{Explain}
\begin{enumerate}
\item variable noun \\
A \textbf{paradigm} is a model for something which explains it or shows how it can be produced.
 \textit{
	\begin{itemize}
	\item ...a new paradigm of production.
	\end{itemize}
}
\item countable noun \\
A \textbf{paradigm} is a clear and typical example of something.
 \textit{
	\begin{itemize}
	\item He had become the paradigm of the successful man.
	\end{itemize}
}
\end{enumerate}

\section*{promising}
{\large \color{blue}  }
\subsection*{Explain}
\begin{enumerate}
\item adjective \\
Someone or something that is \textbf{promising}  seems  likely to be very good or successful .
 \textit{
	\begin{itemize}
	\item A school has honoured one of its brightest and most promising former pupils.
	\end{itemize}
}
\end{enumerate}

\section*{parliament}
{\large \color{blue}  parliaments  }
\subsection*{Explain}
\begin{enumerate}
\item countable noun \\
The \textbf{parliament} of some countries, for example Britain, is the group of people who make or change its laws , and decide what policies the country should follow .
 \textit{
	\begin{itemize}
	\item Parliament today approved the policy, but it has not yet become law.
	\end{itemize}
}
\item countable noun \\
A particular \textbf{parliament} is a particular period of time in which a parliament is doing its work, between two
elections or between two periods of holiday .
 \textit{
	\begin{itemize}
	\item The legislation is expected to be passed in the next parliament.
	\end{itemize}
}
\end{enumerate}

\section*{regular}
{\large \color{blue}  regulars  }
\subsection*{Explain}
\begin{enumerate}
\item adjective \\
\textbf{Regular} events have equal amounts of time between them, so that they happen , for example, at the same time each day or each week .
 \textit{
	\begin{itemize}
	\item Take regular exercise.
	\item Now it's time for our regular look at the world of international sport.
	\item We're going to be meeting there on a regular basis.
	\item The cartridge must be replaced at regular intervals.
	\end{itemize}
}
\item adjective \\
\textbf{Regular} events happen often.
 \textit{
	\begin{itemize}
	\item This condition usually clears up with regular shampooing.
	\end{itemize}
}
\item adjective \\
If you are, for example, a \textbf{regular}  customer at a shop or a \textbf{regular}  visitor to a place, you go there often.
 \textit{
	\begin{itemize}
	\item 'Tell me, Mr Mentakis, was Mrs Savalas one of your regular customers?'.
	\item She has become a regular visitor to Houghton Hall.
	\item ...people who are not regular churchgoers.
	\end{itemize}
}
\item countable noun \\
The \textbf{regulars} at a place or in a team are the people who often go to the place or are often in
the team.
 \textit{
	\begin{itemize}
	\item Regulars at his local pub have set up a fund to help out.
	\item I wasn't one of their regulars.
	\end{itemize}
}
\item adjective \\
You use \textbf{regular} when referring to the thing, person, time, or place that is usually used by someone.
For example, someone's \textbf{regular} place is the place where they usually sit .
 \textit{
	\begin{itemize}
	\item The man sat at his regular table near the window.
	\item ...samples from one of their regular suppliers.
	\end{itemize}
}
\item adjective \\
A \textbf{regular}  rhythm consists of a series of sounds or movements with equal periods of time between them.
 \textit{
	\begin{itemize}
	\item ...a very regular beat.
	\item He stood in the doorway, listening to her quiet, regular breathing.
	\end{itemize}
}
\item adjective \\
\textbf{Regular} is used to mean 'normal'.
 \textit{
	\begin{itemize}
	\item It looks and feels like a regular guitar.
	\item He describes himself as just a regular guy from suburban Chicago.
	\end{itemize}
}
\item adjective \\
In some restaurants , a \textbf{regular} drink or quantity of food is of medium size.
 \textit{
	\begin{itemize}
	\item ...a cheeseburger and regular fries.
	\end{itemize}
}
\item adjective \\
A \textbf{regular} pattern or arrangement consists of a series of things with equal spaces between them.
 \textit{
	\begin{itemize}
	\item ...sandy hillocks that look as if they've been scattered in a regular pattern on
the ground.
	\item ...regular rows of wooden huts.
	\end{itemize}
}
\item adjective \\
If something has a \textbf{regular} shape, both halves are the same and it has straight edges or a smooth outline .
 \textit{
	\begin{itemize}
	\item ...some regular geometrical shape.
	\end{itemize}
}
\item adjective \\
\textbf{Regular}  troops are professional soldiers who are a permanent part of an official national army.
 \textbf{Regulars} are regular troops.
 \textit{
	\begin{itemize}
	\item Most schemes attempt to reduce the cost of defence through a smaller regular army.
	\item Only about a third of the reinforcements will be regular troops.
	\item ...the presence of a garrison of British regulars.
	\end{itemize}
}
\item adjective \\
In grammar , a \textbf{regular} verb, noun, or adjective inflects in the same way as most verbs, nouns, or adjectives in the language.
 \textit{
	\begin{itemize}
	\end{itemize}
}
\end{enumerate}

\section*{poke}
{\large \color{blue}  pokes  poking  poked  }
\subsection*{Explain}
\begin{enumerate}
\item verb \\
If you \textbf{poke} someone or something, you quickly push them with your finger or with a sharp object.
 \textbf{Poke} is also a noun .
 \textit{
	\begin{itemize}
	\item Lindy poked him in the ribs.
	\item John smiled at them and gave Richard a playful poke.
	\end{itemize}
}
\item verb \\
If you \textbf{poke} one thing \textbf{into} another, you push the first thing into the second thing.
 \textit{
	\begin{itemize}
	\item He poked his finger into the hole.
	\end{itemize}
}
\item verb \\
If something \textbf{pokes out of} or \textbf{through} another thing, you can see part of it appearing from behind or underneath the other thing.
 \textit{
	\begin{itemize}
	\item He saw the dog's twitching nose poke out of the basket.
	\item His fingers poked through the worn tips of his gloves.
	\end{itemize}
}
\item verb \\
If you \textbf{poke} your head through an opening or if it \textbf{pokes} through an opening, you push it through, often so that you can see something more
 easily .
 \textit{
	\begin{itemize}
	\item Julie tapped on my door and poked her head in.
	\item We hadn't been able to poke our heads out and see what was going on.
	\item Raymond's head poked through the doorway.
	\end{itemize}
}
\end{enumerate}

\section*{relevant}
{\large \color{blue}  }
\subsection*{Explain}
\begin{enumerate}
\item adjective \\
Something that is \textbf{relevant}  \textbf{to} a situation or person is important or significant in that situation or to that person.
 \textit{
	\begin{itemize}
	\item Is socialism still relevant to people's lives?
	\item We have passed all relevant information on to the police.
	\end{itemize}
}
\item adjective \\
\textbf{The relevant} thing of a particular kind is the one that is appropriate .
 \textit{
	\begin{itemize}
	\item Make sure you enclose all the relevant certificates.
	\end{itemize}
}
\end{enumerate}

\section*{postcard}
{\large \color{blue}  postcards  }
\subsection*{Explain}
\begin{enumerate}
\item countable noun \\
A \textbf{postcard} is a piece of thin card, often with a picture on one side, which you can write on and send to people without using an envelope.
 \textit{
	\begin{itemize}
	\end{itemize}
}
\end{enumerate}

\section*{shady}
{\large \color{blue}  shadier  shadiest  }
\subsection*{Explain}
\begin{enumerate}
\item adjective \\
You can describe a place as \textbf{shady} when you like the fact that it is sheltered from bright  sunlight , for example by trees or buildings.
 \textit{
	\begin{itemize}
	\item After flowering, place the pot in a shady spot in the garden.
	\item The rooms are admirably cool and shady after the hot brown monotony of the countryside.
	\end{itemize}
}
\item adjective \\
\textbf{Shady} trees provide a lot of shade.
 \textit{
	\begin{itemize}
	\item Clara had been reading in a lounge chair under a shady tree.
	\end{itemize}
}
\item adjective \\
You can describe activities as \textbf{shady} when you think that they might be dishonest or illegal . You can also use \textbf{shady} to describe people who are involved in such activities.
 \textit{
	\begin{itemize}
	\item The company was notorious for shady deals.
	\item Joseph watched a shady-looking bunch playing cards aboard a Mississippi steamer.
	\end{itemize}
}
\end{enumerate}

\section*{rain}
{\large \color{blue}  rains  raining  rained  }
\subsection*{Explain}
\begin{enumerate}
\item uncountable noun \\
\textbf{Rain} is water that falls from the clouds in small drops.
 \textit{
	\begin{itemize}
	\item I hope you didn't get soaked standing out in the rain.
	\item A spot of rain fell on her hand.
	\end{itemize}
}
\item plural noun \\
In countries where rain only falls in certain seasons , this rain is referred to as \textbf{the rains} .
 \textit{
	\begin{itemize}
	\item ...the spring, when the rains came.
	\item The rains have failed again in the Horn of Africa.
	\end{itemize}
}
\item verb \\
When rain falls, you can say that \textbf{it is raining} .
 \textit{
	\begin{itemize}
	\item It rained the whole weekend.
	\item It was raining hard, and she hadn't an umbrella.
	\end{itemize}
}
\item verb \\
If someone \textbf{rains}  blows , kicks , or bombs  \textbf{on} a person or place, the person or place is attacked by many blows, kicks, or bombs. You can also say that blows, kicks, or bombs \textbf{rain on} a person or place.
 \textbf{Rain down} means the same as rain .
 \textit{
	\begin{itemize}
	\item The opponents were raining blows on each other long after the bell had gone.
	\item Rockets, mortars and artillery rounds rained on buildings.
	\item Fighter aircraft rained down high explosives.
	\item Grenades and mortars rained down on the city.
	\end{itemize}
}
\item singular noun \\
A \textbf{rain of} things is a large number of things that fall from the sky at the same time.
 \textit{
	\begin{itemize}
	\item A rain of stones descended on the police.
	\end{itemize}
}
\item  \\
 it never rains but it pours \textit{
	\begin{itemize}
	\end{itemize}
}
\item  \\
 as right as rain \textit{
	\begin{itemize}
	\end{itemize}
}
\item  \\
 rain or shine \textit{
	\begin{itemize}
	\end{itemize}
}
\end{enumerate}

\section*{sick}
{\large \color{blue}  sicker  sickest  }
\subsection*{Explain}
\begin{enumerate}
\item adjective \\
If you are \textbf{sick} , you are ill. \textbf{Sick} usually means physically ill, but it can sometimes be used to mean mentally ill.
 \textbf{The sick} are people who are sick.
 \textit{
	\begin{itemize}
	\item He's very sick. He needs medication.
	\item She found herself with two small children, a sick husband, and no money.
	\item He was not evil, but he was sick.
	\item There were no doctors to treat the sick.
	\end{itemize}
}
\item adjective \\
If you are \textbf{sick} , the food that you have eaten  comes up from your stomach and out of your mouth . If you \textbf{feel}  \textbf{sick} , you feel as if you are going to be sick.
 \textit{
	\begin{itemize}
	\item She got up and was sick in the handbasin.
	\item The very thought of food made him feel sick.
	\item Orange juice makes him sick so don't give it to him.
	\end{itemize}
}
\item uncountable noun \\
\textbf{Sick} is vomit.
 \textit{
	\begin{itemize}
	\end{itemize}
}
\item adjective \\
If you say that you are \textbf{sick of} something or \textbf{sick and tired of} it, you are emphasizing that you are very annoyed by it and want it to stop .
 \textit{
	\begin{itemize}
	\item I am sick and tired of hearing all these people moaning.
	\item Most people here are sick of violence.
	\end{itemize}
}
\item adjective \\
If you describe something such as a joke or story as \textbf{sick} , you mean that it deals with death or suffering in an unpleasantly humorous way.
 \textit{
	\begin{itemize}
	\item ...a sick joke about a cat.
	\item That's really sick.
	\end{itemize}
}
\item  \\
 make sb sick \textit{
	\begin{itemize}
	\end{itemize}
}
\item  \\
 off sick \textit{
	\begin{itemize}
	\end{itemize}
}
\item  \\
 worried sick \textit{
	\begin{itemize}
	\end{itemize}
}
\end{enumerate}

\section*{software}
{\large \color{blue}  }
\subsection*{Explain}
\begin{enumerate}
\item uncountable noun \\
Computer programs are referred to as \textbf{software} . Compare  hardware .
 \textit{
	\begin{itemize}
	\item ...the people who write the software for big computer projects.
	\end{itemize}
}
\end{enumerate}

\section*{significant}
{\large \color{blue}  }
\subsection*{Explain}
\begin{enumerate}
\item adjective \\
A \textbf{significant} amount or effect is large enough to be important or affect a situation to a noticeable  degree .
 \textit{
	\begin{itemize}
	\item A small but significant number of 11-year-olds are illiterate.
	\item ...foods that offer a significant amount of protein.
	\item It is the first drug that seems to have a very significant effect on this disease.
	\end{itemize}
}
\item adjective \\
A \textbf{significant}  fact , event, or thing is one that is important or shows something.
 \textit{
	\begin{itemize}
	\item Time would appear to be the significant factor in this whole drama.
	\item ...a very significant piece of legislation.
	\item I think it was significant that he never knew his own father.
	\end{itemize}
}
\item graded adjective \\
A \textbf{significant} action or gesture is intended to have a special meaning.
 \textit{
	\begin{itemize}
	\item Mrs Bycraft gave Rose a significant glance.
	\end{itemize}
}
\end{enumerate}

\section*{thorn}
{\large \color{blue}  thorns  }
\subsection*{Explain}
\begin{enumerate}
\item countable noun \\
\textbf{Thorns} are the sharp points on some plants and trees, for example on a rose  bush .
 \textit{
	\begin{itemize}
	\item Roses will always have thorns but with care they can be avoided.
	\end{itemize}
}
\item variable noun \\
A \textbf{thorn} or a \textbf{thorn bush} or a \textbf{thorn tree} is a bush or tree which has a lot of thorns on it.
 \textit{
	\begin{itemize}
	\item ...the shade of a thorn bush.
	\item ...the thorn and bramble thickets.
	\end{itemize}
}
\item  \\
 thorn in your side/flesh \textit{
	\begin{itemize}
	\end{itemize}
}
\end{enumerate}

\section*{sympathetic}
{\large \color{blue}  }
\subsection*{Explain}
\begin{enumerate}
\item adjective \\
If you are \textbf{sympathetic} to someone who is in a bad  situation , you are kind to them and show that you understand their feelings.
 \textit{
	\begin{itemize}
	\item She was very sympathetic to the problems of adult students.
	\item It may be that he sees you only as a sympathetic friend.
	\end{itemize}
}
\item adjective \\
If you are \textbf{sympathetic}  \textbf{to} a proposal or action, you approve of it and are willing to support it.
 \textit{
	\begin{itemize}
	\item She met people in London who were sympathetic to her cause.
	\item His speeches against corruption may find a sympathetic hearing among voters.
	\end{itemize}
}
\item adjective \\
You describe someone as \textbf{sympathetic} when you like them and approve of the way that they behave .
 \textit{
	\begin{itemize}
	\item She sounds a most sympathetic character.
	\end{itemize}
}
\end{enumerate}

\section*{tiger}
{\large \color{blue}  tigers  }
\subsection*{Explain}
\begin{enumerate}
\item countable noun \\
A \textbf{tiger} is a large fierce animal belonging to the cat family. Tigers are orange with black stripes.
 \textit{
	\begin{itemize}
	\end{itemize}
}
\end{enumerate}

\section*{toxic}
{\large \color{blue}  }
\subsection*{Explain}
\begin{enumerate}
\item adjective \\
A \textbf{toxic} substance is poisonous.
 \textit{
	\begin{itemize}
	\item ...the cost of cleaning up toxic waste.
	\item These products are not toxic to humans.
	\end{itemize}
}
\end{enumerate}

\section*{toilet}
{\large \color{blue}  toilets  }
\subsection*{Explain}
\begin{enumerate}
\item countable noun \\
A \textbf{toilet} is a large bowl with a seat , or a platform with a hole , which is connected to a water system and which you use when you want to get  rid of urine or faeces from your body.
 \textit{
	\begin{itemize}
	\item She made Tina flush the pills down the toilet.
	\end{itemize}
}
\item countable noun \\
A \textbf{toilet} is a room in a house or public building that contains a toilet.
 \textit{
	\begin{itemize}
	\item Annette ran and locked herself in the toilet.
	\item Fred never uses public toilets.
	\end{itemize}
}
\item  \\
 go to the toilet \textit{
	\begin{itemize}
	\end{itemize}
}
\end{enumerate}

\section*{useful}
{\large \color{blue}  }
\subsection*{Explain}
\begin{enumerate}
\item adjective \\
If something is \textbf{useful} , you can use it to do something or to help you in some way.
 \textit{
	\begin{itemize}
	\item The slow cooker is very useful for people who go out all day.
	\item Hypnotherapy can be useful in helping you give up smoking.
	\item The police gained a great deal of useful information about the organization.
	\end{itemize}
}
\item  \\
 come in useful \textit{
	\begin{itemize}
	\end{itemize}
}
\end{enumerate}

\section*{tomorrow}
{\large \color{blue}  tomorrows  }
\subsection*{Explain}
\begin{enumerate}
\item adverb \\
You use \textbf{tomorrow} to refer to the day after today.
 \textbf{Tomorrow} is also a noun .
 \textit{
	\begin{itemize}
	\item Bye, see you tomorrow.
	\item The first official results will be announced tomorrow.
	\item What's on your agenda for tomorrow?
	\item He will play for the team in tomorrow's match against England.
	\item Tomorrow is Christmas Day.
	\end{itemize}
}
\item adverb \\
You can refer to the future, especially the near future, as \textbf{tomorrow} .
 \textbf{Tomorrow} is also a noun.
 \textit{
	\begin{itemize}
	\item What is education going to look like tomorrow?
	\item ...tomorrow's computer industry.
	\item Experiences in the past become a part of us, affecting our tomorrows.
	\end{itemize}
}
\end{enumerate}

\section*{valid}
{\large \color{blue}  }
\subsection*{Explain}
\begin{enumerate}
\item adjective \\
A \textbf{valid} argument, comment , or idea is based on sensible  reasoning .
 \textit{
	\begin{itemize}
	\item They put forward many valid reasons for not exporting.
	\item It is valid to consider memory the oldest mental skill, from which all others derive.
	\item He recognized the valid points that both sides were making.
	\end{itemize}
}
\item adjective \\
Something that is \textbf{valid} is important or serious enough to make it worth  saying or doing.
 \textit{
	\begin{itemize}
	\item Most designers share the unspoken belief that fashion is a valid form of visual art.
	\end{itemize}
}
\item adjective \\
If a ticket or other document is \textbf{valid} , it can be used and will be accepted by people in authority.
 \textit{
	\begin{itemize}
	\item For foreign holidays you will need a valid passport.
	\item All tickets are valid for two months.
	\end{itemize}
}
\end{enumerate}

\section*{transition}
{\large \color{blue}  transitions  transitioning  transitioned  }
\subsection*{Explain}
\begin{enumerate}
\item variable noun \\
\textbf{Transition} is the process in which something changes from one state to another.
 \textit{
	\begin{itemize}
	\item The transition to a multi-party democracy is proving to be difficult.
	\item ...a period of transition.
	\end{itemize}
}
\item verb \\
To \textbf{transition}  \textbf{from} one state or activity to another means to move gradually from one to the other.
 \textit{
	\begin{itemize}
	\item The country has begun transitioning from a military dictatorship to a budding democracy.
	\item The company transitioned to an intellectual property company.
	\end{itemize}
}
\item verb \\
To \textbf{transition} means to start  living your life as a person of a different gender. 
 \textit{
	\begin{itemize}
	\item He confirmed in an interview with ABC that he is transitioning to life as a woman.
	\end{itemize}
}
\item variable noun \\
\textbf{Transition} is the process of starting to live your life as a person of a different gender.
 \textit{
	\begin{itemize}
	\item She has made a TV series about her gender transition and how she is adjusting to
her new life.
	\item He started gender transition treatment last year.
	\end{itemize}
}
\end{enumerate}

\section*{known}
{\large \color{blue}  }
\subsection*{Explain}
\begin{enumerate}
\item  \\
\textbf{Known} is the past  participle of know .
 \textit{
	\begin{itemize}
	\end{itemize}
}
\item adjective \\
You use \textbf{known} to describe someone or something that is clearly  recognized by or familiar to all people or to a particular group of people.
 \textit{
	\begin{itemize}
	\item ...He was a known drug dealer.
	\item He became one of the best known actors of his day.
	\item Lead was one of the metals known to the ancient world.
	\item This plant has long been known for its medicinal qualities.
	\item The sport is still little known.
	\end{itemize}
}
\item adjective \\
If someone or something is \textbf{known for} a particular achievement or feature , they are familiar to many people because of that achievement or feature.
 \textit{
	\begin{itemize}
	\item He is better known for his film and TV work.
	\end{itemize}
}
\item  \\
 to let it be known \textit{
	\begin{itemize}
	\end{itemize}
}
\end{enumerate}

\section*{wheel}
{\large \color{blue}  wheels  wheeling  wheeled  }
\subsection*{Explain}
\begin{enumerate}
\item countable noun \\
The \textbf{wheels} of a vehicle are the circular objects which are fixed underneath it and which enable it to move along the ground.
 \textit{
	\begin{itemize}
	\item The car wheels spun and slipped on some oil on the road.
	\end{itemize}
}
\item countable noun \\
A \textbf{wheel} is a circular object which forms a part of a machine, usually a moving part.
 \textit{
	\begin{itemize}
	\item ...an eighteenth century mill with a water wheel.
	\end{itemize}
}
\item countable noun \\
\textbf{The wheel} of a car or other vehicle is the circular object that is used to steer it. \textbf{The wheel} is used in expressions to talk about who is driving a vehicle. For example , if someone is \textbf{at the wheel} of a car, they are driving it.
 \textit{
	\begin{itemize}
	\item My co-pilot suddenly grabbed the wheel.
	\item Curtis got behind the wheel and they started back toward the cottage.
	\item Roberto handed Flynn the keys and let him take the wheel.
	\end{itemize}
}
\item plural noun \\
People sometimes refer to a car as \textbf{wheels} .
 \textit{
	\begin{itemize}
	\item 'Do you own a house?'—'No. But I have wheels.'
	\end{itemize}
}
\item verb \\
If you \textbf{wheel} an object that has wheels somewhere , you push it along.
 \textit{
	\begin{itemize}
	\item He wheeled his bike into the alley at the side of the house.
	\item They wheeled her out on the stretcher.
	\end{itemize}
}
\item verb \\
If something such as a group of animals or birds \textbf{wheels} , it moves in a circle .
 \textit{
	\begin{itemize}
	\item A flock of crows wheeled overhead.
	\end{itemize}
}
\item verb \\
If you \textbf{wheel} around, you turn around suddenly where you are standing , often because you are surprised , shocked , or angry .
 \textit{
	\begin{itemize}
	\item He wheeled around to face her.
	\item She wheeled sharply and headed for the check-out counter.
	\end{itemize}
}
\item singular noun \\
You use \textbf{wheel} in expressions such as \textbf{the wheel of fortune} to refer to the changes that take place in life, especially when you are referring to the fact that the same situations occur more than once.
 \textit{
	\begin{itemize}
	\item The wheel of fortune will swing round again; in politics, it always does.
	\item In his view the wheel of history could not be turned back.
	\end{itemize}
}
\item plural noun \\
People talk about \textbf{the wheels of} an organization or system to mean the way in which it operates.
 \textit{
	\begin{itemize}
	\item He knows the wheels of administration turn slowly.
	\end{itemize}
}
\item  \\
 wheels within wheels \textit{
	\begin{itemize}
	\end{itemize}
}
\end{enumerate}

\section*{wife}
{\large \color{blue}  wives  }
\subsection*{Explain}
\begin{enumerate}
\item countable noun \\
Someone's \textbf{wife} is the woman they are married to.
 \textit{
	\begin{itemize}
	\item He married his wife Jane 37 years ago.
	\item The woman was the wife of a film director.
	\end{itemize}
}
\end{enumerate}

\section*{alike}
{\large \color{blue}  }
\subsection*{Explain}
\begin{enumerate}
\item adjective \\
If two or more things are \textbf{alike} , they are similar in some way.
 \textit{
	\begin{itemize}
	\item We looked very alike.
	\end{itemize}
}
\item adverb \\
\textbf{Alike} means in a similar way.
 \textit{
	\begin{itemize}
	\item They even dressed alike.
	\item ...their assumption that all men and women think alike.
	\end{itemize}
}
\item adverb \\
You use \textbf{alike} after mentioning two or more people, groups, or things in order to emphasize that you are referring to both or all of them.
 \textit{
	\begin{itemize}
	\item The techniques are being applied almost everywhere by big and small firms alike.
	\end{itemize}
}
\end{enumerate}

\section*{bill}
{\large \color{blue}  bills  billing  billed  }
\subsection*{Explain}
\begin{enumerate}
\item countable noun \\
A \textbf{bill} is a written statement of money that you owe for goods or services.
 \textit{
	\begin{itemize}
	\item They couldn't afford to pay the bills.
	\item He paid his bill for the newspapers promptly.
	\item ...phone bills.
	\end{itemize}
}
\item verb \\
If you \textbf{bill} someone \textbf{for} goods or services you have provided them with, you give or send them a bill stating
how much money they owe you for these goods or services.
 \textit{
	\begin{itemize}
	\item Are you going to bill me for this?
	\end{itemize}
}
\item singular noun \\
\textbf{The bill} in a restaurant is a piece of paper on which the price of the meal you have just eaten is written and which you are given before you pay.
 \textit{
	\begin{itemize}
	\end{itemize}
}
\item countable noun \\
A \textbf{bill} is a piece of paper money.
 \textit{
	\begin{itemize}
	\item ...a large quantity of U.S. dollar bills.
	\end{itemize}
}
\item countable noun \\
In government, a \textbf{bill} is a formal statement of a proposed new law that is discussed and then voted on.
 \textit{
	\begin{itemize}
	\item This is the toughest crime bill that Congress has passed in a decade.
	\item The bill was approved by a large majority.
	\end{itemize}
}
\item singular noun \\
The \textbf{bill} of a show or concert is a list of the entertainers who will take part in it.
 \textit{
	\begin{itemize}
	\end{itemize}
}
\item verb \\
If someone \textbf{is billed}  \textbf{to} appear in a particular show, it has been advertised that they are going to be in it.
 \textit{
	\begin{itemize}
	\item She was billed to play the Red Queen in Snow White.
	\end{itemize}
}
\item verb \\
If you \textbf{bill} a person or event \textbf{as} a particular thing, you advertise them in a way that makes people think they have particular qualities or abilities.
 \textit{
	\begin{itemize}
	\item They bill it as Britain's most exciting museum.
	\end{itemize}
}
\item countable noun \\
A bird's \textbf{bill} is its beak.
 \textit{
	\begin{itemize}
	\end{itemize}
}
\item  \\
 fit the bill \textit{
	\begin{itemize}
	\end{itemize}
}
\item  \\
 to foot the bill \textit{
	\begin{itemize}
	\end{itemize}
}
\item  \\
 a clean bill of health \textit{
	\begin{itemize}
	\end{itemize}
}
\end{enumerate}

\section*{attractive}
{\large \color{blue}  }
\subsection*{Explain}
\begin{enumerate}
\item adjective \\
A person who is \textbf{attractive} is pleasant to look at.
 \textit{
	\begin{itemize}
	\item She's a very attractive woman.
	\item I thought he was very attractive and obviously very intelligent.
	\item He was always immensely attractive to women.
	\end{itemize}
}
\item adjective \\
Something that is \textbf{attractive} has a pleasant appearance or sound.
 \textit{
	\begin{itemize}
	\item The flat was small but attractive, if rather shabby.
	\item The creamy white flowers are attractive in the spring.
	\end{itemize}
}
\item adjective \\
You can describe something as \textbf{attractive} when it seems  worth having or doing.
 \textit{
	\begin{itemize}
	\item Co-operation was more than just an attractive option, it was an obligation.
	\item Smoking can still seem attractive to many young people.
	\end{itemize}
}
\end{enumerate}

\section*{cafe}
{\large \color{blue}  cafés  }
\subsection*{Explain}
\begin{enumerate}
\item countable noun \\
A \textbf{café} is a place where you can buy drinks, simple meals, and snacks , but, in Britain, not usually alcoholic drinks.
 \textit{
	\begin{itemize}
	\end{itemize}
}
\item countable noun \\
A street  \textbf{café} or a pavement  \textbf{café} is a café which has tables and chairs on the pavement outside it where people can eat and drink.
 \textit{
	\begin{itemize}
	\item ...an Italian street café.
	\item ...sidewalk cafés and boutiques.
	\end{itemize}
}
\end{enumerate}

\section*{automatic}
{\large \color{blue}  automatics  }
\subsection*{Explain}
\begin{enumerate}
\item adjective \\
An \textbf{automatic} machine or device is one which has controls that enable it to perform a task without needing to be constantly operated by a person. \textbf{Automatic}  methods and processes involve the use of such machines.
 \textit{
	\begin{itemize}
	\item Modern trains have automatic doors.
	\end{itemize}
}
\item countable noun \\
An \textbf{automatic} is a gun that keeps  firing  shots until you stop  pulling the trigger.
 \textit{
	\begin{itemize}
	\item He drew his automatic and began running in the direction of the sounds.
	\item The gunmen opened fire with automatic weapons.
	\end{itemize}
}
\item countable noun \\
An \textbf{automatic} is a car in which the gears change automatically as the car's speed  increases or decreases .
 \textit{
	\begin{itemize}
	\end{itemize}
}
\item adjective \\
An \textbf{automatic} action is one that you do without thinking about it.
 \textit{
	\begin{itemize}
	\item All of the automatic body functions, even breathing, are affected.
	\end{itemize}
}
\item adjective \\
If something such as an action or a punishment is \textbf{automatic} , it happens without people needing to think about it because it is the result of a fixed  rule or method.
 \textit{
	\begin{itemize}
	\item Those drivers should face an automatic charge of manslaughter.
	\end{itemize}
}
\end{enumerate}

\section*{chart}
{\large \color{blue}  charts  charting  charted  }
\subsection*{Explain}
\begin{enumerate}
\item countable noun \\
A \textbf{chart} is a diagram , picture , or graph which is intended to make information easier to understand .
 \textit{
	\begin{itemize}
	\item Male unemployment was 14.2%, compared with 5.8% for women (see chart on next page).
	\item The chart below shows our top 10 choices.
	\end{itemize}
}
\item countable noun \\
A \textbf{chart} is a map of the sea or stars .
 \textit{
	\begin{itemize}
	\item ...charts of Greek waters.
	\end{itemize}
}
\item verb \\
If you \textbf{chart} an area of land , sea, or sky , or a feature in that area, you make a map of the area or show the feature in it.
 \textit{
	\begin{itemize}
	\item Portuguese explorers had charted the west coast of Africa as far as Sierra Leone.
	\item Ptolemy charted more than 1000 stars in 48 constellations.
	\item These seas have been well charted.
	\end{itemize}
}
\item countable noun \\
\textbf{The}  \textbf{charts} are the official  lists that show which songs have had the most downloads or which CDs have sold the most copies each week .
 \textit{
	\begin{itemize}
	\item This album confirmed The Orb's status as national stars, going straight to Number
One in the charts.
	\item They topped both the U.S. singles and album charts at the same time.
	\end{itemize}
}
\item verb \\
If you \textbf{chart} the development or progress of something, you observe it and record or show it. You can also  say that a report or graph \textbf{charts} the development or progress of something.
 \textit{
	\begin{itemize}
	\item One GP has charted a dramatic rise in local childhood asthma since the motorway was
built nearby.
	\item This magnificent show charts his meteoric rise from 'small town' country singer to
top international Rock idol.
	\item Bulletin boards charted each executive's progress.
	\end{itemize}
}
\item verb \\
If a person or plan \textbf{charts} a course of action , they describe what should be done in order to achieve something or to make progress in the future .
 \textit{
	\begin{itemize}
	\item We've charted a possible way forward.
	\item NATO had charted a new course for stability and cooperation in Europe.
	\item Your future is already neatly planned and charted.
	\end{itemize}
}
\end{enumerate}

\section*{back}
{\large \color{blue}  }
\subsection*{Explain}
\begin{enumerate}
\item adverb \\
If you move \textbf{back} , you move in the opposite direction to the one in which you are facing or in which
you were moving before.
 \textit{
	\begin{itemize}
	\item The photographers drew back to let us view the body.
	\item She stepped back from the door expectantly.
	\item He pushed her away and she fell back on the wooden bench.
	\item She pushes back her chair and stands.
	\end{itemize}
}
\item adverb \\
If you go \textbf{back}  somewhere , you return to where you were before.
 \textit{
	\begin{itemize}
	\item I went back to bed.
	\item I'm due back in London by late afternoon.
	\item Smith changed his mind and moved back home.
	\item I'll be back as soon as I can.
	\item He made a round-trip to the terminal and back.
	\end{itemize}
}
\item adverb \\
If someone or something is \textbf{back} in a particular state, they were in that state before and are now in it again.
 \textit{
	\begin{itemize}
	\item The rail company said it expected services to get slowly back to normal.
	\item Denise hopes to be back at work by the time her daughter is one.
	\item Having recently bought an old typewriter, I am now trying to bring it back into working
order.
	\end{itemize}
}
\item adverb \\
If you give or put something \textbf{back} , you return it to the person who had it or to the place where it was before you took
it. If you get or take something \textbf{back} , you then have it again after not having it for a while.
 \textit{
	\begin{itemize}
	\item She handed the knife back.
	\item Put it back in the freezer.
	\item You'll get your money back.
	\end{itemize}
}
\item adverb \\
If you put a clock or watch \textbf{back} , you change the time shown on it so that it shows an earlier time, for example when
the time changes to winter time or standard time.
 \textit{
	\begin{itemize}
	\end{itemize}
}
\item adverb \\
If you write or call \textbf{back} , you write to or phone someone after they have written to or phoned you. If you look \textbf{back} at someone, you look at them after they have started looking at you.
 \textit{
	\begin{itemize}
	\item They wrote back to me and they told me that I didn't have to do it.
	\item If the phone rings, say you'll call back after dinner.
	\item Lee looked at Theodora. She stared back.
	\end{itemize}
}
\item adverb \\
You can say that you go or come \textbf{back}  \textbf{to} a particular point in a conversation to show that you are mentioning or discussing it again.
 \textit{
	\begin{itemize}
	\item Can I come back to the question of policing once again?
	\item To come back to what I said in the Introduction, in the nineteenth century Spain
was fully a part of Europe.
	\item Going back to the school, how many staff are there?
	\end{itemize}
}
\item adverb \\
If something is or comes \textbf{back} , it is fashionable again after it has been unfashionable for some time.
 \textit{
	\begin{itemize}
	\item Short skirts are back.
	\item Consensus politics could easily come back into fashion.
	\end{itemize}
}
\item adverb \\
If someone or something is kept or situated \textbf{back}  \textbf{from} a place, they are at a distance away from it.
 \textit{
	\begin{itemize}
	\item Keep back from the edge of the platform.
	\item I'm a few miles back from the border.
	\item He started for Dot's bedroom and Myrtle held him back.
	\end{itemize}
}
\item adverb \\
If something is held or tied  \textbf{back} , it is held or tied so that it does not hang loosely over something.
 \textit{
	\begin{itemize}
	\item Her hair was tied back.
	\item The curtains were held back by tassels.
	\end{itemize}
}
\item adverb \\
If you lie or sit  \textbf{back} , you move your body backwards into a relaxed sloping or flat position, with your head and body resting on something.
 \textit{
	\begin{itemize}
	\item She lay back and stared at the ceiling.
	\item She leaned back in her chair and smiled.
	\end{itemize}
}
\item adverb \\
If you look or shout  \textbf{back} at someone or something, you turn to look or shout at them when they are behind you.
 \textit{
	\begin{itemize}
	\item Nick looked back over his shoulder and then stopped, frowning.
	\item He called back to her.
	\end{itemize}
}
\item adverb \\
You use \textbf{back} in expressions like \textbf{back in London} or \textbf{back at the house} when you are giving an account, to show that you are going to start talking about what happened or was happening in the place you mention.
 \textit{
	\begin{itemize}
	\item Meanwhile, back in London, Palace Pictures was collapsing.
	\item Later, back at home, the telephone rang.
	\end{itemize}
}
\item adverb \\
If you talk about something that happened \textbf{back} in the past or several years \textbf{back} , you are emphasizing that it happened quite a long time ago .
 \textit{
	\begin{itemize}
	\item The story starts back in 1950, when I was five.
	\item I was in St. Lucia back in January of this year.
	\item He contributed £50m to the project a few years back.
	\end{itemize}
}
\item adverb \\
If you think  \textbf{back}  \textbf{to} something that happened in the past, you remember it or try to remember it.
 \textit{
	\begin{itemize}
	\item I thought back to the time in 1975 when my son was desperately ill.
	\item My mind flew back to stories I had heard about Vinnie.
	\end{itemize}
}
\item  \\
 back and forth \textit{
	\begin{itemize}
	\end{itemize}
}
\end{enumerate}

\section*{coffee}
{\large \color{blue}  coffees  }
\subsection*{Explain}
\begin{enumerate}
\item variable noun \\
\textbf{Coffee} is a hot drink made with water and ground or powdered coffee beans .
 A \textbf{coffee} is a cup of coffee.
 \textit{
	\begin{itemize}
	\item Would you like some coffee?
	\item Newman poured more black coffee.
	\item I made a coffee.
	\end{itemize}
}
\item variable noun \\
\textbf{Coffee} is the roasted beans or powder from which the drink is made.
 \textit{
	\begin{itemize}
	\item Brazil harvested 28m bags of coffee in 1991.
	\item ...superior quality coffee.
	\end{itemize}
}
\end{enumerate}

\section*{backward}
{\large \color{blue}  }
\subsection*{Explain}
\begin{enumerate}
\item adjective \\
A \textbf{backward}  movement or look is in the direction that your back is facing . Some people use backwards for this meaning .
 \textit{
	\begin{itemize}
	\item He turned and walked out without a backward glance.
	\item He did a backward flip.
	\end{itemize}
}
\item adjective \\
If someone takes a \textbf{backward}  step , they do something that does not change or improve their situation , but causes them to go back a stage .
 \textit{
	\begin{itemize}
	\item He didn't want to take a backward step at this point in his career.
	\end{itemize}
}
\item adjective \\
A \textbf{backward}  country or society does not have modern  industries and machines .
 \textit{
	\begin{itemize}
	\item We need to accelerate the pace of change in our backward country.
	\end{itemize}
}
\item adjective \\
A \textbf{backward}  child has difficulty in learning. This use could cause offence .
 \textit{
	\begin{itemize}
	\item I was slow to walk and talk and my parents thought I was backward.
	\end{itemize}
}
\end{enumerate}

\section*{commission}
{\large \color{blue}  commissions  commissioning  commissioned  }
\subsection*{Explain}
\begin{enumerate}
\item verb \\
If you \textbf{commission} something or \textbf{commission} someone \textbf{to} do something, you formally arrange for someone to do a piece of work for you.
 \textbf{Commission} is also a noun .
 \textit{
	\begin{itemize}
	\item The Ministry of Agriculture commissioned a study into low-input farming.
	\item You can commission them to paint something especially for you.
	\item ...specially commissioned reports.
	\item Our china can be bought off the shelf or by commission.
	\item Parliament has set up a commission to investigate football-related violence.
	\end{itemize}
}
\item countable noun \\
A \textbf{commission} is a piece of work that someone is asked to do and is paid for.
 \textit{
	\begin{itemize}
	\item Just a few days ago, I finished a commission.
	\end{itemize}
}
\item variable noun \\
\textbf{Commission} is a sum of money paid to a salesperson for every sale that he or she makes. If a salesperson is paid \textbf{on}  \textbf{commission} , the amount they receive  depends on the amount they sell .
 \textit{
	\begin{itemize}
	\item The salespeople work on commission only.
	\item He also got a commission for bringing in new clients.
	\end{itemize}
}
\item uncountable noun \\
If a bank or other company charges \textbf{commission} , they charge a fee for providing a service, for example for exchanging money or issuing an insurance  policy .
 \textit{
	\begin{itemize}
	\item Travel agents charge 1 per cent commission on sterling cheques.
	\item Sellers pay a fixed commission fee.
	\end{itemize}
}
\item countable noun \\
A \textbf{commission} is a group of people who have been appointed to find out about something or to control something.
 \textit{
	\begin{itemize}
	\item The authorities have been asked to set up a commission to investigate the murders.
	\item ...the Press Complaints Commission.
	\end{itemize}
}
\item uncountable noun \\
\textbf{The}  \textbf{commission}  \textbf{of} a crime is the act of committing a crime.
 \textit{
	\begin{itemize}
	\item Anyone using a gun in the commission of a crime should be given an additional penalty.
	\end{itemize}
}
\item countable noun \\
If a member of the armed  forces receives a \textbf{commission} , he or she becomes an officer.
 \textit{
	\begin{itemize}
	\item He accepted a commission as a naval officer.
	\end{itemize}
}
\item verb \\
If a member of the armed forces \textbf{is commissioned} , he or she is made an officer.
 \textit{
	\begin{itemize}
	\item He was commissioned as second lieutenant in the Air Force.
	\item Only commissioned officers qualify for the Military Cross.
	\end{itemize}
}
\item  \\
 out of commission \textit{
	\begin{itemize}
	\end{itemize}
}
\end{enumerate}

\section*{brilliant}
{\large \color{blue}  }
\subsection*{Explain}
\begin{enumerate}
\item adjective \\
A \textbf{brilliant} person, idea , or performance is extremely  clever or skilful .
 \textit{
	\begin{itemize}
	\item She had a brilliant mind.
	\item Her brilliant performance had earned her two Golden Globes.
	\end{itemize}
}
\item adjective \\
You can say that something is \textbf{brilliant} when you are very pleased about it or think that it is very good.
 \textit{
	\begin{itemize}
	\item If you get a chance to see the show, do go–it's brilliant.
	\item My sister's given me this brilliant book.
	\end{itemize}
}
\item adjective \\
A \textbf{brilliant}  career or success is very successful .
 \textit{
	\begin{itemize}
	\item He served four years in prison, emerging to find his brilliant career in ruins.
	\item The raid was a brilliant success.
	\end{itemize}
}
\item adjective \\
A \textbf{brilliant} colour is extremely bright .
 \textit{
	\begin{itemize}
	\item The woman had brilliant green eyes.
	\item ...a brilliant white open-necked shirt.
	\end{itemize}
}
\item adjective \\
You describe light, or something that reflects light, as \textbf{brilliant} when it shines very brightly.
 \textit{
	\begin{itemize}
	\item The event was held in brilliant sunshine.
	\item It was 250 million times more brilliant than the Sun.
	\end{itemize}
}
\end{enumerate}

\section*{committee}
{\large \color{blue}  committees  }
\subsection*{Explain}
\begin{enumerate}
\item countable noun \\
A \textbf{committee} is a group of people who meet to make decisions or plans for a larger group or organization that they represent.
 \textit{
	\begin{itemize}
	\item ...a committee of ministers.
	\item He sat on the firm's management committee.
	\item ...the Committee for Safety in Medicine.
	\item My reasons were stated in writing and circulated to all committee members.
	\end{itemize}
}
\end{enumerate}

\section*{coordinate}
{\large \color{blue}  }
\subsection*{Explain}
\begin{enumerate}
\item verb \\
1.  2.  3.  4.  5.  \textit{
	\begin{itemize}
	\end{itemize}
}
\item noun \\
6.  7.  \textit{
	\begin{itemize}
	\end{itemize}
}
\item adjective \\
8.  9.  10.  \textit{
	\begin{itemize}
	\item coordinate geometry
	\end{itemize}
}
\end{enumerate}

\section*{compass}
{\large \color{blue}  compasses  }
\subsection*{Explain}
\begin{enumerate}
\item countable noun \\
A \textbf{compass} is an instrument that you use for finding directions. It has a dial and a magnetic needle that always points to the north.
 \textit{
	\begin{itemize}
	\item We had to rely on a compass and a lot of luck to get here.
	\end{itemize}
}
\item plural noun \\
\textbf{Compasses} are a hinged  V-shaped instrument that you use for drawing circles.
 \textit{
	\begin{itemize}
	\end{itemize}
}
\item countable noun \\
If something is within \textbf{the}  \textbf{compass}  \textbf{of} something or someone, it is within their limits or abilities .
 \textit{
	\begin{itemize}
	\item Within the compass of a normal sized book such a comprehensive survey was not practicable.
	\end{itemize}
}
\end{enumerate}

\section*{coordinate}
{\large \color{blue}  }
\subsection*{Explain}
\begin{enumerate}
\item verb \\
1.  2.  3.  4.  5.  \textit{
	\begin{itemize}
	\end{itemize}
}
\item noun \\
6.  7.  \textit{
	\begin{itemize}
	\end{itemize}
}
\item adjective \\
8.  9.  10.  \textit{
	\begin{itemize}
	\item coordinate geometry
	\end{itemize}
}
\end{enumerate}

\section*{diagram}
{\large \color{blue}  diagrams  diagramming  diagrammed  }
\subsection*{Explain}
\begin{enumerate}
\item countable noun \\
A \textbf{diagram} is a simple  drawing which consists mainly of lines and is used, for example , to explain how a machine works.
 \textit{
	\begin{itemize}
	\item ...a circuit diagram.
	\item You can reduce long explanations to simple charts or diagrams.
	\end{itemize}
}
\item verb \\
To \textbf{diagram} something means to draw a diagram of it or to explain it using a diagram.
 \textit{
	\begin{itemize}
	\item The sound waves of the voice could be diagramed as in B.
	\item ...diagramming the movement of a system's variables.
	\end{itemize}
}
\end{enumerate}

\section*{downward}
{\large \color{blue}  }
\subsection*{Explain}
\begin{enumerate}
\item adjective \\
A \textbf{downward}  movement or look is directed towards a lower place or a lower level.
 \textit{
	\begin{itemize}
	\item ...a firm downward movement of the hands.
	\end{itemize}
}
\item adjective \\
If you refer to a \textbf{downward}  trend , you mean that something is decreasing or that a situation is getting  worse .
 \textit{
	\begin{itemize}
	\item The downward trend in home ownership is likely to continue.
	\item ...a decline in the economy, resulting in a general downward spiral.
	\end{itemize}
}
\end{enumerate}

\section*{flap}
{\large \color{blue}  flaps  flapping  flapped  }
\subsection*{Explain}
\begin{enumerate}
\item verb \\
If something such as a piece of cloth or paper \textbf{flaps} or if you \textbf{flap} it, it moves quickly up and down or from side to side.
 \textit{
	\begin{itemize}
	\item Grey sheets flapped on the clothes line.
	\item They would flap bath towels from their balconies as they chatted.
	\end{itemize}
}
\item verb \\
If a bird or insect  \textbf{flaps} its wings or if its wings \textbf{flap} , the wings move quickly up and down.
 \textit{
	\begin{itemize}
	\item The bird flapped its wings furiously.
	\item A pigeon emerges, wings flapping noisily, from the tower.
	\end{itemize}
}
\item verb \\
If you \textbf{flap} your arms, you move them quickly up and down as if they were the wings of a bird.
 \textit{
	\begin{itemize}
	\item ...a kid running and flapping her arms.
	\end{itemize}
}
\item countable noun \\
A \textbf{flap} of cloth or skin, for example , is a flat piece of it that can move freely up and down or from side to side because
it is held or attached by only one edge.
 \textit{
	\begin{itemize}
	\item He drew back the tent flap and strode out into the blizzard.
	\item ...a loose flap of skin.
	\end{itemize}
}
\item countable noun \\
A \textbf{flap} on the wing of an aircraft is an area along the edge of the wing that can be raised or lowered to control the movement of the aircraft.
 \textit{
	\begin{itemize}
	\item ...the sudden slowing as the flaps were lowered.
	\end{itemize}
}
\item countable noun \\
A \textbf{flap} is a sudden noise or movement made by a bird's wing or by a piece of paper or cloth when it flaps.
 \textit{
	\begin{itemize}
	\item Nothing to be heard but the soft flap of a silk banner.
	\end{itemize}
}
\item singular noun \\
Someone who is \textbf{in}  \textbf{a flap} is in a state of great excitement , worry , or panic.
 \textit{
	\begin{itemize}
	\item Why did people get in a flap over nuclear energy?
	\item Wherever he goes there's always a flap.
	\end{itemize}
}
\end{enumerate}

\section*{early}
{\large \color{blue}  earlier  earliest  }
\subsection*{Explain}
\begin{enumerate}
\item adverb \\
\textbf{Early}  means before the usual time that a particular event or activity  happens .
 \textbf{Early} is also an adjective .
 \textit{
	\begin{itemize}
	\item I knew I had to get up early.
	\item Why do we have to go to bed so early?
	\item I decided that I was going to take early retirement.
	\item I planned an early night.
	\end{itemize}
}
\item adjective \\
\textbf{Early} means near the beginning of a day , week , year , or other period of time.
 \textbf{Early} is also an adverb .
 \textit{
	\begin{itemize}
	\item ...in the 1970s and the early 1980s.
	\item ...a few weeks in early summer.
	\item She was in her early teens.
	\item ...the early hours of Saturday morning.
	\item We'll hope to see you some time early next week.
	\item ...early in the season.
	\end{itemize}
}
\item adverb \\
\textbf{Early} means before the time that was arranged or expected.
 \textbf{Early} is also an adjective.
 \textit{
	\begin{itemize}
	\item She arrived early to secure a place at the front.
	\item The first snow came a month earlier than usual.
	\item I'm always early.
	\end{itemize}
}
\item adjective \\
\textbf{Early} means near the beginning of a period in history , or in the history of something such as the world, a society , or an activity.
 \textit{
	\begin{itemize}
	\item ...the early stages of pregnancy.
	\item ...Fassbinder's early films.
	\item ...the early days of the occupation.
	\item It's too early to declare his efforts a success.
	\end{itemize}
}
\item adjective \\
\textbf{Early} means near the beginning of something such as a piece of work or a process.
 \textbf{Early} is also an adverb.
 \textit{
	\begin{itemize}
	\item ...the book's early chapters.
	\item ...an incident which occurred much earlier in the game.
	\end{itemize}
}
\item adjective \\
\textbf{Early}  refers to plants which flower or crop before or at the beginning of the main  season .
 \textbf{Early} is also an adverb.
 \textit{
	\begin{itemize}
	\item ...these early cabbages and cauliflowers.
	\item ...early flowering shrubs.
	\end{itemize}
}
\item adjective \\
\textbf{Early}  reports or indications of something are the first reports or indications about it.
 \textit{
	\begin{itemize}
	\item The early indications look encouraging.
	\item Earlier reports that troops opened fire are now being denied.
	\end{itemize}
}
\item  \\
 as early as \textit{
	\begin{itemize}
	\end{itemize}
}
\item  \\
 it's early days \textit{
	\begin{itemize}
	\end{itemize}
}
\end{enumerate}

\section*{elder}
{\large \color{blue}  elders  }
\subsection*{Explain}
\begin{enumerate}
\item adjective \\
\textbf{The}  \textbf{elder}  \textbf{of} two people is the one who was born first.
 \textit{
	\begin{itemize}
	\item ...his elder brother.
	\item ...the elder of her two daughters.
	\end{itemize}
}
\item countable noun \\
A person's \textbf{elder} is someone who is older than them, especially someone quite a lot older.
 \textit{
	\begin{itemize}
	\item The young have no respect for their elders.
	\end{itemize}
}
\item countable noun \\
In some societies , an \textbf{elder} is one of the respected older people who have influence and authority.
 \textit{
	\begin{itemize}
	\item ...tribal elders.
	\end{itemize}
}
\item countable noun \\
In some Christian churches, an \textbf{elder} is one of the people who hold a position of responsibility , but not usually a minister .
 \textit{
	\begin{itemize}
	\item He is now an elder of the village church.
	\end{itemize}
}
\item variable noun \\
An \textbf{elder} is a bush or small tree which has groups of small white flowers and black berries .
 \textit{
	\begin{itemize}
	\end{itemize}
}
\end{enumerate}

\section*{instance}
{\large \color{blue}  instances  }
\subsection*{Explain}
\begin{enumerate}
\item phrase \\
You use \textbf{for instance} to introduce a particular event, situation , or person that is an example of what you are talking about.
 \textit{
	\begin{itemize}
	\item There are a number of improvements; for instance, both mouse buttons can now be used.
	\item Straining to lift heavy weights for instance can lead to a rise in blood pressure
whilst the activity continues.
	\end{itemize}
}
\item countable noun \\
An \textbf{instance} is a particular example or occurrence of something.
 \textit{
	\begin{itemize}
	\item She cited an instance where their training had been a marvelous help in dealing with
problems.
	\item ...an investigation into a serious instance of corruption.
	\end{itemize}
}
\item  \\
 in the first instance \textit{
	\begin{itemize}
	\end{itemize}
}
\item  \\
 at sbs instance \textit{
	\begin{itemize}
	\end{itemize}
}
\end{enumerate}

\section*{elegant}
{\large \color{blue}  }
\subsection*{Explain}
\begin{enumerate}
\item adjective \\
If you describe a person or thing as \textbf{elegant} , you mean that they are pleasing and graceful in appearance or style.
 \textit{
	\begin{itemize}
	\item Patricia looked beautiful and elegant as always.
	\item ...an elegant restaurant.
	\end{itemize}
}
\item adjective \\
If you describe a piece of writing , an idea , or a plan as \textbf{elegant} , you mean that it is simple, clear , and clever .
 \textit{
	\begin{itemize}
	\item The document impressed me with its elegant simplicity.
	\end{itemize}
}
\end{enumerate}

\section*{intellectual}
{\large \color{blue}  intellectuals  }
\subsection*{Explain}
\begin{enumerate}
\item adjective \\
\textbf{Intellectual} means involving a person's ability to think and to understand  ideas and information .
 \textit{
	\begin{itemize}
	\item High levels of lead could damage the intellectual development of children.
	\item He has written seven thrillers, and clearly enjoys intellectual pursuits.
	\end{itemize}
}
\item countable noun \\
An \textbf{intellectual} is someone who spends a lot of time studying and thinking about complicated ideas.
 \textbf{Intellectual} is also an adjective .
 \textit{
	\begin{itemize}
	\item ...teachers, artists and other intellectuals.
	\item They were very intellectual and witty.
	\item ...an intellectual elite.
	\end{itemize}
}
\end{enumerate}

\section*{learning}
{\large \color{blue}  }
\subsection*{Explain}
\begin{enumerate}
\item uncountable noun \\
\textbf{Learning} is the process of gaining knowledge through studying.
 \textit{
	\begin{itemize}
	\item The brochure described the library as the focal point of learning on the campus.
	\end{itemize}
}
\end{enumerate}

\section*{gorgeous}
{\large \color{blue}  }
\subsection*{Explain}
\begin{enumerate}
\item adjective \\
If you say that something is \textbf{gorgeous} , you mean that it gives you a lot of pleasure or is very attractive .
 \textit{
	\begin{itemize}
	\item ...gorgeous mountain scenery.
	\item It's a gorgeous day.
	\item Some of the Renaissance buildings are gorgeous.
	\end{itemize}
}
\item adjective \\
If you describe someone as \textbf{gorgeous} , you mean that you find them very sexually attractive.
 \textit{
	\begin{itemize}
	\item The cosmetics industry uses gorgeous women to sell its skincare products.
	\item All the girls in my house are mad about Ryan, they think he's gorgeous.
	\end{itemize}
}
\item adjective \\
If you describe things such as clothes and colours as \textbf{gorgeous} , you mean they are bright , rich , and impressive .
 \textit{
	\begin{itemize}
	\item ...a red-haired man in the gorgeous uniform of a Marshal of the Empire.
	\end{itemize}
}
\end{enumerate}

\section*{librarian}
{\large \color{blue}  librarians  }
\subsection*{Explain}
\begin{enumerate}
\item countable noun \\
A \textbf{librarian} is a person who is in charge of a library or who has been specially trained to work in a library.
 \textit{
	\begin{itemize}
	\end{itemize}
}
\end{enumerate}

\section*{graceful}
{\large \color{blue}  }
\subsection*{Explain}
\begin{enumerate}
\item adjective \\
Someone or something that is \textbf{graceful} moves in a smooth and controlled way which is attractive to watch .
 \textit{
	\begin{itemize}
	\item His movements were so graceful they seemed effortless.
	\item ...graceful ballerinas.
	\end{itemize}
}
\item adjective \\
Something that is \textbf{graceful} is attractive because it has a pleasing shape or style.
 \textit{
	\begin{itemize}
	\item A graceful medieval cathedral.
	\item His handwriting, from earliest young manhood, was flowing and graceful.
	\end{itemize}
}
\item adjective \\
If a person's behaviour is \textbf{graceful} , it is polite , kind , and pleasant , especially in a difficult  situation .
 \textit{
	\begin{itemize}
	\item Aubrey could think of no graceful way to escape Corbet's company.
	\item He was charming, cheerful, and graceful under pressure.
	\end{itemize}
}
\end{enumerate}

\section*{library}
{\large \color{blue}  libraries  }
\subsection*{Explain}
\begin{enumerate}
\item countable noun \\
A public  \textbf{library} is a building where things such as books, newspapers , videos , and music are kept for people to read , use, or borrow .
 \textit{
	\begin{itemize}
	\item ...the local library.
	\item She issued them library cards.
	\end{itemize}
}
\item countable noun \\
A private  \textbf{library} is a collection of things such as books or music, that is normally only used with the permission of the owner .
 \textit{
	\begin{itemize}
	\item My thanks go to the British School of Osteopathy, for the use of their library.
	\end{itemize}
}
\item countable noun \\
In some large houses \textbf{the}  \textbf{library} is the room where most of the books are kept.
 \textit{
	\begin{itemize}
	\item Guests were rarely entertained in the library.
	\end{itemize}
}
\end{enumerate}

\section*{matter}
{\large \color{blue}  matters  mattering  mattered  }
\subsection*{Explain}
\begin{enumerate}
\item countable noun \\
A \textbf{matter} is a task , situation , or event which you have to deal with or think about, especially one that involves problems .
 \textit{
	\begin{itemize}
	\item It was clear that she wanted to discuss some private matter.
	\item Until the matter is resolved, the athletes will be ineligible to compete.
	\item Don't you think this is now a matter for the police?
	\item Business matters drew him to Paris.
	\end{itemize}
}
\item plural noun \\
You use \textbf{matters} to refer to the situation you are talking about, especially when something is affecting the situation in some way.
 \textit{
	\begin{itemize}
	\item The new system should improve matters.
	\item If it would facilitate matters, I would be happy to come to New York.
	\item Matters took an unexpected turn.
	\end{itemize}
}
\item singular noun \\
If you say that a situation is \textbf{a}  \textbf{matter}  \textbf{of} a particular thing, you mean that that is the most important thing to be done or considered when you are involved in the situation or explaining it.
 \textit{
	\begin{itemize}
	\item History is always a matter of interpretation.
	\item Observance of the law is a matter of principle for us.
	\item After that, life became a matter of defying school rules.
	\item Jack had attended these meetings as a matter of routine for years.
	\end{itemize}
}
\item uncountable noun \\
Printed  \textbf{matter} consists of books, newspapers , and other texts that are printed. Reading  \textbf{matter} consists of things that are suitable for reading, such as books and newspapers.
 \textit{
	\begin{itemize}
	\item Better education created an ever-larger demand for printed matter.
	\item ...a rich variety of reading matter.
	\end{itemize}
}
\item uncountable noun \\
\textbf{Matter} is the physical part of the universe consisting of solids , liquids, and gases .
 \textit{
	\begin{itemize}
	\item A proton is an elementary particle of matter.
	\item He has spent his career studying how matter behaves.
	\end{itemize}
}
\item uncountable noun \\
You use \textbf{matter} to refer to a particular type of substance.
 \textit{
	\begin{itemize}
	\item They feed mostly on decaying vegetable matter.
	\item ...waste matter from industries.
	\end{itemize}
}
\item singular noun \\
You use \textbf{matter} in expressions such as ' \textbf{What's the matter?} ' or ' \textbf{Is anything the matter?} ' when you think that someone has a problem and you want to know what it is.
 \textit{
	\begin{itemize}
	\item Carole, what's the matter? You don't seem happy.
	\item What's the matter with your office?
	\item She told him there was nothing the matter.
	\end{itemize}
}
\item singular noun \\
You use \textbf{matter} in expressions such as ' \textbf{a matter of weeks} ' when you are emphasizing how small an amount is or how short a period of time is.
 \textit{
	\begin{itemize}
	\item Within a matter of days she was back at work.
	\item He expected to be at East Grinstead station in a matter of hours.
	\item This time the journey was short, a matter of four or five miles up into the hills.
	\end{itemize}
}
\item verb \\
If you say that something does not \textbf{matter} , you mean that it is not important to you because it does not have an effect on you
or on a particular situation.
 \textit{
	\begin{itemize}
	\item A lot of the food goes on the floor but that doesn't matter.
	\item As for Laura and me, the colour of our skin has never mattered.
	\item As long as staff are smart, it does not matter how long their hair is.
	\item Does it matter that people don't know this?
	\item Money is the only thing that matters to them.
	\end{itemize}
}
\item  \\
 another matter/a different matter \textit{
	\begin{itemize}
	\end{itemize}
}
\item  \\
 as a matter of \textit{
	\begin{itemize}
	\end{itemize}
}
\item  \\
 no easy matter \textit{
	\begin{itemize}
	\end{itemize}
}
\item  \\
 that's the end of the matter/that's an end to the matter \textit{
	\begin{itemize}
	\end{itemize}
}
\item  \\
 the fact of the matter/the truth of the matter \textit{
	\begin{itemize}
	\end{itemize}
}
\item  \\
 for that matter \textit{
	\begin{itemize}
	\end{itemize}
}
\item  \\
 it doesn't matter \textit{
	\begin{itemize}
	\end{itemize}
}
\item  \\
 it doesn't matter \textit{
	\begin{itemize}
	\end{itemize}
}
\item  \\
 no laughing matter \textit{
	\begin{itemize}
	\end{itemize}
}
\item  \\
 make matters worse \textit{
	\begin{itemize}
	\end{itemize}
}
\item  \\
 no matter \textit{
	\begin{itemize}
	\end{itemize}
}
\item  \\
 no matter \textit{
	\begin{itemize}
	\end{itemize}
}
\item  \\
 no matter what \textit{
	\begin{itemize}
	\end{itemize}
}
\item  \\
 a matter of opinion \textit{
	\begin{itemize}
	\end{itemize}
}
\item  \\
 a matter of time \textit{
	\begin{itemize}
	\end{itemize}
}
\end{enumerate}

\section*{moan}
{\large \color{blue}  moans  moaning  moaned  }
\subsection*{Explain}
\begin{enumerate}
\item verb \\
If you \textbf{moan} , you make a low sound, usually because you are unhappy or in pain .
 \textbf{Moan} is also a noun .
 \textit{
	\begin{itemize}
	\item Tony moaned in his sleep and then turned over on his side.
	\item 'My head, my head,' he moaned. 'I can't see.'
	\item Suddenly she gave a low, choking moan and began to tremble violently.
	\item ...her moan of sorrow.
	\end{itemize}
}
\item verb \\
To \textbf{moan}  means to complain or speak in a way which shows that you are very unhappy.
 \textit{
	\begin{itemize}
	\item I used to moan if I didn't get at least six hours' sleep at night.
	\item ...moaning about the weather.
	\item They moan on a lot about money.
	\item Meg moans, 'I hated it!'
	\item The gardener was moaning that he had another garden to do later that morning.
	\end{itemize}
}
\item countable noun \\
A \textbf{moan} is a complaint.
 \textit{
	\begin{itemize}
	\item They have been listening to people's gripes, moans and praise.
	\end{itemize}
}
\item  \\
 have a moan \textit{
	\begin{itemize}
	\end{itemize}
}
\item countable noun \\
A \textbf{moan} is a low noise .
 \textit{
	\begin{itemize}
	\item ...the occasional moan of the wind round the corners of the house.
	\item ...the moan of distant traffic.
	\end{itemize}
}
\end{enumerate}

\section*{latter}
{\large \color{blue}  }
\subsection*{Explain}
\begin{enumerate}
\item pronoun \\
When two people, things, or groups have just been mentioned, you can refer to the second of them as \textbf{the latter} .
 \textbf{Latter} is also an adjective .
 \textit{
	\begin{itemize}
	\item He tracked down his cousin and uncle. The latter was sick.
	\item There are the people who speak after they think and the people who think while they're
speaking. Mike definitely belongs in the latter category.
	\end{itemize}
}
\item adjective \\
You use \textbf{latter} to describe the later part of a period of time or event .
 \textit{
	\begin{itemize}
	\item He is getting into the latter years of his career.
	\item The latter part of the debate concentrated on abortion.
	\end{itemize}
}
\end{enumerate}

\section*{object}
{\large \color{blue}  objects  objecting  objected  }
\subsection*{Explain}
\begin{enumerate}
\item countable noun \\
An \textbf{object} is anything that has a fixed shape or form, that you can touch or see, and that is not alive .
 \textit{
	\begin{itemize}
	\item He squinted his eyes as though he were studying an object on the horizon.
	\item ...an object the shape of a coconut.
	\item In the cosy consulting room the children are surrounded by familiar objects.
	\end{itemize}
}
\item countable noun \\
The \textbf{object} of what someone is doing is their aim or purpose.
 \textit{
	\begin{itemize}
	\item The object of the exercise is to raise money for the charity.
	\item He made it his object in life to find the island.
	\item My object was to publish a scholarly work on Peter Mourne.
	\end{itemize}
}
\item countable noun \\
The \textbf{object of} a particular feeling or reaction is the person or thing it is directed towards or that causes it.
 \textit{
	\begin{itemize}
	\item The object of her hatred was a 24-year-old model.
	\item The object of great interest at the Temple was a large marble tower built in memory
of Buddha.
	\item She knew that she was an object of pity among her friends.
	\end{itemize}
}
\item countable noun \\
In grammar , the \textbf{object} of a verb or a preposition is the word or phrase which completes the structure begun by the verb or preposition.
 \textit{
	\begin{itemize}
	\end{itemize}
}
\item verb \\
If you \textbf{object} to something, you express your dislike or disapproval of it.
 \textit{
	\begin{itemize}
	\item A lot of people will object to the book.
	\item Cullen objected that his small staff would be unable to handle the added work.
	\item We objected strongly but were outvoted.
	\item 'Hey, I don't know what you're talking about,' Russo objected.
	\end{itemize}
}
\item  \\
 money is no object \textit{
	\begin{itemize}
	\end{itemize}
}
\end{enumerate}

\section*{magnificent}
{\large \color{blue}  }
\subsection*{Explain}
\begin{enumerate}
\item adjective \\
If you say that something or someone is \textbf{magnificent} , you mean that you think they are extremely good, beautiful , or impressive.
 \textit{
	\begin{itemize}
	\item ...a magnificent country house in wooded grounds.
	\item ...magnificent views over the San Fernando Valley.
	\item She is magnificent at making you feel you can talk quite naturally to her.
	\end{itemize}
}
\end{enumerate}

\section*{owner}
{\large \color{blue}  owners  }
\subsection*{Explain}
\begin{enumerate}
\item countable noun \\
The \textbf{owner} of something is the person to whom it belongs.
 \textit{
	\begin{itemize}
	\item The owner of the store was sweeping his floor when I walked in.
	\item Every pet owner knows their animal has its own personality.
	\item New owners will have to wait until September before moving in.
	\end{itemize}
}
\end{enumerate}

\section*{old}
{\large \color{blue}  older  oldest  }
\subsection*{Explain}
\begin{enumerate}
\item adjective \\
Someone who is \textbf{old} has lived for many years and is no longer young.
 \textbf{The old} are people who are old. This use could cause offence .
 \textit{
	\begin{itemize}
	\item ...a white-haired old man.
	\item He was considered too old for the job.
	\item ...providing a caring response for the needs of the old.
	\end{itemize}
}
\item adjective \\
You use \textbf{old} to talk about how many days, weeks , months , or years someone or something has lived or existed.
 \textit{
	\begin{itemize}
	\item He was abandoned by his father when he was three months old.
	\item The paintings in the chapel were perhaps a thousand years old.
	\item How old are you now?
	\item These weren't young kids, they were as old as I was.
	\item Bill was six years older than David.
	\end{itemize}
}
\item adjective \\
Something that is \textbf{old} has existed for a long time.
 \textit{
	\begin{itemize}
	\item She loved the big old house.
	\item These books must be very old.
	\item ...an old Arab proverb.
	\item ...her old habit of criticizing his speech.
	\end{itemize}
}
\item adjective \\
Something that is \textbf{old} is no longer in good condition because of its age or because it has been used a lot .
 \textit{
	\begin{itemize}
	\item He took a bunch of keys from the pocket of his old corduroy trousers.
	\item ...an old toothbrush.
	\end{itemize}
}
\item adjective \\
You use \textbf{old} to refer to something that is no longer used, that no longer exists, or that has been replaced by something else.
 \textit{
	\begin{itemize}
	\item The old road had disappeared under grass and heather.
	\item Although the old secret police have been abolished, the military police still exist.
	\item ...avoiding the corruption and ineffectiveness of the old parties.
	\end{itemize}
}
\item adjective \\
You use \textbf{old} to refer to something that used to belong to you, or to a person or thing that used
to have a particular role in your life.
 \textit{
	\begin{itemize}
	\item I'll make up the bed in your old room.
	\item I still have affection for my old school.
	\item Mark was heartbroken when Jane returned to her old boyfriend.
	\end{itemize}
}
\item adjective \\
An \textbf{old}  friend , enemy , or rival is someone who has been your friend, enemy, or rival for a long time.
 \textit{
	\begin{itemize}
	\item I called my old friend John Horner.
	\item Mr Brownson, I assure you King's an old enemy of mine.
	\item The French and English are old rivals.
	\end{itemize}
}
\item adjective \\
You can use \textbf{old} to express affection when talking to or about someone you know .
 \textit{
	\begin{itemize}
	\item Are you all right, old chap?
	\item Good old Bergen would do him the favor.
	\end{itemize}
}
\item  \\
 any old \textit{
	\begin{itemize}
	\end{itemize}
}
\item  \\
 in the old days \textit{
	\begin{itemize}
	\end{itemize}
}
\item  \\
 the good old days \textit{
	\begin{itemize}
	\end{itemize}
}
\item  \\
 of old \textit{
	\begin{itemize}
	\end{itemize}
}
\end{enumerate}

\section*{pasture}
{\large \color{blue}  pastures  }
\subsection*{Explain}
\begin{enumerate}
\item variable noun \\
\textbf{Pasture} is land with grass growing on it for farm animals to eat .
 \textit{
	\begin{itemize}
	\item The cows are out now, grazing in the pasture.
	\item ...mountain pastures.
	\end{itemize}
}
\item  \\
 pastures new \textit{
	\begin{itemize}
	\end{itemize}
}
\item  \\
 put sth out to pasture \textit{
	\begin{itemize}
	\end{itemize}
}
\item  \\
 put sb out to pasture \textit{
	\begin{itemize}
	\end{itemize}
}
\end{enumerate}

\section*{orthodox}
{\large \color{blue}  }
\subsection*{Explain}
\begin{enumerate}
\item adjective \\
\textbf{Orthodox}  beliefs , methods , or systems are ones which are accepted or used by most people.
 \textit{
	\begin{itemize}
	\item Payne gained a reputation for sound, if orthodox, views.
	\item Many of these ideas are now being incorporated into orthodox medical treatment.
	\item ...orthodox police methods.
	\end{itemize}
}
\item adjective \\
If you describe someone as \textbf{orthodox} , you mean that they hold the older and more traditional  ideas of their religion or party.
 \textit{
	\begin{itemize}
	\item ...orthodox Jews.
	\item ...orthodox communists.
	\end{itemize}
}
\item adjective \\
The \textbf{Orthodox} churches are Christian churches in Eastern  Europe which separated from the western church in the eleventh century.
 \textit{
	\begin{itemize}
	\item ...the Greek Orthodox Church.
	\end{itemize}
}
\end{enumerate}

\section*{peace}
{\large \color{blue}  }
\subsection*{Explain}
\begin{enumerate}
\item uncountable noun \\
If countries or groups involved in a war or violent  conflict are discussing  \textbf{peace} , they are talking to each other in order to try to end the conflict.
 \textit{
	\begin{itemize}
	\item Peace talks broke up without agreement last week.
	\item Leaders of some rival factions signed a peace agreement last week.
	\item They hope the treaty will bring peace and stability to the region.
	\end{itemize}
}
\item uncountable noun \\
If there is \textbf{peace} in a country or in the world, there are no wars or violent conflicts going on.
 \textit{
	\begin{itemize}
	\item The President spoke of a shared commitment to world peace and economic development.
	\item ...the Nobel Peace Prize.
	\end{itemize}
}
\item uncountable noun \\
If you disapprove of weapons , especially  nuclear weapons, you can use \textbf{peace} to refer to campaigns and other activities intended to reduce their numbers or stop their use.
 \textit{
	\begin{itemize}
	\item Two peace campaigners were accused of causing damage to an F1 11 nuclear bomber.
	\item He campaigned for peace and against the spread of nuclear weapons.
	\end{itemize}
}
\item uncountable noun \\
If you have \textbf{peace} , you are not being disturbed , and you are in calm , quiet surroundings.
 \textit{
	\begin{itemize}
	\item All I want is to have some peace and quiet.
	\item One more question and I'll leave you in peace.
	\end{itemize}
}
\item uncountable noun \\
If you have a feeling of \textbf{peace} , you feel  contented and calm and not at all worried . You can also  say that you are \textbf{at peace} .
 \textit{
	\begin{itemize}
	\item I had a wonderful feeling of peace and serenity when I saw him.
	\item The peace of the Lord be always with you.
	\item I know you will never be at peace until you have discovered where your brother is.
	\end{itemize}
}
\item uncountable noun \\
If there is \textbf{peace} among a group of people, they live or work together in a friendly way and do not quarrel . You can also say that people live or work \textbf{in peace with} each other.
 \textit{
	\begin{itemize}
	\item ...a period of relative peace in the country's industrial relations.
	\item If you can't live in peace with your little brother then get out of the house.
	\end{itemize}
}
\item countable noun \\
\textbf{The}  \textbf{Peace of} a particular place is a treaty or an agreement that was signed there, bringing an end to a war.
 \textit{
	\begin{itemize}
	\item The Peace of Ryswick was signed in September 1697.
	\end{itemize}
}
\item  \\
 to hold your peace \textit{
	\begin{itemize}
	\end{itemize}
}
\item  \\
 to keep the peace \textit{
	\begin{itemize}
	\end{itemize}
}
\item  \\
 to keep the peace \textit{
	\begin{itemize}
	\end{itemize}
}
\item  \\
 make (one's) peace \textit{
	\begin{itemize}
	\end{itemize}
}
\item  \\
 peace of mind \textit{
	\begin{itemize}
	\end{itemize}
}
\item  \\
 to rest in peace \textit{
	\begin{itemize}
	\end{itemize}
}
\item  \\
 at peace with \textit{
	\begin{itemize}
	\end{itemize}
}
\end{enumerate}

\section*{plausible}
{\large \color{blue}  }
\subsection*{Explain}
\begin{enumerate}
\item adjective \\
An explanation or statement that is \textbf{plausible}  seems  likely to be true or valid.
 \textit{
	\begin{itemize}
	\item A more plausible explanation would seem to be that people are fed up with the Conservative
government.
	\item That explanation seems entirely plausible to me.
	\end{itemize}
}
\item adjective \\
If you say that someone is \textbf{plausible} , you mean that they seem to be telling the truth and to be sincere and honest .
 \textit{
	\begin{itemize}
	\item He was so plausible that he conned everybody.
	\end{itemize}
}
\end{enumerate}

\section*{physicist}
{\large \color{blue}  physicists  }
\subsection*{Explain}
\begin{enumerate}
\item countable noun \\
A \textbf{physicist} is a person who does research  connected with physics or who studies physics.
 \textit{
	\begin{itemize}
	\item ...a nuclear physicist.
	\end{itemize}
}
\end{enumerate}

\section*{previous}
{\large \color{blue}  }
\subsection*{Explain}
\begin{enumerate}
\item adjective \\
A \textbf{previous} event or thing is one that happened or existed before the one that you are talking about.
 \textit{
	\begin{itemize}
	\item I'm a lot happier than I was in my previous job.
	\item He has no previous convictions.
	\end{itemize}
}
\item adjective \\
You refer to the period of time or the thing immediately before the one that you are talking about as the \textbf{previous} one.
 \textit{
	\begin{itemize}
	\item It was a surprisingly dry day after the rain of the previous week.
	\item He recalled exactly what Bob had told him the previous night.
	\end{itemize}
}
\end{enumerate}

\section*{physics}
{\large \color{blue}  }
\subsection*{Explain}
\begin{enumerate}
\item uncountable noun \\
\textbf{Physics} is the scientific study of forces such as heat, light, sound, pressure , gravity , and electricity, and the way that they affect objects.
 \textit{
	\begin{itemize}
	\item ...the laws of physics.
	\item ...experiments in particle physics.
	\end{itemize}
}
\end{enumerate}

\section*{prior}
{\large \color{blue}  priors  }
\subsection*{Explain}
\begin{enumerate}
\item adjective \\
You use \textbf{prior} to indicate that something has already  happened , or must happen, before another event takes place.
 \textit{
	\begin{itemize}
	\item He claimed he had no prior knowledge of the protest.
	\item The Constitution requires the president to seek the prior approval of Congress for
military action.
	\item For the prior year, they reported net income of $1.1 million.
	\end{itemize}
}
\item adjective \\
A \textbf{prior}  claim or duty is more important than other claims or duties and needs to be dealt with first.
 \textit{
	\begin{itemize}
	\item The firm I wanted to use had prior commitments.
	\end{itemize}
}
\item countable noun \\
A \textbf{prior} is a monk who is in charge of a priory or a monk who is the second most important person in a monastery.
 \textit{
	\begin{itemize}
	\end{itemize}
}
\item  \\
 prior to sth \textit{
	\begin{itemize}
	\end{itemize}
}
\end{enumerate}

\section*{pope}
{\large \color{blue}  popes  }
\subsection*{Explain}
\begin{enumerate}
\item countable noun \\
\textbf{The}  \textbf{Pope} is the head of the Roman Catholic Church.
 \textit{
	\begin{itemize}
	\item ...the Pope's message to the people.
	\item ...Pope John Paul II.
	\end{itemize}
}
\end{enumerate}

\section*{rigid}
{\large \color{blue}  }
\subsection*{Explain}
\begin{enumerate}
\item adjective \\
Laws, rules , or systems that are \textbf{rigid} cannot be changed or varied , and are therefore considered to be rather severe.
 \textit{
	\begin{itemize}
	\item Several colleges in our study have rigid rules about student conduct.
	\item Hospital routines for nurses are very rigid.
	\end{itemize}
}
\item adjective \\
If you disapprove of someone because you think they are not willing to change their way of thinking or behaving , you can describe them as \textbf{rigid} .
 \textit{
	\begin{itemize}
	\item She was a fairly rigid person who had strong religious views.
	\item My father is very rigid in his thinking.
	\end{itemize}
}
\item adjective \\
A \textbf{rigid} substance or object is stiff and does not bend, stretch , or twist  easily .
 \textit{
	\begin{itemize}
	\item ...rigid plastic containers.
	\item These plates are fairly rigid.
	\end{itemize}
}
\item graded adjective \\
If someone goes  \textbf{rigid} , their body becomes very straight and stiff, usually as a result of shock or fear .
 \textit{
	\begin{itemize}
	\item I went rigid with shock.
	\item Andrew went rigid when he saw a dog, any dog, anywhere.
	\end{itemize}
}
\end{enumerate}

\section*{provision}
{\large \color{blue}  provisions  }
\subsection*{Explain}
\begin{enumerate}
\item uncountable noun \\
The \textbf{provision}  \textbf{of} something is the act of giving it or making it available to people who need or want it.
 \textit{
	\begin{itemize}
	\item The department is responsible for the provision of residential care services.
	\item ...nursery provision for children with special needs.
	\end{itemize}
}
\item variable noun \\
If you make \textbf{provision for} something that might  happen or that might need to be done , you make arrangements to deal with it.
 \textit{
	\begin{itemize}
	\item Mr King asked if it had ever occurred to her to make provision for her own pension.
	\item There is no provision for funding performance-related pay rises.
	\end{itemize}
}
\item uncountable noun \\
If you make \textbf{provision for} someone, you support them financially and make sure that they have the things that they need.
 \textit{
	\begin{itemize}
	\item Special provision should be made for children.
	\item There are very generous provisions for the mother.
	\end{itemize}
}
\item countable noun \\
A \textbf{provision} in a law or an agreement is an arrangement which is included in it.
 \textit{
	\begin{itemize}
	\item There was a provision in his contract that would return him two-thirds of his deposit.
	\item The bill's provision for the sale and purchase of land faces stiff opposition.
	\end{itemize}
}
\item plural noun \\
\textbf{Provisions} are supplies of food.
 \textit{
	\begin{itemize}
	\item On board were enough provisions for two weeks.
	\end{itemize}
}
\end{enumerate}

\section*{senior}
{\large \color{blue}  seniors  }
\subsection*{Explain}
\begin{enumerate}
\item adjective \\
The \textbf{senior} people in an organization or profession have the highest and most important  jobs .
 \textit{
	\begin{itemize}
	\item ...senior officials in the Israeli government.
	\item ...the company's senior management.
	\item Television and radio needed many more women in senior jobs.
	\end{itemize}
}
\item adjective \\
If someone is \textbf{senior}  \textbf{to} you in an organization or profession, they have a higher and more important job than you or they are considered to be superior to you because they have worked there for longer and have more experience .
 Your \textbf{seniors} are the people who are senior to you.
 \textit{
	\begin{itemize}
	\item The position had to be filled by an officer senior to Haig.
	\item Williams felt himself to be senior to all of them.
	\item He was described by his seniors as a model officer.
	\end{itemize}
}
\item singular noun \\
\textbf{Senior} is used when indicating how much older one person is than another. For example , if someone is ten years your \textbf{senior} , they are ten years older than you.
 \textit{
	\begin{itemize}
	\item She became involved with a married man many years her senior.
	\end{itemize}
}
\item countable noun \\
\textbf{Seniors} are students in a high school, university, or college who are the oldest and who
have reached an advanced level in their studies .
 \textit{
	\begin{itemize}
	\end{itemize}
}
\item adjective \\
If you take part in a sport at \textbf{senior} level, you take part in competitions with adults and people who have reached a high degree of achievement in that sport.
 \textit{
	\begin{itemize}
	\item This will be his fifth international championship and his third at senior level.
	\end{itemize}
}
\item countable noun \\
In sports such as golf and tennis , a \textbf{senior} is a professional  player who is fairly old and who plays in special competitions against other older players.
 \textit{
	\begin{itemize}
	\end{itemize}
}
\end{enumerate}

\section*{ridge}
{\large \color{blue}  ridges  }
\subsection*{Explain}
\begin{enumerate}
\item countable noun \\
A \textbf{ridge} is a long, narrow piece of raised land.
 \textit{
	\begin{itemize}
	\end{itemize}
}
\item countable noun \\
A \textbf{ridge} is a raised line on a flat surface.
 \textit{
	\begin{itemize}
	\item ...the bony ridge of the eye socket.
	\end{itemize}
}
\end{enumerate}

\section*{rock}
{\large \color{blue}  rocks  rocking  rocked  }
\subsection*{Explain}
\begin{enumerate}
\item uncountable noun \\
\textbf{Rock} is the hard substance which the Earth is made of.
 \textit{
	\begin{itemize}
	\item The hills above the valley are bare rock.
	\item A little way below the ridge was an outcrop of rock that made a rough shelter.
	\end{itemize}
}
\item countable noun \\
A \textbf{rock} is a large piece of rock that sticks up out of the ground or the sea, or that has
broken away from a mountain or a cliff .
 \textit{
	\begin{itemize}
	\item She sat cross-legged on the rock.
	\item ...the sound of the sea crashing against the rocks.
	\item He and two friends were climbing a rock face when they heard cries for help.
	\end{itemize}
}
\item countable noun \\
A \textbf{rock} is a piece of rock that is small enough for you to pick up.
 \textit{
	\begin{itemize}
	\item She bent down, picked up a rock and threw it into the trees.
	\end{itemize}
}
\item verb \\
When something \textbf{rocks} or when you \textbf{rock} it, it moves slowly and regularly backwards and forwards or from side to side.
 \textit{
	\begin{itemize}
	\item His body rocked from side to side with the train.
	\item He stood a few moments, rocking back and forwards on his heels.
	\item She sat on the porch and rocked the baby.
	\end{itemize}
}
\item verb \\
If an explosion or an earthquake  \textbf{rocks} a building or an area, it causes the building or area to shake. You can also  say that the building or area \textbf{rocks} .
 \textit{
	\begin{itemize}
	\item Three people were injured yesterday when an explosion rocked one of Britain's best
known film studios.
	\item ...a country that's rocked by dozens of earthquakes every year.
	\item As the buildings rocked under heavy shell-fire, he took refuge in the cellars.
	\end{itemize}
}
\item verb \\
If an event or a piece of news  \textbf{rocks} a group or society, it shocks them or makes them feel less secure .
 \textit{
	\begin{itemize}
	\item His death rocked the fashion business.
	\item ...the latest scandal to rock the monarchy.
	\item Wall Street was rocked by the news and shares fell 4.3 per cent by the end of trading.
	\end{itemize}
}
\item uncountable noun \\
\textbf{Rock} is loud music with a strong beat that is usually played and sung by a small group of people using instruments such as electric  guitars and drums .
 \textit{
	\begin{itemize}
	\item He once told an interviewer that he didn't even like rock music.
	\item ...a rock concert.
	\item ...famous rock stars.
	\end{itemize}
}
\item uncountable noun \\
\textbf{Rock} is a sweet that is made in long, hard sticks and is often sold in towns by the sea
in Britain.
 \textit{
	\begin{itemize}
	\item ...a stick of rock.
	\end{itemize}
}
\item  \\
 to be caught between a rock and a hard place \textit{
	\begin{itemize}
	\end{itemize}
}
\item  \\
 on the rocks \textit{
	\begin{itemize}
	\end{itemize}
}
\item  \\
 on the rocks \textit{
	\begin{itemize}
	\end{itemize}
}
\end{enumerate}

\section*{smooth}
{\large \color{blue}  smoother  smoothest  smooths  smoothing  smoothed  }
\subsection*{Explain}
\begin{enumerate}
\item adjective \\
A \textbf{smooth} surface has no roughness, lumps , or holes.
 \textit{
	\begin{itemize}
	\item ...a rich cream that keeps skin soft and smooth.
	\item ...a smooth surface such as glass.
	\item The flagstones beneath their feet were worn smooth by centuries of use.
	\end{itemize}
}
\item adjective \\
A \textbf{smooth} liquid or mixture has been mixed  well so that it has no lumps.
 \textit{
	\begin{itemize}
	\item Continue whisking until the mixture looks smooth and creamy.
	\item Blend the cornflour to a smooth paste with a little cold water.
	\end{itemize}
}
\item adjective \\
If you describe a drink such as wine, whisky , or coffee as \textbf{smooth} , you mean that it is not bitter and is pleasant to drink.
 \textit{
	\begin{itemize}
	\item This makes the coffee much smoother.
	\end{itemize}
}
\item adjective \\
A \textbf{smooth} line or movement has no sudden breaks or changes in direction or speed .
 \textit{
	\begin{itemize}
	\item This exercise is done in one smooth motion.
	\item ...the smooth curve of the trunk.
	\end{itemize}
}
\item adjective \\
A \textbf{smooth}  ride , flight , or sea crossing is very comfortable because there are no unpleasant movements.
 \textit{
	\begin{itemize}
	\item The active suspension system gives the car a very smooth ride.
	\end{itemize}
}
\item adjective \\
You use \textbf{smooth} to describe something that is going well and is free of problems or trouble .
 \textit{
	\begin{itemize}
	\item Political hopes for a swift and smooth transition to democracy have been dashed.
	\item A number of problems marred the smooth running of this event.
	\end{itemize}
}
\item adjective \\
If you describe a man as \textbf{smooth} , you mean that he is extremely smart , confident , and polite , often in a way that you find rather unpleasant.
 \textit{
	\begin{itemize}
	\item Twelve extremely good-looking, smooth young men have been picked as finalists.
	\item He was the smoothest and probably the most powerful chief of staff in political memory.
	\end{itemize}
}
\item verb \\
If you \textbf{smooth} something, you move your hands over its surface to make it smooth and flat.
 \textit{
	\begin{itemize}
	\item She stood up and smoothed down her frock.
	\item Bardo smoothed his moustache.
	\end{itemize}
}
\item verb \\
If you \textbf{smooth} something somewhere , you use your hands to spread it there.
 \textit{
	\begin{itemize}
	\item She smoothed the lotion across his shoulder blades.
	\item His fingers smoothed the hair back from her face.
	\end{itemize}
}
\item  \\
 smooth the path/way \textit{
	\begin{itemize}
	\end{itemize}
}
\item  \\
 take the rough with the smooth \textit{
	\begin{itemize}
	\end{itemize}
}
\end{enumerate}

\section*{season}
{\large \color{blue}  seasons  seasoning  seasoned  }
\subsection*{Explain}
\begin{enumerate}
\item countable noun \\
The \textbf{seasons} are the main periods into which a year can be divided and which each have their own
 typical weather conditions.
 \textit{
	\begin{itemize}
	\item Autumn's my favourite season.
	\item ...the only region of Brazil where all four seasons are clearly defined.
	\item ...the rainy season.
	\end{itemize}
}
\item countable noun \\
You can use \textbf{season} to refer to the period during each year when a particular activity or event takes place. For
 example , the planting  \textbf{season} is the period when a particular plant or crop is planted.
 \textit{
	\begin{itemize}
	\item ...birds arriving for the breeding season.
	\item For law students, autumn brings the recruiting season.
	\end{itemize}
}
\item countable noun \\
You can use \textbf{season} to refer to the period when a particular fruit, vegetable , or other food is ready for eating and is widely available .
 \textit{
	\begin{itemize}
	\item The plum season is about to begin.
	\item Now British asparagus is in season.
	\end{itemize}
}
\item countable noun \\
You can use \textbf{season} to refer to a fixed period during each year when a particular sport is played.
 \textit{
	\begin{itemize}
	\item ...the baseball season.
	\item It is his first race this season.
	\end{itemize}
}
\item countable noun \\
A \textbf{season} is a period in which a play or show, or a series of plays or shows, is performed
in one place.
 \textit{
	\begin{itemize}
	\item ...a season of three new plays.
	\item ...the Royal Ballet's summer season.
	\end{itemize}
}
\item countable noun \\
A \textbf{season}  \textbf{of} films is several of them shown as a series because they are connected in some way.
 \textit{
	\begin{itemize}
	\item ...a season of films by America's preeminent documentary maker, Ken Burns.
	\end{itemize}
}
\item countable noun \\
The holiday or vacation  \textbf{season} is the time when most people have their holiday.
 \textit{
	\begin{itemize}
	\item ...the peak holiday season.
	\item There are discos and clubs but these are often closed out of season.
	\end{itemize}
}
\item verb \\
If you \textbf{season} food with salt, pepper, or spices, you add them to it in order to improve its flavour .
 \textit{
	\begin{itemize}
	\item Season the meat with salt and pepper.
	\item I believe in seasoning food before putting it on the table.
	\end{itemize}
}
\item verb \\
If wood \textbf{is seasoned} , it is made suitable for making into furniture or for burning , usually by being allowed to dry out gradually.
 \textit{
	\begin{itemize}
	\item Ensure that new wood has been seasoned.
	\end{itemize}
}
\item  \\
 in season \textit{
	\begin{itemize}
	\end{itemize}
}
\end{enumerate}

\section*{shepherd}
{\large \color{blue}  shepherds  shepherding  shepherded  }
\subsection*{Explain}
\begin{enumerate}
\item countable noun \\
A \textbf{shepherd} is a person, especially a man, whose job is to look after sheep.
 \textit{
	\begin{itemize}
	\end{itemize}
}
\item verb \\
If you \textbf{are shepherded}  somewhere , someone takes you there to make sure that you arrive at the right place safely.
 \textit{
	\begin{itemize}
	\item She was shepherded by her guards up the rear ramp of the aircraft.
	\end{itemize}
}
\end{enumerate}

\section*{spontaneous}
{\large \color{blue}  }
\subsection*{Explain}
\begin{enumerate}
\item adjective \\
\textbf{Spontaneous} acts are not planned or arranged , but are done because someone suddenly  wants to do them.
 \textit{
	\begin{itemize}
	\item Their spontaneous outbursts of song were accompanied by lively music.
	\item I joined in the spontaneous applause.
	\end{itemize}
}
\item adjective \\
A \textbf{spontaneous} event happens because of processes within something rather than being caused by things outside
it.
 \textit{
	\begin{itemize}
	\item I had another spontaneous miscarriage at around the 16th to 18th week.
	\item ...a spontaneous explosion.
	\end{itemize}
}
\end{enumerate}

\section*{storage}
{\large \color{blue}  }
\subsection*{Explain}
\begin{enumerate}
\item uncountable noun \\
If you refer to the \textbf{storage} of something, you mean that it is kept in a special place until it is needed .
 \textit{
	\begin{itemize}
	\item ...the storage of toxic waste.
	\item Some of the space will at first be used for storage.
	\item The collection has been in storage for decades.
	\end{itemize}
}
\item uncountable noun \\
\textbf{Storage} is the process of storing data in a computer.
 \textit{
	\begin{itemize}
	\item His task is to ensure the fair use and storage of personal information held on computer.
	\item ...data-storage devices.
	\end{itemize}
}
\end{enumerate}

\section*{superior}
{\large \color{blue}  superiors  }
\subsection*{Explain}
\begin{enumerate}
\item adjective \\
If one thing or person is \textbf{superior}  \textbf{to} another, the first is better than the second .
 \textit{
	\begin{itemize}
	\item We have a relationship infinitely superior to those of many of our friends.
	\item ...a woman greatly superior to her husband in education and sensitivity.
	\item Long-term stock market investments have produced superior returns compared with cash
deposits.
	\end{itemize}
}
\item adjective \\
If you describe something as \textbf{superior} , you mean that it is good, and better than other things of the same kind .
 \textit{
	\begin{itemize}
	\item A few years ago it was virtually impossible to find superior quality coffee in local
shops.
	\item Lulu was said to be of very superior intelligence.
	\end{itemize}
}
\item adjective \\
A \textbf{superior} person or thing is more important than another person or thing in the same organization or system.
 \textit{
	\begin{itemize}
	\item ...negotiations between the mutineers and their superior officers.
	\item Locally passed laws are of superior authority to those laws passed in the capital.
	\end{itemize}
}
\item countable noun \\
Your \textbf{superior} in an organization that you work for is a person who has a higher rank than you.
 \textit{
	\begin{itemize}
	\item Other army units are completely surrounded and cut-off from communication with their
superiors.
	\item The company president, and my immediate superior, was the dynamic Harry Stokes.
	\end{itemize}
}
\item adjective \\
If you describe someone as \textbf{superior} , you disapprove of them because they behave as if they are better, more important, or more intelligent than other people.
 \textit{
	\begin{itemize}
	\item Finch gave a superior smile.
	\item You can stand there and feel superior as you point and laugh at them.
	\end{itemize}
}
\item adjective \\
If one group of people has \textbf{superior}  numbers to another group, the first has more people than the second, and therefore has an
 advantage over it.
 \textit{
	\begin{itemize}
	\item The demonstrators fled when they saw the authorities' superior numbers.
	\item His men were far superior numerically.
	\end{itemize}
}
\item countable noun \\
If you describe someone as your \textbf{superior} in a particular activity , you mean that they are better than you at that activity.
 \textit{
	\begin{itemize}
	\item Anthony sometimes felt that his mistress was his superior in will-power.
	\item His rival was probably his superior in comic roles.
	\end{itemize}
}
\end{enumerate}

\section*{substance}
{\large \color{blue}  substances  }
\subsection*{Explain}
\begin{enumerate}
\item countable noun \\
A \textbf{substance} is a solid, powder , liquid , or gas with particular properties.
 \textit{
	\begin{itemize}
	\item Ethylene glycol is a poisonous substance found in antifreeze.
	\item The substance that's causing the problem comes from the barley.
	\end{itemize}
}
\item uncountable noun \\
\textbf{Substance} is the quality of being important or significant .
 \textit{
	\begin{itemize}
	\item It's questionable whether anything of substance has been achieved.
	\item Syria will attend only if the negotiations deal with issues of substance.
	\end{itemize}
}
\item singular noun \\
\textbf{The substance of} what someone says or writes is the main thing that they are trying to say.
 \textit{
	\begin{itemize}
	\item The substance of his discussions doesn't really matter.
	\end{itemize}
}
\item uncountable noun \\
If you say that something has no \textbf{substance} , you mean that it is not true .
 \textit{
	\begin{itemize}
	\item There is no substance in any of these allegations.
	\end{itemize}
}
\item  \\
 of substance \textit{
	\begin{itemize}
	\end{itemize}
}
\end{enumerate}

\section*{unanimous}
{\large \color{blue}  }
\subsection*{Explain}
\begin{enumerate}
\item adjective \\
When a group of people are \textbf{unanimous} , they all agree about something or all vote for the same thing.
 \textit{
	\begin{itemize}
	\item Editors were unanimous in their condemnation of the proposals.
	\item They were unanimous that Chortlesby Manor must be preserved.
	\end{itemize}
}
\item adjective \\
A \textbf{unanimous} vote, decision , or agreement is one in which all the people involved agree.
 \textit{
	\begin{itemize}
	\item ...the unanimous vote for Petra as President.
	\item Their decision was unanimous.
	\end{itemize}
}
\end{enumerate}

\section*{thing}
{\large \color{blue}  things  }
\subsection*{Explain}
\begin{enumerate}
\item countable noun \\
You can use \textbf{thing} to refer to any object, feature , or event when you cannot, need not, or do not want to refer to it more precisely.
 \textit{
	\begin{itemize}
	\item 'What's that thing in the middle of the fountain?'—'Some kind of statue, I guess.'
	\item She was in the middle of clearing the breakfast things.
	\item If you could change one thing about yourself, what would it be?
	\item A strange thing happened.
	\item We get blamed for all kinds of things.
	\end{itemize}
}
\item countable noun \\
\textbf{Thing} is used in lists and descriptions to give examples or to increase the range of what you are referring to.
 \textit{
	\begin{itemize}
	\item These are genetic disorders. They are things like muscular dystrophy and haemophilia.
	\item The Earth is made mainly of iron and silicon and things like that.
	\item Keep big things such as bikes or iPods for birthdays or Christmas.
	\item You can spot them fairly easily because of their short haircuts and things.
	\end{itemize}
}
\item countable noun \\
\textbf{Thing} is often used after an adjective , where it would also be possible just to use the adjective. For example, you can say  \textbf{it's a different thing}  instead of \textbf{it's different} .
 \textit{
	\begin{itemize}
	\item Of course, literacy isn't the same thing as intelligence.
	\item To be a parent is a terribly difficult thing.
	\item Perhaps it's a good thing that Dizzy retired.
	\end{itemize}
}
\item singular noun \\
\textbf{Thing} is often used instead of the pronouns 'anything,' or 'everything' in order to emphasize what you are saying .
 \textit{
	\begin{itemize}
	\item It isn't going to solve a single thing.
	\item Don't you worry about a thing.
	\item 'It's all here,' she said. 'Every damn thing.'
	\end{itemize}
}
\item countable noun \\
\textbf{Thing} is used in expressions such as \textbf{such a thing} or \textbf{things like that} , especially in negative statements, in order to emphasize the bad or difficult situation you are referring back to.
 \textit{
	\begin{itemize}
	\item I don't believe he would tell Leo such a thing.
	\item 'Are you accusing me of being a thief?'—'I have done no such thing, Tony.'.
	\item How do you actually go about discovering a thing like that?
	\item I'm trying to cope. These things happen. You have to cope.
	\end{itemize}
}
\item countable noun \\
You can use \textbf{thing} to refer in a vague way to a situation, activity, or idea, especially when you want to suggest that it is not very important.
 \textit{
	\begin{itemize}
	\item I'm a bit unsettled tonight. This war thing's upsetting me.
	\item These folks clearly take this ballroom thing very seriously.
	\item ...the man who had spoken dismissively of the 'vision thing' when running for the
presidency in 1988.
	\end{itemize}
}
\item countable noun \\
You can use \textbf{thing} when you are referring to something that you are uncertain or vague about, after mentioning something that it resembles or could possibly be.
 \textit{
	\begin{itemize}
	\item She'd actually taken it home and she put it in this jar thing.
	\item The captain of the submarine has got this periscope thing.
	\end{itemize}
}
\item countable noun \\
You often use \textbf{thing} to indicate to the person you are addressing that you are about to mention something important, or something that you particularly want them to know .
 \textit{
	\begin{itemize}
	\item One thing I am sure of was that she was scared.
	\item The first thing parents want to know is: will the baby survive?
	\item The funny thing is that the rest of us have known that for years.
	\item The most important thing to remember about fish is to buy it really fresh.
	\end{itemize}
}
\item countable noun \\
\textbf{Thing} is often used to refer back to something that has just been mentioned, either to
emphasize it or to give more information about it.
 \textit{
	\begin{itemize}
	\item I never wanted to be normal. It was not a thing I ever thought desirable.
	\item The Captain stretched his left leg on one of the empty chairs. He knew it was not
a polite thing to do.
	\end{itemize}
}
\item countable noun \\
A \textbf{thing} is a physical object that is considered as having no life of its own.
 \textit{
	\begin{itemize}
	\item It's not a thing, Beauchamp. It's a human being!
	\end{itemize}
}
\item countable noun \\
\textbf{Thing} is used to refer to something, especially a physical object, when you want to express
contempt or anger towards it.
 \textit{
	\begin{itemize}
	\item This thing's virtually useless.
	\item They're armed with sub-machine-guns or machine-pistols or whatever you call those
things.
	\item Turn that thing off!
	\end{itemize}
}
\item countable noun \\
You can call a person or an animal a particular \textbf{thing} when you want to mention a particular quality that they have and express your feelings
towards them, usually affectionate feelings.
 \textit{
	\begin{itemize}
	\item You really are quite a clever little thing.
	\item Oh you lucky thing!
	\end{itemize}
}
\item plural noun \\
Your \textbf{things} are your clothes or possessions.
 \textit{
	\begin{itemize}
	\item Sara told him to take all his things and not to return.
	\item Is there anything you'd like to borrow, before your own things are unpacked?
	\end{itemize}
}
\item plural noun \\
\textbf{Things} can refer to the situation or life in general and the way it is changing or affecting
you.
 \textit{
	\begin{itemize}
	\item Everyone agrees things are getting better.
	\item A change of ownership might improve things.
	\item How are things going?
	\end{itemize}
}
\item plural noun \\
\textbf{Things} can refer to a particular aspect of life, such as the physical or spiritual aspect.
 \textit{
	\begin{itemize}
	\item ...a movement away from the things of this world to the things of the spirit.
	\item I think I'm more aware now of some spiritual things and I do believe in good and
evil.
	\end{itemize}
}
\item countable noun \\
You can refer to something that is too frightening , strange , or horrible to describe  clearly as a \textbf{thing} .
 \textit{
	\begin{itemize}
	\item ...John W. Campbell, author of 'The Thing From Another World.'
	\end{itemize}
}
\item singular noun \\
If you say that something is \textbf{the thing} , you mean that it is fashionable or popular .
 \textit{
	\begin{itemize}
	\item I feel under pressure to go out and get drunk because it's the thing to do.
	\item They were obviously of the opinion that his taste was not quite the thing.
	\end{itemize}
}
\item  \\
 in all things \textit{
	\begin{itemize}
	\end{itemize}
}
\item  \\
 be all things to all men/people \textit{
	\begin{itemize}
	\end{itemize}
}
\item  \\
 do the decent/democratic/right/wrong/honourable thing \textit{
	\begin{itemize}
	\end{itemize}
}
\item  \\
 the done thing \textit{
	\begin{itemize}
	\end{itemize}
}
\item  \\
 first thing \textit{
	\begin{itemize}
	\end{itemize}
}
\item  \\
 have a thing about \textit{
	\begin{itemize}
	\end{itemize}
}
\item  \\
 it is a good/bad thing to \textit{
	\begin{itemize}
	\end{itemize}
}
\item  \\
 make a thing about/out of \textit{
	\begin{itemize}
	\end{itemize}
}
\item  \\
 be one thing \textit{
	\begin{itemize}
	\end{itemize}
}
\item  \\
 for one thing \textit{
	\begin{itemize}
	\end{itemize}
}
\item  \\
 one thing and another \textit{
	\begin{itemize}
	\end{itemize}
}
\item  \\
 it is just/simply one of those things \textit{
	\begin{itemize}
	\end{itemize}
}
\item  \\
 one thing led to another \textit{
	\begin{itemize}
	\end{itemize}
}
\item  \\
 do your own thing \textit{
	\begin{itemize}
	\end{itemize}
}
\item  \\
 a thing of the past \textit{
	\begin{itemize}
	\end{itemize}
}
\item  \\
 seeing/hearing things \textit{
	\begin{itemize}
	\end{itemize}
}
\item  \\
 no such thing \textit{
	\begin{itemize}
	\end{itemize}
}
\item  \\
 the thing is \textit{
	\begin{itemize}
	\end{itemize}
}
\item  \\
 just the thing/the very thing \textit{
	\begin{itemize}
	\end{itemize}
}
\item  \\
 a thing or two \textit{
	\begin{itemize}
	\end{itemize}
}
\end{enumerate}

\section*{upward}
{\large \color{blue}  }
\subsection*{Explain}
\begin{enumerate}
\item adjective \\
An \textbf{upward}  movement or look is directed towards a higher place or a higher level.
 \textit{
	\begin{itemize}
	\item She started once again on the steep upward climb.
	\item She gave him a quick, upward look, then lowered her eyes.
	\end{itemize}
}
\item adjective \\
If you refer to an \textbf{upward}  trend or an \textbf{upward}  spiral , you mean that something is increasing in quantity or price .
 \textit{
	\begin{itemize}
	\item ...the Army's concern that the upward trend in the numbers avoiding military service
may continue.
	\item Oil prices continued an upward swing this morning.
	\item ...if prices continue their inexorably upward spiral.
	\end{itemize}
}
\end{enumerate}

\section*{uniform}
{\large \color{blue}  uniforms  }
\subsection*{Explain}
\begin{enumerate}
\item variable noun \\
A \textbf{uniform} is a special set of clothes which some people, for example soldiers or the police, wear to work in and which some children wear at school .
 \textit{
	\begin{itemize}
	\item The town police wear dark blue uniforms and flat caps.
	\item Philippe was in uniform, wearing a pistol holster on his belt.
	\item She will probably take great pride in wearing school uniform.
	\end{itemize}
}
\item countable noun \\
You can refer to the particular style of clothing which a group of people wear to show they belong to a group or a movement as their \textbf{uniform} .
 \textit{
	\begin{itemize}
	\item Mark's is the uniform of the young male traveller–green army trousers, T-shirt and
shirt.
	\end{itemize}
}
\item adjective \\
If something is \textbf{uniform} , it does not vary , but is even and regular throughout.
 \textit{
	\begin{itemize}
	\item Chips should be cut into uniform size and thickness.
	\item The results after applying the fake tan were uniform.
	\item The price rises will not be uniform across the country.
	\end{itemize}
}
\item adjective \\
If you describe a number of things as \textbf{uniform} , you mean that they are all the same.
 \textit{
	\begin{itemize}
	\item Along each wall stretched uniform green metal filing cabinets.
	\item Shrimp are raised in long uniform ponds, frozen in the nearby packing plant and shipped
north.
	\end{itemize}
}
\end{enumerate}

\section*{voluntary}
{\large \color{blue}  }
\subsection*{Explain}
\begin{enumerate}
\item adjective \\
\textbf{Voluntary} actions or activities are done because someone chooses to do them and not because they have been forced to do them.
 \textit{
	\begin{itemize}
	\item Attention is drawn to a special voluntary course in Commercial French.
	\item The scheme, due to begin next month, will be voluntary.
	\end{itemize}
}
\item adjective \\
\textbf{Voluntary} work is done by people who are not paid for it, but who do it because they want to do it.
 \textit{
	\begin{itemize}
	\item In her spare time she does voluntary work.
	\item He'd been working at the local hostel on a voluntary basis.
	\end{itemize}
}
\item adjective \\
A \textbf{voluntary}  worker is someone who does work without being paid for it, because they want to do it.
 \textit{
	\begin{itemize}
	\item Apna Arts has achieved more with voluntary workers in three years than most organisations
with paid workers have achieved in ten.
	\item We depend solely upon our voluntary helpers.
	\end{itemize}
}
\item adjective \\
A \textbf{voluntary} organization is controlled and organized by the people who have chosen to work for it, often without being paid, rather than receiving  help or money from the government.
 \textit{
	\begin{itemize}
	\item Some local authorities and voluntary organizations also run workshops for people
with disabilities.
	\item It has been largely through the voluntary sector that the needs of victims have been
met.
	\item ...a voluntary hostel for ex-offenders.
	\end{itemize}
}
\end{enumerate}

\section*{usage}
{\large \color{blue}  usages  }
\subsection*{Explain}
\begin{enumerate}
\item uncountable noun \\
\textbf{Usage} is the way in which words are actually used in particular contexts , especially with regard to their meanings .
 \textit{
	\begin{itemize}
	\item The word 'undertaker' had long been in common usage.
	\item He was a stickler for the correct usage of English.
	\end{itemize}
}
\item countable noun \\
A \textbf{usage} is a meaning that a word has or a way in which it can be used.
 \textit{
	\begin{itemize}
	\item It's very definitely a usage which has come over to Britain from America.
	\end{itemize}
}
\item uncountable noun \\
\textbf{Usage} is the degree to which something is used or the way in which it is used.
 \textit{
	\begin{itemize}
	\item Parts of the motor wore out because of constant usage.
	\item If your water usage is very small, it may be worthwhile opting for a meter.
	\end{itemize}
}
\end{enumerate}

\section*{worldwide}
{\large \color{blue}  }
\subsection*{Explain}
\begin{enumerate}
\item adverb \\
If something exists or happens  \textbf{worldwide} , it exists or happens throughout the world.
 \textbf{Worldwide} is also an adjective .
 \textit{
	\begin{itemize}
	\item His books have sold more than 20 million copies worldwide.
	\item Worldwide, an enormous amount of research effort goes into military technology.
	\item Today, doctors are fearing a worldwide epidemic.
	\end{itemize}
}
\end{enumerate}

\section*{use}
{\large \color{blue}  uses  using  used  }
\subsection*{Explain}
\begin{enumerate}
\item verb \\
If you \textbf{use} something, you do something with it in order to do a job or to achieve a particular result or effect .
 \textit{
	\begin{itemize}
	\item Trim off the excess pastry using a sharp knife.
	\item He had simply used a little imagination.
	\item Officials used loud hailers to call for calm.
	\item The show uses Zondo's trial and execution as its framework.
	\end{itemize}
}
\item verb \\
If you \textbf{use} a supply of something, you finish it so that none of it is left .
 \textbf{Use up}  means the same as use1 .
 \textit{
	\begin{itemize}
	\item You used all the ice cubes and didn't put the ice trays back.
	\item They've never had anything spare–they've always used it all.
	\item It isn't them who use up the world's resources.
	\item We were breathing really fast, and using the air up quickly.
	\end{itemize}
}
\item verb \\
If someone \textbf{uses} drugs, they take drugs regularly, especially  illegal ones.
 \textit{
	\begin{itemize}
	\item He denied he had used drugs.
	\end{itemize}
}
\item verb \\
You can  say that someone \textbf{uses} the toilet or bathroom as a polite way of saying that they go to the toilet.
 \textit{
	\begin{itemize}
	\item Wash your hands after using the toilet.
	\item He asked whether he could use my bathroom.
	\end{itemize}
}
\item verb \\
If you \textbf{use} a particular word or expression, you say or write it, because it has the meaning that you want to express .
 \textit{
	\begin{itemize}
	\item The judge liked using the word 'wicked' of people he had sent to jail.
	\item When Johnson talks about cuts, he uses words like 'target price' and 'efficiency
payments'.
	\end{itemize}
}
\item verb \\
If you \textbf{use} a particular name, you call yourself by that name, especially when it is not the name that you usually call yourself.
 \textit{
	\begin{itemize}
	\item Now I use a false name if I'm meeting people for the first time.
	\item I didn't want to use my married name because we've split.
	\end{itemize}
}
\item verb \\
If you say that someone \textbf{uses} people, you disapprove of them because they make others do things for them in order to benefit or gain some advantage from it, and not because they care about the other people.
 \textit{
	\begin{itemize}
	\item Be careful she's not just using you.
	\item Why do I have the feeling I'm being used again?
	\end{itemize}
}
\end{enumerate}

\section*{young}
{\large \color{blue}  younger  youngest  }
\subsection*{Explain}
\begin{enumerate}
\item adjective \\
A \textbf{young} person, animal, or plant has not lived or existed for very long and is not yet mature .
 \textbf{The young} are people who are young.
 \textit{
	\begin{itemize}
	\item In Scotland, young people can marry at 16.
	\item You weren't so very young when she died; you were old enough to remember.
	\item ...a field of young barley.
	\item He played with his younger brother.
	\item The association is advising pregnant women, the very young and the elderly to avoid
such foods.
	\end{itemize}
}
\item adjective \\
You use \textbf{young} to describe a time when a person or thing was young.
 \textit{
	\begin{itemize}
	\item In her younger days my mother had been a successful fashionwear saleswoman.
	\end{itemize}
}
\item adjective \\
Someone who is \textbf{young} in appearance or behaviour looks or behaves as if they are young.
 \textit{
	\begin{itemize}
	\item I was twenty-three, I suppose, and young for my age.
	\item He seemed to me very young and very lonely.
	\end{itemize}
}
\item plural noun \\
The \textbf{young} of an animal are its babies .
 \textit{
	\begin{itemize}
	\item The hen may not be able to feed its young.
	\end{itemize}
}
\end{enumerate}

\section*{waiter}
{\large \color{blue}  waiters  }
\subsection*{Explain}
\begin{enumerate}
\item countable noun \\
A \textbf{waiter} is a man who works in a restaurant, serving people with food and drink .
 \textit{
	\begin{itemize}
	\end{itemize}
}
\end{enumerate}

\section*{conquest}
{\large \color{blue}  conquests  }
\subsection*{Explain}
\begin{enumerate}
\item uncountable noun \\
\textbf{Conquest} is the act of conquering a country or group of people.
 \textit{
	\begin{itemize}
	\item He had led the conquest of southern Poland in 1939.
	\item After the Norman Conquest the forest became a royal hunting preserve.
	\item Jerusalem has seen endless conquests and occupations.
	\end{itemize}
}
\item countable noun \\
\textbf{Conquests} are lands that have been conquered in war .
 \textit{
	\begin{itemize}
	\item He had realized that Britain could not have peace unless she returned at least some
of her former conquests.
	\end{itemize}
}
\item countable noun \\
If someone makes a \textbf{conquest} , they succeed in attracting and usually sleeping with another person. You usually use \textbf{conquest} when you want to indicate that this relationship is not important to the person concerned .
 \textit{
	\begin{itemize}
	\item Despite his conquests, he remains lonely and isolated.
	\item ...men who boast about their sexual conquests to all their friends.
	\end{itemize}
}
\item countable noun \\
You can refer to the person that someone has succeeded in attracting as their \textbf{conquest} .
 \textit{
	\begin{itemize}
	\item Pushkin was a womaniser whose conquests included everyone from prostitutes to princesses.
	\end{itemize}
}
\item singular noun \\
\textbf{The}  \textbf{conquest}  \textbf{of} something such as a problem is success in ending it or dealing with it.
 \textit{
	\begin{itemize}
	\item The conquest of inflation has been the Government's overriding economic priority
for nearly 15 years.
	\item ...the conquest of cancer.
	\end{itemize}
}
\end{enumerate}

\section*{alternate}
{\large \color{blue}  alternates  alternating  alternated  }
\subsection*{Explain}
\begin{enumerate}
\item verb \\
When you \textbf{alternate} two things, you keep using one then the other. When one thing \textbf{alternates}  \textbf{with} another, the first regularly occurs after the other.
 \textit{
	\begin{itemize}
	\item Her aggressive moods alternated with gentle or more co-operative states.
	\item The three acts will alternate as headliners throughout the tour.
	\item Now you just alternate layers of that mixture and eggplant.
	\item The band alternated romantic love songs with bouncy dance numbers.
	\item ...an imaginative novel, with alternating chapters presenting each partner's point
of view.
	\end{itemize}
}
\item adjective \\
\textbf{Alternate} actions, events , or processes regularly occur after each other.
 \textit{
	\begin{itemize}
	\item They were streaked with alternate bands of colour.
	\end{itemize}
}
\item adjective \\
If something happens on \textbf{alternate}  days , it happens on one day, then happens on every second day after that. In the same
 way , something can happen in \textbf{alternate}  weeks , years , or other periods of time.
 \textit{
	\begin{itemize}
	\item Lesley had agreed to Jim going skiing in alternate years.
	\end{itemize}
}
\item adjective \\
You use \textbf{alternate} to describe a plan , idea , or system which is different from the one already in operation and can be used instead of it.
 \textit{
	\begin{itemize}
	\item His group was forced to turn back and take an alternate route.
	\item ...alternate forms of medical treatment.
	\end{itemize}
}
\item countable noun \\
An \textbf{alternate} is a person or thing that replaces another, and can act or be used instead of them.
 \textit{
	\begin{itemize}
	\item In most jurisdictions, twelve jurors and two alternates are chosen.
	\item ... meats and meat alternates.
	\end{itemize}
}
\item adjective \\
\textbf{Alternate} is sometimes used, especially in American  English , instead of alternative in meanings 2, 3, 4, and 5.
 \textit{
	\begin{itemize}
	\item ...an alternate lifestyle.
	\end{itemize}
}
\end{enumerate}

\section*{cream}
{\large \color{blue}  creams  creaming  creamed  }
\subsection*{Explain}
\begin{enumerate}
\item uncountable noun \\
\textbf{Cream} is a thick yellowish-white liquid taken from milk. You can use it in cooking or put it on fruit
or desserts .
 \textit{
	\begin{itemize}
	\item ...strawberries and cream.
	\end{itemize}
}
\item uncountable noun \\
\textbf{Cream} is used in the names of soups that contain cream or milk.
 \textit{
	\begin{itemize}
	\item ...cream of mushroom soup.
	\end{itemize}
}
\item variable noun \\
A \textbf{cream} is a substance that you rub into your skin, for example to keep it soft or to heal or protect it.
 \textit{
	\begin{itemize}
	\item Gently apply the cream to the affected areas.
	\item ...sun protection creams.
	\end{itemize}
}
\item colour \\
Something that is \textbf{cream} is yellowish-white in colour.
 \textit{
	\begin{itemize}
	\item ...cream silk stockings.
	\item ...a cream-coloured Persian cat.
	\end{itemize}
}
\item singular noun \\
\textbf{Cream} is used in expressions such as \textbf{the cream of society} and \textbf{the cream of British athletes} to refer to the best people or things of a particular kind .
 \textit{
	\begin{itemize}
	\item The Ball was attended by the cream of Hollywood society.
	\item ...the cream of Chicago's 200 jazz and blues clubs.
	\end{itemize}
}
\end{enumerate}

\section*{ample}
{\large \color{blue}  ampler  amplest  }
\subsection*{Explain}
\begin{enumerate}
\item adjective \\
If there is an \textbf{ample} amount of something, there is enough of it and usually some extra .
 \textit{
	\begin{itemize}
	\item There'll be ample opportunity to relax, swim and soak up some sun.
	\item The design of the ground floor created ample space for a good-sized kitchen.
	\end{itemize}
}
\item graded adjective \\
If you describe someone's figure as \textbf{ample} , you mean that they are large in a pleasant or attractive way.
 \textit{
	\begin{itemize}
	\item ...a young mother with a baby resting against her ample bosom.
	\end{itemize}
}
\end{enumerate}

\section*{credential}
{\large \color{blue}  }
\subsection*{Explain}
\begin{enumerate}
\item noun \\
1.  2.  \textit{
	\begin{itemize}
	\end{itemize}
}
\item adjective \\
3.  \textit{
	\begin{itemize}
	\end{itemize}
}
\end{enumerate}

\section*{busy}
{\large \color{blue}  busier  busiest  busies  busying  busied  }
\subsection*{Explain}
\begin{enumerate}
\item adjective \\
When you are \textbf{busy} , you are working  hard or concentrating on a task , so that you are not free to do anything else.
 \textit{
	\begin{itemize}
	\item What is it? I'm busy.
	\item They are busy preparing for a hectic day's activity on Saturday.
	\item Rachel said she would be too busy to come.
	\item Phil Martin is an exceptionally busy man.
	\end{itemize}
}
\item adjective \\
A \textbf{busy} time is a period of time during which you have a lot of things to do.
 \textit{
	\begin{itemize}
	\item It'll have to wait. This is our busiest time.
	\item Even with her busy schedule she finds time to watch TV.
	\item I had a busy day and was rather tired.
	\end{itemize}
}
\item adjective \\
If you say that someone is \textbf{busy}  thinking or worrying about something, you mean that it is taking all their attention , often to such an extent that they are unable to think about anything else.
 \textit{
	\begin{itemize}
	\item I'm so busy worrying about all the wrong things that I'm not focusing on the right
ones.
	\item Most people are too busy with their own troubles to give much help.
	\end{itemize}
}
\item verb \\
If you \textbf{busy}  \textbf{yourself} with something, you occupy yourself by dealing with it.
 \textit{
	\begin{itemize}
	\item He busied himself with the camera.
	\item She busied herself getting towels ready.
	\item For a while Kathryn busied herself in the kitchen.
	\end{itemize}
}
\item adjective \\
A \textbf{busy} place is full of people who are doing things or moving about.
 \textit{
	\begin{itemize}
	\item The Strand is one of London's busiest and most affluent streets.
	\item The ward was busy and Amy hardly had time to talk.
	\end{itemize}
}
\item adjective \\
When a telephone line is \textbf{busy} , you cannot make your call because the line is already being used by someone else.
 \textit{
	\begin{itemize}
	\item I tried to reach him, but the line was busy.
	\end{itemize}
}
\end{enumerate}

\section*{destiny}
{\large \color{blue}  destinies  }
\subsection*{Explain}
\begin{enumerate}
\item countable noun \\
A person's \textbf{destiny} is everything that happens to them during their life, including what will happen in the future, especially when it is considered to be controlled by someone or something else.
 \textit{
	\begin{itemize}
	\item We are masters of our own destiny.
	\item It is my destiny one day to be king.
	\end{itemize}
}
\item uncountable noun \\
\textbf{Destiny} is the force which some people believe controls the things that happen to you in your life.
 \textit{
	\begin{itemize}
	\item Is it destiny that brings people together, or is it accident?
	\end{itemize}
}
\end{enumerate}

\section*{cloudy}
{\large \color{blue}  cloudier  cloudiest  }
\subsection*{Explain}
\begin{enumerate}
\item adjective \\
If it is \textbf{cloudy} , there are a lot of clouds in the sky .
 \textit{
	\begin{itemize}
	\item ...a windy, cloudy day.
	\end{itemize}
}
\item adjective \\
A \textbf{cloudy} liquid is less clear than it should be.
 \textit{
	\begin{itemize}
	\end{itemize}
}
\item graded adjective \\
Ideas , opinions , or issues that are \textbf{cloudy} are confused or uncertain .
 \textit{
	\begin{itemize}
	\item ...an absurdly cloudy political debate.
	\item The legal position is very cloudy.
	\end{itemize}
}
\end{enumerate}

\section*{detective}
{\large \color{blue}  detectives  }
\subsection*{Explain}
\begin{enumerate}
\item countable noun \\
A \textbf{detective} is someone whose job is to discover what has happened in a crime or other situation and to find the people involved. Some detectives work in the police force and others work privately.
 \textit{
	\begin{itemize}
	\item Detectives are appealing for witnesses who may have seen anything suspicious.
	\item She hired a private detective in an attempt to find her daughter.
	\end{itemize}
}
\item title noun \\
In Britain ' \textbf{detective} ' is used before words such as ' constable ' or ' sergeant ', and in the U.S. the word ' \textbf{detective} ' is used on its own, to indicate that a police officer is a member of the department concerned with investigating crimes.
 \textit{
	\begin{itemize}
	\item ...Detective Inspector Ian Mosley.
	\item ...Detective Nardosa of the New York City Police Department.
	\end{itemize}
}
\item adjective \\
A \textbf{detective}  novel or story is one in which a detective tries to solve a crime.
 \textit{
	\begin{itemize}
	\end{itemize}
}
\item  \\
 detective work \textit{
	\begin{itemize}
	\end{itemize}
}
\end{enumerate}

\section*{cooperative}
{\large \color{blue}  }
\subsection*{Explain}
\begin{enumerate}
\item adjective \\
1.  2.  3.  \textit{
	\begin{itemize}
	\end{itemize}
}
\item noun \\
4.  5.  \textit{
	\begin{itemize}
	\end{itemize}
}
\end{enumerate}

\section*{diamond}
{\large \color{blue}  diamonds  }
\subsection*{Explain}
\begin{enumerate}
\item variable noun \\
A \textbf{diamond} is a hard, bright , precious  stone which is clear and colourless. Diamonds are used in jewellery and for cutting very hard substances .
 \textit{
	\begin{itemize}
	\item ...a pair of diamond earrings.
	\item ...a sphere made of diamond without impurity or flaw.
	\end{itemize}
}
\item plural noun \\
\textbf{Diamonds} are jewellery such as necklaces and rings which have diamonds set into them.
 \textit{
	\begin{itemize}
	\item Nicole loves wearing her diamonds, even with jeans and a white T-shirt.
	\end{itemize}
}
\item countable noun \\
A \textbf{diamond} is a shape with four straight sides of equal length where the opposite angles are the same, but none of the angles is equal to 90°: ♦.
 \textit{
	\begin{itemize}
	\item He formed his hands into the shape of a diamond.
	\end{itemize}
}
\item uncountable noun \\
\textbf{Diamonds} is one of the four suits of cards in a pack of playing cards. Each card in the suit is marked with one or more red symbols in
the shape of a diamond.
 A \textbf{diamond} is a playing card of this suit.
 \textit{
	\begin{itemize}
	\item He drew the seven of diamonds.
	\item ...win the ace of clubs and play a diamond.
	\end{itemize}
}
\item countable noun \\
In baseball , the \textbf{diamond} is the diamond-shaped area of the playing field between the four bases.
 \textit{
	\begin{itemize}
	\end{itemize}
}
\end{enumerate}

\section*{cosmic}
{\large \color{blue}  }
\subsection*{Explain}
\begin{enumerate}
\item adjective \\
\textbf{Cosmic} means occurring in, or coming from, the part of space that lies outside Earth and its atmosphere .
 \textit{
	\begin{itemize}
	\item ...cosmic radiation.
	\item ...cosmic debris.
	\end{itemize}
}
\item adjective \\
\textbf{Cosmic} means belonging or relating to the universe.
 \textit{
	\begin{itemize}
	\item ...the cosmic laws governing our world.
	\item ...humanity's place in the cosmic order of things.
	\end{itemize}
}
\end{enumerate}

\section*{edition}
{\large \color{blue}  editions  }
\subsection*{Explain}
\begin{enumerate}
\item countable noun \\
An \textbf{edition} is a particular version of a book, magazine , or newspaper that is printed at one time.
 \textit{
	\begin{itemize}
	\item A paperback edition is now available at bookshops.
	\end{itemize}
}
\item countable noun \\
An \textbf{edition} is the total number of copies of a particular book or newspaper that are printed at one time.
 \textit{
	\begin{itemize}
	\item The second edition was published only in America.
	\end{itemize}
}
\item countable noun \\
An \textbf{edition} is a single television or radio programme that is one of a series about a particular subject .
 \textit{
	\begin{itemize}
	\item They appeared on an edition of BBC2's Arena.
	\end{itemize}
}
\end{enumerate}

\section*{dangerous}
{\large \color{blue}  }
\subsection*{Explain}
\begin{enumerate}
\item adjective \\
If something is \textbf{dangerous} , it is able or likely to hurt or harm you.
 \textit{
	\begin{itemize}
	\item It's a dangerous stretch of road.
	\item ...dangerous drugs.
	\item It's dangerous to jump to early conclusions.
	\end{itemize}
}
\end{enumerate}

\section*{example}
{\large \color{blue}  examples  }
\subsection*{Explain}
\begin{enumerate}
\item countable noun \\
An \textbf{example}  \textbf{of} something is a particular situation , object, or person which shows that what is being claimed is true .
 \textit{
	\begin{itemize}
	\item The doctors gave numerous examples of patients being expelled from hospital.
	\item Listed below are just a few examples of some of the family benefits available.
	\end{itemize}
}
\item countable noun \\
An \textbf{example}  \textbf{of} a particular class of objects or styles is something that has many of the typical features of such a class or style, and that you consider  clearly  represents it.
 \textit{
	\begin{itemize}
	\item Symphonies 103 and 104 stand as perfect examples of early symphonic construction.
	\item The plaque illustrated in Figure 1 is an example of his work at this time.
	\end{itemize}
}
\item  \\
 for example \textit{
	\begin{itemize}
	\end{itemize}
}
\item countable noun \\
If you refer to a person or their behaviour as an \textbf{example}  \textbf{to} other people, you mean that he or she behaves in a good or correct way that other people should copy .
 \textit{
	\begin{itemize}
	\item He is a model professional and an example to the younger lads.
	\item Their example shows us what we are all capable of.
	\end{itemize}
}
\item countable noun \\
In a dictionary  entry , an \textbf{example} is a phrase or sentence which shows how a particular word is used.
 \textit{
	\begin{itemize}
	\item The examples are unique to this dictionary.
	\end{itemize}
}
\item  \\
 follow someone's example \textit{
	\begin{itemize}
	\end{itemize}
}
\item  \\
 make an example of someone \textit{
	\begin{itemize}
	\end{itemize}
}
\item  \\
 set an example \textit{
	\begin{itemize}
	\end{itemize}
}
\end{enumerate}

\section*{decisive}
{\large \color{blue}  }
\subsection*{Explain}
\begin{enumerate}
\item adjective \\
If a fact , action, or event is \textbf{decisive} , it makes it certain that there will be a particular result.
 \textit{
	\begin{itemize}
	\item ...his decisive victory in the presidential elections.
	\item The election campaign has now entered its final, decisive phase.
	\item The meeting between Molotov, Bidault and Bevin was decisive.
	\end{itemize}
}
\item adjective \\
If someone is \textbf{decisive} , they have or show an ability to make quick decisions in a difficult or complicated  situation .
 \textit{
	\begin{itemize}
	\item He should give way to a more imaginative, more decisive leader.
	\end{itemize}
}
\end{enumerate}

\section*{exception}
{\large \color{blue}  exceptions  }
\subsection*{Explain}
\begin{enumerate}
\item countable noun \\
An \textbf{exception} is a particular thing, person, or situation that is not included in a general statement , judgment , or rule.
 \textit{
	\begin{itemize}
	\item Few guitarists can sing as well as they can play; Eddie, however, is an exception.
	\item There were no floral offerings at the ceremony, with the exception of a single red
rose.
	\item The law makes no exceptions.
	\item With few exceptions, guests are booked for week-long visits.
	\end{itemize}
}
\item  \\
 no exception \textit{
	\begin{itemize}
	\end{itemize}
}
\item  \\
 the exception that proves the rule \textit{
	\begin{itemize}
	\end{itemize}
}
\item  \\
 take exception to something \textit{
	\begin{itemize}
	\end{itemize}
}
\item  \\
 with the exception of \textit{
	\begin{itemize}
	\end{itemize}
}
\item  \\
 without exception \textit{
	\begin{itemize}
	\end{itemize}
}
\end{enumerate}

\section*{diverse}
{\large \color{blue}  }
\subsection*{Explain}
\begin{enumerate}
\item adjective \\
If a group or range of things is \textbf{diverse} , it is made up of a wide variety of things.
 \textit{
	\begin{itemize}
	\item ...shops selling a diverse range of gifts.
	\item Society is now much more diverse than ever before.
	\end{itemize}
}
\item adjective \\
\textbf{Diverse} people or things are very different from each other.
 \textit{
	\begin{itemize}
	\item Jones has a much more diverse and perhaps younger audience.
	\end{itemize}
}
\end{enumerate}

\section*{fate}
{\large \color{blue}  fates  }
\subsection*{Explain}
\begin{enumerate}
\item uncountable noun \\
\textbf{Fate} is a power that some people believe controls and decides everything that happens , in a way that cannot be prevented or changed. You can also  refer to \textbf{the fates} .
 \textit{
	\begin{itemize}
	\item I see no use quarrelling with fate.
	\item ...the fickleness of fate.
	\item It was just one of those times when you wonder whether the fates conspire against
you.
	\end{itemize}
}
\item countable noun \\
A person's or thing's \textbf{fate} is what happens to them.
 \textit{
	\begin{itemize}
	\item The Russian Parliament will hold a special session later this month to decide his
fate.
	\item He seems for a moment to be again holding the fate of the country in his hands.
	\item The Casino, where she had often danced, had suffered a similar fate.
	\item ...the terrible fate awaiting humanity.
	\end{itemize}
}
\item  \\
 to seal someone's fate \textit{
	\begin{itemize}
	\end{itemize}
}
\end{enumerate}

\section*{faithful}
{\large \color{blue}  faithfuls  }
\subsection*{Explain}
\begin{enumerate}
\item adjective \\
Someone who is \textbf{faithful}  \textbf{to} a person, organization , idea , or activity remains firm in their belief in them or support for them.
 \textbf{The faithful} are people who are faithful to someone or something.
 \textit{
	\begin{itemize}
	\item She had been faithful to her promise to guard this secret.
	\item Older Americans are among this country's most faithful voters.
	\item He spends his time making speeches at factories or gatherings of the Party faithful.
	\end{itemize}
}
\item adjective \\
Someone who is \textbf{faithful}  \textbf{to} their husband , wife , or lover does not have a sexual relationship with anyone else.
 \textit{
	\begin{itemize}
	\item She insisted that she had remained faithful to her husband.
	\item I'm very faithful when I love someone.
	\end{itemize}
}
\item plural noun \\
\textbf{The faithful} are the group of people who believe in a particular religion .
 \textit{
	\begin{itemize}
	\item The faithful revered him then as a prophet.
	\end{itemize}
}
\item adjective \\
A \textbf{faithful}  account , translation , or copy of something represents or reproduces the original accurately.
 \textit{
	\begin{itemize}
	\item Colin Welland's screenplay is faithful to the novel.
	\item ...faithful copies of household items used in the mid-1800s.
	\end{itemize}
}
\item  \\
 old faithful \textit{
	\begin{itemize}
	\end{itemize}
}
\end{enumerate}

\section*{freight}
{\large \color{blue}  freights  freighting  freighted  }
\subsection*{Explain}
\begin{enumerate}
\item uncountable noun \\
\textbf{Freight} is the movement of goods by lorries , trains , ships , or aeroplanes .
 \textit{
	\begin{itemize}
	\item France derives 16% of revenue from air freight.
	\end{itemize}
}
\item uncountable noun \\
\textbf{Freight} is goods that are transported by lorries, trains, ships, or aeroplanes.
 \textit{
	\begin{itemize}
	\item ...26 tons of freight.
	\item 90% of managers wanted to see more freight carried by rail.
	\end{itemize}
}
\item verb \\
When goods \textbf{are freighted} , they are transported in large quantities over a long distance .
 \textit{
	\begin{itemize}
	\item From these ports the grain is freighted down to Addis Ababa.
	\end{itemize}
}
\end{enumerate}

\section*{fat}
{\large \color{blue}  fatter  fattest  fats  }
\subsection*{Explain}
\begin{enumerate}
\item adjective \\
If you say that a person or animal is \textbf{fat} , you mean that they have a lot of flesh on their body and that they weigh too much. You usually use the word \textbf{fat} when you think that this is a bad thing.
 \textit{
	\begin{itemize}
	\item I could eat what I liked without getting fat.
	\item After five minutes, the fat woman in the seat in front of me was asleep.
	\end{itemize}
}
\item uncountable noun \\
\textbf{Fat} is the extra flesh that animals and humans have under their skin, which is used to store energy and to help keep them warm .
 \textit{
	\begin{itemize}
	\item Because you're not burning calories, everything you eat turns to fat.
	\end{itemize}
}
\item variable noun \\
\textbf{Fat} is a solid or liquid substance obtained from animals or vegetables, which is used
in cooking.
 \textit{
	\begin{itemize}
	\item When you use oil or fat for cooking, use as little as possible.
	\item ...vegetable fats, such as coconut oil and palm oil.
	\end{itemize}
}
\item variable noun \\
\textbf{Fat} is a substance contained in foods such as meat , cheese , and butter which forms an energy store in your body.
 \textit{
	\begin{itemize}
	\item An easy way to cut the amount of fat in your diet is to avoid eating red meats.
	\item Most low-fat yogurts are about 40 calories per 100g.
	\end{itemize}
}
\item adjective \\
A \textbf{fat} object, especially a book, is very thick or wide .
 \textit{
	\begin{itemize}
	\item ...'Europe in Figures', a fat book published on September 22nd.
	\item He took out his fat wallet and peeled off some notes.
	\end{itemize}
}
\item adjective \\
A \textbf{fat}  profit or fee is a large one.
 \textit{
	\begin{itemize}
	\item They are set to make a big fat profit.
	\end{itemize}
}
\item  \\
 fat chance \textit{
	\begin{itemize}
	\end{itemize}
}
\item  \\
 grow fat \textit{
	\begin{itemize}
	\end{itemize}
}
\item  \\
 a fat lot of good/use/help \textit{
	\begin{itemize}
	\end{itemize}
}
\end{enumerate}

\section*{friend}
{\large \color{blue}  friends  friending  friended  }
\subsection*{Explain}
\begin{enumerate}
\item countable noun \\
A \textbf{friend} is someone who you know well and like, but who is not related to you.
 \textit{
	\begin{itemize}
	\item I had a long talk about this with my best friend.
	\item She never was a close friend of mine.
	\item ...Sara's old friend, Ogden.
	\end{itemize}
}
\item plural noun \\
If you are \textbf{friends}  \textbf{with} someone, you are their friend and they are yours.
 \textit{
	\begin{itemize}
	\item I still wanted to be friends with Alison.
	\item We remained good friends.
	\item Sally and I became friends.
	\end{itemize}
}
\item plural noun \\
The \textbf{friends}  \textbf{of} a country, cause, organization , or a famous  politician are the people and organizations who help and support them.
 \textit{
	\begin{itemize}
	\item ...the friends of capitalism.
	\item ...The Friends of Birmingham Royal Ballet.
	\end{itemize}
}
\item countable noun \\
If one country refers to another as a \textbf{friend} , they mean that the other country is not an enemy of theirs.
 \textit{
	\begin{itemize}
	\item Do Italy's friends and partners have to accept the situation?
	\end{itemize}
}
\item  \\
 to make friends \textit{
	\begin{itemize}
	\end{itemize}
}
\item verb \\
If you \textbf{friend} someone, you ask them to be your friend on a social media website, so that you can see each other's posts .
 \textit{
	\begin{itemize}
	\item People you have friended on Facebook could be complete strangers in real life.
	\item He friended dozens of other graduates of his college.
	\end{itemize}
}
\end{enumerate}

\section*{foul}
{\large \color{blue}  fouler  foulest  fouls  fouling  fouled  }
\subsection*{Explain}
\begin{enumerate}
\item adjective \\
If you describe something as \textbf{foul} , you mean it is dirty and smells or tastes unpleasant.
 \textit{
	\begin{itemize}
	\item ...foul polluted water.
	\item The smell was quite foul.
	\end{itemize}
}
\item adjective \\
\textbf{Foul} language is offensive and contains swear words or rude words.
 \textit{
	\begin{itemize}
	\item He was sent off for using foul language in a match last Sunday.
	\item He had a foul mouth.
	\end{itemize}
}
\item adjective \\
If someone has a \textbf{foul}  temper or is in a \textbf{foul}  mood , they become angry or violent very suddenly and easily .
 \textit{
	\begin{itemize}
	\item Collins was in a foul mood even before the interviews began.
	\end{itemize}
}
\item adjective \\
\textbf{Foul} weather is unpleasant, windy , and stormy .
 \textit{
	\begin{itemize}
	\end{itemize}
}
\item verb \\
If a place \textbf{is fouled} by someone or something, they make it dirty.
 \textit{
	\begin{itemize}
	\item Two oil-related accidents have fouled the ocean and the skies there.
	\end{itemize}
}
\item verb \\
If an animal \textbf{fouls} a place, it drops faeces onto the ground.
 \textit{
	\begin{itemize}
	\item It is an offence to let your dog foul a footpath.
	\end{itemize}
}
\item verb \\
If a machine or vehicle \textbf{fouls} part of its mechanism or if something such as a rope  \textbf{fouls} the mechanism, the mechanism can no longer work properly because something has become
 twisted or knotted around it.
 \textit{
	\begin{itemize}
	\item The freighter fouled its propeller in fishing nets.
	\end{itemize}
}
\item verb \\
In a game or sport, if a player \textbf{fouls} another player, they touch them or block them in a way which is not allowed according to the rules.
 \textit{
	\begin{itemize}
	\item He was sent off for fouling the striker.
	\end{itemize}
}
\item countable noun \\
A \textbf{foul} is an act in a game or sport that is not allowed according to the rules.
 \textbf{Foul} is also an adjective .
 \textit{
	\begin{itemize}
	\item He has committed more fouls than any other player this season.
	\item ...a foul tackle.
	\end{itemize}
}
\item  \\
 cry foul \textit{
	\begin{itemize}
	\end{itemize}
}
\item  \\
 by fair means or foul \textit{
	\begin{itemize}
	\end{itemize}
}
\item  \\
 to fall foul of \textit{
	\begin{itemize}
	\end{itemize}
}
\end{enumerate}

\section*{gold}
{\large \color{blue}  golds  }
\subsection*{Explain}
\begin{enumerate}
\item uncountable noun \\
\textbf{Gold} is a valuable , yellow-coloured metal that is used for making jewellery and ornaments , and as an international  currency .
 \textit{
	\begin{itemize}
	\item ...a sapphire set in gold.
	\item The price of gold was going up.
	\item ...gold coins.
	\end{itemize}
}
\item uncountable noun \\
\textbf{Gold} is jewellery and other things that are made of gold.
 \textit{
	\begin{itemize}
	\item We handed over all our gold and money.
	\end{itemize}
}
\item colour \\
Something that is \textbf{gold} is a bright yellow colour, and is often shiny .
 \textit{
	\begin{itemize}
	\item I'd been wearing Michel's black and gold shirt.
	\end{itemize}
}
\item variable noun \\
A \textbf{gold} is the same as a gold medal .
 \textit{
	\begin{itemize}
	\item His ambition was to win gold.
	\item This Saturday the British star is going for gold and a new world record.
	\end{itemize}
}
\item  \\
 good as gold \textit{
	\begin{itemize}
	\end{itemize}
}
\item  \\
 a heart of gold \textit{
	\begin{itemize}
	\end{itemize}
}
\item  \\
 pot of gold \textit{
	\begin{itemize}
	\end{itemize}
}
\end{enumerate}

\section*{fruitful}
{\large \color{blue}  }
\subsection*{Explain}
\begin{enumerate}
\item adjective \\
Something that is \textbf{fruitful} produces good and useful results.
 \textit{
	\begin{itemize}
	\item We had a long, happy, fruitful relationship.
	\item The talks had been fruitful, but much remained to be done.
	\end{itemize}
}
\item adjective \\
\textbf{Fruitful} land or trees produce a lot of crops .
 \textit{
	\begin{itemize}
	\item ...a landscape that was fruitful and lush.
	\end{itemize}
}
\end{enumerate}

\section*{imperative}
{\large \color{blue}  imperatives  }
\subsection*{Explain}
\begin{enumerate}
\item adjective \\
If it is \textbf{imperative} that something is done , that thing is extremely important and must be done.
 \textit{
	\begin{itemize}
	\item It was imperative that he act as naturally as possible.
	\item That's why it is imperative to know what your rights are at such a time.
	\item The events of the past few days make it imperative for her to act.
	\end{itemize}
}
\item countable noun \\
An \textbf{imperative} is something that is extremely important and must be done.
 \textit{
	\begin{itemize}
	\item The most important political imperative is to limit the number of U.S. casualties.
	\item ...the needs of those unable to respond to the imperatives of an enterprise culture.
	\end{itemize}
}
\item singular noun \\
In grammar , a clause that is in \textbf{the imperative} , or in \textbf{the imperative} mood, contains the base form of a verb and usually has no subject . Examples are ' Go  away ' and ' Please be careful '. Clauses of this kind are typically used to tell someone to do something.
 \textit{
	\begin{itemize}
	\end{itemize}
}
\item countable noun \\
An \textbf{imperative} is a verb in the base form that is used, usually without a subject, in an imperative
clause.
 \textit{
	\begin{itemize}
	\end{itemize}
}
\end{enumerate}

\section*{glorious}
{\large \color{blue}  }
\subsection*{Explain}
\begin{enumerate}
\item adjective \\
Something that is \textbf{glorious} is very beautiful and impressive .
 \textit{
	\begin{itemize}
	\item ...a glorious rainbow in the air.
	\item She had missed the glorious blooms of the Mediterranean spring.
	\item ...a glorious Edwardian opera house.
	\end{itemize}
}
\item adjective \\
If you describe something as \textbf{glorious} , you are emphasizing that it is wonderful and it makes you feel very happy .
 \textit{
	\begin{itemize}
	\item The win revived glorious memories of his championship-winning days.
	\item We opened the windows and let in the glorious evening air.
	\end{itemize}
}
\item adjective \\
A \textbf{glorious}  career , victory , or occasion involves great fame or success .
 \textit{
	\begin{itemize}
	\item Harrison had a glorious career spanning more than six decades.
	\item Her future could be more glorious even than her past.
	\end{itemize}
}
\item adjective \\
\textbf{Glorious}  weather is hot and sunny .
 \textit{
	\begin{itemize}
	\item I got dressed and emerged into glorious sunshine.
	\item The sun was out again, and it was a glorious day.
	\end{itemize}
}
\end{enumerate}

\section*{lateral}
{\large \color{blue}  }
\subsection*{Explain}
\begin{enumerate}
\item adjective \\
\textbf{Lateral} means relating to the sides of something, or moving in a sideways  direction .
 \textit{
	\begin{itemize}
	\item McKinnon estimated the lateral movement of the bridge to be between four and six
inches.
	\end{itemize}
}
\end{enumerate}

\section*{initiative}
{\large \color{blue}  initiatives  }
\subsection*{Explain}
\begin{enumerate}
\item countable noun \\
An \textbf{initiative} is an important act or statement that is intended to solve a problem .
 \textit{
	\begin{itemize}
	\item Government initiatives to help young people have been inadequate.
	\item There's talk of a new peace initiative.
	\end{itemize}
}
\item singular noun \\
In a fight or contest , if you have \textbf{the initiative} , you are in a better position than your opponents to decide what to do next .
 \textit{
	\begin{itemize}
	\item We have the initiative; we intend to keep it.
	\item He paused enough to consider the options but never so long as to lose the initiative.
	\end{itemize}
}
\item uncountable noun \\
If you have \textbf{initiative} , you have the ability to decide what to do next and to do it, without needing other people to tell you what to do.
 \textit{
	\begin{itemize}
	\item She was disappointed by his lack of initiative.
	\item ...workers who are able to sort out problems on their own initiative.
	\end{itemize}
}
\item  \\
 take the initiative \textit{
	\begin{itemize}
	\end{itemize}
}
\end{enumerate}

\section*{levy}
{\large \color{blue}  levies  levying  levied  }
\subsection*{Explain}
\begin{enumerate}
\item countable noun \\
A \textbf{levy} is a sum of money that you have to pay , for example as a tax to the government .
 \textit{
	\begin{itemize}
	\item ...an annual motorway levy on all drivers.
	\end{itemize}
}
\item verb \\
If a government or organization  \textbf{levies} a tax or other sum of money, it demands it from people or organizations.
 \textit{
	\begin{itemize}
	\item They levied religious taxes on Christian commercial transactions.
	\item Taxes should not be levied without the authority of Parliament.
	\end{itemize}
}
\end{enumerate}

\section*{legitimate}
{\large \color{blue}  legitimates  legitimating  legitimated  }
\subsection*{Explain}
\begin{enumerate}
\item adjective \\
Something that is \textbf{legitimate} is acceptable according to the law.
 \textit{
	\begin{itemize}
	\item The French government has condemned the coup in Haiti and has demanded the restoration
of the legitimate government.
	\item The government will not seek to disrupt the legitimate business activities of the
defendant.
	\end{itemize}
}
\item adjective \\
If you say that something such as a feeling or claim is \textbf{legitimate} , you think that it is reasonable and justified .
 \textit{
	\begin{itemize}
	\item That's a perfectly legitimate fear.
	\item The New York Times has a legitimate claim to be a national newspaper.
	\end{itemize}
}
\item adjective \\
A \textbf{legitimate} child is one whose parents were married before he or she was born.
 \textit{
	\begin{itemize}
	\item We only married in order that the child should be legitimate.
	\end{itemize}
}
\item verb \\
To \textbf{legitimate} something means the same as to legitimize it.
 \textit{
	\begin{itemize}
	\item We want to legitimate this process by passing a law.
	\end{itemize}
}
\end{enumerate}

\section*{limb}
{\large \color{blue}  limbs  }
\subsection*{Explain}
\begin{enumerate}
\item countable noun \\
Your \textbf{limbs} are your arms and legs.
 \textit{
	\begin{itemize}
	\item She would be able to stretch out her cramped limbs and rest for a few hours.
	\end{itemize}
}
\item countable noun \\
The \textbf{limbs} of a tree are its branches.
 \textit{
	\begin{itemize}
	\item This entire rickety structure was hanging from the limb of an enormous leafy tree.
	\end{itemize}
}
\item  \\
 out on a limb \textit{
	\begin{itemize}
	\end{itemize}
}
\item  \\
 tear sb limb from limb \textit{
	\begin{itemize}
	\end{itemize}
}
\end{enumerate}

\section*{metric}
{\large \color{blue}  }
\subsection*{Explain}
\begin{enumerate}
\item adjective \\
\textbf{Metric} means relating to the metric system.
 \textit{
	\begin{itemize}
	\item Around 180,000 metric tons of food aid is required.
	\item Converting metric measurements to U.S. equivalents is easy.
	\end{itemize}
}
\end{enumerate}

\section*{lung}
{\large \color{blue}  lungs  }
\subsection*{Explain}
\begin{enumerate}
\item countable noun \\
Your \textbf{lungs} are the two organs inside your chest which fill with air when you breathe in.
 \textit{
	\begin{itemize}
	\end{itemize}
}
\end{enumerate}

\section*{military}
{\large \color{blue}  militaries  }
\subsection*{Explain}
\begin{enumerate}
\item adjective \\
\textbf{Military}  means relating to the armed forces of a country.
 \textit{
	\begin{itemize}
	\item Military action may become necessary.
	\item The president is sending in almost 20,000 military personnel to help with the relief
efforts.
	\item ...last year's military coup.
	\end{itemize}
}
\item adjective \\
\textbf{Military} means relating to or belonging to the army, rather than to the navy or the air force.
 \textit{
	\begin{itemize}
	\item The attack has caused severe damage to American naval and military forces.
	\end{itemize}
}
\item countable noun \\
\textbf{The military} are the armed forces of a country, especially  officers of high rank .
 \textit{
	\begin{itemize}
	\item The bombing has been far more widespread than the military will admit.
	\item Did you serve in the military?
	\end{itemize}
}
\item adjective \\
\textbf{Military} means well-organized , controlled, or neat , in a way that is typical of a soldier.
 \textit{
	\begin{itemize}
	\item Your working day will need to be organized with military precision.
	\item He has a military bearing, never failing to carry himself erect.
	\end{itemize}
}
\end{enumerate}

\section*{metal}
{\large \color{blue}  metals  }
\subsection*{Explain}
\begin{enumerate}
\item variable noun \\
\textbf{Metal} is a hard substance such as iron, steel, gold, or lead.
 \textit{
	\begin{itemize}
	\item ...pieces of furniture in wood, metal and glass.
	\item He hit his head against a metal bar.
	\end{itemize}
}
\end{enumerate}

\section*{minus}
{\large \color{blue}  minuses  }
\subsection*{Explain}
\begin{enumerate}
\item conjunction \\
You use \textbf{minus} to show that one number or quantity is being subtracted from another.
 \textit{
	\begin{itemize}
	\item One minus one is zero.
	\item They've been promised their full July salary minus the hardship payment.
	\end{itemize}
}
\item adjective \\
\textbf{Minus} before a number or quantity means that the number or quantity is less than zero.
 \textit{
	\begin{itemize}
	\item The aircraft was subjected to temperatures of minus 65 degrees and plus 120 degrees.
	\end{itemize}
}
\item  \\
Teachers use \textbf{minus} in grading work in schools and colleges . ' B minus' is not as good as 'B', but is a better grade than 'C'.
 \textit{
	\begin{itemize}
	\item I'm giving him a B minus.
	\end{itemize}
}
\item preposition \\
To be \textbf{minus} something means not to have that thing.
 \textit{
	\begin{itemize}
	\item The film company collapsed, leaving Chris jobless and minus his life savings.
	\end{itemize}
}
\item countable noun \\
A \textbf{minus} is a disadvantage.
 \textit{
	\begin{itemize}
	\item The minuses far outweigh that possible gain.
	\item The plusses and minuses were about equal.
	\item None of these minus points will have been mentioned.
	\end{itemize}
}
\item  \\
 plus or minus \textit{
	\begin{itemize}
	\end{itemize}
}
\end{enumerate}

\section*{money}
{\large \color{blue}  monies  moneys  }
\subsection*{Explain}
\begin{enumerate}
\item uncountable noun \\
\textbf{Money} is the coins or bank notes that you use to buy things, or the sum that you have in a bank account .
 \textit{
	\begin{itemize}
	\item A lot of the money that you pay at the cinema goes back to the film distributors.
	\item Players should be allowed to earn money from advertising.
	\item She probably had more money but she didn't spend it.
	\item ...discounts and money saving offers.
	\end{itemize}
}
\item plural noun \\
\textbf{Monies} is used to refer to several separate sums of money that form part of a larger amount that is received or spent .
 \textit{
	\begin{itemize}
	\item We drew up a schedule of payments for the rest of the monies owed.
	\item ...the investment and management of monies by pension funds.
	\end{itemize}
}
\item  \\
 to have money to burn \textit{
	\begin{itemize}
	\end{itemize}
}
\item  \\
 in the money \textit{
	\begin{itemize}
	\end{itemize}
}
\item  \\
 make money \textit{
	\begin{itemize}
	\end{itemize}
}
\item  \\
 to put your money where your mouth is \textit{
	\begin{itemize}
	\end{itemize}
}
\item  \\
 the smart money \textit{
	\begin{itemize}
	\end{itemize}
}
\item  \\
 money talks \textit{
	\begin{itemize}
	\end{itemize}
}
\item  \\
 to throw money at something \textit{
	\begin{itemize}
	\end{itemize}
}
\item  \\
 to throw good money after bad \textit{
	\begin{itemize}
	\end{itemize}
}
\item  \\
 (get your) money's worth \textit{
	\begin{itemize}
	\end{itemize}
}
\end{enumerate}

\section*{nephew}
{\large \color{blue}  nephews  }
\subsection*{Explain}
\begin{enumerate}
\item countable noun \\
Someone's \textbf{nephew} is the son of their sister or brother.
 \textit{
	\begin{itemize}
	\item I am planning a 25th birthday party for my nephew.
	\end{itemize}
}
\end{enumerate}

\section*{multiple}
{\large \color{blue}  multiples  }
\subsection*{Explain}
\begin{enumerate}
\item adjective \\
You use \textbf{multiple} to describe things that consist of many parts, involve many people, or have many uses.
 \textit{
	\begin{itemize}
	\item He died of multiple injuries.
	\item The most common multiple births are twins, two babies born at the same time.
	\end{itemize}
}
\item countable noun \\
If one number is a \textbf{multiple of} a smaller number, it can be exactly divided by that smaller number.
 \textit{
	\begin{itemize}
	\item Their numerical system, derived from the Babylonians, was based on multiples of the
number six.
	\end{itemize}
}
\item countable noun \\
A \textbf{multiple} or a \textbf{multiple store} is a shop with a lot of branches in different towns.
 \textit{
	\begin{itemize}
	\item It made it almost impossible for the smaller retailer to compete against the multiples.
	\end{itemize}
}
\end{enumerate}

\section*{niece}
{\large \color{blue}  nieces  }
\subsection*{Explain}
\begin{enumerate}
\item countable noun \\
Someone's \textbf{niece} is the daughter of their sister or brother.
 \textit{
	\begin{itemize}
	\item ...his niece from America, the daughter of his eldest sister.
	\end{itemize}
}
\end{enumerate}

\section*{muscular}
{\large \color{blue}  }
\subsection*{Explain}
\begin{enumerate}
\item adjective \\
\textbf{Muscular} means involving or affecting your muscles.
 \textit{
	\begin{itemize}
	\item As a general rule, all muscular effort is enhanced by breathing in as the effort
is made.
	\item Early symptoms include anorexia, muscular weakness and fatigue.
	\end{itemize}
}
\item adjective \\
If a person or their body is \textbf{muscular} , they are very fit and strong, and have firm muscles which are not covered with a lot of fat .
 \textit{
	\begin{itemize}
	\item Like most female athletes, she was lean and muscular.
	\item ...his tanned muscular legs.
	\end{itemize}
}
\end{enumerate}

\section*{order}
{\large \color{blue}  }
\subsection*{Explain}
\begin{enumerate}
\item phrase \\
If you do something \textbf{in order to}  achieve a particular thing or \textbf{in order that} something can happen , you do it because you want to achieve that thing.
 \textit{
	\begin{itemize}
	\item Most schools are extremely unwilling to cut down on staff in order to cut costs.
	\item ...asking them to risk their lives in order that the rest of us can sleep better.
	\end{itemize}
}
\item phrase \\
If someone must be in a particular situation \textbf{in order to} achieve something they want, they cannot achieve that thing if they are not in that
situation.
 \textit{
	\begin{itemize}
	\item We need to get rid of the idea that we must be liked all the time in order to be
worthwhile.
	\item They need hostages in order to bargain with the government.
	\end{itemize}
}
\item phrase \\
If something must happen \textbf{in order for} something else to happen, the second thing cannot happen if the first thing does
not happen.
 \textit{
	\begin{itemize}
	\item In order for our muscles to work efficiently they need oxygen, which is provided
by the lungs.
	\end{itemize}
}
\end{enumerate}

\section*{nominal}
{\large \color{blue}  }
\subsection*{Explain}
\begin{enumerate}
\item adjective \\
You use \textbf{nominal} to indicate that someone or something is supposed to have a particular identity or status , but in reality does not have it.
 \textit{
	\begin{itemize}
	\item As he was still not allowed to run a company, his partner became its nominal head.
	\item I was brought up a nominal Christian.
	\end{itemize}
}
\item adjective \\
A \textbf{nominal}  price or sum of money is very small in comparison with the real cost or value of the thing that is being bought or sold .
 \textit{
	\begin{itemize}
	\item I am prepared to sell my shares at a nominal price.
	\item All the ferries carry bicycles free or for a nominal charge.
	\end{itemize}
}
\item adjective \\
In economics , the \textbf{nominal} value, rate , or level of something is the one expressed in terms of current prices or figures , without taking into account  general  changes in prices that take place over time.
 \textit{
	\begin{itemize}
	\item Inflation would be lower and so nominal rates would be rather more attractive in
real terms.
	\item In 1990 personal incomes grew a nominal 6.8 per cent.
	\end{itemize}
}
\end{enumerate}

\section*{ownership}
{\large \color{blue}  }
\subsection*{Explain}
\begin{enumerate}
\item uncountable noun \\
\textbf{Ownership} of something is the state of owning it.
 \textit{
	\begin{itemize}
	\item They decided to relax their rules on the foreign ownership of their airlines.
	\item ...the growth of home ownership in Britain.
	\item He said that anyone trying to export goods without proof of ownership would have
them seized.
	\end{itemize}
}
\end{enumerate}

\section*{numerous}
{\large \color{blue}  }
\subsection*{Explain}
\begin{enumerate}
\item adjective \\
If people or things are \textbf{numerous} , they exist or are present in large numbers .
 \textit{
	\begin{itemize}
	\item Sex crimes were just as numerous as they are today.
	\item Despite numerous attempts to diet, her weight soared.
	\end{itemize}
}
\end{enumerate}

\section*{plate}
{\large \color{blue}  plates  }
\subsection*{Explain}
\begin{enumerate}
\item countable noun \\
A \textbf{plate} is a round or oval flat dish that is used to hold food.
 A \textbf{plate}  \textbf{of} food is the amount of food on the plate.
 \textit{
	\begin{itemize}
	\item Anita pushed her plate away; she had eaten virtually nothing.
	\item ...a huge plate of spaghetti.
	\end{itemize}
}
\item countable noun \\
A \textbf{plate} is a flat piece of metal, especially on machinery or a building.
 \textit{
	\begin{itemize}
	\end{itemize}
}
\item countable noun \\
A \textbf{plate} is a small, flat piece of metal with someone's name written on it, which you usually
 find beside the front door of an office or house.
 \textit{
	\begin{itemize}
	\end{itemize}
}
\item plural noun \\
On a road vehicle, the \textbf{plates} are the panels at the front and back which display the license number in the United States, and the registration number in Britain.
 \textit{
	\begin{itemize}
	\item ...dusty-looking cars with New Jersey plates.
	\end{itemize}
}
\item uncountable noun \\
\textbf{Plate} is dishes, bowls , and cups that are made of precious metal, especially silver or gold.
 \textit{
	\begin{itemize}
	\item ...gold and silver plate, jewellery, and roomfuls of antique furniture.
	\end{itemize}
}
\item countable noun \\
In printing, a \textbf{plate} is a sheet of metal which is carved or specially treated with chemicals so that it can be used to print text or pictures.
 \textit{
	\begin{itemize}
	\end{itemize}
}
\item countable noun \\
In photography , a \textbf{plate} is a thin sheet of glass that is covered with a layer of chemicals which react to the light and on which an image can be formed.
 \textit{
	\begin{itemize}
	\end{itemize}
}
\item countable noun \\
A \textbf{plate} in a book is a picture or photograph which takes up a whole page and is usually printed on better quality paper than the rest of the book.
 \textit{
	\begin{itemize}
	\item Fermor's book has 55 colour plates.
	\end{itemize}
}
\item countable noun \\
In a microscope , the \textbf{plate} is a small rectangular piece of glass onto which you put a small amount of the substance that you want to look at. You then slide the plate under the microscope to look at the substance.
 \textit{
	\begin{itemize}
	\end{itemize}
}
\item countable noun \\
A dental  \textbf{plate} is a piece of plastic which is shaped to fit  inside a person's mouth and which a set of false teeth is attached to.
 \textit{
	\begin{itemize}
	\end{itemize}
}
\item countable noun \\
In geology , a \textbf{plate} is a large piece of the Earth's surface, perhaps as large as a continent , which moves very slowly.
 \textit{
	\begin{itemize}
	\item The United States Geological Survey has revealed that the earthquake was not caused
by a simple horizontal movement of one plate past another.
	\end{itemize}
}
\item countable noun \\
In baseball , \textbf{the}  \textbf{plate} is the same as the \textbf{home plate} .
 \textit{
	\begin{itemize}
	\end{itemize}
}
\item  \\
 have enough on one's plate/have a lot on one's plate \textit{
	\begin{itemize}
	\end{itemize}
}
\item  \\
 hand sth to sb on a plate \textit{
	\begin{itemize}
	\end{itemize}
}
\end{enumerate}

\section*{overall}
{\large \color{blue}  overalls  }
\subsection*{Explain}
\begin{enumerate}
\item adjective \\
You use \textbf{overall} to indicate that you are talking about a situation in general or about the whole of something.
 \textbf{Overall} is also an adverb .
 \textit{
	\begin{itemize}
	\item ...the overall rise in unemployment.
	\item Cut down your overall amount of physical activity.
	\item It is usually the woman who assumes overall care of the baby.
	\item Overall, I like Connie. I think she's great.
	\item Overall I was disappointed.
	\item The college has few ways to assess the quality of education overall.
	\end{itemize}
}
\item plural noun \\
\textbf{Overalls} consist of a single piece of clothing that combines  trousers and a jacket . You wear overalls over your clothes in order to protect them while you are working .
 \textit{
	\begin{itemize}
	\item ...workers in blue overalls.
	\end{itemize}
}
\item plural noun \\
\textbf{Overalls} are trousers that are attached to a piece of cloth which covers your chest and which has straps  going over your shoulders .
 \textit{
	\begin{itemize}
	\item An elderly man dressed in faded overalls took the witness stand.
	\end{itemize}
}
\item countable noun \\
An \textbf{overall} is a piece of clothing shaped like a coat that you wear over your clothes in order to protect them while you are working.
 \textit{
	\begin{itemize}
	\end{itemize}
}
\end{enumerate}

\section*{pneumonia}
{\large \color{blue}  }
\subsection*{Explain}
\begin{enumerate}
\item uncountable noun \\
\textbf{Pneumonia} is a serious disease which affects your lungs and makes it difficult for you to breathe .
 \textit{
	\begin{itemize}
	\item She nearly died of pneumonia.
	\end{itemize}
}
\end{enumerate}

\section*{quiet}
{\large \color{blue}  quieter  quietest  quiets  quieting  quieted  }
\subsection*{Explain}
\begin{enumerate}
\item adjective \\
Someone or something that is \textbf{quiet} makes only a small amount of noise.
 \textit{
	\begin{itemize}
	\item Tania kept the children reasonably quiet and contented.
	\item A quiet murmur passed through the classroom.
	\item The airlines have invested enormous sums in new, quieter aircraft.
	\end{itemize}
}
\item adjective \\
If a place is \textbf{quiet} , there is very little noise there.
 \textit{
	\begin{itemize}
	\item She was received in a small, quiet office.
	\item The street was unnaturally quiet.
	\end{itemize}
}
\item adjective \\
If a place, situation , or time is \textbf{quiet} , there is no excitement , activity, or trouble .
 \textit{
	\begin{itemize}
	\item ...a quiet rural backwater.
	\item It is very quiet without him.
	\item While he wanted Los Angeles and partying, she wanted a quiet life.
	\item The city is now relatively quiet but there's palpable anger as people cope with shortages.
	\end{itemize}
}
\item uncountable noun \\
\textbf{Quiet} is silence .
 \textit{
	\begin{itemize}
	\item He called for quiet and announced that the next song was in our honor.
	\item Jeremy wants some peace and quiet before his big match.
	\end{itemize}
}
\item adjective \\
If you are \textbf{quiet} , you are not saying anything.
 \textit{
	\begin{itemize}
	\item I told them to be quiet and go to sleep.
	\item I just went quiet, embarrassed, and couldn't answer.
	\item They were both quiet for a while. Then Charlie said: 'I must go.'.
	\item Then a voice called out, 'Quiet, everybody, please!'
	\end{itemize}
}
\item adjective \\
If you refer , for example , to someone's \textbf{quiet}  confidence or \textbf{quiet}  despair , you mean that they do not say much about the way they are feeling .
 \textit{
	\begin{itemize}
	\item He has a quiet confidence in his ability.
	\item All through his life he has shown a quiet determination to get things done.
	\end{itemize}
}
\item adjective \\
You describe activities as \textbf{quiet} when they happen in secret or in such a way that people do not notice them.
 \textit{
	\begin{itemize}
	\item The Swedes had sought his freedom through quiet diplomacy.
	\item Then it was back to the house for a quiet celebration.
	\item Can I have a quiet word with you, son?
	\end{itemize}
}
\item adjective \\
A \textbf{quiet} person behaves in a calm way and is not easily made angry or upset .
 \textit{
	\begin{itemize}
	\item He's a nice quiet man.
	\end{itemize}
}
\item graded adjective \\
You describe colours or clothes as \textbf{quiet} when they are not bright or not very noticeable .
 \textit{
	\begin{itemize}
	\item They dress in quiet colors so as not to call attention to themselves.
	\end{itemize}
}
\item verb \\
If someone or something \textbf{quiets} or if you \textbf{quiet} them, they become less noisy , less active , or silent.
 \textit{
	\begin{itemize}
	\item The wind dropped and the sea quieted.
	\item He quieted the crowd with a gesture.
	\end{itemize}
}
\item verb \\
To \textbf{quiet}  fears or complaints means to persuade people that there is no good reason for them.
 \textit{
	\begin{itemize}
	\item Music seemed to quiet her anxiety and loneliness.
	\end{itemize}
}
\item  \\
 go quietly \textit{
	\begin{itemize}
	\end{itemize}
}
\item  \\
 keep quiet about sth/keep sth quiet \textit{
	\begin{itemize}
	\end{itemize}
}
\item  \\
 on the quiet \textit{
	\begin{itemize}
	\end{itemize}
}
\end{enumerate}

\section*{poverty}
{\large \color{blue}  }
\subsection*{Explain}
\begin{enumerate}
\item uncountable noun \\
\textbf{Poverty} is the state of being extremely  poor .
 \textit{
	\begin{itemize}
	\item ...people living in absolute poverty.
	\item Garvey died in loneliness and poverty.
	\end{itemize}
}
\item singular noun \\
You can use \textbf{poverty} to refer to any situation in which there is not enough of something or its quality is poor.
 \textit{
	\begin{itemize}
	\item Britain has suffered from a poverty of ambition.
	\item ...a poverty of ideas.
	\end{itemize}
}
\end{enumerate}

\section*{reasonable}
{\large \color{blue}  }
\subsection*{Explain}
\begin{enumerate}
\item adjective \\
If you think that someone is fair and sensible you can  say that they are \textbf{reasonable} .
 \textit{
	\begin{itemize}
	\item He's a reasonable sort of chap.
	\item Oh, come on, be reasonable.
	\end{itemize}
}
\item adjective \\
If you say that a decision or action is \textbf{reasonable} , you mean that it is fair and sensible.
 \textit{
	\begin{itemize}
	\item ...a perfectly reasonable decision.
	\item At the time, what he'd done had seemed reasonable.
	\end{itemize}
}
\item adjective \\
If you say that an expectation or explanation is \textbf{reasonable} , you mean that there are good reasons why it may be correct .
 \textit{
	\begin{itemize}
	\item It seems reasonable to expect rapid urban growth.
	\end{itemize}
}
\item adjective \\
If you say that the price of something is \textbf{reasonable} , you mean that it is fair and not too high .
 \textit{
	\begin{itemize}
	\item You get an interesting meal for a reasonable price.
	\item His fees were quite reasonable.
	\end{itemize}
}
\item adjective \\
You can use \textbf{reasonable} to describe something that is fairly good, but not very good.
 \textit{
	\begin{itemize}
	\item The boy answered him in reasonable French.
	\item He had never been able to make a reasonable living from his writing.
	\end{itemize}
}
\item adjective \\
A \textbf{reasonable}  amount of something is a fairly large amount of it.
 \textit{
	\begin{itemize}
	\item They will need a reasonable amount of desk area and good light.
	\end{itemize}
}
\end{enumerate}

\section*{profile}
{\large \color{blue}  profiles  profiling  profiled  }
\subsection*{Explain}
\begin{enumerate}
\item countable noun \\
Your \textbf{profile} is the outline of your face as it is seen when someone is looking at you from the side.
 \textit{
	\begin{itemize}
	\item His handsome profile was turned away from us.
	\end{itemize}
}
\item uncountable noun \\
If you see someone \textbf{in profile} , you see them from the side.
 \textit{
	\begin{itemize}
	\item This picture shows the girl in profile.
	\end{itemize}
}
\item countable noun \\
A \textbf{profile of} someone is a short article or programme in which their life and character are described .
 \textit{
	\begin{itemize}
	\item A newspaper published profiles of the candidates.
	\end{itemize}
}
\item verb \\
To \textbf{profile} someone means to give an account of that person's life and character.
 \textit{
	\begin{itemize}
	\item Tamar Golan, a Paris-based journalist, profiles the rebel leader.
	\end{itemize}
}
\item countable noun \\
Your \textbf{profile} on a social media website is the part where you post your name, picture , and personal information.
 \textit{
	\begin{itemize}
	\item He lied about himself on his profile.
	\end{itemize}
}
\item  \\
 high profile \textit{
	\begin{itemize}
	\end{itemize}
}
\end{enumerate}

\section*{redundant}
{\large \color{blue}  }
\subsection*{Explain}
\begin{enumerate}
\item adjective \\
If you are made \textbf{redundant} , your employer  tells you to leave because your job is no longer necessary or because your employer cannot
 afford to keep paying you.
 \textit{
	\begin{itemize}
	\item My husband was made redundant late last year.
	\item ...a redundant miner.
	\end{itemize}
}
\item adjective \\
Something that is \textbf{redundant} is no longer needed because its job is being done by something else or because its job is no longer necessary
or useful .
 \textit{
	\begin{itemize}
	\item Changes in technology may mean that once-valued skills are now redundant.
	\item ...the conversion of redundant buildings to residential use.
	\end{itemize}
}
\end{enumerate}

\section*{pyramid}
{\large \color{blue}  pyramids  }
\subsection*{Explain}
\begin{enumerate}
\item countable noun \\
\textbf{Pyramids} are ancient stone buildings with four triangular sloping sides. The most famous pyramids are those built in ancient Egypt to contain the bodies of their kings and queens .
 \textit{
	\begin{itemize}
	\item We set off to see the Pyramids and Sphinx.
	\end{itemize}
}
\item countable noun \\
A \textbf{pyramid} is a shape, object, or pile of things with a flat base and sloping triangular sides that meet at a point.
 \textit{
	\begin{itemize}
	\item On a plate in front of him was piled a pyramid of flat white biscuits.
	\end{itemize}
}
\item countable noun \\
You can describe something as a \textbf{pyramid} when it is organized so that there are fewer people at each level as you go towards the top.
 \textit{
	\begin{itemize}
	\item The structure of the group is more like a loose net than a hierarchical pyramid.

	\end{itemize}
}
\end{enumerate}

\section*{respective}
{\large \color{blue}  }
\subsection*{Explain}
\begin{enumerate}
\item adjective \\
\textbf{Respective} means relating or belonging separately to the individual people you have just mentioned .
 \textit{
	\begin{itemize}
	\item Steve and I were at very different stages in our respective careers.
	\item They went into their respective bedrooms to pack.
	\end{itemize}
}
\end{enumerate}

\section*{quality}
{\large \color{blue}  qualities  }
\subsection*{Explain}
\begin{enumerate}
\item uncountable noun \\
The \textbf{quality} of something is how good or bad it is.
 \textit{
	\begin{itemize}
	\item Everyone can greatly improve the quality of life.
	\item Other services vary dramatically in quality.
	\item ...high-quality paper and plywood.
	\end{itemize}
}
\item uncountable noun \\
Something of \textbf{quality} is of a high standard.
 \textit{
	\begin{itemize}
	\item ...a college of quality.
	\item In our work, quality is paramount.
	\item We have been successful because we are offering a quality service.
	\end{itemize}
}
\item countable noun \\
Someone's \textbf{qualities} are the good characteristics that they have which are part of their nature.
 \textit{
	\begin{itemize}
	\item Sometimes you wonder where your kids get their good qualities.
	\item He wanted to introduce mature people with leadership qualities.
	\item A job analysis should also include what skills and personal qualities are required.
	\end{itemize}
}
\item countable noun \\
You can describe a particular characteristic of a person or thing as a \textbf{quality} .
 \textit{
	\begin{itemize}
	\item ...a childlike quality.
	\item ...the pretentious quality of the poetry.
	\item Thyme tea can be used by adults for its antiseptic qualities.
	\end{itemize}
}
\item adjective \\
The \textbf{quality papers} or the \textbf{quality press} are the more serious  newspapers which give detailed  accounts of world events , as well as reports on business , culture , and society .
 \textit{
	\begin{itemize}
	\item Even the quality papers agreed that it was a triumph.
	\end{itemize}
}
\end{enumerate}

\section*{safe}
{\large \color{blue}  safer  safest  safes  }
\subsection*{Explain}
\begin{enumerate}
\item adjective \\
Something that is \textbf{safe} does not cause physical harm or danger.
 \textit{
	\begin{itemize}
	\item Officials arrived to assess whether it is safe to bring emergency food supplies into
the city.
	\item Most foods that we eat are safe for birds.
	\item ...a safe and reliable birth control option.
	\item ...a programme to make nuclear reactors safer.
	\end{itemize}
}
\item adjective \\
If a person or thing is \textbf{safe}  \textbf{from} something, they cannot be harmed or damaged by it.
 \textit{
	\begin{itemize}
	\item In the future people can go to a football match knowing that they are safe from hooliganism.
	\item There are lots of gadgets you can install to make your home safer.
	\end{itemize}
}
\item adjective \\
If you are \textbf{safe} , you have not been harmed, or you are not in danger of being harmed.
 \textit{
	\begin{itemize}
	\item Where is Sophy? Is she safe?
	\item A baby boy is safe after rescue workers pulled him from a 12-foot-deep construction
hole.
	\end{itemize}
}
\item adjective \\
A \textbf{safe} place is one where it is unlikely that any harm, damage, or unpleasant things will  happen to the people or things that are there.
 \textit{
	\begin{itemize}
	\item Many refugees have fled to safer areas.
	\item The elimination of all nuclear weapons would make the world a safer place.
	\item We shall take the treasure away to a safe place.
	\end{itemize}
}
\item adjective \\
If people or things have a \textbf{safe}  journey , they reach their destination without harm, damage, or unpleasant things happening to them.
 \textit{
	\begin{itemize}
	\item 'I'm heading back home for the weekend.'—'Have a safe journey.'
	\item ...the U.N. plan to deploy 500 troops to ensure the safe delivery of food and other
supplies.
	\end{itemize}
}
\item adjective \\
If you are at a \textbf{safe}  distance from something or someone, you are far enough away from them to avoid any danger, harm, or unpleasant effects .
 \textit{
	\begin{itemize}
	\item I shall conceal myself at a safe distance from the battlefield.
	\item He thinks he can find a way to vaccinate the elephants from a safe distance.
	\end{itemize}
}
\item adjective \\
If something you have or expect to obtain is \textbf{safe} , you cannot lose it or be prevented from having it.
 \textit{
	\begin{itemize}
	\item We as consumers need to feel confident that our jobs are safe before we will spend
spare cash.
	\item Is the National Health Service safe with the Conservative party?
	\item Rovers made the game safe with a spectacular second goal in the 84th minute.
	\end{itemize}
}
\item adjective \\
A \textbf{safe}  course of action is one in which there is very little risk of loss or failure .
 \textit{
	\begin{itemize}
	\item Electricity shares are still a safe investment.
	\item Baldwin called the 1935 election a year early to get the country behind rearmament
- not the politically safe option.
	\end{itemize}
}
\item adjective \\
If you disapprove of something because you think it is not very exciting or original , you can  describe it as \textbf{safe} .
 \textit{
	\begin{itemize}
	\item ...frustrated artists who became lawyers at an early age because it seemed a safe
option.
	\item Rock'n'roll has become so commercialised and safe since punk.
	\end{itemize}
}
\item adjective \\
If \textbf{it is}  \textbf{safe}  \textbf{to}  say or assume something, you can say it with very little risk of being wrong .
 \textit{
	\begin{itemize}
	\item It is safe to say that some students make no effort to do quality work in school.
	\item The only safe assumption is that the world's financial markets will have to find
solutions themselves.
	\end{itemize}
}
\item graded adjective \\
If you say to someone that their secret is \textbf{safe}  \textbf{with} you, you are promising not to tell it to anyone.
 \textit{
	\begin{itemize}
	\item Don't worry, Mr Palin, your secret is safe with me.
	\end{itemize}
}
\item countable noun \\
A \textbf{safe} is a strong metal cupboard with special locks, in which you keep money, jewellery , or other valuable things.
 \textit{
	\begin{itemize}
	\item The files are now in a safe to which only he has the key.
	\end{itemize}
}
\item  \\
 in safe hands / safe in someone's hands \textit{
	\begin{itemize}
	\end{itemize}
}
\item  \\
 safe as houses \textit{
	\begin{itemize}
	\end{itemize}
}
\item  \\
 make somewhere safe for something \textit{
	\begin{itemize}
	\end{itemize}
}
\item  \\
 to play safe \textit{
	\begin{itemize}
	\end{itemize}
}
\item  \\
 to be on the safe side \textit{
	\begin{itemize}
	\end{itemize}
}
\item  \\
 it's better to be safe than sorry \textit{
	\begin{itemize}
	\end{itemize}
}
\item  \\
 safe and sound \textit{
	\begin{itemize}
	\end{itemize}
}
\end{enumerate}

\section*{reptile}
{\large \color{blue}  reptiles  }
\subsection*{Explain}
\begin{enumerate}
\item countable noun \\
\textbf{Reptiles} are a group of cold-blooded animals which have skins covered with small hard plates called scales and lay eggs. Snakes, lizards, and crocodiles are reptiles.
 \textit{
	\begin{itemize}
	\end{itemize}
}
\end{enumerate}

\section*{secondary}
{\large \color{blue}  }
\subsection*{Explain}
\begin{enumerate}
\item adjective \\
If you describe something as \textbf{secondary} , you mean that it is less important than something else.
 \textit{
	\begin{itemize}
	\item The street erupted in a huge explosion, with secondary explosions in the adjoining
buildings.
	\item They argue that human rights considerations are now of only secondary importance.
	\item The actual damage to the brain cells is secondary to the damage caused to the blood
supply.
	\end{itemize}
}
\item adjective \\
\textbf{Secondary}  diseases or infections  happen as a result of another disease or infection that has already happened.
 \textit{
	\begin{itemize}
	\item He had kidney cancer, with secondary tumours in the brain and lungs.
	\end{itemize}
}
\item adjective \\
\textbf{Secondary} education is given to pupils between the ages of 11 or 12 and 17 or 18.
 \textit{
	\begin{itemize}
	\item She went to a 'very minor' private school for the whole of her secondary education.
	\end{itemize}
}
\end{enumerate}

\section*{routine}
{\large \color{blue}  routines  }
\subsection*{Explain}
\begin{enumerate}
\item variable noun \\
A \textbf{routine} is the usual series of things that you do at a particular time. A \textbf{routine} is also the practice of regularly doing things in a fixed order.
 \textit{
	\begin{itemize}
	\item The players had to change their daily routine and lifestyle.
	\item They include the floor exercises as a regular part of their fitness routine.
	\item He checked up on you as a matter of routine.
	\end{itemize}
}
\item adjective \\
You use \textbf{routine} to describe activities that are done as a normal part of a job or process.
 \textit{
	\begin{itemize}
	\item ...a series of routine medical tests.
	\item The operator has to be able to carry out routine maintenance of the machine.
	\end{itemize}
}
\item adjective \\
A \textbf{routine}  situation , action, or event is one which seems completely ordinary , rather than interesting , exciting , or different .
 \textit{
	\begin{itemize}
	\item So many days are routine and uninteresting, especially in winter.
	\item ...this routine thriller about a CIA man and a KGB operative.
	\end{itemize}
}
\item variable noun \\
You use \textbf{routine} to refer to a way of life that is uninteresting and ordinary, or hardly  ever changes.
 \textit{
	\begin{itemize}
	\item ...the mundane routine of her life.
	\item Family holidays are meant to be a break from routine.
	\end{itemize}
}
\item countable noun \\
A \textbf{routine} is a computer program, or part of a program, that performs a specific function.
 \textit{
	\begin{itemize}
	\item ... an installation routine.
	\end{itemize}
}
\item countable noun \\
A \textbf{routine} is a short sequence of jokes , remarks , actions, or movements that forms part of a longer performance .
 \textit{
	\begin{itemize}
	\item ... like a Marx Brothers routine.
	\item ...an athletic dance routine.
	\end{itemize}
}
\end{enumerate}

\section*{solemn}
{\large \color{blue}  }
\subsection*{Explain}
\begin{enumerate}
\item adjective \\
Someone or something that is \textbf{solemn} is very serious rather than cheerful or humorous .
 \textit{
	\begin{itemize}
	\item His solemn little face broke into smiles.
	\item He looked solemn.
	\end{itemize}
}
\item adjective \\
A \textbf{solemn}  promise or agreement is one that you make in a very formal , sincere way.
 \textit{
	\begin{itemize}
	\item ...a solemn pledge that he would never remarry.
	\end{itemize}
}
\end{enumerate}

\section*{scout}
{\large \color{blue}  scouts  scouting  scouted  }
\subsection*{Explain}
\begin{enumerate}
\item countable noun \\
A \textbf{scout} is someone who is sent to an area of countryside to find out the position of an enemy army .
 \textit{
	\begin{itemize}
	\item They sent two men out in front as scouts.
	\end{itemize}
}
\item countable noun \\
A \textbf{scout} is the same as a talent scout .
 \textit{
	\begin{itemize}
	\end{itemize}
}
\item verb \\
If you \textbf{scout}  somewhere  \textbf{for} something, you go through that area searching for it.
 \textit{
	\begin{itemize}
	\item I wouldn't have time to scout the area for junk.
	\item A team of four was sent to scout for a nuclear test site.
	\item I have people scouting the hills already.
	\end{itemize}
}
\end{enumerate}

\section*{spare}
{\large \color{blue}  spares  sparing  spared  }
\subsection*{Explain}
\begin{enumerate}
\item adjective \\
You use \textbf{spare} to describe something that is the same as things that you are already using, but that you do not need yet and are keeping ready in case another one is needed.
 \textbf{Spare} is also a noun .
 \textit{
	\begin{itemize}
	\item If possible keep a spare pair of glasses accessible in case your main pair is broken
or lost.
	\item Don't forget to take a few spare batteries.
	\item He could have taken a spare key.
	\item The wagons carried spare ammunition.
	\item Give me the trunk key and I'll get the spare.
	\end{itemize}
}
\item adjective \\
You use \textbf{spare} to describe something that is not being used by anyone, and is therefore available for someone to use.
 \textit{
	\begin{itemize}
	\item They don't have a lot of spare cash.
	\item The spare bedroom is on the second floor.
	\item There was hardly a spare inch of space to be found.
	\end{itemize}
}
\item verb \\
If you have something such as time, money, or space \textbf{to spare} , you have some extra time, money, or space that you have not used or which you do not need.
 \textit{
	\begin{itemize}
	\item You got here with ninety seconds to spare.
	\item It's not as if he has money to spare.
	\item The car suddenly darted ahead, squeezing past him with only inches to spare.
	\item Miranda has drive and energy to spare and has now taken on an even bigger challenge.
	\end{itemize}
}
\item verb \\
If you \textbf{spare} time or another resource  \textbf{for} a particular purpose, you make it available for that purpose.
 \textit{
	\begin{itemize}
	\item She said that she could only spare 35 minutes for our meeting.
	\item He's a very busy man, and it's good of him to spare the time to visit.
	\item He suggested that his country could not spare the troops for such an operation.
	\end{itemize}
}
\item verb \\
If a person or a place \textbf{is spared} , they are not harmed, even though other people or places have been.
 \textit{
	\begin{itemize}
	\item We have lost everything, but thank God, our lives have been spared.
	\item Not a man was spared.
	\item The area was largely spared from the famine.
	\end{itemize}
}
\item verb \\
If you \textbf{spare} someone an unpleasant  experience , you prevent them from suffering it.
 \textit{
	\begin{itemize}
	\item I wanted to spare Frances the embarrassment of discussing this subject.
	\item Prisoners are spared the indignity of wearing uniforms.
	\item Spare me the gory details.
	\item She's just trying to spare Shawna's feelings.
	\item The policy has not spared the farming community from severe financial pressure.
	\end{itemize}
}
\item graded adjective \\
Someone who is described as \textbf{spare} is tall and not at all fat .
 \textit{
	\begin{itemize}
	\item She was thin and spare, with a sharply intelligent face.
	\end{itemize}
}
\item graded adjective \\
Something such as a room that is \textbf{spare} is very plain with no unnecessary  features .
 \textit{
	\begin{itemize}
	\item Inside, the two small rooms were spare and neat, stripped bare of ornaments.
	\end{itemize}
}
\item  \\
 spare no effort \textit{
	\begin{itemize}
	\end{itemize}
}
\item  \\
 spare a thought for \textit{
	\begin{itemize}
	\end{itemize}
}
\end{enumerate}

\section*{shuttle}
{\large \color{blue}  shuttles  shuttling  shuttled  }
\subsection*{Explain}
\begin{enumerate}
\item countable noun \\
A \textbf{shuttle} is the same as a space shuttle .
 \textit{
	\begin{itemize}
	\end{itemize}
}
\item countable noun \\
A \textbf{shuttle} is a plane , bus, or train which makes frequent journeys between two places.
 \textit{
	\begin{itemize}
	\item ...the BA shuttle to Glasgow.
	\item ...shuttle flights between London and Manchester.
	\end{itemize}
}
\item verb \\
If someone or something \textbf{shuttles} or \textbf{is shuttled} from one place to another place, they frequently go from one place to the other.
 \textit{
	\begin{itemize}
	\item He and colleagues have shuttled back and forth between the three capitals.
	\item Machine parts were also being shuttled across the border without authorisation.
	\end{itemize}
}
\item countable noun \\
A \textbf{shuttle} is a piece of equipment used in weaving. It takes a thread backwards and forwards over the other threads in order to make a piece of cloth .
 \textit{
	\begin{itemize}
	\end{itemize}
}
\end{enumerate}

\section*{spectacular}
{\large \color{blue}  spectaculars  }
\subsection*{Explain}
\begin{enumerate}
\item adjective \\
Something that is \textbf{spectacular} is very impressive or dramatic.
 \textit{
	\begin{itemize}
	\item ...spectacular views of the Sugar Loaf Mountain.
	\item They have revamped the business with spectacular success.
	\item The results have been spectacular.
	\end{itemize}
}
\item countable noun \\
A \textbf{spectacular} is a show or performance which is very grand and impressive.
 \textit{
	\begin{itemize}
	\item ...a television spectacular.
	\item ...one of the world's great sporting spectaculars.
	\end{itemize}
}
\end{enumerate}

\section*{slum}
{\large \color{blue}  slums  slumming  slummed  }
\subsection*{Explain}
\begin{enumerate}
\item countable noun \\
A \textbf{slum} is an area of a city where living conditions are very bad and where the houses are in bad condition.
 \textit{
	\begin{itemize}
	\item ...a slum area of St Louis.
	\item ...inner-city slums in the old cities of the north and east.
	\end{itemize}
}
\item verb \\
If someone \textbf{is slumming it} or \textbf{is slumming} , they are spending time in a place or in conditions that are at a much lower social level than they
are used to.
 \textit{
	\begin{itemize}
	\item ...rich kids slumming it.
	\item ...aristocratic types who enjoyed slumming around in musty old castles.
	\end{itemize}
}
\end{enumerate}

\section*{splendid}
{\large \color{blue}  }
\subsection*{Explain}
\begin{enumerate}
\item adjective \\
If you say that something is \textbf{splendid} , you mean that it is very good.
 \textit{
	\begin{itemize}
	\item The book includes a wealth of splendid photographs.
	\item Our house has got a splendid view across to the Cotswolds.
	\item I found him to be splendid company during the hour of our acquaintance.
	\end{itemize}
}
\item adjective \\
If you describe a building or work of art as \textbf{splendid} , you mean that it is beautiful , impressive , and extremely  well made.
 \textit{
	\begin{itemize}
	\item ...a splendid Victorian mansion.
	\end{itemize}
}
\item exclamation \\
You can say ' \textbf{splendid} ' in a conversation to indicate that you approve of a particular situation or something that someone has said .
 \textit{
	\begin{itemize}
	\item 'I was thinking I might do a lemon cream sauce and baked potatoes.' 'Splendid!' Midge
applauded.
	\end{itemize}
}
\end{enumerate}

\section*{soap}
{\large \color{blue}  soaps  soaping  soaped  }
\subsection*{Explain}
\begin{enumerate}
\item variable noun \\
\textbf{Soap} is a substance that you use with water for washing yourself or sometimes for washing clothes .
 \textit{
	\begin{itemize}
	\item ...a bar of lavender soap.
	\item ...a large packet of soap powder.
	\item ...a soap bubble.
	\end{itemize}
}
\item verb \\
If you \textbf{soap}  \textbf{yourself} , you rub soap on your body in order to wash yourself.
 \textit{
	\begin{itemize}
	\item She soaped herself all over.
	\end{itemize}
}
\item countable noun \\
A \textbf{soap} is the same as a soap opera .
 \textit{
	\begin{itemize}
	\end{itemize}
}
\end{enumerate}

\section*{superfluous}
{\large \color{blue}  }
\subsection*{Explain}
\begin{enumerate}
\item adjective \\
Something that is \textbf{superfluous} is unnecessary or is no longer needed .
 \textit{
	\begin{itemize}
	\item My presence at the afternoon's proceedings was superfluous.
	\item I rid myself of many superfluous belongings and habits that bothered me.
	\end{itemize}
}
\end{enumerate}

\section*{stall}
{\large \color{blue}  stalls  stalling  stalled  }
\subsection*{Explain}
\begin{enumerate}
\item verb \\
If a process \textbf{stalls} , or if someone or something \textbf{stalls} it, the process stops but may continue at a later time.
 \textit{
	\begin{itemize}
	\item They will try to stall the bill until the current session ends.
	\item ...but the peace process stalled.
	\item Negotiations remained stalled yesterday in New York.
	\end{itemize}
}
\item verb \\
If you \textbf{stall} , you try to avoid doing something until later.
 \textit{
	\begin{itemize}
	\item Some parties have accused the governor of stalling.
	\item Thomas had spent all week stalling over his decision.
	\end{itemize}
}
\item verb \\
If you \textbf{stall} someone, you prevent them from doing something until a later time.
 \textit{
	\begin{itemize}
	\item Shop manager Brian Steel stalled the man until the police arrived.
	\end{itemize}
}
\item verb \\
If a vehicle \textbf{stalls} or if you accidentally \textbf{stall} it, the engine stops suddenly .
 \textit{
	\begin{itemize}
	\item The engine stalled.
	\item Your foot falls off the pedal and you stall the car.
	\end{itemize}
}
\item countable noun \\
A \textbf{stall} is a large table on which you put goods that you want to sell, or information that you want to give people.
 \textit{
	\begin{itemize}
	\item ...market stalls selling local fruits.
	\end{itemize}
}
\item plural noun \\
\textbf{The stalls} in a theatre or concert  hall are the seats on the ground floor directly in front of the stage.
 \textit{
	\begin{itemize}
	\end{itemize}
}
\item countable noun \\
A \textbf{stall} is a small enclosed area in a room which is used for a particular purpose, for example a shower .
 \textit{
	\begin{itemize}
	\end{itemize}
}
\end{enumerate}

\section*{synthetic}
{\large \color{blue}  }
\subsection*{Explain}
\begin{enumerate}
\item adjective \\
\textbf{Synthetic} products are made from chemicals or artificial substances rather than from natural ones.
 \textit{
	\begin{itemize}
	\item Boots made from synthetic materials can usually be washed in a machine.
	\item ...synthetic rubber.
	\end{itemize}
}
\end{enumerate}

\section*{version}
{\large \color{blue}  versions  }
\subsection*{Explain}
\begin{enumerate}
\item countable noun \\
A \textbf{version}  \textbf{of} something is a particular form of it in which some details are different from earlier or later forms.
 \textit{
	\begin{itemize}
	\item ...an updated version of his book.
	\item Ludo is a version of an ancient Indian racing game.
	\item The second-hand version is a poor copy of the original.
	\end{itemize}
}
\item countable noun \\
Someone's \textbf{version}  \textbf{of} an event is their own description of it, especially when it is different to other people's.
 \textit{
	\begin{itemize}
	\item Yesterday afternoon the White House put out a new version of events.
	\item She and her friends wanted to go public with their version of the incident.
	\item There have been widely differing versions in the newspapers about the prison siege.
	\end{itemize}
}
\end{enumerate}

\section*{various}
{\large \color{blue}  }
\subsection*{Explain}
\begin{enumerate}
\item adjective \\
If you say that there are \textbf{various} things, you mean there are several different things of the type mentioned .
 \textit{
	\begin{itemize}
	\item His plan is to spread the capital between various building society accounts.
	\item The school has received various grants from the education department.
	\end{itemize}
}
\item adjective \\
If a number of things are described as \textbf{various} , they are very different from one another.
 \textit{
	\begin{itemize}
	\item The methods are many and various.
	\item ...the country's rich and various heritage.
	\end{itemize}
}
\end{enumerate}

\section*{wire}
{\large \color{blue}  wires  wiring  wired  }
\subsection*{Explain}
\begin{enumerate}
\item variable noun \\
A \textbf{wire} is a long thin piece of metal that is used to fasten things or to carry electric
current.
 \textit{
	\begin{itemize}
	\item ...fine copper wire.
	\item ...gadgets which detect electrical wires, pipes and timbers in walls.
	\end{itemize}
}
\item countable noun \\
A \textbf{wire} is a cable which carries power or signals from one place to another.
 \textit{
	\begin{itemize}
	\item I ripped out the phone wire that ran through to his office.
	\item ...the voltage of the overhead wires.
	\end{itemize}
}
\item verb \\
If you \textbf{wire} something such as a building or piece of equipment , you put wires inside it so that electricity or signals can pass into or through it.
 \textbf{Wire up} means the same as wire .
 \textit{
	\begin{itemize}
	\item ...learning to wire and plumb the house herself.
	\item Lamps should be safely wired.
	\item Each of the homes has a security system and is wired for cable television.
	\item ...a badly wired appliance.
	\item He was helping wire up the Channel Tunnel last season.
	\item Wire the thermometers up to trigger off an alarm bell if the temperature drops.
	\item Security experts wired up dozens of expensive plants to the main alarm system at
his mansion.
	\end{itemize}
}
\item countable noun \\
A \textbf{wire} is the same as a telegram .
 \textit{
	\begin{itemize}
	\end{itemize}
}
\item verb \\
If you \textbf{wire} a person, you send them a telegram.
 \textit{
	\begin{itemize}
	\item He wired the chairman immediately.
	\item They wired back a long list of books.
	\item If I get another tummy ache, I will wire you to come.
	\end{itemize}
}
\item verb \\
If you \textbf{wire} an amount of money to a person or place, you tell a bank to send it to the person or place using a telegram message.
 \textit{
	\begin{itemize}
	\item I'm wiring you some money.
	\item They arranged to wire the money from the United States.
	\item Investigators say nearly $100,000 was wired into the suspect's bank accounts.
	\end{itemize}
}
\item  \\
 to the wire \textit{
	\begin{itemize}
	\end{itemize}
}
\end{enumerate}

\section*{awful}
{\large \color{blue}  }
\subsection*{Explain}
\begin{enumerate}
\item adjective \\
If you say that someone or something is \textbf{awful} , you dislike that person or thing or you think that they are not very good .
 \textit{
	\begin{itemize}
	\item We met and I thought he was awful.
	\item I couldn't stand London! Bloody awful place.
	\item ...an awful smell of paint.
	\item Even if the weather's awful there's lots to do.
	\item Jeans look awful on me.
	\end{itemize}
}
\item adjective \\
If you say that something is \textbf{awful} , you mean that it is extremely unpleasant, shocking , or bad.
 \textit{
	\begin{itemize}
	\item Her injuries were massive. It was awful.
	\item Some of their offences are so awful they would chill the blood.
	\end{itemize}
}
\item adjective \\
If you look or feel  \textbf{awful} , you look or feel ill .
 \textit{
	\begin{itemize}
	\item I hardly slept at all and felt pretty awful.
	\item I looked awful and felt quite shaky.
	\end{itemize}
}
\item adjective \\
You can use \textbf{awful} with noun groups that refer to an amount in order to emphasize how large that amount is.
 \textit{
	\begin{itemize}
	\item I've got an awful lot of work to do.
	\end{itemize}
}
\item adverb \\
You can use \textbf{awful} with adjectives that describe a quality in order to emphasize that particular quality.
 \textit{
	\begin{itemize}
	\item Gosh, you're awful pretty.
	\item You know, 10 years sounds like an awful long time.
	\end{itemize}
}
\end{enumerate}

\section*{amplifier}
{\large \color{blue}  amplifiers  }
\subsection*{Explain}
\begin{enumerate}
\item countable noun \\
An \textbf{amplifier} is an electronic device in a radio or stereo system which causes sounds or signals to get  louder .
 \textit{
	\begin{itemize}
	\end{itemize}
}
\end{enumerate}

\section*{bare}
{\large \color{blue}  barer  barest  bares  baring  bared  }
\subsection*{Explain}
\begin{enumerate}
\item adjective \\
If a part of your body is \textbf{bare} , it is not covered by any clothing.
 \textit{
	\begin{itemize}
	\item She was wearing only a thin robe over a flimsy nightdress, and her feet were bare.
	\item She had bare arms and a bare neck.
	\end{itemize}
}
\item adjective \\
A \textbf{bare} surface is not covered or decorated with anything.
 \textit{
	\begin{itemize}
	\item They would have liked bare wooden floors throughout the house.
	\end{itemize}
}
\item adjective \\
If a tree or a branch is \textbf{bare} , it has no leaves on it.
 \textit{
	\begin{itemize}
	\item ...an old, twisted tree, its bark shaggy, many of its limbs brittle and bare.
	\end{itemize}
}
\item adjective \\
If a room , cupboard , or shelf is \textbf{bare} , it is empty .
 \textit{
	\begin{itemize}
	\item His fridge was bare apart from three very withered tomatoes.
	\item He led me through to a bare, draughty interviewing room.
	\end{itemize}
}
\item adjective \\
An area of ground that is \textbf{bare} has no plants growing on it.
 \textit{
	\begin{itemize}
	\item That's probably the most bare, bleak, barren and inhospitable island I've ever seen.
	\end{itemize}
}
\item adjective \\
If someone gives you the \textbf{bare}  facts or the \textbf{barest}  details of something, they tell you only the most basic and important things.
 \textit{
	\begin{itemize}
	\item Newspaper reporters were given nothing but the bare facts by the Superintendent in
charge of the investigation.
	\end{itemize}
}
\item adjective \\
If you talk about the \textbf{bare}  minimum or the \textbf{bare}  essentials , you mean the very least that is necessary .
 \textit{
	\begin{itemize}
	\item The army would try to hold the western desert with a bare minimum of forces.
	\item These are the bare essentials you'll need to dress your baby during the first few
months.
	\end{itemize}
}
\item adjective \\
\textbf{Bare} is used in front of an amount to emphasize how small it is.
 \textit{
	\begin{itemize}
	\item Sales are growing for premium wines, but at a bare 2 percent a year.
	\end{itemize}
}
\item verb \\
If you \textbf{bare} something, you uncover it and show it.
 \textit{
	\begin{itemize}
	\item Walsh bared his teeth in a grin.
	\end{itemize}
}
\item  \\
 with one's bare hands \textit{
	\begin{itemize}
	\end{itemize}
}
\item  \\
 to lay something bare \textit{
	\begin{itemize}
	\end{itemize}
}
\item  \\
 lay sth bare \textit{
	\begin{itemize}
	\end{itemize}
}
\item  \\
 to bare one's soul \textit{
	\begin{itemize}
	\end{itemize}
}
\end{enumerate}

\section*{bachelor}
{\large \color{blue}  bachelors  }
\subsection*{Explain}
\begin{enumerate}
\item countable noun \\
A \textbf{bachelor} is a man who has never  married .
 \textit{
	\begin{itemize}
	\end{itemize}
}
\end{enumerate}

\section*{critical}
{\large \color{blue}  }
\subsection*{Explain}
\begin{enumerate}
\item adjective \\
A \textbf{critical} time, factor , or situation is extremely  important .
 \textit{
	\begin{itemize}
	\item The incident happened at a critical point in the campaign.
	\item Environmentalists say a critical factor in the city's pollution is its population.
	\item He says setting priorities is of critical importance.
	\item How you finance a business is critical to the success of your venture.
	\end{itemize}
}
\item adjective \\
A \textbf{critical} situation is very serious and dangerous .
 \textit{
	\begin{itemize}
	\item The authorities are considering an airlift if the situation becomes critical.
	\item Its day-to-day finances are in a critical state.
	\end{itemize}
}
\item adjective \\
If a person is \textbf{critical} or in a \textbf{critical} condition in hospital , they are seriously ill.
 \textit{
	\begin{itemize}
	\item Ten of the injured are said to be in critical condition.
	\end{itemize}
}
\item adjective \\
To be \textbf{critical}  \textbf{of} someone or something means to criticize them.
 \textit{
	\begin{itemize}
	\item His report is highly critical of the trial judge.
	\item ...a few dozen intellectuals who've been critical of the regime.
	\item He has apologised for critical remarks he made about the referee.
	\end{itemize}
}
\item adjective \\
A \textbf{critical}  approach to something involves examining and judging it carefully.
 \textit{
	\begin{itemize}
	\item We need to become critical text-readers.
	\item Marx's work was more than a critical study of capitalist production.
	\item ...the critical analysis of political ideas.
	\end{itemize}
}
\item adjective \\
If something or someone receives  \textbf{critical}  acclaim , critics say that they are very good.
 \textit{
	\begin{itemize}
	\item The film met with considerable critical and public acclaim.
	\item The show was also a resounding critical success.
	\end{itemize}
}
\end{enumerate}

\section*{bottom}
{\large \color{blue}  bottoms  bottoming  bottomed  }
\subsection*{Explain}
\begin{enumerate}
\item countable noun \\
\textbf{The}  \textbf{bottom}  \textbf{of} something is the lowest or deepest part of it.
 \textit{
	\begin{itemize}
	\item He sat at the bottom of the stairs.
	\item Answers can be found at the bottom of page 8.
	\item ...the bottom of the sea.
	\end{itemize}
}
\item adjective \\
The \textbf{bottom} thing or layer in a series of things or layers is the lowest one.
 \textit{
	\begin{itemize}
	\item There's an extra duvet in the bottom drawer of the cupboard.
	\end{itemize}
}
\item countable noun \\
\textbf{The}  \textbf{bottom}  \textbf{of} an object is the flat surface at its lowest point. You can also  refer to the inside or outside of this surface as the \textbf{bottom} .
 \textit{
	\begin{itemize}
	\item Spread the onion slices on the bottom of the dish.
	\item ...the bottom of their shoes.
	\item ...a suitcase with a false bottom.
	\end{itemize}
}
\item singular noun \\
If you say that \textbf{the bottom} has dropped or fallen out of a market or industry , you mean that people have stopped  buying the products it sells.
 \textit{
	\begin{itemize}
	\item The bottom had fallen out of the city's property market.
	\end{itemize}
}
\item singular noun \\
\textbf{The bottom}  \textbf{of} a street or garden is the end farthest away from you or from your house.
 \textit{
	\begin{itemize}
	\item ...the Cathedral at the bottom of the street.
	\end{itemize}
}
\item singular noun \\
\textbf{The bottom}  \textbf{of} a table is the end farthest away from where you are sitting . \textbf{The bottom}  \textbf{of} a bed is the end where you usually rest your feet.
 \textit{
	\begin{itemize}
	\item Malone sat down on the bottom of the bed.
	\end{itemize}
}
\item singular noun \\
\textbf{The bottom}  \textbf{of} an organization or career structure is the lowest level in it, where new employees often start .
 \textit{
	\begin{itemize}
	\item He had worked in the theatre for many years, starting at the bottom.
	\item ...a contract researcher at the bottom of the pay scale.
	\end{itemize}
}
\item singular noun \\
If someone is \textbf{bottom} or at \textbf{the bottom} in a survey , test , or league , their performance is worse than that of all the other people involved.
 \textit{
	\begin{itemize}
	\item He was always bottom of the class.
	\item The team is close to bottom of the League.
	\end{itemize}
}
\item countable noun \\
Your \textbf{bottom} is the part of your body that you sit on.
 \textit{
	\begin{itemize}
	\item If there was one thing she could change about her body it would be her bottom.
	\end{itemize}
}
\item countable noun \\
The lower part of a bikini , tracksuit , or pair of pyjamas can be referred to as the \textbf{bottoms} or the \textbf{bottom} .
 \textit{
	\begin{itemize}
	\item She wore blue tracksuit bottoms.
	\item ...a skimpy bikini bottom.
	\end{itemize}
}
\item  \\
 at bottom \textit{
	\begin{itemize}
	\end{itemize}
}
\item  \\
 at the bottom of sth \textit{
	\begin{itemize}
	\end{itemize}
}
\item  \\
 from the bottom of one's heart \textit{
	\begin{itemize}
	\end{itemize}
}
\item  \\
 get to the bottom of sth \textit{
	\begin{itemize}
	\end{itemize}
}
\item  \\
 bottoms up \textit{
	\begin{itemize}
	\end{itemize}
}
\end{enumerate}

\section*{curious}
{\large \color{blue}  }
\subsection*{Explain}
\begin{enumerate}
\item adjective \\
If you are \textbf{curious}  \textbf{about} something, you are interested in it and want to know more about it.
 \textit{
	\begin{itemize}
	\item Steve was intensely curious about the world I came from.
	\item Children are naturally curious.
	\item ...a group of curious villagers.
	\end{itemize}
}
\item adjective \\
If you describe something as \textbf{curious} , you mean that it is unusual or difficult to understand .
 \textit{
	\begin{itemize}
	\item There is a curious thing about her writings in this period.
	\item The pageant promises to be a curious mixture of the ancient and modern.
	\item The naval high command's response to these developments is rather curious.
	\end{itemize}
}
\end{enumerate}

\section*{bubble}
{\large \color{blue}  bubbles  bubbling  bubbled  }
\subsection*{Explain}
\begin{enumerate}
\item countable noun \\
\textbf{Bubbles} are small balls of air or gas in a liquid.
 \textit{
	\begin{itemize}
	\item Ink particles attach themselves to air bubbles and rise to the surface.
	\item ...a bubble of gas trapped under the surface.
	\end{itemize}
}
\item countable noun \\
A \textbf{bubble} is a hollow ball of soapy liquid that is floating in the air or standing on a surface.
 \textit{
	\begin{itemize}
	\item With soap and water, bubbles and boats, children love bathtime.
	\end{itemize}
}
\item countable noun \\
A \textbf{bubble} is a situation in which large numbers of people want to buy  shares in a company that is new or not yet financially successful , and pay more than the shares are worth . When it becomes clear that the shares are worth less than people paid for them, you can say that the \textbf{bubble} has burst .
 \textit{
	\begin{itemize}
	\item This is the point when a rising market turns into a speculative bubble.
	\item They vie to cash in before the bubble bursts.
	\end{itemize}
}
\item countable noun \\
In a cartoon , a speech  \textbf{bubble} is the shape which surrounds the words that a character is thinking or saying .
 \textit{
	\begin{itemize}
	\end{itemize}
}
\item verb \\
When a liquid \textbf{bubbles} , bubbles move in it, for example because it is boiling or moving quickly.
 \textit{
	\begin{itemize}
	\item Heat the seasoned stock until it is bubbling.
	\item The coffeepot bubbled, filling the room with fragrance.
	\item The fermenting wine has bubbled up and over the top.
	\item Danny looked down at the stream bubbling through the trees nearby.
	\end{itemize}
}
\item verb \\
If something \textbf{bubbles} , it is very busy or lively .
 \textit{
	\begin{itemize}
	\item The book bubbles with appreciation of the joys of the sport.
	\item The show bubbles like pink champagne with pretty sets and enchanting dance routines.
	\end{itemize}
}
\item verb \\
A feeling , influence , or activity that \textbf{is bubbling}  away  continues to occur.
 \textit{
	\begin{itemize}
	\item ...political tensions that have been bubbling away for years.
	\item Rumours of financial scandals have come bubbling back to the surface.
	\item Retail sales and car sales have been bubbling along, quite nicely, for some months.
	\end{itemize}
}
\item verb \\
Someone who \textbf{is bubbling with} a good feeling is so full of it that they keep  expressing the way they feel to everyone around them.
 \textbf{Bubble over} means the same as bubble .
 \textbf{Bubble} is also a noun .
 \textit{
	\begin{itemize}
	\item She came to the phone bubbling with excitement.
	\item She came back bubbling with ideas.
	\item He was quite tireless, bubbling over with vitality.
	\item As she spoke she felt a bubble of optimism rising inside her.
	\end{itemize}
}
\end{enumerate}

\section*{delicate}
{\large \color{blue}  }
\subsection*{Explain}
\begin{enumerate}
\item adjective \\
Something that is \textbf{delicate} is small and beautifully shaped.
 \textit{
	\begin{itemize}
	\item He had delicate hands.
	\item ...an evergreen tree with large flame-coloured leaves and delicate blossom.
	\end{itemize}
}
\item adjective \\
Something that is \textbf{delicate} has a colour, taste, or smell which is pleasant and not strong or intense .
 \textit{
	\begin{itemize}
	\item Young haricot beans have a tender texture and a delicate, subtle flavour.
	\item The colours are delicate and shimmering.
	\end{itemize}
}
\item adjective \\
If something is \textbf{delicate} , it is easy to harm , damage, or break , and needs to be handled or treated carefully.
 \textit{
	\begin{itemize}
	\item Although the coral looks hard, it is very delicate.
	\item ...a washing machine catering for every fabric–even the most delicate.
	\end{itemize}
}
\item adjective \\
Someone who is \textbf{delicate} is not healthy and strong, and becomes ill easily.
 \textit{
	\begin{itemize}
	\item She was physically delicate and psychologically unstable.
	\end{itemize}
}
\item adjective \\
You use \textbf{delicate} to describe a situation , problem , matter , or discussion that needs to be dealt with carefully and sensitively in order to avoid  upsetting things or offending people.
 \textit{
	\begin{itemize}
	\item The members are afraid of upsetting the delicate balance of political interests.
	\item This sensitive book tackles the delicate issue of adoption with care and simplicity.
	\item She turned to Mary Ann. 'This is kind of delicate. Would you excuse us for a moment?'
	\end{itemize}
}
\item adjective \\
A \textbf{delicate}  task , movement, action, or product needs or shows  great  skill and attention to detail .
 \textit{
	\begin{itemize}
	\item ...a long and delicate operation carried out at a hospital in Florence.
	\item Each motion must be delicate and precise, involving tiny movements.
	\end{itemize}
}
\end{enumerate}

\section*{clothing}
{\large \color{blue}  }
\subsection*{Explain}
\begin{enumerate}
\item uncountable noun \\
\textbf{Clothing} is the things that people wear.
 \textit{
	\begin{itemize}
	\item Some locals offered food and clothing to the refugees.
	\item What is your favourite item of clothing?
	\item Wear protective clothing.
	\item ...the clothing industry.
	\end{itemize}
}
\end{enumerate}

\section*{dirty}
{\large \color{blue}  dirtier  dirtiest  dirties  dirtying  dirtied  }
\subsection*{Explain}
\begin{enumerate}
\item adjective \\
If something is \textbf{dirty} , it is marked or covered with stains, spots , or mud , and needs to be cleaned .
 \textit{
	\begin{itemize}
	\item She still did not like the woman who had dirty fingernails.
	\item The dress had been brightly coloured, but it was stained and dirty now.
	\end{itemize}
}
\item verb \\
To \textbf{dirty} something means to cause it to become dirty.
 \textit{
	\begin{itemize}
	\item He was afraid the dog's hairs might dirty the seats.
	\item A little girl dirties her clothing by climbing a tree.
	\end{itemize}
}
\item adjective \\
If you describe an action as \textbf{dirty} , you disapprove of it and consider it unfair, immoral , or dishonest.
 \textbf{Dirty} is also an adverb .
 \textit{
	\begin{itemize}
	\item The gunman had been hired by a rival Mafia family to do the dirty deed.
	\item Jim Browne is the kind of fellow who can fight dirty.
	\end{itemize}
}
\item adjective \\
If you describe something such as a joke , a book , or someone's language as \textbf{dirty} , you mean that it refers to sex in a way that some people find  offensive .
 \textbf{Dirty} is also an adverb.
 \textit{
	\begin{itemize}
	\item They told dirty jokes and sang raucous ballads.
	\item I'm often asked whether the men talk dirty to me. The answer is no.
	\end{itemize}
}
\item adjective \\
\textbf{Dirty} is used before words of criticism to emphasize that you do not approve of someone or something.
 \textit{
	\begin{itemize}
	\item You dirty liar.
	\end{itemize}
}
\item  \\
 to wash your dirty linen in public \textit{
	\begin{itemize}
	\end{itemize}
}
\item  \\
 dirty look \textit{
	\begin{itemize}
	\end{itemize}
}
\item  \\
 dirty old man \textit{
	\begin{itemize}
	\end{itemize}
}
\item  \\
 do someone's dirty work \textit{
	\begin{itemize}
	\end{itemize}
}
\item  \\
 dirty weekend \textit{
	\begin{itemize}
	\end{itemize}
}
\item  \\
 a dirty word \textit{
	\begin{itemize}
	\end{itemize}
}
\end{enumerate}

\section*{court}
{\large \color{blue}  courts  }
\subsection*{Explain}
\begin{enumerate}
\item countable noun \\
A \textbf{court} is a place where legal matters are decided by a judge and jury or by a magistrate .
 \textit{
	\begin{itemize}
	\item At this rate, we could find ourselves in the divorce courts!
	\item ...a county court judge.
	\item He was deported on a court order following a conviction for armed robbery.
	\item The 28-year-old striker was in court last week for breaking a rival player's jaw.
	\end{itemize}
}
\item countable noun \\
You can refer to the people in a court, especially the judge, jury, or magistrates, as a \textbf{court} .
 \textit{
	\begin{itemize}
	\item A court at Tampa, Florida has convicted five officials on fraud charges.
	\end{itemize}
}
\item countable noun \\
A \textbf{court} is an area in which you play a game such as tennis, basketball , badminton , or squash.
 \textit{
	\begin{itemize}
	\item The hotel has several tennis and squash courts.
	\item She watched a few of the games while waiting to go on court.
	\end{itemize}
}
\item countable noun \\
The \textbf{court} of a king or queen is the place where he or she lives and carries out ceremonial or administrative duties.
 \textit{
	\begin{itemize}
	\item She came to visit England, where she was presented at the court of James I.
	\item Their family was certainly well regarded at court.
	\end{itemize}
}
\item noun, in names \\
In Britain, \textbf{Court} is used in the names of large houses and blocks of flats.
 \textit{
	\begin{itemize}
	\item ...7 Ivebury Court, Latimer Rd, London W10 6RA.
	\end{itemize}
}
\item  \\
 go to court/ take sb to court \textit{
	\begin{itemize}
	\end{itemize}
}
\item  \\
 your day in court \textit{
	\begin{itemize}
	\end{itemize}
}
\item  \\
 to hold court \textit{
	\begin{itemize}
	\end{itemize}
}
\item  \\
 to laugh someone out of court \textit{
	\begin{itemize}
	\end{itemize}
}
\item  \\
 out of court \textit{
	\begin{itemize}
	\end{itemize}
}
\end{enumerate}

\section*{distant}
{\large \color{blue}  }
\subsection*{Explain}
\begin{enumerate}
\item adjective \\
\textbf{Distant}  means very far away.
 \textit{
	\begin{itemize}
	\item The mountains rolled away to a distant horizon.
	\item ...the war in that distant land.
	\end{itemize}
}
\item adjective \\
You use \textbf{distant} to describe a time or event that is very far away in the future or in the past .
 \textit{
	\begin{itemize}
	\item There is little doubt, however, that things will improve in the not too distant future.
	\item Last summer's drought is a distant memory.
	\end{itemize}
}
\item adjective \\
A \textbf{distant}  relative is one who you are not closely related to.
 \textit{
	\begin{itemize}
	\item He's a distant relative of the mayor.
	\item They were distant cousins.
	\end{itemize}
}
\item adjective \\
If you describe someone as \textbf{distant} , you mean that you find them cold and unfriendly .
 \textit{
	\begin{itemize}
	\item He found her cold, ice-like and distant.
	\item He is direct and courteous but distant.
	\end{itemize}
}
\item adjective \\
If you describe someone as \textbf{distant} , you mean that they are not concentrating on what they are doing because they are thinking about other things.
 \textit{
	\begin{itemize}
	\item There was a distant look in her eyes from time to time, her thoughts elsewhere.
	\end{itemize}
}
\end{enumerate}

\section*{dog}
{\large \color{blue}  dogs  dogging  dogged  }
\subsection*{Explain}
\begin{enumerate}
\item countable noun \\
A \textbf{dog} is a very common four-legged animal that is often kept by people as a pet or to guard or hunt . There are many different breeds of dog.
 \textit{
	\begin{itemize}
	\item Outside, a dog was barking.
	\item The dog growled again.
	\item The British are renowned as a nation of dog lovers.
	\end{itemize}
}
\item countable noun \\
You use \textbf{dog} to refer to a male dog, or to the male of some related species such as wolves or foxes .
 \textit{
	\begin{itemize}
	\item Is this a dog or a bitch?
	\item ...a dog fox.
	\end{itemize}
}
\item countable noun \\
If someone calls a man a \textbf{dog} , they strongly disapprove of him.
 \textit{
	\begin{itemize}
	\end{itemize}
}
\item countable noun \\
People use \textbf{dog} to refer to something that they consider unsatisfactory or of poor quality.
 \textit{
	\begin{itemize}
	\item It's a real dog.
	\end{itemize}
}
\item countable noun \\
If someone, especially a man, calls a woman or girl a \textbf{dog} , they mean that she is very ugly , unattractive, or boring.
 \textit{
	\begin{itemize}
	\end{itemize}
}
\item verb \\
If problems or injuries  \textbf{dog} you, they are with you all the time.
 \textit{
	\begin{itemize}
	\item The problems that have dogged him all year are just a temporary setback.
	\item His career has been dogged by bad luck.
	\end{itemize}
}
\item plural noun \\
\textbf{The dogs} is a sports meeting where dogs, especially greyhounds , race and people bet on which dog will win.
 \textit{
	\begin{itemize}
	\end{itemize}
}
\item  \\
 dog's dinner/breakfast \textit{
	\begin{itemize}
	\end{itemize}
}
\item  \\
 dog eat dog \textit{
	\begin{itemize}
	\end{itemize}
}
\item  \\
 going to the dogs \textit{
	\begin{itemize}
	\end{itemize}
}
\item  \\
 to let sleeping dogs lie \textit{
	\begin{itemize}
	\end{itemize}
}
\item  \\
 you can't teach an old dog new tricks \textit{
	\begin{itemize}
	\end{itemize}
}
\end{enumerate}

\section*{dramatic}
{\large \color{blue}  }
\subsection*{Explain}
\begin{enumerate}
\item adjective \\
A \textbf{dramatic} change or event happens  suddenly and is very noticeable and surprising .
 \textit{
	\begin{itemize}
	\item Changes in sea level could have a dramatic effect.
	\item This policy has led to a dramatic increase in our prison populations.
	\end{itemize}
}
\item adjective \\
A \textbf{dramatic} action, event, or situation is exciting and impressive .
 \textit{
	\begin{itemize}
	\item He witnessed many dramatic escapes as people jumped from as high as the fourth floor.
	\item Their arrival was dramatic and exciting.
	\end{itemize}
}
\item adjective \\
You use \textbf{dramatic} to describe things connected with or relating to the theatre , drama, or plays.
 \textit{
	\begin{itemize}
	\item ...a dramatic arts major in college.
	\item I had no thoughts of making a dramatic film. I was working in documentary.
	\end{itemize}
}
\end{enumerate}

\section*{dose}
{\large \color{blue}  doses  dosing  dosed  }
\subsection*{Explain}
\begin{enumerate}
\item countable noun \\
A \textbf{dose}  \textbf{of} medicine or a drug is a measured amount of it which is intended to be taken at one time.
 \textit{
	\begin{itemize}
	\item One dose of penicillin can wipe out the infection.
	\end{itemize}
}
\item countable noun \\
You can refer to an amount of something as a \textbf{dose}  \textbf{of} that thing, especially when you want to emphasize that there is a great deal of it.
 \textit{
	\begin{itemize}
	\item She was born with a healthy dose of self-confidence.
	\item The West is getting a heavy dose of snow and rain today.
	\end{itemize}
}
\item verb \\
If you \textbf{dose} a person or animal \textbf{with} medicine, you give them an amount of it.
 \textbf{Dose up} means the same as dose .
 \textit{
	\begin{itemize}
	\item The doctor fixed the rib, dosed him heavily with drugs, and said he would probably
get better.
	\item I dosed myself with quinine.
	\item I dosed him up with Valium.
	\end{itemize}
}
\end{enumerate}

\section*{expensive}
{\large \color{blue}  }
\subsection*{Explain}
\begin{enumerate}
\item adjective \\
If something is \textbf{expensive} , it costs a lot of money.
 \textit{
	\begin{itemize}
	\item Fuel's so expensive in this country.
	\item I get very nervous because I'm using a lot of expensive equipment.
	\end{itemize}
}
\end{enumerate}

\section*{dress}
{\large \color{blue}  dresses  dressing  dressed  }
\subsection*{Explain}
\begin{enumerate}
\item countable noun \\
A \textbf{dress} is a piece of clothing worn by a woman or girl . It covers her body and part of her legs.
 \textit{
	\begin{itemize}
	\item She was wearing a black dress.
	\end{itemize}
}
\item uncountable noun \\
You can refer to clothes worn by men or women as \textbf{dress} .
 \textit{
	\begin{itemize}
	\item He's usually smart in his dress.
	\item ...hundreds of Cambodians in traditional dress.
	\end{itemize}
}
\item verb \\
When you \textbf{dress} or \textbf{dress}  \textbf{yourself} , you put on clothes.
 \textit{
	\begin{itemize}
	\item He told Sarah to wait while he dressed.
	\item Sue had dressed herself neatly for work.
	\end{itemize}
}
\item verb \\
If you \textbf{dress} someone, for example a child, you put clothes on them.
 \textit{
	\begin{itemize}
	\item She bathed her and dressed her in clean clothes.
	\end{itemize}
}
\item verb \\
If someone \textbf{dresses} in a particular way, they wear clothes of a particular style or colour.
 \textit{
	\begin{itemize}
	\item He dresses in a way that lets everyone know he's got authority.
	\item She used to dress in jeans.
	\end{itemize}
}
\item verb \\
If you \textbf{dress}  \textbf{for} something, you put on special clothes for it.
 \textit{
	\begin{itemize}
	\item We don't dress for dinner here.
	\end{itemize}
}
\item verb \\
When someone \textbf{dresses} a wound, they clean it and cover it.
 \textit{
	\begin{itemize}
	\item The poor child never cried or protested when I was dressing her wounds.
	\end{itemize}
}
\item verb \\
If you \textbf{dress} a salad , you cover it with a mixture of oil, vinegar , and herbs or flavourings .
 \textit{
	\begin{itemize}
	\item Scatter the tomato over, then dress the salad.
	\item ...a bowl of dressed salad.
	\end{itemize}
}
\item verb \\
To \textbf{dress}  meat , chicken , or fish means to prepare it for cooking by cleaning it and removing the parts that
you cannot eat.
 \textit{
	\begin{itemize}
	\item Her mother dressed the meat.
	\item ...dressed crab.
	\end{itemize}
}
\end{enumerate}

\section*{gloomy}
{\large \color{blue}  gloomier  gloomiest  }
\subsection*{Explain}
\begin{enumerate}
\item adjective \\
If a place is \textbf{gloomy} , it is almost dark so that you cannot see very well .
 \textit{
	\begin{itemize}
	\item Inside it's gloomy after all that sunshine.
	\item ...this huge gloomy church.
	\end{itemize}
}
\item adjective \\
If people are \textbf{gloomy} , they are unhappy and have no hope .
 \textit{
	\begin{itemize}
	\item Graduates are feeling gloomy about the jobs market.
	\end{itemize}
}
\item adjective \\
If a situation is \textbf{gloomy} , it does not give you much hope of success or happiness.
 \textit{
	\begin{itemize}
	\item ...a gloomy picture of an economy sliding into recession.
	\item Officials say the outlook for next year is gloomy.
	\end{itemize}
}
\end{enumerate}

\section*{fox}
{\large \color{blue}  foxes  foxing  foxed  }
\subsection*{Explain}
\begin{enumerate}
\item countable noun \\
A \textbf{fox} is a wild animal which looks like a dog and has reddish-brown fur, a pointed face and ears, and a thick tail. Foxes eat smaller animals.
 \textit{
	\begin{itemize}
	\end{itemize}
}
\item verb \\
If you \textbf{are foxed} by something, you cannot understand it or solve it.
 \textit{
	\begin{itemize}
	\item I admit I was foxed for some time.
	\item Only once did we hit on a question which foxed one of the experts.
	\item They're a bit foxed by the colours of the riders' jerseys and hats.
	\end{itemize}
}
\item singular noun \\
If you describe someone as a \textbf{fox} , you mean they are very clever and deceitful .
 \textit{
	\begin{itemize}
	\item Enrico was too good, an old fox, cunning.
	\end{itemize}
}
\end{enumerate}

\section*{furnace}
{\large \color{blue}  furnaces  }
\subsection*{Explain}
\begin{enumerate}
\item countable noun \\
A \textbf{furnace} is a container or enclosed space in which a very hot fire is made, for example to melt metal, burn  rubbish , or produce steam.
 \textit{
	\begin{itemize}
	\end{itemize}
}
\item singular noun \\
If you say that a place is \textbf{a}  \textbf{furnace} , you mean that it is very hot there.
 \textit{
	\begin{itemize}
	\item How can we walk? It's a furnace out there.
	\end{itemize}
}
\end{enumerate}

\section*{good}
{\large \color{blue}  better  best  }
\subsection*{Explain}
\begin{enumerate}
\item adjective \\
\textbf{Good} means pleasant or enjoyable.
 \textit{
	\begin{itemize}
	\item We had a really good time together.
	\item I know they would have a better life here.
	\item There's nothing better than a good cup of hot coffee.
	\item It's so good to hear your voice after all this time.
	\end{itemize}
}
\item adjective \\
\textbf{Good} means of a high quality, standard, or level.
 \textit{
	\begin{itemize}
	\item Exercise is just as important to health as good food.
	\item His parents wanted Raymond to have the best possible education.
	\item The train's average speed was no better than that of our bicycles.
	\item ...good quality furniture.
	\end{itemize}
}
\item adjective \\
If you are \textbf{good at} something, you are skilful and successful at doing it.
 \textit{
	\begin{itemize}
	\item He was very good at his work.
	\item I'm not very good at singing.
	\item He is one of the best players in the world.
	\item I always played football with my older brother because I was good for my age.
	\end{itemize}
}
\item adjective \\
If you describe a piece of news, an action, or an effect as \textbf{good} , you mean that it is likely to result in benefit or success.
 \textit{
	\begin{itemize}
	\item On balance biotechnology should be good news for developing countries.
	\item I had the good fortune to be selected.
	\item This is not a good example to set other children.
	\item I think the response was good.
	\end{itemize}
}
\item adjective \\
A \textbf{good} idea, reason, method, or decision is a sensible or valid one.
 \textit{
	\begin{itemize}
	\item They thought it was a good idea to make some offenders do community service.
	\item There is good reason to doubt this.
	\item Could you give me some advice on the best way to do this?
	\end{itemize}
}
\item adjective \\
If you say that \textbf{it is good}  \textbf{that} something should happen or \textbf{good}  \textbf{to} do something, you mean it is desirable , acceptable, or right.
 \textit{
	\begin{itemize}
	\item I think it's good that some people are going.
	\item It is always best to choose organically grown foods if possible.
	\end{itemize}
}
\item adjective \\
A \textbf{good}  estimate or indication of something is an accurate one.
 \textit{
	\begin{itemize}
	\item We have a fairly good idea of what's going on.
	\item This is a much better indication of what a school is really like.
	\item Laboratory tests are not always a good guide to what happens in the world.
	\end{itemize}
}
\item adjective \\
If you get a \textbf{good} deal or a \textbf{good} price when you buy or sell something, you receive a lot in exchange for what you give.
 \textit{
	\begin{itemize}
	\item Whether such properties are a good deal will depend on individual situations.
	\item The merchandise is reasonably priced and offers exceptionally good value.
	\end{itemize}
}
\item adjective \\
If something is \textbf{good for} a person or organization, it benefits them.
 \textit{
	\begin{itemize}
	\item Rain water was once considered to be good for the complexion.
	\item Nancy chose the product because it is better for the environment.
	\end{itemize}
}
\item singular noun \\
If something is done for \textbf{the}  \textbf{good} of a person or organization, it is done in order to benefit them.
 \textit{
	\begin{itemize}
	\item Furlaud urged him to resign for the good of the country.
	\item Victims want to see justice done not just for themselves, but for the greater good
of society.
	\item I'm only telling you this for your own good!
	\end{itemize}
}
\item uncountable noun \\
If someone or something is \textbf{no good} or is \textbf{not any good} , they are not satisfactory or are of a low standard.
 \textit{
	\begin{itemize}
	\item If the weather's no good then I won't take any pictures.
	\item I was never any good at maths.
	\end{itemize}
}
\item uncountable noun \\
If you say that doing something is \textbf{no good} or does \textbf{not} do \textbf{any good} , you mean that doing it is not of any use or will not bring any success.
 \textit{
	\begin{itemize}
	\item It's no good worrying about it now.
	\item We gave them water and kept them warm, but it didn't do any good.
	\item There is no way to measure these effects; the chances are it did some good.
	\end{itemize}
}
\item uncountable noun \\
\textbf{Good} is what is considered to be right according to moral standards or religious beliefs.
 \textit{
	\begin{itemize}
	\item Good and evil may co-exist within one family.
	\end{itemize}
}
\item adjective \\
Someone who is \textbf{good} is morally correct in their attitudes and behaviour.
 \textit{
	\begin{itemize}
	\item The president is a good man.
	\item For me to think I'm any better than a homeless person on the street is ridiculous.
	\end{itemize}
}
\item adjective \\
Someone, especially a child, who is \textbf{good}  obeys rules and instructions and behaves in a socially correct way.
 \textit{
	\begin{itemize}
	\item The children were very good.
	\item I'm going to be a good boy now.
	\item Both boys had good manners, politely shaking hands.
	\end{itemize}
}
\item adjective \\
Someone who is \textbf{good} is kind and thoughtful .
 \textit{
	\begin{itemize}
	\item You are good to me.
	\item Her good intentions were thwarted almost immediately.
	\item Just ask the Admiral if he will be good enough to drop me a note.
	\end{itemize}
}
\item adjective \\
Someone who is in a \textbf{good}  mood is cheerful and pleasant to be with.
 \textit{
	\begin{itemize}
	\item People were in a pretty good mood.
	\item He exudes natural charm and good humour.
	\item A relaxation session may put you in a better frame of mind.
	\end{itemize}
}
\item adjective \\
If people are \textbf{good} friends, they get on well together and are very close.
 \textit{
	\begin{itemize}
	\item She and Gavin are good friends.
	\item She's my best friend, and I really love her.
	\end{itemize}
}
\item adjective \\
A person's \textbf{good} eye, arm, or leg is the one that is healthy and strong, if the other one is injured or weak.
 \textit{
	\begin{itemize}
	\item He turned his good eye on me and laughed.
	\end{itemize}
}
\item adjective \\
You use \textbf{good} to emphasize the great extent or degree of something.
 \textit{
	\begin{itemize}
	\item We waited a good fifteen minutes.
	\item This whole thing's got a good bit more dangerous.
	\end{itemize}
}
\item convention \\
You say ' \textbf{Good} ' or ' \textbf{Very good} ' to express pleasure, satisfaction , or agreement with something that has been said or done, especially when you are
in a position of authority.
 \textit{
	\begin{itemize}
	\item 'Are you all right?'—'I'm fine.'—'Good. So am I.'
	\item Oh good, Tom's just come in.
	\item 'Strike Force Three are here, sir.'—'Good.'
	\end{itemize}
}
\item  \\
 as good as \textit{
	\begin{itemize}
	\end{itemize}
}
\item  \\
 the common good \textit{
	\begin{itemize}
	\end{itemize}
}
\item  \\
 do sb good \textit{
	\begin{itemize}
	\end{itemize}
}
\item  \\
 for good \textit{
	\begin{itemize}
	\end{itemize}
}
\item  \\
 good for you/him/her/them \textit{
	\begin{itemize}
	\end{itemize}
}
\item  \\
 be good for \textit{
	\begin{itemize}
	\end{itemize}
}
\item  \\
 it's a good job \textit{
	\begin{itemize}
	\end{itemize}
}
\item  \\
 make good \textit{
	\begin{itemize}
	\end{itemize}
}
\item  \\
 make good \textit{
	\begin{itemize}
	\end{itemize}
}
\item  \\
 make good \textit{
	\begin{itemize}
	\end{itemize}
}
\item  \\
 as good as new \textit{
	\begin{itemize}
	\end{itemize}
}
\item  \\
 good old \textit{
	\begin{itemize}
	\end{itemize}
}
\item  \\
 the good life \textit{
	\begin{itemize}
	\end{itemize}
}
\end{enumerate}

\section*{judge}
{\large \color{blue}  judges  judging  judged  }
\subsection*{Explain}
\begin{enumerate}
\item countable noun \\
A \textbf{judge} is the person in a court of law who decides how the law should be applied , for example how criminals should be punished .
 \textit{
	\begin{itemize}
	\item The judge adjourned the hearing until next Tuesday.
	\item Judge Mr Justice Schiemann jailed him for life.
	\end{itemize}
}
\item countable noun \\
A \textbf{judge} is a person who decides who will be the winner of a competition.
 \textit{
	\begin{itemize}
	\item A panel of judges is now selecting the finalists.
	\end{itemize}
}
\item verb \\
If you \textbf{judge} something such as a competition, you decide who or what is the winner.
 \textit{
	\begin{itemize}
	\item Colin Mitchell will judge the entries each week.
	\item Entrants will be judged in two age categories: 5-10 years and 11-14 years.
	\item A grade B judge could only be allowed to judge alongside a qualified grade A judge.
	\end{itemize}
}
\item verb \\
If you \textbf{judge} something or someone, you form an opinion about them after you have examined the evidence or thought carefully about them.
 \textit{
	\begin{itemize}
	\item It will take a few more years to judge the impact of these ideas.
	\item I am ready to judge any book on its merits.
	\item It's for other people to judge how much I have improved.
	\item The U.N. withdrew its relief personnel because it judged the situation too dangerous.
	\item I judged it to be one of the worst programmes ever screened.
	\item The doctor judged that the man's health had, up to the time of the wound, been good.
	\item This may or may not be judged as reasonable.
	\end{itemize}
}
\item verb \\
If you \textbf{judge} something, you guess its amount , size , or value or you guess what it is.
 \textit{
	\begin{itemize}
	\item It is important to judge the weight of your washing load correctly.
	\item I judged him to be about forty.
	\item Though the shoreline could be dimly seen, it was impossible to judge how far away
it was.
	\item I would judge that my earnings as a teacher have, over the years, been considerably
below those of Mr Foot.
	\end{itemize}
}
\item countable noun \\
If someone is a good \textbf{judge}  \textbf{of} something, they understand it and can make sensible  decisions about it. If someone is a bad  \textbf{judge}  \textbf{of} something, they cannot do this.
 \textit{
	\begin{itemize}
	\item I'm a pretty good judge of character.
	\end{itemize}
}
\item  \\
 judging by/judging from/to judge from \textit{
	\begin{itemize}
	\end{itemize}
}
\item  \\
 as far as one can judge \textit{
	\begin{itemize}
	\end{itemize}
}
\end{enumerate}

\section*{grim}
{\large \color{blue}  grimmer  grimmest  }
\subsection*{Explain}
\begin{enumerate}
\item adjective \\
A situation or piece of information that is \textbf{grim} is unpleasant, depressing , and difficult to accept .
 \textit{
	\begin{itemize}
	\item They painted a grim picture of growing crime.
	\item There was further grim economic news yesterday.
	\item The mood could not have been grimmer.
	\end{itemize}
}
\item adjective \\
A place that is \textbf{grim} is unattractive and depressing in appearance.
 \textit{
	\begin{itemize}
	\item The city might be grim at first, but there is a vibrancy and excitement.
	\item ...the tower blocks on the city's grim edges.
	\end{itemize}
}
\item adjective \\
If a person or their behaviour is \textbf{grim} , they are very serious , usually because they are worried about something.
 \textit{
	\begin{itemize}
	\item She was a grim woman with a turned-down mouth.
	\item Her expression was grim and unpleasant.
	\end{itemize}
}
\item adjective \\
If you say that something is \textbf{grim} , you think that it is very bad , ugly , or depressing.
 \textit{
	\begin{itemize}
	\item Things were pretty grim for a time.
	\end{itemize}
}
\end{enumerate}

\section*{junk}
{\large \color{blue}  junks  junking  junked  }
\subsection*{Explain}
\begin{enumerate}
\item uncountable noun \\
\textbf{Junk} is old and used goods that have little value and that you do not want any more.
 \textit{
	\begin{itemize}
	\item Rose finds her furniture in junk shops.
	\item What are you going to do with all that junk, Larry?
	\end{itemize}
}
\item uncountable noun \\
In computing , \textbf{junk}  refers to unwanted emails that have been sent to a large number of people or organizations, usually as advertising .
 \textit{
	\begin{itemize}
	\item ...an increased threat from junk email.
	\end{itemize}
}
\item verb \\
If you \textbf{junk} something, you get  rid of it or stop using it.
 \textit{
	\begin{itemize}
	\item We junk 10 million pieces of furniture every year in the UK.
	\item The Socialists junked dogma when they came to office in 1982.
	\end{itemize}
}
\item countable noun \\
A \textbf{junk} is a Chinese sailing boat that has a flat bottom and square sails.
 \textit{
	\begin{itemize}
	\end{itemize}
}
\end{enumerate}

\section*{jolly}
{\large \color{blue}  jollier  jolliest  }
\subsection*{Explain}
\begin{enumerate}
\item adjective \\
Someone who is \textbf{jolly} is happy and cheerful in their appearance or behaviour.
 \textit{
	\begin{itemize}
	\item She was a jolly, kindhearted woman.
	\end{itemize}
}
\item adjective \\
A \textbf{jolly} event is lively and enjoyable.
 \textit{
	\begin{itemize}
	\item I was looking forward to a jolly party.
	\item She had a very jolly time in Korea.
	\end{itemize}
}
\item adverb \\
\textbf{Jolly} is sometimes used to emphasize an adjective or adverb .
 \textit{
	\begin{itemize}
	\item She was jolly good at jigsaws.
	\item It was jolly hard work, but I loved it.
	\end{itemize}
}
\item  \\
 jolly well \textit{
	\begin{itemize}
	\end{itemize}
}
\end{enumerate}

\section*{law}
{\large \color{blue}  laws  }
\subsection*{Explain}
\begin{enumerate}
\item singular noun \\
\textbf{The law} is a system of rules that a society or government develops in order to deal with crime , business agreements, and social relationships. You can also use the \textbf{law} to refer to the people who work in this system.
 \textit{
	\begin{itemize}
	\item Obscene and threatening phone calls are against the law.
	\item He had broken the law on election funding and illegally received money from abroad.
	\item There must be changes in the law to stop this sort of thing happening.
	\item The book analyses why women kill and how the law treats them.
	\end{itemize}
}
\item uncountable noun \\
\textbf{Law} is used to refer to a particular branch of the law, such as \textbf{criminal law} or \textbf{company law} .
 \textit{
	\begin{itemize}
	\item He was a professor of criminal law at Harvard University law school.
	\item Under international law, diplomats living in foreign countries are exempt from criminal
prosecution.
	\item Important questions of constitutional law were involved.
	\end{itemize}
}
\item countable noun \\
A \textbf{law} is one of the rules in a system of law which deals with a particular type of agreement,
relationship, or crime.
 \textit{
	\begin{itemize}
	\item ...the country's liberal political asylum law.
	\item The law was passed on a second vote.
	\end{itemize}
}
\item plural noun \\
\textbf{The}  \textbf{laws}  \textbf{of} an organization or activity are its rules, which are used to organize and control it.
 \textit{
	\begin{itemize}
	\item ...the laws of the Church of England.
	\item Match officials should not tolerate such behaviour but instead enforce the laws of
the game.
	\end{itemize}
}
\item countable noun \\
A \textbf{law} is a rule or set of rules for good behaviour which is considered right and important
by the majority of people for moral , religious, or emotional  reasons .
 \textit{
	\begin{itemize}
	\item ...inflexible moral laws.
	\end{itemize}
}
\item countable noun \\
A \textbf{law} is a natural process in which a particular event or thing always leads to a particular result.
 \textit{
	\begin{itemize}
	\item The laws of nature are absolute.
	\end{itemize}
}
\item countable noun \\
A \textbf{law} is a scientific rule that someone has invented to explain a particular natural process.
 \textit{
	\begin{itemize}
	\item ...the law of gravity.
	\end{itemize}
}
\item uncountable noun \\
\textbf{Law} or \textbf{the law} is all the professions which deal with advising people about the law, representing people in court, or giving decisions and punishments .
 \textit{
	\begin{itemize}
	\item A career in law is becoming increasingly attractive to young people.
	\item Nearly 100 law firms are being referred to the Solicitors' Disciplinary Tribunal.
	\end{itemize}
}
\item uncountable noun \\
\textbf{Law} is the study of systems of law and how laws work.
 \textit{
	\begin{itemize}
	\item He came to Oxford and studied law.
	\item He holds a law degree from Bristol University.
	\end{itemize}
}
\item  \\
 above the law \textit{
	\begin{itemize}
	\end{itemize}
}
\item  \\
 law of averages \textit{
	\begin{itemize}
	\end{itemize}
}
\item  \\
 by law \textit{
	\begin{itemize}
	\end{itemize}
}
\item  \\
 go to law \textit{
	\begin{itemize}
	\end{itemize}
}
\item  \\
 to lay down the law \textit{
	\begin{itemize}
	\end{itemize}
}
\item  \\
 take the law into your own hands \textit{
	\begin{itemize}
	\end{itemize}
}
\item  \\
 a law unto yourself \textit{
	\begin{itemize}
	\end{itemize}
}
\end{enumerate}

\section*{many}
{\large \color{blue}  }
\subsection*{Explain}
\begin{enumerate}
\item determiner \\
You use \textbf{many} to indicate that you are talking about a large number of people or things.
 \textbf{Many} is also a pronoun.
 \textbf{Many} is also a quantifier .
 \textbf{Many} is also an adjective .
 \textit{
	\begin{itemize}
	\item I don't think many people would argue with that.
	\item Not many films are made in Finland.
	\item Do you keep many books and papers and memorabilia?
	\item Many holidaymakers had avoided the worst of the delays by consulting tourist offices.
	\item Acting is definitely a young person's profession in many ways.
	\item We stood up, thinking through the possibilities. There weren't many.
	\item So, once we have cohabited, why do many of us feel the need to get married?
	\item It seems there are not very many of them left in the sea.
	\item In many of these neighborhoods a lot of people don't have telephones.
	\item Among his many hobbies was the breeding of fine horses.
	\item The possibilities are many.
	\end{itemize}
}
\item adverb \\
You use \textbf{many} in expressions such as 'not many', 'not very many', and 'too many' when replying to questions about numbers of things or people.
 \textit{
	\begin{itemize}
	\item 'How many of the songs that dealt with this theme became hit songs?'—'Not very many.'.
	\item How many years is it since we've seen each other? Too many, anyway.
	\end{itemize}
}
\item predeterminer \\
You use \textbf{many}  followed by 'a' and a noun to emphasize that there are a lot of people or things involved in something.
 \textit{
	\begin{itemize}
	\item Many a mother tries to act out her unrealized dreams through her daughter.
	\item I have spent many a happy hour in the hills.
	\end{itemize}
}
\item determiner \\
You use \textbf{many} after 'how' to ask questions about numbers or quantities . You use \textbf{many} after 'how' in reported  clauses to talk about numbers or quantities.
 \textbf{Many} is also a pronoun.
 \textit{
	\begin{itemize}
	\item How many years have you been here?
	\item No-one knows how many people have been killed since the war began.
	\item How many do you need?
	\end{itemize}
}
\item determiner \\
You use \textbf{many} with 'as' when you are comparing numbers of things or people.
 \textbf{Many} is also a pronoun.
 \textit{
	\begin{itemize}
	\item I've always entered as many photo competitions as I can.
	\item We produced ten times as many tractors as the United States.
	\item Let the child try on as many as she likes.
	\end{itemize}
}
\item pronoun \\
You use \textbf{many} to mean 'many people'.
 \textit{
	\begin{itemize}
	\item Iris Murdoch was regarded by many as a supremely good and serious writer.
	\end{itemize}
}
\item singular noun \\
\textbf{The many} means a large group of people, especially the ordinary people in society , considered as separate from a particular small group.
 \textit{
	\begin{itemize}
	\item The printing press gave power to a few to change the world for the many.
	\item He wanted to create a society of opportunity where benefits became available to the
many.
	\end{itemize}
}
\item  \\
 as many as \textit{
	\begin{itemize}
	\end{itemize}
}
\item  \\
 a good many/a great many \textit{
	\begin{itemize}
	\end{itemize}
}
\end{enumerate}

\section*{legislation}
{\large \color{blue}  }
\subsection*{Explain}
\begin{enumerate}
\item uncountable noun \\
\textbf{Legislation} consists of a law or laws passed by a government .
 \textit{
	\begin{itemize}
	\item ...a letter calling for legislation to protect women's rights.
	\end{itemize}
}
\end{enumerate}

\section*{merry}
{\large \color{blue}  merrier  merriest  }
\subsection*{Explain}
\begin{enumerate}
\item adjective \\
If you describe someone's character or behaviour as \textbf{merry} , you mean that they are happy and cheerful.
 \textit{
	\begin{itemize}
	\item He was much loved for his merry nature.
	\item From the house come the bursts of merry laughter.
	\item Merry black eyes glinted at them.
	\end{itemize}
}
\item graded adjective \\
A \textbf{merry} sound or sight makes you feel cheerful.
 \textit{
	\begin{itemize}
	\item ...the merry sounds of a seven-piece brass band.
	\item She was humming a merry little tune.
	\end{itemize}
}
\item adjective \\
If you get  \textbf{merry} , you get slightly drunk.
 \textit{
	\begin{itemize}
	\item They went off to Glengarriff to get merry.
	\end{itemize}
}
\item adjective \\
Some people use \textbf{merry} to emphasize something that they are saying , often when they want to express  disapproval or humour .
 \textit{
	\begin{itemize}
	\item It hasn't stopped the British Navy proceeding on its merry way.
	\item In the merry world of American lawyers it is the simplest thing in the world to start
an action.
	\end{itemize}
}
\item  \\
 Merry Christmas \textit{
	\begin{itemize}
	\end{itemize}
}
\item  \\
 make merry \textit{
	\begin{itemize}
	\end{itemize}
}
\end{enumerate}

\section*{night}
{\large \color{blue}  nights  }
\subsection*{Explain}
\begin{enumerate}
\item variable noun \\
The \textbf{night} is the part of each day when the sun has set and it is dark outside, especially the time when people are sleeping .
 \textit{
	\begin{itemize}
	\item He didn't sleep a wink all night.
	\item The fighting began in the late afternoon and continued all night.
	\item Our reporter spent the night crossing the border from Austria into Slovenia.
	\item Finally night fell.
	\end{itemize}
}
\item countable noun \\
The \textbf{night} is the period of time between the end of the afternoon and the time that you go to bed, especially the time when you relax before going to bed.
 \textit{
	\begin{itemize}
	\item So whose party was it last night?
	\item Demiris took Catherine to dinner the following night.
	\end{itemize}
}
\item countable noun \\
A particular  \textbf{night} is a particular evening when a special event  takes place, such as a show or a play .
 \textit{
	\begin{itemize}
	\item The first night crowd packed the building.
	\item ...election night.
	\end{itemize}
}
\item  \\
 at night \textit{
	\begin{itemize}
	\end{itemize}
}
\item  \\
 at night \textit{
	\begin{itemize}
	\end{itemize}
}
\item  \\
 day and night/night and day \textit{
	\begin{itemize}
	\end{itemize}
}
\item  \\
 early night \textit{
	\begin{itemize}
	\end{itemize}
}
\end{enumerate}

\section*{militant}
{\large \color{blue}  militants  }
\subsection*{Explain}
\begin{enumerate}
\item adjective \\
You use \textbf{militant} to describe people who believe in something very strongly and are active in trying to bring about political or social  change , often in extreme  ways that other people find  unacceptable .
 \textbf{Militant} is also a noun .
 \textit{
	\begin{itemize}
	\item Militant mineworkers in the Ukraine have voted for a one-day stoppage next month.
	\item ...one of the most active militant groups.
	\item The militants might still find some new excuse to call a strike.
	\end{itemize}
}
\end{enumerate}

\section*{obedience}
{\large \color{blue}  }
\subsection*{Explain}
\begin{enumerate}
\item noun \\
1.  2.  3.  4.  \textit{
	\begin{itemize}
	\end{itemize}
}
\end{enumerate}

\section*{oil}
{\large \color{blue}  oils  oiling  oiled  }
\subsection*{Explain}
\begin{enumerate}
\item variable noun \\
\textbf{Oil} is a smooth, thick liquid that is used as a fuel and for making the parts of machines move smoothly. Oil is found underground .
 \textit{
	\begin{itemize}
	\item The company buys and sells about 600,000 barrels of oil a day.
	\item ...the rapid rise in prices for oil and petrol.
	\item ...a small oil lamp.
	\end{itemize}
}
\item verb \\
If you \textbf{oil} something, you put oil onto or into it, for example to make it work smoothly or to protect it.
 \textit{
	\begin{itemize}
	\item A crew of assistants oiled and adjusted the release mechanism until it worked perfectly.
	\item The leather may need to be oiled every two to three weeks in order to retain its
suppleness.
	\end{itemize}
}
\item variable noun \\
\textbf{Oil} is a smooth, thick liquid made from plants and is often used for cooking .
 \textit{
	\begin{itemize}
	\item Combine the beans, chopped mint and olive oil in a large bowl.
	\item Drop the slices into the oil and fry until golden brown.
	\end{itemize}
}
\item variable noun \\
\textbf{Oil} is a smooth, thick liquid, often with a pleasant  smell , that you rub into your skin or add to your bath .
 \textit{
	\begin{itemize}
	\item Try a hot bath with some relaxing bath oil.
	\end{itemize}
}
\item countable noun \\
\textbf{Oils} are oil paintings .
 \textit{
	\begin{itemize}
	\item Her colourful oils and works on paper have a naive, dreamlike quality.
	\end{itemize}
}
\item plural noun \\
When an artist paints in \textbf{oils} , he or she uses oil paints.
 \textit{
	\begin{itemize}
	\item When she paints in oils she always uses the same range of colours.
	\end{itemize}
}
\item  \\
 to pour oil on troubled waters \textit{
	\begin{itemize}
	\end{itemize}
}
\item  \\
 to oil the wheels \textit{
	\begin{itemize}
	\end{itemize}
}
\end{enumerate}

\section*{progressive}
{\large \color{blue}  progressives  }
\subsection*{Explain}
\begin{enumerate}
\item adjective \\
Someone who is \textbf{progressive} or has \textbf{progressive}  ideas has modern ideas about how things should be done, rather than traditional ones.
 A \textbf{progressive} is someone who is progressive.
 \textit{
	\begin{itemize}
	\item ...a progressive businessman who had voted for Roosevelt in 1932 and 1936.
	\item Willan was able to point to the progressive changes he had already introduced.
	\item The children go to a progressive school.
	\item The Republicans were deeply split between progressives and conservatives.
	\end{itemize}
}
\item adjective \\
A \textbf{progressive} change happens gradually over a period of time.
 \textit{
	\begin{itemize}
	\item One prominent symptom of the disease is progressive loss of memory.
	\item ...the progressive development of a common foreign and security policy.
	\end{itemize}
}
\item adjective \\
In grammar , \textbf{progressive} means the same as continuous .
 \textit{
	\begin{itemize}
	\end{itemize}
}
\end{enumerate}

\section*{outskirts}
{\large \color{blue}  }
\subsection*{Explain}
\begin{enumerate}
\item plural noun \\
\textbf{The outskirts}  \textbf{of} a city or town are the parts of it that are farthest  away from its centre.
 \textit{
	\begin{itemize}
	\item Hours later we reached the outskirts of New York.
	\end{itemize}
}
\end{enumerate}

\section*{punctual}
{\large \color{blue}  }
\subsection*{Explain}
\begin{enumerate}
\item adjective \\
If you are \textbf{punctual} , you do something or arrive somewhere at the right time and are not late.
 \textit{
	\begin{itemize}
	\item He's always very punctual. I'll see if he's here yet.
	\end{itemize}
}
\end{enumerate}

\section*{oven}
{\large \color{blue}  ovens  }
\subsection*{Explain}
\begin{enumerate}
\item countable noun \\
An \textbf{oven} is a device for cooking that is like a box with a door . You heat it and cook food inside it.
 \textit{
	\begin{itemize}
	\item Put the onions and ginger in the oven and let them roast for thirty minutes.
	\end{itemize}
}
\end{enumerate}

\section*{red}
{\large \color{blue}  reds  redder  reddest  }
\subsection*{Explain}
\begin{enumerate}
\item colour \\
Something that is \textbf{red} is the colour of blood or fire.
 \textit{
	\begin{itemize}
	\item ...a bunch of red roses.
	\item She had small hands with nails painted bright red.
	\end{itemize}
}
\item adjective \\
If you say that someone's face is \textbf{red} , you mean that it is redder than its normal colour, because they are embarrassed , angry , or out of breath .
 \textit{
	\begin{itemize}
	\item With a bright red face I was forced to admit that I had no real idea.
	\item She was red with shame.
	\end{itemize}
}
\item adjective \\
You describe someone's hair as \textbf{red} when it is between red and brown in colour.
 \textit{
	\begin{itemize}
	\item ...a girl with red hair.
	\item He is still vain enough to dye his hair red.
	\end{itemize}
}
\item adjective \\
Your \textbf{red} blood cells or \textbf{red}  corpuscles are the cells in your blood which carry oxygen around your body.
 \textit{
	\begin{itemize}
	\end{itemize}
}
\item variable noun \\
You can refer to red wine as \textbf{red} .
 \textit{
	\begin{itemize}
	\item The spicy flavours in these dishes call for reds rather than whites.
	\end{itemize}
}
\item countable noun \\
If you refer to someone as a \textbf{red} or a \textbf{Red} , you disapprove of the fact that they are a communist, a socialist, or have left-wing ideas.
 \textit{
	\begin{itemize}
	\item They're all so terrified of Reds.
	\end{itemize}
}
\item  \\
 in the red \textit{
	\begin{itemize}
	\end{itemize}
}
\item  \\
 to see red \textit{
	\begin{itemize}
	\end{itemize}
}
\end{enumerate}

\section*{paint}
{\large \color{blue}  paints  painting  painted  }
\subsection*{Explain}
\begin{enumerate}
\item variable noun \\
\textbf{Paint} is a coloured liquid that you put onto a surface with a brush in order to protect the surface or to make it look  nice , or that you use to produce a picture.
 \textit{
	\begin{itemize}
	\item ...a pot of red paint.
	\item They saw some large letters in white paint.
	\item ...water-based artist's paints.
	\end{itemize}
}
\item singular noun \\
On a wall or object, \textbf{the paint} is the covering of dried paint on it.
 \textit{
	\begin{itemize}
	\item The paint was peeling on the window frames.
	\item They'll probably scrape the paint off and make it look like a regular patrol car.
	\end{itemize}
}
\item verb \\
If you \textbf{paint} a wall or an object, you cover it with paint.
 \textit{
	\begin{itemize}
	\item They started to mend the woodwork and paint the walls.
	\item I made a guitar and painted it red.
	\item ...painted furniture.
	\end{itemize}
}
\item verb \\
If you \textbf{paint} something or \textbf{paint} a picture of it, you produce a picture of it using paint.
 \textit{
	\begin{itemize}
	\item He is painting a huge volcano.
	\item Why do people paint pictures?
	\item I had come here to paint.
	\end{itemize}
}
\item verb \\
When you \textbf{paint} a design or message on a surface, you put it on the surface using paint.
 \textit{
	\begin{itemize}
	\item ...a machine for painting white lines down roads.
	\item They went around painting rude slogans on cars.
	\item The recesses are decorated with gold stars, with smaller stars painted along the
edges.
	\end{itemize}
}
\item verb \\
If a woman \textbf{paints} her lips or nails , she puts a coloured cosmetic on them.
 \textit{
	\begin{itemize}
	\item She propped the mirror against her handbag and began to paint her lips.
	\item She painted her fingernails bright red.
	\end{itemize}
}
\item verb \\
If you \textbf{paint} a grim or vivid picture of something, you give a description of it that is grim or vivid.
 \textit{
	\begin{itemize}
	\item The report paints a grim picture of life there.
	\item He went on to paint a rosy picture about how much has already been accomplished.
	\end{itemize}
}
\end{enumerate}

\section*{remote}
{\large \color{blue}  remoter  remotest  }
\subsection*{Explain}
\begin{enumerate}
\item adjective \\
\textbf{Remote} areas are far away from cities and places where most people live , and are therefore difficult to get to.
 \textit{
	\begin{itemize}
	\item Landslides have cut off many villages in remote areas.
	\item ...a remote farm in the Yorkshire dales.
	\end{itemize}
}
\item adjective \\
The \textbf{remote}  past or \textbf{remote}  future is a time that is many years distant from the present .
 \textit{
	\begin{itemize}
	\item Slabs of rock had slipped sideways in the remote past, and formed this hole.
	\end{itemize}
}
\item adjective \\
If something is \textbf{remote}  \textbf{from} a particular subject or area of experience , it is not relevant to it because it is very different.
 \textit{
	\begin{itemize}
	\item This government depends on the wishes of a few who are remote from the people.
	\item Teenagers are forced to study subjects that seem remote from their daily lives.
	\end{itemize}
}
\item adjective \\
If you say that there is a \textbf{remote}  possibility or chance that something will  happen , you are emphasizing that there is only a very small chance that it will happen.
 \textit{
	\begin{itemize}
	\item I use a sunscreen whenever there is even a remote possibility that I will be in the
sun.
	\item The chances of his surviving are pretty remote.
	\end{itemize}
}
\item adjective \\
If you describe someone as \textbf{remote} , you mean that they behave as if they do not want to be friendly or closely involved with other people.
 \textit{
	\begin{itemize}
	\item She looked so beautiful, and at the same time so remote.
	\end{itemize}
}
\end{enumerate}

\section*{projector}
{\large \color{blue}  projectors  }
\subsection*{Explain}
\begin{enumerate}
\item countable noun \\
A \textbf{projector} is a machine that projects images onto a screen or wall.
 \textit{
	\begin{itemize}
	\item The chain is introducing digital projectors and broadcasts of live sports.
	\end{itemize}
}
\end{enumerate}

\section*{rigorous}
{\large \color{blue}  }
\subsection*{Explain}
\begin{enumerate}
\item adjective \\
A test, system, or procedure that is \textbf{rigorous} is very thorough and strict.
 \textit{
	\begin{itemize}
	\item The selection process is based on rigorous tests of competence and experience.
	\item ...a rigorous system of blood analysis.
	\item ...rigorous military training.
	\end{itemize}
}
\item adjective \\
If someone is \textbf{rigorous} in the way that they do something, they are very careful and thorough.
 \textit{
	\begin{itemize}
	\item He is rigorous in his control of expenditure.
	\end{itemize}
}
\end{enumerate}

\section*{rabbit}
{\large \color{blue}  rabbits  rabbiting  rabbited  }
\subsection*{Explain}
\begin{enumerate}
\item countable noun \\
A \textbf{rabbit} is a small furry animal with long ears. Rabbits are sometimes kept as pets , or live  wild in holes in the ground.
 \textbf{Rabbit} is the flesh of this animal eaten as food.
 \textit{
	\begin{itemize}
	\item ...rabbit stew.
	\end{itemize}
}
\end{enumerate}

\section*{sarcastic}
{\large \color{blue}  }
\subsection*{Explain}
\begin{enumerate}
\item adjective \\
Someone who is \textbf{sarcastic}  says or does the opposite of what they really mean in order to mock or insult someone.
 \textit{
	\begin{itemize}
	\item She poked fun at people's shortcomings with sarcastic remarks.
	\end{itemize}
}
\end{enumerate}

\section*{reel}
{\large \color{blue}  reels  reeling  reeled  }
\subsection*{Explain}
\begin{enumerate}
\item countable noun \\
A \textbf{reel} is a cylindrical object around which you wrap something such as cinema film, magnetic tape, fishing line, or cotton thread.
 \textit{
	\begin{itemize}
	\item ...a 30m reel of cable.
	\end{itemize}
}
\item countable noun \\
You can talk about a \textbf{reel} as a way of referring to all the scenes in a film which fit onto one reel of film.
 \textit{
	\begin{itemize}
	\item I shall not reveal the movie's final reel.
	\end{itemize}
}
\item verb \\
If someone \textbf{reels} , they move about in an unsteady way as if they are going to fall.
 \textit{
	\begin{itemize}
	\item He was reeling a little. He must be very drunk.
	\item He lost his balance and reeled back.
	\item I stood up and almost fell, reeling against the deck rail.
	\end{itemize}
}
\item verb \\
If you \textbf{are}  \textbf{reeling} from a shock, you are feeling extremely surprised or upset because of it.
 \textit{
	\begin{itemize}
	\item I'm still reeling from the shock of hearing of it.
	\item It left us reeling with disbelief.
	\end{itemize}
}
\item verb \\
If you say that your brain or your mind  \textbf{is reeling} , you mean that you are very confused because you have too many things to think about.
 \textit{
	\begin{itemize}
	\item His mind reeled at the question.
	\end{itemize}
}
\item countable noun \\
A \textbf{reel} is a type of fast Scottish dance, or fast country dance.
 \textit{
	\begin{itemize}
	\end{itemize}
}
\end{enumerate}

\section*{serious}
{\large \color{blue}  }
\subsection*{Explain}
\begin{enumerate}
\item adjective \\
\textbf{Serious}  problems or situations are very bad and cause people to be worried or afraid .
 \textit{
	\begin{itemize}
	\item Crime is an increasingly serious problem in modern society.
	\item The government still face very serious difficulties.
	\item Doctors said his condition was serious but stable.
	\end{itemize}
}
\item adjective \\
\textbf{Serious} matters are important and deserve  careful and thoughtful consideration .
 \textit{
	\begin{itemize}
	\item I regard this as a serious matter.
	\item Don't laugh boy. This is serious.
	\item ...the serious business of running the country.
	\end{itemize}
}
\item adjective \\
When important matters are dealt with in a \textbf{serious} way, they are given careful and thoughtful consideration.
 \textit{
	\begin{itemize}
	\item We had never discussed marriage in any serious way.
	\item It was a question which deserved serious consideration.
	\item ...serious discussions.
	\end{itemize}
}
\item adjective \\
\textbf{Serious}  music or literature requires concentration to understand or appreciate it.
 \textit{
	\begin{itemize}
	\item ...serious classical music.
	\item There is no point reviewing a blockbuster as you might review a serious novel.
	\end{itemize}
}
\item adjective \\
If someone is \textbf{serious}  \textbf{about} something, they are sincere about what they are saying , doing, or intending to do.
 \textit{
	\begin{itemize}
	\item You really are serious about this, aren't you?
	\item I hope you're not serious.
	\end{itemize}
}
\item adjective \\
\textbf{Serious} people are thoughtful and quiet , and do not laugh very often.
 \textit{
	\begin{itemize}
	\item He's quite a serious person.
	\item She looked at me with big, serious eyes.
	\end{itemize}
}
\item adjective \\
\textbf{Serious}  money is a very large amount of money.
 \textit{
	\begin{itemize}
	\item He started earning serious money only in the sixties.
	\end{itemize}
}
\end{enumerate}

\section*{river}
{\large \color{blue}  rivers  }
\subsection*{Explain}
\begin{enumerate}
\item countable noun \\
A \textbf{river} is a large amount of fresh water flowing continuously in a long line across the land.
 \textit{
	\begin{itemize}
	\item ...a chemical works on the banks of the river.
	\item ...boating on the River Danube.
	\end{itemize}
}
\end{enumerate}

\section*{severe}
{\large \color{blue}  severer  severest  }
\subsection*{Explain}
\begin{enumerate}
\item adjective \\
You use \textbf{severe} to indicate that something bad or undesirable is great or intense .
 \textit{
	\begin{itemize}
	\item ...a business with severe cash flow problems.
	\item I suffered from severe bouts of depression.
	\item Steve passed out on the floor and woke up blinded and in severe pain.
	\item Shortages of professional staff are very severe in some places.
	\end{itemize}
}
\item adjective \\
\textbf{Severe}  punishments or criticisms are very strong or harsh.
 \textit{
	\begin{itemize}
	\item This was a dreadful crime and a severe sentence is necessary.
	\item Before she could reply, my mother launched into a severe reprimand.
	\end{itemize}
}
\item graded adjective \\
If you describe the appearance of someone or something as \textbf{severe} , you do not like its plain appearance and lack of decoration .
 \textit{
	\begin{itemize}
	\item ...wearing her felt hats and severe grey suits.
	\item The cushions add a touch of colour in a room that might otherwise look severe.
	\end{itemize}
}
\end{enumerate}

\section*{scale}
{\large \color{blue}  scales  scaling  scaled  }
\subsection*{Explain}
\begin{enumerate}
\item singular noun \\
If you refer to the \textbf{scale} of something, you are referring to its size or extent, especially when it is very big .
 \textit{
	\begin{itemize}
	\item However, he underestimates the scale of the problem.
	\item You may feel dwarfed by the sheer scale of the place.
	\item The break-down of law and order could result in killing on a massive scale.
	\item The British aid programme is small in scale.
	\end{itemize}
}
\item countable noun \\
A \textbf{scale} is a set of levels or numbers which are used in a particular system of measuring things or are used when comparing things.
 \textit{
	\begin{itemize}
	\item ...an earthquake measuring five-point-five on the Richter scale.
	\item The patient rates the therapies on a scale of zero to ten.
	\item The higher up the social scale they are, the more the men have to lose.
	\end{itemize}
}
\item countable noun \\
A pay \textbf{scale} or \textbf{scale}  \textbf{of}  fees is a list that shows how much someone should be paid, depending , for example, on their age or what work they do.
 \textit{
	\begin{itemize}
	\item ...those on the high end of the pay scale.
	\item A Registered Osteopath will be pleased to tell you his scale of fees before you decide
on a consultation.
	\end{itemize}
}
\item countable noun \\
The \textbf{scale} of a map , plan, or model is the relationship between the size of something in the map, plan,
or model and its size in the real world.
 \textit{
	\begin{itemize}
	\item The map, on a scale of 1:10,000, shows over 5,000 individual paths.
	\end{itemize}
}
\item adjective \\
A \textbf{scale} model or \textbf{scale}  replica of a building or object is a model of it which is smaller than the real thing but
has all the same parts and features .
 \textit{
	\begin{itemize}
	\item Franklin made his mother an intricately detailed scale model of the house.
	\end{itemize}
}
\item countable noun \\
In music, a \textbf{scale} is a fixed sequence of musical notes, each one higher than the next , which begins at a particular note.
 \textit{
	\begin{itemize}
	\item ...the scale of C major.
	\end{itemize}
}
\item countable noun \\
The \textbf{scales} of a fish or reptile are the small, flat pieces of hard skin that cover its body.
 \textit{
	\begin{itemize}
	\end{itemize}
}
\item plural noun \\
\textbf{Scales} are a piece of equipment used for weighing things, for example for weighing amounts
of food that you need in order to make a particular meal .
 \textit{
	\begin{itemize}
	\item ...a pair of kitchen scales.
	\item ...bathroom scales.
	\item I step on the scales practically every morning.
	\end{itemize}
}
\item verb \\
If you \textbf{scale} something such as a mountain or a wall, you climb up it or over it.
 \textit{
	\begin{itemize}
	\item ...Rebecca Stephens, the first British woman to scale Everest.
	\item The men scaled a wall and climbed down scaffolding on the other side.
	\end{itemize}
}
\item  \\
 out of scale \textit{
	\begin{itemize}
	\end{itemize}
}
\item  \\
 to scale \textit{
	\begin{itemize}
	\end{itemize}
}
\end{enumerate}

\section*{sociable}
{\large \color{blue}  }
\subsection*{Explain}
\begin{enumerate}
\item adjective \\
\textbf{Sociable} people are friendly and enjoy  talking to other people.
 \textit{
	\begin{itemize}
	\item She was, and remained, extremely sociable, enjoying dancing, golf and bicycling.
	\item Some children have more sociable personalities than others.
	\end{itemize}
}
\end{enumerate}

\section*{scrutiny}
{\large \color{blue}  }
\subsection*{Explain}
\begin{enumerate}
\item uncountable noun \\
If a person or thing is under \textbf{scrutiny} , they are being studied or observed very carefully.
 \textit{
	\begin{itemize}
	\item His private life came under media scrutiny.
	\item The President promised a government open to public scrutiny.
	\end{itemize}
}
\end{enumerate}

\section*{stern}
{\large \color{blue}  sterner  sternest  sterns  }
\subsection*{Explain}
\begin{enumerate}
\item adjective \\
\textbf{Stern} words or actions are very severe.
 \textit{
	\begin{itemize}
	\item She issued a stern warning to those who persist in violence.
	\item He said stern measures would be taken against the killers.
	\item Michael gave the dog a stern look.
	\end{itemize}
}
\item adjective \\
Someone who is \textbf{stern} is very serious and strict.
 \textit{
	\begin{itemize}
	\item Her father was stern and hard to please.
	\end{itemize}
}
\item countable noun \\
The \textbf{stern} of a boat is the back part of it.
 \textit{
	\begin{itemize}
	\end{itemize}
}
\item  \\
 made of sterner stuff \textit{
	\begin{itemize}
	\end{itemize}
}
\end{enumerate}

\section*{service}
{\large \color{blue}  services  servicing  serviced  }
\subsection*{Explain}
\begin{enumerate}
\item countable noun \\
A \textbf{service} is something that the public needs, such as transport, communications  facilities , hospitals , or energy supplies, which is provided in a planned and organized way by the government
or an official body.
 \textit{
	\begin{itemize}
	\item Britain still boasts the cheapest postal service.
	\item We have started a campaign for better nursery and school services.
	\item The authorities have said they will attempt to maintain essential services.
	\end{itemize}
}
\item countable noun \\
You can sometimes refer to an organization or private company as a particular \textbf{service} when it provides something for the public or acts on behalf of the government.
 \textit{
	\begin{itemize}
	\item ...the BBC World Service.
	\item ...Careers Advisory Services.
	\end{itemize}
}
\item countable noun \\
If an organization or company provides a particular \textbf{service} , they can do a particular job or a type of work for you.
 \textit{
	\begin{itemize}
	\item The kitchen maintains a twenty-four-hour service and can be contacted via Reception.
	\item The larger firm was capable of providing a better range of services.
	\end{itemize}
}
\item plural noun \\
\textbf{Services} are activities such as tourism , banking , and selling things which are part of a country's economy , but are not concerned with producing or manufacturing goods.
 \textit{
	\begin{itemize}
	\item Mining rose by 9.1%, manufacturing by 9.4% and services by 4.3%.
	\item ...the doctrine that a highly developed service sector was the sign of a modern economy.
	\end{itemize}
}
\item uncountable noun \\
The level or standard of \textbf{service} provided by an organization or company is the amount or quality of the work it can
do for you.
 \textit{
	\begin{itemize}
	\item Taking risks is the only way employees can provide effective and efficient customer
service.
	\item The current level of service will be maintained during the holidays.
	\end{itemize}
}
\item countable noun \\
A bus or train \textbf{service} is a route or regular journey that is part of a transport system.
 \textit{
	\begin{itemize}
	\item A bus service operates between Bolton and Salford.
	\end{itemize}
}
\item plural noun \\
Your \textbf{services} are the things that you do or the skills that you use in your job, which other people
 find  useful and are usually willing to pay you for.
 \textit{
	\begin{itemize}
	\item I have obtained the services of a top photographer to take our pictures.
	\item The performers have all offered their services free of charge.
	\end{itemize}
}
\item uncountable noun \\
If you refer to someone's \textbf{service} or \textbf{services}  \textbf{to} a particular organization or activity, you mean that they have done a lot of work for it or spent a lot of their time on it.
 \textit{
	\begin{itemize}
	\item You've given a lifetime of service to athletics.
	\item More than half his long service in parliament has been as a cabinet minister.
	\item ...the two policemen, who have a total of 31 years' service between them.
	\item He was awarded the OBE in 1990 for services to fashion.
	\end{itemize}
}
\item countable noun \\
\textbf{The}  \textbf{Services} are the army, the navy , and the air force.
 \textit{
	\begin{itemize}
	\item In June 1945, Britain still had forty-five per cent of its workforce in the Services
and munitions industries.
	\end{itemize}
}
\item uncountable noun \\
\textbf{Service} is the work done by people or equipment in the army, navy, or air force, for example
during a war.
 \textit{
	\begin{itemize}
	\item The regiment was recruited from the Highlands specifically for service in India.
	\item ...an aircraft carrier that saw service in World War II.
	\end{itemize}
}
\item uncountable noun \\
When you receive \textbf{service} in a restaurant, hotel, or shop, an employee asks you what you want or gives you what you have ordered.
 \textit{
	\begin{itemize}
	\item Bill was £68 including service and a couple of bar drinks and wine.
	\item ...clean stores with respectful service and fair prices.
	\end{itemize}
}
\item countable noun \\
A \textbf{service} is a religious ceremony that takes place in a church.
 \textit{
	\begin{itemize}
	\item After the hour-long service, his body was taken to a cemetery in the south of the
city.
	\item ...the church in which the President was attending morning service.
	\end{itemize}
}
\item countable noun \\
A \textbf{dinner}  \textbf{service} or a \textbf{tea}  \textbf{service} is a complete set of plates, cups, saucers , and other pieces of china.
 \textit{
	\begin{itemize}
	\item ...a 60-piece dinner service.
	\end{itemize}
}
\item countable noun \\
A \textbf{services} is a place beside a motorway where you can buy petrol and other things, or have a meal .
 \textit{
	\begin{itemize}
	\item They had to pull up, possibly go to a motorway services or somewhere like that.
	\item We have repeatedly told planners that services are vital on a motorway.
	\end{itemize}
}
\item countable noun \\
In tennis , badminton , and some other sports, when it is your \textbf{service} , it is your turn to serve.
 \textit{
	\begin{itemize}
	\item She conceded just three points on her service during the first set.
	\end{itemize}
}
\item adjective \\
\textbf{Service} is used to describe the parts of a building or structure that are used by the staff who clean , repair , or look after it, and are not usually used by the public.
 \textit{
	\begin{itemize}
	\item He wheeled the trolley down the corridor and disappeared with it into the service
lift.
	\item ...the bigger tunnels, which run either side of the service tunnel.
	\end{itemize}
}
\item uncountable noun \\
If someone is in \textbf{service} , they are working as a servant.
 \textit{
	\begin{itemize}
	\item If a young woman did not have a dowry, she went into domestic service.
	\end{itemize}
}
\item verb \\
If you have a vehicle or machine \textbf{serviced} , you arrange for someone to examine , adjust , and clean it so that it will keep working efficiently and safely.
 \textbf{Service} is also a noun .
 \textit{
	\begin{itemize}
	\item I had my car serviced at the local garage.
	\item Make sure that all gas fires and central heating boilers are serviced annually.
	\item The car needs a service.
	\item The company sends a service engineer to fix the disk drive before it fails.
	\end{itemize}
}
\item verb \\
If a country or organization \textbf{services} its debts, it pays the interest on them.
 \textit{
	\begin{itemize}
	\item Almost a quarter of the country's export earnings go to service a foreign debt of
$29 billion.
	\end{itemize}
}
\item verb \\
If someone or something \textbf{services} an organization, a project, or a group of people, they provide it with the things
that it needs in order to function properly or effectively.
 \textit{
	\begin{itemize}
	\item There are now 400 staff at headquarters, servicing our regional work.
	\item Fossil fuels such as oil and gas will service our needs for some considerable time
to come.
	\end{itemize}
}
\item  \\
 at the service of \textit{
	\begin{itemize}
	\end{itemize}
}
\item  \\
 at your service \textit{
	\begin{itemize}
	\end{itemize}
}
\item  \\
 do someone a service \textit{
	\begin{itemize}
	\end{itemize}
}
\item  \\
 in service / out of service \textit{
	\begin{itemize}
	\end{itemize}
}
\item  \\
 of service \textit{
	\begin{itemize}
	\end{itemize}
}
\end{enumerate}

\section*{shopkeeper}
{\large \color{blue}  shopkeepers  }
\subsection*{Explain}
\begin{enumerate}
\item countable noun \\
A \textbf{shopkeeper} is a person who owns or manages a small shop.
 \textit{
	\begin{itemize}
	\end{itemize}
}
\end{enumerate}

\section*{superb}
{\large \color{blue}  }
\subsection*{Explain}
\begin{enumerate}
\item adjective \\
If something is \textbf{superb} , its quality is very good indeed.
 \textit{
	\begin{itemize}
	\item There is a superb 18-hole golf course 6 miles away.
	\item The waters are crystal clear and offer a superb opportunity for swimming.
	\end{itemize}
}
\item adjective \\
If you say that someone has \textbf{superb}  confidence , control, or skill , you mean that they have very great confidence, control, or skill.
 \textit{
	\begin{itemize}
	\item With superb skill he managed to make a perfect landing.
	\end{itemize}
}
\end{enumerate}

\section*{stale}
{\large \color{blue}  staler  stalest  }
\subsection*{Explain}
\begin{enumerate}
\item adjective \\
\textbf{Stale} food is no longer fresh or good to eat .
 \textit{
	\begin{itemize}
	\item Their daily diet consisted of a lump of stale bread, a bowl of rice and stale water.
	\end{itemize}
}
\item adjective \\
\textbf{Stale} air or a \textbf{stale}  smells is unpleasant because it is no longer fresh.
 \textit{
	\begin{itemize}
	\item A layer of smoke hung low in the stale air.
	\item ...the smell of stale sweat.
	\end{itemize}
}
\item graded adjective \\
If you feel  \textbf{stale} , you are bored because you have no new ideas or enthusiasm for what you are doing.
 \textit{
	\begin{itemize}
	\item I believe in progression, in taking risks, in never getting stale.
	\end{itemize}
}
\item adjective \\
If you say that a place, an activity , or an idea is \textbf{stale} , you mean that it has become boring because it is always the same.
 \textit{
	\begin{itemize}
	\item Her relationship with Mark has become stale.
	\item The Government, he said, were sticking to stale ideas.
	\end{itemize}
}
\end{enumerate}

\section*{supreme}
{\large \color{blue}  }
\subsection*{Explain}
\begin{enumerate}
\item adjective \\
\textbf{Supreme} is used in the title of a person or an official group to indicate that they are at the highest level in a particular organization or system.
 \textit{
	\begin{itemize}
	\item MacArthur was Supreme Commander for the allied powers in the Pacific.
	\item ...the Supreme Court.
	\item ...the Supreme Being.
	\end{itemize}
}
\item adjective \\
You use \textbf{supreme} to emphasize that a quality or thing is very great.
 \textit{
	\begin{itemize}
	\item Her approval was of supreme importance.
	\item The lady conspired to seize supreme power.
	\end{itemize}
}
\end{enumerate}

\section*{stock}
{\large \color{blue}  stocks  stocking  stocked  }
\subsection*{Explain}
\begin{enumerate}
\item countable noun \\
\textbf{Stocks} are shares in the ownership of a company, or investments on which a fixed amount of interest will be paid.
 \textit{
	\begin{itemize}
	\item ...the buying and selling of stocks and shares.
	\item As stock prices have dropped, so too has bank capital.
	\end{itemize}
}
\item uncountable noun \\
A company's \textbf{stock} is the amount of money which the company has through selling shares.
 \textit{
	\begin{itemize}
	\item The stock was valued in the market at $460 million.
	\item The Fisher family holds 40% of the stock.
	\end{itemize}
}
\item verb \\
If a shop \textbf{stocks} particular goods, it keeps a supply of them to sell.
 \textit{
	\begin{itemize}
	\item The shop stocks everything from chocolate to recycled loo paper.
	\end{itemize}
}
\item uncountable noun \\
A shop's \textbf{stock} is the total amount of goods which it has available to sell.
 \textit{
	\begin{itemize}
	\item We took the decision to withdraw a quantity of stock from sale.
	\end{itemize}
}
\item verb \\
If you \textbf{stock} something such as a cupboard , shelf , or room, you fill it with food or other things.
 \textbf{Stock up} means the same as stock .
 \textit{
	\begin{itemize}
	\item I worked stocking shelves in a grocery store.
	\item Some families stocked their cellars with food and water.
	\item The kitchen cupboard was stocked with tins of soup.
	\item I had to stock the boat up with food.
	\item Customers travel from hundreds of miles away to stock up their deep freezes.
	\item You can stock up the freezer with ice cream ready for the next sunny day
	\end{itemize}
}
\item countable noun \\
If you have a \textbf{stock}  \textbf{of} things, you have a supply of them stored in a place ready to be used.
 \textit{
	\begin{itemize}
	\item Keep a stock of essentials such as bread in the freezer.
	\item Stocks of ammunition were running low.
	\end{itemize}
}
\item singular noun \\
The \textbf{stock} of something is the total amount of it that is available in a particular area.
 \textit{
	\begin{itemize}
	\item ...the stock of accommodation available to be rented.
	\end{itemize}
}
\item uncountable noun \\
If you are from a particular \textbf{stock} , you are descended from a particular group of people.
 \textit{
	\begin{itemize}
	\item We are both from working class stock.
	\item The World Service no longer reflects the interests of listeners of British stock.
	\end{itemize}
}
\item plural noun \\
\textbf{Stock} are cattle, sheep, pigs , or other animals which are kept by a farmer , usually ones which have been specially bred.
 \textit{
	\begin{itemize}
	\item I am carefully selecting the breeding stock.
	\item His herd of 170 dairy cattle and 200 young stock are kept on the land.
	\end{itemize}
}
\item adjective \\
A \textbf{stock}  answer , expression, or way of doing something is one that is very commonly used, especially because people cannot be bothered to think of something new.
 \textit{
	\begin{itemize}
	\item My boss had a stock response–'If it ain't broke, don't fix it!'.
	\item National security is the stock excuse for keeping things confidential.
	\end{itemize}
}
\item variable noun \\
\textbf{Stock} is a liquid, usually made by boiling meat, bones, or vegetables in water, that is
used to give flavour to soups and sauces .
 \textit{
	\begin{itemize}
	\end{itemize}
}
\item plural noun \\
In former times, the \textbf{stocks} were an instrument of punishment. A criminal's hands and legs were locked into holes in a wooden frame while people threw things at them.
 \textit{
	\begin{itemize}
	\end{itemize}
}
\item  \\
 in stock/out of stock \textit{
	\begin{itemize}
	\end{itemize}
}
\item  \\
 take stock \textit{
	\begin{itemize}
	\end{itemize}
}
\end{enumerate}

\section*{tame}
{\large \color{blue}  tamer  tamest  tames  taming  tamed  }
\subsection*{Explain}
\begin{enumerate}
\item adjective \\
A \textbf{tame} animal or bird is one that is not afraid of humans.
 \textit{
	\begin{itemize}
	\item The deer never became tame; they would run away if you approached them.
	\end{itemize}
}
\item adjective \\
If you say that something or someone is \textbf{tame} , you are criticizing them for being weak and uninteresting , rather than forceful or shocking .
 \textit{
	\begin{itemize}
	\item Some of today's political demonstrations look rather tame.
	\item The report was pretty tame stuff.
	\end{itemize}
}
\item verb \\
If someone \textbf{tames} a wild animal or bird, they train it not to be afraid of humans and to do what they say.
 \textit{
	\begin{itemize}
	\item The Amazons were believed to have been the first to tame horses.
	\end{itemize}
}
\item verb \\
If you \textbf{tame} someone or something that is dangerous , uncontrolled , or likely to cause trouble , you bring them under control.
 \textit{
	\begin{itemize}
	\item Two regiments of cavalry were called out to tame the crowds.
	\end{itemize}
}
\end{enumerate}

\section*{stove}
{\large \color{blue}  stoves  }
\subsection*{Explain}
\begin{enumerate}
\item countable noun \\
A \textbf{stove} is a piece of equipment which provides heat, either for cooking or for heating a room .
 \textit{
	\begin{itemize}
	\item She put the kettle on the gas stove.
	\end{itemize}
}
\item countable noun \\
A \textbf{stove} is the top of a cooker .
 \textit{
	\begin{itemize}
	\end{itemize}
}
\end{enumerate}

\section*{technical}
{\large \color{blue}  }
\subsection*{Explain}
\begin{enumerate}
\item adjective \\
\textbf{Technical} means involving the sorts of machines, processes, and materials that are used in industry , transport , and communications .
 \textit{
	\begin{itemize}
	\item In order to reach this limit a number of technical problems will have to be solved.
	\item ...jobs that require technical knowledge.
	\item Many technical experts at the time had doubts about the technology.
	\end{itemize}
}
\item adjective \\
You use \textbf{technical} to describe the practical skills and methods used to do an activity such as an art, a craft , or a sport .
 \textit{
	\begin{itemize}
	\item Their technical ability is exceptional.
	\item In the realm of sculpture too, the technical skill of foreign artists was long recognised.
	\end{itemize}
}
\item adjective \\
\textbf{Technical} language involves using special words to describe the details of a specialized activity.
 \textit{
	\begin{itemize}
	\item The technical term for sunburn is erythema.
	\item He's just written a book: large format, nicely illustrated and not too technical.
	\end{itemize}
}
\end{enumerate}

\section*{terrific}
{\large \color{blue}  }
\subsection*{Explain}
\begin{enumerate}
\item adjective \\
If you describe something or someone as \textbf{terrific} , you are very pleased with them or very impressed by them.
 \textit{
	\begin{itemize}
	\item What a terrific idea!
	\item Everybody there was having a terrific time.
	\item You look terrific, Ann. You really do.
	\end{itemize}
}
\item adjective \\
\textbf{Terrific} means very great in amount, degree , or intensity .
 \textit{
	\begin{itemize}
	\item He did a terrific amount of fundraising.
	\item All of a sudden there was a terrific bang and a flash of smoke.
	\end{itemize}
}
\end{enumerate}

\section*{tanker}
{\large \color{blue}  tankers  }
\subsection*{Explain}
\begin{enumerate}
\item countable noun \\
A \textbf{tanker} is a very large ship used for transporting large quantities of gas or liquid, especially oil.
 \textit{
	\begin{itemize}
	\item An oil tanker has run aground.
	\end{itemize}
}
\item countable noun \\
A \textbf{tanker} is a large truck , railway  vehicle , or aircraft used for transporting large quantities of a substance.
 \textit{
	\begin{itemize}
	\item ...an accident involving a petrol tanker on the M27, east of Southampton.
	\end{itemize}
}
\end{enumerate}

\section*{tiny}
{\large \color{blue}  tinier  tiniest  }
\subsection*{Explain}
\begin{enumerate}
\item adjective \\
Something or someone that is \textbf{tiny} is extremely small.
 \textit{
	\begin{itemize}
	\item The living room is tiny.
	\item Though she was tiny, she had a very loud voice.
	\item The crop represents a tiny fraction of U.S. production.
	\end{itemize}
}
\end{enumerate}

\section*{temple}
{\large \color{blue}  temples  }
\subsection*{Explain}
\begin{enumerate}
\item countable noun \\
A \textbf{temple} is a building used for the worship of a god or gods, especially in the Buddhist and Hindu  religions , and in ancient Greek and Roman times.
 \textit{
	\begin{itemize}
	\item ...a small Hindu temple.
	\item ...the Temple of Diana at Ephesus.
	\end{itemize}
}
\item countable noun \\
Your \textbf{temples} are the flat parts on each side of the front part of your head, near your forehead .
 \textit{
	\begin{itemize}
	\item Threads of silver ran through his beard and the hair at his temples.
	\end{itemize}
}
\end{enumerate}

\section*{vocal}
{\large \color{blue}  }
\subsection*{Explain}
\begin{enumerate}
\item adjective \\
You say that people are \textbf{vocal} when they speak forcefully about something that they feel strongly about.
 \textit{
	\begin{itemize}
	\item He has been very vocal in his displeasure over the results.
	\item A public inquiry earlier this year produced vocal opposition from residents.
	\end{itemize}
}
\item adjective \\
\textbf{Vocal}  means involving the use of the human voice, especially in singing.
 \textit{
	\begin{itemize}
	\item ...a wider range of vocal styles.
	\item ...vocal training.
	\end{itemize}
}
\end{enumerate}

\section*{unit}
{\large \color{blue}  units  }
\subsection*{Explain}
\begin{enumerate}
\item countable noun \\
If you consider something as a \textbf{unit} , you consider it as a single, complete thing.
 \textit{
	\begin{itemize}
	\item Agriculture was based in the past on the family as a unit.
	\end{itemize}
}
\item countable noun \\
A \textbf{unit} is a group of people who work together at a specific job , often in a particular place.
 \textit{
	\begin{itemize}
	\item ...the health services research unit.
	\item The results from this unit are staggering.
	\end{itemize}
}
\item countable noun \\
A \textbf{unit} is a group within an armed force or police force, whose members fight or work together or carry out a particular task .
 \textit{
	\begin{itemize}
	\item One secret military unit tried to contaminate the drinking water of the refugees.
	\item Two small Marine units are trapped inside the city.
	\end{itemize}
}
\item countable noun \\
A \textbf{unit} is a small machine which has a particular function, often part of a larger machine.
 \textit{
	\begin{itemize}
	\item The unit plugs into any TV set.
	\end{itemize}
}
\item countable noun \\
A \textbf{unit} of measurement is a fixed standard quantity, length, or weight that is used for measuring things.
The litre , the centimetre , and the ounce are all units.
 \textit{
	\begin{itemize}
	\end{itemize}
}
\item countable noun \\
A \textbf{unit} is one of the parts that a textbook is divided into.
 \textit{
	\begin{itemize}
	\end{itemize}
}
\end{enumerate}

\section*{waterproof}
{\large \color{blue}  waterproofs  waterproofing  waterproofed  }
\subsection*{Explain}
\begin{enumerate}
\item adjective \\
Something which is \textbf{waterproof} does not let water pass through it.
 \textit{
	\begin{itemize}
	\item Take waterproof clothing–Orkney weather is unpredictable.
	\item Designed to be completely waterproof, the lights are manufactured from heavy duty
plastic.
	\end{itemize}
}
\item countable noun \\
\textbf{Waterproofs} are items of clothing which do not let water in.
 \textit{
	\begin{itemize}
	\item For staying dry you'll want nice lightweight waterproofs to wear over your leathers.
	\end{itemize}
}
\item verb \\
If something \textbf{is waterproofed} , it is treated so that water cannot pass through it or damage it.
 \textit{
	\begin{itemize}
	\item The whole boat has been totally waterproofed.
	\item Waterproofed fabric pants are more expensive than plastic pants.
	\end{itemize}
}
\end{enumerate}

\section*{variation}
{\large \color{blue}  variations  }
\subsection*{Explain}
\begin{enumerate}
\item countable noun \\
A \textbf{variation on} something is the same thing presented in a slightly different form.
 \textit{
	\begin{itemize}
	\item This delicious variation on an omelette is quick and easy to prepare.
	\item Many theories on punishment exist, all of which are variations on a theme.
	\end{itemize}
}
\item variable noun \\
A \textbf{variation} is a change or slight  difference in a level , amount, or quantity .
 \textit{
	\begin{itemize}
	\item The survey found a wide variation in the prices charged for canteen food.
	\item Every day without variation my grandfather ate a plate of cold ham.
	\end{itemize}
}
\end{enumerate}

\section*{wicked}
{\large \color{blue}  }
\subsection*{Explain}
\begin{enumerate}
\item adjective \\
You use \textbf{wicked} to describe someone or something that is very bad and deliberately harmful to people.
 \textit{
	\begin{itemize}
	\item She described the shooting as a wicked attack.
	\item She flew at me, shouting how wicked and evil I was.
	\end{itemize}
}
\item adjective \\
If you describe someone or something as \textbf{wicked} , you mean that they are rather  naughty , but in a way that you find  attractive or enjoyable.
 \textit{
	\begin{itemize}
	\item She had a wicked sense of humour.
	\item I adore white chocolate, and I always feel very wicked when eating it.
	\end{itemize}
}
\end{enumerate}

\section*{weekend}
{\large \color{blue}  weekends  }
\subsection*{Explain}
\begin{enumerate}
\item countable noun \\
A \textbf{weekend} is Saturday and Sunday.
 \textit{
	\begin{itemize}
	\item She had agreed to have dinner with him in town the following weekend.
	\item He told me to give you a call over the weekend.
	\end{itemize}
}
\end{enumerate}

\section*{annual}
{\large \color{blue}  annuals  }
\subsection*{Explain}
\begin{enumerate}
\item adjective \\
\textbf{Annual}  events  happen once every year.
 \textit{
	\begin{itemize}
	\item ...the annual conference of Britain's trade union movement.
	\item In its annual report, UNICEF says at least 40,000 children die every day.
	\end{itemize}
}
\item adjective \\
\textbf{Annual}  quantities or rates relate to a period of one year.
 \textit{
	\begin{itemize}
	\item Annual costs, tuition and fees, £1,600.
	\item The electronic and printing unit has annual sales of about $80 million.
	\end{itemize}
}
\item countable noun \\
An \textbf{annual} is a book or magazine that is published once a year.
 \textit{
	\begin{itemize}
	\item I looked for Wyman's picture in my high-school annual.
	\item He tried the various almanacs, annuals and gazettes which were held in the library.
	\end{itemize}
}
\item countable noun \\
An \textbf{annual} is a plant that grows and dies within one year.
 \textit{
	\begin{itemize}
	\item The simplest way to deal with these hardy annuals is to sow them where they are to
flower.
	\end{itemize}
}
\end{enumerate}

\section*{air}
{\large \color{blue}  airs  airing  aired  }
\subsection*{Explain}
\begin{enumerate}
\item uncountable noun \\
\textbf{Air} is the mixture of gases which forms the Earth's atmosphere and which we breathe .
 \textit{
	\begin{itemize}
	\item Draughts help to circulate air.
	\item Keith opened the window and leaned out into the cold air.
	\item ...water and air pollutants.
	\end{itemize}
}
\item singular noun \\
\textbf{The}  \textbf{air} is the space around things or above the ground.
 \textit{
	\begin{itemize}
	\item Government troops broke up the protest by firing their guns in the air.
	\item Smoke seemed to hang in the air.
	\end{itemize}
}
\item uncountable noun \\
\textbf{Air} is used to refer to travel in aircraft.
 \textit{
	\begin{itemize}
	\item Air travel will continue to grow at about 6% per year.
	\item Casualties had to be brought to hospital by air.
	\end{itemize}
}
\item countable noun \\
An \textbf{air} is a simple tune which can be easily recognized and remembered .
 \textit{
	\begin{itemize}
	\end{itemize}
}
\item singular noun \\
If you say that someone or something has a particular \textbf{air} , you mean that they give this general impression .
 \textit{
	\begin{itemize}
	\item Jennifer regarded him with an air of amusement.
	\item The meal gave the occasion an almost festive air.
	\end{itemize}
}
\item plural noun \\
If you say that someone is putting on \textbf{airs} or giving themselves \textbf{airs} , you are criticizing them for behaving as if they are better than other people.
 \textit{
	\begin{itemize}
	\item We're poor and we never put on airs.
	\end{itemize}
}
\item verb \\
If a broadcasting company \textbf{airs} a television or radio programme, they show it on television or broadcast it on the
radio.
 \textit{
	\begin{itemize}
	\item Tonight PBS will air a documentary called 'Democracy In Action'.
	\end{itemize}
}
\item verb \\
If you \textbf{air} your opinions, you make them known to people.
 \textit{
	\begin{itemize}
	\item They sat for more than six hours, and both sides agreed they had aired all their
differences.
	\item The whole issue was thoroughly aired at the meeting.
	\end{itemize}
}
\item verb \\
If you \textbf{air} a room or building, you let  fresh air into it.
 \textit{
	\begin{itemize}
	\item One day a week her mother systematically cleaned and aired each room.
	\end{itemize}
}
\item verb \\
If you \textbf{air} clothing or bedding , you put it somewhere warm to make sure that it is completely dry.
 \textit{
	\begin{itemize}
	\item I ironed the shirts myself, aired them and placed them in drawers in his room.
	\end{itemize}
}
\item  \\
 to clear the air \textit{
	\begin{itemize}
	\end{itemize}
}
\item  \\
 airs and graces \textit{
	\begin{itemize}
	\end{itemize}
}
\item  \\
 in the air \textit{
	\begin{itemize}
	\end{itemize}
}
\item  \\
 on the air \textit{
	\begin{itemize}
	\end{itemize}
}
\item  \\
 into thin air/out of thin air \textit{
	\begin{itemize}
	\end{itemize}
}
\item  \\
 up in the air \textit{
	\begin{itemize}
	\end{itemize}
}
\item  \\
 to be walking on air \textit{
	\begin{itemize}
	\end{itemize}
}
\end{enumerate}

\section*{aural}
{\large \color{blue}  }
\subsection*{Explain}
\begin{enumerate}
\item adjective \\
\textbf{Aural} means related to the sense of hearing. Compare  acoustic .
 \textit{
	\begin{itemize}
	\item He became famous as an inventor of astonishing visual and aural effects.
	\end{itemize}
}
\end{enumerate}

\section*{bald}
{\large \color{blue}  balder  baldest  }
\subsection*{Explain}
\begin{enumerate}
\item adjective \\
Someone who is \textbf{bald} has little or no hair on the top of their head.
 \textit{
	\begin{itemize}
	\item The man's bald head was beaded with sweat.
	\item She is going bald.
	\end{itemize}
}
\item adjective \\
If a tyre is \textbf{bald} , its surface has worn down and it is no longer safe to use.
 \textit{
	\begin{itemize}
	\end{itemize}
}
\item adjective \\
A \textbf{bald}  statement is in plain language and contains no extra  explanation or information .
 \textit{
	\begin{itemize}
	\item The announcement came in a bald statement from the official news agency.
	\item The bald truth is he's just not happy.
	\end{itemize}
}
\end{enumerate}

\section*{bureaucracy}
{\large \color{blue}  bureaucracies  }
\subsection*{Explain}
\begin{enumerate}
\item countable noun \\
A \textbf{bureaucracy} is an administrative system operated by a large number of officials.
 \textit{
	\begin{itemize}
	\item State bureaucracies can tend to stifle enterprise and initiative.
	\end{itemize}
}
\item uncountable noun \\
\textbf{Bureaucracy}  refers to all the rules and procedures followed by government departments and similar organizations, especially when you think that these are complicated and cause long delays .
 \textit{
	\begin{itemize}
	\item People usually complain about having to deal with too much bureaucracy.
	\end{itemize}
}
\end{enumerate}

\section*{consecutive}
{\large \color{blue}  }
\subsection*{Explain}
\begin{enumerate}
\item adjective \\
\textbf{Consecutive} periods of time or events happen one after the other without interruption.
 \textit{
	\begin{itemize}
	\item The Cup was won for the third consecutive year by the Toronto Maple Leafs.
	\item ...two consecutive wet British summers.
	\end{itemize}
}
\end{enumerate}

\section*{chamber}
{\large \color{blue}  chambers  }
\subsection*{Explain}
\begin{enumerate}
\item countable noun \\
A \textbf{chamber} is a large room, especially one that is used for formal meetings.
 \textit{
	\begin{itemize}
	\item We are going to make sure we are in the council chamber every time he speaks.
	\end{itemize}
}
\item countable noun \\
You can refer to a country's parliament or to one section of it as a \textbf{chamber} .
 \textit{
	\begin{itemize}
	\item More than 80 parties are contesting seats in the two-chamber parliament.
	\item His government has only a 16-seat majority in the Chamber of Deputies.
	\end{itemize}
}
\item countable noun \\
A \textbf{chamber} is a room designed and equipped for a particular purpose .
 \textit{
	\begin{itemize}
	\item For many, the dentist's surgery remains a torture chamber.
	\end{itemize}
}
\item countable noun \\
A \textbf{chamber} is a hollow place inside the body of a person or animal, or inside a plant.
 \textit{
	\begin{itemize}
	\end{itemize}
}
\item plural noun \\
The offices used by judges and barristers are referred to as \textbf{chambers} .
 \textit{
	\begin{itemize}
	\end{itemize}
}
\end{enumerate}

\section*{continuous}
{\large \color{blue}  }
\subsection*{Explain}
\begin{enumerate}
\item adjective \\
A \textbf{continuous} process or event continues for a period of time without stopping .
 \textit{
	\begin{itemize}
	\item Residents report that they heard continuous gunfire.
	\item ...all employees who had a record of five years' continuous employment with the firm.
	\item There is a continuous stream of phone calls.
	\end{itemize}
}
\item adjective \\
A \textbf{continuous} line or surface has no gaps or holes in it.
 \textit{
	\begin{itemize}
	\item ...a continuous line of boats.
	\item ...the continuous frieze of sculpted figures.
	\end{itemize}
}
\item adjective \\
In English grammar , \textbf{continuous}  verb groups are formed using the auxiliary 'be' and the present  participle of a verb, as in 'I'm feeling a bit  tired ' and 'She had been watching them for some time'. Continuous verb groups are used especially when you are focusing on a particular moment . Compare  simple .
 \textit{
	\begin{itemize}
	\end{itemize}
}
\end{enumerate}

\section*{character}
{\large \color{blue}  characters  }
\subsection*{Explain}
\begin{enumerate}
\item countable noun \\
The \textbf{character} of a person or place consists of all the qualities they have that make them distinct from other people or places.
 \textit{
	\begin{itemize}
	\item Perhaps there is a negative side to his character that you haven't seen yet.
	\item The character of this country has been formed by immigration.
	\end{itemize}
}
\item singular noun \\
If something has a particular \textbf{character} , it has a particular quality.
 \textit{
	\begin{itemize}
	\item The financial concessions granted to British Aerospace were, he said, of a precarious
character.
	\item The state farms were semi-military in character.
	\end{itemize}
}
\item singular noun \\
You can use \textbf{character} to refer to the qualities that people from a particular place are believed to have.
 \textit{
	\begin{itemize}
	\item Individuality is a valued and inherent part of the British character.
	\end{itemize}
}
\item countable noun \\
You use \textbf{character} to say what kind of person someone is. For example, if you say that someone is a strange  \textbf{character} , you mean they are strange.
 \textit{
	\begin{itemize}
	\item It's that kind of courage and determination that makes him such a remarkable character.
	\item What a sad character that Nigel is.
	\end{itemize}
}
\item variable noun \\
Your \textbf{character} is your personality , especially how reliable and honest you are. If someone is \textbf{of} good \textbf{character} , they are reliable and honest. If they are \textbf{of}  bad  \textbf{character} , they are unreliable and dishonest .
 \textit{
	\begin{itemize}
	\item He's begun a series of personal attacks on my character.
	\item Mr Bartman was a man of good character.
	\end{itemize}
}
\item uncountable noun \\
If you say that someone has \textbf{character} , you mean that they have the ability to deal effectively with difficult , unpleasant , or dangerous situations.
 \textit{
	\begin{itemize}
	\item She showed real character in her attempts to win over the crowd.
	\item I didn't know Ron had that much strength of character.
	\end{itemize}
}
\item uncountable noun \\
If you say that a place has \textbf{character} , you mean that it has an interesting or unusual quality which makes you notice it and like it.
 \textit{
	\begin{itemize}
	\item An ugly shopping centre stands across from one of the few buildings with character.
	\end{itemize}
}
\item countable noun \\
The \textbf{characters} in a film, book, or play are the people that it is about.
 \textit{
	\begin{itemize}
	\item The film is autobiographical and the central character is played by Collard himself.
	\item He's made the characters believable.
	\end{itemize}
}
\item countable noun \\
If you say that someone is a \textbf{character} , you mean that they are interesting, unusual, or amusing .
 \textit{
	\begin{itemize}
	\item He'll be sadly missed. He was a real character.
	\end{itemize}
}
\item countable noun \\
A \textbf{character} is a letter, number, or other symbol that is written or printed.
 \textit{
	\begin{itemize}
	\end{itemize}
}
\item  \\
 in character/out of character \textit{
	\begin{itemize}
	\end{itemize}
}
\end{enumerate}

\section*{daily}
{\large \color{blue}  dailies  }
\subsection*{Explain}
\begin{enumerate}
\item adverb \\
If something happens  \textbf{daily} , it happens every day.
 \textbf{Daily} is also an adjective .
 \textit{
	\begin{itemize}
	\item Cathay Pacific flies daily non-stop to Hong Kong from Heathrow.
	\item The Visitor Centre is open daily 8.30 a.m.–4.30 p.m.
	\item They held daily press briefings.
	\end{itemize}
}
\item adjective \\
\textbf{Daily}  quantities or rates relate to a period of one day.
 \textit{
	\begin{itemize}
	\item ...a diet containing adequate daily amounts of fresh fruit.
	\item Our average daily turnover is about £300.
	\end{itemize}
}
\item countable noun \\
A \textbf{daily} is a newspaper that is published every day of the week except Sunday .
 \textbf{Daily} is also an adjective.
 \textit{
	\begin{itemize}
	\item Copies of the local daily had been scattered on a table.
	\item He studied the daily papers.
	\end{itemize}
}
\item  \\
 daily life \textit{
	\begin{itemize}
	\end{itemize}
}
\end{enumerate}

\section*{college}
{\large \color{blue}  colleges  }
\subsection*{Explain}
\begin{enumerate}
\item variable noun \\
A \textbf{college} is an institution where students study after they have left school.
 \textit{
	\begin{itemize}
	\item Their daughter Joanna is doing business studies at a local college.
	\item Stephanie took up making jewellery after leaving art college this summer.
	\item He is now a professor of economics at Western New England College in Springfield,
Massachusetts.
	\end{itemize}
}
\item countable noun \\
A \textbf{college} is one of the institutions which some British universities are divided into.
 \textit{
	\begin{itemize}
	\item He was educated at Balliol College, Oxford.
	\end{itemize}
}
\item countable noun \\
At some universities in the United  States , \textbf{colleges} are divisions which offer  degrees in particular subjects.
 \textit{
	\begin{itemize}
	\item ...a professor at the University of Florida College of Law.
	\end{itemize}
}
\item countable noun \\
\textbf{College} is used in Britain in the names of some secondary schools which charge fees .
 \textit{
	\begin{itemize}
	\item In 1854, Cheltenham Ladies' College became the first girls' public school.
	\end{itemize}
}
\item countable noun \\
A \textbf{college} of a particular kind is an organized group of people who have special duties and powers.
 \textit{
	\begin{itemize}
	\item He is a member of the Royal College of Physicians.
	\item There is a college of international supervisors working together.
	\end{itemize}
}
\end{enumerate}

\section*{decent}
{\large \color{blue}  }
\subsection*{Explain}
\begin{enumerate}
\item adjective \\
\textbf{Decent} is used to describe something which is considered to be of an acceptable  standard or quality.
 \textit{
	\begin{itemize}
	\item Nearby is a village with a decent pub.
	\item He didn't get a decent explanation.
	\item The lack of a decent education did not defeat Rey.
	\end{itemize}
}
\item adjective \\
\textbf{Decent} is used to describe something which is morally correct or acceptable.
 \textit{
	\begin{itemize}
	\item But, after a decent interval, trade relations began to return to normal.
	\item As soon as it was decent, he turned and left the cemetery.
	\end{itemize}
}
\item adjective \\
\textbf{Decent} people are honest and behave in a way that most people approve of.
 \textit{
	\begin{itemize}
	\item The majority of people around here are decent people.
	\item The jury will see what a decent guy he is.
	\end{itemize}
}
\item  \\
 do the decent thing \textit{
	\begin{itemize}
	\end{itemize}
}
\end{enumerate}

\section*{comment}
{\large \color{blue}  comments  commenting  commented  }
\subsection*{Explain}
\begin{enumerate}
\item verb \\
If you \textbf{comment}  \textbf{on} something, you give your opinion about it or you give an explanation for it.
 \textit{
	\begin{itemize}
	\item So far, Mr Cook has not commented on these reports.
	\item Stratford police refuse to comment on whether anyone has been arrested.
	\item You really can't comment till you know the facts.
	\item 'I'm always happy with new developments,' he commented.
	\item Stuart commented that this was very true.
	\end{itemize}
}
\item variable noun \\
A \textbf{comment} is something that you say which expresses your opinion of something or which gives an explanation of it.
 \textit{
	\begin{itemize}
	\item He made his comments at a news conference in Amsterdam.
	\item I was wondering whether you had any comments about that?
	\item There's been no comment so far from police about the allegations.
	\item The Prime Minister, who is abroad, was not available for comment.
	\end{itemize}
}
\item singular noun \\
If an event or situation is a \textbf{comment}  \textbf{on} something, it reveals something about that thing, usually something bad .
 \textit{
	\begin{itemize}
	\item He argues that family problems are typically a comment on some unresolved issues
in the family.
	\end{itemize}
}
\item  \\
 no comment \textit{
	\begin{itemize}
	\end{itemize}
}
\end{enumerate}

\section*{difficult}
{\large \color{blue}  }
\subsection*{Explain}
\begin{enumerate}
\item adjective \\
Something that is \textbf{difficult} is not easy to do, understand, or deal with.
 \textit{
	\begin{itemize}
	\item Hobart found it difficult to get her first book published.
	\item The lack of childcare provisions made it difficult for single mothers to get jobs.
	\item It was a very difficult decision to make.
	\item We're living in difficult times.
	\end{itemize}
}
\item adjective \\
Someone who is \textbf{difficult}  behaves in an unreasonable and unhelpful  way .
 \textit{
	\begin{itemize}
	\item I had a feeling you were going to be difficult about this.
	\end{itemize}
}
\end{enumerate}

\section*{definition}
{\large \color{blue}  definitions  }
\subsection*{Explain}
\begin{enumerate}
\item countable noun \\
A \textbf{definition} is a statement giving the meaning of a word or expression , especially in a dictionary .
 \textit{
	\begin{itemize}
	\item There is no general agreement on a standard definition of intelligence.
	\item A nice meal with friends is my definition of a good time.
	\end{itemize}
}
\item uncountable noun \\
\textbf{Definition} is the quality of being clear and distinct .
 \textit{
	\begin{itemize}
	\item Give your brows extra definition with eyebrow pencil.
	\item The speakers criticised his new programme for lack of definition.
	\end{itemize}
}
\end{enumerate}

\section*{every}
{\large \color{blue}  }
\subsection*{Explain}
\begin{enumerate}
\item determiner \\
You use \textbf{every} to indicate that you are referring to all the members of a group or all the parts of something and not only some of
them.
 \textbf{Every} is also an adjective .
 \textit{
	\begin{itemize}
	\item Every village has a green, a church, a pub and a manor house.
	\item Record every expenditure you make.
	\item ...Mediterranean fish of every shape and hue.
	\item We need help, every kind of help.
	\item ...recipes for every occasion.
	\item His every utterance will be scrutinized.
	\item He will find his every step more harshly spotlighted than has been the case previously.
	\end{itemize}
}
\item determiner \\
You use \textbf{every} in order to say how often something happens or to indicate that something happens at regular  intervals .
 \textit{
	\begin{itemize}
	\item We were made to attend meetings every day.
	\item A burglary occurs every three minutes in London.
	\item She will need to have the therapy repeated every few months.
	\item They meet here every Friday morning.
	\end{itemize}
}
\item determiner \\
You use \textbf{every} in front of a number when you are saying what proportion of people or things something happens to or applies to.
 \textit{
	\begin{itemize}
	\item Human beings spend about eight out of every 24 hours sleeping.
	\item About one in every 20 people have clinical depression.
	\item He said Africa was suffering badly from deforestation: for every ten trees cut down,
only one was planted.
	\end{itemize}
}
\item determiner \\
You can use \textbf{every} before some nouns, for example ' sign ', ' effort ', ' reason ', and ' intention ' in order to emphasize what you are saying.
 \textit{
	\begin{itemize}
	\item Like most of those on the dance floor, they give every sign of delight.
	\item I think that there is every chance that you will succeed.
	\item Make every effort to visit a person suffering a significant loss, rather than writing
or telephoning.
	\item Every care has been taken in compiling this list.
	\end{itemize}
}
\item adjective \\
If you say that someone's \textbf{every}  whim , wish , or desire  will be satisfied , you are emphasizing that everything they want will happen or be provided .
 \textit{
	\begin{itemize}
	\item Dozens of servants had catered to his every whim.
	\end{itemize}
}
\item  \\
 every now and then etc \textit{
	\begin{itemize}
	\end{itemize}
}
\item  \\
 every other day/every second day etc \textit{
	\begin{itemize}
	\end{itemize}
}
\end{enumerate}

\section*{entity}
{\large \color{blue}  entities  }
\subsection*{Explain}
\begin{enumerate}
\item countable noun \\
An \textbf{entity} is something that exists separately from other things and has a clear  identity of its own.
 \textit{
	\begin{itemize}
	\item ...the earth as a living entity.
	\item They did not see government and society as two separate entities.
	\end{itemize}
}
\end{enumerate}

\section*{financial}
{\large \color{blue}  }
\subsection*{Explain}
\begin{enumerate}
\item adjective \\
\textbf{Financial} means relating to or involving money.
 \textit{
	\begin{itemize}
	\item The company is in financial difficulties.
	\item ...the government's financial advisers.
	\end{itemize}
}
\end{enumerate}

\section*{estate}
{\large \color{blue}  estates  }
\subsection*{Explain}
\begin{enumerate}
\item countable noun \\
An \textbf{estate} is a large area of land in the country which is owned by a person, family, or organization .
 \textit{
	\begin{itemize}
	\item ...a shooting party on Lord Wyville's estate in Yorkshire.
	\end{itemize}
}
\item countable noun \\
People sometimes use \textbf{estate} to refer to a housing estate or an industrial estate.
 \textit{
	\begin{itemize}
	\item He used to live on the estate.
	\end{itemize}
}
\item countable noun \\
Someone's \textbf{estate} is all the money and property that they leave behind them when they die .
 \textit{
	\begin{itemize}
	\item His estate was valued at $150,000.
	\end{itemize}
}
\item countable noun \\
An \textbf{estate} is the same as an estate car .
 \textit{
	\begin{itemize}
	\end{itemize}
}
\end{enumerate}

\section*{firm}
{\large \color{blue}  firms  firming  firmed  firmer  firmest  }
\subsection*{Explain}
\begin{enumerate}
\item countable noun \\
A \textbf{firm} is an organization which sells or produces something or which provides a service
which people pay for.
 \textit{
	\begin{itemize}
	\item The firm's employees were expecting large bonuses.
	\item ...a firm of heating engineers.
	\end{itemize}
}
\item adjective \\
If something is \textbf{firm} , it does not change much in shape when it is pressed but is not completely hard.
 \textit{
	\begin{itemize}
	\item Fruit should be firm and in excellent condition.
	\item Choose a soft, medium or firm mattress to suit their individual needs.
	\end{itemize}
}
\item adjective \\
If something is \textbf{firm} , it does not shake or move when you put weight or pressure on it, because it is strongly made or securely
 fastened .
 \textit{
	\begin{itemize}
	\item If you have to climb up, use a firm platform or a sturdy ladder.
	\end{itemize}
}
\item adjective \\
If someone's grip is \textbf{firm} or if they perform a physical action in a \textbf{firm} way, they do it with quite a lot of force or pressure but also in a controlled way.
 \textit{
	\begin{itemize}
	\item The quick handshake was firm and cool.
	\item He managed to grasp the metal, get a firm grip of it and heave his body upwards.
	\end{itemize}
}
\item adjective \\
If you describe someone as \textbf{firm} , you mean they behave in a way that shows that they are not going to change their mind, or that they are the person who is in control.
 \textit{
	\begin{itemize}
	\item She had to be firm with him. 'I don't want to see you again.'
	\item Perhaps they need the guiding hand of a firm father figure.
	\end{itemize}
}
\item adjective \\
A \textbf{firm}  decision or opinion is definite and unlikely to change.
 \textit{
	\begin{itemize}
	\item He made a firm decision to leave Fort Multry by boat.
	\item It is my firm belief that partnership between police and the public is absolutely
necessary.
	\end{itemize}
}
\item adjective \\
\textbf{Firm}  evidence or information is based on facts and so is likely to be true .
 \textit{
	\begin{itemize}
	\item This man may have killed others but unfortunately we have no firm evidence.
	\item There's unlikely to be firm news about the convoy's progress for some time.
	\end{itemize}
}
\item adjective \\
You use \textbf{firm} to describe control or a basis or position when it is strong and unlikely to be ended or removed.
 \textit{
	\begin{itemize}
	\item A goalkeeping mistake put Dagenham in firm control of the first half.
	\item The company, a household name in the States, has a firm foothold in the British market.
	\end{itemize}
}
\item graded adjective \\
If people are \textbf{firm}  friends , they have been close friends for a long time and their friendship is likely to continue .
 \textit{
	\begin{itemize}
	\item The couple met about two years ago and soon became firm friends.
	\end{itemize}
}
\item adjective \\
If a price, value, or currency is \textbf{firm} , it is not decreasing in value or amount.
 \textit{
	\begin{itemize}
	\item Cotton prices remain firm and demand is strong.
	\item The shares held firm at 280p.
	\item Firm prices and stability will allow both producers and consumers to plan confidently.
	\end{itemize}
}
\item verb \\
If you \textbf{firm}  soil around a plant, you press it so that it is fairly solid rather than loose .
 \textit{
	\begin{itemize}
	\item Firm more soil over the roots and water thoroughly.
	\end{itemize}
}
\item  \\
 to stand firm \textit{
	\begin{itemize}
	\end{itemize}
}
\end{enumerate}

\section*{experiment}
{\large \color{blue}  experiments  experimenting  experimented  }
\subsection*{Explain}
\begin{enumerate}
\item variable noun \\
An \textbf{experiment} is a scientific test which is done in order to discover what happens to something in particular conditions.
 \textit{
	\begin{itemize}
	\item He carried out a series of experiments on the properties of plants.
	\item ...a proposed new law on animal experiments.
	\item This question can be answered only by experiment.
	\end{itemize}
}
\item verb \\
If you \textbf{experiment with} something or \textbf{experiment on} it, you do a scientific test on it in order to discover what happens to it in particular
conditions.
 \textit{
	\begin{itemize}
	\item In 1857 Mendel started experimenting with peas in his monastery garden.
	\item The scientists have already experimented at each other's test sites.
	\end{itemize}
}
\item variable noun \\
An \textbf{experiment} is the trying out of a new idea or method in order to see what it is like and what effects it has.
 \textit{
	\begin{itemize}
	\item As an experiment, we bought Ted a watch.
	\item ...the country's five year experiment in democracy.
	\item She needs plenty of room for experiment in her life.
	\end{itemize}
}
\item verb \\
To \textbf{experiment}  means to try out a new idea or method to see what it is like and what effects it has.
 \textit{
	\begin{itemize}
	\item ...if you like cooking and have the time to experiment.
	\item He believes that students should be encouraged to experiment with bold ideas.
	\end{itemize}
}
\end{enumerate}

\section*{frequent}
{\large \color{blue}  frequents  frequenting  frequented  }
\subsection*{Explain}
\begin{enumerate}
\item adjective \\
If something is \textbf{frequent} , it happens often.
 \textit{
	\begin{itemize}
	\item Bordeaux is on the main Paris-Madrid line so there are frequent trains.
	\item He is a frequent visitor to the house.
	\end{itemize}
}
\item verb \\
If someone \textbf{frequents} a particular place, they regularly go there.
 \textit{
	\begin{itemize}
	\item I hear he frequents the Cajun restaurant in Hampstead.
	\end{itemize}
}
\end{enumerate}

\section*{institute}
{\large \color{blue}  institutes  instituting  instituted  }
\subsection*{Explain}
\begin{enumerate}
\item countable noun \\
An \textbf{institute} is an organization set up to do a particular type of work, especially research or teaching . You can also use \textbf{institute} to refer to the building the organization occupies .
 \textit{
	\begin{itemize}
	\item ...the National Cancer Institute.
	\item ...an elite research institute devoted to computer software.
	\item Directly in front of the institute is Kelly Ingram Park.
	\end{itemize}
}
\item verb \\
If you \textbf{institute} a system, rule, or course of action, you start it.
 \textit{
	\begin{itemize}
	\item We will institute a number of measures to better safeguard the public.
	\item Hormone replacement therapy is very important and should be instituted early.
	\end{itemize}
}
\end{enumerate}

\section*{immune}
{\large \color{blue}  }
\subsection*{Explain}
\begin{enumerate}
\item adjective \\
If you are \textbf{immune}  \textbf{to} a particular disease, you cannot be affected by it.
 \textit{
	\begin{itemize}
	\item This blood test will show whether or not you're immune to the disease.
	\item Most adults are immune to rubella.
	\end{itemize}
}
\item adjective \\
An \textbf{immune}  response or reaction is a reaction by the body's immune system to something harmful that is affecting it.
 \textit{
	\begin{itemize}
	\item It is hoped the procedure will trigger an immune response that will wipe out HIV-infected
cells while leaving non-infected cells unharmed.
	\end{itemize}
}
\item adjective \\
If you are \textbf{immune}  \textbf{to} something that happens or is done, you are not affected by it.
 \textit{
	\begin{itemize}
	\item Whilst Marc did gradually harden himself to the poverty, he did not become immune
to the sight of death.
	\item Football is not immune to economic recession.
	\end{itemize}
}
\item adjective \\
Someone or something that is \textbf{immune}  \textbf{from} a particular process or situation is able to escape it.
 \textit{
	\begin{itemize}
	\item Members of the Bundestag are immune from prosecution for corruption.
	\item No one is immune from scandal.
	\end{itemize}
}
\end{enumerate}

\section*{journal}
{\large \color{blue}  journals  }
\subsection*{Explain}
\begin{enumerate}
\item countable noun \\
A \textbf{journal} is a magazine , especially one that deals with a specialized subject.
 \textit{
	\begin{itemize}
	\item All our results are published in scientific journals.
	\end{itemize}
}
\item countable noun \\
A \textbf{journal} is a daily or weekly newspaper. The word journal is often used in the name of the paper .
 \textit{
	\begin{itemize}
	\item He was a newspaperman for The New York Times and some other journals.
	\item ...The Wall Street Journal.
	\end{itemize}
}
\item countable noun \\
A \textbf{journal} is an account which you write of your daily activities.
 \textit{
	\begin{itemize}
	\item Sara confided to her journal.
	\item On the plane he wrote in his journal.
	\end{itemize}
}
\end{enumerate}

\section*{implicit}
{\large \color{blue}  }
\subsection*{Explain}
\begin{enumerate}
\item adjective \\
Something that is \textbf{implicit} is expressed in an indirect way.
 \textit{
	\begin{itemize}
	\item This is seen as an implicit warning not to continue with military action.
	\item The specific reference to the latter phenomenon was only implicit in the text.
	\end{itemize}
}
\item adjective \\
If a quality or element is \textbf{implicit in} something, it is involved in it or is shown by it.
 \textit{
	\begin{itemize}
	\item ...the delays implicit in formal council meetings.
	\item Try and learn from the lessons implicit in the failure of your marriage.
	\end{itemize}
}
\item adjective \\
If you say that someone has an \textbf{implicit}  belief or faith in something, you mean that they have complete faith in it and no doubts at all.
 \textit{
	\begin{itemize}
	\item He had implicit faith in the noble intentions of the Emperor.
	\end{itemize}
}
\end{enumerate}

\section*{landlord}
{\large \color{blue}  landlords  }
\subsection*{Explain}
\begin{enumerate}
\item countable noun \\
Someone's \textbf{landlord} is the man who allows them to live or work in a building which he owns, in return for rent .
 \textit{
	\begin{itemize}
	\item His landlord doubled the rent.
	\end{itemize}
}
\item countable noun \\
The \textbf{landlord} of a pub is the man who owns or runs it, or the husband of the person who owns or runs it.
 \textit{
	\begin{itemize}
	\item The landlord refused to serve him because he considered him too drunk.
	\end{itemize}
}
\item countable noun \\
A \textbf{landlord} is the man who owns or runs a boarding house or inn .
 \textit{
	\begin{itemize}
	\end{itemize}
}
\end{enumerate}

\section*{inside}
{\large \color{blue}  insides  }
\subsection*{Explain}
\begin{enumerate}
\item preposition \\
Something or someone that is \textbf{inside} a place, container , or object is in it or is surrounded by it.
 \textbf{Inside} is also an adverb .
 \textbf{Inside} is also an adjective .
 \textit{
	\begin{itemize}
	\item Inside the passport was a folded slip of paper.
	\item There is a phone inside the entrance hall.
	\item The couple chatted briefly on the doorstep before going inside.
	\item He ripped open the envelope and read what was inside.
	\item I could hear music coming from inside.
	\item At a table inside, a man and woman were awaiting her.
	\item Inside, a few workers went about their work.
	\item ...four-berth inside cabins with en-suite bathroom and shower.
	\end{itemize}
}
\item countable noun \\
The \textbf{inside} of something is the part or area that its sides surround or contain.
 \textbf{Inside} is also an adjective.
 \textbf{Inside} is also an adverb.
 \textit{
	\begin{itemize}
	\item The doors were locked from the inside.
	\item I painted the inside of the house.
	\item Kiwi fruit can be eaten by cutting off the tops and scooping out the insides with
a teaspoon.
	\item The popular papers all have photo features on their inside pages.
	\item The potato cakes should be crisp outside and meltingly soft inside.
	\end{itemize}
}
\item adverb \\
You can say that someone is \textbf{inside} when they are in prison.
 \textit{
	\begin{itemize}
	\item He's been inside three times.
	\end{itemize}
}
\item adjective \\
On a wide road, the \textbf{inside}  lane is the one which is closest to the edge of the road. Compare  outside .
 \textbf{Inside} is also a noun .
 \textit{
	\begin{itemize}
	\item I was driving up at seventy miles an hour on the inside lane on the motorway.
	\item I overtook Charlie on the inside.
	\end{itemize}
}
\item adjective \\
\textbf{Inside}  information is obtained from someone who is involved in a situation and therefore knows a lot about it.
 \textit{
	\begin{itemize}
	\item Sloane used inside diplomatic information to make himself rich.
	\item It's fascinating to get the inside story so many years after this incident.
	\end{itemize}
}
\item preposition \\
If you are \textbf{inside} an organization, you belong to it.
 \textbf{Inside} is also an adjective.
 \textbf{Inside} is also a noun.
 \textit{
	\begin{itemize}
	\item 75 percent of chief executives come from inside the company.
	\item He hasn't looked very carefully into what was happening inside the ruling party.
	\item ...a recent book about the inside world of pro football.
	\item McAvoy was convinced he could control things from the inside but he lost control.
	\end{itemize}
}
\item plural noun \\
Your \textbf{insides} are your internal organs, especially your stomach.
 \textit{
	\begin{itemize}
	\end{itemize}
}
\item adverb \\
If you say that someone has a feeling  \textbf{inside} , you mean that they have it but have not expressed it.
 \textbf{Inside} is also a preposition .
 \textbf{Inside} is also a noun.
 \textit{
	\begin{itemize}
	\item There is nothing left inside–no words, no anger, no tears.
	\item Do you get a feeling inside when you write something you like?
	\item He felt a great weight of sorrow inside him.
	\item There was a little anger inside me.
	\item What is needed is a change from the inside, a real change in outlook and attitude.
	\end{itemize}
}
\item preposition \\
If you do something \textbf{inside} a particular time, you do it before the end of that time.
 \textit{
	\begin{itemize}
	\item They should have everything working inside an hour.
	\item New Zealand were ahead inside five minutes.
	\end{itemize}
}
\item  \\
 inside out \textit{
	\begin{itemize}
	\end{itemize}
}
\item  \\
 inside out \textit{
	\begin{itemize}
	\end{itemize}
}
\item  \\
 turn inside out \textit{
	\begin{itemize}
	\end{itemize}
}
\end{enumerate}

\section*{leisure}
{\large \color{blue}  }
\subsection*{Explain}
\begin{enumerate}
\item uncountable noun \\
\textbf{Leisure} is the time when you are not working and you can relax and do things that you enjoy .
 \textit{
	\begin{itemize}
	\item ...a relaxing way to fill my leisure time.
	\item ...one of Britain's most popular leisure activities.
	\end{itemize}
}
\item  \\
 at leisure/at sb's leisure \textit{
	\begin{itemize}
	\end{itemize}
}
\end{enumerate}

\section*{masculine}
{\large \color{blue}  }
\subsection*{Explain}
\begin{enumerate}
\item adjective \\
\textbf{Masculine} qualities and things relate to or are considered typical of men, in contrast to women .
 \textit{
	\begin{itemize}
	\item ...masculine characteristics like a husky voice and facial hair.
	\item ...masculine pride.
	\end{itemize}
}
\item adjective \\
If you say that someone or something is \textbf{masculine} , you mean that they have qualities such as strength or confidence which are considered typical of men.
 \textit{
	\begin{itemize}
	\item ...her aggressive, masculine image.
	\item The Duke's study was very masculine, with deep red wall-covering and dark oak shelving.
	\end{itemize}
}
\item adjective \\
In some languages, a \textbf{masculine} noun, pronoun , or adjective has a different form from a feminine or neuter one, or behaves in a different way .
 \textit{
	\begin{itemize}
	\end{itemize}
}
\end{enumerate}

\section*{mud}
{\large \color{blue}  }
\subsection*{Explain}
\begin{enumerate}
\item uncountable noun \\
\textbf{Mud} is a sticky  mixture of earth and water.
 \textit{
	\begin{itemize}
	\item His uniform was crumpled, untidy, splashed with mud.
	\item Their lorry got stuck in the mud.
	\end{itemize}
}
\end{enumerate}

\section*{medical}
{\large \color{blue}  medicals  }
\subsection*{Explain}
\begin{enumerate}
\item adjective \\
\textbf{Medical} means relating to illness and injuries and to their treatment or prevention .
 \textit{
	\begin{itemize}
	\item Several police officers received medical treatment for cuts and bruises.
	\item ...the medical profession.
	\end{itemize}
}
\item countable noun \\
A \textbf{medical} is a thorough examination of your body by a doctor , for example before you start a new job .
 \textit{
	\begin{itemize}
	\end{itemize}
}
\end{enumerate}

\section*{officer}
{\large \color{blue}  officers  }
\subsection*{Explain}
\begin{enumerate}
\item countable noun \\
In the armed forces, an \textbf{officer} is a person in a position of authority.
 \textit{
	\begin{itemize}
	\item ...a retired British army officer.
	\item He was an officer in the Cadet Corps.
	\end{itemize}
}
\item countable noun \\
An \textbf{officer} is a person who has a responsible position in an organization, especially a government organization.
 \textit{
	\begin{itemize}
	\item ...a local authority education officer.
	\end{itemize}
}
\item countable noun \\
Members of the police force can be referred to as \textbf{officers} .
 \textit{
	\begin{itemize}
	\item ...senior officers in the West Midlands police force.
	\item Thank you, Officer.
	\end{itemize}
}
\end{enumerate}

\section*{monthly}
{\large \color{blue}  monthlies  }
\subsection*{Explain}
\begin{enumerate}
\item adjective \\
A \textbf{monthly}  event or publication  happens or appears every month.
 \textbf{Monthly} is also an adverb .
 \textit{
	\begin{itemize}
	\item Many people are now having trouble making their monthly house payments.
	\item Kidscape runs monthly workshops for teachers.
	\item ...Young Guard, a monthly journal founded in 1922.
	\item In some areas the property price can rise monthly.
	\end{itemize}
}
\item countable noun \\
You can refer to a publication that is published monthly as a \textbf{monthly} .
 \textit{
	\begin{itemize}
	\item ...Scallywag, a London satirical monthly.
	\item ...the Nairobi Law Monthly.
	\end{itemize}
}
\item adjective \\
\textbf{Monthly}  quantities or rates relate to a period of one month.
 \textit{
	\begin{itemize}
	\item The monthly rent for a two-bedroom flat would be £953.33.
	\item Monthly interest costs vary.
	\end{itemize}
}
\end{enumerate}

\section*{official}
{\large \color{blue}  officials  }
\subsection*{Explain}
\begin{enumerate}
\item adjective \\
\textbf{Official} means approved by the government or by someone in authority.
 \textit{
	\begin{itemize}
	\item According to the official figures, over one thousand people died during the revolution.
	\item An official announcement is expected in the next few days.
	\item A report in the official police newspaper gave no reason for the move.
	\end{itemize}
}
\item adjective \\
\textbf{Official}  activities are carried out by a person in authority as part of their job .
 \textit{
	\begin{itemize}
	\item The President is in Brazil for an official two-day visit.
	\end{itemize}
}
\item adjective \\
\textbf{Official} things are used by a person in authority as part of their job.
 \textit{
	\begin{itemize}
	\item ...the official residence of the Head of State.
	\end{itemize}
}
\item adjective \\
If you describe someone's explanation or reason for something as the \textbf{official} explanation, you are suggesting that it is probably not true , but is used because the real explanation is embarrassing .
 \textit{
	\begin{itemize}
	\item The official explanation for the cancellation is that there are no premises available.
	\item The official reason given for the President's absence was sickness.
	\end{itemize}
}
\item countable noun \\
An \textbf{official} is a person who holds a position of authority in an organization.
 \textit{
	\begin{itemize}
	\item A senior U.N. official hopes to visit Baghdad this month.
	\end{itemize}
}
\item countable noun \\
An \textbf{official} at a sports  event is a referee , umpire , or other person who checks that the players  follow the rules .
 \textit{
	\begin{itemize}
	\end{itemize}
}
\end{enumerate}

\section*{more}
{\large \color{blue}  }
\subsection*{Explain}
\begin{enumerate}
\item determiner \\
You use \textbf{more} to indicate that there is a greater amount of something than before or than average , or than something else. You can use 'a little', 'a lot ', 'a bit ', ' far ', and 'much' in front of \textbf{more} .
 \textbf{More} is also a pronoun.
 \textbf{More} is also a quantifier .
 \textit{
	\begin{itemize}
	\item More and more people are surviving heart attacks.
	\item He spent more time perfecting his dance moves instead of gym work.
	\item ...teaching more children foreign languages other than English.
	\item It's a good idea to give adolescents a little more information than they ask for.
	\item As the level of work increased from light to heavy, workers ate more.
	\item He had four hundred dollars in his pocket. Billy had more.
	\item Employees may have to take on more of their own medical costs.
	\item The urgent need to bolster the reforms is beginning to demand more of his attention.
	\end{itemize}
}
\item phrase \\
You use \textbf{more than} before a number or amount to say that the actual number or amount is even greater.
 \textit{
	\begin{itemize}
	\item The Afghan authorities say the airport had been closed for more than a year.
	\item ...classy leather and silk jackets at more than £250.
	\item ...a survey of more than 1,500 schools.
	\end{itemize}
}
\item adverb \\
You use \textbf{more} to indicate that something or someone has a greater amount of a quality than they
used to or than is average or usual .
 \textit{
	\begin{itemize}
	\item Prison conditions have become more brutal.
	\item We can satisfy our basic wants more easily than in the past.
	\end{itemize}
}
\item adverb \\
If you say that something is \textbf{more} one thing \textbf{than} another, you mean that it is like the first thing rather than the second .
 \textit{
	\begin{itemize}
	\item The exhibition at Boston's Museum of Fine Arts is more a production than it is a
museum display.
	\item He's more like a film star than a life-guard, really.
	\item She looked more sad than in pain.
	\item Sue screamed, not loudly, more in surprise than terror.
	\item She's more of a social animal than me.
	\end{itemize}
}
\item adverb \\
If you do something \textbf{more} than before or \textbf{more} than someone else, you do it to a greater extent or more often.
 \textit{
	\begin{itemize}
	\item When we are tired, tense, depressed or unwell, we feel pain much more.
	\item What impressed me more was that she knew Tennessee Williams.
	\end{itemize}
}
\item adverb \\
You can use \textbf{more} to indicate that something continues to happen for a further period of time.
 \textit{
	\begin{itemize}
	\item Things might have been different if I'd talked a bit more.
	\end{itemize}
}
\item adverb \\
You use \textbf{more} to indicate that something is repeated . For example , if you do something 'once more', you do it again once.
 \textit{
	\begin{itemize}
	\item This train would stop twice more in the suburbs before rolling southeast toward Munich.
	\item The breathing exercises should be repeated several times more.
	\end{itemize}
}
\item determiner \\
You use \textbf{more} to refer to an additional thing or amount. You can use 'a little', 'a lot', 'a bit', 'far'
and 'much' in front of \textbf{more} .
 \textbf{More} is also an adjective.
 \textbf{More} is also a pronoun.
 \textit{
	\begin{itemize}
	\item They needed more time to consider whether to hold an inquiry.
	\item We stayed in Danville two more days.
	\item Are you sure you wouldn't like some more coffee?
	\item Oxfam has appealed to western nations to do more to help the refugees.
	\item 'None of them are very nice folks.'—'Tell me more.'
	\end{itemize}
}
\item pronoun \\
You can use \textbf{more} in expressions like 'no more, no less' and 'neither more nor less' to indicate that what you are
 saying is exactly  true or correct .
 \textit{
	\begin{itemize}
	\item I told him the truth. No more, no less.
	\item I'm sixty-two. I feel sixty-two, neither more nor less.
	\end{itemize}
}
\item adverb \\
You use \textbf{more} in conversations when you want to draw someone's attention to something interesting or important that you are about to say.
 \textit{
	\begin{itemize}
	\item The way we dress reflects who we are and, more interestingly, who we wish we could
be.
	\item More seriously for him, there are members who say he is wrong on this issue.
	\end{itemize}
}
\item  \\
 more and more \textit{
	\begin{itemize}
	\end{itemize}
}
\item  \\
 more or less \textit{
	\begin{itemize}
	\end{itemize}
}
\item  \\
 more than \textit{
	\begin{itemize}
	\end{itemize}
}
\item  \\
 more than \textit{
	\begin{itemize}
	\end{itemize}
}
\item  \\
 no more than/not more than \textit{
	\begin{itemize}
	\end{itemize}
}
\item  \\
 nothing more than \textit{
	\begin{itemize}
	\end{itemize}
}
\item  \\
 what is more \textit{
	\begin{itemize}
	\end{itemize}
}
\end{enumerate}

\section*{poem}
{\large \color{blue}  poems  }
\subsection*{Explain}
\begin{enumerate}
\item countable noun \\
A \textbf{poem} is a piece of writing in which the words are chosen for their beauty and sound and are carefully arranged , often in short lines which rhyme.
 \textit{
	\begin{itemize}
	\end{itemize}
}
\end{enumerate}

\section*{poet}
{\large \color{blue}  poets  }
\subsection*{Explain}
\begin{enumerate}
\item countable noun \\
A \textbf{poet} is a person who writes poems .
 \textit{
	\begin{itemize}
	\item He was a painter and poet.
	\end{itemize}
}
\end{enumerate}

\section*{negative}
{\large \color{blue}  negatives  }
\subsection*{Explain}
\begin{enumerate}
\item adjective \\
A fact, situation, or experience that is \textbf{negative} is unpleasant, depressing , or harmful .
 \textit{
	\begin{itemize}
	\item The news from overseas is overwhelmingly negative.
	\item All this had an extremely negative effect on the criminal justice system.
	\end{itemize}
}
\item adjective \\
If someone is \textbf{negative} or has a \textbf{negative}  attitude , they consider only the bad aspects of a situation, rather than the good ones.
 \textit{
	\begin{itemize}
	\item When asked for your views about your current job, on no account must you be negative
about it.
	\item Why does the media present such a negative view of this splendid city?
	\end{itemize}
}
\item adjective \\
A \textbf{negative}  reply or decision indicates the answer 'no'.
 \textit{
	\begin{itemize}
	\item Dr Velayati gave a vague but negative response.
	\item Upon a negative decision, the applicant loses the protection offered by Belgian law.
	\item The Tory response to that was negative.
	\end{itemize}
}
\item countable noun \\
A \textbf{negative} is a word, expression, or gesture that means 'no' or 'not'.
 \textit{
	\begin{itemize}
	\item In the past we have heard only negatives when it came to following a healthy diet.
	\end{itemize}
}
\item adjective \\
In grammar , a \textbf{negative}  clause contains a word such as 'not', ' never ', or 'no-one'.
 \textit{
	\begin{itemize}
	\end{itemize}
}
\item adjective \\
If a medical test or scientific test is \textbf{negative} , it shows no evidence of the medical condition or substance that you are looking for.
 \textit{
	\begin{itemize}
	\item So far 57 have taken the test and all have been negative.
	\item ...negative test results.
	\end{itemize}
}
\item countable noun \\
In photography, a \textbf{negative} is an image that shows dark areas as light and light areas as dark. Negatives are
made from a camera film, and are used to print photographs.
 \textit{
	\begin{itemize}
	\end{itemize}
}
\item adjective \\
A \textbf{negative} charge or current has the same electrical charge as an electron.
 \textit{
	\begin{itemize}
	\item Stimulate the site of greatest pain with a small negative current.
	\end{itemize}
}
\item adjective \\
A \textbf{negative} number, quantity, or measurement is less than zero.
 \textit{
	\begin{itemize}
	\item The weakest students can end up with a negative score.
	\end{itemize}
}
\item  \\
 in the negative \textit{
	\begin{itemize}
	\end{itemize}
}
\item  \\
 in the negative \textit{
	\begin{itemize}
	\end{itemize}
}
\end{enumerate}

\section*{poetry}
{\large \color{blue}  }
\subsection*{Explain}
\begin{enumerate}
\item uncountable noun \\
Poems , considered as a form of literature, are referred to as \textbf{poetry} .
 \textit{
	\begin{itemize}
	\item ...Russian poetry.
	\item His first encounter with poetry had been the Tennyson given him by his father.
	\item Since when have you been interested in poetry?
	\end{itemize}
}
\item uncountable noun \\
You can describe something very beautiful as \textbf{poetry} .
 \textit{
	\begin{itemize}
	\item His music is purer poetry than a poem in words.
	\end{itemize}
}
\end{enumerate}

\section*{neither}
{\large \color{blue}  }
\subsection*{Explain}
\begin{enumerate}
\item conjunction \\
You use \textbf{neither} in front of the first of two or more words or expressions when you are linking two or more things which are not true or do not happen . The other thing is introduced by 'nor'.
 \textit{
	\begin{itemize}
	\item Professor Hisamatsu spoke neither English nor German.
	\item The play is neither as funny nor as disturbing as Tabori thinks it is.
	\end{itemize}
}
\item determiner \\
You use \textbf{neither} to refer to each of two things or people, when you are making a negative  statement that includes both of them.
 \textbf{Neither} is also a quantifier .
 \textbf{Neither} is also a pronoun.
 \textit{
	\begin{itemize}
	\item At first, neither man could speak.
	\item Neither of us felt like going out.
	\item They both smiled; neither seemed likely to be aware of my absence for long.
	\end{itemize}
}
\item conjunction \\
If you say that one person or thing does not do something and \textbf{neither} does another, what you say is true of all the people or things that you are mentioning .
 \textit{
	\begin{itemize}
	\item I never learned to swim and neither did they.
	\item I don't have all the answers and neither do you.
	\end{itemize}
}
\item conjunction \\
You use \textbf{neither} after a negative statement to emphasize that you are introducing another negative statement.
 \textit{
	\begin{itemize}
	\item I can't ever recall Dad hugging me. Neither did I sit on his knee.
	\end{itemize}
}
\item  \\
 neither here nor there \textit{
	\begin{itemize}
	\end{itemize}
}
\end{enumerate}

\section*{practitioner}
{\large \color{blue}  practitioners  }
\subsection*{Explain}
\begin{enumerate}
\item countable noun \\
Doctors are sometimes  referred to as \textbf{practitioners} or \textbf{medical practitioners} .
 \textit{
	\begin{itemize}
	\end{itemize}
}
\end{enumerate}

\section*{noisy}
{\large \color{blue}  noisier  noisiest  }
\subsection*{Explain}
\begin{enumerate}
\item adjective \\
A \textbf{noisy} person or thing makes a lot of loud or unpleasant noise.
 \textit{
	\begin{itemize}
	\item ...my noisy old typewriter.
	\item His daughter was very active and noisy in the mornings.
	\end{itemize}
}
\item adjective \\
A \textbf{noisy} place is full of a lot of loud or unpleasant noise.
 \textit{
	\begin{itemize}
	\item It's a noisy place with film clips showing constantly on one of the cafe's giant
screens.
	\item The baggage hall was crowded and noisy.
	\end{itemize}
}
\item adjective \\
If you describe someone as \textbf{noisy} , you are critical of them for trying to attract  attention to their views by frequently and forcefully discussing them.
 \textit{
	\begin{itemize}
	\item It might, at last, silence the small but noisy intellectual clique.
	\end{itemize}
}
\end{enumerate}

\section*{pupil}
{\large \color{blue}  pupils  }
\subsection*{Explain}
\begin{enumerate}
\item countable noun \\
The \textbf{pupils} of a school are the children who go to it.
 \textit{
	\begin{itemize}
	\item Many secondary schools in Wales have over 1,000 pupils.
	\item Eleanor was a reluctant, anxious pupil.
	\end{itemize}
}
\item countable noun \\
A \textbf{pupil} of a painter , musician , or other expert is someone who studies under that expert and learns his or her skills .
 \textit{
	\begin{itemize}
	\item ...the only drawing firmly attributed to Cesare Magni (1511-1534), a pupil of Leonardo
da Vinci.
	\end{itemize}
}
\item countable noun \\
The \textbf{pupils} of your eyes are the small, round , black holes in the centre of them.
 \textit{
	\begin{itemize}
	\end{itemize}
}
\end{enumerate}

\section*{religion}
{\large \color{blue}  religions  }
\subsection*{Explain}
\begin{enumerate}
\item uncountable noun \\
\textbf{Religion} is belief in a god or gods and the activities that are connected with this belief, such as praying or worshipping in a building such as a church or temple .
 \textit{
	\begin{itemize}
	\item ...his understanding of Indian philosophy and religion.
	\item Do avoid potentially contentious subjects such as religion, sex or politics.
	\end{itemize}
}
\item countable noun \\
A \textbf{religion} is a particular system of belief in a god or gods and the activities that are connected
with this system.
 \textit{
	\begin{itemize}
	\item ...the Christian religion.
	\end{itemize}
}
\item  \\
 get religion \textit{
	\begin{itemize}
	\end{itemize}
}
\end{enumerate}

\section*{polar}
{\large \color{blue}  }
\subsection*{Explain}
\begin{enumerate}
\item adjective \\
\textbf{Polar}  means near the North and South Poles.
 \textit{
	\begin{itemize}
	\item ...the rigours of life in the polar regions.
	\item Warmth melted some of the polar ice.
	\item ...polar explorers.
	\end{itemize}
}
\item adjective \\
\textbf{Polar} is used to describe things which are completely opposite in character, quality, or type.
 \textit{
	\begin{itemize}
	\item In many ways, Brett and Bernard are polar opposites.
	\item ...economists at polar ends of the politico-economic spectrum.
	\end{itemize}
}
\end{enumerate}

\section*{private}
{\large \color{blue}  privates  }
\subsection*{Explain}
\begin{enumerate}
\item adjective \\
\textbf{Private}  industries and services are owned or controlled by an individual person or a commercial  company , rather than by the state or an official organization.
 \textit{
	\begin{itemize}
	\item ...a joint venture with private industry.
	\item Bupa runs private hospitals in Britain.
	\item Brazil says its constitution forbids the private ownership of energy assets.
	\end{itemize}
}
\item adjective \\
\textbf{Private} individuals are acting only for themselves, and are not representing any group, company, or organization.
 \textit{
	\begin{itemize}
	\item ...the law's insistence that private citizens are not permitted to have weapons.
	\item The King was on a private visit to enable him to pray at the tombs of his ancestors.
	\item The family tried to bring a private prosecution against him for assault.
	\end{itemize}
}
\item adjective \\
Your \textbf{private} things belong only to you, or may only be used by you.
 \textit{
	\begin{itemize}
	\item The landowners have had to sell their private aircraft.
	\item They want more State control over private property.
	\item There are 76 individually furnished bedrooms, all with private bathrooms.
	\item He later travelled with the Prince of Wales as his private secretary.
	\end{itemize}
}
\item adjective \\
\textbf{Private} places or gatherings may be attended only by a particular group of people, rather than by the general public.
 \textit{
	\begin{itemize}
	\item 673 private golf clubs took part in a recent study.
	\item The door is marked 'Private'.
	\item He was buried in a private ceremony in Liverpool.
	\end{itemize}
}
\item adjective \\
\textbf{Private}  meetings , discussions , and other activities  involve only a small number of people, and very little information about them is given to other people.
 \textit{
	\begin{itemize}
	\item Don't bug private conversations, and don't buy papers that reprint them.
	\end{itemize}
}
\item adjective \\
Your \textbf{private}  \textbf{life} is that part of your life that is concerned with your personal  relationships and activities, rather than with your work or business .
 \textit{
	\begin{itemize}
	\item I've always kept my private and professional life separate.
	\item My private affairs are no one's business but my own.
	\end{itemize}
}
\item adjective \\
Your \textbf{private}  thoughts or feelings are ones that you do not talk about to other people.
 \textit{
	\begin{itemize}
	\item We all felt as if we were intruding on his private grief.
	\item It's just that it's something very private, and I simply can't talk about it.
	\end{itemize}
}
\item adjective \\
You can use \textbf{private} to describe  situations or activities that are understood only by the people involved in them, and not by anyone else.
 \textit{
	\begin{itemize}
	\item Chinese waiters stood in a cluster, sharing a private joke.
	\item As many as 40 per cent of twins have a private language that excludes the rest of
the family.
	\end{itemize}
}
\item adjective \\
If you describe a place as \textbf{private} , or as somewhere where you can be \textbf{private} , you mean that it is a quiet place and you can be alone there without being disturbed .
 \textit{
	\begin{itemize}
	\item It was the only reasonably private place they could find.
	\item ...a very attractive country house set within a uniquely beautiful and private position.
	\item We were alone, completely private, with not even Angela present.
	\end{itemize}
}
\item adjective \\
If you describe someone as a \textbf{private} person, you mean that they are very quiet by nature and do not reveal their thoughts and feelings to other people.
 \textit{
	\begin{itemize}
	\item She has always been a rather private person.
	\item Gould was an intensely private individual.
	\end{itemize}
}
\item adjective \\
You can use \textbf{private} to describe lessons that are not part of ordinary  school activity, and which are given by a teacher to an individual pupil or a small group, usually in return for payment .
 \textit{
	\begin{itemize}
	\item Martial arts: Private lessons: £8 per hour.
	\item ...Donald Tovey, who took her as his private pupil for the piano.
	\end{itemize}
}
\item countable noun \\
A \textbf{private} is a soldier of the lowest rank in an army or the marines.
 \textit{
	\begin{itemize}
	\item One gunner in each battery was an NCO and the rest were privates.
	\item ...Private Martin Ferguson.
	\end{itemize}
}
\item plural noun \\
Your \textbf{privates} are your genitals.
 \textit{
	\begin{itemize}
	\item You should wash your feet and your privates every day.
	\end{itemize}
}
\item  \\
 in private \textit{
	\begin{itemize}
	\end{itemize}
}
\end{enumerate}

\section*{scholar}
{\large \color{blue}  scholars  }
\subsection*{Explain}
\begin{enumerate}
\item countable noun \\
A \textbf{scholar} is a person who studies an academic  subject and knows a lot about it.
 \textit{
	\begin{itemize}
	\item The library attracts thousands of scholars and researchers.
	\item ...an influential Islamic scholar.
	\end{itemize}
}
\item countable noun \\
You can use the word \textbf{scholar} to refer to someone who learns things at school in a particular way. For example , if someone is a good \textbf{scholar} , they are good at learning things.
 \textit{
	\begin{itemize}
	\item She could be a good scholar if she didn't let her mind wander so much.
	\end{itemize}
}
\item countable noun \\
A \textbf{scholar} is a student who has obtained a scholarship .
 \textit{
	\begin{itemize}
	\item He came to Oxford as a Rhodes scholar and studied law.
	\end{itemize}
}
\end{enumerate}

\section*{quarterly}
{\large \color{blue}  quarterlies  }
\subsection*{Explain}
\begin{enumerate}
\item adjective \\
A \textbf{quarterly}  event  happens  four times a year , at intervals of three months.
 \textbf{Quarterly} is also an adverb .
 \textit{
	\begin{itemize}
	\item ...the latest Bank of Japan quarterly survey of 5,000 companies.
	\item It makes no difference whether dividends are paid quarterly or annually.
	\end{itemize}
}
\item countable noun \\
A \textbf{quarterly} is a magazine that is published four times a year, at intervals of three months.
 \textit{
	\begin{itemize}
	\item The quarterly had been a forum for sound academic debate.
	\item ...'Foreign Policy', a quarterly journal published in Paris.
	\end{itemize}
}
\end{enumerate}

\section*{school}
{\large \color{blue}  schools  schooling  schooled  }
\subsection*{Explain}
\begin{enumerate}
\item variable noun \\
A \textbf{school} is a place where children are educated. You usually refer to this place as \textbf{school} when you are talking about the time that children spend there and the activities that they do there.
 \textit{
	\begin{itemize}
	\item ...a boy who was in my class at school.
	\item Even the good students say homework is what they most dislike about school.
	\item I took the kids for a picnic in the park after school.
	\item ...a school built in the Sixties.
	\item He favors extending the school day and school year.
	\item ...two boys wearing school uniform.
	\end{itemize}
}
\item countable noun \\
A \textbf{school} is the pupils or staff at a school.
 \textit{
	\begin{itemize}
	\item Deirdre, the whole school's going to hate you.
	\item ...a children's writing competition open to schools or individuals.
	\end{itemize}
}
\item countable noun \\
A privately-run place where a particular skill or subject is taught can be referred to as a \textbf{school} .
 \textit{
	\begin{itemize}
	\item ...a riding school and equestrian centre near Chepstow.
	\item ...the Kingsley School of English.
	\end{itemize}
}
\item variable noun \\
A university , college , or university department specializing in a particular type of subject can be referred
to as a \textbf{school} .
 \textit{
	\begin{itemize}
	\item ...a lecturer in the School of Veterinary Medicine.
	\item Stella, 21, is at art school training to be a fashion designer.
	\end{itemize}
}
\item uncountable noun \\
\textbf{School} is used to refer to university or college.
 \textit{
	\begin{itemize}
	\item Moving rapidly through school, he graduated Phi Beta Kappa from the University of
Kentucky at age 18.
	\end{itemize}
}
\item countable noun \\
A particular \textbf{school}  \textbf{of} writers, artists, or thinkers is a group of them whose work, opinions , or theories are similar.
 \textit{
	\begin{itemize}
	\item ...the Chicago school of economists.
	\item O'Keeffe was influenced by various painters but she was never a member of any school.
	\end{itemize}
}
\item countable noun \\
A \textbf{school of} fish or dolphins is a large group of them moving through water together.
 \textit{
	\begin{itemize}
	\end{itemize}
}
\item verb \\
If you \textbf{school} someone \textbf{in} something, you train or educate them to have a certain skill, type of behaviour , or way of thinking .
 \textit{
	\begin{itemize}
	\item Many mothers schooled their daughters in the myth of female inferiority.
	\item He is schooled to spot trouble.
	\end{itemize}
}
\item verb \\
To \textbf{school} a child means to educate him or her.
 \textit{
	\begin{itemize}
	\item She's been schooling her kids herself.
	\end{itemize}
}
\item verb \\
If you \textbf{school} a horse, you train it so that it can be ridden in races or competitions .
 \textit{
	\begin{itemize}
	\item She bought him as a £1,000 colt of six months and schooled him.
	\end{itemize}
}
\item  \\
 of the old school \textit{
	\begin{itemize}
	\end{itemize}
}
\end{enumerate}

\section*{resolute}
{\large \color{blue}  }
\subsection*{Explain}
\begin{enumerate}
\item adjective \\
If you describe someone as \textbf{resolute} , you approve of them because they are very determined not to change their mind or not to give up a course of action.
 \textit{
	\begin{itemize}
	\item Voters perceive him as a decisive and resolute international leader.
	\item He described the situation as very dangerous and called for resolute action.
	\end{itemize}
}
\end{enumerate}

\section*{semester}
{\large \color{blue}  semesters  }
\subsection*{Explain}
\begin{enumerate}
\item countable noun \\
In colleges and universities in some countries, a \textbf{semester} is one of the two main periods into which the year is divided.
 \textit{
	\begin{itemize}
	\end{itemize}
}
\end{enumerate}

\section*{resultant}
{\large \color{blue}  }
\subsection*{Explain}
\begin{enumerate}
\item adjective \\
\textbf{Resultant} means caused by the event just mentioned .
 \textit{
	\begin{itemize}
	\item At least a quarter of a million people have died in the fighting and the resultant
famines.
	\end{itemize}
}
\end{enumerate}

\section*{situated}
{\large \color{blue}  }
\subsection*{Explain}
\begin{enumerate}
\item adjective \\
If something is \textbf{situated} in a particular place or position, it is in that place or position.
 \textit{
	\begin{itemize}
	\item His hotel is situated in one of the loveliest places on the Loire.
	\item The pain was situated above and around the eyes.
	\item The new store is better situated to attract customers.
	\end{itemize}
}
\end{enumerate}

\section*{shoulder}
{\large \color{blue}  shoulders  shouldering  shouldered  }
\subsection*{Explain}
\begin{enumerate}
\item countable noun \\
Your \textbf{shoulders} are between your neck and the tops of your arms.
 \textit{
	\begin{itemize}
	\item She led him to an armchair, with her arm round his shoulder.
	\item He glanced over his shoulder and saw me watching him.
	\end{itemize}
}
\item countable noun \\
The \textbf{shoulders} of a piece of clothing are the parts that cover your shoulders.
 \textit{
	\begin{itemize}
	\item ...extravagant fashions with padded shoulders.
	\end{itemize}
}
\item plural noun \\
When you talk about someone's problems or responsibilities, you can say that they carry them \textbf{on} their \textbf{shoulders} .
 \textit{
	\begin{itemize}
	\item No one suspected the anguish he carried on his shoulders.
	\item I hope he understands the burden that's on his shoulders.
	\end{itemize}
}
\item verb \\
If you \textbf{shoulder} the responsibility or the blame for something, you accept it.
 \textit{
	\begin{itemize}
	\item He has had to shoulder the responsibility of his father's mistakes.
	\item Some of the blame for the disastrous night must be shouldered by the promoters.
	\end{itemize}
}
\item verb \\
If you \textbf{shoulder} something heavy , you put it across one of your shoulders so that you can carry it more easily .
 \textit{
	\begin{itemize}
	\item The rest of the group shouldered their bags, gritted their teeth and set off.
	\item He shouldered his bike and walked across the finish line.
	\end{itemize}
}
\item verb \\
If you \textbf{shoulder} someone \textbf{aside} or if you \textbf{shoulder} your \textbf{way}  somewhere , you push past people roughly using your shoulder.
 \textit{
	\begin{itemize}
	\item The policemen rushed past him, shouldering him aside.
	\item She could do nothing to stop him as he shouldered his way into the house.
	\item He shouldered past Harlech and opened the door.
	\end{itemize}
}
\item variable noun \\
A \textbf{shoulder} is a joint of meat from the upper part of the front  leg of an animal.
 \textit{
	\begin{itemize}
	\item ...shoulder of lamb.
	\end{itemize}
}
\item  \\
 a shoulder to cry on \textit{
	\begin{itemize}
	\end{itemize}
}
\item  \\
 head and shoulders \textit{
	\begin{itemize}
	\end{itemize}
}
\item  \\
 look over one's shoulder \textit{
	\begin{itemize}
	\end{itemize}
}
\item  \\
 shoulder to shoulder \textit{
	\begin{itemize}
	\end{itemize}
}
\item  \\
 shoulder to shoulder \textit{
	\begin{itemize}
	\end{itemize}
}
\end{enumerate}

\section*{sturdy}
{\large \color{blue}  sturdier  sturdiest  }
\subsection*{Explain}
\begin{enumerate}
\item adjective \\
Someone or something that is \textbf{sturdy}  looks strong and is unlikely to be easily  injured or damaged .
 \textit{
	\begin{itemize}
	\item She was a short, sturdy woman in her early sixties.
	\item The camera was mounted on a sturdy tripod.
	\end{itemize}
}
\end{enumerate}

\section*{soil}
{\large \color{blue}  soils  soiling  soiled  }
\subsection*{Explain}
\begin{enumerate}
\item variable noun \\
\textbf{Soil} is the substance on the surface of the earth in which plants grow .
 \textit{
	\begin{itemize}
	\item We have the most fertile soil in Europe.
	\item ...regions with sandy soils.
	\end{itemize}
}
\item uncountable noun \\
You can use \textbf{soil} in expressions  like  \textbf{British soil} to refer to a country's territory .
 \textit{
	\begin{itemize}
	\item The issue of foreign troops on Turkish soil is a sensitive one.
	\end{itemize}
}
\item verb \\
If you \textbf{soil} something, you make it dirty.
 \textit{
	\begin{itemize}
	\item Young people don't want to do things that soil their hands.
	\item He raised his eyes slightly as though her words might somehow soil him.
	\end{itemize}
}
\end{enumerate}

\section*{thorough}
{\large \color{blue}  }
\subsection*{Explain}
\begin{enumerate}
\item adjective \\
A \textbf{thorough} action or activity is one that is done very carefully and in a detailed way so that nothing is forgotten .
 \textit{
	\begin{itemize}
	\item We are making a thorough investigation.
	\item This very thorough survey goes back to 1784.
	\item How thorough is the assessment?
	\end{itemize}
}
\item adjective \\
Someone who is \textbf{thorough} is always very careful in their work, so that nothing is forgotten.
 \textit{
	\begin{itemize}
	\item Martin would be a good judge, I thought. He was calm and thorough.
	\item The men were expert, thorough and careful.
	\end{itemize}
}
\item adjective \\
\textbf{Thorough} is used to emphasize the great  degree or extent of something.
 \textit{
	\begin{itemize}
	\item I was a thorough little academic snob.
	\item We regard the band as a thorough shambles.
	\end{itemize}
}
\end{enumerate}

\section*{swamp}
{\large \color{blue}  swamps  swamping  swamped  }
\subsection*{Explain}
\begin{enumerate}
\item variable noun \\
A \textbf{swamp} is an area of very wet land with wild plants growing in it.
 \textit{
	\begin{itemize}
	\end{itemize}
}
\item verb \\
If something \textbf{swamps} a place or object, it fills it with water.
 \textit{
	\begin{itemize}
	\item A rogue wave swamped the boat.
	\item The Ventura river burst its banks, swamping a mobile home park.
	\end{itemize}
}
\item verb \\
If you \textbf{are swamped} by things or people, you have more of them than you can deal with.
 \textit{
	\begin{itemize}
	\item He is swamped with work.
	\item The railway station was swamped with thousands of families trying to flee the city.
	\end{itemize}
}
\end{enumerate}

\section*{tough}
{\large \color{blue}  tougher  toughest  toughs  toughing  toughed  }
\subsection*{Explain}
\begin{enumerate}
\item adjective \\
A \textbf{tough} person is strong and determined , and can tolerate difficulty or suffering .
 \textit{
	\begin{itemize}
	\item He built up a reputation as a tough businessman.
	\item She is tough and ambitious.
	\end{itemize}
}
\item adjective \\
If you describe someone as \textbf{tough} , you mean that they are rough and violent .
 A \textbf{tough} is a tough person.
 \textit{
	\begin{itemize}
	\item He had shot three people dead, earning himself a reputation as a tough guy.
	\item Three burly toughs elbowed their way to the front.
	\end{itemize}
}
\item adjective \\
A \textbf{tough} place or area is considered to have a lot of crime and violence .
 \textit{
	\begin{itemize}
	\item She doesn't seem cut out for this tough neighbourhood.
	\item Arthur grew up in a tough city.
	\end{itemize}
}
\item adjective \\
A \textbf{tough} way of life or period of time is difficult or full of suffering.
 \textit{
	\begin{itemize}
	\item She had a pretty tough childhood.
	\item It's been a tough day.
	\item He was having a really tough time at work.
	\end{itemize}
}
\item adjective \\
A \textbf{tough}  task or problem is difficult to do or solve .
 \textit{
	\begin{itemize}
	\item It was a very tough decision but we feel we made the right one.
	\item Whoever wins the election is going to have a tough job getting the economy back on
its feet.
	\item It may be tough to raise cash.
	\item Change is often tough to deal with.
	\end{itemize}
}
\item adjective \\
\textbf{Tough}  policies or actions are strict and firm.
 \textit{
	\begin{itemize}
	\item He is known for taking a tough line on security.
	\item He announced tough measures to limit the money supply.
	\end{itemize}
}
\item adjective \\
A \textbf{tough} substance is strong, and difficult to break , cut , or tear .
 \textit{
	\begin{itemize}
	\item In industry, diamond can form a tough, non-corrosive coating for tools.
	\item ...dark brown beans with a rather tough outer skin.
	\end{itemize}
}
\item adjective \\
\textbf{Tough}  meat is difficult to cut and chew .
 \textit{
	\begin{itemize}
	\item The steak was tough and the peas were like bullets.
	\end{itemize}
}
\item  \\
 hang tough \textit{
	\begin{itemize}
	\end{itemize}
}
\end{enumerate}

\section*{term}
{\large \color{blue}  terms  terming  termed  }
\subsection*{Explain}
\begin{enumerate}
\item  \\
 in terms of \textit{
	\begin{itemize}
	\end{itemize}
}
\item  \\
 in particular terms \textit{
	\begin{itemize}
	\end{itemize}
}
\item countable noun \\
A \textbf{term} is a word or expression with a specific meaning , especially one which is used in relation to a particular subject.
 \textit{
	\begin{itemize}
	\item Myocardial infarction is the medical term for a heart attack.
	\end{itemize}
}
\item verb \\
If you say that something \textbf{is termed} a particular thing, you mean that that is what people call it or that is their opinion of it.
 \textit{
	\begin{itemize}
	\item He had been termed a temporary employee.
	\item He termed the war a humanitarian nightmare.
	\end{itemize}
}
\item variable noun \\
A \textbf{term} is one of the periods of time that a school, college, or university divides the year
into.
 \textit{
	\begin{itemize}
	\item ...the summer term.
	\item ...the last day of term.
	\end{itemize}
}
\item countable noun \\
A \textbf{term} is a period of time between two elections during which a particular party or government is in power.
 \textit{
	\begin{itemize}
	\item He won a fourth term of office in the election.
	\end{itemize}
}
\item countable noun \\
A \textbf{term} is a period of time that someone spends doing a particular job or in a particular place.
 \textit{
	\begin{itemize}
	\item ...a 12 month term of service.
	\item Offenders will be liable to a seven-year prison term.
	\end{itemize}
}
\item countable noun \\
A \textbf{term} is the period for which a legal contract or insurance policy is valid .
 \textit{
	\begin{itemize}
	\item Premiums are guaranteed throughout the term of the policy.
	\end{itemize}
}
\item uncountable noun \\
The \textbf{term} of a woman's pregnancy is the nine  month period that it lasts . \textbf{Term} is also used to refer to the end of the nine month period.
 \textit{
	\begin{itemize}
	\item Two of her pregnancies failed to reach full term.
	\item Women over 40 seem to be just as capable of carrying a baby to term as younger women.
	\end{itemize}
}
\item plural noun \\
The \textbf{terms} of an agreement, treaty , or other arrangement are the conditions that must be accepted by the people involved in it.
 \textit{
	\begin{itemize}
	\item They are discussing the terms of the peace agreement.
	\item Mayor Rendell imposed the new contract terms.
	\end{itemize}
}
\item  \\
 come to terms with \textit{
	\begin{itemize}
	\end{itemize}
}
\item  \\
 on equal terms/on the same terms \textit{
	\begin{itemize}
	\end{itemize}
}
\item  \\
 on good terms \textit{
	\begin{itemize}
	\end{itemize}
}
\item  \\
 in the long term \textit{
	\begin{itemize}
	\end{itemize}
}
\item  \\
 on your terms \textit{
	\begin{itemize}
	\end{itemize}
}
\item  \\
 think in terms of \textit{
	\begin{itemize}
	\end{itemize}
}
\end{enumerate}

\section*{disastrous}
{\large \color{blue}  }
\subsection*{Explain}
\begin{enumerate}
\item adjective \\
A \textbf{disastrous} event has extremely bad  consequences and effects.
 \textit{
	\begin{itemize}
	\item ...the recent, disastrous earthquake.
	\item The effect on coffee prices has been disastrous for the producers.
	\end{itemize}
}
\item adjective \\
If you describe something as \textbf{disastrous} , you mean that it was very unsuccessful .
 \textit{
	\begin{itemize}
	\item England's cricketers have had another disastrous day.
	\item ...their disastrous performance in the general election of 1906.
	\end{itemize}
}
\end{enumerate}

\section*{clue}
{\large \color{blue}  clues  }
\subsection*{Explain}
\begin{enumerate}
\item countable noun \\
A \textbf{clue}  \textbf{to} a problem or mystery is something that helps you to find the answer to it.
 \textit{
	\begin{itemize}
	\item Geneticists in Canada have discovered a clue to the puzzle of why our cells get old
and die.
	\item How a man shaves may be a telling clue to his age.
	\end{itemize}
}
\item countable noun \\
A \textbf{clue} is an object or piece of information that helps someone solve a crime .
 \textit{
	\begin{itemize}
	\item The vital clue to the killer's identity was his nickname, Peanuts.
	\end{itemize}
}
\item countable noun \\
A \textbf{clue} in a crossword or game is information which is given to help you to find the answer to a question .
 \textit{
	\begin{itemize}
	\end{itemize}
}
\item  \\
 haven't a clue \textit{
	\begin{itemize}
	\end{itemize}
}
\end{enumerate}

\section*{dubious}
{\large \color{blue}  }
\subsection*{Explain}
\begin{enumerate}
\item adjective \\
If you describe something as \textbf{dubious} , you mean that you do not consider it to be completely honest , safe , or reliable .
 \textit{
	\begin{itemize}
	\item This claim seems to us to be rather dubious.
	\item Soho was still a highly dubious area.
	\item Those figures alone are a dubious basis for such a conclusion.
	\end{itemize}
}
\item adjective \\
If you are \textbf{dubious}  \textbf{about} something, you are not completely sure about it and have not yet made up your mind about it.
 \textit{
	\begin{itemize}
	\item My parents were dubious about it at first but we soon convinced them.
	\end{itemize}
}
\item adjective \\
If you say that someone has the \textbf{dubious}  honour or the \textbf{dubious}  pleasure  \textbf{of} doing something, you are indicating that what they are doing is not an honour or
pleasure at all, but is, in fact, unpleasant or bad .
 \textit{
	\begin{itemize}
	\item Nagy has the dubious honour of being the first athlete to be banned in this way.
	\end{itemize}
}
\end{enumerate}

\section*{construction}
{\large \color{blue}  constructions  }
\subsection*{Explain}
\begin{enumerate}
\item uncountable noun \\
\textbf{Construction} is the building of things such as houses , factories , roads , and bridges .
 \textit{
	\begin{itemize}
	\item He'd already started construction on a hunting lodge.
	\item ...the only nuclear power station under construction in Britain.
	\item ...the downturn in the construction industry.
	\item Quincy wants a job in construction.
	\end{itemize}
}
\item uncountable noun \\
The \textbf{construction} of something such as a vehicle or machine is the making of it.
 \textit{
	\begin{itemize}
	\item ...companies who have long experience in the construction of those types of equipment.
	\item With the exception of teak, this is the finest wood for boat construction.
	\end{itemize}
}
\item uncountable noun \\
The \textbf{construction} of something such as a system is the creation of it.
 \textit{
	\begin{itemize}
	\item ...the construction of a just system of criminal justice.
	\end{itemize}
}
\item countable noun \\
You can refer to an object that has been built or made as a \textbf{construction} .
 \textit{
	\begin{itemize}
	\item The British pavilion is an impressive steel and glass construction the size of Westminster
Abbey.
	\end{itemize}
}
\item uncountable noun \\
You use \textbf{construction} to refer to the structure of something and the way it has been built or made.
 \textit{
	\begin{itemize}
	\item The Shakers believed that furniture should be plain, simple, useful, practical and
of sound construction.
	\item The chairs were light in construction yet extremely strong.
	\end{itemize}
}
\item countable noun \\
The \textbf{construction} that you put on what someone says or does is your interpretation of what it means .
 \textit{
	\begin{itemize}
	\item The denial was limited to rejecting the construction put on his remarks.
	\item He put the wrong construction on what he saw.
	\end{itemize}
}
\item countable noun \\
A grammatical  \textbf{construction} is a particular arrangement of words in a sentence, clause, or phrase.
 \textit{
	\begin{itemize}
	\item Avoid complex verbal constructions.
	\end{itemize}
}
\end{enumerate}

\section*{elementary}
{\large \color{blue}  }
\subsection*{Explain}
\begin{enumerate}
\item adjective \\
Something that is \textbf{elementary} is very simple and basic .
 \textit{
	\begin{itemize}
	\item ...elementary computer skills.
	\end{itemize}
}
\end{enumerate}

\section*{driver}
{\large \color{blue}  drivers  }
\subsection*{Explain}
\begin{enumerate}
\item countable noun \\
The \textbf{driver} of a vehicle is the person who is driving it.
 \textit{
	\begin{itemize}
	\item The driver got out of his van.
	\item ...a taxi driver.
	\end{itemize}
}
\item countable noun \\
A \textbf{driver} is a computer program that controls a device such as a printer .
 \textit{
	\begin{itemize}
	\item ...printer driver software.
	\end{itemize}
}
\end{enumerate}

\section*{exempt}
{\large \color{blue}  exempts  exempting  exempted  }
\subsection*{Explain}
\begin{enumerate}
\item adjective \\
If someone or something is \textbf{exempt}  \textbf{from} a particular  rule , duty , or obligation, they do not have to follow it or do it.
 \textbf{Exempt} is also a combining form.
 \textit{
	\begin{itemize}
	\item Men in college were exempt from military service.
	\item Children under two years are exempt.
	\item The fund was in danger of losing its tax-exempt status.
	\end{itemize}
}
\item verb \\
To \textbf{exempt} a person or thing \textbf{from} a particular rule, duty, or obligation means to state officially that they are not bound or affected by it.
 \textit{
	\begin{itemize}
	\item South Carolina claimed the power to exempt its citizens from the obligation to obey
federal law.
	\item Companies with fifty-five or fewer employees would be exempted from the requirements.
	\end{itemize}
}
\end{enumerate}

\section*{fast}
{\large \color{blue}  faster  fastest  fasts  fasting  fasted  }
\subsection*{Explain}
\begin{enumerate}
\item adjective \\
\textbf{Fast} means happening , moving, or doing something at great speed. You also use \textbf{fast} in questions or statements about speed.
 \textbf{Fast} is also an adverb .
 \textit{
	\begin{itemize}
	\item ...fast cars with flashing lights and sirens.
	\item Brindley was known as a very, very fast driver.
	\item The party aims to attract votes from the business and professional communities, which
want a faster pace of political reform.
	\item The only question is how fast the process will be.
	\item They work terrifically fast.
	\item It would be nice to go faster and break the world record.
	\item He thinks they're not adapting fast enough.
	\item Barnes also knows that he is fast running out of time.
	\item How fast were you driving?
	\item How fast would the disease develop?
	\end{itemize}
}
\item adverb \\
You use \textbf{fast} to say that something happens without any delay .
 \textbf{Fast} is also an adjective .
 \textit{
	\begin{itemize}
	\item When you've got a crisis like this you need professional help–fast!
	\item We'd appreciate your leaving as fast as possible.
	\item That would be an astonishingly fast action on the part of the Congress.
	\end{itemize}
}
\item adjective \\
If a watch or clock is \textbf{fast} , it is showing a time that is later than the real time.
 \textit{
	\begin{itemize}
	\item That clock's an hour fast.
	\end{itemize}
}
\item  \\
 to hold fast \textit{
	\begin{itemize}
	\end{itemize}
}
\item adverb \\
If you hold \textbf{fast} to a principle or idea, or if you stand  \textbf{fast} , you do not change your mind about it, even though people are trying to persuade you to.
 \textit{
	\begin{itemize}
	\item We can only try to hold fast to the age-old values of honesty, decency and concern
for others.
	\item He told supporters to stand fast over the next few vital days.
	\end{itemize}
}
\item adjective \\
If colours or dyes are \textbf{fast} , they do not come out of the fabrics they are used on when they get  wet .
 \textit{
	\begin{itemize}
	\item The fabric was ironed to make the colours fast.
	\end{itemize}
}
\item adjective \\
A \textbf{fast} way of life is one which involves a lot of enjoyable and expensive or dangerous activities.
 \textit{
	\begin{itemize}
	\item Life in Detroit no longer satisfied him; he wanted the fast life of California.
	\end{itemize}
}
\item verb \\
If you \textbf{fast} , you eat no food for a period of time, usually for either religious or medical  reasons , or as a protest .
 \textbf{Fast} is also a noun .
 \textit{
	\begin{itemize}
	\item I fasted for a day and half and asked God to help me.
	\item The fast is broken at sunset, traditionally with dates and water.
	\end{itemize}
}
\item  \\
 fast asleep \textit{
	\begin{itemize}
	\end{itemize}
}
\item  \\
 to play fast and loose \textit{
	\begin{itemize}
	\end{itemize}
}
\item  \\
 to pull a fast one \textit{
	\begin{itemize}
	\end{itemize}
}
\end{enumerate}

\section*{hardship}
{\large \color{blue}  hardships  }
\subsection*{Explain}
\begin{enumerate}
\item variable noun \\
\textbf{Hardship} is a situation in which your life is difficult or unpleasant , often because you do not have enough money.
 \textit{
	\begin{itemize}
	\item Many people are suffering economic hardship.
	\item One of the worst hardships is having so little time to spend with one's family.
	\end{itemize}
}
\end{enumerate}

\section*{forthcoming}
{\large \color{blue}  }
\subsection*{Explain}
\begin{enumerate}
\item adjective \\
A \textbf{forthcoming} event is planned to happen  soon .
 \textit{
	\begin{itemize}
	\item ...his opponents in the forthcoming elections.
	\end{itemize}
}
\item adjective \\
If something that you want , need , or expect is \textbf{forthcoming} , it is given to you or it happens.
 \textit{
	\begin{itemize}
	\item They promised that the money would be forthcoming.
	\item We must first see some real evidence. So far it has not been forthcoming.
	\item One source predicts no major shift in policy will be forthcoming at the committee
hearings.
	\end{itemize}
}
\item adjective \\
If you say that someone is \textbf{forthcoming} , you mean that they willingly give information when you ask them.
 \textit{
	\begin{itemize}
	\end{itemize}
}
\end{enumerate}

\section*{house}
{\large \color{blue}  houses  housing  housed  }
\subsection*{Explain}
\begin{enumerate}
\item countable noun \\
A \textbf{house} is a building in which people live, usually the people belonging to one family.
 \textit{
	\begin{itemize}
	\item She has moved to a smaller house.
	\item ...her parents' house in Warwickshire.
	\end{itemize}
}
\item singular noun \\
You can refer to all the people who live together in a house as \textbf{the}  \textbf{house} .
 \textit{
	\begin{itemize}
	\item If he set his alarm clock for midnight, it would wake the whole house.
	\end{itemize}
}
\item countable noun \\
\textbf{House} is used in the names of types of places where people go to eat and drink.
 \textit{
	\begin{itemize}
	\item ...a steak house.
	\item ...an old Salzburg coffee house.
	\end{itemize}
}
\item countable noun \\
\textbf{House} is used in the names of types of companies, especially ones which publish books, lend money, or design clothes.
 \textit{
	\begin{itemize}
	\item Many of the clothes come from the world's top fashion houses.
	\item Eventually she was fired from her job at a publishing house.
	\end{itemize}
}
\item countable noun \\
\textbf{House} is sometimes used in the names of office buildings and large private homes or expensive houses.
 \textit{
	\begin{itemize}
	\item I was to go to the very top floor of Bush House in Aldwych.
	\item ...Harewood House near Leeds.
	\end{itemize}
}
\item countable noun \\
You can refer to the two main bodies of Britain's parliament and the United States of America's legislature as \textbf{the}  \textbf{House} or a \textbf{House} .
 \textit{
	\begin{itemize}
	\item Some members of the House and Senate worked all day yesterday.
	\item The Republicans have majorities in both Houses.
	\end{itemize}
}
\item singular noun \\
You can refer to all the people at a debate as \textbf{the}  \textbf{house} .
 \textit{
	\begin{itemize}
	\item The club is planning a public debate on 'This house believes that journalism has
not gained from the introduction of new technology'.
	\end{itemize}
}
\item countable noun \\
In a British school, a \textbf{house} is a group of children of different ages who compete against other groups in sports and other activities. Each house usually has a name.
 \textit{
	\begin{itemize}
	\item He was a prefect and house captain.
	\end{itemize}
}
\item countable noun \\
A \textbf{house} is a family which has been or will be important for many generations , especially the family of a king or queen .
 \textit{
	\begin{itemize}
	\item ...the Saudi Royal House.
	\item ...the House of Windsor.
	\end{itemize}
}
\item countable noun \\
The \textbf{house} is the part of a theatre, cinema, or other place of entertainment where the audience sits . You can also refer to the audience at a particular performance as the \textbf{house} .
 \textit{
	\begin{itemize}
	\item They played in front of a packed house.
	\end{itemize}
}
\item adjective \\
A restaurant's \textbf{house} wine is the cheapest wine it sells, which is not listed by name on the wine list.
 \textit{
	\begin{itemize}
	\item Tweed ordered a carafe of the house wine.
	\item ...a bottle of house red or white.
	\end{itemize}
}
\item verb \\
To \textbf{house} someone means to provide a house or flat for them to live in.
 \textit{
	\begin{itemize}
	\item Part III of the Housing Act 1985 imposes duties on local authorities to house homeless
people.
	\item Regrettably we have to house families in these inadequate flats.
	\end{itemize}
}
\item verb \\
A building or container that \textbf{houses} something is the place where it is located or from where it operates.
 \textit{
	\begin{itemize}
	\item The château itself is open to the public and houses a museum of motorcycles and cars.
	\item Many years later, the temple erected in her name was used to house the Roman mint.
	\end{itemize}
}
\item verb \\
If you say that a building \textbf{houses} a number of people, you mean that is the place where they live or where they are
 staying .
 \textit{
	\begin{itemize}
	\item The building will house twelve boys and eight girls.
	\item Their villas housed army officers now.
	\end{itemize}
}
\item  \\
 to bring the house down \textit{
	\begin{itemize}
	\end{itemize}
}
\item  \\
 like a house on fire \textit{
	\begin{itemize}
	\end{itemize}
}
\item  \\
 to keep house \textit{
	\begin{itemize}
	\end{itemize}
}
\item  \\
 on the house \textit{
	\begin{itemize}
	\end{itemize}
}
\item  \\
 get/set one's house in order \textit{
	\begin{itemize}
	\end{itemize}
}
\end{enumerate}

\section*{frank}
{\large \color{blue}  franker  frankest  franks  franking  franked  }
\subsection*{Explain}
\begin{enumerate}
\item adjective \\
If someone is \textbf{frank} , they state or express things in an open and honest way.
 \textit{
	\begin{itemize}
	\item 'It is clear that my client has been less than frank with me,' said his lawyer.
	\item They had a frank discussion about the issue.
	\end{itemize}
}
\item verb \\
When a letter or parcel \textbf{is franked} , it is marked with a symbol that shows that the proper charge has been paid or that no stamp is needed .
 \textit{
	\begin{itemize}
	\item The letter was franked in London on August 6.
	\item ...a self-addressed, franked envelope.
	\end{itemize}
}
\item  \\
 to be frank \textit{
	\begin{itemize}
	\end{itemize}
}
\end{enumerate}

\section*{inhabitant}
{\large \color{blue}  inhabitants  }
\subsection*{Explain}
\begin{enumerate}
\item countable noun \\
The \textbf{inhabitants} of a place are the people who live there.
 \textit{
	\begin{itemize}
	\item ...the inhabitants of Glasgow.
	\end{itemize}
}
\end{enumerate}

\section*{furious}
{\large \color{blue}  }
\subsection*{Explain}
\begin{enumerate}
\item adjective \\
Someone who is \textbf{furious} is extremely angry.
 \textit{
	\begin{itemize}
	\item He is furious at the way his wife has been treated.
	\item I am furious that it has taken so long to uncover what really happened.
	\end{itemize}
}
\item adjective \\
\textbf{Furious} is also used to describe something that is done with great energy, effort , speed, or violence .
 \textit{
	\begin{itemize}
	\item A furious gunbattle ensued.
	\end{itemize}
}
\end{enumerate}

\section*{inventory}
{\large \color{blue}  inventories  }
\subsection*{Explain}
\begin{enumerate}
\item countable noun \\
An \textbf{inventory} is a written list of all the objects in a particular place.
 \textit{
	\begin{itemize}
	\item Before starting, he made an inventory of everything that was to stay.
	\end{itemize}
}
\item variable noun \\
An \textbf{inventory} is a supply or stock of something.
 \textit{
	\begin{itemize}
	\item ...one inventory of twelve sails for each yacht.
	\end{itemize}
}
\end{enumerate}

\section*{gay}
{\large \color{blue}  gays  gayer  gayest  }
\subsection*{Explain}
\begin{enumerate}
\item adjective \\
A \textbf{gay} person is homosexual.
 \textit{
	\begin{itemize}
	\item ...the gay community.
	\end{itemize}
}
\item adjective \\
A \textbf{gay} person is fun to be with because they are lively and cheerful .
 \textit{
	\begin{itemize}
	\item I am happy and free, in good health, gay and cheerful.
	\end{itemize}
}
\item adjective \\
A \textbf{gay} object is brightly coloured and pretty to look at.
 \textit{
	\begin{itemize}
	\item I like gay, relaxing paintings.
	\end{itemize}
}
\end{enumerate}

\section*{limitation}
{\large \color{blue}  limitations  }
\subsection*{Explain}
\begin{enumerate}
\item uncountable noun \\
The \textbf{limitation} of something is the act or process of controlling or reducing it.
 \textit{
	\begin{itemize}
	\item All the talk had been about the limitation of nuclear weapons.
	\item ...damage limitation.
	\end{itemize}
}
\item variable noun \\
A \textbf{limitation}  \textbf{on} something is a rule or decision which prevents that thing from growing or extending beyond certain limits.
 \textit{
	\begin{itemize}
	\item ...a limitation on the tax deductions for people who make more than $100,000 a year.
	\item There is to be no limitation on the number of opposition parties.
	\end{itemize}
}
\item plural noun \\
If you talk about the \textbf{limitations} of someone or something, you mean that they can only do some things and not others,
or cannot do something very well .
 \textit{
	\begin{itemize}
	\item The theory is a useful tool, but it has limitations.
	\item Parents are too likely to blame schools for the educational limitations of their
children.
	\end{itemize}
}
\item variable noun \\
A \textbf{limitation} is a fact or situation that allows only some actions and makes others impossible .
 \textit{
	\begin{itemize}
	\item This drug has one important limitation. Its effects only last six hours.
	\item ...an acute disc collapse in the spine, causing limitation of movement.
	\end{itemize}
}
\end{enumerate}

\section*{line}
{\large \color{blue}  lines  lining  lined  }
\subsection*{Explain}
\begin{enumerate}
\item countable noun \\
A \textbf{line} is a long thin mark which is drawn or painted on a surface.
 \textit{
	\begin{itemize}
	\item Draw a line down that page's center.
	\item ...a dotted line.
	\item The ball had clearly crossed the line.
	\end{itemize}
}
\item countable noun \\
The \textbf{lines} on someone's skin, especially on their face, are long thin marks that appear there
as they grow older.
 \textit{
	\begin{itemize}
	\item He has a large, generous face with deep lines.
	\item ...fine lines and wrinkles.
	\end{itemize}
}
\item countable noun \\
A \textbf{line} of people or things is a number of them arranged one behind the other or side by
side.
 \textit{
	\begin{itemize}
	\item The sparse line of spectators noticed nothing unusual.
	\end{itemize}
}
\item countable noun \\
A \textbf{line} of people or vehicles is a number of them that are waiting one behind another, for
example in order to buy something or to go in a particular direction.
 \textit{
	\begin{itemize}
	\item Children clutching empty bowls form a line.
	\end{itemize}
}
\item countable noun \\
A \textbf{line} of a piece of writing is one of the rows of words, numbers, or other symbols in it.
 \textit{
	\begin{itemize}
	\item The next line should read: Five days, 23.5 hours.
	\item Tina wouldn't have read more than three lines.
	\end{itemize}
}
\item plural noun \\
In school, if a child is given \textbf{lines} , he or she is punished by being made to write out a sentence many times or to write out a passage from a
book.
 \textit{
	\begin{itemize}
	\end{itemize}
}
\item countable noun \\
A \textbf{line} of a poem, song, or play is a group of words that are spoken or sung together. If
an actor \textbf{learns} his or her \textbf{lines} for a play or film, they learn what they have to say .
 \textit{
	\begin{itemize}
	\item ...a line from Shakespeare's Othello: 'one that loved not wisely but too well'.
	\item Every time I sing that line, I have to compete with that bloody trombone!
	\item Learning lines is very easy. Acting is very difficult.
	\end{itemize}
}
\item countable noun \\
A particular type of \textbf{line} in a conversation is a remark that is intended to have a particular effect.
 \textit{
	\begin{itemize}
	\item 'In time perhaps you'll marry again'. 'That's a great line, coming from you!'.
	\item ...chat-up lines like 'You've got beautiful eyes'.
	\end{itemize}
}
\item variable noun \\
You can refer to a long piece of wire, string, or cable as a \textbf{line} when it is used for a particular purpose.
 \textit{
	\begin{itemize}
	\item She put her washing on the line.
	\item ...a piece of fishing-line.
	\item The winds downed power lines.
	\end{itemize}
}
\item countable noun \\
A \textbf{line} is a connection which makes it possible for two people to speak to each other on
the phone .
 \textit{
	\begin{itemize}
	\item The phone lines went dead.
	\item It's not a very good line. Shall we call you back Susan?
	\item She's on the line from her home in Boston.
	\end{itemize}
}
\item countable noun \\
You can use \textbf{line} to refer to a phone number which you can ring in order to get information or advice .
 \textit{
	\begin{itemize}
	\item ...the 24-hours information line.
	\item ...details from Lesbian Line.
	\end{itemize}
}
\item countable noun \\
A \textbf{line} is a route, especially a dangerous or secret one, along which people move or send
messages or supplies.
 \textit{
	\begin{itemize}
	\item The American continent's geography severely limited the lines of attack.
	\item Negotiators say they're keeping communication lines open.
	\item ...the guerrillas' main supply lines.
	\end{itemize}
}
\item countable noun \\
The \textbf{line} in which something or someone moves is the particular route that they take, especially
when they keep moving straight ahead .
 \textit{
	\begin{itemize}
	\item Walk in a straight line.
	\item The wings were at right angles to the line of flight.
	\end{itemize}
}
\item countable noun \\
A \textbf{line} is a particular route, involving the same stations, roads, or stops along which a
train or bus service regularly operates.
 \textit{
	\begin{itemize}
	\item They've got to ride all the way to the end of the line.
	\item Fires halted service on two commuter lines for several hours.
	\item I would be able to stay on the Piccadilly Line and get off the tube at South Kensington.
	\end{itemize}
}
\item countable noun \\
A railway \textbf{line} consists of the pieces of metal and wood which form the track that the trains travel
along.
 \textit{
	\begin{itemize}
	\item Leaves on the line are an expensive problem for the railways.
	\end{itemize}
}
\item countable noun \\
A shipping, air, or bus \textbf{line} is a company which provides services for transporting people or goods by sea, air,
or bus.
 \textit{
	\begin{itemize}
	\item The Foreign Office offered to pay the shipping line all the costs of diverting the
ship to Bermuda.
	\end{itemize}
}
\item countable noun \\
You can use \textbf{line} to refer to the edge, outline , or shape of an object or a person's body.
 \textit{
	\begin{itemize}
	\item The garden has an informal feel to soften the architectural lines of the conservatory.
	\item ...a sculptured evening dress that follows the lines of the body.
	\end{itemize}
}
\item countable noun \\
A state or county \textbf{line} is a boundary between two states or counties.
 \textit{
	\begin{itemize}
	\item ...the California state line.
	\end{itemize}
}
\item countable noun \\
You can use \textbf{lines} to refer to the set of physical defences or the soldiers that have been established
along the boundary of an area occupied by an army.
 \textit{
	\begin{itemize}
	\item Their unit was shelling the German lines only seven miles away.
	\item ...the stupendous fortification they called the Maginot Line.
	\end{itemize}
}
\item countable noun \\
The particular \textbf{line} that a person has towards a problem is the attitude that they have towards it. For
example, if someone takes a \textbf{hard line} on something, they have a firm strict policy which they refuse to change.
 \textit{
	\begin{itemize}
	\item Forty members of the governing Conservative party rebelled, voting against the government
line.
	\item Most members of Parliament took a hard line on this issue.
	\end{itemize}
}
\item countable noun \\
You can use \textbf{line} to refer to the way in which someone's thoughts or activities develop, particularly
if it is logical .
 \textit{
	\begin{itemize}
	\item Our discussion in the previous chapter continues this line of thinking.
	\item What are some of the practical benefits likely to be of this line of research?
	\end{itemize}
}
\item plural noun \\
If you say that something happens \textbf{along} particular \textbf{lines} , or \textbf{on} particular \textbf{lines} , you are giving a general summary or approximate account of what happens, which may not be correct in every detail.
 \textit{
	\begin{itemize}
	\item There followed praise along the lines of 'Hey, this coffee is fantastic!'.
	\item He'd said something along those lines already.
	\item Our forecast was on the right lines.
	\item The main lines of the plan were reduced expenditure and fewer government controls.
	\end{itemize}
}
\item plural noun \\
If something is organized \textbf{on} particular \textbf{lines} , or \textbf{along} particular \textbf{lines} , it is organized according to that method or principle.
 \textit{
	\begin{itemize}
	\item ...so-called autonomous republics based on ethnic lines.
	\item ...reorganising old factories to work along Japanese lines.
	\end{itemize}
}
\item countable noun \\
Your \textbf{line}  \textbf{of} business or work is the kind of work that you do.
 \textit{
	\begin{itemize}
	\item So what was your father's line of business?
	\item In my line of work I often get home too late for dinner.
	\end{itemize}
}
\item singular noun \\
If someone says that something is \textbf{your}  \textbf{line} , or that it is \textbf{in your line} , they mean that it is the sort of thing that you often do because you enjoy doing it.
 \textit{
	\begin{itemize}
	\item Wild guesses aren't much in my line.
	\item Perhaps doing voluntary work is more your line?
	\end{itemize}
}
\item countable noun \\
A \textbf{line} is a particular type of product that a company makes or sells.
 \textit{
	\begin{itemize}
	\item His best selling line is the cheapest lager at £1.99.
	\end{itemize}
}
\item singular noun \\
You can use \textbf{line} to refer to something connected with a particular activity. For example, something
 \textbf{in the sports line} is connected with sports.
 \textit{
	\begin{itemize}
	\item Most kids can do something in the art line.
	\end{itemize}
}
\item countable noun \\
In a factory, a \textbf{line} is an arrangement of workers or machines where a product passes from one worker to
another until it is finished.
 \textit{
	\begin{itemize}
	\item ...a production line capable of producing three different products.
	\end{itemize}
}
\item countable noun \\
You can use \textbf{line} to refer to all the generations of a family, especially when you are considering the social status or the physical
characteristics that the various members inherit .
 \textit{
	\begin{itemize}
	\item ...the old Welsh royal line descended from Arthur and Uther Pendragon.
	\item This title will only pass down through the male line.
	\end{itemize}
}
\item countable noun \\
You can use \textbf{line} when you are referring to a number of people who are ranked according to status.
 \textit{
	\begin{itemize}
	\item Nicholas Paul Patrick was seventh in the line of succession to the throne.
	\item The line of command went from head office in Chicago to a regional boss and then
down to a country boss and finally to a local-office managing-partner.
	\item ...the man who stands next in line for the presidency.
	\end{itemize}
}
\item countable noun \\
A particular \textbf{line}  \textbf{of} people or things is a series of them that has existed over a period of time, when
they have all been similar in some way, or done similar things.
 \textit{
	\begin{itemize}
	\item We were part of a long line of artists.
	\item It's the latest in a long line of tragedies.
	\end{itemize}
}
\item verb \\
If people or things \textbf{line} a road, room, or other place, they are present in large numbers along its edges or
sides.
 \textit{
	\begin{itemize}
	\item Thousands of local people lined the streets and clapped as the procession went by.
	\item ...a square lined with pubs and clubs.
	\end{itemize}
}
\item verb \\
If you \textbf{line} a wall, container, or other object, you put a layer of something such as leaves or
paper on the inside surface of it in order to make it stronger, warmer, or cleaner .
 \textit{
	\begin{itemize}
	\item Scoop the blanket weed out and use it to line hanging baskets.
	\item Female bears tend to line their dens with leaves or grass.
	\end{itemize}
}
\item verb \\
If something \textbf{lines} a container or area, especially an area inside a person, animal, or plant, it forms
a layer on the inside surface.
 \textit{
	\begin{itemize}
	\item ...the muscles that line the intestines.
	\end{itemize}
}
\item  \\
 along the line/down the line \textit{
	\begin{itemize}
	\end{itemize}
}
\item  \\
 down the line \textit{
	\begin{itemize}
	\end{itemize}
}
\item  \\
 to draw the line \textit{
	\begin{itemize}
	\end{itemize}
}
\item  \\
 draw a line \textit{
	\begin{itemize}
	\end{itemize}
}
\item  \\
 to drop someone a line \textit{
	\begin{itemize}
	\end{itemize}
}
\item  \\
 in the line of duty \textit{
	\begin{itemize}
	\end{itemize}
}
\item  \\
 the first line of \textit{
	\begin{itemize}
	\end{itemize}
}
\item  \\
 in line \textit{
	\begin{itemize}
	\end{itemize}
}
\item  \\
 in/into line \textit{
	\begin{itemize}
	\end{itemize}
}
\item  \\
 in/into line \textit{
	\begin{itemize}
	\end{itemize}
}
\item  \\
 stand/wait in line \textit{
	\begin{itemize}
	\end{itemize}
}
\item  \\
 in line/into line \textit{
	\begin{itemize}
	\end{itemize}
}
\item  \\
 on line \textit{
	\begin{itemize}
	\end{itemize}
}
\item  \\
 on line \textit{
	\begin{itemize}
	\end{itemize}
}
\item  \\
 on the line \textit{
	\begin{itemize}
	\end{itemize}
}
\item  \\
 out of line \textit{
	\begin{itemize}
	\end{itemize}
}
\item  \\
 out of line \textit{
	\begin{itemize}
	\end{itemize}
}
\item  \\
 out of line \textit{
	\begin{itemize}
	\end{itemize}
}
\item  \\
 to read between the lines \textit{
	\begin{itemize}
	\end{itemize}
}
\end{enumerate}

\section*{indifferent}
{\large \color{blue}  }
\subsection*{Explain}
\begin{enumerate}
\item adjective \\
If you accuse someone of being \textbf{indifferent}  \textbf{to} something, you mean that they have a complete lack of interest in it.
 \textit{
	\begin{itemize}
	\item People have become indifferent to the suffering of others.
	\end{itemize}
}
\item adjective \\
If you describe something or someone as \textbf{indifferent} , you mean that their standard or quality is not very good, and often quite  bad .
 \textit{
	\begin{itemize}
	\item She had starred in several very indifferent movies.
	\item Much of the food we eat is of very poor or indifferent quality.
	\end{itemize}
}
\end{enumerate}

\section*{microphone}
{\large \color{blue}  microphones  }
\subsection*{Explain}
\begin{enumerate}
\item countable noun \\
A \textbf{microphone} is a device that is used to make sounds louder or to record them.
 \textit{
	\begin{itemize}
	\end{itemize}
}
\end{enumerate}

\section*{intact}
{\large \color{blue}  }
\subsection*{Explain}
\begin{enumerate}
\item adjective \\
Something that is \textbf{intact} is complete and has not been damaged or changed.
 \textit{
	\begin{itemize}
	\item Most of the cargo was left intact after the explosion.
	\item If the family unit is still intact, the patient frequently does very well.
	\end{itemize}
}
\end{enumerate}

\section*{orphan}
{\large \color{blue}  orphans  orphaned  }
\subsection*{Explain}
\begin{enumerate}
\item countable noun \\
An \textbf{orphan} is a child whose parents are dead.
 \textit{
	\begin{itemize}
	\item I'm an orphan and pretty much grew up on my own.
	\item ...a young orphan girl brought up by peasants.
	\end{itemize}
}
\item passive verb \\
If a child \textbf{is orphaned} , their parents die , or their remaining parent dies.
 \textit{
	\begin{itemize}
	\item Jones was orphaned at the age of ten, and taken in by next-door neighbours.
	\item Some ten million children have been orphaned by the disease.
	\item ...a fifteen-year-old boy left orphaned by the recent disaster.
	\end{itemize}
}
\end{enumerate}

\section*{intermittent}
{\large \color{blue}  }
\subsection*{Explain}
\begin{enumerate}
\item adjective \\
Something that is \textbf{intermittent}  happens occasionally rather than continuously.
 \textit{
	\begin{itemize}
	\item After three hours of intermittent rain, the game was abandoned.
	\end{itemize}
}
\end{enumerate}

\section*{parachute}
{\large \color{blue}  parachutes  parachuting  parachuted  }
\subsection*{Explain}
\begin{enumerate}
\item countable noun \\
A \textbf{parachute} is a device which enables a person to jump from an aircraft and float safely to the ground . It consists of a large piece of thin  cloth  attached to your body by strings .
 \textit{
	\begin{itemize}
	\item They fell 41,000 ft. before opening their parachutes.
	\item U.N. troops could be landed by helicopter or even by parachute.
	\end{itemize}
}
\item verb \\
If a person \textbf{parachutes} or someone \textbf{parachutes} them somewhere , they jump from an aircraft using a parachute.
 \textit{
	\begin{itemize}
	\item He was a courier for the Polish underground and parachuted into Warsaw.
	\item He was parachuted in.
	\end{itemize}
}
\item verb \\
To \textbf{parachute} something somewhere means to drop it somewhere by parachute.
 \textit{
	\begin{itemize}
	\item Planes parachuted food and water into the rugged mountainous border region.
	\item Supplies were parachuted into the mountains.
	\end{itemize}
}
\item verb \\
If a person \textbf{parachutes}  \textbf{into} an organization or if they \textbf{are parachuted}  \textbf{into} it, they are brought in suddenly in order to help it.
 \textit{
	\begin{itemize}
	\item ...a consultant who parachutes into corporations and helps provide strategic thinking.
	\item There was intense speculation 18 months ago that the former foreign secretary might
be parachuted into the Scottish Parliament.
	\end{itemize}
}
\end{enumerate}

\section*{late}
{\large \color{blue}  later  latest  }
\subsection*{Explain}
\begin{enumerate}
\item adverb \\
\textbf{Late}  means near the end of a day , week , year , or other period of time.
 \textbf{Late} is also an adjective .
 \textit{
	\begin{itemize}
	\item It was late in the afternoon.
	\item She had to work late at night.
	\item His autobiography was written late in life.
	\item The case is expected to end late next week.
	\item Since late last year the border area has been the scene of heavy fighting.
	\item The talks eventually broke down in late spring.
	\item He was in his late 20s.
	\item ...the late 1960s.
	\end{itemize}
}
\item adjective \\
If it is \textbf{late} , it is near the end of the day or it is past the time that you feel something should have been done .
 \textit{
	\begin{itemize}
	\item It was very late and the streets were deserted.
	\item We've got to go now. It's getting late.
	\end{itemize}
}
\item adverb \\
\textbf{Late} means after the time that was arranged or expected.
 \textbf{Late} is also an adjective.
 \textit{
	\begin{itemize}
	\item Steve arrived late.
	\item The talks began some fifteen minutes late.
	\item We got up late.
	\item His campaign got off to a late start.
	\item We were a little late.
	\item The train was 40 minutes late.
	\item He's a half hour late.
	\end{itemize}
}
\item adverb \\
\textbf{Late} means after the usual time that a particular  event or activity  happens .
 \textbf{Late} is also an adjective.
 \textit{
	\begin{itemize}
	\item We went to bed very late.
	\item He married late.
	\item They had a late lunch in a cafe.
	\item He was a very late developer.
	\end{itemize}
}
\item adjective \\
You use \textbf{late} when you are talking about someone who is dead , especially someone who has died recently.
 \textit{
	\begin{itemize}
	\item ...my late husband.
	\item ...the late Mr Parkin.
	\end{itemize}
}
\item adjective \\
Someone who is \textbf{late of} a particular place or institution  lived or worked there until recently.
 \textit{
	\begin{itemize}
	\item ...Cousin Zachary, late of Bellevue Avenue.
	\item The restaurant is managed by Angelo, late of the Savoy Grill.
	\end{itemize}
}
\item  \\
 better late than never \textit{
	\begin{itemize}
	\end{itemize}
}
\item  \\
 late in the day \textit{
	\begin{itemize}
	\end{itemize}
}
\item  \\
 of late \textit{
	\begin{itemize}
	\end{itemize}
}
\item  \\
 too late \textit{
	\begin{itemize}
	\end{itemize}
}
\end{enumerate}

\section*{parameter}
{\large \color{blue}  parameters  }
\subsection*{Explain}
\begin{enumerate}
\item countable noun \\
\textbf{Parameters} are factors or limits which affect the way that something can be done or made.
 \textit{
	\begin{itemize}
	\item That would be enough to make sure we fell within the parameters of our loan agreement.
	\item A person's stride length has certain parameters.
	\end{itemize}
}
\end{enumerate}

\section*{latin}
{\large \color{blue}  Latins  }
\subsection*{Explain}
\begin{enumerate}
\item uncountable noun \\
\textbf{Latin} is the language which the ancient Romans used to speak .
 \textit{
	\begin{itemize}
	\end{itemize}
}
\item adjective \\
\textbf{Latin}  countries are countries where Spanish , or perhaps  Portuguese , Italian , or French , is spoken. You can  also use \textbf{Latin} to refer to things and people that come from these countries.
 \textit{
	\begin{itemize}
	\item Cuba was one of the least Catholic of the Latin countries.
	\item The enthusiasm for Latin music is worldwide.
	\end{itemize}
}
\item countable noun \\
\textbf{Latins} are people who come from countries where Spanish, or perhaps Portuguese, Italian,
or French, are spoken or whose families come from one of these countries.
 \textit{
	\begin{itemize}
	\item They are role models for thousands of young Latins.
	\end{itemize}
}
\end{enumerate}

\section*{participant}
{\large \color{blue}  participants  }
\subsection*{Explain}
\begin{enumerate}
\item countable noun \\
The \textbf{participants} in an activity are the people who take part in it.
 \textit{
	\begin{itemize}
	\item 40 of the course participants are offered employment with the company.
	\item You are expected to be an active participant.
	\end{itemize}
}
\end{enumerate}

\section*{prayer}
{\large \color{blue}  prayers  }
\subsection*{Explain}
\begin{enumerate}
\item uncountable noun \\
\textbf{Prayer} is the activity of speaking to God .
 \textit{
	\begin{itemize}
	\item They had joined a religious order and dedicated their lives to prayer and good works.
	\item The night was spent in prayer.
	\end{itemize}
}
\item countable noun \\
A \textbf{prayer} is the words a person says when they speak to God.
 \textit{
	\begin{itemize}
	\item They should take a little time and say a prayer for the people on both sides.
	\item ...prayers of thanksgiving.
	\end{itemize}
}
\item countable noun \\
You can  refer to a strong hope that you have as your \textbf{prayer} .
 \textit{
	\begin{itemize}
	\item This drug could be the answer to our prayers.
	\end{itemize}
}
\item plural noun \\
A short  religious  service at which people gather to pray can be referred to as \textbf{prayers} .
 \textit{
	\begin{itemize}
	\item He promised that the boy would be back at school in time for evening prayers.
	\item ...Muslims attending prayers in the main mosque.
	\end{itemize}
}
\item  \\
 to have not got a prayer \textit{
	\begin{itemize}
	\end{itemize}
}
\end{enumerate}

\section*{modern}
{\large \color{blue}  moderns  }
\subsection*{Explain}
\begin{enumerate}
\item adjective \\
\textbf{Modern}  means relating to the present time, for example the present decade or present century.
 \textit{
	\begin{itemize}
	\item ...the problem of materialism in modern society.
	\item ...the risks facing every modern marriage.
	\item It's the sort of thing that would be very difficult to prove in any modern court
of law.
	\end{itemize}
}
\item adjective \\
Something that is \textbf{modern} is new and involves the latest  ideas or equipment .
 \textit{
	\begin{itemize}
	\item Modern technology has opened our eyes to many things.
	\item In many ways, it was a very modern school for its time.
	\item As the country's economy prospered, it was bound to want a modern army.
	\end{itemize}
}
\item adjective \\
People are sometimes  described as \textbf{modern} when they have opinions or ways of behaviour that have not yet been accepted by most people in a society .
 \textit{
	\begin{itemize}
	\item They were very modern Tories in almost every sense.
	\item She is very modern in outlook.
	\end{itemize}
}
\item adjective \\
\textbf{Modern} is used to describe styles of art, dance , music, and architecture that have developed in recent times, in contrast to classical styles.
 The \textbf{moderns} are artists who follow modern styles.
 \textit{
	\begin{itemize}
	\item ...a modern dance company.
	\item ...the Museum of Modern Art.
	\item I don't have much time for the moderns. Chaucer's my favourite.
	\end{itemize}
}
\end{enumerate}

\section*{resident}
{\large \color{blue}  residents  }
\subsection*{Explain}
\begin{enumerate}
\item countable noun \\
The \textbf{residents} of a house or area are the people who live there.
 \textit{
	\begin{itemize}
	\item ...building low-cost homes for local residents.
	\item More than 10 percent of the city's residents live below the poverty line.
	\end{itemize}
}
\item adjective \\
Someone who is \textbf{resident}  \textbf{in} a country or a town lives there.
 \textit{
	\begin{itemize}
	\item He moved to Belgium to live with his son, who had been resident in Brussels since
1997.
	\end{itemize}
}
\item adjective \\
A \textbf{resident} doctor or teacher lives in the place where he or she works.
 \textit{
	\begin{itemize}
	\item The morning after your arrival, you meet with the resident physician.
	\end{itemize}
}
\item countable noun \\
A \textbf{resident} or a \textbf{resident} doctor is a doctor who is receiving a period of specialized training in a hospital after leaving university.
 \textit{
	\begin{itemize}
	\end{itemize}
}
\item adjective \\
If an institution has a \textbf{resident} specialist, that specialist works for the institution.
 \textit{
	\begin{itemize}
	\item Having begun her career at Gray's Pottery, she stayed there as resident designer
for seven years.
	\end{itemize}
}
\end{enumerate}

\section*{nearby}
{\large \color{blue}  }
\subsection*{Explain}
\begin{enumerate}
\item adverb \\
If something is \textbf{nearby,} it is only a short distance away.
 \textbf{Nearby} is also an adjective .
 \textit{
	\begin{itemize}
	\item He might easily have been seen by someone who lived nearby.
	\item He spoke softly to a couple standing nearby.
	\item There is less expensive accommodation nearby.
	\item There were one or two suspicious looks from nearby.
	\item At a nearby table a man was complaining in a loud voice.
	\item ...the nearby village of Crowthorne.
	\end{itemize}
}
\end{enumerate}

\section*{senate}
{\large \color{blue}  Senates  }
\subsection*{Explain}
\begin{enumerate}
\item proper noun \\
\textbf{The Senate} is the smaller and more important of the two parts of the parliament in some countries, for example the United  States and Australia.
 \textit{
	\begin{itemize}
	\item The Senate is expected to pass the bill shortly.
	\item ...a Senate committee.
	\end{itemize}
}
\item proper noun \\
\textbf{Senate} or \textbf{the Senate} is the governing council at some universities.
 \textit{
	\begin{itemize}
	\item By the time I was Vice Chancellor, Senate had become a much larger and a much more
democratic body.
	\item The new bill would remove student representation from the university Senate.
	\end{itemize}
}
\end{enumerate}

\section*{oval}
{\large \color{blue}  ovals  }
\subsection*{Explain}
\begin{enumerate}
\item adjective \\
\textbf{Oval} things have a shape that is like a circle but is wider in one direction than the other.
 \textbf{Oval} is also a noun .
 \textit{
	\begin{itemize}
	\item He was a man in his late thirties, with fine, dark hair and a pale oval face.
	\item ...the small oval framed picture of a little boy.
	\item Using 2 spoons, mould the cheese into small balls or ovals.
	\end{itemize}
}
\end{enumerate}

\section*{senator}
{\large \color{blue}  senators  }
\subsection*{Explain}
\begin{enumerate}
\item countable noun \\
A \textbf{senator} is a member of a political Senate, for example in the United  States or Australia .
 \textit{
	\begin{itemize}
	\end{itemize}
}
\end{enumerate}

\section*{sight}
{\large \color{blue}  sights  sighting  sighted  }
\subsection*{Explain}
\begin{enumerate}
\item uncountable noun \\
Someone's \textbf{sight} is their ability to see.
 \textit{
	\begin{itemize}
	\item My sight is failing, and I can't see to read any more.
	\item I use the sense of sound much more than the sense of sight.
	\end{itemize}
}
\item singular noun \\
\textbf{The sight of} something is the act of seeing it or an occasion on which you see it.
 \textit{
	\begin{itemize}
	\item I faint at the sight of blood.
	\item The sight of her entering a room could flood her with anger.
	\end{itemize}
}
\item countable noun \\
A \textbf{sight} is something that you see.
 \textit{
	\begin{itemize}
	\item The practice of hanging clothes across the street is a common sight in many parts
of the city.
	\item We encountered the pathetic sight of a family packing up its home.
	\item Among the most spectacular sights are the great sea-bird colonies.
	\end{itemize}
}
\item verb \\
If you \textbf{sight} someone or something, you suddenly see them, often briefly.
 \textit{
	\begin{itemize}
	\item The security forces sighted a group of young men that had crossed the border.
	\item A fleet of French ships was sighted in the North Sea.
	\end{itemize}
}
\item countable noun \\
The \textbf{sights} of a weapon such as a rifle are the part which helps you aim it more accurately.
 \textit{
	\begin{itemize}
	\end{itemize}
}
\item plural noun \\
\textbf{The}  \textbf{sights} are the places that are interesting to see and that are often visited by tourists .
 \textit{
	\begin{itemize}
	\item We'd toured the sights of Paris.
	\item I am going to show you the sights of our wonderful city.
	\item Once at Elgin day-trippers visit a number of local sights.
	\end{itemize}
}
\item adverb \\
You can use \textbf{a sight} to mean a lot . For example , if you say that something is \textbf{a sight}  worse than it was before, you are emphasizing that it is much worse than it was.
 \textit{
	\begin{itemize}
	\item He's been no more difficult than most children and a sight better than some I could
mention.
	\item We aren't doing anything different. We're just doing it a damn sight quicker.
	\end{itemize}
}
\item  \\
 catch sight of someone \textit{
	\begin{itemize}
	\end{itemize}
}
\item  \\
 at first sight \textit{
	\begin{itemize}
	\end{itemize}
}
\item  \\
 in sight/within sight/out of sight \textit{
	\begin{itemize}
	\end{itemize}
}
\item  \\
 in sight/ within sight \textit{
	\begin{itemize}
	\end{itemize}
}
\item  \\
 to lose sight of \textit{
	\begin{itemize}
	\end{itemize}
}
\item  \\
 know someone by sight \textit{
	\begin{itemize}
	\end{itemize}
}
\item  \\
 out of sight, out of mind \textit{
	\begin{itemize}
	\end{itemize}
}
\item  \\
 on sight \textit{
	\begin{itemize}
	\end{itemize}
}
\item  \\
 not a pretty sight \textit{
	\begin{itemize}
	\end{itemize}
}
\item  \\
 set one's sights on something \textit{
	\begin{itemize}
	\end{itemize}
}
\item  \\
 sight unseen \textit{
	\begin{itemize}
	\end{itemize}
}
\end{enumerate}

\section*{pregnant}
{\large \color{blue}  }
\subsection*{Explain}
\begin{enumerate}
\item adjective \\
If a woman or female animal is \textbf{pregnant} , she has a baby or babies developing in her body.
 \textit{
	\begin{itemize}
	\item She got pregnant soon after they married.
	\item Tina was pregnant with her first child.
	\item ...a pregnant woman.
	\end{itemize}
}
\item adjective \\
A \textbf{pregnant}  silence or moment has a special meaning which is not obvious but which people are aware of.
 \textit{
	\begin{itemize}
	\item There was a long, pregnant silence.
	\item ...a deceptive peace, pregnant with invisible threats.
	\end{itemize}
}
\end{enumerate}

\section*{sister}
{\large \color{blue}  sisters  }
\subsection*{Explain}
\begin{enumerate}
\item countable noun \\
Your \textbf{sister} is a girl or woman who has the same parents as you.
 \textit{
	\begin{itemize}
	\item His sister Sarah helped him.
	\item ...Vanessa Bell, the sister of Virginia Woolf.
	\item I didn't know you had a sister.
	\end{itemize}
}
\item countable noun \\
\textbf{Sister} is a title given to a woman who belongs to a religious community .
 \textit{
	\begin{itemize}
	\item Sister Francesca entered the chapel.
	\item ...the Hospice of the Sisters of Charity at Lourdes.
	\end{itemize}
}
\item countable noun \\
A \textbf{sister} is a senior female nurse who supervises part of a hospital .
 \textit{
	\begin{itemize}
	\item Ask to speak to the sister on the ward.
	\item Sister Middleton followed the coffee trolley.
	\end{itemize}
}
\item countable noun \\
You can  describe a woman as your \textbf{sister} if you feel a connection with her, for example because she belongs to the same race , religion , country , or profession .
 \textit{
	\begin{itemize}
	\item Modern woman has been freed from many of the duties that befell her sisters in times
past.
	\item ...our Jewish brothers and sisters.
	\end{itemize}
}
\item adjective \\
You can use \textbf{sister} to describe something that is of the same type or is connected in some way to another thing you have mentioned . For example, if a company has a \textbf{sister} company, they are connected.
 \textit{
	\begin{itemize}
	\item ...the International Monetary Fund and its sister organisation, the World Bank.
	\item ...Voyager 2 and its sister ship, Voyager 1.
	\end{itemize}
}
\end{enumerate}

\section*{pure}
{\large \color{blue}  purer  purest  }
\subsection*{Explain}
\begin{enumerate}
\item adjective \\
A \textbf{pure} substance is not mixed with anything else.
 \textit{
	\begin{itemize}
	\item ...a carton of pure orange juice.
	\end{itemize}
}
\item adjective \\
Something that is \textbf{pure} is clean and does not contain any harmful substances.
 \textit{
	\begin{itemize}
	\item In remote regions, the air is pure and the crops are free of poisonous insecticides.
	\item ...demands for purer and cleaner river water.
	\end{itemize}
}
\item adjective \\
If you describe something such as a colour, a sound, or a type of light as \textbf{pure} , you mean that it is very clear and represents a perfect  example of its type.
 \textit{
	\begin{itemize}
	\item ...flowers in a whole range of blues with the occasional pure white.
	\end{itemize}
}
\item adjective \\
If you describe a form of art or a philosophy as \textbf{pure} , you mean that it is produced or practised  according to a standard or form that is expected of it.
 \textit{
	\begin{itemize}
	\item Nicholson never swerved from his aim of making pure and simple art.
	\end{itemize}
}
\item adjective \\
\textbf{Pure}  science or \textbf{pure}  research is concerned only with theory and not with how this theory can be used in practical
ways.
 \textit{
	\begin{itemize}
	\item Physics isn't just about pure science with no immediate applications.
	\item They did not approach their subject solely as a matter of 'pure' theory.
	\end{itemize}
}
\item adjective \\
\textbf{Pure} means complete and total .
 \textit{
	\begin{itemize}
	\item The old man turned to give her a look of pure surprise.
	\item To sleep on my own and not hear the boys snore or grunt was pure bliss.
	\end{itemize}
}
\item graded adjective \\
A person, especially a woman, who is described as \textbf{pure} is considered to be morally good, especially because they have no sexual  experience or sexual thoughts .
 \textit{
	\begin{itemize}
	\item She was baptized and she was pure and clean of sin.
	\end{itemize}
}
\item  \\
 pure and simple \textit{
	\begin{itemize}
	\end{itemize}
}
\end{enumerate}

\section*{specification}
{\large \color{blue}  specifications  }
\subsection*{Explain}
\begin{enumerate}
\item countable noun \\
A \textbf{specification} is a requirement which is clearly stated, for example about the necessary  features in the design of something.
 \textit{
	\begin{itemize}
	\item Troll's exclusive, personalized luggage is made to our own exacting specifications
in heavy-duty PVC/nylon.
	\item Legislation will require U.K. petrol companies to meet an E.U. specification for
petrol.
	\item ...officials constrained by rigid job specifications.
	\end{itemize}
}
\end{enumerate}

\section*{stranger}
{\large \color{blue}  strangers  }
\subsection*{Explain}
\begin{enumerate}
\item countable noun \\
A \textbf{stranger} is someone you have never  met before.
 \textit{
	\begin{itemize}
	\item Telling a complete stranger about your life is difficult.
	\item Sometimes I feel like I'm living with a stranger.
	\end{itemize}
}
\item plural noun \\
If two people are \textbf{strangers} , they do not know each other.
 \textit{
	\begin{itemize}
	\item The women knew nothing of the dead girl. They were strangers.
	\end{itemize}
}
\item countable noun \\
If you are a \textbf{stranger} in a place, you do not know the place well .
 \textit{
	\begin{itemize}
	\item 'You don't know much about our town, do you?'—'No, I'm a stranger here.'
	\end{itemize}
}
\item countable noun \\
If you are a \textbf{stranger}  \textbf{to} something, you have had no experience of it or do not understand it.
 \textit{
	\begin{itemize}
	\item He is no stranger to controversy.
	\item We were both strangers to diplomatic life.
	\end{itemize}
}
\end{enumerate}

\section*{rapid}
{\large \color{blue}  }
\subsection*{Explain}
\begin{enumerate}
\item adjective \\
A \textbf{rapid} change is one that happens very quickly.
 \textit{
	\begin{itemize}
	\item ...the country's rapid economic growth in the 1980s.
	\item This signals a rapid change of mind by the government.
	\item ...the rapid decline in the birth rate.
	\end{itemize}
}
\item adjective \\
A \textbf{rapid} movement is one that is very fast.
 \textit{
	\begin{itemize}
	\item He walked at a rapid pace along Charles Street.
	\item ...the St John Ambulance Air Wing, formed to provide for the rapid transport of
patients in urgent need of specialist attention.
	\item Breathing becomes more rapid and sweating starts.
	\end{itemize}
}
\end{enumerate}

\section*{string}
{\large \color{blue}  strings  stringing  strung  }
\subsection*{Explain}
\begin{enumerate}
\item variable noun \\
\textbf{String} is thin rope made of twisted threads, used for tying things together or tying up parcels .
 \textit{
	\begin{itemize}
	\item He held out a small bag tied with string.
	\item ...a shiny metallic coin on a string.
	\end{itemize}
}
\item countable noun \\
A \textbf{string}  \textbf{of} things is a number of them on a piece of string, thread, or wire.
 \textit{
	\begin{itemize}
	\item She wore a string of pearls around her neck.
	\item ...a string of fairy lights.
	\end{itemize}
}
\item countable noun \\
A \textbf{string}  \textbf{of} places or objects is a number of them that form a line.
 \textit{
	\begin{itemize}
	\item The landscape is broken only by a string of villages.
	\item A string of five rowing boats set out from the opposite bank.
	\end{itemize}
}
\item countable noun \\
A \textbf{string}  \textbf{of} similar events is a series of them that happen one after the other.
 \textit{
	\begin{itemize}
	\item The incident was the latest in a string of attacks.
	\item Between 1940 and 1943 he had a string of 62 consecutive victories.
	\end{itemize}
}
\item countable noun \\
The \textbf{strings} on a musical instrument such as a violin or guitar are the thin pieces of wire or
 nylon stretched across it that make sounds when the instrument is played.
 \textit{
	\begin{itemize}
	\item He went off to change a guitar string.
	\item ...a twenty-one-string harp.
	\end{itemize}
}
\item plural noun \\
\textbf{The}  \textbf{strings} are the section of an orchestra which consists of stringed instruments played with a bow .
 \textit{
	\begin{itemize}
	\item The strings provided a melodic background to the passages played by the soloist.
	\item There was a 20-member string section.
	\end{itemize}
}
\item countable noun \\
In computing , a \textbf{string} is a particular series of letters, numbers, symbols, or spaces, for example a word
or phrase that you want to search for in a document.
 \textit{
	\begin{itemize}
	\end{itemize}
}
\item verb \\
If you \textbf{string} something somewhere , you hang it up between two or more objects.
 \textbf{String up} means the same as string .
 \textit{
	\begin{itemize}
	\item He had strung a banner across the wall.
	\item People were stringing up decorations on the fronts of their homes.
	\end{itemize}
}
\item  \\
 a string to one's bow \textit{
	\begin{itemize}
	\end{itemize}
}
\item  \\
 no strings \textit{
	\begin{itemize}
	\end{itemize}
}
\item  \\
 to pull strings \textit{
	\begin{itemize}
	\end{itemize}
}
\end{enumerate}

\section*{realistic}
{\large \color{blue}  }
\subsection*{Explain}
\begin{enumerate}
\item adjective \\
If you are \textbf{realistic} about a situation , you recognize and accept its true  nature and try to deal with it in a practical way.
 \textit{
	\begin{itemize}
	\item Police have to be realistic about violent crime.
	\item It's only realistic to acknowledge that something, some time, will go wrong.
	\item ...a realistic view of what we can afford.
	\end{itemize}
}
\item adjective \\
Something such as a goal or target that is \textbf{realistic} is one which you can sensibly expect to achieve .
 \textit{
	\begin{itemize}
	\item Elections are scheduled for next year but many doubt this is a realistic goal.
	\item A more realistic figure is eleven million.
	\item Establish deadlines that are more realistic.
	\end{itemize}
}
\item adjective \\
You say that a painting , story , or film is \textbf{realistic} when the people and things in it are like people and things in real life.
 \textit{
	\begin{itemize}
	\item ...extraordinarily realistic paintings of Indians.
	\item The language is foul and the violence horribly realistic.
	\end{itemize}
}
\end{enumerate}

\section*{suggestion}
{\large \color{blue}  suggestions  }
\subsection*{Explain}
\begin{enumerate}
\item countable noun \\
If you make a \textbf{suggestion} , you put  forward an idea or plan for someone to think about.
 \textit{
	\begin{itemize}
	\item The dietitian was helpful, making suggestions as to how I could improve my diet.
	\item Perhaps he'd followed her suggestion of a stroll to the river.
	\item I have lots of suggestions for the park's future.
	\end{itemize}
}
\item countable noun \\
A \textbf{suggestion} is something that a person says which implies that something is the case .
 \textit{
	\begin{itemize}
	\item We reject any suggestion that the law needs amending.
	\item There are suggestions that he might be supported by the Socialists.
	\end{itemize}
}
\item singular noun \\
If there is no \textbf{suggestion}  \textbf{that} something is the case, there is no reason to think that it is the case.
 \textit{
	\begin{itemize}
	\item There is no suggestion whatsoever that the two sides are any closer to agreeing.
	\item There is absolutely no suggestion of any mainstream political party involvement.
	\end{itemize}
}
\item countable noun \\
If there is a \textbf{suggestion}  \textbf{of} something, there is a slight amount or sign of it.
 \textit{
	\begin{itemize}
	\item ...that fashionably faint suggestion of a tan.
	\item ...a firm, well-sprung mattress with not one suggestion of a sag.
	\end{itemize}
}
\item uncountable noun \\
\textbf{Suggestion} means giving people a particular idea by associating it with other ideas.
 \textit{
	\begin{itemize}
	\item The power of suggestion is very strong.
	\end{itemize}
}
\end{enumerate}

\section*{sheer}
{\large \color{blue}  sheerer  sheerest  }
\subsection*{Explain}
\begin{enumerate}
\item adjective \\
You can use \textbf{sheer} to emphasize that a state or situation is complete and does not involve or is not mixed with anything else.
 \textit{
	\begin{itemize}
	\item His music is sheer delight.
	\item Sheer chance quite often plays an important part in sparking off an idea.
	\item ...acts of sheer desperation.
	\end{itemize}
}
\item adjective \\
A \textbf{sheer}  cliff or drop is extremely steep or completely vertical .
 \textit{
	\begin{itemize}
	\item There was a sheer drop just outside my window.
	\item A young man plunged from a sheer rock face to his death.
	\end{itemize}
}
\item adjective \\
\textbf{Sheer} material is very thin, light, and delicate .
 \textit{
	\begin{itemize}
	\item ...sheer black tights.
	\end{itemize}
}
\end{enumerate}

\section*{surname}
{\large \color{blue}  surnames  }
\subsection*{Explain}
\begin{enumerate}
\item countable noun \\
Your \textbf{surname} is the name that you share with other members of your family. In English  speaking  countries and many other countries it is your last name.
 \textit{
	\begin{itemize}
	\item She'd never known his surname.
	\item Although they share a surname they are not related.
	\end{itemize}
}
\end{enumerate}

\section*{skeptical}
{\large \color{blue}  }
\subsection*{Explain}
\begin{enumerate}
\end{enumerate}

\section*{sceptical}
{\large \color{blue}  }
\subsection*{Explain}
\begin{enumerate}
\item adjective \\
If you are \textbf{sceptical}  \textbf{about} something, you have doubts about it.
 \textit{
	\begin{itemize}
	\item Other archaeologists are sceptical about his findings.
	\item ...scientists who are sceptical of global warming and its alleged consequences.
	\item The party has always had a cautious and sceptical attitude to Europe.
	\end{itemize}
}
\end{enumerate}

\section*{tenant}
{\large \color{blue}  tenants  }
\subsection*{Explain}
\begin{enumerate}
\item countable noun \\
A \textbf{tenant} is someone who pays rent for the place they live in, or for land or buildings that they use.
 \textit{
	\begin{itemize}
	\item Regulations placed clear obligations on the landlord for the benefit of the tenant.
	\item Landowners frequently left the management of their estates to tenant farmers.
	\end{itemize}
}
\end{enumerate}

\section*{social}
{\large \color{blue}  socials  }
\subsection*{Explain}
\begin{enumerate}
\item adjective \\
\textbf{Social} means relating to society or to the way society is organized.
 \textit{
	\begin{itemize}
	\item ...the worst effects of unemployment, low pay and other social problems.
	\item ...long-term social change.
	\item ...the acceptance that social conditions influenced crime.
	\item ...changing social attitudes.
	\item ...the tightly woven social fabric of small towns.
	\item ...research into housing and social policy.
	\end{itemize}
}
\item adjective \\
\textbf{Social} means relating to the status or rank that someone has in society.
 \textit{
	\begin{itemize}
	\item Higher education is unequally distributed across social classes.
	\item The guests came from all social backgrounds.
	\item Morisot and Degas moved in the same social circles.
	\item ...a prosperous upper-middle-class couple with social aspirations.
	\end{itemize}
}
\item adjective \\
\textbf{Social} means relating to leisure activities that involve meeting other people.
 \textit{
	\begin{itemize}
	\item We ought to organize more social events.
	\item Social activities might include walking tours of the Old Town.
	\end{itemize}
}
\item countable noun \\
A \textbf{social} is a party , dance , or informal gathering that is organized for the members of a club or institution .
 \textit{
	\begin{itemize}
	\item ...church socials.
	\end{itemize}
}
\item adjective \\
\textbf{Social} animals live in groups and do things together.
 \textit{
	\begin{itemize}
	\item These endangered gentle giants are highly social animals.
	\item ...social insects like bees and ants.
	\end{itemize}
}
\end{enumerate}

\section*{thread}
{\large \color{blue}  threads  threading  threaded  }
\subsection*{Explain}
\begin{enumerate}
\item variable noun \\
\textbf{Thread} or a \textbf{thread} is a long very thin piece of a material such as cotton, nylon , or silk , especially one that is used in sewing.
 \textit{
	\begin{itemize}
	\item This time I'll do it properly with a needle and thread.
	\item ...a tiny Nepalese hat embroidered with golden threads.
	\end{itemize}
}
\item countable noun \\
The \textbf{thread} of an argument , a story , or a situation is an aspect of it that connects all the different parts together.
 \textit{
	\begin{itemize}
	\item The thread running through many of these proposals was the theme of opportunity.
	\item All religions are united by the common threads of fighting evil and helping others.
	\item The possible consequences so filled his mind that he lost the thread of Wan Da's
narrative.
	\end{itemize}
}
\item countable noun \\
A \textbf{thread}  \textbf{of} something such as liquid, light, or colour is a long thin line or piece of it.
 \textit{
	\begin{itemize}
	\item A thin, glistening thread of moisture ran along the rough concrete sill.
	\item ...Venetian glass decorated with embedded threads of white.
	\item ...a corpulent man with threads of black hair plastered across his brow.
	\end{itemize}
}
\item plural noun \\
You can refer to clothes as \textbf{threads} .
 \textit{
	\begin{itemize}
	\item ...a cheap place to pick up natty threads.
	\end{itemize}
}
\item countable noun \\
The \textbf{thread} on a screw, or on something such as a lid or a pipe , is the raised spiral line of metal or plastic around it which allows it to be fixed in place by twisting.
 \textit{
	\begin{itemize}
	\item The screw threads will be able to get a good grip.
	\end{itemize}
}
\item countable noun \\
On the internet, a \textbf{thread} is a series of messages from different people about a particular subject.
 \textit{
	\begin{itemize}
	\item I saw the post but I didn't read the thread below it.
	\end{itemize}
}
\item verb \\
If you \textbf{thread} your \textbf{way} through a group of people or things, or \textbf{thread}  \textbf{through} it, you move through it carefully or slowly, changing direction frequently as you
move.
 \textit{
	\begin{itemize}
	\item Slowly, she threaded her way back through the moving mass of bodies.
	\item ...threading our way past little boats.
	\item We threaded through a network of back streets.
	\end{itemize}
}
\item verb \\
If you \textbf{thread} a long thin object \textbf{through} something, you pass it through one or more holes or narrow spaces.
 \textit{
	\begin{itemize}
	\item ...threading the laces through the eyelets of his shoes.
	\item Air ducts and electrical cables were threaded through the complex structure.
	\item These instruments allow doctors to thread microscopic telescopes into the digestive
tract.
	\end{itemize}
}
\item verb \\
If you \textbf{thread} small objects such as beads onto a string or thread, you join them together by pushing the string through them.
 \textit{
	\begin{itemize}
	\item Wipe the mushrooms clean and thread them on a string.
	\end{itemize}
}
\item verb \\
When you \textbf{thread} a needle , you put a piece of thread through the hole in the top of the needle in order to
sew with it.
 \textit{
	\begin{itemize}
	\item I sit down, thread a needle, snip off an old button.
	\end{itemize}
}
\item  \\
 to hang by a thread \textit{
	\begin{itemize}
	\end{itemize}
}
\item  \\
 pick up the threads of sth \textit{
	\begin{itemize}
	\end{itemize}
}
\end{enumerate}

\section*{such}
{\large \color{blue}  }
\subsection*{Explain}
\begin{enumerate}
\item determiner \\
You use \textbf{such} to refer back to the thing or person that you have just mentioned , or a thing or person like the one that you have just mentioned. You use \textbf{such as} and \textbf{such...as} to introduce a reference to the person or thing that has just been mentioned.
 \textbf{Such} is also a predeterminer.
 \textbf{Such} is also used before \textbf{be} .
 \textbf{As such} is also used.
 \textbf{Such as} is also used.
 \textit{
	\begin{itemize}
	\item There have been previous attempts at coups. We regard such methods as entirely unacceptable.
	\item You're being made to choose. Such choices as this are a by-product of freedom.
	\item There'd be no telling how John would react to such news as this.
	\item If your request is for information about a child, please contact the Registrar to
find out how to make such a request.
	\item She told us her family make her pay rent. We could not believe such a thing.
	\item How can we make sense of such a story as this?
	\item We are scared because we are being watched–such is the atmosphere in Pristina and
other cities in Kosovo.
	\item There should be a law ensuring products tested on animals have to be labelled as
such.
	\item Issues such as these were not really his concern.
	\item I wouldn't see another chance such as this in my lifetime.
	\end{itemize}
}
\item determiner \\
You use \textbf{such...as} to link something or someone with a clause in which you give a description of the kind of thing or person that you mean.
 \textbf{Such as} is also used.
 \textit{
	\begin{itemize}
	\item Each member of the alliance agrees to take such action as it deems necessary.
	\item Britain is not enjoying such prosperity as it was in the mid-1980s.
	\item Children do not use inflections such as are used in mature adult speech.
	\end{itemize}
}
\item determiner \\
You use \textbf{such...as} to introduce one or more examples of the kind of thing or person that you have just mentioned.
 \textbf{Such as} is also used.
 \textit{
	\begin{itemize}
	\item He was said to have written such books as The Day of Locusts and Miss Lovely Hearts.
	\item ...such careers as teaching, nursing, hairdressing and catering.
	\item ...delays caused by such things as bad weather or industrial disputes.
	\item ...serious offences, such as assault on a police officer.
	\item He definitely wants to perform further tests, such as a biopsy and some x-rays.
	\item When I get tired, such as when I'm working on my computer, I turn to biscuits.
	\end{itemize}
}
\item determiner \\
You use \textbf{such} before noun groups to emphasize the extent of something or to emphasize that something is remarkable .
 \textbf{Such} is also a predeterminer.
 \textit{
	\begin{itemize}
	\item I think most of us don't want to read what's in the newspaper anyway in such detail.
	\item One will never be able to understand why these political issues can acquire such
force.
	\item The economy was not in such bad shape, he says.
	\item You know the health service is in such a state and it's getting desperate now.
	\item He had such a way with the ladies.
	\item It was such a pleasant surprise.
	\item He's such a sweet boy, isn't he.
	\end{itemize}
}
\item predeterminer \\
You use \textbf{such...that} in order to emphasize the degree of something by mentioning the result or consequence of it.
 \textbf{Such} is also a determiner .
 \textbf{Such} is also used after \textbf{be} .
 \textit{
	\begin{itemize}
	\item His tongue swelled up to such a size that he could no longer speak clearly.
	\item These changes take place over such a long time that we don't see them happening.
	\item He was in such a hurry that he almost pushed me over on the stairs.
	\item She looked at him in such distress that he had to look away.
	\item Though Vivaldi had earned a great deal in his lifetime, his extravagance was such
that he died in poverty.
	\item He kept thinking the pain was such that he would faint.
	\end{itemize}
}
\item determiner \\
You use \textbf{such...that} or \textbf{such...as} in order to say what the result or consequence of something that you have just mentioned is.
 \textbf{Such} is also a predeterminer.
 \textbf{Such} is also used after \textbf{be} .
 \textit{
	\begin{itemize}
	\item The operation uncovered such dealing in stolen property that police pressed for changes
in the law.
	\item He could put an idea in such a way that Alan would believe it was his own.
	\item OFSTED's brief is such that it can conduct any inquiry or provide any advice which
the Secretary of State requires.
	\end{itemize}
}
\item  \\
 such and such \textit{
	\begin{itemize}
	\end{itemize}
}
\item  \\
 such as it is/such as they are \textit{
	\begin{itemize}
	\end{itemize}
}
\item  \\
 as such \textit{
	\begin{itemize}
	\end{itemize}
}
\item  \\
 as such \textit{
	\begin{itemize}
	\end{itemize}
}
\end{enumerate}

\section*{topic}
{\large \color{blue}  topics  }
\subsection*{Explain}
\begin{enumerate}
\item countable noun \\
A \textbf{topic} is a particular subject that you discuss or write about.
 \textit{
	\begin{itemize}
	\item The weather is a constant topic of conversation in Britain.
	\item The main topic for discussion is political union.
	\item They offer tips on topics such as home safety.
	\end{itemize}
}
\end{enumerate}

\section*{superficial}
{\large \color{blue}  }
\subsection*{Explain}
\begin{enumerate}
\item adjective \\
If you describe someone as \textbf{superficial} , you disapprove of them because they do not think deeply, and have little understanding of anything serious or important .
 \textit{
	\begin{itemize}
	\item This guy is a superficial yuppie with no intellect whatsoever.
	\item The tone of his book is consistently negative, occasionally arrogant, and often superficial.
	\end{itemize}
}
\item adjective \\
If you describe something such as an action, feeling , or relationship as \textbf{superficial} , you mean that it includes only the simplest and most obvious  aspects of that thing, and not those aspects which require more effort to deal with or understand .
 \textit{
	\begin{itemize}
	\item Their arguments do not withstand the most superficial scrutiny.
	\item His roommate had been pleasant on a superficial level.
	\item Father had no more than a superficial knowledge of music.
	\end{itemize}
}
\item adjective \\
\textbf{Superficial} is used to describe the appearance of something or the impression that it gives, especially if its real  nature is very different .
 \textit{
	\begin{itemize}
	\item Despite these superficial resemblances, this is a darker work than her earlier novels.
	\item Spain may well look different but the changes are superficial.
	\end{itemize}
}
\item adjective \\
\textbf{Superficial}  injuries are not very serious, and affect only the surface of the body. You can also describe damage to an object as \textbf{superficial} .
 \textit{
	\begin{itemize}
	\item The 69-year-old clergyman escaped with superficial wounds.
	\item The explosion caused superficial damage to the fortified house.
	\end{itemize}
}
\item adjective \\
The \textbf{superficial}  layers of the skin are the ones nearest the surface.
 \textit{
	\begin{itemize}
	\item ...superficial blood vessels in the forearm.
	\end{itemize}
}
\end{enumerate}

\section*{trick}
{\large \color{blue}  tricks  tricking  tricked  }
\subsection*{Explain}
\begin{enumerate}
\item countable noun \\
A \textbf{trick} is an action that is intended to deceive someone.
 \textit{
	\begin{itemize}
	\item We are playing a trick on a man who keeps bothering me.
	\end{itemize}
}
\item verb \\
If someone \textbf{tricks} you, they deceive you, often in order to make you do something.
 \textit{
	\begin{itemize}
	\item Stephen is going to be pretty upset when he finds out how you tricked him.
	\item She was said to have tricked him into going to a warehouse at night in the hope of
securing a lucrative deal.
	\item His real purpose is to trick his way into your home to see what he can steal.
	\end{itemize}
}
\item countable noun \\
A \textbf{trick} is a clever or skilful action that someone does in order to entertain people.
 \textit{
	\begin{itemize}
	\item He shows me card tricks.
	\end{itemize}
}
\item countable noun \\
A \textbf{trick} is a clever way of doing something.
 \textit{
	\begin{itemize}
	\item Tiffany revamped her sitting room with simple decorative tricks.
	\item It is not just a little trick you can pick up in half an hour.
	\end{itemize}
}
\item  \\
 do the trick \textit{
	\begin{itemize}
	\end{itemize}
}
\item  \\
 every trick in the book \textit{
	\begin{itemize}
	\end{itemize}
}
\item  \\
 a trick of the light \textit{
	\begin{itemize}
	\end{itemize}
}
\item  \\
 sb doesn't miss a trick \textit{
	\begin{itemize}
	\end{itemize}
}
\item  \\
 tricks of the trade \textit{
	\begin{itemize}
	\end{itemize}
}
\item  \\
 up to one's tricks \textit{
	\begin{itemize}
	\end{itemize}
}
\end{enumerate}

\section*{turbulent}
{\large \color{blue}  }
\subsection*{Explain}
\begin{enumerate}
\item adjective \\
A \textbf{turbulent} time, place, or relationship is one in which there is a lot of change, confusion , and disorder .
 \textit{
	\begin{itemize}
	\item They had been together for five or six turbulent years of rows and reconciliations.
	\item The minister announced that he was taking a rest from the turbulent world of politics.
	\end{itemize}
}
\item adjective \\
\textbf{Turbulent} water or air contains strong  currents which change direction  suddenly .
 \textit{
	\begin{itemize}
	\item I had to have a boat that could handle turbulent seas.
	\end{itemize}
}
\end{enumerate}

\section*{view}
{\large \color{blue}  views  viewing  viewed  }
\subsection*{Explain}
\begin{enumerate}
\item countable noun \\
Your \textbf{views} on something are the beliefs or opinions that you have about it, for example whether you think it is good, bad , right, or wrong .
 \textit{
	\begin{itemize}
	\item Neither of them had strong views on politics.
	\item I take the view that she should be stopped as soon as possible.
	\item My own view is absolutely clear. What I did was right.
	\item You should also make your views known to your local MP.
	\end{itemize}
}
\item singular noun \\
Your \textbf{view}  \textbf{of} a particular subject is the way that you understand and think about it.
 \textit{
	\begin{itemize}
	\item The drama takes an idealistic, even a naive view of the subject.
	\item The whole point was to get away from a Christian-centred view of religion.
	\item In the old animistic world view, people believed that nature was organised by invisible
souls.
	\end{itemize}
}
\item verb \\
If you \textbf{view} something in a particular way, you think of it in that way.
 \textit{
	\begin{itemize}
	\item First-generation Americans view the United States as a land of golden opportunity.
	\item Abigail's mother Linda views her daughter's talent with a mixture of pride and worry.
	\item Sectors in the economy can be viewed in a variety of ways.
	\item We would view favourably any sensible suggestion for maintaining the business.
	\end{itemize}
}
\item countable noun \\
The \textbf{view} from a window or high place is everything which can be seen from that place, especially when it is considered to be beautiful .
 \textit{
	\begin{itemize}
	\item The view from our window was one of beautiful green countryside.
	\item Each of the rooms has a superb view of Pissouri Bay.
	\end{itemize}
}
\item singular noun \\
If you have a \textbf{view}  \textbf{of} something, you can see it.
 \textit{
	\begin{itemize}
	\item He stood up to get a better view of the blackboard.
	\item He stopped in the doorway, blocking her view.
	\end{itemize}
}
\item uncountable noun \\
You use \textbf{view} in expressions to do with being able to see something. For example, if something is \textbf{in view} , you can see it. If something is \textbf{in full view of everyone} , everyone can see it.
 \textit{
	\begin{itemize}
	\item She was lying there in full view of anyone who walked by.
	\item A group of riders came into view on the dirt road.
	\item On South Main Street, a huge brick building looms into view.
	\end{itemize}
}
\item verb \\
If you \textbf{view} something, you look at it for a particular purpose.
 \textit{
	\begin{itemize}
	\item They came back to view the house again.
	\item Twenty-five thousand mourners passed to view the body.
	\end{itemize}
}
\item verb \\
If you \textbf{view} a television programme , video , or film, you watch it.
 \textit{
	\begin{itemize}
	\item We have viewed the footage of the incident.
	\item 'Elizabeth R', a TV portrait of the Queen, had record viewing figures.
	\end{itemize}
}
\item uncountable noun \\
\textbf{View}  refers to the way in which a piece of text or graphics is displayed on a computer screen .
 \textit{
	\begin{itemize}
	\item To see the current document in full-page view, click the Page Zoom Full button.
	\end{itemize}
}
\item  \\
 to take a dim view \textit{
	\begin{itemize}
	\end{itemize}
}
\item  \\
 in my view \textit{
	\begin{itemize}
	\end{itemize}
}
\item  \\
 in view of \textit{
	\begin{itemize}
	\end{itemize}
}
\item  \\
 in view \textit{
	\begin{itemize}
	\end{itemize}
}
\item  \\
 to take the long view \textit{
	\begin{itemize}
	\end{itemize}
}
\item  \\
 on view \textit{
	\begin{itemize}
	\end{itemize}
}
\item  \\
 with a view to \textit{
	\begin{itemize}
	\end{itemize}
}
\end{enumerate}

\section*{date}
{\large \color{blue}  dates  dating  dated  }
\subsection*{Explain}
\begin{enumerate}
\item countable noun \\
A \textbf{date} is a specific time that can be named , for example a particular day or a particular year.
 \textit{
	\begin{itemize}
	\item What's the date today?
	\item You will need to give the dates you wish to stay and the number of rooms you require.
	\end{itemize}
}
\item verb \\
If you \textbf{date} something, you give or discover the date when it was made or when it began .
 \textit{
	\begin{itemize}
	\item You cannot date the carving and it is difficult to date the stone itself.
	\item I think we can date the decline of Western Civilization quite precisely.
	\item Archaeologists have dated the fort to the reign of Emperor Antoninus Pius.
	\end{itemize}
}
\item verb \\
When you \textbf{date} something such as a letter or a cheque , you write that day's date on it.
 \textit{
	\begin{itemize}
	\item Once the decision is reached, he can date and sign the sheet.
	\item The letter is dated 2 July 1993.
	\end{itemize}
}
\item singular noun \\
If you want to refer to an event without saying  exactly when it will  happen or when it happened, you can say that it will happen or happened \textbf{at} some \textbf{date} in the future or past .
 \textit{
	\begin{itemize}
	\item Retain copies of all correspondence, since you may need them at a later date.
	\item He did leave open the possibility of direct American aid at some unspecified date
in the future.
	\item He was content for her wedding to be at some date between July and September.
	\end{itemize}
}
\item  \\
 to date \textit{
	\begin{itemize}
	\end{itemize}
}
\item verb \\
If something \textbf{dates} , it goes out of fashion and becomes unacceptable to modern  tastes .
 \textit{
	\begin{itemize}
	\item A black coat always looks smart and will never date.
	\item This album has hardly dated at all.
	\end{itemize}
}
\item verb \\
If your ideas , what you say, or the things that you like or can remember  \textbf{date} you, they show that you are quite  old or older than the people you are with.
 \textit{
	\begin{itemize}
	\item It's going to date me now. I attended that school from 1969 to 1972.
	\end{itemize}
}
\item countable noun \\
A \textbf{date} is an appointment to meet someone or go out with them, especially someone with whom you are having, or may  soon have, a romantic  relationship .
 \textit{
	\begin{itemize}
	\item I have a date with Bob.
	\item I think we should make a date to go and see Gwendolen soon.
	\end{itemize}
}
\item countable noun \\
If you have a date with someone with whom you are having, or may soon have, a romantic
relationship, you can refer to that person as your \textbf{date} .
 \textit{
	\begin{itemize}
	\item He lied to Essie, saying his date was one of the girls in the show.
	\end{itemize}
}
\item verb \\
If you \textbf{are dating} someone, you go out with them regularly because you are having, or may soon have,
a romantic relationship with them. You can also say that two people \textbf{are dating} .
 \textit{
	\begin{itemize}
	\item For a year I dated a woman who was a research assistant.
	\item They've been dating for three months.
	\item In high school, he did not date very much.
	\end{itemize}
}
\item countable noun \\
A \textbf{date} is a small, dark-brown, sticky fruit with a stone  inside . Dates grow on palm trees in hot  countries .
 \textit{
	\begin{itemize}
	\end{itemize}
}
\end{enumerate}

\section*{vision}
{\large \color{blue}  visions  }
\subsection*{Explain}
\begin{enumerate}
\item countable noun \\
Your \textbf{vision}  \textbf{of} a future situation or society is what you imagine or hope it would be like , if things were very different from the way they are now .
 \textit{
	\begin{itemize}
	\item I have a vision of a society that is free of exploitation and injustice.
	\item That's my vision of how the world could be.
	\item Turning that vision into a practical reality is not easy.
	\end{itemize}
}
\item countable noun \\
If you have a \textbf{vision}  \textbf{of} someone in a particular situation, you imagine them in that situation, for example because you are worried that it might  happen , or hope that it will happen.
 \textit{
	\begin{itemize}
	\item He had a vision of Cheryl, slumped on a plastic chair in the waiting-room.
	\item Maybe you had visions of being surrounded by happy, smiling children.
	\end{itemize}
}
\item countable noun \\
A \textbf{vision} is the experience of seeing something that other people cannot see, for example in
a religious experience or as a result of madness or taking  drugs .
 \textit{
	\begin{itemize}
	\item It was on 24th June 1981 that young villagers first reported seeing the Virgin Mary
in a vision.
	\end{itemize}
}
\item uncountable noun \\
Your \textbf{vision} is your ability to see clearly with your eyes.
 \textit{
	\begin{itemize}
	\item It causes blindness or serious loss of vision.
	\item In spite of his otherwise excellent vision, he found he was colour-blind.
	\end{itemize}
}
\item uncountable noun \\
Your \textbf{vision} is everything that you can see from a particular place or position .
 \textit{
	\begin{itemize}
	\item Jane blocked Cross's vision and he could see nothing.
	\item I saw other indistinct shapes that stayed out of vision.
	\end{itemize}
}
\end{enumerate}

\section*{aware}
{\large \color{blue}  }
\subsection*{Explain}
\begin{enumerate}
\item adjective \\
If you are \textbf{aware}  \textbf{of} something, you know about it.
 \textit{
	\begin{itemize}
	\item Smokers are well aware of the dangers to their own health.
	\item He should have been aware of what his junior officers were doing.
	\item Some people may not be aware that this was a problem.
	\end{itemize}
}
\item adjective \\
If you are \textbf{aware}  \textbf{of} something, you realize that it is present or is happening because you hear it, see it, smell it, or feel it.
 \textit{
	\begin{itemize}
	\item She was acutely aware of the noise of the city.
	\item Jane was suddenly aware that she was digging her nails into her thigh.
	\end{itemize}
}
\item adjective \\
Someone who is \textbf{aware}  notices what is happening around them or happening in the place where they live .
 \textit{
	\begin{itemize}
	\item They are politically very aware.
	\end{itemize}
}
\end{enumerate}

\section*{bright}
{\large \color{blue}  brights  brighter  brightest  }
\subsection*{Explain}
\begin{enumerate}
\item adjective \\
A \textbf{bright} colour is strong and noticeable , and not dark .
 \textit{
	\begin{itemize}
	\item ...a bright red dress.
	\item ...the bright uniforms of the guards parading at Buckingham Palace.
	\end{itemize}
}
\item adjective \\
A \textbf{bright} light, object, or place is shining strongly or is full of light.
 \textit{
	\begin{itemize}
	\item ...a bright October day.
	\item She leaned forward, her eyes bright with excitement.
	\end{itemize}
}
\item adjective \\
If you describe someone as \textbf{bright} , you mean that they are quick at learning things.
 \textit{
	\begin{itemize}
	\item I was convinced that he was brighter than average.
	\end{itemize}
}
\item adjective \\
A \textbf{bright}  idea is clever and original .
 \textit{
	\begin{itemize}
	\item There are lots of books crammed with bright ideas.
	\item Ford had the bright idea of paying workers enough to buy cars.
	\end{itemize}
}
\item adjective \\
If someone looks or sounds \textbf{bright} , they look or sound cheerful and lively .
 \textit{
	\begin{itemize}
	\item The boy was so bright and animated.
	\item 'May I help you?' said a bright American voice over the telephone.
	\end{itemize}
}
\item adjective \\
If the future is \textbf{bright} , it is likely to be pleasant or successful .
 \textit{
	\begin{itemize}
	\item Both had successful careers and the future looked bright.
	\item There are much brighter prospects for a comprehensive settlement than before.
	\end{itemize}
}
\item countable noun \\
\textbf{The}  \textbf{brights} on a car or other vehicle are the headlights when they are switched on fully .
 \textit{
	\begin{itemize}
	\end{itemize}
}
\item  \\
 to look on the bright side \textit{
	\begin{itemize}
	\end{itemize}
}
\end{enumerate}

\section*{brittle}
{\large \color{blue}  }
\subsection*{Explain}
\begin{enumerate}
\item adjective \\
An object or substance that is \textbf{brittle} is hard but easily broken.
 \textit{
	\begin{itemize}
	\item Pine is brittle and breaks.
	\item ...the dry, brittle ends of the hair.
	\end{itemize}
}
\item adjective \\
If you describe a situation , relationship , or someone's mood as \textbf{brittle} , you mean that it is unstable , and may easily change.
 \textit{
	\begin{itemize}
	\item They are nurturing a diplomatic relationship that is dangerously brittle but cannot
be allowed to fail.
	\item This may help to undermine the brittle truce that currently exists.
	\end{itemize}
}
\item graded adjective \\
Someone who is \textbf{brittle}  seems rather sharp and insensitive and says things which are likely to hurt other people's feelings .
 \textit{
	\begin{itemize}
	\item His father, for all his brittle wit, was also a deeply sentimental man.
	\end{itemize}
}
\item graded adjective \\
A \textbf{brittle} sound is short, loud , and sharp.
 \textit{
	\begin{itemize}
	\item Myrtle gave a brittle laugh.
	\end{itemize}
}
\end{enumerate}

\section*{concrete}
{\large \color{blue}  concretes  concreting  concreted  }
\subsection*{Explain}
\begin{enumerate}
\item uncountable noun \\
\textbf{Concrete} is a substance used for building which is made by mixing  together cement, sand, small stones, and water.
 \textit{
	\begin{itemize}
	\item The posts have to be set in concrete.
	\item They had lain on sleeping bags on the concrete floor.
	\item ...concrete barriers.
	\end{itemize}
}
\item verb \\
When you \textbf{concrete} something such as a path , you cover it with concrete.
 \textit{
	\begin{itemize}
	\item He merely cleared and concreted the floors.
	\end{itemize}
}
\item adjective \\
You use \textbf{concrete} to indicate that something is definite and specific.
 \textit{
	\begin{itemize}
	\item He had no concrete evidence.
	\item There were no concrete proposals on the table.
	\item I must have something to tell him. Something concrete.
	\end{itemize}
}
\item adjective \\
A \textbf{concrete} object is a real , physical object.
 \textit{
	\begin{itemize}
	\item ...using concrete objects to teach addition and subtraction.
	\end{itemize}
}
\item adjective \\
A \textbf{concrete} noun is a noun that refers to a physical object rather than to a quality or idea .
 \textit{
	\begin{itemize}
	\end{itemize}
}
\item  \\
 set in concrete/embedded in concrete \textit{
	\begin{itemize}
	\end{itemize}
}
\end{enumerate}

\section*{castle}
{\large \color{blue}  castles  }
\subsection*{Explain}
\begin{enumerate}
\item countable noun \\
A \textbf{castle} is a large building with thick , high walls . Castles were built by important people, such as kings , in former times, especially for protection during wars and battles .
 \textit{
	\begin{itemize}
	\end{itemize}
}
\item countable noun \\
In chess , a \textbf{castle} is a piece that can be moved forwards , backwards , or sideways .
 \textit{
	\begin{itemize}
	\end{itemize}
}
\end{enumerate}

\section*{constant}
{\large \color{blue}  constants  }
\subsection*{Explain}
\begin{enumerate}
\item adjective \\
You use \textbf{constant} to describe something that happens all the time or is always there.
 \textit{
	\begin{itemize}
	\item Inflation is a constant threat.
	\item He has been her constant companion for the last four months.
	\end{itemize}
}
\item adjective \\
If an amount or level is \textbf{constant} , it stays the same over a particular period of time.
 \textit{
	\begin{itemize}
	\item The average speed of the winds remained constant.
	\end{itemize}
}
\item countable noun \\
A \textbf{constant} is a thing or value that always stays the same.
 \textit{
	\begin{itemize}
	\item In the world of fashion it sometimes seems that the only constant is ceaseless change.
	\item Two significant constants have been found in a number of research studies.
	\end{itemize}
}
\end{enumerate}

\section*{cell}
{\large \color{blue}  cells  }
\subsection*{Explain}
\begin{enumerate}
\item countable noun \\
A \textbf{cell} is the smallest part of an animal or plant that is able to function independently.
Every animal or plant is made up of millions of cells.
 \textit{
	\begin{itemize}
	\item Those cells divide and give many other different types of cells.
	\item ...blood cells.
	\item Soap destroys the cell walls of bacteria.
	\end{itemize}
}
\item countable noun \\
A \textbf{cell} is a small room in which a prisoner is locked . A \textbf{cell} is also a small room in which a monk or nun lives.
 \textit{
	\begin{itemize}
	\end{itemize}
}
\item countable noun \\
You can refer to a small group of people within a larger organization as a \textbf{cell} .
 \textit{
	\begin{itemize}
	\item ...Communist Party cells.
	\end{itemize}
}
\end{enumerate}

\section*{costly}
{\large \color{blue}  costlier  costliest  }
\subsection*{Explain}
\begin{enumerate}
\item adjective \\
If you say that something is \textbf{costly} , you mean that it costs a lot of money, often more than you would want to pay.
 \textit{
	\begin{itemize}
	\item Having professionally-made curtains can be costly, so why not make your own?
	\end{itemize}
}
\item adjective \\
If you describe someone's action or mistake as \textbf{costly} , you mean that it results in a serious  disadvantage for them, for example the loss of a large amount of money or the loss of their reputation .
 \textit{
	\begin{itemize}
	\item Psychometric tests can save organizations from grim and costly mistakes.
	\item This sort of scandal in international banking has been politically costly.
	\end{itemize}
}
\end{enumerate}

\section*{challenge}
{\large \color{blue}  challenges  challenging  challenged  }
\subsection*{Explain}
\begin{enumerate}
\item variable noun \\
A \textbf{challenge} is something new and difficult which requires great effort and determination .
 \textit{
	\begin{itemize}
	\item I like a big challenge and they don't come much bigger than this.
	\item The new government's first challenge is the economy.
	\end{itemize}
}
\item  \\
 to rise to the challenge \textit{
	\begin{itemize}
	\end{itemize}
}
\item variable noun \\
A \textbf{challenge}  \textbf{to} something is a questioning of its truth or value. A \textbf{challenge}  \textbf{to} someone is a questioning of their authority.
 \textit{
	\begin{itemize}
	\item The demonstrators have now made a direct challenge to the authority of the government.
	\end{itemize}
}
\item verb \\
If you \textbf{challenge}  ideas or people, you question their truth, value, or authority.
 \textit{
	\begin{itemize}
	\item Democratic leaders have challenged the president to sign the bill.
	\item The move was immediately challenged by two of the republics.
	\item I challenged him on the hypocrisy of his political attitudes.
	\end{itemize}
}
\item verb \\
If you \textbf{challenge} someone, you invite them to fight or compete with you in some way.
 \textbf{Challenge} is also a noun .
 \textit{
	\begin{itemize}
	\item A mum challenged her to a fight after their daughters fell out.
	\item He left a note at the scene of the crime, challenging detectives to catch him.
	\item We challenged a team who called themselves 'College Athletes'.
	\item A third presidential candidate emerged to mount a serious challenge.
	\end{itemize}
}
\item verb \\
If someone \textbf{is challenged} by a guard , they are ordered to stop and say who they are or why they are there.
 \textit{
	\begin{itemize}
	\item The men apparently opened fire after they were challenged by a patrol.
	\end{itemize}
}
\end{enumerate}

\section*{dear}
{\large \color{blue}  dearer  dearest  dears  }
\subsection*{Explain}
\begin{enumerate}
\item adjective \\
You use \textbf{dear} to describe someone or something that you feel affection for.
 \textit{
	\begin{itemize}
	\item Mrs Cavendish is a dear friend of mine.
	\item At last I am back at my dear little desk.
	\end{itemize}
}
\item adjective \\
If something is \textbf{dear to} you or \textbf{dear to} your \textbf{heart} , you care deeply about it.
 \textit{
	\begin{itemize}
	\item His family life was very dear to him.
	\item This is a subject very dear to the hearts of academics up and down the country.
	\end{itemize}
}
\item adjective \\
You use \textbf{dear} in expressions such as ' \textbf{my dear fellow} ', ' \textbf{dear girl} ', or ' \textbf{my dear Richard} ' when you are addressing someone whom you know and are fond of. You can also use expressions like this in a rude way to indicate that you think you are superior to the person you are addressing.
 \textit{
	\begin{itemize}
	\item Of course, Toby, my dear fellow, of course.
	\item Take as long as you like, dear boy.
	\end{itemize}
}
\item adjective \\
\textbf{Dear} is written at the beginning of a letter, followed by the name or title of the person you are writing to.
 \textit{
	\begin{itemize}
	\item Dear Peter, I have been thinking about you so much during the past few days.
	\end{itemize}
}
\item convention \\
In British English, you begin  formal letters with ' \textbf{Dear Sir} ' or ' \textbf{Dear Madam} '. In American English, you begin them with 'Sir' or 'Madam'.
 \textit{
	\begin{itemize}
	\item 'Dear sir,' she began.
	\end{itemize}
}
\item countable noun \\
You can call someone \textbf{dear} as a sign of affection.
 \textit{
	\begin{itemize}
	\item You're a lot like me, dear.
	\item 'Good night, my dears,' she called to us as we closed her door behind us.
	\end{itemize}
}
\item exclamation \\
You can use \textbf{dear} in expressions such as ' \textbf{oh dear} ', ' \textbf{dear me} ', and ' \textbf{dear, dear} ' when you are sad , disappointed , or surprised about something.
 \textit{
	\begin{itemize}
	\item 'Oh dear, oh dear.' McKinnon sighed. 'You, too.'
	\item Outside, Bruce glanced at his watch: 'Dear me, nearly one o'clock.'
	\end{itemize}
}
\item countable noun \\
You can call someone a \textbf{dear} when you are fond of them and think that they are nice .
 \textit{
	\begin{itemize}
	\item He's such a dear.
	\end{itemize}
}
\item adjective \\
If you say that something is \textbf{dear} , you mean that it costs a lot of money, usually more than you can afford or more than you think it should cost.
 \textit{
	\begin{itemize}
	\item Clothes here are much dearer than in the States.
	\item They're too dear.
	\end{itemize}
}
\item  \\
 to cost someone dear \textit{
	\begin{itemize}
	\end{itemize}
}
\end{enumerate}

\section*{city}
{\large \color{blue}  cities  }
\subsection*{Explain}
\begin{enumerate}
\item countable noun \\
A \textbf{city} is a large town.
 \textit{
	\begin{itemize}
	\item ...the city of Bologna.
	\item ...a busy city centre.
	\end{itemize}
}
\end{enumerate}

\section*{definite}
{\large \color{blue}  }
\subsection*{Explain}
\begin{enumerate}
\item adjective \\
If something such as a decision or an arrangement is \textbf{definite} , it is firm and clear , and unlikely to be changed.
 \textit{
	\begin{itemize}
	\item It's too soon to give a definite answer.
	\item Her Royal Highness has definite views about most things.
	\item She made no definite plans for her future.
	\end{itemize}
}
\item adjective \\
\textbf{Definite}  evidence or information is true , rather than being someone's opinion or guess .
 \textit{
	\begin{itemize}
	\item We didn't have any definite proof.
	\item If you have any definite news of my husband, please let me know.
	\item The police had nothing definite against her.
	\end{itemize}
}
\item adjective \\
You use \textbf{definite} to emphasize the strength of your opinion or belief .
 \textit{
	\begin{itemize}
	\item There has already been a definite improvement.
	\item That's a very definite possibility.
	\end{itemize}
}
\item adjective \\
Someone who is \textbf{definite}  behaves or talks in a firm, confident way.
 \textit{
	\begin{itemize}
	\item Mary is very definite about this.
	\end{itemize}
}
\item graded adjective \\
A \textbf{definite} shape or colour is clear and noticeable .
 \textit{
	\begin{itemize}
	\item Studying his face in the bathroom mirror he wished he had more definite features.
	\end{itemize}
}
\end{enumerate}

\section*{component}
{\large \color{blue}  components  }
\subsection*{Explain}
\begin{enumerate}
\item countable noun \\
The \textbf{components} of something are the parts that it is made of.
 \textit{
	\begin{itemize}
	\item Enriched uranium is a key component of a nuclear weapon.
	\item The management plan has four main components.
	\item They were automotive component suppliers to motor manufacturers.
	\end{itemize}
}
\item adjective \\
The \textbf{component} parts of something are the parts that make it up.
 \textit{
	\begin{itemize}
	\item First we have to break the system down into its component parts.
	\item They manufacture component parts for engines.
	\end{itemize}
}
\end{enumerate}

\section*{evident}
{\large \color{blue}  }
\subsection*{Explain}
\begin{enumerate}
\item adjective \\
If something is \textbf{evident} , you notice it easily and clearly .
 \textit{
	\begin{itemize}
	\item His footprints were clearly evident in the heavy dust.
	\item The threat of inflation is already evident in bond prices.
	\item ...the best-publicised cases of evident injustice.
	\end{itemize}
}
\item adjective \\
You use \textbf{evident} to show that you are certain about a situation or fact and your interpretation of it.
 \textit{
	\begin{itemize}
	\item It was evident that she had once been a beauty.
	\item The cities are bombarded day after day in an evident effort to force their surrender.
	\end{itemize}
}
\end{enumerate}

\section*{detail}
{\large \color{blue}  details  detailing  detailed  }
\subsection*{Explain}
\begin{enumerate}
\item countable noun \\
The \textbf{details}  \textbf{of} something are its individual features or elements.
 \textit{
	\begin{itemize}
	\item The details of the plan are still being worked out.
	\item No details of the discussions have been given.
	\item I recall every detail of the party.
	\end{itemize}
}
\item plural noun \\
\textbf{Details} about someone or something are facts or pieces of information about them.
 \textit{
	\begin{itemize}
	\item See the bottom of this page for details of how to apply for this exciting offer.
	\item Full details will be announced soon.
	\end{itemize}
}
\item countable noun \\
A \textbf{detail} is a minor point or aspect of something, as opposed to the central ones.
 \textit{
	\begin{itemize}
	\item Only minor details now remain to be settled.
	\end{itemize}
}
\item uncountable noun \\
You can refer to the small features of something which are often not noticed as \textbf{detail} .
 \textit{
	\begin{itemize}
	\item We like his attention to detail and his enthusiasm.
	\end{itemize}
}
\item countable noun \\
A \textbf{detail} of a picture is a part of it that is printed separately and perhaps made bigger , so that smaller features can be clearly  seen .
 \textit{
	\begin{itemize}
	\end{itemize}
}
\item verb \\
If you \textbf{detail} things, you list them or give information about them.
 \textit{
	\begin{itemize}
	\item The report detailed the human rights abuses committed during the war.
	\end{itemize}
}
\item verb \\
If someone \textbf{is detailed}  \textbf{to} do a task or job , they are officially ordered to do it.
 \textit{
	\begin{itemize}
	\item He detailed a constable to take it to the Incident Room.
	\end{itemize}
}
\item  \\
 go into details \textit{
	\begin{itemize}
	\end{itemize}
}
\item  \\
 in detail \textit{
	\begin{itemize}
	\end{itemize}
}
\end{enumerate}

\section*{excellent}
{\large \color{blue}  }
\subsection*{Explain}
\begin{enumerate}
\item adjective \\
Something that is \textbf{excellent} is very good indeed.
 \textit{
	\begin{itemize}
	\item The recording quality is excellent.
	\item Luckily, Sue is very efficient and does an excellent job as Fred's personal assistant.
	\end{itemize}
}
\item exclamation \\
Some people say ' \textbf{Excellent!} ' to show that they approve of something.
 \textit{
	\begin{itemize}
	\end{itemize}
}
\end{enumerate}

\section*{duration}
{\large \color{blue}  }
\subsection*{Explain}
\begin{enumerate}
\item uncountable noun \\
The \textbf{duration}  \textbf{of} an event or state is the time during which it happens or exists .
 \textit{
	\begin{itemize}
	\item He was given the task of protecting her for the duration of the trial.
	\item Courses are of two years' duration.
	\end{itemize}
}
\item  \\
 for the duration \textit{
	\begin{itemize}
	\end{itemize}
}
\end{enumerate}

\section*{exotic}
{\large \color{blue}  }
\subsection*{Explain}
\begin{enumerate}
\item adjective \\
Something that is \textbf{exotic} is unusual and interesting, usually because it comes from or is related to a distant country.
 \textit{
	\begin{itemize}
	\item ...brilliantly coloured, exotic flowers.
	\item She flits from one exotic location to another.
	\end{itemize}
}
\end{enumerate}

\section*{economics}
{\large \color{blue}  }
\subsection*{Explain}
\begin{enumerate}
\item uncountable noun \\
\textbf{Economics} is the study of the way in which money, industry , and trade are organized in a society.
 \textit{
	\begin{itemize}
	\item He gained a first class Honours degree in economics.
	\end{itemize}
}
\item uncountable noun \\
The \textbf{economics} of a society or industry is the system of organizing money and trade in it.
 \textit{
	\begin{itemize}
	\item He is regarded as a committed supporter of a radical free-market economics policy.
	\item ...the economics of the third world.
	\end{itemize}
}
\end{enumerate}

\section*{exercise}
{\large \color{blue}  exercises  exercising  exercised  }
\subsection*{Explain}
\begin{enumerate}
\item verb \\
If you \textbf{exercise} something such as your authority, your rights, or a good quality, you use it or put
it into effect.
 \textbf{Exercise} is also a noun .
 \textit{
	\begin{itemize}
	\item They are merely exercising their right to free speech.
	\item He insisted that he would exercise presidential powers to the full.
	\item Britain has warned travellers to exercise prudence and care.
	\item ...the exercise of political and economic power.
	\item Leadership does not rest on the exercise of force alone.
	\end{itemize}
}
\item verb \\
When you \textbf{exercise} , you move your body energetically in order to get fit and to remain  healthy .
 \textbf{Exercise} is also a noun.
 \textit{
	\begin{itemize}
	\item She exercises two or three times a week.
	\item Never keep on exercising if you have even the slightest chest pain.
	\item Exercising the body does a great deal to improve one's health.
	\item Lack of exercise can lead to feelings of depression and exhaustion.
	\item Aerobic exercise moves our entire body and uses most major muscles.
	\end{itemize}
}
\item verb \\
If a movement or activity \textbf{exercises} a part of your body, it keeps it strong , healthy, or in good condition.
 \textit{
	\begin{itemize}
	\item They call rowing the perfect sport. It exercises every major muscle group.
	\end{itemize}
}
\item countable noun \\
\textbf{Exercises} are a series of movements or actions which you do in order to get fit, remain healthy,
or practise for a particular physical activity.
 \textit{
	\begin{itemize}
	\item These stomach exercises will tighten abdominal muscles.
	\item I do special neck and shoulder exercises.
	\item That's when I try to meditate or do some deep-breathing exercises.
	\end{itemize}
}
\item countable noun \\
\textbf{Exercises} are military activities and operations which are not part of a real war, but which allow the armed forces to practise for a real war.
 \textit{
	\begin{itemize}
	\item General Powell predicted that in the future it might even be possible to stage joint
military exercises.
	\item The military truck was taking 14 men on exercise in a remote area of Norway.
	\end{itemize}
}
\item countable noun \\
An \textbf{exercise} is a short activity or piece of work that you do, for example in school, which is designed to help you learn a particular skill.
 \textit{
	\begin{itemize}
	\item Try working through the opening exercises in this chapter.
	\item He took up piano lessons, combining standard classical exercises with his own attempts
at Gershwin.
	\end{itemize}
}
\item plural noun \\
\textbf{Exercises} is an official ceremony, in which people make speeches and present awards .
 \textit{
	\begin{itemize}
	\item Vicky and Gary attended the graduation exercises at Columbia.
	\end{itemize}
}
\item countable noun \\
If you describe an activity as an \textbf{exercise}  \textbf{in} a particular quality or result, you mean that it has that quality or result, especially when it was not intended to have it.
 \textit{
	\begin{itemize}
	\item Her morning was an exercise in indecision. She tried on everything in her closet
but couldn't decide what to wear.
	\item As an exercise in stating the obvious, this could scarcely be faulted.
	\item Think what a waste of taxpayers' money the whole exercise was.
	\end{itemize}
}
\item verb \\
If something \textbf{exercises} you or your mind , you think or talk about it a great deal , especially because you are worried or concerned about it.
 \textit{
	\begin{itemize}
	\item This has been a major problem exercising the minds of scientists around the world.
	\item The proper role of appeal judges is an issue that has long exercised the finest legal
minds.
	\end{itemize}
}
\end{enumerate}

\section*{fortunate}
{\large \color{blue}  }
\subsection*{Explain}
\begin{enumerate}
\item adjective \\
If you say that someone or something is \textbf{fortunate} , you mean that they are lucky.
 \textit{
	\begin{itemize}
	\item He was extremely fortunate to survive.
	\item Central London is fortunate in having so many large parks and open spaces.
	\item It was fortunate that the water was shallow.
	\item She is in the fortunate position of having plenty of choice.
	\end{itemize}
}
\end{enumerate}

\section*{fabric}
{\large \color{blue}  fabrics  }
\subsection*{Explain}
\begin{enumerate}
\item variable noun \\
\textbf{Fabric} is cloth or other material produced by weaving together cotton , nylon , wool , silk , or other threads . Fabrics are used for making things such as clothes , curtains , and sheets .
 \textit{
	\begin{itemize}
	\item ...small squares of red cotton fabric.
	\item Whatever your colour scheme, there's a fabric to match.
	\end{itemize}
}
\item singular noun \\
The \textbf{fabric} of a society or system is its basic structure, with all the customs and beliefs that make it work successfully.
 \textit{
	\begin{itemize}
	\item The fabric of society has been deeply damaged by the previous regime.
	\item Years of civil war have wrecked the country's infrastructure and destroyed its social
fabric.
	\end{itemize}
}
\item singular noun \\
The \textbf{fabric} of a building is its walls , roof , and the materials with which it is built .
 \textit{
	\begin{itemize}
	\item Condensation will eventually cause the fabric of the building to rot away.
	\end{itemize}
}
\end{enumerate}

\section*{fragile}
{\large \color{blue}  }
\subsection*{Explain}
\begin{enumerate}
\item adjective \\
If you describe a situation as \textbf{fragile} , you mean that it is weak or uncertain , and unlikely to be able to resist  strong  pressure or attack .
 \textit{
	\begin{itemize}
	\item ...moves that will place added strain on an already fragile economy.
	\item The Prime Minister's fragile government was on the brink of collapse.
	\item His overall condition remained fragile.
	\end{itemize}
}
\item adjective \\
Something that is \textbf{fragile} is easily broken or damaged .
 \textit{
	\begin{itemize}
	\item He leaned back in his fragile chair.
	\end{itemize}
}
\item graded adjective \\
Something that is \textbf{fragile} is very delicate or fine in appearance .
 \textit{
	\begin{itemize}
	\end{itemize}
}
\item graded adjective \\
If someone feels  \textbf{fragile} , they feel weak, for example because they are ill or have drunk too much alcohol .
 \textit{
	\begin{itemize}
	\item He felt irritated and strangely fragile, as if he were recovering from a severe bout
of flu.
	\end{itemize}
}
\end{enumerate}

\section*{gentleman}
{\large \color{blue}  gentlemen  }
\subsection*{Explain}
\begin{enumerate}
\item countable noun \\
A \textbf{gentleman} is a man who comes from a family of high social standing .
 \textit{
	\begin{itemize}
	\item ...this wonderful portrait of English gentleman Joseph Greenway.
	\end{itemize}
}
\item countable noun \\
If you say that a man is a \textbf{gentleman} , you mean he is polite and educated , and can be trusted .
 \textit{
	\begin{itemize}
	\item He was always such a gentleman.
	\end{itemize}
}
\item countable noun \\
You can address men as \textbf{gentlemen} , or refer politely to them as \textbf{gentlemen} .
 \textit{
	\begin{itemize}
	\item This way, please, ladies and gentlemen.
	\item It seems this gentleman was waiting for the doctor.
	\end{itemize}
}
\item  \\
 a gentleman's agreement \textit{
	\begin{itemize}
	\end{itemize}
}
\end{enumerate}

\section*{heroic}
{\large \color{blue}  heroics  }
\subsection*{Explain}
\begin{enumerate}
\item adjective \\
If you describe a person or their actions as \textbf{heroic} , you admire them because they show  extreme bravery.
 \textit{
	\begin{itemize}
	\item His heroic deeds were celebrated in every corner of India.
	\end{itemize}
}
\item adjective \\
If you describe an action or event as \textbf{heroic} , you admire it because it involves great  effort or determination to succeed .
 \textit{
	\begin{itemize}
	\item The company has made heroic efforts at cost reduction.
	\item He finally faltered in the last game of a heroic match.
	\item Their dogged single-mindedness is almost heroic.
	\end{itemize}
}
\item adjective \\
\textbf{Heroic} means being or relating to the hero of a story .
 \textit{
	\begin{itemize}
	\item ...the book's central, heroic figure.
	\end{itemize}
}
\item plural noun \\
\textbf{Heroics} are actions involving bravery, courage , or determination.
 \textit{
	\begin{itemize}
	\item ...the man whose aerial heroics helped save the helicopter pilot.
	\item England need heroics from the captain now.
	\end{itemize}
}
\item plural noun \\
If you describe someone's actions or plans as \textbf{heroics} , you think that they are foolish or dangerous because they are too difficult or brave for the situation in which they occur.
 \textit{
	\begin{itemize}
	\item He said his advice was: 'No heroics, stay within the law'.
	\item Cut it out, Perry. You've performed your heroics. It's all over now.
	\end{itemize}
}
\end{enumerate}

\section*{glass}
{\large \color{blue}  glasses  }
\subsection*{Explain}
\begin{enumerate}
\item uncountable noun \\
\textbf{Glass} is a hard transparent substance that is used to make things such as windows and bottles.
 \textit{
	\begin{itemize}
	\item ...a pane of glass.
	\item ...a sliding glass door.
	\end{itemize}
}
\item countable noun \\
A \textbf{glass} is a container made from glass, which you can drink from and which does not have a handle .
 The contents of a glass can be referred to as a \textbf{glass}  \textbf{of} something.
 \textit{
	\begin{itemize}
	\item Grossman raised the glass to his lips.
	\item ...a glass of milk.
	\end{itemize}
}
\item uncountable noun \\
\textbf{Glass} is used to mean objects made of glass, for example drinking containers and bowls .
 \textit{
	\begin{itemize}
	\item There's a glittering array of glass to choose from at markets.
	\end{itemize}
}
\item plural noun \\
\textbf{Glasses} are two lenses in a frame that some people wear in front of their eyes in order to help them see  better .
 \textit{
	\begin{itemize}
	\item He took off his glasses.
	\end{itemize}
}
\end{enumerate}

\section*{inherent}
{\large \color{blue}  }
\subsection*{Explain}
\begin{enumerate}
\item adjective \\
The \textbf{inherent} qualities of something are the necessary and natural parts of it.
 \textit{
	\begin{itemize}
	\item Stress is an inherent part of dieting.
	\item ...the dangers inherent in an outbreak of war.
	\end{itemize}
}
\end{enumerate}

\section*{government}
{\large \color{blue}  governments  }
\subsection*{Explain}
\begin{enumerate}
\item countable noun \\
The \textbf{government} of a country is the group of people who are responsible for governing it.
 \textit{
	\begin{itemize}
	\item The Government has insisted that confidence is needed before the economy can improve.
	\item ...democratic governments in countries like Britain and the U.S..
	\item ...fighting between government forces and left-wing rebels.
	\end{itemize}
}
\item uncountable noun \\
\textbf{Government} consists of the activities, methods , and principles involved in governing a country or other political unit.
 \textit{
	\begin{itemize}
	\item The first four years of government were completely disastrous.
	\item ...our system of government.
	\end{itemize}
}
\end{enumerate}

\section*{integral}
{\large \color{blue}  }
\subsection*{Explain}
\begin{enumerate}
\item adjective \\
Something that is an \textbf{integral} part of something is an essential part of that thing.
 \textit{
	\begin{itemize}
	\item Rituals and festivals form an integral part of every human society.
	\item Anxiety is integral to the human condition.
	\end{itemize}
}
\end{enumerate}

\section*{holder}
{\large \color{blue}  holders  }
\subsection*{Explain}
\begin{enumerate}
\item countable noun \\
A \textbf{holder} is someone who owns or has something.
 \textit{
	\begin{itemize}
	\item This season the club has had 73,500 season-ticket holders.
	\item ...the record holder in the 3, 000 metres steeplechase.
	\item ...the holders of the European Football Championship.
	\end{itemize}
}
\item countable noun \\
A \textbf{holder} is a container in which you put an object, usually in order to protect it or to keep it in place.
 \textit{
	\begin{itemize}
	\item ...a toothbrush holder.
	\item ...a cigarette holder.
	\end{itemize}
}
\end{enumerate}

\section*{international}
{\large \color{blue}  internationals  }
\subsection*{Explain}
\begin{enumerate}
\item adjective \\
\textbf{International} means between or involving different countries.
 \textit{
	\begin{itemize}
	\item ...an international agreement against exporting arms to that country.
	\item ...Kuwait International Airport.
	\item ...emergency aid from the international community.
	\end{itemize}
}
\item countable noun \\
In sport , an \textbf{international} is a game that is played between teams representing two different countries.
 \textit{
	\begin{itemize}
	\item ...the midweek international against England.
	\end{itemize}
}
\item countable noun \\
An \textbf{international} is a member of a country's sports team.
 \textit{
	\begin{itemize}
	\item ...a former England international.
	\end{itemize}
}
\end{enumerate}

\section*{longitude}
{\large \color{blue}  longitudes  }
\subsection*{Explain}
\begin{enumerate}
\item variable noun \\
The \textbf{longitude} of a place is its distance to the west or east of a line  passing through Greenwich . Compare  latitude .
 \textbf{Longitude} is also an adjective .
 \textit{
	\begin{itemize}
	\item He noted the latitude and longitude, then made a mark on the admiralty chart.
	\item A similar feature is found at 13 degrees North between 230 degrees and 250 degrees
longitude.
	\end{itemize}
}
\end{enumerate}

\section*{invaluable}
{\large \color{blue}  }
\subsection*{Explain}
\begin{enumerate}
\item adjective \\
If you describe something as \textbf{invaluable} , you mean that it is extremely  useful .
 \textit{
	\begin{itemize}
	\item I was able to gain invaluable experience over that year.
	\item The research should prove invaluable in the study of linguistics.
	\item Their advice was invaluable to me at that stage of my work.
	\end{itemize}
}
\end{enumerate}

\section*{loom}
{\large \color{blue}  looms  looming  loomed  }
\subsection*{Explain}
\begin{enumerate}
\item verb \\
If something \textbf{looms}  \textbf{over} you, it appears as a large or unclear shape, often in a frightening way.
 \textit{
	\begin{itemize}
	\item Vincent loomed over me, as pale and grey as a tombstone.
	\item ...the bleak mountains that loomed out of the blackness and towered around us.
	\end{itemize}
}
\item verb \\
If a worrying or threatening situation or event \textbf{is looming} , it seems likely to happen  soon .
 \textit{
	\begin{itemize}
	\item Another government spending crisis is looming in the United States.
	\item The threat of renewed civil war looms ahead.
	\item ...the looming threat of recession.
	\end{itemize}
}
\item  \\
 to loom large \textit{
	\begin{itemize}
	\end{itemize}
}
\item countable noun \\
A \textbf{loom} is a machine that is used for weaving thread into cloth.
 \textit{
	\begin{itemize}
	\end{itemize}
}
\end{enumerate}

\section*{lucky}
{\large \color{blue}  luckier  luckiest  }
\subsection*{Explain}
\begin{enumerate}
\item adjective \\
You say that someone is \textbf{lucky} when they have something that is very desirable or when they are in a very desirable situation .
 \textit{
	\begin{itemize}
	\item I am luckier than most. I have a job.
	\item I consider myself the luckiest man on the face of the Earth.
	\item He is incredibly lucky to be alive.
	\item Those who are lucky enough to be wealthy have a duty to give to the hungry.
	\end{itemize}
}
\item adjective \\
Someone who is \textbf{lucky}  seems to always have good luck .
 \textit{
	\begin{itemize}
	\item Some people are born lucky aren't they?
	\item He had always been lucky at cards.
	\end{itemize}
}
\item adjective \\
If you describe an action or experience as \textbf{lucky} , you mean that it was good or successful , and that it happened by chance and not as a result of planning or preparation .
 \textit{
	\begin{itemize}
	\item They admit they are now desperate for a lucky break.
	\item The man is very lucky that he had help so quickly.
	\end{itemize}
}
\item adjective \\
A \textbf{lucky}  object is something that people believe  helps them to be successful.
 \textit{
	\begin{itemize}
	\item He did not have on his other lucky charm, a pair of green socks.
	\end{itemize}
}
\item  \\
 sb will be lucky \textit{
	\begin{itemize}
	\end{itemize}
}
\item  \\
 count oneself lucky \textit{
	\begin{itemize}
	\end{itemize}
}
\item  \\
 lucky sb \textit{
	\begin{itemize}
	\end{itemize}
}
\item  \\
 strike lucky \textit{
	\begin{itemize}
	\end{itemize}
}
\item  \\
 third time lucky (and other numbers) \textit{
	\begin{itemize}
	\end{itemize}
}
\end{enumerate}

\section*{obvious}
{\large \color{blue}  }
\subsection*{Explain}
\begin{enumerate}
\item adjective \\
If something is \textbf{obvious} , it is easy to see or understand.
 \textit{
	\begin{itemize}
	\item ...the need to rectify what is an obvious injustice.
	\item Determining how he will conduct his presidency isn't quite so obvious.
	\end{itemize}
}
\item adjective \\
If you describe something that someone says as \textbf{obvious} , you are being critical of it because you think it is unnecessary or shows lack of imagination .
 \textit{
	\begin{itemize}
	\item There are some very obvious phrases that we should know better than to use.
	\end{itemize}
}
\end{enumerate}

\section*{manager}
{\large \color{blue}  managers  }
\subsection*{Explain}
\begin{enumerate}
\item countable noun \\
A \textbf{manager} is a person who is responsible for running part of or the whole of a business organization.
 \textit{
	\begin{itemize}
	\item The chef, staff and managers are all Chinese.
	\item ...Linda Emery, marketing manager for Wall's sausages.
	\item ...a retired bank manager.
	\end{itemize}
}
\item countable noun \\
The \textbf{manager} of a pop star or other entertainer is the person who looks after their business interests .
 \textit{
	\begin{itemize}
	\end{itemize}
}
\item countable noun \\
The \textbf{manager} of a sports team is the person responsible for training the players and organizing the way they play. In American English, \textbf{manager} is only used for baseball ; in other sports, coach is used instead .
 \textit{
	\begin{itemize}
	\end{itemize}
}
\end{enumerate}

\section*{odd}
{\large \color{blue}  odder  oddest  }
\subsection*{Explain}
\begin{enumerate}
\item adjective \\
If you describe someone or something as \textbf{odd} , you think that they are strange or unusual.
 \textit{
	\begin{itemize}
	\item He'd always been odd, but not to this extent.
	\item What an odd coincidence that he should have known your family.
	\item Something odd began to happen.
	\end{itemize}
}
\item  \\
 See also  odd-looking \textit{
	\begin{itemize}
	\end{itemize}
}
\item adjective \\
You use \textbf{odd} before a noun to indicate that you are not mentioning the type, size , or quality of something because it is not important .
 \textit{
	\begin{itemize}
	\item ...moving from place to place where she could find the odd bit of work.
	\item He had various odd cleaning jobs around the place.
	\item I knew that Alan liked the odd drink.
	\end{itemize}
}
\item adverb \\
You use \textbf{odd} after a number to indicate that it is only approximate .
 \textit{
	\begin{itemize}
	\item How many pages was it, 500 odd?
	\item He has now appeared in sixty odd films.
	\item 'How long have you lived here?'—'Twenty odd years.'
	\end{itemize}
}
\item adjective \\
\textbf{Odd} numbers, such as 3 and 17, are those which cannot be divided  exactly by the number two.
 \textit{
	\begin{itemize}
	\item The odd numbers are on the left as you walk up the street.
	\item There's an odd number of candidates.
	\end{itemize}
}
\item adjective \\
You say that two things are \textbf{odd} when they do not belong to the same set or pair.
 \textit{
	\begin{itemize}
	\item I'm wearing odd socks today by the way.
	\end{itemize}
}
\item  \\
 the odd man out/the odd woman out/the odd one out \textit{
	\begin{itemize}
	\end{itemize}
}
\end{enumerate}

\section*{module}
{\large \color{blue}  modules  }
\subsection*{Explain}
\begin{enumerate}
\item countable noun \\
A \textbf{module} is one of the separate parts of a course taught at a college or university .
 \textit{
	\begin{itemize}
	\item These courses cover a twelve-week period and are organised into three four-week modules.
	\end{itemize}
}
\item countable noun \\
A \textbf{module} is a part of a spacecraft which can operate by itself, often away from the rest of the spacecraft.
 \textit{
	\begin{itemize}
	\item A rescue plan could be achieved by sending an unmanned module to the space station.
	\end{itemize}
}
\item countable noun \\
A \textbf{module} is one of a set of parts from which some buildings are made. Each module is made
separately, and the completed modules are then joined together to form the building.
 \textit{
	\begin{itemize}
	\end{itemize}
}
\item countable noun \\
A \textbf{module} is a part of a machine , especially a computer, which performs a particular function .
 \textit{
	\begin{itemize}
	\end{itemize}
}
\end{enumerate}

\section*{plain}
{\large \color{blue}  plainer  plainest  plains  }
\subsection*{Explain}
\begin{enumerate}
\item adjective \\
A \textbf{plain} object, surface, or fabric is entirely in one colour and has no pattern, design, or writing on it.
 \textit{
	\begin{itemize}
	\item In general, a plain carpet makes a room look bigger.
	\item He placed the paper in a plain envelope.
	\item He wore a plain blue shirt, open at the collar.
	\end{itemize}
}
\item adjective \\
Something that is \textbf{plain} is very simple in style.
 \textit{
	\begin{itemize}
	\item Bronwen's dress was plain but it hung well on her.
	\item It was a plain, grey stone house, distinguished mainly by its largely unspoilt simplicity.
	\end{itemize}
}
\item adjective \\
If a fact, situation, or statement is \textbf{plain} , it is easy to recognize or understand .
 \textit{
	\begin{itemize}
	\item It was plain to him that I was having a nervous breakdown.
	\item He's made it plain that he loves the game and wants to be involved still.
	\end{itemize}
}
\item adjective \\
If you describe someone as \textbf{plain} , you think they look  ordinary and not at all beautiful .
 \textit{
	\begin{itemize}
	\item ...a shy, rather plain girl with a pale complexion.
	\end{itemize}
}
\item countable noun \\
A \textbf{plain} is a large flat area of land with very few trees on it.
 \textit{
	\begin{itemize}
	\item Once there were 70 million buffalo on the plains.
	\end{itemize}
}
\item adverb \\
You can use \textbf{plain} before an adjective in order to emphasize it.
 \textbf{Plain} is also used before a noun .
 \textit{
	\begin{itemize}
	\item The food was just plain terrible.
	\item Is it love of publicity or plain stupidity on her part?
	\end{itemize}
}
\item adjective \\
You can use \textbf{plain} before a name to emphasize how simple and ordinary it is, especially when you are comparing it with another more unusual or impressive name.
 \textit{
	\begin{itemize}
	\item Why couldn't they call you plain Ann or Alice like the rest?
	\end{itemize}
}
\item  \\
 in plain clothes \textit{
	\begin{itemize}
	\end{itemize}
}
\end{enumerate}

\section*{nail}
{\large \color{blue}  nails  nailing  nailed  }
\subsection*{Explain}
\begin{enumerate}
\item countable noun \\
A \textbf{nail} is a thin  piece of metal with one pointed end and one flat end. You hit the flat end with a hammer in order to push the nail into something such as a wall .
 \textit{
	\begin{itemize}
	\item A mirror hung on a nail above the washstand.
	\item He hammered the nail into the branch.
	\end{itemize}
}
\item verb \\
If you \textbf{nail} something somewhere , you fix it there using one or more nails.
 \textit{
	\begin{itemize}
	\item Frank put the first plank down and nailed it in place.
	\item They nail shut the front door.
	\item The windows were all nailed shut.
	\end{itemize}
}
\item countable noun \\
Your \textbf{nails} are the thin hard parts that grow at the ends of your fingers and toes.
 \textit{
	\begin{itemize}
	\item Keep your nails short and your hands clean.
	\end{itemize}
}
\item verb \\
To \textbf{nail} someone means to catch them and prove that they have been breaking the law .
 \textit{
	\begin{itemize}
	\item The prosecution still managed to nail him for robberies at the homes of leading industrialists.
	\end{itemize}
}
\item verb \\
If you \textbf{nail} something, you do it extremely  well or successfully.
 \textit{
	\begin{itemize}
	\item She had nailed the art of making us laugh.
	\item He tried, but I'm not sure he totally nailed it.
	\end{itemize}
}
\item  \\
 as hard as nails/hard as nails \textit{
	\begin{itemize}
	\end{itemize}
}
\item  \\
 to hit the nail on the head \textit{
	\begin{itemize}
	\end{itemize}
}
\end{enumerate}

\section*{positive}
{\large \color{blue}  }
\subsection*{Explain}
\begin{enumerate}
\item adjective \\
If you are \textbf{positive}  \textbf{about} things, you are hopeful and confident , and think of the good aspects of a situation rather than the bad ones.
 \textit{
	\begin{itemize}
	\item Be positive about your future and get on with living a normal life.
	\item He became much more positive and was soon back in full-time employment.
	\item ...a positive frame of mind.
	\end{itemize}
}
\item adjective \\
A \textbf{positive} fact, situation, or experience is pleasant and helpful to you in some way.
 \textbf{The positive} in a situation is the good and pleasant aspects of it.
 \textit{
	\begin{itemize}
	\item The parting from his sister had a positive effect on John.
	\item Working abroad should be an exciting and positive experience for all concerned.
	\item Work on the positive, creating beautiful, loving and fulfilling relationships.
	\end{itemize}
}
\item adjective \\
If you make a \textbf{positive}  decision or take \textbf{positive} action, you do something definite in order to deal with a task or problem.
 \textit{
	\begin{itemize}
	\item There are positive changes that should be implemented in the rearing of animals.
	\item Having a good diet gives me a sense that I'm doing something positive and that I'm
in control.
	\end{itemize}
}
\item adjective \\
A \textbf{positive} response to something indicates agreement, approval , or encouragement.
 \textit{
	\begin{itemize}
	\item Police have had a positive response to appeals for information from motorists travelling
on the M40.
	\end{itemize}
}
\item adjective \\
If you are \textbf{positive} about something, you are completely sure about it.
 \textit{
	\begin{itemize}
	\item I'm as positive as I can be about it.
	\item 'She's never late. You sure she said eight?'—'Positive.'
	\end{itemize}
}
\item adjective \\
\textbf{Positive}  evidence gives definite proof of the truth or identity of something.
 \textit{
	\begin{itemize}
	\item There was no positive evidence that any birth defects had arisen as a result of Vitamin
A intake.
	\end{itemize}
}
\item adjective \\
If a medical or scientific test is \textbf{positive} , it shows that something has happened or is present.
 \textit{
	\begin{itemize}
	\item If the test is positive, a course of antibiotics may be prescribed.
	\item He was stripped of his gold medal after testing positive for steroids.
	\end{itemize}
}
\item adjective \\
You can use \textbf{positive} to emphasize a noun.
 \textit{
	\begin{itemize}
	\item Good day to you, Bernard! It's a positive delight to see you.
	\item He was in a positive fury.
	\item The man was being a positive embarrassment.
	\end{itemize}
}
\item adjective \\
A \textbf{positive} number is greater than zero.
 \textit{
	\begin{itemize}
	\item It's really a simple numbers game with negative and positive numbers.
	\end{itemize}
}
\item adjective \\
If something has a \textbf{positive} electrical charge, it has the same charge as a proton and the opposite charge to
an electron.
 \textit{
	\begin{itemize}
	\end{itemize}
}
\end{enumerate}

\section*{necklace}
{\large \color{blue}  necklaces  necklacing  necklaced  }
\subsection*{Explain}
\begin{enumerate}
\item countable noun \\
A \textbf{necklace} is a piece of jewellery such as a chain or a string of beads which someone, usually a woman, wears round their neck.
 \textit{
	\begin{itemize}
	\item ...a diamond necklace and matching earrings.
	\end{itemize}
}
\item verb \\
To \textbf{necklace} someone means to kill them by putting a tyre soaked in petrol around their neck and then setting fire to it.
 \textit{
	\begin{itemize}
	\item Alleged strike breakers had their houses petrol-bombed or were hacked to death or
necklaced.
	\end{itemize}
}
\end{enumerate}

\section*{queer}
{\large \color{blue}  queerer  queerest  queers  }
\subsection*{Explain}
\begin{enumerate}
\item adjective \\
Something that is \textbf{queer} is strange.
 \textit{
	\begin{itemize}
	\item If you ask me, there's something a bit queer going on.
	\end{itemize}
}
\item countable noun \\
People sometimes  call homosexual men \textbf{queers} .
 \textbf{Queer} is also an adjective .
 \textit{
	\begin{itemize}
	\item ...queer men.
	\end{itemize}
}
\item adjective \\
\textbf{Queer}  means relating to homosexual people, and is used by some homosexuals.
 \textit{
	\begin{itemize}
	\item ...contemporary queer culture.
	\item ...queer activism.
	\end{itemize}
}
\end{enumerate}

\section*{pearl}
{\large \color{blue}  pearls  }
\subsection*{Explain}
\begin{enumerate}
\item countable noun \\
A \textbf{pearl} is a hard round object which is shiny and usually creamy-white in colour. Pearls grow  inside the shell of an oyster and are used for making expensive  jewellery .
 \textit{
	\begin{itemize}
	\item She wore a string of pearls at her throat.
	\item I put on the pearl earrings Daddy had bought me.
	\end{itemize}
}
\item adjective \\
\textbf{Pearl} is used to describe something which looks like a pearl.
 \textit{
	\begin{itemize}
	\item ...tiny pearl buttons.
	\end{itemize}
}
\item  \\
 to cast pearls before swine \textit{
	\begin{itemize}
	\end{itemize}
}
\item  \\
 pearls of wisdom \textit{
	\begin{itemize}
	\end{itemize}
}
\end{enumerate}

\section*{sensible}
{\large \color{blue}  }
\subsection*{Explain}
\begin{enumerate}
\item adjective \\
\textbf{Sensible} actions or decisions are good because they are based on reasons  rather than emotions .
 \textit{
	\begin{itemize}
	\item It might be sensible to get a solicitor.
	\item The sensible thing is to leave them alone.
	\item ...sensible advice.
	\end{itemize}
}
\item adjective \\
\textbf{Sensible} people behave in a sensible way.
 \textit{
	\begin{itemize}
	\item She was a sensible girl and did not panic.
	\item Oh come on, let's be sensible about this.
	\item I'm trying to persuade you to be more sensible.
	\end{itemize}
}
\item adjective \\
\textbf{Sensible}  shoes or clothes are practical and strong rather than fashionable and attractive .
 \textit{
	\begin{itemize}
	\item Wear loose clothing and sensible footwear.
	\end{itemize}
}
\end{enumerate}

\section*{pickup}
{\large \color{blue}  }
\subsection*{Explain}
\begin{enumerate}
\item noun \\
1.  2.  3.  4.  5.  6.  7.  8.  9.  \textit{
	\begin{itemize}
	\end{itemize}
}
\item adjective \\
10.  11.  12.  \textit{
	\begin{itemize}
	\item a pickup jazz band or baseball team
	\end{itemize}
}
\end{enumerate}

\section*{singular}
{\large \color{blue}  }
\subsection*{Explain}
\begin{enumerate}
\item adjective \\
The \textbf{singular} form of a word is the form that is used when referring to one person or thing.
 \textit{
	\begin{itemize}
	\item ...the fifteen case endings of the singular form of the Finnish noun.
	\item The word 'you' can be singular or plural.
	\end{itemize}
}
\item singular noun \\
\textbf{The singular} of a noun is the form of it that is used to refer to one person or thing.
 \textit{
	\begin{itemize}
	\item The singular of Inuit is Inuk.
	\end{itemize}
}
\item adjective \\
\textbf{Singular} means very great and remarkable.
 \textit{
	\begin{itemize}
	\item ...a smile of singular sweetness.
	\item Barre was quickly drawn into the electoral arena, although with singular lack of
success.
	\end{itemize}
}
\item adjective \\
If you describe someone or something as \textbf{singular} , you mean that they are strange or unusual.
 \textit{
	\begin{itemize}
	\item Cardinal Meschia was without doubt a singular character.
	\item Where he got that singular notion I just can't think.
	\end{itemize}
}
\end{enumerate}

\section*{poison}
{\large \color{blue}  poisons  poisoning  poisoned  }
\subsection*{Explain}
\begin{enumerate}
\item variable noun \\
\textbf{Poison} is a substance that harms or kills people or animals if they swallow it or absorb it.
 \textit{
	\begin{itemize}
	\item Poison from the weaver fish causes paralysis, swelling, and nausea.
	\item Mercury is a known poison.
	\end{itemize}
}
\item verb \\
If someone \textbf{poisons} another person, they kill the person or make them ill by giving them poison.
 \textit{
	\begin{itemize}
	\item The rumours that she had poisoned him could never be proved.
	\end{itemize}
}
\item verb \\
If you \textbf{are poisoned}  \textbf{by} a substance, it makes you very ill and sometimes kills you.
 \textit{
	\begin{itemize}
	\item Employees were taken to hospital yesterday after being poisoned by fumes.
	\item Toxic waste could endanger lives and poison fish.
	\end{itemize}
}
\item verb \\
If someone \textbf{poisons} a food , drink , or weapon , they add poison to it so that it can be used to kill someone.
 \textit{
	\begin{itemize}
	\item They considered poisoning his food.
	\end{itemize}
}
\item verb \\
To \textbf{poison} water, air , or land  means to damage it with harmful substances such as chemicals.
 \textit{
	\begin{itemize}
	\item ...the textile and fibre industries that poison the water and use vast amounts of
natural resources.
	\item The land has been completely poisoned by chemicals.
	\item ...dying forests, poisoned rivers and lakes.
	\end{itemize}
}
\item verb \\
Something that \textbf{poisons} a good  situation or relationship  spoils it or destroys it.
 \textit{
	\begin{itemize}
	\item The whole atmosphere has really been poisoned.
	\item ...ill-feeling that will poison further talk of a common foreign policy.
	\end{itemize}
}
\item  \\
 poison sb's mind \textit{
	\begin{itemize}
	\end{itemize}
}
\end{enumerate}

\section*{slim}
{\large \color{blue}  slimmer  slimmest  slims  slimming  slimmed  }
\subsection*{Explain}
\begin{enumerate}
\item adjective \\
A \textbf{slim} person has an attractively thin and well-shaped body.
 \textit{
	\begin{itemize}
	\item The young woman was tall and slim.
	\item He is attractive, of slim build, with blue eyes.
	\end{itemize}
}
\item verb \\
If you \textbf{are slimming} , you are trying to make yourself thinner and lighter by eating less food.
 \textbf{Slim down} means the same as slim .
 \textit{
	\begin{itemize}
	\item Some people will gain weight, no matter how hard they try to slim.
	\item It makes sense to eat a reasonably balanced diet when slimming.
	\item Doctors have told Benny to slim down.
	\item ...salon treatments that claim to slim down thighs.
	\end{itemize}
}
\item adjective \\
A \textbf{slim} book, wallet , or other object is thinner than usual .
 \textit{
	\begin{itemize}
	\item The slim booklets describe a range of services and facilities.
	\item He published only three slim volumes of verse in his short life.
	\end{itemize}
}
\item adjective \\
A \textbf{slim}  chance or possibility is a very small one.
 \textit{
	\begin{itemize}
	\item There's still a slim chance that he may become Prime Minister.
	\end{itemize}
}
\item verb \\
If an organization \textbf{slims} its products or workers , it reduces the number of them that it has.
 \textit{
	\begin{itemize}
	\item The company recently slimmed its product line.
	\end{itemize}
}
\end{enumerate}

\section*{policy}
{\large \color{blue}  policies  }
\subsection*{Explain}
\begin{enumerate}
\item variable noun \\
A \textbf{policy} is a set of ideas or plans that is used as a basis for making decisions , especially in politics , economics , or business.
 \textit{
	\begin{itemize}
	\item ...plans which include changes in foreign policy and economic reforms.
	\item ...the U.N.'s policy-making body.
	\end{itemize}
}
\item countable noun \\
An official organization's \textbf{policy} on a particular issue or towards a country is their attitude and actions regarding that issue or country.
 \textit{
	\begin{itemize}
	\item ...its no-strings aid policy towards Africa.
	\item ...the government's policy on repatriation.
	\item ...the corporation's policy of forbidding building on common land.
	\end{itemize}
}
\item countable noun \\
An insurance \textbf{policy} is a document which shows the agreement that you have made with an insurance company.
 \textit{
	\begin{itemize}
	\item You are advised to read the small print of household and motor insurance policies.
	\end{itemize}
}
\end{enumerate}

\section*{soft}
{\large \color{blue}  softer  softest  }
\subsection*{Explain}
\begin{enumerate}
\item adjective \\
Something that is \textbf{soft} is pleasant to touch, and not rough or hard.
 \textit{
	\begin{itemize}
	\item Regular use of a body lotion will keep the skin soft and supple.
	\item When it's dry, brush the hair using a soft, nylon baby brush.
	\item ...warm, soft, white towels.
	\end{itemize}
}
\item adjective \\
Something that is \textbf{soft} changes shape or bends easily when you press it.
 \textit{
	\begin{itemize}
	\item She lay down on the soft, comfortable bed.
	\item Add enough milk to form a soft dough.
	\item ...soft cheese.
	\end{itemize}
}
\item adjective \\
Something that has a \textbf{soft} appearance has smooth curves rather than sharp or distinct edges.
 \textit{
	\begin{itemize}
	\item This is a smart, yet soft and feminine look.
	\item ...the soft skin on the baby's face.
	\end{itemize}
}
\item adjective \\
Something that is \textbf{soft} is very gentle and has no force. For example, a \textbf{soft} sound or voice is quiet and not harsh. A \textbf{soft} light or colour is pleasant to look at because it is not bright.
 \textit{
	\begin{itemize}
	\item There was a soft tapping on my door.
	\item When he woke again he could hear soft music.
	\item ...a soft Irish accent.
	\item ...soft muted colours.
	\item A soft spring rain had fallen all day.
	\end{itemize}
}
\item adjective \\
If you are \textbf{soft}  \textbf{on} someone, you do not treat them as strictly or severely as you should do.
 \textit{
	\begin{itemize}
	\item The president says the measure is soft and weak on criminals.
	\item He had initially thought Byrnes too soft with the Russians.
	\end{itemize}
}
\item adjective \\
If you say that someone has a \textbf{soft}  \textbf{heart} , you mean that they are sensitive and sympathetic towards other people.
 \textit{
	\begin{itemize}
	\item Her rather tough and worldly exterior hides a very soft and sensitive heart.
	\end{itemize}
}
\item adjective \\
You use \textbf{soft} to describe a way of life that is easy and involves very little work.
 \textit{
	\begin{itemize}
	\item The regime at Latchmere could be seen as a soft option.
	\item There is no way that 20 years of soft living could be lost in the first 30 minutes'
exercise.
	\end{itemize}
}
\item adjective \\
\textbf{Soft} drugs are drugs, such as cannabis, which are illegal but which many people do not consider to be strong or harmful .
 \textit{
	\begin{itemize}
	\end{itemize}
}
\item adjective \\
A \textbf{soft}  target is a place or person that can easily be attacked.
 \textit{
	\begin{itemize}
	\item People who carry a lot of cash about are a very soft target.
	\end{itemize}
}
\item adjective \\
\textbf{Soft} water does not contain much of the mineral calcium and so makes bubbles easily when you use soap.
 \textit{
	\begin{itemize}
	\end{itemize}
}
\item  \\
 to have a soft spot for someone \textit{
	\begin{itemize}
	\end{itemize}
}
\end{enumerate}

\section*{politician}
{\large \color{blue}  politicians  }
\subsection*{Explain}
\begin{enumerate}
\item countable noun \\
A \textbf{politician} is a person whose job is in politics, especially a member of parliament or congress .
 \textit{
	\begin{itemize}
	\item They have arrested a number of leading opposition politicians.
	\end{itemize}
}
\end{enumerate}

\section*{solid}
{\large \color{blue}  solids  }
\subsection*{Explain}
\begin{enumerate}
\item adjective \\
A \textbf{solid} substance or object stays the same shape whether it is in a container or not.
 \textit{
	\begin{itemize}
	\item ...the potential of greatly reducing our solid waste problem.
	\item He did not eat solid food for several weeks.
	\end{itemize}
}
\item countable noun \\
A \textbf{solid} is a substance that stays the same shape whether it is in a container or not.
 \textit{
	\begin{itemize}
	\item Solids turn to liquids at certain temperatures.
	\item ...the decomposition of solids.
	\end{itemize}
}
\item adjective \\
A substance that is \textbf{solid} is very hard or firm.
 \textit{
	\begin{itemize}
	\item The snow had melted, but the lake was still frozen solid.
	\item The concrete will stay as solid as a rock.
	\end{itemize}
}
\item adjective \\
A \textbf{solid} object or mass does not have a space inside it, or holes or gaps in it.
 \textit{
	\begin{itemize}
	\item ...a tunnel carved through 50ft of solid rock.
	\item ...a solid wall of multicoloured trees.
	\item ...a solid mass of colour.
	\item The car park was absolutely packed solid with people.
	\end{itemize}
}
\item adjective \\
If an object is made of \textbf{solid}  gold or \textbf{solid} wood, for example , it is made of gold or wood all the way through, rather than just on the outside.
 \textit{
	\begin{itemize}
	\item The taps appeared to be made of solid gold.
	\item ...solid wood doors.
	\item ...solid pine furniture.
	\end{itemize}
}
\item adjective \\
A structure that is \textbf{solid} is strong and is not likely to collapse or fall over.
 \textit{
	\begin{itemize}
	\item Banks are built to look solid to reassure their customers.
	\item The car feels very solid.
	\end{itemize}
}
\item adjective \\
If you describe someone as \textbf{solid} , you mean that they are very reliable and respectable .
 \textit{
	\begin{itemize}
	\item You want a partner who is solid and stable.
	\item Mr Zuma had a solid reputation as a grass roots organiser.
	\end{itemize}
}
\item adjective \\
\textbf{Solid}  evidence or information is reliable because it is based on facts .
 \textit{
	\begin{itemize}
	\item We don't have good solid information on where the people are.
	\item Some solid evidence was what was required.
	\item He has a solid alibi.
	\end{itemize}
}
\item adjective \\
You use \textbf{solid} to describe something such as advice or a piece of work which is useful and reliable.
 \textit{
	\begin{itemize}
	\item The CIU provides churches with solid advice on a wide range of subjects.
	\item All I am looking for is a good solid performance.
	\item I've always felt that solid experience would stand me in good stead.
	\end{itemize}
}
\item adjective \\
You use \textbf{solid} to describe something such as the basis for a policy or support for an organization when it is strong, because it has been developed carefully
and slowly.
 \textit{
	\begin{itemize}
	\item I am determined to build on this solid foundation.
	\item ...a nominee with solid support within the party.
	\item ...Washington's attempt to build a solid international coalition.
	\end{itemize}
}
\item adjective \\
If you do something for a \textbf{solid} period of time, you do it without any pause or interruption throughout that time.
 \textit{
	\begin{itemize}
	\item We had worked together for two solid years.
	\end{itemize}
}
\end{enumerate}

\section*{politics}
{\large \color{blue}  }
\subsection*{Explain}
\begin{enumerate}
\item plural noun \\
\textbf{Politics} are the actions or activities concerned with achieving and using power in a country or society. The verb that follows  \textbf{politics} may be either singular or plural .
 \textit{
	\begin{itemize}
	\item The key question in British politics was how long the prime minister could survive.
	\item He quickly involved himself in local politics.
	\item ...a crucial watershed in the politics of the German right.
	\item Politics is by no means the only arena in which women are excelling.
	\end{itemize}
}
\item plural noun \\
Your \textbf{politics} are your beliefs about how a country ought to be governed .
 \textit{
	\begin{itemize}
	\item My politics are well to the left of centre.
	\end{itemize}
}
\item uncountable noun \\
\textbf{Politics} is the study of the ways in which countries are governed.
 \textit{
	\begin{itemize}
	\item He began studying politics and medieval history.
	\item ...young politics graduates.
	\end{itemize}
}
\item plural noun \\
\textbf{Politics} can be used to talk about the ways that power is shared in an organization and the ways it is affected by personal relationships between people who work together. The verb that follows \textbf{politics} may be either singular or plural.
 \textit{
	\begin{itemize}
	\item You need to understand how office politics influence the working environment.
	\end{itemize}
}
\end{enumerate}

\section*{possession}
{\large \color{blue}  possessions  }
\subsection*{Explain}
\begin{enumerate}
\item uncountable noun \\
If you are \textbf{in}  \textbf{possession}  \textbf{of} something, you have it, because you have obtained it or because it belongs to you.
 \textit{
	\begin{itemize}
	\item Those documents are now in the possession of the Guardian.
	\item We should go up and take possession of the land.
	\item He was also charged with illegal possession of firearms.
	\item Religious pamphlets were found in their possession.
	\end{itemize}
}
\item countable noun \\
Your \textbf{possessions} are the things that you own or have with you at a particular time.
 \textit{
	\begin{itemize}
	\item People had lost their homes and all their possessions.
	\item She had tidied away her possessions.
	\end{itemize}
}
\item countable noun \\
A country's \textbf{possessions} are countries or territories that it controls.
 \textit{
	\begin{itemize}
	\item All of them were French possessions at one time or another.
	\item ...Britain's imperial possessions.
	\end{itemize}
}
\item uncountable noun \\
\textbf{Possession} by an evil spirit is the situation when a person's mind and body is controlled by an evil spirit.
 \textit{
	\begin{itemize}
	\item They were convinced the girls' behaviour was due to possession by the devil.
	\end{itemize}
}
\end{enumerate}

\section*{straight}
{\large \color{blue}  straighter  straightest  straights  }
\subsection*{Explain}
\begin{enumerate}
\item adjective \\
A \textbf{straight} line or edge continues in the same direction and does not bend or curve.
 \textbf{Straight} is also an adverb .
 \textit{
	\begin{itemize}
	\item Keep the boat in a straight line.
	\item Using the straight edge as a guide, trim the cloth to size.
	\item His teeth were perfectly straight.
	\item There wasn't a single straight wall in the building.
	\item Stand straight and stretch the left hand to the right foot.
	\item Turn right and just basically walk straight, right over the river.
	\end{itemize}
}
\item adjective \\
\textbf{Straight} hair has no curls or waves in it.
 \textit{
	\begin{itemize}
	\item Grace had long straight dark hair which she wore in a bun.
	\end{itemize}
}
\item adverb \\
You use \textbf{straight} to indicate that the way from one place to another is very direct, with no changes
of direction.
 \textit{
	\begin{itemize}
	\item The ball fell straight to the feet of the striker.
	\item He finished his conversation and stood up, looking straight at me.
	\item Straight ahead were the low cabins of the motel.
	\end{itemize}
}
\item adverb \\
If you go \textbf{straight} to a place, you go there immediately.
 \textit{
	\begin{itemize}
	\item As always, we went straight to the experts for advice.
	\item We'll go to a meeting in Birmingham and come straight back.
	\end{itemize}
}
\item adjective \\
If you give someone a \textbf{straight}  answer , you answer them clearly and honestly .
 \textbf{Straight} is also an adverb.
 \textit{
	\begin{itemize}
	\item What a shifty arguer he is, refusing ever to give a straight answer.
	\item I lost my temper and told him straight that I hadn't been looking for any job.
	\end{itemize}
}
\item adjective \\
\textbf{Straight} means following one after the other, with no gaps or intervals .
 \textbf{Straight} is also an adverb.
 \textit{
	\begin{itemize}
	\item They'd won 12 straight games before they lost.
	\item He called from Weddington, having been there for 31 hours straight.
	\end{itemize}
}
\item adjective \\
A \textbf{straight}  choice or a \textbf{straight} fight involves only two people or things.
 \textit{
	\begin{itemize}
	\item It's a straight choice between low-paid jobs and no jobs.
	\item Each has several times beaten the other in a straight fight.
	\end{itemize}
}
\item adjective \\
If you describe someone as \textbf{straight} , you mean that they are normal and conventional, for example in their opinions and
in the way they live.
 \textit{
	\begin{itemize}
	\item Dorothy was described as a very straight woman, a very strict Christian who was married
to her job.
	\end{itemize}
}
\item adjective \\
If you describe someone as \textbf{straight} , you mean that they are heterosexual rather than homosexual .
 \textbf{Straight} is also a noun .
 \textit{
	\begin{itemize}
	\item His sexual orientation was a lot more gay than straight.
	\item Marty of New York describes herself as a straight female.
	\item ...a standard of sexual conduct that applies equally to gays and straights.
	\end{itemize}
}
\item adjective \\
A \textbf{straight} drink, especially an alcoholic drink, has not had another liquid such as water added to it.
 \textit{
	\begin{itemize}
	\item ...a large straight whiskey without ice.
	\item Children should not drink fruit juices straight.
	\end{itemize}
}
\item countable noun \\
On a racetrack, a \textbf{straight} is a section of the track that is straight, rather than curved.
 \textit{
	\begin{itemize}
	\item Our cars were clearly too slow along the straights.
	\item I went to overtake him on the back straight on the last lap.
	\end{itemize}
}
\item  \\
 get sth straight \textit{
	\begin{itemize}
	\end{itemize}
}
\item  \\
 go straight \textit{
	\begin{itemize}
	\end{itemize}
}
\item  \\
 on the straight and narrow \textit{
	\begin{itemize}
	\end{itemize}
}
\end{enumerate}

\section*{practice}
{\large \color{blue}  practices  }
\subsection*{Explain}
\begin{enumerate}
\item countable noun \\
You can refer to something that people do regularly as a \textbf{practice} .
 \textit{
	\begin{itemize}
	\item Some firms have cut workers' pay below the level set in their contract, a practice
that is illegal in Germany.
	\item The Prime Minister demanded a public inquiry into bank practices.
	\end{itemize}
}
\item variable noun \\
\textbf{Practice}  means doing something regularly in order to be able to do it better . A \textbf{practice} is one of these periods of doing something.
 \textit{
	\begin{itemize}
	\item She was taking all three of her daughters to basketball practice every day.
	\item ...the hard practice necessary to develop from a learner to an accomplished musician.
	\item The defending world racing champion recorded the fastest time in a final practice
today.
	\end{itemize}
}
\item uncountable noun \\
The work done by doctors and lawyers is referred to as the \textbf{practice} of medicine and law. People's religious activities are referred to as the \textbf{practice} of a religion .
 \textit{
	\begin{itemize}
	\item ...the practice of internal medicine.
	\item I eventually realized I had to change my attitude toward medical practice.
	\item ...a law guaranteeing the people freedom of conscience and religious practice.
	\end{itemize}
}
\item countable noun \\
A doctor's or lawyer's \textbf{practice} is his or her business , often shared with other doctors or lawyers.
 \textit{
	\begin{itemize}
	\item The new doctor's practice was miles away from where I lived.
	\item My law practice isn't the most important thing in my life, you know.
	\end{itemize}
}
\item  \\
 in practice \textit{
	\begin{itemize}
	\end{itemize}
}
\item  \\
 normal practice/standard practice \textit{
	\begin{itemize}
	\end{itemize}
}
\item  \\
 out of practice \textit{
	\begin{itemize}
	\end{itemize}
}
\item  \\
 practice makes perfect \textit{
	\begin{itemize}
	\end{itemize}
}
\item  \\
 put into practice \textit{
	\begin{itemize}
	\end{itemize}
}
\end{enumerate}

\section*{strange}
{\large \color{blue}  stranger  strangest  }
\subsection*{Explain}
\begin{enumerate}
\item adjective \\
Something that is \textbf{strange} is unusual or unexpected , and makes you feel  slightly  nervous or afraid .
 \textit{
	\begin{itemize}
	\item Then a strange thing happened.
	\item There was something strange about the flickering blue light.
	\item It's strange how things turn out.
	\end{itemize}
}
\item adjective \\
A \textbf{strange} place is one that you have never been to before. A \textbf{strange} person is someone that you have never met before.
 \textit{
	\begin{itemize}
	\item I ended up alone in a strange city.
	\item She was faced with a new job, in unfamiliar surroundings with strange people.
	\end{itemize}
}
\item graded adjective \\
If you feel \textbf{strange} , you have an unpleasant or uncomfortable  feeling , either physical or emotional .
 \textit{
	\begin{itemize}
	\item I felt all dizzy and strange.
	\end{itemize}
}
\end{enumerate}

\section*{rack}
{\large \color{blue}  racks  racking  racked  }
\subsection*{Explain}
\begin{enumerate}
\item countable noun \\
A \textbf{rack} is a frame or shelf , usually with bars or hooks , that is used for holding things or for hanging things on.
 \textit{
	\begin{itemize}
	\item My rucksack was too big for the luggage rack.
	\item You have to fight to reach the racks of clothes but the bargains are amazing.
	\end{itemize}
}
\item verb \\
If someone \textbf{is racked}  \textbf{by} something such as illness or anxiety , it causes them great suffering or pain .
 \textit{
	\begin{itemize}
	\item His already infirm body was racked by high fever.
	\item The country is now racked by three violent separatist movements.
	\item ...a teenager racked with guilt and anxiety.
	\end{itemize}
}
\item  \\
 rack one's brain \textit{
	\begin{itemize}
	\end{itemize}
}
\item  \\
 on the rack \textit{
	\begin{itemize}
	\end{itemize}
}
\item  \\
 go to rack and ruin \textit{
	\begin{itemize}
	\end{itemize}
}
\item  \\
 off the rack \textit{
	\begin{itemize}
	\end{itemize}
}
\end{enumerate}

\section*{sure}
{\large \color{blue}  surer  surest  }
\subsection*{Explain}
\begin{enumerate}
\item adjective \\
If you are \textbf{sure} that something is true , you are certain that it is true. If you are not \textbf{sure} about something, you do not know for certain what the true situation is.
 \textit{
	\begin{itemize}
	\item He'd never been in a class before and he was not even sure that he should have been
teaching.
	\item The president has never been sure which direction he wanted to go in on this issue.
	\item She was no longer sure how she felt about him.
	\item It is impossible to be sure about the value of land.
	\end{itemize}
}
\item adjective \\
If someone is \textbf{sure of}  getting something, they will  definitely get it or they think they will definitely get it.
 \textit{
	\begin{itemize}
	\item It's better to pay so that you can be sure of getting quality.
	\item It is the self-assurance of the new generation which makes them sure of their success.
	\end{itemize}
}
\item phrase \\
If you say that something \textbf{is sure to}  happen , you are emphasizing your belief that it will happen.
 \textit{
	\begin{itemize}
	\item With over 80 beaches to choose from, you are sure to find a place to lay your towel.
	\item Anyone who goes food shopping without a list is sure to forget the things they really
need.
	\end{itemize}
}
\item adjective \\
\textbf{Sure} is used to emphasize that something such as a sign or ability is reliable or accurate .
 \textit{
	\begin{itemize}
	\item Sharpe's leg and shoulder began to ache, a sure sign of rain.
	\item She has a sure grasp of social issues such as literacy, poverty and child care.
	\end{itemize}
}
\item adjective \\
If you tell someone to \textbf{be sure}  \textbf{to} do something, you mean that they must not forget to do it.
 \textit{
	\begin{itemize}
	\item Be sure to read about how mozzarella is made, on page 65.
	\item Be sure you get your daily quota of calcium.
	\end{itemize}
}
\item convention \\
\textbf{Sure} is an informal way of saying 'yes' or 'all right '.
 \textit{
	\begin{itemize}
	\item 'He rang you?'—'Sure. Last night.'
	\item 'I'd like to be alone, O.K?'—'Sure. O.K.'
	\item 'We'll phone and you can make an appointment'—'Sure. What time do you want to go?'
	\end{itemize}
}
\item adverb \\
You can use \textbf{sure} in order to emphasize what you are saying.
 \textit{
	\begin{itemize}
	\item 'Has the whole world just gone crazy?'—'Sure looks that way, doesn't it.'
	\item It sure is hot, he thought.
	\end{itemize}
}
\item  \\
 sure enough \textit{
	\begin{itemize}
	\end{itemize}
}
\item  \\
 for sure \textit{
	\begin{itemize}
	\end{itemize}
}
\item  \\
 make sure \textit{
	\begin{itemize}
	\end{itemize}
}
\item  \\
 make sure \textit{
	\begin{itemize}
	\end{itemize}
}
\item  \\
 sure thing \textit{
	\begin{itemize}
	\end{itemize}
}
\item  \\
 sure thing \textit{
	\begin{itemize}
	\end{itemize}
}
\item  \\
 to be sure \textit{
	\begin{itemize}
	\end{itemize}
}
\item  \\
 be sure of oneself \textit{
	\begin{itemize}
	\end{itemize}
}
\end{enumerate}

\section*{regime}
{\large \color{blue}  regimes  }
\subsection*{Explain}
\begin{enumerate}
\item countable noun \\
If you refer to a government or system of running a country as a \textbf{regime} , you are critical of it because you think it is not democratic and uses unacceptable  methods .
 \textit{
	\begin{itemize}
	\item ...the collapse of the Fascist regime at the end of the war.
	\item The emerging capitalist order was giving rise to harsh regimes.
	\end{itemize}
}
\item countable noun \\
A \textbf{regime} is the way that something such as an institution , company , or economy is run , especially when it involves tough or severe action.
 \textit{
	\begin{itemize}
	\item The authorities moved him to the less rigid regime of an open prison.
	\item ...a drastic regime of economic reform and financial discipline.
	\end{itemize}
}
\item countable noun \\
A \textbf{regime} is a set of rules about food, exercise , or beauty that some people follow in order to stay  healthy or attractive .
 \textit{
	\begin{itemize}
	\item He has a new fitness regime to strengthen his back.
	\end{itemize}
}
\end{enumerate}

\section*{susceptible}
{\large \color{blue}  }
\subsection*{Explain}
\begin{enumerate}
\item adjective \\
If you are \textbf{susceptible to} something or someone, you are very likely to be influenced by them.
 \textit{
	\begin{itemize}
	\item Young people are the most susceptible to advertisements.
	\item James was extremely susceptible to flattery.
	\item He was, she believes, unusually susceptible to women.
	\end{itemize}
}
\item adjective \\
If you are \textbf{susceptible to} a disease or injury , you are very likely to be affected by it.
 \textit{
	\begin{itemize}
	\item Walking with weights makes the shoulders very susceptible to injury.
	\item Diesel exhaust is particularly aggravating to many susceptible individuals.
	\end{itemize}
}
\item graded adjective \\
A \textbf{susceptible} person is very easily influenced emotionally.
 \textit{
	\begin{itemize}
	\item Hers was a susceptible nature.
	\end{itemize}
}
\end{enumerate}

\section*{seal}
{\large \color{blue}  seals  sealing  sealed  }
\subsection*{Explain}
\begin{enumerate}
\item verb \\
When you \textbf{seal} an envelope, you close it by folding part of it over and sticking it down, so that it cannot be opened without being torn .
 \textit{
	\begin{itemize}
	\item He sealed the envelope and put on a stamp.
	\item Write your letter and seal it in a blank envelope.
	\item A courier was despatched with two sealed envelopes.
	\end{itemize}
}
\item verb \\
If you \textbf{seal} a container or an opening, you cover it with something in order to prevent air, liquid,
or other material getting in or out. If you \textbf{seal} something \textbf{in} a container, you put it inside and then close the container tightly.
 \textit{
	\begin{itemize}
	\item She merely filled the containers, sealed them with a cork, and pasted on labels.
	\item A woman picks them up and seals them in plastic bags.
	\item ...a lid to seal in heat and keep food moist.
	\item ...a hermetically sealed, leak-proof packet.
	\end{itemize}
}
\item countable noun \\
The \textbf{seal} on a container or opening is the part where it has been sealed.
 \textit{
	\begin{itemize}
	\item When assembling the pie, wet the edges where the two crusts join, to form a seal.
	\end{itemize}
}
\item countable noun \\
A \textbf{seal} is a device or a piece of material, for example in a machine, which closes an opening
tightly so that air, liquid, or other substances cannot get in or out.
 \textit{
	\begin{itemize}
	\item Check seals on fridges and freezers regularly.
	\end{itemize}
}
\item countable noun \\
A \textbf{seal} is something such as a piece of sticky paper or wax that is fixed to a container or door and must be broken before the container or door can be opened.
 \textit{
	\begin{itemize}
	\item The seal on the box broke when it fell from its hiding-place.
	\item Protestors banged on the sides of the lorry and broke customs seals on the doors.
	\end{itemize}
}
\item countable noun \\
A \textbf{seal} is a special mark or design, for example on a document, representing someone or something.
It may be used to show that something is genuine or officially approved.
 \textit{
	\begin{itemize}
	\item ...a supply of note paper bearing the Presidential seal.
	\item The best wines are entitled to a numbered seal of quality.
	\end{itemize}
}
\item verb \\
If someone in authority \textbf{seals} an area, they stop people entering or passing through it, for example by placing barriers in the way.
 \textbf{Seal off} means the same as seal1 .
 \textit{
	\begin{itemize}
	\item The soldiers were deployed to help police seal the border.
	\item A wide area round the building is sealed to all traffic except the emergency services.
	\item Police and troops sealed off the area after the attack.
	\item Soldiers there are going to seal the airport off.
	\end{itemize}
}
\item verb \\
To \textbf{seal} something means to make it definite or confirm how it is going to be.
 \textit{
	\begin{itemize}
	\item McLaren are close to sealing a deal with Renault.
	\item The election will seal his destiny one way or the other.
	\item His artistic character was sealed by his experiences of the First World War.
	\end{itemize}
}
\item  \\
 set/put the seal on \textit{
	\begin{itemize}
	\end{itemize}
}
\item  \\
 under seal \textit{
	\begin{itemize}
	\end{itemize}
}
\end{enumerate}

\section*{upright}
{\large \color{blue}  uprights  }
\subsection*{Explain}
\begin{enumerate}
\item adjective \\
If you are sitting or standing  \textbf{upright} , you are sitting or standing with your back  straight , rather than bending or lying down.
 \textit{
	\begin{itemize}
	\item Helen sat upright in her chair.
	\item ...those who had managed to remain upright.
	\item Jerrold pulled himself upright on the bed.
	\item He moved into an upright position.
	\end{itemize}
}
\item adjective \\
An \textbf{upright}  vacuum  cleaner or freezer is tall rather than wide .
 \textit{
	\begin{itemize}
	\item ...the latest state-of-the-art upright vacuum cleaners.
	\end{itemize}
}
\item adjective \\
An \textbf{upright}  chair has a straight back and no arms .
 \textit{
	\begin{itemize}
	\item He was sitting on an upright chair beside his bed, reading.
	\end{itemize}
}
\item countable noun \\
You can refer to vertical posts or the vertical parts of an object as \textbf{uprights} .
 \textit{
	\begin{itemize}
	\item ...the uprights of a four-poster bed.
	\end{itemize}
}
\item adjective \\
You can describe people as \textbf{upright} when they are careful to follow  acceptable  rules of behaviour and behave in a moral  way .
 \textit{
	\begin{itemize}
	\item ...a very upright, trustworthy man.
	\end{itemize}
}
\end{enumerate}

\section*{spring}
{\large \color{blue}  springs  springing  sprang  sprung  }
\subsection*{Explain}
\begin{enumerate}
\item variable noun \\
\textbf{Spring} is the season between winter and summer when the weather becomes warmer and plants start to grow again.
 \textit{
	\begin{itemize}
	\item We planted bulbs to flower in spring.
	\item The Labor government of Western Australia has an election due next spring.
	\item We met again in the spring of 1977.
	\item The apricot plant provides delicate, white spring flowers.
	\end{itemize}
}
\item countable noun \\
A \textbf{spring} is a spiral of wire which returns to its original shape after it is pressed or pulled .
 \textit{
	\begin{itemize}
	\item As the mattress wears, the springs soften and do not support your spine.
	\item Both springs in the fuel pump were broken.
	\end{itemize}
}
\item countable noun \\
A \textbf{spring} is a place where water comes up through the ground. It is also the water that comes
from that place.
 \textit{
	\begin{itemize}
	\item To the north are the hot springs of Banyas de Sant Loan.
	\end{itemize}
}
\item verb \\
When a person or animal \textbf{springs} , they jump upwards or forwards suddenly or quickly.
 \textit{
	\begin{itemize}
	\item He sprang to his feet, grabbing his keys off the coffee table.
	\item Outside each door a guard sprang to attention as they approached.
	\item Throwing back the sheet, he sprang from the bed.
	\item The lion roared once and sprang.
	\end{itemize}
}
\item verb \\
If something \textbf{springs} in a particular direction, it moves suddenly and quickly.
 \textit{
	\begin{itemize}
	\item Sadly when the lid of the boot sprang open, it was empty.
	\end{itemize}
}
\item verb \\
If things or people \textbf{spring into action} or \textbf{spring to life} , they suddenly start being active or suddenly come into existence.
 \textit{
	\begin{itemize}
	\item When she contacted me at the beginning of August to enlist support, Sharon and I
sprang into action.
	\item ...new industries which had sprung into life during the 1920s.
	\end{itemize}
}
\item verb \\
If one thing \textbf{springs from} another thing, it is the result of it.
 \textit{
	\begin{itemize}
	\item Ethiopia's art springs from her early Christian as well as her Muslim heritage.
	\item His anger sprang from his suffering.
	\end{itemize}
}
\item verb \\
If a boat or container \textbf{springs a leak} , water or some other liquid starts coming in or out through a crack .
 \textit{
	\begin{itemize}
	\item The yacht has sprung a leak in the hull.
	\end{itemize}
}
\item verb \\
If you \textbf{spring} some news or a surprise  \textbf{on} someone, you tell them something that they did not expect to hear , without warning them.
 \textit{
	\begin{itemize}
	\item The two superpower leaders sprang a surprise at a ceremony in the White House yesterday
by signing a trade deal.
	\item Mclaren sprang a new idea on him.
	\end{itemize}
}
\end{enumerate}

\section*{variable}
{\large \color{blue}  variables  }
\subsection*{Explain}
\begin{enumerate}
\item adjective \\
Something that is \textbf{variable} changes quite often, and there usually seems to be no fixed  pattern to these changes.
 \textit{
	\begin{itemize}
	\item The potassium content of foodstuffs is very variable.
	\item There was a bit of a wind and it was blowing onshore, variable, but quite strong.
	\item ...a variable rate of interest.
	\end{itemize}
}
\item countable noun \\
A \textbf{variable} is a factor that can change in quality, quantity, or size , which you have to take into account in a situation .
 \textit{
	\begin{itemize}
	\item Decisions could be made on the basis of price, delivery dates, or any other variable.
	\item Other variables in making forecasts include the weather and the economic climate.
	\end{itemize}
}
\item countable noun \\
A \textbf{variable} is a quantity that can have any one of a set of values.
 \textit{
	\begin{itemize}
	\item It is conventional to place the independent variable on the right-hand side of an
equation.
	\end{itemize}
}
\end{enumerate}

\section*{statesman}
{\large \color{blue}  statesmen  }
\subsection*{Explain}
\begin{enumerate}
\item countable noun \\
A \textbf{statesman} is an important and experienced politician, especially one who is widely known and respected.
 \textit{
	\begin{itemize}
	\item Hamilton is a great statesman and political thinker.
	\end{itemize}
}
\end{enumerate}

\section*{vertical}
{\large \color{blue}  verticals  }
\subsection*{Explain}
\begin{enumerate}
\item adjective \\
Something that is \textbf{vertical}  stands or points straight up.
 \textit{
	\begin{itemize}
	\item The climber inched up a vertical wall of rock.
	\item The gadget can be attached to any vertical or near vertical surface.
	\end{itemize}
}
\item singular noun \\
\textbf{The vertical} is the direction that points straight up, at an angle of 90 degrees to a flat surface.
 \textit{
	\begin{itemize}
	\item Pluto seems to have suffered a major collision that tipped it 122 degrees from the
vertical.
	\end{itemize}
}
\item countable noun \\
A \textbf{vertical} is a line or structure that is vertical.
 \textit{
	\begin{itemize}
	\item As long as the verticals align, the design will look regular.
	\end{itemize}
}
\end{enumerate}

\section*{tissue}
{\large \color{blue}  tissues  }
\subsection*{Explain}
\begin{enumerate}
\item uncountable noun \\
In animals and plants, \textbf{tissue} consists of cells that are similar to each other in appearance and that have the same function.
 \textit{
	\begin{itemize}
	\item As we age we lose muscle tissue.
	\item Athletes have hardly any fatty tissue.
	\item All the cells and tissues in the body benefit from the increased intake of oxygen.
	\end{itemize}
}
\item uncountable noun \\
\textbf{Tissue} or \textbf{tissue paper} is thin paper that is used for wrapping things that are easily  damaged , such as objects made of glass or china .
 \textit{
	\begin{itemize}
	\end{itemize}
}
\item countable noun \\
A \textbf{tissue} is a piece of thin soft paper that you use to blow your nose .
 \textit{
	\begin{itemize}
	\item ...a box of tissues.
	\end{itemize}
}
\end{enumerate}

\section*{vulnerable}
{\large \color{blue}  }
\subsection*{Explain}
\begin{enumerate}
\item adjective \\
Someone who is \textbf{vulnerable} is weak and without protection , with the result that they are easily hurt physically or emotionally.
 \textit{
	\begin{itemize}
	\item Old people are particularly vulnerable members of our society.
	\end{itemize}
}
\item adjective \\
If a person, animal, or plant is \textbf{vulnerable}  \textbf{to} a disease, they are more likely to get it than other people, animals, or plants.
 \textit{
	\begin{itemize}
	\item People with high blood pressure are especially vulnerable to diabetes.
	\item Plants that are growing vigorously are less likely to be vulnerable to disease.
	\end{itemize}
}
\item adjective \\
Something that is \textbf{vulnerable} can be easily harmed or affected by something bad .
 \textit{
	\begin{itemize}
	\item Their tanks would be vulnerable to attack from the air.
	\item ...a table showing which cars are most vulnerable to theft.
	\item Goodyear could be vulnerable in a prolonged economic slump.
	\end{itemize}
}
\end{enumerate}

\section*{academic}
{\large \color{blue}  academics  }
\subsection*{Explain}
\begin{enumerate}
\item adjective \\
\textbf{Academic} is used to describe things that relate to the work done in schools , colleges, and universities, especially work which involves studying and reasoning rather than practical or technical skills .
 \textit{
	\begin{itemize}
	\item Their academic standards are high.
	\item I was terrible at school and left with few academic qualifications.
	\end{itemize}
}
\item adjective \\
\textbf{Academic} is used to describe things that relate to schools, colleges, and universities.
 \textit{
	\begin{itemize}
	\item ...the start of the last academic year.
	\item I'd had enough of academic life.
	\end{itemize}
}
\item adjective \\
\textbf{Academic} is used to describe work, or a school, college, or university, that places emphasis on studying and reasoning rather than on practical or technical skills.
 \textit{
	\begin{itemize}
	\item The author has settled for a more academic approach.
	\item I went to a school that was very academic.
	\end{itemize}
}
\item adjective \\
Someone who is \textbf{academic} is good at studying.
 \textit{
	\begin{itemize}
	\item The system is failing most disastrously among less academic children.
	\end{itemize}
}
\item countable noun \\
An \textbf{academic} is a member of a university or college who teaches or does research .
 \textit{
	\begin{itemize}
	\end{itemize}
}
\item adjective \\
You can say that a discussion or situation is \textbf{academic} if you think it is not important because it has no real  effect or cannot happen .
 \textit{
	\begin{itemize}
	\item This was not an academic exercise–soldiers' lives were at risk.
	\item Such is the size of the problem that these arguments are purely academic.
	\end{itemize}
}
\end{enumerate}

\section*{beard}
{\large \color{blue}  beards  }
\subsection*{Explain}
\begin{enumerate}
\item countable noun \\
A man's \textbf{beard} is the hair that grows on his chin and cheeks .
 \textit{
	\begin{itemize}
	\item He's decided to grow a beard.
	\item ...Charlie's bushy black beard.
	\end{itemize}
}
\end{enumerate}

\section*{alone}
{\large \color{blue}  }
\subsection*{Explain}
\begin{enumerate}
\item adjective \\
When you are \textbf{alone} , you are not with any other people.
 \textbf{Alone} is also an adverb .
 \textit{
	\begin{itemize}
	\item There is nothing so frightening as to be alone in a combat situation.
	\item He was all alone in the middle of the hall.
	\item She has lived alone in this house for almost five years now.
	\item He was sitting alone at a table reading a newspaper.
	\end{itemize}
}
\item adjective \\
If one person is \textbf{alone}  \textbf{with} another person, or if two or more people are \textbf{alone} , they are together , without anyone else present .
 \textit{
	\begin{itemize}
	\item I couldn't imagine why he would want to be alone with me.
	\item My brother and I were alone with Vincent.
	\end{itemize}
}
\item adjective \\
If you say that you are \textbf{alone} or feel  \textbf{alone} , you mean that nobody who is with you, or nobody at all, cares about you.
 \textit{
	\begin{itemize}
	\item Never in her life had she felt so alone, so abandoned.
	\item He found himself alone in a hostile world.
	\end{itemize}
}
\item adverb \\
You say that one person or thing \textbf{alone} does something when you are emphasizing that only one person or thing is involved .
 \textit{
	\begin{itemize}
	\item You alone should determine what is right for you.
	\item They were convicted on forensic evidence alone.
	\end{itemize}
}
\item adverb \\
If you say that one person or thing \textbf{alone} is responsible for part of an amount , you are emphasizing the size of that part and the size of the total amount.
 \textit{
	\begin{itemize}
	\item The BBC alone is sending 300 technicians, directors and commentators.
	\item Megastars like Jack Nicholson, who made £50 million from Batman alone, are unlikely
to be affected.
	\end{itemize}
}
\item adjective \\
If someone is \textbf{alone}  \textbf{in} doing something, they are the only person doing it, and so are different from other people.
 \textbf{Alone} is also an adverb.
 \textit{
	\begin{itemize}
	\item If you sometimes feel pain at the front of your shins after running, you are far
from alone.
	\item Am I alone in thinking that this scandal should finish his career?
	\item Alone among the great Victorian novelists, Hardy lived long enough to see his work
adapted for the screen.
	\item I alone was sane, I thought, in a world of crazy people.
	\end{itemize}
}
\item adverb \\
When someone does something \textbf{alone} , they do it without help from other people.
 \textit{
	\begin{itemize}
	\item Bringing up a child alone should give you a sense of achievement.
	\item He was working alone and did not have an accomplice.
	\end{itemize}
}
\item  \\
 go it alone \textit{
	\begin{itemize}
	\end{itemize}
}
\end{enumerate}

\section*{carrot}
{\large \color{blue}  carrots  }
\subsection*{Explain}
\begin{enumerate}
\item variable noun \\
\textbf{Carrots} are long, thin , orange-coloured vegetables. They grow under the ground , and have green  shoots above the ground.
 \textit{
	\begin{itemize}
	\end{itemize}
}
\item countable noun \\
Something that is offered to people in order to persuade them to do something can be referred to as a \textbf{carrot} . Something that is meant to persuade people not to do something can be referred to in the same sentence as a ' stick '.
 \textit{
	\begin{itemize}
	\item They will be set targets, with a carrot of extra cash and pay if they achieve them.
	\item Why the new emphasis on sticks instead of diplomatic carrots?
	\end{itemize}
}
\end{enumerate}

\section*{amiable}
{\large \color{blue}  }
\subsection*{Explain}
\begin{enumerate}
\item adjective \\
Someone who is \textbf{amiable} is friendly and pleasant to be with.
 \textit{
	\begin{itemize}
	\item She had been surprised at how amiable and polite he had seemed.
	\end{itemize}
}
\end{enumerate}

\section*{cushion}
{\large \color{blue}  cushions  cushioning  cushioned  }
\subsection*{Explain}
\begin{enumerate}
\item countable noun \\
A \textbf{cushion} is a fabric  case filled with soft  material , which you put on a seat to make it more comfortable .
 \textit{
	\begin{itemize}
	\item ...a velvet cushion.
	\end{itemize}
}
\item countable noun \\
A \textbf{cushion} is a soft pad or barrier , especially one that protects something.
 \textit{
	\begin{itemize}
	\item The company provides a styrofoam cushion to protect the tablets during shipping.
	\end{itemize}
}
\item verb \\
Something that \textbf{cushions} an object when it hits something protects it by reducing the force of the impact .
 \textit{
	\begin{itemize}
	\item ...a giant airbag to cushion your landing.
	\item The suspension is designed to cushion passengers from the effects of rough roads.
	\end{itemize}
}
\item verb \\
To \textbf{cushion} the effect of something unpleasant means to reduce it.
 \textit{
	\begin{itemize}
	\item He was trying to cushion the blow of this terrible news.
	\item The price rises will be cushioned by welfare benefits.
	\item The subsidies are designed to cushion farmers against unpredictable weather.
	\end{itemize}
}
\item countable noun \\
Something that is a \textbf{cushion}  \textbf{against} something unpleasant reduces its effect.
 \textit{
	\begin{itemize}
	\item Housing benefit provides a cushion against hardship.
	\end{itemize}
}
\end{enumerate}

\section*{barren}
{\large \color{blue}  }
\subsection*{Explain}
\begin{enumerate}
\item adjective \\
A \textbf{barren}  landscape is dry and bare, and has very few plants and no trees.
 \textit{
	\begin{itemize}
	\item ...the country's landscape of high barren mountains.
	\end{itemize}
}
\item adjective \\
\textbf{Barren} land consists of soil that is so poor that plants cannot grow in it.
 \textit{
	\begin{itemize}
	\item He also wants to use the water to irrigate barren desert land.
	\end{itemize}
}
\item adjective \\
If you describe something such as an activity or a period of your life as \textbf{barren} , you mean that you achieve no success during it or that it has no useful results.
 \textit{
	\begin{itemize}
	\item ...politics that are banal and barren of purpose.
	\item ...the player, who ended a 14-month barren spell by winning the Tokyo event in October.
	\item As the leaves of autumn wither and fall, so has my own life become barren.
	\end{itemize}
}
\item adjective \\
If you describe a room or a place as \textbf{barren} , you do not like it because it has almost no furniture or other objects in it.
 \textit{
	\begin{itemize}
	\item The room was austere, nearly barren of furniture or decoration.
	\item Six stale loaves of brown bread formed a dark blot on the otherwise barren shelves.
	\end{itemize}
}
\item adjective \\
A \textbf{barren} woman or female animal is unable to have babies .
 \textit{
	\begin{itemize}
	\item He prayed that his barren wife would one day have a child.
	\item ...a three-year-old barren mare.
	\end{itemize}
}
\end{enumerate}

\section*{director}
{\large \color{blue}  directors  }
\subsection*{Explain}
\begin{enumerate}
\item countable noun \\
The \textbf{director} of a play, film, or television programme is the person who decides how it will  appear on stage or screen , and who tells the actors and technical staff what to do.
 \textit{
	\begin{itemize}
	\end{itemize}
}
\item countable noun \\
In some organizations and public  authorities , the person in charge is referred to as \textbf{the}  \textbf{director} .
 \textit{
	\begin{itemize}
	\item ...the director of the intensive care unit at Guy's Hospital.
	\item ...the Director of Public Prosecutions.
	\item She has just been appointed artistic director of Queensland Theatre Company.
	\end{itemize}
}
\item countable noun \\
The \textbf{directors} of a company are its most senior  managers , who meet regularly to make important  decisions about how it will be run .
 \textit{
	\begin{itemize}
	\item He served on the board of directors of a local bank.
	\end{itemize}
}
\item countable noun \\
The \textbf{director} of an orchestra or choir is the person who is conducting it.
 \textit{
	\begin{itemize}
	\end{itemize}
}
\end{enumerate}

\section*{blank}
{\large \color{blue}  blanks  blanking  blanked  }
\subsection*{Explain}
\begin{enumerate}
\item adjective \\
Something that is \textbf{blank} has nothing on it.
 \textit{
	\begin{itemize}
	\item We could put some of the pictures over on that blank wall over there.
	\item He tore a blank page from his notebook.
	\item ... a blank screen.
	\end{itemize}
}
\item countable noun \\
A \textbf{blank} is a space which is left in a piece of writing or on a printed form for you to fill
in particular information.
 \textit{
	\begin{itemize}
	\item Put a word in each blank to complete the sentence.
	\end{itemize}
}
\item adjective \\
If you look  \textbf{blank} , your face shows no feeling, understanding, or interest.
 \textit{
	\begin{itemize}
	\item Abbot looked blank. 'I don't quite follow, sir.'.
	\item His daughter gave him a blank look.
	\end{itemize}
}
\item singular noun \\
If your mind or memory is \textbf{a blank} , you cannot think of anything or remember anything.
 \textit{
	\begin{itemize}
	\item I'm sorry, but my mind is a blank.
	\item I came round in hospital and did not know where I was. Everything was a complete
blank.
	\end{itemize}
}
\item countable noun \\
\textbf{Blanks} are gun  cartridges which contain explosive but do not contain a bullet , so that they cause no harm when the gun is fired.
 \textit{
	\begin{itemize}
	\item ...a starter pistol which only fires blanks.
	\end{itemize}
}
\item  \\
 to draw a blank \textit{
	\begin{itemize}
	\end{itemize}
}
\item  \\
 go blank \textit{
	\begin{itemize}
	\end{itemize}
}
\end{enumerate}

\section*{draft}
{\large \color{blue}  drafts  drafting  drafted  }
\subsection*{Explain}
\begin{enumerate}
\item countable noun \\
A \textbf{draft} is an early version of a letter, book, or speech.
 \textit{
	\begin{itemize}
	\item I rewrote his rough draft, which was published under my name.
	\item I emailed a first draft of this article to him.
	\item ...a draft report from a major U.S. university.
	\item ...a draft law.
	\end{itemize}
}
\item verb \\
When you \textbf{draft} a letter, book, or speech, you write the first version of it.
 \textit{
	\begin{itemize}
	\item He drafted a standard letter to the editors.
	\item The legislation was drafted by House Democrats.
	\end{itemize}
}
\item verb \\
If you \textbf{are drafted} , you are ordered to serve in the armed forces, usually for a limited period of time.
 \textit{
	\begin{itemize}
	\item During the Second World War, he was drafted into the U.S. Army.
	\item He wasn't drafted for the war; he volunteered for the Navy.
	\end{itemize}
}
\item verb \\
If people \textbf{are drafted}  \textbf{into} a place, they are moved there to do a particular job .
 \textit{
	\begin{itemize}
	\item Extra police have been drafted into the town after the violence.
	\item The manager will make a special plea to draft the player into his squad as a replacement.
	\end{itemize}
}
\item singular noun \\
\textbf{The draft} is the practice of ordering people to serve in the armed forces, usually for a limited period of time.
 \textit{
	\begin{itemize}
	\item ...his effort to avoid the draft.
	\end{itemize}
}
\item countable noun \\
A \textbf{draft} is a written order for payment of money by a bank, especially from one bank to another.
 \textit{
	\begin{itemize}
	\item The money was payable by a draft drawn by the home.
	\item Ten days later Carmen received a bank draft for a plane ticket.
	\end{itemize}
}
\end{enumerate}

\section*{blind}
{\large \color{blue}  blinds  blinding  blinded  }
\subsection*{Explain}
\begin{enumerate}
\item adjective \\
Someone who is \textbf{blind} is unable to see because their eyes are damaged.
 \textbf{The blind} are people who are blind. This use could cause offence .
 \textit{
	\begin{itemize}
	\item I started helping him run the business when he went blind.
	\item How would you explain colour to a blind person?
	\item He was a teacher of the blind.
	\end{itemize}
}
\item verb \\
If something \textbf{blinds} you, it makes you unable to see, either for a short time or permanently.
 \textit{
	\begin{itemize}
	\item The sun hit the windscreen, momentarily blinding him.
	\end{itemize}
}
\item adjective \\
If you are \textbf{blind}  \textbf{with} something such as tears or a bright light, you are unable to see for a short time because of the tears or
light.
 \textit{
	\begin{itemize}
	\item Her mother groped for the back of the chair, her eyes blind with tears.
	\end{itemize}
}
\item adjective \\
If you say that someone is \textbf{blind to} a fact or a situation, you mean that they ignore it or are unaware of it, although you think that they should take notice of it or be aware of it.
 \textit{
	\begin{itemize}
	\item David's good looks and impeccable manners had always made her blind to his faults.
	\item All the time I was blind to your suffering.
	\end{itemize}
}
\item verb \\
If something \textbf{blinds} you \textbf{to} the real situation, it prevents you from realizing that it exists or from understanding it properly.
 \textit{
	\begin{itemize}
	\item He never allowed his love of Australia to blind him to his countrymen's faults.
	\end{itemize}
}
\item adjective \\
You can describe someone's beliefs or actions as \textbf{blind} when you think that they seem to take no notice of important facts or behave in an unreasonable way.
 \textit{
	\begin{itemize}
	\item ...her blind faith in the wisdom of the Church.
	\item Lesley yelled at him with blind, hating rage.
	\end{itemize}
}
\item adjective \\
A \textbf{blind}  corner is one that you cannot see round because something is blocking your view.
 \textit{
	\begin{itemize}
	\item He tried to overtake three cars on a blind corner and crashed head-on into a lorry.
	\end{itemize}
}
\item adjective \\
A \textbf{blind} wall or building is one which has no windows or doors .
 \textit{
	\begin{itemize}
	\item I remembered a huddle of stone buildings with blind walls.
	\end{itemize}
}
\item countable noun \\
A \textbf{blind} is a roll of cloth or paper which you can pull down over a window as a covering.
 \textit{
	\begin{itemize}
	\end{itemize}
}
\item  \\
 to turn a blind eye \textit{
	\begin{itemize}
	\end{itemize}
}
\end{enumerate}

\section*{drug}
{\large \color{blue}  drugs  drugging  drugged  }
\subsection*{Explain}
\begin{enumerate}
\item countable noun \\
A \textbf{drug} is a chemical which is given to people in order to treat or prevent an illness or disease.
 \textit{
	\begin{itemize}
	\item The drug will be useful to hundreds of thousands of infected people.
	\item ...the drug companies.
	\end{itemize}
}
\item countable noun \\
\textbf{Drugs} are substances that some people take because of their pleasant effects, but which
are usually illegal .
 \textit{
	\begin{itemize}
	\item His mother was on drugs, on cocaine.
	\item She was sure Leo was taking drugs.
	\item ...the problem of drug abuse.
	\end{itemize}
}
\item verb \\
If you \textbf{drug} a person or animal, you give them a chemical substance in order to make them sleepy or unconscious .
 \textit{
	\begin{itemize}
	\item They drugged the guard dog with doped meatballs.
	\item She was drugged and robbed.
	\item He grew tired, and drifted off into a drugged sleep.
	\end{itemize}
}
\item verb \\
If food or drink \textbf{is drugged} , a chemical substance is added to it in order to make someone sleepy or unconscious when they eat or drink it.
 \textit{
	\begin{itemize}
	\item I wonder now if that drink had been drugged.
	\item Anyone could have drugged that wine.
	\item A tourist was robbed after being given a drugged orange.
	\end{itemize}
}
\end{enumerate}

\section*{gauge}
{\large \color{blue}  gauges  gauging  gauged  }
\subsection*{Explain}
\begin{enumerate}
\item verb \\
If you \textbf{gauge} the speed or strength of something, or if you gauge an amount, you measure or calculate it, often by using a device of some kind.
 \textit{
	\begin{itemize}
	\item He gauged the wind at over thirty knots.
	\item Distance is gauged by journey time rather than miles.
	\end{itemize}
}
\item countable noun \\
A \textbf{gauge} is a device that measures the amount or quantity of something and shows the amount
measured.
 \textit{
	\begin{itemize}
	\item ...temperature gauges.
	\item ...pressure gauges.
	\end{itemize}
}
\item verb \\
If you \textbf{gauge} people's actions, feelings, or intentions in a particular situation, you carefully consider and judge them.
 \textit{
	\begin{itemize}
	\item ...as he gauged possible enemy moves and his own responses.
	\item His mood can be gauged by his reaction to the most trivial of incidents.
	\end{itemize}
}
\item singular noun \\
A \textbf{gauge}  \textbf{of} someone's feelings or a situation is a fact or event that can be used to judge them.
 \textit{
	\begin{itemize}
	\item The index is the government's chief gauge of future economic activity.
	\end{itemize}
}
\item countable noun \\
A \textbf{gauge} is the distance between the two rails on a railway line.
 \textit{
	\begin{itemize}
	\item ...a narrow gauge railway.
	\end{itemize}
}
\item countable noun \\
A \textbf{gauge} is the thickness of something, especially metal or wire.
 \textit{
	\begin{itemize}
	\end{itemize}
}
\end{enumerate}

\section*{dependent}
{\large \color{blue}  }
\subsection*{Explain}
\begin{enumerate}
\item adjective \\
To be \textbf{dependent}  \textbf{on} something or someone means to need them in order to succeed or be able to survive .
 \textit{
	\begin{itemize}
	\item The local economy is overwhelmingly dependent on oil and gas extraction.
	\item Up to two million people there are dependent on food aid.
	\item Britain became increasingly dependent upon American technology.
	\item In his own way, he was dependent on her.
	\item Just 26 per cent of households are married couples with dependent children.
	\end{itemize}
}
\item adjective \\
If one thing is \textbf{dependent}  \textbf{on} another, the first thing will be affected or determined by the second .
 \textit{
	\begin{itemize}
	\item How we cope with new roles is largely dependent on previous experience.
	\item The treatment of infertility is largely dependent on the ability of couples to pay.
	\end{itemize}
}
\end{enumerate}

\section*{grass}
{\large \color{blue}  grasses  grassing  grassed  }
\subsection*{Explain}
\begin{enumerate}
\item variable noun \\
\textbf{Grass} is a very common plant consisting of large numbers of thin, spiky , green leaves that cover the surface of the ground.
 \textit{
	\begin{itemize}
	\item Small things stirred in the grass around the tent.
	\item The lawn contained a mixture of grasses.
	\end{itemize}
}
\item singular noun \\
If you talk about \textbf{the grass} , you are referring to an area of ground that is covered with grass, for example in your garden .
 \textit{
	\begin{itemize}
	\item In the old days, there were strict fines for walking on the grass or missing a study
period.
	\item I'm going to cut the grass.
	\end{itemize}
}
\item uncountable noun \\
\textbf{Grass} is the same as marijuana .
 \textit{
	\begin{itemize}
	\item I started smoking grass when I was about sixteen.
	\end{itemize}
}
\item verb \\
If you say that one person \textbf{grasses on} another, the first person tells the police or other authorities about something criminal or wrong which the second person has done.
 \textbf{Grass up} means the same as grass .
 \textit{
	\begin{itemize}
	\item His sister wants him to grass on the members of his own gang.
	\item He was repeatedly attacked by other inmates, who accused him of grassing.
	\item How many of them are going to grass up their own kids to the police?
	\end{itemize}
}
\item countable noun \\
A \textbf{grass} is someone who tells the police or other authorities about criminal activities that
they know about.
 \textit{
	\begin{itemize}
	\end{itemize}
}
\item  \\
 the grass is greener \textit{
	\begin{itemize}
	\end{itemize}
}
\item  \\
 put out to grass \textit{
	\begin{itemize}
	\end{itemize}
}
\end{enumerate}

\section*{dull}
{\large \color{blue}  duller  dullest  dulls  dulling  dulled  }
\subsection*{Explain}
\begin{enumerate}
\item adjective \\
If you describe someone or something as \textbf{dull} , you mean they are not interesting or exciting .
 \textit{
	\begin{itemize}
	\item They are both nice people but can be rather dull.
	\item I felt she found me boring and dull.
	\item The documentary lasts for more than two-and-a-half hours, and there is scarcely a
dull minute.
	\end{itemize}
}
\item adjective \\
Someone or something that is \textbf{dull} is not very lively or energetic .
 \textit{
	\begin{itemize}
	\item The body's natural rhythms mean we all feel dull and sleepy between 1 and 3pm.
	\end{itemize}
}
\item adjective \\
A \textbf{dull} colour or light is not bright.
 \textit{
	\begin{itemize}
	\item The stamp was a dull blue colour.
	\end{itemize}
}
\item adjective \\
You say the weather is \textbf{dull} when it is very cloudy.
 \textit{
	\begin{itemize}
	\item It's always dull and raining.
	\end{itemize}
}
\item adjective \\
\textbf{Dull} sounds are not very clear or loud.
 \textit{
	\begin{itemize}
	\item The long whining whistle of a shell was followed by the dull boom of the explosion.
	\item The coffin closed with a dull thud.
	\end{itemize}
}
\item adjective \\
\textbf{Dull}  feelings are weak and not intense.
 \textit{
	\begin{itemize}
	\item The pain, usually a dull ache, gets worse with exercise.
	\item I realized with a kind of dull shock that I didn't recognize a single name.
	\end{itemize}
}
\item adjective \\
If a knife or blade is \textbf{dull} , it is not sharp .
 \textit{
	\begin{itemize}
	\end{itemize}
}
\item verb \\
If something \textbf{dulls} or if it \textbf{is dulled} , it becomes less intense, bright, or lively.
 \textit{
	\begin{itemize}
	\item Her eyes dulled and she gazed blankly.
	\item He can dull your senses with facts and figures.
	\item Share prices and trading have been dulled by worries over the war.
	\end{itemize}
}
\end{enumerate}

\section*{guideline}
{\large \color{blue}  guidelines  }
\subsection*{Explain}
\begin{enumerate}
\item countable noun \\
If an organization  issues  \textbf{guidelines}  \textbf{on} something, it issues official  advice about how to do it.
 \textit{
	\begin{itemize}
	\item The government should issue clear guidelines on the content of religious education.
	\item The Internet Advertising Bureau has guidelines for advertising companies to follow.
	\end{itemize}
}
\item countable noun \\
A \textbf{guideline} is something that can be used to help you plan your actions or to form an opinion about something.
 \textit{
	\begin{itemize}
	\item The effects of the sun can be significantly reduced if we follow certain guidelines.
	\item A written IQ test is merely a guideline.
	\end{itemize}
}
\end{enumerate}

\section*{empirical}
{\large \color{blue}  }
\subsection*{Explain}
\begin{enumerate}
\item adjective \\
\textbf{Empirical}  evidence or study relies on practical experience rather than theories.
 \textit{
	\begin{itemize}
	\item There is no empirical evidence to support his thesis.
	\end{itemize}
}
\end{enumerate}

\section*{heading}
{\large \color{blue}  headings  }
\subsection*{Explain}
\begin{enumerate}
\item countable noun \\
A \textbf{heading} is the title of a piece of writing , which is written or printed at the top of the page.
 \textit{
	\begin{itemize}
	\item ...helpful chapter headings.
	\end{itemize}
}
\end{enumerate}

\section*{empty}
{\large \color{blue}  emptier  emptiest  empties  emptying  emptied  }
\subsection*{Explain}
\begin{enumerate}
\item adjective \\
An \textbf{empty} place, vehicle , or container is one that has no people or things in it.
 \textit{
	\begin{itemize}
	\item The room was bare and empty.
	\item ...empty cans of lager.
	\item The roads were nearly empty of traffic.
	\end{itemize}
}
\item adjective \\
An \textbf{empty}  gesture , threat , or relationship has no real value or meaning .
 \textit{
	\begin{itemize}
	\item His father threatened to throw him out, but he knew it was an empty threat.
	\item ...to ensure the event is not perceived as an empty gesture.
	\end{itemize}
}
\item adjective \\
If you describe a person's life or a period of time as \textbf{empty} , you mean that nothing interesting or valuable  happens in it.
 \textit{
	\begin{itemize}
	\item My life was very hectic but empty before I met him.
	\end{itemize}
}
\item adjective \\
If you \textbf{feel}  \textbf{empty} , you feel unhappy and have no energy, usually because you are very tired or have just experienced something upsetting .
 \textit{
	\begin{itemize}
	\item I felt empty and hollow; defeated.
	\item I feel so empty, my life just doesn't seem worth living any more.
	\end{itemize}
}
\item verb \\
If you \textbf{empty} a container, or \textbf{empty} something out of it, you remove its contents, especially by tipping it up.
 \textit{
	\begin{itemize}
	\item I emptied the wastepaper basket.
	\item Empty the noodles and liquid into a serving bowl.
	\item He emptied the contents out into the palm of his hand.
	\end{itemize}
}
\item verb \\
If someone \textbf{empties} a room or place, or if it \textbf{empties} , everyone that is in it goes  away .
 \textit{
	\begin{itemize}
	\item The stadium emptied at the end of the first day of athletics.
	\item ...a woman who could empty a pub full of drunks just by lifting one fist.
	\end{itemize}
}
\item verb \\
A river or canal that \textbf{empties into} a lake , river, or sea  flows into it.
 \textit{
	\begin{itemize}
	\item The Washougal empties into the Columbia River near Portland.
	\end{itemize}
}
\item countable noun \\
\textbf{Empties} are bottles or containers which no longer have anything in them.
 \textit{
	\begin{itemize}
	\end{itemize}
}
\end{enumerate}

\section*{herb}
{\large \color{blue}  herbs  }
\subsection*{Explain}
\begin{enumerate}
\item countable noun \\
A \textbf{herb} is a plant whose leaves are used in cooking to add  flavour to food, or as a medicine.
 \textit{
	\begin{itemize}
	\end{itemize}
}
\end{enumerate}

\section*{exceptional}
{\large \color{blue}  }
\subsection*{Explain}
\begin{enumerate}
\item adjective \\
You use \textbf{exceptional} to describe someone or something that has a particular quality, usually a good quality, to an
unusually high degree .
 \textit{
	\begin{itemize}
	\item ...children with exceptional ability.
	\item His translation is exceptional in its poetic quality.
	\end{itemize}
}
\item adjective \\
\textbf{Exceptional}  situations and incidents are unusual and only likely to happen very infrequently.
 \textit{
	\begin{itemize}
	\item ...if the courts hold that this case is exceptional.
	\item Magistrates would have the discretion to impose a community order.
	\end{itemize}
}
\end{enumerate}

\section*{indication}
{\large \color{blue}  indications  }
\subsection*{Explain}
\begin{enumerate}
\item variable noun \\
An \textbf{indication} is a sign which suggests, for example , what people are thinking or feeling .
 \textit{
	\begin{itemize}
	\item All the indications are that we are going to receive reasonable support from abroad.
	\item He gave no indication that he was ready to compromise.
	\end{itemize}
}
\end{enumerate}

\section*{fertile}
{\large \color{blue}  }
\subsection*{Explain}
\begin{enumerate}
\item adjective \\
Land or soil that is \textbf{fertile} is able to support the growth of a large number of strong  healthy plants.
 \textit{
	\begin{itemize}
	\item ...fertile soil.
	\item ...the rolling fertile countryside of East Cork.
	\end{itemize}
}
\item adjective \\
A \textbf{fertile}  mind or imagination is able to produce a lot of good, original  ideas .
 \textit{
	\begin{itemize}
	\item ...a product of Flynn's fertile imagination.
	\item A chess player must have a fertile imagination and rich sense of fantasy.
	\end{itemize}
}
\item adjective \\
A situation or environment that is \textbf{fertile} in relation to a particular activity or feeling  encourages the activity or feeling.
 \textit{
	\begin{itemize}
	\item ...a fertile breeding ground for this kind of violent racism.
	\end{itemize}
}
\item adjective \\
A person or animal that is \textbf{fertile} is able to reproduce and have babies or young .
 \textit{
	\begin{itemize}
	\item The operation cannot be reversed to make her fertile again.
	\end{itemize}
}
\end{enumerate}

\section*{label}
{\large \color{blue}  labels  labelling  labelled  }
\subsection*{Explain}
\begin{enumerate}
\item countable noun \\
A \textbf{label} is a piece of paper or plastic that is attached to an object in order to give information about it.
 \textit{
	\begin{itemize}
	\item He peered at the label on the bottle.
	\end{itemize}
}
\item verb \\
If something \textbf{is labelled} , a label is attached to it giving information about it.
 \textit{
	\begin{itemize}
	\item The stuff has never been properly logged and labelled.
	\item Meat labelled 'Scotch Beef' sells for a premium in supermarkets.
	\item All the products are labelled with comprehensive instructions.
	\end{itemize}
}
\item verb \\
If you say that someone or something \textbf{is labelled}  \textbf{as} a particular thing, you mean that people generally describe them that way and you think that this is unfair .
 \textit{
	\begin{itemize}
	\item Too often the press are labelled as bad boys.
	\item Certain estates are labelled as undesirable.
	\item They are afraid to contact the social services in case they are labelled a problem
family.
	\item If you venture from 'feminine' standards, you are labelled aggressive and hostile.
	\end{itemize}
}
\item countable noun \\
If you say that someone gets a particular \textbf{label} , you mean that people show  disapproval of them by describing them with a critical word or phrase.
 \textit{
	\begin{itemize}
	\item They remain on course for the label of worst Super League side.
	\end{itemize}
}
\item countable noun \\
You can refer to a company that produces and sells CDs as a particular \textbf{label} .
 \textit{
	\begin{itemize}
	\item It was on the Virgin label.
	\end{itemize}
}
\end{enumerate}

\section*{gentle}
{\large \color{blue}  gentler  gentlest  }
\subsection*{Explain}
\begin{enumerate}
\item adjective \\
Someone who is \textbf{gentle} is kind , mild, and calm .
 \textit{
	\begin{itemize}
	\item My son was a quiet and gentle man.
	\item Michael's voice was gentle and consoling.
	\end{itemize}
}
\item adjective \\
\textbf{Gentle} actions or movements are performed in a calm and controlled manner, with little force.
 \textit{
	\begin{itemize}
	\item ...a gentle game of tennis.
	\item His movements were gentle and deliberate.
	\end{itemize}
}
\item adjective \\
If you describe the weather , especially the wind , as \textbf{gentle} , you mean it is pleasant and calm and not harsh or violent .
 \textit{
	\begin{itemize}
	\item The blustery winds of spring had dropped to a gentle breeze.
	\end{itemize}
}
\item adjective \\
A \textbf{gentle}  slope or curve is not steep or severe .
 \textit{
	\begin{itemize}
	\item ...gentle, rolling meadows.
	\item There were two passes over 13,000 feet but the slopes were gentle.
	\end{itemize}
}
\item adjective \\
A \textbf{gentle} heat is a fairly low heat.
 \textit{
	\begin{itemize}
	\item Cook for 30 minutes over a gentle heat.
	\end{itemize}
}
\end{enumerate}

\section*{lawn}
{\large \color{blue}  lawns  }
\subsection*{Explain}
\begin{enumerate}
\item variable noun \\
A \textbf{lawn} is an area of grass that is kept  cut short and is usually part of someone's garden or backyard, or part of a park .
 \textit{
	\begin{itemize}
	\item They were sitting on the lawn under a large beech tree.
	\end{itemize}
}
\end{enumerate}

\section*{golden}
{\large \color{blue}  }
\subsection*{Explain}
\begin{enumerate}
\item adjective \\
Something that is \textbf{golden} is bright yellow in colour.
 \textit{
	\begin{itemize}
	\item She combed and arranged her golden hair.
	\item ...an endless golden beach.
	\end{itemize}
}
\item adjective \\
\textbf{Golden} things are made of gold.
 \textit{
	\begin{itemize}
	\item ...a golden chain with a golden locket.
	\end{itemize}
}
\item adjective \\
If you describe something as \textbf{golden} , you mean it is wonderful because it is likely to be successful and rewarding , or because it is the best of its kind .
 \textit{
	\begin{itemize}
	\item He says there's a golden opportunity for peace which must be seized.
	\end{itemize}
}
\item  \\
 golden boy/girl \textit{
	\begin{itemize}
	\end{itemize}
}
\end{enumerate}

\section*{meadow}
{\large \color{blue}  meadows  }
\subsection*{Explain}
\begin{enumerate}
\item countable noun \\
A \textbf{meadow} is a field which has grass and flowers growing in it.
 \textit{
	\begin{itemize}
	\end{itemize}
}
\end{enumerate}

\section*{greedy}
{\large \color{blue}  greedier  greediest  }
\subsection*{Explain}
\begin{enumerate}
\item adjective \\
If you describe someone as \textbf{greedy} , you mean that they want to have more of something such as food or money than is necessary or fair .
 \textit{
	\begin{itemize}
	\item He attacked greedy bosses for awarding themselves big rises.
	\item She is greedy and selfish.
	\end{itemize}
}
\end{enumerate}

\section*{nonsense}
{\large \color{blue}  }
\subsection*{Explain}
\begin{enumerate}
\item uncountable noun \\
If you say that something spoken or written is \textbf{nonsense} , you mean that you consider it to be untrue or silly .
 \textit{
	\begin{itemize}
	\item Most orthodox doctors however dismiss this as complete nonsense.
	\item ...all that poetic nonsense about love.
	\item 'I'm putting on weight.'—'Nonsense my dear.'
	\end{itemize}
}
\item variable noun \\
You can use \textbf{nonsense} to refer to something that you think is foolish or that you disapprove of.
 \textit{
	\begin{itemize}
	\item Surely it is an economic nonsense to deplete the world of natural resources.
	\item I think there is a limit to how much of this nonsense people are going to put up
with.
	\end{itemize}
}
\item uncountable noun \\
You can refer to spoken or written words that do not mean anything because they do
not make sense as \textbf{nonsense} .
 \textit{
	\begin{itemize}
	\item ...a children's nonsense poem by Charles E Carryl.
	\end{itemize}
}
\item  \\
 make a nonsense of sth \textit{
	\begin{itemize}
	\end{itemize}
}
\end{enumerate}

\section*{hollow}
{\large \color{blue}  hollows  hollowing  hollowed  }
\subsection*{Explain}
\begin{enumerate}
\item adjective \\
Something that is \textbf{hollow} has a space inside it, as opposed to being solid all the way through.
 \textit{
	\begin{itemize}
	\item ...a hollow tree.
	\item ...a hollow cylinder.
	\end{itemize}
}
\item adjective \\
A surface that is \textbf{hollow}  curves  inwards .
 \textit{
	\begin{itemize}
	\item He looked young, dark and sharp-featured, with hollow cheeks.
	\end{itemize}
}
\item countable noun \\
A \textbf{hollow} is a hole inside a tree.
 \textit{
	\begin{itemize}
	\item I made my home there, in the hollow of a dying elm.
	\end{itemize}
}
\item countable noun \\
A \textbf{hollow} is an area that is lower than the surrounding surface.
 \textit{
	\begin{itemize}
	\item Below him the town lay warm in the hollow of the hill.
	\item ...where water gathers in a hollow and forms a pond.
	\end{itemize}
}
\item adjective \\
If you describe a statement , situation , or person as \textbf{hollow} , you mean they have no real  value , worth , or effectiveness.
 \textit{
	\begin{itemize}
	\item Any threat to bring in the police is a hollow one.
	\item ...a hollow man who's coasted on charm for far too long.
	\end{itemize}
}
\item adjective \\
If someone gives a \textbf{hollow}  laugh , they laugh in a way that shows that they do not really  find something amusing .
 \textit{
	\begin{itemize}
	\item Murray Pick's hollow laugh had no mirth in it.
	\end{itemize}
}
\item adjective \\
A \textbf{hollow} sound is dull and echoing .
 \textit{
	\begin{itemize}
	\item ...the hollow sound of a gunshot.
	\end{itemize}
}
\item verb \\
If something \textbf{is hollowed} , its surface is made to curve inwards or downwards .
 \textit{
	\begin{itemize}
	\item The mule's back was hollowed by the weight of its burden.
	\item ...her high, elegantly hollowed cheekbones.
	\end{itemize}
}
\end{enumerate}

\section*{pad}
{\large \color{blue}  pads  padding  padded  }
\subsection*{Explain}
\begin{enumerate}
\item countable noun \\
A \textbf{pad} is a fairly thick, flat piece of a material such as cloth or rubber. Pads are used, for example,
to clean things, to protect things, or to change their shape.
 \textit{
	\begin{itemize}
	\item He withdrew the needle and placed a pad of cotton-wool over the spot.
	\item ...a scouring pad.
	\item ...a flowered dress with shoulder pads.
	\end{itemize}
}
\item countable noun \\
A \textbf{pad}  \textbf{of} paper is a number of pieces of paper which are fixed together along the top or the
side, so that each piece can be torn off when it has been used.
 \textit{
	\begin{itemize}
	\item She wrote on a pad of paper.
	\item Have a pad and pencil ready and jot down some of your thoughts.
	\item 'Here's your ticket,' he said, and he tore it off the pad.
	\end{itemize}
}
\item verb \\
When someone \textbf{pads}  somewhere , they walk there with steps that are fairly quick , light, and quiet .
 \textit{
	\begin{itemize}
	\item Freddy speaks very quietly and pads around in soft velvet slippers.
	\item ...a dog padding through the streets.
	\item I often bumped into him as he padded the corridors.
	\end{itemize}
}
\item countable noun \\
A \textbf{pad} is a platform or an area of flat, hard ground where helicopters take off and land or rockets are launched.
 \textit{
	\begin{itemize}
	\item ...a little round helicopter pad.
	\item ...a landing pad on the back of the ship.
	\item Journalists report seeing a fire on the pad after the launch.
	\end{itemize}
}
\item countable noun \\
People can refer to the place where they live as their \textbf{pad} , especially if it is a flat.
 \textit{
	\begin{itemize}
	\item I moved on round the big house to reach my pad.
	\item It wouldn't have occurred to me to get myself a bachelor pad.
	\end{itemize}
}
\item countable noun \\
The \textbf{pads}  \textbf{of} a person's fingers and toes or of an animal's feet are the soft, fleshy parts of them.
 \textit{
	\begin{itemize}
	\item Tap your cheeks all over with the pads of your fingers.
	\end{itemize}
}
\item verb \\
If you \textbf{pad} something, you put something soft in it or over it in order to make it less hard,
to protect it, or to give it a different shape.
 \textit{
	\begin{itemize}
	\item Pad the back of a car seat with a pillow.
	\item I can tell you I always padded my bras.
	\end{itemize}
}
\end{enumerate}

\section*{idle}
{\large \color{blue}  idles  idling  idled  }
\subsection*{Explain}
\begin{enumerate}
\item adjective \\
If people who were working are \textbf{idle} , they have no jobs or work.
 \textit{
	\begin{itemize}
	\item Employees have been idle almost a month because of shortages.
	\end{itemize}
}
\item adjective \\
If machines or factories are \textbf{idle} , they are not working or being used.
 \textit{
	\begin{itemize}
	\item Now the machine is lying idle.
	\item ...factories that had been idle for years.
	\end{itemize}
}
\item adjective \\
If you say that someone is \textbf{idle} , you disapprove of them because they are not doing anything and you think they should be.
 \textit{
	\begin{itemize}
	\item ...idle bureaucrats who spent the day reading newspapers.
	\item I never met such an idle bunch of workers in all my life!
	\end{itemize}
}
\item adjective \\
\textbf{Idle} is used to describe something that you do for no particular reason , often because you have nothing better to do.
 \textit{
	\begin{itemize}
	\item Brian kept up the idle chatter for another five minutes.
	\item ...idle curiosity.
	\end{itemize}
}
\item adjective \\
If you say that \textbf{it is idle}  \textbf{to} do something, you mean that it is not worth doing it, because it will not achieve anything.
 \textit{
	\begin{itemize}
	\item It would be idle to pretend the system is perfect.
	\end{itemize}
}
\item adjective \\
You refer to an \textbf{idle}  threat or boast when you do not think the person making it will or can do what they say.
 \textit{
	\begin{itemize}
	\item It was more of an idle threat than anything.
	\item His statement isn't merely an idle boast.
	\end{itemize}
}
\item verb \\
If you \textbf{idle} , you spend time in a lazy way, doing nothing in particular.
 \textit{
	\begin{itemize}
	\item When they reached his house, Scobie idled a bit, finishing his cigarette.
	\item We spent many hours idling in one of the cafes that line three sides of the tiny
piazza.
	\item He idled around afterwards, window shopping until about 5pm.
	\end{itemize}
}
\item verb \\
To \textbf{idle} a factory or other place of work means to close it down because there is no work to do or because the workers are on strike .
 \textit{
	\begin{itemize}
	\item ...idled assembly plants.
	\end{itemize}
}
\item verb \\
To \textbf{idle} workers means to stop them working.
 \textit{
	\begin{itemize}
	\item The strike has idled about 55,000 machinists.
	\end{itemize}
}
\item verb \\
If an engine or vehicle \textbf{is idling} , the engine is running slowly and quietly because it is not in gear , and the vehicle is not moving.
 \textit{
	\begin{itemize}
	\item Beyond a stand of trees a small plane idled.
	\item Her Daimler limo waits with its engine idling.
	\end{itemize}
}
\end{enumerate}

\section*{pepper}
{\large \color{blue}  peppers  peppering  peppered  }
\subsection*{Explain}
\begin{enumerate}
\item uncountable noun \\
\textbf{Pepper} is a hot-tasting spice which is used to flavour food.
 \textit{
	\begin{itemize}
	\item Season with salt and pepper.
	\item ...freshly ground black pepper.
	\end{itemize}
}
\item countable noun \\
A \textbf{pepper} , or in American English a \textbf{bell pepper} , is a hollow  green , red, or yellow vegetable with seeds  inside it.
 \textit{
	\begin{itemize}
	\end{itemize}
}
\item verb \\
If something \textbf{is peppered with} small objects, a lot of those objects hit it.
 \textit{
	\begin{itemize}
	\item He was wounded in both legs and severely peppered with shrapnel.
	\item Suddenly the garden was peppered with pellets.
	\end{itemize}
}
\item verb \\
If something \textbf{is peppered}  \textbf{with} things, it has a lot of those things in it or on it.
 \textit{
	\begin{itemize}
	\item While her English was correct, it was peppered with French phrases.
	\item Outside, the road was peppered with glass.
	\item Yachts peppered the tranquil waters of Botafogo Bay.
	\end{itemize}
}
\end{enumerate}

\section*{legal}
{\large \color{blue}  }
\subsection*{Explain}
\begin{enumerate}
\item adjective \\
\textbf{Legal} is used to describe things that relate to the law.
 \textit{
	\begin{itemize}
	\item He vowed to take legal action.
	\item ...the British legal system.
	\item I sought legal advice on this.
	\item ...the legal profession.
	\end{itemize}
}
\item adjective \\
An action or situation that is \textbf{legal} is allowed or required by law.
 \textit{
	\begin{itemize}
	\item What I did was perfectly legal.
	\item ...drivers who have more than the legal limit of alcohol.
	\end{itemize}
}
\end{enumerate}

\section*{pharmacy}
{\large \color{blue}  pharmacies  }
\subsection*{Explain}
\begin{enumerate}
\item countable noun \\
A \textbf{pharmacy} is a shop or a department in a shop where medicines are sold or given out.
 \textit{
	\begin{itemize}
	\item Make sure you understand exactly how to take your medicines before you leave the
pharmacy.
	\item ...the pharmacy section of the drugstore.
	\end{itemize}
}
\item uncountable noun \\
\textbf{Pharmacy} is the job or the science of preparing medicines.
 \textit{
	\begin{itemize}
	\item He spent four years studying pharmacy.
	\end{itemize}
}
\end{enumerate}

\section*{monetary}
{\large \color{blue}  }
\subsection*{Explain}
\begin{enumerate}
\item adjective \\
\textbf{Monetary}  means relating to money, especially the total amount of money in a country.
 \textit{
	\begin{itemize}
	\item Some countries tighten monetary policy to avoid inflation.
	\item The courts will be asked to place a monetary value on his unfinished career.
	\end{itemize}
}
\end{enumerate}

\section*{pill}
{\large \color{blue}  pills  }
\subsection*{Explain}
\begin{enumerate}
\item countable noun \\
\textbf{Pills} are small solid  round masses of medicine or vitamins that you swallow without chewing .
 \textit{
	\begin{itemize}
	\item Why do I have to take all these pills?
	\item ...sleeping pills.
	\end{itemize}
}
\item singular noun \\
If a woman is \textbf{on}  \textbf{the pill} , she takes a special pill that prevents her from becoming  pregnant .
 \textit{
	\begin{itemize}
	\item She had been on the pill for three years.
	\item ...the contraceptive pill.
	\end{itemize}
}
\item  \\
 a bitter pill \textit{
	\begin{itemize}
	\end{itemize}
}
\item  \\
 sweeten the pill \textit{
	\begin{itemize}
	\end{itemize}
}
\end{enumerate}

\section*{monotonous}
{\large \color{blue}  }
\subsection*{Explain}
\begin{enumerate}
\item adjective \\
Something that is \textbf{monotonous} is very boring because it has a regular , repeated  pattern which never changes.
 \textit{
	\begin{itemize}
	\item It's monotonous work, like most factory jobs.
	\item The food may get a bit monotonous, but there'll be enough of it.
	\end{itemize}
}
\end{enumerate}

\section*{prescription}
{\large \color{blue}  prescriptions  }
\subsection*{Explain}
\begin{enumerate}
\item countable noun \\
A \textbf{prescription} is the piece of paper on which your doctor writes an order for medicine and which you give to a chemist or pharmacist to get the medicine.
 \textit{
	\begin{itemize}
	\item You will have to take your prescription to a chemist.
	\end{itemize}
}
\item countable noun \\
A \textbf{prescription} is a medicine which a doctor has told you to take.
 \textit{
	\begin{itemize}
	\item The prescription Ackerman gave me isn't doing any good.
	\end{itemize}
}
\item countable noun \\
A \textbf{prescription} is a proposal or a plan which gives ideas about how to solve a problem or improve a situation .
 \textit{
	\begin{itemize}
	\item ...the economic prescriptions of Ireland's main political parties.
	\end{itemize}
}
\end{enumerate}

\section*{remedy}
{\large \color{blue}  remedies  remedying  remedied  }
\subsection*{Explain}
\begin{enumerate}
\item countable noun \\
A \textbf{remedy} is a successful way of dealing with a problem .
 \textit{
	\begin{itemize}
	\item The remedy lies in the hands of the government.
	\item ...a remedy for economic ills.
	\end{itemize}
}
\item countable noun \\
A \textbf{remedy} is something that is intended to cure you when you are ill or in pain .
 \textit{
	\begin{itemize}
	\item ...natural remedies to help overcome winter infections.
	\end{itemize}
}
\item verb \\
If you \textbf{remedy} something that is wrong or harmful , you correct it or improve it.
 \textit{
	\begin{itemize}
	\item A great deal has been done internally to remedy the situation.
	\end{itemize}
}
\end{enumerate}

\section*{obedient}
{\large \color{blue}  }
\subsection*{Explain}
\begin{enumerate}
\item adjective \\
A person or animal who is \textbf{obedient} does what they are told to do.
 \textit{
	\begin{itemize}
	\item He was very respectful at home and obedient to his parents.
	\item What a sweet, obedient little girl she was in the sixth grade.
	\end{itemize}
}
\end{enumerate}

\section*{revolution}
{\large \color{blue}  revolutions  }
\subsection*{Explain}
\begin{enumerate}
\item countable noun \\
A \textbf{revolution} is a successful  attempt by a large group of people to change the political system of their country by force.
 \textit{
	\begin{itemize}
	\item The period since the revolution has been one of political turmoil.
	\item ...after the French Revolution.
	\item ...before the 1917 Revolution.
	\end{itemize}
}
\item countable noun \\
A \textbf{revolution} in a particular area of human activity is an important change in that area.
 \textit{
	\begin{itemize}
	\item The nineteenth century witnessed a revolution in ship design and propulsion.
	\item ...the industrial revolution.
	\end{itemize}
}
\end{enumerate}

\section*{peaceful}
{\large \color{blue}  }
\subsection*{Explain}
\begin{enumerate}
\item adjective \\
\textbf{Peaceful}  activities and situations do not involve war.
 \textit{
	\begin{itemize}
	\item He has attempted to find a peaceful solution to the conflict.
	\item They emphasised that their equipment was for peaceful and not military purposes.
	\end{itemize}
}
\item adjective \\
\textbf{Peaceful}  occasions  happen without violence or serious  disorder .
 \textit{
	\begin{itemize}
	\item The farmers staged a noisy but peaceful protest outside the headquarters of the organization.
	\item Reports say that polling was orderly and peaceful.
	\end{itemize}
}
\item adjective \\
\textbf{Peaceful} people are not violent and try to avoid  quarrelling or fighting with other people.
 \textit{
	\begin{itemize}
	\item ...warriors who killed or enslaved the peaceful farmers.
	\end{itemize}
}
\item adjective \\
A \textbf{peaceful} place or time is quiet , calm, and free from disturbance .
 \textit{
	\begin{itemize}
	\item ...a peaceful Georgian house in the heart of Dorset.
	\item Mornings are usually quiet and peaceful in Hueytown.
	\end{itemize}
}
\item graded adjective \\
Someone who \textbf{feels} or \textbf{looks peaceful} feels or looks calm and free from worry .
 \textit{
	\begin{itemize}
	\item I feel relaxed and peaceful.
	\item The animals look peaceful and happy.
	\end{itemize}
}
\end{enumerate}

\section*{salvation}
{\large \color{blue}  }
\subsection*{Explain}
\begin{enumerate}
\item uncountable noun \\
In Christianity , \textbf{salvation} is the fact that Christ has saved a person from evil .
 \textit{
	\begin{itemize}
	\item The church's message of salvation has changed the lives of many.
	\end{itemize}
}
\item uncountable noun \\
The \textbf{salvation} of someone or something is the act of saving them from harm, destruction , or an unpleasant  situation .
 \textit{
	\begin{itemize}
	\item It became clear that the situation was beyond salvation.
	\end{itemize}
}
\item singular noun \\
If someone or something is your \textbf{salvation} , they are responsible for saving you from harm, destruction, or an unpleasant situation.
 \textit{
	\begin{itemize}
	\item The country's salvation lies in forcing through democratic reforms.
	\item I consider books my salvation.
	\end{itemize}
}
\end{enumerate}

\section*{physical}
{\large \color{blue}  physicals  }
\subsection*{Explain}
\begin{enumerate}
\item adjective \\
\textbf{Physical} qualities, actions, or things are connected with a person's body, rather than with their mind.
 \textit{
	\begin{itemize}
	\item ...the physical and mental problems caused by the illness.
	\item Physical activity promotes good health.
	\item The attraction between them is physical.
	\end{itemize}
}
\item adjective \\
\textbf{Physical} things are real things that can be touched and seen , rather than ideas or spoken words.
 \textit{
	\begin{itemize}
	\item All he needed was a little room, a physical and spiritual place to call home.
	\item ...physical evidence to support the story.
	\item ...the physical similarities among the towns.
	\end{itemize}
}
\item adjective \\
\textbf{Physical} means relating to the structure, size , or shape of something that can be touched and seen.
 \textit{
	\begin{itemize}
	\item ...the physical characteristics of the terrain.
	\item ...the physical properties (weight, volume, hardness, etc.) of a substance.
	\end{itemize}
}
\item adjective \\
\textbf{Physical} means connected with physics or the laws of physics.
 \textit{
	\begin{itemize}
	\item ...the physical laws of combustion and thermodynamics.
	\end{itemize}
}
\item adjective \\
Someone who is \textbf{physical} touches people a lot , either in an affectionate way or in a rough way.
 \textit{
	\begin{itemize}
	\item We decided that in the game we would be physical and aggressive.
	\end{itemize}
}
\item adjective \\
\textbf{Physical} is used in expressions such as \textbf{physical love} and \textbf{physical relationships} to refer to sexual relationships between people.
 \textit{
	\begin{itemize}
	\item the book celebrated the sublime joys of physical love.
	\item It had been years since they had shared any meaningful form of physical relationship.
	\end{itemize}
}
\item countable noun \\
A \textbf{physical} is a medical  examination , done in order to see if someone is fit and well enough to do a particular job or to join the army .
 \textit{
	\begin{itemize}
	\item Bob failed his physical.
	\item Routine physicals are done by a nurse.
	\end{itemize}
}
\end{enumerate}

\section*{saucer}
{\large \color{blue}  saucers  }
\subsection*{Explain}
\begin{enumerate}
\item countable noun \\
A \textbf{saucer} is a small curved  plate on which you stand a cup.
 \textit{
	\begin{itemize}
	\end{itemize}
}
\end{enumerate}

\section*{practical}
{\large \color{blue}  practicals  }
\subsection*{Explain}
\begin{enumerate}
\item adjective \\
The \textbf{practical}  aspects of something involve real  situations and events, rather than just ideas and theories .
 \textit{
	\begin{itemize}
	\item We can offer you practical suggestions on how to increase the fibre in your daily
diet.
	\item This practical guidebook teaches you about relaxation, coping skills, and time management.
	\end{itemize}
}
\item adjective \\
You describe people as \textbf{practical} when they make sensible  decisions and deal effectively with problems .
 \textit{
	\begin{itemize}
	\item You were always so practical, Maria.
	\item How could she be so practical when he'd just told her something so shattering?
	\item He lacked any of the practical common sense essential in management.
	\end{itemize}
}
\item adjective \\
\textbf{Practical} ideas and methods are likely to be effective or successful in a real situation.
 \textit{
	\begin{itemize}
	\item Although the causes of cancer are being uncovered, we do not yet have any practical
way to prevent it.
	\item It is not easy to make practical suggestions for helping her.
	\end{itemize}
}
\item adjective \\
You can describe clothes and things in your house as \textbf{practical} when they are suitable for a particular purpose rather than just being fashionable or attractive .
 \textit{
	\begin{itemize}
	\item Our clothes are lightweight, fashionable, practical for holidays.
	\end{itemize}
}
\item countable noun \\
A \textbf{practical} is an examination or a lesson in which you make things or do experiments rather than simply  writing  answers to questions .
 \textit{
	\begin{itemize}
	\end{itemize}
}
\end{enumerate}

\section*{sign}
{\large \color{blue}  signs  signing  signed  }
\subsection*{Explain}
\begin{enumerate}
\item countable noun \\
A \textbf{sign} is a mark or shape that always has a particular meaning , for example in mathematics or music.
 \textit{
	\begin{itemize}
	\item Equations are generally written with a two-bar equals sign.
	\end{itemize}
}
\item countable noun \\
A \textbf{sign} is a movement of your arms, hands , or head which is intended to have a particular meaning.
 \textit{
	\begin{itemize}
	\item They gave Lavalle the thumbs-up sign.
	\item He made a sign of assent.
	\end{itemize}
}
\item verb \\
If you \textbf{sign} , you communicate with someone using sign language. If a programme or performance  \textbf{is signed} , someone uses sign language so that deaf people can understand it.
 \textit{
	\begin{itemize}
	\item All programmes will be either 'signed' or subtitled.
	\end{itemize}
}
\item countable noun \\
A \textbf{sign} is a piece of wood, metal, or plastic with words or pictures on it. Signs give you information about something, or give you a warning or an instruction .
 \textit{
	\begin{itemize}
	\item ...a sign saying that the highway was closed because of snow.
	\item Over his head, he held a cardboard sign saying 'Free Hugs' in big, black letters.
	\item As soon as the seat belt sign had been switched off, we rushed out.
	\end{itemize}
}
\item variable noun \\
If there is a \textbf{sign}  \textbf{of} something, there is something which shows that it exists or is happening .
 \textit{
	\begin{itemize}
	\item They are prepared to hand back a hundred prisoners of war a day as a sign of good
will.
	\item His face and movements rarely betrayed a sign of nerves.
	\item Your blood would have been checked for any sign of kidney failure.
	\end{itemize}
}
\item verb \\
When you \textbf{sign} a document, you write your name on it, usually at the end or in a special space. You do this to indicate that you have written the document, that you agree with what is written, or that you were present as a witness .
 \textit{
	\begin{itemize}
	\item World leaders are expected to sign a treaty pledging to increase environmental protection.
	\item Before an operation the patient will be asked to sign a consent form.
	\end{itemize}
}
\item verb \\
If an organization \textbf{signs} someone or if someone \textbf{signs} for an organization, they sign a contract agreeing to work for that organization for a specified period of time.
 \textit{
	\begin{itemize}
	\item The Blues' boss planned to sign a big-name striker in January.
	\item The band then signed to Slash Records.
	\end{itemize}
}
\item countable noun \\
In astrology , a \textbf{sign} or a \textbf{sign}  \textbf{of the zodiac} is one of the twelve areas into which the heavens are divided.
 \textit{
	\begin{itemize}
	\item The New Moon takes place in your opposite sign of Libra on the 15th.
	\end{itemize}
}
\item  \\
 no sign of someone \textit{
	\begin{itemize}
	\end{itemize}
}
\item  \\
 signed and sealed \textit{
	\begin{itemize}
	\end{itemize}
}
\end{enumerate}

\section*{qualitative}
{\large \color{blue}  }
\subsection*{Explain}
\begin{enumerate}
\item adjective \\
\textbf{Qualitative} means relating to the nature or standard of something, rather than to its quantity.
 \textit{
	\begin{itemize}
	\item There are qualitative differences in the way children and adults think.
	\item That's the whole difference between quantitative and qualitative research.
	\end{itemize}
}
\end{enumerate}

\section*{slogan}
{\large \color{blue}  slogans  }
\subsection*{Explain}
\begin{enumerate}
\item countable noun \\
A \textbf{slogan} is a short phrase that is easy to remember . Slogans are used in advertisements and by political  parties and other organizations who want people to remember what they are saying or selling .
 \textit{
	\begin{itemize}
	\item They could campaign on the slogan 'We'll take less of your money'.
	\item ...a group of angry demonstrators shouting slogans.
	\end{itemize}
}
\end{enumerate}

\section*{radioactive}
{\large \color{blue}  }
\subsection*{Explain}
\begin{enumerate}
\item adjective \\
Something that is \textbf{radioactive} contains a substance that produces energy in the form of powerful and harmful  rays .
 \textit{
	\begin{itemize}
	\item The government has been storing radioactive waste at Fernald for 50 years.
	\end{itemize}
}
\end{enumerate}

\section*{south}
{\large \color{blue}  }
\subsection*{Explain}
\begin{enumerate}
\item uncountable noun \\
\textbf{The}  \textbf{south} is the direction which is on your right when you are looking towards the direction where the sun  rises .
 \textit{
	\begin{itemize}
	\item The town lies ten miles to the south of here.
	\item All around him, from east to west, north to south, the stars glittered in the heavens.
	\end{itemize}
}
\item singular noun \\
\textbf{The}  \textbf{south}  \textbf{of} a place, country, or region is the part which is in the south.
 \textit{
	\begin{itemize}
	\item ...holidays in the south of France.
	\end{itemize}
}
\item adverb \\
If you go  \textbf{south} , you travel towards the south.
 \textit{
	\begin{itemize}
	\item We did an extremely fast U-turn and shot south up the Boulevard St. Michel.
	\item He went south to climb Taishan, a mountain sacred to the Chinese.
	\end{itemize}
}
\item adverb \\
Something that is \textbf{south}  \textbf{of} a place is positioned to the south of it.
 \textit{
	\begin{itemize}
	\item They now own and operate a farm 50 miles south of Rochester.
	\item I was living in a house just south of Market Street.
	\end{itemize}
}
\item adjective \\
The \textbf{south}  edge , corner , or part of a place or country is the part which is towards the south.
 \textit{
	\begin{itemize}
	\item ...the south coast of Alderney.
	\end{itemize}
}
\item adjective \\
' \textbf{South} ' is used in the names of some countries, states, and regions in the south of a larger
area.
 \textit{
	\begin{itemize}
	\item Next week the President will visit five South American countries in six days.
	\item ...the states of Mississippi and South Carolina.
	\end{itemize}
}
\item adjective \\
A \textbf{south} wind is a wind that blows from the south.
 \textit{
	\begin{itemize}
	\end{itemize}
}
\item singular noun \\
\textbf{The South} is used to refer to the poorer , less developed countries of the world.
 \textit{
	\begin{itemize}
	\item The South is poorer than the North, and the divide is growing.
	\end{itemize}
}
\end{enumerate}

\section*{religious}
{\large \color{blue}  }
\subsection*{Explain}
\begin{enumerate}
\item adjective \\
You use \textbf{religious} to describe things that are connected with religion or with one particular religion.
 \textit{
	\begin{itemize}
	\item ...religious groups.
	\item ...different religious beliefs.
	\end{itemize}
}
\item adjective \\
Someone who is \textbf{religious} has a strong  belief in a god or gods.
 \textit{
	\begin{itemize}
	\item They are both very religious and felt it was a gift from God.
	\end{itemize}
}
\end{enumerate}

\section*{specimen}
{\large \color{blue}  specimens  }
\subsection*{Explain}
\begin{enumerate}
\item countable noun \\
A \textbf{specimen} is a single plant or animal which is an example of a particular species or type and is examined by scientists .
 \textit{
	\begin{itemize}
	\item 200,000 specimens of fungus are kept at the Komarov Botanical Institute.
	\item ...North American fossil specimens.
	\item Collectors will pay $50,000 to $1 million for a rare specimen.
	\end{itemize}
}
\item countable noun \\
A \textbf{specimen}  \textbf{of} something is an example of it which gives an idea of what the whole of it is like .
 \textit{
	\begin{itemize}
	\item Job applicants have to submit a specimen of handwriting.
	\item ...a specimen bank note.
	\end{itemize}
}
\item countable noun \\
A \textbf{specimen} is a small quantity of someone's urine, blood, or other body fluid which is examined in a medical  laboratory , in order to find out if they are ill or if they have been drinking  alcohol or taking  drugs .
 \textit{
	\begin{itemize}
	\item He refused to provide a specimen.
	\item If your urine specimen shows the presence of bacteria, you'll be prescribed antibiotics.
	\end{itemize}
}
\item countable noun \\
You can use \textbf{specimen} to refer to someone who has a quality of a particular kind .
 \textit{
	\begin{itemize}
	\item He is a fine specimen of his class.
	\end{itemize}
}
\end{enumerate}

\section*{ritual}
{\large \color{blue}  rituals  }
\subsection*{Explain}
\begin{enumerate}
\item variable noun \\
A \textbf{ritual} is a religious service or other ceremony which involves a series of actions performed in a fixed order.
 \textit{
	\begin{itemize}
	\item This is the most ancient, and holiest of the Shinto rituals.
	\item These ceremonies were already part of pre-Christian ritual in Mexico.
	\end{itemize}
}
\item adjective \\
\textbf{Ritual} activities happen as part of a ritual or tradition .
 \textit{
	\begin{itemize}
	\item ...fastings and ritual dancing.
	\item ...an act of ritual suicide.
	\end{itemize}
}
\item variable noun \\
A \textbf{ritual} is a way of behaving or a series of actions which people regularly carry out in a particular situation , because it is their custom to do so.
 \textit{
	\begin{itemize}
	\item The whole Italian culture revolves around the ritual of eating.
	\item Cocktails at the Plaza was a nightly ritual of their sophisticated world.
	\end{itemize}
}
\item adjective \\
You can describe something as a \textbf{ritual} action when it is done in exactly the same way whenever a particular situation occurs.
 \textit{
	\begin{itemize}
	\item I realized that here the conventions required me to make the ritual noises.
	\end{itemize}
}
\end{enumerate}

\section*{spelling}
{\large \color{blue}  spellings  }
\subsection*{Explain}
\begin{enumerate}
\item countable noun \\
A \textbf{spelling} is the correct order of the letters in a word.
 \textit{
	\begin{itemize}
	\item In most languages adjectives have slightly different spellings for masculine and
feminine.
	\item If we got a spelling wrong we were forced to get a dictionary out.
	\end{itemize}
}
\item uncountable noun \\
\textbf{Spelling} is the ability to spell words in the correct way. It is also an attempt to spell a word in the correct way.
 \textit{
	\begin{itemize}
	\item His spelling is very bad.
	\item ...basic skills in reading, writing, grammar and spelling.
	\item Spelling mistakes are often just the result of haste.
	\end{itemize}
}
\end{enumerate}

\section*{shallow}
{\large \color{blue}  shallower  shallowest  }
\subsection*{Explain}
\begin{enumerate}
\item adjective \\
A \textbf{shallow}  container , hole , or area of water measures only a short distance from the top to the bottom .
 \textit{
	\begin{itemize}
	\item Put the milk in a shallow dish.
	\item The water is quite shallow for some distance.
	\end{itemize}
}
\item adjective \\
If you describe a person, piece of work, or idea as \textbf{shallow} , you disapprove of them because they do not show or involve any serious or careful  thought .
 \textit{
	\begin{itemize}
	\item I think he is shallow, vain and untrustworthy.
	\item The evening news is often criticized for being shallow.
	\end{itemize}
}
\item adjective \\
If your breathing is \textbf{shallow} , you take only a very small amount of air into your lungs at each breath .
 \textit{
	\begin{itemize}
	\item She began to hear her own taut, shallow breathing.
	\end{itemize}
}
\end{enumerate}

\section*{standard}
{\large \color{blue}  standards  }
\subsection*{Explain}
\begin{enumerate}
\item countable noun \\
A \textbf{standard} is a level of quality or achievement , especially a level that is thought to be acceptable.
 \textit{
	\begin{itemize}
	\item The standard of professional cricket has never been lower.
	\item There will be new national standards for hospital cleanliness.
	\end{itemize}
}
\item countable noun \\
A \textbf{standard} is something that you use in order to judge the quality of something else.
 \textit{
	\begin{itemize}
	\item ...systems that were by later standards absurdly primitive.
	\end{itemize}
}
\item plural noun \\
\textbf{Standards} are moral principles which affect people's attitudes and behaviour.
 \textit{
	\begin{itemize}
	\item My father has always had high moral standards.
	\end{itemize}
}
\item adjective \\
You use \textbf{standard} to describe things which are usual and normal.
 \textit{
	\begin{itemize}
	\item It was standard practice for untrained clerks to advise in serious cases such as
murder.
	\item No other executive car can offer you the same level of standard equipment at this
price.
	\end{itemize}
}
\item adjective \\
A \textbf{standard} work or text on a particular subject is one that is widely read and often recommended .
 \textit{
	\begin{itemize}
	\end{itemize}
}
\end{enumerate}

\section*{strawberry}
{\large \color{blue}  strawberries  }
\subsection*{Explain}
\begin{enumerate}
\item countable noun \\
A \textbf{strawberry} is a small red fruit which is soft and juicy and has tiny yellow seeds on its skin.
 \textit{
	\begin{itemize}
	\item ...strawberries and cream.
	\item ...homemade strawberry jam.
	\end{itemize}
}
\end{enumerate}

\section*{sole}
{\large \color{blue}  soles  }
\subsection*{Explain}
\begin{enumerate}
\item adjective \\
The \textbf{sole} thing or person of a particular type is the only one of that type.
 \textit{
	\begin{itemize}
	\item Their sole aim is to destabilize the Indian government.
	\end{itemize}
}
\item adjective \\
If you have \textbf{sole}  charge or ownership of something, you are the only person in charge of it or who owns it.
 \textit{
	\begin{itemize}
	\item Many women are left as the sole providers in families after their husband has died.
	\item Chief Hart had sole control over that fund.
	\end{itemize}
}
\item countable noun \\
The \textbf{sole} of your foot or of a shoe or sock is the underneath surface of it.
 \textit{
	\begin{itemize}
	\item ...shoes with rubber soles.
	\item He had burned the sole of his foot.
	\end{itemize}
}
\item countable noun \\
A \textbf{sole} is a kind of flat  fish that you can eat .
 \textbf{Sole} is this fish eaten as food.
 \textit{
	\begin{itemize}
	\end{itemize}
}
\end{enumerate}

\section*{tablet}
{\large \color{blue}  tablets  }
\subsection*{Explain}
\begin{enumerate}
\item countable noun \\
A \textbf{tablet} is a small solid round mass of medicine which you swallow .
 \textit{
	\begin{itemize}
	\item It's time for your tablets, dear.
	\item It is never a good idea to take sleeping tablets regularly.
	\end{itemize}
}
\item countable noun \\
A \textbf{tablet} is a small flat computer that you operate by touching the screen.
 \textit{
	\begin{itemize}
	\item ...a free guide to the best tablets on the market.
	\end{itemize}
}
\item countable noun \\
Clay  \textbf{tablets} or stone \textbf{tablets} are the flat pieces of clay or stone which people used to write on before paper was
 invented .
 \textit{
	\begin{itemize}
	\item He also studied the ancient stone tablets from around the pyramids.
	\end{itemize}
}
\end{enumerate}

\section*{tiresome}
{\large \color{blue}  }
\subsection*{Explain}
\begin{enumerate}
\item adjective \\
If you describe someone or something as \textbf{tiresome} , you mean that you find them irritating or boring.
 \textit{
	\begin{itemize}
	\item ...the tiresome old lady next door.
	\item It would be too tiresome to wait in the queue.
	\end{itemize}
}
\end{enumerate}

\section*{tag}
{\large \color{blue}  tags  tagging  tagged  }
\subsection*{Explain}
\begin{enumerate}
\item countable noun \\
A \textbf{tag} is a small piece of card or cloth which is attached to an object or person and has
information about that object or person on it.
 \textit{
	\begin{itemize}
	\item Staff wore name tags.
	\item ...baggage tags.
	\end{itemize}
}
\item countable noun \\
An electronic \textbf{tag} is a device that is firmly attached to someone or something and sets off an alarm if that person or thing moves away or is removed.
 \textit{
	\begin{itemize}
	\item A hospital is to fit new-born babies with electronic tags to foil kidnappers.
	\item Sometimes, they've snapped off the security tag and just taken the one shoe.
	\end{itemize}
}
\item verb \\
If you \textbf{tag} something, you attach something to it or mark it so that it can be identified later.
 \textit{
	\begin{itemize}
	\item Professor Orr has developed interesting ways of tagging chemical molecules using
existing laboratory lasers.
	\item The most important trees were tagged to protect them from being damaged by machinery.
	\end{itemize}
}
\item countable noun \\
You can refer to a phrase that is used to describe someone or something as a \textbf{tag} .
 \textit{
	\begin{itemize}
	\item Jazz was losing its elitist tag and gaining a much broader audience.
	\end{itemize}
}
\item verb \\
If you \textbf{tag} someone in a particular way, you keep describing them using a particular phrase or
 thinking of them as a particular thing.
 \textit{
	\begin{itemize}
	\item ...the pundits were still tagging him with that age-old label, 'best of a bad bunch'.
	\item She has always lived in John's house and is still tagged 'Dad's girlfriend' by his
children.
	\end{itemize}
}
\item countable noun \\
A \textbf{tag} is a short phrase or saying that you quote from a book, speech, or piece of writing.
 \textit{
	\begin{itemize}
	\end{itemize}
}
\item uncountable noun \\
\textbf{Tag} is a children's game in which one child chases the others and tries to touch them.
 \textit{
	\begin{itemize}
	\end{itemize}
}
\end{enumerate}

\section*{vacant}
{\large \color{blue}  }
\subsection*{Explain}
\begin{enumerate}
\item adjective \\
If something is \textbf{vacant} , it is not being used by anyone.
 \textit{
	\begin{itemize}
	\item Half way down the coach was a vacant seat.
	\item In every major city there are more vacant buildings than there are homeless people.
	\end{itemize}
}
\item adjective \\
If a job or position is \textbf{vacant} , no one is doing it or in it at present, and people can apply for it.
 \textit{
	\begin{itemize}
	\item A number of senior people were regarded as likely to occupy the now vacant post.
	\item The post of chairman has been vacant for some time.
	\end{itemize}
}
\item adjective \\
A \textbf{vacant}  look or expression is one that suggests that someone does not understand something or that they are not thinking about anything in particular.
 \textit{
	\begin{itemize}
	\item She had a kind of vacant look on her face.
	\end{itemize}
}
\end{enumerate}

\section*{tale}
{\large \color{blue}  tales  }
\subsection*{Explain}
\begin{enumerate}
\item countable noun \\
A \textbf{tale} is a story, often involving magic or exciting events.
 \textit{
	\begin{itemize}
	\item ...a collection of stories, poems and folk tales.
	\item ...the tales of King Arthur and his Round Table.
	\end{itemize}
}
\item countable noun \\
You can refer to an interesting, exciting, or dramatic account of a real event as a \textbf{tale} .
 \textit{
	\begin{itemize}
	\item The media have been filled with tales of horror and loss resulting from Monday's
earthquake.
	\item He tells me long tales about my mother.
	\end{itemize}
}
\item  \\
 live to tell the tale \textit{
	\begin{itemize}
	\end{itemize}
}
\item  \\
 tell tales \textit{
	\begin{itemize}
	\end{itemize}
}
\end{enumerate}

\section*{virtual}
{\large \color{blue}  }
\subsection*{Explain}
\begin{enumerate}
\item adjective \\
You can use \textbf{virtual} to indicate that something is so nearly  true that for most purposes it can be regarded as true.
 \textit{
	\begin{itemize}
	\item Argentina came to a virtual standstill while the game was being played.
	\item He claimed to be a virtual prisoner in his own home.
	\item ...conditions of virtual slavery.
	\end{itemize}
}
\item adjective \\
\textbf{Virtual} objects and activities are generated by a computer to simulate  real objects and activities.
 \textit{
	\begin{itemize}
	\item Up to four players can compete in a virtual world of role playing.
	\item The site provided a virtual meeting place for activists.
	\end{itemize}
}
\end{enumerate}

\section*{tape}
{\large \color{blue}  tapes  taping  taped  }
\subsection*{Explain}
\begin{enumerate}
\item uncountable noun \\
\textbf{Tape} is a narrow plastic strip covered with a magnetic substance. It is used to record sounds, pictures , and computer  information .
 \textit{
	\begin{itemize}
	\item Tape is expensive and loses sound quality every time it is copied.
	\item Many students declined to be interviewed on tape.
	\end{itemize}
}
\item countable noun \\
A \textbf{tape} is a cassette or spool with magnetic tape wound  round it.
 \textit{
	\begin{itemize}
	\item ...a cassette tape.
	\item Her brother once found an old tape of her music hidden in the back of a drawer.
	\end{itemize}
}
\item verb \\
If you \textbf{tape} music, sounds, or television pictures, you record them so that you can watch or listen to them later .
 \textit{
	\begin{itemize}
	\item She has just taped an interview.
	\item He shouldn't be taping without the singer's permission.
	\item ...taped evidence from prisoners.
	\end{itemize}
}
\item variable noun \\
A \textbf{tape} is a strip of cloth used to tie things together or to identify who a piece of clothing belongs to.
 \textit{
	\begin{itemize}
	\item The books were all tied up with tape.
	\end{itemize}
}
\item countable noun \\
A \textbf{tape} is a ribbon that is stretched across the finishing line of a race.
 \textit{
	\begin{itemize}
	\item ...the finishing tape.
	\end{itemize}
}
\item uncountable noun \\
\textbf{Tape} is a sticky strip of plastic used for sticking things together.
 \textit{
	\begin{itemize}
	\item ...strong adhesive tape.
	\end{itemize}
}
\item verb \\
If you \textbf{tape} one thing to another, you attach it using sticky tape.
 \textit{
	\begin{itemize}
	\item I taped the base of the feather onto the velvet.
	\item There are notes from years ago taped to the walls.
	\item The envelope has been tampered with and then taped shut again.
	\end{itemize}
}
\end{enumerate}

\section*{void}
{\large \color{blue}  voids  voiding  voided  }
\subsection*{Explain}
\begin{enumerate}
\item countable noun \\
If you describe a situation or a feeling as a \textbf{void} , you mean that it seems empty because there is nothing interesting or worthwhile about it.
 \textit{
	\begin{itemize}
	\item His death has left a void in the cricketing world which can never be filled.
	\item ...an aching void of loneliness.
	\end{itemize}
}
\item countable noun \\
You can describe a large or frightening space as a \textbf{void} .
 \textit{
	\begin{itemize}
	\item He stared into the dark void where the battle had been fought.
	\item The ship moved silently through the black void.
	\item Observers have found in the universe giant voids about 500,000,000 light-years across.
	\end{itemize}
}
\item adjective \\
Something that is \textbf{void} or \textbf{null and void} is officially  considered to have no value or authority .
 \textit{
	\begin{itemize}
	\item The original elections were declared void by the former military ruler.
	\item The agreement will be considered null and void.
	\end{itemize}
}
\item adjective \\
If you are \textbf{void of} something, you do not have any of it.
 \textit{
	\begin{itemize}
	\item He rose, his face void of emotion as he walked towards the door.
	\item The treaty is now void of absolute commitments.
	\end{itemize}
}
\item verb \\
To \textbf{void} something means to officially say that it is not valid .
 \textit{
	\begin{itemize}
	\item The Supreme Court threw out the confession and voided his conviction for murder.
	\end{itemize}
}
\end{enumerate}

\section*{tea}
{\large \color{blue}  teas  }
\subsection*{Explain}
\begin{enumerate}
\item variable noun \\
\textbf{Tea} is a drink made by adding hot water to tea leaves or tea bags . Many people add  milk to the drink and some add sugar .
 A cup of tea can be referred to as a \textbf{tea} .
 \textit{
	\begin{itemize}
	\item ...a cup of tea.
	\item Would you like some tea?
	\item Four or five men were drinking tea from flasks.
	\item Would anybody like a tea or coffee?
	\end{itemize}
}
\item mass noun \\
Drinks such as mint  \textbf{tea} or chamomile \textbf{tea} are made by pouring hot water on the dried leaves of the particular plant or flower.
 \textit{
	\begin{itemize}
	\end{itemize}
}
\item variable noun \\
The chopped dried leaves of the plant that tea is made from is referred to as \textbf{tea} .
 \textit{
	\begin{itemize}
	\item ...a packet of tea.
	\item America imports about 190 million pounds of tea a year.
	\item Earl Grey, Darjeeling and Jasmine are best-selling traditional teas.
	\end{itemize}
}
\item variable noun \\
\textbf{Tea} is a meal some people eat in the late afternoon. It consists of food such as sandwiches and cakes, with tea to drink.
 \textit{
	\begin{itemize}
	\item I'm doing the sandwiches for tea.
	\item I took her to tea at the Ritz.
	\end{itemize}
}
\item variable noun \\
Some people refer to the main meal that they eat in the early part of the evening as \textbf{tea} .
 \textit{
	\begin{itemize}
	\item At five o'clock he comes back for his tea.
	\end{itemize}
}
\item  \\
 sb's cup of tea \textit{
	\begin{itemize}
	\end{itemize}
}
\end{enumerate}

\section*{absurd}
{\large \color{blue}  }
\subsection*{Explain}
\begin{enumerate}
\item adjective \\
If you say that something is \textbf{absurd} , you are criticizing it because you think that it is ridiculous or that it does not make sense .
 \textbf{The absurd} is something that is absurd.
 \textit{
	\begin{itemize}
	\item It is absurd to be discussing compulsory redundancy policies for teachers.
	\item I've known clients of mine go to absurd lengths, just to avoid paying me a few pounds.
	\item That's absurd.
	\item Parkinson had a sharp eye for the absurd.
	\end{itemize}
}
\end{enumerate}

\section*{anecdote}
{\large \color{blue}  anecdotes  }
\subsection*{Explain}
\begin{enumerate}
\item variable noun \\
An \textbf{anecdote} is a short, amusing account of something that has happened .
 \textit{
	\begin{itemize}
	\item Pete was telling them an anecdote about their mother.
	\item He has a talent for recollection and anecdote.
	\end{itemize}
}
\item variable noun \\
\textbf{Anecdotes} are individual accounts of something that are not reliable  evidence .
 \textit{
	\begin{itemize}
	\item The image of the fox as a pest is grossly exaggerated in anecdote and folklore.
	\end{itemize}
}
\end{enumerate}

\section*{actual}
{\large \color{blue}  }
\subsection*{Explain}
\begin{enumerate}
\item adjective \\
You use \textbf{actual} to emphasize that you are referring to something real or genuine.
 \textit{
	\begin{itemize}
	\item The segments are filmed using either local actors or the actual people involved.
	\item In this country, the actual number of miscarriages in humans is never fully recorded.
	\end{itemize}
}
\item adjective \\
You use \textbf{actual} to contrast the important  aspect of something with a less important aspect.
 \textit{
	\begin{itemize}
	\item She had compiled pages of notes, but she had not yet gotten down to doing the actual
writing.
	\item The exercises in this chapter can guide you, but it will be up to you to do the actual
work.
	\end{itemize}
}
\end{enumerate}

\section*{area}
{\large \color{blue}  areas  }
\subsection*{Explain}
\begin{enumerate}
\item countable noun \\
An \textbf{area} is a particular part of a town, a country, a region, or the world.
 \textit{
	\begin{itemize}
	\item ...the large number of community groups in the area.
	\item She works in a rural area off the beaten track.
	\item ...mountainous areas of Europe, Asia, North and South America.
	\end{itemize}
}
\item countable noun \\
Your \textbf{area} is the part of a town, country, or region where you live . An organization's \textbf{area} is the part of a town, country, or region that it is responsible for.
 \textit{
	\begin{itemize}
	\item Local authorities have been responsible for the running of schools in their areas.
	\item If there is an election in your area, you should go and vote.
	\end{itemize}
}
\item countable noun \\
A particular \textbf{area} is a piece of land or part of a building that is used for a particular activity.
 \textit{
	\begin{itemize}
	\item ...a picnic area.
	\item ...the main check-in area located in Terminal 1.
	\end{itemize}
}
\item countable noun \\
An \textbf{area} is a particular place on a surface or object, for example on your body.
 \textit{
	\begin{itemize}
	\item You will notice that your baby has two soft areas on the top of his head.
	\end{itemize}
}
\item variable noun \\
The \textbf{area} of a surface such as a piece of land is the amount of flat space or ground that it
covers, measured in square units.
 \textit{
	\begin{itemize}
	\item The islands cover a total area of 625.6 square kilometers.
	\item Although large in area, the flat did not have many rooms.
	\end{itemize}
}
\item countable noun \\
You can use \textbf{area} to refer to a particular subject or topic , or to a particular part of a larger, more general situation or activity.
 \textit{
	\begin{itemize}
	\item ...the politically sensitive area of old age pensions.
	\item ...the internationalization of the economy and all other areas of society.
	\item She wants to be involved in every area of your life.
	\end{itemize}
}
\item countable noun \\
On a football  pitch , \textbf{the area} is the same as the penalty area .
 \textit{
	\begin{itemize}
	\end{itemize}
}
\end{enumerate}

\section*{assistant}
{\large \color{blue}  assistants  }
\subsection*{Explain}
\begin{enumerate}
\item adjective \\
\textbf{Assistant} is used in front of titles or jobs to indicate a slightly  lower  rank . For example , an assistant director is one rank lower than a director in an organization.
 \textit{
	\begin{itemize}
	\item ...the Assistant Secretary of Defense.
	\item ...a young assistant professor at Harvard.
	\end{itemize}
}
\item countable noun \\
Someone's \textbf{assistant} is a person who helps them in their work.
 \textit{
	\begin{itemize}
	\item Kalan called his assistant, Hashim, to take over while he went out.
	\item The salesman had been accompanied to the meeting by an assistant.
	\end{itemize}
}
\item countable noun \\
An \textbf{assistant} is a person who works in a shop selling things to customers .
 \textit{
	\begin{itemize}
	\item The assistant took the book and checked the price on the back cover.
	\item She got a job as a sales assistant selling handbags.
	\end{itemize}
}
\end{enumerate}

\section*{axis}
{\large \color{blue}  axes  }
\subsection*{Explain}
\begin{enumerate}
\item countable noun \\
An \textbf{axis} is an imaginary line through the middle of something.
 \textit{
	\begin{itemize}
	\end{itemize}
}
\item countable noun \\
An \textbf{axis} of a graph is one of the two lines on which the scales of measurement are marked.
 \textit{
	\begin{itemize}
	\end{itemize}
}
\end{enumerate}

\section*{bleak}
{\large \color{blue}  bleaker  bleakest  }
\subsection*{Explain}
\begin{enumerate}
\item adjective \\
If a situation is \textbf{bleak} , it is bad , and seems  unlikely to improve .
 \textit{
	\begin{itemize}
	\item The immediate outlook remains bleak.
	\item Many predicted a bleak future.
	\end{itemize}
}
\item adjective \\
If you describe a place as \textbf{bleak} , you mean that it looks cold, empty , and unattractive .
 \textit{
	\begin{itemize}
	\item The island's pretty bleak.
	\item ...bleak inner-city streets.
	\end{itemize}
}
\item adjective \\
When the weather is \textbf{bleak} , it is cold, dull , and unpleasant .
 \textit{
	\begin{itemize}
	\item The weather can be quite bleak on the coast.
	\end{itemize}
}
\item adjective \\
If someone looks or sounds \textbf{bleak} , they look or sound depressed , as if they have no hope or energy .
 \textit{
	\begin{itemize}
	\item His face was bleak.
	\item Alberg gave him a bleak stare.
	\end{itemize}
}
\end{enumerate}

\section*{background}
{\large \color{blue}  backgrounds  }
\subsection*{Explain}
\begin{enumerate}
\item countable noun \\
Your \textbf{background} is the kind of family you come from and the kind of education you have had. It can also  refer to such things as your social and racial  origins , your financial  status , or the type of work experience that you have.
 \textit{
	\begin{itemize}
	\item She came from a working-class background.
	\item His background was in engineering.
	\end{itemize}
}
\item countable noun \\
The \textbf{background} to an event or situation consists of the facts that explain what caused it.
 \textit{
	\begin{itemize}
	\item The background to the current troubles is provided by the dire state of the country's
economy.
	\item The meeting takes place against a background of continuing political violence.
	\item ...background information.
	\end{itemize}
}
\item singular noun \\
\textbf{The}  \textbf{background} is sounds, such as music, which you can hear but which you are not listening to with your full attention .
 \textit{
	\begin{itemize}
	\item I kept hearing the sound of applause in the background.
	\item ...police sirens wailing in the background.
	\item The background music was provided by an accordion player.
	\end{itemize}
}
\item countable noun \\
You can use \textbf{background} to refer to the things in a picture or scene that are less noticeable or important than the main things or people in it.
 \textit{
	\begin{itemize}
	\item ...roses patterned on a blue background.
	\item Paint the background tones lighter and the colours cooler.
	\end{itemize}
}
\end{enumerate}

\section*{brisk}
{\large \color{blue}  brisker  briskest  }
\subsection*{Explain}
\begin{enumerate}
\item adjective \\
A \textbf{brisk} activity or action is done quickly and in an energetic way.
 \textit{
	\begin{itemize}
	\item Taking a brisk walk can often induce a feeling of well-being.
	\item The horse broke into a brisk trot.
	\end{itemize}
}
\item adjective \\
If trade or business is \textbf{brisk} , things are being sold very quickly and a lot of money is being made.
 \textit{
	\begin{itemize}
	\item Vendors were doing a brisk trade in souvenirs.
	\item Its sales had been brisk since July.
	\end{itemize}
}
\item adjective \\
If the weather is \textbf{brisk} , it is cold and fresh .
 \textit{
	\begin{itemize}
	\item ...a typically brisk winter's day on the South Coast.
	\item The breeze was cool, brisk and invigorating.
	\end{itemize}
}
\item adjective \\
Someone who is \textbf{brisk}  behaves in a busy , confident way which shows that they want to get things done quickly.
 \textit{
	\begin{itemize}
	\item The Chief summoned me downstairs. He was brisk and businesslike.
	\item She is noted for her brisk handling of business.
	\end{itemize}
}
\end{enumerate}

\section*{bacon}
{\large \color{blue}  }
\subsection*{Explain}
\begin{enumerate}
\item uncountable noun \\
\textbf{Bacon} is salted or smoked meat which comes from the back or sides of a pig.
 \textit{
	\begin{itemize}
	\item ...bacon and eggs.
	\item ...smoked streaky bacon.
	\end{itemize}
}
\item  \\
 bring home the bacon \textit{
	\begin{itemize}
	\end{itemize}
}
\item  \\
 bring home the bacon \textit{
	\begin{itemize}
	\end{itemize}
}
\item  \\
 to save someone's bacon \textit{
	\begin{itemize}
	\end{itemize}
}
\end{enumerate}

\section*{brutal}
{\large \color{blue}  }
\subsection*{Explain}
\begin{enumerate}
\item adjective \\
A \textbf{brutal} act or person is cruel and violent .
 \textit{
	\begin{itemize}
	\item He was the victim of a very brutal murder.
	\item ...the brutal suppression of anti-government protests.
	\item Jensen is a dangerous man, and can be very brutal and reckless.
	\end{itemize}
}
\item adjective \\
If someone expresses something unpleasant with \textbf{brutal}  honesty or frankness, they express it in a clear and accurate way, without attempting to disguise its unpleasantness.
 \textit{
	\begin{itemize}
	\item It was refreshing to talk about themselves and their feelings with brutal honesty.
	\item He took an anguished breath. He had to be brutal and say it.
	\end{itemize}
}
\item adjective \\
\textbf{Brutal} is used to describe things that have an unpleasant effect on people, especially when there is no attempt by anyone to reduce their effect.
 \textit{
	\begin{itemize}
	\item The dip in prices this summer will be brutal.
	\item The afternoon sun had been brutal.
	\item The 20th century brought brutal change to some countries.
	\end{itemize}
}
\end{enumerate}

\section*{barrier}
{\large \color{blue}  barriers  }
\subsection*{Explain}
\begin{enumerate}
\item countable noun \\
A \textbf{barrier} is something such as a rule , law , or policy that makes it difficult or impossible for something to happen or be achieved .
 \textit{
	\begin{itemize}
	\item Duties and taxes are the most obvious barrier to free trade.
	\end{itemize}
}
\item countable noun \\
A \textbf{barrier} is a problem that prevents two people or groups from agreeing , communicating , or working with each other.
 \textit{
	\begin{itemize}
	\item There is no reason why love shouldn't cross the age barrier.
	\item She had been waiting for Simon to break down the barrier between them.
	\item When you get involved in sports and athletes, a lot of the racial barriers are broken
down.
	\end{itemize}
}
\item countable noun \\
A \textbf{barrier} is something such as a fence or wall that is put in place to prevent people from moving easily from one area to another.
 \textit{
	\begin{itemize}
	\item The demonstrators broke through heavy police barriers.
	\item As each woman reached the barrier one of the men glanced at her papers.
	\end{itemize}
}
\item countable noun \\
A \textbf{barrier} is an object or layer that physically prevents something from moving from one place to another.
 \textit{
	\begin{itemize}
	\item ...a severe storm, which destroyed a natural barrier between the house and the lake.
	\item The packaging must provide an effective barrier to prevent contamination of the product.
	\end{itemize}
}
\item singular noun \\
You can refer to a particular number or amount as a \textbf{barrier} when you think it is significant , because it is difficult or unusual to go above it.
 \textit{
	\begin{itemize}
	\item They are fearful that unemployment will soon break the barrier of three million.
	\item The Popular Front failed, as expected, to pass the 5 per cent barrier.
	\end{itemize}
}
\end{enumerate}

\section*{certain}
{\large \color{blue}  }
\subsection*{Explain}
\begin{enumerate}
\item adjective \\
If you are \textbf{certain} about something, you firmly believe it is true and have no doubt about it. If you are not \textbf{certain} about something, you do not have definite  knowledge about it.
 \textit{
	\begin{itemize}
	\item She's absolutely certain she's going to make it in the world.
	\item We are not certain whether the appendix had already burst or not.
	\item It wasn't a balloon–I'm certain of that.
	\end{itemize}
}
\item adjective \\
If you say that something is \textbf{certain}  \textbf{to}  happen , you mean that it will definitely happen.
 \textit{
	\begin{itemize}
	\item However, the scheme is certain to meet opposition from fishermen's leaders.
	\item It's not certain they'll accept the Front's candidate if he wins.
	\item Brazil need to beat Uruguay to be certain of a place in the finals.
	\item The Prime Minister is heading for certain defeat if he forces a vote.
	\item Victory looked certain.
	\end{itemize}
}
\item adjective \\
If you say that something is \textbf{certain} , you firmly believe that it is true, or have definite knowledge about it.
 \textit{
	\begin{itemize}
	\item One thing is certain, both have the utmost respect for each other.
	\item It is certain that Rodney arrived the previous day.
	\end{itemize}
}
\item graded adjective \\
If you have \textbf{certain} knowledge, you know that a particular thing is true.
 \textit{
	\begin{itemize}
	\item He had been there four times to my certain knowledge.
	\end{itemize}
}
\item  \\
 for certain \textit{
	\begin{itemize}
	\end{itemize}
}
\item  \\
 make certain \textit{
	\begin{itemize}
	\end{itemize}
}
\end{enumerate}

\section*{battle}
{\large \color{blue}  battles  battling  battled  }
\subsection*{Explain}
\begin{enumerate}
\item variable noun \\
A \textbf{battle} is a violent fight between groups of people, especially one between military forces during a war .
 \textit{
	\begin{itemize}
	\item ...the victory of King William III at the Battle of the Boyne.
	\item ...after a gun battle between police and drug traffickers.
	\item ...men who die in battle.
	\end{itemize}
}
\item countable noun \\
A \textbf{battle} is a conflict in which different people or groups compete in order to achieve success or control .
 \textit{
	\begin{itemize}
	\item The political battle over the pre-budget report promises to be a bitter one.
	\item ...the eternal battle between good and evil in the world.
	\item ...a macho battle for supremacy.
	\item He was appalled to discover members of the board fighting damaging personal battles.
	\end{itemize}
}
\item countable noun \\
You can use \textbf{battle} to refer to someone's efforts to achieve something in spite of very difficult  circumstances .
 \textit{
	\begin{itemize}
	\item ...the battle against crime.
	\item She has fought a constant battle with her weight.
	\item Greg lost his brave battle against cancer two years ago.
	\end{itemize}
}
\item verb \\
To \textbf{battle}  \textbf{with} an opposing group means to take part in a fight or contest against them. In American English, you can also  say that one group or person \textbf{is battling} another.
 \textit{
	\begin{itemize}
	\item Thousands of people battled with police and several were reportedly wounded.
	\item The sides must battle again for a quarter-final place on December 16.
	\item They're also battling the government to win compensation.
	\end{itemize}
}
\item verb \\
To \textbf{battle} means to try  hard to do something in spite of very difficult circumstances. In British English, you
 \textbf{battle}  \textbf{against} something or \textbf{with} something. In American English, you \textbf{battle} something.
 \textit{
	\begin{itemize}
	\item Doctors battled throughout the night to save her life.
	\item ...a lone yachtsman returning from his months of battling with the elements.
	\item In Wyoming, firefighters are still battling the two blazes.
	\end{itemize}
}
\item  \\
 do battle \textit{
	\begin{itemize}
	\end{itemize}
}
\item  \\
 half the battle \textit{
	\begin{itemize}
	\end{itemize}
}
\item  \\
 the battle lines are drawn \textit{
	\begin{itemize}
	\end{itemize}
}
\item  \\
 to fight a losing battle \textit{
	\begin{itemize}
	\end{itemize}
}
\item  \\
 battle it out \textit{
	\begin{itemize}
	\end{itemize}
}
\item  \\
 lose/win the battle, win/lose the war \textit{
	\begin{itemize}
	\end{itemize}
}
\item  \\
 battle of wills \textit{
	\begin{itemize}
	\end{itemize}
}
\item  \\
 battle of wits \textit{
	\begin{itemize}
	\end{itemize}
}
\end{enumerate}

\section*{classical}
{\large \color{blue}  }
\subsection*{Explain}
\begin{enumerate}
\item adjective \\
You use \textbf{classical} to describe something that is traditional in form, style, or content .
 \textit{
	\begin{itemize}
	\item Fokine did not change the steps of classical ballet; instead he found new ways of
using them.
	\item ...the scientific attitude of Smith and earlier classical economists.
	\end{itemize}
}
\item adjective \\
\textbf{Classical} music is music that is considered to be serious and of lasting value.
 \textit{
	\begin{itemize}
	\end{itemize}
}
\item adjective \\
\textbf{Classical} is used to describe things which relate to the ancient Greek or Roman civilizations.
 \textit{
	\begin{itemize}
	\item ...the healers of ancient Egypt and classical Greece.
	\item It's a technological achievement that is unrivalled in the classical world.
	\item ...classical architecture.
	\end{itemize}
}
\item adjective \\
A \textbf{classical} language is a form of a language that was used in ancient times and is now no longer used, or only used in formal  writing .
 \textit{
	\begin{itemize}
	\item ...a line of classical Arabic poetry.
	\end{itemize}
}
\end{enumerate}

\section*{considerable}
{\large \color{blue}  }
\subsection*{Explain}
\begin{enumerate}
\item adjective \\
\textbf{Considerable} means great in amount or degree .
 \textit{
	\begin{itemize}
	\item To be without Pearce would be a considerable blow.
	\item Doing it properly makes considerable demands on our time.
	\item Vets' fees can be considerable, even for routine visits.
	\end{itemize}
}
\end{enumerate}

\section*{bread}
{\large \color{blue}  breads  breading  breaded  }
\subsection*{Explain}
\begin{enumerate}
\item variable noun \\
\textbf{Bread} is a very common food made from flour, water, and yeast.
 \textit{
	\begin{itemize}
	\item ...a loaf of bread.
	\item ...bread and butter.
	\item There is more fibre in wholemeal bread than in white bread.
	\end{itemize}
}
\item uncountable noun \\
If you earn your \textbf{bread} doing a particular job or activity , you earn your money doing it.
 \textit{
	\begin{itemize}
	\item There's not a living soul in Colorado who doesn't depend for his bread on silver.
	\end{itemize}
}
\item verb \\
If food such as fish or meat  \textbf{is breaded} , it is covered in tiny pieces of dry bread called breadcrumbs. It can then be fried or grilled .
 \textit{
	\begin{itemize}
	\item It is important that food be breaded just minutes before frying.
	\end{itemize}
}
\end{enumerate}

\section*{cordial}
{\large \color{blue}  cordials  }
\subsection*{Explain}
\begin{enumerate}
\item adjective \\
\textbf{Cordial} means friendly.
 \textit{
	\begin{itemize}
	\item He had never known him to be so chatty and cordial.
	\item He said the two countries had close and cordial relations.
	\end{itemize}
}
\item variable noun \\
\textbf{Cordial} is a sweet non-alcoholic drink made from fruit juice .
 \textit{
	\begin{itemize}
	\item ...fruit cordials.
	\end{itemize}
}
\item mass noun \\
A \textbf{cordial} is a strong  alcoholic drink with a sweet taste . You drink it after a meal .
 \textit{
	\begin{itemize}
	\end{itemize}
}
\end{enumerate}

\section*{brick}
{\large \color{blue}  bricks  bricking  bricked  }
\subsection*{Explain}
\begin{enumerate}
\item variable noun \\
\textbf{Bricks} are rectangular blocks of baked clay used for building walls, which are usually red or brown . \textbf{Brick} is the material made up of these blocks.
 \textit{
	\begin{itemize}
	\item She built bookshelves out of bricks and planks.
	\item ...a tiny garden surrounded by high brick walls.
	\end{itemize}
}
\item singular noun \\
If you say that someone is \textbf{a brick} , you mean that they have helped you or supported you when you were in a difficult  situation .
 \textit{
	\begin{itemize}
	\item You were a brick, a real friend in need.
	\end{itemize}
}
\item  \\
 be banging one's head against a brick wall \textit{
	\begin{itemize}
	\end{itemize}
}
\item  \\
 hit/come up against a brick wall \textit{
	\begin{itemize}
	\end{itemize}
}
\item  \\
 bricks and mortar \textit{
	\begin{itemize}
	\end{itemize}
}
\end{enumerate}

\section*{cruel}
{\large \color{blue}  crueller  cruellest  }
\subsection*{Explain}
\begin{enumerate}
\item adjective \\
Someone who is \textbf{cruel} deliberately causes pain or distress to people or animals.
 \textit{
	\begin{itemize}
	\item Children can be so cruel.
	\item Don't you think it's cruel to cage a creature up?
	\end{itemize}
}
\item adjective \\
A situation or event that is \textbf{cruel} is very harsh and causes people distress.
 \textit{
	\begin{itemize}
	\item ...struggling to survive in a cruel world with which they cannot cope.
	\item By a cruel irony, his horse came down on a flat part of the course.
	\end{itemize}
}
\end{enumerate}

\section*{cheek}
{\large \color{blue}  cheeks  }
\subsection*{Explain}
\begin{enumerate}
\item countable noun \\
Your \textbf{cheeks} are the sides of your face below your eyes.
 \textit{
	\begin{itemize}
	\item Tears were running down her cheeks.
	\item She kissed him lightly on both cheeks.
	\end{itemize}
}
\item singular noun \\
You say that someone has a \textbf{cheek} when you are annoyed or shocked at something unreasonable that they have done.
 \textit{
	\begin{itemize}
	\item I'm amazed they had the cheek to ask in the first place.
	\item I still think it's a bit of a cheek sending a voucher rather than a refund.
	\item The cheek of it, lying to me like that!
	\end{itemize}
}
\item  \\
 turn the other cheek \textit{
	\begin{itemize}
	\end{itemize}
}
\end{enumerate}

\section*{desolate}
{\large \color{blue}  desolates  desolating  desolated  }
\subsection*{Explain}
\begin{enumerate}
\item adjective \\
A \textbf{desolate} place is empty of people and lacking in comfort .
 \textit{
	\begin{itemize}
	\item ...a desolate landscape of flat green fields broken by marsh.
	\item Half-ruined, hardly a building untouched, it's a desolate place.
	\end{itemize}
}
\item adjective \\
If someone is \textbf{desolate} , they feel very sad , alone , and without hope.
 \textit{
	\begin{itemize}
	\item He was desolate without her.
	\end{itemize}
}
\item verb \\
If something \textbf{desolates} you, it upsets you and makes you very unhappy .
 \textit{
	\begin{itemize}
	\item Their inclination to wait and demand more resources desolated President Lincoln.
	\end{itemize}
}
\end{enumerate}

\section*{conflict}
{\large \color{blue}  conflicts  conflicting  conflicted  }
\subsection*{Explain}
\begin{enumerate}
\item uncountable noun \\
\textbf{Conflict} is serious disagreement and argument about something important . If two people or groups are \textbf{in}  \textbf{conflict} , they have had a serious disagreement or argument and have not yet reached  agreement .
 \textit{
	\begin{itemize}
	\item Try to keep any conflict between you and your ex-partner to a minimum.
	\item Employees already are in conflict with management over job cuts.
	\item The two companies came into conflict.
	\end{itemize}
}
\item uncountable noun \\
\textbf{Conflict} is a state of mind in which you find it impossible to make a decision .
 \textit{
	\begin{itemize}
	\item ...the anguish of his own inner conflict.
	\end{itemize}
}
\item variable noun \\
\textbf{Conflict} is fighting between countries or groups of people.
 \textit{
	\begin{itemize}
	\item ...talks aimed at ending four decades of conflict.
	\item The National Security Council has met to discuss ways of preventing a military conflict.
	\end{itemize}
}
\item variable noun \\
A \textbf{conflict} is a serious difference between two or more beliefs , ideas, or interests. If two beliefs, ideas, or interests are \textbf{in}  \textbf{conflict} , they are very different .
 \textit{
	\begin{itemize}
	\item There is a conflict between what they are doing and what you want.
	\item Do you feel any conflict of loyalties?
	\item The two objectives are in conflict.
	\end{itemize}
}
\item verb \\
If ideas, beliefs, or accounts  \textbf{conflict} , they are very different from each other and it seems impossible for them to exist  together or to each be true .
 \textit{
	\begin{itemize}
	\item Personal ethics and professional ethics sometimes conflict.
	\item He held firm opinions which usually conflicted with my own.
	\item There are conflicting reports about the identity of the hostage.
	\item ...three powers with conflicting interests.
	\end{itemize}
}
\end{enumerate}

\section*{durable}
{\large \color{blue}  }
\subsection*{Explain}
\begin{enumerate}
\item adjective \\
Something that is \textbf{durable} is strong and lasts a long time without breaking or becoming  weaker .
 \textit{
	\begin{itemize}
	\item Bone china is strong and durable.
	\end{itemize}
}
\end{enumerate}

\section*{contempt}
{\large \color{blue}  }
\subsection*{Explain}
\begin{enumerate}
\item uncountable noun \\
If you have \textbf{contempt}  \textbf{for} someone or something, you have no respect for them or think that they are unimportant .
 \textit{
	\begin{itemize}
	\item He has contempt for those beyond his immediate family circle.
	\item I hope voters will treat his advice with the contempt it deserves.
	\end{itemize}
}
\item uncountable noun \\
\textbf{Contempt} means the same as contempt of court .
 \textit{
	\begin{itemize}
	\item Mr. Kelly was sentenced to six months in prison for contempt.
	\end{itemize}
}
\item  \\
 hold sb/sth in contempt \textit{
	\begin{itemize}
	\end{itemize}
}
\end{enumerate}

\section*{economic}
{\large \color{blue}  }
\subsection*{Explain}
\begin{enumerate}
\item adjective \\
\textbf{Economic} means concerned with the organization of the money, industry , and trade of a country, region, or society .
 \textit{
	\begin{itemize}
	\item ...Poland's radical economic reforms.
	\item The pace of economic growth is picking up.
	\end{itemize}
}
\item adjective \\
If something is \textbf{economic} , it produces a profit.
 \textit{
	\begin{itemize}
	\item The new system may be more economic but will lead to a decline in programme quality.
	\end{itemize}
}
\end{enumerate}

\section*{correlate}
{\large \color{blue}  correlates  correlating  correlated  }
\subsection*{Explain}
\begin{enumerate}
\item verb \\
If one thing \textbf{correlates}  \textbf{with} another, there is a close similarity or connection between them, often because one thing causes the other. You can also  say that two things \textbf{correlate} .
 \textit{
	\begin{itemize}
	\item Obesity correlates with increased risk for hypertension and stroke.
	\item The political opinions of spouses correlate more closely than their heights.
	\item The loss of respect for British science is correlated to reduced funding.
	\item At the highest executive levels, earnings and performance aren't always correlated.
	\end{itemize}
}
\item verb \\
If you \textbf{correlate} things, you work out the way in which they are connected or the way they influence each other.
 \textit{
	\begin{itemize}
	\item The report correlated the stock market values of the companies with their losses.

	\item Lieutenant Ryan closed his eyes, first mentally viewing the different crime scenes,
then correlating the data.
	\end{itemize}
}
\end{enumerate}

\section*{equal}
{\large \color{blue}  equals  equalling  equalled  }
\subsection*{Explain}
\begin{enumerate}
\item adjective \\
If two things are \textbf{equal} or if one thing is \textbf{equal}  \textbf{to} another, they are the same in size, number, standard , or value .
 \textit{
	\begin{itemize}
	\item Investors can borrow an amount equal to the property's purchase price.
	\item ...in a population having equal numbers of men and women.
	\item Research and teaching are of equal importance.
	\end{itemize}
}
\item adjective \\
If different groups of people have \textbf{equal} rights or are given  \textbf{equal}  treatment , they have the same rights or are treated the same as each other, however different they are.
 \textit{
	\begin{itemize}
	\item We will be justly demanding equal rights at work.
	\item ...the commitment to equal opportunities.
	\item ...new legislation allowing companies to compete on equal terms.
	\end{itemize}
}
\item adjective \\
If you say that people are \textbf{equal} , you mean that they have or should have the same rights and opportunities as each other.
 \textit{
	\begin{itemize}
	\item We are equal in every way.
	\item We teach our children that everyone is equal under the law.
	\end{itemize}
}
\item countable noun \\
Someone who is your \textbf{equal} has the same ability, status, or rights as you have.
 \textit{
	\begin{itemize}
	\item She was one of the boys, their equal.
	\item You should have married somebody more your equal.
	\end{itemize}
}
\item adjective \\
If someone is \textbf{equal to} a particular  job or situation , they have the necessary ability, strength, or courage to deal successfully with it.
 \textit{
	\begin{itemize}
	\item She was determined that she would be equal to any test the corporation put to them.
	\item The guards were equal to anything.
	\end{itemize}
}
\item link verb \\
If something \textbf{equals} a particular number or amount , it is the same as that amount or the equivalent of that amount.
 \textit{
	\begin{itemize}
	\item 9 percent interest less 7 percent inflation equals 2 percent.
	\item The average pay rise equalled 1.41 times inflation.
	\end{itemize}
}
\item verb \\
To \textbf{equal} something or someone means to be as good or as great as them.
 \textit{
	\begin{itemize}
	\item The victory equalled Southend's best in history.
	\item No amount of money can equal memories like that.
	\end{itemize}
}
\item  \\
 has no equal \textit{
	\begin{itemize}
	\end{itemize}
}
\item  \\
 other things being equal \textit{
	\begin{itemize}
	\end{itemize}
}
\end{enumerate}

\section*{discrepancy}
{\large \color{blue}  discrepancies  }
\subsection*{Explain}
\begin{enumerate}
\item variable noun \\
If there is a \textbf{discrepancy}  \textbf{between} two things that ought to be the same, there is a noticeable  difference between them.
 \textit{
	\begin{itemize}
	\item ...the discrepancy between press and radio reports.
	\item ...major discrepancies in payments made to claimants in similar circumstances.
	\end{itemize}
}
\end{enumerate}

\section*{flour}
{\large \color{blue}  flours  flouring  floured  }
\subsection*{Explain}
\begin{enumerate}
\item variable noun \\
\textbf{Flour} is a white or brown powder that is made by grinding grain. It is used to make bread , cakes , and pastry .
 \textit{
	\begin{itemize}
	\end{itemize}
}
\item verb \\
If you \textbf{flour} cooking equipment or food, you cover it with flour.
 \textit{
	\begin{itemize}
	\item Lightly flour a rolling pin.
	\item Remove the dough from the bowl and put it on a floured surface.
	\end{itemize}
}
\end{enumerate}

\section*{explicit}
{\large \color{blue}  }
\subsection*{Explain}
\begin{enumerate}
\item adjective \\
Something that is \textbf{explicit} is expressed or shown clearly and openly, without any attempt to hide anything.
 \textit{
	\begin{itemize}
	\item ...sexually explicit scenes in films and books.
	\item ...explicit references to age in recruitment advertising.
	\end{itemize}
}
\item adjective \\
If you are \textbf{explicit}  \textbf{about} something, you speak about it very openly and clearly.
 \textit{
	\begin{itemize}
	\item He was explicit about his intention to overhaul the party's internal voting system.
	\end{itemize}
}
\end{enumerate}

\section*{gum}
{\large \color{blue}  gums  gumming  gummed  }
\subsection*{Explain}
\begin{enumerate}
\item variable noun \\
\textbf{Gum} is a substance, usually tasting of mint , which you chew for a long time but do not swallow .
 \textit{
	\begin{itemize}
	\end{itemize}
}
\item countable noun \\
Your \textbf{gums} are the areas of firm , pink  flesh  inside your mouth , which your teeth grow out of.
 \textit{
	\begin{itemize}
	\item The toothbrush gently removes plaque without damaging the gums.
	\item ...gum disease.
	\end{itemize}
}
\item variable noun \\
\textbf{Gum} is a type of glue that is used to stick two pieces of paper together.
 \textit{
	\begin{itemize}
	\item He was holding up a pound note that had been torn in half and stuck together with
gum.
	\end{itemize}
}
\item adjective \\
If two things are \textbf{gummed}  \textbf{together} , they are stuck together.
 \textit{
	\begin{itemize}
	\item It is a mild infection in which a baby's eyelashes can become gummed together.
	\end{itemize}
}
\item mass noun \\
\textbf{Gum} is a sticky substance which comes from the eucalyptus tree or from various other trees and shrubs .
 \textit{
	\begin{itemize}
	\end{itemize}
}
\end{enumerate}

\section*{feudal}
{\large \color{blue}  }
\subsection*{Explain}
\begin{enumerate}
\item adjective \\
\textbf{Feudal} means relating to the system or the time of feudalism.
 \textit{
	\begin{itemize}
	\item ...the emperor and his feudal barons.
	\end{itemize}
}
\end{enumerate}

\section*{handle}
{\large \color{blue}  handles  handling  handled  }
\subsection*{Explain}
\begin{enumerate}
\item countable noun \\
A \textbf{handle} is a small round object or a lever that is attached to a door and is used for opening and closing it.
 \textit{
	\begin{itemize}
	\item I turned the handle and found the door was open.
	\end{itemize}
}
\item countable noun \\
A \textbf{handle} is the part of an object such as a tool , bag , or cup that you hold in order to be able to pick up and use the object.
 \textit{
	\begin{itemize}
	\item The handle of a cricket bat protruded from under his arm.
	\item ...a broom handle.
	\end{itemize}
}
\item verb \\
If you say that someone can \textbf{handle} a problem or situation , you mean that they have the ability to deal with it successfully.
 \textit{
	\begin{itemize}
	\item To tell the truth, I don't know if I can handle the job.
	\item She cannot handle pressure.
	\item You must learn how to handle your feelings.
	\end{itemize}
}
\item verb \\
If you talk about the way that someone \textbf{handles} a problem or situation, you mention whether or not they are successful in achieving the result they want .
 \textit{
	\begin{itemize}
	\item I think I would handle a meeting with Mr. Siegel very badly.
	\item She admitted to herself she didn't know how to handle the problem.
	\end{itemize}
}
\item verb \\
If you \textbf{handle} a particular area of work, you have responsibility for it.
 \textit{
	\begin{itemize}
	\item She handled travel arrangements for the press corps during the presidential campaign.
	\item The investigation is being handled by Scotland Yard's anti terrorist branch.
	\end{itemize}
}
\item verb \\
When you \textbf{handle} something such as a weapon , vehicle , or animal, you use it or control it, especially by using your hands.
 \textit{
	\begin{itemize}
	\item I had never handled an automatic.
	\end{itemize}
}
\item verb \\
If something such as a vehicle \textbf{handles}  well , it is easy to use or control.
 \textit{
	\begin{itemize}
	\item His ship had handled like a dream!
	\end{itemize}
}
\item verb \\
When you \textbf{handle} something, you hold it or move it with your hands.
 \textit{
	\begin{itemize}
	\item Wear rubber gloves when handling cat litter.
	\end{itemize}
}
\item singular noun \\
If you have \textbf{a}  \textbf{handle}  \textbf{on} a subject or problem, you have a way of approaching it that helps you to understand it or deal with it.
 \textit{
	\begin{itemize}
	\item When you have got a handle on your anxiety you can begin to control it.
	\end{itemize}
}
\item  \\
 to fly off the handle \textit{
	\begin{itemize}
	\end{itemize}
}
\end{enumerate}

\section*{honest}
{\large \color{blue}  }
\subsection*{Explain}
\begin{enumerate}
\item adjective \\
If you describe someone as \textbf{honest} , you mean that they always  tell the truth , and do not try to deceive people or break the law .
 \textit{
	\begin{itemize}
	\item My dad was the most honest man I ever met.
	\item I know she's honest and reliable.
	\end{itemize}
}
\item adjective \\
If you are \textbf{honest} in a particular situation , you tell the complete truth or give your sincere  opinion , even if this is not very pleasant .
 \textit{
	\begin{itemize}
	\item I was honest about what I was doing.
	\item He had been honest with her and she had tricked him!
	\item What do you think of the school, in your honest opinion?
	\end{itemize}
}
\item adverb \\
You say ' \textbf{honest} ' before or after a statement to emphasize that you are telling the truth and that you want people to believe you.
 \textit{
	\begin{itemize}
	\item I'm not sure, honest.
	\end{itemize}
}
\item  \\
 honest to God \textit{
	\begin{itemize}
	\end{itemize}
}
\item  \\
 to be honest \textit{
	\begin{itemize}
	\end{itemize}
}
\end{enumerate}

\section*{intersection}
{\large \color{blue}  intersections  }
\subsection*{Explain}
\begin{enumerate}
\item countable noun \\
An \textbf{intersection} is a place where roads or other lines meet or cross .
 \textit{
	\begin{itemize}
	\item ...at the intersection of two main canals.
	\item ...a busy highway intersection.
	\end{itemize}
}
\end{enumerate}

\section*{inverse}
{\large \color{blue}  }
\subsection*{Explain}
\begin{enumerate}
\item adjective \\
If there is an \textbf{inverse} relationship between two things, one of them becomes larger as the other becomes
smaller.
 \textit{
	\begin{itemize}
	\item The tension grew in inverse proportion to the distance from their final destination.
	\end{itemize}
}
\item singular noun \\
\textbf{The inverse} of something is its exact opposite.
 \textbf{Inverse} is also an adjective .
 \textit{
	\begin{itemize}
	\item There is no sign that you bothered to consider the inverse of your logic.
	\item The hologram can be flipped to show the inverse image.
	\end{itemize}
}
\end{enumerate}

\section*{lemon}
{\large \color{blue}  lemons  }
\subsection*{Explain}
\begin{enumerate}
\item variable noun \\
A \textbf{lemon} is a bright yellow fruit with very sour  juice . Lemons grow on trees in warm countries .
 \textit{
	\begin{itemize}
	\item ...a slice of lemon.
	\item ...oranges, lemons and other citrus fruits.
	\item ...lemon juice.
	\end{itemize}
}
\item uncountable noun \\
\textbf{Lemon} is a drink that tastes of lemons.
 \textit{
	\begin{itemize}
	\end{itemize}
}
\item colour \\
\textbf{Lemon} is the same as lemon yellow .
 \textit{
	\begin{itemize}
	\end{itemize}
}
\item countable noun \\
If you think that something is a failure , or not as good or as useful as it should be, you can say that it is a \textbf{lemon} .
 \textit{
	\begin{itemize}
	\item He took a little test drive and agreed the car was a lemon.
	\end{itemize}
}
\item countable noun \\
If you think that someone looks  foolish because they are shy or slow to take action, you can say that they are \textbf{like a}  \textbf{lemon} .
 \textit{
	\begin{itemize}
	\item I just stood there like a lemon.
	\end{itemize}
}
\end{enumerate}

\section*{linear}
{\large \color{blue}  }
\subsection*{Explain}
\begin{enumerate}
\item adjective \\
A \textbf{linear} process or development is one in which something changes or progresses  straight from one stage to another, and has a starting point and an ending point.
 \textit{
	\begin{itemize}
	\item Her novel subverts the conventions of linear narrative.
	\item ...the linear view of time.
	\end{itemize}
}
\item adjective \\
A \textbf{linear} shape or form consists of straight lines.
 \textit{
	\begin{itemize}
	\item ...the sharp, linear designs of the Seventies and Eighties.
	\end{itemize}
}
\item adjective \\
\textbf{Linear}  movement or force occurs in a straight line rather than in a curve .
 \textit{
	\begin{itemize}
	\end{itemize}
}
\end{enumerate}

\section*{mask}
{\large \color{blue}  masks  masking  masked  }
\subsection*{Explain}
\begin{enumerate}
\item countable noun \\
A \textbf{mask} is a piece of cloth or other material, which you wear over your face so that people
cannot see who you are, or so that you look like someone or something else.
 \textit{
	\begin{itemize}
	\item The gunman, whose mask had slipped, fled.
	\item ...actors wearing masks.
	\end{itemize}
}
\item countable noun \\
A \textbf{mask} is a piece of cloth or other material that you wear over all or part of your face
to protect you from germs or harmful substances.
 \textit{
	\begin{itemize}
	\item You must wear goggles and a mask that will protect you against the fumes.
	\item She wore a surgical mask and rubber gloves while she worked with the samples.
	\end{itemize}
}
\item countable noun \\
If you describe someone's behaviour as a \textbf{mask} , you mean that they do not show their real feelings or character.
 \textit{
	\begin{itemize}
	\item His mask of detachment cracked, and she saw for an instant an angry and violent man.
	\end{itemize}
}
\item countable noun \\
A \textbf{mask} is a thick cream or paste made of various substances, which you spread over your face and leave for some time
in order to improve your skin.
 \textit{
	\begin{itemize}
	\item This mask leaves your complexion feeling soft and supple.
	\end{itemize}
}
\item verb \\
If you \textbf{mask} your feelings, you deliberately do not show them in your behaviour, so that people
cannot know what you really  feel .
 \textit{
	\begin{itemize}
	\item Mr Straw has, in public at least, masked his disappointment.
	\end{itemize}
}
\item verb \\
If one thing \textbf{masks} another, it prevents people from noticing or recognizing the other thing.
 \textit{
	\begin{itemize}
	\item A thick grey cloud masked the sun.
	\item Too much salt masks the true flavour of the food.
	\item The healthy trade figures mask a much gloomier picture.
	\end{itemize}
}
\end{enumerate}

\section*{lonely}
{\large \color{blue}  lonelier  loneliest  }
\subsection*{Explain}
\begin{enumerate}
\item adjective \\
Someone who is \textbf{lonely} is unhappy because they are alone or do not have anyone they can  talk to.
 \textbf{The lonely} are people who are lonely.
 \textit{
	\begin{itemize}
	\item ...lonely people who just want to talk.
	\item I feel lonelier in the middle of London than I do on my boat in the middle of nowhere.
	\item He looks for the lonely, the lost, the unloved.
	\end{itemize}
}
\item adjective \\
A \textbf{lonely}  situation or period of time is one in which you feel unhappy because you are alone or do not have anyone to talk to.
 \textit{
	\begin{itemize}
	\item I desperately needed something to occupy me during those long, lonely nights.
	\item ...her lonely childhood.
	\end{itemize}
}
\item adjective \\
A \textbf{lonely} place is one where very few people come .
 \textit{
	\begin{itemize}
	\item It felt like the loneliest place in the world.
	\item ...dark, lonely streets.
	\end{itemize}
}
\end{enumerate}

\section*{orange}
{\large \color{blue}  oranges  }
\subsection*{Explain}
\begin{enumerate}
\item colour \\
Something that is \textbf{orange} is of a colour between red and yellow.
 \textit{
	\begin{itemize}
	\item ...men in bright orange uniforms.
	\end{itemize}
}
\item variable noun \\
An \textbf{orange} is a round juicy fruit with a thick , orange coloured skin.
 \textit{
	\begin{itemize}
	\item An orange a day will give you all the vitamin C you need.
	\item ...orange trees.
	\item ...fresh orange juice.
	\end{itemize}
}
\item uncountable noun \\
\textbf{Orange} is a drink that is made from or tastes of oranges.
 \textit{
	\begin{itemize}
	\item ...cola or orange.
	\end{itemize}
}
\end{enumerate}

\section*{mutual}
{\large \color{blue}  }
\subsection*{Explain}
\begin{enumerate}
\item adjective \\
You use \textbf{mutual} to describe a situation , feeling , or action that is experienced, felt , or done by both of two people mentioned .
 \textit{
	\begin{itemize}
	\item The East and the West can work together for their mutual benefit and progress.
	\item It's plain that he adores his daughter, and the feeling is mutual.
	\end{itemize}
}
\item adjective \\
You use \textbf{mutual} to describe something such as an interest which two or more people share.
 \textit{
	\begin{itemize}
	\item They do, however, share a mutual interest in design.
	\item We were introduced by a mutual friend.
	\end{itemize}
}
\item adjective \\
If a building  society or an insurance company has \textbf{mutual}  status , it is not owned by shareholders but by its customers , who receive a share of the profits.
 \textit{
	\begin{itemize}
	\item Britain's third-largest building society abandoned its mutual status and became a
bank.
	\end{itemize}
}
\end{enumerate}

\section*{panel}
{\large \color{blue}  panels  }
\subsection*{Explain}
\begin{enumerate}
\item countable noun \\
A \textbf{panel} is a small group of people who are chosen to do something, for example to discuss something in public or to make a decision .
 \textit{
	\begin{itemize}
	\item He assembled a panel of scholars to advise him.
	\item All the writers on the panel agreed Quinn's book should be singled out for special
praise.
	\item The advisory panel disagreed with the decision.
	\end{itemize}
}
\item countable noun \\
A \textbf{panel} is a flat rectangular piece of wood or other material that forms part of a larger object such as a door.
 \textit{
	\begin{itemize}
	\item ...the frosted glass panel set in the centre of the door.
	\end{itemize}
}
\item countable noun \\
A control \textbf{panel} or instrument \textbf{panel} is a board or surface which contains switches and controls to operate a machine or piece of equipment.
 \textit{
	\begin{itemize}
	\item The equipment was extremely sophisticated and was monitored from a central control-panel.
	\item They had failed to recognise signs on their instrument panel indicating a serious
problem.
	\end{itemize}
}
\end{enumerate}

\section*{political}
{\large \color{blue}  }
\subsection*{Explain}
\begin{enumerate}
\item adjective \\
\textbf{Political} means relating to the way power is achieved and used in a country or society .
 \textit{
	\begin{itemize}
	\item All other political parties there have been completely banned.
	\item The Canadian government is facing another political crisis.
	\item ...a democratic political system.
	\item Abortion is once again a controversial political and moral issue.
	\end{itemize}
}
\item  \\
 See also  party political \textit{
	\begin{itemize}
	\end{itemize}
}
\item adjective \\
Someone who is \textbf{political} is interested or involved in politics and holds strong  beliefs about it.
 \textit{
	\begin{itemize}
	\item Oh I'm not political, I take no interest in politics.
	\item This play is very political.
	\end{itemize}
}
\end{enumerate}

\section*{patience}
{\large \color{blue}  }
\subsection*{Explain}
\begin{enumerate}
\item uncountable noun \\
If you have \textbf{patience} , you are able to stay  calm and not get  annoyed , for example when something takes a long time, or when someone is not doing what you want them to do.
 \textit{
	\begin{itemize}
	\item He doesn't have the patience to wait.
	\item It was exacting work and required all his patience.
	\end{itemize}
}
\item uncountable noun \\
\textbf{Patience} is a card game for only one player.
 \textit{
	\begin{itemize}
	\item He would often sit and play patience.
	\end{itemize}
}
\item  \\
 to try someone's patience \textit{
	\begin{itemize}
	\end{itemize}
}
\end{enumerate}

\section*{portable}
{\large \color{blue}  portables  }
\subsection*{Explain}
\begin{enumerate}
\item adjective \\
A \textbf{portable}  machine or device is designed to be easily carried or moved.
 \textit{
	\begin{itemize}
	\item There was a little portable television switched on behind the bar.
	\item I always carry a portable computer with me.
	\end{itemize}
}
\item countable noun \\
A \textbf{portable} is something such as a television, radio , or computer which can be easily carried or moved.
 \textit{
	\begin{itemize}
	\item The majority of people listen to music on portables or in cars.
	\end{itemize}
}
\end{enumerate}

\section*{pillar}
{\large \color{blue}  pillars  }
\subsection*{Explain}
\begin{enumerate}
\item countable noun \\
A \textbf{pillar} is a tall solid structure, which is usually used to support part of a building.
 \textit{
	\begin{itemize}
	\item ...the pillars supporting the roof.
	\end{itemize}
}
\item countable noun \\
If something is the \textbf{pillar}  \textbf{of} a system or agreement , it is the most important part of it or what makes it strong and successful .
 \textit{
	\begin{itemize}
	\item The pillar of her economic policy was keeping tight control over money supply.
	\item ...the last pillar of apartheid.
	\end{itemize}
}
\item countable noun \\
If you describe someone as a \textbf{pillar of}  society or as a \textbf{pillar}  \textbf{of} the community , you approve of them because they play an important and active part in society or in the community.
 \textit{
	\begin{itemize}
	\item My father had been a pillar of the community.
	\item ...well-respected pillars of society.
	\end{itemize}
}
\end{enumerate}

\section*{precious}
{\large \color{blue}  }
\subsection*{Explain}
\begin{enumerate}
\item adjective \\
If you say that something such as a resource is \textbf{precious} , you mean that it is valuable and should not be wasted or used badly .
 \textit{
	\begin{itemize}
	\item After four months in foreign parts, every hour at home was precious.
	\item A family break allows you to spend precious time together.
	\item Water is becoming an increasingly precious resource.
	\end{itemize}
}
\item adjective \\
\textbf{Precious} objects and materials are worth a lot of money because they are rare .
 \textit{
	\begin{itemize}
	\item ...jewellery and precious objects belonging to her mother.
	\end{itemize}
}
\item adjective \\
If something is \textbf{precious} to you, you regard it as important and do not want to lose it.
 \textit{
	\begin{itemize}
	\item Her family's support is particularly precious to Josie.
	\item Mary left her most precious possession–a small bookcase–to her niece.
	\end{itemize}
}
\item adjective \\
People sometimes use \textbf{precious} to emphasize their dislike for things which other people think are important.
 \textit{
	\begin{itemize}
	\item You don't care about anything but yourself and your precious face.
	\end{itemize}
}
\item graded adjective \\
If you describe someone as \textbf{precious} , you mean that they behave in a formal and unnatural way.
 \textit{
	\begin{itemize}
	\end{itemize}
}
\item  \\
 precious little/precious few \textit{
	\begin{itemize}
	\end{itemize}
}
\end{enumerate}

\section*{pole}
{\large \color{blue}  poles  }
\subsection*{Explain}
\begin{enumerate}
\item countable noun \\
A \textbf{pole} is a long thin piece of wood or metal, used especially for supporting things.
 \textit{
	\begin{itemize}
	\item The truck crashed into a telegraph pole.
	\item He reached up with a hooked pole to roll down the metal shutter.
	\end{itemize}
}
\item countable noun \\
The earth's \textbf{poles} are the two opposite ends of its axis, its most northern and southern points.
 \textit{
	\begin{itemize}
	\item For six months of the year, there is hardly any light at the poles.
	\end{itemize}
}
\item countable noun \\
The two \textbf{poles} of a range of qualities, opinions, or beliefs are the completely opposite qualities,
opinions, or beliefs at either end of the range.
 \textit{
	\begin{itemize}
	\item The two politicians represent opposite poles of the political spectrum.
	\end{itemize}
}
\item  \\
 poles apart \textit{
	\begin{itemize}
	\end{itemize}
}
\end{enumerate}

\section*{rash}
{\large \color{blue}  rashes  }
\subsection*{Explain}
\begin{enumerate}
\item adjective \\
If someone is \textbf{rash} or does \textbf{rash} things, they act without thinking carefully first, and therefore make mistakes or behave foolishly.
 \textit{
	\begin{itemize}
	\item It would be rash to rely on such evidence.
	\item The Prime Minister is making no rash promises.
	\item Don't do anything rash until the feelings subside.
	\end{itemize}
}
\item countable noun \\
A \textbf{rash} is an area of red  spots that appears on your skin when you are ill or have a bad  reaction to something that you have eaten or touched .
 \textit{
	\begin{itemize}
	\item I noticed a rash on my leg.
	\item He may break out in a rash when he eats these nuts.
	\item The symptoms include skin rashes, fever, and painful joints.
	\end{itemize}
}
\item singular noun \\
If you talk about a \textbf{rash of} events or things, you mean a large number of unpleasant events or undesirable things, which have happened or appeared within a short period of time.
 \textit{
	\begin{itemize}
	\item ...one of the few major airlines left untouched by the industry's rash of takeovers.
	\item Now a rash of scruffy little shops bordered one side of the street.
	\end{itemize}
}
\end{enumerate}

\section*{pump}
{\large \color{blue}  pumps  pumping  pumped  }
\subsection*{Explain}
\begin{enumerate}
\item countable noun \\
A \textbf{pump} is a machine or device that is used to force a liquid or gas to flow in a particular
direction.
 \textit{
	\begin{itemize}
	\item ...pumps that circulate the fuel around in the engine.
	\item There was no water in the building, just a pump in the courtyard.
	\item You'll need a bicycle pump to keep the tyres topped up with air.
	\end{itemize}
}
\item verb \\
To \textbf{pump} a liquid or gas in a particular direction means to force it to flow in that direction
using a pump.
 \textit{
	\begin{itemize}
	\item It's not enough to get rid of raw sewage by pumping it out to sea.
	\item The money raised will be used to dig bore holes to pump water into the dried-up lake.
	\item ...drill rigs that are busy pumping natural gas.
	\item Age diminishes the heart's ability to pump harder and faster under exertion.
	\end{itemize}
}
\item countable noun \\
A petrol or gas \textbf{pump} is a machine with a tube attached to it that you use to fill a car with petrol.
 \textit{
	\begin{itemize}
	\item There are already long queues of vehicles at petrol pumps.
	\item ...gas pumps.
	\end{itemize}
}
\item verb \\
If someone \textbf{has} their stomach  \textbf{pumped} , doctors remove the contents of their stomach, for example because they have swallowed  poison or drugs.
 \textit{
	\begin{itemize}
	\item She was released from hospital yesterday after having her stomach pumped.
	\end{itemize}
}
\item verb \\
If you \textbf{pump} money or other resources  \textbf{into} something such as a project or an industry , you invest a lot of money or resources in it.
 \textit{
	\begin{itemize}
	\item The Government needs to pump more money into community care.
	\end{itemize}
}
\item verb \\
If you \textbf{pump} someone \textbf{about} something, you keep asking them questions in order to get information.
 \textit{
	\begin{itemize}
	\item He ran in every five minutes to pump me about the case.
	\item He must have pumped Janey for details.
	\item Stop trying to pump information out of me.
	\end{itemize}
}
\item verb \\
To \textbf{pump} bullets \textbf{into} someone means to fire a lot of bullets into them very quickly.
 \textit{
	\begin{itemize}
	\end{itemize}
}
\item countable noun \\
\textbf{Pumps} are canvas shoes with flat rubber soles which people wear for sports and leisure .
 \textit{
	\begin{itemize}
	\end{itemize}
}
\item countable noun \\
\textbf{Pumps} are women's shoes that do not cover the top part of the foot and are usually made
of plain  leather .
 \textit{
	\begin{itemize}
	\end{itemize}
}
\item  \\
 to prime the pump \textit{
	\begin{itemize}
	\end{itemize}
}
\end{enumerate}

\section*{reciprocal}
{\large \color{blue}  }
\subsection*{Explain}
\begin{enumerate}
\item adjective \\
A \textbf{reciprocal} action or agreement involves two people or groups who do the same thing to each other or agree to help each another in a similar way.
 \textit{
	\begin{itemize}
	\item They expected a reciprocal gesture before more hostages could be freed.
	\item Many countries have reciprocal agreements for health care.
	\end{itemize}
}
\end{enumerate}

\section*{rail}
{\large \color{blue}  rails  railing  railed  }
\subsection*{Explain}
\begin{enumerate}
\item countable noun \\
A \textbf{rail} is a horizontal bar attached to posts or fixed round the edge of something as a fence
or support.
 \textit{
	\begin{itemize}
	\item They had to walk across an emergency footbridge, holding onto a rope that served
as a rail.
	\item She gripped the thin rail in the lift.
	\end{itemize}
}
\item countable noun \\
A \textbf{rail} is a horizontal bar that you hang things on.
 \textit{
	\begin{itemize}
	\item ...frocks hanging from a rail.
	\item This pair of curtains will fit a rail up to 7ft 6in wide.
	\end{itemize}
}
\item countable noun \\
\textbf{Rails} are the steel bars which trains run on.
 \textit{
	\begin{itemize}
	\item The train left the rails but somehow forced its way back onto the line.
	\end{itemize}
}
\item uncountable noun \\
If you travel or send something \textbf{by}  \textbf{rail} , you travel or send it on a train.
 \textit{
	\begin{itemize}
	\item The president traveled by rail to his home town.
	\item ...the electric rail link between Manchester and Sheffield.
	\end{itemize}
}
\item verb \\
If you \textbf{rail} against something, you criticize it loudly and angrily.
 \textit{
	\begin{itemize}
	\item He railed against hypocrisy and greed.
	\item I'd cursed him and railed at him.
	\end{itemize}
}
\item  \\
 back on the rails \textit{
	\begin{itemize}
	\end{itemize}
}
\item  \\
 go off the rails \textit{
	\begin{itemize}
	\end{itemize}
}
\end{enumerate}

\section*{relative}
{\large \color{blue}  relatives  }
\subsection*{Explain}
\begin{enumerate}
\item countable noun \\
Your \textbf{relatives} are the members of your family.
 \textit{
	\begin{itemize}
	\item Do relatives of yours still live in Siberia?
	\item Get a relative to look after the children.
	\end{itemize}
}
\item adjective \\
You use \textbf{relative} to say that something is true to a certain degree , especially when compared with other things of the same kind .
 \textit{
	\begin{itemize}
	\item The fighting resumed after a period of relative calm.
	\item It is a cancer that can be cured with relative ease.
	\item Pedestrian zones mean that children can play in relative safety.
	\end{itemize}
}
\item adjective \\
You use \textbf{relative} when you are comparing the quality or size of two things.
 \textit{
	\begin{itemize}
	\item They chatted about the relative merits of London and Paris as places to live.
	\item I reflected on the relative importance of education in 50 countries.
	\item ...the relative strength of the central and state governments.
	\end{itemize}
}
\item  \\
 relative to sth \textit{
	\begin{itemize}
	\end{itemize}
}
\item adjective \\
If you say that something is \textbf{relative} , you mean that it needs to be considered and judged in relation to other things.
 \textit{
	\begin{itemize}
	\item Fitness is relative; one must always ask 'Fit for what?'.
	\item Truth is relative.
	\end{itemize}
}
\item countable noun \\
If one animal, plant, language, or invention is a \textbf{relative}  \textbf{of} another, they have both developed from the same type of animal, plant, language,
or invention.
 \textit{
	\begin{itemize}
	\item The pheasant is a close relative of the Guinea hen.
	\end{itemize}
}
\end{enumerate}

\section*{relativity}
{\large \color{blue}  }
\subsection*{Explain}
\begin{enumerate}
\item uncountable noun \\
The theory of \textbf{relativity} is Einstein's theory concerning  space , time, and motion.
 \textit{
	\begin{itemize}
	\end{itemize}
}
\end{enumerate}

\section*{resemblance}
{\large \color{blue}  resemblances  }
\subsection*{Explain}
\begin{enumerate}
\item variable noun \\
If there is a \textbf{resemblance} between two people or things, they are similar to each other.
 \textit{
	\begin{itemize}
	\item There was a remarkable resemblance between him and Pete.
	\item Our tour prices bore little resemblance to those in the holiday brochures.
	\end{itemize}
}
\end{enumerate}

\section*{ridiculous}
{\large \color{blue}  }
\subsection*{Explain}
\begin{enumerate}
\item adjective \\
If you say that something or someone is \textbf{ridiculous} , you mean that they are very foolish .
 \textit{
	\begin{itemize}
	\item It is ridiculous to suggest we are having a romance.
	\item It was an absolutely ridiculous decision.
	\end{itemize}
}
\end{enumerate}

\section*{reverse}
{\large \color{blue}  reverses  reversing  reversed  }
\subsection*{Explain}
\begin{enumerate}
\item verb \\
When someone or something \textbf{reverses} a decision , policy , or trend , they change it to the opposite decision, policy, or trend.
 \textit{
	\begin{itemize}
	\item They have made it clear they will not reverse the decision to increase prices.
	\item The rise, the first in 10 months, reversed the downward trend in the jobless rate.
	\end{itemize}
}
\item verb \\
If you \textbf{reverse} the order of a set of things, you arrange them in the opposite order, so that the first thing comes  last .
 \textit{
	\begin{itemize}
	\item Simply reversing the order of the questions made it more logical. .
	\item The normal word order is reversed in passive sentences.
	\end{itemize}
}
\item verb \\
If you \textbf{reverse} the positions or functions of two things, you change them so that each thing has the position or function that
the other one had.
 \textit{
	\begin{itemize}
	\item He reversed the position of the two stamps.
	\end{itemize}
}
\item verb \\
When a car  \textbf{reverses} or when you \textbf{reverse} it, the car is driven backwards.
 \textit{
	\begin{itemize}
	\item Another car reversed out of the drive.
	\item He reversed and drove away.
	\item He reversed his car straight at the policeman.
	\end{itemize}
}
\item uncountable noun \\
If your car is \textbf{in}  \textbf{reverse} , you have changed gear so that you can drive it backwards.
 \textit{
	\begin{itemize}
	\item He lurched the car in reverse along the ruts to the access road.
	\end{itemize}
}
\item adjective \\
\textbf{Reverse}  means opposite to what you expect or to what has just been described .
 \textit{
	\begin{itemize}
	\item The wrong attitude will have exactly the reverse effect.
	\end{itemize}
}
\item singular noun \\
If you say that one thing is \textbf{the reverse} of another, you are emphasizing that the first thing is the complete opposite of the second thing.
 \textit{
	\begin{itemize}
	\item There is no evidence that spectators want longer cricket matches. Quite the reverse.
	\item I expected a dense and detailed autobiography. The reverse is true. The book is short
and spare.
	\end{itemize}
}
\item countable noun \\
A \textbf{reverse} is a serious  failure or defeat.
 \textit{
	\begin{itemize}
	\item It's clear that the party of the former Prime Minister has suffered a major reverse.
	\end{itemize}
}
\item singular noun \\
\textbf{The}  \textbf{reverse} or \textbf{the}  \textbf{reverse}  \textbf{side} of a flat  object which has two sides is the less important or the other side.
 \textit{
	\begin{itemize}
	\item Write your address on the reverse of the cheque.
	\end{itemize}
}
\item  \\
 in reverse \textit{
	\begin{itemize}
	\end{itemize}
}
\item  \\
 reverse the charges \textit{
	\begin{itemize}
	\end{itemize}
}
\end{enumerate}

\section*{ruthless}
{\large \color{blue}  }
\subsection*{Explain}
\begin{enumerate}
\item adjective \\
If you say that someone is \textbf{ruthless} , you mean that you disapprove of them because they are very harsh or cruel , and will do anything that is necessary to achieve what they want .
 \textit{
	\begin{itemize}
	\item The President was ruthless in dealing with any hint of internal political dissent.
	\item ...an invasion by a ruthless totalitarian power.
	\item The late newspaper tycoon is condemned for his ruthless treatment of employees.
	\end{itemize}
}
\item adjective \\
A \textbf{ruthless} action or activity is done forcefully and thoroughly, without much concern for its effects on other people.
 \textit{
	\begin{itemize}
	\item Her lawyers have been ruthless in thrashing out a settlement.
	\item Successfully merging two banks requires a fast and ruthless attack on costs.
	\end{itemize}
}
\end{enumerate}

\section*{shaft}
{\large \color{blue}  shafts  }
\subsection*{Explain}
\begin{enumerate}
\item countable noun \\
A \textbf{shaft} is a long vertical passage , for example for a lift.
 \textit{
	\begin{itemize}
	\item He was found dead at the bottom of a lift shaft.
	\item ...old mine shafts.
	\end{itemize}
}
\item countable noun \\
In a machine, a \textbf{shaft} is a rod that turns round continually in order to transfer movement in the machine.
 \textit{
	\begin{itemize}
	\item ...a drive shaft.
	\item ...the propeller shaft.
	\end{itemize}
}
\item countable noun \\
A \textbf{shaft} is a long thin piece of wood or metal that forms part of a spear, axe, golf club,
or other object.
 \textit{
	\begin{itemize}
	\item ...golf clubs with steel shafts.
	\end{itemize}
}
\item countable noun \\
A \textbf{shaft}  \textbf{of} light is a beam of light, for example sunlight  shining through an opening .
 \textit{
	\begin{itemize}
	\item A brilliant shaft of sunlight burst through the doorway.
	\end{itemize}
}
\end{enumerate}

\section*{same}
{\large \color{blue}  }
\subsection*{Explain}
\begin{enumerate}
\item adjective \\
If two or more things, actions, or qualities are \textbf{the same} , or if one is \textbf{the same}  \textbf{as} another, they are very like each other in some way.
 \textit{
	\begin{itemize}
	\item The houses were all the same–square, close to the street, needing paint.
	\item In essence, all computers are the same.
	\item People with the same experience in the job should be paid the same.
	\item Driving a boat is not the same as driving a car.
	\item I want my son to wear the same clothes as everyone else at the school.
	\item Bihar had a population roughly the same as that of England.
	\end{itemize}
}
\item  \\
 the same as \textit{
	\begin{itemize}
	\end{itemize}
}
\item adjective \\
You use \textbf{same} to indicate that you are referring to only one place, time, or thing, and not to
different ones.
 \textit{
	\begin{itemize}
	\item Bernard works at the same institution as Arlette.
	\item It's impossible to get everybody together at the same time.
	\item Members of his staff learn to work the same 13-hour days that he imposes on himself.
	\item John just told me that your birthday is on the same day as mine.
	\item ...business people who spoke the same language as himself.
	\item Gary plays football with the other children of the same age.
	\end{itemize}
}
\item adjective \\
Something that is still  \textbf{the same} has not changed in any way.
 \textit{
	\begin{itemize}
	\item It has been rare for the environment to stay the same for very long.
	\item Only 17% said the economy would improve, but 25% believed it would stay the same.
	\end{itemize}
}
\item pronoun \\
You use \textbf{the same} to refer to something that has previously been mentioned or suggested .
 \textbf{Same} is also an adjective .
 \textit{
	\begin{itemize}
	\item We made the decision which was right for us. Other parents must do the same.
	\item In the United States small specialised bookshops survive quite well. The same applies
to small publishers.
	\item We like him very much and he says the same about us.
	\item Tom Wood is a player I admire because he is so honest, and Chris Robshaw has that
same quality.
	\end{itemize}
}
\item pronoun \\
You use \textbf{same} to refer to something that has already been mentioned in a document such as a business letter or bill .
 \textit{
	\begin{itemize}
	\item Wrist watches: £5. Inscription of same: £25.
	\end{itemize}
}
\item convention \\
You say ' \textbf{same here} ' in order to suggest that you feel the same way about something as the person who has just spoken to you, or that you have done the same thing.
 \textit{
	\begin{itemize}
	\item 'Nice to meet you,' said Michael. 'Same here,' said Mary Ann.
	\item 'I hate going into stores.'—'Same here,' said William.
	\end{itemize}
}
\item convention \\
You say ' \textbf{same to you} ' in response to someone who wishes you well with something.
 \textit{
	\begin{itemize}
	\item 'Have a nice Easter.'—'And the same to you Bridie.'
	\item 'Goodbye, then, and thanks. Good luck.'—'The same to you.'
	\end{itemize}
}
\item  \\
 same again \textit{
	\begin{itemize}
	\end{itemize}
}
\item  \\
 all the same/just the same \textit{
	\begin{itemize}
	\end{itemize}
}
\item  \\
 all the same to me \textit{
	\begin{itemize}
	\end{itemize}
}
\item  \\
 one and the same \textit{
	\begin{itemize}
	\end{itemize}
}
\item  \\
 the same/the very same \textit{
	\begin{itemize}
	\end{itemize}
}
\end{enumerate}

\section*{tar}
{\large \color{blue}  tars  tarring  tarred  }
\subsection*{Explain}
\begin{enumerate}
\item uncountable noun \\
\textbf{Tar} is a thick black sticky substance that is used especially for making roads .
 \textit{
	\begin{itemize}
	\item The oil has hardened to tar.
	\item They drove across the river to New Hampshire on a hot tar road.
	\end{itemize}
}
\item uncountable noun \\
\textbf{Tar} is one of the poisonous substances contained in tobacco .
 \textit{
	\begin{itemize}
	\end{itemize}
}
\item  \\
 tarred with the same brush \textit{
	\begin{itemize}
	\end{itemize}
}
\end{enumerate}

\section*{sincere}
{\large \color{blue}  }
\subsection*{Explain}
\begin{enumerate}
\item adjective \\
If you say that someone is \textbf{sincere} , you approve of them because they really mean the things they say. You can also  describe someone's behaviour and beliefs as \textbf{sincere} .
 \textit{
	\begin{itemize}
	\item He's sincere in his views.
	\item He accepted her apologies as sincere.
	\item There was a sincere expression of friendliness on both their faces.
	\end{itemize}
}
\end{enumerate}

\section*{tree}
{\large \color{blue}  trees  }
\subsection*{Explain}
\begin{enumerate}
\item countable noun \\
A \textbf{tree} is a tall plant that has a hard trunk, branches, and leaves.
 \textit{
	\begin{itemize}
	\item I planted those apple trees.
	\item ...a variety of shrubs and trees.
	\end{itemize}
}
\item  \\
 to be barking up the wrong tree \textit{
	\begin{itemize}
	\end{itemize}
}
\item  \\
 can't see the wood for the trees \textit{
	\begin{itemize}
	\end{itemize}
}
\end{enumerate}

\section*{southern}
{\large \color{blue}  }
\subsection*{Explain}
\begin{enumerate}
\item adjective \\
\textbf{Southern} means in or from the south of a region, state, or country.
 \textit{
	\begin{itemize}
	\item The Everglades National Park stretches across the southern tip of Florida.
	\item ...a place where you can sample southern cuisine.
	\end{itemize}
}
\end{enumerate}

\section*{veil}
{\large \color{blue}  veils  }
\subsection*{Explain}
\begin{enumerate}
\item countable noun \\
A \textbf{veil} is a piece of thin soft  cloth that women sometimes wear over their heads and which can also cover their face.
 \textit{
	\begin{itemize}
	\item She's got long fair hair, but she's got a veil over it.
	\item She swathes her face in a veil of decorative muslin.
	\end{itemize}
}
\item countable noun \\
You can refer to something that hides or partly hides a situation or activity as a \textbf{veil} .
 \textit{
	\begin{itemize}
	\item The country is ridding itself of its disgraced prime minister in a veil of secrecy.
	\item The chilling facts behind this veil of silence were slow to emerge.
	\end{itemize}
}
\item countable noun \\
You can refer to something that you can partly see through, for example a mist , as a \textbf{veil} .
 \textit{
	\begin{itemize}
	\item The eruption has left a thin veil of dust in the upper atmosphere.
	\item He recognized the coast of England through a veil of mist.
	\item Bright moonlight shines through a thin veil of clouds.
	\end{itemize}
}
\item  \\
 to draw a veil over something \textit{
	\begin{itemize}
	\end{itemize}
}
\end{enumerate}

\section*{substantial}
{\large \color{blue}  }
\subsection*{Explain}
\begin{enumerate}
\item adjective \\
\textbf{Substantial} means large in amount or degree .
 \textit{
	\begin{itemize}
	\item The party has just lost office and with it a substantial number of seats.
	\item That is a very substantial improvement in the present situation.
	\end{itemize}
}
\item graded adjective \\
A \textbf{substantial} building is large and strongly built .
 \textit{
	\begin{itemize}
	\item ...those fortunate enough to have a fairly substantial property to sell.
	\end{itemize}
}
\end{enumerate}

\section*{vest}
{\large \color{blue}  vests  vesting  vested  }
\subsection*{Explain}
\begin{enumerate}
\item countable noun \\
A \textbf{vest} is a piece of underwear which you can wear on the top  half of your body in order to keep  warm .
 \textit{
	\begin{itemize}
	\end{itemize}
}
\item countable noun \\
A \textbf{vest} is a sleeveless piece of clothing with buttons which people usually wear over a shirt .
 \textit{
	\begin{itemize}
	\end{itemize}
}
\item verb \\
If something \textbf{is vested}  \textbf{in} you, or if you \textbf{are vested}  \textbf{with} it, it is given to you as a right or responsibility .
 \textit{
	\begin{itemize}
	\item All authority was vested in the woman, who discharged every kind of public duty.
	\item The mass media have been vested with significant power in modern societies.
	\item There's an extraordinary amount of power vested in us.
	\end{itemize}
}
\end{enumerate}

\section*{tentative}
{\large \color{blue}  }
\subsection*{Explain}
\begin{enumerate}
\item adjective \\
\textbf{Tentative}  agreements , plans , or arrangements are not definite or certain, but have been made as a first step .
 \textit{
	\begin{itemize}
	\item Political leaders have reached a tentative agreement.
	\item Such theories are still very tentative.
	\end{itemize}
}
\item adjective \\
If someone is \textbf{tentative} , they are cautious and not very confident because they are uncertain or afraid .
 \textit{
	\begin{itemize}
	\item My first attempts at complaining were rather tentative.
	\item She did not return his tentative smile.
	\end{itemize}
}
\end{enumerate}

\section*{war}
{\large \color{blue}  wars  }
\subsection*{Explain}
\begin{enumerate}
\item variable noun \\
A \textbf{war} is a period of fighting or conflict between countries or states.
 \textit{
	\begin{itemize}
	\item He spent part of the war in the National Guard.
	\item ...matters of war and peace.
	\item They've been at war for the last fifteen years.
	\end{itemize}
}
\item variable noun \\
\textbf{War} is intense  economic  competition between countries or organizations.
 \textit{
	\begin{itemize}
	\item The most important thing is to reach an agreement and to avoid a trade war.
	\end{itemize}
}
\item variable noun \\
If you make \textbf{war}  \textbf{on} someone or something that you are opposed to, you do things to stop them succeeding .
 \textit{
	\begin{itemize}
	\item She has been involved in the war against organised crime.
	\item ...if the United States is to be successful in its war on corruption.
	\end{itemize}
}
\item  \\
 be in the wars \textit{
	\begin{itemize}
	\end{itemize}
}
\item  \\
 go to war \textit{
	\begin{itemize}
	\end{itemize}
}
\item  \\
 war of words \textit{
	\begin{itemize}
	\end{itemize}
}
\end{enumerate}

\section*{thick}
{\large \color{blue}  thicker  thickest  }
\subsection*{Explain}
\begin{enumerate}
\item adjective \\
Something that is \textbf{thick} has a large distance between its two opposite sides.
 \textit{
	\begin{itemize}
	\item For breakfast I had a thick slice of bread and syrup.
	\item He wore glasses with thick rims.
	\item This material is very thick and this needle is not strong enough to go through it.
	\end{itemize}
}
\item adjective \\
You can use \textbf{thick} to talk or ask about how wide or deep something is.
 \textbf{Thick} is also a combining form.
 \textit{
	\begin{itemize}
	\item The folder was two inches thick.
	\item How thick are these walls?
	\item ...a finger as thick as a sausage.
	\item His life was saved by a quarter-inch-thick bullet-proof steel screen.
	\end{itemize}
}
\item adjective \\
If something that consists of several things is \textbf{thick} , it has a large number of them very close together.
 \textit{
	\begin{itemize}
	\item She inherited our father's thick, wavy hair.
	\item They walked through thick forest.
	\end{itemize}
}
\item adjective \\
If something is \textbf{thick with} another thing, the first thing is full of or covered with the second.
 \textit{
	\begin{itemize}
	\item The air is thick with acrid smoke from the fires.
	\item She ate scones thick with butter.
	\end{itemize}
}
\item adjective \\
\textbf{Thick} clothes are made from heavy cloth , so that they will keep you warm in cold  weather .
 \textit{
	\begin{itemize}
	\item In the winter she wears thick socks, Wellington boots and gloves.
	\item She wore a thick tartan skirt and a red cashmere sweater.
	\end{itemize}
}
\item adjective \\
\textbf{Thick}  smoke , fog , or cloud is difficult to see through.
 \textit{
	\begin{itemize}
	\item The smoke was bluish-black and thick.
	\item It wasn't very thick fog.
	\end{itemize}
}
\item adjective \\
\textbf{Thick} liquids are fairly  stiff and solid and do not flow easily .
 \textit{
	\begin{itemize}
	\item They had to battle through thick mud to reach construction workers.
	\item The sauce is thick and rich so don't bother trying to diet.
	\end{itemize}
}
\item adjective \\
If someone's voice is \textbf{thick} , they are not speaking clearly , for example because they are ill , upset , or drunk.
 \textit{
	\begin{itemize}
	\item When he spoke his voice was thick with bitterness.
	\end{itemize}
}
\item adjective \\
A \textbf{thick} accent is very obvious and easy to identify .
 \textit{
	\begin{itemize}
	\item He answered our questions in English but with a thick accent.
	\item 'What do you want?' a teenage girl demanded in a thick German accent.
	\end{itemize}
}
\item adjective \\
If you describe someone as \textbf{thick} , you think they are stupid .
 \textit{
	\begin{itemize}
	\item How could she have been so thick?
	\end{itemize}
}
\item  \\
 thick and fast \textit{
	\begin{itemize}
	\end{itemize}
}
\item  \\
 in the thick of \textit{
	\begin{itemize}
	\end{itemize}
}
\item  \\
 through thick and thin \textit{
	\begin{itemize}
	\end{itemize}
}
\end{enumerate}

\section*{warfare}
{\large \color{blue}  }
\subsection*{Explain}
\begin{enumerate}
\item uncountable noun \\
\textbf{Warfare} is the activity of fighting a war.
 \textit{
	\begin{itemize}
	\item ...the threat of chemical warfare.
	\end{itemize}
}
\item uncountable noun \\
\textbf{Warfare} is sometimes used to refer to any violent struggle or conflict.
 \textit{
	\begin{itemize}
	\item Much of the violence is related to gang warfare.
	\item At times party rivalries have broken out into open warfare.
	\end{itemize}
}
\end{enumerate}

\section*{urban}
{\large \color{blue}  }
\subsection*{Explain}
\begin{enumerate}
\item adjective \\
\textbf{Urban} means belonging to, or relating to, a town or city.
 \textit{
	\begin{itemize}
	\item Most of the population is an urban population.
	\item Most urban areas are close to a park.
	\item ...urban planning.
	\end{itemize}
}
\end{enumerate}

\section*{whisky}
{\large \color{blue}  whiskies  }
\subsection*{Explain}
\begin{enumerate}
\item variable noun \\
\textbf{Whisky} is a strong  alcoholic drink made, especially in Scotland , from grain such as barley or rye .
 A \textbf{whisky} is a glass of whisky.
 \textit{
	\begin{itemize}
	\item ...a bottle of whisky.
	\item ...expensive whiskies and brandies.
	\item She handed him a whisky.
	\end{itemize}
}
\end{enumerate}

\section*{account}
{\large \color{blue}  accounts  accounting  accounted  }
\subsection*{Explain}
\begin{enumerate}
\item countable noun \\
If you have an \textbf{account} with a bank or a similar organization, you have an arrangement to leave your money
there and take some out when you need it.
 \textit{
	\begin{itemize}
	\item Some banks make it difficult to open an account.
	\item I had two accounts with the bank, a savings account and a current account.
	\end{itemize}
}
\item countable noun \\
In business, a regular customer of a company can be referred to as an \textbf{account} , especially when the customer is another company.
 \textit{
	\begin{itemize}
	\item The Glasgow-based marketing agency has won two Edinburgh accounts.
	\end{itemize}
}
\item countable noun \\
\textbf{Accounts} are detailed records of all the money that a person or business receives and spends .
 \textit{
	\begin{itemize}
	\item He kept detailed accounts.
	\item ...an account book.
	\end{itemize}
}
\item countable noun \\
An \textbf{account} is a written or spoken report of something that has happened .
 \textit{
	\begin{itemize}
	\item He gave a detailed account of what happened on the fateful night.
	\item And that, according to some accounts I have read, is why he adopted the name.
	\end{itemize}
}
\item countable noun \\
An \textbf{account}  \textbf{of} something is a theory which is intended to explain or describe it.
 \textit{
	\begin{itemize}
	\item This basic utilitarian model gives a relatively unsophisticated account of human
behaviour.
	\item Science, on Weber's account, is an essentially value-free activity.
	\end{itemize}
}
\item verb \\
If you say that something \textbf{is accounted} a particular thing, you are reporting someone's judgment or opinion that it is that thing.
 \textit{
	\begin{itemize}
	\item The opening day of the battle was, nevertheless, accounted a success.
	\end{itemize}
}
\item  \\
 by/from all accounts \textit{
	\begin{itemize}
	\end{itemize}
}
\item  \\
 give a good account of oneself \textit{
	\begin{itemize}
	\end{itemize}
}
\item  \\
 of no/little account \textit{
	\begin{itemize}
	\end{itemize}
}
\item  \\
 on account \textit{
	\begin{itemize}
	\end{itemize}
}
\item  \\
 on account of \textit{
	\begin{itemize}
	\end{itemize}
}
\item  \\
 on someone's account \textit{
	\begin{itemize}
	\end{itemize}
}
\item  \\
 on someone's account \textit{
	\begin{itemize}
	\end{itemize}
}
\item  \\
 on no account \textit{
	\begin{itemize}
	\end{itemize}
}
\item  \\
 on that/this account \textit{
	\begin{itemize}
	\end{itemize}
}
\item  \\
 by their own account \textit{
	\begin{itemize}
	\end{itemize}
}
\item  \\
 on one's own account \textit{
	\begin{itemize}
	\end{itemize}
}
\item  \\
 on one's own account \textit{
	\begin{itemize}
	\end{itemize}
}
\item  \\
 settle (one's) accounts \textit{
	\begin{itemize}
	\end{itemize}
}
\item  \\
 take into account/take account of \textit{
	\begin{itemize}
	\end{itemize}
}
\item  \\
 be called/held/brought to account \textit{
	\begin{itemize}
	\end{itemize}
}
\end{enumerate}

\section*{adventure}
{\large \color{blue}  adventures  adventuring  adventured  }
\subsection*{Explain}
\begin{enumerate}
\item countable noun \\
If someone has an \textbf{adventure} , they become involved in an unusual , exciting, and rather dangerous journey or series of events.
 \textit{
	\begin{itemize}
	\item I set off for a new adventure in the United States.
	\end{itemize}
}
\item uncountable noun \\
\textbf{Adventure} is excitement and willingness to do new, unusual, or rather dangerous things.
 \textit{
	\begin{itemize}
	\item Their cultural backgrounds gave them a spirit of adventure.
	\item ...a feeling of adventure and excitement.
	\end{itemize}
}
\item verb \\
If you \textbf{adventure}  somewhere , you go somewhere new, unusual, and exciting.
 \textit{
	\begin{itemize}
	\item The group has adventured as far as the Austrian alps.
	\end{itemize}
}
\end{enumerate}

\section*{appropriate}
{\large \color{blue}  appropriates  appropriating  appropriated  }
\subsection*{Explain}
\begin{enumerate}
\item adjective \\
Something that is \textbf{appropriate} is suitable or acceptable for a particular situation .
 \textit{
	\begin{itemize}
	\item It is appropriate that Irish names dominate the list.
	\item Dress neatly and attractively in an outfit appropriate to the job.
	\item The teacher can then take appropriate action.
	\end{itemize}
}
\item verb \\
If someone \textbf{appropriates} something which does not belong to them, they take it, usually without the right to do so.
 \textit{
	\begin{itemize}
	\item Several other newspapers have appropriated the idea.
	\item The land was simply appropriated by the rulers.
	\end{itemize}
}
\item verb \\
If a government or organization  \textbf{appropriates} an amount of money for a particular purpose, it reserves it for that purpose.
 \textit{
	\begin{itemize}
	\item The legislature authorized the raising of an army and appropriated money to supply
it with weapons.
	\end{itemize}
}
\end{enumerate}

\section*{bell}
{\large \color{blue}  bells  }
\subsection*{Explain}
\begin{enumerate}
\item countable noun \\
A \textbf{bell} is a device that makes a ringing sound and is used to give a signal or to attract people's attention .
 \textit{
	\begin{itemize}
	\item I had just enough time to finish eating before the bell rang and I was off to my
first class.
	\item I've been ringing the door bell, there's no answer.
	\end{itemize}
}
\item countable noun \\
A \textbf{bell} is a hollow metal object shaped like a cup which has a piece hanging inside it that hits the sides and makes a sound.
 \textit{
	\begin{itemize}
	\item Church bells tolled yesterday in remembrance of the five girls who were killed.
	\end{itemize}
}
\item  \\
 as clear as a bell \textit{
	\begin{itemize}
	\end{itemize}
}
\item  \\
 give sb a bell \textit{
	\begin{itemize}
	\end{itemize}
}
\item  \\
 sth rings a bell \textit{
	\begin{itemize}
	\end{itemize}
}
\item  \\
 as sound as a bell \textit{
	\begin{itemize}
	\end{itemize}
}
\end{enumerate}

\section*{cheap}
{\large \color{blue}  cheaper  cheapest  }
\subsection*{Explain}
\begin{enumerate}
\item adjective \\
Goods or services that are \textbf{cheap} cost less money than usual or than you expected .
 \textit{
	\begin{itemize}
	\item I'm going to live off campus if I can find somewhere cheap enough.
	\item Smoke detectors are cheap and easy to put up.
	\item Running costs are coming down because of cheaper fuel.
	\item They served breakfast all day and sold it cheap.
	\end{itemize}
}
\item adjective \\
If you describe goods as \textbf{cheap} , you mean they cost less money than similar products but their quality is poor.
 \textit{
	\begin{itemize}
	\item Don't resort to cheap copies; save up for the real thing.
	\item ...a tight suit made of some cheap material.
	\end{itemize}
}
\item adjective \\
If you describe the cost of someone's work as \textbf{cheap} , you disapprove of the way people are taking  advantage of a situation to pay someone less than they should for the work that they do.
 \textit{
	\begin{itemize}
	\item ...unscrupulous employers who treat children as a cheap source of labour.
	\end{itemize}
}
\item adjective \\
If you describe someone's remarks or actions as \textbf{cheap} , you mean that they are unkindly or insincerely using a situation to benefit themselves or to harm someone else.
 \textit{
	\begin{itemize}
	\item These tests will inevitably be used by politicians to make cheap political points.
	\end{itemize}
}
\item adjective \\
If you describe someone as \textbf{cheap} , you are criticizing them for being unwilling to spend money.
 \textit{
	\begin{itemize}
	\item Oh, please, Dad, just this once don't be cheap.
	\end{itemize}
}
\item  \\
 life is cheap \textit{
	\begin{itemize}
	\end{itemize}
}
\item  \\
 on the cheap \textit{
	\begin{itemize}
	\end{itemize}
}
\end{enumerate}

\section*{bone}
{\large \color{blue}  bones  boning  boned  }
\subsection*{Explain}
\begin{enumerate}
\item variable noun \\
Your \textbf{bones} are the hard parts inside your body which together form your skeleton.
 \textit{
	\begin{itemize}
	\item Many passengers suffered broken bones.
	\item Stephen fractured a thigh bone.
	\item The body is made up primarily of bone, muscle, and fat.
	\item She scooped the chicken bones back into the stewpot.
	\end{itemize}
}
\item verb \\
If you \textbf{bone} a piece of meat or fish , you remove the bones from it before cooking it.
 \textit{
	\begin{itemize}
	\item Make sure that you do not pierce the skin when boning the chicken thighs.
	\item The boned fish is so easy to serve.
	\end{itemize}
}
\item adjective \\
A \textbf{bone}  tool or ornament is made of bone.
 \textit{
	\begin{itemize}
	\item ...a small, expensive pocketknife with a bone handle.
	\end{itemize}
}
\item  \\
 bare bones \textit{
	\begin{itemize}
	\end{itemize}
}
\item  \\
 close to the bone \textit{
	\begin{itemize}
	\end{itemize}
}
\item  \\
 to feel something in your bones \textit{
	\begin{itemize}
	\end{itemize}
}
\item  \\
 make no bones \textit{
	\begin{itemize}
	\end{itemize}
}
\item  \\
 make no bones \textit{
	\begin{itemize}
	\end{itemize}
}
\item  \\
 skin and bone \textit{
	\begin{itemize}
	\end{itemize}
}
\item  \\
 to cut something to the bone \textit{
	\begin{itemize}
	\end{itemize}
}
\item  \\
 to the bone \textit{
	\begin{itemize}
	\end{itemize}
}
\end{enumerate}

\section*{clockwise}
{\large \color{blue}  }
\subsection*{Explain}
\begin{enumerate}
\item adverb \\
When something is moving \textbf{clockwise} , it is moving in a circle in the same direction as the hands on a clock.
 \textbf{Clockwise} is also an adjective .
 \textit{
	\begin{itemize}
	\item He told the children to start moving clockwise around the room.
	\item Gently swing your right arm in a clockwise direction.
	\end{itemize}
}
\end{enumerate}

\section*{calcium}
{\large \color{blue}  }
\subsection*{Explain}
\begin{enumerate}
\item uncountable noun \\
\textbf{Calcium} is a soft white element which is found in bones and teeth, and also in limestone , chalk , and marble .
 \textit{
	\begin{itemize}
	\end{itemize}
}
\end{enumerate}

\section*{complex}
{\large \color{blue}  complexes  }
\subsection*{Explain}
\begin{enumerate}
\item adjective \\
Something that is \textbf{complex} has many different parts, and is therefore often difficult to understand .
 \textit{
	\begin{itemize}
	\item ...in-depth coverage of today's complex issues.
	\item ...a complex system of voting.
	\item ...her complex personality.
	\item ...complex machines.
	\end{itemize}
}
\item adjective \\
In grammar , a \textbf{complex} sentence contains one or more subordinate clauses as well as a main clause. Compare  compound , , simple .
 \textit{
	\begin{itemize}
	\end{itemize}
}
\item countable noun \\
A \textbf{complex} is a group of buildings designed for a particular purpose, or one large building
divided into several smaller areas.
 \textit{
	\begin{itemize}
	\item ...plans for constructing a new stadium and leisure complex.
	\item ...a complex of offices and flats.
	\end{itemize}
}
\item countable noun \\
A \textbf{complex}  \textbf{of} things is a group or system of things that are connected with each other in a complicated way.
 \textit{
	\begin{itemize}
	\item ...the complex of clans which occupied the land.
	\item ...the military-industrial complex.
	\end{itemize}
}
\item countable noun \\
If someone has a \textbf{complex} about something, they have a mental or emotional problem relating to it, often because of an unpleasant  experience in the past .
 \textit{
	\begin{itemize}
	\item I have never had a complex about my height.
	\item ...a deranged attacker, driven by a persecution complex.
	\end{itemize}
}
\end{enumerate}

\section*{campaign}
{\large \color{blue}  campaigns  campaigning  campaigned  }
\subsection*{Explain}
\begin{enumerate}
\item countable noun \\
A \textbf{campaign} is a planned set of activities that people carry out over a period of time in order to achieve something such as social or political
change.
 \textit{
	\begin{itemize}
	\item During his election campaign he promised to put the economy back on its feet.
	\item ...the campaign against public smoking.
	\end{itemize}
}
\item verb \\
If someone \textbf{campaigns}  \textbf{for} something, they carry out a planned set of activities over a period of time in order
to achieve their aim.
 \textit{
	\begin{itemize}
	\item We are campaigning for law reform.
	\item Mr Burns has actively campaigned against a hostel being set up here.
	\item They have been campaigning to improve the legal status of women.
	\end{itemize}
}
\item countable noun \\
In a war , a \textbf{campaign} is a series of planned movements carried out by armed forces.
 \textit{
	\begin{itemize}
	\item The allies are intensifying their air campaign.
	\item ...a bombing campaign.
	\end{itemize}
}
\end{enumerate}

\section*{conservative}
{\large \color{blue}  conservatives  }
\subsection*{Explain}
\begin{enumerate}
\item adjective \\
A \textbf{Conservative}  politician or voter is a member of or votes for the Conservative Party in Britain .
 \textbf{Conservative} is also a noun .
 \textit{
	\begin{itemize}
	\item Most Conservative MPs appear happy with the government's reassurances.
	\item ...disenchanted Conservative voters.
	\item In 1951, the Conservatives were returned to power.
	\end{itemize}
}
\item adjective \\
Someone who is \textbf{conservative} has right-wing  views .
 \textbf{Conservative} is also a noun.
 \textit{
	\begin{itemize}
	\item ...counties whose citizens invariably support the most conservative candidate in
any election.
	\item The new judge is 50-year-old David Suitor who's regarded as a conservative.
	\end{itemize}
}
\item adjective \\
Someone who is \textbf{conservative} or has \textbf{conservative} ideas is unwilling to accept changes and new ideas.
 \textit{
	\begin{itemize}
	\item It is essentially a narrow and conservative approach to child care.
	\end{itemize}
}
\item adjective \\
If someone dresses in a \textbf{conservative} way, their clothes are conventional in style.
 \textit{
	\begin{itemize}
	\item The girl was well dressed, as usual, though in a more conservative style.
	\end{itemize}
}
\item adjective \\
A \textbf{conservative}  estimate or guess is one in which you are cautious and estimate or guess a low amount which is probably less than the real amount.
 \textit{
	\begin{itemize}
	\item A conservative estimate of the bill, so far, is about £22,000.
	\item This guess is probably on the conservative side.
	\end{itemize}
}
\end{enumerate}

\section*{choice}
{\large \color{blue}  choices  choicer  choicest  }
\subsection*{Explain}
\begin{enumerate}
\item countable noun \\
If there is a \textbf{choice}  \textbf{of} things, there are several of them and you can choose the one you want .
 \textit{
	\begin{itemize}
	\item It's available in a choice of colours.
	\item At lunchtime, there's a choice between the buffet or the set menu.
	\item Club Sportif offer a wide choice of holidays.
	\end{itemize}
}
\item countable noun \\
Your \textbf{choice} is someone or something that you choose from a range of things.
 \textit{
	\begin{itemize}
	\item Although he was only joking, his choice of words made Rodney angry.
	\end{itemize}
}
\item adjective \\
\textbf{Choice} means of very high quality.
 \textit{
	\begin{itemize}
	\item ...Fortnum and Mason's choicest chocolates.
	\end{itemize}
}
\item  \\
 have no choice/ have little choice \textit{
	\begin{itemize}
	\end{itemize}
}
\item  \\
 of one's choice \textit{
	\begin{itemize}
	\end{itemize}
}
\item  \\
 of choice \textit{
	\begin{itemize}
	\end{itemize}
}
\end{enumerate}

\section*{conspicuous}
{\large \color{blue}  }
\subsection*{Explain}
\begin{enumerate}
\item adjective \\
If someone or something is \textbf{conspicuous} , people can see or notice them very easily .
 \textit{
	\begin{itemize}
	\item He spent his money in a conspicuous way on fast cars and luxury holidays.
	\item You may feel tearful in situations where you feel conspicuous.
	\end{itemize}
}
\item  \\
 conspicuous by one's absence \textit{
	\begin{itemize}
	\end{itemize}
}
\end{enumerate}

\section*{clock}
{\large \color{blue}  clocks  clocking  clocked  }
\subsection*{Explain}
\begin{enumerate}
\item countable noun \\
A \textbf{clock} is an instrument, for example in a room or on the outside of a building, that shows what time of day it is.
 \textit{
	\begin{itemize}
	\item He was conscious of a clock ticking.
	\item He also repairs clocks and watches.
	\item The hands of the clock on the wall moved with a slight click.
	\item ...a digital clock.
	\end{itemize}
}
\item countable noun \\
A time \textbf{clock} in a factory or office is a device that is used to record the hours that people work. Each worker puts a special card into the device when they arrive and leave, and the times are recorded on the card.
 \textit{
	\begin{itemize}
	\item Government workers were made to punch time clocks morning, noon and night.
	\end{itemize}
}
\item countable noun \\
In a car, \textbf{the}  \textbf{clock} is the instrument that shows the speed of the car or the distance it has travelled .
 \textit{
	\begin{itemize}
	\item The car had 160,000 miles on the clock.
	\item At 240 mph the needle went off the clock.
	\end{itemize}
}
\item verb \\
To \textbf{clock} a particular time or speed in a race means to reach that time or speed.
 \textit{
	\begin{itemize}
	\item Elliott clocked the fastest time this year for the 800 metres.
	\item The yacht swayed in 40-knot winds, clocking speeds of 17 knots at times.
	\end{itemize}
}
\item verb \\
If something or someone \textbf{is clocked}  \textbf{at} a particular time or speed, their time or speed is measured at that level.
 \textit{
	\begin{itemize}
	\item He has been clocked at 11 seconds for 100 metres.
	\item 170-mile-an-hour winds were clocked on a mountaintop in North Carolina.
	\end{itemize}
}
\item verb \\
If you \textbf{clock} something, you notice or see it.
 \textit{
	\begin{itemize}
	\item I walked past that gate hundreds of times before I clocked it.
	\end{itemize}
}
\item  \\
 against the clock \textit{
	\begin{itemize}
	\end{itemize}
}
\item  \\
 to beat the clock \textit{
	\begin{itemize}
	\end{itemize}
}
\item  \\
 round the clock/around the clock \textit{
	\begin{itemize}
	\end{itemize}
}
\item  \\
 turn the clock back/put the clock back \textit{
	\begin{itemize}
	\end{itemize}
}
\item  \\
 watch the clock \textit{
	\begin{itemize}
	\end{itemize}
}
\end{enumerate}

\section*{dumb}
{\large \color{blue}  dumber  dumbest  dumbs  dumbing  dumbed  }
\subsection*{Explain}
\begin{enumerate}
\item adjective \\
Someone who is \textbf{dumb} is completely unable to speak.
 \textit{
	\begin{itemize}
	\item ...a young deaf and dumb man.
	\end{itemize}
}
\item adjective \\
If someone is \textbf{dumb} on a particular  occasion , they cannot speak because they are angry , shocked , or surprised .
 \textit{
	\begin{itemize}
	\item We were all struck dumb for a minute.
	\end{itemize}
}
\item adjective \\
If you call a person \textbf{dumb} , you mean that they are stupid or foolish.
 \textit{
	\begin{itemize}
	\item I've met a lot of dumb people.
	\item The questions were set up to make her look dumb.
	\end{itemize}
}
\item adjective \\
If you say that something is \textbf{dumb} , you think that it is silly and annoying .
 \textit{
	\begin{itemize}
	\item I came up with this dumb idea.
	\end{itemize}
}
\item adjective \\
Something that is \textbf{dumb} is done or expressed without words .
 \textit{
	\begin{itemize}
	\item An expression of dumb recognition wiggled across her features.
	\end{itemize}
}
\end{enumerate}

\section*{constituent}
{\large \color{blue}  constituents  }
\subsection*{Explain}
\begin{enumerate}
\item countable noun \\
A \textbf{constituent} is someone who lives in a particular constituency, especially someone who is able to vote in an election .
 \textit{
	\begin{itemize}
	\end{itemize}
}
\item countable noun \\
A \textbf{constituent}  \textbf{of} a mixture , substance, or system is one of the things from which it is formed.
 \textit{
	\begin{itemize}
	\item Caffeine is the active constituent of drinks such as tea and coffee.
	\end{itemize}
}
\item adjective \\
The \textbf{constituent} parts of something are the things from which it is formed.
 \textit{
	\begin{itemize}
	\item ...a plan to split the company into its constituent parts and sell them separately.
	\item ...the leaders of Russia's constituent republics.
	\end{itemize}
}
\end{enumerate}

\section*{eminent}
{\large \color{blue}  }
\subsection*{Explain}
\begin{enumerate}
\item adjective \\
An \textbf{eminent} person is well-known and respected , especially because they are good at their profession .
 \textit{
	\begin{itemize}
	\item ...an eminent scientist.
	\end{itemize}
}
\end{enumerate}

\section*{dial}
{\large \color{blue}  dials  dialling  dialled  }
\subsection*{Explain}
\begin{enumerate}
\item countable noun \\
A \textbf{dial} is the part of a machine or instrument such as a clock or watch which shows you the time or a measurement that has been recorded .
 \textit{
	\begin{itemize}
	\item The luminous dial on the clock showed five minutes to seven.
	\item The dials of most barometers are inscribed with weather terms.
	\end{itemize}
}
\item countable noun \\
A \textbf{dial} is a control on a device or piece of equipment which you can move in order to adjust the setting , for example to select or change the frequency on a radio or the temperature of a heater .
 \textit{
	\begin{itemize}
	\item He turned the dial on the radio.
	\item The heat dial was set at 150 degrees.
	\end{itemize}
}
\item countable noun \\
On some telephones, especially older ones, \textbf{the}  \textbf{dial} is the disc on the front that you turn with your finger to choose the number that you want to call. The disc has holes in it, and numbers or letters behind the holes.
 \textit{
	\begin{itemize}
	\item ...turning the dial on the phone.
	\end{itemize}
}
\item verb \\
If you \textbf{dial} or if you \textbf{dial} a number, you turn the dial or press the buttons on a telephone in order to phone someone.
 \textit{
	\begin{itemize}
	\item He lifted the phone and dialled her number.
	\item He dialled, and spoke briefly to the duty officer.
	\end{itemize}
}
\end{enumerate}

\section*{ethnic}
{\large \color{blue}  }
\subsection*{Explain}
\begin{enumerate}
\item adjective \\
\textbf{Ethnic}  means  connected with or relating to different racial or cultural groups of people.
 \textit{
	\begin{itemize}
	\item ...a survey of Britain's ethnic minorities.
	\item ...ethnic tensions.
	\end{itemize}
}
\item adjective \\
You can use \textbf{ethnic} to describe people who belong to a particular racial or cultural group but who, usually, do not live in the country where most members of that group live.
 \textit{
	\begin{itemize}
	\item There are still several million ethnic Germans in Russia.
	\end{itemize}
}
\item adjective \\
\textbf{Ethnic}  clothing , music , or food is characteristic of the traditions of a particular ethnic group, and different
from what is usually found in modern  Western culture.
 \textit{
	\begin{itemize}
	\item ...the original flavours of ethnic dishes.
	\item ...a magnificent range of ethnic fabrics.
	\end{itemize}
}
\end{enumerate}

\section*{dot}
{\large \color{blue}  dots  dotting  dotted  }
\subsection*{Explain}
\begin{enumerate}
\item countable noun \\
A \textbf{dot} is a very small round mark, for example one that is used as the top part of the letter 'i', as a full  stop , or as a decimal point.
 \textit{
	\begin{itemize}
	\end{itemize}
}
\item countable noun \\
You can refer to something that you can see in the distance and that looks like a small round mark as a \textbf{dot} .
 \textit{
	\begin{itemize}
	\item Soon they were only dots above the hard line of the horizon.
	\end{itemize}
}
\item verb \\
When things \textbf{dot} a place or an area, they are scattered or spread all over it.
 \textit{
	\begin{itemize}
	\item Small coastal towns dot the landscape.
	\end{itemize}
}
\item  \\
 on the dot \textit{
	\begin{itemize}
	\end{itemize}
}
\item  \\
 dot the i's and cross the t's \textit{
	\begin{itemize}
	\end{itemize}
}
\item  \\
 the year dot \textit{
	\begin{itemize}
	\end{itemize}
}
\end{enumerate}

\section*{exclusive}
{\large \color{blue}  exclusives  }
\subsection*{Explain}
\begin{enumerate}
\item adjective \\
If you describe something as \textbf{exclusive} , you mean that it is limited to people who have a lot of money or who belong to a high social class, and is therefore not available to everyone.
 \textit{
	\begin{itemize}
	\item He is already a member of Britain's most exclusive club.
	\item The City was criticised for being too exclusive and uncompetitive.
	\end{itemize}
}
\item adjective \\
Something that is \textbf{exclusive} is used or owned by only one person or group, and not shared with anyone else.
 \textit{
	\begin{itemize}
	\item Our group will have exclusive use of a 60-foot boat.
	\item Many of their cheeses are exclusive to our stores in Britain.
	\end{itemize}
}
\item adjective \\
If a newspaper, magazine , or broadcasting  organization describes one of its reports as \textbf{exclusive} , they mean that it is a special report which does not appear in any other publication or on any other channel .
 An \textbf{exclusive} is an exclusive article or report.
 \textit{
	\begin{itemize}
	\item He told the magazine in an exclusive interview: 'All my problems stem from drink'.
	\item Some papers thought they had an exclusive.
	\end{itemize}
}
\item adjective \\
If a company  states that its prices , goods, or services are \textbf{exclusive}  \textbf{of} something, that thing is not included in the stated price, although it usually still has to be paid for.
 \textit{
	\begin{itemize}
	\item All charges for service are exclusive of value added tax.
	\item Skiing weekends cost £58 (exclusive of travel and accommodation).
	\end{itemize}
}
\item  \\
 mutually exclusive \textit{
	\begin{itemize}
	\end{itemize}
}
\end{enumerate}

\section*{friday}
{\large \color{blue}  Fridays  }
\subsection*{Explain}
\begin{enumerate}
\item variable noun \\
\textbf{Friday} is the day after Thursday and before Saturday .
 \textit{
	\begin{itemize}
	\item Mr Cook is intending to go to Brighton on Friday.
	\item The weekly series starts next Friday.
	\item I get home at half seven on Fridays.
	\item He left Heathrow airport on Friday morning.
	\end{itemize}
}
\end{enumerate}

\section*{galaxy}
{\large \color{blue}  galaxies  }
\subsection*{Explain}
\begin{enumerate}
\item countable noun \\
A \textbf{galaxy} is an extremely large group of stars and planets that extends over many billions of light years .
 \textit{
	\begin{itemize}
	\item Astronomers have discovered a distant galaxy.
	\end{itemize}
}
\item proper noun \\
\textbf{The Galaxy} is the extremely large group of stars and planets to which the Earth and the Solar System belong.
 \textit{
	\begin{itemize}
	\item The Galaxy consists of 100 billion stars.
	\end{itemize}
}
\item singular noun \\
If you talk about \textbf{a galaxy of} people from a particular profession , you mean a group of them who are all famous or important .
 \textit{
	\begin{itemize}
	\item He is one of a small galaxy of Dutch stars on German television.
	\end{itemize}
}
\end{enumerate}

\section*{fragrant}
{\large \color{blue}  }
\subsection*{Explain}
\begin{enumerate}
\item adjective \\
Something that is \textbf{fragrant} has a pleasant, sweet smell.
 \textit{
	\begin{itemize}
	\item ...fragrant oils and perfumes.
	\item The air was fragrant with the smell of orange blossoms.
	\end{itemize}
}
\end{enumerate}

\section*{hook}
{\large \color{blue}  hooks  hooking  hooked  }
\subsection*{Explain}
\begin{enumerate}
\item countable noun \\
A \textbf{hook} is a bent piece of metal or plastic that is used for catching or holding things, or for hanging things up.
 \textit{
	\begin{itemize}
	\item One of his jackets hung from a hook.
	\item ...curtain hooks.
	\item He felt a fish pull at his hook.
	\end{itemize}
}
\item verb \\
If you \textbf{hook} one thing \textbf{to} another, you attach it there using a hook. If something \textbf{hooks}  somewhere , it can be hooked there.
 \textit{
	\begin{itemize}
	\item Paul hooked his tractor to the car and pulled it to safety.
	\item ...one of those can openers that hooked onto the wall.
	\end{itemize}
}
\item verb \\
If you \textbf{hook} your arm, leg, or foot round an object, you place it like a hook round the object
in order to move it or hold it.
 \textit{
	\begin{itemize}
	\item She latched on to his arm, hooking her other arm around a tree.
	\item I hooked my left arm over the side of the dinghy.
	\end{itemize}
}
\item verb \\
If you \textbf{hook} a fish, you catch it with a hook on the end of a line.
 \textit{
	\begin{itemize}
	\item At the first cast I hooked a huge fish.
	\end{itemize}
}
\item countable noun \\
A \textbf{hook} is a short sharp blow with your fist that you make with your elbow bent, usually in a boxing  match .
 \textit{
	\begin{itemize}
	\item He was knocked down by a left hook in the first round.
	\end{itemize}
}
\item verb \\
If you \textbf{are hooked into} something, or \textbf{hook into} something, you get involved with it.
 \textit{
	\begin{itemize}
	\item I'm guessing again now because I'm not hooked into the political circles.
	\item Eager to hook into a career but can't find one right for you?
	\end{itemize}
}
\item verb \\
If you \textbf{hook}  \textbf{into} the internet , you make a connection with the internet on a particular occasion so that you can use it.
 \textbf{Hook up} means the same as hook .
 \textit{
	\begin{itemize}
	\item ...an interactive media tent where people will be able to hook into the internet.
	\item It has no mobile connectivity, which means that users must rely on wi-fi to hook
up to the internet.
	\end{itemize}
}
\item  \\
 to let someone off the hook \textit{
	\begin{itemize}
	\end{itemize}
}
\item  \\
 off the hook \textit{
	\begin{itemize}
	\end{itemize}
}
\item  \\
 ringing off the hook \textit{
	\begin{itemize}
	\end{itemize}
}
\end{enumerate}

\section*{heavy}
{\large \color{blue}  heavier  heaviest  heavies  }
\subsection*{Explain}
\begin{enumerate}
\item adjective \\
Something that is \textbf{heavy}  weighs a lot .
 \textit{
	\begin{itemize}
	\item These scissors are awfully heavy.
	\item Gosh, that was a heavy bag!
	\item The mud stuck to her boots, making her feet heavy and her legs tired.
	\end{itemize}
}
\item adjective \\
You use \textbf{heavy} to ask or talk about how much someone or something weighs.
 \textit{
	\begin{itemize}
	\item How heavy are you?
	\item Protons are nearly 2000 times as heavy as electrons.
	\end{itemize}
}
\item adjective \\
\textbf{Heavy} means great in amount, degree, or intensity.
 \textit{
	\begin{itemize}
	\item Heavy fighting has been going on.
	\item The State fails to recognize the heavy responsibility that parents take on.
	\item He worried about her heavy drinking.
	\item ...lengthy jail sentences and heavy fines.
	\item The traffic along Fitzjohn's Avenue was heavy.
	\end{itemize}
}
\item adjective \\
Someone or something that is \textbf{heavy} is solid in appearance or structure, or is made of a thick material.
 \textit{
	\begin{itemize}
	\item We talk in her Belgrade flat, full of heavy old brown furniture.
	\item He was short and heavy.
	\item Put the sugar and water in a heavy pan and heat slowly.
	\item ...a heavy cream silk blouse.
	\end{itemize}
}
\item graded adjective \\
A \textbf{heavy} substance is thick in texture .
 \textit{
	\begin{itemize}
	\item It is advisable to mix coarse grit into heavy soil to improve drainage.
	\item ...11 million gallons of heavy crude oil.
	\end{itemize}
}
\item adjective \\
A \textbf{heavy} meal is large in amount and often difficult to digest .
 \textit{
	\begin{itemize}
	\item He had been feeling drowsy, the effect of an unusually heavy meal.
	\end{itemize}
}
\item adjective \\
Something that is \textbf{heavy with} things is full of them or loaded with them.
 \textit{
	\begin{itemize}
	\item The air is heavy with moisture.
	\item She brought in a tray heavy with elegant sandwiches, scones and cakes.
	\end{itemize}
}
\item adjective \\
If a person's breathing is \textbf{heavy} , it is very loud and deep.
 \textit{
	\begin{itemize}
	\item Her breathing became slow and heavy.
	\end{itemize}
}
\item adjective \\
A \textbf{heavy} movement or action is done with a lot of force or pressure.
 \textit{
	\begin{itemize}
	\item ...a heavy blow on the back of the skull.
	\item The plane made a heavy landing.
	\end{itemize}
}
\item adjective \\
A \textbf{heavy} machine or piece of military equipment is very large and very powerful.
 \textit{
	\begin{itemize}
	\item ...government militia backed by tanks and heavy artillery.
	\item ...armoured personnel carriers and other heavy vehicles.
	\end{itemize}
}
\item adjective \\
If you describe a period of time or a schedule as \textbf{heavy} , you mean it involves a lot of work.
 \textit{
	\begin{itemize}
	\item It's been a heavy day and I'm tired.
	\item Hopefully, Max would be able to spend a few days with them, depending on his heavy
schedule.
	\end{itemize}
}
\item adjective \\
\textbf{Heavy} work requires a lot of strength or energy.
 \textit{
	\begin{itemize}
	\item The business is thriving and Philippa employs two full-timers for the heavy work.
	\end{itemize}
}
\item adjective \\
If you say that something is \textbf{heavy on} another thing, you mean that it uses a lot of that thing or too much of that thing.
 \textit{
	\begin{itemize}
	\item Tanks are heavy on fuel and destructive to roads.
	\item ...salads heavy on carrots.
	\end{itemize}
}
\item adjective \\
Air or weather that is \textbf{heavy} is unpleasantly still , hot, and damp .
 \textit{
	\begin{itemize}
	\item The outside air was heavy and moist and sultry.
	\end{itemize}
}
\item graded adjective \\
If you describe a person's face as \textbf{heavy} , you mean that it looks sad, tired , or unfriendly .
 \textit{
	\begin{itemize}
	\end{itemize}
}
\item adjective \\
If your heart is \textbf{heavy} , you are sad about something.
 \textit{
	\begin{itemize}
	\item Mr Maddison handed over his resignation letter with a heavy heart.
	\end{itemize}
}
\item adjective \\
A situation that is \textbf{heavy} is serious and difficult to cope with.
 \textit{
	\begin{itemize}
	\item I don't want any more of that heavy stuff.
	\end{itemize}
}
\item countable noun \\
A \textbf{heavy} is a large strong man who is employed to protect a person or place, often by using
violence.
 \textit{
	\begin{itemize}
	\item They had employed heavies to evict shop squatters from neighbouring sites.
	\end{itemize}
}
\end{enumerate}

\section*{hydrogen}
{\large \color{blue}  }
\subsection*{Explain}
\begin{enumerate}
\item uncountable noun \\
\textbf{Hydrogen} is a colourless gas that is the lightest and commonest element in the universe.
 \textit{
	\begin{itemize}
	\end{itemize}
}
\end{enumerate}

\section*{important}
{\large \color{blue}  }
\subsection*{Explain}
\begin{enumerate}
\item adjective \\
Something that is \textbf{important} is very significant , is highly valued, or is necessary .
 \textit{
	\begin{itemize}
	\item Her sons are the most important thing in her life.
	\item The planned general strike represents an important economic challenge to the government.
	\item This gold is every bit as important to me as it is to you.
	\item It's important to answer her questions as honestly as you can.
	\item It was important that he rest.
	\end{itemize}
}
\item adjective \\
Someone who is \textbf{important} has influence or power within a society or a particular group.
 \textit{
	\begin{itemize}
	\item He was the most important person on the island.
	\item ...an important figure in the media world.
	\end{itemize}
}
\end{enumerate}

\section*{key}
{\large \color{blue}  keys  keying  keyed  }
\subsection*{Explain}
\begin{enumerate}
\item countable noun \\
A \textbf{key} is a specially shaped piece of metal that you place in a lock and turn in order to
open or lock a door, or to start or stop the engine of a vehicle.
 \textit{
	\begin{itemize}
	\item They put the key in the door and entered.
	\item She reached for her coat and car keys.
	\end{itemize}
}
\item countable noun \\
The \textbf{keys} on a computer keyboard or typewriter are the buttons that you press in order to operate
it.
 \textit{
	\begin{itemize}
	\end{itemize}
}
\item countable noun \\
The \textbf{keys} of a piano or organ are the long narrow pieces of wood or plastic that you press in order to
play it.
 \textit{
	\begin{itemize}
	\end{itemize}
}
\item variable noun \\
In music, a \textbf{key} is a scale of musical notes that starts on one specific note.
 \textit{
	\begin{itemize}
	\item ...the key of A minor.
	\end{itemize}
}
\item countable noun \\
The \textbf{key} on a map or diagram or in a technical book is a list of the symbols or abbreviations used and their meanings.
 \textit{
	\begin{itemize}
	\item You will find a key at the front of the book.
	\end{itemize}
}
\item adjective \\
The \textbf{key} person or thing in a group is the most important one.
 \textit{
	\begin{itemize}
	\item He is expected to be the key witness at the trial.
	\item Education is likely to be a key issue in the next election.
	\end{itemize}
}
\item countable noun \\
The \textbf{key}  \textbf{to} a desirable situation or result is the way in which it can be achieved.
 \textit{
	\begin{itemize}
	\item The key to success is to be ready from the start.
	\item Diet and relaxation are two important keys to good health.
	\end{itemize}
}
\end{enumerate}

\section*{intricate}
{\large \color{blue}  }
\subsection*{Explain}
\begin{enumerate}
\item adjective \\
You use \textbf{intricate} to describe something that has many small parts or details .
 \textit{
	\begin{itemize}
	\item ...intricate patterns and motifs.
	\end{itemize}
}
\end{enumerate}

\section*{layoff}
{\large \color{blue}  layoffs  }
\subsection*{Explain}
\begin{enumerate}
\item countable noun \\
When there are \textbf{layoffs} in a company, workers are told by their employers to leave their job , usually because there is no more work for them in the company.
 \textit{
	\begin{itemize}
	\item Store closures will result in layoffs of an estimated 2,000 employees.
	\end{itemize}
}
\item countable noun \\
A \textbf{layoff} is a period of time in which people do not work or take part in their normal activities,
often because they are resting or are injured .
 \textit{
	\begin{itemize}
	\item They both made full recoveries after lengthy injury layoffs.
	\end{itemize}
}
\end{enumerate}

\section*{invisible}
{\large \color{blue}  invisibles  }
\subsection*{Explain}
\begin{enumerate}
\item adjective \\
If you describe something as \textbf{invisible} , you mean that it cannot be seen, for example because it is transparent , hidden, or very small.
 \textit{
	\begin{itemize}
	\item The lines were so finely etched as to be invisible from a distance.
	\item The belt is invisible even under the thinnest garments.
	\end{itemize}
}
\item adjective \\
You can use \textbf{invisible} when you are talking about something that cannot be seen but has a definite  effect . In this sense , \textbf{invisible} is often used before a noun which refers to something that can usually be seen.
 \textit{
	\begin{itemize}
	\item Parents fear they might overstep these invisible boundaries.
	\item Her father's face had suddenly tightened as though he was being strangled by invisible
hands.
	\end{itemize}
}
\item adjective \\
If you say that you feel  \textbf{invisible} , you are complaining that you are being ignored by other people. If you say that a particular  problem or situation is \textbf{invisible} , you are complaining that it is not being considered or dealt with.
 \textit{
	\begin{itemize}
	\item It was strange, how invisible a clerk could feel.
	\item The problems of the poor are largely invisible.
	\end{itemize}
}
\item adjective \\
In stories , \textbf{invisible} people or things have a magic  quality which makes people unable to see them.
 \textit{
	\begin{itemize}
	\item ...The Invisible Man.
	\end{itemize}
}
\item adjective \\
In economics , \textbf{invisible} earnings are the money that a country makes as a result of services such as banking and tourism , rather than by producing goods.
 \textit{
	\begin{itemize}
	\item Tourism is Britain's single biggest invisible export.
	\item The revenue from tourism is the biggest single item in the country's invisible earnings.
	\item The invisible trade surplus was £900 million lower than reported.
	\end{itemize}
}
\item plural noun \\
\textbf{Invisibles} are services such as banking and tourism, which provide a country's invisible earnings.
 \textit{
	\begin{itemize}
	\end{itemize}
}
\end{enumerate}

\section*{limit}
{\large \color{blue}  limits  limiting  limited  }
\subsection*{Explain}
\begin{enumerate}
\item countable noun \\
A \textbf{limit} is the greatest amount, extent, or degree of something that is possible .
 \textit{
	\begin{itemize}
	\item Her love for him was being tested to its limits.
	\item There is no limit to how much fresh fruit you can eat in a day.
	\item Firefighters are being stretched to the limit as fire sweeps through the state.
	\end{itemize}
}
\item countable noun \\
A \textbf{limit} of a particular kind is the largest or smallest amount of something such as time or money that is allowed
because of a rule , law , or decision .
 \textit{
	\begin{itemize}
	\item The three month time limit will be up in mid-June.
	\item The economic affairs minister announced limits on petrol sales.
	\end{itemize}
}
\item countable noun \\
The \textbf{limit} of an area is its boundary or edge.
 \textit{
	\begin{itemize}
	\item ...the city limits of Baghdad.
	\end{itemize}
}
\item plural noun \\
\textbf{The}  \textbf{limits}  \textbf{of} a situation are the facts involved in it which make only some actions or results possible.
 \textit{
	\begin{itemize}
	\item She has to work within the limits of a fairly tight budget.
	\item He outlined the limits of British power.
	\end{itemize}
}
\item verb \\
If you \textbf{limit} something, you prevent it from becoming  greater than a particular amount or degree.
 \textit{
	\begin{itemize}
	\item He limited payments on the country's foreign debt.
	\item Place numbers are limited to 25 on both tours, so please book early.
	\end{itemize}
}
\item verb \\
If you \textbf{limit}  \textbf{yourself} to something, or if someone or something \textbf{limits} you, the number of things that you have or do is reduced .
 \textit{
	\begin{itemize}
	\item It is now accepted that men should limit themselves to 20 units of alcohol a week.
	\item Voters cut councillors' pay and limited them to one staff member each.
	\end{itemize}
}
\item verb \\
If something \textbf{is limited}  \textbf{to} a particular place or group of people, it exists only in that place, or is had or done only by that group.
 \textit{
	\begin{itemize}
	\item The protests were not limited to New York.
	\item Entry to this prize draw is limited to U.K. residents.
	\end{itemize}
}
\item  \\
 off limits \textit{
	\begin{itemize}
	\end{itemize}
}
\item  \\
 off limits \textit{
	\begin{itemize}
	\end{itemize}
}
\item  \\
 be over the limit \textit{
	\begin{itemize}
	\end{itemize}
}
\item  \\
 the sky is the limit \textit{
	\begin{itemize}
	\end{itemize}
}
\item  \\
 within limits \textit{
	\begin{itemize}
	\end{itemize}
}
\end{enumerate}

\section*{likely}
{\large \color{blue}  likelier  likeliest  }
\subsection*{Explain}
\begin{enumerate}
\item adjective \\
You use \textbf{likely} to indicate that something is probably the case or will probably happen in a particular situation .
 \textbf{Likely} is also an adverb .
 \textit{
	\begin{itemize}
	\item Experts say a 'yes' vote is still the likely outcome.
	\item If this is your first baby, it's far more likely that you'll get to the hospital
too early.
	\item Francis thought it likely John still loved her.
	\item Profit will most likely have risen by about £25 million.
	\item Very likely he'd told them he had American business interests.
	\end{itemize}
}
\item adjective \\
If someone or something is \textbf{likely}  \textbf{to} do a particular thing, they will very probably do it.
 \textit{
	\begin{itemize}
	\item In the meantime the war of nerves seems likely to continue.
	\item Adolescents who watched more than two hours of TV a day were much more likely to
be overweight.
	\end{itemize}
}
\item adjective \\
A \textbf{likely} person, place, or thing is one that will probably be suitable for a particular purpose.
 \textit{
	\begin{itemize}
	\item At one point he had seemed a likely candidate to become Prime Minister.
	\item We aimed the microscope at a likely looking target.
	\end{itemize}
}
\item  \\
 not likely \textit{
	\begin{itemize}
	\end{itemize}
}
\end{enumerate}

\section*{microscope}
{\large \color{blue}  microscopes  }
\subsection*{Explain}
\begin{enumerate}
\item countable noun \\
A \textbf{microscope} is a scientific instrument which makes very small objects look  bigger so that more detail can be seen .
 \textit{
	\begin{itemize}
	\end{itemize}
}
\item  \\
 under the microscope \textit{
	\begin{itemize}
	\end{itemize}
}
\end{enumerate}

\section*{mute}
{\large \color{blue}  mutes  muting  muted  }
\subsection*{Explain}
\begin{enumerate}
\item adjective \\
Someone who is \textbf{mute} is silent for a particular  reason and does not speak.
 \textbf{Mute} is also an adverb .
 \textit{
	\begin{itemize}
	\item He was mute, distant, and indifferent.
	\item I threw a mute look of appeal at Paula.
	\item He could watch her standing mute by the phone.
	\item He sat mute, speechless with ecstasy, gazing into the sky.
	\end{itemize}
}
\item adjective \\
Someone who is \textbf{mute} is unable to speak.
 \textit{
	\begin{itemize}
	\item Marianna, the duke's daughter, became mute after a shock.
	\end{itemize}
}
\item verb \\
If someone \textbf{mutes} something such as their feelings or their activities , they reduce the strength or intensity of them.
 \textit{
	\begin{itemize}
	\item The corruption does not seem to have muted the country's prolonged economic boom.
	\end{itemize}
}
\item verb \\
If you \textbf{mute} a noise or sound, you lower its volume or make it less distinct .
 \textit{
	\begin{itemize}
	\item They begin to mute their voices, not be as assertive.
	\item At first the wooded hillsides muted the sounds.
	\end{itemize}
}
\item countable noun \\
A \textbf{mute} is a device which can be used to make a musical instrument produce a quieter , softer sound.
 \textit{
	\begin{itemize}
	\end{itemize}
}
\end{enumerate}

\section*{monday}
{\large \color{blue}  Mondays  }
\subsection*{Explain}
\begin{enumerate}
\item variable noun \\
\textbf{Monday} is the day after Sunday and before Tuesday .
 \textit{
	\begin{itemize}
	\item I went back to work on Monday.
	\item The attack took place last Monday.
	\item I'm usually here on Mondays and Fridays.
	\item The deaths on Monday afternoon were being treated as accidental.
	\end{itemize}
}
\end{enumerate}

\section*{narrative}
{\large \color{blue}  narratives  }
\subsection*{Explain}
\begin{enumerate}
\item countable noun \\
A \textbf{narrative} is a story or an account of a series of events.
 \textit{
	\begin{itemize}
	\item ...a fast-moving narrative.
	\item Sloan began his narrative with the day of the murder.
	\end{itemize}
}
\item uncountable noun \\
\textbf{Narrative} is the description of a series of events, usually in a novel .
 \textit{
	\begin{itemize}
	\item Neither author was very strong on narrative.
	\item ...Nye's simple narrative style.
	\end{itemize}
}
\end{enumerate}

\section*{pen}
{\large \color{blue}  pens  penning  penned  }
\subsection*{Explain}
\begin{enumerate}
\item countable noun \\
A \textbf{pen} is a long thin object which you use to write in ink.
 \textit{
	\begin{itemize}
	\end{itemize}
}
\item verb \\
If someone \textbf{pens} a letter , article , or book, they write it.
 \textit{
	\begin{itemize}
	\item I really intended to pen this letter to you early this morning.
	\item She penned a short memo to his private secretary.
	\end{itemize}
}
\item countable noun \\
A \textbf{pen} is also a small area with a fence  round it in which farm animals are kept for a short time.
 \textit{
	\begin{itemize}
	\item ...a holding pen for sheep.
	\item He wasn't sure exactly how a fox could have got into the sheep's pen.
	\end{itemize}
}
\item verb \\
If people or animals \textbf{are penned}  somewhere or \textbf{are penned up} , they are forced to remain in a very small area.
 \textit{
	\begin{itemize}
	\item ...to drive the cattle back to the house so they could be milked and penned for the
night.
	\item The goats are penned in and fodder has to be cut and carried each day.
	\item I don't have to stay in my room penned up like a prisoner.
	\end{itemize}
}
\item countable noun \\
People sometimes  say  \textbf{the pen} to refer to a prison .
 \textit{
	\begin{itemize}
	\end{itemize}
}
\item  \\
 put pen to paper \textit{
	\begin{itemize}
	\end{itemize}
}
\end{enumerate}

\section*{narrow}
{\large \color{blue}  narrower  narrowest  narrows  narrowing  narrowed  }
\subsection*{Explain}
\begin{enumerate}
\item adjective \\
Something that is \textbf{narrow} measures a very small distance from one side to the other, especially compared to its length or height .
 \textit{
	\begin{itemize}
	\item ...through the town's narrow streets.
	\item She had long, narrow feet.
	\item ...the narrow strip of land joining the peninsula to the rest of the island.
	\end{itemize}
}
\item verb \\
If something \textbf{narrows} , it becomes less wide .
 \textit{
	\begin{itemize}
	\item The wide track narrows before crossing another stream.
	\end{itemize}
}
\item verb \\
If your eyes \textbf{narrow} or if you \textbf{narrow} your eyes, you almost close them, for example because you are angry or because you are trying to concentrate on something.
 \textit{
	\begin{itemize}
	\item Coggins' eyes narrowed angrily. 'You think I'd tell you?'
	\item He paused and narrowed his eyes in concentration.
	\end{itemize}
}
\item adjective \\
If you describe someone's ideas, attitudes , or beliefs as \textbf{narrow} , you disapprove of them because they are restricted in some way, and often ignore the more important aspects of an argument or situation.
 \textit{
	\begin{itemize}
	\item ...a narrow and outdated view of family life.
	\item I would have preferred somebody who had wider ideas, and he was rather narrow.
	\end{itemize}
}
\item verb \\
If something \textbf{narrows} or if you \textbf{narrow} it, its extent or range becomes smaller.
 \textit{
	\begin{itemize}
	\item Most recent opinion polls suggest that the gap between the two main parties has narrowed.
	\item Negotiators narrowed their differences over federal spending for anti-drug programs.
	\end{itemize}
}
\item adjective \\
If you have a \textbf{narrow}  victory , you succeed in winning but only by a small amount.
 \textit{
	\begin{itemize}
	\item Delegates have voted by a narrow majority in favour of considering electoral reform.
	\end{itemize}
}
\item adjective \\
If you have a \textbf{narrow} escape, something unpleasant  nearly  happens to you.
 \textit{
	\begin{itemize}
	\item Two police officers had a narrow escape when separatists attacked their vehicles.
	\end{itemize}
}
\end{enumerate}

\section*{piano}
{\large \color{blue}  pianos  }
\subsection*{Explain}
\begin{enumerate}
\item variable noun \\
A \textbf{piano} is a large musical instrument with a row of black and white keys. When you press these keys with your fingers , little hammers hit  wire strings inside the piano which vibrate to produce musical notes.
 \textit{
	\begin{itemize}
	\item I taught myself how to play the piano.
	\item He started piano lessons at the age of 7.
	\item ...sonatas for cello and piano.
	\item ...Rachmaninov's Fourth Piano Concerto.
	\end{itemize}
}
\item adverb \\
A piece of music that is played \textbf{piano} is played quietly .
 \textit{
	\begin{itemize}
	\end{itemize}
}
\end{enumerate}

\section*{noticeable}
{\large \color{blue}  }
\subsection*{Explain}
\begin{enumerate}
\item adjective \\
Something that is \textbf{noticeable} is very obvious , so that it is easy to see, hear , or recognize .
 \textit{
	\begin{itemize}
	\item It is noticeable that trees planted next to houses usually lean away from the house
wall.
	\item The most noticeable effect of these changes is in the way people are now working
together.
	\end{itemize}
}
\end{enumerate}

\section*{pint}
{\large \color{blue}  pints  }
\subsection*{Explain}
\begin{enumerate}
\item countable noun \\
A \textbf{pint} is a unit of measurement for liquids. In Britain , it is equal to 568 cubic  centimetres or one eighth of an imperial gallon. In America , it is equal to 473 cubic centimetres or one eighth of an American gallon.
 \textit{
	\begin{itemize}
	\item ...a pint of milk.
	\item The military requested 6,000 pints of blood from the American Red Cross.
	\item ...glasses which can hold a full pint.
	\end{itemize}
}
\item countable noun \\
If you go for a \textbf{pint} , you go to the pub to drink a pint of beer or more.
 \textit{
	\begin{itemize}
	\item He sits down and reads the paper, then goes out for a pint.
	\end{itemize}
}
\end{enumerate}

\section*{offensive}
{\large \color{blue}  offensives  }
\subsection*{Explain}
\begin{enumerate}
\item adjective \\
Something that is \textbf{offensive} upsets or embarrasses people because it is rude or insulting.
 \textit{
	\begin{itemize}
	\item Some friends of his found the play horribly offensive.
	\item ...offensive remarks which called into question the integrity of my firm.
	\end{itemize}
}
\item countable noun \\
A military \textbf{offensive} is a carefully planned attack made by a large group of soldiers .
 \textit{
	\begin{itemize}
	\item Its latest military offensive against rebel forces is aimed at re-opening trade routes.
	\item The armed forces have launched offensives to recapture lost ground.
	\end{itemize}
}
\item countable noun \\
If you conduct an \textbf{offensive} , you take strong action to show how angry you are about something or how much you disapprove of something.
 \textit{
	\begin{itemize}
	\item Republicans acknowledged that they had little choice but to mount an all-out offensive
on the Democratic nominee.
	\item ...a diplomatic offensive.
	\end{itemize}
}
\item adjective \\
In sports such as American football or basketball , the \textbf{offensive}  team is the team which has possession of the ball and is trying to score .
 \textit{
	\begin{itemize}
	\item The worst-ever defeat of this team proved once again that Stanford can be one of
the most explosive offensive teams in the country.
	\end{itemize}
}
\item  \\
 go on the offensive \textit{
	\begin{itemize}
	\end{itemize}
}
\end{enumerate}

\section*{province}
{\large \color{blue}  provinces  }
\subsection*{Explain}
\begin{enumerate}
\item countable noun \\
A \textbf{province} is a large section of a country which has its own administration .
 \textit{
	\begin{itemize}
	\item ...the Algarve, Portugal's southernmost province.
	\end{itemize}
}
\item plural noun \\
\textbf{The}  \textbf{provinces} are all the parts of a country except the part where the capital is situated .
 \textit{
	\begin{itemize}
	\item The government plans to transfer some 30,000 government jobs from the capital to
the provinces.
	\end{itemize}
}
\item singular noun \\
If you say that a subject or activity is a particular person's \textbf{province} , you mean that this person has a special  interest in it, a special knowledge of it, or a special responsibility for it.
 \textit{
	\begin{itemize}
	\item Arvo avoided committing himself. 'I'm afraid that's not my province,' he replied.
	\item Industrial research is the province of the Department of Trade and Industry.
	\end{itemize}
}
\end{enumerate}

\section*{plural}
{\large \color{blue}  plurals  }
\subsection*{Explain}
\begin{enumerate}
\item adjective \\
The \textbf{plural} form of a word is the form that is used when referring to more than one person or
thing.
 \textit{
	\begin{itemize}
	\item 'Data' is the Latin plural form of 'datum'.
	\item ...his use of the plural pronoun 'we'.
	\end{itemize}
}
\item countable noun \\
The \textbf{plural} of a noun is the form of it that is used to refer to more than one person or thing.
 \textit{
	\begin{itemize}
	\item What is the plural of 'person'?
	\item ...irregular plurals.
	\end{itemize}
}
\item graded adjective \\
A \textbf{plural}  society or system involves different  kinds of people.
 \textit{
	\begin{itemize}
	\item Britain is a plural society in which the secular predominates.
	\item His government has pledged to move the country towards a plural democracy.
	\end{itemize}
}
\end{enumerate}

\section*{rainbow}
{\large \color{blue}  rainbows  }
\subsection*{Explain}
\begin{enumerate}
\item countable noun \\
A \textbf{rainbow} is an arch of different colours that you can sometimes  see in the sky when it is raining.
 \textit{
	\begin{itemize}
	\item Oh look, a rainbow!
	\item ...silk brocade of every colour of the rainbow.
	\end{itemize}
}
\item countable noun \\
A \textbf{rainbow} of colours is a wide  range of bright colours.
 \textit{
	\begin{itemize}
	\item ...a rainbow of coloured cushions.
	\end{itemize}
}
\item  \\
 the end of the rainbow \textit{
	\begin{itemize}
	\end{itemize}
}
\end{enumerate}

\section*{probable}
{\large \color{blue}  }
\subsection*{Explain}
\begin{enumerate}
\item adjective \\
If you say that something is \textbf{probable} , you mean that it is likely to be true or likely to happen.
 \textit{
	\begin{itemize}
	\item It is probable that the medication will suppress the symptom without treating the
condition.
	\item The probable result is that asset prices will again rise rapidly.
	\item An airline official said a bomb was the incident's most probable cause.
	\end{itemize}
}
\item adjective \\
You can use \textbf{probable} to describe a role or function that someone or something is likely to have.
 \textit{
	\begin{itemize}
	\item The Socialists united behind their probable presidential candidate.
	\end{itemize}
}
\end{enumerate}

\section*{saturday}
{\large \color{blue}  Saturdays  }
\subsection*{Explain}
\begin{enumerate}
\item variable noun \\
\textbf{Saturday} is the day after Friday and before Sunday .
 \textit{
	\begin{itemize}
	\item She had a call from him on Saturday morning at the studio.
	\item They had a 3-1 win against Liverpool last Saturday.
	\item The overnight train runs only on Saturdays.
	\item It was Saturday evening and I was getting ready to go out.
	\end{itemize}
}
\end{enumerate}

\section*{proper}
{\large \color{blue}  }
\subsection*{Explain}
\begin{enumerate}
\item adjective \\
You use \textbf{proper} to describe things that you consider to be real and satisfactory rather than inadequate in some way.
 \textit{
	\begin{itemize}
	\item Two out of five people lack a proper job.
	\item I always cook a proper evening meal.
	\end{itemize}
}
\item adjective \\
The \textbf{proper} thing is the one that is correct or most suitable .
 \textit{
	\begin{itemize}
	\item The Supreme Court will ensure that the proper procedures have been followed.
	\item He helped to put things in their proper place.
	\end{itemize}
}
\item adjective \\
If you say that a way of behaving is \textbf{proper} , you mean that it is considered socially acceptable and right.
 \textit{
	\begin{itemize}
	\item In those days it was not thought entirely proper for a woman to be on the stage.
	\item It is right and proper to do this.
	\end{itemize}
}
\item adjective \\
You can add  \textbf{proper} after a word to indicate that you are referring to the central and most important part of a place, event, or object and want to distinguish it from other things which are not regarded as being important or
central to it.
 \textit{
	\begin{itemize}
	\item A distinction must be made between archaeology proper and science-based archaeology.
	\end{itemize}
}
\end{enumerate}

\section*{skeleton}
{\large \color{blue}  skeletons  }
\subsection*{Explain}
\begin{enumerate}
\item countable noun \\
Your \textbf{skeleton} is the framework of bones in your body.
 \textit{
	\begin{itemize}
	\item ...a human skeleton.
	\end{itemize}
}
\item adjective \\
A \textbf{skeleton}  staff is the smallest number of staff necessary in order to run an organization or service.
 \textit{
	\begin{itemize}
	\item Only a skeleton staff remains to show anyone interested around the site.
	\end{itemize}
}
\item countable noun \\
The \textbf{skeleton} of something such as a building or a plan is its basic framework.
 \textit{
	\begin{itemize}
	\item Only skeletons of buildings remained.
	\item ...a skeleton of policy guidelines.
	\end{itemize}
}
\item  \\
 a skeleton in the cupboard \textit{
	\begin{itemize}
	\end{itemize}
}
\end{enumerate}

\section*{racial}
{\large \color{blue}  }
\subsection*{Explain}
\begin{enumerate}
\item adjective \\
\textbf{Racial}  describes things relating to people's race.
 \textit{
	\begin{itemize}
	\item ...the protection of national and racial minorities.
	\item ...the elimination of racial discrimination.
	\end{itemize}
}
\end{enumerate}

\section*{star}
{\large \color{blue}  stars  starring  starred  }
\subsection*{Explain}
\begin{enumerate}
\item countable noun \\
A \textbf{star} is a large ball of burning gas in space. Stars appear to us as small points of light in the sky on clear nights.
 \textit{
	\begin{itemize}
	\item The night was dark, the stars hidden behind cloud.
	\end{itemize}
}
\item countable noun \\
You can refer to a shape or an object as a \textbf{star} when it has four, five, or more points sticking out of it in a regular  pattern .
 \textit{
	\begin{itemize}
	\item Children at school receive coloured stars for work well done.
	\end{itemize}
}
\item countable noun \\
You can say how many \textbf{stars} something such as a hotel or restaurant has as a way of talking about its quality, which is often indicated by a number of star-shaped symbols. The
more stars something has, the better it is.
 \textit{
	\begin{itemize}
	\item ...five star hotels.
	\end{itemize}
}
\item countable noun \\
Famous  actors , musicians , and sports players are often referred to as \textbf{stars} .
 \textit{
	\begin{itemize}
	\item ...Gemma, 41, star of the TV series Pennies From Heaven.
	\item By now Murphy is Hollywood's top male comedy star.
	\item Some football stars are very wealthy.
	\end{itemize}
}
\item verb \\
If an actor or actress  \textbf{stars in} a play or film, he or she has one of the most important parts in it.
 \textit{
	\begin{itemize}
	\item I starred in a pantomime called Puss in Boots.
	\item He's starred in dozens of films.
	\end{itemize}
}
\item verb \\
If a play or film \textbf{stars} a famous actor or actress, he or she has one of the most important parts in it.
 \textit{
	\begin{itemize}
	\item The comedy starred the young actor as an Irish policeman.
	\item ...a Hollywood film, The Secret of Santa Vittoria, directed by Stanley Kramer and
starring Anthony Quinn.
	\end{itemize}
}
\item plural noun \\
Predictions about people's lives which are based on astrology and appear regularly in a newspaper or magazine are sometimes referred to as \textbf{the}  \textbf{stars} .
 \textit{
	\begin{itemize}
	\item There was nothing in my stars to say I'd have travel problems!
	\end{itemize}
}
\item  \\
 to thank your lucky stars \textit{
	\begin{itemize}
	\end{itemize}
}
\end{enumerate}

\section*{scientific}
{\large \color{blue}  }
\subsection*{Explain}
\begin{enumerate}
\item adjective \\
\textbf{Scientific} is used to describe things that relate to science or to a particular science.
 \textit{
	\begin{itemize}
	\item ...federal financing of basic scientific research , especially in the fields of
health and national security.
	\item ...the use of animals in scientific experiments.
	\item ...scientific instruments.
	\end{itemize}
}
\item adjective \\
If you do something in a \textbf{scientific} way, you do it carefully and thoroughly, using experiments or tests .
 \textit{
	\begin{itemize}
	\item It's not a scientific way to test their opinions.
	\item ...the scientific study of capitalist development.
	\end{itemize}
}
\end{enumerate}

\section*{steel}
{\large \color{blue}  steels  steeling  steeled  }
\subsection*{Explain}
\begin{enumerate}
\item variable noun \\
\textbf{Steel} is a very strong metal which is made mainly from iron. Steel is used for making many things, for example  bridges , buildings, vehicles, and cutlery .
 \textit{
	\begin{itemize}
	\item ...steel pipes.
	\item ...the iron and steel industry.
	\item ...a fall in demand for cement, bricks, steel and glass.
	\item The front wall is made of corrugated steel.
	\end{itemize}
}
\item uncountable noun \\
\textbf{Steel} is used to refer to the industry that produces steel and items made of steel.
 \textit{
	\begin{itemize}
	\item ...a three-month study of European steel.
	\item The company has interests in steel and other products.
	\end{itemize}
}
\item verb \\
If you \textbf{steel}  \textbf{yourself} , you prepare to deal with something unpleasant .
 \textit{
	\begin{itemize}
	\item Those involved are steeling themselves for the coming battle.
	\item I was steeling myself to call round when Simon arrived.
	\end{itemize}
}
\end{enumerate}

\section*{slender}
{\large \color{blue}  }
\subsection*{Explain}
\begin{enumerate}
\item adjective \\
A \textbf{slender} person is attractively thin and graceful .
 \textit{
	\begin{itemize}
	\item She was slender, with delicate wrists and ankles.
	\item ...a tall, slender figure in a straw hat.
	\item He gazed at her slender neck.
	\end{itemize}
}
\item adjective \\
You can use \textbf{slender} to describe a situation which exists but only to a very small degree .
 \textit{
	\begin{itemize}
	\item The United States held a slender lead.
	\item He has won a vote of confidence but only by a slender majority.
	\item ...the first slender hope of peace.
	\end{itemize}
}
\end{enumerate}

\section*{stomach}
{\large \color{blue}  stomachs  stomaching  stomached  }
\subsection*{Explain}
\begin{enumerate}
\item countable noun \\
Your \textbf{stomach} is the organ inside your body where food is digested before it moves into the intestines .
 \textit{
	\begin{itemize}
	\item He had an upset stomach.
	\item My stomach is completely full.
	\end{itemize}
}
\item countable noun \\
You can refer to the front part of your body below your waist as your \textbf{stomach} .
 \textit{
	\begin{itemize}
	\item The children lay down on their stomachs.
	\item ...stomach muscles.
	\end{itemize}
}
\item countable noun \\
If the front part of your body below your waist feels  uncomfortable because you are feeling  worried or frightened , you can refer to it as your \textbf{stomach} .
 \textit{
	\begin{itemize}
	\item His stomach was in knots.
	\end{itemize}
}
\item countable noun \\
If you say that someone has a strong  \textbf{stomach} , you mean that they are not disgusted by things that disgust most other people.
 \textit{
	\begin{itemize}
	\item Surgery often demands actual physical strength, as well as the possession of a strong
stomach.
	\end{itemize}
}
\item verb \\
If you cannot \textbf{stomach} something, you cannot accept it because you dislike it or disapprove of it.
 \textit{
	\begin{itemize}
	\item I could never stomach the cruelty involved in the wounding of animals.
	\end{itemize}
}
\item  \\
 on an empty stomach \textit{
	\begin{itemize}
	\end{itemize}
}
\item  \\
 not have the stomach \textit{
	\begin{itemize}
	\end{itemize}
}
\item  \\
 sick to one's stomach \textit{
	\begin{itemize}
	\end{itemize}
}
\item  \\
 turn someone's stomach \textit{
	\begin{itemize}
	\end{itemize}
}
\end{enumerate}

\section*{slight}
{\large \color{blue}  slighter  slightest  slights  slighting  slighted  }
\subsection*{Explain}
\begin{enumerate}
\item adjective \\
Something that is \textbf{slight} is very small in degree or quantity.
 \textit{
	\begin{itemize}
	\item Doctors say he has made a slight improvement.
	\item We have a slight problem.
	\item A slight smile flickered over his face.
	\item He's not the slightest bit worried.
	\end{itemize}
}
\item adjective \\
A \textbf{slight} person has a fairly thin and delicate looking body.
 \textit{
	\begin{itemize}
	\item She is smaller and slighter than Christie.
	\item ...a slight, bespectacled figure.
	\end{itemize}
}
\item verb \\
If you \textbf{are slighted} , someone does or says something that insults you by treating you as if your views or feelings are not important.
 \textbf{Slight} is also a noun .
 \textit{
	\begin{itemize}
	\item They felt slighted by not being adequately consulted.
	\item It isn't a slight on my husband that I enjoy my evening class.
	\end{itemize}
}
\item  \\
 in the slightest \textit{
	\begin{itemize}
	\end{itemize}
}
\end{enumerate}

\section*{strategy}
{\large \color{blue}  strategies  }
\subsection*{Explain}
\begin{enumerate}
\item variable noun \\
A \textbf{strategy} is a general plan or set of plans intended to achieve something, especially over a long period.
 \textit{
	\begin{itemize}
	\item The group hope to agree a strategy for policing the area.
	\item What should our marketing strategy have achieved?
	\item Community involvement is now integral to company strategy.
	\end{itemize}
}
\item uncountable noun \\
\textbf{Strategy} is the art of planning the best way to gain an advantage or achieve success, especially in war.
 \textit{
	\begin{itemize}
	\item I've just been explaining the basic principles of strategy to my generals.
	\end{itemize}
}
\end{enumerate}

\section*{sly}
{\large \color{blue}  }
\subsection*{Explain}
\begin{enumerate}
\item adjective \\
A \textbf{sly}  look , expression , or remark shows that you know something that other people do not know or that was meant to be a secret .
 \textit{
	\begin{itemize}
	\item His lips were spread in a sly smile.
	\item He gave me a sly, meaningful look.
	\end{itemize}
}
\item adjective \\
If you describe someone as \textbf{sly} , you disapprove of them because they keep their feelings or intentions  hidden and are clever at deceiving people.
 \textit{
	\begin{itemize}
	\item She is devious and sly and manipulative.
	\item He's a sly old beggar if ever there was one.
	\end{itemize}
}
\item  \\
 on the sly \textit{
	\begin{itemize}
	\end{itemize}
}
\end{enumerate}

\section*{sunday}
{\large \color{blue}  Sundays  }
\subsection*{Explain}
\begin{enumerate}
\item variable noun \\
\textbf{Sunday} is the day after Saturday and before Monday .
 \textit{
	\begin{itemize}
	\item I thought we might go for a drive on Sunday.
	\item Naomi went to church in Granville last Sunday.
	\item The buses run every 10 minutes even on Sundays.
	\item It was Sunday afternoon when I got a call from Rob.
	\end{itemize}
}
\end{enumerate}

\section*{sovereign}
{\large \color{blue}  sovereigns  }
\subsection*{Explain}
\begin{enumerate}
\item adjective \\
A \textbf{sovereign} state or country is independent and not under the authority of any other country.
 \textit{
	\begin{itemize}
	\item Lithuania and Armenia signed a treaty in Vilnius recognising each other as independent
sovereign states.
	\item The Federation declared itself to be a sovereign republic.
	\end{itemize}
}
\item adjective \\
\textbf{Sovereign} is used to describe the person or institution that has the highest power in a country.
 \textit{
	\begin{itemize}
	\item Sovereign power will continue to lie with the Supreme People's Assembly.
	\end{itemize}
}
\item countable noun \\
A \textbf{sovereign} is a king , queen , or other royal  ruler of a country.
 \textit{
	\begin{itemize}
	\item The British sovereign is also the head of the Church of England.
	\end{itemize}
}
\end{enumerate}

\section*{thought}
{\large \color{blue}  thoughts  }
\subsection*{Explain}
\begin{enumerate}
\item  \\
\textbf{Thought} is the past  tense and past participle of think .
 \textit{
	\begin{itemize}
	\end{itemize}
}
\item countable noun \\
A \textbf{thought} is an idea that you have in your mind .
 \textit{
	\begin{itemize}
	\item The thought of Nick made her throat tighten.
	\item I tormented myself with the thought that life was just too comfortable.
	\item He pushed the thought from his mind.
	\item I've just had a thought.
	\end{itemize}
}
\item plural noun \\
A person's \textbf{thoughts} are their mind, or all the ideas in their mind when they are concentrating on one particular thing.
 \textit{
	\begin{itemize}
	\item I jumped to my feet so my thoughts wouldn't start to wander.
	\item Usually at this time our thoughts are on Christmas.
	\item If he wasn't there physically, he was always in her thoughts.
	\end{itemize}
}
\item plural noun \\
A person's \textbf{thoughts} are their opinions on a particular subject.
 \textit{
	\begin{itemize}
	\item Many of you have written to us to express your thoughts on the conflict.
	\item Mr Goodman, do you have any thoughts on that?
	\end{itemize}
}
\item uncountable noun \\
\textbf{Thought} is the activity of thinking, especially deeply, carefully, or logically.
 \textit{
	\begin{itemize}
	\item Alice had been so deep in thought that she had walked past her car without even seeing
it.
	\item He had given some thought to what she had told him.
	\item After much thought I decided to end my marriage.
	\item ...the differences between his thought processes and ours.
	\end{itemize}
}
\item countable noun \\
A \textbf{thought} is an intention, hope , or reason for doing something.
 \textit{
	\begin{itemize}
	\item Sarah's first thought was to run back and get Max.
	\item They had no thought of surrender.
	\item Morris has now banished all thoughts of retirement.
	\end{itemize}
}
\item singular noun \\
A \textbf{thought} is an act of kindness or an offer of help ; used especially when you are thanking someone, or expressing  admiration of someone.
 \textit{
	\begin{itemize}
	\item 'Would you like to move into the ward?'—'A kind thought, but no, thank you.'
	\item 'She has given them this seven hundred pounds.' 'What a lovely thought.'
	\end{itemize}
}
\item uncountable noun \\
\textbf{Thought} is the group of ideas and beliefs which belongs, for example , to a particular religion , philosophy , science , or political  party .
 \textit{
	\begin{itemize}
	\item Aristotle's scientific theories dominated Western thought for fifteen hundred years.
	\item This school of thought argues that depression is best treated by drugs.
	\end{itemize}
}
\end{enumerate}

\section*{striking}
{\large \color{blue}  }
\subsection*{Explain}
\begin{enumerate}
\item adjective \\
Something that is \textbf{striking} is very noticeable or unusual .
 \textit{
	\begin{itemize}
	\item The most striking feature of those statistics is the high proportion of suicides.
	\item He bears a striking resemblance to Lenin.
	\item ...her striking good looks.
	\end{itemize}
}
\item adjective \\
Someone who is \textbf{striking} is very attractive , in a noticeable way.
 \textit{
	\begin{itemize}
	\item She was a striking woman with long blonde hair.
	\end{itemize}
}
\end{enumerate}

\section*{throat}
{\large \color{blue}  throats  }
\subsection*{Explain}
\begin{enumerate}
\item countable noun \\
Your \textbf{throat} is the back of your mouth and the top part of the tubes that go down into your stomach and your lungs .
 \textit{
	\begin{itemize}
	\item She had a sore throat.
	\item As she stared at him, she felt her throat go dry.
	\end{itemize}
}
\item countable noun \\
Your \textbf{throat} is the front part of your neck.
 \textit{
	\begin{itemize}
	\item His striped tie was loosened at his throat.
	\end{itemize}
}
\item  \\
 to clear your throat \textit{
	\begin{itemize}
	\end{itemize}
}
\item  \\
 to ram something down someone's throat \textit{
	\begin{itemize}
	\end{itemize}
}
\item  \\
 at each other's throats \textit{
	\begin{itemize}
	\end{itemize}
}
\item  \\
 to stick in your throat \textit{
	\begin{itemize}
	\end{itemize}
}
\end{enumerate}

\section*{terrible}
{\large \color{blue}  }
\subsection*{Explain}
\begin{enumerate}
\item adjective \\
A \textbf{terrible}  experience or situation is very serious or very unpleasant.
 \textit{
	\begin{itemize}
	\item Tens of thousands more suffered terrible injuries in the world's worst industrial
disaster.
	\item I often have the most terrible nightmares.
	\item Prison life, he told me, was terrible.
	\end{itemize}
}
\item graded adjective \\
If you \textbf{feel terrible} , you feel extremely  ill or unhappy . If you tell someone that they \textbf{look terrible} , you mean that they look as if they are extremely ill or unhappy.
 \textit{
	\begin{itemize}
	\item He did feel terrible at the time but seems to be fine now.
	\item Are you all right? You look terrible. Are you sick?
	\end{itemize}
}
\item adjective \\
If something is \textbf{terrible} , it is very bad or of very poor quality.
 \textit{
	\begin{itemize}
	\item She admits her French is terrible.
	\end{itemize}
}
\item adjective \\
You use \textbf{terrible} to emphasize the great  extent or degree of something.
 \textit{
	\begin{itemize}
	\item I was a terrible fool, you know. I remember that now.
	\item Her death is a terrible waste.
	\end{itemize}
}
\end{enumerate}

\section*{thursday}
{\large \color{blue}  Thursdays  }
\subsection*{Explain}
\begin{enumerate}
\item variable noun \\
\textbf{Thursday} is the day after Wednesday and before Friday .
 \textit{
	\begin{itemize}
	\item On Thursday Barrett invited me for a drink.
	\item The local elections will be held this Thursday.
	\item We go and do the weekly shopping every Thursday morning.
	\item I'm always terribly busy on Thursdays.
	\item We go and do the weekly shopping every Thursday morning.
	\end{itemize}
}
\end{enumerate}

\section*{timid}
{\large \color{blue}  }
\subsection*{Explain}
\begin{enumerate}
\item adjective \\
\textbf{Timid} people are shy, nervous , and have no courage or confidence in themselves.
 \textit{
	\begin{itemize}
	\item A timid child, Isabella had learned obedience at an early age.
	\end{itemize}
}
\item adjective \\
If you describe someone's attitudes or actions as \textbf{timid} , you are criticizing them for being too cautious or slow to act, because they are nervous about the possible  consequences of their actions.
 \textit{
	\begin{itemize}
	\item The President's critics say he has been too timid in responding to changing international
developments.
	\item The newspaper called the plan timid and unimaginative.
	\end{itemize}
}
\end{enumerate}

\section*{tremble}
{\large \color{blue}  trembles  trembling  trembled  }
\subsection*{Explain}
\begin{enumerate}
\item verb \\
If you \textbf{tremble} , you shake slightly because you are frightened or cold.
 \textbf{Tremble} is also a noun .
 \textit{
	\begin{itemize}
	\item His mouth became dry, his eyes widened, and he began to tremble all over.
	\item Gil was white and trembling with anger.
	\item With trembling fingers, he removed the camera from his pocket.
	\item I will never forget the look on the patient's face, the tremble in his hand.
	\end{itemize}
}
\item verb \\
If something \textbf{trembles} , it shakes slightly.
 \textit{
	\begin{itemize}
	\item He felt the earth tremble under him.
	\item The leaves trembled in the trees.
	\end{itemize}
}
\item verb \\
If your voice  \textbf{trembles} , it sounds unsteady and uncertain , usually because you are upset or nervous .
 \textbf{Tremble} is also a noun.
 \textit{
	\begin{itemize}
	\item His voice trembled, on the verge of tears.
	\item 'Please understand this,' she began, a tremble in her voice.
	\end{itemize}
}
\end{enumerate}

\section*{valuable}
{\large \color{blue}  }
\subsection*{Explain}
\begin{enumerate}
\item adjective \\
If you describe something or someone as \textbf{valuable} , you mean that they are very useful and helpful .
 \textit{
	\begin{itemize}
	\item Many of our teachers also have valuable academic links with Heidelberg University.
	\item Here are a few valuable tips that will help you look your best.
	\item The experience was very valuable.
	\end{itemize}
}
\item adjective \\
\textbf{Valuable} objects are objects which are worth a lot of money.
 \textit{
	\begin{itemize}
	\item Just because a camera is old does not mean it is valuable.
	\item ...valuable books.
	\end{itemize}
}
\end{enumerate}

\section*{tuesday}
{\large \color{blue}  Tuesdays  }
\subsection*{Explain}
\begin{enumerate}
\item variable noun \\
\textbf{Tuesday} is the day after Monday and before Wednesday .
 \textit{
	\begin{itemize}
	\item He phoned on Tuesday, just before you came.
	\item Talks are likely to start next Tuesday.
	\item On Tuesdays and Saturdays the market comes to town.
	\item They left Zeebrugge on Tuesday evening.
	\end{itemize}
}
\end{enumerate}

\section*{visible}
{\large \color{blue}  }
\subsection*{Explain}
\begin{enumerate}
\item adjective \\
If something is \textbf{visible} , it can be seen .
 \textit{
	\begin{itemize}
	\item The warning lights were clearly visible.
	\item They found a bacterium visible to the human eye.
	\item The meadows are hardly visible from the house.
	\end{itemize}
}
\item adjective \\
You use \textbf{visible} to describe something or someone that people notice or recognize .
 \textit{
	\begin{itemize}
	\item The most visible sign of the intensity of the crisis is unemployment.
	\item He was making a visible effort to control himself.
	\item She has become a highly visible artist.
	\end{itemize}
}
\item adjective \\
In economics , \textbf{visible} earnings are the money that a country makes as a result of producing goods, rather
than from services such as banking and tourism .
 \textit{
	\begin{itemize}
	\item In the U.K. visible imports have traditionally been greater than visible exports.
	\end{itemize}
}
\end{enumerate}

\section*{wednesday}
{\large \color{blue}  Wednesdays  }
\subsection*{Explain}
\begin{enumerate}
\item variable noun \\
\textbf{Wednesday} is the day after Tuesday and before Thursday .
 \textit{
	\begin{itemize}
	\item Come and have supper with us on Wednesday, if you're free.
	\item Did you happen to see her leave last Wednesday?
	\item David always collects Alistair from school on Wednesdays.
	\item On a Wednesday afternoon, the shop was shut.
	\end{itemize}
}
\end{enumerate}

\section*{visual}
{\large \color{blue}  visuals  }
\subsection*{Explain}
\begin{enumerate}
\item adjective \\
\textbf{Visual} means relating to sight , or to things that you can see.
 \textit{
	\begin{itemize}
	\item ...the graphic visual depiction of violence.
	\item ...music, film, dance, and the visual arts.
	\item ...visual jokes.
	\end{itemize}
}
\item countable noun \\
A \textbf{visual} is something such as a picture , diagram , or piece of film that is used to show or explain something.
 \textit{
	\begin{itemize}
	\item Remember you want your visuals to reinforce your message, not detract from what you
are saying.
	\end{itemize}
}
\end{enumerate}

\section*{week}
{\large \color{blue}  weeks  }
\subsection*{Explain}
\begin{enumerate}
\item countable noun \\
A \textbf{week} is a period of seven days. Some people consider that a week starts on Monday and ends on Sunday.
 \textit{
	\begin{itemize}
	\item I had a letter from my mother last week.
	\item This has been on my mind all week.
	\item I know a wonderful restaurant where we can have lunch next week.
	\end{itemize}
}
\item countable noun \\
A \textbf{week} is a period of about seven days.
 \textit{
	\begin{itemize}
	\item Her mother stayed for another two weeks.
	\item Only 12 weeks ago he underwent major heart transplant surgery.
	\item Three million people will visit theatres in the annual six-week season.
	\end{itemize}
}
\item countable noun \\
Your working  \textbf{week} is the hours that you spend at work during a week.
 \textit{
	\begin{itemize}
	\item It is not unusual for women to work a 40-hour week.
	\item ...workers on a three-day week.
	\end{itemize}
}
\item singular noun \\
\textbf{The week} is the part of the week that does not include Saturday and Sunday.
 \textit{
	\begin{itemize}
	\item ...the hard work of looking after the children during the week.
	\item They arrived at the weekend and gave three concerts in the week.
	\end{itemize}
}
\item countable noun \\
You use \textbf{week} in expressions such as 'a week on Monday', 'a week next  Tuesday ', and ' tomorrow week' to mean  exactly one week after the day that you mention .
 \textit{
	\begin{itemize}
	\item The deadline to publish the document is a week tomorrow.
	\item The 800 metre final is on Monday week.
	\end{itemize}
}
\item countable noun \\
You use \textbf{week} in expressions such as 'a week last Monday', 'a week ago this Tuesday', and 'a week ago yesterday ' to mean exactly one week before the day that you mention.
 \textit{
	\begin{itemize}
	\item She returned home to Leicestershire a week last Sunday.
	\end{itemize}
}
\item  \\
 week in week out \textit{
	\begin{itemize}
	\end{itemize}
}
\end{enumerate}

\section*{abrupt}
{\large \color{blue}  }
\subsection*{Explain}
\begin{enumerate}
\item adjective \\
An \textbf{abrupt} change or action is very sudden, often in a way which is unpleasant .
 \textit{
	\begin{itemize}
	\item Rosie's idyllic world came to an abrupt end when her parents' marriage broke up.
	\item The recession brought an abrupt halt to this happiness.
	\item His abrupt departure is bound to raise questions.
	\end{itemize}
}
\item adjective \\
Someone who is \textbf{abrupt} speaks in a rather rude , unfriendly way.
 \textit{
	\begin{itemize}
	\item He was abrupt to the point of rudeness.
	\item Cross was a little taken aback by her abrupt manner.
	\end{itemize}
}
\end{enumerate}

\section*{assurance}
{\large \color{blue}  assurances  }
\subsection*{Explain}
\begin{enumerate}
\item variable noun \\
If you give someone an \textbf{assurance}  \textbf{that} something is true or will happen , you say that it is definitely true or will definitely happen, in order to make them feel less worried .
 \textit{
	\begin{itemize}
	\item He would like an assurance that other forces will not move into the territory.
	\item ...the assurance of being loved and valued as a member of the household.
	\end{itemize}
}
\item uncountable noun \\
If you do something \textbf{with}  \textbf{assurance} , you do it with a feeling of confidence and certainty.
 \textit{
	\begin{itemize}
	\item Masur led the orchestra with assurance.
	\item The E.U. is now acquiring greater assurance and authority.
	\end{itemize}
}
\item uncountable noun \\
\textbf{Assurance} is insurance that provides cover in the event of death.
 \textit{
	\begin{itemize}
	\item ...endowment assurance.
	\end{itemize}
}
\end{enumerate}

\section*{adverse}
{\large \color{blue}  }
\subsection*{Explain}
\begin{enumerate}
\item adjective \\
\textbf{Adverse}  decisions , conditions, or effects are unfavourable to you.
 \textit{
	\begin{itemize}
	\item The police said Mr Hadfield's decision would have no adverse effect on the progress
of the investigation.
	\item Despite the adverse conditions, the road was finished in just eight months.
	\end{itemize}
}
\end{enumerate}

\section*{autumn}
{\large \color{blue}  autumns  }
\subsection*{Explain}
\begin{enumerate}
\item variable noun \\
\textbf{Autumn} is the season between summer and winter when the weather becomes cooler and the leaves fall off the trees.
 \textit{
	\begin{itemize}
	\item We are always plagued by wasps in autumn.
	\item A final vote will take place next autumn.
	\item ...in the autumn of 2000.
	\item Her hair was the colour of autumn leaves.
	\end{itemize}
}
\end{enumerate}

\section*{alive}
{\large \color{blue}  }
\subsection*{Explain}
\begin{enumerate}
\item adjective \\
If people or animals are \textbf{alive} , they are not dead .
 \textit{
	\begin{itemize}
	\item She does not know if he is alive or dead.
	\item They kept her alive on a life support machine.
	\end{itemize}
}
\item adjective \\
If you say that someone seems  \textbf{alive} , you mean that they seem to be very lively and to enjoy everything that they do.
 \textit{
	\begin{itemize}
	\item Our relationship made me feel more alive.
	\item I never expected to feel so alive in my life again.
	\end{itemize}
}
\item adjective \\
If an activity , organization , or situation is \textbf{alive} , it continues to exist or function .
 \textit{
	\begin{itemize}
	\item The big factories are trying to stay alive by cutting costs.
	\item Both communities have a tradition of keeping history alive.
	\end{itemize}
}
\item adjective \\
If a place is \textbf{alive}  \textbf{with} something, there are a lot of people or things there and it seems busy or exciting .
 \textit{
	\begin{itemize}
	\item The river was alive with birds.
	\item The street was alive with the sounds of the soldiers.
	\end{itemize}
}
\item adjective \\
If you are \textbf{alive}  \textbf{to} a situation or problem , you are aware of it and realize how important it is.
 \textit{
	\begin{itemize}
	\item You must be alive to opportunity!
	\item He was alive to what he was doing.
	\end{itemize}
}
\item  \\
 come alive \textit{
	\begin{itemize}
	\end{itemize}
}
\item  \\
 come/bring alive \textit{
	\begin{itemize}
	\end{itemize}
}
\item  \\
 to be eaten alive \textit{
	\begin{itemize}
	\end{itemize}
}
\item  \\
 alive and kicking \textit{
	\begin{itemize}
	\end{itemize}
}
\item  \\
 alive and well \textit{
	\begin{itemize}
	\end{itemize}
}
\end{enumerate}

\section*{banana}
{\large \color{blue}  bananas  }
\subsection*{Explain}
\begin{enumerate}
\item variable noun \\
\textbf{Bananas} are long curved fruit with yellow  skins .
 \textit{
	\begin{itemize}
	\item ...a bunch of bananas.
	\end{itemize}
}
\item adjective \\
If someone is behaving in a silly or crazy way, or if they become extremely  angry , you can say that they are going  \textbf{bananas} .
 \textit{
	\begin{itemize}
	\item People went bananas with boredom.
	\item Adamson's going to go bananas on this one.
	\end{itemize}
}
\end{enumerate}

\section*{apt}
{\large \color{blue}  }
\subsection*{Explain}
\begin{enumerate}
\item adjective \\
An \textbf{apt}  remark , description , or choice is especially suitable.
 \textit{
	\begin{itemize}
	\item The words of this report are as apt today as in 1929.
	\item ...an apt description of the situation.
	\end{itemize}
}
\item adjective \\
If someone is \textbf{apt}  \textbf{to} do something, they often do it and so it is likely that they will do it again.
 \textit{
	\begin{itemize}
	\item She was apt to raise her voice and wave her hands about.
	\item This type of weather is apt to be more common in winter.
	\end{itemize}
}
\end{enumerate}

\section*{bowling}
{\large \color{blue}  }
\subsection*{Explain}
\begin{enumerate}
\item uncountable noun \\
\textbf{Bowling} is a game in which you roll a heavy ball down a narrow  track towards a group of wooden objects and try to knock down as many of them as possible .
 \textit{
	\begin{itemize}
	\item I go bowling for relaxation.
	\end{itemize}
}
\item uncountable noun \\
In a sport such as cricket , \textbf{bowling} is the action or activity of bowling the ball towards the batsman.
 \textit{
	\begin{itemize}
	\end{itemize}
}
\end{enumerate}

\section*{beautiful}
{\large \color{blue}  }
\subsection*{Explain}
\begin{enumerate}
\item adjective \\
A \textbf{beautiful} person is very attractive to look at.
 \textit{
	\begin{itemize}
	\item She was a very beautiful woman.
	\item To me he is the most beautiful child in the world.
	\end{itemize}
}
\item adjective \\
If you describe something as \textbf{beautiful} , you mean that it is very attractive or pleasing.
 \textit{
	\begin{itemize}
	\item New England is beautiful.
	\item It was a beautiful morning.
	\item He has beautiful manners.
	\end{itemize}
}
\item adjective \\
You can describe something that someone does as \textbf{beautiful} when they do it very skilfully.
 \textit{
	\begin{itemize}
	\item That's a beautiful shot!
	\end{itemize}
}
\end{enumerate}

\section*{cap}
{\large \color{blue}  caps  capping  capped  }
\subsection*{Explain}
\begin{enumerate}
\item countable noun \\
A \textbf{cap} is a soft, flat hat with a curved part at the front which is called a peak. Caps
are usually worn by men and boys .
 \textit{
	\begin{itemize}
	\item ...a dark-blue baseball cap.
	\end{itemize}
}
\item countable noun \\
A \textbf{cap} is a special hat which is worn as part of a uniform .
 \textit{
	\begin{itemize}
	\item ...a frontier guard in olive-grey uniform and a peaked cap.
	\end{itemize}
}
\item verb \\
If a sports player \textbf{is capped} , they are chosen to represent their country in a team game such as football , rugby , or cricket .
 \textit{
	\begin{itemize}
	\item Rees, 32, has been capped for England 23 times.
	\item ...England's most capped rugby union player.
	\end{itemize}
}
\item countable noun \\
If a sports player represents their country in a team game such as football, rugby,
or cricket, you can say that they have been awarded a \textbf{cap} .
 \textit{
	\begin{itemize}
	\item He will win his first cap for Wales in Sunday's Test match against Australia.
	\end{itemize}
}
\item countable noun \\
You can refer to someone who is representing their country for the first time in a
team game such as football, rugby, or cricket, as a new \textbf{cap} .
 \textit{
	\begin{itemize}
	\item New Zealand - who have one new cap - won last year's fixture 29-9.
	\end{itemize}
}
\item verb \\
If the government \textbf{caps} an organization, council , or budget , it limits the amount of money that the organization or council is allowed to spend , or limits the size of the budget.
 \textit{
	\begin{itemize}
	\item The Secretary of State for Environment has the power to cap councils which spend
excessively.
	\item Nearly half of all local councils face being capped.
	\end{itemize}
}
\item countable noun \\
The \textbf{cap} of a bottle is its lid.
 \textit{
	\begin{itemize}
	\item She unscrewed the cap of her water bottle and gave him a drink.
	\end{itemize}
}
\item countable noun \\
A \textbf{cap} is a circular  rubber device that a woman places inside her vagina to prevent herself from becoming  pregnant .
 \textit{
	\begin{itemize}
	\end{itemize}
}
\item verb \\
If you \textbf{cap} one thing \textbf{with} another, you put the other thing on top.
 \textit{
	\begin{itemize}
	\item They had capped the roof with plywood.
	\item ...homemade scones capped with cream.
	\end{itemize}
}
\item verb \\
If someone says that a good or bad event \textbf{caps} a series of events, they mean it is the final event in the series, and the other events were also good or bad.
 \textit{
	\begin{itemize}
	\item The win capped a fine tournament for the Irish team.
	\end{itemize}
}
\item verb \\
If someone's teeth \textbf{are capped} , covers are fixed over them so that they look better .
 \textit{
	\begin{itemize}
	\item He suddenly smiled, revealing teeth that had recently been capped.
	\item I had my teeth capped.
	\end{itemize}
}
\item countable noun \\
A \textbf{cap} is a small amount of explosive that is wrapped in paper. Caps are often used in toy guns .
 \textit{
	\begin{itemize}
	\end{itemize}
}
\item  \\
 cap in hand \textit{
	\begin{itemize}
	\end{itemize}
}
\end{enumerate}

\section*{brave}
{\large \color{blue}  braver  bravest  braves  braving  braved  }
\subsection*{Explain}
\begin{enumerate}
\item adjective \\
Someone who is \textbf{brave} is willing to do things which are dangerous , and does not show  fear in difficult or dangerous situations .
 \textit{
	\begin{itemize}
	\item He was not brave enough to report the loss of the documents.
	\item ...those brave people who dared to challenge the Stalinist regimes.
	\end{itemize}
}
\item verb \\
If you \textbf{brave}  unpleasant or dangerous conditions, you deliberately expose yourself to them, usually in order to achieve something.
 \textit{
	\begin{itemize}
	\item Thousands have braved icy rain to demonstrate their support.
	\end{itemize}
}
\item countable noun \\
A \textbf{brave} is a young Native American man, especially one who is good at fighting .
 \textit{
	\begin{itemize}
	\end{itemize}
}
\item  \\
 put on a brave face/put a brave face on \textit{
	\begin{itemize}
	\end{itemize}
}
\end{enumerate}

\section*{category}
{\large \color{blue}  categories  }
\subsection*{Explain}
\begin{enumerate}
\item countable noun \\
If people or things are divided into \textbf{categories} , they are divided into groups in such a way that the members of each group are similar
to each other in some way.
 \textit{
	\begin{itemize}
	\item This book clearly falls into the category of fictionalised autobiography.
	\item The tables were organised into six different categories.
	\item Designer wedding dresses make wedding fashion a separate category from mainstream
fashion.
	\end{itemize}
}
\end{enumerate}

\section*{crazy}
{\large \color{blue}  crazier  craziest  crazies  }
\subsection*{Explain}
\begin{enumerate}
\item adjective \\
If you describe someone or something as \textbf{crazy} , you think they are very foolish or strange.
 \textit{
	\begin{itemize}
	\item People thought they were all crazy to try to make money from manufacturing.
	\item That's why he's got so caught up with this crazy idea about Mr. Trancas.
	\item ...that crazy, mixed-up world out there.
	\end{itemize}
}
\item adjective \\
Someone who is \textbf{crazy} is insane.
 \textbf{Crazy} is also a noun .
 \textit{
	\begin{itemize}
	\item If I sat home and worried about all this stuff, I'd go crazy.
	\item He strides around the room beaming like a crazy man.
	\item Outside, mumbling, was one of New York's ever-present crazies.
	\end{itemize}
}
\item adjective \\
If you are \textbf{crazy about} something, you are very enthusiastic about it. If you are \textbf{not}  \textbf{crazy}  \textbf{about} something, you do not like it.
 \textbf{Crazy} is also a combining form.
 \textit{
	\begin{itemize}
	\item He's still crazy about both his work and his hobbies.
	\item I'm also not crazy about the initial terms of the deal.
	\item This city is football-crazy and deserves a top side.
	\end{itemize}
}
\item adjective \\
If you are \textbf{crazy about} someone, you are deeply in love with them.
 \textit{
	\begin{itemize}
	\item None of that matters, because we're crazy about each other.
	\end{itemize}
}
\item adjective \\
If something or someone makes you \textbf{crazy} or drives you \textbf{crazy} , they make you extremely annoyed or upset .
 \textit{
	\begin{itemize}
	\item This sitting around is driving me crazy.
	\item When Jock woke up and found you gone he went crazy.
	\end{itemize}
}
\item  \\
 like crazy \textit{
	\begin{itemize}
	\end{itemize}
}
\end{enumerate}

\section*{champagne}
{\large \color{blue}  champagnes  }
\subsection*{Explain}
\begin{enumerate}
\item variable noun \\
\textbf{Champagne} is an expensive French white wine with bubbles in. It is often drunk to celebrate something.
 \textit{
	\begin{itemize}
	\end{itemize}
}
\item  \\
 champagne corks pop \textit{
	\begin{itemize}
	\end{itemize}
}
\end{enumerate}

\section*{delicious}
{\large \color{blue}  }
\subsection*{Explain}
\begin{enumerate}
\item adjective \\
Food that is \textbf{delicious} has a very pleasant taste.
 \textit{
	\begin{itemize}
	\item There's always a wide selection of delicious meals to choose from.
	\item Pecan nuts are delicious both raw and cooked.
	\end{itemize}
}
\item adjective \\
If you describe something as \textbf{delicious} , you mean that it is very pleasant.
 \textit{
	\begin{itemize}
	\item There is a delicious irony in all this.
	\item ...that delicious feeling of surprise.
	\end{itemize}
}
\end{enumerate}

\section*{complication}
{\large \color{blue}  complications  }
\subsection*{Explain}
\begin{enumerate}
\item countable noun \\
A \textbf{complication} is a problem or difficulty that makes a situation harder to deal with.
 \textit{
	\begin{itemize}
	\item The age difference was a complication to the relationship.
	\item An added complication is the growing concern for the environment.
	\end{itemize}
}
\item countable noun \\
A \textbf{complication} is a medical problem that occurs as a result of another illness or disease.
 \textit{
	\begin{itemize}
	\item Blindness is a common complication of diabetes.
	\item He died of complications from a heart attack.
	\end{itemize}
}
\end{enumerate}

\section*{dense}
{\large \color{blue}  denser  densest  }
\subsection*{Explain}
\begin{enumerate}
\item adjective \\
Something that is \textbf{dense} contains a lot of things or people in a small area.
 \textit{
	\begin{itemize}
	\item Where Bucharest now stands, there once was a large, dense forest.
	\item Its fur is short, dense and silky.
	\item They thrust their way through the dense crowd.
	\end{itemize}
}
\item adjective \\
\textbf{Dense}  fog or smoke is difficult to see through because it is very heavy and dark.
 \textit{
	\begin{itemize}
	\item A dense column of smoke rose several miles into the air.
	\end{itemize}
}
\item adjective \\
In science , a \textbf{dense} substance is very heavy in relation to its volume .
 \textit{
	\begin{itemize}
	\item ...a small dense star.
	\end{itemize}
}
\item graded adjective \\
If you describe writing or a film as \textbf{dense} , you mean that it is difficult to understand because it contains a lot of information and ideas .
 \textit{
	\begin{itemize}
	\item His prose is vigorous and dense, occasionally to the point of obscurity.
	\end{itemize}
}
\item adjective \\
If you say that someone is \textbf{dense} , you mean that you think they are stupid and that they take a long time to understand simple things.
 \textit{
	\begin{itemize}
	\item He's not a bad man, just a bit dense.
	\end{itemize}
}
\end{enumerate}

\section*{conservation}
{\large \color{blue}  }
\subsection*{Explain}
\begin{enumerate}
\item uncountable noun \\
\textbf{Conservation} is saving and protecting the environment.
 \textit{
	\begin{itemize}
	\item ...a four-nation regional meeting on elephant conservation.
	\item ...tree-planting and other conservation projects.
	\end{itemize}
}
\item uncountable noun \\
\textbf{Conservation} is saving and protecting historical objects or works of art such as paintings , sculptures , or buildings.
 \textit{
	\begin{itemize}
	\item The second image was discovered during conservation of the painting.
	\end{itemize}
}
\item uncountable noun \\
The \textbf{conservation} of a supply of something is the careful use of it so that it lasts for a long time.
 \textit{
	\begin{itemize}
	\item ...projects aimed at promoting energy conservation.
	\item ...rules concerning the conservation of fishery resources.
	\end{itemize}
}
\end{enumerate}

\section*{different}
{\large \color{blue}  }
\subsection*{Explain}
\begin{enumerate}
\item adjective \\
If two people or things are \textbf{different} , they are not like each other in one or more ways.
 In British English, people sometimes  say that one thing is \textbf{different to} another. Some people consider this use to be incorrect .
 People sometimes say that one thing is \textbf{different than} another. This use is often considered incorrect in British English, but it is acceptable in American English.
 \textit{
	\begin{itemize}
	\item London was different from most European capitals.
	\item If he'd attended music school, how might things have been different?
	\item We have totally different views.
	\item My approach is totally different to his.
	\item We're not really any different than they are.
	\item ...a style of advertising that's different than the rest of the country.
	\end{itemize}
}
\item adjective \\
You use \textbf{different} to indicate that you are talking about two or more separate and distinct things of the same kind .
 \textit{
	\begin{itemize}
	\item Different countries specialised in different products.
	\item The number of calories in different brands of drinks varies enormously.
	\end{itemize}
}
\item adjective \\
You can describe something as \textbf{different} when it is unusual and not like others of the same kind.
 \textit{
	\begin{itemize}
	\item This recipe is certainly interesting and different.
	\end{itemize}
}
\end{enumerate}

\section*{convenience}
{\large \color{blue}  conveniences  }
\subsection*{Explain}
\begin{enumerate}
\item uncountable noun \\
If something is done for your \textbf{convenience} , it is done in a way that is useful or suitable for you.
 \textit{
	\begin{itemize}
	\item He was happy to make a detour for her convenience.
	\item ...the need to put the rights of citizens above the convenience of elected officials.
	\end{itemize}
}
\item countable noun \\
If you describe something as a \textbf{convenience} , you mean that it is very useful.
 \textit{
	\begin{itemize}
	\item Mail order is a convenience for buyers who are too busy to shop.
	\end{itemize}
}
\item countable noun \\
\textbf{Conveniences} are pieces of equipment designed to make your life easier .
 \textit{
	\begin{itemize}
	\item ...21st-century conveniences such as a bathroom for each of its two bedrooms.
	\item ...an apartment with all the modern conveniences.
	\end{itemize}
}
\item countable noun \\
A public \textbf{convenience} is a building containing toilets which is provided in a public place for anyone to use.
 \textit{
	\begin{itemize}
	\item ...the cubicles of a public convenience.
	\end{itemize}
}
\end{enumerate}

\section*{divine}
{\large \color{blue}  divines  divining  divined  }
\subsection*{Explain}
\begin{enumerate}
\item adjective \\
You use \textbf{divine} to describe something that is provided by or relates to a god or goddess .
 \textit{
	\begin{itemize}
	\item He suggested that the civil war had been a divine punishment.
	\item ...divine inspiration.
	\end{itemize}
}
\item countable noun \\
A \textbf{divine} is a priest who specializes in the study of God and religion.
 \textit{
	\begin{itemize}
	\end{itemize}
}
\item graded adjective \\
People use \textbf{divine} to express their pleasure or enjoyment of something.
 \textit{
	\begin{itemize}
	\item 'Isn't it divine?' she said. 'I wish I had the right sort of brooch to lend you for
it.'.
	\item Darling how lovely to see you, you look simply divine.
	\end{itemize}
}
\item verb \\
If you \textbf{divine} something, you discover or learn it by guessing.
 \textit{
	\begin{itemize}
	\item ...the child's ability to divine the needs of its parents and respond to them.
	\item From this he divined that she did not like him much.
	\end{itemize}
}
\item verb \\
If you \textbf{divine} , you try to find underground supplies of water or minerals , using a special rod or pair of rods.
 \textit{
	\begin{itemize}
	\item The only reason I was divining for water was because of the drought.
	\item ...a divining rod.
	\end{itemize}
}
\end{enumerate}

\section*{duplicate}
{\large \color{blue}  duplicates  duplicating  duplicated  }
\subsection*{Explain}
\begin{enumerate}
\item verb \\
If you \textbf{duplicate} something that has already been done , you repeat or copy it.
 \textbf{Duplicate} is also a noun .
 \textit{
	\begin{itemize}
	\item His task will be to duplicate his success overseas here at home.
	\item Scientists hope that their findings may be duplicated elsewhere.
	\item Charles scored again, with an exact duplicate of his first goal.
	\end{itemize}
}
\item verb \\
To \textbf{duplicate} something which has been written , drawn , or recorded onto tape  means to make exact copies of it.
 \textbf{Duplicate} is also a noun.
 \textit{
	\begin{itemize}
	\item She found Ned alone in the photocopy room, duplicating some articles.
	\item ...a business which duplicates video and cinema tapes for the movie makers.
	\item I'm on my way to Switzerland, but I've lost my card. I've got to get a duplicate.
	\end{itemize}
}
\item adjective \\
\textbf{Duplicate} is used to describe things that have been made as an exact copy of other things, usually in order to
 serve the same purpose .
 \textit{
	\begin{itemize}
	\item He let himself in with a duplicate key.
	\item ...a duplicate copy of the loan contract.
	\end{itemize}
}
\end{enumerate}

\section*{flexible}
{\large \color{blue}  }
\subsection*{Explain}
\begin{enumerate}
\item adjective \\
A \textbf{flexible} object or material can be bent easily without breaking.
 \textit{
	\begin{itemize}
	\item ...brushes with long, flexible bristles.
	\end{itemize}
}
\item adjective \\
Something or someone that is \textbf{flexible} is able to change easily and adapt to different conditions and circumstances as they occur.
 \textit{
	\begin{itemize}
	\item Look for software that's flexible enough for a range of abilities.
	\item ...flexible working hours.
	\end{itemize}
}
\end{enumerate}

\section*{easter}
{\large \color{blue}  Easters  }
\subsection*{Explain}
\begin{enumerate}
\item variable noun \\
\textbf{Easter} is a Christian festival when Jesus Christ's return to life is celebrated. It is celebrated on a Sunday in March or April .
 \textit{
	\begin{itemize}
	\item 'Happy Easter,' he yelled.
	\item ...the first Easter morning.
	\end{itemize}
}
\item variable noun \\
\textbf{Easter} is the period of several days around and including Easter Sunday.
 \textit{
	\begin{itemize}
	\item They usually have a walking holiday at Easter.
	\item She spends her Easter holidays taking groups of children to France.
	\item The government declared Easter Monday a public holiday.
	\end{itemize}
}
\end{enumerate}

\section*{front}
{\large \color{blue}  fronts  fronting  fronted  }
\subsection*{Explain}
\begin{enumerate}
\item countable noun \\
\textbf{The}  \textbf{front}  \textbf{of} something is the part of it that faces you, or that faces forward, or that you normally see or use.
 \textit{
	\begin{itemize}
	\item One man sat in an armchair, and the other sat on the front of the desk.
	\item Stand at the front of the line.
	\item Her cotton dress had ripped down the front.
	\end{itemize}
}
\item countable noun \\
\textbf{The}  \textbf{front}  \textbf{of} a building is the side or part of it that faces the street.
 \textit{
	\begin{itemize}
	\item Attached to the front of the house, there was a large veranda.
	\end{itemize}
}
\item singular noun \\
A person's or animal's \textbf{front} is the part of their body between their head and their legs that is on the opposite
side to their back.
 \textit{
	\begin{itemize}
	\item If you lie your baby on his front, he'll lift his head and chest up.
	\end{itemize}
}
\item adjective \\
\textbf{Front} is used to refer to the side or part of something that is towards the front or nearest to the front.
 \textit{
	\begin{itemize}
	\item I went out there on the front porch.
	\item She was only six and still missing her front teeth.
	\item Children may be tempted to climb into the front seat while the car is in motion.
	\end{itemize}
}
\item adjective \\
The \textbf{front}  page of a newspaper is the outside of the first page, where the main news stories are printed.
 \textit{
	\begin{itemize}
	\item The Guardian's front page carries a photograph of the two foreign ministers.
	\item The story made the front page of most of the newspapers.
	\end{itemize}
}
\item singular noun \\
\textbf{The front} is a road next to the sea in a seaside town.
 \textit{
	\begin{itemize}
	\item ...a stroll on the front.
	\item Amy went out for a last walk along the sea front.
	\end{itemize}
}
\item countable noun \\
In a war, the \textbf{front} is a line where two opposing armies are facing each other.
 \textit{
	\begin{itemize}
	\item Sonja's husband is fighting at the front.
	\end{itemize}
}
\item countable noun \\
If you say that something is happening on a particular \textbf{front} , you mean that it is happening with regard to a particular situation or field of
activity.
 \textit{
	\begin{itemize}
	\item ...research across a wide academic front.
	\item We're moving forward on a variety of fronts.
	\end{itemize}
}
\item countable noun \\
If someone puts on a particular kind of \textbf{front} , they pretend to have a particular quality.
 \textit{
	\begin{itemize}
	\item Michael kept up a brave front both to the world and in his home.
	\item His laugh-a-minute image is just a front to hide his deep unhappiness.
	\end{itemize}
}
\item countable noun \\
An organization or activity that is \textbf{a}  \textbf{front}  \textbf{for} one that is illegal or secret is used to hide it.
 \textit{
	\begin{itemize}
	\item ...a firm later identified by the police as a front for crime syndicates.
	\item He said the present civilian government is just a front for the old military regime.
	\end{itemize}
}
\item countable noun \\
In relation to the weather , a \textbf{front} is a line where a mass of cold air meets a mass of warm air.
 \textit{
	\begin{itemize}
	\item The snow signaled the arrival of a front, and a high-pressure area seemed to be settling
in.
	\item A very active cold front brought dramatic weather changes to Kansas on Wednesday.
	\end{itemize}
}
\item noun, in names \\
\textbf{Front} is often used in the titles of political organizations with a particular aim .
 \textit{
	\begin{itemize}
	\item ...the People's Liberation Front.
	\end{itemize}
}
\item verb \\
A building or an area of land that \textbf{fronts} a particular place or \textbf{fronts}  \textbf{onto} it is next to it and faces it.
 \textit{
	\begin{itemize}
	\item ...real estate, which includes undeveloped land fronting the city convention center.
	\item There are some delightful Victorian houses fronting onto the pavement.
	\item ...quaint cottages fronted by lawns and flowerbeds.
	\end{itemize}
}
\item verb \\
The person who \textbf{fronts} an organization is the most senior person in it.
 \textit{
	\begin{itemize}
	\item He fronted a formidable band of fighters.
	\item The public relations operation has been fronted by Mr Hayward.
	\end{itemize}
}
\item verb \\
The person who \textbf{fronts} a pop group or rock band is the main singer.
 \textit{
	\begin{itemize}
	\item She didn't want to be seen as a token woman fronting a band.
	\item Queen were three great musicians fronted by a showman of genius.
	\end{itemize}
}
\item  \\
 in front \textit{
	\begin{itemize}
	\end{itemize}
}
\item  \\
 in front \textit{
	\begin{itemize}
	\end{itemize}
}
\item  \\
 in front of \textit{
	\begin{itemize}
	\end{itemize}
}
\item  \\
 in front of \textit{
	\begin{itemize}
	\end{itemize}
}
\item  \\
 on the home front/on the domestic front \textit{
	\begin{itemize}
	\end{itemize}
}
\end{enumerate}

\section*{fuse}
{\large \color{blue}  fuses  fusing  fused  }
\subsection*{Explain}
\begin{enumerate}
\item countable noun \\
A \textbf{fuse} is a safety device in an electric plug or circuit. It contains a piece of wire which
melts when there is a fault so that the flow of electricity  stops .
 \textit{
	\begin{itemize}
	\item The fuse blew as he pressed the button to start the motor.
	\item Remove the circuit fuse before beginning electrical work.
	\end{itemize}
}
\item verb \\
When an electric device \textbf{fuses} or when you \textbf{fuse} it, it stops working because of a fault.
 \textit{
	\begin{itemize}
	\item The wire snapped at the wall plug and the light fused.
	\item Rainwater had fused the bulbs.
	\end{itemize}
}
\item countable noun \\
A \textbf{fuse} is a device on a bomb or firework which delays the explosion so that people can move a safe  distance  away .
 \textit{
	\begin{itemize}
	\item A bomb was deactivated at the last moment, after the fuse had been lit.
	\end{itemize}
}
\item verb \\
When things \textbf{fuse} or \textbf{are fused} , they join together physically or chemically, usually to become one thing. You can
 also  say that one thing \textbf{fuses} with another.
 \textit{
	\begin{itemize}
	\item The skull bones fuse between the ages of fifteen and twenty-five.
	\item Conception occurs when a single sperm fuses with an egg.
	\item Manufactured glass is made by fusing various types of sand.
	\item Their solution was to isolate specific clones of B cells and fuse them with cancer
cells.
	\item The flakes seem to fuse together and produce ice crystals.
	\end{itemize}
}
\item verb \\
If something \textbf{fuses} two different qualities, ideas , or things, or if they \textbf{fuse} , they join together, especially in order to form a pleasing or satisfactory  combination .
 \textit{
	\begin{itemize}
	\item His music fused the rhythms of jazz with classical forms.
	\item They have fused two different types of entertainment, the circus and the rock concert.
	\item Past and present fuse.
	\end{itemize}
}
\item  \\
 blow a fuse \textit{
	\begin{itemize}
	\end{itemize}
}
\item  \\
 light the fuse \textit{
	\begin{itemize}
	\end{itemize}
}
\item  \\
 on a short fuse/have a short fuse \textit{
	\begin{itemize}
	\end{itemize}
}
\end{enumerate}

\section*{gracious}
{\large \color{blue}  }
\subsection*{Explain}
\begin{enumerate}
\item adjective \\
If you describe someone, especially someone you think is superior to you, as \textbf{gracious} , you mean that they are very well-mannered and pleasant .
 \textit{
	\begin{itemize}
	\item She is a lovely and gracious woman.
	\end{itemize}
}
\item adjective \\
If you describe the behaviour of someone in a position of authority as \textbf{gracious} , you mean that they behave in a polite and considerate way.
 \textit{
	\begin{itemize}
	\item She closed with a gracious speech of thanks.
	\end{itemize}
}
\item adjective \\
You use \textbf{gracious} to describe the comfortable way of life of wealthy people.
 \textit{
	\begin{itemize}
	\item He drove through the gracious suburbs with the swimming pools and tennis courts.
	\end{itemize}
}
\item  \\
 good gracious \textit{
	\begin{itemize}
	\end{itemize}
}
\end{enumerate}

\section*{gravity}
{\large \color{blue}  }
\subsection*{Explain}
\begin{enumerate}
\item uncountable noun \\
\textbf{Gravity} is the force which causes things to drop to the ground.
 \textit{
	\begin{itemize}
	\item Arrows would continue to fly forward forever were it not for gravity, which brings
them down to earth.
	\end{itemize}
}
\item uncountable noun \\
The \textbf{gravity}  \textbf{of} a situation or event is its extreme importance or seriousness.
 \textit{
	\begin{itemize}
	\item They deserve punishment which matches the gravity of their crime.
	\item Not all acts of vengeance are of equal gravity.
	\end{itemize}
}
\item uncountable noun \\
The \textbf{gravity} of someone's behaviour or speech is the extremely  serious way in which they behave or speak .
 \textit{
	\begin{itemize}
	\item There was an appealing gravity to everything she said.
	\end{itemize}
}
\end{enumerate}

\section*{guarantee}
{\large \color{blue}  guarantees  guaranteeing  guaranteed  }
\subsection*{Explain}
\begin{enumerate}
\item verb \\
If one thing \textbf{guarantees} another, the first is certain to cause the second thing to happen .
 \textit{
	\begin{itemize}
	\item Surplus resources alone do not guarantee growth.
	\item ...a man whose fame guarantees that his calls will nearly always be returned.
	\end{itemize}
}
\item countable noun \\
Something that is a \textbf{guarantee}  \textbf{of} something else makes it certain that it will happen or that it is true .
 \textit{
	\begin{itemize}
	\item A famous old name on a firm is not necessarily a guarantee of quality.
	\item There is still no guarantee that a formula could be found.
	\end{itemize}
}
\item verb \\
If you \textbf{guarantee} something, you promise that it will definitely happen, or that you will do or provide it for someone.
 \textbf{Guarantee} is also a noun .
 \textit{
	\begin{itemize}
	\item Most states guarantee the right to free and adequate education.
	\item All students are guaranteed campus accommodation for their first year.
	\item We guarantee that you will find a community with which to socialise.
	\item We guarantee to refund your money if you are not delighted with your purchase.
	\item ...a guaranteed income of £3.6 million.
	\item The Editor can give no guarantee that they will fulfil their obligations.
	\item California's state Constitution includes a guarantee of privacy.
	\end{itemize}
}
\item countable noun \\
A \textbf{guarantee} is a written promise by a company to replace or repair a product free of charge if it has any faults within a particular time.
 \textit{
	\begin{itemize}
	\item Whatever a guarantee says, if something is faulty, you can still claim your rights
from the shop.
	\item It was still under guarantee.
	\end{itemize}
}
\item verb \\
If a company \textbf{guarantees} its product or work, they provide a guarantee for it.
 \textit{
	\begin{itemize}
	\item Some builders guarantee their work.
	\item All Dreamland's electric blankets are guaranteed for three years.
	\item ... parts of guaranteed quality.
	\end{itemize}
}
\item countable noun \\
A \textbf{guarantee} is money or something valuable which you give to someone to show that you will do what you have promised.
 \textit{
	\begin{itemize}
	\item They had to leave a deposit as a guarantee of returning to do their military service.
	\end{itemize}
}
\end{enumerate}

\section*{gross}
{\large \color{blue}  grosser  grossest  grosses  grossing  grossed  }
\subsection*{Explain}
\begin{enumerate}
\item adjective \\
You use \textbf{gross} to describe something unacceptable or unpleasant to a very great amount, degree , or intensity .
 \textit{
	\begin{itemize}
	\item The company were guilty of gross negligence.
	\item ...an act of gross injustice.
	\end{itemize}
}
\item adjective \\
If you say that someone's speech or behaviour is \textbf{gross} , you think it is very rude or unacceptable.
 \textit{
	\begin{itemize}
	\item He abused the Admiral in the grossest terms.
	\item I feel disgusted and wonder how I could ever have been so gross.
	\end{itemize}
}
\item adjective \\
If you describe something as \textbf{gross} , you think it is very unpleasant.
 \textit{
	\begin{itemize}
	\item I spat them out because they tasted so gross.
	\item He wears really gross holiday outfits.
	\end{itemize}
}
\item adjective \\
If you describe someone as \textbf{gross} , you mean that they are extremely fat and unattractive .
 \textit{
	\begin{itemize}
	\item I only resist things like chocolate if I feel really gross.
	\end{itemize}
}
\item adjective \\
\textbf{Gross} means the total amount of something, especially money, before any has been taken away .
 \textbf{Gross} is also an adverb .
 \textit{
	\begin{itemize}
	\item ...a fixed rate account guaranteeing 10.4% gross interest or 7.8% net until October.
	\item Interest is paid gross, rather than having tax deducted.
	\item ...a father earning £20,000 gross a year.
	\end{itemize}
}
\item adjective \\
\textbf{Gross} means the total amount of something, after all the relevant amounts have been added together.
 \textit{
	\begin{itemize}
	\item National Savings gross sales in June totalled £709 million.
	\end{itemize}
}
\item adjective \\
\textbf{Gross} means the total weight of something, including its container or wrapping .
 \textit{
	\begin{itemize}
	\end{itemize}
}
\item verb \\
If a person or a business \textbf{grosses} a particular amount of money, they earn that amount of money before tax has been
taken away.
 \textit{
	\begin{itemize}
	\item I'm a factory worker who grossed £9,900 last year.
	\item So far the films have grossed more than £590 million.
	\end{itemize}
}
\item number \\
A \textbf{gross} is a group of 144 things.
 \textit{
	\begin{itemize}
	\item He ordered twelve gross of the disks.
	\end{itemize}
}
\end{enumerate}

\section*{importance}
{\large \color{blue}  }
\subsection*{Explain}
\begin{enumerate}
\item uncountable noun \\
The \textbf{importance} of something is its quality of being significant , valued, or necessary in a particular situation .
 \textit{
	\begin{itemize}
	\item We have always stressed the importance of economic reform.
	\item Safety is of paramount importance.
	\end{itemize}
}
\item uncountable noun \\
\textbf{Importance} means having influence , power, or status.
 \textit{
	\begin{itemize}
	\end{itemize}
}
\end{enumerate}

\section*{holy}
{\large \color{blue}  holier  holiest  }
\subsection*{Explain}
\begin{enumerate}
\item adjective \\
If you describe something as \textbf{holy} , you mean that it is considered to be special because it is connected with God or a particular religion .
 \textit{
	\begin{itemize}
	\item To them, as to all Christians, this is a holy place.
	\item ...Yom Kippur, the holiest day in the Jewish calendar.
	\end{itemize}
}
\item adjective \\
A \textbf{holy} person is a religious leader or someone who leads a religious life.
 \textit{
	\begin{itemize}
	\item The Indians think of him as a holy man, a combination of doctor and priest.
	\end{itemize}
}
\item adjective \\
\textbf{Holy} is used in exclamations such as 'Holy cow !' and 'Holy smoke !' to express an emotion such as surprise or panic .
 \textit{
	\begin{itemize}
	\end{itemize}
}
\end{enumerate}

\section*{insurance}
{\large \color{blue}  insurances  }
\subsection*{Explain}
\begin{enumerate}
\item variable noun \\
\textbf{Insurance} is an arrangement in which you pay  money to a company , and they pay money to you if something unpleasant  happens to you, for example if your property is stolen or damaged, or if you get a serious  illness .
 \textit{
	\begin{itemize}
	\item The insurance company paid out for the stolen jewellery and silver.
	\item We recommend that you take out travel insurance on all holidays.
	\end{itemize}
}
\item variable noun \\
If you do something as \textbf{insurance}  \textbf{against} something unpleasant happening , you do it to protect yourself in case the unpleasant thing happens.
 \textit{
	\begin{itemize}
	\item The country needs a defence capability as insurance against the unexpected.
	\end{itemize}
}
\end{enumerate}

\section*{hopeful}
{\large \color{blue}  hopefuls  }
\subsection*{Explain}
\begin{enumerate}
\item adjective \\
If you are \textbf{hopeful} , you are fairly  confident that something that you want to happen  will happen.
 \textit{
	\begin{itemize}
	\item I am hopeful this misunderstanding will be rectified very quickly.
	\item Surgeons were hopeful of saving the sight in Sara's left eye.
	\end{itemize}
}
\item adjective \\
If something such as a sign or event is \textbf{hopeful} , it makes you feel that what you want to happen will happen.
 \textit{
	\begin{itemize}
	\item The result of the election in is yet another hopeful sign that peace could come to
the Middle East.
	\item ...hopeful forecasts that the economy will improve.
	\end{itemize}
}
\item adjective \\
A \textbf{hopeful} action is one that you do in the hope that you will get what you want to get.
 \textit{
	\begin{itemize}
	\item We've chartered the plane in the hopeful anticipation that the government will allow
them to leave.
	\end{itemize}
}
\item countable noun \\
If you refer to someone as a \textbf{hopeful} , you mean that they are hoping and trying to achieve success in a particular career , election , or competition .
 \textit{
	\begin{itemize}
	\item On the show, young hopefuls are given the opportunity to work in different aspects
of the fashion world.
	\end{itemize}
}
\end{enumerate}

\section*{loan}
{\large \color{blue}  loans  loaning  loaned  }
\subsection*{Explain}
\begin{enumerate}
\item countable noun \\
A \textbf{loan} is a sum of money that you borrow .
 \textit{
	\begin{itemize}
	\item The country has no access to foreign loans or financial aid.
	\item The president wants to make it easier for small businesses to get bank loans.
	\item ...loan repayments.
	\end{itemize}
}
\item singular noun \\
If someone gives you a \textbf{loan of} something, you borrow it from them.
 \textit{
	\begin{itemize}
	\item I am in need of a loan of a bike for a few weeks.
	\item He had offered the loan of his small villa at Cap Ferrat.
	\end{itemize}
}
\item verb \\
If you \textbf{loan} something to someone, you lend it to them.
 \textbf{Loan out} means the same as loan .
 \textit{
	\begin{itemize}
	\item He had kindly offered to loan us all the plants required for the exhibit.
	\item We were approached by the Royal Yachting Association to see if we would loan our
boat to them.
	\item It is common practice for clubs to loan out players to sides in the lower divisions.
	\item The ground was loaned out for numerous events including pop concerts.
	\end{itemize}
}
\item  \\
 on loan \textit{
	\begin{itemize}
	\end{itemize}
}
\item  \\
 on loan \textit{
	\begin{itemize}
	\end{itemize}
}
\end{enumerate}

\section*{imperial}
{\large \color{blue}  }
\subsection*{Explain}
\begin{enumerate}
\item adjective \\
\textbf{Imperial} is used to refer to things or people that are or were connected with an empire.
 \textit{
	\begin{itemize}
	\item ...the Imperial Palace in Tokyo.
	\item They executed Russia's imperial family in 1918.
	\end{itemize}
}
\item adjective \\
The \textbf{imperial} system of measurement uses inches, feet , and yards to measure length, ounces and pounds to measure weight, and pints and gallons to measure volume .
 \textit{
	\begin{itemize}
	\end{itemize}
}
\end{enumerate}

\section*{option}
{\large \color{blue}  options  }
\subsection*{Explain}
\begin{enumerate}
\item countable noun \\
An \textbf{option} is something that you can choose to do in preference to one or more alternatives .
 \textit{
	\begin{itemize}
	\item He's argued from the start that America and its allies are putting too much emphasis
on the military option.
	\item What other options do you have?
	\end{itemize}
}
\item singular noun \\
If you have the \textbf{option} of doing something, you can choose whether to do it or not.
 \textit{
	\begin{itemize}
	\item Criminals are given the option of going to jail or facing public humiliation.
	\item We had no option but to abandon the meeting.
	\end{itemize}
}
\item countable noun \\
In business , an \textbf{option} is an agreement or contract that gives someone the right to buy or sell something such as property or shares at a future date.
 \textit{
	\begin{itemize}
	\item Each bank has granted the other an option on 19.9% of its shares.
	\end{itemize}
}
\item countable noun \\
An \textbf{option} is one of a number of subjects which a student can choose to study as a part of his or her course .
 \textit{
	\begin{itemize}
	\item Several options are offered for the student's senior year.
	\end{itemize}
}
\item  \\
 to keep your options open \textit{
	\begin{itemize}
	\end{itemize}
}
\item  \\
 soft option \textit{
	\begin{itemize}
	\end{itemize}
}
\end{enumerate}

\section*{impressive}
{\large \color{blue}  }
\subsection*{Explain}
\begin{enumerate}
\item adjective \\
Something that is \textbf{impressive}  impresses you, for example because it is great in size or degree , or is done with a great deal of skill .
 \textit{
	\begin{itemize}
	\item It is an impressive achievement.
	\item The film's special effects are particularly impressive.
	\end{itemize}
}
\end{enumerate}

\section*{paragraph}
{\large \color{blue}  paragraphs  }
\subsection*{Explain}
\begin{enumerate}
\item countable noun \\
A \textbf{paragraph} is a section of a piece of writing. A paragraph always  begins on a new line and contains at least one sentence .
 \textit{
	\begin{itemize}
	\item The length of a paragraph depends on the information it conveys.
	\item Paragraph 81 sets out the rules that should apply if a gift is accepted.
	\end{itemize}
}
\end{enumerate}

\section*{indoor}
{\large \color{blue}  }
\subsection*{Explain}
\begin{enumerate}
\item adjective \\
\textbf{Indoor}  activities or things are ones that happen or are used inside a building and not outside.
 \textit{
	\begin{itemize}
	\item No smoking in any indoor facilities.
	\item ...an indoor market.
	\item ...indoor plants.
	\end{itemize}
}
\end{enumerate}

\section*{intimate}
{\large \color{blue}  intimates  intimating  intimated  }
\subsection*{Explain}
\begin{enumerate}
\item adjective \\
If you have an \textbf{intimate} friendship with someone, you know them very well and like them a lot .
 An \textbf{intimate} is an intimate friend.
 \textit{
	\begin{itemize}
	\item I discussed with my intimate friends whether I would immediately have a baby.
	\item They are to have an autumn wedding, an intimate of the couple confides.
	\end{itemize}
}
\item adjective \\
If two people are in an \textbf{intimate} relationship, they are involved with each other in a loving or sexual way.
 \textit{
	\begin{itemize}
	\item I just won't discuss my intimate relationships.
	\item ...their intimate moments with their boyfriends.
	\end{itemize}
}
\item adjective \\
An \textbf{intimate}  conversation or detail , for example , is very personal and private.
 \textit{
	\begin{itemize}
	\item He wrote about the intimate details of his family life.
	\item I hate to interrupt your intimate conversation but we do have an assignment to discuss.
	\end{itemize}
}
\item adjective \\
If you use \textbf{intimate} to describe an occasion or the atmosphere of a place, you like it because it is quiet and pleasant , and seems  suitable for close conversations between friends.
 \textit{
	\begin{itemize}
	\item ...an intimate candlelit dinner for two.
	\end{itemize}
}
\item adjective \\
An \textbf{intimate}  connection between ideas or organizations , for example, is a very strong  link between them.
 \textit{
	\begin{itemize}
	\item ...an intimate connection between madness and wisdom.
	\item France has kept the most intimate links with its former African territories.
	\end{itemize}
}
\item adjective \\
An \textbf{intimate} knowledge of something is a deep and detailed knowledge of it.
 \textit{
	\begin{itemize}
	\item He surprised me with his intimate knowledge of Kierkegaard and Schopenhauer.
	\end{itemize}
}
\item verb \\
If you \textbf{intimate} something, you say it in an indirect way.
 \textit{
	\begin{itemize}
	\item He went on to intimate that he was indeed contemplating a shake-up of the company.
	\item He had intimated to the French and Russians his readiness to come to a settlement.
	\end{itemize}
}
\end{enumerate}

\section*{perfume}
{\large \color{blue}  perfumes  perfuming  perfumed  }
\subsection*{Explain}
\begin{enumerate}
\item variable noun \\
\textbf{Perfume} is a pleasant-smelling liquid which women  put on their skin to make themselves smell  nice .
 \textit{
	\begin{itemize}
	\item The hall smelled of her mother's perfume.
	\item ...a bottle of perfume.
	\item ...the manufacture of soaps and perfumes.
	\end{itemize}
}
\item variable noun \\
\textbf{Perfume} is the ingredient that is added to some products to make them smell nice.
 \textit{
	\begin{itemize}
	\item ...a delicate white soap without perfume.
	\item ...a perfume for skin creams.
	\end{itemize}
}
\item countable noun \\
The \textbf{perfume} of something is the pleasant smell it has.
 \textit{
	\begin{itemize}
	\item ...the perfume of roses.
	\item There were two lemon trees and I paused to enjoy their perfume.
	\end{itemize}
}
\item verb \\
If the smell of something \textbf{perfumes} a place or area, it makes it smell nice.
 \textit{
	\begin{itemize}
	\item Flowers started to perfume the air.
	\item As they bake, they perfume the whole house with the aroma of apples and spices.
	\item ...gardens perfumed with jasmine.
	\end{itemize}
}
\item verb \\
If something is used to \textbf{perfume} a product, it is added to the product to make it smell nice.
 \textit{
	\begin{itemize}
	\item The oil is used to flavour and perfume soaps, foam baths, and scents.
	\item ...shower gel perfumed with the popular Paris fragrance.
	\end{itemize}
}
\end{enumerate}

\section*{limp}
{\large \color{blue}  limps  limping  limped  limper  limpest  }
\subsection*{Explain}
\begin{enumerate}
\item verb \\
If a person or animal \textbf{limps} , they walk with difficulty or in an uneven way because one of their legs or feet is hurt .
 \textbf{Limp} is also a noun .
 \textit{
	\begin{itemize}
	\item I wasn't badly hurt, but I injured my thigh and had to limp.
	\item He had to limp off with a leg injury.
	\item A stiff knee following surgery forced her to walk with a limp.
	\end{itemize}
}
\item verb \\
If you say that something such as an organization, process, or vehicle \textbf{limps}  \textbf{along} , you mean that it continues slowly or with difficulty, for example because it has been weakened or damaged .
 \textit{
	\begin{itemize}
	\item In recent years the newspaper had been limping along on limited resources.
	\item A British battleship, which had been damaged severely in the battle of Crete, came
limping into Pearl Harbor.
	\end{itemize}
}
\item adjective \\
If you describe something as \textbf{limp} , you mean that it is soft or weak when it should be firm or strong .
 \textit{
	\begin{itemize}
	\item She was told to reject applicants with limp handshakes.
	\item A residue can build up on the hair shaft, leaving the hair limp and dull looking.
	\end{itemize}
}
\item adjective \\
If someone is \textbf{limp} , their body has no strength and is not moving, for example because they are asleep or unconscious .
 \textit{
	\begin{itemize}
	\item He carried her limp body into the room and laid her on the bed.
	\item He hit his head against a rock and went limp.
	\end{itemize}
}
\end{enumerate}

\section*{plantation}
{\large \color{blue}  plantations  }
\subsection*{Explain}
\begin{enumerate}
\item countable noun \\
A \textbf{plantation} is a large piece of land, especially in a tropical country, where crops such as rubber, coffee , tea , or sugar are grown.
 \textit{
	\begin{itemize}
	\item ...banana plantations in Costa Rica.
	\end{itemize}
}
\item countable noun \\
A \textbf{plantation} is a large number of trees that have been planted together.
 \textit{
	\begin{itemize}
	\item ...a plantation of almond trees.
	\end{itemize}
}
\end{enumerate}

\section*{linguistic}
{\large \color{blue}  linguistics  }
\subsection*{Explain}
\begin{enumerate}
\item adjective \\
\textbf{Linguistic}  abilities or ideas relate to language or linguistics.
 \textit{
	\begin{itemize}
	\item ...linguistic skills.
	\item ...linguistic theory.
	\end{itemize}
}
\item uncountable noun \\
\textbf{Linguistics} is the study of the way in which language works.
 \textit{
	\begin{itemize}
	\item Modern linguistics emerged as a distinct field in the nineteenth century.
	\item ...applied linguistics.
	\end{itemize}
}
\end{enumerate}

\section*{repair}
{\large \color{blue}  repairs  repairing  repaired  }
\subsection*{Explain}
\begin{enumerate}
\item verb \\
If you \textbf{repair} something that has been damaged or is not working properly, you mend it.
 \textit{
	\begin{itemize}
	\item Goldsmith has repaired the roof to ensure the house is wind-proof.
	\item The cost of repairing earthquake damage could be more than seven-thousand-million
dollars.
	\item A woman drove her car to the garage to have it repaired.
	\end{itemize}
}
\item verb \\
If you \textbf{repair} a relationship or someone's reputation after it has been damaged, you do something to improve it.
 \textit{
	\begin{itemize}
	\item The government continued to try to repair the damage caused by the minister's interview.
	\item The first and most important thing was to repair my relationship with my father.
	\end{itemize}
}
\item variable noun \\
A \textbf{repair} is something that you do to mend a machine , building , piece of clothing , or other thing that has been damaged or is not working properly.
 \textit{
	\begin{itemize}
	\item Many people don't know how to carry out repairs on their cars.
	\item Many of the buildings are in need of repair.
	\item There is no doubt now that her relationship is beyond repair.
	\end{itemize}
}
\item verb \\
If someone \textbf{repairs to} a particular place, they go there.
 \textit{
	\begin{itemize}
	\item We then repaired to the pavilion for lunch.
	\end{itemize}
}
\item  \\
 in good/bad repair \textit{
	\begin{itemize}
	\end{itemize}
}
\end{enumerate}

\section*{lively}
{\large \color{blue}  livelier  liveliest  }
\subsection*{Explain}
\begin{enumerate}
\item adjective \\
You can describe someone as \textbf{lively} when they behave in an enthusiastic and cheerful way.
 \textit{
	\begin{itemize}
	\item She had a sweet, lively personality.
	\item Josephine was bright, lively and cheerful.
	\end{itemize}
}
\item adjective \\
A \textbf{lively} event or a \textbf{lively}  discussion , for example , has lots of interesting and exciting things happening or being said in it.
 \textit{
	\begin{itemize}
	\item It turned out to be a very interesting session with a lively debate.
	\item Their 4–1 win in Honduras was a particularly lively affair.
	\end{itemize}
}
\item adjective \\
Someone who has a \textbf{lively} mind is intelligent and interested in a lot of different things.
 \textit{
	\begin{itemize}
	\item She was a very well educated girl with a lively mind, a girl with ambition.
	\item ...her very lively imagination.
	\end{itemize}
}
\item graded adjective \\
A \textbf{lively}  feeling or awareness is a strong or enthusiastic one.
 \textit{
	\begin{itemize}
	\item The papers also show a lively interest in European developments.
	\end{itemize}
}
\end{enumerate}

\section*{repetition}
{\large \color{blue}  repetitions  }
\subsection*{Explain}
\begin{enumerate}
\item variable noun \\
If there is a \textbf{repetition}  \textbf{of} an event, usually an undesirable event, it happens again.
 \textit{
	\begin{itemize}
	\item The government has taken measures to prevent a repetition of last year's confrontation.
	\item They don't want the repetition of the past mistakes.
	\end{itemize}
}
\item variable noun \\
\textbf{Repetition} means using the same words again.
 \textit{
	\begin{itemize}
	\item He could also have cut out much of the repetition and thus saved many pages.
	\end{itemize}
}
\end{enumerate}

\section*{living}
{\large \color{blue}  livings  }
\subsection*{Explain}
\begin{enumerate}
\item countable noun \\
The work that you do for a \textbf{living} is the work that you do in order to earn the money that you need .
 \textit{
	\begin{itemize}
	\item Father never talked about what he did for a living.
	\item He earns his living doing all kinds of things.
	\end{itemize}
}
\item uncountable noun \\
You use \textbf{living} when you are talking about the quality of people's daily lives.
 \textit{
	\begin{itemize}
	\item Olivia has always been a model of healthy living.
	\item ...the stresses of urban living.
	\end{itemize}
}
\item adjective \\
You use \textbf{living} to talk about the places where people relax when they are not working .
 \textit{
	\begin{itemize}
	\item The spacious living quarters were on the second floor.
	\item The study links the main living area to the kitchen.
	\end{itemize}
}
\item plural noun \\
\textbf{The living} are people who are alive, rather than people who have died .
 \textit{
	\begin{itemize}
	\item The young man is dead. We have only to consider the living.
	\end{itemize}
}
\item  \\
 to scrape a living \textit{
	\begin{itemize}
	\end{itemize}
}
\end{enumerate}

\section*{reservation}
{\large \color{blue}  reservations  }
\subsection*{Explain}
\begin{enumerate}
\item variable noun \\
If you have \textbf{reservations}  \textbf{about} something, you are not sure that it is entirely good or right.
 \textit{
	\begin{itemize}
	\item I told him my main reservation about his film was the ending.
	\item After three days, the strikers' demands were met almost without reservation.
	\end{itemize}
}
\item countable noun \\
If you make a \textbf{reservation} , you arrange for something such as a table in a restaurant or a room in a hotel to be kept for you.
 \textit{
	\begin{itemize}
	\item He went to the desk to make a reservation.
	\item Accommodation is restricted so a reservation is essential.
	\end{itemize}
}
\item countable noun \\
A \textbf{reservation} is an area of land that is kept separate for a particular group of people to live in.
 \textit{
	\begin{itemize}
	\item Seventeen thousand Native Americans live on this reservation.
	\end{itemize}
}
\end{enumerate}

\section*{retention}
{\large \color{blue}  }
\subsection*{Explain}
\begin{enumerate}
\item uncountable noun \\
The \textbf{retention}  \textbf{of} something is the keeping of it.
 \textit{
	\begin{itemize}
	\item They supported the retention of a strong central government.
	\end{itemize}
}
\end{enumerate}

\section*{musical}
{\large \color{blue}  musicals  }
\subsection*{Explain}
\begin{enumerate}
\item adjective \\
You use \textbf{musical} to indicate that something is connected with playing or studying music.
 \textit{
	\begin{itemize}
	\item We have a wealth of musical talent in this region.
	\item Stan Getz's musical career spanned five decades.
	\item London's musical life might become as exciting as Berlin's.
	\end{itemize}
}
\item countable noun \\
A \textbf{musical} is a play or film that uses singing and dancing in the story .
 \textit{
	\begin{itemize}
	\item ...London's smash hit musical Miss Saigon.
	\end{itemize}
}
\item adjective \\
Someone who is \textbf{musical} has a natural  ability and interest in music.
 \textit{
	\begin{itemize}
	\item I came from a musical family.
	\end{itemize}
}
\item adjective \\
Sounds that are \textbf{musical} are light and pleasant to hear .
 \textit{
	\begin{itemize}
	\item He had a soft, almost musical voice.
	\end{itemize}
}
\end{enumerate}

\section*{sausage}
{\large \color{blue}  sausages  }
\subsection*{Explain}
\begin{enumerate}
\item variable noun \\
A \textbf{sausage} consists of minced meat, usually pork, mixed with other ingredients and is contained in a tube made of skin or a similar  material .
 \textit{
	\begin{itemize}
	\item ...sausages and chips.
	\end{itemize}
}
\end{enumerate}

\section*{mysterious}
{\large \color{blue}  }
\subsection*{Explain}
\begin{enumerate}
\item adjective \\
Someone or something that is \textbf{mysterious} is strange and is not known about or understood .
 \textit{
	\begin{itemize}
	\item He died in mysterious circumstances.
	\item A mysterious illness confined him to bed for over a month.
	\item The whole thing seems very mysterious.
	\item He began to feel sympathy for this slightly mysterious man.
	\end{itemize}
}
\item adjective \\
If someone is \textbf{mysterious} about something, they deliberately do not talk much about it, sometimes because they want to make people more interested in it.
 \textit{
	\begin{itemize}
	\item As for his job–well, he was very mysterious about it.
	\end{itemize}
}
\end{enumerate}

\section*{science}
{\large \color{blue}  sciences  }
\subsection*{Explain}
\begin{enumerate}
\item uncountable noun \\
\textbf{Science} is the study of the nature and behaviour of natural things and the knowledge that we obtain about them.
 \textit{
	\begin{itemize}
	\item The best discoveries in science are very simple.
	\item ...science and technology.
	\end{itemize}
}
\item countable noun \\
A \textbf{science} is a particular branch of science such as physics , chemistry , or biology .
 \textit{
	\begin{itemize}
	\item Physics is the best example of a science which has developed strong, abstract theories.
	\item ...the science of microbiology.
	\end{itemize}
}
\item countable noun \\
A \textbf{science} is the study of some aspect of human behaviour, for example  sociology or anthropology .
 \textit{
	\begin{itemize}
	\item ...the modern science of psychology.
	\end{itemize}
}
\end{enumerate}

\section*{nervous}
{\large \color{blue}  }
\subsection*{Explain}
\begin{enumerate}
\item adjective \\
If someone is \textbf{nervous} , they are frightened or worried about something that is happening or might  happen , and show this in their behaviour .
 \textit{
	\begin{itemize}
	\item The party has become deeply nervous about its prospects of winning the next election.
	\item She described Mr Hutchinson as nervous and jumpy after his wife's disappearance.
	\end{itemize}
}
\item adjective \\
A \textbf{nervous} person is very tense and easily  upset .
 \textit{
	\begin{itemize}
	\item She was apparently a very nervous woman, and that affected her career.
	\end{itemize}
}
\item adjective \\
A \textbf{nervous}  illness or condition is one that affects your emotions and your mental state.
 \textit{
	\begin{itemize}
	\item The number of nervous disorders was rising in the region.
	\item He developed nervous problems after people began repeatedly correcting him.
	\end{itemize}
}
\end{enumerate}

\section*{scientist}
{\large \color{blue}  scientists  }
\subsection*{Explain}
\begin{enumerate}
\item countable noun \\
A \textbf{scientist} is someone who has studied science and whose job is to teach or do research in science.
 \textit{
	\begin{itemize}
	\item Scientists have collected more data than expected.
	\end{itemize}
}
\end{enumerate}

\section*{nice}
{\large \color{blue}  nicer  nicest  }
\subsection*{Explain}
\begin{enumerate}
\item adjective \\
If you say that something is \textbf{nice} , you mean that you find it attractive , pleasant, or enjoyable.
 \textit{
	\begin{itemize}
	\item I think silk ties can be quite nice.
	\item It's nice to be here together again.
	\item We had a nice meal with a bottle of champagne.
	\end{itemize}
}
\item adjective \\
If you say that it is \textbf{nice of} someone to say or do something, you are saying that they are being kind and thoughtful . This is often used as a way of thanking someone.
 \textit{
	\begin{itemize}
	\item It's awfully nice of you to come all this way to see me.
	\item 'How are your boys?'—'How nice of you to ask.'
	\item This has been so nice, so terribly kind of you.
	\end{itemize}
}
\item adjective \\
If you say that someone is \textbf{nice} , you mean that you like them because they are friendly and pleasant.
 \textit{
	\begin{itemize}
	\item I've met your father and he's rather nice.
	\item He was a nice fellow, very quiet and courteous.
	\end{itemize}
}
\item adjective \\
If you are \textbf{nice}  \textbf{to} people, you are friendly, pleasant, or polite towards them.
 \textit{
	\begin{itemize}
	\item She met Mr and Mrs Ricciardi, who were very nice to her.
	\end{itemize}
}
\item adjective \\
When the weather is \textbf{nice} , it is warm and pleasant.
 \textit{
	\begin{itemize}
	\item He nodded to us and said, 'Nice weather we're having.'
	\end{itemize}
}
\item adjective \\
You can use \textbf{nice} to emphasize a particular quality that you like.
 \textit{
	\begin{itemize}
	\item Once they are a nice dark golden brown, turn them over.
	\item People have got used to nice glossy magazines.
	\item Add the oats to thicken the mixture and stir until it is nice and creamy.
	\item I'll explain it nice and simply so you can understand.
	\end{itemize}
}
\item adjective \\
A \textbf{nice} point or distinction is very clear , precise, and based on good reasoning .
 \textit{
	\begin{itemize}
	\item Those are nice academic arguments, but what about the immediate future?
	\end{itemize}
}
\item adjective \\
You can use \textbf{nice} when you are greeting people. For example , you can say ' \textbf{Nice to meet you} ', ' \textbf{Nice to have met you} ', or ' \textbf{Nice to see you} '.
 \textit{
	\begin{itemize}
	\item Good morning. Nice to meet you and thanks for being with us this weekend.
	\item 'It's so nice to see you,' said Charles.
	\end{itemize}
}
\item  \\
 nice one \textit{
	\begin{itemize}
	\end{itemize}
}
\end{enumerate}

\section*{seed}
{\large \color{blue}  seeds  seeding  seeded  }
\subsection*{Explain}
\begin{enumerate}
\item variable noun \\
A \textbf{seed} is the small, hard part of a plant from which a new plant grows .
 \textit{
	\begin{itemize}
	\item ...a packet of cabbage seed.
	\item I sow the seed in pots of soil-based compost.
	\item ...sunflower seeds.
	\end{itemize}
}
\item verb \\
If you \textbf{seed} a piece of land , you plant seeds in it.
 \textit{
	\begin{itemize}
	\item Men mowed the wide lawns and seeded them.
	\item The primroses should begin to seed themselves down the steep hillside.
	\item ...his newly seeded lawns.
	\end{itemize}
}
\item plural noun \\
You can  refer to the \textbf{seeds of} something when you want to talk about the beginning of a feeling or process that gradually develops and becomes  stronger or more important .
 \textit{
	\begin{itemize}
	\item He raised questions meant to plant seeds of doubts in the minds of jurors.
	\item He considered that there were, in these developments, the seeds of a new moral order.
	\end{itemize}
}
\item countable noun \\
In sports such as tennis or badminton , a \textbf{seed} is a player who has been ranked  according to his or her ability .
 \textit{
	\begin{itemize}
	\item ...He is Wimbledon's top seed and the world No.1.
	\item In the final, the third seed defeated the reigning champion.
	\end{itemize}
}
\item verb \\
When a player or a team \textbf{is seeded} in a sports competition , they are ranked according to their ability.
 \textit{
	\begin{itemize}
	\item In the UEFA Cup the top 16 sides are seeded for the first round.
	\item He could be seeded second at the French Open.
	\item The top four seeded nations are through to the semi-finals.
	\end{itemize}
}
\item  \\
 go to seed \textit{
	\begin{itemize}
	\end{itemize}
}
\item  \\
 to go to seed/run to seed \textit{
	\begin{itemize}
	\end{itemize}
}
\end{enumerate}

\section*{outdoor}
{\large \color{blue}  }
\subsection*{Explain}
\begin{enumerate}
\item adjective \\
\textbf{Outdoor} activities or things happen or are used outside and not in a building.
 \textit{
	\begin{itemize}
	\item If you enjoy outdoor activities, this is the trip for you.
	\item There were outdoor cafes on almost every block.
	\end{itemize}
}
\end{enumerate}

\section*{segment}
{\large \color{blue}  segments  segmenting  segmented  }
\subsection*{Explain}
\begin{enumerate}
\item countable noun \\
A \textbf{segment of} something is one part of it, considered separately from the rest .
 \textit{
	\begin{itemize}
	\item ...the poorer segments of society.
	\item ...the third segment of his journey.
	\end{itemize}
}
\item countable noun \\
A \textbf{segment} of fruit such as an orange or grapefruit is one of the sections into which it is easily divided.
 \textit{
	\begin{itemize}
	\end{itemize}
}
\item countable noun \\
A \textbf{segment} of a circle is one of the two parts into which it is divided when you draw a straight line through it.
 \textit{
	\begin{itemize}
	\end{itemize}
}
\item countable noun \\
A \textbf{segment} of a market is one part of it, considered separately from the rest.
 \textit{
	\begin{itemize}
	\item Three-to-five day cruises are the fastest-growing segment of the market.
	\item Women's tennis is the market leader in a growing market segment–women's sports.
	\end{itemize}
}
\item verb \\
If a company  \textbf{segments} a market, it divides it into separate parts, usually in order to improve marketing opportunities .
 \textit{
	\begin{itemize}
	\item The big multinational companies can segment the world markets into national ones.
	\end{itemize}
}
\end{enumerate}

\section*{outstanding}
{\large \color{blue}  }
\subsection*{Explain}
\begin{enumerate}
\item adjective \\
If you describe someone or something as \textbf{outstanding} , you think that they are very remarkable and impressive .
 \textit{
	\begin{itemize}
	\item Derartu is an outstanding athlete and deserved to win.
	\item ...an area of outstanding natural beauty.
	\item He was outstanding at tennis and golf.
	\end{itemize}
}
\item adjective \\
Money that is \textbf{outstanding} has not yet been paid and is still owed to someone.
 \textit{
	\begin{itemize}
	\item The total debt outstanding is $70 billion.
	\item You have to pay your outstanding bill before joining the scheme.
	\end{itemize}
}
\item adjective \\
\textbf{Outstanding} issues or problems have not yet been resolved .
 \textit{
	\begin{itemize}
	\item We still have some outstanding issues to resolve.
	\end{itemize}
}
\item adjective \\
\textbf{Outstanding}  means very important or obvious .
 \textit{
	\begin{itemize}
	\item The company is an outstanding example of a small business that grew into a big one.
	\item His mother, whose influence on his development was outstanding, came of a distinguished
American family.
	\end{itemize}
}
\end{enumerate}

\section*{selection}
{\large \color{blue}  selections  }
\subsection*{Explain}
\begin{enumerate}
\item uncountable noun \\
\textbf{Selection} is the act of selecting one or more people or things from a group.
 \textit{
	\begin{itemize}
	\item ...Darwin's principles of natural selection.
	\item Dr. Sullivan's selection to head the Department of Health was greeted with satisfaction.
	\item The children have to sit a tough selection test.
	\end{itemize}
}
\item countable noun \\
A \textbf{selection}  \textbf{of} people or things is a set of them that have been selected from a larger group.
 \textit{
	\begin{itemize}
	\item ...this selection of popular songs.
	\item ...a dramatic rendition of selections from Dickens' A Christmas Carol.
	\end{itemize}
}
\item countable noun \\
The \textbf{selection}  \textbf{of} goods in a shop is the particular range of goods that it has available and from which you can choose what you want .
 \textit{
	\begin{itemize}
	\item It offers the widest selection of antiques of every description in a one-day market.
	\end{itemize}
}
\end{enumerate}

\section*{prime}
{\large \color{blue}  primes  priming  primed  }
\subsection*{Explain}
\begin{enumerate}
\item adjective \\
You use \textbf{prime} to describe something that is most important in a situation .
 \textit{
	\begin{itemize}
	\item Political stability, meanwhile, will be a prime concern.
	\item It could be a prime target for guerrilla attack.
	\item The police will see me as the prime suspect!
	\item She is the prime candidate to take over his job.
	\end{itemize}
}
\item adjective \\
You use \textbf{prime} to describe something that is of the best possible quality.
 \textit{
	\begin{itemize}
	\item It was one of the City's prime sites, near the Stock Exchange.
	\end{itemize}
}
\item adjective \\
You use \textbf{prime} to describe an example of a particular kind of thing that is absolutely  typical .
 \textit{
	\begin{itemize}
	\item Marianne North was a prime example of Victorian womanhood of the more adventurous
kind.
	\end{itemize}
}
\item uncountable noun \\
If someone or something is \textbf{in} their \textbf{prime} , they are at the stage in their existence when they are at their strongest , most active , or most successful .
 \textit{
	\begin{itemize}
	\item Maybe I'm just coming into my prime now.
	\item She was in her intellectual prime.
	\item We've had a series of athletes trying to come back well past their prime.
	\item ...young persons in the prime of life.
	\end{itemize}
}
\item verb \\
If you \textbf{prime} someone \textbf{to} do something, you prepare them to do it, for example by giving them information about
it beforehand.
 \textit{
	\begin{itemize}
	\item Claire wished she'd primed Sarah beforehand.
	\item Arnold primed her for her duties.
	\item The press corps was primed to leap to the defense of the fired officials.
	\end{itemize}
}
\item verb \\
If someone \textbf{primes} a bomb or a gun, they prepare it so that it is ready to explode or fire .
 \textit{
	\begin{itemize}
	\item He was priming the bomb to go off in an hour's time.
	\item Tom keeps a primed 10-foot shotgun in his office.
	\end{itemize}
}
\end{enumerate}

\section*{slipper}
{\large \color{blue}  slippers  }
\subsection*{Explain}
\begin{enumerate}
\item countable noun \\
\textbf{Slippers} are loose , soft shoes that you wear at home .
 \textit{
	\begin{itemize}
	\end{itemize}
}
\end{enumerate}

\section*{prominent}
{\large \color{blue}  }
\subsection*{Explain}
\begin{enumerate}
\item adjective \\
Someone who is \textbf{prominent} is important .
 \textit{
	\begin{itemize}
	\item ...a prominent member of the Law Society.
	\item ...the children of very prominent or successful parents.
	\end{itemize}
}
\item adjective \\
Something that is \textbf{prominent} is very noticeable or is an important part of something else.
 \textit{
	\begin{itemize}
	\item Here the window plays a prominent part in the design.
	\item ...Romania's most prominent independent newspaper.
	\end{itemize}
}
\end{enumerate}

\section*{sort}
{\large \color{blue}  sorts  sorting  sorted  }
\subsection*{Explain}
\begin{enumerate}
\item countable noun \\
If you talk about a particular  \textbf{sort}  \textbf{of} something, you are talking about a class of things that have particular features in common and that belong to a larger group of related things.
 \textit{
	\begin{itemize}
	\item What sort of school did you go to?
	\item There are so many different sorts of mushrooms available these days.
	\item A dozen trees of various sorts were planted.
	\item He had a nice, serious sort of smile.
	\item That's just the sort of abuse that he will be investigating.
	\item Eddie was playing a game of some sort.
	\item It is the last time I will take on this sort of work.
	\item Let's have some more articles of this sort.
	\end{itemize}
}
\item singular noun \\
You describe someone as a particular \textbf{sort} when you are describing their character.
 \textit{
	\begin{itemize}
	\item He seemed to be just the right sort for the job.
	\item She was a very vigorous sort of person.
	\item What sort of men were they?
	\end{itemize}
}
\item verb \\
If you \textbf{sort} things, you separate them into different  classes , groups, or places, for example so that you can do different things with them.
 \textit{
	\begin{itemize}
	\item He sorted the materials into their folders.
	\item The students are sorted into three ability groups.
	\item He unlatched the box and sorted through the papers.
	\item I sorted the laundry.
	\end{itemize}
}
\item verb \\
If you get a problem or the details of something \textbf{sorted} , you do what is necessary to solve the problem or organize the details.
 \textit{
	\begin{itemize}
	\item I'm trying to get my script sorted.
	\item These problems have now been sorted.
	\end{itemize}
}
\item  \\
 all sorts \textit{
	\begin{itemize}
	\end{itemize}
}
\item  \\
 of sorts/a sort \textit{
	\begin{itemize}
	\end{itemize}
}
\item  \\
 sort of \textit{
	\begin{itemize}
	\end{itemize}
}
\item  \\
 out of sorts \textit{
	\begin{itemize}
	\end{itemize}
}
\end{enumerate}

\section*{sacred}
{\large \color{blue}  }
\subsection*{Explain}
\begin{enumerate}
\item adjective \\
Something that is \textbf{sacred} is believed to be holy and to have a special  connection with God .
 \textit{
	\begin{itemize}
	\item The owl is sacred for many Californian Indian people.
	\item ...shrines and sacred places.
	\end{itemize}
}
\item adjective \\
Something connected with religion or used in religious ceremonies is described as \textbf{sacred} .
 \textit{
	\begin{itemize}
	\item ...sacred art.
	\item ...sacred songs or music.
	\end{itemize}
}
\item adjective \\
You can describe something as \textbf{sacred} when it is regarded as too important to be changed or interfered with.
 \textit{
	\begin{itemize}
	\item My memories are sacred.
	\item He said the unity of the country was sacred.
	\end{itemize}
}
\end{enumerate}

\section*{technology}
{\large \color{blue}  technologies  }
\subsection*{Explain}
\begin{enumerate}
\item variable noun \\
\textbf{Technology}  refers to methods, systems, and devices which are the result of scientific knowledge being used for practical purposes .
 \textit{
	\begin{itemize}
	\item Technology is changing fast.
	\item They should be allowed to wait for cheaper technologies to be developed.
	\item ...nuclear weapons technology.
	\end{itemize}
}
\end{enumerate}

\section*{steep}
{\large \color{blue}  steeper  steepest  steeps  steeping  steeped  }
\subsection*{Explain}
\begin{enumerate}
\item adjective \\
A \textbf{steep} slope rises at a very sharp  angle and is difficult to go up.
 \textit{
	\begin{itemize}
	\item San Francisco is built on 40 hills and some are very steep.
	\item ...a narrow, steep-sided valley.
	\end{itemize}
}
\item adjective \\
A \textbf{steep}  increase or decrease in something is a very big increase or decrease.
 \textit{
	\begin{itemize}
	\item Consumers are rebelling at steep price increases.
	\end{itemize}
}
\item adjective \\
If you say that the price of something is \textbf{steep} , you mean that it is expensive .
 \textit{
	\begin{itemize}
	\item The annual premium can be a little steep, but will be well worth it if your dog is
injured.
	\end{itemize}
}
\item verb \\
To \textbf{steep} food \textbf{in} a liquid means to put the food in the liquid for some time so that the food gets  flavour from the liquid.
 \textit{
	\begin{itemize}
	\item It's a drink made by steeping pineapple rind in water.
	\item ...green beans steeped in olive oil.
	\end{itemize}
}
\end{enumerate}

\section*{tribe}
{\large \color{blue}  tribes  }
\subsection*{Explain}
\begin{enumerate}
\item countable noun \\
\textbf{Tribe} is sometimes used to refer to a group of people of the same race , language, and customs , especially in a developing country. Some people disapprove of this use.
 \textit{
	\begin{itemize}
	\item ...three-hundred members of the Xhosa tribe.
	\item ...a map of Maryland marked with the names of Indian tribes.
	\end{itemize}
}
\item countable noun \\
You can use \textbf{tribe} to refer to a group of people who are all doing the same thing or who all behave in the same way.
 \textit{
	\begin{itemize}
	\item ...tribes of talented young people.
	\end{itemize}
}
\end{enumerate}

\section*{strenuous}
{\large \color{blue}  }
\subsection*{Explain}
\begin{enumerate}
\item adjective \\
A \textbf{strenuous} activity or action involves a lot of energy or effort.
 \textit{
	\begin{itemize}
	\item Avoid strenuous exercise in the evening.
	\item These trips were strenuous, and the couple did not enjoy them.
	\item Strenuous efforts had been made to improve conditions in the jail.
	\item Despite strenuous objections by the right wing, the grant was agreed.
	\end{itemize}
}
\end{enumerate}

\section*{variety}
{\large \color{blue}  varieties  }
\subsection*{Explain}
\begin{enumerate}
\item uncountable noun \\
If something has \textbf{variety} , it consists of things which are different from each other.
 \textit{
	\begin{itemize}
	\item Susan's idea of freedom was to have variety in her lifestyle.
	\item I know no store anywhere in the world that has such variety and display.
	\item The music itself has so much variety.
	\end{itemize}
}
\item singular noun \\
A \textbf{variety}  \textbf{of} things is a number of different kinds or examples of the same thing.
 \textit{
	\begin{itemize}
	\item West Hampstead has a variety of good shops and supermarkets.
	\item The island offers such a wide variety of scenery and wildlife.
	\item People change their mind for a variety of reasons.
	\end{itemize}
}
\item countable noun \\
A \textbf{variety}  \textbf{of} something is a type of it.
 \textit{
	\begin{itemize}
	\item I'm always pleased to try out a new variety.
	\item She has 12 varieties of old-fashioned roses.
	\end{itemize}
}
\item uncountable noun \\
\textbf{Variety} is a type of entertainment which includes many different kinds of acts in the same
 show .
 \textit{
	\begin{itemize}
	\item ...a variety show of music, comedy, and magic.
	\end{itemize}
}
\end{enumerate}

\section*{urgent}
{\large \color{blue}  }
\subsection*{Explain}
\begin{enumerate}
\item adjective \\
If something is \textbf{urgent} , it needs to be dealt with as soon as possible .
 \textit{
	\begin{itemize}
	\item There is an urgent need for food and water.
	\item He had urgent business in New York.
	\end{itemize}
}
\item adjective \\
If you speak in an \textbf{urgent} way, you show that you are anxious for people to notice something or to do something.
 \textit{
	\begin{itemize}
	\item His voice was low and urgent.
	\item His mother leaned forward and spoke to him in urgent undertones.
	\end{itemize}
}
\end{enumerate}

\section*{weight}
{\large \color{blue}  weights  weighting  weighted  }
\subsection*{Explain}
\begin{enumerate}
\item variable noun \\
The \textbf{weight} of a person or thing is how heavy they are, measured in units such as kilograms,
pounds, or tons .
 \textit{
	\begin{itemize}
	\item What is your height and weight?
	\item This reduced the weight of the load.
	\item Turkeys can reach enormous weights of up to 50 pounds.
	\end{itemize}
}
\item uncountable noun \\
A person's or thing's \textbf{weight} is the fact that they are very heavy.
 \textit{
	\begin{itemize}
	\item His weight was harming his health.
	\item Despite the vehicle's size and weight, it is not difficult to drive.
	\end{itemize}
}
\item singular noun \\
If you move your \textbf{weight} , you change position so that most of the pressure of your body is on a particular
part of your body.
 \textit{
	\begin{itemize}
	\item He shifted his weight from one foot to the other.
	\item He kept the weight from his left leg.
	\end{itemize}
}
\item countable noun \\
\textbf{Weights} are objects which weigh a known amount and which people lift as a form of exercise .
 \textit{
	\begin{itemize}
	\item I was in the gym lifting weights.
	\end{itemize}
}
\item countable noun \\
\textbf{Weights} are metal objects which weigh a known amount and which are used on a set of scales to weigh other things.
 \textit{
	\begin{itemize}
	\end{itemize}
}
\item countable noun \\
You can refer to a heavy object as a \textbf{weight} , especially when you have to lift it.
 \textit{
	\begin{itemize}
	\item Straining to lift heavy weights can lead to a rise in blood pressure.
	\end{itemize}
}
\item verb \\
If you \textbf{weight} something, you make it heavier by adding something to it, for example in order to stop it from moving easily .
 \textit{
	\begin{itemize}
	\item It can be sewn into curtain hems to weight the curtain and so allow it to hang better.
	\end{itemize}
}
\item verb \\
If you \textbf{weight} things, you give them different values according to how important or significant they are.
 \textit{
	\begin{itemize}
	\item ...a computer program which weights the different transitions according to their
likelihood.
	\item Responses were weighted by region to more accurately reflect the population. .
	\end{itemize}
}
\item variable noun \\
If something is given a particular \textbf{weight} , it is given a particular value according to how important or significant it is.
 \textit{
	\begin{itemize}
	\item The scientists involved put different weight on the conclusions of different models.
	\item We had this understanding that courses were roughly the same weight.
	\end{itemize}
}
\item uncountable noun \\
If you talk about \textbf{the weight of} something, you mean that it is large in amount or has great power, which means that
it is difficult to oppose or fight against.
 \textit{
	\begin{itemize}
	\item The weight of expectation was getting to them.
	\item Companies found themselves collapsing under the weight of debts.
	\end{itemize}
}
\item uncountable noun \\
If someone or something gives \textbf{weight} to what a person says , thinks , or does, they emphasize its significance .
 \textit{
	\begin{itemize}
	\item The fact that he is gone has given more weight to fears that he may try to launch
a civil war.
	\item Do you think, perhaps, that what happened today might lend weight to that criticism?
	\end{itemize}
}
\item uncountable noun \\
If you give something or someone \textbf{weight} , you consider them to be very important or influential in a particular situation.
 \textit{
	\begin{itemize}
	\item This might have been avoided had ministers placed more weight on scientific advice.

	\item ...the overwhelming weight Freud assigned parents in our development.
	\end{itemize}
}
\item singular noun \\
If you feel a \textbf{weight} on you, you have a problem or a responsibility that is difficult for you to manage and that you are very worried about.
 \textit{
	\begin{itemize}
	\item The relief was indescribable. A great weight lifted from me.
	\end{itemize}
}
\item  \\
 to carry weight \textit{
	\begin{itemize}
	\end{itemize}
}
\item  \\
 worth your weight in gold \textit{
	\begin{itemize}
	\end{itemize}
}
\item  \\
 to pull your weight \textit{
	\begin{itemize}
	\end{itemize}
}
\item  \\
 to throw your weight about \textit{
	\begin{itemize}
	\end{itemize}
}
\item  \\
 throw one's weight behind \textit{
	\begin{itemize}
	\end{itemize}
}
\end{enumerate}

\section*{absent}
{\large \color{blue}  absents  absenting  absented  }
\subsection*{Explain}
\begin{enumerate}
\item adjective \\
If someone or something is \textbf{absent}  \textbf{from} a place or situation where they should be or where they usually are, they are not there.
 \textit{
	\begin{itemize}
	\item He has been absent from his desk for two weeks.
	\item The pictures, too, were absent from the walls.
	\item Evans was absent without leave from his Hong Kong-based regiment.
	\end{itemize}
}
\item adjective \\
If someone appears  \textbf{absent} , they are not paying attention because they are thinking about something else.
 \textit{
	\begin{itemize}
	\item 'Nothing,' Rosie said in an absent way.
	\end{itemize}
}
\item adjective \\
An \textbf{absent}  parent does not live with his or her children.
 \textit{
	\begin{itemize}
	\item ...absent fathers who fail to pay towards the costs of looking after their children.
	\end{itemize}
}
\item verb \\
If someone \textbf{absents}  \textbf{themselves from} a place where they should be or where they usually are, they do not go there or they do not stay there.
 \textit{
	\begin{itemize}
	\item She was old enough to absent herself from the lunch table if she chose.
	\item He pleaded guilty before a court martial to absenting himself without leave.
	\end{itemize}
}
\item preposition \\
If you say that \textbf{absent} one thing, another thing will  happen , you mean that if the first thing does not happen, the second thing will happen.
 \textit{
	\begin{itemize}
	\item Absent a solution, people like Sue Godfrey will just keep on fighting.
	\end{itemize}
}
\end{enumerate}

\section*{appetite}
{\large \color{blue}  appetites  }
\subsection*{Explain}
\begin{enumerate}
\item variable noun \\
Your \textbf{appetite} is your desire to eat .
 \textit{
	\begin{itemize}
	\item He has a healthy appetite.
	\item Symptoms are a slight fever, headache and loss of appetite.
	\end{itemize}
}
\item countable noun \\
Someone's \textbf{appetite}  \textbf{for} something is their strong desire for it.
 \textit{
	\begin{itemize}
	\item ...his appetite for success.
	\item ...Americans' growing appetite for scandal.
	\item She gave him just enough information to whet his appetite.
	\end{itemize}
}
\end{enumerate}

\section*{absolute}
{\large \color{blue}  absolutes  }
\subsection*{Explain}
\begin{enumerate}
\item adjective \\
\textbf{Absolute} means total and complete.
 \textit{
	\begin{itemize}
	\item It's not really suited to absolute beginners.
	\item A sick person needs absolute confidence and trust in a doctor.
	\end{itemize}
}
\item adjective \\
You use \textbf{absolute} to emphasize something that you are saying .
 \textit{
	\begin{itemize}
	\item About 12 inches wide is the absolute minimum you should consider.
	\item I think it's absolute nonsense.
	\end{itemize}
}
\item adjective \\
An \textbf{absolute}  ruler has complete power and authority over his or her country.
 \textit{
	\begin{itemize}
	\item He ruled with absolute power.
	\item ...the doctrine of absolute monarchy based upon divine right.
	\end{itemize}
}
\item adjective \\
\textbf{Absolute} is used to say that something is definite and will not change even if circumstances change.
 \textit{
	\begin{itemize}
	\item John brought the absolute proof that we needed.
	\item They had given an absolute assurance that it would be kept secret.
	\end{itemize}
}
\item adjective \\
An amount that is expressed in \textbf{absolute} terms is expressed as a fixed amount rather than referring to variable factors such as what you earn or the effects of inflation .
 \textit{
	\begin{itemize}
	\item In absolute terms British wages remain low by European standards.
	\end{itemize}
}
\item adjective \\
\textbf{Absolute} rules and principles are believed to be true , right, or relevant in all situations.
 \textit{
	\begin{itemize}
	\item There are no absolute rules.
	\item ...certain assumptions which are accepted without question as absolute truths.
	\end{itemize}
}
\item countable noun \\
An \textbf{absolute} is a rule or principle that is believed to be true, right, or relevant in all situations.
 \textit{
	\begin{itemize}
	\item We tend to think in absolutes.
	\end{itemize}
}
\end{enumerate}

\section*{bait}
{\large \color{blue}  baits  baiting  baited  }
\subsection*{Explain}
\begin{enumerate}
\item variable noun \\
\textbf{Bait} is food which you put on a hook or in a trap in order to catch fish or animals.
 \textit{
	\begin{itemize}
	\end{itemize}
}
\item verb \\
If you \textbf{bait} a hook or trap, you put bait on it or in it.
 \textit{
	\begin{itemize}
	\item He baited his hook with pie.
	\item The boys dug pits and baited them so that they could spear their prey.
	\item ...baited lures.
	\end{itemize}
}
\item variable noun \\
To use something as \textbf{bait}  means to use it to trick or persuade someone to do something.
 \textit{
	\begin{itemize}
	\item Service stations use petrol as a bait to lure drivers into the restaurants and other
facilities.
	\item Television programmes are essentially bait to attract an audience for advertisements.
	\end{itemize}
}
\item verb \\
If you \textbf{bait} someone, you deliberately try to make them angry by teasing them.
 \textit{
	\begin{itemize}
	\item He delighted in baiting his mother.
	\end{itemize}
}
\item  \\
 take the bait \textit{
	\begin{itemize}
	\end{itemize}
}
\end{enumerate}

\section*{anonymous}
{\large \color{blue}  }
\subsection*{Explain}
\begin{enumerate}
\item adjective \\
If you remain \textbf{anonymous} when you do something, you do not let people know that you were the person who did it.
 \textit{
	\begin{itemize}
	\item You can remain anonymous if you wish.
	\item An anonymous benefactor stepped in to provide the prize money.
	\item ...anonymous phone calls.
	\end{itemize}
}
\item adjective \\
Something that is \textbf{anonymous} does not reveal who you are.
 \textit{
	\begin{itemize}
	\item Of course, that would have to be by anonymous vote.
	\end{itemize}
}
\item adjective \\
If you describe a place as \textbf{anonymous} , you dislike it because it has no unusual or interesting  features and seems  unwelcoming .
 \textit{
	\begin{itemize}
	\item ...the most anonymous part of north-west Washington.
	\item It's nice to stay in a home rather than in an anonymous holiday villa.
	\end{itemize}
}
\end{enumerate}

\section*{basin}
{\large \color{blue}  basins  }
\subsection*{Explain}
\begin{enumerate}
\item countable noun \\
A \textbf{basin} is a large or deep  bowl that you use for holding liquids, or for mixing or storing food.
 A \textbf{basin}  \textbf{of} something such as water is an amount of it that is contained in a basin.
 \textit{
	\begin{itemize}
	\item Place the eggs and sugar in a large basin.
	\item ...a pudding basin.
	\item We were given a basin of water to wash our hands in.
	\end{itemize}
}
\item countable noun \\
A \textbf{basin} is the same as a washbasin .
 \textit{
	\begin{itemize}
	\item ...a cast-iron bath with a matching basin and WC.
	\end{itemize}
}
\item countable noun \\
The \textbf{basin} of a large river is the area of land around it from which streams  run down into it.
 \textit{
	\begin{itemize}
	\item ...the Amazon basin.
	\end{itemize}
}
\item countable noun \\
In geography , a \textbf{basin} is a particular region of the world where the earth's surface is lower than in other places.
 \textit{
	\begin{itemize}
	\item ...countries around the Pacific Basin.
	\end{itemize}
}
\item countable noun \\
A \textbf{basin} is a partially enclosed area of deep water where boats or ships are kept .
 \textit{
	\begin{itemize}
	\end{itemize}
}
\end{enumerate}

\section*{bankrupt}
{\large \color{blue}  bankrupts  bankrupting  bankrupted  }
\subsection*{Explain}
\begin{enumerate}
\item adjective \\
People or organizations that go  \textbf{bankrupt} do not have enough money to pay their debts.
 \textit{
	\begin{itemize}
	\item If the firm cannot sell its products, it will go bankrupt.
	\item He was declared bankrupt after failing to pay a £114m loan guarantee.
	\end{itemize}
}
\item verb \\
To \textbf{bankrupt} a person or organization means to make them go bankrupt.
 \textit{
	\begin{itemize}
	\item The move to the market nearly bankrupted the firm and its director.
	\item Uninsured people can be bankrupted by big medical bills.
	\end{itemize}
}
\item countable noun \\
A \textbf{bankrupt} is a person who has been declared bankrupt by a court of law .
 \textit{
	\begin{itemize}
	\end{itemize}
}
\item adjective \\
If you say that something is \textbf{bankrupt} , you are emphasizing that it lacks any value or worth .
 \textit{
	\begin{itemize}
	\item He really thinks that European civilisation is morally bankrupt.
	\end{itemize}
}
\end{enumerate}

\section*{biscuit}
{\large \color{blue}  biscuits  }
\subsection*{Explain}
\begin{enumerate}
\item countable noun \\
A \textbf{biscuit} is a small flat cake that is crisp and usually sweet.
 \textit{
	\begin{itemize}
	\end{itemize}
}
\item countable noun \\
A \textbf{biscuit} is a small round dry cake that is made with baking  powder , baking soda , or yeast .
 \textit{
	\begin{itemize}
	\end{itemize}
}
\item  \\
 to take the biscuit \textit{
	\begin{itemize}
	\end{itemize}
}
\end{enumerate}

\section*{better}
{\large \color{blue}  betters  bettering  bettered  }
\subsection*{Explain}
\begin{enumerate}
\item  \\
\textbf{Better} is the comparative of good .
 \textit{
	\begin{itemize}
	\end{itemize}
}
\item  \\
\textbf{Better} is the comparative of well2 .
 \textit{
	\begin{itemize}
	\end{itemize}
}
\item adverb \\
If you like one thing \textbf{better}  \textbf{than} another, you like it more.
 \textit{
	\begin{itemize}
	\item I like your interpretation better than the one I was taught.
	\item I'd like nothing better than to join you girls.
	\item They liked it better when it rained.
	\end{itemize}
}
\item adjective \\
If you are \textbf{better} after an illness or injury , you have recovered from it. If you feel  \textbf{better} , you no longer feel so ill .
 \textit{
	\begin{itemize}
	\item He is much better now, he's fine.
	\item The doctors were saying there wasn't much hope of me getting better.
	\end{itemize}
}
\item phrase \\
You use \textbf{had better} or \textbf{'d better} when you are advising , warning , or threatening someone, or expressing an opinion about what should happen .
 In spoken  English , people sometimes use \textbf{better} without 'had' or 'be' before it. It has the same meaning .
 \textit{
	\begin{itemize}
	\item It's half past two. I think we had better go home.
	\item You'd better run if you're going to get your ticket.
	\item He'd better not try to fool me.
	\item You better not say too much aloud.
	\end{itemize}
}
\item pronoun \\
If you say that you expect or deserve  \textbf{better} , you mean that you expect or deserve a higher  standard of achievement , behaviour , or treatment from people than they have shown you.
 \textit{
	\begin{itemize}
	\item We expect better of you in the future.
	\item Our long-suffering mining communities deserve better than this.
	\end{itemize}
}
\item plural noun \\
Your \textbf{betters} are people who have a higher status or rank than you do.
 \textit{
	\begin{itemize}
	\item Sit down and be quiet in front of your elders and betters.
	\end{itemize}
}
\item verb \\
If someone \textbf{betters} a high achievement or standard, they achieve something higher.
 \textit{
	\begin{itemize}
	\item He recorded a time of 4 minutes 23, bettering the old record of 4-24.
	\item As an account of adolescence it could hardly be bettered.
	\end{itemize}
}
\item verb \\
If you \textbf{better} your situation , you improve your social status or the quality of your life . If you \textbf{better}  \textbf{yourself} , you improve your social status.
 \textit{
	\begin{itemize}
	\item Others dreamed of owning land and of bettering their social position.
	\item Our parents chose to come here with the hope of bettering themselves.
	\end{itemize}
}
\item  \\
\textbf{Better} is used to form the comparative of compound  adjectives  beginning with ' good ' and ' well .' For example , the comparative of ' well-off ' is 'better-off.'
 \textit{
	\begin{itemize}
	\end{itemize}
}
\item  \\
 be better doing sth/it is better doing sth \textit{
	\begin{itemize}
	\end{itemize}
}
\item  \\
 to change for the better \textit{
	\begin{itemize}
	\end{itemize}
}
\item  \\
 to get the better of sb \textit{
	\begin{itemize}
	\end{itemize}
}
\item  \\
 to get the better of sb \textit{
	\begin{itemize}
	\end{itemize}
}
\item  \\
 to know better \textit{
	\begin{itemize}
	\end{itemize}
}
\item  \\
 to know better \textit{
	\begin{itemize}
	\end{itemize}
}
\item  \\
 be better off \textit{
	\begin{itemize}
	\end{itemize}
}
\item  \\
 go one better \textit{
	\begin{itemize}
	\end{itemize}
}
\item  \\
 that's better \textit{
	\begin{itemize}
	\end{itemize}
}
\item  \\
 so much the better \textit{
	\begin{itemize}
	\end{itemize}
}
\item  \\
 the bigger/sooner/smaller etc the better \textit{
	\begin{itemize}
	\end{itemize}
}
\item  \\
 the better to do sth \textit{
	\begin{itemize}
	\end{itemize}
}
\item  \\
 to think better of it \textit{
	\begin{itemize}
	\end{itemize}
}
\item  \\
 for better or worse \textit{
	\begin{itemize}
	\end{itemize}
}
\end{enumerate}

\section*{bonus}
{\large \color{blue}  bonuses  }
\subsection*{Explain}
\begin{enumerate}
\item countable noun \\
A \textbf{bonus} is an extra amount of money that is added to someone's pay, usually because they have worked very hard .
 \textit{
	\begin{itemize}
	\item Workers receive a large part of their pay in the form of bonuses and overtime.
	\item ...a £15 bonus.
	\item ...a special bonus payment.
	\end{itemize}
}
\item countable noun \\
A \textbf{bonus} is something good that you get in addition to something else, and which you would not usually expect.
 \textit{
	\begin{itemize}
	\item We felt we might finish third. Any better would be a bonus.
	\item It has the added bonus of containing 30 per cent less fat than ordinary cheese.
	\end{itemize}
}
\item countable noun \\
A \textbf{bonus} is a sum of money that an insurance  company pays to its policyholders, for example a percentage of the company's profits.
 \textit{
	\begin{itemize}
	\item These returns will not be enough to meet the payment of annual bonuses to policyholders.
	\end{itemize}
}
\end{enumerate}

\section*{circular}
{\large \color{blue}  circulars  }
\subsection*{Explain}
\begin{enumerate}
\item adjective \\
Something that is \textbf{circular} is shaped like a circle.
 \textit{
	\begin{itemize}
	\item ...a circular hole twelve feet wide and two feet deep.
	\item Using a circular motion, massage gently.
	\end{itemize}
}
\item adjective \\
A \textbf{circular}  journey or route is one in which you go to a place and return by a different route.
 \textit{
	\begin{itemize}
	\item Both sides of the river can be explored on this circular walk.
	\end{itemize}
}
\item adjective \\
A \textbf{circular} argument or theory is not valid because it uses a statement to prove something which is then used to prove the statement.
 \textit{
	\begin{itemize}
	\end{itemize}
}
\item countable noun \\
A \textbf{circular} is an official letter or advertisement that is sent to a large number of people at the same time.
 \textit{
	\begin{itemize}
	\item The proposal has been widely publicised in information circulars sent to newspapers.
	\end{itemize}
}
\end{enumerate}

\section*{cake}
{\large \color{blue}  cakes  caking  caked  }
\subsection*{Explain}
\begin{enumerate}
\item variable noun \\
A \textbf{cake} is a sweet food made by baking a mixture of flour, eggs, sugar, and fat in an oven . Cakes may be large and cut into slices or small and intended for one person only.
 \textit{
	\begin{itemize}
	\item ...a piece of cake.
	\item Would you like some chocolate cake?
	\item ...little cakes with white icing.
	\end{itemize}
}
\item countable noun \\
Food that is formed into flat round shapes before it is cooked can be referred to as \textbf{cakes} .
 \textit{
	\begin{itemize}
	\item ...fish cakes.
	\item ...home-made potato cakes.
	\end{itemize}
}
\item countable noun \\
A \textbf{cake}  \textbf{of} soap is a small block of it.
 \textit{
	\begin{itemize}
	\item ...a small cake of lime-scented soap.
	\end{itemize}
}
\item verb \\
If something such as blood or mud  \textbf{cakes} , it changes from a thick  liquid to a dry layer or lump .
 \textit{
	\begin{itemize}
	\item The blood had begun to cake and turn brown.
	\end{itemize}
}
\item  \\
 to have your cake and eat it \textit{
	\begin{itemize}
	\end{itemize}
}
\item  \\
 to sell like hot cakes \textit{
	\begin{itemize}
	\end{itemize}
}
\item  \\
 a piece of cake \textit{
	\begin{itemize}
	\end{itemize}
}
\item  \\
 take the cake \textit{
	\begin{itemize}
	\end{itemize}
}
\end{enumerate}

\section*{compact}
{\large \color{blue}  compacts  compacting  compacted  }
\subsection*{Explain}
\begin{enumerate}
\item adjective \\
\textbf{Compact} things are small or take up very little space. You use this word when you think this is a good quality.
 \textit{
	\begin{itemize}
	\item ...my compact office in Washington.
	\item ...the new, more compact Czech government.
	\end{itemize}
}
\item adjective \\
A \textbf{compact} person is small but strong .
 \textit{
	\begin{itemize}
	\item He was compact, probably no taller than me.
	\item He looked physically very powerful, athletic in a compact way.
	\end{itemize}
}
\item adjective \\
A \textbf{compact}  cassette , camera , or car is a small type of cassette, camera, or car.
 \textit{
	\begin{itemize}
	\end{itemize}
}
\item verb \\
To \textbf{compact} something means to press it so that it becomes more solid.
 \textit{
	\begin{itemize}
	\item The Smith boy was compacting the trash.
	\item The soil settles and is compacted by the winter rain.
	\end{itemize}
}
\item countable noun \\
A \textbf{compact} is a small, flat case that contains face powder and a mirror.
 \textit{
	\begin{itemize}
	\end{itemize}
}
\end{enumerate}

\section*{canoe}
{\large \color{blue}  canoes  }
\subsection*{Explain}
\begin{enumerate}
\item countable noun \\
A \textbf{canoe} is a small, narrow boat that you move through the water using a stick with a wide  end  called a paddle.
 \textit{
	\begin{itemize}
	\end{itemize}
}
\end{enumerate}

\section*{confidential}
{\large \color{blue}  }
\subsection*{Explain}
\begin{enumerate}
\item adjective \\
Information that is \textbf{confidential} is meant to be kept secret or private.
 \textit{
	\begin{itemize}
	\item She accused them of leaking confidential information about her private life.
	\item We'll take good care and keep what you've told us strictly confidential, Mr. Lane.
	\end{itemize}
}
\item adjective \\
If you talk to someone in a \textbf{confidential} way, you talk to them quietly because what you are saying is secret or private.
 \textit{
	\begin{itemize}
	\item All of this is delivered in a warm, confidential tone.
	\item His face suddenly turned solemn, his voice confidential.
	\end{itemize}
}
\end{enumerate}

\section*{captive}
{\large \color{blue}  captives  }
\subsection*{Explain}
\begin{enumerate}
\item adjective \\
A \textbf{captive} person or animal is being kept  imprisoned or enclosed .
 A \textbf{captive} is someone who is captive.
 \textit{
	\begin{itemize}
	\item Her heart had begun to pound inside her chest like a captive animal.
	\item He described the difficulties of surviving for four months as a captive.
	\end{itemize}
}
\item adjective \\
A \textbf{captive} audience is a group of people who are not free to leave a certain place and so have to watch or listen . A \textbf{captive}  market is a group of people who cannot choose whether or where to buy things.
 \textit{
	\begin{itemize}
	\item We all performed dances before a captive audience of parents and patrons.
	\item Airlines consider business travellers a captive market.
	\end{itemize}
}
\item  \\
 take sb captive/hold sb captive \textit{
	\begin{itemize}
	\end{itemize}
}
\end{enumerate}

\section*{deadly}
{\large \color{blue}  deadlier  deadliest  }
\subsection*{Explain}
\begin{enumerate}
\item adjective \\
If something is \textbf{deadly} , it is likely or able to cause someone's death, or has already caused someone's death.
 \textit{
	\begin{itemize}
	\item He was acquitted on charges of assault with a deadly weapon.
	\item ...a deadly disease currently affecting dolphins.
	\item Passive smoking can be deadly too.
	\item The authorities are looking into last week's deadly gas explosions.
	\end{itemize}
}
\item adjective \\
If you describe a person or their behaviour as \textbf{deadly} , you mean that they will do or say anything to get what they want , without caring about other people.
 \textit{
	\begin{itemize}
	\item His mother's voice was one he knew; ice cold and deadly.
	\item The Duchess levelled a deadly look at Nikko.
	\end{itemize}
}
\item adjective \\
If you describe someone or something as \textbf{deadly} , you mean that you think they are very dull and boring.
 \textit{
	\begin{itemize}
	\item She finds these parties deadly.
	\end{itemize}
}
\item adverb \\
You can use \textbf{deadly} to emphasize that something has a particular quality, especially an unpleasant or undesirable quality.
 \textit{
	\begin{itemize}
	\item Broadcast news was accurate and reliable but deadly dull.
	\item The north wind was bitter and deadly cold.
	\item The United States had been deadly serious in its threat of military action.
	\end{itemize}
}
\item adjective \\
A \textbf{deadly}  situation has unpleasant or dangerous  consequences .
 \textit{
	\begin{itemize}
	\item ...the deadly combination of low expectations and low achievement.
	\item It is here that most students fall into a subtle and deadly trap.
	\end{itemize}
}
\item adjective \\
\textbf{Deadly}  enemies or rivals  fight or compete with each other in a very aggressive way.
 \textit{
	\begin{itemize}
	\item The two became deadly enemies.
	\item That would make the competition between rival suppliers even deadlier.
	\end{itemize}
}
\item graded adjective \\
In sport , \textbf{deadly} players and actions are extremely skilful and successful .
 \textit{
	\begin{itemize}
	\item ...the fastest and deadliest bowlers in world cricket today.
	\end{itemize}
}
\end{enumerate}

\section*{client}
{\large \color{blue}  clients  }
\subsection*{Explain}
\begin{enumerate}
\item countable noun \\
A \textbf{client} of a professional person or organization is a person or company that receives a service
from them in return for payment .
 \textit{
	\begin{itemize}
	\item ...a solicitor and his client.
	\item The company required clients to pay substantial fees in advance.
	\end{itemize}
}
\end{enumerate}

\section*{destructive}
{\large \color{blue}  }
\subsection*{Explain}
\begin{enumerate}
\item adjective \\
Something that is \textbf{destructive} causes or is capable of causing great damage , harm , or injury .
 \textit{
	\begin{itemize}
	\item ...the awesome destructive power of nuclear weapons.
	\item Guilt can be very destructive.
	\end{itemize}
}
\end{enumerate}

\section*{emperor}
{\large \color{blue}  emperors  }
\subsection*{Explain}
\begin{enumerate}
\item countable noun \\
An \textbf{emperor} is a man who rules an empire or is the head of state in an empire.
 \textit{
	\begin{itemize}
	\end{itemize}
}
\end{enumerate}

\section*{earnest}
{\large \color{blue}  }
\subsection*{Explain}
\begin{enumerate}
\item  \\
 in earnest \textit{
	\begin{itemize}
	\end{itemize}
}
\item adjective \\
\textbf{Earnest} people are very serious and sincere in what they say or do, because they think that their actions and beliefs are important.
 \textit{
	\begin{itemize}
	\item Ella was a pious, earnest woman.
	\item His expression is as earnest when he smiles as when he is arguing.
	\item Despite their earnest efforts, they still struggle to win support.
	\end{itemize}
}
\item  \\
 in earnest \textit{
	\begin{itemize}
	\end{itemize}
}
\end{enumerate}

\section*{envelope}
{\large \color{blue}  envelopes  }
\subsection*{Explain}
\begin{enumerate}
\item countable noun \\
An \textbf{envelope} is the rectangular paper cover in which you send a letter to someone through the post .
 \textit{
	\begin{itemize}
	\end{itemize}
}
\item  \\
 push the envelope \textit{
	\begin{itemize}
	\end{itemize}
}
\end{enumerate}

\section*{elderly}
{\large \color{blue}  }
\subsection*{Explain}
\begin{enumerate}
\item adjective \\
You use \textbf{elderly} as a polite way of saying that someone is old.
 \textbf{The elderly} are people who are old. This use could cause offence .
 \textit{
	\begin{itemize}
	\item ...an elderly couple.
	\item Many of those most affected are elderly.
	\item The elderly are a formidable force in any election.
	\end{itemize}
}
\item graded adjective \\
If you describe an object as \textbf{elderly} , you are referring , often in a humorous way, to the fact that it is rather old or old-fashioned and not as good or efficient as a new one would be.
 \textit{
	\begin{itemize}
	\item Some of those artillery pieces look a little elderly.
	\end{itemize}
}
\end{enumerate}

\section*{escape}
{\large \color{blue}  escapes  escaping  escaped  }
\subsection*{Explain}
\begin{enumerate}
\item verb \\
If you \textbf{escape}  \textbf{from} a place, you succeed in getting away from it.
 \textit{
	\begin{itemize}
	\item A prisoner has escaped from a jail in northern England.
	\item They are reported to have escaped to the other side of the border.
	\item He was fatally wounded as he tried to escape.
	\end{itemize}
}
\item countable noun \\
Someone's \textbf{escape} is the act of escaping from a particular place or situation.
 \textit{
	\begin{itemize}
	\item The man made his escape.
	\end{itemize}
}
\item verb \\
You can say that you \textbf{escape} when you survive something such as an accident .
 \textbf{Escape} is also a noun .
 \textit{
	\begin{itemize}
	\item The two officers were extremely lucky to escape serious injury.
	\item The man's girlfriend managed to escape unhurt.
	\item He narrowly escaped with his life when he was attacked by a bear.
	\item I hear you had a very narrow escape on the bridge.
	\end{itemize}
}
\item countable noun \\
If something is an \textbf{escape} , it is a way of avoiding difficulties or responsibilities .
 \textit{
	\begin{itemize}
	\item But for me television is an escape.
	\item ...an escape from the depressing realities of wartime.
	\end{itemize}
}
\item adjective \\
You can use \textbf{escape} to describe things which allow you to avoid difficulties or problems . For example, an \textbf{escape route} is an activity or opportunity that lets you improve your situation. An \textbf{escape clause} is part of an agreement that allows you to avoid having to do something that you
do not want to do.
 \textit{
	\begin{itemize}
	\item We all need the occasional escape route from the boring, routine aspects of our lives.
	\item This is a wonderful escape clause for dishonest employers everywhere.
	\end{itemize}
}
\item verb \\
If something \textbf{escapes} you or \textbf{escapes} your attention , you do not know about it, do not remember it, or do not notice it.
 \textit{
	\begin{itemize}
	\item It was an actor whose name escapes me for the moment.
	\end{itemize}
}
\item verb \\
When gas, liquid, or heat \textbf{escapes} , it comes out from a pipe , container, or place.
 \textit{
	\begin{itemize}
	\item Leave a vent open to let some moist air escape.
	\end{itemize}
}
\end{enumerate}

\section*{enthusiastic}
{\large \color{blue}  }
\subsection*{Explain}
\begin{enumerate}
\item adjective \\
If you are \textbf{enthusiastic}  \textbf{about} something, you show how much you like or enjoy it by the way that you behave and talk .
 \textit{
	\begin{itemize}
	\item Tom was very enthusiastic about the place.
	\item He knew much about pictures and fine furniture, and was an enthusiastic gardener.
	\end{itemize}
}
\end{enumerate}

\section*{faith}
{\large \color{blue}  faiths  }
\subsection*{Explain}
\begin{enumerate}
\item uncountable noun \\
If you have \textbf{faith}  \textbf{in} someone or something, you feel  confident about their ability or goodness .
 \textit{
	\begin{itemize}
	\item She had placed a great deal of faith in Mr Penleigh.
	\item People have lost faith in the British Parliament.
	\end{itemize}
}
\item countable noun \\
A \textbf{faith} is a particular religion, for example  Christianity , Buddhism , or Islam .
 \textit{
	\begin{itemize}
	\item He established a reputation as a steadfast defender of the Catholic faith.
	\end{itemize}
}
\item uncountable noun \\
\textbf{Faith} is strong religious belief in a particular God.
 \textit{
	\begin{itemize}
	\item Umberto Eco's loss of his own religious faith is reflected in his novels.
	\end{itemize}
}
\item  \\
 break faith with \textit{
	\begin{itemize}
	\end{itemize}
}
\item  \\
 in good faith \textit{
	\begin{itemize}
	\end{itemize}
}
\item  \\
 keep faith with \textit{
	\begin{itemize}
	\end{itemize}
}
\end{enumerate}

\section*{fatal}
{\large \color{blue}  }
\subsection*{Explain}
\begin{enumerate}
\item adjective \\
A \textbf{fatal} action has very undesirable effects.
 \textit{
	\begin{itemize}
	\item She knew it was fatal to try to argue with Stephen.
	\item He made the fatal mistake of compromising early.
	\item It would deal a fatal blow to his fading chances of success.
	\end{itemize}
}
\item adjective \\
A \textbf{fatal}  accident or illness causes someone's death.
 \textit{
	\begin{itemize}
	\item ...the fatal stabbing of a police sergeant.
	\item A hospital spokesman said she had suffered a fatal heart attack.
	\end{itemize}
}
\end{enumerate}

\section*{food}
{\large \color{blue}  foods  }
\subsection*{Explain}
\begin{enumerate}
\item variable noun \\
\textbf{Food} is what people and animals eat.
 \textit{
	\begin{itemize}
	\item Enjoy your food.
	\item ...supplies of food and water.
	\item ...emergency food aid.
	\item ...frozen foods.
	\end{itemize}
}
\item  \\
 be off your food \textit{
	\begin{itemize}
	\end{itemize}
}
\item  \\
 food for thought \textit{
	\begin{itemize}
	\end{itemize}
}
\end{enumerate}

\section*{genuine}
{\large \color{blue}  }
\subsection*{Explain}
\begin{enumerate}
\item adjective \\
\textbf{Genuine} is used to describe people and things that are exactly what they appear to be, and are not false or an imitation .
 \textit{
	\begin{itemize}
	\item He is an inspiration and a genuine hero.
	\item ...genuine leather.
	\item They're convinced the picture is genuine.
	\end{itemize}
}
\item adjective \\
\textbf{Genuine}  refers to things such as emotions that are real and not pretended.
 \textit{
	\begin{itemize}
	\item There was genuine joy in this room.
	\item If this offer is genuine I will gladly accept it.
	\end{itemize}
}
\item adjective \\
If you describe a person as \textbf{genuine} , you approve of them because they are honest , truthful , and sincere in the way they live and in their relationships with other people.
 \textit{
	\begin{itemize}
	\item She is very caring and very genuine.
	\end{itemize}
}
\end{enumerate}

\section*{fountain}
{\large \color{blue}  fountains  }
\subsection*{Explain}
\begin{enumerate}
\item countable noun \\
A \textbf{fountain} is an ornamental  feature in a pool or lake which consists of a long narrow stream of water that is forced up into the air by a pump .
 \textit{
	\begin{itemize}
	\end{itemize}
}
\item countable noun \\
A \textbf{fountain}  \textbf{of} a liquid is an amount of it which is sent up into the air and falls  back .
 \textit{
	\begin{itemize}
	\item The volcano spewed a fountain of molten rock 650 feet in the air.
	\end{itemize}
}
\item countable noun \\
If you describe a person or thing as a \textbf{fountain of} something, you mean they are an important source of it and supply a lot of it.
 \textit{
	\begin{itemize}
	\item You are a fountain of ideas.
	\end{itemize}
}
\end{enumerate}

\section*{gradual}
{\large \color{blue}  }
\subsection*{Explain}
\begin{enumerate}
\item adjective \\
A \textbf{gradual} change or process occurs in small stages over a long period of time, rather than
 suddenly .
 \textit{
	\begin{itemize}
	\item Losing weight is a slow, gradual process.
	\item You can expect her progress at school to be gradual rather than brilliant.
	\end{itemize}
}
\end{enumerate}

\section*{future}
{\large \color{blue}  futures  }
\subsection*{Explain}
\begin{enumerate}
\item singular noun \\
\textbf{The future} is the period of time that will come after the present , or the things that will happen then.
 \textit{
	\begin{itemize}
	\item The spokesman said no decision on the proposal was likely in the immediate future.
	\item He was making plans for the future.
	\item I had little time to think about what the future held for me.
	\end{itemize}
}
\item adjective \\
\textbf{Future} things will happen or exist after the present time.
 \textit{
	\begin{itemize}
	\item By taking action now we can ensure that future generations will not be put at risk.
	\item ...a report on the future role of local government.
	\item ...the future King and Queen.
	\end{itemize}
}
\item countable noun \\
Someone's \textbf{future} , or \textbf{the}  \textbf{future}  \textbf{of} something, is what will happen to them or what they will do after the present time.
 \textit{
	\begin{itemize}
	\item His future as prime minister depends on the outcome of the elections.
	\item Graeme is a supremely talented cricketer with a splendid future in the game.
	\item ...a proposed national conference on the country's political future.
	\item Young people are an investment for our future.
	\end{itemize}
}
\item countable noun \\
If you say that someone or something has \textbf{a}  \textbf{future} , you mean that they are likely to be successful or to survive .
 \textit{
	\begin{itemize}
	\item These abandoned children have now got a future.
	\item There's no future in this relationship.
	\end{itemize}
}
\item plural noun \\
When people trade in \textbf{futures} , they buy  stocks and shares , commodities such as coffee or oil , or foreign  currency at a price that is agreed at the time of purchase for items which are delivered some time in the future.
 \textit{
	\begin{itemize}
	\item This report could spur some buying in corn futures when the market opens today.
	\item Futures prices recovered from sharp early declines to end with moderate losses.
	\end{itemize}
}
\item adjective \\
In grammar , the \textbf{future} tense of a verb is the one used to talk about things that are going to happen. In English, this applies to verb groups consisting of 'will' or ' shall ' and the base form of a verb. The \textbf{future perfect} tense of a verb is used to talk about things that will have happened at some time
in the future.
 \textit{
	\begin{itemize}
	\end{itemize}
}
\item  \\
 in (the) future \textit{
	\begin{itemize}
	\end{itemize}
}
\item  \\
 what the future holds \textit{
	\begin{itemize}
	\end{itemize}
}
\item  \\
 sb's future lies swh \textit{
	\begin{itemize}
	\end{itemize}
}
\end{enumerate}

\section*{hostile}
{\large \color{blue}  }
\subsection*{Explain}
\begin{enumerate}
\item adjective \\
If you are \textbf{hostile}  \textbf{to} another person or an idea , you disagree with them or disapprove of them, often showing this in your behaviour .
 \textit{
	\begin{itemize}
	\item Many people felt he would be hostile to the idea of foreign intervention.
	\item The West has gradually relaxed its hostile attitude to this influential state.
	\item The Governor faced hostile crowds when he visited the town yesterday.
	\end{itemize}
}
\item adjective \\
Someone who is \textbf{hostile} is unfriendly and aggressive .
 \textit{
	\begin{itemize}
	\item They usually relate in a cold and hostile way to the world.
	\item The prisoner eyed him in hostile silence.
	\end{itemize}
}
\item adjective \\
\textbf{Hostile}  situations and conditions make it difficult for you to achieve something.
 \textit{
	\begin{itemize}
	\item ...some of the most hostile climatic conditions in the world.
	\item The world's trading environment is likely to become increasingly hostile.
	\end{itemize}
}
\item adjective \\
A \textbf{hostile} takeover bid is one that is opposed by the company that is being bid for.
 \textit{
	\begin{itemize}
	\item The Malaysian tycoon launched a hostile bid.
	\end{itemize}
}
\item adjective \\
In a war, you use \textbf{hostile} to describe your enemy's forces, organizations, weapons , land, and activities.
 \textit{
	\begin{itemize}
	\item The city is encircled by a hostile army.
	\item They were in hostile territory.
	\item ...hostile aircraft.
	\end{itemize}
}
\end{enumerate}

\section*{ghost}
{\large \color{blue}  ghosts  ghosting  ghosted  }
\subsection*{Explain}
\begin{enumerate}
\item countable noun \\
A \textbf{ghost} is the spirit of a dead person that someone believes they can see or feel .
 \textit{
	\begin{itemize}
	\item ...the ghost of Marie Antoinette.
	\item The village is haunted by the ghosts of the dead children.
	\end{itemize}
}
\item countable noun \\
The \textbf{ghost of} something, especially of something bad that has happened , is the memory of it.
 \textit{
	\begin{itemize}
	\item The Rams have finally laid the ghost of seasons past to rest.
	\item ...the ghost of anti-Americanism.
	\end{itemize}
}
\item singular noun \\
If there is a \textbf{ghost of} something, that thing is so faint or weak that it hardly exists.
 \textit{
	\begin{itemize}
	\item He gave the ghost of a smile.
	\item The sun was warm and there was just a ghost of a breeze from the north-west.
	\end{itemize}
}
\item verb \\
If a book or other piece of writing  \textbf{is ghosted} , it is written by a writer for another person, for example a politician or sportsman , who then publishes it as his or her own work.
 \textit{
	\begin{itemize}
	\item I published his autobiography, which was very competently ghosted by a woman journalist
from the Daily Mail.
	\item I ghosted his weekly rugby column for the Telegraph.
	\end{itemize}
}
\item  \\
 a ghost of a chance \textit{
	\begin{itemize}
	\end{itemize}
}
\item  \\
 to give up the ghost \textit{
	\begin{itemize}
	\end{itemize}
}
\end{enumerate}

\section*{lower}
{\large \color{blue}  lowers  lowering  lowered  }
\subsection*{Explain}
\begin{enumerate}
\item adjective \\
You can use \textbf{lower} to refer to the bottom one of a pair of things.
 \textit{
	\begin{itemize}
	\item She bit her lower lip.
	\item ...the lower deck of the bus.
	\item The upper layer of felt should overlap the lower.
	\item ...the lower of the two holes.
	\end{itemize}
}
\item adjective \\
You can use \textbf{lower} to refer to the bottom part of something.
 \textit{
	\begin{itemize}
	\item Use a small cushion to help give support to the lower back.
	\item ...fires which started in the lower part of a tower block.
	\end{itemize}
}
\item adjective \\
You can use \textbf{lower} to refer to people or things that are less important than similar people or things.
 \textit{
	\begin{itemize}
	\item Already the awards are causing resentment in the lower ranks of council officers.
	\item The nation's highest court reversed the lower court's decision.
	\item The higher orders of society must rule the lower.
	\end{itemize}
}
\item verb \\
If you \textbf{lower} something, you move it slowly downwards .
 \textit{
	\begin{itemize}
	\item Two reporters had to help lower the coffin into the grave.
	\item Sokolowski lowered himself into the black leather chair.
	\item 'No movies of me getting out of the pool, boys.' They dutifully lowered their cameras.
	\end{itemize}
}
\item verb \\
If you \textbf{lower} something, you make it less in amount, degree , value, or quality.
 \textit{
	\begin{itemize}
	\item The bank has lowered interest rates by 2 percent.
	\item This drug lowers cholesterol levels by binding fats in the intestine.
	\end{itemize}
}
\item verb \\
If someone \textbf{lowers} their head or eyes, they look downwards, for example because they are sad or embarrassed .
 \textit{
	\begin{itemize}
	\item She lowered her head and brushed past photographers as she went back inside.
	\item She lowered her gaze to the hands in her lap.
	\end{itemize}
}
\item verb \\
If you say that you would not \textbf{lower}  \textbf{yourself} by doing something, you mean that you would not behave in a way that would make you or other people respect you less.
 \textit{
	\begin{itemize}
	\item Don't lower yourself, don't be the way they are.
	\item I've got no qualms about lowering myself to Lemmer's level to get what I want.
	\end{itemize}
}
\item verb \\
If you \textbf{lower} your voice or if your voice \textbf{lowers} , you speak more quietly .
 \textit{
	\begin{itemize}
	\item The man moved closer, lowering his voice.
	\item His voice lowers confidentially.
	\end{itemize}
}
\end{enumerate}

\section*{grocer}
{\large \color{blue}  grocers  }
\subsection*{Explain}
\begin{enumerate}
\item countable noun \\
A \textbf{grocer} is a shopkeeper who sells foods such as flour , sugar , and tinned foods.
 \textit{
	\begin{itemize}
	\end{itemize}
}
\item countable noun \\
A \textbf{grocer} or a \textbf{grocer's} is a shop where foods such as flour, sugar, and tinned foods are sold.
 \textit{
	\begin{itemize}
	\end{itemize}
}
\end{enumerate}

\section*{malignant}
{\large \color{blue}  }
\subsection*{Explain}
\begin{enumerate}
\item adjective \\
A \textbf{malignant} tumour or disease is out of control and likely to cause death .
 \textit{
	\begin{itemize}
	\item She developed a malignant breast tumour.
	\end{itemize}
}
\item adjective \\
If you say that someone is \textbf{malignant} , you think they are cruel and like to cause harm.
 \textit{
	\begin{itemize}
	\item He said that we were evil, malignant and mean.
	\item ...a community over-run by a malignant minority indulging in crime and violence.
	\end{itemize}
}
\end{enumerate}

\section*{minor}
{\large \color{blue}  minors  minoring  minored  }
\subsection*{Explain}
\begin{enumerate}
\item adjective \\
You use \textbf{minor} when you want to describe something that is less important, serious , or significant than other things in a group or situation.
 \textit{
	\begin{itemize}
	\item She is known in Italy for a number of minor roles in films.
	\item Western officials say the problem is minor, and should be quickly overcome.
	\end{itemize}
}
\item adjective \\
A \textbf{minor}  illness or operation is not likely to be dangerous to someone's life or health .
 \textit{
	\begin{itemize}
	\item Sarah had been plagued continually by a series of minor illnesses.
	\item His mother had to go to the hospital for minor surgery.
	\end{itemize}
}
\item adjective \\
In European music, a \textbf{minor} scale is one in which the third note is three semitones higher than the first.
 \textit{
	\begin{itemize}
	\item ...the unfinished sonata movement in F minor.
	\end{itemize}
}
\item countable noun \\
A \textbf{minor} is a person who is still legally a child. In Britain and most states in the United States, people are minors until they reach the age of eighteen .
 \textit{
	\begin{itemize}
	\item The approach has virtually ended cigarette sales to minors.
	\end{itemize}
}
\item countable noun \\
At a university or college in the United States, a student's \textbf{minor} is a subject that they are studying in addition to their main subject, or major.
 \textit{
	\begin{itemize}
	\end{itemize}
}
\item countable noun \\
At a university or college in the United States, if a student is, for example, a geology  \textbf{minor} , they are studying geology as well as their main subject.
 \textit{
	\begin{itemize}
	\end{itemize}
}
\item verb \\
If a student at a university or college in the United States \textbf{minors}  \textbf{in} a particular subject, they study it in addition to their main subject.
 \textit{
	\begin{itemize}
	\item I'm minoring in computer science.
	\end{itemize}
}
\end{enumerate}

\section*{lawyer}
{\large \color{blue}  lawyers  }
\subsection*{Explain}
\begin{enumerate}
\item countable noun \\
A \textbf{lawyer} is a person who is qualified to advise people about the law and represent them in court.
 \textit{
	\begin{itemize}
	\item Prosecution and defense lawyers are expected to deliver closing arguments next week.
	\end{itemize}
}
\end{enumerate}

\section*{mortal}
{\large \color{blue}  mortals  }
\subsection*{Explain}
\begin{enumerate}
\item adjective \\
If you refer to the fact that people are \textbf{mortal} , you mean that they have to die and cannot live for ever .
 \textit{
	\begin{itemize}
	\item A man is deliberately designed to be mortal. He grows, he ages, and he dies.
	\end{itemize}
}
\item countable noun \\
You can describe someone as a \textbf{mortal} when you want to say that they are an ordinary person.
 \textit{
	\begin{itemize}
	\item Tickets seem unobtainable to the ordinary mortal.
	\item ...impossible needs for any mere mortal to meet.
	\end{itemize}
}
\item adjective \\
You can use \textbf{mortal} to show that something is very serious or may cause death.
 \textit{
	\begin{itemize}
	\item The police were defending themselves and others against mortal danger.
	\item Broadcasting was regarded at the time as the mortal enemy of live music-making.
	\end{itemize}
}
\item adjective \\
You can use \textbf{mortal} to emphasize that a feeling is extremely great or severe .
 \textit{
	\begin{itemize}
	\item When self-esteem is high, we lose our mortal fear of jealousy.
	\end{itemize}
}
\end{enumerate}

\section*{letter}
{\large \color{blue}  letters  lettering  lettered  }
\subsection*{Explain}
\begin{enumerate}
\item countable noun \\
If you write a \textbf{letter} to someone, you write a message on paper and send it to them, usually by post.
 \textit{
	\begin{itemize}
	\item I had received a letter from a very close friend.
	\item ...a letter of resignation.
	\item Our long courtship had been conducted mostly by letter.
	\end{itemize}
}
\item countable noun \\
\textbf{Letters} are written symbols which represent one of the sounds in a language.
 \textit{
	\begin{itemize}
	\item ...the letters of the alphabet.
	\item ...the letter E.
	\end{itemize}
}
\item countable noun \\
If a student  earns a \textbf{letter} in sports or athletics by being part of the university or college  team , they are entitled to wear on their jacket the initial letter of the name of their university or college.
 \textit{
	\begin{itemize}
	\item Valerie earned letters in three sports: volleyball, basketball, and field hockey.
	\end{itemize}
}
\item verb \\
If a student \textbf{letters} in sports or athletics by being part of the university or college team, they are
entitled to wear on their jacket the initial letter of the name of their university
or college.
 \textit{
	\begin{itemize}
	\item Burkoth lettered in soccer.
	\end{itemize}
}
\item  \\
 the letter of the law \textit{
	\begin{itemize}
	\end{itemize}
}
\item  \\
 to the letter \textit{
	\begin{itemize}
	\end{itemize}
}
\end{enumerate}

\section*{naughty}
{\large \color{blue}  naughtier  naughtiest  }
\subsection*{Explain}
\begin{enumerate}
\item adjective \\
If you say that a child is \textbf{naughty} , you mean that they behave  badly or do not do what they are told .
 \textit{
	\begin{itemize}
	\item Girls, you're being very naughty.
	\item You naughty boy, you gave me such a fright.
	\end{itemize}
}
\item adjective \\
You can describe  books , pictures , or words as \textbf{naughty} when they are slightly  rude or related to sex .
 \textit{
	\begin{itemize}
	\item You know what little boys are like with naughty words.
	\item ...saucy TV shows, crammed full of naughty innuendo.
	\end{itemize}
}
\end{enumerate}

\section*{lion}
{\large \color{blue}  lions  }
\subsection*{Explain}
\begin{enumerate}
\item countable noun \\
A \textbf{lion} is a large wild member of the cat family that is found in Africa. Lions have yellowish fur , and male lions have long hair on their head and neck .
 \textit{
	\begin{itemize}
	\end{itemize}
}
\end{enumerate}

\section*{next}
{\large \color{blue}  }
\subsection*{Explain}
\begin{enumerate}
\item ordinal number \\
The \textbf{next}  period of time, event , person, or thing is the one that comes immediately after the present one or after the previous one.
 \textit{
	\begin{itemize}
	\item I got up early the next morning.
	\item ...the next available flight.
	\item Who will be the next prime minister?
	\item I want my next child born at home.
	\item Many senior citizens have very few visitors from one week to the next.
	\item And then Captain Charles sings, 'Don't ever laugh when a hearse goes by or you will
be the next to die.'
	\end{itemize}
}
\item determiner \\
You use \textbf{next} in expressions such as \textbf{next Friday} , \textbf{next day} and \textbf{next year} to refer , for example , to the first Friday, day, or year that comes after the present or previous one.
 \textbf{Next} is also an adjective .
 \textbf{Next} is also a pronoun .
 \textit{
	\begin{itemize}
	\item Let's plan a big night next week.
	\item He retires next January.
	\item Next day the E.U. summit strengthened their ultimatum.
	\item I shall be 26 years old on Friday next.
	\item He predicted the region's economy would grow both this year and next.
	\end{itemize}
}
\item adjective \\
\textbf{The}  \textbf{next} place or person is the one that is nearest to you or that is the first one that you
come to.
 \textit{
	\begin{itemize}
	\item Grace sighed so heavily that Trish could hear it in the next room.
	\item The man in the next chair was asleep.
	\item Stop at the next corner. I'm getting out.
	\end{itemize}
}
\item adverb \\
The thing that happens  \textbf{next} is the thing that happens immediately after something else.
 \textit{
	\begin{itemize}
	\item Next, close your eyes then screw them up tight.
	\item I don't know what to do next.
	\item The news is next.
	\end{itemize}
}
\item adverb \\
When you \textbf{next} do something, you do it for the first time since you last did it.
 \textit{
	\begin{itemize}
	\item I next saw him at his house in Berkshire.
	\item When we next met, he was much more jovial.
	\end{itemize}
}
\item adverb \\
You use \textbf{next} to say that something has more of a particular  quality than all other things except one. For example, the thing that is \textbf{next}  best is the one that is the best except for one other thing.
 \textit{
	\begin{itemize}
	\item He didn't have a son; I think he felt that a grandson is the next best thing.
	\item At least three times more daffodils are grown than in Holland, the next largest grower.
	\end{itemize}
}
\item  \\
 after next \textit{
	\begin{itemize}
	\end{itemize}
}
\item  \\
 as the next \textit{
	\begin{itemize}
	\end{itemize}
}
\item  \\
 the next thing sb knows \textit{
	\begin{itemize}
	\end{itemize}
}
\item  \\
 next to \textit{
	\begin{itemize}
	\end{itemize}
}
\item  \\
 next to \textit{
	\begin{itemize}
	\end{itemize}
}
\item  \\
 next to \textit{
	\begin{itemize}
	\end{itemize}
}
\end{enumerate}

\section*{medal}
{\large \color{blue}  medals  }
\subsection*{Explain}
\begin{enumerate}
\item countable noun \\
A \textbf{medal} is a small metal disc which is given as an award for bravery or as a prize in a sporting event.
 \textit{
	\begin{itemize}
	\end{itemize}
}
\end{enumerate}

\section*{nuclear}
{\large \color{blue}  }
\subsection*{Explain}
\begin{enumerate}
\item adjective \\
\textbf{Nuclear}  means relating to the nuclei of atoms, or to the energy released when these nuclei are split or combined .
 \textit{
	\begin{itemize}
	\item ...a nuclear power station.
	\item ...nuclear energy.
	\item ...nuclear physics.
	\end{itemize}
}
\item adjective \\
\textbf{Nuclear} means relating to weapons that explode by using the energy released when the nuclei of atoms are split or combined.
 \textit{
	\begin{itemize}
	\item They rejected a demand for the removal of all nuclear weapons from U.K. soil.
	\item ...nuclear testing.
	\end{itemize}
}
\end{enumerate}

\section*{pose}
{\large \color{blue}  poses  posing  posed  }
\subsection*{Explain}
\begin{enumerate}
\item verb \\
If something \textbf{poses} a problem or a danger , it is the cause of that problem or danger.
 \textit{
	\begin{itemize}
	\item This could pose a threat to jobs in the coal industry.
	\item His ill health poses serious problems for the future.
	\end{itemize}
}
\item verb \\
If you \textbf{pose} a question, you ask it. If you \textbf{pose} an issue that needs  considering , you mention the issue.
 \textit{
	\begin{itemize}
	\item When I finally posed the question, 'Why?' he merely shrugged.
	\item ...the moral issues posed by new technologies.
	\end{itemize}
}
\item verb \\
If you \textbf{pose as} someone, you pretend to be that person in order to deceive people.
 \textit{
	\begin{itemize}
	\item Industrial spies posed as flight attendants.
	\end{itemize}
}
\item verb \\
If you \textbf{pose for} a photograph or painting, you stay in a particular  position so that someone can photograph you or paint you.
 \textit{
	\begin{itemize}
	\item Before going into their meeting the six foreign ministers posed for photographs.
	\end{itemize}
}
\item verb \\
You can say that people \textbf{are posing} when you think that they are behaving in an insincere or exaggerated  way because they want to make a particular impression on other people.
 \textit{
	\begin{itemize}
	\item He criticized them for dressing outrageously and posing pretentiously.
	\end{itemize}
}
\item countable noun \\
A \textbf{pose} is a particular way that you stand , sit , or lie , for example when you are being photographed or painted.
 \textit{
	\begin{itemize}
	\item We have had several preliminary sittings in various poses.
	\end{itemize}
}
\item countable noun \\
A \textbf{pose} is an insincere or exaggerated way of behaving that is intended to make a particular impression on other people.
 \textit{
	\begin{itemize}
	\item In many writers modesty is a pose, but in Ford it seems to have been genuine.
	\end{itemize}
}
\end{enumerate}

\section*{particular}
{\large \color{blue}  }
\subsection*{Explain}
\begin{enumerate}
\item adjective \\
You use \textbf{particular} to emphasize that you are talking about one thing or one kind of thing rather than other similar ones.
 \textit{
	\begin{itemize}
	\item I remembered a particular story about a postman who was a murderer.
	\item I have to know exactly why it is I'm doing a particular job.
	\item ...if there are particular things you're interested in.
	\end{itemize}
}
\item adjective \\
If a person or thing has a \textbf{particular} quality or possession , it is distinct and belongs only to them.
 \textit{
	\begin{itemize}
	\item I have a particular responsibility to ensure I make the right decision.
	\end{itemize}
}
\item adjective \\
You can use \textbf{particular} to emphasize that something is greater or more intense than usual .
 \textit{
	\begin{itemize}
	\item Particular emphasis will be placed on oral language training.
	\end{itemize}
}
\item adjective \\
If you say that someone is \textbf{particular} , you mean that they choose things and do things very carefully, and are not easily  satisfied .
 \textit{
	\begin{itemize}
	\item Ted was very particular about the colors he used.
	\end{itemize}
}
\item  \\
 in particular \textit{
	\begin{itemize}
	\end{itemize}
}
\item  \\
 nothing in particular \textit{
	\begin{itemize}
	\end{itemize}
}
\end{enumerate}

\section*{prize}
{\large \color{blue}  prizes  prizing  prized  }
\subsection*{Explain}
\begin{enumerate}
\item countable noun \\
A \textbf{prize} is money or something valuable that is given to someone who has the best  results in a competition or game, or as a reward for doing good work.
 \textit{
	\begin{itemize}
	\item You must claim your prize by phoning our claims line.
	\item He won first prize at the Leeds Piano Competition.
	\item He was awarded the Nobel Prize for Physics in 1985.
	\item They were going all out for the prize-money, £6,500 for the winning team.
	\end{itemize}
}
\item adjective \\
You use \textbf{prize} to describe things that are of such good quality that they win prizes or deserve to win prizes.
 \textit{
	\begin{itemize}
	\item ...a prize bull.
	\item ...prize blooms.
	\end{itemize}
}
\item countable noun \\
You can  refer to someone or something as a \textbf{prize} when people consider them to be of great value or importance .
 \textit{
	\begin{itemize}
	\item With no lands of his own, he was no great matrimonial prize.
	\end{itemize}
}
\item verb \\
Something that \textbf{is prized} is wanted and admired because it is considered to be very valuable or very good quality.
 \textit{
	\begin{itemize}
	\item Military figures, made out of lead are prized by collectors.
	\item His Fender Stratocaster remains one of his most prized possessions.
	\end{itemize}
}
\item verb \\
If you \textbf{prize} something open or \textbf{prize} it away from a surface , you force it to open or force it to come away from the surface.
 \textit{
	\begin{itemize}
	\item He tried to prize the dog's mouth open.
	\item I prised off the metal rim surrounding one of the dials.
	\item He held on tight but she prised it from his fingers.
	\end{itemize}
}
\item verb \\
If you \textbf{prize} something such as information  \textbf{out of} someone, you persuade them to tell you although they may be very unwilling to.
 \textit{
	\begin{itemize}
	\item Alison and I had to prize conversation out of him.
	\end{itemize}
}
\end{enumerate}

\section*{primitive}
{\large \color{blue}  }
\subsection*{Explain}
\begin{enumerate}
\item adjective \\
\textbf{Primitive} means belonging to a society in which people live in a very simple way, usually without industries or a writing system.
 \textit{
	\begin{itemize}
	\item ...studies of primitive societies.
	\item ...primitive tribes.
	\end{itemize}
}
\item adjective \\
\textbf{Primitive} means belonging to a very early period in the development of an animal or plant.
 \textit{
	\begin{itemize}
	\item ...primitive whales.
	\item Primitive humans needed to be able to react like this to escape from dangerous animals.
	\item It is a primitive instinct to flee a place of danger.
	\end{itemize}
}
\item adjective \\
If you describe something as \textbf{primitive} , you mean that it is very simple in style or very old-fashioned .
 \textit{
	\begin{itemize}
	\item The conditions are primitive by any standards.
	\item The primitive surgery of those days left him virtually deaf in one ear.
	\item It's using some rather primitive technology.
	\end{itemize}
}
\end{enumerate}

\section*{pulse}
{\large \color{blue}  pulses  pulsing  pulsed  }
\subsection*{Explain}
\begin{enumerate}
\item countable noun \\
Your \textbf{pulse} is the regular beating of blood through your body, which you can feel when you touch particular parts of your body, especially your wrist.
 \textit{
	\begin{itemize}
	\item Mahoney's pulse was racing, and he felt confused.
	\end{itemize}
}
\item countable noun \\
In music, a \textbf{pulse} is a regular beat, which is often produced by a drum .
 \textit{
	\begin{itemize}
	\item ...the repetitive pulse of the music.
	\end{itemize}
}
\item countable noun \\
A \textbf{pulse}  \textbf{of}  electrical current, light, or sound is a temporary increase in its level.
 \textit{
	\begin{itemize}
	\item The switch works by passing a pulse of current between the tip and the surface.
	\end{itemize}
}
\item singular noun \\
If you refer to \textbf{the pulse of} a group in society, you mean the ideas , opinions , or feelings they have at a particular time.
 \textit{
	\begin{itemize}
	\item I love the way he is so absolutely in tune with the pulse of his audience.
	\end{itemize}
}
\item verb \\
If something \textbf{pulses} , it moves, appears , or makes a sound with a strong regular rhythm .
 \textit{
	\begin{itemize}
	\item His temples pulsed a little, threatening a headache.
	\item It was a slow, pulsing rhythm that seemed to sway languidly in the air.
	\end{itemize}
}
\item plural noun \\
Some seeds which can be cooked and eaten are called  \textbf{pulses} , for example peas, beans, and lentils.
 \textit{
	\begin{itemize}
	\end{itemize}
}
\item  \\
 finger on the pulse \textit{
	\begin{itemize}
	\end{itemize}
}
\item  \\
 take someone's pulse/feel someone's pulse \textit{
	\begin{itemize}
	\end{itemize}
}
\end{enumerate}

\section*{real}
{\large \color{blue}  }
\subsection*{Explain}
\begin{enumerate}
\item adjective \\
Something that is \textbf{real}  actually exists and is not imagined , invented , or theoretical.
 \textit{
	\begin{itemize}
	\item No, it wasn't a dream. It was real.
	\item Legends grew up around a great many figures, both real and fictitious.
	\end{itemize}
}
\item adjective \\
If something is \textbf{real}  \textbf{to} someone, they experience it as though it really exists or happens , even though it does not.
 \textit{
	\begin{itemize}
	\item Whitechild's life becomes increasingly real to the reader.
	\end{itemize}
}
\item adjective \\
A material or object that is \textbf{real} is natural or functioning, and not artificial or an imitation .
 \textit{
	\begin{itemize}
	\item ...the smell of real leather.
	\item Who's to know if they're real guns or not?
	\item Desmond did not believe the diamond was real.
	\end{itemize}
}
\item adjective \\
You can use \textbf{real} to describe someone or something that has all the characteristics or qualities that such a person
or thing typically has.
 \textit{
	\begin{itemize}
	\item ...his first real girlfriend.
	\item He's not a real artist.
	\item The only real job I'd ever had was as manager of the local cafe.
	\end{itemize}
}
\item adjective \\
You can use \textbf{real} to describe something that is the true or original thing of its kind, in contrast to one that someone wants you to believe is true.
 \textit{
	\begin{itemize}
	\item This was the real reason for her call.
	\item Her real name had been Miriam Pinckus.
	\end{itemize}
}
\item adjective \\
You can use \textbf{real} to describe something that is the most important or typical part of a thing.
 \textit{
	\begin{itemize}
	\item When he talks, he only gives glimpses of his real self.
	\item The smart executive has people he can trust doing all the real work.
	\item ...a solo journey to discover the real America.
	\end{itemize}
}
\item adjective \\
You can use \textbf{real} when you are talking about a situation or feeling to emphasize that it exists and is important or serious .
 \textit{
	\begin{itemize}
	\item Global warming is a real problem.
	\item The prospect of civil war is very real.
	\item There was never any real danger of the children being affected.
	\item Political defeat seemed a real possibility at the end of 1981.
	\item At least they have a real chance to find work.
	\end{itemize}
}
\item adjective \\
You can use \textbf{real} to emphasize a quality that is genuine and sincere .
 \textit{
	\begin{itemize}
	\item You've been drifting from job to job without any real commitment.
	\item I thought we were the team who showed real determination to win.
	\end{itemize}
}
\item adjective \\
You can use \textbf{real} before nouns to emphasize your description of something or someone.
 \textit{
	\begin{itemize}
	\item 'It's a fabulous deal, a real bargain.'
	\item 'You must think I'm a real idiot.'
	\end{itemize}
}
\item adjective \\
The \textbf{real}  cost or value of something is its cost or value after other amounts have been added or
 subtracted and when factors such as the level of inflation have been considered.
 \textit{
	\begin{itemize}
	\item ...the real cost of borrowing.
	\end{itemize}
}
\item adverb \\
You can use \textbf{real} to emphasize an adjective or adverb .
 \textit{
	\begin{itemize}
	\item He is finding prison life 'real tough'.
	\item I don't think you are trying real hard.
	\end{itemize}
}
\item  \\
 for real \textit{
	\begin{itemize}
	\end{itemize}
}
\item  \\
 for real \textit{
	\begin{itemize}
	\end{itemize}
}
\item  \\
 the real thing \textit{
	\begin{itemize}
	\end{itemize}
}
\end{enumerate}

\section*{pursuit}
{\large \color{blue}  pursuits  }
\subsection*{Explain}
\begin{enumerate}
\item uncountable noun \\
Your \textbf{pursuit of} something is your attempts at achieving it. If you do something \textbf{in pursuit of} a particular result, you do it in order to achieve that result.
 \textit{
	\begin{itemize}
	\item ...a young man whose relentless pursuit of excellence is conducted with determination.
	\item ...individuals who impoverish their families in pursuit of some dream.
	\end{itemize}
}
\item uncountable noun \\
The \textbf{pursuit of} an activity, interest, or plan consists of all the things that you do when you are carrying it out.
 \textit{
	\begin{itemize}
	\item The vigorous pursuit of policies is no guarantee of success.
	\end{itemize}
}
\item uncountable noun \\
Someone who is \textbf{in pursuit of} a person, vehicle, or animal is chasing them.
 \textit{
	\begin{itemize}
	\item ...a police officer who drove a patrol car at more than 120mph in pursuit of a motor
cycle.
	\end{itemize}
}
\item singular noun \\
In cycling and skating , \textbf{the pursuit} is a race in which two competitors or teams  start on opposite sides of a circular track and try to catch up with each other.
 \textit{
	\begin{itemize}
	\item Moreau took gold in the five-kilometre individual pursuit competition.
	\end{itemize}
}
\item countable noun \\
Your \textbf{pursuits} are your activities, usually activities that you enjoy when you are not working .
 \textit{
	\begin{itemize}
	\item They both love outdoor pursuits.
	\item His favourite childhood pursuits were sailing, swimming and cycling.
	\end{itemize}
}
\item  \\
 in hot pursuit \textit{
	\begin{itemize}
	\end{itemize}
}
\end{enumerate}

\section*{round}
{\large \color{blue}  }
\subsection*{Explain}
\begin{enumerate}
\item preposition \\
To be positioned \textbf{round} a place or object means to surround it or be on all sides of it. To move \textbf{round} a place means to go along its edge, back to the point where you started.
 \textbf{Round} is also an adverb .
 \textit{
	\begin{itemize}
	\item They were sitting round the kitchen table.
	\item The nightdress has handmade lace round the armholes and neckline.
	\item All round us was desert.
	\item I shivered and pulled my scarf more tightly round my neck.
	\item He tramped hurriedly round the lake towards the garden.
	\item ...cycling round and round the park.
	\item Visibility was good all round.
	\item The goldfish swam round and round in their tiny bowls.
	\end{itemize}
}
\item preposition \\
If you move \textbf{round} a corner or obstacle , you move to the other side of it. If you look \textbf{round} a corner or obstacle, you look to see what is on the other side.
 \textit{
	\begin{itemize}
	\item Suddenly a car came round a corner on the opposite side.
	\item Stay on the left-hand pavement to follow a road downhill round a curve.
	\item One of his men tapped and looked round the door.
	\end{itemize}
}
\item preposition \\
You use \textbf{round} to say that something happens in or relates to different parts of a place, or is near a place.
 \textbf{Round} is also an adverb.
 \textit{
	\begin{itemize}
	\item He happens to own half the land round here.
	\item I think he has earned the respect of leaders all round the world.
	\item She's been on at me for weeks to show her round the stables.
	\item They need some way of getting round the country.
	\item Shirley found someone to show them round.
	\item So you're going to have a look round?
	\end{itemize}
}
\item adverb \\
If a wheel or object spins  \textbf{round} , it turns on its axis.
 \textit{
	\begin{itemize}
	\item Holes can be worn remarkably quickly by a wheel going round at 60mph.
	\item Stars appeared everywhere, spinning round and round, faster and faster.
	\end{itemize}
}
\item adverb \\
If you turn \textbf{round} , you turn so that you are facing or going in the opposite direction.
 \textit{
	\begin{itemize}
	\item She paused, but did not turn round.
	\item The end result was that the ship had to turn round, and go back home.
	\item The wind veered round to the east.
	\item Tricia looked round in surprise.
	\end{itemize}
}
\item adverb \\
If you move things \textbf{round} , you move them so they are in different places.
 \textit{
	\begin{itemize}
	\item He will be glad to refurnish where possible, change things round and redecorate.
	\item I've already moved things round a bit to make it easier for him.
	\end{itemize}
}
\item adverb \\
If you hand or pass something \textbf{round} , it is passed from person to person in a group.
 \textbf{Round} is also a preposition .
 \textit{
	\begin{itemize}
	\item John handed round the plate of sandwiches.
	\item Coffee was being passed round.
	\item They started handing the microphone out round the girls at the front.
	\item The word is passed round the industry if you think there's a troublesome driver.
	\end{itemize}
}
\item adverb \\
If you go \textbf{round} to someone's house, you visit them.
 \textbf{Round} is also a preposition in non-standard English.
 \textit{
	\begin{itemize}
	\item I think we should go round and tell Kevin to turn his music down.
	\item He came round with a bottle of champagne.
	\item I went round my friend's house.
	\end{itemize}
}
\item adverb \\
You use \textbf{round} in informal expressions such as \textbf{sit round} or \textbf{hang round} when you are saying that someone is spending time in a place and is not doing anything very important.
 \textbf{Round} is also a preposition.
 \textit{
	\begin{itemize}
	\item As we sat round chatting, I began to think I'd made a mistake.
	\item I was running round all hyped up.
	\item She would spend the day hanging round street corners.
	\item Leonard pottered round the greenhouse, tying up canes for the tomatoes.
	\end{itemize}
}
\item preposition \\
If something is built or based \textbf{round} a particular idea, that idea is the basis for it.
 \textit{
	\begin{itemize}
	\item That was for a design built round an existing American engine.
	\item The core of the festival's programme centres round performances of new and 20th century
work.
	\end{itemize}
}
\item preposition \\
If you get \textbf{round} a problem or difficulty, you find a way of dealing with it.
 \textit{
	\begin{itemize}
	\item Don't just immediately give up but think about ways round a problem.
	\item There are ways of getting round most things!
	\end{itemize}
}
\item adverb \\
If you win someone \textbf{round} , or if they come \textbf{round} , they change their mind about something and start agreeing with you.
 \textit{
	\begin{itemize}
	\item He did his best to talk me round, but I wouldn't speak to him.
	\item The Chandler twins were coming round to the same opinion.
	\end{itemize}
}
\item adverb \\
You use \textbf{round} in expressions such as \textbf{this time round} or \textbf{to come round} when you are describing something that has happened before or things that happen
regularly.
 \textit{
	\begin{itemize}
	\item At least two directors were expected to vote to increase rates this time round .
	\item Of course, it isn't the same first time round.
	\item We were very keen when the 1954 Rally came round.
	\end{itemize}
}
\item preposition \\
You can use \textbf{round} to give the measurement of the outside of something that is shaped like a circle
or a cylinder.
 \textbf{Round} is also an adverb.
 \textit{
	\begin{itemize}
	\item I'm about two inches larger round the waist.
	\item ...forty-eight inches round the hips.
	\item It's six feet high and five feet round.
	\end{itemize}
}
\item adverb \\
You use \textbf{round} in front of times or amounts to indicate that they are approximate .
 \textit{
	\begin{itemize}
	\item I go to bed round 11:00 at night.
	\end{itemize}
}
\item  \\
 round about \textit{
	\begin{itemize}
	\end{itemize}
}
\item  \\
 all round \textit{
	\begin{itemize}
	\end{itemize}
}
\item  \\
 go round and round \textit{
	\begin{itemize}
	\end{itemize}
}
\item  \\
 all year round \textit{
	\begin{itemize}
	\end{itemize}
}
\end{enumerate}

\section*{reliance}
{\large \color{blue}  }
\subsection*{Explain}
\begin{enumerate}
\item uncountable noun \\
A person's or thing's \textbf{reliance}  \textbf{on} something is the fact that they need it and often cannot live or work without it.
 \textit{
	\begin{itemize}
	\item ...the country's increasing reliance on foreign aid.
	\end{itemize}
}
\end{enumerate}

\section*{scarce}
{\large \color{blue}  scarcer  scarcest  }
\subsection*{Explain}
\begin{enumerate}
\item adjective \\
If something is \textbf{scarce} , there is not enough of it.
 \textit{
	\begin{itemize}
	\item Food was scarce and expensive.
	\item Jobs are becoming increasingly scarce.
	\item ...the allocation of scarce resources.
	\end{itemize}
}
\item  \\
 make oneself scarce \textit{
	\begin{itemize}
	\end{itemize}
}
\end{enumerate}

\section*{royalty}
{\large \color{blue}  royalties  }
\subsection*{Explain}
\begin{enumerate}
\item uncountable noun \\
The members of royal families are sometimes  referred to as \textbf{royalty} .
 \textit{
	\begin{itemize}
	\item Royalty and government leaders from all around the world are gathering together.
	\item ...a ceremony attended by royalty.
	\end{itemize}
}
\item plural noun \\
\textbf{Royalties} are payments made to authors and musicians when their work is sold or performed. They usually receive a fixed percentage of the profits from these sales or performances.
 \textit{
	\begin{itemize}
	\item I lived on the royalties on my book.
	\end{itemize}
}
\item countable noun \\
Payments made to someone whose invention, idea , or property is used by a commercial company can be referred to as \textbf{royalties} .
 \textit{
	\begin{itemize}
	\item The royalties enabled the inventor to re-establish himself in business.
	\end{itemize}
}
\end{enumerate}

\section*{special}
{\large \color{blue}  specials  }
\subsection*{Explain}
\begin{enumerate}
\item adjective \\
Someone or something that is \textbf{special} is better or more important than other people or things.
 \textit{
	\begin{itemize}
	\item You're very special to me, darling.
	\item There are strong arguments for holidays at Easter and Christmas because these are
special occasions.
	\item Just to see him was something special.
	\item The famous author was going to be a special guest lecturer on the campus.
	\end{itemize}
}
\item adjective \\
\textbf{Special} means different from normal .
 \textit{
	\begin{itemize}
	\item The committee can waive this three-year rule in special cases.
	\item So you didn't notice anything special about him?
	\item On her birthday, she invites her dearest friends over for a special meal.
	\item ...'Little Scarlet' strawberry jam, made from a special variety of strawberry.
	\end{itemize}
}
\item adjective \\
You use \textbf{special} to describe someone who is officially  appointed or who has a particular position specially created for them.
 \textit{
	\begin{itemize}
	\item Due to his wife's illness, he returned to the State Department as special adviser
to the President.
	\item He is a special correspondent for Newsweek magazine.
	\end{itemize}
}
\item adjective \\
\textbf{Special}  institutions are for people who have serious  physical or mental  problems .
 \textit{
	\begin{itemize}
	\item Police are still searching for a convicted rapist, who escaped from Broadmoor special
hospital yesterday.
	\end{itemize}
}
\item adjective \\
You use \textbf{special} to describe something that relates to one particular person, group, or place.
 \textit{
	\begin{itemize}
	\item Every anxious person will have his or her own special problems or fears.
	\item ...it requires a very special brand of courage to fight dictators.
	\end{itemize}
}
\item countable noun \\
A \textbf{special} is a product, programme , or meal which is not normally  available , or which is made for a particular purpose.
 \textit{
	\begin{itemize}
	\item ...complaints about the BBC's Hallowe'en special, 'Ghostwatch'.
	\item Grocery stores have to offer enough specials to bring people into the store.
	\item ...talk shows and news specials.
	\end{itemize}
}
\end{enumerate}

\section*{scar}
{\large \color{blue}  scars  scarring  scarred  }
\subsection*{Explain}
\begin{enumerate}
\item countable noun \\
A \textbf{scar} is a mark on the skin which is left after a wound has healed.
 \textit{
	\begin{itemize}
	\item He had a scar on his forehead.
	\item ...facial injuries which have left permanent scars.
	\end{itemize}
}
\item verb \\
If your skin \textbf{is scarred} , it is badly marked as a result of a wound.
 \textit{
	\begin{itemize}
	\item He was scarred for life during a pub fight.
	\item His scarred face crumpled with pleasure.
	\end{itemize}
}
\item verb \\
If a surface \textbf{is scarred} , it is damaged and there are ugly marks on it.
 \textit{
	\begin{itemize}
	\item The arena was scarred by deep muddy ruts.
	\item ...scarred wooden table tops.
	\end{itemize}
}
\item countable noun \\
If an unpleasant  physical or emotional experience leaves a \textbf{scar} on someone, it has a permanent effect on their mind .
 \textit{
	\begin{itemize}
	\item The early years of fear and the hostility left a deep scar on the young boy.
	\end{itemize}
}
\item verb \\
If an unpleasant physical or emotional experience \textbf{scars} you, it has a permanent effect on your mind.
 \textit{
	\begin{itemize}
	\item This is something that's going to scar him forever.
	\end{itemize}
}
\end{enumerate}

\section*{statistical}
{\large \color{blue}  }
\subsection*{Explain}
\begin{enumerate}
\item adjective \\
\textbf{Statistical} means relating to the use of statistics.
 \textit{
	\begin{itemize}
	\item The report contains a great deal of statistical information.
	\item We need to back that suspicion up with statistical proof.
	\end{itemize}
}
\end{enumerate}

\section*{scholarship}
{\large \color{blue}  scholarships  }
\subsection*{Explain}
\begin{enumerate}
\item countable noun \\
If you get a \textbf{scholarship} to a school or university , your studies are paid for by the school or university or by some other organization .
 \textit{
	\begin{itemize}
	\item He got a scholarship to the Pratt Institute of Art.
	\item ...scholarships for women over 30.
	\end{itemize}
}
\item uncountable noun \\
\textbf{Scholarship} is serious academic study and the knowledge that is obtained from it.
 \textit{
	\begin{itemize}
	\item I want to take advantage of your lifetime of scholarship.
	\end{itemize}
}
\end{enumerate}

\section*{stubborn}
{\large \color{blue}  }
\subsection*{Explain}
\begin{enumerate}
\item adjective \\
Someone who is \textbf{stubborn} or who behaves in a \textbf{stubborn} way is determined to do what they want and is very unwilling to change their mind .
 \textit{
	\begin{itemize}
	\item He is a stubborn character used to getting his own way.
	\item His face was set in an expression of stubborn determination.
	\end{itemize}
}
\item adjective \\
A \textbf{stubborn}  stain or problem is difficult to remove or to deal with.
 \textit{
	\begin{itemize}
	\item This treatment removes the most stubborn stains.
	\item The first and most stubborn problem was that of reductions in the number of aircraft.
	\end{itemize}
}
\end{enumerate}

\section*{signal}
{\large \color{blue}  signals  signalling  signalled  }
\subsection*{Explain}
\begin{enumerate}
\item countable noun \\
A \textbf{signal} is a gesture, sound, or action which is intended to give a particular message to the person who sees or hears it.
 \textit{
	\begin{itemize}
	\item They fired three distress signals.
	\item As soon as it was dark, Mrs Evans gave the signal.
	\item You mustn't fire without my signal.
	\end{itemize}
}
\item verb \\
If you \textbf{signal}  \textbf{to} someone, you make a gesture or sound in order to send them a particular message.
 \textit{
	\begin{itemize}
	\item The United manager was to be seen frantically signalling to McClair.
	\item He stood up, signalling to the officer that he had finished with his client.
	\item She signalled a passing taxi and ordered him to take her to the rue Marengo.
	\end{itemize}
}
\item countable noun \\
If an event or action is a \textbf{signal}  \textbf{of} something, it suggests that this thing exists or is going to happen .
 \textit{
	\begin{itemize}
	\item Kurdish leaders saw the visit as an important signal of support.
	\item The first warning signals came in March.
	\item The Red Cross is withdrawing its staff until they receive clear signals from all
sides that their presence is welcomed.
	\end{itemize}
}
\item verb \\
If someone or something \textbf{signals} an event, they suggest that the event is happening or likely to happen.
 \textit{
	\begin{itemize}
	\item She will be signalling massive changes in energy policy.
	\item Britain was signalling its readiness to have the embargo lifted.
	\item The outcome of the meeting could signal whether there is a political will to begin
negotiating.
	\end{itemize}
}
\item countable noun \\
A \textbf{signal} is a piece of equipment beside a railway , which indicates to train  drivers whether they should stop the train or not.
 \textit{
	\begin{itemize}
	\end{itemize}
}
\item countable noun \\
A \textbf{signal} is a series of radio waves, light waves, or changes in electrical current which may carry information.
 \textit{
	\begin{itemize}
	\item ...high-frequency radio signals.
	\end{itemize}
}
\item adjective \\
You use \textbf{signal} to describe a success or failure when you are emphasizing the fact that it has occurred and are indicating that the consequences are significant .
 \textit{
	\begin{itemize}
	\item His final round was a signal triumph in a career marked by many sweet moments.
	\item ...Marlantes's signal failure to master some of the more central aspects of novel
writing, including character, pace and style.
	\end{itemize}
}
\end{enumerate}

\section*{successful}
{\large \color{blue}  }
\subsection*{Explain}
\begin{enumerate}
\item adjective \\
Something that is \textbf{successful}  achieves what it was intended to achieve. Someone who is \textbf{successful} achieves what they intended to achieve.
 \textit{
	\begin{itemize}
	\item How successful will this new treatment be?
	\item I am looking forward to a long and successful partnership with him.
	\item She has been comparatively successful in maintaining her privacy.
	\end{itemize}
}
\item adjective \\
Something that is \textbf{successful} is popular or makes a lot of money.
 \textit{
	\begin{itemize}
	\item ...the hugely successful movie that brought Robert Redford an Oscar for his directing.
	\item One of the keys to successful business is careful planning.
	\end{itemize}
}
\item adjective \\
Someone who is \textbf{successful} achieves a high position in what they do, for example in business or politics .
 \textit{
	\begin{itemize}
	\item Women do not necessarily have to imitate men to be successful in business.
	\item She is a successful lawyer.
	\end{itemize}
}
\end{enumerate}

\section*{slot}
{\large \color{blue}  slots  slotting  slotted  }
\subsection*{Explain}
\begin{enumerate}
\item countable noun \\
A \textbf{slot} is a narrow opening in a machine or container , for example a hole that you put coins in to make a machine work.
 \textit{
	\begin{itemize}
	\item He dropped a coin into the slot and dialed.
	\end{itemize}
}
\item verb \\
If you \textbf{slot} something into something else, or if it \textbf{slots} into it, you put it into a space where it fits.
 \textit{
	\begin{itemize}
	\item The seatbelt buckle has red LED lights to indicate where to slot the belt in.
	\item The car seat belt slotted into place easily.
	\item She slotted in a fresh filter.
	\end{itemize}
}
\item countable noun \\
A \textbf{slot} in a schedule or scheme is a place in it where an activity can take place.
 \textit{
	\begin{itemize}
	\item Visitors can book a time slot a week or more in advance.
	\item The first episode occupies a peak evening viewing slot.
	\end{itemize}
}
\end{enumerate}

\section*{thermal}
{\large \color{blue}  thermals  }
\subsection*{Explain}
\begin{enumerate}
\item adjective \\
\textbf{Thermal}  means relating to or caused by heat or by changes in temperature.
 \textit{
	\begin{itemize}
	\item ...thermal power stations.
	\item ...financial assistance with repair, thermal insulation and improvements to homes
through Government grants.
	\end{itemize}
}
\item adjective \\
\textbf{Thermal}  streams or baths contain water which is naturally hot or warm.
 \textit{
	\begin{itemize}
	\item Volcanic activity has created thermal springs and boiling mud pools.
	\end{itemize}
}
\item adjective \\
\textbf{Thermal}  clothes are specially designed to keep you warm in cold  weather .
 \textbf{Thermals} are thermal clothes.
 \textit{
	\begin{itemize}
	\item My feet were like blocks of ice despite the thermal socks.
	\item I put on my thermal leggings, long socks and the rest of my clothes.
	\item Have you got your thermals on?
	\end{itemize}
}
\item countable noun \\
A \textbf{thermal} is a movement of rising warm air.
 \textit{
	\begin{itemize}
	\item Birds use thermals to lift them through the air.
	\end{itemize}
}
\end{enumerate}

\section*{solo}
{\large \color{blue}  solos  }
\subsection*{Explain}
\begin{enumerate}
\item adjective \\
You use \textbf{solo} to indicate that someone does something alone rather than with other people.
 \textbf{Solo} is also an adverb .
 \textit{
	\begin{itemize}
	\item He had just completed his final solo album.
	\item ...his spectacular solo goal.
	\item She had long since quit the band for a solo career.
	\item Charles Lindbergh became the very first person to fly solo across the Atlantic.
	\end{itemize}
}
\item countable noun \\
A \textbf{solo} is a piece of music or a dance  performed by one person.
 \textit{
	\begin{itemize}
	\item The original version featured a guitar solo.
	\end{itemize}
}
\end{enumerate}

\section*{sword}
{\large \color{blue}  swords  }
\subsection*{Explain}
\begin{enumerate}
\item countable noun \\
A \textbf{sword} is a weapon with a handle and a long sharp blade.
 \textit{
	\begin{itemize}
	\end{itemize}
}
\item  \\
 to cross swords \textit{
	\begin{itemize}
	\end{itemize}
}
\item  \\
 a double-edged sword \textit{
	\begin{itemize}
	\end{itemize}
}
\end{enumerate}

\section*{tropical}
{\large \color{blue}  }
\subsection*{Explain}
\begin{enumerate}
\item adjective \\
\textbf{Tropical} means belonging to or typical of the tropics.
 \textit{
	\begin{itemize}
	\item ...tropical diseases.
	\item ...a plan to preserve the world's tropical forests.
	\end{itemize}
}
\item adjective \\
\textbf{Tropical} weather is hot and damp weather that people believe to be typical of the tropics.
 \textit{
	\begin{itemize}
	\end{itemize}
}
\end{enumerate}

\section*{trade}
{\large \color{blue}  trades  trading  traded  }
\subsection*{Explain}
\begin{enumerate}
\item uncountable noun \\
\textbf{Trade} is the activity of buying, selling, or exchanging goods or services between people,
firms, or countries.
 \textit{
	\begin{itemize}
	\item The ministry had direct control over every aspect of foreign trade.
	\item ...negotiations on a new international trade agreement.
	\item Texas has a long history of trade with Mexico.
	\end{itemize}
}
\item verb \\
When people, firms, or countries \textbf{trade} , they buy, sell, or exchange goods or services between themselves.
 \textit{
	\begin{itemize}
	\item They may refuse to trade, even when offered attractive prices.
	\item Australia and New Zealand trade extensively with each other.
	\item He has been trading in antique furniture for 25 years.
	\end{itemize}
}
\item countable noun \\
A \textbf{trade} is a particular area of business or industry.
 \textit{
	\begin{itemize}
	\item They've completely ruined the tourist trade for the next few years.
	\item ...the arms trade.
	\end{itemize}
}
\item countable noun \\
Someone's \textbf{trade} is the kind of work that they do, especially when they have been trained to do it over a period of time.
 \textit{
	\begin{itemize}
	\item He learnt his trade as a diver in the North Sea.
	\item Allyn was a jeweller by trade.
	\item She is a patron of small businesses and trades.
	\end{itemize}
}
\item verb \\
If someone \textbf{trades} one thing \textbf{for} another or if two people \textbf{trade} things, they agree to exchange one thing for the other thing.
 \textbf{Trade} is also a noun .
 \textit{
	\begin{itemize}
	\item They traded land for goods and money.
	\item He still claims the arms weren't traded for hostages.
	\item Kids used to trade baseball cards.
	\item They suspected that Neville had traded secret information with Mr Foster.
	\item I am willing to make a trade with you.
	\item It wouldn't exactly have been a fair trade.
	\end{itemize}
}
\item verb \\
If you \textbf{trade} places \textbf{with} someone or if the two of you \textbf{trade} places, you move into the other person's position or situation , and they move into yours.
 \textit{
	\begin{itemize}
	\item Mike asked George to trade places with him so he could ride with Tod.
	\item Kennedy mischievously suggested that professors ought to trade jobs for a time with
janitors.
	\item The receiver and the quarterback are going to trade positions.
	\end{itemize}
}
\item verb \\
In professional sports, for example  football or baseball , if a player \textbf{is traded} from one team to another, they leave one team and begin playing for another.
 \textit{
	\begin{itemize}
	\item He was traded from the Giants to the Yankees.
	\item The A's have not won a game since they traded him.
	\end{itemize}
}
\item verb \\
If two people or groups \textbf{trade} something such as blows , insults , or jokes , they hit each other, insult each other, or tell each other jokes.
 \textit{
	\begin{itemize}
	\item Children would settle disputes by trading punches or insults in the schoolyard.
	\item They traded artillery fire with government forces inside the city.
	\end{itemize}
}
\end{enumerate}

\section*{vicious}
{\large \color{blue}  }
\subsection*{Explain}
\begin{enumerate}
\item adjective \\
A \textbf{vicious} person or a \textbf{vicious}  blow is violent and cruel.
 \textit{
	\begin{itemize}
	\item He was a cruel and vicious man.
	\item He suffered a vicious attack by a gang of white youths.
	\item The blow was so sudden and vicious that he dropped to his knees.
	\end{itemize}
}
\item adjective \\
A \textbf{vicious}  remark is cruel and intended to upset someone.
 \textit{
	\begin{itemize}
	\item It is a deliberate, nasty and vicious attack on a young man's character.
	\end{itemize}
}
\end{enumerate}

\section*{triumph}
{\large \color{blue}  triumphs  triumphing  triumphed  }
\subsection*{Explain}
\begin{enumerate}
\item variable noun \\
A \textbf{triumph} is a great  success or achievement, often one that has been gained with a lot of skill or effort .
 \textit{
	\begin{itemize}
	\item The championships proved to be a personal triumph for the coach.
	\item Cataract operations are a triumph of modern surgery.
	\item In the moment of triumph I felt uneasy.
	\end{itemize}
}
\item uncountable noun \\
\textbf{Triumph} is a feeling of great satisfaction and pride resulting from a success or victory.
 \textit{
	\begin{itemize}
	\item Her sense of triumph was short-lived.
	\item He was laughing with triumph.
	\end{itemize}
}
\item verb \\
If someone or something \textbf{triumphs} , they gain complete success, control, or victory, often after a long or difficult  struggle .
 \textit{
	\begin{itemize}
	\item All her life, Kelly had stuck with difficult tasks and challenges, and triumphed.
	\item The whole world looked to her as a symbol of good triumphing over evil.
	\end{itemize}
}
\end{enumerate}

\section*{bloody}
{\large \color{blue}  bloodier  bloodiest  bloodies  bloodying  bloodied  }
\subsection*{Explain}
\begin{enumerate}
\item adjective \\
\textbf{Bloody} is used by some people to emphasize what they are saying , especially when they are angry .
 \textit{
	\begin{itemize}
	\end{itemize}
}
\item adjective \\
If you describe a situation or event as \textbf{bloody} , you mean that it is very violent and a lot of people are killed .
 \textit{
	\begin{itemize}
	\item Forty-three demonstrators were killed in bloody clashes.
	\item They came to power after a bloody civil war.
	\end{itemize}
}
\item adjective \\
You can describe someone or something as \textbf{bloody} if they are covered in a lot of blood.
 \textit{
	\begin{itemize}
	\item He was arrested last October, still carrying a bloody knife.
	\item Yulka's fingers were bloody and cracked.
	\end{itemize}
}
\item verb \\
If you have \textbf{bloodied} part of your body, there is blood on it, usually because you have had an accident or you have been attacked .
 \textit{
	\begin{itemize}
	\item One of our children fell and bloodied his knee.
	\item She stared at her own bloodied hands, unable to think or move.
	\end{itemize}
}
\item passive verb \\
If someone or something \textbf{is bloodied} by an experience , they are hurt or damaged by it.
 \textit{
	\begin{itemize}
	\item She'd been bloodied in love.
	\item The reinsurance market has been bloodied by disasters in the U.S.
	\end{itemize}
}
\end{enumerate}

\section*{advance}
{\large \color{blue}  advances  advancing  advanced  }
\subsection*{Explain}
\begin{enumerate}
\item verb \\
To \textbf{advance} means to move forward, often in order to attack someone.
 \textit{
	\begin{itemize}
	\item The Allies began advancing on the city in 1943.
	\item The water is advancing at a rate of 5cm a day.
	\item ...a picture of a man throwing himself before an advancing tank.
	\end{itemize}
}
\item verb \\
To \textbf{advance} means to make progress, especially in your knowledge of something.
 \textit{
	\begin{itemize}
	\item Medical technology has advanced considerably.
	\end{itemize}
}
\item verb \\
If you \textbf{advance} someone a sum of money, you lend it to them, or pay it to them earlier than arranged.
 \textit{
	\begin{itemize}
	\item I advanced him some money, which he would repay on our way home.
	\item The bank advanced $1.2 billion to help the country with debt repayments.
	\end{itemize}
}
\item countable noun \\
An \textbf{advance} is money which is lent or paid to someone before they would normally receive it.
 \textit{
	\begin{itemize}
	\item She was paid a £100,000 advance for her next two novels.
	\end{itemize}
}
\item verb \\
To \textbf{advance} an event, or the time or date of an event, means to bring it forward to an earlier time or date.
 \textit{
	\begin{itemize}
	\item Too much protein in the diet may advance the ageing process.
	\item The country's election commission has advanced the date of the election by three
days.
	\end{itemize}
}
\item verb \\
If you \textbf{advance} a cause, interest, or claim , you support it and help to make it successful .
 \textit{
	\begin{itemize}
	\item When not producing art of his own, Oliver was busy advancing the work of others.
	\end{itemize}
}
\item verb \\
When a theory or argument  \textbf{is advanced} , it is put forward for discussion .
 \textit{
	\begin{itemize}
	\item Many theories have been advanced as to why some women suffer from depression.
	\item An explanation has now been advanced by scientists.
	\end{itemize}
}
\item variable noun \\
An \textbf{advance} is a forward movement of people or vehicles, usually as part of a military operation.
 \textit{
	\begin{itemize}
	\item ...an advance on enemy positions.
	\item The defences are intended to obstruct any advance by tanks and other vehicles.
	\end{itemize}
}
\item plural noun \\
If you make \textbf{advances} to someone, you try to start a sexual relationship with them.
 \textit{
	\begin{itemize}
	\item Mark had for some time been making advances towards her.
	\item She rejected his advances during the trip to Cannes.
	\end{itemize}
}
\item variable noun \\
An \textbf{advance} in a particular subject or activity is progress in understanding it or in doing it well .
 \textit{
	\begin{itemize}
	\item ...the technological advances of the last four decades.
	\item Their progress at work was mirrored by their children's educational advance.
	\end{itemize}
}
\item singular noun \\
If something is an \textbf{advance}  \textbf{on} what was previously available or done, it is better in some way.
 \textit{
	\begin{itemize}
	\item This could be an advance on the present situation.
	\end{itemize}
}
\item adjective \\
\textbf{Advance}  booking , notice , or warning is done or given before an event happens .
 \textit{
	\begin{itemize}
	\item They don't normally give any advance notice about which building they're going to
inspect.
	\item The event received little advance publicity.
	\end{itemize}
}
\item adjective \\
An \textbf{advance} party or group is a small group of people who go on ahead of the main group.
 \textit{
	\begin{itemize}
	\item The 20-strong advance party will be followed by another 600 soldiers.
	\end{itemize}
}
\item  \\
 in advance of \textit{
	\begin{itemize}
	\end{itemize}
}
\item  \\
 in advance \textit{
	\begin{itemize}
	\end{itemize}
}
\end{enumerate}

\section*{broad}
{\large \color{blue}  broader  broadest  broads  }
\subsection*{Explain}
\begin{enumerate}
\item adjective \\
Something that is \textbf{broad} is wide .
 \textit{
	\begin{itemize}
	\item His shoulders were broad and his waist narrow.
	\item The hills rise green and sheer above the broad river.
	\item ...a broad expanse of green lawn.
	\end{itemize}
}
\item adjective \\
A \textbf{broad}  smile is one in which your mouth is stretched very wide because you are very pleased or amused .
 \textit{
	\begin{itemize}
	\item He greeted them with a wave and a broad smile.
	\end{itemize}
}
\item adjective \\
You use \textbf{broad} to describe something that includes a large number of different things or people.
 \textit{
	\begin{itemize}
	\item A broad range of issues was discussed.
	\item ...a broad coalition of workers, peasants, students and middle class professionals.
	\end{itemize}
}
\item adjective \\
You use \textbf{broad} to describe a word or meaning which covers or refers to a wide range of different things.
 \textit{
	\begin{itemize}
	\item The term Wissenschaft has a much broader meaning than the English word 'science'.
	\item ...restructuring in the broad sense of the word.
	\end{itemize}
}
\item adjective \\
You use \textbf{broad} to describe a feeling or opinion that is shared by many people, or by people of many different kinds .
 \textit{
	\begin{itemize}
	\item The agreement won broad support in the U.S. Congress.
	\item ...a film with broad appeal.
	\end{itemize}
}
\item adjective \\
A \textbf{broad}  description or idea is general rather than detailed.
 \textit{
	\begin{itemize}
	\item These documents provided a broad outline of the Society's development.
	\item We have discussed in broad terms the course of action appropriate at each stage.
	\end{itemize}
}
\item adjective \\
A \textbf{broad}  hint is a very obvious hint.
 \textit{
	\begin{itemize}
	\item They've been giving broad hints about what to expect.
	\end{itemize}
}
\item adjective \\
A \textbf{broad}  accent is strong and noticeable .
 \textit{
	\begin{itemize}
	\item ...a Briton who spoke in a broad Yorkshire accent.
	\end{itemize}
}
\item countable noun \\
Some men refer to women as \textbf{broads} . This use could cause offence .
 \textit{
	\begin{itemize}
	\end{itemize}
}
\end{enumerate}

\section*{amount}
{\large \color{blue}  amounts  amounting  amounted  }
\subsection*{Explain}
\begin{enumerate}
\item variable noun \\
The \textbf{amount}  \textbf{of} something is how much there is, or how much you have, need , or get .
 \textit{
	\begin{itemize}
	\item He needs that amount of money to survive.
	\item I still do a certain amount of work for them.
	\item Postal money orders are available in amounts up to $700.
	\end{itemize}
}
\item verb \\
If something \textbf{amounts}  \textbf{to} a particular total, all the parts of it add up to that total.
 \textit{
	\begin{itemize}
	\item Consumer spending on sports-related items amounted to £9.75 billion.
	\end{itemize}
}
\item  \\
 any amount of something \textit{
	\begin{itemize}
	\end{itemize}
}
\end{enumerate}

\section*{competitive}
{\large \color{blue}  }
\subsection*{Explain}
\begin{enumerate}
\item adjective \\
\textbf{Competitive} is used to describe  situations or activities in which people or firms compete with each other.
 \textit{
	\begin{itemize}
	\item Only by keeping down costs will America maintain its competitive advantage over other
countries.
	\item Japan is a highly competitive market system.
	\item Universities are very competitive for the best students.
	\end{itemize}
}
\item adjective \\
A \textbf{competitive} person is eager to be more successful than other people.
 \textit{
	\begin{itemize}
	\item He has always been ambitious and fiercely competitive.
	\item I'm a very competitive person and I was determined not be beaten.
	\end{itemize}
}
\item adjective \\
Goods or services that are at a \textbf{competitive} price or rate are likely to be bought , because they are less expensive than other goods of the same kind .
 \textit{
	\begin{itemize}
	\item Homes for sale at competitive prices will secure interest from serious purchasers.
	\item ...a travel company specialising in amazingly competitive rates for flights.
	\end{itemize}
}
\end{enumerate}

\section*{analogue}
{\large \color{blue}  analogues  }
\subsection*{Explain}
\begin{enumerate}
\item countable noun \\
If one thing is an \textbf{analogue}  \textbf{of} another, it is similar in some way.
 \textit{
	\begin{itemize}
	\item No model can ever be a perfect analogue of nature itself.
	\end{itemize}
}
\item adjective \\
\textbf{Analogue}  technology involves measuring, storing , or recording an infinitely variable amount of information by using physical quantities such as voltage.
 \textit{
	\begin{itemize}
	\item The analogue signals from the video tape are converted into digital code.
	\end{itemize}
}
\item adjective \\
An \textbf{analogue}  watch or clock  shows what it is measuring with a pointer on a dial rather than with a number display . Compare  digital .
 \textit{
	\begin{itemize}
	\end{itemize}
}
\end{enumerate}

\section*{cool}
{\large \color{blue}  cooler  coolest  cools  cooling  cooled  }
\subsection*{Explain}
\begin{enumerate}
\item adjective \\
Something that is \textbf{cool} has a temperature which is low but not very low.
 \textit{
	\begin{itemize}
	\item I felt a current of cool air.
	\item The water was slightly cooler than a child's bath.
	\item The vaccines were kept cool in refrigerators.
	\end{itemize}
}
\item adjective \\
If it is \textbf{cool} , or if a place is \textbf{cool} , the temperature of the air is low but not very low.
 \textbf{Cool} is also a noun .
 \textit{
	\begin{itemize}
	\item Thank goodness it's cool in here.
	\item Store grains and cereals in a cool, dry place.
	\item ...a cool November evening.
	\item She walked into the cool of the hallway.
	\end{itemize}
}
\item adjective \\
Clothing that is \textbf{cool} is made of thin material so that you do not become too hot in hot weather .
 \textit{
	\begin{itemize}
	\item In warm weather, you should wear clothing that is cool and comfortable.
	\end{itemize}
}
\item adjective \\
\textbf{Cool} colours are light colours which give an impression of coolness.
 \textit{
	\begin{itemize}
	\item Choose a cool colour such as cream.
	\item The drawing-room was a cool silver green.
	\end{itemize}
}
\item verb \\
When something \textbf{cools} or when you \textbf{cool} it, it becomes lower in temperature.
 To \textbf{cool down} means the same as to cool .
 \textit{
	\begin{itemize}
	\item Drain the meat and allow it to cool.
	\item Huge fans will have to cool the concrete floor to keep it below 150 degrees.
	\item ...a cooling breeze.
	\item Avoid putting your car away until the engine has cooled down.
	\item The other main way the body cools itself down is by panting.
	\end{itemize}
}
\item verb \\
When a feeling or emotion \textbf{cools} , or when you \textbf{cool} it, it becomes less powerful .
 \textit{
	\begin{itemize}
	\item Within a few minutes tempers had cooled.
	\item His weird behaviour had cooled her passion.
	\end{itemize}
}
\item adjective \\
If you say that a person or their behaviour is \textbf{cool} , you mean that they are calm and unemotional , especially in a difficult situation.
 \textit{
	\begin{itemize}
	\item He was marvellously cool again, smiling as if nothing had happened.
	\item At that, Reno lost her cool composure.
	\end{itemize}
}
\item adjective \\
If you say that a person or their behaviour is \textbf{cool} , you mean that they are unfriendly or not enthusiastic .
 \textit{
	\begin{itemize}
	\item I didn't like him at all. I thought he was cool, aloof, and arrogant.
	\item The idea met with a cool response.
	\item He was given a cool reception.
	\end{itemize}
}
\item adjective \\
If you say that a person or their behaviour is \textbf{cool} , you mean that they are fashionable and attractive .
 \textit{
	\begin{itemize}
	\item He was trying to be really cool and trendy.
	\item ...some 15-year-old kid who thinks it's cool to do heroin.
	\end{itemize}
}
\item adjective \\
If you say that someone is \textbf{cool}  \textbf{about} something, you mean that they accept it and are not angry or upset about it.
 \textit{
	\begin{itemize}
	\item Bev was really cool about it all.
	\end{itemize}
}
\item adjective \\
If you say that something is \textbf{cool} , you think it is very good.
 \textit{
	\begin{itemize}
	\item Kathleen gave me a really cool dress.
	\end{itemize}
}
\item adjective \\
You can use \textbf{cool} to emphasize that an amount or figure is very large, especially when it has been obtained easily .
 \textit{
	\begin{itemize}
	\item Columbia recently re-signed the band for a cool $30 million.
	\end{itemize}
}
\item  \\
 cool it \textit{
	\begin{itemize}
	\end{itemize}
}
\item  \\
 keep your cool \textit{
	\begin{itemize}
	\end{itemize}
}
\item  \\
 play it cool \textit{
	\begin{itemize}
	\end{itemize}
}
\end{enumerate}

\section*{analogy}
{\large \color{blue}  analogies  }
\subsection*{Explain}
\begin{enumerate}
\item countable noun \\
If you make or draw an \textbf{analogy}  \textbf{between} two things, you show that they are similar in some way.
 \textit{
	\begin{itemize}
	\item Once again, Hockett draws an analogy with American football
	\end{itemize}
}
\end{enumerate}

\section*{crisp}
{\large \color{blue}  crisper  crispest  crisps  crisping  crisped  }
\subsection*{Explain}
\begin{enumerate}
\item adjective \\
Food that is \textbf{crisp} is pleasantly hard , or has a pleasantly hard surface.
 \textit{
	\begin{itemize}
	\item Bake the potatoes for 15 minutes, till they're nice and crisp.
	\item ...crisp bacon.
	\item ...crisp lettuce.
	\end{itemize}
}
\item verb \\
If food \textbf{crisps} or if you \textbf{crisp} it, it becomes pleasantly hard, for example because you have heated it at a high temperature .
 \textit{
	\begin{itemize}
	\item Cook the bacon until it begins to crisp.
	\item Spread breadcrumbs on a dry baking sheet and crisp them in the oven.
	\end{itemize}
}
\item countable noun \\
\textbf{Crisps} are very thin slices of fried potato that are eaten cold as a snack.
 \textit{
	\begin{itemize}
	\item ...a packet of crisps.
	\item ...cheese and onion potato crisps.
	\end{itemize}
}
\item adjective \\
Weather that is pleasantly fresh, cold, and dry can be described as \textbf{crisp} .
 \textit{
	\begin{itemize}
	\item ...a crisp autumn day.
	\end{itemize}
}
\item adjective \\
\textbf{Crisp}  cloth or paper is clean and has no creases in it.
 \textit{
	\begin{itemize}
	\item He wore a panama hat and a crisp white suit.
	\item I slipped between the crisp clean sheets.
	\item ...crisp banknotes.
	\end{itemize}
}
\item graded adjective \\
Leaves or snow that make a loud  noise when you walk on them can be described as \textbf{crisp} .
 \textit{
	\begin{itemize}
	\item ...crisp autumn leaves.
	\item He crunched through the crisp snow.
	\end{itemize}
}
\item graded adjective \\
If you describe someone's writing or speech as \textbf{crisp} , you mean they write or speak very clearly , without mentioning  unnecessary  details . This may make them seem  unfriendly .
 \textit{
	\begin{itemize}
	\item 'Very well,' I said, adopting a crisp authoritative tone.
	\end{itemize}
}
\item  \\
 be burnt to a crisp \textit{
	\begin{itemize}
	\end{itemize}
}
\end{enumerate}

\section*{beauty}
{\large \color{blue}  beauties  }
\subsection*{Explain}
\begin{enumerate}
\item uncountable noun \\
\textbf{Beauty} is the state or quality of being beautiful .
 \textit{
	\begin{itemize}
	\item ...an area of outstanding natural beauty.
	\item Everyone admired her elegance and her beauty.
	\end{itemize}
}
\item countable noun \\
A \textbf{beauty} is a beautiful woman.
 \textit{
	\begin{itemize}
	\item She is known as a great beauty.
	\end{itemize}
}
\item countable noun \\
You can say that something is a \textbf{beauty} when you think it is very good.
 \textit{
	\begin{itemize}
	\item The pass was a real beauty, but the shot was poor.
	\end{itemize}
}
\item countable noun \\
The \textbf{beauties} of something are its attractive qualities or features.
 \textit{
	\begin{itemize}
	\item He was beginning to enjoy the beauties of nature.
	\end{itemize}
}
\item adjective \\
\textbf{Beauty} is used to describe people, products, and activities that are concerned with making women look beautiful.
 \textit{
	\begin{itemize}
	\item Additional beauty treatments can be booked in advance.
	\end{itemize}
}
\item countable noun \\
If you say that a particular feature is \textbf{the}  \textbf{beauty}  \textbf{of} something, you mean that this feature is what makes the thing so good.
 \textit{
	\begin{itemize}
	\item There would be no effect on animals–that's the beauty of such water-based materials.
	\end{itemize}
}
\end{enumerate}

\section*{desirable}
{\large \color{blue}  }
\subsection*{Explain}
\begin{enumerate}
\item adjective \\
Something that is \textbf{desirable} is worth having or doing because it is useful , necessary , or popular .
 \textit{
	\begin{itemize}
	\item Prolonged negotiation was not desirable.
	\item The crowd moved indoors for what were deemed the most desirable items.
	\end{itemize}
}
\item adjective \\
Someone who is \textbf{desirable} is considered to be sexually attractive.
 \textit{
	\begin{itemize}
	\item ...the young women whom his classmates thought most desirable.
	\end{itemize}
}
\end{enumerate}

\section*{bomb}
{\large \color{blue}  bombs  bombing  bombed  }
\subsection*{Explain}
\begin{enumerate}
\item countable noun \\
A \textbf{bomb} is a device which explodes and damages or destroys a large area.
 \textit{
	\begin{itemize}
	\item Bombs went off at two London train stations.
	\item It's not known who planted the bomb.
	\item Most of the bombs fell in the south.
	\item There were two bomb explosions in the city overnight.
	\end{itemize}
}
\item singular noun \\
Nuclear  weapons are sometimes  referred to as \textbf{the bomb} .
 \textit{
	\begin{itemize}
	\item They are generally thought to have the bomb.
	\end{itemize}
}
\item verb \\
When people \textbf{bomb} a place, they attack it with bombs.
 \textit{
	\begin{itemize}
	\item Airforce jets bombed the airport.
	\end{itemize}
}
\end{enumerate}

\section*{domestic}
{\large \color{blue}  domestics  }
\subsection*{Explain}
\begin{enumerate}
\item adjective \\
\textbf{Domestic}  political activities, events, and situations  happen or exist within one particular country.
 \textit{
	\begin{itemize}
	\item ...over 100 domestic flights a day to 15 U.K. destinations.
	\item ...sales in the domestic market.
	\end{itemize}
}
\item  \\
 See also  gross domestic product \textit{
	\begin{itemize}
	\end{itemize}
}
\item adjective \\
\textbf{Domestic}  duties and activities are concerned with the running of a home and family.
 \textit{
	\begin{itemize}
	\item ...a plan for sharing domestic chores.
	\end{itemize}
}
\item adjective \\
\textbf{Domestic}  items and services are intended to be used in people's homes rather than in factories or offices .
 \textit{
	\begin{itemize}
	\item ...domestic appliances.
	\end{itemize}
}
\item adjective \\
A \textbf{domestic} situation or atmosphere is one which involves a family and their home.
 \textit{
	\begin{itemize}
	\item It was a scene of such domestic bliss.
	\item I was called out to attend a domestic dispute.
	\end{itemize}
}
\item graded adjective \\
Someone who is \textbf{domestic} enjoys being at home and running a family.
 \textit{
	\begin{itemize}
	\item She was kind and domestic and put her family before her part-time job.
	\end{itemize}
}
\item adjective \\
A \textbf{domestic} animal is one that is not wild and is kept either on a farm to produce food or in someone's home as a pet.
 \textit{
	\begin{itemize}
	\item ...a domestic cat.
	\end{itemize}
}
\item countable noun \\
A \textbf{domestic} , a \textbf{domestic help} , or a \textbf{domestic worker} is a person who is paid to come to help with the work that has to be done in a house such as the cleaning , washing , and ironing .
 \textit{
	\begin{itemize}
	\end{itemize}
}
\end{enumerate}

\section*{capital}
{\large \color{blue}  capitals  }
\subsection*{Explain}
\begin{enumerate}
\item uncountable noun \\
\textbf{Capital} is a large sum of money which you use to start a business, or which you invest in order to make more money.
 \textit{
	\begin{itemize}
	\item Companies are having difficulty in raising capital.
	\item A large amount of capital is invested in all these branches.
	\end{itemize}
}
\item uncountable noun \\
You can use \textbf{capital} to refer to buildings or machinery which are necessary to produce goods or to make companies more efficient , but which do not make money directly .
 \textit{
	\begin{itemize}
	\item ...capital equipment that could have served to increase production.
	\item ...capital investment.
	\end{itemize}
}
\item uncountable noun \\
\textbf{Capital} is the part of an amount of money borrowed or invested which does not include interest.
 \textit{
	\begin{itemize}
	\item With a conventional repayment mortgage, the repayments consist of both capital and
interest.
	\end{itemize}
}
\item countable noun \\
The \textbf{capital} of a country is the city or town where its government or parliament  meets .
 \textit{
	\begin{itemize}
	\item ...Kathmandu, the capital of Nepal.
	\end{itemize}
}
\item countable noun \\
If a place is \textbf{the}  \textbf{capital}  \textbf{of} a particular industry or activity, it is the place that is most famous for it, because it happens in that place more than anywhere else.
 \textit{
	\begin{itemize}
	\item Colmar has long been considered the capital of the wine trade.
	\item ...New York, the fashion capital of the world.
	\end{itemize}
}
\item countable noun \\
\textbf{Capitals} or \textbf{capital letters} are written or printed letters in the form which is used at the beginning of sentences or names. ' T ', ' B ', and ' F ' are capitals.
 \textit{
	\begin{itemize}
	\item The name and address are written in capitals.
	\end{itemize}
}
\item adjective \\
A \textbf{capital}  offence is one that is so serious that the person who commits it can be punished by death.
 \textit{
	\begin{itemize}
	\item Espionage is a capital offence in this country.
	\item ...Americans wrongly convicted of capital crimes.
	\end{itemize}
}
\item countable noun \\
A \textbf{capital} is the top part of a stone column, which is sometimes decorated with stone leaves or other patterns.
 \textit{
	\begin{itemize}
	\end{itemize}
}
\item  \\
 make capital of sth/make capital out of sth \textit{
	\begin{itemize}
	\end{itemize}
}
\item  \\
 with a capital \textit{
	\begin{itemize}
	\end{itemize}
}
\end{enumerate}

\section*{easy}
{\large \color{blue}  easier  easiest  }
\subsection*{Explain}
\begin{enumerate}
\item adjective \\
If a job or action is \textbf{easy} , you can do it without difficulty or effort, because it is not complicated and causes no problems .
 \textit{
	\begin{itemize}
	\item The shower is easy to install.
	\item It's easy to get a seat at the best shows in town.
	\item This is not an easy task.
	\item The home is situated within easy access of shops and other facilities.
	\end{itemize}
}
\item adjective \\
If you describe an action or activity as \textbf{easy} , you mean that it is done in a confident , relaxed way. If someone is \textbf{easy}  \textbf{about} something, they feel relaxed and confident about it.
 \textit{
	\begin{itemize}
	\item She is laughing and joking and making easy conversation with people she has never
met before.
	\item He was an easy person to talk to.
	\item ...when you are both feeling a little easier about the break up of your relationship.
	\end{itemize}
}
\item adjective \\
If you say that someone has an \textbf{easy} life, you mean that they live comfortably without any problems or worries .
 \textit{
	\begin{itemize}
	\item She has not had an easy life.
	\end{itemize}
}
\item adjective \\
If you say that something is \textbf{easy} or too \textbf{easy} , you are criticizing someone because they have done the most obvious or least difficult thing, and have not considered the situation carefully enough.
 \textit{
	\begin{itemize}
	\item That's easy for you to say.
	\item It was all too easy to believe it.
	\end{itemize}
}
\item graded adjective \\
You use \textbf{easy} in expressions such as \textbf{easy on the ear} or \textbf{easy on the eye} when you are describing things that are pleasant and that do not need much effort to be enjoyed or done.
 \textit{
	\begin{itemize}
	\item The music sounds like an advert–easy on the ear but bland and forgettable.
	\item The layout should be clear and easy on the eye.
	\item ...a low-impact form of aerobic exercise that's easy on the joints.
	\end{itemize}
}
\item adjective \\
If you describe someone or something as \textbf{easy prey} or as an \textbf{easy target} , you mean that they can easily be attacked or criticized.
 \textit{
	\begin{itemize}
	\item Tourists have become easy prey.
	\item Leaders are easy targets for critics, because they have visibility.
	\end{itemize}
}
\item  \\
 easy come, easy go \textit{
	\begin{itemize}
	\end{itemize}
}
\item  \\
 easy does it \textit{
	\begin{itemize}
	\end{itemize}
}
\item  \\
 go easy on sth \textit{
	\begin{itemize}
	\end{itemize}
}
\item  \\
 go/be easy on sb \textit{
	\begin{itemize}
	\end{itemize}
}
\item  \\
 easier said than done \textit{
	\begin{itemize}
	\end{itemize}
}
\item  \\
 take it easy \textit{
	\begin{itemize}
	\end{itemize}
}
\end{enumerate}

\section*{captain}
{\large \color{blue}  captains  captaining  captained  }
\subsection*{Explain}
\begin{enumerate}
\item title noun \\
In the army, navy, and some other armed forces, a \textbf{captain} is an officer of middle rank.
 \textit{
	\begin{itemize}
	\item ...Captain Mark Phillips.
	\item ...a captain in the British army.
	\item Are all your weapons in place, Captain?
	\end{itemize}
}
\item countable noun \\
The \textbf{captain}  \textbf{of} a sports team is the player in charge of it.
 \textit{
	\begin{itemize}
	\item ...Mickey Thomas, the captain of Wrexham football club.
	\item The former Australia cricket captain offers a unique insight into his nation's sporting
psyche.
	\end{itemize}
}
\item countable noun \\
The \textbf{captain} of a ship is the sailor in charge of it.
 \textit{
	\begin{itemize}
	\item ...the captain of the aircraft carrier Saratoga.
	\end{itemize}
}
\item countable noun \\
The \textbf{captain} of an aeroplane is the pilot in charge of it.
 \textit{
	\begin{itemize}
	\end{itemize}
}
\item countable noun \\
In the United  States and some other countries, a \textbf{captain} is a police officer or firefighter of fairly senior rank.
 \textit{
	\begin{itemize}
	\end{itemize}
}
\item verb \\
If you \textbf{captain} a team or a ship, you are the captain of it.
 \textit{
	\begin{itemize}
	\item Two months later, he captained Pakistan to victory in the World Cup.
	\item I did once dream of becoming the first woman to captain an ocean liner.
	\end{itemize}
}
\end{enumerate}

\section*{epidemic}
{\large \color{blue}  epidemics  }
\subsection*{Explain}
\begin{enumerate}
\item countable noun \\
If there is an \textbf{epidemic}  \textbf{of} a particular disease somewhere , it affects a very large number of people there and spreads quickly to other areas.
 \textit{
	\begin{itemize}
	\item A flu epidemic is sweeping through Moscow.
	\item ...a killer epidemic of yellow fever.
	\end{itemize}
}
\item countable noun \\
If an activity that you disapprove of is increasing or spreading rapidly, you can refer to this as an \textbf{epidemic}  \textbf{of} that activity.
 \textit{
	\begin{itemize}
	\item ...an epidemic of serial killings.
	\item Drug experts say it could spell the end of the crack epidemic.
	\end{itemize}
}
\end{enumerate}

\section*{cave}
{\large \color{blue}  caves  caving  caved  }
\subsection*{Explain}
\begin{enumerate}
\item countable noun \\
A \textbf{cave} is a large hole in the side of a cliff or hill , or one that is under the ground.
 \textit{
	\begin{itemize}
	\item ...a cave more than 1,000 feet deep.
	\end{itemize}
}
\end{enumerate}

\section*{fashionable}
{\large \color{blue}  }
\subsection*{Explain}
\begin{enumerate}
\item adjective \\
Something or someone that is \textbf{fashionable} is popular or approved of at a particular time.
 \textit{
	\begin{itemize}
	\item It became fashionable to eat certain kinds of fish.
	\item Chelsea Harbour is renowned for its fashionable restaurants.
	\end{itemize}
}
\end{enumerate}

\section*{concert}
{\large \color{blue}  concerts  }
\subsection*{Explain}
\begin{enumerate}
\item countable noun \\
A \textbf{concert} is a performance of music.
 \textit{
	\begin{itemize}
	\item ...a short concert of piano music.
	\item I've been to plenty of live rock concerts.
	\item ...a new concert hall.
	\end{itemize}
}
\item  \\
 in concert \textit{
	\begin{itemize}
	\end{itemize}
}
\item  \\
 in concert \textit{
	\begin{itemize}
	\end{itemize}
}
\end{enumerate}

\section*{fluent}
{\large \color{blue}  }
\subsection*{Explain}
\begin{enumerate}
\item adjective \\
Someone who is \textbf{fluent}  \textbf{in} a particular language can speak the language easily and correctly. You can also  say that someone speaks \textbf{fluent}  French , Chinese , or some other language.
 \textit{
	\begin{itemize}
	\item She studied eight foreign languages but is fluent in only six of them.
	\item He speaks fluent Russian.
	\end{itemize}
}
\item adjective \\
If your speech , reading , or writing is \textbf{fluent} , you speak, read , or write easily, smoothly, and clearly with no mistakes .
 \textit{
	\begin{itemize}
	\item He had emerged from being a hesitant and unsure candidate into a fluent debater.
	\end{itemize}
}
\end{enumerate}

\section*{difference}
{\large \color{blue}  differences  }
\subsection*{Explain}
\begin{enumerate}
\item countable noun \\
The \textbf{difference} between two things is the way in which they are unlike each other.
 \textit{
	\begin{itemize}
	\item That is the fundamental difference between the two societies.
	\item There is no difference between the sexes.
	\item ...the vast difference in size.
	\item We do have problems here. The difference is that people try to help each other.
	\end{itemize}
}
\item singular noun \\
A \textbf{difference} between two quantities is the amount by which one quantity is less than the other.
 \textit{
	\begin{itemize}
	\item The difference is 8532.
	\end{itemize}
}
\item countable noun \\
If people have their \textbf{differences} about something, they disagree about it.
 \textit{
	\begin{itemize}
	\item The two communities are learning how to resolve their differences.
	\end{itemize}
}
\item  \\
 make a/no difference \textit{
	\begin{itemize}
	\end{itemize}
}
\item  \\
 split the difference \textit{
	\begin{itemize}
	\end{itemize}
}
\item  \\
 with a difference \textit{
	\begin{itemize}
	\end{itemize}
}
\item  \\
 difference of opinion \textit{
	\begin{itemize}
	\end{itemize}
}
\end{enumerate}

\section*{fluid}
{\large \color{blue}  fluids  }
\subsection*{Explain}
\begin{enumerate}
\item variable noun \\
A \textbf{fluid} is a liquid.
 \textit{
	\begin{itemize}
	\item The blood vessels may leak fluid, which distorts vision.
	\item Make sure that you drink plenty of fluids.
	\item ...fluid retention.
	\end{itemize}
}
\item adjective \\
\textbf{Fluid} movements or lines or designs are smooth and graceful .
 \textit{
	\begin{itemize}
	\item The forehand stroke should be fluid and well balanced.
	\item ...long, fluid dresses.
	\end{itemize}
}
\item adjective \\
A situation that is \textbf{fluid} is unstable and is likely to change often.
 \textit{
	\begin{itemize}
	\item The situation is extremely fluid and it can be changing from day to day.
	\end{itemize}
}
\end{enumerate}

\section*{empire}
{\large \color{blue}  empires  }
\subsection*{Explain}
\begin{enumerate}
\item countable noun \\
An \textbf{empire} is a number of individual  nations that are all controlled by the government or ruler of one particular country.
 \textit{
	\begin{itemize}
	\item ...the Roman Empire.
	\end{itemize}
}
\item countable noun \\
You can refer to a group of companies controlled by one person as an \textbf{empire} .
 \textit{
	\begin{itemize}
	\item ...the big Mondadori publishing empire.
	\end{itemize}
}
\end{enumerate}

\section*{generous}
{\large \color{blue}  }
\subsection*{Explain}
\begin{enumerate}
\item adjective \\
A \textbf{generous} person gives more of something, especially money, than is usual or expected .
 \textit{
	\begin{itemize}
	\item German banks are more generous in their lending.
	\item The gift is generous by any standards.
	\end{itemize}
}
\item adjective \\
A \textbf{generous} person is friendly , helpful , and willing to see the good qualities in someone or something.
 \textit{
	\begin{itemize}
	\item He was always generous in sharing his enormous knowledge.
	\item He was generous enough to congratulate his successor on his decision.
	\end{itemize}
}
\item adjective \\
A \textbf{generous} amount of something is much larger than is usual or necessary .
 \textit{
	\begin{itemize}
	\item ...a generous six weeks of annual holiday.
	\item He should be able to keep his room tidy with the generous amount of storage space.
	\end{itemize}
}
\end{enumerate}

\section*{executive}
{\large \color{blue}  executives  }
\subsection*{Explain}
\begin{enumerate}
\item countable noun \\
An \textbf{executive} is someone who is employed by a business at a senior level. Executives decide what the business should do, and ensure that it is done.
 \textit{
	\begin{itemize}
	\item ...an advertising executive.
	\item She is a senior bank executive.
	\end{itemize}
}
\item adjective \\
The \textbf{executive} sections and tasks of an organization are concerned with the making of decisions and with ensuring that decisions are carried out.
 \textit{
	\begin{itemize}
	\item A successful job search needs to be as well organised as any other executive task.
	\item I will not take an executive role, but rather become a consultant.
	\end{itemize}
}
\item adjective \\
\textbf{Executive} goods are expensive goods designed or intended for executives and other people at a similar social or economic level.
 \textit{
	\begin{itemize}
	\item ...an executive briefcase.
	\item ...a chain of shops specialising in pricey executive toys.
	\item ...executive cars.
	\end{itemize}
}
\item singular noun \\
The \textbf{executive}  committee or board of an organization is a committee within that organization which has the authority
to make decisions and ensures that these decisions are carried out.
 \textit{
	\begin{itemize}
	\item He sits on the executive committee that manages Lloyds.
	\item Some executive members have called for his resignation.
	\item ...the executive of the National Union of Students.
	\end{itemize}
}
\item singular noun \\
\textbf{The executive} is the part of the government of a country that is concerned with carrying out decisions
or orders, as opposed to the part that makes laws or the part that deals with criminals .
 \textit{
	\begin{itemize}
	\item The government, the executive and the judiciary are supposed to be separate.
	\item The matter should be resolved by the executive branch of government.
	\end{itemize}
}
\end{enumerate}

\section*{glad}
{\large \color{blue}  }
\subsection*{Explain}
\begin{enumerate}
\item adjective \\
If you are \textbf{glad} about something, you are happy and pleased about it.
 \textit{
	\begin{itemize}
	\item I'm glad I relented in the end.
	\item The people seem genuinely glad to see you.
	\item I ought to be glad about what happened.
	\item I'd be glad if the boys slept a little longer so I could do some ironing.
	\end{itemize}
}
\item adjective \\
If you say that you will be \textbf{glad}  \textbf{to} do something, usually for someone else, you mean that you are willing and eager to do it.
 \textit{
	\begin{itemize}
	\item I'll be glad to show you everything.
	\item We should be glad to answer any questions.
	\end{itemize}
}
\item  \\
 glad tidings \textit{
	\begin{itemize}
	\end{itemize}
}
\end{enumerate}

\section*{flood}
{\large \color{blue}  floods  flooding  flooded  }
\subsection*{Explain}
\begin{enumerate}
\item variable noun \\
If there is a \textbf{flood} , a large amount of water covers an area which is usually dry, for example when a river flows over its banks or a pipe  bursts .
 \textit{
	\begin{itemize}
	\item More than 70 people were killed in the floods, caused when a dam burst.
	\item This is the type of flood dreaded by cavers.
	\item Over 25 people drowned when a schoolbus tried to cross a river and flood waters swept
through.
	\end{itemize}
}
\item verb \\
If something such as a river or a burst pipe \textbf{floods} an area that is usually dry or if the area \textbf{floods} , it becomes covered with water.
 \textit{
	\begin{itemize}
	\item The Chicago River flooded the city's underground tunnel system.
	\item The kitchen flooded.
	\end{itemize}
}
\item verb \\
If a river \textbf{floods} , it overflows, especially after very heavy rain .
 \textit{
	\begin{itemize}
	\item ...the relentless rain that caused twenty rivers to flood.
	\item Many streams have flooded their banks, making some roads impassable.
	\end{itemize}
}
\item countable noun \\
If you say that a \textbf{flood}  \textbf{of} people or things arrive  somewhere , you are emphasizing that a very large number of them arrive there.
 \textit{
	\begin{itemize}
	\item ...a flood of new university graduates.
	\item He received a flood of letters from irate constituents.
	\end{itemize}
}
\item verb \\
If you say that people or things \textbf{flood} into a place, you are emphasizing that they arrive there in large numbers .
 \textit{
	\begin{itemize}
	\item Enquiries flooded in from all over the world.
	\item They flooded out of the ground in their thousands long before the final whistle.
	\end{itemize}
}
\item verb \\
If you \textbf{flood} a place \textbf{with} a particular type of thing, or if a particular type of thing \textbf{floods} a place, the place becomes full of so many of them that it cannot hold or deal with any more.
 \textit{
	\begin{itemize}
	\item ...a policy aimed at flooding Europe with exports.
	\item Brokers expect the markets to be flooded with the shares.
	\item German cameras at knock-down prices flooded the British market.
	\end{itemize}
}
\item verb \\
If an emotion , feeling , or thought  \textbf{floods} you, you suddenly  feel it very intensely. If feelings or memories  \textbf{flood back} , you suddenly remember them very clearly .
 \textit{
	\begin{itemize}
	\item A wave of happiness flooded me.
	\item Mary Ann was flooded with relief.
	\item It was probably the shock which had brought all the memories flooding back.
	\end{itemize}
}
\item verb \\
If light  \textbf{floods} a place or \textbf{floods} into it, it suddenly fills it.
 \textit{
	\begin{itemize}
	\item The afternoon light flooded the little rooms.
	\item Morning sunshine flooded in through the open curtains.
	\end{itemize}
}
\item  \\
 in flood \textit{
	\begin{itemize}
	\end{itemize}
}
\item  \\
 floods of tears \textit{
	\begin{itemize}
	\end{itemize}
}
\end{enumerate}

\section*{healthy}
{\large \color{blue}  healthier  healthiest  }
\subsection*{Explain}
\begin{enumerate}
\item adjective \\
Someone who is \textbf{healthy} is well and is not suffering from any illness .
 \textit{
	\begin{itemize}
	\item Most of us need to lead more balanced lives to be healthy and happy.
	\item She had a normal pregnancy and delivered a healthy child.
	\end{itemize}
}
\item adjective \\
If a feature or quality that you have is \textbf{healthy} , it makes you look well or shows that you are well.
 \textit{
	\begin{itemize}
	\item ...the glow of healthy skin.
	\item ...young adults with healthy appetites.
	\end{itemize}
}
\item adjective \\
Something that is \textbf{healthy} is good for your health.
 \textit{
	\begin{itemize}
	\item ...a great healthy outdoor pursuit.
	\item ...a healthy diet.
	\end{itemize}
}
\item adjective \\
A \textbf{healthy}  organization or system is successful .
 \textit{
	\begin{itemize}
	\item ...an economically healthy socialist state.
	\end{itemize}
}
\item adjective \\
A \textbf{healthy} amount of something is a large amount that shows success .
 \textit{
	\begin{itemize}
	\item He predicts a continuation of healthy profits in the current financial year.
	\item ...a healthy bank account.
	\end{itemize}
}
\item adjective \\
If you have a \textbf{healthy}  attitude about something, you show good sense .
 \textit{
	\begin{itemize}
	\item She has a refreshingly healthy attitude to work.
	\item It's very healthy to be afraid when there's something to be afraid of.
	\end{itemize}
}
\end{enumerate}

\section*{forehead}
{\large \color{blue}  foreheads  }
\subsection*{Explain}
\begin{enumerate}
\item countable noun \\
Your \textbf{forehead} is the area at the front of your head between your eyebrows and your hair .
 \textit{
	\begin{itemize}
	\end{itemize}
}
\end{enumerate}

\section*{gesture}
{\large \color{blue}  gestures  gesturing  gestured  }
\subsection*{Explain}
\begin{enumerate}
\item countable noun \\
A \textbf{gesture} is a movement that you make with a part of your body, especially your hands, to express emotion or information .
 \textit{
	\begin{itemize}
	\item Sarah made a menacing gesture with her fist.
	\item He throws his hands open in a gesture which clearly indicates his relief.
	\end{itemize}
}
\item countable noun \\
A \textbf{gesture} is something that you say or do in order to express your attitude or intentions, often something that you know  will not have much effect .
 \textit{
	\begin{itemize}
	\item I wasn't expecting a fortune, just a gesture of goodwill.
	\item There's not greater gesture of love than having someone's name tattooed on your body.
	\item As a gesture to security, cars were fitted with special locks.
	\end{itemize}
}
\item verb \\
If you \textbf{gesture} , you use movements of your hands or head in order to tell someone something or draw their attention to something.
 \textit{
	\begin{itemize}
	\item I gestured towards the boathouse, and he looked inside.
	\item He gestures, gesticulates, and moves with the grace of a dancer.
	\end{itemize}
}
\end{enumerate}

\section*{marine}
{\large \color{blue}  marines  }
\subsection*{Explain}
\begin{enumerate}
\item countable noun \\
A \textbf{marine} is a member of an armed force, for example the US Marine Corps or the Royal Marines, who is specially trained for military  duties at sea as well as on land.
 \textit{
	\begin{itemize}
	\end{itemize}
}
\item adjective \\
\textbf{Marine} is used to describe things relating to the sea or to the animals and plants that live in the sea.
 \textit{
	\begin{itemize}
	\item ...breeding grounds for marine life.
	\item ...research in marine biology.
	\end{itemize}
}
\item adjective \\
\textbf{Marine} is used to describe things relating to ships and their movement at sea.
 \textit{
	\begin{itemize}
	\item ...a solicitor specialising in marine law.
	\item ...marine insurance claims.
	\end{itemize}
}
\end{enumerate}

\section*{hole}
{\large \color{blue}  holes  holing  holed  }
\subsection*{Explain}
\begin{enumerate}
\item countable noun \\
A \textbf{hole} is a hollow space in something solid, with an opening on one side.
 \textit{
	\begin{itemize}
	\item He took a shovel, dug a hole, and buried his once-prized possessions.
	\item The builders had cut holes into the soft stone to support the ends of the beams.
	\item ...a 60ft hole.
	\end{itemize}
}
\item countable noun \\
A \textbf{hole} is an opening in something that goes right through it.
 \textit{
	\begin{itemize}
	\item These tiresome creatures eat holes in the leaves.
	\item Armed robbers broke into the jeweller's through a hole in the wall.
	\item ...kids with holes in the knees of their jeans.
	\end{itemize}
}
\item countable noun \\
A \textbf{hole} is the home or hiding place of a mouse , rabbit , or other small animal.
 \textit{
	\begin{itemize}
	\item ...a rabbit hole.
	\end{itemize}
}
\item countable noun \\
A \textbf{hole}  \textbf{in} a law, theory, or argument is a fault or weakness that it has.
 \textit{
	\begin{itemize}
	\item There were some holes in that theory, some unanswered questions.
	\end{itemize}
}
\item countable noun \\
If you refer to a place as a \textbf{hole} , you are emphasizing that you think it is very unpleasant .
 \textit{
	\begin{itemize}
	\item Why don't you leave this awful hole and come to live with me?
	\end{itemize}
}
\item countable noun \\
A \textbf{hole} is also one of the nine or eighteen sections of a golf course.
 \textit{
	\begin{itemize}
	\item I played nine holes with Gary today.
	\end{itemize}
}
\item countable noun \\
A \textbf{hole} is one of the places on a golf course that the ball must drop into, usually marked by a flag .
 \textit{
	\begin{itemize}
	\end{itemize}
}
\item verb \\
If you \textbf{hole} in a game of golf, you hit the ball so that it goes into the hole.
 \textit{
	\begin{itemize}
	\item He holed from nine feet at the 18th.
	\item Frost holed a bunker shot from 50 feet to snatch the title by one stroke.
	\end{itemize}
}
\item verb \\
If something such as a building or ship \textbf{is holed} , holes are made in it by guns or other weapons .
 \textit{
	\begin{itemize}
	\item Blocks of flats have been holed and some shells have fallen within the historic ramparts.
	\end{itemize}
}
\item  \\
 need sth/sb like a hole in the head \textit{
	\begin{itemize}
	\end{itemize}
}
\item  \\
 in a hole \textit{
	\begin{itemize}
	\end{itemize}
}
\item  \\
 a hole in one \textit{
	\begin{itemize}
	\end{itemize}
}
\item  \\
 pick holes \textit{
	\begin{itemize}
	\end{itemize}
}
\end{enumerate}

\section*{noble}
{\large \color{blue}  nobles  nobler  noblest  }
\subsection*{Explain}
\begin{enumerate}
\item adjective \\
If you say that someone is a \textbf{noble} person, you admire and respect them because they are unselfish and morally good.
 \textit{
	\begin{itemize}
	\item He was an upright and noble man who was always willing to help in any way he could.
	\item I wanted so much to believe he was pure and noble.
	\end{itemize}
}
\item adjective \\
If you say that something is a \textbf{noble}  idea , goal , or action, you admire it because it is based on high moral principles .
 \textit{
	\begin{itemize}
	\item He had implicit faith in the noble intentions of the Emperor.
	\item We'll always justify our actions with noble sounding theories.
	\item Their cause was noble.
	\end{itemize}
}
\item adjective \\
If you describe something as \textbf{noble} , you think that its appearance or quality is very impressive , making it superior to other things of its type.
 \textit{
	\begin{itemize}
	\item ...the great parks with their noble trees.
	\item The horse is the noblest animal.
	\end{itemize}
}
\item adjective \\
\textbf{Noble} means belonging to a high social class and having a title.
 \textit{
	\begin{itemize}
	\item ...rich and noble families.
	\item Although he was of noble birth, he lived as a poor man.
	\end{itemize}
}
\item countable noun \\
In former times, people who belonged to a high social class and had titles such as
'Baron' or 'Duke' were referred to as \textbf{nobles} .
 \textit{
	\begin{itemize}
	\item More and more nobles made Moscow their home during Catherine's reign.
	\end{itemize}
}
\end{enumerate}

\section*{insight}
{\large \color{blue}  insights  }
\subsection*{Explain}
\begin{enumerate}
\item variable noun \\
If you gain  \textbf{insight} or an \textbf{insight}  \textbf{into} a complex situation or problem, you gain an accurate and deep understanding of it.
 \textit{
	\begin{itemize}
	\item The project would give scientists new insights into what is happening to the Earth's
atmosphere.
	\item I hope that this talk has given you some insight into our work.
	\end{itemize}
}
\item uncountable noun \\
If someone has \textbf{insight} , they are able to understand complex situations.
 \textit{
	\begin{itemize}
	\item He was a man with considerable insight.
	\end{itemize}
}
\end{enumerate}

\section*{notorious}
{\large \color{blue}  }
\subsection*{Explain}
\begin{enumerate}
\item adjective \\
To be \textbf{notorious} means to be well-known for something bad.
 \textit{
	\begin{itemize}
	\item ...an area notorious for crime and violence.
	\item She told us the story of one of Britain's most notorious country house murders.
	\end{itemize}
}
\end{enumerate}

\section*{messenger}
{\large \color{blue}  messengers  }
\subsection*{Explain}
\begin{enumerate}
\item countable noun \\
A \textbf{messenger} takes a message to someone, or takes messages regularly as their job .
 \textit{
	\begin{itemize}
	\item There will be a messenger at the airport to collect the photographs from our courier.
	\item He gave the instruction for the document to be sent by messenger.
	\end{itemize}
}
\end{enumerate}

\section*{painful}
{\large \color{blue}  }
\subsection*{Explain}
\begin{enumerate}
\item adjective \\
If a part of your body is \textbf{painful} , it hurts because it is injured or because there is something wrong with it.
 \textit{
	\begin{itemize}
	\item Her glands were swollen and painful.
	\item Sampras awaits the results of a bone scan on a painful left shin.
	\end{itemize}
}
\item adjective \\
If something such as an illness , injury , or operation is \textbf{painful} , it causes you a lot of physical pain.
 \textit{
	\begin{itemize}
	\item ...a painful back injury.
	\item Sunburn is painful and potentially dangerous.
	\end{itemize}
}
\item adjective \\
Situations , memories , or experiences that are \textbf{painful} are difficult and unpleasant to deal with, and often make you feel  sad and upset .
 \textit{
	\begin{itemize}
	\item Remarks like that brought back painful memories.
	\item ...the painful transition to democracy.
	\item She finds it too painful to return there without him.
	\end{itemize}
}
\item adjective \\
If a performance or interview is \textbf{painful} , it is so bad that it makes you feel embarrassed for the people taking part in it.
 \textit{
	\begin{itemize}
	\item The interview was painful to watch.
	\end{itemize}
}
\end{enumerate}

\section*{monarch}
{\large \color{blue}  monarchs  }
\subsection*{Explain}
\begin{enumerate}
\item countable noun \\
The \textbf{monarch} of a country is the king, queen, emperor, or empress .
 \textit{
	\begin{itemize}
	\end{itemize}
}
\end{enumerate}

\section*{partial}
{\large \color{blue}  }
\subsection*{Explain}
\begin{enumerate}
\item adjective \\
You use \textbf{partial} to refer to something that is not complete or whole .
 \textit{
	\begin{itemize}
	\item He managed to reach a partial agreement with both republics.
	\item ...a partial ban on the use of cars in the city.
	\item ...partial blindness.
	\end{itemize}
}
\item adjective \\
If you are \textbf{partial to} something, you like it.
 \textit{
	\begin{itemize}
	\item He's partial to sporty women with blue eyes.
	\item Mollie confesses she is rather partial to pink.
	\item I am partial to baking cookies.
	\end{itemize}
}
\item adjective \\
Someone who is \textbf{partial} supports a particular person or thing, for example in a competition or dispute , instead of being completely fair .
 \textit{
	\begin{itemize}
	\item I might be accused of being partial.
	\end{itemize}
}
\end{enumerate}

\section*{music}
{\large \color{blue}  }
\subsection*{Explain}
\begin{enumerate}
\item uncountable noun \\
\textbf{Music} is the pattern of sounds produced by people singing or playing instruments.
 \textit{
	\begin{itemize}
	\item ...classical music.
	\item ...the music of George Gershwin.
	\item ...a mixture of music, dance, cabaret and children's theatre.
	\item ...a music critic for the New York Times.
	\end{itemize}
}
\item uncountable noun \\
\textbf{Music} is the art of creating or performing music.
 \textit{
	\begin{itemize}
	\item He went on to study music, specialising in the clarinet.
	\item ...a music lesson.
	\end{itemize}
}
\item uncountable noun \\
\textbf{Music} is the symbols written on paper which represent musical sounds.
 \textit{
	\begin{itemize}
	\item He's never been able to read music.
	\end{itemize}
}
\item  \\
 music to your ears \textit{
	\begin{itemize}
	\end{itemize}
}
\item  \\
 face the music \textit{
	\begin{itemize}
	\end{itemize}
}
\end{enumerate}

\section*{passive}
{\large \color{blue}  }
\subsection*{Explain}
\begin{enumerate}
\item adjective \\
If you describe someone as \textbf{passive} , you mean that they do not take action but instead  let things happen to them.
 \textit{
	\begin{itemize}
	\item His passive attitude made things easier for me.
	\item Even passive acceptance of the regime was a kind of collaboration.
	\end{itemize}
}
\item adjective \\
A \textbf{passive} activity involves watching , looking at, or listening to things rather than doing things.
 \textit{
	\begin{itemize}
	\item They want less passive ways of filling their time.
	\item ...the passive enjoyment one gets from looking at a painting or sculpture.
	\end{itemize}
}
\item adjective \\
\textbf{Passive}  resistance involves showing  opposition to the people in power in your country by not co-operating with them and protesting in non-violent ways.
 \textit{
	\begin{itemize}
	\item When police arrived, the protesters used passive resistance to continue their protest.

	\end{itemize}
}
\item singular noun \\
In grammar , \textbf{the passive} or \textbf{the passive voice} is formed using 'be' and the past  participle of a verb. The subject of a passive clause does not perform the action expressed by the verb but is affected by it. For example , in 'He's been murdered ', the verb is in the passive. Compare  active .
 \textit{
	\begin{itemize}
	\end{itemize}
}
\end{enumerate}

\section*{musician}
{\large \color{blue}  musicians  }
\subsection*{Explain}
\begin{enumerate}
\item countable noun \\
A \textbf{musician} is a person who plays a musical instrument as their job or hobby .
 \textit{
	\begin{itemize}
	\item He was a brilliant musician.
	\item ...one of Britain's best known rock musicians.
	\end{itemize}
}
\end{enumerate}

\section*{patriotic}
{\large \color{blue}  }
\subsection*{Explain}
\begin{enumerate}
\item adjective \\
Someone who is \textbf{patriotic} loves their country and feels very loyal towards it.
 \textit{
	\begin{itemize}
	\item Woosnam was fiercely patriotic.
	\item The crowd sang 'Land of Hope and Glory' and other patriotic songs.
	\end{itemize}
}
\end{enumerate}

\section*{obsession}
{\large \color{blue}  obsessions  }
\subsection*{Explain}
\begin{enumerate}
\item variable noun \\
If you say that someone has an \textbf{obsession} with a person or thing, you think they are spending too much time thinking about them.
 \textit{
	\begin{itemize}
	\item She would try to forget her obsession with Christopher.
	\item 95% of patients know their obsessions are irrational.
	\end{itemize}
}
\end{enumerate}

\section*{pop}
{\large \color{blue}  pops  popping  popped  }
\subsection*{Explain}
\begin{enumerate}
\item uncountable noun \\
\textbf{Pop} is modern music that usually has a strong rhythm and uses electronic equipment.
 \textit{
	\begin{itemize}
	\item ...the perfect combination of Caribbean rhythms, European pop, and American soul.
	\item We don't want to be a pop band, we want to be a serious group.
	\item ...a life-size poster of a pop star.
	\item I know nothing about pop music.
	\end{itemize}
}
\item uncountable noun \\
You can refer to fizzy drinks such as lemonade as \textbf{pop} .
 \textit{
	\begin{itemize}
	\item He still visits the village shop for buns and fizzy pop.
	\item ...glass pop bottles.
	\end{itemize}
}
\item countable noun \\
\textbf{Pop} is used to represent a short sharp sound, for example the sound made by bursting
a balloon or by pulling a cork out of a bottle .
 \textit{
	\begin{itemize}
	\item Each corn kernel will make a loud pop when cooked.
	\item His back tyre just went pop on a motorway.
	\end{itemize}
}
\item verb \\
If something \textbf{pops} , it makes a short sharp sound.
 \textit{
	\begin{itemize}
	\item He untwisted the wire off the champagne bottle, and the cork popped and shot to the
ceiling.
	\end{itemize}
}
\item verb \\
If your eyes \textbf{pop} , you look very surprised or excited when you see something.
 \textit{
	\begin{itemize}
	\item My eyes popped at the sight of the rich variety of food on show.
	\end{itemize}
}
\item verb \\
If you \textbf{pop} something somewhere , you put it there quickly.
 \textit{
	\begin{itemize}
	\item Marianne got a couple of mugs from the dresser and popped a teabag into each of them.
	\item He plucked a purple grape from the bunch and popped it in his mouth.
	\end{itemize}
}
\item verb \\
If you \textbf{pop} somewhere, you go there for a short time.
 \textit{
	\begin{itemize}
	\item He does pop down to the pub, but he seldom stays longer than an hour.
	\item Wendy popped in for a quick bite to eat on Monday night.
	\end{itemize}
}
\item countable noun \\
Some people call their father  \textbf{pop} .
 \textit{
	\begin{itemize}
	\item I looked at Pop and he had big tears in his eyes.
	\item Yes, Pop, I made a big mistake–you and Mark made me realize that.
	\end{itemize}
}
\end{enumerate}

\section*{pension}
{\large \color{blue}  pensions  pensioning  pensioned  }
\subsection*{Explain}
\begin{enumerate}
\item countable noun \\
Someone who has a \textbf{pension}  receives a regular sum of money from the state or from a former employer because they have retired or because
they are widowed or have a disability .
 \textit{
	\begin{itemize}
	\item ...struggling by on a pension.
	\item ...a company pension scheme.
	\end{itemize}
}
\end{enumerate}

\section*{popular}
{\large \color{blue}  }
\subsection*{Explain}
\begin{enumerate}
\item adjective \\
Something that is \textbf{popular} is enjoyed or liked by a lot of people.
 \textit{
	\begin{itemize}
	\item This is the most popular ball game ever devised.
	\item Chocolate sauce is always popular with youngsters.
	\end{itemize}
}
\item adjective \\
Someone who is \textbf{popular} is liked by most people, or by most people in a particular group.
 \textit{
	\begin{itemize}
	\item He remained the most popular politician in France.
	\item He was not only talented but immensely popular with his colleagues.
	\end{itemize}
}
\item adjective \\
\textbf{Popular} newspapers, television  programmes , or forms of art are aimed at ordinary people and not at experts or intellectuals .
 \textit{
	\begin{itemize}
	\item Once again the popular press in Britain has been rife with stories about their marriage.
	\item ...one of the classics of modern popular music.
	\item ...the popular culture of his native Mexico.
	\end{itemize}
}
\item adjective \\
\textbf{Popular}  ideas , feelings , or attitudes are approved of or held by most people.
 \textit{
	\begin{itemize}
	\item Contrary to popular belief, the oil companies can't control the price of crude.
	\item The military government has been unable to win popular support.
	\item Popular anger has been expressed in demonstrations.
	\end{itemize}
}
\item adjective \\
\textbf{Popular} is used to describe  political activities which involve the ordinary people of a country, and not just members of
political parties.
 \textit{
	\begin{itemize}
	\item The late President Ferdinand Marcos was overthrown by a popular uprising in 1986.
	\end{itemize}
}
\end{enumerate}

\section*{posture}
{\large \color{blue}  postures  posturing  postured  }
\subsection*{Explain}
\begin{enumerate}
\item variable noun \\
Your \textbf{posture} is the position in which you stand or sit .
 \textit{
	\begin{itemize}
	\item You can make your stomach look flatter instantly by improving your posture.
	\item Exercise, fresh air, and good posture are all helpful.
	\item Sit in a relaxed upright posture.
	\end{itemize}
}
\item countable noun \\
A \textbf{posture} is an attitude that you have towards something.
 \textit{
	\begin{itemize}
	\item The military machine is ready to change its defensive posture to one prepared for
action.
	\item None of the banks changed their posture on the deal as a result of the inquiry.
	\end{itemize}
}
\item verb \\
You can say that someone \textbf{is posturing} when you disapprove of their behaviour because you think they are trying to give a particular impression in order to deceive people.
 \textit{
	\begin{itemize}
	\item She says the President may just be posturing.
	\end{itemize}
}
\end{enumerate}

\section*{prevalent}
{\large \color{blue}  }
\subsection*{Explain}
\begin{enumerate}
\item adjective \\
A condition, practice, or belief that is \textbf{prevalent} is common.
 \textit{
	\begin{itemize}
	\item This condition is more prevalent in women than in men.
	\item The prevalent view is that interest rates will fall.
	\end{itemize}
}
\end{enumerate}

\section*{predecessor}
{\large \color{blue}  predecessors  }
\subsection*{Explain}
\begin{enumerate}
\item countable noun \\
Your \textbf{predecessor} is the person who had your job before you.
 \textit{
	\begin{itemize}
	\item He maintained that he learned everything he knew from his predecessor.
	\end{itemize}
}
\item countable noun \\
The \textbf{predecessor} of an object or machine is the object or machine that came before it in a sequence or process of development .
 \textit{
	\begin{itemize}
	\item The car is some 40mm shorter than its predecessor.
	\end{itemize}
}
\end{enumerate}

\section*{premier}
{\large \color{blue}  premiers  }
\subsection*{Explain}
\begin{enumerate}
\item countable noun \\
The leader of the government of a country is sometimes  referred to as the country's \textbf{premier} .
 \textit{
	\begin{itemize}
	\item ...Australian premier Malcolm Turnbull.
	\end{itemize}
}
\item adjective \\
\textbf{Premier} is used to describe something that is considered to be the best or most important thing of a particular type.
 \textit{
	\begin{itemize}
	\item ...the country's premier opera company.
	\end{itemize}
}
\end{enumerate}

\section*{remarkable}
{\large \color{blue}  }
\subsection*{Explain}
\begin{enumerate}
\item adjective \\
Someone or something that is \textbf{remarkable} is unusual or special in a way that makes people notice them and be surprised or impressed .
 \textit{
	\begin{itemize}
	\item He was a remarkable man.
	\item It was a remarkable achievement.
	\item It is quite remarkable that doctors have been so wrong about this.
	\end{itemize}
}
\end{enumerate}

\section*{premise}
{\large \color{blue}  premises  }
\subsection*{Explain}
\begin{enumerate}
\item plural noun \\
The \textbf{premises} of a business or an institution are all the buildings and land that it occupies in one place.
 \textit{
	\begin{itemize}
	\item There is a kitchen on the premises.
	\item The business moved to premises in Brompton Road.
	\end{itemize}
}
\item countable noun \\
A \textbf{premise} is something that you suppose is true and that you use as a basis for developing an idea .
 \textit{
	\begin{itemize}
	\item The premise is that schools will work harder to improve if they must compete.
	\item The programme started from the premise that men and women are on equal terms in this
society.
	\end{itemize}
}
\end{enumerate}

\section*{romantic}
{\large \color{blue}  romantics  }
\subsection*{Explain}
\begin{enumerate}
\item adjective \\
Someone who is \textbf{romantic} or does \textbf{romantic} things says and does things that make their wife , husband , girlfriend , or boyfriend  feel  special and loved.
 \textit{
	\begin{itemize}
	\item When we're together, all he talks about is business. I wish he were more romantic.
	\item They enjoyed a romantic dinner for two at one of their favourite restaurants.
	\end{itemize}
}
\item adjective \\
\textbf{Romantic} means connected with sexual love.
 \textit{
	\begin{itemize}
	\item ...his early romantic experiences.
	\item He was not interested in a romantic relationship with Ingrid.
	\end{itemize}
}
\item adjective \\
A \textbf{romantic} play, film, or story  describes or represents a love affair .
 \textit{
	\begin{itemize}
	\item It is a lovely romantic comedy, well worth seeing.
	\item ...romantic novels.
	\end{itemize}
}
\item adjective \\
If you say that someone has a \textbf{romantic} view or idea of something, you are critical of them because their view of it is unrealistic and they think that thing is better or more exciting than it really is.
 A \textbf{romantic} is a person who has romantic views.
 \textit{
	\begin{itemize}
	\item He has a romantic view of rural society.
	\item I don't have any romantic notions about having a baby. It's a really tough job.
	\item You're a hopeless romantic.
	\end{itemize}
}
\item adjective \\
Something that is \textbf{romantic} is beautiful in a way that strongly affects your feelings.
 \textit{
	\begin{itemize}
	\item Seacliff House is one of the most romantic ruins in Scotland.
	\item ...romantic images from travel brochures.
	\end{itemize}
}
\item adjective \\
\textbf{Romantic} means connected with the artistic movement of the eighteenth and nineteenth centuries which was concerned with the expression of the individual's feelings and
 emotions .
 \textit{
	\begin{itemize}
	\item ...the poems and prose of the English romantic poets.
	\end{itemize}
}
\end{enumerate}

\section*{president}
{\large \color{blue}  presidents  }
\subsection*{Explain}
\begin{enumerate}
\item title noun \\
The \textbf{president} of a country that has no king or queen is the person who is the head of state of that country.
 \textit{
	\begin{itemize}
	\item ...President Zuma.
	\item The White House says the president would veto the bill.
	\end{itemize}
}
\item countable noun \\
The \textbf{president} of an organization is the person who has the highest position in it.
 \textit{
	\begin{itemize}
	\item Research and marketing operations will the job of the president of the new company.
	\item ...Alexandre de Merode, the president of the medical commission.
	\end{itemize}
}
\end{enumerate}

\section*{secret}
{\large \color{blue}  secrets  }
\subsection*{Explain}
\begin{enumerate}
\item adjective \\
If something is \textbf{secret} , it is known about by only a small number of people, and is not told or shown to anyone else.
 \textit{
	\begin{itemize}
	\item Soldiers have been training at a secret location.
	\item The police have been trying to keep the documents secret.
	\end{itemize}
}
\item  \\
 See also  top secret \textit{
	\begin{itemize}
	\end{itemize}
}
\item countable noun \\
A \textbf{secret} is a fact that is known by only a small number of people, and is not told to anyone else.
 \textit{
	\begin{itemize}
	\item I think he enjoyed keeping our love a secret.
	\item I didn't want anyone to know about it, it was my secret.
	\end{itemize}
}
\item singular noun \\
If you say that a particular way of doing things is \textbf{the secret}  \textbf{of}  achieving something, you mean that it is the best or only way to achieve it.
 \textit{
	\begin{itemize}
	\item The secret of success is honesty and fair dealing.
	\item I learned something about writing. The secret is to say less than you need.
	\end{itemize}
}
\item countable noun \\
Something's \textbf{secrets} are the things about it which have never been fully  explained .
 \textit{
	\begin{itemize}
	\item We have an opportunity now to really unlock the secrets of the universe.
	\item The past is riddled with deep dark secrets.
	\end{itemize}
}
\item  \\
 in secret \textit{
	\begin{itemize}
	\end{itemize}
}
\item  \\
 to keep a secret \textit{
	\begin{itemize}
	\end{itemize}
}
\item  \\
 make no secret \textit{
	\begin{itemize}
	\end{itemize}
}
\end{enumerate}

\section*{razor}
{\large \color{blue}  razors  }
\subsection*{Explain}
\begin{enumerate}
\item countable noun \\
A \textbf{razor} is a tool that people use for shaving.
 \textit{
	\begin{itemize}
	\end{itemize}
}
\end{enumerate}

\section*{sore}
{\large \color{blue}  sorer  sorest  sores  }
\subsection*{Explain}
\begin{enumerate}
\item adjective \\
If part of your body is \textbf{sore} , it causes you pain and discomfort .
 \textit{
	\begin{itemize}
	\item It's years since I've had a sore throat like I did last night.
	\item My chest is still sore from the surgery.
	\end{itemize}
}
\item adjective \\
If you are \textbf{sore} about something, you are angry and upset about it.
 \textit{
	\begin{itemize}
	\item The result is that they are now all feeling very sore at you.
	\item They are sore about losing to England in the quarter-finals.
	\end{itemize}
}
\item countable noun \\
A \textbf{sore} is a painful place on the body where the skin is infected .
 \textit{
	\begin{itemize}
	\end{itemize}
}
\item  \\
 a sore point \textit{
	\begin{itemize}
	\end{itemize}
}
\end{enumerate}

\section*{rebellion}
{\large \color{blue}  rebellions  }
\subsection*{Explain}
\begin{enumerate}
\item variable noun \\
A \textbf{rebellion} is a violent organized action by a large group of people who are trying to change their country's political system.
 \textit{
	\begin{itemize}
	\item The British soon put down the rebellion.
	\item ...the ruthless and brutal suppression of rebellion.
	\end{itemize}
}
\item variable noun \\
A situation in which politicians  show their opposition to their own party's policies can be referred to as a \textbf{rebellion} .
 \textit{
	\begin{itemize}
	\item The Prime Minister faced his first Commons rebellion since the election.
	\end{itemize}
}
\end{enumerate}

\section*{streamline}
{\large \color{blue}  streamlines  streamlining  streamlined  }
\subsection*{Explain}
\begin{enumerate}
\item verb \\
To \textbf{streamline} an organization or process means to make it more efficient by removing unnecessary parts of it.
 \textit{
	\begin{itemize}
	\item They're making efforts to streamline their normally cumbersome bureaucracy.
	\item They say things should be better now that they have streamlined application procedures.
	\end{itemize}
}
\end{enumerate}

\section*{relish}
{\large \color{blue}  relishes  relishing  relished  }
\subsection*{Explain}
\begin{enumerate}
\item verb \\
If you \textbf{relish} something, you get a lot of enjoyment from it.
 \textbf{Relish} is also a noun .
 \textit{
	\begin{itemize}
	\item I relish the challenge of doing jobs that others turn down.
	\item He ate quietly, relishing his meal.
	\item The three men ate with relish.
	\end{itemize}
}
\item verb \\
If you \textbf{relish} the idea , thought , or prospect of something, you are looking forward to it very much.
 \textit{
	\begin{itemize}
	\item Jacqueline is not relishing the prospect of another spell in prison.
	\item He relished the idea of getting some cash.
	\end{itemize}
}
\item variable noun \\
\textbf{Relish} is a sauce or pickle that you eat with other food in order to give the other food more flavour.
 \textit{
	\begin{itemize}
	\end{itemize}
}
\end{enumerate}

\section*{tired}
{\large \color{blue}  }
\subsection*{Explain}
\begin{enumerate}
\item adjective \\
If you are \textbf{tired} , you feel that you want to rest or sleep .
 \textit{
	\begin{itemize}
	\item Michael is tired and he has to rest after his long trip.
	\end{itemize}
}
\item adjective \\
You can  describe a part of your body as \textbf{tired} if it looks or feels as if you need to rest it or to sleep.
 \textit{
	\begin{itemize}
	\item Cucumber is good for soothing tired eyes.
	\item My arms are tired, and my back is tense.
	\end{itemize}
}
\item adjective \\
If you are \textbf{tired of} something, you do not want it to continue because you are bored of it or unhappy with it.
 \textit{
	\begin{itemize}
	\item I am tired of all the speculation.
	\item I was tired of being a bookkeeper.
	\end{itemize}
}
\item adjective \\
If you describe something as \textbf{tired} , you are critical of it because you have heard it or seen it many times.
 \textit{
	\begin{itemize}
	\item I didn't want to hear another one of his tired excuses.
	\item What we see at Westminster is a tired old ritual.
	\end{itemize}
}
\end{enumerate}

\section*{sprinkle}
{\large \color{blue}  sprinkles  sprinkling  sprinkled  }
\subsection*{Explain}
\begin{enumerate}
\item verb \\
If you \textbf{sprinkle} a thing \textbf{with} something such as a liquid or powder, you scatter the liquid or powder over it.
 \textit{
	\begin{itemize}
	\item Sprinkle the meat with salt and place in the pan.
	\item At the festival, candles are blessed and sprinkled with holy water.
	\item Cheese can be sprinkled on egg or vegetable dishes.
	\end{itemize}
}
\item verb \\
If something \textbf{is sprinkled}  \textbf{with}  particular things, it has a few of them throughout it and they are far  apart from each other.
 \textit{
	\begin{itemize}
	\item Unfortunately, the text is sprinkled with errors.
	\item Men in green army uniforms are sprinkled throughout the huge auditorium.
	\end{itemize}
}
\item verb \\
If \textbf{it}  \textbf{is sprinkling} , it is raining very lightly.
 \textit{
	\begin{itemize}
	\end{itemize}
}
\end{enumerate}

\section*{tolerant}
{\large \color{blue}  }
\subsection*{Explain}
\begin{enumerate}
\item adjective \\
If you describe someone as \textbf{tolerant} , you approve of the fact that they allow other people to say and do as they like , even if they do not agree with or like it.
 \textit{
	\begin{itemize}
	\item They need to be tolerant of different points of view.
	\item Other changes include more tolerant attitudes to unmarried couples having children.
	\end{itemize}
}
\item adjective \\
If a plant, animal, or machine is \textbf{tolerant of} particular conditions or types of treatment , it is able to bear them without being damaged or hurt .
 \textit{
	\begin{itemize}
	\item ...plants which are more tolerant of dry conditions.
	\item Alpine strawberries are tolerant of most soils.
	\end{itemize}
}
\end{enumerate}

\section*{transparent}
{\large \color{blue}  }
\subsection*{Explain}
\begin{enumerate}
\item adjective \\
If an object or substance is \textbf{transparent} , you can see through it.
 \textit{
	\begin{itemize}
	\item ...a sheet of transparent coloured plastic.
	\item I looked at his thin face with its almost transparent skin.
	\end{itemize}
}
\item adjective \\
If a situation , system, or activity is \textbf{transparent} , it is easily understood or recognized.
 \textit{
	\begin{itemize}
	\item We are now striving hard to establish a transparent parliamentary democracy.
	\item The company has to make its accounts and operations as transparent as possible.
	\end{itemize}
}
\item adjective \\
You use \textbf{transparent} to describe a statement or action that is obviously  dishonest or wrong , and that you think  will not deceive people.
 \textit{
	\begin{itemize}
	\item He thought he could fool people with transparent deceptions.
	\end{itemize}
}
\end{enumerate}

\section*{superstition}
{\large \color{blue}  superstitions  }
\subsection*{Explain}
\begin{enumerate}
\item variable noun \\
\textbf{Superstition} is belief in things that are not real or possible , for example  magic .
 \textit{
	\begin{itemize}
	\item Fortune-telling is a very much debased art surrounded by superstition.
	\item The phantom of the merry-go-round is just a local superstition.
	\end{itemize}
}
\end{enumerate}

\section*{vain}
{\large \color{blue}  vainer  vainest  }
\subsection*{Explain}
\begin{enumerate}
\item adjective \\
A \textbf{vain}  attempt or action is one that fails to achieve what was intended .
 \textit{
	\begin{itemize}
	\item The drafting committee worked through the night in a vain attempt to finish on schedule.
	\item I was singing in a vain effort to cheer him up.
	\end{itemize}
}
\item adjective \\
If you describe a hope that something will happen as a \textbf{vain} hope, you mean that there is no chance of it happening .
 \textit{
	\begin{itemize}
	\item He glanced around in the vain hope that there were no witnesses.
	\end{itemize}
}
\item adjective \\
If you describe someone as \textbf{vain} , you are critical of their extreme  pride in their own beauty, intelligence , or other good qualities.
 \textit{
	\begin{itemize}
	\item I think he is shallow, vain and untrustworthy.
	\end{itemize}
}
\item  \\
 in vain \textit{
	\begin{itemize}
	\end{itemize}
}
\item  \\
 in vain \textit{
	\begin{itemize}
	\end{itemize}
}
\end{enumerate}

\section*{tone}
{\large \color{blue}  tones  toning  toned  }
\subsection*{Explain}
\begin{enumerate}
\item countable noun \\
The \textbf{tone} of a sound is its particular quality.
 \textit{
	\begin{itemize}
	\item Cross could hear him speaking in low tones to Sarah.
	\item ...the clear tone of the bell.
	\end{itemize}
}
\item countable noun \\
Someone's \textbf{tone} is a quality in their voice which shows what they are feeling or thinking .
 \textit{
	\begin{itemize}
	\item I still didn't like his tone of voice.
	\item Suddenly he laughed again, this time with a cold, sharp tone.
	\item Her tone implied that her patience was limited.
	\end{itemize}
}
\item singular noun \\
The \textbf{tone} of a speech or piece of writing is its style and the opinions or ideas expressed
in it.
 \textit{
	\begin{itemize}
	\item The spokesman said the tone of the letter was very friendly.
	\item His comments to reporters were conciliatory in tone.
	\item The whole tone of the President's speech was one of continuity and stability.
	\end{itemize}
}
\item singular noun \\
The \textbf{tone} of a place or an event is its general atmosphere .
 \textit{
	\begin{itemize}
	\item The high tone of the occasion was assured by the presence of a dozen wealthy patrons.
	\item The front desk, with its friendly, helpful staff, sets the tone for the rest of the
store.
	\end{itemize}
}
\item uncountable noun \\
The \textbf{tone} of someone's body, especially their muscles, is its degree of firmness and strength.
 \textit{
	\begin{itemize}
	\item ...stretch exercises that aim to improve muscle tone.
	\item Keeping your muscles strong and in tone helps you to avoid back problems.
	\end{itemize}
}
\item verb \\
Something that \textbf{tones} your body makes it firm and strong.
 \textbf{Tone up} means the same as tone .
 \textit{
	\begin{itemize}
	\item This movement lengthens your spine and tones the spinal nerves.
	\item Try these toning exercises before you start the day.
	\item ...finely toned muscular bodies.
	\item Exercise tones up your body.
	\item Although it's not strenuous exercise, you feel toned-up, supple and relaxed.
	\end{itemize}
}
\item variable noun \\
A \textbf{tone} is one of the lighter, darker, or brighter shades of the same colour.
 \textit{
	\begin{itemize}
	\item Each brick also varies slightly in tone, texture and size.
	\item I'm a cheery sort of person, so I like cheerful tones.
	\item ...two-tone, striped wallpaper.
	\end{itemize}
}
\item verb \\
If one thing \textbf{tones}  \textbf{with} another, the two things look nice together because their colours are similar in quality or brightness .
 \textbf{Tone in} means the same as tone .
 \textit{
	\begin{itemize}
	\item Her sister toned with her in a turquoise print dress.
	\item The bowls tone in cleverly with the mugs.
	\end{itemize}
}
\item singular noun \\
A \textbf{tone} is one of the sounds that you hear when you are using a phone , for example the sound that tells you that a number is engaged or busy , or no longer exists.
 \textit{
	\begin{itemize}
	\end{itemize}
}
\item countable noun \\
A \textbf{tone} is a difference in pitch between two musical notes equal to two semitones .
 \textit{
	\begin{itemize}
	\end{itemize}
}
\item  \\
 lower the tone of sth \textit{
	\begin{itemize}
	\end{itemize}
}
\end{enumerate}

\section*{weary}
{\large \color{blue}  wearies  wearying  wearied  wearier  weariest  }
\subsection*{Explain}
\begin{enumerate}
\item adjective \\
If you are \textbf{weary} , you are very tired.
 \textit{
	\begin{itemize}
	\item Rachel looked pale and weary.
	\item ...a weary traveller.
	\item He managed a weary smile.
	\end{itemize}
}
\item adjective \\
If you are \textbf{weary of} something, you have become tired of it and have lost your enthusiasm for it.
 \textit{
	\begin{itemize}
	\item They're getting awfully weary of this silly war.
	\item She was weary of being alone.
	\end{itemize}
}
\item verb \\
If you \textbf{weary} of something or it \textbf{wearies} you, you become tired of it and lose your enthusiasm for it.
 \textit{
	\begin{itemize}
	\item The public had wearied of his repeated warnings of a revolution that never seemed
to start.
	\item He had wearied of teaching in state universities.
	\item The political hysteria soon wearied him and he dropped the newspaper to the floor.
	\end{itemize}
}
\end{enumerate}

\section*{total}
{\large \color{blue}  totals  totalling  totalled  }
\subsection*{Explain}
\begin{enumerate}
\item countable noun \\
A \textbf{total} is the number that you get when you add several numbers  together or when you count how many things there are in a group.
 \textit{
	\begin{itemize}
	\item The companies have a total of 1,776 employees.
	\end{itemize}
}
\item adjective \\
The \textbf{total} number or cost of something is the number or cost that you get when you add together or count all
the parts in it.
 \textit{
	\begin{itemize}
	\item There could begin to be a decline in the total number of babies born each year.
	\item The total cost of the project would be more than $240 million.
	\end{itemize}
}
\item  \\
 in total \textit{
	\begin{itemize}
	\end{itemize}
}
\item verb \\
If several numbers or things \textbf{total} a certain figure , that figure is the total of all the numbers or all the things.
 \textit{
	\begin{itemize}
	\item The unit's exports will total $85 million this year.
	\item They will compete for prizes totalling nearly £3000.
	\end{itemize}
}
\item verb \\
When you \textbf{total} a set of numbers or objects , you add them all together.
 \textit{
	\begin{itemize}
	\item They haven't totalled the exact figures.
	\end{itemize}
}
\item adjective \\
You can use \textbf{total} to emphasize that something is as great in extent , degree , or amount as it possibly can be.
 \textit{
	\begin{itemize}
	\item You were a total failure if you hadn't married by the time you were about twenty-three.
	\item There was an almost total lack of management control.
	\item Why should we trust a total stranger?
	\item I have total confidence that things will change.
	\end{itemize}
}
\end{enumerate}

\section*{traitor}
{\large \color{blue}  traitors  }
\subsection*{Explain}
\begin{enumerate}
\item countable noun \\
If you call someone a \textbf{traitor} , you mean that they have betrayed beliefs that they used to hold , or that their friends hold, by their words or actions .
 \textit{
	\begin{itemize}
	\item Some say he's a traitor to the working class.
	\end{itemize}
}
\item countable noun \\
If someone is a \textbf{traitor} , they betray their country or a group of which they are a member by helping its enemies , especially during time of war .
 \textit{
	\begin{itemize}
	\item ...rumours that there were traitors among us who were sending messages to the enemy.
	\end{itemize}
}
\end{enumerate}

\section*{worthwhile}
{\large \color{blue}  }
\subsection*{Explain}
\begin{enumerate}
\item adjective \\
If something is \textbf{worthwhile} , it is enjoyable or useful , and worth the time, money, or effort that is spent on it.
 \textit{
	\begin{itemize}
	\item The President's trip to Washington this week seems to have been worthwhile.
	\item ...a worthwhile movie that was compelling enough to watch again.
	\item It might be worthwhile to consider your attitude to an insurance policy.
	\end{itemize}
}
\end{enumerate}

\section*{valve}
{\large \color{blue}  valves  }
\subsection*{Explain}
\begin{enumerate}
\item countable noun \\
A \textbf{valve} is a device attached to a pipe or a tube which controls the flow of air or liquid through the pipe or tube.
 \textit{
	\begin{itemize}
	\end{itemize}
}
\item countable noun \\
A \textbf{valve} is a small piece of tissue in your heart or in a vein which controls the flow of blood and keeps it flowing in one direction only.
 \textit{
	\begin{itemize}
	\item He also has problems with a heart valve.
	\end{itemize}
}
\end{enumerate}

\section*{auxiliary}
{\large \color{blue}  auxiliaries  }
\subsection*{Explain}
\begin{enumerate}
\item countable noun \\
An \textbf{auxiliary} is a person who is employed to assist other people in their work. Auxiliaries are often medical  workers or members of the armed  forces .
 \textit{
	\begin{itemize}
	\item Nursing auxiliaries provide basic care, but are not qualified nurses.
	\end{itemize}
}
\item adjective \\
\textbf{Auxiliary}  staff and troops assist other staff and troops.
 \textit{
	\begin{itemize}
	\item The government's first concern was to augment the army and auxiliary forces.
	\end{itemize}
}
\item adjective \\
\textbf{Auxiliary}  equipment is extra equipment that is available for use when necessary .
 \textit{
	\begin{itemize}
	\item ...an auxiliary motor.
	\item ...auxiliary fuel tanks.
	\end{itemize}
}
\item countable noun \\
An \textbf{auxiliary} is an organization that is connected with, but less important than, another organization; for example , an organization for the wives of the members of the main organization.
 \textit{
	\begin{itemize}
	\item The restaurant is operated by the Palo Alto Auxiliary for the benefit of the Lucile
Salter Packard Children's Hospital.
	\end{itemize}
}
\item countable noun \\
In grammar , an \textbf{auxiliary} or \textbf{auxiliary verb} is a verb which is used with a main verb, for example to form different  tenses or to make the verb passive . In English , the basic auxiliary verbs are 'be', 'have', and 'do'. Modal verbs such as ' can ' and ' will ' are also  sometimes  called auxiliaries.
 \textit{
	\begin{itemize}
	\end{itemize}
}
\end{enumerate}

\section*{ancestor}
{\large \color{blue}  ancestors  }
\subsection*{Explain}
\begin{enumerate}
\item countable noun \\
Your \textbf{ancestors} are the people from whom you are descended.
 \textit{
	\begin{itemize}
	\item ...our daily lives, so different from those of our ancestors.
	\item He could trace his ancestors back seven hundred years.
	\end{itemize}
}
\item countable noun \\
An \textbf{ancestor}  \textbf{of} something modern is an earlier thing from which it developed .
 \textit{
	\begin{itemize}
	\item The direct ancestor of the modern cat was the Kaffir cat of ancient Egypt.
	\item The immediate ancestor of rock 'n' roll is rhythm-and-blues.
	\end{itemize}
}
\end{enumerate}

\section*{awkward}
{\large \color{blue}  }
\subsection*{Explain}
\begin{enumerate}
\item adjective \\
An \textbf{awkward}  situation is embarrassing and difficult to deal with.
 \textit{
	\begin{itemize}
	\item I was the first to ask him awkward questions but there'll be harder ones to come.
	\item There was an awkward moment as couples decided whether to stand next to their partners.
	\end{itemize}
}
\item adjective \\
Something that is \textbf{awkward}  \textbf{to} use or carry is difficult to use or carry because of its design . A job that is \textbf{awkward} is difficult to do.
 \textit{
	\begin{itemize}
	\item It was small but heavy enough to make it awkward to carry.
	\item Full-size tripods can be awkward, especially if you're shooting a low-level subject.
	\end{itemize}
}
\item adjective \\
An \textbf{awkward} movement or position is uncomfortable or clumsy.
 \textit{
	\begin{itemize}
	\item Amy made an awkward gesture with her hands.
	\end{itemize}
}
\item adjective \\
Someone who feels  \textbf{awkward}  behaves in a shy or embarrassed way .
 \textit{
	\begin{itemize}
	\item Women frequently say that they feel awkward taking the initiative in sex.
	\item He was rather awkward with his godson.
	\end{itemize}
}
\item adjective \\
If you say that someone is \textbf{awkward} , you are critical of them because you find them unreasonable and difficult to live with or deal with.
 \textit{
	\begin{itemize}
	\item She's got to an age where she is being awkward.
	\end{itemize}
}
\end{enumerate}

\section*{aunt}
{\large \color{blue}  aunts  }
\subsection*{Explain}
\begin{enumerate}
\item countable noun \\
Someone's \textbf{aunt} is the sister of their mother or father, or the wife of their uncle.
 \textit{
	\begin{itemize}
	\item She wrote to her aunt in America.
	\item It was a present from Aunt Vera.
	\end{itemize}
}
\end{enumerate}

\section*{basic}
{\large \color{blue}  }
\subsection*{Explain}
\begin{enumerate}
\item adjective \\
You use \textbf{basic} to describe things, activities, and principles that are very important or necessary , and on which others depend .
 \textit{
	\begin{itemize}
	\item One of the most basic requirements for any form of angling is a sharp hook.
	\item ...the basic skills of reading, writing and communicating.
	\item ...the basic laws of physics.
	\item Access to justice is a basic right.
	\end{itemize}
}
\item adjective \\
\textbf{Basic} goods and services are very simple ones which every human being needs . You can also refer to people's \textbf{basic} needs for such goods and services.
 \textit{
	\begin{itemize}
	\item ...shortages of even the most basic foodstuffs.
	\item Hospitals lack even basic drugs for surgical operations.
	\item ...the basic needs of food and water.
	\end{itemize}
}
\item adjective \\
If one thing is \textbf{basic}  \textbf{to} another, it is absolutely necessary to it, and the second thing cannot exist, succeed , or be imagined without it.
 \textit{
	\begin{itemize}
	\item ...an oily liquid, basic to the manufacture of a host of other chemical substances.
	\item There are certain ethical principles that are basic to all the great religions.
	\end{itemize}
}
\item adjective \\
You can use \textbf{basic} to emphasize that you are referring to what you consider to be the most important aspect of a situation, and that you are not concerned with less important details .
 \textit{
	\begin{itemize}
	\item There are three basic types of tea.
	\item The basic design changed little from that patented by Edison more than 100 years
ago.
	\item The basic point is that sanctions cannot be counted on to produce a sure result.
	\end{itemize}
}
\item adjective \\
You can use \textbf{basic} to describe something that is very simple in style and has only the most necessary
 features , without any luxuries .
 \textit{
	\begin{itemize}
	\item We provide 2-person tents and basic cooking and camping equipment.
	\item ...the extremely basic hotel room.
	\end{itemize}
}
\item adjective \\
\textbf{Basic} is used to describe a price or someone's income when this does not include any additional amounts.
 \textit{
	\begin{itemize}
	\item ...an increase of more than twenty per cent on the basic pay of a typical worker.
	\item The basic retirement pension will go up by £1.95 a week.
	\item The basic price for a 10-minute call is only £2.49.
	\end{itemize}
}
\item adjective \\
The \textbf{basic} rate of income tax is the lowest or most common rate, which applies to people who earn  average incomes.
 \textit{
	\begin{itemize}
	\item All this is to be done without big rises in the basic level of taxation.
	\item ...a basic-rate taxpayer.
	\end{itemize}
}
\end{enumerate}

\section*{bond}
{\large \color{blue}  bonds  bonding  bonded  }
\subsection*{Explain}
\begin{enumerate}
\item countable noun \\
A \textbf{bond}  \textbf{between} people is a strong feeling of friendship, love , or shared beliefs and experiences that unites them.
 \textit{
	\begin{itemize}
	\item The experience created a very special bond between us.
	\item ...the bond that linked them.
	\end{itemize}
}
\item verb \\
When people \textbf{bond}  \textbf{with} each other, they form a relationship based on love or shared beliefs and experiences.
You can also say that people \textbf{bond} or that something \textbf{bonds} them.
 \textit{
	\begin{itemize}
	\item Belinda was having difficulty bonding with the baby.
	\item They all bonded while writing graffiti together.
	\item What had bonded them instantly and so completely was their similar background.
	\item The players are bonded by a spirit that is rarely seen in an English team.
	\end{itemize}
}
\item countable noun \\
A \textbf{bond}  \textbf{between} people or groups is a close connection that they have with each other, for example because they have a special agreement.
 \textit{
	\begin{itemize}
	\item ...the strong bond between church and nation.
	\item ...her political bond with the American president.
	\end{itemize}
}
\item plural noun \\
\textbf{Bonds} are feelings, duties, or customs that force you to behave in a particular way.
 \textit{
	\begin{itemize}
	\item Freed from the bonds of convention, the mind responds with new solutions.
	\item We must, somehow, find a way to loosen the bonds of tradition.
	\end{itemize}
}
\item countable noun \\
A \textbf{bond} between two things is the way in which they stick to one another or are joined in some way.
 \textit{
	\begin{itemize}
	\item The superglue may not create a bond with some plastics.
	\item The molecule contains four carbon atoms with a triple bond between two of them.
	\end{itemize}
}
\item verb \\
When one thing \textbf{bonds}  \textbf{with} another, it sticks to it or becomes joined to it in some way. You can also say that
two things \textbf{bond}  \textbf{together} , or that something \textbf{bonds} them \textbf{together} .
 \textit{
	\begin{itemize}
	\item Diamond may be strong in itself, but it does not bond well with other materials.
	\item In graphite sheets, carbon atoms bond together in rings.
	\item Strips of wood are bonded together and moulded by machine.
	\end{itemize}
}
\item countable noun \\
When a government or company issues a \textbf{bond} , it borrows money from investors. The certificate which is issued to investors who lend money is also called a \textbf{bond} .
 \textit{
	\begin{itemize}
	\item Most of it will be financed by government bonds.
	\item ...the recent sharp decline in bond prices.
	\end{itemize}
}
\end{enumerate}

\section*{clumsy}
{\large \color{blue}  clumsier  clumsiest  }
\subsection*{Explain}
\begin{enumerate}
\item adjective \\
A \textbf{clumsy} person moves or handles things in a careless , awkward way, often so that things are knocked over or broken.
 \textit{
	\begin{itemize}
	\item I'd never seen a clumsier, less coordinated boxer.
	\item Unfortunately, I was still very clumsy behind the wheel of the jeep.
	\end{itemize}
}
\item adjective \\
A \textbf{clumsy} action or statement is not skilful or is likely to upset people.
 \textit{
	\begin{itemize}
	\item The action seemed a clumsy attempt to topple the Janata Dal government.
	\item He denied the announcement was clumsy and insensitive.
	\end{itemize}
}
\item graded adjective \\
An object that is \textbf{clumsy} is not neat in design or appearance , and is often awkward to use.
 \textit{
	\begin{itemize}
	\item The keyboard is a large and clumsy instrument as far as portable computers are concerned.
	\item It was a clumsy looking aeroplane.
	\end{itemize}
}
\end{enumerate}

\section*{brow}
{\large \color{blue}  brows  }
\subsection*{Explain}
\begin{enumerate}
\item countable noun \\
Your \textbf{brow} is your forehead.
 \textit{
	\begin{itemize}
	\item He wiped his brow with the back of his hand.
	\item She wrinkled her brow inquisitively.
	\end{itemize}
}
\item countable noun \\
Your \textbf{brows} are your eyebrows .
 \textit{
	\begin{itemize}
	\item He had thick brown hair and shaggy brows.
	\end{itemize}
}
\item countable noun \\
The \textbf{brow}  \textbf{of} a hill is the top part of it.
 \textit{
	\begin{itemize}
	\item He was on the look-out just below the brow of the hill.
	\end{itemize}
}
\end{enumerate}

\section*{compatible}
{\large \color{blue}  }
\subsection*{Explain}
\begin{enumerate}
\item adjective \\
If things, for example systems, ideas , and beliefs , are \textbf{compatible} , they work well together or can exist together successfully.
 \textit{
	\begin{itemize}
	\item Free enterprise, he argued, was compatible with Russian values and traditions.
	\item Marriage and the life I live just don't seem compatible.
	\end{itemize}
}
\item adjective \\
If you say that you are \textbf{compatible} with someone, you mean that you have a good relationship with them because you have similar opinions and interests .
 \textit{
	\begin{itemize}
	\item Mildred and I are very compatible. She's interested in the things that interest me.
	\item You should find a doctor with whom you are compatible and feel comfortable.
	\end{itemize}
}
\item adjective \\
If one make of computer or computer equipment is \textbf{compatible}  \textbf{with} another make, especially IBM, they can be used together and can use the same software .
 \textit{
	\begin{itemize}
	\end{itemize}
}
\end{enumerate}

\section*{bully}
{\large \color{blue}  bullies  bullying  bullied  }
\subsection*{Explain}
\begin{enumerate}
\item countable noun \\
A \textbf{bully} is someone who often hurts or frightens other people.
 \textit{
	\begin{itemize}
	\item I fell victim to the office bully.
	\item He's a coward and a bully who confuses physical strength with manhood.
	\end{itemize}
}
\item verb \\
If someone \textbf{bullies} you, they often do or say things to hurt or frighten you.
 \textit{
	\begin{itemize}
	\item I wasn't going to let him bully me.
	\item I asked her if she was bullied by the other children.
	\end{itemize}
}
\item verb \\
If someone \textbf{bullies} you \textbf{into} something, they make you do it by using force or threats.
 \textit{
	\begin{itemize}
	\item We think an attempt to bully them into submission would be counterproductive.
	\item She used to bully me into doing my schoolwork.
	\item The government says it will not be bullied by the press.
	\end{itemize}
}
\end{enumerate}

\section*{cabin}
{\large \color{blue}  cabins  }
\subsection*{Explain}
\begin{enumerate}
\item countable noun \\
A \textbf{cabin} is a small room in a ship or boat.
 \textit{
	\begin{itemize}
	\item He showed her to a small cabin.
	\end{itemize}
}
\item countable noun \\
A \textbf{cabin} is one of the areas inside a plane .
 \textit{
	\begin{itemize}
	\item He sat quietly in the First Class cabin, looking tired.
	\end{itemize}
}
\item countable noun \\
A \textbf{cabin} is a small wooden house, especially one in an area of forests or mountains .
 \textit{
	\begin{itemize}
	\item ...a log cabin.
	\end{itemize}
}
\end{enumerate}

\section*{eligible}
{\large \color{blue}  }
\subsection*{Explain}
\begin{enumerate}
\item adjective \\
Someone who is \textbf{eligible}  \textbf{to} do something is qualified or able to do it, for example because they are old enough.
 \textit{
	\begin{itemize}
	\item Almost half the population are eligible to vote in today's election.
	\item You could be eligible for a university scholarship.
	\end{itemize}
}
\item adjective \\
An \textbf{eligible} man or woman is not yet married and is thought by many people to be a suitable  partner .
 \textit{
	\begin{itemize}
	\item He's the most eligible bachelor in Japan.
	\end{itemize}
}
\end{enumerate}

\section*{champion}
{\large \color{blue}  champions  championing  championed  }
\subsection*{Explain}
\begin{enumerate}
\item countable noun \\
A \textbf{champion} is someone who has won the first prize in a competition, contest , or fight .
 \textit{
	\begin{itemize}
	\item ...a former Commonwealth champion.
	\item Kasparov became world champion.
	\item ...a champion boxer and skier.
	\end{itemize}
}
\item countable noun \\
If you are a \textbf{champion}  \textbf{of} a person, a cause, or a principle , you support or defend them.
 \textit{
	\begin{itemize}
	\item He received acclaim as a champion of the oppressed.
	\item He was once known as a champion of social reform.
	\end{itemize}
}
\item verb \\
If you \textbf{champion} a person, a cause, or a principle, you support or defend them.
 \textit{
	\begin{itemize}
	\item He passionately championed the poor.
	\item The amendments had been championed by pro-democracy activists.
	\end{itemize}
}
\end{enumerate}

\section*{extravagant}
{\large \color{blue}  }
\subsection*{Explain}
\begin{enumerate}
\item adjective \\
Someone who is \textbf{extravagant} spends more money than they can afford or uses more of something than is reasonable .
 \textit{
	\begin{itemize}
	\item We are not extravagant; restaurant meals are a luxury and designer clothes are out.
	\item I hope you don't think I'm extravagant but I've had the electric fire on for most
of the day.
	\end{itemize}
}
\item adjective \\
Something that is \textbf{extravagant}  costs more money than you can afford or uses more of something than is reasonable.
 \textit{
	\begin{itemize}
	\item Her Aunt Sallie gave her an uncharacteristically extravagant gift.
	\item Baking a whole cheese in pastry may seem extravagant.
	\item ...her extravagant lifestyle.
	\end{itemize}
}
\item adjective \\
\textbf{Extravagant}  behaviour is extreme behaviour that is often done for a particular effect.
 \textit{
	\begin{itemize}
	\item He was extravagant in his admiration of Hellas.
	\item They may make extravagant shows of generosity.
	\end{itemize}
}
\item adjective \\
\textbf{Extravagant}  claims or ideas are unrealistic or impractical .
 \textit{
	\begin{itemize}
	\item They have to compete by adorning their products with ever more extravagant claims.
	\item Don't be afraid to consider apparently extravagant ideas.
	\end{itemize}
}
\item graded adjective \\
\textbf{Extravagant}  entertainments or designs are elaborate and impressive .
 \textit{
	\begin{itemize}
	\item ...the wildest and most extravagant London parties.
	\item ...painting extravagant and bold designs onto wooden frames.
	\end{itemize}
}
\end{enumerate}

\section*{combination}
{\large \color{blue}  combinations  }
\subsection*{Explain}
\begin{enumerate}
\item countable noun \\
A \textbf{combination}  \textbf{of} things is a mixture of them.
 \textit{
	\begin{itemize}
	\item ...a fantastic combination of colours.
	\item ...the combination of science and art.
	\end{itemize}
}
\end{enumerate}

\section*{famous}
{\large \color{blue}  }
\subsection*{Explain}
\begin{enumerate}
\item adjective \\
Someone or something that is \textbf{famous} is very well known.
 \textit{
	\begin{itemize}
	\item New Orleans is famous for its cuisine.
	\item ...England's most famous landscape artist, John Constable.
	\end{itemize}
}
\end{enumerate}

\section*{courage}
{\large \color{blue}  }
\subsection*{Explain}
\begin{enumerate}
\item uncountable noun \\
\textbf{Courage} is the quality shown by someone who decides to do something difficult or dangerous , even though they may be afraid .
 \textit{
	\begin{itemize}
	\item He has impressed everyone with his authority and personal courage.
	\item They do not have the courage to apologise for their actions.
	\end{itemize}
}
\item  \\
 the courage of your convictions \textit{
	\begin{itemize}
	\end{itemize}
}
\end{enumerate}

\section*{feeble}
{\large \color{blue}  feebler  feeblest  }
\subsection*{Explain}
\begin{enumerate}
\item adjective \\
If you describe someone or something as \textbf{feeble} , you mean that they are weak.
 \textit{
	\begin{itemize}
	\item He told them he was old and feeble and was not able to walk so far.
	\item The feeble light of a tin lamp.
	\end{itemize}
}
\item graded adjective \\
If you describe someone as \textbf{feeble} , you are criticizing them because they are afraid of taking  strong action or seem to make no effort .
 \textit{
	\begin{itemize}
	\item He said that the Government had been feeble.
	\item ...some rather feeble traditionalists.
	\end{itemize}
}
\item adjective \\
If you describe something that someone says as \textbf{feeble} , you mean that it is not very good or convincing .
 \textit{
	\begin{itemize}
	\item This is a particularly feeble argument.
	\end{itemize}
}
\end{enumerate}

\section*{courtyard}
{\large \color{blue}  courtyards  }
\subsection*{Explain}
\begin{enumerate}
\item countable noun \\
A \textbf{courtyard} is an open area of ground which is surrounded by buildings or walls.
 \textit{
	\begin{itemize}
	\item They walked through the arch and into the cobbled courtyard.
	\end{itemize}
}
\end{enumerate}

\section*{fundamental}
{\large \color{blue}  }
\subsection*{Explain}
\begin{enumerate}
\item adjective \\
You use \textbf{fundamental} to describe things, activities , and principles that are very important or essential . They affect the basic nature of other things or are the most important element upon which other things depend .
 \textit{
	\begin{itemize}
	\item Our constitution embodies all the fundamental principles of democracy.
	\item A fundamental human right is being withheld from these people.
	\item Technical skill is a fundamental basis for most, if not all, great art.
	\end{itemize}
}
\item adjective \\
You use \textbf{fundamental} to describe something which exists at a deep and basic level , and is therefore likely to continue .
 \textit{
	\begin{itemize}
	\item But on this question, the two leaders have very fundamental differences.
	\end{itemize}
}
\item adjective \\
If one thing \textbf{is fundamental to} another, it is absolutely  necessary to it, and the second thing cannot exist, succeed , or be imagined without it.
 \textit{
	\begin{itemize}
	\item Communication is fundamental to human society.
	\item The method they pioneered remains fundamental to research into the behaviour of nerve
cells.
	\end{itemize}
}
\item adjective \\
You can use \textbf{fundamental} to show that you are referring to what you consider to be the most important aspect of a situation , and that you are not concerned with less important details .
 \textit{
	\begin{itemize}
	\item The fundamental problem lies in their inability to distinguish between reality and
invention.
	\item It was not simply a practical matter, but a fundamental question of principle.
	\end{itemize}
}
\item adjective \\
\textbf{Fundamental}  research into a subject is concerned with gaining  knowledge about the subject itself, rather than its practical aspects.
 \textit{
	\begin{itemize}
	\item Industry leaders want scientists to engage in fundamental research, not applied research.
	\end{itemize}
}
\end{enumerate}

\section*{declaration}
{\large \color{blue}  declarations  }
\subsection*{Explain}
\begin{enumerate}
\item countable noun \\
A \textbf{declaration} is an official announcement or statement.
 \textit{
	\begin{itemize}
	\item They will sign the declaration tomorrow.
	\item The opening speeches sounded more like declarations of war than offerings of peace.
	\item ...the issues arising from their declaration of independence.
	\end{itemize}
}
\item countable noun \\
A \textbf{declaration} is a firm , emphatic statement which shows that you have no doubts about what you are saying .
 \textit{
	\begin{itemize}
	\item She needed time to adjust to Clive's declaration.
	\item ...declarations of undying love.
	\end{itemize}
}
\item countable noun \\
A \textbf{declaration} is a written statement about something which you have signed and which can be used as evidence in a court of law.
 \textit{
	\begin{itemize}
	\item On the customs declaration, the sender labeled the freight as agricultural machinery.
	\item They will ask you to sign a declaration allowing your doctor to disclose your medical
details.
	\end{itemize}
}
\end{enumerate}

\section*{global}
{\large \color{blue}  }
\subsection*{Explain}
\begin{enumerate}
\item adjective \\
You can use \textbf{global} to describe something that happens in all parts of the world or affects all parts of the world.
 \textit{
	\begin{itemize}
	\item ...a global ban on nuclear testing.
	\item ...one of the most successful organizations fighting child poverty on a global scale.
	\end{itemize}
}
\item adjective \\
A \textbf{global}  view or vision of a situation is one in which all the different aspects of it are considered.
 \textit{
	\begin{itemize}
	\item They are confident their leader is taking a global view on important issues.
	\item ...a global vision of contemporary societies.
	\end{itemize}
}
\end{enumerate}

\section*{dye}
{\large \color{blue}  dyes  dyeing  dyed  }
\subsection*{Explain}
\begin{enumerate}
\item verb \\
If you \textbf{dye} something such as hair or cloth , you change its colour by soaking it in a special liquid.
 \textit{
	\begin{itemize}
	\item The women prepared, spun and dyed the wool.
	\item She had dyed black hair.
	\end{itemize}
}
\item variable noun \\
\textbf{Dye} is a substance made from plants or chemicals which is mixed into a liquid and used to change the colour of something such as cloth or hair.
 \textit{
	\begin{itemize}
	\item ...bottles of hair dye.
	\end{itemize}
}
\end{enumerate}

\section*{energy}
{\large \color{blue}  energies  }
\subsection*{Explain}
\begin{enumerate}
\item uncountable noun \\
\textbf{Energy} is the ability and strength to do active  physical things and the feeling that you are full of physical power and life.
 \textit{
	\begin{itemize}
	\item He was saving his energy for next week's race in Belgium.
	\item We try to boost our energy by eating.
	\end{itemize}
}
\item uncountable noun \\
\textbf{Energy} is determination and enthusiasm about doing things.
 \textit{
	\begin{itemize}
	\item At 54 years old her energy and looks are magnificent.
	\item You have drive and energy for those things you are interested in.
	\end{itemize}
}
\item countable noun \\
Your \textbf{energies} are your efforts and attention , which you can direct towards a particular aim .
 \textit{
	\begin{itemize}
	\item She had started to devote her energies to teaching rather than performing.
	\item We must concentrate our energies on treating addiction first.
	\end{itemize}
}
\item uncountable noun \\
\textbf{Energy} is the power from sources such as electricity and coal that makes machines work or provides heat.
 \textit{
	\begin{itemize}
	\item ...those who favour nuclear energy.
	\item Oil shortages have brought on an energy crisis.
	\item It doesn't take much to improve the energy efficiency of your home.
	\end{itemize}
}
\end{enumerate}

\section*{ideal}
{\large \color{blue}  ideals  }
\subsection*{Explain}
\begin{enumerate}
\item countable noun \\
An \textbf{ideal} is a principle , idea, or standard that seems very good and worth  trying to achieve .
 \textit{
	\begin{itemize}
	\item The party has drifted too far from its socialist ideals.
	\item I tried to live up to my ideal of myself.
	\end{itemize}
}
\item singular noun \\
Your \textbf{ideal}  \textbf{of} something is the person or thing that seems to you to be the best possible example of it.
 \textit{
	\begin{itemize}
	\item ...the Japanese ideal of beauty.
	\item Throughout his career she remained his feminine ideal.
	\end{itemize}
}
\item adjective \\
The \textbf{ideal} person or thing for a particular task or purpose is the best possible person or thing for it.
 \textit{
	\begin{itemize}
	\item She decided that I was the ideal person to take over the job.
	\item I really love the area and see it as an ideal place to start my managerial career.
	\item The conditions were ideal for racing.
	\end{itemize}
}
\item adjective \\
An \textbf{ideal}  society or world is the best possible one that you can imagine .
 \textit{
	\begin{itemize}
	\item We do not live in an ideal world.
	\item In an ideal world, there would be no such thing as rubbish.
	\item Their ideal society collapsed around them into the Terror and then into the Counterrevolution.
	\end{itemize}
}
\end{enumerate}

\section*{eyebrow}
{\large \color{blue}  eyebrows  }
\subsection*{Explain}
\begin{enumerate}
\item countable noun \\
Your \textbf{eyebrows} are the lines of hair which grow above your eyes.
 \textit{
	\begin{itemize}
	\end{itemize}
}
\item  \\
 raise an eyebrow \textit{
	\begin{itemize}
	\end{itemize}
}
\end{enumerate}

\section*{lofty}
{\large \color{blue}  loftier  loftiest  }
\subsection*{Explain}
\begin{enumerate}
\item adjective \\
A \textbf{lofty}  ideal or ambition is noble, important , and admirable .
 \textit{
	\begin{itemize}
	\item It was a bank that started out with grand ideas and lofty ideals.
	\item Amid the chaos, he had lofty aims.
	\end{itemize}
}
\item adjective \\
A \textbf{lofty} building or room is very high.
 \textit{
	\begin{itemize}
	\item ...a light, lofty apartment in the suburbs of Salzburg.
	\item Victorian houses can seem cold with their lofty ceilings and rambling rooms.
	\end{itemize}
}
\item adjective \\
If you say that someone behaves in a \textbf{lofty} way, you are critical of them for behaving in a proud and rather unpleasant way, as if they think they are very important.
 \textit{
	\begin{itemize}
	\item ...the lofty disdain he often expresses for his profession.
	\item ...lofty contempt.
	\end{itemize}
}
\end{enumerate}

\section*{fee}
{\large \color{blue}  fees  }
\subsection*{Explain}
\begin{enumerate}
\item countable noun \\
A \textbf{fee} is a sum of money that you pay to be allowed to do something.
 \textit{
	\begin{itemize}
	\item He hadn't paid his television licence fee.
	\end{itemize}
}
\item countable noun \\
A \textbf{fee} is the amount of money that a person or organization is paid for a particular job or service that they provide.
 \textit{
	\begin{itemize}
	\item Find out how much your surveyor's and solicitor's fees will be.
	\end{itemize}
}
\end{enumerate}

\section*{grammar}
{\large \color{blue}  grammars  }
\subsection*{Explain}
\begin{enumerate}
\item uncountable noun \\
\textbf{Grammar} is the ways that words can be put  together in order to make sentences .
 \textit{
	\begin{itemize}
	\item He doesn't have mastery of the basic rules of grammar.
	\item ...the difference between Sanskrit and English grammar.
	\end{itemize}
}
\item uncountable noun \\
Someone's \textbf{grammar} is the way in which they obey or do not obey the rules of grammar when they write or speak .
 \textit{
	\begin{itemize}
	\item His vocabulary was sound and his grammar excellent.
	\item ...a deterioration in spelling and grammar among teenagers.
	\end{itemize}
}
\item countable noun \\
A \textbf{grammar} is a book that describes the rules of a language.
 \textit{
	\begin{itemize}
	\item ...an advanced English grammar.
	\end{itemize}
}
\item variable noun \\
A particular \textbf{grammar} is a particular theory that is intended to explain the rules of a language.
 \textit{
	\begin{itemize}
	\item Transformational grammars are more restrictive.
	\end{itemize}
}
\end{enumerate}

\section*{missionary}
{\large \color{blue}  missionaries  }
\subsection*{Explain}
\begin{enumerate}
\item countable noun \\
A \textbf{missionary} is a Christian who has been sent to a foreign country to teach people about Christianity .
 \textit{
	\begin{itemize}
	\end{itemize}
}
\item adjective \\
\textbf{Missionary} is used to describe the activities of missionaries.
 \textit{
	\begin{itemize}
	\item You should be in missionary work.
	\end{itemize}
}
\item adjective \\
If you refer to someone's enthusiasm for an activity or belief as \textbf{missionary}  zeal , you are emphasizing that they are very enthusiastic about it.
 \textit{
	\begin{itemize}
	\item She had a kind of missionary zeal about bringing culture to the masses.
	\end{itemize}
}
\end{enumerate}

\section*{optical}
{\large \color{blue}  }
\subsection*{Explain}
\begin{enumerate}
\item adjective \\
\textbf{Optical} devices, processes, and effects involve or relate to vision, light, or images .
 \textit{
	\begin{itemize}
	\item ...optical telescopes.
	\item ...the optical effects of volcanic dust in the stratosphere.
	\end{itemize}
}
\end{enumerate}

\section*{guest}
{\large \color{blue}  guests  guesting  guested  }
\subsection*{Explain}
\begin{enumerate}
\item countable noun \\
A \textbf{guest} is someone who is visiting you or is at an event because you have invited them.
 \textit{
	\begin{itemize}
	\item She was a guest at the wedding.
	\item Their guests sipped drinks on the veranda.
	\end{itemize}
}
\item countable noun \\
A \textbf{guest} is someone who visits a place or organization or appears on a radio or television  show because they have been invited to do so.
 \textit{
	\begin{itemize}
	\item ...a frequent chat show guest.
	\item Dr Gerald Jeffers is the guest speaker.
	\item They met when she made a guest appearance in his TV show.
	\end{itemize}
}
\item countable noun \\
A \textbf{guest} is someone who is staying in a hotel.
 \textit{
	\begin{itemize}
	\item I was the only hotel guest.
	\item Hotels operate a collection service for their guests from the airports.
	\end{itemize}
}
\item verb \\
To \textbf{guest} means to take part in a performance or a match as a guest, rather than as a regular member.
 \textit{
	\begin{itemize}
	\item He guested for one or two League sides.
	\item The band have recently guested on records by other artists.
	\end{itemize}
}
\item  \\
 be my guest \textit{
	\begin{itemize}
	\end{itemize}
}
\end{enumerate}

\section*{overnight}
{\large \color{blue}  overnights  overnighting  overnighted  }
\subsection*{Explain}
\begin{enumerate}
\item adverb \\
If something happens  \textbf{overnight} , it happens throughout the night or at some point during the night.
 \textbf{Overnight} is also an adjective .
 \textit{
	\begin{itemize}
	\item The weather remained calm overnight.
	\item The decision was reached overnight.
	\item Travel and overnight accommodation are included.
	\item Overnight buying also helped the dollar.
	\end{itemize}
}
\item adverb \\
You can say that something happens \textbf{overnight} when it happens very quickly and unexpectedly.
 \textbf{Overnight} is also an adjective.
 \textit{
	\begin{itemize}
	\item The rules are not going to change overnight.
	\item He's realistic enough to know he's not going to succeed overnight.
	\item Almost overnight, she had aged ten years and become fat.
	\item In 1970 he became an overnight success in America.
	\end{itemize}
}
\item adjective \\
\textbf{Overnight}  bags or clothes are ones that you take when you go and stay somewhere for one or two nights.
 \textit{
	\begin{itemize}
	\item He realized he'd left his overnight bag at Mary's house.
	\end{itemize}
}
\item verb \\
If you \textbf{overnight} somewhere, you spend the night there.
 \textbf{Overnight} is also a noun .
 \textit{
	\begin{itemize}
	\item They had told her she would be overnighting in Sydney.
	\item Overnights can be arranged.
	\end{itemize}
}
\end{enumerate}

\section*{howl}
{\large \color{blue}  howls  howling  howled  }
\subsection*{Explain}
\begin{enumerate}
\item verb \\
If an animal such as a wolf or a dog  \textbf{howls} , it makes a long, loud , crying sound.
 \textbf{Howl} is also a noun .
 \textit{
	\begin{itemize}
	\item Somewhere in the streets beyond a dog suddenly howled, baying at the moon.
	\item The dog let out a savage howl and, wheeling round, flew at him.
	\end{itemize}
}
\item verb \\
If a person \textbf{howls} , they make a long, loud cry expressing pain, anger , or unhappiness.
 \textbf{Howl} is also a noun.
 \textit{
	\begin{itemize}
	\item He howled like a wounded animal as blood spurted from the gash.
	\item The baby was howling for her 3am feed.
	\item With a howl of rage, he grabbed the neck of a broken bottle and advanced.
	\end{itemize}
}
\item verb \\
When the wind \textbf{howls} , it blows  hard and makes a loud noise.
 \textit{
	\begin{itemize}
	\item The wind howled all night, but I slept a little.
	\item It sank in a howling gale.
	\end{itemize}
}
\item verb \\
If you \textbf{howl} something, you say it in a very loud voice .
 \textit{
	\begin{itemize}
	\item 'Get away, get away, get away' he howled.
	\item The crowd howled its approval.
	\end{itemize}
}
\item verb \\
If you \textbf{howl}  \textbf{with} laughter, you laugh very loudly.
 \textbf{Howl} is also a noun.
 \textit{
	\begin{itemize}
	\item Joe, Pink, and Booker howled with delight.
	\item The crowd howled, delirious.
	\item His stories caused howls of laughter.
	\end{itemize}
}
\end{enumerate}

\section*{preliminary}
{\large \color{blue}  preliminaries  }
\subsection*{Explain}
\begin{enumerate}
\item adjective \\
\textbf{Preliminary} activities or discussions take place at the beginning of an event, often as a form of preparation.
 \textit{
	\begin{itemize}
	\item Preliminary results show the Republican party with 11 percent of the vote.
	\item ...preliminary talks on the future of the bases.
	\end{itemize}
}
\item countable noun \\
A \textbf{preliminary} is something that you do at the beginning of an activity, often as a form of preparation.
 \textit{
	\begin{itemize}
	\item It had taken about ten minutes to cover the preliminaries.
	\item A background check is normally a preliminary to a presidential appointment.
	\end{itemize}
}
\item countable noun \\
A \textbf{preliminary} is the first part of a competition to see who will  go on to the main competition.
 \textit{
	\begin{itemize}
	\item The winner of each preliminary goes through to the final.
	\end{itemize}
}
\end{enumerate}

\section*{illustration}
{\large \color{blue}  illustrations  }
\subsection*{Explain}
\begin{enumerate}
\item countable noun \\
An \textbf{illustration} is an example or a story which is used to make a point clear .
 \textit{
	\begin{itemize}
	\item ...a perfect illustration of the way Britain absorbs and adapts external influences.
	\end{itemize}
}
\item countable noun \\
An \textbf{illustration} in a book is a picture , design, or diagram .
 \textit{
	\begin{itemize}
	\item She looked like a princess in a nineteenth-century illustration.
	\end{itemize}
}
\end{enumerate}

\section*{professional}
{\large \color{blue}  professionals  }
\subsection*{Explain}
\begin{enumerate}
\item adjective \\
\textbf{Professional} means relating to a person's work, especially work that requires  special  training .
 \textit{
	\begin{itemize}
	\item His professional career started at Liverpool University.
	\end{itemize}
}
\item adjective \\
\textbf{Professional} people have jobs that require advanced  education or training.
 \textbf{Professional} is also a noun .
 \textit{
	\begin{itemize}
	\item ...highly qualified professional people like doctors and engineers.
	\item My father wanted me to become a professional and have more stability.
	\end{itemize}
}
\item adjective \\
You use \textbf{professional} to describe people who do a particular thing to earn  money  rather than as a hobby .
 \textbf{Professional} is also a noun.
 \textit{
	\begin{itemize}
	\item This has been my worst time for injuries since I started as a professional footballer.
	\item The veteran golfer has played in every Major Championship since he turned professional.
	\item He had been a professional since March 1985.
	\end{itemize}
}
\item adjective \\
\textbf{Professional}  sports are played for money rather than as a hobby.
 \textit{
	\begin{itemize}
	\item ...an art student who had played professional football for a short time.
	\end{itemize}
}
\item adjective \\
If you say that something that someone does or produces is \textbf{professional} , you approve of it because you think that it is of a very high  standard .
 \textbf{Professional} is also a noun.
 \textit{
	\begin{itemize}
	\item They run it with a truly professional but personal touch.
	\item ...a dedicated professional who worked harmoniously with the cast and crew.
	\end{itemize}
}
\end{enumerate}

\section*{language}
{\large \color{blue}  languages  }
\subsection*{Explain}
\begin{enumerate}
\item countable noun \\
A \textbf{language} is a system of communication which consists of a set of sounds and written symbols which are used by the people
of a particular country or region for talking or writing.
 \textit{
	\begin{itemize}
	\item ...the English language.
	\item Students are expected to master a second language.
	\item Holidays are for seeing the sights, hearing the language and savouring the smells.
	\end{itemize}
}
\item uncountable noun \\
\textbf{Language} is the use of a system of communication which consists of a set of sounds or written
symbols.
 \textit{
	\begin{itemize}
	\item Students examined how children acquire language.
	\item Language is not art but both are forms of human behavior.
	\end{itemize}
}
\item uncountable noun \\
You can refer to the words used in connection with a particular subject as \textbf{the}  \textbf{language}  \textbf{of} that subject.
 \textit{
	\begin{itemize}
	\item ...the language of business.
	\end{itemize}
}
\item uncountable noun \\
You can refer to someone's use of rude words or swearing as \textbf{bad language} when you find it offensive .
 \textit{
	\begin{itemize}
	\item Television companies tend to censor bad language in feature films.
	\item There's a girl gonna be in the club, so you guys watch your language.
	\end{itemize}
}
\item uncountable noun \\
The \textbf{language} of a piece of writing or speech is the style in which it is written or spoken.
 \textit{
	\begin{itemize}
	\item ...a booklet summarising it in plain language.
	\item The tone of his language was diplomatic and polite.
	\item Mr Harris has not been afraid to use language that many in his party despise.
	\end{itemize}
}
\item variable noun \\
You can use \textbf{language} to refer to various means of communication involving recognizable symbols, non-verbal
sounds, or actions.
 \textit{
	\begin{itemize}
	\item Some sign languages are very sophisticated means of communication.
	\item ...the digital language of computers.
	\end{itemize}
}
\end{enumerate}

\section*{prospective}
{\large \color{blue}  }
\subsection*{Explain}
\begin{enumerate}
\item adjective \\
You use \textbf{prospective} to describe someone who wants to be the thing mentioned or who is likely to be the thing mentioned.
 \textit{
	\begin{itemize}
	\item The story should act as a warning to other prospective buyers.
	\item When his prospective employers heard his history, they said they wouldn't hire him.
	\end{itemize}
}
\item adjective \\
You use \textbf{prospective} to describe something that is likely to happen  soon .
 \textit{
	\begin{itemize}
	\item ...the terms of the prospective deal.
	\end{itemize}
}
\end{enumerate}

\section*{laundry}
{\large \color{blue}  laundries  }
\subsection*{Explain}
\begin{enumerate}
\item uncountable noun \\
\textbf{Laundry} is used to refer to clothes, sheets , and towels that are about to be washed, are being washed, or have just been washed.
 \textit{
	\begin{itemize}
	\item I'll do your laundry.
	\item ...the room where I hang the laundry.
	\item He'd put his dirty laundry in the clothes basket.
	\end{itemize}
}
\item countable noun \\
A \textbf{laundry} is a firm that washes and irons clothes, sheets, and towels for people.
 \textit{
	\begin{itemize}
	\item We had to have the washing done at the laundry.
	\end{itemize}
}
\item countable noun \\
A \textbf{laundry} or a \textbf{laundry room} is a room in a house , hotel , or institution where clothes, sheets, and towels are washed.
 \textit{
	\begin{itemize}
	\item He worked in the laundry at Oxford prison.
	\end{itemize}
}
\end{enumerate}

\section*{radical}
{\large \color{blue}  radicals  }
\subsection*{Explain}
\begin{enumerate}
\item adjective \\
\textbf{Radical} changes and differences are very important and great in degree .
 \textit{
	\begin{itemize}
	\item The country needs a period of calm without more surges of radical change.
	\item The Football League has announced its proposals for a radical reform of the way football
is run in England.
	\end{itemize}
}
\item adjective \\
\textbf{Radical} people believe that there should be great changes in society and try to bring about these changes.
 A \textbf{radical} is someone who has radical views.
 \textit{
	\begin{itemize}
	\item ...threats by left-wing radical groups to disrupt the proceedings.
	\item ...political tension between radical and conservative politicians.
	\end{itemize}
}
\end{enumerate}

\section*{mystery}
{\large \color{blue}  mysteries  }
\subsection*{Explain}
\begin{enumerate}
\item countable noun \\
A \textbf{mystery} is something that is not understood or known about.
 \textit{
	\begin{itemize}
	\item The source of the gunshots still remains a mystery.
	\item ...the mysteries of mental breakdown.
	\end{itemize}
}
\item uncountable noun \\
If you talk about the \textbf{mystery} of someone or something, you are talking about how difficult they are to understand or know about, especially when this gives them a rather  strange or magical quality.
 \textit{
	\begin{itemize}
	\item She's a lady of mystery.
	\item It is an elaborate ceremony, shrouded in mystery.
	\end{itemize}
}
\item adjective \\
A \textbf{mystery} person or thing is one whose identity or nature is not known.
 \textit{
	\begin{itemize}
	\item The mystery hero immediately alerted police after spotting a bomb.
	\item ...a mystery prize of up to £1,000.
	\end{itemize}
}
\item countable noun \\
A \textbf{mystery} is a story in which strange things happen that are not explained until the end.
 \textit{
	\begin{itemize}
	\item His fourth novel is a murder mystery set in London.
	\end{itemize}
}
\end{enumerate}

\section*{rational}
{\large \color{blue}  }
\subsection*{Explain}
\begin{enumerate}
\item adjective \\
\textbf{Rational}  decisions and thoughts are based on reason rather than on emotion .
 \textit{
	\begin{itemize}
	\item He's asking you to look at both sides of the case and come to a rational decision.
	\item Mary was able to short-circuit her stress response by keeping her thoughts calm and
rational.
	\end{itemize}
}
\item adjective \\
A \textbf{rational} person is someone who is sensible and is able to make decisions based on intelligent thinking rather than on emotion.
 \textit{
	\begin{itemize}
	\item Did he come across as a sane rational person?
	\item Rachel looked calmer and more rational now.
	\end{itemize}
}
\end{enumerate}

\section*{myth}
{\large \color{blue}  myths  }
\subsection*{Explain}
\begin{enumerate}
\item variable noun \\
A \textbf{myth} is a well-known story which was made up in the past to explain natural events or to justify religious beliefs or social customs.
 \textit{
	\begin{itemize}
	\item There is a famous Greek myth in which Icarus flew too near to the Sun.
	\item ...the world of magic and of myth.
	\end{itemize}
}
\item variable noun \\
If you describe a belief or explanation as a \textbf{myth} , you mean that many people believe it but it is actually  untrue .
 \textit{
	\begin{itemize}
	\item Contrary to the popular myth, women are not reckless spendthrifts.
	\end{itemize}
}
\end{enumerate}

\section*{recipient}
{\large \color{blue}  recipients  }
\subsection*{Explain}
\begin{enumerate}
\item countable noun \\
The \textbf{recipient} of something is the person who receives it.
 \textit{
	\begin{itemize}
	\item ...the largest recipient of American foreign aid.
	\item A suppressed immune system puts a transplant recipient at risk of other infections.
	\end{itemize}
}
\end{enumerate}

\section*{nerve}
{\large \color{blue}  nerves  nerving  nerved  }
\subsection*{Explain}
\begin{enumerate}
\item countable noun \\
\textbf{Nerves} are long thin fibres that transmit  messages between your brain and other parts of your body.
 \textit{
	\begin{itemize}
	\item ...spinal nerves.
	\item ...in cases where the nerve fibres are severed.
	\end{itemize}
}
\item plural noun \\
If you refer to someone's \textbf{nerves} , you mean their ability to cope with problems such as stress , worry , and danger .
 \textit{
	\begin{itemize}
	\item Jill's nerves are stretched to breaking point.
	\item I can be very patient, and then I can burst if my nerves are worn out.
	\end{itemize}
}
\item plural noun \\
You can refer to someone's feelings of anxiety or tension as \textbf{nerves} .
 \textit{
	\begin{itemize}
	\item I just played badly. It wasn't nerves.
	\end{itemize}
}
\item uncountable noun \\
\textbf{Nerve} is the courage that you need in order to do something difficult or dangerous .
 \textit{
	\begin{itemize}
	\item The brandy made him choke, but it restored his nerve.
	\item He never got up enough nerve to meet me.
	\end{itemize}
}
\item verb \\
If you \textbf{nerve}  \textbf{yourself} to do something difficult or frightening , you prepare yourself for it by trying to be brave .
 \textit{
	\begin{itemize}
	\item I nerved myself to face the pain.
	\end{itemize}
}
\item  \\
 get on sb's nerves \textit{
	\begin{itemize}
	\end{itemize}
}
\item  \\
 have a nerve \textit{
	\begin{itemize}
	\end{itemize}
}
\item  \\
 hold one's nerve/keep ones's nerves \textit{
	\begin{itemize}
	\end{itemize}
}
\item  \\
 live on one's nerves \textit{
	\begin{itemize}
	\end{itemize}
}
\item  \\
 to lose your nerve \textit{
	\begin{itemize}
	\end{itemize}
}
\item  \\
 touch a raw nerve \textit{
	\begin{itemize}
	\end{itemize}
}
\end{enumerate}

\section*{savage}
{\large \color{blue}  savages  savaging  savaged  }
\subsection*{Explain}
\begin{enumerate}
\item adjective \\
Someone or something that is \textbf{savage} is extremely cruel , violent , and uncontrolled .
 \textit{
	\begin{itemize}
	\item This was a savage attack on a defenceless young girl.
	\item ...the savage wave of violence that swept the country in November 1987.
	\item ...a savage dog lunging at the end of a chain.
	\end{itemize}
}
\item countable noun \\
If you refer to people as \textbf{savages} , you dislike them because you think that they do not have an advanced society and are violent.
 \textit{
	\begin{itemize}
	\item ...their conviction that the area was a frozen desert peopled with uncouth savages.
	\end{itemize}
}
\item verb \\
If someone \textbf{is savaged} by a dog or other animal, the animal attacks them violently.
 \textit{
	\begin{itemize}
	\item The animal then turned on him and he was savaged to death.
	\end{itemize}
}
\item verb \\
If someone or something that they have done \textbf{is savaged} by another person, that person criticizes them severely.
 \textit{
	\begin{itemize}
	\item The show had already been savaged by critics.
	\item Speakers called for clearer direction and savaged the Chancellor.
	\end{itemize}
}
\end{enumerate}

\section*{onion}
{\large \color{blue}  onions  }
\subsection*{Explain}
\begin{enumerate}
\item variable noun \\
An \textbf{onion} is a round vegetable with a brown  skin that grows  underground . It has many white layers on its inside which have a strong , sharp  smell and taste.
 \textit{
	\begin{itemize}
	\item Will you chop an onion up for me?
	\item It is made with fresh minced meat, cooked with onion and a rich tomato sauce.
	\item ...onion soup.
	\end{itemize}
}
\end{enumerate}

\section*{successive}
{\large \color{blue}  }
\subsection*{Explain}
\begin{enumerate}
\item adjective \\
\textbf{Successive} means happening or existing one after another without a break .
 \textit{
	\begin{itemize}
	\item Jackson was the winner for a second successive year.
	\item ...the failure of successive governments to co-ordinate transport policy.
	\end{itemize}
}
\end{enumerate}

\section*{palace}
{\large \color{blue}  palaces  }
\subsection*{Explain}
\begin{enumerate}
\item countable noun \\
A \textbf{palace} is a very large impressive house, especially one which is the official home of a king , queen , or president .
 \textit{
	\begin{itemize}
	\item ...Buckingham Palace.
	\item They entered the palace courtyard.
	\end{itemize}
}
\item singular noun \\
When the members of a royal palace make an announcement through an official spokesperson, they can be referred to as \textbf{the Palace} .
 \textit{
	\begin{itemize}
	\item The Palace will not comment on questions about the family's private life.
	\end{itemize}
}
\item countable noun \\
You can refer to any large splendid house or other building as a \textbf{palace} .
 \textit{
	\begin{itemize}
	\item They'd bought a huge barn with some land and planned to turn it into a palace.
	\end{itemize}
}
\end{enumerate}

\section*{sweet}
{\large \color{blue}  sweeter  sweetest  sweets  }
\subsection*{Explain}
\begin{enumerate}
\item adjective \\
\textbf{Sweet} food and drink contains a lot of sugar.
 \textit{
	\begin{itemize}
	\item ...a mug of sweet tea.
	\item If the sauce seems too sweet, add a dash of vinegar.
	\item ...the sweet taste of wild strawberries.
	\end{itemize}
}
\item countable noun \\
\textbf{Sweets} are small sweet things such as toffees, chocolates , and mints .
 \textit{
	\begin{itemize}
	\end{itemize}
}
\item variable noun \\
A \textbf{sweet} is something sweet, such as fruit or a pudding, that you eat at the end of a meal , especially in a restaurant .
 \textit{
	\begin{itemize}
	\item The sweet was a mousse flavoured with coffee.
	\end{itemize}
}
\item adjective \\
A \textbf{sweet} smell is a pleasant one, for example the smell of a flower.
 \textit{
	\begin{itemize}
	\item ...the sweet smell of her shampoo.
	\item She'd baked some bread which made the air smell sweet.
	\end{itemize}
}
\item adjective \\
If you describe something such as air or water as \textbf{sweet} , you mean that it smells or tastes pleasantly fresh and clean .
 \textit{
	\begin{itemize}
	\item I gulped a breath of sweet air.
	\item ...a stream of sweet water.
	\end{itemize}
}
\item adjective \\
A \textbf{sweet} sound is pleasant, smooth, and gentle.
 \textit{
	\begin{itemize}
	\item Her voice was as soft and sweet as a young girl's.
	\item ...the sweet sounds of Mozart.
	\end{itemize}
}
\item adjective \\
If you describe something as \textbf{sweet} , you mean that it gives you great pleasure and satisfaction .
 \textit{
	\begin{itemize}
	\item There are few things quite as sweet as revenge.
	\item ...the sweet taste of illicit love.
	\item His success was all the sweeter for being at the expense of Europe's most admired
team.
	\end{itemize}
}
\item adjective \\
If you describe someone as \textbf{sweet} , you mean that they are pleasant, kind, and gentle towards other people.
 \textit{
	\begin{itemize}
	\item How sweet of you to think of me!
	\end{itemize}
}
\item adjective \\
If you describe a small person or thing as \textbf{sweet} , you mean that they are attractive in a simple or unsophisticated way.
 \textit{
	\begin{itemize}
	\item ...a sweet little baby girl.
	\item The house was really sweet.
	\end{itemize}
}
\item vocative noun \\
You can address someone as \textbf{sweet} or \textbf{my sweet} if you are very fond of them.
 \textit{
	\begin{itemize}
	\item I am so proud of you, my sweet!
	\end{itemize}
}
\item  \\
 keep sb sweet \textit{
	\begin{itemize}
	\end{itemize}
}
\end{enumerate}

\section*{persuasion}
{\large \color{blue}  persuasions  }
\subsection*{Explain}
\begin{enumerate}
\item uncountable noun \\
\textbf{Persuasion} is the act of persuading someone to do something or to believe that something is true .
 \textit{
	\begin{itemize}
	\item Only after much persuasion from Ellis had she agreed to hold a show at all.
	\item She was using all her powers of persuasion to induce the Griffins to remain in Rollway.
	\end{itemize}
}
\item countable noun \\
If you are \textbf{of} a particular  \textbf{persuasion} , you have a particular belief or set of beliefs.
 \textit{
	\begin{itemize}
	\item It is a national movement and has within it people of all political persuasions.
	\end{itemize}
}
\end{enumerate}

\section*{terminal}
{\large \color{blue}  terminals  }
\subsection*{Explain}
\begin{enumerate}
\item adjective \\
A \textbf{terminal}  illness or disease causes death, often slowly, and cannot be cured .
 \textit{
	\begin{itemize}
	\item ...terminal cancer.
	\item His illness was terminal.
	\end{itemize}
}
\item adjective \\
A \textbf{terminal}  patient is dying of a terminal illness or disease.
 \textit{
	\begin{itemize}
	\item They have started a hospice for terminal patients.
	\end{itemize}
}
\item countable noun \\
A \textbf{terminal} is a place where vehicles, passengers, or goods begin or end a journey .
 \textit{
	\begin{itemize}
	\item Plans are underway for a third terminal at the airport.
	\end{itemize}
}
\item countable noun \\
A computer \textbf{terminal} is a piece of equipment consisting of a keyboard and a screen that is used for putting information into a computer or getting information from it.
 \textit{
	\begin{itemize}
	\item Carl sits at a computer terminal 40 hours a week.
	\end{itemize}
}
\item countable noun \\
On a piece of electrical equipment, a \textbf{terminal} is one of the points where electricity enters or leaves it.
 \textit{
	\begin{itemize}
	\item ...the positive terminal of the battery.
	\end{itemize}
}
\end{enumerate}

\section*{piston}
{\large \color{blue}  pistons  }
\subsection*{Explain}
\begin{enumerate}
\item countable noun \\
A \textbf{piston} is a cylinder or metal disc that is part of an engine. Pistons slide up and down inside  tubes and cause various parts of the engine to move.
 \textit{
	\begin{itemize}
	\end{itemize}
}
\end{enumerate}

\section*{theoretical}
{\large \color{blue}  }
\subsection*{Explain}
\begin{enumerate}
\item adjective \\
A \textbf{theoretical} study or explanation is based on or uses the ideas and abstract  principles that relate to a particular subject, rather than the practical aspects or uses of it.
 \textit{
	\begin{itemize}
	\item ...theoretical physics.
	\end{itemize}
}
\item adjective \\
If you describe a situation as a \textbf{theoretical} one, you mean that although it is supposed to be true or to exist in the way stated, it may not in fact be true or exist in that way.
 \textit{
	\begin{itemize}
	\item This is certainly a theoretical risk but in practice there is seldom a problem.
	\item These fears are purely theoretical.
	\end{itemize}
}
\end{enumerate}

\section*{propaganda}
{\large \color{blue}  }
\subsection*{Explain}
\begin{enumerate}
\item uncountable noun \\
\textbf{Propaganda} is information, often inaccurate information, which a political organization publishes or broadcasts in order to influence people.
 \textit{
	\begin{itemize}
	\item The party adopted an aggressive propaganda campaign against its rivals.
	\item They asked me to help destroy the system and spread propaganda against the government.
	\end{itemize}
}
\end{enumerate}

\section*{trivial}
{\large \color{blue}  }
\subsection*{Explain}
\begin{enumerate}
\item adjective \\
If you describe something as \textbf{trivial} , you think that it is unimportant and not serious .
 \textit{
	\begin{itemize}
	\item The director tried to wave aside these issues as trivial details that could be settled
later.
	\item I don't like to visit the doctor just for something trivial.
	\end{itemize}
}
\end{enumerate}

\section*{roof}
{\large \color{blue}  roofs  }
\subsection*{Explain}
\begin{enumerate}
\item countable noun \\
The \textbf{roof} of a building is the covering on top of it that protects the people and things inside from the weather .
 \textit{
	\begin{itemize}
	\item ...a small stone cottage with a red slate roof.
	\item A pail stood in one corner of the room to catch the drips where the roof leaked.
	\end{itemize}
}
\item countable noun \\
The \textbf{roof} of a car or other vehicle is the top part of it, which protects passengers or goods from the weather.
 \textit{
	\begin{itemize}
	\item The car rolled onto its roof, trapping him.
	\end{itemize}
}
\item countable noun \\
\textbf{The roof of} your mouth is the highest part of the inside of your mouth.
 \textit{
	\begin{itemize}
	\item She clicked her tongue against the roof of her mouth.
	\end{itemize}
}
\item countable noun \\
The \textbf{roof} of an underground  space such as a cave or mine is the highest part of it.
 \textit{
	\begin{itemize}
	\item The cave roof collapsed.
	\end{itemize}
}
\item  \\
 go through the roof \textit{
	\begin{itemize}
	\end{itemize}
}
\item  \\
 to hit the roof \textit{
	\begin{itemize}
	\end{itemize}
}
\item  \\
 roof over one's head \textit{
	\begin{itemize}
	\end{itemize}
}
\item  \\
 to raise the roof \textit{
	\begin{itemize}
	\end{itemize}
}
\item  \\
 under one roof/under the same roof \textit{
	\begin{itemize}
	\end{itemize}
}
\item  \\
 under sb's roof \textit{
	\begin{itemize}
	\end{itemize}
}
\end{enumerate}

\section*{unique}
{\large \color{blue}  }
\subsection*{Explain}
\begin{enumerate}
\item adjective \\
Something that is \textbf{unique} is the only one of its kind .
 \textit{
	\begin{itemize}
	\item Each person's signature is unique.
	\item The area has its own unique language, Catalan.
	\end{itemize}
}
\item adjective \\
You can use \textbf{unique} to describe things that you admire because they are very unusual and special .
 \textit{
	\begin{itemize}
	\item Brett's vocals are just unique.
	\item Kauffman was a woman of unique talent and determination.
	\end{itemize}
}
\item adjective \\
If something is \textbf{unique to} one thing, person, group, or place, it concerns or belongs only to that thing, person, group, or place.
 \textit{
	\begin{itemize}
	\item No one knows for sure why adolescence is unique to humans.
	\item This interesting and charming creature is unique to Borneo.
	\end{itemize}
}
\end{enumerate}

\section*{screen}
{\large \color{blue}  screens  screening  screened  }
\subsection*{Explain}
\begin{enumerate}
\item countable noun \\
A \textbf{screen} is a flat vertical surface on which pictures or words are shown. Television sets and computers have
screens, and films are shown on a screen in cinemas.
 \textit{
	\begin{itemize}
	\end{itemize}
}
\item singular noun \\
You can refer to film or television as \textbf{the screen} .
 \textit{
	\begin{itemize}
	\item Many viewers have strong opinions about violence on the screen.
	\item She was the ideal American teenager, both on and off screen.
	\end{itemize}
}
\item verb \\
When a film or a television programme  \textbf{is screened} , it is shown in the cinema or broadcast on television.
 \textit{
	\begin{itemize}
	\item The series is likely to be screened in January.
	\item TV firms were later banned from screening any pictures of the demo.
	\end{itemize}
}
\item countable noun \\
A \textbf{screen} is a vertical panel which can be moved around. It is used to keep cold air away from part of a room, or to create a smaller area within a room.
 \textit{
	\begin{itemize}
	\item They put a screen in front of me so I couldn't see what was going on.
	\end{itemize}
}
\item verb \\
If something \textbf{is screened}  \textbf{by} another thing, it is behind it and hidden by it.
 \textit{
	\begin{itemize}
	\item Most of the road behind the hotel was screened by a block of flats.
	\end{itemize}
}
\item verb \\
To \textbf{screen}  \textbf{for} a disease means to examine people to make sure that they do not have it.
 \textit{
	\begin{itemize}
	\item ...a quick saliva test that would screen for people at risk of tooth decay.
	\end{itemize}
}
\item verb \\
When an organization \textbf{screens} people who apply to join it, it investigates them to make sure that they are not likely to cause problems .
 \textit{
	\begin{itemize}
	\item They will screen all their candidates.
	\item ...screening procedures for the regiment.
	\end{itemize}
}
\item verb \\
To \textbf{screen} people or luggage means to check them using special  equipment to make sure they are not carrying a weapon or a bomb .
 \textit{
	\begin{itemize}
	\item The airline had been screening baggage on X-ray machines.
	\end{itemize}
}
\item verb \\
If you \textbf{screen} your phone calls, calls made to you are connected to an answering machine or are answered by someone else, so that you can choose whether or not to speak to the people phoning you.
 \textit{
	\begin{itemize}
	\item I employ a secretary to screen my calls.
	\end{itemize}
}
\end{enumerate}

\section*{usual}
{\large \color{blue}  }
\subsection*{Explain}
\begin{enumerate}
\item adjective \\
\textbf{Usual} is used to describe what happens or what is done most often in a particular situation .
 \textbf{Usual} is also a noun .
 \textit{
	\begin{itemize}
	\item It is a neighborhood beset by all the usual inner-city problems.
	\item She's smiling her usual friendly smile.
	\item After lunch there was a little more clearing up to do than usual.
	\item We've had more press coverage in the last three weeks than in the usual three years.
	\item It is usual to tip waiters, porters, guides and drivers.
	\item The barman appeared to take their order. 'Good morning, sir. The usual?'
	\end{itemize}
}
\item  \\
 as usual \textit{
	\begin{itemize}
	\end{itemize}
}
\item  \\
 as usual \textit{
	\begin{itemize}
	\end{itemize}
}
\end{enumerate}

\section*{shampoo}
{\large \color{blue}  shampoos  shampooing  shampooed  }
\subsection*{Explain}
\begin{enumerate}
\item variable noun \\
\textbf{Shampoo} is a soapy liquid that you use for washing your hair.
 \textit{
	\begin{itemize}
	\item ...a bottle of shampoo.
	\item ...bubble baths, soaps and shampoos.
	\end{itemize}
}
\item verb \\
When you \textbf{shampoo} your hair, you wash it using shampoo.
 \textit{
	\begin{itemize}
	\item Shampoo your hair and dry it.
	\end{itemize}
}
\end{enumerate}

\section*{versatile}
{\large \color{blue}  }
\subsection*{Explain}
\begin{enumerate}
\item adjective \\
If you say that a person is \textbf{versatile} , you approve of them because they have many different skills.
 \textit{
	\begin{itemize}
	\item He had been one of the game's most versatile athletes.
	\end{itemize}
}
\item adjective \\
A tool , machine , or material that is \textbf{versatile} can be used for many different purposes .
 \textit{
	\begin{itemize}
	\item Never before has computing been so versatile.
	\item ...a versatile blue chambray skirt.
	\end{itemize}
}
\end{enumerate}

\section*{shelf}
{\large \color{blue}  shelves  }
\subsection*{Explain}
\begin{enumerate}
\item countable noun \\
A \textbf{shelf} is a flat piece of wood, metal, or glass which is attached to a wall or to the sides of a cupboard . Shelves are used for keeping things on.
 \textit{
	\begin{itemize}
	\item He took a book from the shelf.
	\item ...the middle shelf of the oven.
	\end{itemize}
}
\item countable noun \\
A \textbf{shelf} is a section of rock on a cliff or mountain or underwater that sticks out like a shelf.
 \textit{
	\begin{itemize}
	\item The house stands on a shelf of rock among pines.
	\end{itemize}
}
\item  \\
 off the shelf \textit{
	\begin{itemize}
	\end{itemize}
}
\item  \\
 on the shelf \textit{
	\begin{itemize}
	\end{itemize}
}
\end{enumerate}

\section*{weak}
{\large \color{blue}  weaker  weakest  }
\subsection*{Explain}
\begin{enumerate}
\item adjective \\
If someone is \textbf{weak} , they are not healthy or do not have good muscles, so that they cannot move quickly or carry heavy things.
 \textit{
	\begin{itemize}
	\item I was too weak to move or think or speak.
	\item His arms and legs were weak.
	\end{itemize}
}
\item adjective \\
If someone has an organ or sense that is \textbf{weak} , it is not very effective or powerful , or is likely to fail .
 \textit{
	\begin{itemize}
	\item Until the beating, Cantanco's eyesight had been weak, but adequate.
	\item She tired easily and had a weak heart.
	\end{itemize}
}
\item adjective \\
If you describe someone as \textbf{weak} , you mean that they are not very confident or determined, so that they are often frightened or worried , or easily influenced by other people.
 \textit{
	\begin{itemize}
	\item He was a nice doctor, but a weak man who wasn't going to stick his neck out.
	\item You have been conditioned to believe that it is weak to be scared.
	\end{itemize}
}
\item adjective \\
If you describe someone's voice or smile as \textbf{weak} , you mean that it not very loud or big , suggesting that the person lacks confidence , enthusiasm , or physical strength.
 \textit{
	\begin{itemize}
	\item His weak voice was almost inaudible.
	\item He managed a weak smile.
	\end{itemize}
}
\item adjective \\
If an object or surface is \textbf{weak} , it breaks easily and cannot support a lot of weight or resist a lot of strain .
 \textit{
	\begin{itemize}
	\item The owner said the bird may have escaped through a weak spot in the aviary.
	\item Swimming is helpful for bones that are porous and weak.
	\end{itemize}
}
\item adverb \\
A \textbf{weak} physical force does not have much power or intensity.
 \textit{
	\begin{itemize}
	\item The molecules in regular liquids are held together by relatively weak bonds.
	\item Strong winds can turn boats when the tide is weak.
	\item ...the weak winter sun.
	\end{itemize}
}
\item adjective \\
If individuals or groups are \textbf{weak} , they do not have any power or influence.
 \textbf{The weak} are people who are weak.
 \textit{
	\begin{itemize}
	\item The council was too weak to do anything about it.
	\item He voiced his solidarity with the weak and defenceless.
	\end{itemize}
}
\item adjective \\
A \textbf{weak} government or leader does not have much control, and is not prepared or able to act
firmly or severely.
 \textit{
	\begin{itemize}
	\item The changes come after mounting criticism that the government is weak and indecisive.
	\item The chief editorial writer also blames weak leadership for the current crisis.
	\end{itemize}
}
\item adjective \\
If you describe something such a country's currency, economy , industry, or government as \textbf{weak} , you mean that it is not successful , and may be likely to fail or collapse .
 \textit{
	\begin{itemize}
	\item The weak dollar made American goods relative bargains for foreigners.
	\item When the economy is weak, it's very hard for suppliers to raise their prices.
	\end{itemize}
}
\item adjective \\
If something such as an argument or case is \textbf{weak} , it is not convincing or there is little evidence to support it.
 \textit{
	\begin{itemize}
	\item Do you think the prosecution made any particular errors, or did they just have a
weak case?
	\item The claim exposed a weak point in the structure of facts upon which his argument
rested.
	\item The evidence against him was weak and insufficient.
	\end{itemize}
}
\item adjective \\
A \textbf{weak} drink, chemical, or drug contains very little of a particular substance, for example
because a lot of water has been added to it.
 \textit{
	\begin{itemize}
	\item ...a cup of weak tea.
	\item ...a very weak bleach solution.
	\end{itemize}
}
\item adjective \\
Your \textbf{weak} points are the qualities or talents you do not possess, or the things you are not very good at.
 \textit{
	\begin{itemize}
	\item You may very well be asked what your weak points are. Don't try to claim you don't
have any.
	\item Geography was my weak subject.
	\item His short stories tend to be weak on plot.
	\end{itemize}
}
\item graded adjective \\
You can say that someone has a \textbf{weak}  chin to indicate that their chin is not large, especially when you want to suggest that they do not have a strong character.
 \textit{
	\begin{itemize}
	\item She was a plain-looking woman with a weak chin.
	\end{itemize}
}
\end{enumerate}

\section*{speaker}
{\large \color{blue}  speakers  }
\subsection*{Explain}
\begin{enumerate}
\item countable noun \\
A \textbf{speaker} at a meeting , conference , or other gathering is a person who is making a speech or giving a talk .
 \textit{
	\begin{itemize}
	\item Among the speakers at the gathering was the Treasury Secretary.
	\item Bruce Wyatt will be the guest speaker at next month's meeting.
	\item He was not a good speaker.
	\end{itemize}
}
\item countable noun \\
A \textbf{speaker}  \textbf{of} a particular language is a person who speaks it, especially one who speaks it as their first language.
 \textit{
	\begin{itemize}
	\item Most viewers are not native English speakers.
	\item She teaches English to speakers of other languages.
	\end{itemize}
}
\item proper noun \\
In the parliament or legislature of many countries, the \textbf{Speaker} is the person who is in charge of meetings.
 \textit{
	\begin{itemize}
	\item For twenty minutes, the Speaker tried to keep order.
	\item ...the Speaker of the Polish Parliament.
	\item Mr. Speaker, our message to the president is simple.
	\end{itemize}
}
\item countable noun \\
A \textbf{speaker} is a person who is speaking.
 \textit{
	\begin{itemize}
	\item From a simple gesture or the speaker's tone of voice, the Japanese listener gleans
the whole meaning.
	\end{itemize}
}
\item countable noun \\
A \textbf{speaker} is a piece of electrical  equipment , for example part of a radio or set of equipment for playing CDs or tapes , through which sound comes out.
 \textit{
	\begin{itemize}
	\item For a good stereo effect, the speakers should not be too wide apart.
	\end{itemize}
}
\end{enumerate}

\section*{wild}
{\large \color{blue}  wilds  wilder  wildest  }
\subsection*{Explain}
\begin{enumerate}
\item adjective \\
\textbf{Wild} animals or plants live or grow in natural surroundings and are not looked after by
people.
 \textit{
	\begin{itemize}
	\item We saw two more wild cats creeping towards us in the darkness.
	\item The lane was lined with wild flowers.
	\end{itemize}
}
\item adjective \\
\textbf{Wild} land is natural and is not used by people.
 \textit{
	\begin{itemize}
	\item Elmley is one of the few wild areas remaining in the South East.
	\end{itemize}
}
\item plural noun \\
\textbf{The wilds} of a place are the natural areas that are far away from towns.
 \textit{
	\begin{itemize}
	\item They went canoeing in the wilds of Canada.
	\end{itemize}
}
\item adjective \\
\textbf{Wild} is used to describe the weather or the sea when it is stormy .
 \textit{
	\begin{itemize}
	\item The wild weather did not deter some people from swimming in the sea.
	\end{itemize}
}
\item adjective \\
\textbf{Wild} behaviour is uncontrolled , excited, or energetic .
 \textit{
	\begin{itemize}
	\item The children are wild with joy.
	\item As George himself came on stage they went wild.
	\item They marched into town to the wild cheers of the inhabitants.
	\end{itemize}
}
\item adjective \\
If you describe someone or their behaviour as \textbf{wild} , you mean that they behave in a very uncontrolled way.
 \textit{
	\begin{itemize}
	\item When angry or excited, however, he could be wild, profane, and terrifying.
	\item She lived a wild and incredible life.
	\item The house is in a mess after a wild party.
	\end{itemize}
}
\item adjective \\
If someone is \textbf{wild} , they are very angry .
 \textit{
	\begin{itemize}
	\item For a long time I daren't tell him I knew, and when I did he went wild.
	\end{itemize}
}
\item graded adjective \\
If you say that someone has \textbf{wild} eyes or a \textbf{wild} look, you mean that their eyes are wide open and staring because they are frightened , angry, or insane .
 \textit{
	\begin{itemize}
	\item She could see his face now, his eyes wild and his skin glistening with perspiration.
	\item I could not forget the wild look in his eyes.
	\end{itemize}
}
\item adjective \\
A \textbf{wild} idea is unusual or extreme. A \textbf{wild}  guess is one that you make without much thought.
 \textit{
	\begin{itemize}
	\item I was just a kid and full of all sorts of wild ideas.
	\item Browning's prediction is no better than a wild guess.
	\end{itemize}
}
\item  \\
 be wild about \textit{
	\begin{itemize}
	\end{itemize}
}
\item  \\
 in the wild \textit{
	\begin{itemize}
	\end{itemize}
}
\item  \\
 to run wild \textit{
	\begin{itemize}
	\end{itemize}
}
\end{enumerate}

\section*{wash}
{\large \color{blue}  washes  washing  washed  }
\subsection*{Explain}
\begin{enumerate}
\item verb \\
If you \textbf{wash} something, you clean it using water and usually a substance such as soap or detergent .
 \textbf{Wash} is also a noun.
 \textit{
	\begin{itemize}
	\item He got a job washing dishes in a pizza parlour.
	\item The colours gently fade each time you wash the shirt.
	\item It took a long time to wash the mud out of his hair.
	\item Rub down the door and wash off the dust before applying the varnish.
	\item That coat could do with a wash.
	\item The treatment leaves hair glossy and lasts 10 to 16 washes.
	\end{itemize}
}
\item verb \\
If you \textbf{wash} or if you \textbf{wash} part of your body, especially your hands and face, you clean part of your body using soap and water.
 \textbf{Wash} is also a noun.
 \textit{
	\begin{itemize}
	\item They looked as if they hadn't washed in days.
	\item She washed her face with cold water.
	\item You are going to have your dinner, get washed, and go to bed.
	\item She had a wash and changed her clothes.
	\end{itemize}
}
\item verb \\
If a sea or river \textbf{washes}  somewhere , it flows there gently. You can also say that something carried by a sea or river \textbf{washes} or \textbf{is washed} somewhere.
 \textit{
	\begin{itemize}
	\item The sea washed against the shore.
	\item The oil washed ashore on roughly 1000 miles of coastline.
	\item The force of the water washed him back into the cave.
	\end{itemize}
}
\item singular noun \\
\textbf{The wash} of a boat is the wave that it causes on either side as it moves through the water.
 \textit{
	\begin{itemize}
	\item ...the wash from large ships.
	\end{itemize}
}
\item verb \\
If a feeling \textbf{washes}  \textbf{over} you, you suddenly feel it very strongly and cannot control it.
 \textit{
	\begin{itemize}
	\item A wave of self-consciousness can wash over her when someone new enters the room.
	\item The overpowering despair that he'd fought so hard to keep at bay washed through the
boy.
	\end{itemize}
}
\item countable noun \\
A \textbf{wash}  \textbf{of} something such as light or colour is a thin layer of it.
 \textit{
	\begin{itemize}
	\item The lights from the truck sent a wash of pale light over the snow.
	\end{itemize}
}
\item verb \\
If you say that an excuse or idea will not \textbf{wash} , you mean that people will not accept or believe it.
 \textit{
	\begin{itemize}
	\item He said her policies didn't work and the excuses didn't wash.
	\item If they believe that solution would wash with the Haitian people, they are making
a dramatic error.
	\end{itemize}
}
\item  \\
 come out in the wash \textit{
	\begin{itemize}
	\end{itemize}
}
\item  \\
 be in the wash \textit{
	\begin{itemize}
	\end{itemize}
}
\end{enumerate}

\section*{yellow}
{\large \color{blue}  yellows  yellowing  yellowed  }
\subsection*{Explain}
\begin{enumerate}
\item colour \\
Something that is \textbf{yellow} is the colour of lemons , butter , or the middle part of an egg.
 \textit{
	\begin{itemize}
	\item The walls have been painted bright yellow.
	\item Kim opted for cooler blues and yellows in the master bedroom.
	\end{itemize}
}
\item verb \\
If something \textbf{yellows} , it becomes yellow in colour, often because it is old .
 \textit{
	\begin{itemize}
	\item The flesh of his cheeks seemed to have yellowed.
	\item She sat scanning the yellowing pages.
	\end{itemize}
}
\end{enumerate}

\section*{yard}
{\large \color{blue}  yards  }
\subsection*{Explain}
\begin{enumerate}
\item countable noun \\
A \textbf{yard} is a unit of length equal to thirty-six inches or approximately 91.4 centimetres .
 \textit{
	\begin{itemize}
	\item The incident took place about 500 yards from where he was standing.
	\item A few yards away, José Vargas stands beside his small home.
	\item ...a long narrow strip of linen two or three yards long.
	\item ...a yard of silk.
	\end{itemize}
}
\item countable noun \\
A \textbf{yard} is a flat area of concrete or stone that is next to a building and often has a wall around it.
 \textit{
	\begin{itemize}
	\item I saw him standing in the yard.
	\end{itemize}
}
\item countable noun \\
You can refer to a large open area where a particular type of work is done as a \textbf{yard} .
 \textit{
	\begin{itemize}
	\item ...a railway yard.
	\item ...a ship repair yard.
	\end{itemize}
}
\item countable noun \\
A \textbf{yard} is a piece of land next to someone's house , with grass and plants growing in it.
 \textit{
	\begin{itemize}
	\item He dug a hole in our yard on Edgerton Avenue to plant a maple tree when I was born.
	\end{itemize}
}
\end{enumerate}

\section*{accidental}
{\large \color{blue}  }
\subsection*{Explain}
\begin{enumerate}
\item adjective \\
An \textbf{accidental}  event  happens by chance or as the result of an accident , and is not deliberately intended .
 \textit{
	\begin{itemize}
	\item The jury returned a verdict of accidental death.
	\item His hand brushed against hers; it could have been either accidental or deliberate.
	\end{itemize}
}
\end{enumerate}

\section*{bundle}
{\large \color{blue}  bundles  bundling  bundled  }
\subsection*{Explain}
\begin{enumerate}
\item countable noun \\
A \textbf{bundle}  \textbf{of} things is a number of them that are tied together or wrapped in a cloth or bag so that they can be carried or stored .
 \textit{
	\begin{itemize}
	\item She produced a bundle of notes and proceeded to count out one hundred and ninety-five
pounds.
	\item He gathered the bundles of clothing into his arms.
	\item I have about 20 year's magazines tied up in bundles.
	\end{itemize}
}
\item countable noun \\
You can refer to a tiny  baby as a \textbf{bundle} .
 \textit{
	\begin{itemize}
	\end{itemize}
}
\item singular noun \\
If you describe someone as, for example , a \textbf{bundle}  \textbf{of}  fun , you are emphasizing that they are full of fun. If you describe someone as a \textbf{bundle}  \textbf{of} nerves, you are emphasizing that they are very nervous .
 \textit{
	\begin{itemize}
	\item I remember Mickey as a bundle of fun, great to have around.
	\item Life at high school wasn't a bundle of laughs, either.
	\item He confessed to having been a bundle of nerves.
	\end{itemize}
}
\item countable noun \\
If you refer to a \textbf{bundle}  \textbf{of} things, you are emphasizing that there is a wide  range of them.
 \textit{
	\begin{itemize}
	\item The profession offers a bundle of benefits, not least of which is extensive training.
	\end{itemize}
}
\item verb \\
If someone \textbf{is bundled}  somewhere , someone pushes them there in a rough and hurried way.
 \textit{
	\begin{itemize}
	\item He was bundled into a car and driven 50 miles to a police station.
	\item He was bundled in and arrested as soon as he was airborne.
	\end{itemize}
}
\item verb \\
To \textbf{bundle}  software means to sell it together with a computer, or with other hardware or software, as part of a set.
 \textit{
	\begin{itemize}
	\item It's cheaper to buy software bundled with a PC than separately.
	\end{itemize}
}
\item  \\
 cost a bundle \textit{
	\begin{itemize}
	\end{itemize}
}
\end{enumerate}

\section*{accustomed}
{\large \color{blue}  }
\subsection*{Explain}
\begin{enumerate}
\item adjective \\
If you are \textbf{accustomed to} something, you know it so well or have experienced it so often that it seems  natural , unsurprising , or easy to deal with.
 \textit{
	\begin{itemize}
	\item I was accustomed to being the only child at a table full of adults.
	\item She had not yet become accustomed to the fact that she was a rich woman.
	\end{itemize}
}
\item adjective \\
When your eyes become \textbf{accustomed}  \textbf{to} darkness or bright light, they adjust so that you start to be able to see things, after not being able to see properly at first.
 \textit{
	\begin{itemize}
	\item My eyes were becoming accustomed to the gloom.
	\end{itemize}
}
\item adjective \\
You can use \textbf{accustomed} to describe an action that someone usually does, a quality that they usually show , or an object that they usually use.
 \textit{
	\begin{itemize}
	\item He took up his accustomed position with his back to the fire.
	\item Fred acted with his accustomed shrewdness.
	\item His cap was missing from its accustomed peg.
	\end{itemize}
}
\end{enumerate}

\section*{camel}
{\large \color{blue}  camels  }
\subsection*{Explain}
\begin{enumerate}
\item countable noun \\
A \textbf{camel} is a large animal that lives in deserts and is used for carrying goods and people.
Camels have long necks and one or two lumps on their backs called humps.
 \textit{
	\begin{itemize}
	\end{itemize}
}
\end{enumerate}

\section*{ashamed}
{\large \color{blue}  }
\subsection*{Explain}
\begin{enumerate}
\item adjective \\
If someone is \textbf{ashamed} , they feel  embarrassed or guilty because of something they do or they have done , or because of their appearance .
 \textit{
	\begin{itemize}
	\item I felt incredibly ashamed of myself for getting so angry.
	\item She was ashamed that she looked so shabby.
	\end{itemize}
}
\item adjective \\
If you are \textbf{ashamed of} someone, you feel embarrassed to be connected with them, often because of their appearance or because you disapprove of something they have done.
 \textit{
	\begin{itemize}
	\item I've never told this to anyone, but it's true, I was terribly ashamed of my mum.
	\end{itemize}
}
\item adjective \\
If someone is \textbf{ashamed}  \textbf{to} do something, they do not want to do it because they feel embarrassed about it.
 \textit{
	\begin{itemize}
	\item Women are often ashamed to admit they are being abused.
	\end{itemize}
}
\end{enumerate}

\section*{class}
{\large \color{blue}  classes  classing  classed  }
\subsection*{Explain}
\begin{enumerate}
\item countable noun \\
A \textbf{class} is a group of pupils or students who are taught together.
 \textit{
	\begin{itemize}
	\item He had to spend about six months in a class with younger students.
	\item Reducing class sizes should be a top priority.
	\end{itemize}
}
\item countable noun \\
A \textbf{class} is a course of teaching in a particular subject.
 \textit{
	\begin{itemize}
	\item He acquired a law degree by taking classes at night.
	\item I go to dance classes here in New York.
	\end{itemize}
}
\item uncountable noun \\
If you do something \textbf{in class} , you do it during a lesson in school.
 \textit{
	\begin{itemize}
	\item There is lots of reading in class.
	\end{itemize}
}
\item singular noun \\
The students in a school or university who finish their course in a particular year are often referred to as the \textbf{class of} that year.
 \textit{
	\begin{itemize}
	\item These two members of Yale's Class of 2002 never miss a reunion.
	\end{itemize}
}
\item variable noun \\
\textbf{Class} refers to the division of people in a society into groups according to their social status.
 \textit{
	\begin{itemize}
	\item ...the relationship between social classes.
	\item What it will do is create a whole new ruling class.
	\item ...the characteristics of the British class structure.
	\end{itemize}
}
\item countable noun \\
A \textbf{class}  \textbf{of} things is a group of them with similar characteristics.
 \textit{
	\begin{itemize}
	\item Harbour staff noticed that measurements given for the same class of boats often varied.
	\item ...the division of the stars into six classes of brightness.
	\end{itemize}
}
\item verb \\
If someone or something \textbf{is classed as} a particular thing, they are regarded as belonging to that group of things.
 \textit{
	\begin{itemize}
	\item Since the birds inter-breed they cannot be classed as different species.
	\item I class myself as an ordinary working person.
	\item I would class my garden as medium in size.
	\item He was not an explorer but can certainly be classed as a pioneer.
	\end{itemize}
}
\item uncountable noun \\
If you say that someone or something has \textbf{class} , you mean that they are elegant and sophisticated .
 \textit{
	\begin{itemize}
	\item He's got the same style off the pitch as he has on it–sheer class.
	\end{itemize}
}
\item adjective \\
If you describe someone or something as a \textbf{class} person or thing, you mean that they are very good.
 \textit{
	\begin{itemize}
	\item Kite is undoubtedly a class player.
	\end{itemize}
}
\item  \\
 a class act \textit{
	\begin{itemize}
	\end{itemize}
}
\item  \\
 in a class of one's own \textit{
	\begin{itemize}
	\end{itemize}
}
\end{enumerate}

\section*{coalition}
{\large \color{blue}  coalitions  }
\subsection*{Explain}
\begin{enumerate}
\item countable noun \\
A \textbf{coalition} is a government consisting of people from two or more political parties.
 \textit{
	\begin{itemize}
	\item Since June the country has had a coalition government.
	\item It took five months for the coalition to agree on and publish a medium-term economic
programme.
	\end{itemize}
}
\item countable noun \\
A \textbf{coalition} is a group consisting of people from different political or social groups who are
co-operating to achieve a particular aim .
 \textit{
	\begin{itemize}
	\item He had been opposed by a coalition of about 50 civil rights, women's and Latino organizations.
	\end{itemize}
}
\end{enumerate}

\section*{blunt}
{\large \color{blue}  blunter  bluntest  blunts  blunting  blunted  }
\subsection*{Explain}
\begin{enumerate}
\item adjective \\
If you are \textbf{blunt} , you say  exactly what you think without trying to be polite .
 \textit{
	\begin{itemize}
	\item She is blunt about her personal life.
	\item She told the industry in blunt terms that such discrimination is totally unacceptable.
	\end{itemize}
}
\item adjective \\
A \textbf{blunt} object has a rounded or flat end rather than a sharp one.
 \textit{
	\begin{itemize}
	\item One of them had been struck 13 times over the head with a blunt object.
	\end{itemize}
}
\item adjective \\
A \textbf{blunt} knife or blade is no longer sharp and does not cut  well .
 \textit{
	\begin{itemize}
	\end{itemize}
}
\item verb \\
If something \textbf{blunts} an emotion , a feeling or a need , it weakens it.
 \textit{
	\begin{itemize}
	\item The constant repetition of violence has blunted the human response to it.
	\item The passing of time will blunt the pain.
	\end{itemize}
}
\end{enumerate}

\section*{conclusion}
{\large \color{blue}  conclusions  }
\subsection*{Explain}
\begin{enumerate}
\item countable noun \\
When you come to a \textbf{conclusion} , you decide that something is true after you have thought about it carefully and have considered all the relevant  facts .
 \textit{
	\begin{itemize}
	\item Over the years I've come to the conclusion that she's a very great musician.
	\item I know I'm doing the right thing but other people will draw their own conclusions.
	\end{itemize}
}
\item singular noun \\
The \textbf{conclusion} of something is its ending .
 \textit{
	\begin{itemize}
	\item At the conclusion of the programme, I asked the children if they had any questions.
	\end{itemize}
}
\item singular noun \\
The \textbf{conclusion} of a treaty or a business deal is the act of arranging it or agreeing it.
 \textit{
	\begin{itemize}
	\item ...the expected conclusion of a free-trade agreement between the two countries.
	\end{itemize}
}
\item  \\
 a foregone conclusion \textit{
	\begin{itemize}
	\end{itemize}
}
\item  \\
 in conclusion \textit{
	\begin{itemize}
	\end{itemize}
}
\item  \\
 to jump to a conclusion \textit{
	\begin{itemize}
	\end{itemize}
}
\end{enumerate}

\section*{casual}
{\large \color{blue}  }
\subsection*{Explain}
\begin{enumerate}
\item adjective \\
If you are \textbf{casual} , you are, or you pretend to be, relaxed and not very concerned about what is happening or what you are doing.
 \textit{
	\begin{itemize}
	\item It's difficult for me to be casual about anything.
	\item He's an easy-going, friendly young man with a casual sort of attitude towards money.
	\end{itemize}
}
\item adjective \\
A \textbf{casual} event or situation  happens by chance or without planning .
 \textit{
	\begin{itemize}
	\item What you mean as a casual remark could be misinterpreted.
	\item Even a casual observer could notice the tense atmosphere.
	\end{itemize}
}
\item adjective \\
\textbf{Casual} clothes are ones that you normally wear at home or on holiday , and not on formal  occasions .
 \textit{
	\begin{itemize}
	\item I also bought some casual clothes for the weekend.
	\end{itemize}
}
\item adjective \\
\textbf{Casual} work is done for short periods and not on a permanent or regular  basis .
 \textit{
	\begin{itemize}
	\item ...establishments which employ people on a casual basis, such as pubs and restaurants.
	\item It became increasingly expensive to hire casual workers.
	\end{itemize}
}
\end{enumerate}

\section*{consequence}
{\large \color{blue}  consequences  }
\subsection*{Explain}
\begin{enumerate}
\item countable noun \\
The \textbf{consequences}  \textbf{of} something are the results or effects of it.
 \textit{
	\begin{itemize}
	\item Her lawyer said she understood the consequences of her actions and was prepared to
go to jail.
	\item An economic crisis may have tremendous consequences for our global security.
	\end{itemize}
}
\item  \\
 in consequence, as a consequence \textit{
	\begin{itemize}
	\end{itemize}
}
\item  \\
 of (...) consequence \textit{
	\begin{itemize}
	\end{itemize}
}
\item  \\
 take the consequences/face the consequences \textit{
	\begin{itemize}
	\end{itemize}
}
\end{enumerate}

\section*{clean}
{\large \color{blue}  cleaner  cleanest  cleans  cleaning  cleaned  }
\subsection*{Explain}
\begin{enumerate}
\item adjective \\
Something that is \textbf{clean} is free from dirt or unwanted marks.
 \textit{
	\begin{itemize}
	\item He wore his cleanest slacks, a clean shirt and a navy blazer.
	\item Disease has not been a problem because clean water is available.
	\item The metro is efficient and spotlessly clean.
	\item Tiled kitchen floors are easy to keep clean.
	\end{itemize}
}
\item adjective \\
You say that people or animals are \textbf{clean} when they keep themselves or their surroundings clean.
 \textit{
	\begin{itemize}
	\end{itemize}
}
\item adjective \\
A \textbf{clean}  fuel or chemical process does not create many harmful or polluting substances.
 \textit{
	\begin{itemize}
	\item Fans of electric cars say they are clean, quiet and economical.
	\end{itemize}
}
\item verb \\
If you \textbf{clean} something or \textbf{clean} dirt off it, you make it free from dirt and unwanted marks, for example by washing or wiping it. If something \textbf{cleans} easily, it is easy to clean.
 \textbf{Clean} is also a noun .
 \textit{
	\begin{itemize}
	\item Her father cleaned his glasses with a paper napkin.
	\item It took half an hour to clean the orange powder off the bath.
	\item He cleaned the flakes away with his coat sleeve.
	\item Wood flooring not only cleans easily, but it's environmentally friendly into the
bargain.
	\item Give the cooker a good clean.
	\end{itemize}
}
\item verb \\
If you \textbf{clean} a room or house, you make the inside of it and the furniture in it free from dirt and dust .
 \textit{
	\begin{itemize}
	\item My parents cooked and cleaned.
	\item She got up early and cleaned the flat.
	\end{itemize}
}
\item adjective \\
If you describe something such as a book, joke , or lifestyle as \textbf{clean} , you think that they are not sexually immoral or offensive .
 \textit{
	\begin{itemize}
	\item They're trying to show clean, wholesome, decent movies.
	\item Flirting is good clean fun.
	\item He became a model of clean living and Bible Belt virtues.
	\end{itemize}
}
\item adjective \\
If someone has a \textbf{clean}  reputation or record, they have never done anything illegal or wrong .
 \textit{
	\begin{itemize}
	\item Accusations of tax evasion have tarnished his clean image.
	\item You can hire these from most car hire firms, provided you have a clean driving licence.
	\end{itemize}
}
\item adjective \\
A \textbf{clean} game or fight is carried out fairly, according to the rules.
 \textit{
	\begin{itemize}
	\item He called for a clean fight in the election and an end to 'negative campaigning'.
	\item It was a clean match, well refereed.
	\end{itemize}
}
\item graded adjective \\
If you describe a flavour , smell , or colour as \textbf{clean} , you like it because it is light and fresh.
 \textit{
	\begin{itemize}
	\item ...the fresh, clean smell of the sea.
	\item Soft tones of blue and grey create a clean, bright look.
	\end{itemize}
}
\item adjective \\
A \textbf{clean}  sheet of paper has no writing or drawing on it.
 \textit{
	\begin{itemize}
	\item Take a clean sheet of paper and down the left-hand side make a list.
	\end{itemize}
}
\item adjective \\
If you make a \textbf{clean} break or start , you end a situation completely and start again in a different way.
 \textit{
	\begin{itemize}
	\item She wanted to make a clean break from her mother and father.
	\end{itemize}
}
\item adverb \\
\textbf{Clean} is used to emphasize that something was done completely.
 \textit{
	\begin{itemize}
	\item It burned clean through the seat of my overalls.
	\item The thief got clean away with the money.
	\item I clean forgot everything I had prepared.
	\end{itemize}
}
\item graded adjective \\
A \textbf{clean} shape is simple and regular , with definite , smooth  edges .
 \textit{
	\begin{itemize}
	\item He admires the clean lines of Shaker furniture.
	\item The drill should be slowly rotated to ensure a clean hole.
	\end{itemize}
}
\item adjective \\
You can describe an action as \textbf{clean} to indicate that it is carried out simply and quickly without mistakes .
 \textit{
	\begin{itemize}
	\item They were more concerned about the dogs' welfare than a clean getaway.
	\item Paul had arrested countless men like this one before and was expecting a clean, quick
job.
	\end{itemize}
}
\item  \\
 to come clean \textit{
	\begin{itemize}
	\end{itemize}
}
\end{enumerate}

\section*{crane}
{\large \color{blue}  cranes  craning  craned  }
\subsection*{Explain}
\begin{enumerate}
\item countable noun \\
A \textbf{crane} is a large machine that moves heavy things by lifting them in the air .
 \textit{
	\begin{itemize}
	\item The little prefabricated hut was lifted away by a huge crane.
	\end{itemize}
}
\item countable noun \\
A \textbf{crane} is a kind of large bird with a long neck and long legs .
 \textit{
	\begin{itemize}
	\end{itemize}
}
\item verb \\
If you \textbf{crane} your neck or head, you stretch your neck in a particular direction in order to see or hear something better .
 \textit{
	\begin{itemize}
	\item She craned her neck to get a better view.
	\item Children craned to get close to him.
	\item She craned forward to look at me.
	\end{itemize}
}
\end{enumerate}

\section*{coarse}
{\large \color{blue}  coarser  coarsest  }
\subsection*{Explain}
\begin{enumerate}
\item adjective \\
\textbf{Coarse} things have a rough texture because they consist of thick threads or large pieces.
 \textit{
	\begin{itemize}
	\item ...a jacket made of very coarse cloth.
	\item ...a beach of coarse sand.
	\end{itemize}
}
\item adjective \\
If you describe someone as \textbf{coarse} , you mean that he or she talks and behaves in a rude and offensive  way .
 \textit{
	\begin{itemize}
	\item The soldiers did not bother to moderate their coarse humour in her presence.
	\end{itemize}
}
\end{enumerate}

\section*{despair}
{\large \color{blue}  despairs  despairing  despaired  }
\subsection*{Explain}
\begin{enumerate}
\item uncountable noun \\
\textbf{Despair} is the feeling that everything is wrong and that nothing will  improve .
 \textit{
	\begin{itemize}
	\item I looked at my wife in despair.
	\item ...feelings of despair or inadequacy.
	\end{itemize}
}
\item verb \\
If you \textbf{despair} , you feel that everything is wrong and that nothing will improve.
 \textit{
	\begin{itemize}
	\item 'Oh, I despair sometimes,' he says in mock sorrow.
	\item He does despair at much of the press criticism.
	\end{itemize}
}
\item verb \\
If you \textbf{despair of} something, you feel that there is no hope that it will happen or improve. If you \textbf{despair of} someone, you feel that there is no hope that they will improve.
 \textit{
	\begin{itemize}
	\item He wished to earn a living through writing but despaired of doing so.
	\item ...efforts to find homes for people despairing of ever having a roof over their heads.
	\item There are signs that many voters have already despaired of politicians.
	\end{itemize}
}
\end{enumerate}

\section*{cohesive}
{\large \color{blue}  }
\subsection*{Explain}
\begin{enumerate}
\item adjective \\
Something that is \textbf{cohesive} consists of parts that fit together well and form a united  whole .
 \textit{
	\begin{itemize}
	\item It takes an enormous amount of work to make a cohesive album.
	\item Huston had assembled a remarkably cohesive and sympathetic cast.
	\end{itemize}
}
\end{enumerate}

\section*{drawing}
{\large \color{blue}  drawings  }
\subsection*{Explain}
\begin{enumerate}
\item countable noun \\
A \textbf{drawing} is a picture made with a pencil or pen.
 \textit{
	\begin{itemize}
	\item She did a drawing of me.
	\end{itemize}
}
\end{enumerate}

\section*{commercial}
{\large \color{blue}  commercials  }
\subsection*{Explain}
\begin{enumerate}
\item adjective \\
\textbf{Commercial} means involving or relating to the buying and selling of goods.
 \textit{
	\begin{itemize}
	\item Docklands in its heyday was a major centre of industrial and commercial activity.
	\item Attacks were reported on police, vehicles and commercial premises.
	\end{itemize}
}
\item adjective \\
\textbf{Commercial} organizations and activities are concerned with making money or profits, rather than,
for example , with scientific  research or providing a public service.
 \textit{
	\begin{itemize}
	\item The NHS adopted a more commercial and businesslike financial framework.
	\item Conservationists are concerned over the effect of commercial exploitation of forests.
	\item Whether the project will be a commercial success is still uncertain.
	\end{itemize}
}
\item adjective \\
A \textbf{commercial} product is made to be sold to the public.
 \textit{
	\begin{itemize}
	\item They are the leading manufacturer in both defence and commercial products.
	\end{itemize}
}
\item adjective \\
A \textbf{commercial} vehicle is a vehicle used for carrying goods, or passengers who pay.
 \textit{
	\begin{itemize}
	\item Commercial vehicles, coaches and lorries are required by law to be fitted with tachographs.
	\item ...the fastest crossing of the Atlantic by a commercial passenger vessel.
	\end{itemize}
}
\item adjective \\
\textbf{Commercial} television and radio are paid for by the broadcasting of advertisements, rather than by the government.
 \textit{
	\begin{itemize}
	\item He got a job as a programme controller for the local commercial radio station.
	\end{itemize}
}
\item adjective \\
\textbf{Commercial} is used to describe something such as a film or a type of music that it is intended to be popular with the public, and is not very original or of high quality.
 \textit{
	\begin{itemize}
	\item There's a feeling among a lot of people that music has become too commercial.
	\end{itemize}
}
\item countable noun \\
A \textbf{commercial} is an advertisement that is broadcast on television or radio.
 \textit{
	\begin{itemize}
	\item The government has launched a campaign of television commercials and leaflets.
	\end{itemize}
}
\end{enumerate}

\section*{effect}
{\large \color{blue}  effects  effecting  effected  }
\subsection*{Explain}
\begin{enumerate}
\item variable noun \\
The \textbf{effect}  \textbf{of} one thing \textbf{on} another is the change that the first thing causes in the second thing.
 \textit{
	\begin{itemize}
	\item Parents worry about the effect of music on their adolescent's behavior.
	\item The austerity measures will have little immediate effect on the average citizen.
	\item Even minor head injuries can cause long-lasting psychological effects.
	\end{itemize}
}
\item countable noun \\
An \textbf{effect} is an impression that someone creates deliberately, for example in a place or in a piece of writing .
 \textit{
	\begin{itemize}
	\item The whole effect is cool, light and airy.
	\end{itemize}
}
\item plural noun \\
A person's \textbf{effects} are the things that they have with them at a particular time, for example when they
are arrested or admitted to hospital , or the things that they owned when they died .
 \textit{
	\begin{itemize}
	\item His daughters were collecting his effects.
	\end{itemize}
}
\item plural noun \\
The \textbf{effects} in a film are the specially created sounds and scenery .
 \textit{
	\begin{itemize}
	\end{itemize}
}
\item verb \\
If you \textbf{effect} something that you are trying to achieve , you succeed in causing it to happen .
 \textit{
	\begin{itemize}
	\item Prospects for effecting real political change have taken a step backwards.
	\end{itemize}
}
\item  \\
 for effect \textit{
	\begin{itemize}
	\end{itemize}
}
\item  \\
 in effect \textit{
	\begin{itemize}
	\end{itemize}
}
\item  \\
 (put/bring/carry) sth into effect \textit{
	\begin{itemize}
	\end{itemize}
}
\item  \\
 take/come into effect \textit{
	\begin{itemize}
	\end{itemize}
}
\item  \\
 take effect \textit{
	\begin{itemize}
	\end{itemize}
}
\item  \\
 to (good) effect \textit{
	\begin{itemize}
	\end{itemize}
}
\item  \\
 to this/that effect \textit{
	\begin{itemize}
	\end{itemize}
}
\item  \\
 with (immediate) effect/effect from \textit{
	\begin{itemize}
	\end{itemize}
}
\end{enumerate}

\section*{composite}
{\large \color{blue}  composites  }
\subsection*{Explain}
\begin{enumerate}
\item adjective \\
A \textbf{composite} object or item is made up of several different things, parts, or substances.
 \textbf{Composite} is also a noun .
 \textit{
	\begin{itemize}
	\item ...composite pictures with different faces superimposed over one another.
	\item Spain is a composite of diverse traditions and people.
	\end{itemize}
}
\end{enumerate}

\section*{ending}
{\large \color{blue}  endings  }
\subsection*{Explain}
\begin{enumerate}
\item countable noun \\
You can  refer to the last part of a book, story , play , or film as the \textbf{ending} , especially when you are considering the way that the story ends.
 \textit{
	\begin{itemize}
	\item The film has a Hollywood happy ending.
	\end{itemize}
}
\item countable noun \\
The \textbf{ending} of a word is the last part of it.
 \textit{
	\begin{itemize}
	\item ...common word endings, like 'ing' in walking.
	\end{itemize}
}
\end{enumerate}

\section*{conventional}
{\large \color{blue}  }
\subsection*{Explain}
\begin{enumerate}
\item adjective \\
Someone who is \textbf{conventional} has behaviour or opinions that are ordinary and normal .
 \textit{
	\begin{itemize}
	\item ...a respectable married woman with conventional opinions.
	\end{itemize}
}
\item adjective \\
A \textbf{conventional} method or product is one that is usually used or that has been in use for a long
time.
 \textit{
	\begin{itemize}
	\item ...the risks and drawbacks of conventional family planning methods.
	\item This new memory stick holds twice as much information as a conventional pen drive.
	\end{itemize}
}
\item adjective \\
\textbf{Conventional} weapons and wars do not involve nuclear explosives .
 \textit{
	\begin{itemize}
	\item We must reduce the danger of war by controlling nuclear, chemical and conventional
arms.
	\end{itemize}
}
\end{enumerate}

\section*{enthusiasm}
{\large \color{blue}  enthusiasms  }
\subsection*{Explain}
\begin{enumerate}
\item variable noun \\
\textbf{Enthusiasm} is great eagerness to be involved in a particular activity which you like and enjoy or which you think is important .
 \textit{
	\begin{itemize}
	\item His enthusiasm for the game is infectious.
	\item Their skill, enthusiasm and running has got them in the team.
	\end{itemize}
}
\item countable noun \\
An \textbf{enthusiasm} is an activity or subject that interests you very much and that you spend a lot of time on.
 \textit{
	\begin{itemize}
	\item Draw him out about his current enthusiasms and future plans.
	\end{itemize}
}
\end{enumerate}

\section*{fright}
{\large \color{blue}  frights  }
\subsection*{Explain}
\begin{enumerate}
\item uncountable noun \\
\textbf{Fright} is a sudden feeling of fear, especially the fear that you feel when something unpleasant surprises you.
 \textit{
	\begin{itemize}
	\item The steam pipes rattled suddenly, and Franklin uttered a shriek and jumped with fright.
	\item The birds smashed into the top of their cages in fright.
	\item To hide my fright I asked a question.
	\end{itemize}
}
\item countable noun \\
\textbf{A}  \textbf{fright} is an experience which makes you suddenly  afraid .
 \textit{
	\begin{itemize}
	\item The snake picked up its head and stuck out its tongue which gave everyone a fright.
	\item The last time you had a real fright, you nearly crashed the car.
	\end{itemize}
}
\item  \\
 take fright \textit{
	\begin{itemize}
	\end{itemize}
}
\end{enumerate}

\section*{governor}
{\large \color{blue}  governors  }
\subsection*{Explain}
\begin{enumerate}
\item countable noun \\
In some systems of government, a \textbf{governor} is a person who is in charge of the political  administration of a region or state.
 \textit{
	\begin{itemize}
	\item He was governor of the province in the late 1970s.
	\item The governor addressed the New Jersey Assembly.
	\end{itemize}
}
\item countable noun \\
A \textbf{governor} is a member of a committee which controls an organization such as a school or a hospital .
 \textit{
	\begin{itemize}
	\item Governors are using the increased powers given to them to act against incompetent
headteachers.
	\item ...the chairman of the BBC board of governors.
	\end{itemize}
}
\item countable noun \\
In some British institutions , the \textbf{governor} is the most senior official , who is in charge of the institution.
 \textit{
	\begin{itemize}
	\item The incident was reported to the prison governor.
	\end{itemize}
}
\end{enumerate}

\section*{deliberate}
{\large \color{blue}  deliberates  deliberating  deliberated  }
\subsection*{Explain}
\begin{enumerate}
\item adjective \\
If you do something that is \textbf{deliberate} , you planned or decided to do it beforehand , and so it happens on purpose  rather than by chance .
 \textit{
	\begin{itemize}
	\item It has a deliberate policy to introduce world art to Britain.
	\item Witnesses say the firing was deliberate and sustained.
	\end{itemize}
}
\item adjective \\
If a movement or action is \textbf{deliberate} , it is done slowly and carefully.
 \textit{
	\begin{itemize}
	\item His movements were gentle and deliberate.
	\item ...stepping with deliberate slowness up the steep paths.
	\end{itemize}
}
\item verb \\
If you \textbf{deliberate} , you think about something carefully, especially before making a very important  decision .
 \textit{
	\begin{itemize}
	\item She deliberated over the decision for a long time before she made up her mind.
	\item The six-person jury deliberated about two hours before returning with the verdict.
	\item The Court of Criminal Appeals has been deliberating his case for almost two weeks.
	\end{itemize}
}
\end{enumerate}

\section*{heat}
{\large \color{blue}  heats  heating  heated  }
\subsection*{Explain}
\begin{enumerate}
\item verb \\
When you \textbf{heat} something, you raise its temperature, for example by using a flame or a special piece of equipment .
 \textit{
	\begin{itemize}
	\item Meanwhile, heat the tomatoes and oil in a pan.
	\item ...a gas that absorbs the sun's energy and heats the air above it.
	\item ...heated swimming pools.
	\end{itemize}
}
\item uncountable noun \\
\textbf{Heat} is warmth or the quality of being hot.
 \textit{
	\begin{itemize}
	\item The seas store heat and release it gradually during cold periods.
	\item Its leaves drooped a little in the fierce heat of the sun.
	\end{itemize}
}
\item uncountable noun \\
\textbf{The}  \textbf{heat} is very hot weather.
 \textit{
	\begin{itemize}
	\item As an asthmatic, he cannot cope with the heat and humidity.
	\item This heat is killing me.
	\end{itemize}
}
\item uncountable noun \\
The \textbf{heat} of something is the temperature of something that is warm or that is being heated.
 \textit{
	\begin{itemize}
	\item Warm the milk to blood heat.
	\item Adjust the heat of the barbecue by opening and closing the air vents.
	\end{itemize}
}
\item singular noun \\
You use \textbf{heat} to refer to a source of heat, for example a cooking ring or the heating system of a house.
 \textit{
	\begin{itemize}
	\item Immediately remove the pan from the heat.
	\item Some apartment buildings don't have their heat turned on till the end of this week.
	\end{itemize}
}
\item uncountable noun \\
You use \textbf{heat} to refer to a state of strong emotion , especially of anger or excitement.
 \textit{
	\begin{itemize}
	\item It was all done in the heat of the moment and I have certainly learned by my mistake.
	\item 'Look here,' I said, without heat, 'all I did was to walk down a street and sit down.'
	\end{itemize}
}
\item singular noun \\
\textbf{The heat of} a particular activity is the point when there is the greatest activity or excitement.
 \textit{
	\begin{itemize}
	\item Last week, in the heat of the election campaign, the Prime Minister left for America.
	\end{itemize}
}
\item countable noun \\
A \textbf{heat} is one of a series of races or competitions. The winners of a heat take part in another race or competition, against the winners of other
heats.
 \textit{
	\begin{itemize}
	\item ...the heats of the men's 100m breaststroke.
	\end{itemize}
}
\item  \\
 on heat/in heat \textit{
	\begin{itemize}
	\end{itemize}
}
\end{enumerate}

\section*{distinct}
{\large \color{blue}  }
\subsection*{Explain}
\begin{enumerate}
\item adjective \\
If something is \textbf{distinct}  \textbf{from} something else of the same type, it is different or separate from it.
 \textit{
	\begin{itemize}
	\item Engineering and technology are disciplines distinct from one another and from science.
	\item This book is divided into two distinct parts.
	\end{itemize}
}
\item adjective \\
If something is \textbf{distinct} , you can hear , see , or taste it clearly .
 \textit{
	\begin{itemize}
	\item ...to impart a distinct flavor with a minimum of cooking fat.
	\end{itemize}
}
\item adjective \\
If an idea , thought , or intention is \textbf{distinct} , it is clear and definite.
 \textit{
	\begin{itemize}
	\item Now that Tony was no longer present, there was a distinct change in her attitude.
	\item I have distinct memories of him in his last years.
	\end{itemize}
}
\item adjective \\
You can use \textbf{distinct} to emphasize that something is great enough in amount or degree to be noticeable or important .
 \textit{
	\begin{itemize}
	\item Being 6ft 3in tall has some distinct disadvantages!
	\item Another Cup marathon between the two sides is now a distinct possibility.
	\end{itemize}
}
\item  \\
 as distinct from \textit{
	\begin{itemize}
	\end{itemize}
}
\end{enumerate}

\section*{horror}
{\large \color{blue}  horrors  }
\subsection*{Explain}
\begin{enumerate}
\item uncountable noun \\
\textbf{Horror} is a feeling of great  shock , fear, and worry caused by something extremely  unpleasant .
 \textit{
	\begin{itemize}
	\item I felt numb with horror.
	\item As I watched in horror the boat began to power away from me.
	\end{itemize}
}
\item singular noun \\
If you have a \textbf{horror of} something, you are afraid of it or dislike it very much.
 \textit{
	\begin{itemize}
	\item ...his horror of death.
	\end{itemize}
}
\item singular noun \\
The \textbf{horror}  \textbf{of} something, especially something that hurts people, is its very great unpleasantness.
 \textit{
	\begin{itemize}
	\item ...the horror of this most bloody of civil wars.
	\end{itemize}
}
\item countable noun \\
You can  refer to extremely unpleasant or frightening experiences as \textbf{horrors} .
 \textit{
	\begin{itemize}
	\item Can you possibly imagine all the horrors we have undergone since I last wrote you?
	\end{itemize}
}
\item countable noun \\
If you refer to someone or something as a \textbf{horror} , you mean that you think they are very unpleasant or ugly .
 \textit{
	\begin{itemize}
	\item I didn't want to listen. I was a horror. He did well to put up with me.
	\end{itemize}
}
\item adjective \\
A \textbf{horror}  film or story is intended to be very frightening.
 \textit{
	\begin{itemize}
	\item ...a psychological horror film.
	\end{itemize}
}
\item adjective \\
You can refer to an account of a very unpleasant experience or event as a \textbf{horror} story.
 \textit{
	\begin{itemize}
	\item ...a horror story about lost luggage while flying.
	\end{itemize}
}
\item  \\
 horror of horrors \textit{
	\begin{itemize}
	\end{itemize}
}
\end{enumerate}

\section*{elastic}
{\large \color{blue}  elastics  }
\subsection*{Explain}
\begin{enumerate}
\item uncountable noun \\
\textbf{Elastic} is a rubber material that stretches when you pull it and returns to its original size and shape when you let it go . Elastic is often used in clothes to make them fit tightly, for example  round the waist .
 \textit{
	\begin{itemize}
	\item ...a piece of elastic.
	\item ...my plaid Bermuda shorts with the elastic waist.
	\end{itemize}
}
\item adjective \\
Something that is \textbf{elastic} is able to stretch easily and then return to its original size and shape.
 \textit{
	\begin{itemize}
	\item Beat it until the dough is slightly elastic.
	\item ...an elastic rope.
	\end{itemize}
}
\item adjective \\
If ideas , plans , or policies are \textbf{elastic} , they are able to change to suit new circumstances or conditions as they occur.
 \textit{
	\begin{itemize}
	\item ...an elastic interpretation of the rules of boxing.
	\item The period of conversion was elastic, in some cases lasting over twenty years
	\end{itemize}
}
\item countable noun \\
An \textbf{elastic} is a rubber band .
 \textit{
	\begin{itemize}
	\end{itemize}
}
\end{enumerate}

\section*{liner}
{\large \color{blue}  liners  }
\subsection*{Explain}
\begin{enumerate}
\item countable noun \\
A \textbf{liner} is a large ship in which people travel long distances , especially on holiday .
 \textit{
	\begin{itemize}
	\item ...luxury ocean liners.
	\item ...the cruise liner, the QE2.
	\end{itemize}
}
\end{enumerate}

\section*{green}
{\large \color{blue}  greens  greener  greenest  }
\subsection*{Explain}
\begin{enumerate}
\item colour \\
\textbf{Green} is the colour of grass or leaves.
 \textit{
	\begin{itemize}
	\item ...shiny red and green apples.
	\item Yellow and green together make a pale green.
	\end{itemize}
}
\item adjective \\
A place that is \textbf{green} is covered with grass, plants, and trees and not with houses or factories .
 \textit{
	\begin{itemize}
	\item Cairo has only thirteen square centimetres of green space for each inhabitant.
	\end{itemize}
}
\item adjective \\
\textbf{Green} issues and political movements relate to or are concerned with the protection of the environment.
 \textit{
	\begin{itemize}
	\item The power of the Green movement in Germany has made that country a leader in the
drive to recycle more waste materials.
	\end{itemize}
}
\item adjective \\
If you say that someone or something is \textbf{green} , you mean they harm the environment as little as possible.
 \textit{
	\begin{itemize}
	\item ...trying to persuade governments to adopt greener policies.
	\item Our children are being educated to be green in everything they do.
	\end{itemize}
}
\item countable noun \\
\textbf{Greens} are members of green political movements.
 \textit{
	\begin{itemize}
	\item The Greens won a seat.
	\end{itemize}
}
\item countable noun \\
A \textbf{green} is a smooth, flat area of grass around a hole on a golf course.
 \textit{
	\begin{itemize}
	\item ...the 18th green.
	\end{itemize}
}
\item countable noun \\
A \textbf{green} is an area of land covered with grass, especially in a town or in the middle of a village.
 \textit{
	\begin{itemize}
	\item ...the village green.
	\end{itemize}
}
\item countable noun \\
\textbf{Green} is used in the names of places that contain or used to contain an area of grass.
 \textit{
	\begin{itemize}
	\item ...Bethnal Green.
	\end{itemize}
}
\item plural noun \\
You can refer to the cooked leaves of vegetables such as spinach or cabbage as \textbf{greens} .
 \textit{
	\begin{itemize}
	\item Eat your greens.
	\end{itemize}
}
\item graded adjective \\
You can describe fruit and vegetables as \textbf{green} when they are unripe and not ready to be eaten.
 \textit{
	\begin{itemize}
	\item Pick and ripen any green fruits in a warm dark place.
	\end{itemize}
}
\item adjective \\
If you say that someone is \textbf{green} , you mean that they have had very little experience of life or a particular job.
 \textit{
	\begin{itemize}
	\item He was a young lad, very green, very immature.
	\end{itemize}
}
\item  \\
 green with envy \textit{
	\begin{itemize}
	\end{itemize}
}
\item  \\
 to have green fingers \textit{
	\begin{itemize}
	\end{itemize}
}
\end{enumerate}

\section*{harsh}
{\large \color{blue}  harsher  harshest  }
\subsection*{Explain}
\begin{enumerate}
\item adjective \\
\textbf{Harsh}  climates or conditions are very difficult for people, animals, and plants to live in.
 \textit{
	\begin{itemize}
	\item The weather grew harsh, chilly and unpredictable.
	\item ...the harsh desert environment.
	\item ...after the harsh experience of the war.
	\end{itemize}
}
\item adjective \\
\textbf{Harsh} actions or speech are unkind and show no understanding or sympathy .
 \textit{
	\begin{itemize}
	\item He said many harsh and unkind things about his opponents.
	\end{itemize}
}
\item adjective \\
Something that is \textbf{harsh} is so hard, bright , or rough that it seems  unpleasant or harmful .
 \textit{
	\begin{itemize}
	\item Tropical colours may look rather harsh in our dull northern light.
	\item ...harsher detergents that can leave hair brittle.
	\end{itemize}
}
\item adjective \\
\textbf{Harsh}  voices and sounds are ones that are rough and unpleasant to listen to.
 \textit{
	\begin{itemize}
	\item It's a pity she has such a loud harsh voice.
	\end{itemize}
}
\item adjective \\
If you talk about \textbf{harsh}  realities or facts , or the \textbf{harsh}  truth , you are emphasizing that they are true or real , although they are unpleasant and people try to avoid  thinking about them.
 \textit{
	\begin{itemize}
	\item The harsh truth is that luck plays a big part in who will live or die.
	\end{itemize}
}
\end{enumerate}

\section*{marriage}
{\large \color{blue}  marriages  }
\subsection*{Explain}
\begin{enumerate}
\item countable noun \\
A \textbf{marriage} is the relationship between two people who are married .
 \textit{
	\begin{itemize}
	\item In a good marriage, both partners work hard to solve any problems that arise.
	\item When I was 35 my marriage broke up.
	\item His son by his second marriage lives in Paris.
	\end{itemize}
}
\item variable noun \\
A \textbf{marriage} is the act of marrying someone, or the ceremony at which this is done.
 \textit{
	\begin{itemize}
	\item I opposed her marriage to Darryl.
	\end{itemize}
}
\item uncountable noun \\
\textbf{Marriage} is the state of being married.
 \textit{
	\begin{itemize}
	\item Marriage might not suit you.
	\item In twenty years of marriage he has only taken two proper vacations.
	\end{itemize}
}
\end{enumerate}

\section*{hysterical}
{\large \color{blue}  }
\subsection*{Explain}
\begin{enumerate}
\item adjective \\
Someone who is \textbf{hysterical} is in a state of uncontrolled  excitement , anger , or panic .
 \textit{
	\begin{itemize}
	\item The singer had to leave by a side exit to flee 200 hysterical fans.
	\item He made headlines and received hysterical hate mail.
	\end{itemize}
}
\item adjective \\
Someone who is \textbf{hysterical} is in a state of violent and disturbed  emotion that is usually a result of shock .
 \textit{
	\begin{itemize}
	\item I suffered bouts of really hysterical depression.
	\end{itemize}
}
\item adjective \\
\textbf{Hysterical}  laughter is loud and uncontrolled.
 \textit{
	\begin{itemize}
	\item I had to rush to the loo to avoid an attack of hysterical giggles.
	\end{itemize}
}
\item adjective \\
If you describe something or someone as \textbf{hysterical} , you think that they are very funny and they make you laugh a lot .
 \textit{
	\begin{itemize}
	\item Paul Mazursky was Master of Ceremonies, and he was pretty hysterical.
	\end{itemize}
}
\end{enumerate}

\section*{mischief}
{\large \color{blue}  }
\subsection*{Explain}
\begin{enumerate}
\item uncountable noun \\
\textbf{Mischief} is playing harmless  tricks on people or doing things you are not supposed to do. It can also  refer to the desire to do this.
 \textit{
	\begin{itemize}
	\item The little lad was a real handful. He was always up to mischief.
	\item Boys at that age should be able to explore and get into mischief.
	\item His eyes were full of mischief.
	\end{itemize}
}
\item uncountable noun \\
\textbf{Mischief} is behaviour that is intended to cause trouble for people. It can also refer to the trouble that is caused.
 \textit{
	\begin{itemize}
	\item Angry MPs have continued to make mischief.
	\end{itemize}
}
\end{enumerate}

\section*{leading}
{\large \color{blue}  }
\subsection*{Explain}
\begin{enumerate}
\item adjective \\
The \textbf{leading} person or thing in a particular area is the one which is most important or successful .
 \textit{
	\begin{itemize}
	\item ...a leading member of Bristol's Sikh community.
	\item Britain's future as a leading industrial nation depends on investment.
	\end{itemize}
}
\item adjective \\
The \textbf{leading}  role in a play or film is the main role. A \textbf{leading}  lady or man is an actor who plays this role.
 \textit{
	\begin{itemize}
	\end{itemize}
}
\item adjective \\
The \textbf{leading} group, vehicle , or person in a race or procession is the one that is at the front .
 \textit{
	\begin{itemize}
	\end{itemize}
}
\end{enumerate}

\section*{monitor}
{\large \color{blue}  monitors  monitoring  monitored  }
\subsection*{Explain}
\begin{enumerate}
\item verb \\
If you \textbf{monitor} something, you regularly check its development or progress , and sometimes comment on it.
 \textit{
	\begin{itemize}
	\item Officials had not been allowed to monitor the voting.
	\item You need feedback to monitor progress.
	\end{itemize}
}
\item verb \\
If someone \textbf{monitors} radio broadcasts from other countries, they record them or listen carefully to them in order to obtain information.
 \textit{
	\begin{itemize}
	\item Peter Murray is in London and has been monitoring reports out of Monrovia.
	\end{itemize}
}
\item countable noun \\
A \textbf{monitor} is a machine that is used to check or record things, for example processes or substances
 inside a person's body.
 \textit{
	\begin{itemize}
	\item The heart monitor shows low levels of consciousness.
	\end{itemize}
}
\item countable noun \\
A \textbf{monitor} is a screen which is used to display certain kinds of information, for example in
airports or television studios.
 \textit{
	\begin{itemize}
	\item He was watching a game of tennis on a television monitor.
	\end{itemize}
}
\item countable noun \\
A \textbf{monitor} is the screen on a computer.
 \textit{
	\begin{itemize}
	\item Last night you went home without switching off your computer monitor.
	\end{itemize}
}
\item countable noun \\
You can refer to a person who checks that something is done correctly, or that it is fair , as a \textbf{monitor} .
 \textit{
	\begin{itemize}
	\item Government monitors will continue to accompany reporters.
	\end{itemize}
}
\end{enumerate}

\section*{marital}
{\large \color{blue}  }
\subsection*{Explain}
\begin{enumerate}
\item adjective \\
\textbf{Marital} is used to describe things relating to marriage.
 \textit{
	\begin{itemize}
	\item Caroline was keen to make her marital home in London.
	\item Her son had no marital problems.
	\end{itemize}
}
\end{enumerate}

\section*{nightmare}
{\large \color{blue}  nightmares  }
\subsection*{Explain}
\begin{enumerate}
\item countable noun \\
A \textbf{nightmare} is a very frightening dream.
 \textit{
	\begin{itemize}
	\item All the victims still suffered nightmares.
	\item Jane did not eat cheese because it gives her nightmares.
	\end{itemize}
}
\item countable noun \\
If you refer to a situation as a \textbf{nightmare} , you mean that it is very frightening and unpleasant .
 \textit{
	\begin{itemize}
	\item The years in prison were a nightmare.
	\end{itemize}
}
\item countable noun \\
If you refer to a situation as a \textbf{nightmare} , you are saying in a very emphatic way that it is irritating because it causes you a lot of trouble .
 \textit{
	\begin{itemize}
	\item Taking my son Peter to a restaurant was a nightmare.
	\item In practice a graduate tax is an administrative nightmare.
	\end{itemize}
}
\end{enumerate}

\section*{numb}
{\large \color{blue}  numbs  numbing  numbed  }
\subsection*{Explain}
\begin{enumerate}
\item adjective \\
If a part of your body is \textbf{numb} , you cannot feel anything there.
 \textit{
	\begin{itemize}
	\item He could feel his fingers growing numb at their tips.
	\item My legs felt numb and my toes ached.
	\end{itemize}
}
\item adjective \\
If you are \textbf{numb}  \textbf{with} shock, fear , or grief , you are so shocked, frightened , or upset that you cannot think  clearly or feel any emotion .
 \textit{
	\begin{itemize}
	\item The mother, numb with grief, has trouble speaking.
	\item I was so shocked I went numb.
	\end{itemize}
}
\item verb \\
If an event or experience  \textbf{numbs} you, you can no longer think clearly or feel any emotion.
 \textit{
	\begin{itemize}
	\item For a while the shock of Philippe's letter numbed her.
	\item The horror of my experience has numbed my senses.
	\end{itemize}
}
\item  \\
 See also  mind-numbing \textit{
	\begin{itemize}
	\end{itemize}
}
\item verb \\
If cold weather , a drug , or a blow  \textbf{numbs} a part of your body, you can no longer feel anything in it.
 \textit{
	\begin{itemize}
	\item The cold numbed my fingers.
	\item An injection of local anaesthetic is usually given first to numb the area.
	\item She awoke with a numbed feeling in her left leg.
	\end{itemize}
}
\end{enumerate}

\section*{origin}
{\large \color{blue}  origins  }
\subsection*{Explain}
\begin{enumerate}
\item variable noun \\
You can refer to the beginning, cause, or source of something as its \textbf{origin} or \textbf{origins} .
 \textit{
	\begin{itemize}
	\item ...theories about the origin of life.
	\item The disorder in military policy had its origins in Truman's first term.
	\item Their medical problems are basically physical in origin.
	\item Most of the thickeners are of plant origin.
	\end{itemize}
}
\item countable noun \\
When you talk about a person's \textbf{origin} or \textbf{origins} , you are referring to the country, race , or social class of their parents or ancestors .
 \textit{
	\begin{itemize}
	\item Thomas has not forgotten his humble origins.
	\item ...people of Asian origin.
	\item They are forced to return to their country of origin.
	\end{itemize}
}
\end{enumerate}

\section*{occasional}
{\large \color{blue}  }
\subsection*{Explain}
\begin{enumerate}
\item adjective \\
\textbf{Occasional} means happening sometimes, but not regularly or often.
 \textit{
	\begin{itemize}
	\item I've had occasional mild headaches all my life.
	\item Esther used to visit him for the occasional days and weekends.
	\end{itemize}
}
\end{enumerate}

\section*{outcome}
{\large \color{blue}  outcomes  }
\subsection*{Explain}
\begin{enumerate}
\item countable noun \\
The \textbf{outcome} of an activity, process, or situation is the situation that exists at the end of it.
 \textit{
	\begin{itemize}
	\item Mr. Singh said he was pleased with the outcome.
	\item It's too early to know the outcome of her illness.
	\item ...a successful outcome.
	\end{itemize}
}
\end{enumerate}

\section*{profound}
{\large \color{blue}  profounder  profoundest  }
\subsection*{Explain}
\begin{enumerate}
\item adjective \\
You use \textbf{profound} to emphasize that something is very great or intense.
 \textit{
	\begin{itemize}
	\item ...discoveries which had a profound effect on many areas of medicine.
	\item ...profound disagreement.
	\item The overwhelming feeling is just deep, profound shock and anger.
	\item Anna's patriotism was profound.
	\end{itemize}
}
\item adjective \\
A \textbf{profound} idea, work, or person shows great intellectual depth and understanding.
 \textit{
	\begin{itemize}
	\item This is a book full of profound, original and challenging insights.
	\item ...one of the country's most profound minds.
	\end{itemize}
}
\end{enumerate}

\section*{passion}
{\large \color{blue}  passions  }
\subsection*{Explain}
\begin{enumerate}
\item uncountable noun \\
\textbf{Passion} is strong sexual feelings towards someone.
 \textit{
	\begin{itemize}
	\item ...my passion for a dark-haired, slender boy named James.
	\item ...the expression of love and passion.
	\item ...Maggy, the object of his passions.
	\end{itemize}
}
\item uncountable noun \\
\textbf{Passion} is a very strong feeling about something or a strong belief in something.
 \textit{
	\begin{itemize}
	\item He spoke with great passion.
	\item ...the passion and commitment of the Republican candidate.
	\end{itemize}
}
\item countable noun \\
If you have a \textbf{passion for} something, you have a very strong interest in it and like it very much.
 \textit{
	\begin{itemize}
	\item She had a passion for gardening.
	\item His other great passion was Italy.
	\end{itemize}
}
\end{enumerate}

\section*{prompt}
{\large \color{blue}  prompts  prompting  prompted  }
\subsection*{Explain}
\begin{enumerate}
\item verb \\
To \textbf{prompt} someone \textbf{to} do something means to make them decide to do it.
 \textit{
	\begin{itemize}
	\item Japan's recession has prompted consumers to cut back on buying cars.
	\item The need for villagers to control their own destinies has prompted a new plan.
	\end{itemize}
}
\item verb \\
If you \textbf{prompt} someone when they stop speaking, you encourage or help them to continue . If you \textbf{prompt} an actor, you tell them what their next line is when they have forgotten what comes next.
 \textbf{Prompt} is also a noun .
 \textit{
	\begin{itemize}
	\item 'Go on,' the therapist prompted him.
	\item How exactly did he prompt her, Mr Markham?
	\item Her blushes were saved by a prompt from one of her hosts.
	\end{itemize}
}
\item adjective \\
A \textbf{prompt} action is done without any delay.
 \textit{
	\begin{itemize}
	\item It is not too late, but prompt action is needed.
	\item ...an inflammation of the eyeball which needs prompt treatment.
	\end{itemize}
}
\item adjective \\
If you are \textbf{prompt} to do something, you do it without delay or you are not late.
 \textit{
	\begin{itemize}
	\item You have been so prompt in carrying out all these commissions.
	\item We didn't worry because they were always so prompt with their rental payment.
	\end{itemize}
}
\end{enumerate}

\section*{philosopher}
{\large \color{blue}  philosophers  }
\subsection*{Explain}
\begin{enumerate}
\item countable noun \\
A \textbf{philosopher} is a person who studies or writes about philosophy.
 \textit{
	\begin{itemize}
	\item ...the Greek philosopher Plato.
	\end{itemize}
}
\item countable noun \\
If you refer to someone as a \textbf{philosopher} , you mean that they think deeply and seriously about life and other basic matters.
 \textit{
	\begin{itemize}
	\end{itemize}
}
\end{enumerate}

\section*{random}
{\large \color{blue}  }
\subsection*{Explain}
\begin{enumerate}
\item adjective \\
A \textbf{random}  sample or method is one in which all the people or things involved have an equal chance of being chosen.
 \textit{
	\begin{itemize}
	\item The survey used a random sample of two thousand people across England and Wales.
	\item The competitors will be subject to random drug testing.
	\end{itemize}
}
\item adjective \\
If you describe events as \textbf{random} , you mean that they do not seem to follow a definite plan or pattern .
 \textit{
	\begin{itemize}
	\item ...random violence against innocent victims.
	\item Children's words and actions are often fairly random.
	\item ...random variations of the wind.
	\end{itemize}
}
\item  \\
 at random \textit{
	\begin{itemize}
	\end{itemize}
}
\item  \\
 at random \textit{
	\begin{itemize}
	\end{itemize}
}
\end{enumerate}

\section*{philosophy}
{\large \color{blue}  philosophies  }
\subsection*{Explain}
\begin{enumerate}
\item uncountable noun \\
\textbf{Philosophy} is the study or creation of theories about basic things such as the nature of existence , knowledge, and thought, or about how people should live .
 \textit{
	\begin{itemize}
	\item He studied philosophy and psychology at Cambridge.
	\item ...traditional Chinese philosophy.
	\end{itemize}
}
\item countable noun \\
A \textbf{philosophy} is a particular set of ideas that a philosopher has.
 \textit{
	\begin{itemize}
	\item ...the philosophies of Socrates, Plato, and Aristotle.
	\item ...a whole spectrum of political philosophies.
	\end{itemize}
}
\item countable noun \\
A \textbf{philosophy} is a particular theory that someone has about how to live or how to deal with a particular situation.
 \textit{
	\begin{itemize}
	\item The best philosophy is to change your food habits to a low-sugar, high-fibre diet.
	\item When I interviewed Shakira I felt in tune with her philosophy of life.
	\item Annie's work reflects her philosophy that life is full of mysteries.
	\end{itemize}
}
\end{enumerate}

\section*{restless}
{\large \color{blue}  }
\subsection*{Explain}
\begin{enumerate}
\item adjective \\
If you are \textbf{restless} , you are bored , impatient , or dissatisfied , and you want to do something else.
 \textit{
	\begin{itemize}
	\item By 1982, she was restless and needed a new impetus for her talent.
	\item ...a major new initiative to placate the country's restless intellectuals.
	\end{itemize}
}
\item adjective \\
If someone is \textbf{restless} , they keep moving around because they find it difficult to keep still.
 \textit{
	\begin{itemize}
	\item My father seemed very restless and excited.
	\end{itemize}
}
\item adjective \\
If you have a \textbf{restless}  night , you do not sleep properly and when you wake up you feel  tired and uncomfortable .
 \textit{
	\begin{itemize}
	\item The shocking revelations of the 700-page report had caused him several restless nights.
	\item Hurt had spent a restless few hours on the plane from Paris.
	\end{itemize}
}
\end{enumerate}

\section*{rake}
{\large \color{blue}  rakes  raking  raked  }
\subsection*{Explain}
\begin{enumerate}
\item countable noun \\
A \textbf{rake} is a garden tool consisting of a row of metal or wooden teeth attached to a long handle . You can use a rake to make the earth smooth and level before you put plants in,
or to gather leaves together.
 \textit{
	\begin{itemize}
	\end{itemize}
}
\item verb \\
If you \textbf{rake} a surface, you move a rake across it in order to make it smooth and level.
 \textit{
	\begin{itemize}
	\item Rake the soil, press the seed into it, then cover it lightly.
	\item The beach is raked and cleaned daily.
	\end{itemize}
}
\item verb \\
If you \textbf{rake} leaves or ashes, you move them somewhere using a rake or a similar tool.
 \textit{
	\begin{itemize}
	\item I watched the men rake leaves into heaps.
	\item She raked out the ashes from the boiler.
	\end{itemize}
}
\item verb \\
If someone \textbf{rakes} an area \textbf{with} gunfire or \textbf{with} light, they cover it thoroughly by moving the gun or the light across from one side
of the area to another.
 \textit{
	\begin{itemize}
	\item Planes dropped bombs and raked the beach with machine gun fire.
	\item The caravan was raked with bullets.
	\item The headlights raked across a painted sign.
	\end{itemize}
}
\item verb \\
If branches or someone's finger  nails  \textbf{rake} your skin, they scrape across it.
 \textit{
	\begin{itemize}
	\item Ragged fingernails raked her skin.
	\item He found the man's cheeks and raked them with his nails.
	\end{itemize}
}
\item verb \\
If you \textbf{rake through} a pile of objects or rubbish , you search through it thoroughly with your hands .
 \textit{
	\begin{itemize}
	\item Many can survive only by raking through dustbins.
	\end{itemize}
}
\item countable noun \\
If you call a man a \textbf{rake} , you mean that he is rather immoral , for example because he gambles , drinks, or has sexual relationships with many women.
 \textit{
	\begin{itemize}
	\end{itemize}
}
\end{enumerate}

\section*{rotary}
{\large \color{blue}  }
\subsection*{Explain}
\begin{enumerate}
\item adjective \\
\textbf{Rotary} means turning or able to turn round a fixed point.
 \textit{
	\begin{itemize}
	\item ...turning linear into rotary motion.
	\item ...heavy-duty rotary blades.
	\end{itemize}
}
\item adjective \\
\textbf{Rotary} is used in the names of some machines that have parts that turn round a fixed point.
 \textit{
	\begin{itemize}
	\item ...a rotary engine.
	\end{itemize}
}
\end{enumerate}

\section*{result}
{\large \color{blue}  results  resulting  resulted  }
\subsection*{Explain}
\begin{enumerate}
\item countable noun \\
A \textbf{result} is something that happens or exists because of something else that has happened.
 \textit{
	\begin{itemize}
	\item Compensation is available for people who developed asthma as a direct result of their
work.
	\item A real pizza oven gives better results than an ordinary home oven.
	\end{itemize}
}
\item verb \\
If something \textbf{results}  \textbf{in} a particular  situation or event, it causes that situation or event to happen.
 \textit{
	\begin{itemize}
	\item Fifty per cent of road accidents result in head injuries.
	\item One in five hip fractures results in death.
	\end{itemize}
}
\item verb \\
If something \textbf{results}  \textbf{from} a particular event or action, it is caused by that event or action.
 \textit{
	\begin{itemize}
	\item Many hair problems result from what you eat.
	\item Ignore the early warnings and illness could result.
	\end{itemize}
}
\item countable noun \\
A \textbf{result} is the situation that exists at the end of a contest.
 \textit{
	\begin{itemize}
	\item 'What was the result?'—'One-nil to Leeds.'.
	\item The final election results will be announced on Friday.
	\item ...the football results.
	\end{itemize}
}
\item countable noun \\
A \textbf{result} is the number that you get when you do a calculation .
 \textit{
	\begin{itemize}
	\item They found their computers producing different results from exactly the same calculation.
	\end{itemize}
}
\item countable noun \\
Your \textbf{results} are the marks or grades that you get for examinations you have taken .
 \textit{
	\begin{itemize}
	\item Kate's exam results were excellent.
	\end{itemize}
}
\end{enumerate}

\section*{rough}
{\large \color{blue}  rougher  roughest  roughs  roughing  roughed  }
\subsection*{Explain}
\begin{enumerate}
\item adjective \\
If a surface is \textbf{rough} , it is uneven and not smooth.
 \textit{
	\begin{itemize}
	\item His hands were rough and calloused, from years of karate practice.
	\item Grace made her way slowly across the rough ground.
	\end{itemize}
}
\item adjective \\
You say that people or their actions are \textbf{rough} when they use too much force and not enough care or gentleness .
 \textit{
	\begin{itemize}
	\item Rugby's a rough game at the best of times.
	\item They have complained of discrimination and occasional rough treatment.
	\end{itemize}
}
\item adjective \\
A \textbf{rough} area, city, school, or other place is unpleasant and dangerous because there is a lot of violence or crime there.
 \textit{
	\begin{itemize}
	\item It was quite a rough part of our town.
	\end{itemize}
}
\item adjective \\
If you say that someone has had a \textbf{rough} time, you mean that they have had some difficult or unpleasant experiences.
 \textit{
	\begin{itemize}
	\item All women have a rough time in our society.
	\item Tomorrow, he knew, would be a rough day.
	\end{itemize}
}
\item adjective \\
If you feel \textbf{rough} , you feel ill.
 \textit{
	\begin{itemize}
	\item The virus won't go away and the lad is still feeling a bit rough.
	\end{itemize}
}
\item adjective \\
A \textbf{rough}  calculation or guess is approximately correct, but not exact .
 \textit{
	\begin{itemize}
	\item We were only able to make a rough estimate of how much fuel would be required.
	\item As a rough guide, a horse needs 2.5 per cent of his body weight in food every day.
	\end{itemize}
}
\item adjective \\
If you give someone a \textbf{rough} idea, description , or drawing of something, you indicate only the most important features, without
much detail.
 \textit{
	\begin{itemize}
	\item I've got a rough idea of what he looks like.
	\item It often helps to make a rough sketch showing where the vehicles were.
	\end{itemize}
}
\item adjective \\
You can say that something is \textbf{rough} when it is not neat and well made.
 \textit{
	\begin{itemize}
	\item ...a rough wooden table.
	\item ...chairs set in a rough circle in the middle of the room.
	\end{itemize}
}
\item adjective \\
If the sea or the weather at sea is \textbf{rough} , the weather is windy or stormy and there are very big waves.
 \textit{
	\begin{itemize}
	\item A fishing vessel and a cargo ship collided in rough seas.
	\end{itemize}
}
\item  \\
 to sleep rough \textit{
	\begin{itemize}
	\end{itemize}
}
\item verb \\
If you have to \textbf{rough} it, you have to live without the possessions and comforts that you normally have.
 \textit{
	\begin{itemize}
	\item You won't be roughing it; each room comes equipped with a telephone and a 3-channel
radio.
	\end{itemize}
}
\end{enumerate}

\section*{ruler}
{\large \color{blue}  rulers  }
\subsection*{Explain}
\begin{enumerate}
\item countable noun \\
The \textbf{ruler} of a country is the person who rules the country.
 \textit{
	\begin{itemize}
	\item ...the former military ruler of Lesotho.
	\item He was a weak-willed and indecisive ruler.
	\end{itemize}
}
\item countable noun \\
A \textbf{ruler} is a long flat  piece of wood, metal, or plastic with straight edges marked in centimetres or inches. Rulers are used to measure things and to draw straight lines.
 \textit{
	\begin{itemize}
	\end{itemize}
}
\end{enumerate}

\section*{rude}
{\large \color{blue}  ruder  rudest  }
\subsection*{Explain}
\begin{enumerate}
\item adjective \\
When people are \textbf{rude} , they act in an impolite way towards other people or say impolite things about them.
 \textit{
	\begin{itemize}
	\item He's rude to her friends and obsessively jealous.
	\item People were quite often rude about him, often the people he had helped.
	\item Unfair bosses and rude customers make us unhappy on the job.
	\end{itemize}
}
\item adjective \\
\textbf{Rude} is used to describe words and behaviour that are likely to embarrass or offend people, because they relate to sex or to body functions.
 \textit{
	\begin{itemize}
	\item Fred keeps cracking rude jokes with the guests.
	\item Luke made a rude gesture with his finger.
	\end{itemize}
}
\item adjective \\
If someone receives a \textbf{rude}  shock , something unpleasant happens unexpectedly.
 \textit{
	\begin{itemize}
	\item It will come as a rude shock when their salary or income-tax refund cannot be cashed.
	\end{itemize}
}
\item adjective \\
Objects can be described as \textbf{rude} when they are very simply and roughly made.
 \textit{
	\begin{itemize}
	\item Roden had already constructed a rude cabin for himself and his family in case of
necessity.
	\end{itemize}
}
\item  \\
 rude health \textit{
	\begin{itemize}
	\end{itemize}
}
\end{enumerate}

\section*{salt}
{\large \color{blue}  salts  salting  salted  }
\subsection*{Explain}
\begin{enumerate}
\item uncountable noun \\
\textbf{Salt} is a strong-tasting substance, in the form of white powder or crystals , which is used to improve the flavour of food or to preserve it. Salt occurs naturally in sea water.
 \textit{
	\begin{itemize}
	\item Season lightly with salt and pepper.
	\item ...a pinch of salt.
	\end{itemize}
}
\item verb \\
When you \textbf{salt} food, you add salt to it.
 \textit{
	\begin{itemize}
	\item Salt the stock to your taste and leave it simmering very gently.
	\end{itemize}
}
\item countable noun \\
\textbf{Salts} are substances that are formed when an acid reacts with an alkali .
 \textit{
	\begin{itemize}
	\item The rock is rich in mineral salts.
	\end{itemize}
}
\item  \\
 salt of the earth \textit{
	\begin{itemize}
	\end{itemize}
}
\item  \\
 to take something with a pinch of salt \textit{
	\begin{itemize}
	\end{itemize}
}
\item  \\
 worth one's salt \textit{
	\begin{itemize}
	\end{itemize}
}
\item  \\
 to rub salt into the wound \textit{
	\begin{itemize}
	\end{itemize}
}
\end{enumerate}

\section*{second}
{\large \color{blue}  seconds  }
\subsection*{Explain}
\begin{enumerate}
\item countable noun \\
A \textbf{second} is one of the sixty parts that a minute is divided into. People often say ' \textbf{a second} ' or ' \textbf{seconds} ' when they simply mean a very short time.
 \textit{
	\begin{itemize}
	\item For a few seconds nobody said anything.
	\item It only takes forty seconds.
	\item Her orbital speed must be a few hundred meters per second.
	\item Within seconds the other soldiers began firing too.
	\item Seconds later, firemen reached his door.
	\end{itemize}
}
\end{enumerate}

\section*{saving}
{\large \color{blue}  savings  }
\subsection*{Explain}
\begin{enumerate}
\item countable noun \\
A \textbf{saving} is a reduction in the amount of time or money that is used or needed .
 \textit{
	\begin{itemize}
	\item Fill in the form below and you will be making a saving of £6.60 on a one-year subscription.
	\item ...a program of household savings on energy use.
	\end{itemize}
}
\item plural noun \\
Your \textbf{savings} are the money that you have saved, especially in a bank or a building society .
 \textit{
	\begin{itemize}
	\item Her savings were in the Post Office Savings Bank.
	\item ...a savings account.
	\end{itemize}
}
\end{enumerate}

\section*{silent}
{\large \color{blue}  }
\subsection*{Explain}
\begin{enumerate}
\item adjective \\
Someone who is \textbf{silent} is not speaking.
 \textit{
	\begin{itemize}
	\item Trish was silent because she was reluctant to put her thoughts into words.
	\item He spoke no English and was completely silent during the visit.
	\item They both fell silent.
	\end{itemize}
}
\item adjective \\
If you describe someone as a \textbf{silent} person, you mean that they do not talk to people very much, and sometimes give the impression of being unfriendly .
 \textit{
	\begin{itemize}
	\item He was a serious, silent man.
	\end{itemize}
}
\item adjective \\
A place that is \textbf{silent} is completely quiet , with no sound at all. Something that is \textbf{silent} makes no sound at all.
 \textit{
	\begin{itemize}
	\item The room was silent except for John's crunching.
	\item The heavy guns have again fallen silent.
	\end{itemize}
}
\item graded adjective \\
If someone is \textbf{silent about} something, they do not tell people anything about it, because they think it is a private  matter or because they want to keep the information  secret .
 \textit{
	\begin{itemize}
	\item Douglas was noticeably silent about his feelings for his father.
	\end{itemize}
}
\item adjective \\
A \textbf{silent}  emotion or action is not expressed in speech .
 \textit{
	\begin{itemize}
	\item The attacker still stood there, watching her with silent contempt.
	\item She offered a silent prayer of thanks.
	\end{itemize}
}
\item adjective \\
A \textbf{silent} film has pictures usually accompanied by music but does not have the actors ' voices or any other sounds.
 \textit{
	\begin{itemize}
	\item ...one of the famous silent films of Charlie Chaplin.
	\item ...comedy stars of the silent era.
	\end{itemize}
}
\item adjective \\
A \textbf{silent} letter in a word is written but not pronounced. For example , the ' k ' in the word ' know ' is silent.
 \textit{
	\begin{itemize}
	\end{itemize}
}
\end{enumerate}

\section*{setback}
{\large \color{blue}  setbacks  }
\subsection*{Explain}
\begin{enumerate}
\item countable noun \\
A \textbf{setback} is an event that delays your progress or reverses some of the progress that you have made.
 \textit{
	\begin{itemize}
	\item The move represents a setback for the peace process.
	\item He has suffered a serious setback in his political career.
	\end{itemize}
}
\end{enumerate}

\section*{sober}
{\large \color{blue}  sobers  sobering  sobered  }
\subsection*{Explain}
\begin{enumerate}
\item adjective \\
When you are \textbf{sober} , you are not drunk.
 \textit{
	\begin{itemize}
	\item When Dad was sober he was a good father.
	\end{itemize}
}
\item adjective \\
A \textbf{sober} person is serious and thoughtful .
 \textit{
	\begin{itemize}
	\item We are now far more sober and realistic.
	\item It was a room filled with sad, sober faces.
	\item The euphoria is giving way to a more sober assessment of the situation.
	\end{itemize}
}
\item adjective \\
\textbf{Sober} colours and clothes are plain and rather dull.
 \textit{
	\begin{itemize}
	\item He dresses in sober grey suits.
	\item ...sober-suited middle-aged men.
	\end{itemize}
}
\end{enumerate}

\section*{sketch}
{\large \color{blue}  sketches  sketching  sketched  }
\subsection*{Explain}
\begin{enumerate}
\item countable noun \\
A \textbf{sketch} is a drawing that is done quickly without a lot of details . Artists often use sketches as a preparation for a more detailed painting or drawing.
 \textit{
	\begin{itemize}
	\item ...a sketch of a soldier by Orpen.
	\end{itemize}
}
\item verb \\
If you \textbf{sketch} something, you make a quick , rough drawing of it.
 \textit{
	\begin{itemize}
	\item Clare and David Astor are sketching a view of far Spanish hills.
	\item I always sketch with pen and paper.
	\item ...balconies and gates sketched on holidays in Spain and Italy.
	\item Her hobbies were playing the guitar and sketching.
	\end{itemize}
}
\item countable noun \\
A \textbf{sketch}  \textbf{of} a situation , person, or incident is a brief description of it without many details.
 \textit{
	\begin{itemize}
	\item ...thumbnail sketches of heads of state and political figures.
	\item I had a basic sketch of a plan.
	\end{itemize}
}
\item verb \\
If you \textbf{sketch} a situation or incident, you give a short description of it, including only the most important  facts .
 \textbf{Sketch out}  means the same as sketch .
 \textit{
	\begin{itemize}
	\item Cross sketched the story briefly, telling the facts just as they had happened.
	\item Initially you only need to submit a proposal which briefly sketches out your ideas.
	\end{itemize}
}
\item countable noun \\
A \textbf{sketch} is a short humorous piece of acting, usually forming part of a comedy  show .
 \textit{
	\begin{itemize}
	\item ...a five-minute sketch about a folk singer.
	\end{itemize}
}
\end{enumerate}

\section*{sticky}
{\large \color{blue}  stickier  stickiest  }
\subsection*{Explain}
\begin{enumerate}
\item adjective \\
A \textbf{sticky} substance is soft , or thick and liquid , and can stick to other things. \textbf{Sticky} things are covered with a sticky substance.
 \textit{
	\begin{itemize}
	\item ...sticky toffee.
	\item If the dough is sticky, add more flour.
	\item Peel away the sticky paper.
	\end{itemize}
}
\item adjective \\
A \textbf{sticky}  situation involves problems or is embarrassing .
 \textit{
	\begin{itemize}
	\item Inevitably the transition will yield some sticky moments.
	\item Her research was going through a sticky patch.
	\end{itemize}
}
\item adjective \\
\textbf{Sticky} weather is unpleasantly hot and damp .
 \textit{
	\begin{itemize}
	\item ...four desperately hot, sticky days in the middle of August.
	\end{itemize}
}
\item  \\
 to come to a sticky end \textit{
	\begin{itemize}
	\end{itemize}
}
\end{enumerate}

\section*{structure}
{\large \color{blue}  structures  structuring  structured  }
\subsection*{Explain}
\begin{enumerate}
\item variable noun \\
The \textbf{structure}  \textbf{of} something is the way in which it is made, built , or organized .
 \textit{
	\begin{itemize}
	\item The typical family structure of Freud's patients involved two parents and two children.
	\item The chemical structure of this particular molecule is very unusual.
	\end{itemize}
}
\item countable noun \\
A \textbf{structure} is something that consists of parts connected together in an ordered way.
 \textit{
	\begin{itemize}
	\item The feet are highly specialised structures made up of 26 small delicate bones.
	\end{itemize}
}
\item countable noun \\
A \textbf{structure} is something that has been built.
 \textit{
	\begin{itemize}
	\item About half of those funds has gone to repair public roads, structures and bridges.
	\item The house was a handsome four-story brick structure.
	\end{itemize}
}
\item verb \\
If you \textbf{structure} something, you arrange it in a careful , organized pattern or system.
 \textit{
	\begin{itemize}
	\item By structuring the course this way, we produce something companies think is valuable.
	\end{itemize}
}
\end{enumerate}

\section*{subsequent}
{\large \color{blue}  }
\subsection*{Explain}
\begin{enumerate}
\item adjective \\
You use \textbf{subsequent} to describe something that happened or existed after the time or event that has just been referred to.
 \textit{
	\begin{itemize}
	\item ...the increase of population in subsequent years.
	\item Those concerns were overshadowed by subsequent events.
	\end{itemize}
}
\item  \\
 subsequent to \textit{
	\begin{itemize}
	\end{itemize}
}
\end{enumerate}

\section*{terror}
{\large \color{blue}  terrors  }
\subsection*{Explain}
\begin{enumerate}
\item uncountable noun \\
\textbf{Terror} is very great fear.
 \textit{
	\begin{itemize}
	\item I shook with terror whenever I was about to fly in an aeroplane.
	\item The day of terror ended after police used teargas and stormed the house.
	\end{itemize}
}
\item uncountable noun \\
\textbf{Terror} is violence or the threat of violence, especially when it is used for political reasons .
 \textit{
	\begin{itemize}
	\item The bomb attack on the capital could signal the start of a pre-election terror campaign.
	\end{itemize}
}
\item countable noun \\
A \textbf{terror} is something that makes you very frightened .
 \textit{
	\begin{itemize}
	\item As a boy, he had a real terror of facing people.
	\item ...the terrors of violence.
	\end{itemize}
}
\item countable noun \\
If someone describes a child as a \textbf{terror} , they think that he or she is naughty and difficult to control.
 \textit{
	\begin{itemize}
	\item He was a terror. He had been a difficult child for as long as his parents could remember.
	\end{itemize}
}
\item  \\
 hold no terrors for \textit{
	\begin{itemize}
	\end{itemize}
}
\end{enumerate}

\section*{violent}
{\large \color{blue}  }
\subsection*{Explain}
\begin{enumerate}
\item adjective \\
If someone is \textbf{violent} , or if they do something which is \textbf{violent} , they use physical force or weapons to hurt , injure, or kill other people.
 \textit{
	\begin{itemize}
	\item A quarter of current inmates have committed violent crimes.
	\item ...violent anti-government demonstrations.
	\item When I first came here, I was very violent.
	\item Sometimes the men get violent.
	\end{itemize}
}
\item adjective \\
A \textbf{violent}  event  happens  suddenly and with great force.
 \textit{
	\begin{itemize}
	\item A violent impact hurtled her forward.
	\item A violent explosion seemed to jolt the whole ground.
	\end{itemize}
}
\item adjective \\
If you describe something as \textbf{violent} , you mean that it is said , done , or felt very strongly.
 \textit{
	\begin{itemize}
	\item Violent opposition to the plan continues.
	\item He had violent stomach pains.
	\item ...an outburst of violent emotion.
	\end{itemize}
}
\item graded adjective \\
\textbf{Violent}  changes are extreme changes from one state to another.
 \textit{
	\begin{itemize}
	\item Larry began suffering severe headaches and violent mood swings.
	\end{itemize}
}
\item adjective \\
A \textbf{violent}  death is painful and unexpected , usually because the person who dies has been murdered .
 \textit{
	\begin{itemize}
	\item ...an innocent man who had met a violent death.
	\end{itemize}
}
\item adjective \\
A \textbf{violent}  film or television  programme contains a lot of scenes which show violence.
 \textit{
	\begin{itemize}
	\item It was the most violent film that I have ever seen.
	\end{itemize}
}
\item graded adjective \\
If you describe a colour as \textbf{violent} , you mean that it is extremely , and often unpleasantly, bright .
 \textit{
	\begin{itemize}
	\item ...the violent red of dying sunset.
	\end{itemize}
}
\item graded adjective \\
\textbf{Violent}  weather is extremely stormy and windy .
 \textit{
	\begin{itemize}
	\item A violent storm had struck the area.
	\end{itemize}
}
\end{enumerate}

\section*{threat}
{\large \color{blue}  threats  }
\subsection*{Explain}
\begin{enumerate}
\item variable noun \\
A \textbf{threat}  \textbf{to} a person or thing is a danger that something unpleasant  might  happen to them. A \textbf{threat} is also the cause of this danger.
 \textit{
	\begin{itemize}
	\item Some couples see single women as a threat to their relationships.
	\item The Hurricane Center warns people not to take the threat of tropical storms lightly.
	\item All countries in the region had the right to protect themselves against external
threat.
	\end{itemize}
}
\item countable noun \\
A \textbf{threat} is a statement by someone that they will do something unpleasant, especially if you do not do what they want .
 \textit{
	\begin{itemize}
	\item He may be forced to carry out his threat to resign.
	\item The writer remains in hiding after threats by former officials of the ousted dictatorship.
	\item The last journalist to interview him received a death threat.
	\end{itemize}
}
\item  \\
 under threat \textit{
	\begin{itemize}
	\end{itemize}
}
\end{enumerate}

\section*{wonderful}
{\large \color{blue}  }
\subsection*{Explain}
\begin{enumerate}
\item adjective \\
If you describe something or someone as \textbf{wonderful} , you think they are extremely good.
 \textit{
	\begin{itemize}
	\item The cold, misty air felt wonderful on his face.
	\item It's wonderful to see you.
	\item I've always thought he was a wonderful actor.
	\end{itemize}
}
\end{enumerate}

\section*{zeal}
{\large \color{blue}  }
\subsection*{Explain}
\begin{enumerate}
\item uncountable noun \\
\textbf{Zeal} is great enthusiasm , especially in connection with work, religion , or politics .
 \textit{
	\begin{itemize}
	\item ...his zeal for teaching.
	\item Mr Lopez approached his task with a religious zeal.
	\end{itemize}
}
\end{enumerate}

\section*{best}
{\large \color{blue}  }
\subsection*{Explain}
\begin{enumerate}
\item  \\
\textbf{Best} is the superlative of well2 .
 \textit{
	\begin{itemize}
	\item What's the best thing to do when I get a cold sore?
	\item It's not the best place to live if you wish to develop your knowledge and love of
mountains.
	\end{itemize}
}
\item  \\
\textbf{Best} is the superlative of good .
 \textit{
	\begin{itemize}
	\item He was best known as a writer on mystical subjects.
	\end{itemize}
}
\item singular noun \\
\textbf{The best} is used to refer to things of the highest quality or standard .
 \textit{
	\begin{itemize}
	\item We offer only the best to our clients.
	\item He'll have the best of care.
	\end{itemize}
}
\item singular noun \\
Someone's \textbf{best} is the greatest effort or highest achievement or standard that they are capable of.
 \textit{
	\begin{itemize}
	\item Miss Blockey was at her best when she played the piano.
	\item One needs to be a first-class driver to get the best out of that sort of machinery.
	\end{itemize}
}
\item singular noun \\
If you say that something is \textbf{the best} that can be done or hoped for, you think it is the most pleasant , successful , or useful thing that can be done or hoped for.
 \textit{
	\begin{itemize}
	\item A draw seems the best they can hope for.
	\item The best we can do is try to stay cool and muddle through.
	\end{itemize}
}
\item adverb \\
If you like something \textbf{best} or like it \textbf{the best} , you prefer it.
 \textit{
	\begin{itemize}
	\item The thing I liked best about the show was the music.
	\item Mother liked it best when Daniel got money.
	\item What was the role you loved the best?
	\end{itemize}
}
\item  \\
\textbf{Best} is used to form the superlative of compound  adjectives  beginning with ' good ' and ' well '. For example , the superlative of ' well-known ' is 'best-known'.
 \textit{
	\begin{itemize}
	\end{itemize}
}
\item  \\
 all the best \textit{
	\begin{itemize}
	\end{itemize}
}
\item  \\
 best of all \textit{
	\begin{itemize}
	\end{itemize}
}
\item  \\
 as best one can \textit{
	\begin{itemize}
	\end{itemize}
}
\item  \\
 at best \textit{
	\begin{itemize}
	\end{itemize}
}
\item  \\
 to do one's best \textit{
	\begin{itemize}
	\end{itemize}
}
\item  \\
 for the best \textit{
	\begin{itemize}
	\end{itemize}
}
\item  \\
 the best of friends \textit{
	\begin{itemize}
	\end{itemize}
}
\item  \\
 had best \textit{
	\begin{itemize}
	\end{itemize}
}
\item  \\
 to know best \textit{
	\begin{itemize}
	\end{itemize}
}
\item  \\
 to look one's best \textit{
	\begin{itemize}
	\end{itemize}
}
\item  \\
 to make the best of sth \textit{
	\begin{itemize}
	\end{itemize}
}
\end{enumerate}

\section*{atom}
{\large \color{blue}  atoms  }
\subsection*{Explain}
\begin{enumerate}
\item countable noun \\
An \textbf{atom} is the smallest amount of a substance that can take part in a chemical reaction.
 \textit{
	\begin{itemize}
	\item A methane molecule is composed of one carbon atom attached to four hydrogens.
	\end{itemize}
}
\end{enumerate}

\section*{black}
{\large \color{blue}  blacker  blackest  blacks  blacking  blacked  }
\subsection*{Explain}
\begin{enumerate}
\item colour \\
Something that is \textbf{black} is of the darkest colour that there is, the colour of the sky at night when there is no light at all.
 \textit{
	\begin{itemize}
	\item She was wearing a black coat with a white collar.
	\item He had thick black hair.
	\item I wear a lot of black.
	\item He was dressed all in black.
	\end{itemize}
}
\item adjective \\
A \textbf{black} person belongs to a race of people with dark skins, especially a race from Africa.
 \textit{
	\begin{itemize}
	\item He worked for the rights of black people.
	\item Sherry is black, tall, slender and soft-spoken.
	\item ...the traditions of the black community.
	\end{itemize}
}
\item countable noun \\
Black people are sometimes  referred to as \textbf{blacks} . This use could cause offence .
 \textit{
	\begin{itemize}
	\item There are about thirty-one million blacks in the U.S..
	\end{itemize}
}
\item adjective \\
\textbf{Black} coffee or tea has no milk or cream added to it.
 \textit{
	\begin{itemize}
	\item A cup of black tea or black coffee contains no calories.
	\item I drink coffee black.
	\end{itemize}
}
\item adjective \\
If you describe a situation as \textbf{black} , you are emphasizing that it is very bad indeed.
 \textit{
	\begin{itemize}
	\item It was, he said later, one of the blackest days of his political career.
	\item The future for the industry looks even blacker.
	\end{itemize}
}
\item adjective \\
If someone is in a \textbf{black}  mood , they feel very miserable and depressed .
 \textit{
	\begin{itemize}
	\item In late 1975, she fell into a black depression.
	\item Her mood was blacker than ever.
	\end{itemize}
}
\item graded adjective \\
You use \textbf{black} to describe things that you consider to be very cruel or wicked.
 \textit{
	\begin{itemize}
	\item I think their crime is a blacker one than mere exploitation.
	\item ...the blackest laws in the country's history.
	\end{itemize}
}
\item adjective \\
\textbf{Black}  humour involves jokes about sad or difficult situations.
 \textit{
	\begin{itemize}
	\item 'So you can all go over there and get shot,' he said, with the sort of black humour
common among British troops here.
	\item It's a black comedy of racial prejudice, mistaken identity and thwarted expectations.
	\end{itemize}
}
\item adjective \\
People who believe in \textbf{black}  magic believe that it is possible to communicate with evil  spirits .
 \textit{
	\begin{itemize}
	\item He was also alleged to have conducted black magic ceremonies.
	\item The King was unjustly accused of practising the black arts.
	\end{itemize}
}
\item verb \\
If someone \textbf{blacks} another person's eye, they punch or hit that person in the eye, causing it to bruise and look black.
 \textit{
	\begin{itemize}
	\item Her brother blacked her eye.
	\item He was trying to hide his two blacked eyes.
	\end{itemize}
}
\item  \\
 be black and blue \textit{
	\begin{itemize}
	\end{itemize}
}
\item  \\
 be in the black \textit{
	\begin{itemize}
	\end{itemize}
}
\item  \\
 black look \textit{
	\begin{itemize}
	\end{itemize}
}
\item  \\
 the new black \textit{
	\begin{itemize}
	\end{itemize}
}
\item  \\
 the new black \textit{
	\begin{itemize}
	\end{itemize}
}
\end{enumerate}

\section*{bang}
{\large \color{blue}  bangs  banging  banged  }
\subsection*{Explain}
\begin{enumerate}
\item countable noun \\
A \textbf{bang} is a sudden loud noise such as the noise of an explosion.
 \textit{
	\begin{itemize}
	\item I heard four or five loud bangs.
	\item She slammed the door with a bang.
	\item The television went bang.
	\end{itemize}
}
\item verb \\
If something \textbf{bangs} , it makes a sudden loud noise, once or several times.
 \textit{
	\begin{itemize}
	\item The engine spat and banged.
	\end{itemize}
}
\item verb \\
If you \textbf{bang} a door or if it \textbf{bangs} , it closes suddenly with a loud noise.
 \textit{
	\begin{itemize}
	\item ...the sound of doors banging.
	\item All up and down the street the windows bang shut.
	\item The wind banged a door somewhere.
	\end{itemize}
}
\item verb \\
If you \textbf{bang}  \textbf{on} something or if you \textbf{bang} it, you hit it hard, making a loud noise.
 \textit{
	\begin{itemize}
	\item We could bang on the desks and shout till they let us out.
	\item There is no point in shouting or banging the table.
	\end{itemize}
}
\item verb \\
If you \textbf{bang} something on something or if you \textbf{bang} it down, you quickly and violently put it on a surface , because you are angry .
 \textit{
	\begin{itemize}
	\item She banged his dinner on the table.
	\item He banged down the telephone.
	\end{itemize}
}
\item verb \\
If you \textbf{bang} a part of your body, you accidentally knock it against something and hurt yourself.
 \textbf{Bang} is also a noun .
 \textit{
	\begin{itemize}
	\item She'd fainted and banged her head.
	\item He hurried into the hall, banging his shin against a chair in the darkness.
	\item ...a nasty bang on the head.
	\end{itemize}
}
\item verb \\
If you \textbf{bang}  \textbf{into} something or someone, you bump or knock them hard, usually because you are not looking where you are going .
 \textit{
	\begin{itemize}
	\item I didn't mean to bang into you.
	\item Various men kept banging into me in the narrow corridor.
	\end{itemize}
}
\item plural noun \\
\textbf{Bangs} are hair which is cut so that it hangs over your forehead.
 \textit{
	\begin{itemize}
	\item My bangs were cut short, but the rest of my hair was long.
	\end{itemize}
}
\item adverb \\
You can use \textbf{bang} to emphasize  expressions that indicate an exact  position or an exact time.
 \textit{
	\begin{itemize}
	\item ...bang in the middle of the track.
	\item For once you leave bang on time for work.
	\end{itemize}
}
\item  \\
 bang goes sth \textit{
	\begin{itemize}
	\end{itemize}
}
\item  \\
 with a bang \textit{
	\begin{itemize}
	\end{itemize}
}
\end{enumerate}

\section*{colonial}
{\large \color{blue}  colonials  }
\subsection*{Explain}
\begin{enumerate}
\item adjective \\
\textbf{Colonial} means relating to countries that are colonies, or to colonialism .
 \textit{
	\begin{itemize}
	\item ...the 31st anniversary of Jamaica's independence from British colonial rule.
	\item ...the colonial civil service.
	\end{itemize}
}
\item countable noun \\
People who have lived for a long time in a colony but who belong to the colonizing country are sometimes  referred to as \textbf{colonials} .
 \textit{
	\begin{itemize}
	\item ...a group of ex-colonials.
	\end{itemize}
}
\item adjective \\
A \textbf{Colonial} building or piece of furniture was built or made in a style that was popular in America in the 17th and 18th centuries.
 \textit{
	\begin{itemize}
	\item ...the white colonial houses on the north side of the campus.
	\item I sat on the Colonial bench that was just to the left of the office doorway.
	\end{itemize}
}
\end{enumerate}

\section*{bridge}
{\large \color{blue}  bridges  bridging  bridged  }
\subsection*{Explain}
\begin{enumerate}
\item countable noun \\
A \textbf{bridge} is a structure that is built over a railway, river, or road so that people or vehicles
can cross from one side to the other.
 \textit{
	\begin{itemize}
	\item He walked back over the railway bridge.
	\item ...the Golden Gate Bridge.
	\end{itemize}
}
\item countable noun \\
A \textbf{bridge} between two places is a piece of land that joins or connects them.
 \textit{
	\begin{itemize}
	\item ...a land bridge linking Serbian territories.
	\end{itemize}
}
\item verb \\
To \textbf{bridge} the gap between two people or things means to reduce it or get  rid of it.
 \textit{
	\begin{itemize}
	\item It is unlikely that the two sides will be able to bridge their differences.
	\end{itemize}
}
\item verb \\
Something that \textbf{bridges} the gap between two very different things has some of the qualities of each of these
things.
 \textit{
	\begin{itemize}
	\item ...the singer who bridged the gap between pop music and opera.
	\end{itemize}
}
\item countable noun \\
If something or someone acts as a \textbf{bridge} between two people, groups, or things, they connect them.
 \textit{
	\begin{itemize}
	\item We hope this book will act as a bridge between doctor and patient.
	\item They saw themselves as a bridge to peace.
	\end{itemize}
}
\item countable noun \\
\textbf{The}  \textbf{bridge} is the place on a ship from which it is steered .
 \textit{
	\begin{itemize}
	\end{itemize}
}
\item countable noun \\
The \textbf{bridge} of your nose is the thin top part of it, between your eyes.
 \textit{
	\begin{itemize}
	\item On the bridge of his hooked nose was a pair of gold rimless spectacles.
	\end{itemize}
}
\item countable noun \\
The \textbf{bridge} of a pair of glasses is the part that rests on your nose.
 \textit{
	\begin{itemize}
	\end{itemize}
}
\item countable noun \\
The \textbf{bridge} of a violin, guitar, or other stringed instrument is the small piece of wood under the strings that holds them up.
 \textit{
	\begin{itemize}
	\end{itemize}
}
\item countable noun \\
A \textbf{bridge} is a piece of metal or plastic that holds false teeth in place by connecting them to natural teeth.
 \textit{
	\begin{itemize}
	\end{itemize}
}
\item uncountable noun \\
\textbf{Bridge} is a card game for four players in which the players begin by declaring how many tricks they expect to win.
 \textit{
	\begin{itemize}
	\end{itemize}
}
\item  \\
 burn one's bridges \textit{
	\begin{itemize}
	\end{itemize}
}
\end{enumerate}

\section*{comic}
{\large \color{blue}  comics  }
\subsection*{Explain}
\begin{enumerate}
\item adjective \\
If you describe something as \textbf{comic} , you mean that it makes you laugh , and is often intended to make you laugh.
 \textit{
	\begin{itemize}
	\item The novel is comic and tragic.
	\item Most of these trips had exciting or comic moments.
	\end{itemize}
}
\item adjective \\
\textbf{Comic} is used to describe comedy as a form of entertainment , and the actors and entertainers who perform it.
 \textit{
	\begin{itemize}
	\item Grodin is a fine comic actor.
	\item ...a comic opera.
	\end{itemize}
}
\item countable noun \\
A \textbf{comic} is an entertainer who tells  jokes in order to make people laugh.
 \textit{
	\begin{itemize}
	\end{itemize}
}
\item countable noun \\
A \textbf{comic} is a magazine that contains stories told in pictures .
 \textit{
	\begin{itemize}
	\item Joe loved to read 'Superman' comics.
	\end{itemize}
}
\item plural noun \\
\textbf{The}  \textbf{comics} is the part of a newspaper that contains the comic strips.
 \textit{
	\begin{itemize}
	\end{itemize}
}
\end{enumerate}

\section*{cause}
{\large \color{blue}  causes  causing  caused  }
\subsection*{Explain}
\begin{enumerate}
\item countable noun \\
The \textbf{cause}  \textbf{of} an event, usually a bad event, is the thing that makes it happen .
 \textit{
	\begin{itemize}
	\item Smoking is the biggest preventable cause of death and disease.
	\item The causes are a complex blend of local and national tensions.
	\end{itemize}
}
\item verb \\
To \textbf{cause} something, usually something bad, means to make it happen.
 \textit{
	\begin{itemize}
	\item Attempts to limit family size among some minorities are likely to cause problems.
	\item This was a genuine mistake, but it did cause me some worry.
	\item ...a protein that gets into animal cells and attacks other proteins, causing disease
to spread.
	\item Experts are assessing the damage caused by a fire at the aircraft factory.
	\end{itemize}
}
\item uncountable noun \\
If you have \textbf{cause}  \textbf{for} a particular feeling or action, you have good reasons for feeling it or doing it.
 \textit{
	\begin{itemize}
	\item Only a few people can find any cause for celebration.
	\item Both had much cause to be grateful.
	\end{itemize}
}
\item countable noun \\
A \textbf{cause} is an aim or principle which a group of people supports or is fighting for.
 \textit{
	\begin{itemize}
	\item Refusing to have one leader has not helped the cause.
	\end{itemize}
}
\item  \\
 cause and effect \textit{
	\begin{itemize}
	\end{itemize}
}
\item  \\
 to make common cause with someone \textit{
	\begin{itemize}
	\end{itemize}
}
\item  \\
 in a good cause/for a good cause \textit{
	\begin{itemize}
	\end{itemize}
}
\end{enumerate}

\section*{dark}
{\large \color{blue}  darker  darkest  }
\subsection*{Explain}
\begin{enumerate}
\item adjective \\
When it is \textbf{dark} , there is not enough light to see properly, for example because it is night.
 \textit{
	\begin{itemize}
	\item When she awoke it was evening and already dark.
	\item It was too dark inside to see much.
	\item People usually draw the curtains once it gets dark.
	\item She snapped off the light and made her way back through the dark kitchen.
	\end{itemize}
}
\item singular noun \\
\textbf{The dark} is the lack of light in a place.
 \textit{
	\begin{itemize}
	\item Her mother was sitting in the dark by the stove in her rocking chair.
	\item I've always been afraid of the dark.
	\end{itemize}
}
\item adjective \\
If you describe something as \textbf{dark} , you mean that it is black in colour, or a shade that is close to black.
 \textit{
	\begin{itemize}
	\item He wore a dark suit and carried a black attaché case.
	\item The heavy dark table is inlaid with lighter wood.
	\end{itemize}
}
\item adjective \\
When you use \textbf{dark} to describe a colour, you are referring to a shade of that colour which is close to black, or seems to have some black in it.
 \textit{
	\begin{itemize}
	\item She was wearing a dark blue dress.
	\end{itemize}
}
\item adjective \\
If someone has \textbf{dark} hair, eyes , or skin, they have brown or black hair, eyes, or skin.
 \textit{
	\begin{itemize}
	\item He had dark, curly hair.
	\item Leo went on, his dark eyes wide with pity and concern.
	\end{itemize}
}
\item adjective \\
If you describe a white person as \textbf{dark} , you mean that they have brown or black hair, and often a brownish skin.
 \textit{
	\begin{itemize}
	\item He's gorgeous – tall and dark.
	\end{itemize}
}
\item adjective \\
A \textbf{dark} period of time is unpleasant or frightening .
 \textit{
	\begin{itemize}
	\item Once again there's talk of very dark days ahead.
	\item This was the darkest period of the war.
	\end{itemize}
}
\item adjective \\
A \textbf{dark} place or area is mysterious and not fully known about.
 \textit{
	\begin{itemize}
	\item The spacecraft is set to throw new light on to a dark corner of the solar system.
	\item ...the dark recesses of the mind.
	\end{itemize}
}
\item adjective \\
\textbf{Dark}  thoughts are sad , and show that you are expecting something unpleasant to happen .
 \textit{
	\begin{itemize}
	\item Troy's chatter kept me from thinking dark thoughts.
	\end{itemize}
}
\item adjective \\
\textbf{Dark}  looks or remarks make you think that the person giving them wants to harm you or that something horrible is going to happen.
 \textit{
	\begin{itemize}
	\item Garin shot him a dark glance, as if in warning.
	\item ...dark threats.
	\end{itemize}
}
\item adjective \\
If you describe something as \textbf{dark} , you mean that it is related to things that are serious or unpleasant, rather than light-hearted .
 \textit{
	\begin{itemize}
	\item Their dark humor never failed to astound him.
	\item Nina took a dark pleasure in being the cause of tension.
	\end{itemize}
}
\item  \\
 after dark \textit{
	\begin{itemize}
	\end{itemize}
}
\item  \\
 before dark \textit{
	\begin{itemize}
	\end{itemize}
}
\item  \\
 in the dark \textit{
	\begin{itemize}
	\end{itemize}
}
\item  \\
 a shot in the dark \textit{
	\begin{itemize}
	\end{itemize}
}
\end{enumerate}

\section*{circle}
{\large \color{blue}  circles  circling  circled  }
\subsection*{Explain}
\begin{enumerate}
\item countable noun \\
A \textbf{circle} is a shape consisting of a curved line completely surrounding an area. Every part
of the line is the same distance from the centre of the area.
 \textit{
	\begin{itemize}
	\item The flag was red, with a large white circle in the centre.
	\item I wrote down the number 46 and drew a circle around it.
	\end{itemize}
}
\item countable noun \\
A \textbf{circle}  \textbf{of} something is a round flat piece or area of it.
 \textit{
	\begin{itemize}
	\item Cut out 4 circles of pastry.
	\item ...a circle of yellow light.
	\end{itemize}
}
\item countable noun \\
A \textbf{circle}  \textbf{of} objects or people is a group of them arranged in the shape of a circle.
 \textit{
	\begin{itemize}
	\item The monument consists of a circle of gigantic stones.
	\item We stood in a circle holding hands.
	\end{itemize}
}
\item verb \\
If something \textbf{circles} an object or a place, or \textbf{circles}  \textbf{around} it, it forms a circle around it.
 \textit{
	\begin{itemize}
	\item This is the ring road that circles the city.
	\item ...the long curving driveway that circled around the vast clipped lawn.
	\end{itemize}
}
\item verb \\
If an aircraft or a bird \textbf{circles} or \textbf{circles} something, it moves round in a circle in the air.
 \textit{
	\begin{itemize}
	\item The plane circled, awaiting permission to land.
	\item There were two helicopters circling around.
	\item ...like a hawk circling prey.
	\end{itemize}
}
\item verb \\
To \textbf{circle}  \textbf{around} someone or something, or to \textbf{circle} them, means to move around them.
 \textit{
	\begin{itemize}
	\item Emily kept circling around her mother.
	\item The silent wolves would track and circle them.
	\end{itemize}
}
\item verb \\
If you \textbf{circle} something on a piece of paper, you draw a circle around it.
 \textit{
	\begin{itemize}
	\item Circle the correct answers on the coupon below.
	\end{itemize}
}
\item countable noun \\
You can refer to a group of people as a \textbf{circle} when they meet each other regularly because they are friends or because they belong to the same profession or share the same interests.
 \textit{
	\begin{itemize}
	\item He has a small circle of friends.
	\item Alton has made himself fiercely unpopular in certain circles.
	\end{itemize}
}
\item singular noun \\
In a theatre or cinema , \textbf{the circle} is an area of seats on the upper floor .
 \textit{
	\begin{itemize}
	\end{itemize}
}
\item  \\
 to come full circle \textit{
	\begin{itemize}
	\end{itemize}
}
\item  \\
 go round in circles/go around in circles \textit{
	\begin{itemize}
	\end{itemize}
}
\end{enumerate}

\section*{exact}
{\large \color{blue}  exacts  exacting  exacted  }
\subsection*{Explain}
\begin{enumerate}
\item adjective \\
\textbf{Exact} means correct in every detail. For example , an \textbf{exact}  copy is the same in every detail as the thing it is copied from.
 \textit{
	\begin{itemize}
	\item I don't remember the exact words.
	\item The exact number of protest calls has not been revealed.
	\item It's an exact copy of the one which was found in Ann Alice's room.
	\end{itemize}
}
\item adjective \\
You use \textbf{exact} before a noun to emphasize that you are referring to that particular thing and no other, especially something that has a particular significance .
 \textit{
	\begin{itemize}
	\item I hadn't really thought about it until this exact moment.
	\item Do you really think I could get the exact thing I want?
	\item It may be that you will feel the exact opposite of what you expected.
	\end{itemize}
}
\item adjective \\
If you describe someone as \textbf{exact} , you mean that they are very careful and detailed in their work, thinking , or methods.
 \textit{
	\begin{itemize}
	\item Formal, exact and obstinate, he was also cold, suspicious, touchy and tactless.
	\end{itemize}
}
\item verb \\
When someone \textbf{exacts} something, they demand and obtain it from another person, especially because they
are in a superior or more powerful position.
 \textit{
	\begin{itemize}
	\item Already he has exacted a written apology from the chairman of the commission.
	\item They, too, would be likely to exact a high price for their cooperation.
	\end{itemize}
}
\item verb \\
If someone \textbf{exacts}  revenge  \textbf{on} a person, they have their revenge on them.
 \textit{
	\begin{itemize}
	\item She uses the media to help her exact a terrible revenge.
	\end{itemize}
}
\item verb \\
If something \textbf{exacts} a high price, it has a bad effect on a person or situation.
 \textit{
	\begin{itemize}
	\item The sheer physical effort had exacted a heavy price.
	\item The strain of a violent ground campaign will exact a toll on troops.
	\end{itemize}
}
\item  \\
 to be exact \textit{
	\begin{itemize}
	\end{itemize}
}
\end{enumerate}

\section*{federal}
{\large \color{blue}  federals  }
\subsection*{Explain}
\begin{enumerate}
\item adjective \\
A \textbf{federal} country or system of government is one in which the different states or provinces of the country have important powers to make their own laws and decisions .
 \textit{
	\begin{itemize}
	\item The provinces are to become autonomous regions in the new federal system.
	\end{itemize}
}
\item adjective \\
Some people use \textbf{federal} to describe a system of government which they disapprove of, in which the different states or provinces are controlled by a strong central government.
 \textit{
	\begin{itemize}
	\item He does not believe in a federal Europe with centralising powers.
	\end{itemize}
}
\item adjective \\
\textbf{Federal}  also  means belonging or relating to the national government of a federal country rather than to one of the states within it.
 \textit{
	\begin{itemize}
	\item The federal government controls just 6% of the education budget.
	\item ...a federal judge.
	\end{itemize}
}
\item countable noun \\
\textbf{Federals} are the same as feds .
 \textit{
	\begin{itemize}
	\end{itemize}
}
\end{enumerate}

\section*{duck}
{\large \color{blue}  ducks  ducking  ducked  }
\subsection*{Explain}
\begin{enumerate}
\item variable noun \\
A \textbf{duck} is a very common water bird with short legs, a short neck , and a large flat  beak .
 \textbf{Duck} is the flesh of this bird when it is eaten as food.
 \textit{
	\begin{itemize}
	\item Chickens and ducks scratch around the outbuildings.
	\item ...honey roasted duck.
	\end{itemize}
}
\item countable noun \\
A \textbf{duck} is a female duck. The male is called a drake.
 \textit{
	\begin{itemize}
	\item I brought in one drake and three ducks.
	\end{itemize}
}
\item verb \\
If you \textbf{duck} , you move your head or the top  half of your body quickly downwards to avoid something that might  hit you, or to avoid being seen .
 \textit{
	\begin{itemize}
	\item He ducked in time to save his head from a blow from the poker.
	\item He ducked his head to hide his admiration.
	\item I wanted to duck down and slip past but they saw me.
	\end{itemize}
}
\item verb \\
If you \textbf{duck} something such as a blow, you avoid it by moving your head or body quickly downwards.
 \textit{
	\begin{itemize}
	\item Hans deftly ducked their blows.
	\end{itemize}
}
\item verb \\
If you \textbf{duck} into a place, you move there quickly, often in an attempt to avoid danger or to avoid being seen.
 \textit{
	\begin{itemize}
	\item Matt ducked into his office.
	\item He ducked through the door and looked about frantically.
	\end{itemize}
}
\item verb \\
You say that someone \textbf{ducks} a duty or responsibility when you disapprove of the fact that they avoid it.
 \textit{
	\begin{itemize}
	\item The Opposition reckons the Health Secretary has ducked all the difficult decisions.
	\item He had ducked the confrontation with United Nations inspectors last summer.
	\end{itemize}
}
\item verb \\
If someone \textbf{ducks} someone else, they force them or their head under water for a short time.
 \textit{
	\begin{itemize}
	\item She splashed around in the pool with Mark, rowdily trying to duck him.
	\end{itemize}
}
\item vocative noun \\
Some people call other people \textbf{duck} or \textbf{ducks} as a sign of affection .
 \textit{
	\begin{itemize}
	\item Oh, I am glad to see you, duck.
	\end{itemize}
}
\item  \\
 like water off a duck's back \textit{
	\begin{itemize}
	\end{itemize}
}
\item  \\
 to take to something like a duck to water \textit{
	\begin{itemize}
	\end{itemize}
}
\end{enumerate}

\section*{final}
{\large \color{blue}  finals  }
\subsection*{Explain}
\begin{enumerate}
\item adjective \\
In a series of events , things, or people, the \textbf{final} one is the last one.
 \textit{
	\begin{itemize}
	\item They will hold a meeting in a final attempt to agree a common position.
	\item This is the fifth and probably final day of testimony before the Senate Judiciary
Committee.
	\item On the last Saturday in September, I received a final letter from Clive.
	\end{itemize}
}
\item adjective \\
\textbf{Final} means happening at the end of an event or series of events.
 \textit{
	\begin{itemize}
	\item You must have been on stage until the final curtain.
	\item The countdown to the Notting Hill Carnival is in its final hours.
	\end{itemize}
}
\item adjective \\
You can use \textbf{final} to emphasize that a situation has a particular quality to a very great or severe  degree .
 \textit{
	\begin{itemize}
	\item Only a few go through the final humiliation of meeting the bailiff at the door.
	\end{itemize}
}
\item adjective \\
If a decision or someone's authority is \textbf{final} , it cannot be changed or questioned .
 \textit{
	\begin{itemize}
	\item The judges' decision is final.
	\item The White House has the final say.
	\item I'm not going, and that's final.
	\end{itemize}
}
\item countable noun \\
The \textbf{final} is the last game or contest in a series and decides who is the winner.
 \textit{
	\begin{itemize}
	\item ...the Scottish Cup Final.
	\item He won the men's final at the Singapore Open.
	\end{itemize}
}
\item plural noun \\
\textbf{The}  \textbf{finals} of a sporting  tournament consist of a smaller tournament that includes only players or teams that have won  earlier games. The finals decide the winner of the whole tournament.
 \textit{
	\begin{itemize}
	\item They have a chance of qualifying for the World Cup Finals.
	\end{itemize}
}
\item plural noun \\
When a student takes his or her \textbf{finals} , he or she takes the last and most important  examinations in a university or college  course .
 \textit{
	\begin{itemize}
	\item Anna sat her finals in the summer.
	\end{itemize}
}
\end{enumerate}

\section*{emergency}
{\large \color{blue}  emergencies  }
\subsection*{Explain}
\begin{enumerate}
\item countable noun \\
An \textbf{emergency} is an unexpected and difficult or dangerous  situation , especially an accident , which happens  suddenly and which requires quick action to deal with it.
 \textit{
	\begin{itemize}
	\item He deals with emergencies promptly.
	\item The hospital will cater only for emergencies.
	\end{itemize}
}
\item adjective \\
An \textbf{emergency} action is one that is done or arranged quickly and not in the normal way, because an emergency has occurred.
 \textit{
	\begin{itemize}
	\item The Prime Minister has called an emergency meeting of parliament.
	\item She made an emergency appointment.
	\end{itemize}
}
\item adjective \\
\textbf{Emergency}  equipment or supplies are those intended for use in an emergency.
 \textit{
	\begin{itemize}
	\item The plane is carrying emergency supplies for refugees.
	\item They escaped through an emergency exit and called the police.
	\end{itemize}
}
\end{enumerate}

\section*{fine}
{\large \color{blue}  finer  finest  }
\subsection*{Explain}
\begin{enumerate}
\item adjective \\
You use \textbf{fine} to describe something that you admire and think is very good.
 \textit{
	\begin{itemize}
	\item There is a fine view of the countryside.
	\item This is a fine book.
	\item ...London's finest art deco cinema.
	\end{itemize}
}
\item adjective \\
If you say that you are \textbf{fine} , you mean that you are in good health or reasonably happy .
 \textit{
	\begin{itemize}
	\item Lina is fine and sends you her love and best wishes.
	\end{itemize}
}
\item adjective \\
If you say that something is \textbf{fine} , you mean that it is satisfactory or acceptable.
 \textbf{Fine} is also an adverb .
 \textit{
	\begin{itemize}
	\item The skiing is fine.
	\item Everything was going to be just fine.
	\item It's fine to ask questions as we go along, but it's better if you wait until we have
finished.
	\item All the instruments are working fine.
	\end{itemize}
}
\item convention \\
You say ' \textbf{fine} ' or ' \textbf{that's fine} ' to show that you do not object to an arrangement, action, or situation that has
been suggested .
 \textit{
	\begin{itemize}
	\item If competition is the best way to achieve it, then, fine.
	\item If you don't want to give it to me, that's fine, I don't mind.
	\item 'It'll take me a couple of days.'—'That's fine with me.'
	\end{itemize}
}
\item adjective \\
Something that is \textbf{fine} is very delicate, narrow, or small.
 \textit{
	\begin{itemize}
	\item The heat scorched the fine hairs on her arms.
	\item The ship has come to rest on the fine sand.
	\end{itemize}
}
\item adjective \\
\textbf{Fine} objects or clothing are of good quality, delicate, and expensive .
 \textit{
	\begin{itemize}
	\item We waited in our fine clothes.
	\item She'll wear fine jewellery wherever she goes.
	\end{itemize}
}
\item adjective \\
A \textbf{fine} detail or distinction is very delicate, small, or exact.
 \textit{
	\begin{itemize}
	\item The market likes the broad outline but is reserving judgment on the fine detail.
	\end{itemize}
}
\item adjective \\
A \textbf{fine} person is someone you consider good, moral, and worth admiring.
 \textit{
	\begin{itemize}
	\item I was with fine people doing a good job.
	\item He was an excellent journalist and a very fine man.
	\end{itemize}
}
\item adjective \\
When the weather is \textbf{fine} , the sun is shining and it is not raining .
 \textit{
	\begin{itemize}
	\item He might be doing a spot of gardening if the weather is fine.
	\end{itemize}
}
\end{enumerate}

\section*{frame}
{\large \color{blue}  frames  framing  framed  }
\subsection*{Explain}
\begin{enumerate}
\item countable noun \\
The \textbf{frame} of a picture or mirror is the wood, metal, or plastic that is fitted around it, especially when it is displayed or hung on a wall.
 \textit{
	\begin{itemize}
	\item Estelle kept a photograph of her mother in a silver frame on the kitchen mantelpiece.
	\item ...a pair of picture frames.
	\end{itemize}
}
\item countable noun \\
The \textbf{frame} of an object such as a building, chair , or window is the arrangement of wooden, metal, or plastic bars between which other
material is fitted, and which give the object its strength and shape.
 \textit{
	\begin{itemize}
	\item He supplied housebuilders with modern timber frames.
	\item With difficulty he released the mattress from the metal frame, and groped beneath
it.
	\item We painted our table to match the window frame in the bedroom.
	\end{itemize}
}
\item countable noun \\
The \textbf{frames} of a pair of glasses are all the metal or plastic parts of it, but not the lenses .
 \textit{
	\begin{itemize}
	\item He was wearing new spectacles with gold wire frames.
	\end{itemize}
}
\item countable noun \\
You can refer to someone's body as their \textbf{frame} , especially when you are describing the general shape of their body.
 \textit{
	\begin{itemize}
	\item Their belts are pulled tight against their bony frames.
	\end{itemize}
}
\item countable noun \\
A \textbf{frame} of cinema film is one of the many separate photographs that it consists of.
 \textit{
	\begin{itemize}
	\item Standard 8mm projects at 16 frames per second.
	\end{itemize}
}
\item adjective \\
A \textbf{frame} building is one in which pieces of wood form the most important part of the structure,
rather than bricks or stone.
 \textit{
	\begin{itemize}
	\item He lives in a white-painted frame house behind a picket fence up in Connecticut.
	\end{itemize}
}
\item verb \\
When a picture or photograph \textbf{is framed} , it is put in a frame.
 \textit{
	\begin{itemize}
	\item The picture is now ready to be mounted and framed.
	\item On the wall is a large framed photograph.
	\end{itemize}
}
\item verb \\
If an object \textbf{is framed} by a particular thing, it is surrounded by that thing in a way that makes the object
more striking or attractive to look at.
 \textit{
	\begin{itemize}
	\item The swimming pool is framed by tropical gardens.
	\item An elegant occasional table is framed in the window.
	\end{itemize}
}
\item verb \\
If someone \textbf{frames} something such as a set of rules, a plan, or a system, they create and develop it.
 \textit{
	\begin{itemize}
	\item After the war, a convention was set up to frame a constitution.
	\end{itemize}
}
\item verb \\
If someone \textbf{frames} something in a particular style or kind of language, they express it in that way.
 \textit{
	\begin{itemize}
	\item The story is framed in a format that is part thriller, part love story.
	\item He framed this question three different ways in search of an answer.
	\end{itemize}
}
\item verb \\
If someone \textbf{frames} an innocent person, they make other people think that that person is guilty of a crime, by lying or inventing  evidence .
 \textit{
	\begin{itemize}
	\item I need to find out who tried to frame me.
	\item He claimed that he had been framed by the police.
	\end{itemize}
}
\item  \\
 in the frame \textit{
	\begin{itemize}
	\end{itemize}
}
\end{enumerate}

\section*{foremost}
{\large \color{blue}  }
\subsection*{Explain}
\begin{enumerate}
\item adjective \\
The \textbf{foremost} thing or person in a group is the most important or best .
 \textit{
	\begin{itemize}
	\item He was one of the world's foremost scholars of ancient Indian culture.
	\item Foremost among the military government's enemies are the foreign media.
	\end{itemize}
}
\item  \\
 first and foremost \textit{
	\begin{itemize}
	\end{itemize}
}
\end{enumerate}

\section*{headmaster}
{\large \color{blue}  headmasters  }
\subsection*{Explain}
\begin{enumerate}
\item countable noun \\
A \textbf{headmaster} is a man who is the head teacher of a school.
 \textit{
	\begin{itemize}
	\end{itemize}
}
\end{enumerate}

\section*{genetic}
{\large \color{blue}  }
\subsection*{Explain}
\begin{enumerate}
\item adjective \\
You use \textbf{genetic} to describe something that is concerned with genetics or with genes.
 \textit{
	\begin{itemize}
	\item Cystic fibrosis is the most common fatal genetic disease in the United States.
	\end{itemize}
}
\end{enumerate}

\section*{initial}
{\large \color{blue}  initials  initialling  initialled  }
\subsection*{Explain}
\begin{enumerate}
\item adjective \\
You use \textbf{initial} to describe something that happens at the beginning of a process.
 \textit{
	\begin{itemize}
	\item The initial reaction has been excellent.
	\item The aim of this initial meeting is to clarify the issues.
	\end{itemize}
}
\item countable noun \\
\textbf{Initials} are the capital letters which begin each word of a name. For example , if your full name is Michael  Dennis  Stocks , your initials will be M . D .S.
 \textit{
	\begin{itemize}
	\item ...a silver Porsche car with her initials JB on the side.
	\end{itemize}
}
\item verb \\
If someone \textbf{initials} an official  document , they write their initials on it, for example to show that they have seen it or that they accept or agree with it.
 \textit{
	\begin{itemize}
	\item Would you mind initialing this voucher?
	\item The agreement was initialled in June.
	\end{itemize}
}
\end{enumerate}

\section*{mammal}
{\large \color{blue}  mammals  }
\subsection*{Explain}
\begin{enumerate}
\item countable noun \\
\textbf{Mammals} are animals such as humans, dogs , lions , and whales. In general, female mammals give birth to babies rather than laying  eggs , and feed their young with milk .
 \textit{
	\begin{itemize}
	\end{itemize}
}
\end{enumerate}

\section*{keen}
{\large \color{blue}  keener  keenest  keens  keening  keened  }
\subsection*{Explain}
\begin{enumerate}
\item adjective \\
If you are \textbf{keen}  \textbf{on} doing something, you very much want to do it. If you are \textbf{keen}  \textbf{that} something should happen , you very much want it to happen.
 \textit{
	\begin{itemize}
	\item You're not keen on going, are you?
	\item Both companies were keen on a merger.
	\item I'm very keen that the European Union should be as open as possible to trade from
Russia.
	\item She's still keen to keep in touch.
	\item I am not keen for her to have a bicycle.
	\end{itemize}
}
\item adjective \\
If you are \textbf{keen on} something, you like it a lot and are very enthusiastic about it.
 \textit{
	\begin{itemize}
	\item I got quite keen on the idea.
	\item I wasn't too keen on physics and chemistry.
	\end{itemize}
}
\item adjective \\
You use \textbf{keen} to indicate that someone has a lot of enthusiasm for a particular activity and spends a lot of time doing it.
 \textit{
	\begin{itemize}
	\item She was a keen amateur photographer.
	\end{itemize}
}
\item adjective \\
If you describe someone as \textbf{keen} , you mean that they have an enthusiastic nature and are interested in everything that they do.
 \textit{
	\begin{itemize}
	\item He's a very keen student and works very hard.
	\item You're all very keen.
	\end{itemize}
}
\item adjective \\
A \textbf{keen} interest or emotion is one that is very intense.
 \textit{
	\begin{itemize}
	\item He had retained a keen interest in the progress of the work.
	\item ...his keen sense of loyalty.
	\end{itemize}
}
\item adjective \\
If you are a \textbf{keen}  supporter of a cause, movement, or idea , you support it enthusiastically.
 \textit{
	\begin{itemize}
	\item He's been a keen supporter of the Labour Party all his life.
	\item He is a keen advocate of park-and-ride schemes.
	\end{itemize}
}
\item adjective \\
If you say that someone has a \textbf{keen}  mind , you mean that they are very clever and aware of what is happening around them.
 \textit{
	\begin{itemize}
	\item They described him as a man of keen intellect.
	\item Mr Walsh has a keen appreciation of the priorities of the electorate.
	\item I can see you have a keen sense of humour.
	\end{itemize}
}
\item adjective \\
If you have a \textbf{keen}  eye or ear , you are able to notice things that are difficult to detect .
 \textit{
	\begin{itemize}
	\item ...an amateur artist with a keen eye for detail.
	\item Brand's keen ear caught the trace of an accent.
	\end{itemize}
}
\item graded adjective \\
If you are \textbf{keen on} someone, you find them sexually attractive and want to get to know them better .
 \textit{
	\begin{itemize}
	\item Mick has always been very keen on Carla.
	\end{itemize}
}
\item adjective \\
A \textbf{keen}  fight or competition is one in which the competitors are all trying very hard to win , and it is not easy to predict who will win.
 \textit{
	\begin{itemize}
	\item There is expected to be a keen fight in the local elections.
	\end{itemize}
}
\item adjective \\
\textbf{Keen}  prices are low and competitive.
 \textit{
	\begin{itemize}
	\item The company negotiates very keen prices with their suppliers.
	\end{itemize}
}
\item verb \\
If someone \textbf{keens} , they cry out or make sounds to express their sorrow at someone's death .
 \textit{
	\begin{itemize}
	\item He tossed back his head and keened.
	\item Someone was making a low, keening noise.
	\end{itemize}
}
\item  \\
 mad keen \textit{
	\begin{itemize}
	\end{itemize}
}
\end{enumerate}

\section*{model}
{\large \color{blue}  models  modelling  modelled  }
\subsection*{Explain}
\begin{enumerate}
\item countable noun \\
A \textbf{model} of an object is a physical representation that shows what it looks like or how it works. The model is often smaller than the object it represents.
 \textbf{Model} is also an adjective .
 \textit{
	\begin{itemize}
	\item ...an architect's model of a wooden house.
	\item ...a working scale model of the whole Bay Area.
	\item I made a model out of paper and glue.
	\item I had made a model aeroplane.
	\item ...a model railway.
	\end{itemize}
}
\item countable noun \\
A \textbf{model} is a system that is being used and that people might  want to copy in order to achieve similar results.
 \textit{
	\begin{itemize}
	\item We believe that this is a general model of managerial activity.
	\item He wants companies to follow the European model of social responsibility.
	\end{itemize}
}
\item countable noun \\
A \textbf{model} of a system or process is a theoretical description that can help you understand how the system or process works, or how it might work.
 \textit{
	\begin{itemize}
	\item Darwin eventually put forward a model of biological evolution.
	\item He proposed a model of stress reaction in the body.
	\end{itemize}
}
\item verb \\
If someone such as a scientist  \textbf{models} a system or process, they make an accurate theoretical description of it in order to understand or explain how it works.
 \textit{
	\begin{itemize}
	\item ...the mathematics needed to model a nonlinear system like an atmosphere.
	\end{itemize}
}
\item countable noun \\
If you say that someone or something is \textbf{a}  \textbf{model}  \textbf{of} a particular quality, you are showing approval of them because they have that quality to a large degree.
 \textit{
	\begin{itemize}
	\item A model of good manners, he has conquered any inward fury.
	\item His marriage and family life is a model of propriety.
	\end{itemize}
}
\item adjective \\
You use \textbf{model} to express approval of someone when you think that they perform their role or duties extremely well .
 \textit{
	\begin{itemize}
	\item As a girl she had been a model pupil.
	\item Hospital staff say he is a model patient.
	\end{itemize}
}
\item verb \\
If one thing \textbf{is modelled}  \textbf{on} another, the first thing is made so that it is like the second thing in some way.
 \textit{
	\begin{itemize}
	\item The quota system was modelled on those operated in America and continental Europe.
	\item The program will be modeled after a popular BBC series called 'The Archers'.
	\item She asked the author if she had modelled her hero on anybody in particular.
	\end{itemize}
}
\item verb \\
If you \textbf{model} yourself \textbf{on} someone, you copy the way that they do things, because you admire them and want to be like them.
 \textit{
	\begin{itemize}
	\item There's absolutely nothing wrong in modelling yourself on an older woman.
	\item They will tend to model their behaviour on the teacher's behaviour.
	\end{itemize}
}
\item countable noun \\
A particular \textbf{model} of a machine is a particular version of it.
 \textit{
	\begin{itemize}
	\item To keep the cost down, opt for a basic model.
	\item The model number is 1870/285.
	\end{itemize}
}
\item countable noun \\
An artist's \textbf{model} is a person who stays  still in a particular position so that the artist can make a picture or sculpture of them.
 \textit{
	\begin{itemize}
	\end{itemize}
}
\item verb \\
If someone \textbf{models} for an artist, they stay still in a particular position so that the artist can make
a picture or sculpture of them.
 \textit{
	\begin{itemize}
	\item Tullio has been modelling for Sandra for eleven years.
	\end{itemize}
}
\item countable noun \\
A fashion  \textbf{model} is a person whose job is to display clothes by wearing them.
 \textit{
	\begin{itemize}
	\item ...Paris's top photographic fashion model.
	\end{itemize}
}
\item verb \\
If someone \textbf{models} clothes, they display them by wearing them.
 \textit{
	\begin{itemize}
	\item I wasn't here to model clothes.
	\item She began modelling in Paris aged 15.
	\end{itemize}
}
\item verb \\
If you \textbf{model} shapes or figures , you make them out of a substance such as clay or wood.
 \textit{
	\begin{itemize}
	\item There she began to model in clay.
	\item Sometimes she carved wood and sometimes stone; sometimes she modelled clay.
	\item The artist modelled an appropriate animal for each voice.
	\end{itemize}
}
\end{enumerate}

\section*{lame}
{\large \color{blue}  lamer  lamest  }
\subsection*{Explain}
\begin{enumerate}
\item adjective \\
If someone is \textbf{lame} , they are unable to walk properly because of damage to one or both of their legs.
 \textbf{The lame} are people who are lame. This use could cause offence .
 \textit{
	\begin{itemize}
	\item He was aware that she was lame in one leg.
	\item David had to pull out of the Championships when his horse went lame.
	\item ... the wounded and the lame of the last war.
	\end{itemize}
}
\item adjective \\
If you describe something, for example an excuse , argument , or remark , as \textbf{lame} , you mean that it is poor or weak.
 \textit{
	\begin{itemize}
	\item He mumbled some lame excuse about having gone to sleep.
	\item All our theories sound pretty lame.
	\end{itemize}
}
\end{enumerate}

\section*{mosquito}
{\large \color{blue}  mosquitoes  mosquitos  }
\subsection*{Explain}
\begin{enumerate}
\item countable noun \\
\textbf{Mosquitos} are small flying insects which bite people and animals in order to suck their blood.
 \textit{
	\begin{itemize}
	\end{itemize}
}
\end{enumerate}

\section*{last}
{\large \color{blue}  lasts  lasting  lasted  }
\subsection*{Explain}
\begin{enumerate}
\item determiner \\
You use \textbf{last} in expressions such as \textbf{last Friday,}  \textbf{last night} , and \textbf{last year} to refer , for example , to the most recent Friday, night, or year.
 \textit{
	\begin{itemize}
	\item I got married last July.
	\item He never made it home at all last night.
	\item It is not surprising they did so badly in last year's elections.
	\end{itemize}
}
\item adjective \\
The \textbf{last} event, person, thing, or period of time is the most recent one.
 \textbf{Last} is also a pronoun .
 \textit{
	\begin{itemize}
	\item Much has changed since my last visit.
	\item At the last count inflation was 10.9 per cent.
	\item I split up with my last boyfriend three years ago.
	\item The last few weeks have been hectic.
	\item The next tide, it was announced, would be even higher than the last.
	\end{itemize}
}
\item adverb \\
If something \textbf{last}  happened on a particular occasion , that is the most recent occasion on which it happened.
 \textit{
	\begin{itemize}
	\item When were you there last?
	\item The house is a little more dilapidated than when I last saw it.
	\item Hunting on the trust's 625,000 acres was last debated two years ago.
	\end{itemize}
}
\item ordinal number \\
The \textbf{last} thing, person, event, or period of time is the one that happens or comes after all the others of the same kind .
 \textbf{Last} is also a pronoun.
 \textit{
	\begin{itemize}
	\item This is his last chance as prime minister.
	\item ...the last three pages of the chapter.
	\item She said it was the very last house on the road.
	\item They didn't come last in their league.
	\item I'm not the first employee she has done this to and I probably won't be the last.

	\item The trickiest bits are the last on the list.
	\end{itemize}
}
\item adverb \\
If you do something \textbf{last} , you do it after everyone else does, or after you do everything else.
 \textit{
	\begin{itemize}
	\item I testified last.
	\item I was always picked last for the football team at school.
	\item The foreground, nearest the viewer, is painted last.
	\end{itemize}
}
\item pronoun \\
If you are \textbf{the}  \textbf{last}  \textbf{to} do or know something, everyone else does or knows it before you.
 \textit{
	\begin{itemize}
	\item She was the last to go to bed.
	\item Riccardo and I are always the last to know what's going on.
	\end{itemize}
}
\item adjective \\
\textbf{Last} is used to refer to the only thing, person, or part of something that remains.
 \textbf{Last} is also a noun .
 \textit{
	\begin{itemize}
	\item Jed nodded, finishing off the last piece of pizza.
	\item ...the freeing of the last hostage.
	\item He finished off the last of the coffee.
	\item The last of the ten inmates gave themselves up after twenty eight hours.
	\end{itemize}
}
\item adjective \\
You use \textbf{last} before numbers to refer to a position that someone has reached in a competition after other competitors have been knocked out.
 \textit{
	\begin{itemize}
	\item He reached the last four at Wimbledon.
	\item ...the only woman among the authors making it through to the last six.
	\end{itemize}
}
\item adjective \\
You can use \textbf{last} to indicate that something is extremely  undesirable or unlikely .
 \textbf{Last} is also a pronoun.
 \textit{
	\begin{itemize}
	\item The last thing I wanted to do was teach.
	\item He would be the last person who would do such a thing.
	\item I would be the last to say that science has explained everything.
	\end{itemize}
}
\item pronoun \\
\textbf{The last} you see of someone or \textbf{the last} you hear of them is the final time that you see them or talk to them.
 \textit{
	\begin{itemize}
	\item She disappeared shouting, 'To the river, to the river!' And that was the last we
saw of her.
	\item I had a feeling it would be the last I heard of him.
	\end{itemize}
}
\item verb \\
If an event, situation, or problem  \textbf{lasts} for a particular length of time, it continues to exist or happen for that length
of time.
 \textit{
	\begin{itemize}
	\item The marriage had lasted for less than two years.
	\item The games lasted only half the normal time.
	\item Enjoy it because it won't last.
	\end{itemize}
}
\item verb \\
If something \textbf{lasts} for a particular length of time, it continues to be able to be used for that time,
for example because there is some of it left or because it is in good enough condition.
 \textit{
	\begin{itemize}
	\item You only need a very small blob of glue, so one tube lasts for ages.
	\item The repaired sail lasted less than 24 hours.
	\item The implication is that this battery lasts twice as long as other batteries.
	\item If you build more plastics into cars, the car lasts longer.
	\end{itemize}
}
\item verb \\
You can use \textbf{last} in expressions such as \textbf{last the game} , \textbf{last the course} , and \textbf{last the week} , to indicate that someone manages to take part in an event or situation right to the end, especially when this is very difficult for them.
 To \textbf{last out} means the same as to last .
 \textit{
	\begin{itemize}
	\item They wouldn't have lasted the full game.
	\item I almost lasted the two weeks. I only had a couple of days to do.
	\item It'll be a miracle if the band lasts out the tour.
	\item A breakfast will be served to those who last out till dawn!
	\end{itemize}
}
\item  \\
 at last \textit{
	\begin{itemize}
	\end{itemize}
}
\item  \\
 the sb/sth before last \textit{
	\begin{itemize}
	\end{itemize}
}
\item  \\
 breathe one's last \textit{
	\begin{itemize}
	\end{itemize}
}
\item  \\
 last but one/last but three etc \textit{
	\begin{itemize}
	\end{itemize}
}
\item  \\
 every last \textit{
	\begin{itemize}
	\end{itemize}
}
\item  \\
 last in, first out \textit{
	\begin{itemize}
	\end{itemize}
}
\item  \\
 the last sb heard \textit{
	\begin{itemize}
	\end{itemize}
}
\item  \\
 leave sth until last \textit{
	\begin{itemize}
	\end{itemize}
}
\item  \\
 see the last of sb \textit{
	\begin{itemize}
	\end{itemize}
}
\item  \\
 to the last \textit{
	\begin{itemize}
	\end{itemize}
}
\item  \\
 to the last \textit{
	\begin{itemize}
	\end{itemize}
}
\item  \\
 to the last detail/to the last man \textit{
	\begin{itemize}
	\end{itemize}
}
\end{enumerate}

\section*{ounce}
{\large \color{blue}  ounces  }
\subsection*{Explain}
\begin{enumerate}
\item countable noun \\
An \textbf{ounce} is a unit of weight used in Britain and the USA. There are sixteen ounces in a pound and one ounce is equal to 28.35 grams.
 \textit{
	\begin{itemize}
	\item ...four ounces of sugar.
	\end{itemize}
}
\item singular noun \\
You can  refer to a very small amount of something, such as a quality or characteristic , as an \textbf{ounce} .
 \textit{
	\begin{itemize}
	\item If only my father had possessed an ounce of business sense.
	\item I spent every ounce of energy trying to hide.
	\end{itemize}
}
\end{enumerate}

\section*{learned}
{\large \color{blue}  }
\subsection*{Explain}
\begin{enumerate}
\item adjective \\
A \textbf{learned} person has gained a lot of knowledge by studying.
 \textit{
	\begin{itemize}
	\item He is a serious scholar, a genuinely learned man.
	\end{itemize}
}
\item adjective \\
\textbf{Learned} books or papers have been written by someone who has gained a lot of knowledge by studying.
 \textit{
	\begin{itemize}
	\item This learned book should start a real debate on Western policy towards the Baltics.
	\end{itemize}
}
\item adjective \\
A \textbf{learned}  reaction , response , or ability is one that you acquire from experience or from your environment , not one that you were born with.
 \textit{
	\begin{itemize}
	\item Your anxiety is a learned reaction, conditioned by the events of your life.
	\end{itemize}
}
\end{enumerate}

\section*{pardon}
{\large \color{blue}  pardons  pardoning  pardoned  }
\subsection*{Explain}
\begin{enumerate}
\item  \\
 I beg your pardon \textit{
	\begin{itemize}
	\end{itemize}
}
\item convention \\
People say ' \textbf{I beg your pardon?} ' when they are surprised or offended by something that someone has just said .
 \textit{
	\begin{itemize}
	\item 'Would you get undressed, please?'—'I beg your pardon?'—'Will you get undressed?'
	\end{itemize}
}
\item convention \\
You say ' \textbf{I beg your pardon} ' or ' \textbf{I do beg your pardon} ' as a way of apologizing for accidentally doing something wrong , such as disturbing someone or making a mistake.
 \textit{
	\begin{itemize}
	\item I was impolite and I do beg your pardon.
	\item 'We're meant to do it quarterly actually.'—'Oh, I beg your pardon, I thought it was
monthly.'
	\end{itemize}
}
\item convention \\
Some people say ' \textbf{Pardon me} ' instead of 'Excuse me' when they want to politely get someone's attention or interrupt them.
 \textit{
	\begin{itemize}
	\item Pardon me, are you finished, madam?
	\end{itemize}
}
\item convention \\
You can say things like ' \textbf{Pardon me for asking} ' or ' \textbf{Pardon my frankness} ' as a way of showing you understand that what you are going to say may sound rude .
 \textit{
	\begin{itemize}
	\item That, if you'll pardon my saying so, is neither here nor there.
	\end{itemize}
}
\item convention \\
Some people say things like ' \textbf{If you'll pardon the expression} ' or ' \textbf{Pardon my French} ' just before or after saying something which they think  might offend people.
 \textit{
	\begin{itemize}
	\item It's enough to make you wet yourself, if you'll pardon the expression.
	\end{itemize}
}
\item verb \\
If someone who has been found  guilty of a crime  \textbf{is pardoned} , they are officially  allowed to go  free and are not punished .
 \textbf{Pardon} is also a noun .
 \textit{
	\begin{itemize}
	\item Hundreds of political prisoners were pardoned and released.
	\item He was granted a presidential pardon.
	\end{itemize}
}
\end{enumerate}

\section*{least}
{\large \color{blue}  }
\subsection*{Explain}
\begin{enumerate}
\item  \\
 at least \textit{
	\begin{itemize}
	\end{itemize}
}
\item  \\
 at least \textit{
	\begin{itemize}
	\end{itemize}
}
\item  \\
 at least \textit{
	\begin{itemize}
	\end{itemize}
}
\item  \\
 at least \textit{
	\begin{itemize}
	\end{itemize}
}
\item adjective \\
You use \textbf{the least} to mean a smaller amount than anyone or anything else, or the smallest amount possible .
 \textbf{Least} is also a pronoun .
 \textbf{Least} is also an adverb .
 \textit{
	\begin{itemize}
	\item I try to offend the least amount of people possible.
	\item If you like cheese, go for the ones with the least fat.
	\item On education funding, Japan performs best but spends the least per student.
	\item Damming the river may end up benefitting those who need it the least.
	\end{itemize}
}
\item adverb \\
You use \textbf{least} to indicate that someone or something has less of a particular quality than most other things
of its kind .
 \textit{
	\begin{itemize}
	\item He is the youngest and least experienced player in the team.
	\item He was one of the least warm human beings I had ever met.
	\item ...the least technically accomplished car in Europe.
	\end{itemize}
}
\item adjective \\
You use \textbf{the least} to emphasize the smallness of something, especially when it hardly  exists at all.
 \textit{
	\begin{itemize}
	\item I don't have the least idea of what you're talking about.
	\item They neglect their duty at the least hint of fun elsewhere.
	\item The bosses paid less than they had promised and at the least complaint went to the
police.
	\end{itemize}
}
\item adverb \\
You use \textbf{least} to indicate that something is true or happens to a smaller degree or extent than anything else or at any other time.
 \textit{
	\begin{itemize}
	\item He had a way of throwing her off guard with his charm when she least expected it.
	\end{itemize}
}
\item adjective \\
You use \textbf{least} in structures where you are emphasizing that a particular situation or event is much less important or serious than other possible or actual ones.
 \textit{
	\begin{itemize}
	\item Having to get up at three o'clock every morning was the least of her worries.
	\item Although three days isn't very long, shortage of time was the least of his problems.
	\item At that moment, they were among the least of the concerns of the government.
	\end{itemize}
}
\item pronoun \\
You use \textbf{the least} in structures where you are stating the minimum that should be done in a situation, and suggesting that more should really be done.
 \textit{
	\begin{itemize}
	\item Well, the least you can do, if you won't help me yourself, is to tell me where to
go instead.
	\item The least they could have given me was half a day to rest.
	\item The least his hotel could do is provide a little privacy.
	\end{itemize}
}
\item  \\
 in the least \textit{
	\begin{itemize}
	\end{itemize}
}
\item  \\
 last but not least \textit{
	\begin{itemize}
	\end{itemize}
}
\item  \\
 least of all \textit{
	\begin{itemize}
	\end{itemize}
}
\item  \\
 not least \textit{
	\begin{itemize}
	\end{itemize}
}
\item  \\
 to say the least \textit{
	\begin{itemize}
	\end{itemize}
}
\end{enumerate}

\section*{peach}
{\large \color{blue}  peaches  }
\subsection*{Explain}
\begin{enumerate}
\item countable noun \\
A \textbf{peach} is a soft, round, slightly  furry fruit with sweet yellow flesh and pinky-orange skin. Peaches grow in warm countries.
 \textit{
	\begin{itemize}
	\end{itemize}
}
\item colour \\
Something that is \textbf{peach} is pale pinky-orange in colour.
 \textit{
	\begin{itemize}
	\item ...the romantic Tower Suite, decorated throughout in peach and ivory.
	\item ...a peach silk blouse.
	\end{itemize}
}
\item singular noun \\
If you describe someone or something as a \textbf{peach} , you find them very pleasing or attractive .
 \textit{
	\begin{itemize}
	\item Frank was there and he is a perfect peach.
	\item ...a peach of a goal from the team's captain.
	\end{itemize}
}
\end{enumerate}

\section*{principle}
{\large \color{blue}  principles  }
\subsection*{Explain}
\begin{enumerate}
\item variable noun \\
A \textbf{principle} is a general belief that you have about the way you should behave , which influences your behaviour.
 \textit{
	\begin{itemize}
	\item Buck never allowed himself to be bullied into doing anything that went against his
principles.
	\item ...moral principles.
	\item It's not just a matter of principle.
	\item ...a man of principle.
	\end{itemize}
}
\item countable noun \\
The \textbf{principles}  \textbf{of} a particular theory or philosophy are its basic rules or laws.
 \textit{
	\begin{itemize}
	\item ...a violation of the basic principles of Marxism.
	\item The doctrine was based on three fundamental principles.
	\end{itemize}
}
\item countable noun \\
Scientific  \textbf{principles} are general scientific laws which explain how something happens or works.
 \textit{
	\begin{itemize}
	\item These people lack all understanding of scientific principles.
	\item ...the principles of quantum theory.
	\end{itemize}
}
\item  \\
 in principle \textit{
	\begin{itemize}
	\end{itemize}
}
\item  \\
 in principle \textit{
	\begin{itemize}
	\end{itemize}
}
\item  \\
 on principle \textit{
	\begin{itemize}
	\end{itemize}
}
\end{enumerate}

\section*{optimum}
{\large \color{blue}  }
\subsection*{Explain}
\begin{enumerate}
\item adjective \\
The \textbf{optimum} or \textbf{optimal} level or state of something is the best level or state that it could achieve .
 \textit{
	\begin{itemize}
	\item Aim to do some physical activity three times a week for optimum health.
	\item ...regions in which optimal conditions for farming can be created.
	\end{itemize}
}
\end{enumerate}

\section*{prison}
{\large \color{blue}  prisons  }
\subsection*{Explain}
\begin{enumerate}
\item variable noun \\
A \textbf{prison} is a building where criminals are kept as punishment or where people accused of a crime are kept before their trial.
 \textit{
	\begin{itemize}
	\item The prison's inmates are being kept in their cells.
	\item He was sentenced to life in prison.
	\item They released Mr Mandela from prison in 1990.
	\end{itemize}
}
\end{enumerate}

\section*{original}
{\large \color{blue}  originals  }
\subsection*{Explain}
\begin{enumerate}
\item adjective \\
You use \textbf{original} when referring to something that existed at the beginning of a process or activity, or the characteristics that something
had when it began or was made.
 \textit{
	\begin{itemize}
	\item The original plan was to hold an indefinite stoppage.
	\item He was unable to identify the original name of the site.
	\end{itemize}
}
\item countable noun \\
If something such as a document , a work of art, or a piece of writing is an \textbf{original} , it is not a copy or a later  version .
 \textit{
	\begin{itemize}
	\item When you have filled in the questionnaire, copy it and send the original to your
employer.
	\item For once the sequel is as good as the original.
	\end{itemize}
}
\item adjective \\
An \textbf{original} document or work of art is not a copy.
 \textit{
	\begin{itemize}
	\item ...an original movie poster.
	\end{itemize}
}
\item adjective \\
An \textbf{original} piece of writing or music was written recently and has not been published or performed before.
 \textit{
	\begin{itemize}
	\item ...its policy of commissioning original work.
	\item ...with catchy original songs by Richard Warner.
	\end{itemize}
}
\item adjective \\
If you describe someone or their work as \textbf{original} , you mean that they are very imaginative and have new ideas.
 \textit{
	\begin{itemize}
	\item It is one of the most original works of imagination in the language.
	\item ...an original writer.
	\item ...a chef with an original touch and a measure of inspiration.
	\end{itemize}
}
\item  \\
 in the original/in the original French/etc \textit{
	\begin{itemize}
	\end{itemize}
}
\end{enumerate}

\section*{prototype}
{\large \color{blue}  prototypes  }
\subsection*{Explain}
\begin{enumerate}
\item countable noun \\
A \textbf{prototype} is a new type of machine or device which is not yet ready to be made in large numbers and sold .
 \textit{
	\begin{itemize}
	\item He built a prototype of a machine called the wave rotor.
	\item ...the first prototype aircraft.
	\end{itemize}
}
\item countable noun \\
If you say that someone or something is a \textbf{prototype of} a type of person or thing, you mean that they are the first or most typical one of that type.
 \textit{
	\begin{itemize}
	\item He was the prototype of the elder statesman.
	\end{itemize}
}
\end{enumerate}

\section*{overtime}
{\large \color{blue}  }
\subsection*{Explain}
\begin{enumerate}
\item uncountable noun \\
\textbf{Overtime} is time that you spend doing your job in addition to your normal working hours.
 \textit{
	\begin{itemize}
	\item He would work overtime, without pay, to finish a job.
	\item Union leaders had urged miners to vote in favour of an overtime ban.
	\end{itemize}
}
\item  \\
 work overtime \textit{
	\begin{itemize}
	\end{itemize}
}
\item uncountable noun \\
\textbf{Overtime} is an additional period of time that is added to the end of a sports  match in which the two teams are level , as a way of allowing one of the teams to win .
 \textit{
	\begin{itemize}
	\item They won the championship by defeating their opponents 3–2 in overtime.
	\end{itemize}
}
\end{enumerate}

\section*{rate}
{\large \color{blue}  rates  rating  rated  }
\subsection*{Explain}
\begin{enumerate}
\item countable noun \\
The \textbf{rate} at which something happens is the speed with which it happens.
 \textit{
	\begin{itemize}
	\item The rate at which hair grows can be agonisingly slow.
	\item The world's tropical forests are disappearing at an even faster rate than experts
had thought.
	\end{itemize}
}
\item countable noun \\
The \textbf{rate} at which something happens is the number of times it happens over a period of time or in a particular group.
 \textit{
	\begin{itemize}
	\item New diet books appear at a rate of nearly one a week.
	\item His heart rate was 30 beats per minute slower.
	\item The country has the world's sixth highest unemployment rate.
	\end{itemize}
}
\item countable noun \\
A \textbf{rate} is the amount of money that is charged for goods or services.
 \textit{
	\begin{itemize}
	\item Calls cost 36p per minute cheap rate and 48p at all other times.
	\item ...specially reduced rates for travellers using Gatwick Airport.
	\item After the age of 35, we start losing muscle at the rate of half a pound a year.
	\end{itemize}
}
\item countable noun \\
The \textbf{rate} of taxation or interest is the amount of tax or interest that needs to be paid . It is expressed as a percentage of the amount that is earned , gained as profit , or borrowed .
 \textit{
	\begin{itemize}
	\item The government insisted that it would not be panicked into interest rate cuts.
	\end{itemize}
}
\item verb \\
If you \textbf{rate} someone or something as good or bad , you consider them to be good or bad. You can  also  say that someone or something \textbf{rates} as good or bad.
 \textit{
	\begin{itemize}
	\item Of all the men in the survey, they rate themselves the most responsible.
	\item The film was rated excellent by 90 per cent of children.
	\item Most rated it a hit.
	\item We rate him as one of the best.
	\item She rated the course highly.
	\item Reading books does not rate highly among Britons as a leisure activity.
	\item ...the most highly rated player in English football.
	\end{itemize}
}
\item verb \\
If you \textbf{rate} someone or something, you think that they are good.
 \textit{
	\begin{itemize}
	\item It's flattering to know that other clubs have shown interest and seem to rate me.
	\item Its artistic value failed to move Paddy Clegg. 'I don't know what all the fuss is
about. I didn't rate it at all,' he said.
	\end{itemize}
}
\item passive verb \\
If someone or something \textbf{is rated} at a particular position or rank, they are calculated or considered to be in that
position on a list .
 \textit{
	\begin{itemize}
	\item He is generally rated Italy's No. 3 industrialist.
	\item He came here rated 100th on the tennis computer.
	\end{itemize}
}
\item verb \\
If you say that someone or something \textbf{rates} a particular reaction , you mean that this is the reaction you consider to be appropriate .
 \textit{
	\begin{itemize}
	\item Their national golf championship barely rated a mention.
	\item In those crowded streets her attire did not rate a second glance.
	\end{itemize}
}
\item  \\
 at any rate \textit{
	\begin{itemize}
	\end{itemize}
}
\item  \\
 at any rate \textit{
	\begin{itemize}
	\end{itemize}
}
\item  \\
 at this rate \textit{
	\begin{itemize}
	\end{itemize}
}
\end{enumerate}

\section*{pessimistic}
{\large \color{blue}  }
\subsection*{Explain}
\begin{enumerate}
\item adjective \\
Someone who is \textbf{pessimistic}  thinks that bad things are going to happen .
 \textit{
	\begin{itemize}
	\item Not everyone is so pessimistic about the future.
	\item Hardy has often been criticised for an excessively pessimistic view of life.
	\item ...one of the most pessimistic forecasts of the year.
	\end{itemize}
}
\end{enumerate}

\section*{reason}
{\large \color{blue}  reasons  reasoning  reasoned  }
\subsection*{Explain}
\begin{enumerate}
\item countable noun \\
The \textbf{reason}  \textbf{for} something is a fact or situation which explains why it happens or what causes it to happen.
 \textit{
	\begin{itemize}
	\item There is a reason for every important thing that happens.
	\item Who would have a reason to want to kill her?
	\item ...the reason why Italian tomatoes have so much flavour.
	\item The only reason I went was because I was told to.
	\item My parents came to Germany for business reasons.
	\item The exact locations are being kept secret for reasons of security.
	\end{itemize}
}
\item uncountable noun \\
If you say that you have \textbf{reason}  \textbf{to}  believe something or \textbf{to} have a particular  emotion , you mean that you have evidence for your belief or there is a definite cause of your feeling .
 \textit{
	\begin{itemize}
	\item They had reason to believe there could be trouble.
	\item He had every reason to be upset.
	\item He doesn't trust me. With good reason.
	\end{itemize}
}
\item uncountable noun \\
The ability that people have to think and to make sensible judgments can be referred to as \textbf{reason} .
 \textit{
	\begin{itemize}
	\item ...a conflict between emotion and reason.
	\item Mike is my voice of reason. He thinks logically and points out where I'm going wrong.
	\end{itemize}
}
\item verb \\
If you \textbf{reason}  \textbf{that} something is true , you decide that it is true after thinking carefully about all the facts.
 \textit{
	\begin{itemize}
	\item I reasoned that changing my diet would lower my cholesterol level.
	\item 'Listen,' I reasoned, 'it doesn't take a genius to figure out what Adam's up to.'
	\end{itemize}
}
\item  \\
 for reasons best known to oneself \textit{
	\begin{itemize}
	\end{itemize}
}
\item  \\
 by reason of \textit{
	\begin{itemize}
	\end{itemize}
}
\item  \\
 listen to reason \textit{
	\begin{itemize}
	\end{itemize}
}
\item  \\
 for no reason/ for no reason at all \textit{
	\begin{itemize}
	\end{itemize}
}
\item  \\
 reason for living/reason for being \textit{
	\begin{itemize}
	\end{itemize}
}
\item  \\
 for some reason \textit{
	\begin{itemize}
	\end{itemize}
}
\item  \\
 within reason \textit{
	\begin{itemize}
	\end{itemize}
}
\end{enumerate}

\section*{primary}
{\large \color{blue}  primaries  }
\subsection*{Explain}
\begin{enumerate}
\item adjective \\
You use \textbf{primary} to describe something that is very important.
 \textit{
	\begin{itemize}
	\item That's the primary reason the company's share price has held up so well.
	\item His misunderstanding of language was the primary cause of his other problems.
	\item The family continues to be the primary source of care for people as they grow older.
	\end{itemize}
}
\item adjective \\
\textbf{Primary} education is given to pupils between the ages of 5 and 11.
 \textit{
	\begin{itemize}
	\item Britain did not introduce compulsory primary education until 1880.
	\item Ninety-nine per cent of primary pupils now have hands-on experience of computers.
	\item ...primary teachers.
	\end{itemize}
}
\item adjective \\
\textbf{Primary} is used to describe something that occurs first.
 \textit{
	\begin{itemize}
	\item It is not the primary tumour that kills, but secondary growths elsewhere in the body.
	\item They have been barred from primary bidding for clients.
	\end{itemize}
}
\item countable noun \\
A \textbf{primary} or a \textbf{primary election} is an election in an American state in which people vote for someone to become a candidate for a political office. Compare  general election .
 \textit{
	\begin{itemize}
	\item ...the 1968 New Hampshire primary.
	\item She won the Democratic primary.
	\item New York holds its primary election on Tuesday.
	\end{itemize}
}
\end{enumerate}

\section*{root}
{\large \color{blue}  roots  rooting  rooted  }
\subsection*{Explain}
\begin{enumerate}
\item countable noun \\
The \textbf{roots} of a plant are the parts of it that grow under the ground.
 \textit{
	\begin{itemize}
	\item ...the twisted roots of an apple tree.
	\end{itemize}
}
\item verb \\
If you \textbf{root} a plant or cutting or if it \textbf{roots} , roots form on the bottom of its stem and it starts to grow.
 \textit{
	\begin{itemize}
	\item Most plants will root in about six to eight weeks.
	\item Root the cuttings in a heated propagator.
	\end{itemize}
}
\item adjective \\
\textbf{Root}  vegetables or \textbf{root}  crops are grown for their roots which are large and can be eaten .
 \textit{
	\begin{itemize}
	\item ...root crops such as carrots and potatoes.
	\end{itemize}
}
\item countable noun \\
The \textbf{root} of a hair or tooth is the part of it that is underneath the skin .
 \textit{
	\begin{itemize}
	\item ...decay around the roots of teeth.
	\item ...wax strips which remove hairs cleanly from the root.
	\end{itemize}
}
\item plural noun \\
You can refer to the place or culture that a person or their family comes from as their \textbf{roots} .
 \textit{
	\begin{itemize}
	\item I am proud of my Brazilian roots.
	\item It's 21 years since she first moved to Britain from the Lebanon, but she hasn't forgotten
her roots.
	\end{itemize}
}
\item uncountable noun \\
\textbf{Roots} is used to refer to pop music , especially  reggae , that is strongly influenced by the traditional music of the culture that it originally  came from.
 \textit{
	\begin{itemize}
	\item ...mixing older Jamaican styles such as bluebeat and ska with roots reggae and dub.
	\end{itemize}
}
\item countable noun \\
You can refer to the cause of a problem or of an unpleasant  situation as \textbf{the}  \textbf{root}  \textbf{of} it or \textbf{the}  \textbf{roots}  \textbf{of} it.
 \textit{
	\begin{itemize}
	\item We got to the root of the problem.
	\item This lack of recognition was at the root of the dispute.
	\item His sense of guilt had its roots in his childhood loss of his younger sister.
	\item They were treating symptoms and not the root cause.
	\end{itemize}
}
\item countable noun \\
The \textbf{root} of a word is the part that contains its meaning and to which other parts can be added .
 \textit{
	\begin{itemize}
	\item The word 'secretary' comes from the same Latin root as the word 'secret'.
	\end{itemize}
}
\item verb \\
If you \textbf{root}  \textbf{through} or \textbf{in} something, you search for something by moving other things around.
 \textit{
	\begin{itemize}
	\item She rooted through the bag, found what she wanted, and headed toward the door.
	\item Dogs root in the debris at the roadside.
	\end{itemize}
}
\item  \\
 root and branch \textit{
	\begin{itemize}
	\end{itemize}
}
\item  \\
 put down roots \textit{
	\begin{itemize}
	\end{itemize}
}
\item  \\
 take root \textit{
	\begin{itemize}
	\end{itemize}
}
\end{enumerate}

\section*{principal}
{\large \color{blue}  principals  }
\subsection*{Explain}
\begin{enumerate}
\item adjective \\
\textbf{Principal} means first in order of importance.
 \textit{
	\begin{itemize}
	\item The principal reason for my change of mind is this.
	\item ...the country's principal source of foreign exchange earnings.
	\item Their principal concern is bound to be that of winning the next general election.
	\end{itemize}
}
\item countable noun \\
The \textbf{principal} of a school, or in Britain the \textbf{principal} of a college , is the person in charge of the school or college.
 \textit{
	\begin{itemize}
	\item Donald King is the principal of Dartmouth High School.
	\end{itemize}
}
\end{enumerate}

\section*{sample}
{\large \color{blue}  samples  sampling  sampled  }
\subsection*{Explain}
\begin{enumerate}
\item countable noun \\
A \textbf{sample} of a substance or product is a small quantity of it that shows you what it is like .
 \textit{
	\begin{itemize}
	\item You'll receive samples of paint, curtains and upholstery.
	\item We're giving away 2000 free samples.
	\item They asked me to do some sample drawings.
	\end{itemize}
}
\item countable noun \\
A \textbf{sample} of a substance is a small amount of it that is examined and analysed scientifically.
 \textit{
	\begin{itemize}
	\item They took samples of my blood.
	\item ...urine samples.
	\end{itemize}
}
\item countable noun \\
A \textbf{sample} of people or things is a number of them chosen out of a larger group and then used in tests or used to provide  information about the whole group.
 \textit{
	\begin{itemize}
	\item We based our analysis on a random sample of more than 200 males.
	\end{itemize}
}
\item verb \\
If you \textbf{sample}  food or drink , you taste a small amount of it in order to find out if you like it.
 \textit{
	\begin{itemize}
	\item We sampled a selection of different bottled waters.
	\end{itemize}
}
\item verb \\
If you \textbf{sample} a place or situation , you experience it for a short time in order to find out about it.
 \textit{
	\begin{itemize}
	\item ...the chance to sample a different way of life.
	\end{itemize}
}
\item verb \\
When musicians or pieces of their music  \textbf{are sampled} , parts of their music are used by other musicians in their own work.
 \textit{
	\begin{itemize}
	\item I don't actually mind being sampled as long as people give credit where it's due.
	\end{itemize}
}
\end{enumerate}

\section*{purple}
{\large \color{blue}  purples  }
\subsection*{Explain}
\begin{enumerate}
\item colour \\
Something that is \textbf{purple} is of a reddish-blue colour.
 \textit{
	\begin{itemize}
	\item She wore purple and green silk.
	\item ...sinister dark greens and purples.
	\end{itemize}
}
\item adjective \\
\textbf{Purple prose} or a \textbf{purple}  \textbf{patch} is a piece of writing that contains very elaborate language or images .
 \textit{
	\begin{itemize}
	\item ...passages of purple prose describing intense experiences.
	\end{itemize}
}
\item  \\
 purple patch \textit{
	\begin{itemize}
	\end{itemize}
}
\end{enumerate}

\section*{shorthand}
{\large \color{blue}  }
\subsection*{Explain}
\begin{enumerate}
\item uncountable noun \\
\textbf{Shorthand} is a quick way of writing and uses signs to represent words or syllables . Shorthand is used by secretaries and journalists to write down what someone is saying .
 \textit{
	\begin{itemize}
	\item Ben took notes in shorthand.
	\end{itemize}
}
\item uncountable noun \\
You can use \textbf{shorthand} to mean a quick or simple way of referring to something.
 \textit{
	\begin{itemize}
	\item We've been friends so long we have a kind of shorthand. We don't really need to speak.
	\item The fiction that 'he' is a neutral shorthand for 'he or she' is no longer acceptable
to many.
	\end{itemize}
}
\end{enumerate}

\section*{rare}
{\large \color{blue}  rarer  rarest  }
\subsection*{Explain}
\begin{enumerate}
\item adjective \\
Something that is \textbf{rare} is not common and is therefore interesting or valuable .
 \textit{
	\begin{itemize}
	\item ...the black-necked crane, one of the rarest species in the world.
	\item She collects rare plants.
	\item Do you want to know about a particular rare stamp or rare stamps in general?
	\end{itemize}
}
\item adjective \\
An event or situation that is \textbf{rare} does not occur very often.
 \textit{
	\begin{itemize}
	\item ...on those rare occasions when he did eat alone.
	\item Heart attacks were extremely rare in babies, he said.
	\item It's apparently rare for anyone to have two legs the same length.
	\item I think it's very rare to have big families nowadays.
	\end{itemize}
}
\item adjective \\
You use \textbf{rare} to emphasize an extremely good or remarkable quality.
 \textit{
	\begin{itemize}
	\item Ferris has a rare ability to record her observations on paper.
	\item It was a rare pleasure to see him in action.
	\item ...a leader of rare strength and instinct.
	\end{itemize}
}
\item adjective \\
Meat that is \textbf{rare} is cooked very lightly so that the inside is still red.
 \textit{
	\begin{itemize}
	\item Thick tuna steaks are eaten rare, like beef.
	\item Waiter, I specifically asked for this steak rare.
	\end{itemize}
}
\end{enumerate}

\section*{sad}
{\large \color{blue}  sadder  saddest  }
\subsection*{Explain}
\begin{enumerate}
\item adjective \\
If you are \textbf{sad} , you feel unhappy, usually because something has happened that you do not like.
 \textit{
	\begin{itemize}
	\item The relationship had been important to me and its loss left me feeling sad and empty.
	\item I'm sad that Julie's marriage is on the verge of splitting up.
	\item I'd grown fond of our little house and felt sad to leave it.
	\item I'm sad about my toys getting burned in the fire.
	\end{itemize}
}
\item adjective \\
\textbf{Sad}  stories and \textbf{sad}  news make you feel sad.
 \textit{
	\begin{itemize}
	\item ...a desperately humorous, impossibly sad novel.
	\item I received the sad news that he had been killed in a motor-cycle accident.
	\end{itemize}
}
\item adjective \\
A \textbf{sad} event or situation is unfortunate or undesirable .
 \textit{
	\begin{itemize}
	\item It's a sad truth that children are the biggest victims of passive smoking.
	\end{itemize}
}
\item adjective \\
If you describe someone as \textbf{sad} , you do not have any respect for them and think their behaviour or ideas are ridiculous .
 \textit{
	\begin{itemize}
	\item ...the obsessive rantings of sad old petrolheads.
	\end{itemize}
}
\item  \\
 sad to say \textit{
	\begin{itemize}
	\end{itemize}
}
\end{enumerate}

\section*{spite}
{\large \color{blue}  }
\subsection*{Explain}
\begin{enumerate}
\item  \\
 in spite of sth \textit{
	\begin{itemize}
	\end{itemize}
}
\item phrase \\
If you do something \textbf{in spite of}  \textbf{yourself} , you do it although you did not really  intend to or expect to.
 \textit{
	\begin{itemize}
	\item The blunt comment made Richard laugh in spite of himself.
	\item She was deeply moved and in spite of herself could not help showing it.
	\end{itemize}
}
\item uncountable noun \\
If you do something cruel out of \textbf{spite} , you do it because you want to hurt or upset someone.
 \textit{
	\begin{itemize}
	\item He thinks Dan has vandalised the car out of spite.
	\item Never had she met such spite and pettiness.
	\end{itemize}
}
\item verb \\
If you do something cruel \textbf{to}  \textbf{spite} someone, you do it in order to hurt or upset them.
 \textit{
	\begin{itemize}
	\item You don't want to come because you want to spite me in front of my neighbours.
	\end{itemize}
}
\end{enumerate}

\section*{sharp}
{\large \color{blue}  sharps  sharper  sharpest  }
\subsection*{Explain}
\begin{enumerate}
\item adjective \\
A \textbf{sharp} point or edge is very thin and can cut through things very easily. A \textbf{sharp}  knife , tool, or other object has a point or edge of this kind.
 \textit{
	\begin{itemize}
	\item The other end of the twig is sharpened into a sharp point to use as a toothpick.
	\item Using a sharp knife, cut away the pith and peel from both fruits.
	\item The ground was strewn with sharp-edged pebbles.
	\end{itemize}
}
\item adjective \\
You can describe a shape or an object as \textbf{sharp} if part of it or one end of it comes to a point or forms an angle.
 \textit{
	\begin{itemize}
	\item His nose was thin and sharp.
	\item ...black sharp-toed cowboy boots.
	\end{itemize}
}
\item adjective \\
A \textbf{sharp}  bend or turn is one that changes direction suddenly .
 \textbf{Sharp} is also an adverb .
 \textit{
	\begin{itemize}
	\item I was approaching a fairly sharp bend that swept downhill to the left.
	\item Do not cross the bridge but turn sharp left to go down on to the towpath.
	\end{itemize}
}
\item adjective \\
If you describe someone as \textbf{sharp} , you are praising them because they are quick to notice , hear , understand , or react to things.
 \textit{
	\begin{itemize}
	\item He is very sharp, a quick thinker and swift with repartee.
	\item Gates is known to be a superb analyst with a sharp eye and an excellent memory.
	\end{itemize}
}
\item adjective \\
If someone says something in a \textbf{sharp} way, they say it suddenly and rather firmly or angrily, for example because they
are warning or criticizing you.
 \textit{
	\begin{itemize}
	\item 'Don't contradict your mother,' was Charles's sharp reprimand.
	\item That ruling had drawn sharp criticism from civil rights groups.
	\end{itemize}
}
\item adjective \\
A \textbf{sharp} change, movement, or feeling occurs suddenly, and is great in amount, force, or degree.
 \textit{
	\begin{itemize}
	\item There's been a sharp rise in the rate of inflation.
	\item Tennis requires a lot of short sharp movements.
	\item He felt a sharp pain in the abductor muscle in his right thigh.
	\end{itemize}
}
\item adjective \\
A \textbf{sharp}  difference , image, or sound is very easy to see, hear, or distinguish .
 \textit{
	\begin{itemize}
	\item Many people make a sharp distinction between humans and other animals.
	\item Her reticence was in sharp contrast to the glamour and star status of her predecessors.
	\item All the footmarks are quite sharp and clear.
	\item We heard a voice sing out in a clear, sharp tone.
	\end{itemize}
}
\item adjective \\
A \textbf{sharp} taste or smell is rather strong or bitter, but is often also clear and fresh .
 \textit{
	\begin{itemize}
	\item ...a colourless, almost odourless liquid with a sharp, sweetish taste.
	\item In the hot sun the rain-washed herbs smelled sharp and spicy and sweet all at once.
	\end{itemize}
}
\item graded adjective \\
A \textbf{sharp} wind, or \textbf{sharp}  cold , is so strong or intense that it almost  hurts you when you are exposed to it.
 \textit{
	\begin{itemize}
	\item The wind was not as sharp and cruel as it had been.
	\end{itemize}
}
\item graded adjective \\
\textbf{Sharp} clothes are neat , elegant , and fashionable .
 \textit{
	\begin{itemize}
	\item Now politics is all about the right haircut and a sharp suit.
	\item A sharp dresser, Wyatt is never seen in casual clothes.
	\end{itemize}
}
\item adverb \\
\textbf{Sharp} is used after stating a particular time to show that something happens at exactly the time stated.
 \textit{
	\begin{itemize}
	\item She planned to unlock the store at 8.00 sharp this morning.
	\end{itemize}
}
\item countable noun \\
\textbf{Sharp} is used after a letter representing a musical note to show that the note should be
played or sung  half a tone higher. \textbf{Sharp} is often represented by the symbol ♯.
 \textit{
	\begin{itemize}
	\item A solitary viola plucks a lonely, soft F sharp.
	\end{itemize}
}
\item  \\
 at the sharp end \textit{
	\begin{itemize}
	\end{itemize}
}
\end{enumerate}

\section*{spouse}
{\large \color{blue}  spouses  }
\subsection*{Explain}
\begin{enumerate}
\item countable noun \\
Someone's \textbf{spouse} is the person they are married to.
 \textit{
	\begin{itemize}
	\end{itemize}
}
\end{enumerate}

\section*{short}
{\large \color{blue}  shorter  shortest  }
\subsection*{Explain}
\begin{enumerate}
\item adjective \\
If something is \textbf{short} or lasts for a \textbf{short} time, it does not last very long.
 \textit{
	\begin{itemize}
	\item The announcement was made a short time ago.
	\item How could you do it in such a short period of time?
	\item Mr Annan took a short break before resuming his schedule.
	\item Kemp gave a short laugh.
	\item We had a short meeting.
	\end{itemize}
}
\item adjective \\
If you talk about a \textbf{short}  hour , day, or year, you mean that it seems to have passed very quickly or will seem to pass very quickly.
 \textit{
	\begin{itemize}
	\item For a few short weeks there was peace.
	\item Only five short years later, your money will have grown by $94,000.
	\end{itemize}
}
\item adjective \\
A \textbf{short} speech, letter, or book does not have many words or pages in it.
 \textit{
	\begin{itemize}
	\item ...short extracts from the Bible.
	\item This is a short note to say thank you.
	\end{itemize}
}
\item adjective \\
Someone who is \textbf{short} is not as tall as most people are.
 \textit{
	\begin{itemize}
	\item I'm tall and thin and he's short and fat.
	\item ...a short, elderly woman with grey hair.
	\item He's the shortest of four brothers.
	\end{itemize}
}
\item adjective \\
Something that is \textbf{short} measures only a small amount from one end to the other.
 \textit{
	\begin{itemize}
	\item The city centre and shops are only a short distance away.
	\item A short flight of steps led to a grand doorway.
	\item His black hair was very short.
	\end{itemize}
}
\item adjective \\
If you are \textbf{short of} something or if it is \textbf{short} , you do not have enough of it. If you are running \textbf{short of} something or if it is running \textbf{short} , you do not have much of it left.
 \textit{
	\begin{itemize}
	\item Her father's illness left the family short of money.
	\item Government forces are running short of ammunition and fuel.
	\item Supplies of everything are unreliable; food is short.
	\item Time is running short.
	\end{itemize}
}
\item  \\
 to fall short \textit{
	\begin{itemize}
	\end{itemize}
}
\item phrase \\
\textbf{Short of} a particular thing means except for that thing or without actually doing that thing.
 \textit{
	\begin{itemize}
	\item Short of climbing railings four metres high, there was no way into the garden from
this road.
	\item They have no means, short of civil war, to enforce their will upon the minorities.
	\end{itemize}
}
\item adverb \\
If something is \textbf{cut}  \textbf{short} or \textbf{stops}  \textbf{short} , it is stopped before people expect it to or before it has finished.
 \textit{
	\begin{itemize}
	\item His glittering career was cut short by a heart attack.
	\item Jackson cut short his trip to Africa.
	\end{itemize}
}
\item adjective \\
If a name or abbreviation is \textbf{short for} another name, it is the short version of that name.
 \textit{
	\begin{itemize}
	\item Her friend Kes (short for Kesewa) was in tears.
	\item 'O.O.B.E.' is short for 'Out Of Body Experience'.
	\end{itemize}
}
\item adjective \\
If you have a \textbf{short}  temper , you get angry very easily.
 \textit{
	\begin{itemize}
	\item ...an awkward, self-conscious woman with a short temper.
	\end{itemize}
}
\item adjective \\
If you are \textbf{short with} someone, you speak briefly and rather rudely to them, because you are impatient or angry.
 \textit{
	\begin{itemize}
	\item She seemed nervous or tense, and she was definitely short with me.
	\end{itemize}
}
\item  \\
 to be caught short \textit{
	\begin{itemize}
	\end{itemize}
}
\item  \\
 for short \textit{
	\begin{itemize}
	\end{itemize}
}
\item  \\
 go short \textit{
	\begin{itemize}
	\end{itemize}
}
\item  \\
 in short \textit{
	\begin{itemize}
	\end{itemize}
}
\item  \\
 nothing short of \textit{
	\begin{itemize}
	\end{itemize}
}
\item  \\
 X short of a Y \textit{
	\begin{itemize}
	\end{itemize}
}
\item  \\
 be short on sth \textit{
	\begin{itemize}
	\end{itemize}
}
\item  \\
 to stop short of \textit{
	\begin{itemize}
	\end{itemize}
}
\item  \\
 short time \textit{
	\begin{itemize}
	\end{itemize}
}
\item  \\
 pull/bring sb up short \textit{
	\begin{itemize}
	\end{itemize}
}
\item  \\
 make short work of sb/sth \textit{
	\begin{itemize}
	\end{itemize}
}
\end{enumerate}

\section*{stake}
{\large \color{blue}  stakes  staking  staked  }
\subsection*{Explain}
\begin{enumerate}
\item  \\
 at stake \textit{
	\begin{itemize}
	\end{itemize}
}
\item plural noun \\
The \textbf{stakes}  involved in a contest or a risky  action are the things that can be gained or lost .
 \textit{
	\begin{itemize}
	\item The game was usually played for high stakes between two large groups.
	\item By arresting the organisation's leaders the government has raised the stakes.
	\item For the president the political stakes could hardly have been higher.
	\end{itemize}
}
\item verb \\
If you \textbf{stake} something such as your money or your reputation  \textbf{on} the result of something, you risk your money or reputation on it.
 \textit{
	\begin{itemize}
	\item He has staked his political future on an election victory.
	\item He has staked his reputation on the outcome.
	\end{itemize}
}
\item countable noun \\
If you have a \textbf{stake in} something such as a business , it matters to you, for example because you own part of it or because its success or failure  will  affect you.
 \textit{
	\begin{itemize}
	\item He enjoyed an entrepreneurial role where he had a big financial stake in his own
efforts.
	\end{itemize}
}
\item plural noun \\
You can use \textbf{stakes} to refer to something that is like a contest. For example, you can refer to the choosing of a leader as \textbf{the}  leadership  \textbf{stakes} .
 \textit{
	\begin{itemize}
	\item She won hands down in the glamour stakes.
	\end{itemize}
}
\item countable noun \\
A \textbf{stake} is a pointed wooden post which is pushed into the ground, for example in order to support a young tree.
 \textit{
	\begin{itemize}
	\end{itemize}
}
\item  \\
 to stake a claim \textit{
	\begin{itemize}
	\end{itemize}
}
\end{enumerate}

\section*{stiff}
{\large \color{blue}  stiffer  stiffest  }
\subsection*{Explain}
\begin{enumerate}
\item adjective \\
Something that is \textbf{stiff} is firm or does not bend easily.
 \textit{
	\begin{itemize}
	\item The furniture was stiff, uncomfortable, too delicate, and too neat.
	\item His waterproof trousers were brand new and stiff.
	\item Her fingers were stiff with cold inside her leather gloves.
	\item Clean the mussels with a stiff brush under cold running water.
	\end{itemize}
}
\item adjective \\
Something such as a door or drawer that is \textbf{stiff} does not move as easily as it should.
 \textit{
	\begin{itemize}
	\item Train door handles are stiff so that they cannot be opened accidentally.
	\end{itemize}
}
\item adjective \\
If you are \textbf{stiff} , your muscles or joints  hurt when you move, because of illness or because of too much exercise .
 \textit{
	\begin{itemize}
	\item Massage will relieve tension and stiff muscles.
	\item I'm stiff all over right now–I hope I can recover for tomorrow's race.
	\end{itemize}
}
\item adjective \\
\textbf{Stiff}  behaviour is rather formal and not very friendly or relaxed .
 \textit{
	\begin{itemize}
	\item She looked at him with a stiff smile.
	\item They always seemed a little awkward with each other, a bit stiff and formal.
	\end{itemize}
}
\item adjective \\
\textbf{Stiff} can be used to mean difficult or severe .
 \textit{
	\begin{itemize}
	\item The film faces stiff competition for the Best Film nomination.
	\item Under the country's stiff anti-corruption laws they could face twenty years in jail.
	\end{itemize}
}
\item adjective \\
A \textbf{stiff} drink is a large amount of a strong alcoholic drink.
 \textit{
	\begin{itemize}
	\item ...a stiff whisky.
	\end{itemize}
}
\item graded adjective \\
A \textbf{stiff}  wind is blowing  quite strongly.
 \textit{
	\begin{itemize}
	\item Next morning dawned clear and sunny, with a stiff breeze rustling the trees.
	\end{itemize}
}
\item adverb \\
If you are bored  \textbf{stiff} , worried  \textbf{stiff} , or scared  \textbf{stiff} , you are extremely bored, worried, or scared.
 \textbf{Stiff} is also an adjective .
 \textit{
	\begin{itemize}
	\item Anna tried to look interested. Actually, she was bored stiff.
	\item I was scared stiff when I realized what I'd done.
	\item Even if he bores you stiff, it is good manners not to let him know it.
	\end{itemize}
}
\end{enumerate}

\section*{stuff}
{\large \color{blue}  stuffs  stuffing  stuffed  }
\subsection*{Explain}
\begin{enumerate}
\item uncountable noun \\
You can use \textbf{stuff} to refer to things such as a substance, a collection of things, events, or ideas, or the contents of something in a general way without mentioning the thing itself by name.
 \textit{
	\begin{itemize}
	\item I'd like some coffee, and I don't object to the powdered stuff if it's all you've
got.
	\item I don't know anything about this antique stuff.
	\item 'What do you want to know?'—'About life and stuff.'
	\item Don't tell me you still believe in all that stuff?
	\item He pointed to a duffle bag.'That's my stuff.'
	\end{itemize}
}
\item verb \\
If you \textbf{stuff} something somewhere , you push it there quickly and roughly .
 \textit{
	\begin{itemize}
	\item I stuffed my hands in my pockets.
	\item He stuffed the newspapers into a litter bin and headed down the street.
	\item His pants were stuffed inside the tops of his boots.
	\end{itemize}
}
\item verb \\
If you \textbf{stuff} a container or space \textbf{with} something, you fill it with something or with a quantity of things until it is full.
 \textit{
	\begin{itemize}
	\item He grabbed my purse, opened it and stuffed it full, then gave it back to me.
	\item He still stood behind his cash register stuffing his mouth with popcorn.
	\item ...wallets stuffed with dollars.
	\end{itemize}
}
\item verb \\
If you \textbf{stuff}  \textbf{yourself} , you eat a lot of food.
 \textit{
	\begin{itemize}
	\item I could stuff myself with ten chocolate bars and half an hour later eat a big meal.
	\end{itemize}
}
\item verb \\
If you \textbf{stuff} a bird such as a chicken or a vegetable such as a pepper , you put a mixture of food inside it before cooking it.
 \textit{
	\begin{itemize}
	\item Will you stuff the turkey and shove it in the oven for me?
	\item ...stuffed tomatoes.
	\end{itemize}
}
\item verb \\
If a dead animal \textbf{is stuffed} , it is filled with a substance so that it can be preserved and displayed.
 \textit{
	\begin{itemize}
	\item A pike weighing 29 lb 8 oz taken in 1878 was stuffed and is on display at the estate
office.
	\item He didn't much care for the stuffed animal heads that hung on the walls.
	\end{itemize}
}
\item singular noun \\
If you say that one thing is \textbf{the stuff of} another, you mean that the first thing is a very important feature or characteristic of the second thing, or that the second thing can be based or built
on the first thing.
 \textit{
	\begin{itemize}
	\item The idea that we can be whatever we want has become the stuff of television commercials.
	\end{itemize}
}
\item exclamation \\
If you are angry with someone for something that they have said or done, you might say ' \textbf{Get stuffed!} ' to them.
 \textit{
	\begin{itemize}
	\end{itemize}
}
\item verb \\
\textbf{Stuff} is used in front of nouns to emphasize that you do not care about something, or do not want it.
 \textit{
	\begin{itemize}
	\item Ultimately my attitude was: stuff them.
	\item Stuff your money. We don't want a handout.
	\end{itemize}
}
\item  \\
 do one's stuff \textit{
	\begin{itemize}
	\end{itemize}
}
\item  \\
 know one's stuff \textit{
	\begin{itemize}
	\end{itemize}
}
\end{enumerate}

\section*{supersonic}
{\large \color{blue}  }
\subsection*{Explain}
\begin{enumerate}
\item adjective \\
\textbf{Supersonic} aircraft travel faster than the speed of sound.
 \textit{
	\begin{itemize}
	\item There was a huge bang; it sounded like a supersonic jet.
	\end{itemize}
}
\end{enumerate}

\section*{suite}
{\large \color{blue}  suites  }
\subsection*{Explain}
\begin{enumerate}
\item countable noun \\
A \textbf{suite} is a set of rooms in a hotel or other building .
 \textit{
	\begin{itemize}
	\item They had a fabulous time during their week in a suite at the Paris Hilton.
	\item ...a new suite of offices.
	\end{itemize}
}
\item countable noun \\
A \textbf{suite} is a set of matching armchairs and a sofa .
 \textit{
	\begin{itemize}
	\item ...a three-piece suite.
	\end{itemize}
}
\item countable noun \\
A bathroom  \textbf{suite} is a matching bath , washbasin , and toilet .
 \textit{
	\begin{itemize}
	\end{itemize}
}
\end{enumerate}

\section*{temporary}
{\large \color{blue}  }
\subsection*{Explain}
\begin{enumerate}
\item adjective \\
Something that is \textbf{temporary}  lasts for only a limited time.
 \textit{
	\begin{itemize}
	\item His job here is only temporary.
	\item Most adolescent problems are temporary.
	\item ...a temporary loss of memory.
	\end{itemize}
}
\end{enumerate}

\section*{summer}
{\large \color{blue}  summers  }
\subsection*{Explain}
\begin{enumerate}
\item variable noun \\
\textbf{Summer} is the season between spring and autumn when the weather is usually warm or hot.
 \textit{
	\begin{itemize}
	\item In summer I like to go sailing in Long Island.
	\item I escaped the heatwave in London earlier this summer and flew to Cork.
	\item It was a perfect summer's day.
	\item ...in the summer of 1987.
	\item ...the summer holidays.
	\item He used to spend childhood summers with his grandparents.
	\end{itemize}
}
\end{enumerate}

\section*{tragic}
{\large \color{blue}  }
\subsection*{Explain}
\begin{enumerate}
\item adjective \\
A \textbf{tragic}  event or situation is extremely  sad , usually because it involves  death or suffering .
 \textit{
	\begin{itemize}
	\item It was just a tragic accident.
	\item ...the tragic loss of so many lives.
	\item The circumstances are tragic but we have to act within the law.
	\end{itemize}
}
\item adjective \\
\textbf{Tragic} is used to refer to tragedy as a type of literature .
 \textit{
	\begin{itemize}
	\item ...Michael Henchard, the tragic hero of 'The Mayor of Casterbridge'.
	\end{itemize}
}
\end{enumerate}

\section*{table}
{\large \color{blue}  tables  tabling  tabled  }
\subsection*{Explain}
\begin{enumerate}
\item countable noun \\
A \textbf{table} is a piece of furniture with a flat top that you put things on or sit at.
 \textit{
	\begin{itemize}
	\item She was sitting at the kitchen table eating a currant bun.
	\item I placed his drink on the small table at his elbow.
	\end{itemize}
}
\item countable noun \\
If you ask for a \textbf{table} in a restaurant, you want to have a meal there.
 \textit{
	\begin{itemize}
	\item I'd like a table for two at about 8.30, please.
	\item I booked a table at the Savoy Grill.
	\item You will join us at our table, won't you?
	\end{itemize}
}
\item verb \\
If someone \textbf{tables} a proposal , they say formally that they want it to be discussed at a meeting .
 \textit{
	\begin{itemize}
	\item They've tabled a motion criticising the Government for doing nothing about the problem.
	\end{itemize}
}
\item verb \\
If someone \textbf{tables} a proposal or plan which has been put forward , they decide to discuss it or deal with it at a later date , rather than straight away.
 \textit{
	\begin{itemize}
	\item We will table that for later.
	\end{itemize}
}
\item countable noun \\
A \textbf{table} is a written set of facts and figures arranged in columns and rows .
 \textit{
	\begin{itemize}
	\item Consult the table on page 104.
	\item Other research supports the figures in Table 3.3.
	\end{itemize}
}
\item countable noun \\
A \textbf{table} is a list of the multiplications of numbers between one and twelve . Children often have to learn tables at school.
 \textit{
	\begin{itemize}
	\item He didn't know his eleven-times table.
	\item I attempted to learn my tables.
	\end{itemize}
}
\item  \\
 on the table \textit{
	\begin{itemize}
	\end{itemize}
}
\item  \\
 turn the tables \textit{
	\begin{itemize}
	\end{itemize}
}
\end{enumerate}

\section*{transient}
{\large \color{blue}  transients  }
\subsection*{Explain}
\begin{enumerate}
\item adjective \\
\textbf{Transient} is used to describe a situation that lasts only a short time or is constantly changing.
 \textit{
	\begin{itemize}
	\item ...the transient nature of high fashion.
	\item In most cases, pain is transient.
	\end{itemize}
}
\item countable noun \\
\textbf{Transients} are people who stay in a place for only a short time and then move somewhere else.
 \textit{
	\begin{itemize}
	\item ...a hotel for transients.
	\end{itemize}
}
\end{enumerate}

\section*{truth}
{\large \color{blue}  truths  }
\subsection*{Explain}
\begin{enumerate}
\item uncountable noun \\
\textbf{The}  \textbf{truth} about something is all the facts about it, rather than things that are imagined or invented .
 \textit{
	\begin{itemize}
	\item Is it possible to separate truth from fiction?
	\item I must tell you the truth about this business.
	\item The truth of the matter is that we had no other choice.
	\item In the town very few know the whole truth.
	\item ...judgements of truth or falsity.
	\end{itemize}
}
\item uncountable noun \\
If you say that there is some \textbf{truth}  \textbf{in} a statement or story , you mean that it is true, or at least partly true.
 \textit{
	\begin{itemize}
	\item There is no truth in this story.
	\item Is there any truth to the rumors?
	\item The criticisms have at least an element of truth and validity.
	\end{itemize}
}
\item countable noun \\
A \textbf{truth} is something that is believed to be true.
 \textit{
	\begin{itemize}
	\item It is still a basic truth that women have to work harder than men to get to the same
level.
	\end{itemize}
}
\item  \\
 in truth \textit{
	\begin{itemize}
	\end{itemize}
}
\item  \\
 to tell you the truth \textit{
	\begin{itemize}
	\end{itemize}
}
\end{enumerate}

\section*{ultimate}
{\large \color{blue}  }
\subsection*{Explain}
\begin{enumerate}
\item adjective \\
You use \textbf{ultimate} to describe the final result or aim of a long series of events .
 \textit{
	\begin{itemize}
	\item He said it is still not possible to predict the ultimate outcome.
	\item The ultimate aim is to expand the network further.
	\end{itemize}
}
\item adjective \\
You use \textbf{ultimate} to describe the original  source or cause of something.
 \textit{
	\begin{itemize}
	\item Plants are the ultimate source of all foodstuffs.
	\item The ultimate cause of what's happened seems to have been the advertising campaign.
	\end{itemize}
}
\item adjective \\
You use \textbf{ultimate} to describe the most important or powerful thing of a particular  kind .
 \textit{
	\begin{itemize}
	\item ...the ultimate power of the central government.
	\item Of course, the ultimate authority remained the presidency.
	\item My experience as player, coach and manager has prepared me for this ultimate challenge.
	\end{itemize}
}
\item adjective \\
You use \textbf{ultimate} to describe the most extreme and unpleasant  example of a particular thing.
 \textit{
	\begin{itemize}
	\item Bringing back the death penalty would be the ultimate abuse of human rights.
	\item Treachery was the ultimate sin.
	\item Coleman lives in fear of the ultimate disgrace.
	\end{itemize}
}
\item adjective \\
You use \textbf{ultimate} to describe the best  possible example of a particular thing.
 \textit{
	\begin{itemize}
	\item He is the ultimate English gentleman.
	\item Caviar and oysters on ice are generally considered the ultimate luxury foods.
	\end{itemize}
}
\item  \\
 the ultimate in \textit{
	\begin{itemize}
	\end{itemize}
}
\end{enumerate}

\section*{vacuum}
{\large \color{blue}  vacuums  vacuuming  vacuumed  }
\subsection*{Explain}
\begin{enumerate}
\item countable noun \\
If someone or something creates a \textbf{vacuum} , they leave a place or position which then needs to be filled by another person or thing.
 \textit{
	\begin{itemize}
	\item The collapse of the army left a vacuum in the area.
	\item His presence should fill the power vacuum which has been developing over the past
few days.
	\end{itemize}
}
\item  \\
 in a vacuum \textit{
	\begin{itemize}
	\end{itemize}
}
\item verb \\
If you \textbf{vacuum} something, you clean it using a vacuum cleaner.
 \textit{
	\begin{itemize}
	\item I vacuumed the carpets today.
	\item It's important to vacuum regularly.
	\end{itemize}
}
\item countable noun \\
A \textbf{vacuum} is a space that contains no air or other gas.
 \textit{
	\begin{itemize}
	\item Wind is a current of air caused by a vacuum caused by hot air rising.
	\item The spinning turbine creates a vacuum.
	\end{itemize}
}
\end{enumerate}

\section*{utmost}
{\large \color{blue}  }
\subsection*{Explain}
\begin{enumerate}
\item adjective \\
You can use \textbf{utmost} to emphasize the importance or seriousness of something or to emphasize the way that it is done.
 \textit{
	\begin{itemize}
	\item It is a matter of the utmost urgency to find out what has happened to these people.
	\item Security matters are treated with the utmost seriousness.
	\item You should proceed with the utmost caution.
	\item Utmost care must be taken not to spill any of the contents.
	\end{itemize}
}
\item singular noun \\
If you say that you are doing your \textbf{utmost}  \textbf{to} do something, you are emphasizing that you are trying as hard as you can to do it.
 \textit{
	\begin{itemize}
	\item He would have done his utmost to help her, of that she was certain.
	\item He will try his utmost to help them by means of his conventional medical knowledge.
	\end{itemize}
}
\item  \\
 to the utmost \textit{
	\begin{itemize}
	\end{itemize}
}
\end{enumerate}

\section*{velocity}
{\large \color{blue}  velocities  }
\subsection*{Explain}
\begin{enumerate}
\item variable noun \\
\textbf{Velocity} is the speed at which something moves in a particular direction.
 \textit{
	\begin{itemize}
	\item ...the velocity of light.
	\item ...the velocities at which the stars orbit.
	\item ...high velocity rifles.
	\end{itemize}
}
\end{enumerate}

\section*{vigorous}
{\large \color{blue}  }
\subsection*{Explain}
\begin{enumerate}
\item adjective \\
\textbf{Vigorous}  physical activities involve using a lot of energy , usually to do short and repeated actions.
 \textit{
	\begin{itemize}
	\item Very vigorous exercise can increase the risk of heart attacks.
	\item African dance is vigorous, but full of subtlety.
	\end{itemize}
}
\item adjective \\
A \textbf{vigorous} person does things with great energy and enthusiasm . A \textbf{vigorous}  campaign or activity is done with great energy and enthusiasm.
 \textit{
	\begin{itemize}
	\item Sir Robert was a strong and vigorous politician.
	\item ...the most vigorous critics of the government.
	\item They will take vigorous action to recover the debts.
	\item ...a vigorous campaign against GM food.
	\end{itemize}
}
\item graded adjective \\
A \textbf{vigorous} person is strong and healthy and full of energy.
 \textit{
	\begin{itemize}
	\item He was a vigorous, handsome young man.
	\end{itemize}
}
\end{enumerate}

\section*{wing}
{\large \color{blue}  wings  winging  winged  }
\subsection*{Explain}
\begin{enumerate}
\item countable noun \\
The \textbf{wings} of a bird or insect are the two parts of its body that it uses for flying.
 \textit{
	\begin{itemize}
	\item The bird flapped its wings furiously.
	\item She saw the occasional glimmer of a moth's wings.
	\end{itemize}
}
\item countable noun \\
The \textbf{wings} of an aeroplane are the long flat parts sticking out of its side which support it while it is flying.
 \textit{
	\begin{itemize}
	\end{itemize}
}
\item countable noun \\
A \textbf{wing} of a building is a part of it which sticks out from the main part.
 \textit{
	\begin{itemize}
	\item We were given an office in the empty west wing.
	\item ...in the Child Psychiatry wing of London's Royal Free Hospital.
	\end{itemize}
}
\item countable noun \\
A \textbf{wing} of an organization, especially a political organization, is a group within it which has a particular function or
particular beliefs.
 \textit{
	\begin{itemize}
	\item The military wing of the organisation was banned.
	\item ...the liberal wing of the Democratic Party.
	\end{itemize}
}
\item plural noun \\
In a theatre, \textbf{the wings} are the sides of the stage which are hidden from the audience by curtains or scenery .
 \textit{
	\begin{itemize}
	\item Most nights I watched the start of the play from the wings.
	\end{itemize}
}
\item countable noun \\
In a game such as football or hockey , \textbf{the}  \textbf{left wing} and \textbf{the}  \textbf{right wing} are the areas on the far left and the far right of the pitch. You can also refer
to the players who play in these positions as \textbf{the}  \textbf{left wing} and \textbf{the}  \textbf{right wing} .
 \textit{
	\begin{itemize}
	\end{itemize}
}
\item countable noun \\
A \textbf{wing} of a car is a part of it on the outside which is over one of the wheels.
 \textit{
	\begin{itemize}
	\end{itemize}
}
\item plural noun \\
When pilots get their \textbf{wings} , they become qualified to fly aeroplanes.
 \textit{
	\begin{itemize}
	\item He had no sooner got his wings than the Korean conflict broke out.
	\end{itemize}
}
\item verb \\
If you say that something or someone \textbf{wings} their way somewhere or \textbf{wings} somewhere, you mean that they go there quickly, especially by plane.
 \textit{
	\begin{itemize}
	\item A few moments later they were airborne and winging their way south.
	\item A cash bonanza will be winging its way to the 600,000 members of the scheme.
	\item The first of the airliners winged westwards and home.
	\end{itemize}
}
\item  \\
 to clip someone's wings \textit{
	\begin{itemize}
	\end{itemize}
}
\item  \\
 in the wings \textit{
	\begin{itemize}
	\end{itemize}
}
\item  \\
 to spread your wings \textit{
	\begin{itemize}
	\end{itemize}
}
\item  \\
 take sb under one's wing \textit{
	\begin{itemize}
	\end{itemize}
}
\end{enumerate}

\section*{affluent}
{\large \color{blue}  }
\subsection*{Explain}
\begin{enumerate}
\item adjective \\
If you are \textbf{affluent} , you have a lot of money.
 \textbf{The affluent} are people who are affluent.
 \textit{
	\begin{itemize}
	\item Today's affluent Indian tourists are also big spenders.
	\item ...the affluent neighborhoods of Malibu.
	\item The diet of the affluent has not changed much over the decades.
	\end{itemize}
}
\end{enumerate}

\section*{absence}
{\large \color{blue}  absences  }
\subsection*{Explain}
\begin{enumerate}
\item variable noun \\
Someone's \textbf{absence} from a place is the fact that they are not there.
 \textit{
	\begin{itemize}
	\item ...a bundle of letters which had arrived for me in my absence.
	\item Eleanor would later blame her mother-in-law for her husband's frequent absences.
	\end{itemize}
}
\item singular noun \\
The \textbf{absence} of something from a place is the fact that it is not there or does not exist .
 \textit{
	\begin{itemize}
	\item The presence or absence of clouds can have an important impact on heat transfer.
	\item In the absence of a will the courts decide who the guardian is.
	\end{itemize}
}
\end{enumerate}

\section*{anxious}
{\large \color{blue}  }
\subsection*{Explain}
\begin{enumerate}
\item adjective \\
If you are \textbf{anxious}  \textbf{to} do something or \textbf{anxious}  \textbf{that} something should happen , you very much want to do it or very much want it to happen.
 \textit{
	\begin{itemize}
	\item The miners were anxious to avoid a strike.
	\item He is anxious that there should be no delay.
	\item Those anxious for reform say that the present system is too narrow.
	\end{itemize}
}
\item adjective \\
If you are \textbf{anxious} , you are nervous or worried about something.
 \textit{
	\begin{itemize}
	\item The foreign minister admitted he was still anxious about the situation in the country.
	\item A friend of mine is a very anxious person.
	\end{itemize}
}
\item adjective \\
An \textbf{anxious} time or situation is one during which you feel nervous and worried.
 \textit{
	\begin{itemize}
	\item He told last night of the anxious hours waiting to discover whether his girlfriend
was safe from the earthquake.
	\end{itemize}
}
\end{enumerate}

\section*{acrobat}
{\large \color{blue}  acrobats  }
\subsection*{Explain}
\begin{enumerate}
\item countable noun \\
An \textbf{acrobat} is an entertainer who performs difficult physical acts such as jumping and balancing , especially in a circus .
 \textit{
	\begin{itemize}
	\end{itemize}
}
\end{enumerate}

\section*{arrogant}
{\large \color{blue}  }
\subsection*{Explain}
\begin{enumerate}
\item adjective \\
Someone who is \textbf{arrogant}  behaves in a proud, unpleasant way towards other people because they believe that they are more important than others.
 \textit{
	\begin{itemize}
	\item He was so arrogant.
	\item That sounds arrogant, doesn't it?
	\item ...an air of arrogant indifference.
	\end{itemize}
}
\end{enumerate}

\section*{bribe}
{\large \color{blue}  bribes  bribing  bribed  }
\subsection*{Explain}
\begin{enumerate}
\item countable noun \\
A \textbf{bribe} is a sum of money or something valuable that one person offers or gives to another in order to persuade him or her to do something.
 \textit{
	\begin{itemize}
	\item He was being investigated for receiving bribes.
	\end{itemize}
}
\item verb \\
If one person \textbf{bribes} another, they give them a bribe.
 \textit{
	\begin{itemize}
	\item He was accused of bribing a senior bank official.
	\item The government bribed the workers to be quiet.
	\end{itemize}
}
\end{enumerate}

\section*{articulate}
{\large \color{blue}  articulates  articulating  articulated  }
\subsection*{Explain}
\begin{enumerate}
\item adjective \\
If you describe someone as \textbf{articulate} , you mean that they are able to express their thoughts and ideas  easily and well .
 \textit{
	\begin{itemize}
	\item She is an articulate young woman.
	\item The child was unable to offer an articulate description of what she had witnessed.
	\end{itemize}
}
\item verb \\
When you \textbf{articulate} your ideas or feelings , you express them clearly in words.
 \textit{
	\begin{itemize}
	\item The president has been accused of failing to articulate an overall vision in foreign
affairs.
	\end{itemize}
}
\item verb \\
If you \textbf{articulate} something, you say it very clearly, so that each word or syllable can be heard .
 \textit{
	\begin{itemize}
	\item He articulated each syllable carefully.
	\end{itemize}
}
\end{enumerate}

\section*{candidate}
{\large \color{blue}  candidates  }
\subsection*{Explain}
\begin{enumerate}
\item countable noun \\
A \textbf{candidate} is someone who is being considered for a position, for example someone who is running in an election or applying for a job.
 \textit{
	\begin{itemize}
	\item The Democratic candidate is still leading in the polls.
	\item He is a candidate for the office of Governor.
	\item We all spoke to them and John emerged as the best candidate.
	\end{itemize}
}
\item countable noun \\
A \textbf{candidate} is someone who is taking an examination.
 \textit{
	\begin{itemize}
	\end{itemize}
}
\item countable noun \\
A \textbf{candidate} is someone who is studying for a degree at a college .
 \textit{
	\begin{itemize}
	\end{itemize}
}
\item countable noun \\
A \textbf{candidate} is a person or thing that is regarded as being suitable for a particular purpose or as being likely to do or be a particular thing.
 \textit{
	\begin{itemize}
	\item Investment banking looks a prime candidate for further job losses.
	\item Those who are overweight or indulge in high-salt diets are candidates for hypertension.
	\end{itemize}
}
\end{enumerate}

\section*{asleep}
{\large \color{blue}  }
\subsection*{Explain}
\begin{enumerate}
\item adjective \\
Someone who is \textbf{asleep} is sleeping.
 \textit{
	\begin{itemize}
	\item My four-year-old daughter was asleep on the sofa.
	\end{itemize}
}
\item  \\
 fall asleep \textit{
	\begin{itemize}
	\end{itemize}
}
\item  \\
 fast asleep \textit{
	\begin{itemize}
	\end{itemize}
}
\end{enumerate}

\section*{club}
{\large \color{blue}  clubs  clubbing  clubbed  }
\subsection*{Explain}
\begin{enumerate}
\item countable noun \\
A \textbf{club} is an organization of people interested in a particular activity or subject who usually
meet on a regular  basis .
 \textit{
	\begin{itemize}
	\item ...the Chorlton Conservative Club.
	\item ...a youth club.
	\item He was club secretary.
	\end{itemize}
}
\item countable noun \\
A \textbf{club} is a place where the members of a club meet.
 \textit{
	\begin{itemize}
	\item I stopped in at the club for a drink.
	\end{itemize}
}
\item countable noun \\
A \textbf{club} is a team which competes in sporting competitions .
 \textit{
	\begin{itemize}
	\item ...the New York Yankees baseball club.
	\item ...Liverpool football club.
	\end{itemize}
}
\item countable noun \\
A \textbf{club} is the same as a nightclub .
 \textit{
	\begin{itemize}
	\item It's a big dance hit in the clubs.
	\item ...the London club scene.
	\end{itemize}
}
\item countable noun \\
A \textbf{club} is a long, thin, metal stick with a piece of wood or metal at one end that you use
to hit the ball in golf.
 \textit{
	\begin{itemize}
	\item ...a six-iron club.
	\end{itemize}
}
\item countable noun \\
A \textbf{club} is a thick heavy stick that can be used as a weapon.
 \textit{
	\begin{itemize}
	\item Men armed with knives and clubs attacked his home.
	\end{itemize}
}
\item verb \\
To \textbf{club} a person or animal means to hit them hard with a thick heavy stick or a similar weapon.
 \textit{
	\begin{itemize}
	\item Two thugs clubbed him with baseball bats.
	\item Clubbing baby seals to death for their pelts is wrong.
	\end{itemize}
}
\item uncountable noun \\
\textbf{Clubs} is one of the four suits in a pack of playing cards. Each card in the suit is marked with one or more black symbols:
♣.
 A \textbf{club} is a playing card of this suit.
 \textit{
	\begin{itemize}
	\item ...the ace of clubs.
	\item The next player discarded a club.
	\end{itemize}
}
\end{enumerate}

\section*{blue}
{\large \color{blue}  bluer  bluest  blues  }
\subsection*{Explain}
\begin{enumerate}
\item colour \\
Something that is \textbf{blue} is the colour of the sky on a sunny day.
 \textit{
	\begin{itemize}
	\item There were swallows in the cloudless blue sky.
	\item She fixed her pale blue eyes on her father's.
	\item ...colourful blues and reds.
	\end{itemize}
}
\item plural noun \\
\textbf{The blues} is a type of music which was developed by African American musicians in the southern United States. It is characterized by a slow  tempo and a strong rhythm .
 \textit{
	\begin{itemize}
	\item His singing really does have the depth and the emotional range of the blues.
	\item ...the blues bars of Chicago.
	\end{itemize}
}
\item plural noun \\
If you have got  \textbf{the blues} , you feel  sad and depressed.
 \textit{
	\begin{itemize}
	\item He's been suffering from the blues since losing his job.
	\end{itemize}
}
\item adjective \\
If you are feeling \textbf{blue} , you are feeling sad or depressed, often when there is no particular reason .
 \textit{
	\begin{itemize}
	\item There's no earthly reason for me to feel so blue.
	\end{itemize}
}
\item countable noun \\
A Cambridge \textbf{blue} or an Oxford \textbf{blue} is a man or woman who has played for Cambridge or Oxford University in a particular
sport.
 \textit{
	\begin{itemize}
	\end{itemize}
}
\item adjective \\
\textbf{Blue} films, stories , or jokes are about sex .
 \textit{
	\begin{itemize}
	\item ...a secret stash of porn mags and blue movies.
	\end{itemize}
}
\item  \\
 out of the blue \textit{
	\begin{itemize}
	\end{itemize}
}
\end{enumerate}

\section*{column}
{\large \color{blue}  columns  }
\subsection*{Explain}
\begin{enumerate}
\item countable noun \\
A \textbf{column} is a tall , often decorated  cylinder of stone which is built to honour someone or forms part of a building.
 \textit{
	\begin{itemize}
	\item ...a London landmark, Nelson's Column in Trafalgar Square.
	\end{itemize}
}
\item countable noun \\
A \textbf{column} is something that has a tall narrow shape.
 \textit{
	\begin{itemize}
	\item The explosion sent a column of smoke thousands of feet into the air.
	\end{itemize}
}
\item countable noun \\
A \textbf{column} is a group of people or animals which moves in a long line.
 \textit{
	\begin{itemize}
	\item There were reports of columns of military vehicles appearing on the streets.
	\end{itemize}
}
\item countable noun \\
On a printed page such as a page of a dictionary , newspaper, or printed chart , a \textbf{column} is one of two or more vertical sections which are read  downwards .
 \textit{
	\begin{itemize}
	\item We had stupidly been looking at the wrong column of figures.
	\item In The Dictionary of Quotations, there are no fewer than one and a half columns devoted
to 'kiss'.
	\end{itemize}
}
\item countable noun \\
In a newspaper or magazine , a \textbf{column} is a section that is always  written by the same person or is always about the same topic .
 \textit{
	\begin{itemize}
	\item His name features frequently in the social columns of the tabloid newspapers.
	\item She also writes a regular column for the Times Educational Supplement.
	\end{itemize}
}
\end{enumerate}

\section*{comfortable}
{\large \color{blue}  }
\subsection*{Explain}
\begin{enumerate}
\item adjective \\
If a piece of furniture or an item of clothing is \textbf{comfortable} , it makes you feel physically relaxed when you use it, for example because it is soft .
 \textit{
	\begin{itemize}
	\item ...a comfortable fireside chair.
	\item Trainers are so comfortable to wear.
	\end{itemize}
}
\item adjective \\
If a building or room is \textbf{comfortable} , it makes you feel physically relaxed when you spend time in it, for example because it is warm and has nice furniture.
 \textit{
	\begin{itemize}
	\item A home should be comfortable and friendly.
	\item ...somewhere warm and comfortable.
	\end{itemize}
}
\item adjective \\
If you are \textbf{comfortable} , you are physically relaxed because of the place or position you are sitting or lying in.
 \textit{
	\begin{itemize}
	\item Lie down on your bed and make yourself comfortable.
	\item She tried to manoeuvre her body into a more comfortable position.
	\end{itemize}
}
\item adjective \\
If you say that someone is \textbf{comfortable} , you mean that they have enough money to be able to live without financial  problems .
 \textit{
	\begin{itemize}
	\item 'Is he rich?'—'He's comfortable.'
	\item She came from a stable, comfortable, middle-class family.
	\end{itemize}
}
\item adjective \\
In a race , competition , or election , if you have a \textbf{comfortable}  lead , you are likely to win it easily . If you gain a \textbf{comfortable}  victory or majority , you win easily.
 \textit{
	\begin{itemize}
	\item By half distance we held a comfortable two-lap lead.
	\item He appeared to be heading for a comfortable victory.
	\end{itemize}
}
\item adjective \\
If you feel \textbf{comfortable}  \textbf{with} a particular situation or person, you feel confident and relaxed with them.
 \textit{
	\begin{itemize}
	\item Nervous politicians might well feel more comfortable with a step-by-step approach.
	\item He liked me and I felt comfortable with him.
	\item I'll talk to them, but I won't feel comfortable about it.
	\end{itemize}
}
\item adjective \\
When a sick or injured person is said to be \textbf{comfortable} , they are in a stable physical condition .
 \textit{
	\begin{itemize}
	\item He was described as comfortable in hospital last night.
	\end{itemize}
}
\item adjective \\
A \textbf{comfortable}  life , job , or situation does not cause you any problems or worries .
 \textit{
	\begin{itemize}
	\item ...a comfortable teaching job at a university.
	\item Kohl's retirement looks far from comfortable.
	\end{itemize}
}
\end{enumerate}

\section*{cylinder}
{\large \color{blue}  cylinders  }
\subsection*{Explain}
\begin{enumerate}
\item countable noun \\
A \textbf{cylinder} is an object with flat  circular ends and long straight sides.
 \textit{
	\begin{itemize}
	\item ...a cylinder of foam.
	\item It was recorded on a wax cylinder.
	\end{itemize}
}
\item countable noun \\
A gas \textbf{cylinder} is a cylinder-shaped container in which gas is kept under pressure .
 \textit{
	\begin{itemize}
	\item ...oxygen cylinders.
	\end{itemize}
}
\item countable noun \\
In an engine, a \textbf{cylinder} is a cylinder-shaped part in which a piston moves backwards and forwards .
 \textit{
	\begin{itemize}
	\item ...a 2.5 litre, four-cylinder engine.
	\end{itemize}
}
\end{enumerate}

\section*{common}
{\large \color{blue}  commoner  commonest  commons  }
\subsection*{Explain}
\begin{enumerate}
\item adjective \\
If something is \textbf{common} , it is found in large numbers or it happens often.
 \textit{
	\begin{itemize}
	\item His name was Hansen, a common name in Norway.
	\item Oil pollution is the commonest cause of death for seabirds.
	\item Earthquakes are not common in this part of the world.
	\item It was common practice for prisoners to carve objects from animal bones to pass the
time.
	\end{itemize}
}
\item adjective \\
If something is \textbf{common}  \textbf{to} two or more people or groups, it is done , possessed , or used by them all.
 \textit{
	\begin{itemize}
	\item Moldavians and Romanians share a common language.
	\item Such behaviour is common to all young people.
	\end{itemize}
}
\item adjective \\
When there are more animals or plants of a particular  species than there are of related species, then the first species is called  \textbf{common} .
 \textit{
	\begin{itemize}
	\item ...the common house fly.
	\end{itemize}
}
\item adjective \\
\textbf{Common} is used to indicate that someone or something is of the ordinary kind and not special in any way .
 \textit{
	\begin{itemize}
	\item Common salt is made up of 40% sodium and 60% chloride.
	\end{itemize}
}
\item adjective \\
\textbf{Common}  decency or \textbf{common}  courtesy is the decency or courtesy which most people have. You usually talk about this when someone has not shown these characteristics in their behaviour to show your disapproval of them.
 \textit{
	\begin{itemize}
	\item It is common decency to give your seat to anyone in greater need.
	\item He didn't have the common courtesy to ask permission.
	\end{itemize}
}
\item adjective \\
You can use \textbf{common} to describe  knowledge , an opinion , or a feeling that is shared by people in general .
 \textit{
	\begin{itemize}
	\item It is common knowledge that swimming is one of the best forms of exercise.
	\item ...the common view that acupuncture is only a fringe area of medicine.
	\end{itemize}
}
\item adjective \\
If you describe someone or their behaviour as \textbf{common} , you mean that they show a lack of taste , education , and good manners.
 \textit{
	\begin{itemize}
	\item She might be a little common at times, but she was certainly not boring.
	\end{itemize}
}
\item countable noun \\
A \textbf{common} is an area of grassy land, usually in or near a village or small town , where the public is allowed to go.
 In American  English , \textbf{the commons} is also used.
 \textit{
	\begin{itemize}
	\item We are warning women not to go out on to the common alone.
	\item ...Wimbledon Common.
	\item ...people who have the greatest need for the use of the commons, the public space.
	\end{itemize}
}
\item proper noun \\
\textbf{The Commons} is the same as the House of Commons . The members of the House of Commons can also be referred to as \textbf{the Commons} .
 \textit{
	\begin{itemize}
	\item The Prime Minister is to make a statement in the Commons this afternoon.
	\item The Commons has spent over three months on the bill.
	\end{itemize}
}
\item  \\
 in common \textit{
	\begin{itemize}
	\end{itemize}
}
\item  \\
 in common \textit{
	\begin{itemize}
	\end{itemize}
}
\end{enumerate}

\section*{debt}
{\large \color{blue}  debts  }
\subsection*{Explain}
\begin{enumerate}
\item variable noun \\
A \textbf{debt} is a sum of money that you owe someone.
 \textit{
	\begin{itemize}
	\item Three years later, he is still paying off his debts.
	\item Shrinking economies mean falling tax revenues and more government debt.
	\item ...reducing the country's $18 billion foreign debt.
	\end{itemize}
}
\item uncountable noun \\
\textbf{Debt} is the state of owing money.
 \textit{
	\begin{itemize}
	\item Stress is a main reason for debt.
	\end{itemize}
}
\item countable noun \\
You use \textbf{debt} in expressions such as \textbf{I owe you a debt} or \textbf{I am in your debt} when you are expressing  gratitude for something that someone has done for you.
 \textit{
	\begin{itemize}
	\item He was so good to me that I can never repay the debt I owe him.
	\item I owe a debt of thanks to Joyce Thompson, whose careful and able research was of
great help.
	\item I know I shall feel for ever in her debt.
	\end{itemize}
}
\end{enumerate}

\section*{deficiency}
{\large \color{blue}  deficiencies  }
\subsection*{Explain}
\begin{enumerate}
\item variable noun \\
\textbf{Deficiency}  \textbf{in} something, especially something that your body needs , is not having enough of it.
 \textit{
	\begin{itemize}
	\item They did blood tests on him for signs of vitamin deficiency.
	\item There are serious deficiencies in the numbers of suitable aircraft.
	\end{itemize}
}
\item variable noun \\
A \textbf{deficiency} that someone or something has is a weakness or imperfection in them.
 \textit{
	\begin{itemize}
	\item ...a serious deficiency in our air defence.
	\end{itemize}
}
\end{enumerate}

\section*{cosy}
{\large \color{blue}  cosier  cosiest  }
\subsection*{Explain}
\begin{enumerate}
\item adjective \\
A house or room that is \textbf{cosy} is comfortable and warm.
 \textit{
	\begin{itemize}
	\item Downstairs there's a breakfast room and guests can relax in the cosy bar.
	\end{itemize}
}
\item adjective \\
If you are \textbf{cosy} , you are comfortable and warm.
 \textit{
	\begin{itemize}
	\item They like to make sure their guests are comfortable and cosy.
	\end{itemize}
}
\item adjective \\
You use \textbf{cosy} to describe activities that are pleasant and friendly, and involve people who know each other well .
 \textit{
	\begin{itemize}
	\item ...a cosy chat between friends.
	\item My mood this year is for a cosy, nice and thoroughly wholesome Christmas.
	\end{itemize}
}
\end{enumerate}

\section*{uncomfortable}
{\large \color{blue}  }
\subsection*{Explain}
\begin{enumerate}
\item adjective \\
If you are \textbf{uncomfortable} , you are slightly  worried or embarrassed , and not relaxed and confident .
 \textit{
	\begin{itemize}
	\item The request for money made them feel uncomfortable.
	\item If you are uncomfortable with your counsellor or therapist, you must discuss it.
	\item I feel uncomfortable lying.
	\end{itemize}
}
\item adjective \\
Something that is \textbf{uncomfortable} makes you feel  slight  pain or physical discomfort when you experience it or use it.
 \textit{
	\begin{itemize}
	\item Wigs are hot and uncomfortable to wear constantly.
	\item The Metro journey back to the centre of the town was hot and uncomfortable.
	\item ...an uncomfortable chair.
	\end{itemize}
}
\item adjective \\
If you are \textbf{uncomfortable} , you are not physically content and relaxed, and feel slight pain or discomfort.
 \textit{
	\begin{itemize}
	\item I sometimes feel uncomfortable after eating in the evening.
	\item You may find it uncomfortable to look at bright lights.
	\end{itemize}
}
\item adjective \\
You can describe a situation or fact as \textbf{uncomfortable} when it is difficult to deal with and causes problems and worries.
 \textit{
	\begin{itemize}
	\item It is uncomfortable to think of our own death, but we need to.
	\item Such questions are uncomfortable to answer.
	\item The decree put the president in an uncomfortable position.
	\end{itemize}
}
\end{enumerate}

\section*{dome}
{\large \color{blue}  domes  }
\subsection*{Explain}
\begin{enumerate}
\item countable noun \\
A \textbf{dome} is a round roof.
 \textit{
	\begin{itemize}
	\item ...the dome of St Paul's cathedral.
	\end{itemize}
}
\item countable noun \\
A \textbf{dome} is any object that has a similar shape to a dome.
 \textit{
	\begin{itemize}
	\item ...the dome of the hill.
	\end{itemize}
}
\end{enumerate}

\section*{diligent}
{\large \color{blue}  }
\subsection*{Explain}
\begin{enumerate}
\item adjective \\
Someone who is \textbf{diligent} works hard in a careful and thorough way.
 \textit{
	\begin{itemize}
	\item Meyers is a diligent and prolific worker.
	\item The historical research was impressively diligent.
	\end{itemize}
}
\end{enumerate}

\section*{enemy}
{\large \color{blue}  enemies  }
\subsection*{Explain}
\begin{enumerate}
\item countable noun \\
If someone is your \textbf{enemy} , they hate you or want to harm you.
 \textit{
	\begin{itemize}
	\end{itemize}
}
\item countable noun \\
If someone is your \textbf{enemy} , they are opposed to you and to what you think or do.
 \textit{
	\begin{itemize}
	\item The Government's political enemies were quick to pick up on this series of disasters.
	\end{itemize}
}
\item singular noun \\
\textbf{The}  \textbf{enemy} is an army or other force that is opposed to you in a war, or a country with which your country
is at war.
 \textit{
	\begin{itemize}
	\item The enemy were pursued for two miles.
	\item He searched the skies for enemy bombers.
	\end{itemize}
}
\item countable noun \\
If one thing is the \textbf{enemy of} another thing, the second thing cannot happen or succeed because of the first thing.
 \textit{
	\begin{itemize}
	\item Reform, as we know, is the enemy of revolution.
	\end{itemize}
}
\end{enumerate}

\section*{foolish}
{\large \color{blue}  }
\subsection*{Explain}
\begin{enumerate}
\item adjective \\
If someone's behaviour or action is \textbf{foolish} , it is not sensible and shows a lack of good  judgment .
 \textit{
	\begin{itemize}
	\item It would be foolish to raise hopes unnecessarily.
	\item It is foolish to risk skin cancer.
	\end{itemize}
}
\item adjective \\
If you look or feel  \textbf{foolish} , you look or feel so silly or ridiculous that people are likely to laugh at you.
 \textit{
	\begin{itemize}
	\item I just stood there feeling foolish and watching him.
	\item I didn't want him to look foolish and be laughed at.
	\end{itemize}
}
\end{enumerate}

\section*{feature}
{\large \color{blue}  features  featuring  featured  }
\subsection*{Explain}
\begin{enumerate}
\item countable noun \\
A \textbf{feature}  \textbf{of} something is an interesting or important part or characteristic of it.
 \textit{
	\begin{itemize}
	\item Patriotic songs have long been a feature of Kuwaiti life.
	\item The spacious gardens are a special feature of this property.
	\item Perhaps the most unusual feature in the room is an extraordinary pair of candles.
	\end{itemize}
}
\item plural noun \\
Your \textbf{features} are your eyes , nose, mouth, and other parts of your face.
 \textit{
	\begin{itemize}
	\item His features seemed to change.
	\item Her features were strongly defined.
	\end{itemize}
}
\item verb \\
When something such as a film or exhibition  \textbf{features} a particular person or thing, they are an important part of it.
 \textit{
	\begin{itemize}
	\item It's a great movie and it features a Spanish actor who is going to be a world star
within a year.
	\item The hour-long programme will be updated each week and feature highlights from recent
games.
	\item This spectacular event, now in its 5th year, features a stunning catwalk show.
	\end{itemize}
}
\item verb \\
If someone or something \textbf{features}  \textbf{in} something such as a show , exhibition, or magazine, they are an important part of it.
 \textit{
	\begin{itemize}
	\item Jon featured in one of the show's most thrilling episodes.
	\end{itemize}
}
\item countable noun \\
A \textbf{feature} is a special article in a newspaper or magazine, or a special programme on radio
or television.
 \textit{
	\begin{itemize}
	\item We are delighted to see the Sunday Times running a long feature on breast cancer.
	\item ...a special feature on the fund-raising project.
	\end{itemize}
}
\item countable noun \\
A \textbf{feature} or a \textbf{feature} film or movie is a full-length film about a fictional situation , as opposed to a short film or a documentary .
 \textit{
	\begin{itemize}
	\item ...the first feature-length cartoon, Snow White and the Seven Dwarfs.
	\end{itemize}
}
\item countable noun \\
A geographical \textbf{feature} is something noticeable in a particular area of country, for example a hill , river, or valley .
 \textit{
	\begin{itemize}
	\end{itemize}
}
\end{enumerate}

\section*{formidable}
{\large \color{blue}  }
\subsection*{Explain}
\begin{enumerate}
\item adjective \\
If you describe something or someone as \textbf{formidable} , you mean that you feel  slightly  frightened by them because they are very great or impressive .
 \textit{
	\begin{itemize}
	\item We have a formidable task ahead of us.
	\item Marsalis has a formidable reputation in both jazz and classical music.
	\item She looked every bit as formidable as her mother.
	\end{itemize}
}
\end{enumerate}

\section*{gap}
{\large \color{blue}  gaps  }
\subsection*{Explain}
\begin{enumerate}
\item countable noun \\
A \textbf{gap} is a space between two things or a hole in the middle of something solid .
 \textit{
	\begin{itemize}
	\item He pulled the thick curtains together, leaving just a narrow gap.
	\item ...the wind tearing through gaps in the window frames.
	\end{itemize}
}
\item countable noun \\
A \textbf{gap} is a period of time when you are not busy or when you stop doing something that you normally do.
 \textit{
	\begin{itemize}
	\item There followed a gap of four years, during which William joined the Army.
	\end{itemize}
}
\item countable noun \\
If there is something missing from a situation that prevents it being complete or satisfactory , you can say that there is a \textbf{gap} .
 \textit{
	\begin{itemize}
	\item Hunt has filled the gap left by the departure of Nick Batram.
	\item Like a good businessman, Stewart identified a gap in the market.
	\end{itemize}
}
\item countable noun \\
A \textbf{gap}  \textbf{between} two groups of people, things, or sets of ideas is a big difference between them.
 \textit{
	\begin{itemize}
	\item ...the gap between rich and poor.
	\item America's trade gap widened.
	\item Britain needs to bridge the technology gap between academia and industry.
	\end{itemize}
}
\end{enumerate}

\section*{funny}
{\large \color{blue}  funnier  funniest  funnies  }
\subsection*{Explain}
\begin{enumerate}
\item adjective \\
Someone or something that is \textbf{funny} is amusing and likely to make you smile or laugh .
 \textit{
	\begin{itemize}
	\item Wade was smart and not bad-looking, and he could be funny when he wanted to.
	\item I'll tell you a funny story.
	\end{itemize}
}
\item adjective \\
If you describe something as \textbf{funny} , you think it is strange , surprising , or puzzling .
 \textit{
	\begin{itemize}
	\item Children get some very funny ideas sometimes!
	\item There's something funny about him.
	\item It's funny how love can come and go.
	\end{itemize}
}
\item adjective \\
If you feel  \textbf{funny} , you feel slightly ill.
 \textit{
	\begin{itemize}
	\item My head had begun to ache and my stomach felt funny.
	\end{itemize}
}
\item plural noun \\
\textbf{The funnies} are humorous drawings or a series of humorous drawings in a newspaper or magazine .
 \textit{
	\begin{itemize}
	\end{itemize}
}
\item  \\
 funny business \textit{
	\begin{itemize}
	\end{itemize}
}
\end{enumerate}

\section*{hammer}
{\large \color{blue}  hammers  hammering  hammered  }
\subsection*{Explain}
\begin{enumerate}
\item countable noun \\
A \textbf{hammer} is a tool that consists of a heavy piece of metal at the end of a handle. It is used,
for example, to hit nails into a piece of wood or a wall, or to break things into pieces.
 \textit{
	\begin{itemize}
	\item He used a hammer and chisel to chip away at the wall.
	\end{itemize}
}
\item verb \\
If you \textbf{hammer} an object such as a nail, you hit it with a hammer.
 \textbf{Hammer in} means the same as hammer .
 \textit{
	\begin{itemize}
	\item To avoid damaging the tree, hammer a wooden peg into the hole.
	\item Builders were still hammering outside the window.
	\item The workers kneel on the ground and hammer the small stones in.
	\end{itemize}
}
\item verb \\
If you \textbf{hammer}  \textbf{on} a surface, you hit it several times in order to make a noise , or to emphasize something you are saying when you are angry .
 \textit{
	\begin{itemize}
	\item We had to hammer and shout before they would open up.
	\item A crowd of reporters was hammering on the door.
	\item He hammered his two clenched fists on the table.
	\end{itemize}
}
\item verb \\
If you \textbf{hammer} something such as an idea \textbf{into} people or you \textbf{hammer}  \textbf{at} it, you keep repeating it forcefully so that it will have an effect on people.
 \textit{
	\begin{itemize}
	\item He hammered it into me that I had not suddenly become a rotten goalkeeper.
	\item Recent advertising campaigns from the industry have hammered at these themes.
	\end{itemize}
}
\item verb \\
If you say that someone \textbf{hammers} another person, you mean that they attack, criticize, or punish the other person
severely.
 \textit{
	\begin{itemize}
	\item The report hammers the private motorist.
	\item If we turned up late we would be hammered by everybody.
	\end{itemize}
}
\item passive verb \\
If you say that businesses \textbf{are being hammered} , you mean that they are being unfairly harmed , for example by a change in taxes or by bad  economic conditions.
 \textit{
	\begin{itemize}
	\item Look at the numbers of small businesses that are being hammered unmercifully.
	\item The company has been hammered by the downturn in the construction and motor industries.
	\end{itemize}
}
\item verb \\
In sports, if you say that one player or team \textbf{hammered} another, you mean that the first player or team defeated the second completely and
easily.
 \textit{
	\begin{itemize}
	\item He hammered the young Austrian player in four straight sets.
	\end{itemize}
}
\item verb \\
If someone's heart  \textbf{is hammering} , it is beating very fast , usually because they are frightened .
 \textit{
	\begin{itemize}
	\item My heart was hammering. The footsteps had stopped outside my door.
	\end{itemize}
}
\item countable noun \\
In machines and instruments, a \textbf{hammer} is a part that hits another part. For example, in a gun the hammer causes the explosion which makes the bullet shoot out of it, and in a piano the hammers hit the strings and cause the sounds.
 \textit{
	\begin{itemize}
	\end{itemize}
}
\item countable noun \\
In athletics , a \textbf{hammer} is a heavy weight on a piece of wire, which the athlete throws as far as possible .
 \textbf{The hammer} also refers to the sport of throwing the hammer.
 \textit{
	\begin{itemize}
	\item Events like the hammer and the discus are not traditional crowd-pullers in the West.
	\end{itemize}
}
\item  \\
 hammer and tongs \textit{
	\begin{itemize}
	\end{itemize}
}
\item  \\
 go/come/be under the hammer \textit{
	\begin{itemize}
	\end{itemize}
}
\end{enumerate}

\section*{humble}
{\large \color{blue}  humbler  humblest  humbles  humbling  humbled  }
\subsection*{Explain}
\begin{enumerate}
\item adjective \\
A \textbf{humble} person is not proud and does not believe that they are better than other people.
 \textit{
	\begin{itemize}
	\item He gave a great performance, but he was very humble.
	\item Andy was a humble, courteous and gentle man.
	\item ...a humble apology.
	\end{itemize}
}
\item adjective \\
People with low  social status are sometimes  described as \textbf{humble} .
 \textit{
	\begin{itemize}
	\item Spyros Latsis started his career as a humble fisherman in the Aegean.
	\item He came from a fairly humble, poor background.
	\end{itemize}
}
\item adjective \\
A \textbf{humble} place or thing is ordinary and not special in any way.
 \textit{
	\begin{itemize}
	\item There are restaurants, both humble and expensive, that specialize in them.
	\item Varndell made his own reflector for these shots from a strip of humble kitchen foil.
	\end{itemize}
}
\item adjective \\
People use \textbf{humble} in a phrase such as \textbf{in my humble opinion} as a polite way of emphasizing what they think , even though they do not feel humble about it.
 \textit{
	\begin{itemize}
	\item It is, in my humble opinion, perhaps the best steak restaurant in Great Britain.
	\end{itemize}
}
\item  \\
 to eat humble pie \textit{
	\begin{itemize}
	\end{itemize}
}
\item verb \\
If you \textbf{humble} someone who is more important or powerful than you, you defeat them easily .
 \textit{
	\begin{itemize}
	\item ...the little car company that humbled the industry giants.
	\item The fans could have cried as their team were humbled and humiliated in the first
half.
	\end{itemize}
}
\item verb \\
If something or someone \textbf{humbles} you, they make you realize that you are not as important or good as you thought you were.
 \textit{
	\begin{itemize}
	\item Ted's words humbled me.
	\item I am sure millions of viewers were humbled by this story.
	\end{itemize}
}
\end{enumerate}

\section*{handwriting}
{\large \color{blue}  }
\subsection*{Explain}
\begin{enumerate}
\item uncountable noun \\
Your \textbf{handwriting} is your style of writing with a pen or pencil .
 \textit{
	\begin{itemize}
	\item The address was in Anna's handwriting.
	\item I have to admit that I have bad handwriting.
	\end{itemize}
}
\end{enumerate}

\section*{humid}
{\large \color{blue}  }
\subsection*{Explain}
\begin{enumerate}
\item adjective \\
You use \textbf{humid} to describe an atmosphere or climate that is very damp, and usually very hot .
 \textit{
	\begin{itemize}
	\item Visitors can expect hot and humid conditions.
	\item The day is overcast and humid.
	\end{itemize}
}
\end{enumerate}

\section*{iron}
{\large \color{blue}  irons  ironing  ironed  }
\subsection*{Explain}
\begin{enumerate}
\item uncountable noun \\
\textbf{Iron} is an element which usually takes the form of a hard, dark-grey metal. It is used
to make steel, and also forms part of many tools, buildings, and vehicles. Very small
amounts of iron occur in your blood and in food.
 \textit{
	\begin{itemize}
	\item The huge, iron gate was locked.
	\item ...the highest-grade iron ore deposits in the world.
	\item Some would call these odd pieces of iron and wood 'antiques'.
	\item He was a tall, lanky man with iron-grey hair.
	\end{itemize}
}
\item countable noun \\
An \textbf{iron} is an electrical device with a flat metal base. You heat it until the base is hot, then rub it over clothes to remove creases.
 \textit{
	\begin{itemize}
	\end{itemize}
}
\item verb \\
If you \textbf{iron} clothes, you remove the creases from them using an iron.
 \textit{
	\begin{itemize}
	\item She used to iron his shirts.
	\item ...a freshly ironed shirt.
	\end{itemize}
}
\item adjective \\
You can use \textbf{iron} to describe the character or behaviour of someone who is very firm in their decisions and actions, or who can control their feelings well .
 \textit{
	\begin{itemize}
	\item ...a man of icy nerve and iron will.
	\item She delighted in the nickname, the 'iron lady'.
	\end{itemize}
}
\item adjective \\
\textbf{Iron} is used in expressions such as \textbf{an iron hand} and \textbf{iron discipline} to describe strong, harsh , or unfair methods of control which do not allow people much freedom .
 \textit{
	\begin{itemize}
	\item He died in 1985 after ruling Albania with an iron fist for 40 years.
	\item ...a people living permanently in the iron grip of poverty.
	\end{itemize}
}
\item  \\
 irons in the fire \textit{
	\begin{itemize}
	\end{itemize}
}
\item  \\
 to pump iron \textit{
	\begin{itemize}
	\end{itemize}
}
\end{enumerate}

\section*{humorous}
{\large \color{blue}  }
\subsection*{Explain}
\begin{enumerate}
\item adjective \\
If someone or something is \textbf{humorous} , they are amusing, especially in a clever or witty way.
 \textit{
	\begin{itemize}
	\item He was quite humorous, and I liked that about him.
	\item ...a humorous magazine.
	\end{itemize}
}
\end{enumerate}

\section*{joke}
{\large \color{blue}  jokes  joking  joked  }
\subsection*{Explain}
\begin{enumerate}
\item countable noun \\
A \textbf{joke} is something that is said or done to make you laugh , for example a funny  story .
 \textit{
	\begin{itemize}
	\item He debated whether to make a joke about shooting rabbits, but decided against it.
	\item No one told worse jokes than Claus.
	\end{itemize}
}
\item verb \\
If you \textbf{joke} , you tell funny stories or say  amusing things.
 \textit{
	\begin{itemize}
	\item She would joke about her appearance.
	\item Lorna was laughing and joking with Trevor.
	\item The project was taking so long that Stephen joked that it would never be finished.
	\item 'Well, a beautiful spring Thursday would probably be a nice day to be buried on,'
Nancy joked.
	\end{itemize}
}
\item countable noun \\
A \textbf{joke} is something untrue that you tell another person in order to amuse yourself.
 \textit{
	\begin{itemize}
	\item It was probably just a joke to them, but it wasn't funny to me.
	\end{itemize}
}
\item verb \\
If you \textbf{joke} , you tell someone something that is not true in order to amuse yourself.
 \textit{
	\begin{itemize}
	\item Don't get defensive, Charlie. I was only joking.
	\item 'I wish you made as much fuss of me,' Vera joked.
	\end{itemize}
}
\item singular noun \\
If you say that something or someone is \textbf{a joke} , you think they are ridiculous and do not deserve  respect .
 \textit{
	\begin{itemize}
	\item It's ridiculous, it's pathetic, it's a joke.
	\item The police investigation was a joke. A total cover-up.
	\end{itemize}
}
\item  \\
 beyond a joke \textit{
	\begin{itemize}
	\end{itemize}
}
\item  \\
 make a joke of \textit{
	\begin{itemize}
	\end{itemize}
}
\item  \\
 no joke \textit{
	\begin{itemize}
	\end{itemize}
}
\item  \\
 the joke is on sb \textit{
	\begin{itemize}
	\end{itemize}
}
\item  \\
 can not take a joke \textit{
	\begin{itemize}
	\end{itemize}
}
\item  \\
 you're/you must be/you've got to be joking \textit{
	\begin{itemize}
	\end{itemize}
}
\end{enumerate}

\section*{imaginary}
{\large \color{blue}  }
\subsection*{Explain}
\begin{enumerate}
\item adjective \\
An \textbf{imaginary} person, place, or thing exists only in your mind or in a story , and not in real life.
 \textit{
	\begin{itemize}
	\item Lots of children have imaginary friends.
	\item ...creating an imaginary world.
	\end{itemize}
}
\end{enumerate}

\section*{laughter}
{\large \color{blue}  }
\subsection*{Explain}
\begin{enumerate}
\item uncountable noun \\
\textbf{Laughter} is the sound of people laughing, for example because they are amused or happy .
 \textit{
	\begin{itemize}
	\item Their laughter filled the corridor.
	\item He delivered the line perfectly, and everybody roared with laughter.
	\item ...hysterical laughter.
	\end{itemize}
}
\item uncountable noun \\
\textbf{Laughter} is the fact of laughing, or the feeling of fun and amusement that you have when you are laughing.
 \textit{
	\begin{itemize}
	\item Pantomime is about bringing laughter to thousands.
	\item My interests: eating out, fun nights in, music and laughter.
	\end{itemize}
}
\end{enumerate}

\section*{imaginative}
{\large \color{blue}  }
\subsection*{Explain}
\begin{enumerate}
\item adjective \\
If you describe someone or their ideas as \textbf{imaginative} , you are praising them because they are easily able to think of or create new or exciting things.
 \textit{
	\begin{itemize}
	\item ...an imaginative writer.
	\item ...hundreds of cooking ideas and imaginative recipes.
	\item They should adopt a more imaginative approach.
	\end{itemize}
}
\end{enumerate}

\section*{lens}
{\large \color{blue}  lenses  }
\subsection*{Explain}
\begin{enumerate}
\item countable noun \\
A \textbf{lens} is a thin  curved piece of glass or plastic used in things such as cameras , telescopes , and pairs of glasses. You look through a lens in order to make things look larger, smaller, or clearer.
 \textit{
	\begin{itemize}
	\item ...a camera lens.
	\item I packed your sunglasses with the green lenses.
	\end{itemize}
}
\item countable noun \\
In your eye , the \textbf{lens} is the part behind the pupil that focuses light and helps you to see  clearly .
 \textit{
	\begin{itemize}
	\end{itemize}
}
\end{enumerate}

\section*{indignant}
{\large \color{blue}  }
\subsection*{Explain}
\begin{enumerate}
\item adjective \\
If you are \textbf{indignant} , you are shocked and angry , because you think that something is unjust or unfair .
 \textit{
	\begin{itemize}
	\item He is indignant at suggestions that they were secret agents.
	\item MPs were indignant that the government had not consulted them.
	\item Sheena gave her an indignant look.
	\end{itemize}
}
\end{enumerate}

\section*{notable}
{\large \color{blue}  notables  }
\subsection*{Explain}
\begin{enumerate}
\item adjective \\
Someone or something that is \textbf{notable} is important or interesting.
 \textit{
	\begin{itemize}
	\item The proposed new structure is notable not only for its height, but for its shape.
	\item It is notable that she never allowed the men in her life to eclipse her.
	\item With a few notable exceptions, doctors are a pretty sensible lot.
	\end{itemize}
}
\item countable noun \\
\textbf{Notables} are important or powerful people.
 \textit{
	\begin{itemize}
	\item Elected by local notables for nine years Senators lack the democratic legitimacy
of members of the National Assembly.
	\item The notables include five Senators, two Supreme Court judges and three State Governors.
	\end{itemize}
}
\end{enumerate}

\section*{intense}
{\large \color{blue}  }
\subsection*{Explain}
\begin{enumerate}
\item adjective \\
\textbf{Intense} is used to describe something that is very great or extreme in strength or degree.
 \textit{
	\begin{itemize}
	\item He was sweating from the intense heat.
	\item Suddenly the room filled with intense light.
	\item Stevens's murder was the result of a deep-seated and intense hatred.
	\item His threats become more intense, agitated, and frequent.
	\end{itemize}
}
\item adjective \\
If you describe an activity as \textbf{intense} , you mean that it is very serious and concentrated , and often involves doing a great deal in a short time.
 \textit{
	\begin{itemize}
	\item The battle for third place was intense.
	\item The military on both sides are involved in intense activity.
	\end{itemize}
}
\item adjective \\
If you describe the way someone looks at you as \textbf{intense} , you mean that they look at you very directly and seem to know what you are thinking or feeling.
 \textit{
	\begin{itemize}
	\item I felt so self-conscious under Luke's mother's intense gaze.
	\item He gazed at me with those intense blue eyes.
	\end{itemize}
}
\item adjective \\
If you describe a person as \textbf{intense} , you mean that they appear to concentrate very hard on everything that they do, and they feel and show their emotions in a very extreme way.
 \textit{
	\begin{itemize}
	\item I know he's an intense player, but he does enjoy what he's doing.
	\item She is taller than I imagined, more adult, more intense.
	\end{itemize}
}
\end{enumerate}

\section*{note}
{\large \color{blue}  notes  noting  noted  }
\subsection*{Explain}
\begin{enumerate}
\item countable noun \\
A \textbf{note} is a short letter.
 \textit{
	\begin{itemize}
	\item Stevens wrote him a note asking him to come to his apartment.
	\item I'll have to leave a note for Karen.
	\end{itemize}
}
\item countable noun \\
A \textbf{note} is something that you write down to remind yourself of something.
 \textit{
	\begin{itemize}
	\item I knew that if I didn't make a note I would forget.
	\item Take notes during the consultation as the final written report is very concise.
	\end{itemize}
}
\item countable noun \\
In a book or article , a \textbf{note} is a short piece of additional information.
 \textit{
	\begin{itemize}
	\item See Note 16 on page p. 223.
	\item ...'Exiles' by James Joyce, edited with an Introduction and notes by J C C Mays.
	\end{itemize}
}
\item countable noun \\
A \textbf{note} is a short document that has to be signed by someone and that gives official information about something.
 \textit{
	\begin{itemize}
	\item Since Mr Bennett was going to need some time off work, he asked for a sick note.
	\item I've got half a ton of gravel in the lorry but he won't sign my delivery note.
	\end{itemize}
}
\item countable noun \\
You can refer to a banknote as a \textbf{note} .
 \textit{
	\begin{itemize}
	\item Her husband received a telephone call ordering him to collect £ 40,000 in used notes.
	\item ...a five pound note.
	\end{itemize}
}
\item countable noun \\
In music, a \textbf{note} is the sound of a particular pitch, or a written symbol representing this sound.
 \textit{
	\begin{itemize}
	\item She has a deep voice and doesn't even try for the high notes.
	\item ...the note of D.
	\end{itemize}
}
\item singular noun \\
You can use \textbf{note} to refer to a particular quality in someone's voice that shows how they are feeling.
 \textit{
	\begin{itemize}
	\item There is an unmistakable note of nostalgia in his voice.
	\item It was not difficult for him to catch the note of bitterness in my voice.
	\end{itemize}
}
\item singular noun \\
You can use \textbf{note} to refer to a particular feeling, impression , or atmosphere.
 \textit{
	\begin{itemize}
	\item Yesterday's testimony began on a note of passionate but civilized disagreement.
	\item Somehow he tells these stories without a note of horror.
	\item The furniture strikes a traditional note which is appropriate to its Edwardian setting.
	\end{itemize}
}
\item verb \\
If you \textbf{note} a fact , you become aware of it.
 \textit{
	\begin{itemize}
	\item The White House has noted his promise to support any attack that was designed to
enforce the U.N. resolutions.
	\item Suddenly, I noted that the rain had stopped.
	\item Haig noted how he 'looked pinched and rather tired'.
	\end{itemize}
}
\item verb \\
If you tell someone to \textbf{note} something, you are drawing their attention to it.
 \textit{
	\begin{itemize}
	\item Note the statue to Sallustio Bandini, a prominent Sienese.
	\item Please note that there are a limited number of tickets.
	\end{itemize}
}
\item verb \\
If you \textbf{note} something, you mention it in order to draw people's attention to it.
 \textit{
	\begin{itemize}
	\item The report notes that export and import volumes picked up in leading economies.
	\item The yearbook also noted a sharp drop in reported cases of sexually transmitted disease.
	\end{itemize}
}
\item verb \\
When you \textbf{note} something, you write it down as a record of what has happened .
 \textit{
	\begin{itemize}
	\item 'He has had his tonsils out and has been ill, too', she noted in her diary.
	\item One policeman was clearly visible noting the number plates of passing cars.
	\item A guard came and took our names and noted where each of us was sitting.
	\end{itemize}
}
\item  \\
 to compare notes \textit{
	\begin{itemize}
	\end{itemize}
}
\item  \\
 of note \textit{
	\begin{itemize}
	\end{itemize}
}
\item  \\
 strike a particular note/sound a particular note \textit{
	\begin{itemize}
	\end{itemize}
}
\item  \\
 take note \textit{
	\begin{itemize}
	\end{itemize}
}
\end{enumerate}

\section*{mild}
{\large \color{blue}  milder  mildest  }
\subsection*{Explain}
\begin{enumerate}
\item adjective \\
\textbf{Mild} is used to describe something such as a feeling , attitude , or illness that is not very strong or severe .
 \textit{
	\begin{itemize}
	\item Teddy turned to Mona with a look of mild confusion.
	\item Anna put up a mild protest.
	\item If you have only mild symptoms, try an over-the-counter treatment.
	\end{itemize}
}
\item adjective \\
A \textbf{mild} person is gentle and does not get  angry  easily .
 \textit{
	\begin{itemize}
	\item He is a mild man, who is reasonable almost to the point of blandness.
	\end{itemize}
}
\item adjective \\
\textbf{Mild}  weather is pleasant because it is neither extremely  hot nor extremely cold .
 \textit{
	\begin{itemize}
	\item The area is famous for its very mild winter climate.
	\end{itemize}
}
\item adjective \\
You describe food as \textbf{mild} when it does not taste or smell strong, sharp , or bitter, especially when you like it because of this.
 \textit{
	\begin{itemize}
	\item This cheese has a soft, mild flavour.
	\item ...a mild curry powder.
	\end{itemize}
}
\item graded adjective \\
\textbf{Mild}  soap or washing-up liquid feels pleasant on your skin and does not contain any substances that might make your skin sore .
 \textit{
	\begin{itemize}
	\item Wash your face thoroughly with a mild soap and warm water.
	\end{itemize}
}
\item uncountable noun \\
\textbf{Mild} is a clear , dark-coloured beer.
 \textit{
	\begin{itemize}
	\end{itemize}
}
\end{enumerate}

\section*{notebook}
{\large \color{blue}  notebooks  }
\subsection*{Explain}
\begin{enumerate}
\item countable noun \\
A \textbf{notebook} is a small book for writing notes in.
 \textit{
	\begin{itemize}
	\item He brought out a notebook and pen from his pocket.
	\item ...her reporter's notebook.
	\end{itemize}
}
\item countable noun \\
A \textbf{notebook}  computer is a small personal computer.
 \textit{
	\begin{itemize}
	\item This is a super-slim, lightweight notebook computer.
	\end{itemize}
}
\end{enumerate}

\section*{miserable}
{\large \color{blue}  }
\subsection*{Explain}
\begin{enumerate}
\item adjective \\
If you are \textbf{miserable} , you are very unhappy.
 \textit{
	\begin{itemize}
	\item I took a series of badly paid secretarial jobs which made me really miserable.
	\item She went to bed, miserable and depressed.
	\end{itemize}
}
\item adjective \\
If you describe a place or situation as \textbf{miserable} , you mean that it makes you feel unhappy or depressed.
 \textit{
	\begin{itemize}
	\item There was nothing at all in this miserable place to distract him.
	\end{itemize}
}
\item adjective \\
If you describe the weather as \textbf{miserable} , you mean that it makes you feel depressed, because it is raining or dull .
 \textit{
	\begin{itemize}
	\item It was a grey, wet, miserable day.
	\item It was very cold, damp and miserable.
	\end{itemize}
}
\item adjective \\
If you describe someone as \textbf{miserable} , you mean that you do not like them because they are bad-tempered or unfriendly .
 \textit{
	\begin{itemize}
	\item He always was a miserable man. He never spoke to me nor anybody else.
	\end{itemize}
}
\item adjective \\
You can describe a quantity or quality as \textbf{miserable} when you think that it is much smaller or worse than it ought to be.
 \textit{
	\begin{itemize}
	\item Our speed over the ground was a miserable 2.2 knots.
	\end{itemize}
}
\item adjective \\
A \textbf{miserable}  failure is a very great one.
 \textit{
	\begin{itemize}
	\item The film was a miserable commercial failure both in Italy and in the United States.
	\end{itemize}
}
\end{enumerate}

\section*{overturn}
{\large \color{blue}  overturns  overturning  overturned  }
\subsection*{Explain}
\begin{enumerate}
\item verb \\
If something \textbf{overturns} or if you \textbf{overturn} it, it turns upside down or on its side.
 \textit{
	\begin{itemize}
	\item The lorry veered out of control, overturned and smashed into a wall.
	\item Alex jumped up so violently that he overturned his glass of sherry.
	\item A dozen cartons of books had been overturned and strewn about the floor.
	\item ...a battered overturned boat.
	\end{itemize}
}
\item verb \\
If someone in authority  \textbf{overturns} a legal  decision , they officially  decide that that decision is incorrect or not valid .
 \textit{
	\begin{itemize}
	\item When the Russian parliament overturned his decision, he backed down.
	\item His nine-month sentence was overturned by Appeal Court judge Lord Justice Watkins.
	\end{itemize}
}
\item verb \\
To \textbf{overturn} a government or system means to remove it or destroy it.
 \textit{
	\begin{itemize}
	\item He accused his opponents of wanting to overturn the government.
	\item ...a society where all the old values had been overturned.
	\end{itemize}
}
\end{enumerate}

\section*{modest}
{\large \color{blue}  }
\subsection*{Explain}
\begin{enumerate}
\item adjective \\
A \textbf{modest} house or other building is not large or expensive .
 \textit{
	\begin{itemize}
	\item ...the modest home of a family who lived off the land.
	\item A one-night stay in a modest hotel costs around £35.
	\end{itemize}
}
\item adjective \\
You use \textbf{modest} to describe something such as an amount, rate , or improvement which is fairly small.
 \textit{
	\begin{itemize}
	\item Swiss unemployment rose to the still modest rate of 0.7%.
	\item The democratic reforms have been modest.
	\item You don't get rich, but you can get a modest living out of it.
	\end{itemize}
}
\item adjective \\
If you say that someone is \textbf{modest} , you approve of them because they do not talk much about their abilities or achievements .
 \textit{
	\begin{itemize}
	\item He's modest, as well as being a great player.
	\item She is modest about her achievements.
	\end{itemize}
}
\item adjective \\
You can describe a woman as \textbf{modest} when she avoids doing or wearing anything that might cause men to have sexual  feelings towards her. You can also describe her clothes or behaviour as \textbf{modest} .
 \textit{
	\begin{itemize}
	\item ...cultures in which women are supposed to be modest.
	\end{itemize}
}
\end{enumerate}

\section*{oxygen}
{\large \color{blue}  }
\subsection*{Explain}
\begin{enumerate}
\item uncountable noun \\
\textbf{Oxygen} is a colourless gas that exists in large quantities in the air. All plants and animals need oxygen in order to live .
 \textit{
	\begin{itemize}
	\item The human brain needs to be without oxygen for only four minutes before permanent
damage occurs.
	\end{itemize}
}
\end{enumerate}

\section*{moral}
{\large \color{blue}  morals  }
\subsection*{Explain}
\begin{enumerate}
\item plural noun \\
\textbf{Morals} are principles and beliefs concerning right and wrong behaviour.
 \textit{
	\begin{itemize}
	\item ...Western ideas and morals.
	\item They have no morals.
	\end{itemize}
}
\item adjective \\
\textbf{Moral}  means relating to beliefs about what is right or wrong.
 \textit{
	\begin{itemize}
	\item She describes her own moral dilemma in making the film.
	\item ...matters of church doctrine and moral teaching.
	\item ...the moral issues involved in 'playing God'.
	\end{itemize}
}
\item adjective \\
\textbf{Moral}  courage or duty is based on what you believe is right or acceptable , rather than on what the law  says should be done .
 \textit{
	\begin{itemize}
	\item The Government had a moral, if not a legal duty to pay compensation.
	\item ...his moral courage and sane defence of his philosophy.
	\end{itemize}
}
\item adjective \\
A \textbf{moral} person behaves in a way that is believed by most people to be good and right.
 \textit{
	\begin{itemize}
	\item The people who will be on the committee are moral, cultured, competent people.
	\end{itemize}
}
\item adjective \\
If you give someone \textbf{moral}  support , you encourage them in what they are doing by expressing  approval .
 \textit{
	\begin{itemize}
	\item Moral as well as financial support was what the West should provide.
	\end{itemize}
}
\item countable noun \\
\textbf{The}  \textbf{moral} of a story or event is what you learn from it about how you should or should not behave.
 \textit{
	\begin{itemize}
	\item I think the moral of the story is let the buyer beware.
	\item The moral is that, once cooked, they look the same and taste every bit as good.
	\end{itemize}
}
\end{enumerate}

\section*{passenger}
{\large \color{blue}  passengers  }
\subsection*{Explain}
\begin{enumerate}
\item countable noun \\
A \textbf{passenger} in a vehicle such as a bus , boat, or plane is a person who is travelling in it, but who is not driving it or working on it.
 \textit{
	\begin{itemize}
	\item Mr Fullemann was a passenger in the car when it crashed.
	\item ...a flight from Milan with more than forty passengers on board.
	\end{itemize}
}
\item adjective \\
\textbf{Passenger} is used to describe something that is designed for passengers, rather than for drivers or goods.
 \textit{
	\begin{itemize}
	\item I sat in the passenger seat.
	\item ...a passenger train.
	\end{itemize}
}
\end{enumerate}

\section*{obscure}
{\large \color{blue}  obscurer  obscurest  obscures  obscuring  obscured  }
\subsection*{Explain}
\begin{enumerate}
\item adjective \\
If something or someone is \textbf{obscure} , they are unknown , or are known by only a few people.
 \textit{
	\begin{itemize}
	\item The origin of the custom is obscure.
	\item The hymn was written by an obscure Greek composer.
	\end{itemize}
}
\item adjective \\
Something that is \textbf{obscure} is difficult to understand or deal with, usually because it involves so many parts or details .
 \textit{
	\begin{itemize}
	\item The contracts are written in obscure language.
	\item Richard's statement was disgracefully obscure.
	\end{itemize}
}
\item verb \\
If one thing \textbf{obscures} another, it prevents it from being seen or heard properly.
 \textit{
	\begin{itemize}
	\item Trees obscured his vision; he couldn't see much of the Square's southern half.
	\item One wall of the parliament building is now almost completely obscured by a huge banner.
	\end{itemize}
}
\item verb \\
To \textbf{obscure} something means to make it difficult to understand.
 \textit{
	\begin{itemize}
	\item ...the jargon that frequently obscures educational writing.
	\item This issue has been obscured by recent events.
	\end{itemize}
}
\end{enumerate}

\section*{pencil}
{\large \color{blue}  pencils  pencilling  pencilled  }
\subsection*{Explain}
\begin{enumerate}
\item countable noun \\
A \textbf{pencil} is an object that you write or draw with. It consists of a thin piece of wood with a rod of a
 black or coloured substance through the middle . If you write or draw something \textbf{in pencil} , you do it using a pencil.
 \textit{
	\begin{itemize}
	\item I found a pencil and some blank paper in her desk.
	\item He had written her a note in pencil.
	\end{itemize}
}
\item verb \\
If you \textbf{pencil} a letter or a note, you write it using a pencil.
 \textit{
	\begin{itemize}
	\item He pencilled a note to Joseph Daniels.
	\end{itemize}
}
\end{enumerate}

\section*{ordinary}
{\large \color{blue}  }
\subsection*{Explain}
\begin{enumerate}
\item adjective \\
\textbf{Ordinary} people or things are normal and not special or different in any way.
 \textit{
	\begin{itemize}
	\item I strongly suspect that most ordinary people would agree with me.
	\item It has 25 calories less than ordinary ice cream.
	\item It was just an ordinary weekend for us.
	\end{itemize}
}
\item graded adjective \\
If you describe someone or something as \textbf{ordinary} , you mean they are not special or interesting in any way and may be rather dull .
 \textit{
	\begin{itemize}
	\item I'm just a very ordinary, boring normal guy.
	\item Your life since then must have seemed very ordinary.
	\item ...very ordinary, if very well made, drinking glasses, lamps and tableware.
	\end{itemize}
}
\item  \\
 out of the ordinary \textit{
	\begin{itemize}
	\end{itemize}
}
\end{enumerate}

\section*{perspective}
{\large \color{blue}  perspectives  }
\subsection*{Explain}
\begin{enumerate}
\item countable noun \\
A particular \textbf{perspective} is a particular way of thinking about something, especially one that is influenced by your beliefs or experiences .
 \textit{
	\begin{itemize}
	\item He says the death of his father 18 months ago has given him a new perspective on
life.
	\item ...two different perspectives on the nature of adolescent development.
	\item Most literature on the subject of immigrants in France has been written from the
perspective of the French themselves.
	\item I would like to offer a historical perspective.
	\end{itemize}
}
\item  \\
 in perspective/into perspective/out of perspective \textit{
	\begin{itemize}
	\end{itemize}
}
\item uncountable noun \\
\textbf{Perspective} is the art of making some objects or people in a picture look further away than others.
 \textit{
	\begin{itemize}
	\end{itemize}
}
\end{enumerate}

\section*{plentiful}
{\large \color{blue}  }
\subsection*{Explain}
\begin{enumerate}
\item adjective \\
Things that are \textbf{plentiful} exist in such large amounts or numbers that there is enough for people's wants or needs .
 \textit{
	\begin{itemize}
	\item Fish are plentiful in the lake.
	\item ...a plentiful supply of vegetables and salads and fruits.
	\end{itemize}
}
\end{enumerate}

\section*{plot}
{\large \color{blue}  plots  plotting  plotted  }
\subsection*{Explain}
\begin{enumerate}
\item countable noun \\
A \textbf{plot} is a secret plan by a group of people to do something that is illegal or wrong , usually against a person or a government.
 \textit{
	\begin{itemize}
	\item Security forces have uncovered a plot to overthrow the government.
	\item He was responding to reports of an assassination plot against him.
	\end{itemize}
}
\item verb \\
If people \textbf{plot}  \textbf{to} do something or \textbf{plot} something that is illegal or wrong, they plan secretly to do it.
 \textit{
	\begin{itemize}
	\item Prosecutors in the trial allege the defendants plotted to overthrow the government.
	\item The military were plotting a coup.
	\item They are awaiting trial on charges of plotting against the state.
	\end{itemize}
}
\item verb \\
When people \textbf{plot} a strategy or a course of action, they carefully plan each step of it.
 \textit{
	\begin{itemize}
	\item Yesterday's meeting was intended to plot a survival strategy for the party.
	\item For the next five years she plotted her career.
	\end{itemize}
}
\item variable noun \\
The \textbf{plot} of a film, novel, or play is the connected series of events which make up the story.
 \textit{
	\begin{itemize}
	\end{itemize}
}
\item countable noun \\
A \textbf{plot}  \textbf{of} land is a small piece of land, especially one that has been measured or marked out for a special purpose, such as building houses or growing vegetables .
 \textit{
	\begin{itemize}
	\item I thought that I'd buy myself a small plot of land and build a house on it.
	\item The bottom of the garden was given over to vegetable plots.
	\end{itemize}
}
\item verb \\
When someone \textbf{plots} something on a graph, they mark certain points on it and then join the points up.
 \textit{
	\begin{itemize}
	\item We plot about eight points on the graph.
	\end{itemize}
}
\item verb \\
When someone \textbf{plots} the position or course of a plane or ship, they mark it on a map using instruments to obtain accurate information.
 \textit{
	\begin{itemize}
	\item We were trying to plot the course of the submarine.
	\end{itemize}
}
\item verb \\
If someone \textbf{plots} the progress or development of something, they make a diagram or a plan which shows how it has developed in order
to give some indication of how it will develop in the future .
 \textit{
	\begin{itemize}
	\item They used a computer to plot the movements of everyone in the building.
	\end{itemize}
}
\item  \\
 to lose the plot \textit{
	\begin{itemize}
	\end{itemize}
}
\end{enumerate}

\section*{powerful}
{\large \color{blue}  }
\subsection*{Explain}
\begin{enumerate}
\item adjective \\
A \textbf{powerful} person or organization is able to control or influence people and events .
 \textit{
	\begin{itemize}
	\item You're a powerful man–people will listen to you.
	\item ...Russia and India, two large, powerful countries.
	\item ...Hong Kong's powerful business community.
	\end{itemize}
}
\item adjective \\
You say that someone's body is \textbf{powerful} when it is physically strong .
 \textit{
	\begin{itemize}
	\item Hans flexed his powerful muscles.
	\item It's such a big powerful dog.
	\end{itemize}
}
\item adjective \\
A \textbf{powerful}  machine or substance is effective because it is very strong.
 \textit{
	\begin{itemize}
	\item The more powerful the car the more difficult it is to handle.
	\item ...powerful computer systems.
	\item ...a powerful magnet.
	\end{itemize}
}
\item adjective \\
A \textbf{powerful}  smell is very strong.
 \textit{
	\begin{itemize}
	\item There was a powerful smell of garlic.
	\item ...tiny creamy flowers with a powerful scent.
	\end{itemize}
}
\item adjective \\
A \textbf{powerful}  voice is loud and can be heard from a long way  away .
 \textit{
	\begin{itemize}
	\item At that moment the housekeeper's powerful voice interrupted them, announcing a visitor.
	\end{itemize}
}
\item adjective \\
You describe a piece of writing , speech , or work of art as \textbf{powerful} when it has a strong effect on people's feelings or beliefs .
 \textit{
	\begin{itemize}
	\item ...Bleasdale's powerful 11-part drama about a corrupt city leader.
	\item ...one of the world's most powerful and moving operas, Verdi's 'Otello'.
	\item ...a powerful new style of dance-theatre.
	\end{itemize}
}
\end{enumerate}

\section*{pretext}
{\large \color{blue}  pretexts  }
\subsection*{Explain}
\begin{enumerate}
\item countable noun \\
A \textbf{pretext} is a reason which you pretend has caused you to do something.
 \textit{
	\begin{itemize}
	\item They wanted a pretext for subduing the region by force.
	\item He excused himself on the pretext of a stomach upset.
	\item They would now find some dubious pretext to restart the war.
	\end{itemize}
}
\end{enumerate}

\section*{privilege}
{\large \color{blue}  privileges  privileging  privileged  }
\subsection*{Explain}
\begin{enumerate}
\item countable noun \\
A \textbf{privilege} is a special right or advantage that only one person or group has.
 \textit{
	\begin{itemize}
	\item The Russian Federation has issued a decree abolishing special privileges for government
officials.
	\item ...the ancient powers and privileges of the House of Commons.
	\end{itemize}
}
\item uncountable noun \\
If you talk about \textbf{privilege} , you are talking about the power and advantage that only a small group of people
have, usually because of their wealth or their high social class.
 \textit{
	\begin{itemize}
	\item Pironi was the son of privilege and wealth, and it showed.
	\item Having been born to privilege in old Hollywood, she was carrying on a family tradition
by acting.
	\end{itemize}
}
\item singular noun \\
You can use \textbf{privilege} in expressions such as \textbf{be a privilege} or \textbf{have the privilege} when you want to show your appreciation of someone or something or to show your respect .
 \textit{
	\begin{itemize}
	\item It must be a privilege to know such a man.
	\item I had the privilege of meeting Mandela at the only service of the Order of Merit
he attended.
	\end{itemize}
}
\item verb \\
To \textbf{privilege} someone or something means to treat them better or differently than other people or things rather than treat them all equally.
 \textit{
	\begin{itemize}
	\item ...privileging a structure that rewards the fastest, strongest, and wealthiest among
us.
	\item They are privileging a tiny number to the disadvantage of the rest.
	\end{itemize}
}
\end{enumerate}

\section*{robust}
{\large \color{blue}  }
\subsection*{Explain}
\begin{enumerate}
\item adjective \\
Someone or something that is \textbf{robust} is very strong or healthy .
 \textit{
	\begin{itemize}
	\item More women than men go to the doctor. Perhaps men are more robust or worry less?
	\item We've always specialised in making very robust, simply designed machinery.
	\end{itemize}
}
\item adjective \\
\textbf{Robust}  views or opinions are strongly held and forcefully expressed.
 \textit{
	\begin{itemize}
	\item A British Foreign Office minister has made a robust defence of the agreement.
	\item He has the keen eye and robust approach needed.
	\end{itemize}
}
\end{enumerate}

\section*{purse}
{\large \color{blue}  purses  pursing  pursed  }
\subsection*{Explain}
\begin{enumerate}
\item countable noun \\
A \textbf{purse} is a very small bag that people, especially women, keep their money in.
 \textit{
	\begin{itemize}
	\end{itemize}
}
\item countable noun \\
A \textbf{purse} is a small bag that women carry.
 \textit{
	\begin{itemize}
	\item She reached in her purse for her phone.
	\end{itemize}
}
\item singular noun \\
\textbf{Purse} is used to refer to the total amount of money that a country, family, or group has.
 \textit{
	\begin{itemize}
	\item The money could simply go into the public purse, helping to lower taxes.
	\end{itemize}
}
\item verb \\
If you \textbf{purse} your \textbf{lips} , you move them into a small, rounded shape, usually because you disapprove of something or when you are thinking .
 \textit{
	\begin{itemize}
	\item She pursed her lips in disapproval.
	\end{itemize}
}
\end{enumerate}

\section*{slippery}
{\large \color{blue}  }
\subsection*{Explain}
\begin{enumerate}
\item adjective \\
Something that is \textbf{slippery} is smooth , wet , or oily and is therefore difficult to walk on or to hold .
 \textit{
	\begin{itemize}
	\item The tiled floor was wet and slippery.
	\item Motorists were warned to beware of slippery conditions.
	\end{itemize}
}
\item adjective \\
You can describe someone as \textbf{slippery} if you think that they are dishonest in a clever way and cannot be trusted .
 \textit{
	\begin{itemize}
	\item He is a slippery customer, and should be carefully watched.
	\end{itemize}
}
\item  \\
 slippery slope \textit{
	\begin{itemize}
	\end{itemize}
}
\end{enumerate}

\section*{rust}
{\large \color{blue}  rusts  rusting  rusted  }
\subsection*{Explain}
\begin{enumerate}
\item uncountable noun \\
\textbf{Rust} is a brown substance that forms on iron or steel, for example when it comes into contact with water.
 \textit{
	\begin{itemize}
	\item ...a decaying tractor, red with rust.
	\item Manufacturers are looking into building cars out of plastic to avoid the problem
of rust.
	\end{itemize}
}
\item verb \\
When a metal object \textbf{rusts} , it becomes covered in rust and often loses its strength .
 \textit{
	\begin{itemize}
	\item Copper nails are better than iron nails because the iron rusts.
	\item There was an old rusting bolt on the door.
	\end{itemize}
}
\item colour \\
\textbf{Rust} is sometimes used to describe things that are reddish-brown in colour.
 \textit{
	\begin{itemize}
	\item ...turquoise woodwork with accent colours of rust and ochre.
	\item ...a rust-coloured blouse.
	\end{itemize}
}
\item uncountable noun \\
\textbf{Rust} is a disease which affects plants. It is caused by a fungus.
 \textit{
	\begin{itemize}
	\end{itemize}
}
\end{enumerate}

\section*{strong}
{\large \color{blue}  stronger  strongest  }
\subsection*{Explain}
\begin{enumerate}
\item adjective \\
Someone who is \textbf{strong} is healthy with good muscles and can move or carry heavy things, or do hard physical work.
 \textit{
	\begin{itemize}
	\item I'm not strong enough to carry him.
	\item I feared I wouldn't be able to control such a strong horse.
	\end{itemize}
}
\item adjective \\
Someone who is \textbf{strong} is confident and determined, and is not easily influenced or worried by other people.
 \textit{
	\begin{itemize}
	\item He is sharp and manipulative with a strong personality.
	\item It's up to managers to be strong and do what they believe is right.
	\item Eventually I felt strong enough to look at him.
	\end{itemize}
}
\item adjective \\
\textbf{Strong} objects or materials are not easily broken and can support a lot of weight or resist a lot of strain .
 \textit{
	\begin{itemize}
	\item The vacuum flask has a strong casing, which won't crack or chip.
	\item Glue the mirror in with a strong adhesive.
	\item The fabric is strong enough to withstand harsh processing.
	\end{itemize}
}
\item adjective \\
A \textbf{strong} wind, current, or other force has a lot of power or speed, and can cause heavy things
to move.
 \textit{
	\begin{itemize}
	\item Strong winds and torrential rain combined to make conditions terrible for golfers
in the Scottish Open.
	\item A fairly strong current seemed to be moving the whole boat.
	\item A neutron star has a gravitational field strong enough to generate X-rays.
	\end{itemize}
}
\item adjective \\
A \textbf{strong}  impression or influence has a great effect on someone.
 \textit{
	\begin{itemize}
	\item We're glad if our music makes a strong impression, even if it's a negative one.
	\item There will be a strong incentive to enter into a process of negotiation.
	\item Teenage idols have a strong influence on our children.
	\item We had strong family traditions; we couldn't escape them.
	\end{itemize}
}
\item adjective \\
If you have \textbf{strong} opinions on something or express them using \textbf{strong} words, you have extreme or very definite opinions which you are willing to express or defend .
 \textit{
	\begin{itemize}
	\item It was hard to find a jury who did not already hold strong views on the tragedy.
	\item There has been strong criticism of the military regime.
	\item I am a strong supporter of the NHS.
	\item The newspaper condemned the campaign in extremely strong language.
	\item It's bad judgment, but it's not treason. I think treason is too strong a word.
	\end{itemize}
}
\item adjective \\
If someone in authority takes \textbf{strong} action, they act firmly and severely.
 \textit{
	\begin{itemize}
	\item The government has said it will take strong action against any further strikes.
	\item He has also said he will have to become a strong President to put things right.
	\end{itemize}
}
\item adjective \\
If there is a \textbf{strong} case or argument for something, it is supported by a lot of evidence .
 \textit{
	\begin{itemize}
	\item The testimony presented offered a strong case for acquitting her on grounds of self-defense.
	\item The evidence that such investment promotes growth is strong.
	\end{itemize}
}
\item adjective \\
If there is a \textbf{strong}  possibility or chance that something is true or will happen , it is very likely to be true or to happen.
 \textit{
	\begin{itemize}
	\item There is a strong possibility that the cat ate contaminated pet food.
	\end{itemize}
}
\item adjective \\
Your \textbf{strong} points are your best qualities or talents , or the things you are good at.
 \textit{
	\begin{itemize}
	\item Discretion is not Jeremy's strong point.
	\item Even in the area of foreign policy, his strong suit, he failed to deliver.
	\item Cynics argue that the E.U. is stronger on rhetoric than on concrete action.
	\end{itemize}
}
\item adjective \\
A \textbf{strong}  competitor , candidate , or team is good or likely to succeed .
 \textit{
	\begin{itemize}
	\item She was a strong contender for Britain's rowing team.
	\item Some countries where English is becoming a strong competitor to French, such as Algeria
and Senegal, may see an immediate halt to the council's work.
	\item They've got a strong squad and some great players.
	\item This show has several strengths–notably a strong cast.
	\end{itemize}
}
\item adjective \\
If a relationship or link is \textbf{strong} , it is close and likely to last for a long time.
 \textit{
	\begin{itemize}
	\item He felt he had a relationship strong enough to talk frankly to Sarah.
	\item This has tested our marriage, and we have come through it stronger than ever.
	\item It is fairly easy for executive directors to develop strong ties with investors.
	\end{itemize}
}
\item adjective \\
A \textbf{strong} currency, economy , or industry has a high value or is very successful .
 \textit{
	\begin{itemize}
	\item Investment performance was strong across the board last year.
	\item The local economy is strong and the population is growing.
	\item The company was not financially strong enough to be floated on the Stock Exchange.
	\end{itemize}
}
\item adjective \\
If something is a \textbf{strong} element or part of something else, it is an important or large part of it.
 \textit{
	\begin{itemize}
	\item We are especially encouraged by the strong representation of women in technology.
	\item There is a strong element of truth to each of these explanations.
	\end{itemize}
}
\item adjective \\
You can use \textbf{strong} when you are saying how many people there are in a group. For example, if a group is twenty strong, there are twenty people in it.
 \textit{
	\begin{itemize}
	\item Ukraine indicated that it would establish its own army, 400,000 strong.
	\item ...a 1,000-strong crowd.
	\end{itemize}
}
\item adjective \\
A \textbf{strong} drink, chemical, or drug contains a lot of the particular substance which makes it
effective.
 \textit{
	\begin{itemize}
	\item Strong coffee or tea late at night may cause sleeplessness.
	\item In strong concentrations it can cause nausea and vomiting.
	\end{itemize}
}
\item adjective \\
A \textbf{strong} colour, flavour, smell, sound, or light is intense and easily noticed .
 \textit{
	\begin{itemize}
	\item As she went past there was a gust of strong perfume.
	\item Strong colours would flatter her pale skin and dark hair.
	\item Munster is among the strongest cheeses in France.
	\end{itemize}
}
\item adjective \\
If someone has a \textbf{strong} accent, they speak in a distinctive way that shows very clearly what country or region they come from.
 \textit{
	\begin{itemize}
	\item 'Good, Mr Ryle,' he said in English with a strong French accent.
	\end{itemize}
}
\item adjective \\
You can say that someone has \textbf{strong} features or a \textbf{strong} face if their face has large, distinctive features.
 \textit{
	\begin{itemize}
	\item He had a strong nose and olive-black eyes.
	\end{itemize}
}
\item  \\
 come on strong \textit{
	\begin{itemize}
	\end{itemize}
}
\item  \\
 going strong \textit{
	\begin{itemize}
	\end{itemize}
}
\end{enumerate}

\section*{sacrifice}
{\large \color{blue}  sacrifices  sacrificing  sacrificed  }
\subsection*{Explain}
\begin{enumerate}
\item verb \\
If you \textbf{sacrifice} something that is valuable or important , you give it up, usually to obtain something else for yourself or for other people.
 \textbf{Sacrifice} is also a noun .
 \textit{
	\begin{itemize}
	\item They sacrificed family life to their careers.
	\item Her pride was a small thing to sacrifice for their children's security.
	\item Kitty Aldridge has sacrificed all for her first film.
	\item He sacrificed himself and so saved his country.
	\item She made many sacrifices to get Anita a good education.
	\item He was willing to make any sacrifice for peace.
	\end{itemize}
}
\item verb \\
To \textbf{sacrifice} an animal or person means to kill them in a special religious ceremony as an offering to a god .
 \textbf{Sacrifice} is also a noun.
 \textit{
	\begin{itemize}
	\item The priest sacrificed a chicken.
	\item Two white bulls were sacrificed and a feast was held.
	\item ...animal sacrifices to the gods.
	\end{itemize}
}
\end{enumerate}

\section*{stupid}
{\large \color{blue}  stupider  stupidest  }
\subsection*{Explain}
\begin{enumerate}
\item adjective \\
If you say that someone or something is \textbf{stupid} , you mean that they show a lack of good judgment or intelligence and they are not at all sensible .
 \textit{
	\begin{itemize}
	\item I'll never do anything so stupid again.
	\item I made a stupid mistake.
	\item Your father wouldn't have asked such a stupid question.
	\item If you give him half a chance he can make you look stupid.
	\end{itemize}
}
\item adjective \\
You say that something is \textbf{stupid} to indicate that you do not like it or care about it, or that it annoys you.
 \textit{
	\begin{itemize}
	\item I wouldn't call it art. It's just stupid and tasteless.
	\item Friendship is much more important to me than a stupid old ring!
	\end{itemize}
}
\end{enumerate}

\section*{secretary}
{\large \color{blue}  secretaries  }
\subsection*{Explain}
\begin{enumerate}
\item countable noun \\
A \textbf{secretary} is a person who is employed to do office work, such as typing letters , answering  phone  calls , and arranging  meetings .
 \textit{
	\begin{itemize}
	\end{itemize}
}
\item countable noun \\
The \textbf{secretary} of an organization such as a trade  union , a political  party , or a club is its official manager.
 \textit{
	\begin{itemize}
	\item My grandfather was secretary of the Scottish Miners' Union.
	\end{itemize}
}
\item countable noun \\
The \textbf{secretary} of a company is the person who has the legal  duty of keeping the company's records.
 \textit{
	\begin{itemize}
	\end{itemize}
}
\item countable noun \\
\textbf{Secretary} is used in the titles of ministers and officials who are in charge of main government departments.
 \textit{
	\begin{itemize}
	\item ...the British Foreign Secretary.
	\item ...the former US Defense Secretary.
	\end{itemize}
}
\end{enumerate}

\section*{unexpected}
{\large \color{blue}  }
\subsection*{Explain}
\begin{enumerate}
\item adjective \\
If an event or someone's behaviour is \textbf{unexpected} , it surprises you because you did not think that it was likely to happen .
 \textit{
	\begin{itemize}
	\item His death was totally unexpected.
	\item He made a brief, unexpected appearance at the office.
	\item Help may also come from some unexpected places.
	\item 'Hello,' he said. 'This is an unexpected pleasure.'
	\end{itemize}
}
\end{enumerate}

\section*{spade}
{\large \color{blue}  spades  }
\subsection*{Explain}
\begin{enumerate}
\item countable noun \\
A \textbf{spade} is a tool used for digging, with a flat metal blade and a long handle.
 \textit{
	\begin{itemize}
	\item ... a garden spade.
	\item The girls happily played in the sand with buckets and spades.
	\end{itemize}
}
\item uncountable noun \\
\textbf{Spades} is one of the four suits in a pack of playing cards. Each card in the suit is marked with one or more black symbols:
♠.
 A \textbf{spade} is a playing card of this suit.
 \textit{
	\begin{itemize}
	\item ...the ace of spades.
	\end{itemize}
}
\item  \\
 to call a spade a spade \textit{
	\begin{itemize}
	\end{itemize}
}
\end{enumerate}

\section*{universal}
{\large \color{blue}  universals  }
\subsection*{Explain}
\begin{enumerate}
\item adjective \\
Something that is \textbf{universal} relates to everyone in the world or everyone in a particular group or society .
 \textit{
	\begin{itemize}
	\item The insurance industry has produced its own proposals for universal health care.
	\item The desire to look attractive is universal.
	\end{itemize}
}
\item adjective \\
Something that is \textbf{universal} affects or relates to every part of the world or the universe.
 \textit{
	\begin{itemize}
	\item ...universal diseases.
	\end{itemize}
}
\item countable noun \\
A \textbf{universal} is a principle that applies in all cases or a characteristic that is present in all members of a particular class.
 \textit{
	\begin{itemize}
	\item There are no economic universals.
	\end{itemize}
}
\end{enumerate}

\section*{steward}
{\large \color{blue}  stewards  }
\subsection*{Explain}
\begin{enumerate}
\item countable noun \\
A \textbf{steward} is a man who works on a ship, plane , or train, looking after passengers and serving  meals to them.
 \textit{
	\begin{itemize}
	\end{itemize}
}
\item countable noun \\
A \textbf{steward} is someone who has the responsibility for looking after property.
 \textit{
	\begin{itemize}
	\item The Earl didn't have the money or good judgement to employ a steward to manage the
place for him.
	\end{itemize}
}
\item countable noun \\
A \textbf{steward} is a man or woman who helps to organize a race , march , or other public event.
 \textit{
	\begin{itemize}
	\item The steward at the march stood his ground while the rest of the marchers decided
to run.
	\end{itemize}
}
\end{enumerate}

\section*{warm}
{\large \color{blue}  warmer  warmest  warms  warming  warmed  }
\subsection*{Explain}
\begin{enumerate}
\item adjective \\
Something that is \textbf{warm} has some heat but not enough to be hot.
 \textit{
	\begin{itemize}
	\item Wheat is grown in places which have cold winters and warm, dry summers.
	\item Because it was warm, David wore only a white cotton shirt.
	\item Dissolve the salt in the warm water.
	\end{itemize}
}
\item adjective \\
\textbf{Warm} clothes and blankets are made of a material such as wool which protects you from the cold .
 \textit{
	\begin{itemize}
	\item They have been forced to sleep in the open without food or warm clothing.
	\item The bed had clean sheets and warm blankets.
	\end{itemize}
}
\item adjective \\
\textbf{Warm} colours have red or yellow in them rather than blue or green, and make you feel comfortable and relaxed .
 \textit{
	\begin{itemize}
	\item We hope the colour gives the house a warm and inviting feel.
	\item The basement hallway is painted a warm yellow.
	\end{itemize}
}
\item adjective \\
A \textbf{warm} person is friendly and shows a lot of affection or enthusiasm in their behaviour.
 \textit{
	\begin{itemize}
	\item She was a warm and loving mother.
	\item His familiar warm voice made everybody who knew him feel welcome.
	\item I would like to express my warmest thanks to the doctors.
	\end{itemize}
}
\item verb \\
If you \textbf{warm} a part of your body or if something hot \textbf{warms} it, it stops feeling cold and starts to feel hotter.
 \textit{
	\begin{itemize}
	\item The sun had come out to warm his back.
	\item She went to warm her hands by the log fire.
	\end{itemize}
}
\item verb \\
If you \textbf{warm to} a person or an idea, you become fonder of the person or more interested in the idea.
 \textit{
	\begin{itemize}
	\item Those who got to know him better warmed to his openness and honesty.
	\item Elizabeth warmed to her theme as the letter continued with her favourite lament.
	\end{itemize}
}
\end{enumerate}

\section*{trait}
{\large \color{blue}  traits  }
\subsection*{Explain}
\begin{enumerate}
\item countable noun \\
A \textbf{trait} is a particular characteristic, quality, or tendency that someone or something has.
 \textit{
	\begin{itemize}
	\item Many of our personality traits are developed during those early months.
	\item Creativity is a human trait.
	\end{itemize}
}
\end{enumerate}

\section*{wealthy}
{\large \color{blue}  wealthier  wealthiest  }
\subsection*{Explain}
\begin{enumerate}
\item adjective \\
Someone who is \textbf{wealthy} has a large amount of money, property, or valuable  possessions .
 \textbf{The wealthy} are people who are wealthy.
 \textit{
	\begin{itemize}
	\item ...a wealthy international businessman.
	\item ...a measure to raise income taxes on the wealthy.
	\end{itemize}
}
\end{enumerate}

\section*{trend}
{\large \color{blue}  trends  trending  trended  }
\subsection*{Explain}
\begin{enumerate}
\item countable noun \\
A \textbf{trend} is a change or development towards something new or different.
 \textit{
	\begin{itemize}
	\item This is a growing trend.
	\item ...a trend towards part-time employment.
	\item ...the downward trend in gasoline prices.
	\end{itemize}
}
\item countable noun \\
To set a \textbf{trend} means to do something that becomes accepted or fashionable , and that a lot of other people copy .
 \textit{
	\begin{itemize}
	\item The record has already proved a success and may well start a trend.
	\end{itemize}
}
\item verb \\
If something such as a topic or name \textbf{is trending} , a lot of people are discussing or mentioning it on social media.
 \textit{
	\begin{itemize}
	\item Minutes after the announcement, Heidi was trending on Twitter.
	\item The hashtag #RoyalBaby became the world's number 1 trending topic.
	\end{itemize}
}
\end{enumerate}

\section*{wet}
{\large \color{blue}  wetter  wettest  wets  wetting  wetted  }
\subsection*{Explain}
\begin{enumerate}
\item adjective \\
If something is \textbf{wet} , it is covered in water, rain , sweat , tears , or another liquid.
 \textit{
	\begin{itemize}
	\item He towelled his wet hair.
	\item I lowered myself to the water's edge, getting my feet wet.
	\item My gloves were soaking wet.
	\item I saw his face was wet with tears.
	\end{itemize}
}
\item verb \\
To \textbf{wet} something means to get water or some other liquid over it.
 \textit{
	\begin{itemize}
	\item When assembling the pie, wet the edges where the two crusts join.
	\item Fielding nervously wet his lips and tried to smile.
	\end{itemize}
}
\item adjective \\
If the weather is \textbf{wet} , it is raining.
 \textbf{The wet} is used to mean wet weather.
 \textit{
	\begin{itemize}
	\item If the weather is wet or cold, choose an indoor activity.
	\item It was a miserable wet day.
	\item They had come in from the cold and the wet.
	\item Braking in the wet in heavy traffic is never fun.
	\end{itemize}
}
\item adjective \\
If something such as paint , ink , or cement is \textbf{wet} , it is not yet dry or solid.
 \textit{
	\begin{itemize}
	\item I lay the painting flat to stop the wet paint running.
	\item She rendered the walls in cement and, while it was still wet, applied the shells.
	\end{itemize}
}
\item graded adjective \\
If a child or its nappy or clothing is \textbf{wet} , its nappy or clothing is soaked in urine .
 \textit{
	\begin{itemize}
	\item Change him when he's wet.
	\item Avoid changing a nappy unless it's dirty or very wet.
	\end{itemize}
}
\item verb \\
If people, especially children, \textbf{wet} their beds or clothes or \textbf{wet}  \textbf{themselves} , they urinate in their beds or in their clothes because they cannot stop themselves.
 \textit{
	\begin{itemize}
	\item A quarter of 4-year-olds frequently wet the bed.
	\item To put it plainly, they wet themselves.
	\end{itemize}
}
\item adjective \\
\textbf{Wet} fish is fish that is sold fresh and uncooked , and not frozen or dried.
 \textit{
	\begin{itemize}
	\end{itemize}
}
\item graded adjective \\
If you say that someone is \textbf{wet} , you mean that they are weak and lacking in enthusiasm , energy , or confidence .
 \textit{
	\begin{itemize}
	\item Don't be so wet, Charles.
	\end{itemize}
}
\item countable noun \\
A \textbf{wet} is a Conservative politician who supports moderate political policies and opposes extreme ones.
 \textit{
	\begin{itemize}
	\item The left, the so-called 'wets', thought more state spending would mean more jobs.
	\end{itemize}
}
\item  \\
 wet behind the ears \textit{
	\begin{itemize}
	\end{itemize}
}
\end{enumerate}

\section*{victim}
{\large \color{blue}  victims  }
\subsection*{Explain}
\begin{enumerate}
\item countable noun \\
A \textbf{victim} is someone who has been hurt or killed .
 \textit{
	\begin{itemize}
	\item Not all the victims survived.
	\item Statistically, our chances of being the victims of violent crime are remote.
	\end{itemize}
}
\item countable noun \\
A \textbf{victim} is someone who has suffered as a result of someone else's actions or beliefs , or as a result of unpleasant circumstances.
 \textit{
	\begin{itemize}
	\item He was a victim of racial prejudice.
	\item He described himself and Altman as victims rather than participants in the scandal.
	\item Infectious diseases are spreading among many of the flood victims.
	\end{itemize}
}
\item  \\
 fall victim to \textit{
	\begin{itemize}
	\end{itemize}
}
\end{enumerate}

\section*{brief}
{\large \color{blue}  briefer  briefest  briefs  briefing  briefed  }
\subsection*{Explain}
\begin{enumerate}
\item adjective \\
Something that is \textbf{brief}  lasts for only a short time.
 \textit{
	\begin{itemize}
	\item She once made a brief appearance on television.
	\item This time their visit is brief.
	\end{itemize}
}
\item adjective \\
A \textbf{brief}  speech or piece of writing does not contain too many words or details .
 \textit{
	\begin{itemize}
	\item In a brief statement, he concentrated entirely on international affairs.
	\item Write a very brief description of a typical problem.
	\end{itemize}
}
\item adjective \\
If you are \textbf{brief} , you say what you want to say in as few words as possible .
 \textit{
	\begin{itemize}
	\item Now please be brief–my time is valuable.
	\item I hope to be brief and to the point.
	\end{itemize}
}
\item adjective \\
You can  describe a period of time as \textbf{brief} if you want to emphasize that it is very short.
 \textit{
	\begin{itemize}
	\item For a few brief minutes we forgot the anxiety and anguish.
	\end{itemize}
}
\item plural noun \\
Men's or women's underpants can be referred to as \textbf{briefs} .
 \textit{
	\begin{itemize}
	\item A bra and a pair of briefs lay on the floor.
	\end{itemize}
}
\item verb \\
If someone \textbf{briefs} you, especially about a piece of work or a serious  matter , they give you information that you need before you do it or consider it.
 \textit{
	\begin{itemize}
	\item A department spokesperson briefed reporters.
	\item The Prime Minister has been briefed by her parliamentary aides.
	\end{itemize}
}
\item countable noun \\
If someone gives you a \textbf{brief} , they officially give you responsibility for dealing with a particular thing.
 \textit{
	\begin{itemize}
	\item She joined the company less than two years ago with a brief to turn the studio around.
	\end{itemize}
}
\item  \\
 in brief \textit{
	\begin{itemize}
	\end{itemize}
}
\item  \\
 in brief \textit{
	\begin{itemize}
	\end{itemize}
}
\end{enumerate}

\section*{accuracy}
{\large \color{blue}  }
\subsection*{Explain}
\begin{enumerate}
\item uncountable noun \\
The \textbf{accuracy}  \textbf{of} information or measurements is their quality of being true or correct , even in small details .
 \textit{
	\begin{itemize}
	\item We cannot guarantee the accuracy of these figures.
	\end{itemize}
}
\item uncountable noun \\
If someone or something performs a task , for example  hitting a target , \textbf{with}  \textbf{accuracy} , they do it in an exact way without making a mistake .
 \textit{
	\begin{itemize}
	\item ...weapons that could fire with accuracy at targets 3,000 yards away.
	\item Every bank pays close attention to the speed and accuracy of its staff.
	\end{itemize}
}
\end{enumerate}

\section*{complicated}
{\large \color{blue}  }
\subsection*{Explain}
\begin{enumerate}
\item adjective \\
If you say that something is \textbf{complicated} , you mean it has so many parts or aspects that it is difficult to understand or deal with.
 \textit{
	\begin{itemize}
	\item The situation in Lebanon is very complicated.
	\item ...a very complicated voting system.
	\end{itemize}
}
\end{enumerate}

\section*{affection}
{\large \color{blue}  affections  }
\subsection*{Explain}
\begin{enumerate}
\item uncountable noun \\
If you regard someone or something with \textbf{affection} , you like them and are fond of them.
 \textit{
	\begin{itemize}
	\item She thought of him with affection.
	\item She had developed quite an affection for the place.
	\item ...trying to win their affection.
	\end{itemize}
}
\item plural noun \\
Your \textbf{affections} are your feelings of love or fondness for someone.
 \textit{
	\begin{itemize}
	\item The distant object of his affections is Caroline.
	\item ...her fear of being replaced in his affections.
	\end{itemize}
}
\end{enumerate}

\section*{concise}
{\large \color{blue}  }
\subsection*{Explain}
\begin{enumerate}
\item adjective \\
Something that is \textbf{concise}  says everything that is necessary without using any unnecessary words.
 \textit{
	\begin{itemize}
	\item Burton's text is concise and informative.
	\item Whatever you are writing make sure you are clear, concise, and accurate.
	\end{itemize}
}
\item adjective \\
A \textbf{concise}  edition of a book , especially a dictionary , is shorter than the original edition.
 \textit{
	\begin{itemize}
	\item ...Sotheby's Concise Encyclopedia of Porcelain.
	\end{itemize}
}
\end{enumerate}

\section*{altitude}
{\large \color{blue}  altitudes  }
\subsection*{Explain}
\begin{enumerate}
\item variable noun \\
If something is at a particular \textbf{altitude} , it is at that height above sea level.
 \textit{
	\begin{itemize}
	\item The aircraft had reached its cruising altitude of about 39,000 feet.
	\item The following day I ran my first race at high altitude.
	\end{itemize}
}
\end{enumerate}

\section*{digital}
{\large \color{blue}  }
\subsection*{Explain}
\begin{enumerate}
\item adjective \\
\textbf{Digital} systems record or transmit information in the form of thousands of very small signals.
 \textit{
	\begin{itemize}
	\item Digital technology allowed a rapid expansion in the number of TV channels.
	\end{itemize}
}
\item adjective \\
\textbf{Digital} devices such as watches or clocks give information by displaying numbers rather than by having a pointer which moves
round a dial. Compare  analogue .
 \textit{
	\begin{itemize}
	\item ...a digital display.
	\end{itemize}
}
\end{enumerate}

\section*{aviation}
{\large \color{blue}  }
\subsection*{Explain}
\begin{enumerate}
\item uncountable noun \\
\textbf{Aviation} is the operation and production of aircraft.
 \textit{
	\begin{itemize}
	\end{itemize}
}
\end{enumerate}

\section*{dim}
{\large \color{blue}  dimmer  dimmest  dims  dimming  dimmed  }
\subsection*{Explain}
\begin{enumerate}
\item adjective \\
\textbf{Dim} light is not bright.
 \textit{
	\begin{itemize}
	\item She stood waiting, in the dim light.
	\item Below decks, the lights were dim.
	\end{itemize}
}
\item adjective \\
A \textbf{dim} place is rather dark because there is not much light in it.
 \textit{
	\begin{itemize}
	\item The room was dim and cool and quiet.
	\end{itemize}
}
\item adjective \\
A \textbf{dim}  figure or object is not very easy to see, either because it is in shadow or darkness, or because it is far  away .
 \textit{
	\begin{itemize}
	\item Pete's torch picked out the dim figures of Bob and Chang.
	\end{itemize}
}
\item adjective \\
If you have a \textbf{dim}  memory or understanding of something, it is difficult to remember or is unclear in your mind.
 \textit{
	\begin{itemize}
	\item It seems that the '60s era of social activism is all but a dim memory.
	\end{itemize}
}
\item adjective \\
If the future of something is \textbf{dim} , you have no reason to feel  hopeful or positive about it.
 \textit{
	\begin{itemize}
	\item The prospects for a peaceful solution are dim.
	\end{itemize}
}
\item adjective \\
If you describe someone as \textbf{dim} , you think that they are stupid .
 \textit{
	\begin{itemize}
	\end{itemize}
}
\item verb \\
If you \textbf{dim} a light or if it \textbf{dims} , it becomes less bright.
 \textit{
	\begin{itemize}
	\item Dim the lighting–it is unpleasant to lie with a bright light shining in your eyes.
	\item The houselights dimmed.
	\end{itemize}
}
\item verb \\
If you are driving a car and \textbf{dim} the headlights, you operate a switch that makes them shine  downwards , so that they do not shine directly into the eyes of other drivers .
 \textit{
	\begin{itemize}
	\item Dim your lights behind that car.
	\end{itemize}
}
\item verb \\
If your future, hopes , or emotions  \textbf{dim} or if something \textbf{dims} them, they become less good or less strong .
 \textit{
	\begin{itemize}
	\item Their economic prospects have dimmed.
	\item Forty eight years of marriage have not dimmed the passion between Bill and Helen.
	\end{itemize}
}
\item ergative verb \\
If your eyes \textbf{dim} or \textbf{are dimmed} by something, they become weaker or unable to see clearly.
 \textit{
	\begin{itemize}
	\item Her eyes dimmed with sorrow.
	\item The twinkle in his eyes was dimmed by tears.
	\end{itemize}
}
\item verb \\
If your memories \textbf{dim} or if something \textbf{dims} them, they become less clear in your mind.
 \textit{
	\begin{itemize}
	\item Their memory of what happened has dimmed.
	\item The intervening years had dimmed his memory.
	\end{itemize}
}
\end{enumerate}

\section*{bosom}
{\large \color{blue}  bosoms  }
\subsection*{Explain}
\begin{enumerate}
\item countable noun \\
A woman's breasts are sometimes  referred to as her \textbf{bosom} or her \textbf{bosoms} .
 \textit{
	\begin{itemize}
	\item ...a large young mother with a baby resting against her ample bosom.
	\end{itemize}
}
\item singular noun \\
If you are in \textbf{the bosom}  \textbf{of} your family or of a community , you are among people who love , accept , and protect you.
 \textit{
	\begin{itemize}
	\item Joan was delighted to welcome her boyfriend into the bosom of her large, close-knit
family.
	\end{itemize}
}
\item adjective \\
A \textbf{bosom}  friend is a friend who you know very well and like very much indeed.
 \textit{
	\begin{itemize}
	\item They were bosom friends.
	\item Sakota was her cousin and bosom pal.
	\end{itemize}
}
\item  \\
 to one's bosom \textit{
	\begin{itemize}
	\end{itemize}
}
\end{enumerate}

\section*{energetic}
{\large \color{blue}  }
\subsection*{Explain}
\begin{enumerate}
\item adjective \\
If you are \textbf{energetic} in what you do, you have a lot of enthusiasm and determination .
 \textit{
	\begin{itemize}
	\item Blackwell is 59, strong looking and enormously energetic.
	\item The next government will play an energetic role in seeking multilateral nuclear disarmament.
	\end{itemize}
}
\item adjective \\
An \textbf{energetic} person is very active and does not feel at all tired . An \textbf{energetic} activity involves a lot of physical movement and power.
 \textit{
	\begin{itemize}
	\item Ten year-olds are incredibly energetic.
	\item ...an energetic exercise routine.
	\end{itemize}
}
\end{enumerate}

\section*{breast}
{\large \color{blue}  breasts  }
\subsection*{Explain}
\begin{enumerate}
\item countable noun \\
A woman's \textbf{breasts} are the two soft, round parts on her chest that can produce milk to feed a baby .
 \textit{
	\begin{itemize}
	\item She wears a low-cut dress which reveals her breasts.
	\item As my newborn cuddled at my breast, her tiny fingers stroked my skin.
	\end{itemize}
}
\item countable noun \\
A person's \textbf{breast} is the upper part of his or her chest.
 \textit{
	\begin{itemize}
	\item He struck his breast in a dramatic gesture.
	\end{itemize}
}
\item countable noun \\
The \textbf{breast} is often considered to be the part of your body where your emotions are.
 \textit{
	\begin{itemize}
	\item The verse rose up to fire his breast with inspiration.
	\item ...a sound calculated to arouse the sublimest emotions in the breast of the soldier.
	\end{itemize}
}
\item countable noun \\
A bird's \textbf{breast} is the front part of its body.
 \textit{
	\begin{itemize}
	\item The cock's breast is tinged with chestnut.
	\end{itemize}
}
\item singular noun \\
The \textbf{breast} of a shirt , jacket , or coat is the part which covers the top part of the chest.
 \textit{
	\begin{itemize}
	\item He moved out from beneath an awning, reaching for something inside the breast of
his overcoat.
	\item He reached into his breast pocket for his cigar case.
	\end{itemize}
}
\item variable noun \\
You can refer to piece of meat that is cut from the front of a bird or lamb as \textbf{breast} .
 \textit{
	\begin{itemize}
	\item ...a chicken breast with vegetables.
	\item ...breast of lamb.
	\end{itemize}
}
\item  \\
 to beat one's breast \textit{
	\begin{itemize}
	\end{itemize}
}
\item  \\
 to make a clean breast of it \textit{
	\begin{itemize}
	\end{itemize}
}
\end{enumerate}

\section*{erroneous}
{\large \color{blue}  }
\subsection*{Explain}
\begin{enumerate}
\item adjective \\
Beliefs , opinions , or methods that are \textbf{erroneous} are incorrect or only partly  correct .
 \textit{
	\begin{itemize}
	\item They did nothing to dispel his erroneous belief about the children's paternity.
	\item They have arrived at some erroneous conclusions.
	\end{itemize}
}
\end{enumerate}

\section*{bug}
{\large \color{blue}  bugs  bugging  bugged  }
\subsection*{Explain}
\begin{enumerate}
\item countable noun \\
A \textbf{bug} is an insect or similar small creature .
 \textit{
	\begin{itemize}
	\item We noticed tiny bugs that were all over the walls.
	\item ...a bloodsucking bug which infests poor housing.
	\end{itemize}
}
\item countable noun \\
A \textbf{bug} is an illness which is caused by small organisms such as bacteria.
 \textit{
	\begin{itemize}
	\item I think I've got a bit of a stomach bug.
	\item There was a bug going around at the club.
	\item ...the killer brain bug meningitis.
	\end{itemize}
}
\item countable noun \\
If there is a \textbf{bug} in a computer program, there is a mistake in it.
 \textit{
	\begin{itemize}
	\item There is a bug in the software.
	\end{itemize}
}
\item countable noun \\
A \textbf{bug} is a tiny  hidden microphone which transmits what people are saying .
 \textit{
	\begin{itemize}
	\item There was a bug on the phone.
	\end{itemize}
}
\item verb \\
If someone \textbf{bugs} a place, they hide tiny microphones in it which transmit what people are saying.
 \textit{
	\begin{itemize}
	\item He heard that they were planning to bug his office.
	\item I found out my phone was bugged.
	\end{itemize}
}
\item singular noun \\
You can say that someone has been bitten by a particular \textbf{bug} when they suddenly become very enthusiastic about something.
 \textit{
	\begin{itemize}
	\item I've definitely been bitten by the gardening bug.
	\item Roundhay Park in Leeds was the place I first got the fishing bug.
	\end{itemize}
}
\item verb \\
If someone or something \textbf{bugs} you, they worry or annoy you.
 \textit{
	\begin{itemize}
	\item I only did it to bug my parents.
	\end{itemize}
}
\end{enumerate}

\section*{faint}
{\large \color{blue}  fainter  faintest  faints  fainting  fainted  }
\subsection*{Explain}
\begin{enumerate}
\item adjective \\
A \textbf{faint} sound, colour, mark , feeling, or quality has very little strength or intensity .
 \textit{
	\begin{itemize}
	\item He became aware of the soft, faint sounds of water dripping.
	\item The room held the faint, sweet odour of pipe tobacco.
	\item He could see faint lines in her face.
	\item There was still the faint hope deep within him that she might never need to know.
	\end{itemize}
}
\item adjective \\
A \textbf{faint}  attempt at something is one that is made without proper  effort and with little enthusiasm .
 \textit{
	\begin{itemize}
	\item Caroline made a faint attempt at a laugh.
	\item A faint smile crossed the Monsignor's face and faded quickly.
	\item Ten years ago today the U.S. Center for Disease Control published the first faint
warnings of a worldwide epidemic.
	\end{itemize}
}
\item verb \\
If you \textbf{faint} , you lose consciousness for a short time, especially because you are hungry , or because of pain , heat , or shock .
 \textbf{Faint} is also a noun .
 \textit{
	\begin{itemize}
	\item She suddenly fell forward on to the table and fainted.
	\item I thought he'd faint when I kissed him.
	\item She slumped to the ground in a faint.
	\end{itemize}
}
\item adjective \\
Someone who is \textbf{faint}  feels weak and unsteady as if they are about to lose consciousness.
 \textit{
	\begin{itemize}
	\item Other signs of angina are nausea, sweating, feeling faint and shortness of breath.
	\end{itemize}
}
\end{enumerate}

\section*{capsule}
{\large \color{blue}  capsules  }
\subsection*{Explain}
\begin{enumerate}
\item countable noun \\
A \textbf{capsule} is a very small tube containing powdered or liquid medicine, which you swallow .
 \textit{
	\begin{itemize}
	\item ...cod liver oil capsules.
	\item You can also take red ginseng in convenient tablet or capsule form.
	\end{itemize}
}
\item countable noun \\
A \textbf{capsule} is a small container with a drug or other substance inside it, which is used for medical or scientific purposes.
 \textit{
	\begin{itemize}
	\item They first inserted capsules into the animals' mouths.
	\item The clear capsules start dissolving as soon as they are immersed in the lake.
	\end{itemize}
}
\item countable noun \\
In some plants, a \textbf{capsule} is a part which forms a case or container for seeds, fruit, or spores .
 \textit{
	\begin{itemize}
	\item ...a large shiny brown nut, enclosed in a large spiny seed capsule.
	\end{itemize}
}
\item countable noun \\
A space \textbf{capsule} is the part of a spacecraft in which people travel , and which often separates from the main rocket .
 \textit{
	\begin{itemize}
	\item A Russian space capsule is currently orbiting the Earth.
	\end{itemize}
}
\end{enumerate}

\section*{chest}
{\large \color{blue}  chests  }
\subsection*{Explain}
\begin{enumerate}
\item countable noun \\
Your \textbf{chest} is the top part of the front of your body where your ribs , lungs , and heart are.
 \textit{
	\begin{itemize}
	\item He crossed his arms over his chest.
	\item He was shot in the chest.
	\item He complained of chest pain.
	\end{itemize}
}
\item countable noun \\
A \textbf{chest} is a large, heavy box used for storing things.
 \textit{
	\begin{itemize}
	\item At the very bottom of the chest were his carving tools.
	\item ...a treasure chest.
	\item ...a medicine chest.
	\end{itemize}
}
\item  \\
 get something off your chest \textit{
	\begin{itemize}
	\end{itemize}
}
\end{enumerate}

\section*{handsome}
{\large \color{blue}  }
\subsection*{Explain}
\begin{enumerate}
\item adjective \\
A \textbf{handsome} man has an attractive  face with regular features.
 \textit{
	\begin{itemize}
	\item ...a tall, dark, handsome sheep farmer.
	\end{itemize}
}
\item adjective \\
A \textbf{handsome} woman has an attractive appearance with features that are large and regular rather than small and delicate .
 \textit{
	\begin{itemize}
	\item ...an extremely handsome woman with a beautiful voice.
	\end{itemize}
}
\item adjective \\
A \textbf{handsome}  sum of money is a large or generous amount.
 \textit{
	\begin{itemize}
	\item They will make a handsome profit on the property.
	\end{itemize}
}
\item graded adjective \\
A place such as a building or garden that is \textbf{handsome} is large and well made with an attractive appearance.
 \textit{
	\begin{itemize}
	\item ...the ports of Dubrovnik and Zadar, with their handsome Renaissance buildings.
	\end{itemize}
}
\item adjective \\
If someone has a \textbf{handsome}  win or a \textbf{handsome}  victory , they get many more points or votes than their opponent .
 \textit{
	\begin{itemize}
	\item The opposition won a handsome victory in the election.
	\end{itemize}
}
\end{enumerate}

\section*{disease}
{\large \color{blue}  diseases  }
\subsection*{Explain}
\begin{enumerate}
\item variable noun \\
A \textbf{disease} is an illness which affects people, animals, or plants, for example one which is caused by bacteria or infection.
 \textit{
	\begin{itemize}
	\item ...the rapid spread of disease in the area.
	\item ...illnesses such as heart disease.
	\item Doctors believe they have cured him of the disease.
	\end{itemize}
}
\item countable noun \\
You can refer to a bad  attitude or habit , usually one that a group of people have, as a \textbf{disease} .
 \textit{
	\begin{itemize}
	\end{itemize}
}
\end{enumerate}

\section*{liberal}
{\large \color{blue}  liberals  }
\subsection*{Explain}
\begin{enumerate}
\item adjective \\
Someone who has \textbf{liberal} views believes people should have a lot of freedom in deciding how to behave and think .
 \textbf{Liberal} is also a noun .
 \textit{
	\begin{itemize}
	\item She is known to have liberal views on divorce and contraception.
	\item ...a nation of free-thinking liberals.
	\end{itemize}
}
\item adjective \\
A \textbf{liberal} system allows people or organizations a lot of political or economic freedom.
 \textbf{Liberal} is also a noun.
 \textit{
	\begin{itemize}
	\item ...a liberal democracy with a multiparty political system.
	\item They favour liberal free-market policies.
	\item Price controls go against all the financial principles of free market liberals.
	\end{itemize}
}
\item adjective \\
A \textbf{Liberal}  politician or voter is a member of a Liberal Party or votes for a Liberal Party.
 \textbf{Liberal} is also a noun.
 \textit{
	\begin{itemize}
	\item The Liberal leader has announced his party's withdrawal from the election.
	\item The Liberals hold twenty-three seats on the local council.
	\end{itemize}
}
\item adjective \\
\textbf{Liberal} means giving, using, or taking a lot of something, or existing in large quantities .
 \textit{
	\begin{itemize}
	\item As always he is liberal with his jokes.
	\item She made liberal use of her elder sister's make-up and clothes.
	\end{itemize}
}
\end{enumerate}

\section*{efficiency}
{\large \color{blue}  efficiencies  }
\subsection*{Explain}
\begin{enumerate}
\item uncountable noun \\
\textbf{Efficiency} is the quality of being able to do a task successfully, without wasting time or energy.
 \textit{
	\begin{itemize}
	\item There are many ways to increase agricultural efficiency in the poorer areas of the
world.
	\item ...energy efficiency.
	\end{itemize}
}
\item uncountable noun \\
In physics and engineering , \textbf{efficiency} is the ratio between the amount of energy a machine needs to make it work, and the amount it produces.
 \textit{
	\begin{itemize}
	\end{itemize}
}
\end{enumerate}

\section*{magnetic}
{\large \color{blue}  }
\subsection*{Explain}
\begin{enumerate}
\item adjective \\
If something metal is \textbf{magnetic} , it acts like a magnet.
 \textit{
	\begin{itemize}
	\item ...magnetic particles.
	\end{itemize}
}
\item adjective \\
You use \textbf{magnetic} to describe something that is caused by or relates to the force of magnetism.
 \textit{
	\begin{itemize}
	\item The electrically charged gas particles are affected by magnetic forces.
	\end{itemize}
}
\item adjective \\
You use \textbf{magnetic} to describe tapes and other objects which have a coating of a magnetic substance and contain coded  information that can be read by computers or other machines .
 \textit{
	\begin{itemize}
	\item ...her magnetic strip ID card.
	\item ...magnetic recording tape.
	\end{itemize}
}
\item adjective \\
If you describe something as \textbf{magnetic} , you mean that it is very attractive to people because it has unusual , powerful, and exciting qualities.
 \textit{
	\begin{itemize}
	\item London's creative reputation has had a magnetic effect.
	\item ...the magnetic pull of his looks and her personality.
	\end{itemize}
}
\end{enumerate}

\section*{fancy}
{\large \color{blue}  fancies  fancying  fancied  }
\subsection*{Explain}
\begin{enumerate}
\item verb \\
If you \textbf{fancy} something, you want to have it or to do it.
 \textit{
	\begin{itemize}
	\item What do you fancy doing, anyway?
	\item Do you fancy going to see a movie sometime?
	\item I just fancied a drink.
	\end{itemize}
}
\item countable noun \\
A \textbf{fancy} is a liking or desire for someone or something, especially one that does not last long.
 \textit{
	\begin{itemize}
	\item She did not suspect that his interest was just a passing fancy.
	\end{itemize}
}
\item verb \\
If you \textbf{fancy} someone, you feel attracted to them, especially in a sexual way.
 \textit{
	\begin{itemize}
	\item I think he thinks I fancy him or something.
	\end{itemize}
}
\item verb \\
If you \textbf{fancy}  \textbf{yourself as} a particular kind of person or fancy \textbf{yourself} doing a particular thing, you like the idea of being that kind of person or doing
that thing.
 \textit{
	\begin{itemize}
	\item So you fancy yourself as the boss someday?
	\item I didn't fancy myself wearing a kilt.
	\end{itemize}
}
\item verb \\
If you say that someone \textbf{fancies}  \textbf{themselves as} a particular kind of person, you mean that they think , often wrongly, that they have the good qualities which that kind of person has.
 \textit{
	\begin{itemize}
	\item She fancies herself a bohemian.
	\item She knew Felix fancied himself as a connoisseur.
	\item ...a flighty young woman who really fancies herself.
	\end{itemize}
}
\item verb \\
If you say that you \textbf{fancy} a particular competitor or team in a competition , you think they will win .
 \textit{
	\begin{itemize}
	\item You have to fancy Bath because they are the most consistent team in England.
	\item I fancy England to win through.
	\end{itemize}
}
\item verb \\
If you \textbf{fancy} that something is the case, you think or suppose that it is so.
 \textit{
	\begin{itemize}
	\item When Ferris looked up he fancied that he saw a shadow pass close to the window.
	\item She fancied he was trying to hide a smile.
	\end{itemize}
}
\item variable noun \\
A \textbf{fancy} is an idea that is unlikely , untrue , or imaginary .
 \textit{
	\begin{itemize}
	\item His last book is a bold, at times surrealistic mixture of fact and fancy.
	\item ...a childhood fancy.
	\item ...whims and fancies.
	\end{itemize}
}
\item exclamation \\
You say ' \textbf{fancy} ' or ' \textbf{fancy that} ' when you want to express surprise or disapproval .
 \textit{
	\begin{itemize}
	\item It was very tasteless. Fancy talking like that so soon after his death.
	\item 'Fancy that!' smiled Conti.
	\end{itemize}
}
\item  \\
 take a fancy to sb/sth \textit{
	\begin{itemize}
	\end{itemize}
}
\item  \\
 take/tickle sb's fancy \textit{
	\begin{itemize}
	\end{itemize}
}
\end{enumerate}

\section*{fleet}
{\large \color{blue}  fleets  }
\subsection*{Explain}
\begin{enumerate}
\item countable noun \\
A \textbf{fleet} is a group of ships organized to do something together, for example to fight  battles or to catch fish.
 \textit{
	\begin{itemize}
	\item The damage inflicted upon the British fleet was devastating.
	\item ...restaurants supplied by local fishing fleets.
	\end{itemize}
}
\item countable noun \\
A \textbf{fleet} of vehicles is a group of them, especially when they all belong to a particular organization or business, or when they are all
 going  somewhere together.
 \textit{
	\begin{itemize}
	\item With its own fleet of trucks, the company delivers most orders overnight.
	\item A fleet of ambulances took the injured to hospital.
	\end{itemize}
}
\end{enumerate}

\section*{mental}
{\large \color{blue}  }
\subsection*{Explain}
\begin{enumerate}
\item adjective \\
\textbf{Mental}  means relating to the process of thinking .
 \textit{
	\begin{itemize}
	\item ...the mental development of children.
	\item ...intensive mental effort.
	\end{itemize}
}
\item adjective \\
\textbf{Mental} means relating to the state or the health of a person's mind.
 \textit{
	\begin{itemize}
	\item The mental state that had created her psychosis was no longer present.
	\item ...mental health problems.
	\end{itemize}
}
\item adjective \\
A \textbf{mental} act is one that involves only thinking and not physical action.
 \textit{
	\begin{itemize}
	\item Practise mental arithmetic when you go out shopping.
	\item Graham made a quick mental calculation.
	\item She made a mental note not to sit anywhere near him.
	\end{itemize}
}
\item adjective \\
If you say that someone is \textbf{mental} , you mean that you think they are mad .
 \textit{
	\begin{itemize}
	\item I just said to him 'you must be mental'.
	\end{itemize}
}
\item  \\
 to make a mental note \textit{
	\begin{itemize}
	\end{itemize}
}
\end{enumerate}

\section*{glue}
{\large \color{blue}  glues  glueing  gluing  glued  }
\subsection*{Explain}
\begin{enumerate}
\item variable noun \\
\textbf{Glue} is a sticky substance used for joining things together, often for repairing  broken things.
 \textit{
	\begin{itemize}
	\item ...a tube of glue.
	\item ...high quality glues.
	\end{itemize}
}
\item verb \\
If you \textbf{glue} one object to another, you stick them together using glue.
 \textit{
	\begin{itemize}
	\item Glue the fabric around the window.
	\item The material is cut and glued in place.
	\item They are glued together.
	\end{itemize}
}
\item passive verb \\
If you say that someone \textbf{is glued to} something, you mean that they are giving it all their attention .
 \textit{
	\begin{itemize}
	\item They are all glued to the final episode.
	\end{itemize}
}
\end{enumerate}

\section*{missing}
{\large \color{blue}  }
\subsection*{Explain}
\begin{enumerate}
\item adjective \\
If something is \textbf{missing} , it is not in its usual place, and you cannot find it.
 \textit{
	\begin{itemize}
	\item It was only an hour or so later that I discovered that my gun was missing.
	\item The playing cards had gone missing.
	\end{itemize}
}
\item adjective \\
If a part of something is \textbf{missing} , it has been removed or has come off, and has not been replaced .
 \textit{
	\begin{itemize}
	\item Three buttons were missing from his shirt.
	\end{itemize}
}
\item adjective \\
If you say that something is \textbf{missing} , you mean that it has not been included, and you think that it should have been.
 \textit{
	\begin{itemize}
	\item What is missing, however, is an internal, artistic cohesion.
	\item She had given me an incomplete list. One name was missing from it.
	\end{itemize}
}
\item adjective \\
Someone who is \textbf{missing} cannot be found, and it is not known whether they are alive or dead.
 \textit{
	\begin{itemize}
	\item Five people died in the explosion, and one person is still missing.
	\end{itemize}
}
\end{enumerate}

\section*{golf}
{\large \color{blue}  }
\subsection*{Explain}
\begin{enumerate}
\item uncountable noun \\
\textbf{Golf} is a game in which you use long sticks  called clubs to hit a small, hard ball into holes that are spread out over a large area of grassy land.
 \textit{
	\begin{itemize}
	\end{itemize}
}
\end{enumerate}

\section*{naked}
{\large \color{blue}  }
\subsection*{Explain}
\begin{enumerate}
\item adjective \\
Someone who is \textbf{naked} is not wearing any clothes.
 \textit{
	\begin{itemize}
	\item Kate throws a kimono over her naked body.
	\item The hot paving stones scorched my naked feet.
	\item They stripped me naked.
	\item He stood naked in front of me.
	\end{itemize}
}
\item adjective \\
If an animal or part of an animal is \textbf{naked} , it has no fur or feathers on it.
 \textit{
	\begin{itemize}
	\item The nest contained eight little mice that were naked and blind.
	\end{itemize}
}
\item adjective \\
If you say that someone is \textbf{naked} or feels  \textbf{naked} , you mean they are powerless or have no way of protecting themselves.
 \textit{
	\begin{itemize}
	\item If the reports are accurate, the deal leaves the authorities and the President virtually
naked.
	\end{itemize}
}
\item adjective \\
You can describe an object as \textbf{naked} when it does not have its normal covering.
 \textit{
	\begin{itemize}
	\item ...a naked bulb dangling in a bare room.
	\item The water was heated by a naked gas flame.
	\end{itemize}
}
\item adjective \\
\textbf{Naked}  emotions are easy to recognize , because they are very strongly felt.
 \textit{
	\begin{itemize}
	\item The naked hatred in the woman's face shocked me.
	\item There had been naked misery in her voice when she'd spoken about the letter.
	\end{itemize}
}
\item adjective \\
You can use \textbf{naked} to describe unpleasant or violent actions and behaviour which are not disguised or hidden in any way.
 \textit{
	\begin{itemize}
	\item Naked aggression and an attempt to change frontiers by force could not go unchallenged.
	\item ...violence and the naked pursuit of power.
	\item ...naked greed.
	\end{itemize}
}
\item  \\
 the naked eye \textit{
	\begin{itemize}
	\end{itemize}
}
\end{enumerate}

\section*{health}
{\large \color{blue}  }
\subsection*{Explain}
\begin{enumerate}
\item uncountable noun \\
A person's \textbf{health} is the condition of their body and the extent to which it is free from illness or is able to resist illness.
 \textit{
	\begin{itemize}
	\item Caffeine is bad for your health.
	\end{itemize}
}
\item uncountable noun \\
\textbf{Health} is a state in which a person is not suffering from any illness and is feeling  well .
 \textit{
	\begin{itemize}
	\item In hospital they nursed me back to health.
	\end{itemize}
}
\item  \\
 to drink someone's health \textit{
	\begin{itemize}
	\end{itemize}
}
\item uncountable noun \\
The \textbf{health} of something such as an organization or a system is its success and the fact that it is working well.
 \textit{
	\begin{itemize}
	\item There's no way to predict the future health of the banking industry.
	\end{itemize}
}
\end{enumerate}

\section*{height}
{\large \color{blue}  heights  }
\subsection*{Explain}
\begin{enumerate}
\item variable noun \\
The \textbf{height} of a person or thing is their size or length from the bottom to the top.
 \textit{
	\begin{itemize}
	\item Her weight is about normal for her height.
	\item I am 5'6'' in height.
	\item The wave here has a length of 250 feet and a height of 10 feet.
	\item He was a man of medium height.
	\item ...a garden containing all sorts of trees and shrubs of varying heights and shades.
	\end{itemize}
}
\item uncountable noun \\
\textbf{Height} is the quality of being tall .
 \textit{
	\begin{itemize}
	\item She admits that her height is intimidating for some people.
	\end{itemize}
}
\item variable noun \\
A particular \textbf{height} is the distance that something is above the ground or above something else mentioned .
 \textit{
	\begin{itemize}
	\item At the speed and height at which he was moving, he was never more than half a second
from disaster.
	\item ...a test in which a 6.3 kilogram weight was dropped on it from a height of 1 metre.
	\item The corridors there were painted chocolate-brown to shoulder height.
	\item The chains were at different heights on the wall.
	\end{itemize}
}
\item countable noun \\
A \textbf{height} is a high position or place above the ground.
 \textit{
	\begin{itemize}
	\item From a height, it looks like a desert.
	\item I'm not afraid of heights.
	\item ...the Golan Heights.
	\end{itemize}
}
\item singular noun \\
When an activity, situation , or organization is \textbf{at} its \textbf{height} , it is at its most successful , powerful , or intense .
 \textit{
	\begin{itemize}
	\item During the early sixth century emigration from Britain to Brittany was at its height.
	\item At its height Bletchley Park employed 12,000 people.
	\item He was struck down at the height of his career.
	\item It was freezing up there even at the height of summer.
	\end{itemize}
}
\item singular noun \\
If you say that something is \textbf{the height of} a particular quality, you are emphasizing that it has that quality to the greatest degree  possible .
 \textit{
	\begin{itemize}
	\item The hip-hugging black and white polka-dot dress was the height of fashion.
	\item I think it's the height of bad manners to be dressed badly.
	\item This is the height of hooliganism.
	\end{itemize}
}
\item plural noun \\
If something reaches great \textbf{heights} , it becomes very extreme or intense.
 \textit{
	\begin{itemize}
	\item ...the mid-1980s, when house prices rose to absurd heights.
	\item Recently the speculation has reached new heights.
	\item One wondered what heights of ecstasy the winner reached.
	\end{itemize}
}
\end{enumerate}

\section*{numerical}
{\large \color{blue}  }
\subsection*{Explain}
\begin{enumerate}
\item adjective \\
\textbf{Numerical} means expressed in numbers or relating to numbers.
 \textit{
	\begin{itemize}
	\item Your job is to group them by letter and put them in numerical order.
	\end{itemize}
}
\end{enumerate}

\section*{hike}
{\large \color{blue}  hikes  hiking  hiked  }
\subsection*{Explain}
\begin{enumerate}
\item countable noun \\
A \textbf{hike} is a long walk in the country, especially one that you go on for pleasure.
 \textit{
	\begin{itemize}
	\end{itemize}
}
\item verb \\
If you \textbf{hike} , you go for a long walk in the country.
 \textit{
	\begin{itemize}
	\item You could hike through the Fish River Canyon.
	\item We plan to hike the Samaria Gorge.
	\end{itemize}
}
\item countable noun \\
A \textbf{hike} is a sudden or large increase in prices, rates , taxes , or quantities .
 \textit{
	\begin{itemize}
	\item ...a sudden 1.75 per cent hike in interest rates.
	\item His economic plan, with its tax hikes and spending cuts, will slow the economy.
	\end{itemize}
}
\item verb \\
To \textbf{hike} prices, rates, taxes, or quantities means to increase them suddenly or by a large amount .
 \textbf{Hike up} means the same as hike .
 \textit{
	\begin{itemize}
	\item It has now been forced to hike its rates by 5.25 per cent.
	\item The federal government hiked the tax on hard liquor.
	\item The insurers have started hiking up premiums by huge amounts.
	\item Big banks were hiking their rates up.
	\end{itemize}
}
\end{enumerate}

\section*{precise}
{\large \color{blue}  }
\subsection*{Explain}
\begin{enumerate}
\item adjective \\
You use \textbf{precise} to emphasize that you are referring to an exact thing, rather than something vague .
 \textit{
	\begin{itemize}
	\item At that precise moment I felt sorry for him and didn't want to hurt him.
	\item The precise location of the wreck was discovered in 1988.
	\item He was not clear on the precise nature of his mission.
	\item We will never know the precise details of his death.
	\end{itemize}
}
\item adjective \\
Something that is \textbf{precise} is exact and accurate in all its details .
 \textit{
	\begin{itemize}
	\item They speak very precise English.
	\item He does not talk too much and what he has to say is precise and to the point.
	\end{itemize}
}
\item  \\
 to be precise \textit{
	\begin{itemize}
	\end{itemize}
}
\end{enumerate}

\section*{illness}
{\large \color{blue}  illnesses  }
\subsection*{Explain}
\begin{enumerate}
\item uncountable noun \\
\textbf{Illness} is the fact or experience of being ill.
 \textit{
	\begin{itemize}
	\item If your child shows any signs of illness, take her to the doctor.
	\item Mental illness is still a taboo subject.
	\end{itemize}
}
\item countable noun \\
An \textbf{illness} is a particular disease such as measles or pneumonia .
 \textit{
	\begin{itemize}
	\item She returned to her family home to recover from an illness.
	\end{itemize}
}
\end{enumerate}

\section*{prudent}
{\large \color{blue}  }
\subsection*{Explain}
\begin{enumerate}
\item adjective \\
Someone who is \textbf{prudent} is sensible and careful.
 \textit{
	\begin{itemize}
	\item It is always prudent to start any exercise programme gradually at first.
	\item Being a prudent and cautious person, you realise that the problem must be resolved.
	\end{itemize}
}
\end{enumerate}

\section*{invalid}
{\large \color{blue}  invalids  }
\subsection*{Explain}
\begin{enumerate}
\item countable noun \\
An \textbf{invalid} is someone who needs to be cared for because they have an illness or disability .
 \textit{
	\begin{itemize}
	\item I hate being treated as an invalid.
	\end{itemize}
}
\item adjective \\
If an action, procedure , or document is \textbf{invalid} , it cannot be accepted , because it breaks the law or some official  rule .
 \textit{
	\begin{itemize}
	\item The trial was stopped and the results declared invalid.
	\item We cannot accept liability if you are refused entry because of invalid documents.
	\end{itemize}
}
\item adjective \\
An \textbf{invalid} argument or conclusion is wrong because it is based on a mistake .
 \textit{
	\begin{itemize}
	\item We think that those arguments are rendered invalid by the hard facts on the ground.
	\end{itemize}
}
\end{enumerate}

\section*{quantitative}
{\large \color{blue}  }
\subsection*{Explain}
\begin{enumerate}
\item adjective \\
\textbf{Quantitative} means relating to different sizes or amounts of things.
 \textit{
	\begin{itemize}
	\item ...the advantages of quantitative and qualitative research.
	\item ...the quantitative analysis of migration.
	\end{itemize}
}
\end{enumerate}

\section*{love}
{\large \color{blue}  loves  loving  loved  }
\subsection*{Explain}
\begin{enumerate}
\item verb \\
If you \textbf{love} someone, you feel romantically or sexually attracted to them, and they are very important to you.
 \textit{
	\begin{itemize}
	\item Oh, Amy, I love you.
	\item We love each other. We want to spend our lives together.
	\end{itemize}
}
\item uncountable noun \\
\textbf{Love} is a very strong feeling of affection towards someone who you are romantically or
sexually attracted to.
 \textit{
	\begin{itemize}
	\item Our love for each other has been increased by what we've been through together.
	\item ...a old fashioned love story.
	\item ...an album of love songs.
	\end{itemize}
}
\item verb \\
You say that you \textbf{love} someone when their happiness is very important to you, so that you behave in a kind and caring way towards them.
 \textit{
	\begin{itemize}
	\item You'll never love anyone the way you love your baby.
	\end{itemize}
}
\item uncountable noun \\
\textbf{Love} is the feeling that a person's happiness is very important to you, and the way you
show this feeling in your behaviour towards them.
 \textit{
	\begin{itemize}
	\item My love for all my children is unconditional.
	\item She's got a great capacity for love.
	\end{itemize}
}
\item verb \\
If you \textbf{love} something, you like it very much.
 \textit{
	\begin{itemize}
	\item We loved the food so much, especially the fish dishes.
	\item I loved reading.
	\item ...one of these people that loves to be in the outdoors.
	\item I love it when I hear you laugh.
	\end{itemize}
}
\item verb \\
You can say that you \textbf{love} something when you consider that it is important and want to protect or support it.
 \textit{
	\begin{itemize}
	\item I love my country as you love yours.
	\end{itemize}
}
\item uncountable noun \\
\textbf{Love} is a strong liking for something, or a belief that it is important.
 \textit{
	\begin{itemize}
	\item This is no way to encourage a love of literature.
	\item The French are known for their love of their language.
	\end{itemize}
}
\item countable noun \\
Your \textbf{love} is someone or something that you love.
 \textit{
	\begin{itemize}
	\item 'She is the love of my life,' he said.
	\item Music's one of my great loves.
	\end{itemize}
}
\item verb \\
If you \textbf{would love}  \textbf{to} have or do something, you very much want to have it or do it.
 \textit{
	\begin{itemize}
	\item I would love to play for England again.
	\item I would love a hot bath and clean clothes.
	\item His wife would love him to give up his job.
	\end{itemize}
}
\item countable noun \\
Some people use \textbf{love} as an affectionate way of addressing someone.
 \textit{
	\begin{itemize}
	\item Well, I'll take your word for it then, love.
	\item Don't cry, my love.
	\end{itemize}
}
\item number \\
In tennis, \textbf{love} is a score of zero.
 \textit{
	\begin{itemize}
	\item He beat the Austrian three sets to love.
	\end{itemize}
}
\item convention \\
You can use expressions such as ' \textbf{love} ', ' \textbf{love from} ', and ' \textbf{all my love} ', followed by your name, as an informal way of ending a letter to a friend or relation.
 \textit{
	\begin{itemize}
	\item ...with love from Grandma and Grandpa.
	\end{itemize}
}
\item uncountable noun \\
If you send someone your \textbf{love} , you ask another person, who will soon be speaking or writing to them, to tell them that you are thinking about them with affection.
 \textit{
	\begin{itemize}
	\item Please give her my love.
	\end{itemize}
}
\item  \\
 fall in love \textit{
	\begin{itemize}
	\end{itemize}
}
\item  \\
 fall in love \textit{
	\begin{itemize}
	\end{itemize}
}
\item  \\
 be in love \textit{
	\begin{itemize}
	\end{itemize}
}
\item  \\
 be in love \textit{
	\begin{itemize}
	\end{itemize}
}
\item  \\
 no love lost/little love lost \textit{
	\begin{itemize}
	\end{itemize}
}
\item  \\
 make love \textit{
	\begin{itemize}
	\end{itemize}
}
\item  \\
 for love or money \textit{
	\begin{itemize}
	\end{itemize}
}
\item  \\
 love at first sight \textit{
	\begin{itemize}
	\end{itemize}
}
\end{enumerate}

\section*{recent}
{\large \color{blue}  }
\subsection*{Explain}
\begin{enumerate}
\item adjective \\
A \textbf{recent} event or period of time happened only a short while ago.
 \textit{
	\begin{itemize}
	\item In the most recent attack one man was shot dead and two others were wounded.
	\item Sales have fallen by more than 75 percent in recent years.
	\end{itemize}
}
\end{enumerate}

\section*{lover}
{\large \color{blue}  lovers  }
\subsection*{Explain}
\begin{enumerate}
\item countable noun \\
Someone's \textbf{lover} is someone who they are having a sexual relationship with but are not married to.
 \textit{
	\begin{itemize}
	\item Every Thursday she would meet her lover Leon.
	\item He and Liz became lovers soon after they first met.
	\end{itemize}
}
\item countable noun \\
If you are a \textbf{lover} of something such as animals or the arts, you enjoy them very much and take great pleasure in them.
 \textit{
	\begin{itemize}
	\item She is a great lover of horses and horse racing.
	\item Are you an opera lover?
	\end{itemize}
}
\end{enumerate}

\section*{rotten}
{\large \color{blue}  }
\subsection*{Explain}
\begin{enumerate}
\item adjective \\
If food, wood, or another substance is \textbf{rotten} , it has decayed and can no longer be used.
 \textit{
	\begin{itemize}
	\item The smell outside this building is overwhelming–like rotten eggs.
	\item The front bay window is rotten.
	\end{itemize}
}
\item adjective \\
If you describe something as \textbf{rotten} , you think it is very unpleasant or of very poor quality.
 \textit{
	\begin{itemize}
	\item I personally think it's a rotten idea.
	\item I had a pretty rotten day yesterday.
	\item What rotten luck!
	\end{itemize}
}
\item adjective \\
If you describe someone as \textbf{rotten} , you are insulting them or criticizing them because you think that they are very unpleasant or unkind .
 \textit{
	\begin{itemize}
	\item You rotten swine! How dare you?
	\item That's a rotten thing to say!
	\end{itemize}
}
\item adjective \\
If you feel  \textbf{rotten} , you feel bad , either because you are ill or because you are sorry about something.
 \textit{
	\begin{itemize}
	\item I had glandular fever and spent that year feeling rotten.
	\item Suddenly, Sarah felt rotten about the whole thing.
	\end{itemize}
}
\item adjective \\
You use \textbf{rotten} to emphasize your dislike for something or your anger or annoyance about it.
 \textit{
	\begin{itemize}
	\item She was a rotten coward.
	\item Keep your rotten mouth shut.
	\end{itemize}
}
\end{enumerate}

\section*{mat}
{\large \color{blue}  mats  }
\subsection*{Explain}
\begin{enumerate}
\item countable noun \\
A \textbf{mat} is a small piece of something such as cloth, card , or plastic which you put on a table to protect it from plates or cups .
 \textit{
	\begin{itemize}
	\item The food is served on polished tables with mats.
	\end{itemize}
}
\item countable noun \\
A \textbf{mat} is a small piece of carpet or other thick material which is put on the floor for protection , decoration , or comfort .
 \textit{
	\begin{itemize}
	\item There was a letter on the mat.
	\item Bring a sleeping bag and foam mat.
	\end{itemize}
}
\item countable noun \\
A \textbf{mat}  \textbf{of} something such as grass or moss is a thick, untidy  layer of it.
 \textit{
	\begin{itemize}
	\item The houses are well spaced out, each on its own plot of ground and mat of coarse
grass.
	\item She touched the thick mat of sandy hair on his chest.
	\end{itemize}
}
\end{enumerate}

\section*{shabby}
{\large \color{blue}  shabbier  shabbiest  }
\subsection*{Explain}
\begin{enumerate}
\item adjective \\
\textbf{Shabby} things or places look old and in bad condition.
 \textit{
	\begin{itemize}
	\item His clothes were old and shabby.
	\item He walked past her into a tiny, shabby room.
	\item ...one of the shabbiest and poorest areas of London.
	\end{itemize}
}
\item adjective \\
A person who is \textbf{shabby} is wearing old, worn clothes.
 \textit{
	\begin{itemize}
	\item ...a shabby, tall man with dark eyes.
	\end{itemize}
}
\item adjective \\
If you describe someone's behaviour as \textbf{shabby} , you think they behave in an unfair or unacceptable way.
 \textit{
	\begin{itemize}
	\item It was hard to say why the man deserved such shabby treatment.
	\item I knew it was shabby of me, but I couldn't help feeling slightly disappointed.
	\end{itemize}
}
\end{enumerate}

\section*{navigation}
{\large \color{blue}  }
\subsection*{Explain}
\begin{enumerate}
\item uncountable noun \\
You can refer to the movement of ships as \textbf{navigation} .
 \textit{
	\begin{itemize}
	\item Pack ice around Iceland was becoming a threat to navigation.
	\end{itemize}
}
\end{enumerate}

\section*{silly}
{\large \color{blue}  sillier  silliest  }
\subsection*{Explain}
\begin{enumerate}
\item adjective \\
If you say that someone or something is \textbf{silly} , you mean that they are foolish, childish , or ridiculous .
 \textit{
	\begin{itemize}
	\item My best friend tells me that I am silly to be upset about this.
	\item You silly boy; why did you tramp about so long in the cold?
	\item I thought it would be silly to be too rude at that stage.
	\item That's a silly question.
	\item ...a silly hat.
	\end{itemize}
}
\item adjective \\
If you do something such as laugh or drink \textbf{yourself}  \textbf{silly} , you do it so much that you are unable to think or behave sensibly.
 \textit{
	\begin{itemize}
	\item Poor Donald's been worrying himself silly.
	\end{itemize}
}
\end{enumerate}

\section*{norm}
{\large \color{blue}  norms  }
\subsection*{Explain}
\begin{enumerate}
\item countable noun \\
\textbf{Norms} are ways of behaving that are considered normal in a particular society .
 \textit{
	\begin{itemize}
	\item ...the commonly accepted norms of democracy.
	\item ...a social norm that says drunkenness is inappropriate behaviour.
	\end{itemize}
}
\item singular noun \\
If you say that a situation is \textbf{the norm} , you mean that it is usual and expected.
 \textit{
	\begin{itemize}
	\item Families of six or seven are the norm in Borough Park.
	\item There will be more leases of 15 years than the present norm of 25 years.
	\end{itemize}
}
\item countable noun \\
A \textbf{norm} is an official standard or level that organizations are expected to reach .
 \textit{
	\begin{itemize}
	\item ...an agency which would establish European norms and co-ordinate national policies
to halt pollution.
	\end{itemize}
}
\end{enumerate}

\section*{simple}
{\large \color{blue}  simpler  simplest  }
\subsection*{Explain}
\begin{enumerate}
\item adjective \\
If you describe something as \textbf{simple} , you mean that it is not complicated, and is therefore easy to understand.
 \textit{
	\begin{itemize}
	\item ...simple pictures and diagrams.
	\item ...pages of simple advice on filling in your tax form.
	\item Buddhist ethics are simple but its practices can seem very complex to some.
	\end{itemize}
}
\item adjective \\
If you describe people or things as \textbf{simple} , you mean that they have all the basic or necessary things they require, but nothing extra .
 \textit{
	\begin{itemize}
	\item The Holy Family Church was closed and the parish now celebrates mass in this simple
side chapel.
	\item He ate a simple dinner of rice and beans.
	\item ...the simple pleasures of childhood.
	\item He lives a very simple life for a man who has become incredibly rich.
	\item Nothing is simpler than a cool white shirt.
	\end{itemize}
}
\item adjective \\
If a problem is \textbf{simple} or if its solution is \textbf{simple} , the problem can be solved  easily .
 \textit{
	\begin{itemize}
	\item Some puzzles look difficult but once the solution is known are actually quite simple.
	\item The answer is simple.
	\item I cut my purchases dramatically by the simple expedient of destroying my credit cards.
	\end{itemize}
}
\item adjective \\
A \textbf{simple}  task is easy to do.
 \textit{
	\begin{itemize}
	\item The job itself had been simple enough.
	\item The simplest way to install a shower is to fit one over the bath.
	\end{itemize}
}
\item adjective \\
If you say that someone is \textbf{simple} , you mean that they are not very intelligent and have difficulty  learning things.
 \textit{
	\begin{itemize}
	\item He was simple as a child.
	\end{itemize}
}
\item adjective \\
You use \textbf{simple} to emphasize that the thing you are referring to is the only important or relevant  reason for something.
 \textit{
	\begin{itemize}
	\item His refusal to talk was simple stubbornness.
	\end{itemize}
}
\item adjective \\
In grammar , \textbf{simple}  tenses are ones which are formed without an auxiliary  verb 'be', for example 'I dressed and went for a walk ' and 'This tastes  nice '. \textbf{Simple} verb groups are used especially to refer to completed actions, regular actions, and situations. Compare  continuous .
 \textit{
	\begin{itemize}
	\end{itemize}
}
\item adjective \\
In English grammar, a \textbf{simple}  sentence consists of one main clause . Compare compound , , complex .
 \textit{
	\begin{itemize}
	\end{itemize}
}
\end{enumerate}

\section*{clock}
{\large \color{blue}  clocks  clocking  clocked  }
\subsection*{Explain}
\begin{enumerate}
\item countable noun \\
A \textbf{clock} is an instrument, for example in a room or on the outside of a building, that shows what time of day it is.
 \textit{
	\begin{itemize}
	\item He was conscious of a clock ticking.
	\item He also repairs clocks and watches.
	\item The hands of the clock on the wall moved with a slight click.
	\item ...a digital clock.
	\end{itemize}
}
\item countable noun \\
A time \textbf{clock} in a factory or office is a device that is used to record the hours that people work. Each worker puts a special card into the device when they arrive and leave, and the times are recorded on the card.
 \textit{
	\begin{itemize}
	\item Government workers were made to punch time clocks morning, noon and night.
	\end{itemize}
}
\item countable noun \\
In a car, \textbf{the}  \textbf{clock} is the instrument that shows the speed of the car or the distance it has travelled .
 \textit{
	\begin{itemize}
	\item The car had 160,000 miles on the clock.
	\item At 240 mph the needle went off the clock.
	\end{itemize}
}
\item verb \\
To \textbf{clock} a particular time or speed in a race means to reach that time or speed.
 \textit{
	\begin{itemize}
	\item Elliott clocked the fastest time this year for the 800 metres.
	\item The yacht swayed in 40-knot winds, clocking speeds of 17 knots at times.
	\end{itemize}
}
\item verb \\
If something or someone \textbf{is clocked}  \textbf{at} a particular time or speed, their time or speed is measured at that level.
 \textit{
	\begin{itemize}
	\item He has been clocked at 11 seconds for 100 metres.
	\item 170-mile-an-hour winds were clocked on a mountaintop in North Carolina.
	\end{itemize}
}
\item verb \\
If you \textbf{clock} something, you notice or see it.
 \textit{
	\begin{itemize}
	\item I walked past that gate hundreds of times before I clocked it.
	\end{itemize}
}
\item  \\
 against the clock \textit{
	\begin{itemize}
	\end{itemize}
}
\item  \\
 to beat the clock \textit{
	\begin{itemize}
	\end{itemize}
}
\item  \\
 round the clock/around the clock \textit{
	\begin{itemize}
	\end{itemize}
}
\item  \\
 turn the clock back/put the clock back \textit{
	\begin{itemize}
	\end{itemize}
}
\item  \\
 watch the clock \textit{
	\begin{itemize}
	\end{itemize}
}
\end{enumerate}

\end{document}