\documentclass[twocolumn]{book}

\title{English book}
\author{lucas gao}
\date{\today}
\usepackage{lipsum}
\usepackage{color}
\usepackage{sectsty}
\sectionfont{\color{cyan} \fontsize{18}{20}\selectfont}

\begin{document}
\section*{alien}
{\large \color{blue}  aliens  }
\subsection*{Explain}
\begin{enumerate}
\item adjective \\
\textbf{Alien}  means belonging to a different country, race , or group, usually one you do not like or are frightened of.
 \textit{
	\begin{itemize}
	\item He said they were opposed to the presence of alien forces in the region.
	\end{itemize}
}
\item adjective \\
You use \textbf{alien} to describe something that seems strange and perhaps frightening, because it is not part of your normal  experience .
 \textit{
	\begin{itemize}
	\item His work offers an insight into an alien culture.
	\end{itemize}
}
\item adjective \\
If something is \textbf{alien}  \textbf{to} you or \textbf{to} your normal feelings or behaviour , it is not the way you would normally  feel or behave .
 \textit{
	\begin{itemize}
	\item Such an attitude is alien to many businesspeople.
	\end{itemize}
}
\item countable noun \\
An \textbf{alien} is someone who is not a legal  citizen of the country in which they live .
 \textit{
	\begin{itemize}
	\item Both women had hired illegal aliens for child care.
	\item When war broke out, he was interned as an enemy alien.
	\end{itemize}
}
\item countable noun \\
In science fiction, an \textbf{alien} is a creature from outer  space .
 \textit{
	\begin{itemize}
	\end{itemize}
}
\end{enumerate}

\section*{classify}
{\large \color{blue}  classifies  classifying  classified  }
\subsection*{Explain}
\begin{enumerate}
\item verb \\
To \textbf{classify} things means to divide them into groups or types so that things with similar characteristics
are in the same group.
 \textit{
	\begin{itemize}
	\item It is necessary initially to classify the headaches into certain types.
	\item Rocks can be classified according to their mode of origin.
	\item The coroner immediately classified his death as a suicide.
	\end{itemize}
}
\end{enumerate}

\section*{bird}
{\large \color{blue}  birds  }
\subsection*{Explain}
\begin{enumerate}
\item countable noun \\
A \textbf{bird} is a creature with feathers and wings. Female birds lay  eggs . Most birds can fly .
 \textit{
	\begin{itemize}
	\end{itemize}
}
\item countable noun \\
Some men refer to young women as \textbf{birds} . This use could cause offence .
 \textit{
	\begin{itemize}
	\end{itemize}
}
\item  \\
 the birds and (the) bees \textit{
	\begin{itemize}
	\end{itemize}
}
\item  \\
 do bird \textit{
	\begin{itemize}
	\end{itemize}
}
\item  \\
 birds of a feather \textit{
	\begin{itemize}
	\end{itemize}
}
\item  \\
 get/give so the bird \textit{
	\begin{itemize}
	\end{itemize}
}
\item  \\
 get/give so the bird \textit{
	\begin{itemize}
	\end{itemize}
}
\item  \\
 a bird in the hand \textit{
	\begin{itemize}
	\end{itemize}
}
\item  \\
 little bird \textit{
	\begin{itemize}
	\end{itemize}
}
\item  \\
 old bird \textit{
	\begin{itemize}
	\end{itemize}
}
\item  \\
 rare bird \textit{
	\begin{itemize}
	\end{itemize}
}
\item  \\
 to kill two birds with one stone \textit{
	\begin{itemize}
	\end{itemize}
}
\end{enumerate}

\section*{climb}
{\large \color{blue}  climbs  climbing  climbed  }
\subsection*{Explain}
\begin{enumerate}
\item verb \\
If you \textbf{climb} something such as a tree, mountain, or ladder , or \textbf{climb}  \textbf{up} it, you move towards the top of it. If you \textbf{climb}  \textbf{down} it, you move towards the bottom of it.
 \textbf{Climb} is also a noun .
 \textit{
	\begin{itemize}
	\item Climbing the first hill took half an hour.
	\item He picked up his suitcase and climbed the stairs.
	\item I told her about him climbing up the drainpipe.
	\item Kelly climbed down the ladder into the water.
	\item Children love to climb.
	\item ...an hour's leisurely climb through olive groves and vineyards.
	\end{itemize}
}
\item verb \\
If you \textbf{climb}  somewhere , you move there carefully, for example because you are moving into a small space or trying to avoid  falling .
 \textit{
	\begin{itemize}
	\item The girls hurried outside, climbed into the car, and drove off.
	\item He must have climbed out of his cot.
	\item He climbed down from the cab.
	\end{itemize}
}
\item verb \\
When something such as an aeroplane  \textbf{climbs} , it moves upwards to a higher position. When the sun  \textbf{climbs} , it moves higher in the sky .
 \textit{
	\begin{itemize}
	\item The plane continued to climb until it reached its cruising altitude.
	\end{itemize}
}
\item verb \\
When something \textbf{climbs} , it increases in value or amount.
 \textit{
	\begin{itemize}
	\item The nation's unemployment rate has been climbing steadily since last June.
	\item Prices have climbed by 21% since the beginning of the year.
	\item The FA Cup Final's audience climbed to 12.3 million.
	\item Jaguar shares climbed 43 pence to 510 pence.
	\end{itemize}
}
\end{enumerate}

\section*{brandy}
{\large \color{blue}  brandies  }
\subsection*{Explain}
\begin{enumerate}
\item variable noun \\
\textbf{Brandy} is a strong alcoholic drink. It is often drunk after a meal .
 A \textbf{brandy} is a glass of brandy.
 \textit{
	\begin{itemize}
	\item After a couple of brandies, Michael started telling me his life story.
	\end{itemize}
}
\end{enumerate}

\section*{contradict}
{\large \color{blue}  contradicts  contradicting  contradicted  }
\subsection*{Explain}
\begin{enumerate}
\item verb \\
If you \textbf{contradict} someone, you say that what they have just said is wrong , or suggest that it is wrong by saying something different.
 \textit{
	\begin{itemize}
	\item She dared not contradict him.
	\item His comments appeared to contradict remarks made earlier in the day by the chairman.
	\item He often talks in circles, frequently contradicting himself and often ends up saying
nothing.
	\end{itemize}
}
\item verb \\
If one statement or piece of evidence  \textbf{contradicts} another, the first one makes the second one appear to be wrong.
 \textit{
	\begin{itemize}
	\item The images on screen contradict his claims.
	\item The result seems to contradict a major U.S. study reported last November.
	\end{itemize}
}
\item verb \\
If one policy or situation  \textbf{contradicts} another, there is a conflict between them, and they cannot both exist or be successful .
 \textit{
	\begin{itemize}
	\item Mr Grant feels that the cut-backs contradict the Government's commitment to better
educational standards.
	\end{itemize}
}
\end{enumerate}

\section*{cage}
{\large \color{blue}  cages  }
\subsection*{Explain}
\begin{enumerate}
\item countable noun \\
A \textbf{cage} is a structure of wire or metal bars in which birds or animals are kept.
 \textit{
	\begin{itemize}
	\item I hate to see birds in cages.
	\end{itemize}
}
\item  \\
 rattle so's cage \textit{
	\begin{itemize}
	\end{itemize}
}
\end{enumerate}

\section*{could}
{\large \color{blue}  }
\subsection*{Explain}
\begin{enumerate}
\item modal verb \\
You use \textbf{could} to indicate that someone had the ability to do something. You use \textbf{could not} or \textbf{couldn't} to say that someone was unable to do something.
 \textit{
	\begin{itemize}
	\item For my return journey, I felt I could afford the extra and travel first class.
	\item I could see that something was terribly wrong.
	\item He could not resist telling her the truth.
	\item When I left school at 16, I couldn't read or write.
	\item There was no way she could have coped with a baby around.
	\end{itemize}
}
\item modal verb \\
You use \textbf{could} to indicate that something sometimes happened .
 \textit{
	\begin{itemize}
	\item Though he had a temper and could be nasty, it never lasted.
	\item He could be very pleasant when he wanted to.
	\end{itemize}
}
\item modal verb \\
You use \textbf{could have} to indicate that something was a possibility in the past , although it did not actually happen.
 \textit{
	\begin{itemize}
	\item He could have made a fortune as a lawyer.
	\item You could have been killed!
	\item He did not regret saying what he did but felt that he could have expressed it differently.
	\end{itemize}
}
\item modal verb \\
You use \textbf{could} to indicate that something is possibly true , or that it may possibly happen.
 \textit{
	\begin{itemize}
	\item Doctors told him the disease could have been caused by years of working in smokey
clubs.
	\item An improvement in living standards could be years away.
	\item He was jailed five years ago and could be released next year.
	\end{itemize}
}
\item modal verb \\
You use \textbf{could not} or \textbf{couldn't} to indicate that it is not possible that something is true.
 \textit{
	\begin{itemize}
	\item They argued all the time and thought it couldn't be good for the baby.
	\item Anne couldn't be expected to understand the situation.
	\item He couldn't have been more than fourteen years old.
	\end{itemize}
}
\item modal verb \\
You use \textbf{could} to talk about a possibility, ability, or opportunity that depends on other conditions.
 \textit{
	\begin{itemize}
	\item Their hope was that a new and better country could be born.
	\item I knew that if I spoke to Myra, I could get her to call my father.
	\end{itemize}
}
\item modal verb \\
You use \textbf{could} when you are saying that one thing or situation resembles another.
 \textit{
	\begin{itemize}
	\item The charming characters she draws look like they could have walked out of the 1920s.
	\end{itemize}
}
\item modal verb \\
You use \textbf{could} , or \textbf{couldn't} in questions , when you are making offers and suggestions.
 \textit{
	\begin{itemize}
	\item I could call the local doctor.
	\item We need money right? We could go around and ask if people need odd jobs done or something.
	\item 'It's boring to walk all alone.'—'Couldn't you go for walks with your friends?'.
	\item You could look for a career abroad where environmental jobs are better paid and more
secure.
	\item It would be a good idea if you could do this exercise twice or three times on separate
days.
	\end{itemize}
}
\item modal verb \\
You use \textbf{could} in questions when you are making a polite  request or asking for permission to do something. Speakers sometimes use \textbf{couldn't}  instead of 'could' to show that they realize that their request may be refused .
 \textit{
	\begin{itemize}
	\item Could I stay tonight?
	\item Could I speak to you in private a moment, John?
	\item I wonder if some time I could have a word with you.
	\item Sir, could you please come to the commanding officer's office?
	\item Could we go outside just for a second?
	\item He asked if he could have a cup of coffee.
	\item Couldn't I watch you do it?
	\end{itemize}
}
\item modal verb \\
People sometimes use structures with \textbf{if I could} or \textbf{could I} as polite ways of interrupting someone or of introducing what they are going to say next .
 \textit{
	\begin{itemize}
	\item Well, if I could just interject.
	\item Could I stop you there?
	\item Could I ask you if there have been any further problems?
	\item First of all, could I begin with an apology for a mistake I made last week?
	\end{itemize}
}
\item modal verb \\
You use \textbf{could} to say emphatically that someone ought to do the thing mentioned , especially when you are annoyed because they have not done it. You use \textbf{why couldn't} in questions to express your surprise or annoyance that someone has not done something.
 \textit{
	\begin{itemize}
	\item We've come to see you, so you could at least stand and greet us properly.
	\item Idiot! You could have told me!
	\item He could have written.
	\item Why couldn't she have said something?
	\item But why couldn't he tell me straight out?
	\end{itemize}
}
\item modal verb \\
You use \textbf{could} when you are expressing strong feelings about something by saying that you feel as if you want to do the thing mentioned, although you do not do it.
 \textit{
	\begin{itemize}
	\item I could kill you! I swear I could!
	\item 'Welcome back' was all they said. I could have kissed them!
	\item She could have screamed with tension.
	\end{itemize}
}
\item modal verb \\
You use \textbf{could} after 'if' when talking about something that you do not have the ability or opportunity
to do, but which you are imagining in order to consider what the likely  consequences  might be.
 \textit{
	\begin{itemize}
	\item If I could afford it I'd have four television sets.
	\item If only I could get some sleep, I would be able to cope.
	\end{itemize}
}
\item modal verb \\
You use \textbf{could not} or \textbf{couldn't} with comparatives to emphasize that someone or something has as much as is possible of a particular quality. For
example, if you say 'I couldn't be happier ', you mean that you are extremely happy.
 \textit{
	\begin{itemize}
	\item The rest of the players are a great bunch of lads and I couldn't be happier.
	\item Darling Neville, I couldn't be more pleased for you.
	\item The news couldn't have come at a better time.
	\end{itemize}
}
\item modal verb \\
In speech, you use \textbf{how could} in questions to emphasize that you feel strongly about something bad that has happened.
 \textit{
	\begin{itemize}
	\item How could you allow him to do something like that?
	\item How could I have been so stupid?
	\item How could she do this to me?
	\item How could you have lied to us all these years?
	\end{itemize}
}
\item convention \\
You say ' \textbf{I couldn't} ' to refuse an offer of more food or drink.
 \textit{
	\begin{itemize}
	\item 'More cake?'—'Oh no, I couldn't.'
	\end{itemize}
}
\end{enumerate}

\section*{cluster}
{\large \color{blue}  clusters  clustering  clustered  }
\subsection*{Explain}
\begin{enumerate}
\item countable noun \\
A \textbf{cluster}  \textbf{of} people or things is a small group of them close together.
 \textit{
	\begin{itemize}
	\item ...clusters of men in formal clothes.
	\item There's no town here, just a cluster of shops, cabins and motels at the side of the
highway.
	\end{itemize}
}
\item verb \\
If people \textbf{cluster}  \textbf{together} , they gather together in a small group.
 \textit{
	\begin{itemize}
	\item The passengers clustered together in small groups.
	\item The children clustered around me.
	\end{itemize}
}
\end{enumerate}

\section*{decline}
{\large \color{blue}  declines  declining  declined  }
\subsection*{Explain}
\begin{enumerate}
\item verb \\
If something \textbf{declines} , it becomes less in quantity, importance , or strength .
 \textit{
	\begin{itemize}
	\item The number of staff has declined from 217,000 to 114,000.
	\item Hourly output by workers declined 1.3% in the first quarter.
	\item Union membership and union power are declining fast.
	\item ...a declining birth rate.
	\end{itemize}
}
\item verb \\
If you \textbf{decline} something or \textbf{decline}  \textbf{to} do something, you politely refuse to accept it or do it.
 \textit{
	\begin{itemize}
	\item He declined their invitation.
	\item The band declined to comment on the story.
	\item He offered the boys some coffee. They declined politely.
	\end{itemize}
}
\item variable noun \\
If there is a \textbf{decline}  \textbf{in} something, it becomes less in quantity, importance, or quality.
 \textit{
	\begin{itemize}
	\item There wasn't such a big decline in enrollments after all.
	\item ...Rome's decline in the fifth century.
	\item The first signs of economic decline became visible.
	\end{itemize}
}
\item  \\
 in decline/on the decline \textit{
	\begin{itemize}
	\end{itemize}
}
\item  \\
 into decline \textit{
	\begin{itemize}
	\end{itemize}
}
\end{enumerate}

\section*{code}
{\large \color{blue}  codes  coding  coded  }
\subsection*{Explain}
\begin{enumerate}
\item countable noun \\
A \textbf{code} is a set of rules about how people should behave or about how something must be done .
 \textit{
	\begin{itemize}
	\item ...Article 159 of the Turkish penal code.
	\item ...the code of the Samurai.
	\item ...local building codes.
	\end{itemize}
}
\item countable noun \\
A \textbf{code} is a system of replacing the words in a message with other words or symbols, so that nobody can understand it unless they know the system.
 \textit{
	\begin{itemize}
	\item They used elaborate secret codes, as when the names of trees stood for letters.
	\item If you can't remember your number, write it in code in a diary.
	\end{itemize}
}
\item countable noun \\
A \textbf{code} is a group of numbers or letters which is used to identify something, such as a postal  address or part of a telephone system.
 \textit{
	\begin{itemize}
	\item Callers dialling the wrong area code will not get through.
	\end{itemize}
}
\item countable noun \\
A \textbf{code} is any system of signs or symbols that has a meaning .
 \textit{
	\begin{itemize}
	\item Only one voucher or promotional code can be used per transaction.
	\end{itemize}
}
\item countable noun \\
The genetic  \textbf{code} of a person, animal or plant is the information contained in DNA which determines the structure and function of cells , and the inherited  characteristics of all living things.
 \textit{
	\begin{itemize}
	\item ...the genetic code that determines every bodily feature.
	\end{itemize}
}
\item verb \\
If a gene  \textbf{codes for} something such as a substance or characteristic, it creates or determines it.
 \textit{
	\begin{itemize}
	\item These genes code for proteins that appear to play a role in appetite control.
	\item ...the genes that code for facial appearance.
	\end{itemize}
}
\item verb \\
To \textbf{code} something means to give it a code or to mark it with its code.
 \textit{
	\begin{itemize}
	\item He devised a way of coding every statement uniquely.
	\item The machines were coded so we could tell them apart.
	\end{itemize}
}
\item uncountable noun \\
Computer \textbf{code} is a system or language for expressing information and instructions in a form which can be understood by a computer.
 \textit{
	\begin{itemize}
	\end{itemize}
}
\item verb \\
To \textbf{code} means to write programs and instructions for a computer using computer code.
 \textit{
	\begin{itemize}
	\item Learning to code may be the fastest way into employment for young people.
	\item The course teaches you to code a multi-platform web app in just one day.
	\end{itemize}
}
\end{enumerate}

\section*{descend}
{\large \color{blue}  descends  descending  descended  }
\subsection*{Explain}
\begin{enumerate}
\item verb \\
If you \textbf{descend} or if you \textbf{descend} a staircase, you move downwards from a higher to a lower level.
 \textit{
	\begin{itemize}
	\item Things are cooler and more damp as we descend to the cellar.
	\item She descended one flight of stairs.
	\end{itemize}
}
\item verb \\
When a mood or atmosphere  \textbf{descends}  \textbf{on} a place or on the people there, it affects them by spreading among them.
 \textit{
	\begin{itemize}
	\item An uneasy calm descended on the area.
	\item A reverent hush descended on the multitude.
	\end{itemize}
}
\item verb \\
If a large group of people arrive to see you, especially if their visit is unexpected or causes you a lot of work, you can say that they \textbf{have descended}  \textbf{on} you.
 \textit{
	\begin{itemize}
	\item 3,000 city officials descended on Capitol Hill to lobby for more money.
	\item Curious tourists and reporters from around the globe are descending upon the peaceful
villages.
	\end{itemize}
}
\item verb \\
When night , dusk , or darkness \textbf{descends} , it starts to get  dark .
 \textit{
	\begin{itemize}
	\item Darkness has now descended and the moon and stars shine hazily in the clear sky.
	\end{itemize}
}
\item verb \\
If you say that someone \textbf{descends to} behaviour which you consider unacceptable , you are expressing your disapproval of the fact that they do it.
 \textit{
	\begin{itemize}
	\item We're not going to descend to such methods.
	\item She's got too much dignity to descend to writing anonymous letters.
	\end{itemize}
}
\item verb \\
When you want to emphasize that the situation that someone is entering is very bad , you can say that they \textbf{are descending into} that situation.
 \textit{
	\begin{itemize}
	\item He was ultimately overthrown and the country descended into chaos.
	\end{itemize}
}
\end{enumerate}

\section*{crust}
{\large \color{blue}  crusts  }
\subsection*{Explain}
\begin{enumerate}
\item countable noun \\
The \textbf{crust} on a loaf of bread is the outside part.
 \textit{
	\begin{itemize}
	\end{itemize}
}
\item countable noun \\
A pie's \textbf{crust} is its cooked  pastry .
 \textit{
	\begin{itemize}
	\end{itemize}
}
\item countable noun \\
A \textbf{crust} is a hard layer of something, especially on top of a softer or wetter substance.
 \textit{
	\begin{itemize}
	\item As the water evaporates, a crust of salt is left on the surface of the soil.
	\end{itemize}
}
\item countable noun \\
The Earth's \textbf{crust} is its outer layer.
 \textit{
	\begin{itemize}
	\item Earthquakes leave scars in the Earth's crust.
	\end{itemize}
}
\item  \\
 to earn a crust \textit{
	\begin{itemize}
	\end{itemize}
}
\end{enumerate}

\section*{dictate}
{\large \color{blue}  dictates  dictating  dictated  }
\subsection*{Explain}
\begin{enumerate}
\item verb \\
If you \textbf{dictate} something, you say or read it aloud for someone else to write down.
 \textit{
	\begin{itemize}
	\item Sheldon writes every day of the week, dictating his novels in the morning.
	\item Everything he dictated was signed and sent out the same day.
	\end{itemize}
}
\item verb \\
If someone \textbf{dictates}  \textbf{to} someone else, they tell them what they should do or can do.
 \textit{
	\begin{itemize}
	\item We don't want to dictate to anyone how to live their lives.
	\item What right has one country to dictate the environmental standards of another?
	\item He cannot be allowed to dictate what can and cannot be inspected.
	\item What gives them the right to dictate to us what we should eat?
	\item The officers were more or less able to dictate terms to successive governments.
	\item The rules of court dictate that a defendant is entitled to all evidence which may
help his case.
	\end{itemize}
}
\item verb \\
If one thing \textbf{dictates} another, the first thing causes or influences the second thing.
 \textit{
	\begin{itemize}
	\item The film's budget dictated a tough schedule.
	\item The way in which they dress is dictated by very rigid fashion rules.
	\item Of course, a number of factors will dictate how long an apple tree can survive.
	\item Circumstances dictated that they played a defensive rather than attacking game.
	\end{itemize}
}
\item verb \\
You say that reason or common  sense  \textbf{dictates}  \textbf{that} a particular thing is the case when you believe strongly that it is the case and that reason or common sense will cause other people
to agree .
 \textit{
	\begin{itemize}
	\item Commonsense now dictates that it would be wise to sell a few shares.
	\end{itemize}
}
\item countable noun \\
A \textbf{dictate} is an order which you have to obey .
 \textit{
	\begin{itemize}
	\item Their job is to ensure that the dictates of the Party are followed.
	\end{itemize}
}
\item countable noun \\
\textbf{Dictates} are principles or rules which you consider to be extremely  important .
 \textit{
	\begin{itemize}
	\item We have followed the dictates of our consciences and have done our duty.
	\end{itemize}
}
\end{enumerate}

\section*{daytime}
{\large \color{blue}  }
\subsection*{Explain}
\begin{enumerate}
\item singular noun \\
\textbf{The}  \textbf{daytime} is the part of a day between the time when it gets  light and the time when it gets dark .
 \textit{
	\begin{itemize}
	\item In the daytime he stayed up in his room, sleeping, or listening to music.
	\item Please give a daytime telephone number.
	\item His rant against using car headlights in daytime is quite unjustified.
	\end{itemize}
}
\item adjective \\
\textbf{Daytime}  television and radio is broadcast during the morning and afternoon on weekdays .
 \textit{
	\begin{itemize}
	\item ...ITV's new package of daytime programmes.
	\end{itemize}
}
\end{enumerate}

\section*{differentiate}
{\large \color{blue}  differentiates  differentiating  differentiated  }
\subsection*{Explain}
\begin{enumerate}
\item verb \\
If you \textbf{differentiate}  \textbf{between} things or if you \textbf{differentiate} one thing \textbf{from} another, you recognize or show the difference between them.
 \textit{
	\begin{itemize}
	\item A child may not differentiate between his imagination and the real world.
	\item At this age your baby cannot differentiate one person from another.
	\end{itemize}
}
\item verb \\
A quality or feature that \textbf{differentiates} one thing \textbf{from} another makes the two things different .
 \textit{
	\begin{itemize}
	\item ...distinctive policies that differentiate them from the other parties.
	\end{itemize}
}
\end{enumerate}

\section*{delegate}
{\large \color{blue}  delegates  delegating  delegated  }
\subsection*{Explain}
\begin{enumerate}
\item countable noun \\
A \textbf{delegate} is a person who is chosen to vote or make decisions on behalf of a group of other people, especially at a conference or a meeting.
 \textit{
	\begin{itemize}
	\end{itemize}
}
\item verb \\
If you \textbf{delegate} duties, responsibilities , or power \textbf{to} someone, you give them those duties, those responsibilities, or that power so that
they can act on your behalf.
 \textit{
	\begin{itemize}
	\item He plans to delegate more authority to his deputies.
	\item How many of their activities can be safely and effectively delegated to less trained
staff?
	\item Many employers find it hard to delegate.
	\end{itemize}
}
\item verb \\
If you \textbf{are delegated}  \textbf{to} do something, you are given the duty of acting on someone else's behalf by making decisions, voting, or doing
some particular work.
 \textit{
	\begin{itemize}
	\item Officials have now been delegated to start work on a draft settlement.
	\end{itemize}
}
\end{enumerate}

\section*{disappear}
{\large \color{blue}  disappears  disappearing  disappeared  }
\subsection*{Explain}
\begin{enumerate}
\item verb \\
If you say that someone or something \textbf{disappears} , you mean that you can no longer see them, usually because you or they have changed position.
 \textit{
	\begin{itemize}
	\item The black car drove away from them and disappeared.
	\item Clive disappeared into a room by himself.
	\item The airliner disappeared off their radar.
	\end{itemize}
}
\item verb \\
If someone or something \textbf{disappears} , they go away or are taken away somewhere where nobody can find them.
 \textit{
	\begin{itemize}
	\item ...a Japanese woman who disappeared thirteen years ago.
	\item Janet's sister noticed that small kitchen objects were disappearing.
	\end{itemize}
}
\item verb \\
If something \textbf{disappears} , it stops existing or happening .
 \textit{
	\begin{itemize}
	\item The immediate security threat has disappeared.
	\end{itemize}
}
\end{enumerate}

\section*{deputy}
{\large \color{blue}  deputies  }
\subsection*{Explain}
\begin{enumerate}
\item countable noun \\
A \textbf{deputy} is the second most important person in an organization such as a business or government department . Someone's deputy often acts on their behalf when they are not there.
 \textit{
	\begin{itemize}
	\item ...France's minister for culture and his deputy.
	\item ...the academy's deputy director, Vladimir Kudryatsev.
	\end{itemize}
}
\item countable noun \\
In some parliaments or law-making bodies, the elected members are called  \textbf{deputies} .
 \textit{
	\begin{itemize}
	\item The president appealed to deputies to approve the plan quickly.
	\end{itemize}
}
\end{enumerate}

\section*{discriminate}
{\large \color{blue}  discriminates  discriminating  discriminated  }
\subsection*{Explain}
\begin{enumerate}
\item verb \\
If you can \textbf{discriminate}  \textbf{between} two things, you can recognize that they are different .
 \textit{
	\begin{itemize}
	\item He is incapable of discriminating between a good idea and a terrible one.
	\item The device can discriminate between the cancerous and the normal cells.
	\end{itemize}
}
\item verb \\
To \textbf{discriminate}  \textbf{against} a group of people or \textbf{in favour of} a group of people means to unfairly treat them worse or better than other groups.
 \textit{
	\begin{itemize}
	\item They believe the law discriminates against women.
	\item ...legislation which would discriminate in favour of racial minorities.
	\item The Commission for Racial Equality teaches organisations not to discriminate.
	\end{itemize}
}
\end{enumerate}

\section*{disgrace}
{\large \color{blue}  disgraces  disgracing  disgraced  }
\subsection*{Explain}
\begin{enumerate}
\item uncountable noun \\
If you say that someone is \textbf{in}  \textbf{disgrace} , you are emphasizing that other people disapprove of them and do not respect them because of something that they have done .
 \textit{
	\begin{itemize}
	\item His vice president also had to resign in disgrace.
	\end{itemize}
}
\item singular noun \\
If you say that something is \textbf{a disgrace} , you are emphasizing that it is very bad or wrong , and that you find it completely unacceptable .
 \textit{
	\begin{itemize}
	\item The way the sales were handled was a complete disgrace.
	\item The national airline is a disgrace.
	\end{itemize}
}
\item singular noun \\
You say that someone is \textbf{a disgrace}  \textbf{to} someone else when you want to emphasize that their behaviour causes the other person to feel  ashamed .
 \textit{
	\begin{itemize}
	\item Republican leaders called him a disgrace to the party.
	\item What went on was a scandal. It was a disgrace to Britain.
	\end{itemize}
}
\item verb \\
If you say that someone \textbf{disgraces} someone else, you are emphasizing that their behaviour causes the other person to
feel ashamed.
 \textit{
	\begin{itemize}
	\item I have disgraced my family's name.
	\item I've disgraced myself by the actions I've taken.
	\end{itemize}
}
\end{enumerate}

\section*{distract}
{\large \color{blue}  distracts  distracting  distracted  }
\subsection*{Explain}
\begin{enumerate}
\item verb \\
If something \textbf{distracts} you or your attention \textbf{from} something, it takes your attention away from it.
 \textit{
	\begin{itemize}
	\item Tom admits that playing video games sometimes distracts him from his homework.
	\item Don't let yourself be distracted by fashionable theories.
	\item A disturbance in the street distracted my attention.
	\end{itemize}
}
\end{enumerate}

\section*{disposal}
{\large \color{blue}  }
\subsection*{Explain}
\begin{enumerate}
\item  \\
 at one's disposal \textit{
	\begin{itemize}
	\end{itemize}
}
\item uncountable noun \\
\textbf{Disposal} is the act of getting rid of something that is no longer wanted or needed .
 \textit{
	\begin{itemize}
	\item ...methods for the permanent disposal of radioactive waste.
	\item ...waste disposal sites.
	\end{itemize}
}
\end{enumerate}

\section*{distribute}
{\large \color{blue}  distributes  distributing  distributed  }
\subsection*{Explain}
\begin{enumerate}
\item verb \\
If you \textbf{distribute} things, you hand them or deliver them to a number of people.
 \textit{
	\begin{itemize}
	\item Students shouted slogans and distributed leaflets.
	\item Thousands of soldiers are working to distribute food and blankets to the refugees.
	\item In the move most of the furniture was left to the neighbours or distributed among
friends.
	\end{itemize}
}
\item verb \\
When a company \textbf{distributes} goods, it supplies them to the shops or businesses that sell them.
 \textit{
	\begin{itemize}
	\item We didn't understand how difficult it was to distribute a national paper.
	\end{itemize}
}
\item verb \\
If you \textbf{distribute} things \textbf{among the} members of a group, you share them among those members.
 \textit{
	\begin{itemize}
	\item After his election he distributed major offices among his friends and supporters.
	\end{itemize}
}
\item verb \\
To \textbf{distribute} a substance \textbf{over} something means to scatter it over it.
 \textit{
	\begin{itemize}
	\item Distribute the topping evenly over the fruit.
	\end{itemize}
}
\end{enumerate}

\section*{famine}
{\large \color{blue}  famines  }
\subsection*{Explain}
\begin{enumerate}
\item variable noun \\
\textbf{Famine} is a situation in which large numbers of people have little or no food, and many of them die .
 \textit{
	\begin{itemize}
	\item Thousands of refugees are trapped by war, drought and famine.
	\item The civil war is obstructing distribution of famine relief by aid agencies.
	\end{itemize}
}
\end{enumerate}

\section*{divide}
{\large \color{blue}  divides  dividing  divided  }
\subsection*{Explain}
\begin{enumerate}
\item verb \\
When people or things \textbf{are divided} or \textbf{divide}  \textbf{into} smaller groups or parts, they become separated into smaller parts.
 \textit{
	\begin{itemize}
	\item The physical benefits of exercise can be divided into three factors.
	\item It will be easiest if we divide them into groups.
	\item Divide the pastry in half and roll out each piece.
	\item We divide into pairs and each pair takes a region.
	\item Bacteria reproduce by dividing and making copies of themselves.
	\end{itemize}
}
\item verb \\
If you \textbf{divide} something \textbf{among} people or things, you separate it into several parts or quantities which you distribute
to the people or things.
 \textit{
	\begin{itemize}
	\item Paul divides most of his spare time between the study and his bedroom.
	\item Divide the sauce among 4 bowls.
	\end{itemize}
}
\item verb \\
If you \textbf{divide} a larger number \textbf{by} a smaller number or \textbf{divide} a smaller number \textbf{into} a larger number, you calculate how many times the smaller number can fit  exactly into the larger number.
 \textit{
	\begin{itemize}
	\item Measure the floor area of the greenhouse and divide it by six.
	\end{itemize}
}
\item verb \\
If a border or line \textbf{divides} two areas or \textbf{divides} an area into two, it keeps the two areas separate from each other.
 \textit{
	\begin{itemize}
	\item the decision to divide the country into autonomous regions.
	\item ...the artificial line that divides the city.
	\item ...the long frontier dividing Mexico from the United States.
	\end{itemize}
}
\item verb \\
If people \textbf{divide} over something or if something \textbf{divides} them, it causes strong  disagreement between them.
 \textit{
	\begin{itemize}
	\item She has done more to divide the Conservatives than anyone else.
	\item The democrats are divided over whether to admit him into their group.
	\item The party is likely to divide along ideological lines.
	\end{itemize}
}
\item countable noun \\
A \textbf{divide} is a significant  distinction between two groups, often one that causes conflict .
 \textit{
	\begin{itemize}
	\item ...a deliberate attempt to create a Hindu–Muslim divide in India.
	\end{itemize}
}
\item countable noun \\
A \textbf{divide} is a moment in time or a point in a process when there is a complete change from one situation to another.
 \textit{
	\begin{itemize}
	\item The time had come to cross the great divide between formality and truth.
	\end{itemize}
}
\item countable noun \\
A \textbf{divide} is a line of high ground between areas that are drained by different  rivers .
 \textit{
	\begin{itemize}
	\end{itemize}
}
\item  \\
 divide and rule \textit{
	\begin{itemize}
	\end{itemize}
}
\end{enumerate}

\section*{foreigner}
{\large \color{blue}  foreigners  }
\subsection*{Explain}
\begin{enumerate}
\item countable noun \\
A \textbf{foreigner} is someone who belongs to a country that is not your own.
 \textit{
	\begin{itemize}
	\item They are discouraged from becoming close friends with foreigners.
	\end{itemize}
}
\end{enumerate}

\section*{dominate}
{\large \color{blue}  dominates  dominating  dominated  }
\subsection*{Explain}
\begin{enumerate}
\item verb \\
To \textbf{dominate} a situation  means to be the most powerful or important person or thing in it.
 \textit{
	\begin{itemize}
	\item The book is expected to dominate the best-seller lists.
	\item ...countries where life is dominated by war.
	\item No single factor appears to dominate.
	\end{itemize}
}
\item verb \\
If one country or person \textbf{dominates} another, they have power over them.
 \textit{
	\begin{itemize}
	\item He denied that his country wants to dominate Europe.
	\item Women are no longer dominated by the men in their relationships.
	\item The countries of Eastern Europe immediately started to dominate.
	\end{itemize}
}
\item verb \\
If a building , mountain , or other object \textbf{dominates} an area, it is so large or impressive that you cannot avoid  seeing it.
 \textit{
	\begin{itemize}
	\item It's one of the biggest buildings in this area, and it really dominates this whole
place.
	\item ...its skyline dominated by the central mosque.
	\end{itemize}
}
\end{enumerate}

\section*{haste}
{\large \color{blue}  }
\subsection*{Explain}
\begin{enumerate}
\item uncountable noun \\
\textbf{Haste} is the quality of doing something quickly, sometimes too quickly so that you are
careless and make mistakes .
 \textit{
	\begin{itemize}
	\item In their haste to escape the rising water, they dropped some expensive equipment.
	\item The translations bear the signs of inaccuracy and haste.
	\end{itemize}
}
\item  \\
 in haste \textit{
	\begin{itemize}
	\end{itemize}
}
\item  \\
 make haste \textit{
	\begin{itemize}
	\end{itemize}
}
\end{enumerate}

\section*{drive}
{\large \color{blue}  drives  driving  drove  driven  }
\subsection*{Explain}
\begin{enumerate}
\item verb \\
When you \textbf{drive}  somewhere , you operate a car or other vehicle and control its movement and direction.
 \textit{
	\begin{itemize}
	\item I drove into town and went to a restaurant for dinner.
	\item He put the bags in the car and drove off.
	\item She never learned to drive.
	\item Mrs Glick drove her own car and the girls went in Nancy's convertible.
	\end{itemize}
}
\item verb \\
If you \textbf{drive} someone somewhere, you take them there in a car or other vehicle.
 \textit{
	\begin{itemize}
	\item His daughter Carly drove him to the train station.
	\end{itemize}
}
\item countable noun \\
A \textbf{drive} is a journey in a car or other vehicle.
 \textit{
	\begin{itemize}
	\item I thought we might go for a drive on Sunday.
	\end{itemize}
}
\item countable noun \\
A \textbf{drive} is a wide piece of hard ground, or sometimes a private road, that leads from the
road to a person's house.
 \textit{
	\begin{itemize}
	\end{itemize}
}
\item verb \\
If something \textbf{drives} a machine, it supplies the power that makes it work.
 \textit{
	\begin{itemize}
	\item The current flows into electric motors that drive the wheels.
	\end{itemize}
}
\item uncountable noun \\
\textbf{Drive} is the power supplied by the engine to particular wheels in a car or other vehicle
to make the vehicle move.
 \textit{
	\begin{itemize}
	\item He put the jeep in four-wheel drive and splashed up the slope.
	\end{itemize}
}
\item countable noun \\
You use \textbf{drive} to refer to the mechanical part of a computer which reads the data on disks and tapes , or writes data onto them.
 \textit{
	\begin{itemize}
	\item The easiest way to back up your computer is with an external hard disk drive.
	\end{itemize}
}
\item verb \\
If you \textbf{drive} something such as a nail  \textbf{into} something else, you push it in or hammer it in using a lot of effort.
 \textit{
	\begin{itemize}
	\item I used a sledgehammer to drive the pegs into the ground.
	\item I held it still and drove in a nail.
	\end{itemize}
}
\item verb \\
In games such as cricket, golf , or football , if a player \textbf{drives} a ball somewhere, they kick or hit it there with a lot of force.
 \textit{
	\begin{itemize}
	\item Armstrong drove the ball into the roof of the net.
	\end{itemize}
}
\item countable noun \\
In golf, a \textbf{drive} is the first stroke a player makes from the tee.
 \textit{
	\begin{itemize}
	\item Woosnam sliced his drive into the bushes.
	\end{itemize}
}
\item verb \\
If the wind, rain , or snow  \textbf{drives} in a particular direction, it moves with great force in that direction.
 \textit{
	\begin{itemize}
	\item Rain drove against the window.
	\end{itemize}
}
\item verb \\
If you \textbf{drive} people or animals somewhere, you make them go to or from that place.
 \textit{
	\begin{itemize}
	\item The last offensive drove thousands of people into the hills.
	\item Every summer the shepherds drive the sheep up to pasture.
	\item The smoke also drove mosquitoes away.
	\end{itemize}
}
\item verb \\
To \textbf{drive} someone \textbf{into} a particular state or situation means to force them into that state or situation.
 \textit{
	\begin{itemize}
	\item The recession and hospital bills drove them into bankruptcy.
	\item He nearly drove Elsie mad with his fussing.
	\end{itemize}
}
\item verb \\
The desire or feeling that \textbf{drives} a person \textbf{to} do something, especially something extreme, is the desire or feeling that causes them to do it.
 \textit{
	\begin{itemize}
	\item More than once, depression drove him to attempt suicide.
	\item Jealousy drives people to murder.
	\item ...people who are driven by guilt, resentment and anxiety.
	\item ...a man driven by a pathological need to win.
	\end{itemize}
}
\item uncountable noun \\
If you say that someone has \textbf{drive} , you mean they have energy and determination .
 \textit{
	\begin{itemize}
	\item John will be best remembered for his drive and enthusiasm.
	\end{itemize}
}
\item countable noun \\
A \textbf{drive} is a very strong need or desire in human beings that makes them act in particular
ways.
 \textit{
	\begin{itemize}
	\item ...compelling, dynamic sex drives.
	\end{itemize}
}
\item singular noun \\
A \textbf{drive} is a special effort made by a group of people for a particular purpose.
 \textit{
	\begin{itemize}
	\item They plan to launch a nationwide recruitment drive.
	\item The country is leading the drive towards a low-carbon economy.
	\end{itemize}
}
\item countable noun \\
\textbf{Drive} is used in the names of some streets.
 \textit{
	\begin{itemize}
	\item ...23 Queen's Drive, Malvern, Worcestershire.
	\end{itemize}
}
\item  \\
 what someone is driving at \textit{
	\begin{itemize}
	\end{itemize}
}
\end{enumerate}

\section*{host}
{\large \color{blue}  hosts  hosting  hosted  }
\subsection*{Explain}
\begin{enumerate}
\item countable noun \\
The \textbf{host} at a party is the person who has invited the guests and provides the food, drink, or entertainment .
 \textit{
	\begin{itemize}
	\item Apart from my host, I didn't know a single person there.
	\item Tommy was always the perfect host.
	\end{itemize}
}
\item verb \\
If someone \textbf{hosts} a party, dinner , or other function, they have invited the guests and provide the food, drink, or
entertainment.
 \textit{
	\begin{itemize}
	\item Tonight she hosts a ball for 300 guests.
	\item ...a banquet hosted by the president of Kazakhstan.
	\end{itemize}
}
\item countable noun \\
A country, city, or organization that is the \textbf{host} of an event provides the facilities for that event to take place.
 \textit{
	\begin{itemize}
	\item ...the host of the World Cup.
	\item ...a preliminary qualifying tournament in Andorra involving the host country.
	\end{itemize}
}
\item verb \\
If a country, city, or organization \textbf{hosts} an event, they provide the facilities for the event to take place.
 \textit{
	\begin{itemize}
	\item Cannes hosts the annual film festival.
	\end{itemize}
}
\item  \\
 to play host \textit{
	\begin{itemize}
	\end{itemize}
}
\item countable noun \\
The \textbf{host} of a radio or television show is the person who introduces it and talks to the people who appear in it.
 \textit{
	\begin{itemize}
	\item I am host of a live radio programme.
	\end{itemize}
}
\item verb \\
The person who \textbf{hosts} a radio or television show introduces it and talks to the people who appear in it.
 \textit{
	\begin{itemize}
	\item She also hosts a show on St Petersburg Radio.
	\end{itemize}
}
\item quantifier \\
A \textbf{host}  \textbf{of} things is a lot of them.
 \textit{
	\begin{itemize}
	\item ...a host of problems.
	\item Today we have radios, TVs, and a whole host of gadgets powered by electricity.
	\end{itemize}
}
\item countable noun \\
If an area is \textbf{host}  \textbf{to} living things, those creatures  live and feed in that area.
 \textit{
	\begin{itemize}
	\item Uganda's beautiful highlands are host to a wide range of wildlife.
	\end{itemize}
}
\item countable noun \\
A \textbf{host} or a \textbf{host computer} is the main computer in a network of computers, which controls the most important
 files and programs .
 \textit{
	\begin{itemize}
	\end{itemize}
}
\item countable noun \\
The \textbf{host} of a parasite is the plant or animal which it lives on or inside and from which it gets its food.
 \textit{
	\begin{itemize}
	\item When the eggs hatch the larvae eat the living flesh of the host animal.
	\end{itemize}
}
\item countable noun \\
\textbf{The}  \textbf{Host} is the bread which is used to represent the body of Christ in Christian church services such as Holy  Communion .
 \textit{
	\begin{itemize}
	\end{itemize}
}
\end{enumerate}

\section*{exploit}
{\large \color{blue}  exploits  exploiting  exploited  }
\subsection*{Explain}
\begin{enumerate}
\item verb \\
If you say that someone \textbf{is exploiting} you, you think that they are treating you unfairly by using your work or ideas and giving you very little in return .
 \textit{
	\begin{itemize}
	\item Critics claim he exploited black musicians for personal gain.
	\item ...the plight of the exploited sugar cane workers.
	\end{itemize}
}
\item verb \\
If you say that someone \textbf{is exploiting} a situation, you disapprove of them because they are using it to gain an advantage for themselves, rather than trying to help other people or do what is right.
 \textit{
	\begin{itemize}
	\item The government and its opponents compete to exploit the troubles to their advantage.
	\end{itemize}
}
\item verb \\
If you \textbf{exploit} something, you use it well , and achieve something or gain an advantage from it.
 \textit{
	\begin{itemize}
	\item You'll need a good aerial to exploit the radio's performance.
	\item Cary is hoping to exploit new opportunities in Europe.
	\item So you feel that your skills have never been fully appreciated or exploited?
	\end{itemize}
}
\item verb \\
To \textbf{exploit}  resources or raw materials means to develop them and use them for industry or commercial activities.
 \textit{
	\begin{itemize}
	\item I think we're being very short sighted in not exploiting our own coal.
	\end{itemize}
}
\item countable noun \\
If you refer to someone's \textbf{exploits} , you mean the brave , interesting , or amusing things that they have done.
 \textit{
	\begin{itemize}
	\item His wartime exploits were later made into a film.
	\end{itemize}
}
\end{enumerate}

\section*{idiot}
{\large \color{blue}  idiots  }
\subsection*{Explain}
\begin{enumerate}
\item countable noun \\
If you call someone an \textbf{idiot} , you are showing that you think they are very stupid or have done something very stupid.
 \textit{
	\begin{itemize}
	\item I knew I'd been an idiot to stay there.
	\end{itemize}
}
\item adjective \\
\textbf{Idiot} means stupid.
 \textit{
	\begin{itemize}
	\end{itemize}
}
\item countable noun \\
In the past , people who had something wrong with their brains that made them seem less intelligent , or different from other people, were sometimes called \textbf{idiots} .
 \textit{
	\begin{itemize}
	\end{itemize}
}
\end{enumerate}

\section*{fasten}
{\large \color{blue}  fastens  fastening  fastened  }
\subsection*{Explain}
\begin{enumerate}
\item verb \\
When you \textbf{fasten} something, you close it by means of buttons or a strap , or some other device. If something \textbf{fastens} with buttons or straps, you can close it in this way.
 \textit{
	\begin{itemize}
	\item She got quickly into her Mini and fastened the seat-belt.
	\item Her long fair hair was fastened at the nape of her neck by an elastic band.
	\item ...the dress, which fastens with a long back zip.
	\end{itemize}
}
\item verb \\
If you \textbf{fasten} one thing \textbf{to} another, you attach the first thing to the second , for example with a piece of string or tape .
 \textit{
	\begin{itemize}
	\item There were no instructions on how to fasten the carrying strap to the box.
	\item Mamma fastened the picture on the wall.
	\end{itemize}
}
\item ergative verb \\
If someone or something \textbf{fastens} your attention  \textbf{on} a particular thing or if your attention \textbf{fastens}  \textbf{on} it, you start to concentrate on it rather than on anything else.
 \textit{
	\begin{itemize}
	\item More and more her memory and all her thoughts fastened on one event.
	\item The discovery has fastened public attention on the possibilities of DNA analysis
for resolving mysteries.
	\end{itemize}
}
\item verb \\
If someone or something \textbf{fastens}  \textbf{on} a particular thing, they start to concentrate on it.
 \textit{
	\begin{itemize}
	\item My mind attempted to calm itself by fastening on this trivial detail.
	\item It's a gross over-simplification to fasten on to the red deer as a threat to the
environment.
	\end{itemize}
}
\item verb \\
If someone \textbf{fastens}  \textbf{on} you, they keep  following , talking to, or staying with you, when you want them to go  away .
 \textit{
	\begin{itemize}
	\end{itemize}
}
\end{enumerate}

\section*{impression}
{\large \color{blue}  impressions  }
\subsection*{Explain}
\begin{enumerate}
\item countable noun \\
Your \textbf{impression} of a person or thing is what you think they are like , usually after having seen or heard them. Your \textbf{impression} of a situation is what you think is going on.
 \textit{
	\begin{itemize}
	\item What were your first impressions of college?
	\item My impression is that they are totally out of control.
	\item There was a general impression that tomorrow meant a fresh start.
	\end{itemize}
}
\item singular noun \\
If someone gives you a particular \textbf{impression} , they cause you to believe that something is the case , often when it is not.
 \textit{
	\begin{itemize}
	\item I don't want to give the impression that I'm running away from the charges.
	\item He cleverly inserted mirrors above the window to create an impression of space.
	\end{itemize}
}
\item countable noun \\
An \textbf{impression} is an amusing imitation of someone's behaviour or way of talking , usually someone well-known .
 \textit{
	\begin{itemize}
	\item At college in Glasgow he did impressions of teachers, but was otherwise well behaved.
	\end{itemize}
}
\item countable noun \\
An \textbf{impression} of an object is a mark or outline that it has left after being pressed  hard onto a surface.
 \textit{
	\begin{itemize}
	\item ...the world's oldest fossil impressions of plant life.
	\end{itemize}
}
\item  \\
 make an impression \textit{
	\begin{itemize}
	\end{itemize}
}
\item  \\
 under the impression \textit{
	\begin{itemize}
	\end{itemize}
}
\end{enumerate}

\section*{hail}
{\large \color{blue}  hails  hailing  hailed  }
\subsection*{Explain}
\begin{enumerate}
\item verb \\
If a person, event , or achievement  \textbf{is hailed}  \textbf{as}  important or successful , they are praised  publicly .
 \textit{
	\begin{itemize}
	\item He has been hailed as the best centreback in the land.
	\item U.S. magazines hailed her as the greatest rock'n'roll singer in the world.
	\item The deal was hailed by the Defence Secretary.
	\end{itemize}
}
\item uncountable noun \\
\textbf{Hail} consists of small balls of ice that fall like rain from the sky .
 \textit{
	\begin{itemize}
	\item ...a sharp short-lived storm with heavy hail.
	\end{itemize}
}
\item verb \\
When \textbf{it hails} , hail falls like rain from the sky.
 \textit{
	\begin{itemize}
	\item It started to hail, huge great stones.
	\end{itemize}
}
\item singular noun \\
A \textbf{hail of} things, usually small objects, is a large number of them that hit you at the same time and with great force.
 \textit{
	\begin{itemize}
	\item The victim was hit by a hail of bullets.
	\item The riot police were met with a hail of stones and petrol bombs.
	\end{itemize}
}
\item verb \\
Someone who \textbf{hails from} a particular place was born there or lives there.
 \textit{
	\begin{itemize}
	\item I hail from Brighton.
	\item The band hail from Glasgow.
	\end{itemize}
}
\item verb \\
If someone or something \textbf{hails from} a particular background , they come from it.
 \textit{
	\begin{itemize}
	\item He hails from an affluent background.
	\item This is a film which seems to hail from the hippie era.
	\end{itemize}
}
\item verb \\
If you \textbf{hail} someone, you call to them.
 \textit{
	\begin{itemize}
	\item Jill saw him and hailed him.
	\item Suddenly, a voice hailed us and there was Miss Quigley.
	\end{itemize}
}
\item verb \\
If you \textbf{hail} a taxi , you wave at it in order to stop it because you want the driver to take you somewhere .
 \textit{
	\begin{itemize}
	\item I hurried away to hail a taxi.
	\end{itemize}
}
\item convention \\
\textbf{Hail} is used as a word of greeting.
 \textit{
	\begin{itemize}
	\item Hail to the new champion.
	\end{itemize}
}
\end{enumerate}

\section*{jungle}
{\large \color{blue}  jungles  }
\subsection*{Explain}
\begin{enumerate}
\item variable noun \\
A \textbf{jungle} is a forest in a tropical country where large numbers of tall trees and plants grow very close together.
 \textit{
	\begin{itemize}
	\item ...the mountains and jungles of Papua New Guinea.
	\item The mountain area is covered entirely in dense jungle.
	\item ...a remote jungle area.
	\end{itemize}
}
\item singular noun \\
If you describe a place as \textbf{a}  \textbf{jungle} , you are emphasizing that it is full of lots of things and very untidy .
 \textit{
	\begin{itemize}
	\item ...a jungle of stuffed birds, knick-knacks, potted plants.
	\end{itemize}
}
\item singular noun \\
If you describe a situation as \textbf{a}  \textbf{jungle} , you dislike it because it is complicated and difficult to get what you want from it.
 \textit{
	\begin{itemize}
	\item Social security law and procedure remain a jungle of complex rules.
	\item ...the examination jungle.
	\end{itemize}
}
\item  \\
 law of the jungle \textit{
	\begin{itemize}
	\end{itemize}
}
\item uncountable noun \\
\textbf{Jungle} is a style of dance music with a fast drum  beat .
 \textit{
	\begin{itemize}
	\end{itemize}
}
\end{enumerate}

\section*{lapse}
{\large \color{blue}  lapses  lapsing  lapsed  }
\subsection*{Explain}
\begin{enumerate}
\item countable noun \\
A \textbf{lapse} is a moment or instance of bad behaviour by someone who usually behaves  well .
 \textit{
	\begin{itemize}
	\item On Friday he showed neither decency nor dignity. It was an uncommon lapse.
	\end{itemize}
}
\item countable noun \\
A \textbf{lapse}  \textbf{of} something such as concentration or judgment is a temporary lack of that thing, which can often cause you to make a mistake .
 \textit{
	\begin{itemize}
	\item I had a little lapse of concentration in the middle of the race.
	\item He was a genius and because of it you could accept lapses of taste.
	\item The incident was being seen as a serious security lapse.
	\end{itemize}
}
\item verb \\
If you \textbf{lapse}  \textbf{into} a quiet or inactive state, you stop talking or being active .
 \textit{
	\begin{itemize}
	\item She muttered something unintelligible and lapsed into silence.
	\item Doris Brown closed her eyes and lapsed into sleep.
	\end{itemize}
}
\item verb \\
If someone \textbf{lapses}  \textbf{into} a particular way of speaking, or behaving, they start speaking or behaving in that way, usually for a short period.
 \textbf{Lapse} is also a noun .
 \textit{
	\begin{itemize}
	\item She lapsed into a little girl voice to deliver a nursery rhyme.
	\item Teenagers occasionally find it all too much to cope with and lapse into bad behaviour.
	\item Her lapse into German didn't seem peculiar. After all, it was her native tongue.
	\end{itemize}
}
\item singular noun \\
A \textbf{lapse}  \textbf{of} time is a period that is long enough for a situation to change or for people to have
a different opinion about it.
 \textit{
	\begin{itemize}
	\item ...the restoration of diplomatic relations after a lapse of 24 years.
	\item There is usually a time lapse between receipt of new information and its publication.
	\end{itemize}
}
\item verb \\
If a period of time \textbf{lapses} , it passes.
 \textit{
	\begin{itemize}
	\item Too much time has lapsed for police to now bring charges.
	\end{itemize}
}
\item verb \\
If a situation or legal contract  \textbf{lapses} , it is allowed to end rather than being continued , renewed , or extended.
 \textit{
	\begin{itemize}
	\item Her membership of the Labour Party has lapsed.
	\item Ford allowed the name and trademark to lapse during the Eighties.
	\end{itemize}
}
\item verb \\
If a member of a particular religion \textbf{lapses} , they stop believing in it or stop following its rules and practices.
 \textit{
	\begin{itemize}
	\item I lapsed in my 20s, returned to it, then lapsed again, while writing the life of
historical Jesus.
	\item She calls herself a lapsed Catholic.
	\end{itemize}
}
\end{enumerate}

\section*{ignore}
{\large \color{blue}  ignores  ignoring  ignored  }
\subsection*{Explain}
\begin{enumerate}
\item verb \\
If you \textbf{ignore} someone or something, you pay no attention to them.
 \textit{
	\begin{itemize}
	\item Why are you ignoring me?
	\item The government had ignored his views on the subject.
	\item She ignored legal advice to drop the case.
	\item For two decades her theatrical talents were ignored by the film industry.
	\end{itemize}
}
\item verb \\
If you say that an argument or theory  \textbf{ignores} an important  aspect of a situation , you are criticizing it because it fails to consider that aspect or to take it into account .
 \textit{
	\begin{itemize}
	\item Such arguments ignore the question of where ultimate responsibility lay.
	\end{itemize}
}
\end{enumerate}

\section*{layman}
{\large \color{blue}  laymen  }
\subsection*{Explain}
\begin{enumerate}
\item countable noun \\
A \textbf{layman} is a person who is not trained , qualified , or experienced in a particular subject or activity.
 \textit{
	\begin{itemize}
	\item To the layman, the words 'heart failure' suggest imminent death.
	\item There are basically two types called, in layman's terms, blue and white asbestos.
	\end{itemize}
}
\item countable noun \\
A \textbf{layman} is a man who is involved with the Christian  church but is not a member of the clergy or a monk .
 \textit{
	\begin{itemize}
	\item In 1932, one Boston layman wrote to Archbishop William O'Connell in support of Father
Coughlin.
	\end{itemize}
}
\end{enumerate}

\section*{interact}
{\large \color{blue}  interacts  interacting  interacted  }
\subsection*{Explain}
\begin{enumerate}
\item verb \\
When people \textbf{interact}  \textbf{with} each other or \textbf{interact} , they communicate as they work or spend time together .
 \textit{
	\begin{itemize}
	\item While the other children interacted and played together, Ted ignored them.
	\item ...rhymes and songs to help parents interact with their babies.
	\end{itemize}
}
\item verb \\
When people \textbf{interact}  \textbf{with}  computers , or when computers \textbf{interact}  \textbf{with} other machines , information or instructions are exchanged .
 \textit{
	\begin{itemize}
	\item Millions of people want new, simplified ways of interacting with a computer.
	\item There will be a true global village in which phones, computers and televisions interact.
	\end{itemize}
}
\item verb \\
When one thing \textbf{interacts}  \textbf{with} another or two things \textbf{interact} , the two things affect each other's behaviour or condition.
 \textit{
	\begin{itemize}
	\item You have to understand how cells interact.
	\item Atoms within the fluid interact with the minerals that form the grains.
	\end{itemize}
}
\end{enumerate}

\section*{outfit}
{\large \color{blue}  outfits  outfitting  outfitted  }
\subsection*{Explain}
\begin{enumerate}
\item countable noun \\
An \textbf{outfit} is a set of clothes.
 \textit{
	\begin{itemize}
	\item She was wearing an outfit she'd bought the previous day.
	\item I spent lots of money on smart new outfits for work.
	\end{itemize}
}
\item countable noun \\
You can refer to an organization as an \textbf{outfit} .
 \textit{
	\begin{itemize}
	\item He works for a private security outfit.
	\item We are a professional outfit and we do require payment for our services.
	\end{itemize}
}
\item verb \\
To \textbf{outfit} someone or something means to provide them with equipment for a particular purpose .
 \textit{
	\begin{itemize}
	\item They outfitted him with artificial legs.
	\item I outfitted an attic bedroom as a studio.
	\end{itemize}
}
\end{enumerate}

\section*{interrupt}
{\large \color{blue}  interrupts  interrupting  interrupted  }
\subsection*{Explain}
\begin{enumerate}
\item verb \\
If you \textbf{interrupt} someone who is speaking , you say or do something that causes them to stop .
 \textit{
	\begin{itemize}
	\item Turkin tapped him on the shoulder. 'Sorry to interrupt, Colonel.'.
	\item He tried to speak, but she interrupted him.
	\end{itemize}
}
\item verb \\
If someone or something \textbf{interrupts} a process or activity, they stop it for a period of time.
 \textit{
	\begin{itemize}
	\item He has rightly interrupted his holiday in Spain to return to London.
	\item The match took nearly three hours and was interrupted at times by rain.
	\end{itemize}
}
\item verb \\
If something \textbf{interrupts} a line, surface, or view, it stops it from being continuous or makes it look  irregular .
 \textit{
	\begin{itemize}
	\item Taller plants interrupt the views from the house.
	\end{itemize}
}
\end{enumerate}

\section*{outing}
{\large \color{blue}  outings  }
\subsection*{Explain}
\begin{enumerate}
\item countable noun \\
An \textbf{outing} is a short enjoyable trip, usually with a group of people, away from your home , school, or place of work.
 \textit{
	\begin{itemize}
	\item One evening, she made a rare outing to the local discotheque.
	\item ...families on a Sunday afternoon outing.
	\end{itemize}
}
\item countable noun \\
In sport, an \textbf{outing} is an occasion when a player competes in a particular contest or competition .
 \textit{
	\begin{itemize}
	\item Playing in England's first outing, he suffered a whiplash injury to his neck.
	\end{itemize}
}
\end{enumerate}

\section*{lift}
{\large \color{blue}  lifts  lifting  lifted  }
\subsection*{Explain}
\begin{enumerate}
\item verb \\
If you \textbf{lift} something, you move it to another position, especially upwards.
 \textbf{Lift up} means the same as lift .
 \textit{
	\begin{itemize}
	\item The Colonel lifted the phone and dialed his superior.
	\item She lifted the last of her drink to her lips.
	\item She put her arms around him and lifted him up.
	\item Curious shoppers lifted up their children to take a closer look at the parade.
	\end{itemize}
}
\item verb \\
If you \textbf{lift} a part of your body, you move it to a higher position.
 \textbf{Lift up} means the same as lift .
 \textit{
	\begin{itemize}
	\item Amy lifted her arm to wave. 'Goodbye,' she called.
	\item She lifted her foot and squashed the wasp into the ground.
	\item Tom took his seat again and lifted his feet up on to the railing.
	\item The boys lifted up their legs, indicating they wanted to climb in.
	\end{itemize}
}
\item verb \\
If you \textbf{lift} your eyes or your head, you look up, for example when you have been reading and someone comes into the room.
 \textit{
	\begin{itemize}
	\item When he finished he lifted his eyes and looked out the window.
	\end{itemize}
}
\item verb \\
If people in authority \textbf{lift} a law or rule that prevents people from doing something, they end it.
 \textit{
	\begin{itemize}
	\item The European Commission has urged France to lift its ban on imports of British beef.
	\end{itemize}
}
\item verb \\
If something \textbf{lifts} your spirits or your mood , or if they \textbf{lift} , you start feeling more cheerful .
 \textit{
	\begin{itemize}
	\item He used his incredible sense of humour to lift my spirits.
	\item A brisk walk in the fresh air can lift your mood and dissolve a winter depression.
	\item As soon as she heard the phone ring her spirits lifted.
	\end{itemize}
}
\item singular noun \\
If something gives you a \textbf{lift} , it gives you a feeling of greater  confidence , energy, or enthusiasm .
 \textit{
	\begin{itemize}
	\item My selection for the team has given me a tremendous lift.
	\end{itemize}
}
\item countable noun \\
A \textbf{lift} is a device that carries people or goods up and down inside tall buildings.
 \textit{
	\begin{itemize}
	\item They took the lift to the fourth floor.
	\end{itemize}
}
\item countable noun \\
If you give someone a \textbf{lift}  somewhere , you take them there in your car as a favour to them.
 \textit{
	\begin{itemize}
	\item He had a car and often gave me a lift home.
	\end{itemize}
}
\item verb \\
If a government or organization \textbf{lifts} people or goods in or out of an area, it transports them there by aircraft, especially
when there is a war.
 \textit{
	\begin{itemize}
	\item The army lifted people off rooftops where they had climbed to escape the flooding.
	\item The helicopters are designed to quickly lift soldiers and equipment to the battlefield.
	\end{itemize}
}
\item verb \\
To \textbf{lift} something means to increase its amount or to increase the level or the rate at which
it happens .
 \textit{
	\begin{itemize}
	\item The bank lifted its basic home loans rate to 10.99% from 10.75%.
	\item A barrage would halt the flow upstream and lift the water level.
	\end{itemize}
}
\item verb \\
If fog , cloud , or mist  \textbf{lifts} , it reduces, for example by moving upwards or by becoming less thick.
 \textit{
	\begin{itemize}
	\item The fog had lifted and revealed a warm, sunny day.
	\end{itemize}
}
\item verb \\
If you \textbf{lift} root vegetables or bulbs , you dig them out of the ground.
 \textit{
	\begin{itemize}
	\item Lift carrots on a dry day and pack them horizontally in boxes of damp sand.
	\end{itemize}
}
\end{enumerate}

\section*{overcoat}
{\large \color{blue}  overcoats  }
\subsection*{Explain}
\begin{enumerate}
\item countable noun \\
An \textbf{overcoat} is a thick warm coat that you wear in winter .
 \textit{
	\begin{itemize}
	\end{itemize}
}
\end{enumerate}

\section*{oppose}
{\large \color{blue}  opposes  opposing  opposed  }
\subsection*{Explain}
\begin{enumerate}
\item verb \\
If you \textbf{oppose} someone or \textbf{oppose} their plans or ideas , you disagree with what they want to do and try to prevent them from doing it.
 \textit{
	\begin{itemize}
	\item Mr Taylor was not bitter towards those who had opposed him.
	\item Many parents oppose bilingual education in schools.
	\end{itemize}
}
\end{enumerate}

\section*{package}
{\large \color{blue}  packages  packaging  packaged  }
\subsection*{Explain}
\begin{enumerate}
\item countable noun \\
A \textbf{package} is a small parcel .
 \textit{
	\begin{itemize}
	\item I tore open the package.
	\item ...a package addressed to Miss Claire Montgomery.
	\end{itemize}
}
\item countable noun \\
A \textbf{package} is a small container in which a quantity of something is sold. Packages are either small boxes made of
thin cardboard , or bags or envelopes made of paper or plastic .
 \textit{
	\begin{itemize}
	\item ...a package of doughnuts.
	\item It is listed among the ingredients on the package.
	\end{itemize}
}
\item countable noun \\
A \textbf{package} is a set of proposals that are made by a government or organization and which must be accepted or rejected as a group.
 \textit{
	\begin{itemize}
	\item ...a package of measures aimed at improving child welfare.
	\item They put together an economic aid package.
	\end{itemize}
}
\item verb \\
When a product \textbf{is packaged} , it is put into containers to be sold.
 \textit{
	\begin{itemize}
	\item The beans are then ground and packaged for sale as ground coffee.
	\item Packaged foods have to show a list of ingredients.
	\end{itemize}
}
\item verb \\
If something \textbf{is packaged} in a particular way, it is presented or advertised in that way in order to make it seem  attractive or interesting.
 \textit{
	\begin{itemize}
	\item A city has to be packaged properly to be attractive to tourists.
	\item ...entertainment packaged as information.
	\end{itemize}
}
\item countable noun \\
A \textbf{package}  tour , or in British English a \textbf{package}  holiday , is a holiday arranged by a travel company in which your travel and your accommodation are booked for you.
 \textit{
	\begin{itemize}
	\item If you are on a package holiday, your travel company's rep should act on your behalf.
	\end{itemize}
}
\end{enumerate}

\section*{owe}
{\large \color{blue}  owes  owing  owed  }
\subsection*{Explain}
\begin{enumerate}
\item verb \\
If you \textbf{owe}  money  \textbf{to} someone, they have lent it to you and you have not yet paid it back . You can  also  say that the money \textbf{is owing} .
 \textit{
	\begin{itemize}
	\item The company owes money to more than 60 banks.
	\item Blake already owed him nearly £50.
	\item I'm broke, Livy, and I owe a couple of million dollars.
	\item He could take what was owing for the rent.
	\end{itemize}
}
\item verb \\
If someone or something \textbf{owes} a particular  quality or their success  \textbf{to} a person or thing, they only have it because of that person or thing.
 \textit{
	\begin{itemize}
	\item I always suspected she owed her first job to her friendship with Roger.
	\item He owed his survival to his strength as a swimmer.
	\item The fruit owes its extraordinary aroma to a mixture of three main chemicals.
	\item The city essentially owes its fame and beauty to the Moors who transformed it into
the Muslim capital of Spain.
	\item I owe him my life.
	\end{itemize}
}
\item verb \\
If you say that you \textbf{owe} a great  deal  \textbf{to} someone or something, you mean that they have helped you or influenced you a lot , and you feel very grateful to them.
 \textit{
	\begin{itemize}
	\item As a professional composer, I owe much to Radio 3.
	\item He's been fantastic. I owe him a lot.
	\end{itemize}
}
\item verb \\
If you say that something \textbf{owes} a great deal to a person or thing, you mean that it exists , is successful , or has its particular form mainly because of them.
 \textit{
	\begin{itemize}
	\item She is the first to admit that her career path owes a lot to good fortune.
	\item Mrs Allen's style of cooking owes much to her mother-in-law.
	\end{itemize}
}
\item verb \\
If you say that you \textbf{owe} someone gratitude , respect , or loyalty , you mean that they deserve it from you.
 \textit{
	\begin{itemize}
	\item Perhaps we owe these people more respect.
	\item I owe you an apology. You must have found my attitude very annoying.
	\item I owe a big debt of gratitude to her.
	\end{itemize}
}
\item verb \\
If you say that you \textbf{owe it to} someone to do something, you mean that you should do that thing because they deserve
it.
 \textit{
	\begin{itemize}
	\item I can't go. I owe it to him to stay.
	\item You owe it to yourself to get some professional help.
	\item Of course she would have to send a letter; she owed it to the family.
	\end{itemize}
}
\item  \\
 owing to sth \textit{
	\begin{itemize}
	\end{itemize}
}
\end{enumerate}

\section*{parcel}
{\large \color{blue}  parcels  parcelling  parcelled  }
\subsection*{Explain}
\begin{enumerate}
\item countable noun \\
A \textbf{parcel} is something wrapped in paper, usually so that it can be sent to someone by post .
 \textit{
	\begin{itemize}
	\item ...parcels of food and clothing.
	\item He had a large brown paper parcel under his left arm.
	\end{itemize}
}
\item countable noun \\
A \textbf{parcel of} land is a piece of land.
 \textit{
	\begin{itemize}
	\item These small parcels of land were purchased for the most part by local people.
	\end{itemize}
}
\item countable noun \\
A \textbf{parcel of} things or people is a quantity of them.
 \textit{
	\begin{itemize}
	\item ...acquiring a parcel of financial worries.
	\end{itemize}
}
\item  \\
 part and parcel \textit{
	\begin{itemize}
	\end{itemize}
}
\end{enumerate}

\section*{quiver}
{\large \color{blue}  quivers  quivering  quivered  }
\subsection*{Explain}
\begin{enumerate}
\item verb \\
If something \textbf{quivers} , it shakes with very small movements.
 \textit{
	\begin{itemize}
	\item Her bottom lip quivered and big tears rolled down her cheeks.
	\end{itemize}
}
\item verb \\
If you say that someone or their voice  \textbf{is quivering}  \textbf{with} an emotion such as rage or excitement , you mean that they are strongly affected by this emotion and show it in their appearance or voice.
 \textbf{Quiver} is also a noun .
 \textit{
	\begin{itemize}
	\item Cooper arrived, quivering with rage.
	\item Mack made his voice quiver with fear on these last two words.
	\item I felt a quiver of panic.
	\end{itemize}
}
\item countable noun \\
A \textbf{quiver} is a container for carrying arrows in.
 \textit{
	\begin{itemize}
	\end{itemize}
}
\end{enumerate}

\section*{penalty}
{\large \color{blue}  penalties  }
\subsection*{Explain}
\begin{enumerate}
\item countable noun \\
A \textbf{penalty} is a punishment that someone is given for doing something which is against a law or rule .
 \textit{
	\begin{itemize}
	\item One of those arrested could face the death penalty.
	\item The maximum penalty is up to 7 years' imprisonment or an unlimited fine.
	\end{itemize}
}
\item countable noun \\
In sports such as football and hockey , a \textbf{penalty} is an opportunity to score a goal, which is given to the attacking team if the defending team breaks a rule near their own goal.
 \textit{
	\begin{itemize}
	\item Referee Michael Reed had no hesitation in awarding a penalty.
	\item The team's captain scored a penalty goal.
	\end{itemize}
}
\item countable noun \\
\textbf{The}  \textbf{penalty} that you pay for something you have done is something unpleasant that you experience as a result.
 \textit{
	\begin{itemize}
	\item Why should I pay the penalty for somebody else's mistake?
	\end{itemize}
}
\end{enumerate}

\section*{react}
{\large \color{blue}  reacts  reacting  reacted  }
\subsection*{Explain}
\begin{enumerate}
\item verb \\
When you \textbf{react}  \textbf{to} something that has happened to you, you behave in a particular way because of it.
 \textit{
	\begin{itemize}
	\item They reacted violently to the news.
	\item It's natural to react with disbelief if your child is accused of bullying.
	\item 'How did he react?'—'Very calmly.'
	\end{itemize}
}
\item verb \\
If you \textbf{react against} someone's way of behaving, you deliberately behave in a different way because you do not like the way they behave.
 \textit{
	\begin{itemize}
	\item She reacted against the mindlessness and luxury of their lives.
	\item My father never saved and perhaps I reacted against that.
	\end{itemize}
}
\item verb \\
If you \textbf{react}  \textbf{to} a substance such as a drug , or \textbf{to} something you have touched , you are affected unpleasantly or made ill by it.
 \textit{
	\begin{itemize}
	\item Someone allergic to milk is likely to react to cheese.
	\item He reacted very badly to the radiation therapy.
	\end{itemize}
}
\item verb \\
When one chemical substance \textbf{reacts}  \textbf{with} another, or when two chemical substances \textbf{react} , they combine chemically to form another substance.
 \textit{
	\begin{itemize}
	\item Calcium reacts with water.
	\item Under normal circumstances, these two gases react readily to produce carbon dioxide
and water.
	\end{itemize}
}
\end{enumerate}

\section*{print}
{\large \color{blue}  prints  printing  printed  }
\subsection*{Explain}
\begin{enumerate}
\item verb \\
If someone \textbf{prints} something such as a book or newspaper, they produce it in large quantities using
a machine.
 In American English, \textbf{print up} means the same as print .
 \textit{
	\begin{itemize}
	\item He started to print his own posters to distribute abroad.
	\item ...the company that prints banknotes on behalf of the Bank of England.
	\item Our brochure is printed on environmentally-friendly paper.
	\item Television and radio gave rise to far fewer complaints than did the printed media.
	\item Community workers here are printing up pamphlets for peace demonstrations.
	\item Hey, I know what, I'll get a bumper sticker printed up.
	\end{itemize}
}
\item verb \\
If a newspaper or magazine  \textbf{prints} a piece of writing, it includes it or publishes it.
 \textit{
	\begin{itemize}
	\item We can only print letters which are accompanied by the writer's name and address.
	\item ...a questionnaire printed in the magazine recently.
	\end{itemize}
}
\item verb \\
If numbers, letters, or designs \textbf{are printed}  \textbf{on} a surface, they are put on it in ink or dye using a machine. You can also say that a surface \textbf{is printed}  \textbf{with} numbers, letters, or designs.
 \textit{
	\begin{itemize}
	\item ...the number printed on the receipt.
	\item The company has for some time printed its phone number on its products.
	\item The shirts were printed with a paisley pattern.
	\item 'Ecu' was printed in lower case rather than capital letters.
	\end{itemize}
}
\item countable noun \\
A \textbf{print} is a piece of clothing or material with a pattern printed on it. You can also refer to the pattern itself as a \textbf{print} .
 \textit{
	\begin{itemize}
	\item Her mother wore one of her dark summer prints.
	\item In this living room we've mixed glorious floral prints.
	\item ...multi-coloured print jackets.
	\end{itemize}
}
\item verb \\
When you \textbf{print} a photograph , you produce it from a negative.
 \textit{
	\begin{itemize}
	\item Printing a black-and-white negative on to colour paper produces a similar monochrome
effect.
	\item I selected two negatives to print from.
	\end{itemize}
}
\item countable noun \\
A \textbf{print} is a photograph from a film that has been developed.
 \textit{
	\begin{itemize}
	\item ...black and white prints of Margaret and Jean as children.
	\item ...35mm colour print films.
	\end{itemize}
}
\item countable noun \\
A \textbf{print} of a cinema film is a particular copy or set of copies of it.
 \textit{
	\begin{itemize}
	\item First released in 1957, the movie now appears in a new print.
	\end{itemize}
}
\item countable noun \\
A \textbf{print} is one of a number of copies of a particular picture. It can be either a photograph,
something such as a painting , or a picture made by an artist who puts ink on a prepared surface and presses it against paper.
 \textit{
	\begin{itemize}
	\item ...William Hogarth's famous series of prints.
	\end{itemize}
}
\item uncountable noun \\
\textbf{Print} is used to refer to letters and numbers as they appear on the pages of a book, newspaper, or printed document.
 \textit{
	\begin{itemize}
	\item ...columns of tiny print.
	\item Laser printers give high quality print.
	\end{itemize}
}
\item adjective \\
The \textbf{print} media consists of newspapers and magazines, but not television or radio.
 \textit{
	\begin{itemize}
	\item I have been convinced that the print media are more accurate and more reliable than
television.
	\item ...print journalists.
	\end{itemize}
}
\item verb \\
If you \textbf{print} words, you write in letters that are not joined together and that look like the letters in a book or newspaper.
 \textit{
	\begin{itemize}
	\item Print your name and address on a postcard and send it to us.
	\end{itemize}
}
\item countable noun \\
You can refer to a mark left by someone's foot as a \textbf{print} .
 \textit{
	\begin{itemize}
	\item He crawled from print to print, sniffing at the earth, following the scent left in
the tracks.
	\item ...boot prints.
	\end{itemize}
}
\item countable noun \\
You can refer to invisible marks left by someone's fingers as their \textbf{prints} .
 \textit{
	\begin{itemize}
	\item Fresh prints of both girls were found in the flat.
	\end{itemize}
}
\item  \\
 in print \textit{
	\begin{itemize}
	\end{itemize}
}
\item  \\
 in print \textit{
	\begin{itemize}
	\end{itemize}
}
\item  \\
 out of print \textit{
	\begin{itemize}
	\end{itemize}
}
\item  \\
 small/fine print \textit{
	\begin{itemize}
	\end{itemize}
}
\end{enumerate}

\section*{rebel}
{\large \color{blue}  rebels  rebelling  rebelled  }
\subsection*{Explain}
\begin{enumerate}
\item countable noun \\
\textbf{Rebels} are people who are fighting against their own country's army in order to change the political system there.
 \textit{
	\begin{itemize}
	\item ...fighting between rebels and government forces.
	\item ...rebel forces in Liberia.
	\end{itemize}
}
\item countable noun \\
Politicians who oppose some of their own party's policies can be referred to as \textbf{rebels} .
 \textit{
	\begin{itemize}
	\item The rebels want another 1% cut in interest rates.
	\item ...rebel MPs.
	\end{itemize}
}
\item verb \\
If politicians \textbf{rebel} against one of their own party's policies, they show that they oppose it.
 \textit{
	\begin{itemize}
	\item There are signs that MPs are rebelling against a new regime of austerity at the Commons.
	\item ...MPs planning to rebel over the proposed welfare cuts.
	\end{itemize}
}
\item countable noun \\
You can say that someone is a \textbf{rebel} if you think that they behave differently from other people and have rejected the values of society or of their parents .
 \textit{
	\begin{itemize}
	\item She had been a rebel at school.
	\end{itemize}
}
\item verb \\
When someone \textbf{rebels} , they start to behave differently from other people and reject the values of society or of their
parents.
 \textit{
	\begin{itemize}
	\item The child who rebels is unlikely to be overlooked.
	\item I was very young and rebelling against everything.
	\end{itemize}
}
\end{enumerate}

\section*{pronoun}
{\large \color{blue}  pronouns  }
\subsection*{Explain}
\begin{enumerate}
\item countable noun \\
A \textbf{pronoun} is a word that you use to refer to someone or something when you do not need to use a noun, often because the person or thing has been mentioned earlier . Examples are 'it', 'she', 'something', and 'myself'.
 \textit{
	\begin{itemize}
	\end{itemize}
}
\end{enumerate}

\section*{reckon}
{\large \color{blue}  reckons  reckoning  reckoned  }
\subsection*{Explain}
\begin{enumerate}
\item verb \\
If you \textbf{reckon} that something is true , you think that it is true.
 \textit{
	\begin{itemize}
	\item Toni reckoned that it must be about three o'clock.
	\item He reckoned he was still fond of her.
	\end{itemize}
}
\item verb \\
If you say that something \textbf{is reckoned}  \textbf{to} be true, you mean that people think that it is true.
 \textit{
	\begin{itemize}
	\item The sale has been held up because the price is reckoned to be too high.
	\end{itemize}
}
\item verb \\
If you say that someone \textbf{reckons}  \textbf{to} do something, you mean that they expect to do it.
 \textit{
	\begin{itemize}
	\item The merged banks reckon to raise 4 billion dollars of new equity next year.
	\item Police officers on the case are reckoning to charge someone very shortly.
	\end{itemize}
}
\item verb \\
If something \textbf{is reckoned} to be a particular figure , it is calculated to be roughly that amount.
 \textit{
	\begin{itemize}
	\item The star's surface temperature is reckoned to be minus 75 degrees Celsius.
	\item A proportion of the research, which I reckoned at about 30 percent, was basic research.
	\end{itemize}
}
\end{enumerate}

\section*{refute}
{\large \color{blue}  refutes  refuting  refuted  }
\subsection*{Explain}
\begin{enumerate}
\item verb \\
If you \textbf{refute} an argument , accusation , or theory, you prove that it is wrong or untrue .
 \textit{
	\begin{itemize}
	\item It was the kind of rumour that it is impossible to refute.
	\end{itemize}
}
\item verb \\
If you \textbf{refute} an argument or accusation, you say that it is not true .
 \textit{
	\begin{itemize}
	\item Isabelle is quick to refute any suggestion of intellectual snobbery.
	\end{itemize}
}
\end{enumerate}

\section*{siege}
{\large \color{blue}  sieges  }
\subsection*{Explain}
\begin{enumerate}
\item countable noun \\
A \textbf{siege} is a military or police operation in which soldiers or police surround a place in order to force the people there to come out or give up control of the place.
 \textit{
	\begin{itemize}
	\item We must do everything possible to lift the siege.
	\item They are hopeful of bringing the siege to a peaceful conclusion.
	\item The journalists found a city virtually under siege.
	\end{itemize}
}
\item  \\
 to lay siege to something \textit{
	\begin{itemize}
	\end{itemize}
}
\item  \\
 under siege \textit{
	\begin{itemize}
	\end{itemize}
}
\end{enumerate}

\section*{relay}
{\large \color{blue}  relays  relaying  relayed  }
\subsection*{Explain}
\begin{enumerate}
\item countable noun \\
A \textbf{relay} or a \textbf{relay race} is a race between two or more teams, for example teams of runners or swimmers. Each member of the team runs or swims one section of the race.
 \textit{
	\begin{itemize}
	\item Britain's prospects of beating the United States in the relay looked poor.
	\end{itemize}
}
\item verb \\
To \textbf{relay}  television or radio signals means to send them or broadcast them.
 \textbf{Relay} is also a noun .
 \textit{
	\begin{itemize}
	\item The satellite will be used mainly to relay television programmes.
	\item This system monitors radiation levels and relays the information to a central computer.
	\item The event will be relayed to a giant TV screen a mile away.
	\item More than a thousand people outside listened to a relay of the proceedings.
	\end{itemize}
}
\item countable noun \\
A \textbf{relay} is a piece of equipment that receives television or radio signals from one place and sends them to another
place.
 \textit{
	\begin{itemize}
	\item ...a security system with satellite relays.
	\item ...a television relay station.
	\end{itemize}
}
\item verb \\
If you \textbf{relay} something that has been said to you, you repeat it to another person.
 \textit{
	\begin{itemize}
	\item She relayed the message, then frowned.
	\item The decision will be relayed to Iraq's ambassador at the U.N..
	\end{itemize}
}
\end{enumerate}

\section*{substitute}
{\large \color{blue}  substitutes  substituting  substituted  }
\subsection*{Explain}
\begin{enumerate}
\item verb \\
If you \textbf{substitute} one thing \textbf{for} another, or if one thing \textbf{substitutes}  \textbf{for} another, it takes the place or performs the function of the other thing.
 \textit{
	\begin{itemize}
	\item They were substituting violence for dialogue.
	\item You could always substitute a low-fat soft cheese.
	\item Would phone conversations substitute for cosy chats over lunch or in the pub after
work?
	\item He was substituting for the injured William Wales.
	\end{itemize}
}
\item countable noun \\
A \textbf{substitute} is something that you have or use instead of something else.
 \textit{
	\begin{itemize}
	\item She is seeking a substitute for the very man whose departure made her cry.
	\item ...tests on humans to find a blood substitute made from animal blood.
	\end{itemize}
}
\item countable noun \\
If you say that one thing is no \textbf{substitute}  \textbf{for} another, you mean that it does not have certain desirable  features that the other thing has, and is therefore unsatisfactory . If you say that there is no \textbf{substitute}  \textbf{for} something, you mean that it is the only thing which is really  satisfactory .
 \textit{
	\begin{itemize}
	\item The printed word is no substitute for personal discussion with a great thinker.
	\item There is no substitute for practical experience.
	\end{itemize}
}
\item countable noun \\
In team games such as football , a \textbf{substitute} is a player who is brought into a match to replace another player.
 \textit{
	\begin{itemize}
	\item Coming on as a substitute, he scored four crucial goals.
	\end{itemize}
}
\end{enumerate}

\section*{rise}
{\large \color{blue}  rises  rising  rose  risen  }
\subsection*{Explain}
\begin{enumerate}
\item verb \\
If something \textbf{rises} , it moves upwards.
 \textbf{Rise up} means the same as rise .
 \textit{
	\begin{itemize}
	\item He watched the smoke rise from the chimney.
	\item The powdery dust rose in a cloud around him.
	\item Spray rose up from the surface of the water.
	\item Black dense smoke rose up.
	\end{itemize}
}
\item verb \\
When you \textbf{rise} , you stand up.
 \textbf{Rise up} means the same as rise .
 \textit{
	\begin{itemize}
	\item Luther rose slowly from the chair.
	\item He looked at Livy and Mark, who had risen to greet him.
	\item The only thing I wanted was to rise up from the table and leave this house.
	\end{itemize}
}
\item verb \\
When you \textbf{rise} , you get out of bed.
 \textit{
	\begin{itemize}
	\item Tony had risen early and gone to the cottage to work.
	\end{itemize}
}
\item verb \\
When the sun or moon \textbf{rises} , it appears in the sky .
 \textit{
	\begin{itemize}
	\item He wanted to be over the line of the ridge before the sun had risen.
	\end{itemize}
}
\item verb \\
You can say that something \textbf{rises} when it appears as a large tall shape.
 \textbf{Rise up} means the same as rise .
 \textit{
	\begin{itemize}
	\item The building rose before him, tall and stately.
	\item The towers rise out of a concrete podium.
	\item The White Mountains rose up before me.
	\end{itemize}
}
\item verb \\
If the level of something such as the water in a river \textbf{rises} , it becomes higher.
 \textit{
	\begin{itemize}
	\item The waters continue to rise as more than 1,000 people are evacuated.
	\item ...the tides rise and fall.
	\end{itemize}
}
\item verb \\
If land \textbf{rises} , it slopes upwards.
 \textit{
	\begin{itemize}
	\item He looked up the slope of land that rose from the house.
	\item The ground begins to rise some 20 yards away.
	\item The great house stood on rising ground.
	\end{itemize}
}
\item countable noun \\
A \textbf{rise} is an area of ground that slopes upwards.
 \textit{
	\begin{itemize}
	\item The pub itself was on a rise, commanding views across the countryside.
	\item I climbed to the top of a rise overlooking the ramparts.
	\end{itemize}
}
\item verb \\
If an amount \textbf{rises} , it increases.
 \textit{
	\begin{itemize}
	\item Pre-tax profits rose from £842,000 to £1.82m.
	\item Tourist trips of all kinds in Britain rose by 10.5% between 1977 and 1987.
	\item Exports in June rose 1.5% to a record $30.91 billion.
	\item The number of business failures has risen.
	\item The increase is needed to meet rising costs.
	\end{itemize}
}
\item countable noun \\
A \textbf{rise in} the amount of something is an increase in it.
 \textit{
	\begin{itemize}
	\item ...the prospect of another rise in interest rates.
	\item Foreign nationals have begun leaving because of a sharp rise in violence.
	\end{itemize}
}
\item countable noun \\
A \textbf{rise} is an increase in your wages or your salary.
 \textit{
	\begin{itemize}
	\item He will get a pay rise of nearly £4,000.
	\end{itemize}
}
\item singular noun \\
\textbf{The rise of} a movement or activity is an increase in its popularity or influence.
 \textit{
	\begin{itemize}
	\item ...the rise of women's football.
	\item ...the rise of home ownership.
	\end{itemize}
}
\item verb \\
If the wind \textbf{rises} , it becomes stronger.
 \textbf{Rise up} means the same as rise .
 \textit{
	\begin{itemize}
	\item The wind was still rising, approaching a force nine gale.
	\item Foxworth shivered as the wind rose up and roared through the beech trees.
	\end{itemize}
}
\item verb \\
If a sound \textbf{rises} or if someone's voice \textbf{rises} , it becomes louder or higher.
 \textit{
	\begin{itemize}
	\item 'Bernard?' Her voice rose hysterically.
	\item His voice rose almost to a scream.
	\end{itemize}
}
\item verb \\
If a sound \textbf{rises}  \textbf{from} a group of people, it comes from them.
 \textbf{Rise up} means the same as rise .
 \textit{
	\begin{itemize}
	\item There were low, muffled voices rising from the hallway.
	\item From the people, a cheer rose up.
	\end{itemize}
}
\item verb \\
If an emotion \textbf{rises}  \textbf{in} someone, they suddenly feel it very intensely so that it affects their behaviour.
 \textit{
	\begin{itemize}
	\item A tide of emotion rose and clouded his judgement.
	\item The thought made anger rise in him.
	\end{itemize}
}
\item verb \\
If your colour \textbf{rises} or if a blush  \textbf{rises}  \textbf{in} your cheeks , you turn red because you feel angry , embarrassed , or excited .
 \textit{
	\begin{itemize}
	\item Amy felt the colour rising in her cheeks at the thought.
	\end{itemize}
}
\item verb \\
When the people in a country \textbf{rise} , they try to defeat the government or army that is controlling them.
 \textbf{Rise up} means the same as rise .
 \textit{
	\begin{itemize}
	\item The National Convention has promised armed support to any people who wish to rise
against armed oppression.
	\item He warned that if the government moved against him the people would rise up.
	\item A woman called on the population to rise up against the government.
	\end{itemize}
}
\item verb \\
If someone \textbf{rises}  \textbf{to} a higher position or status, they become more important, successful , or powerful.
 \textbf{Rise up} means the same as rise .
 \textit{
	\begin{itemize}
	\item She is a strong woman who has risen to the top of a deeply sexist organisation.
	\item From an unlikely background he has risen rapidly through the ranks of government.
	\item I started with Hoover 26 years ago in sales and rose up through the ranks.
	\end{itemize}
}
\item singular noun \\
The \textbf{rise} of someone is the process by which they become more important, successful, or powerful.
 \textit{
	\begin{itemize}
	\item Haig's rise was fuelled by an all-consuming sense of patriotic duty.
	\item The group celebrated the regime's rise to power in 1979.
	\end{itemize}
}
\item  \\
 to give rise to \textit{
	\begin{itemize}
	\end{itemize}
}
\end{enumerate}

\section*{surgeon}
{\large \color{blue}  surgeons  }
\subsection*{Explain}
\begin{enumerate}
\item countable noun \\
A \textbf{surgeon} is a doctor who is specially trained to perform surgery.
 \textit{
	\begin{itemize}
	\item ...a heart surgeon.
	\end{itemize}
}
\end{enumerate}

\section*{shout}
{\large \color{blue}  shouts  shouting  shouted  }
\subsection*{Explain}
\begin{enumerate}
\item verb \\
If you \textbf{shout} , you say something very loudly, usually because you want people a long distance  away to hear you or because you are angry .
 \textbf{Shout} is also a noun .
 \textit{
	\begin{itemize}
	\item He had to shout to make himself heard above the near gale-force wind.
	\item 'She's alive!' he shouted triumphantly.
	\item Andrew rushed out of the house, shouting for help.
	\item You don't have to shout at me.
	\item I shouted at mother to get the police.
	\item The driver managed to escape from the vehicle and shout a warning.
	\item The decision was greeted with shouts of protest from opposition MPs.
	\item I heard a distant shout.
	\end{itemize}
}
\item  \\
 in with a shout of \textit{
	\begin{itemize}
	\end{itemize}
}
\item  \\
 it's sb's shout \textit{
	\begin{itemize}
	\end{itemize}
}
\end{enumerate}

\section*{surgery}
{\large \color{blue}  surgeries  }
\subsection*{Explain}
\begin{enumerate}
\item uncountable noun \\
\textbf{Surgery} is medical  treatment in which someone's body is cut  open so that a doctor can repair , remove, or replace a diseased or damaged part.
 \textit{
	\begin{itemize}
	\item His father has just recovered from heart surgery.
	\item Mr Clark underwent five hours of emergency surgery.
	\end{itemize}
}
\item countable noun \\
A \textbf{surgery} is the room or house where a doctor or dentist works.
 \textit{
	\begin{itemize}
	\item Bill was in the doctor's surgery demanding to know what was wrong with him.
	\end{itemize}
}
\item countable noun \\
A doctor's \textbf{surgery} is the period of time each day when a doctor sees  patients at his or her surgery.
 \textit{
	\begin{itemize}
	\item His surgery always ends at eleven.
	\end{itemize}
}
\item countable noun \\
In Britain , when someone such as an MP or a local  councillor holds a \textbf{surgery} , they go to an office where members of the public can come and talk to them about problems or issues that concern them.
 \textit{
	\begin{itemize}
	\end{itemize}
}
\item countable noun \\
A \textbf{surgery} is the room in a hospital where surgeons operate on their patients.
 \textit{
	\begin{itemize}
	\end{itemize}
}
\end{enumerate}

\section*{sip}
{\large \color{blue}  sips  sipping  sipped  }
\subsection*{Explain}
\begin{enumerate}
\item verb \\
If you \textbf{sip} a drink or \textbf{sip at} it, you drink by taking just a small amount at a time.
 \textit{
	\begin{itemize}
	\item Jessica sipped her drink thoughtfully.
	\item He sipped at the glass and then put it down.
	\item She sipped from her coffee mug, watching him over the rim.
	\item He lifted the water-bottle to his lips and sipped.
	\end{itemize}
}
\item countable noun \\
A \textbf{sip} is a small amount of drink that you take into your mouth.
 \textit{
	\begin{itemize}
	\item Harry took a sip of bourbon.
	\item Katherine took another sip from her glass to calm herself.
	\end{itemize}
}
\end{enumerate}

\section*{virgin}
{\large \color{blue}  virgins  }
\subsection*{Explain}
\begin{enumerate}
\item countable noun \\
A \textbf{virgin} is someone who has never had sex .
 \textit{
	\begin{itemize}
	\item I was a virgin until I was thirty years old.
	\item They were both virgins when they met and married.
	\end{itemize}
}
\item adjective \\
You use \textbf{virgin} to describe something such as land that has never been used or spoiled .
 \textit{
	\begin{itemize}
	\item Within 40 years there will be no virgin forest left.
	\item ...a sloping field of virgin snow.
	\end{itemize}
}
\item  \\
 virgin territory \textit{
	\begin{itemize}
	\end{itemize}
}
\item countable noun \\
You can use \textbf{virgin} to describe someone who has never done or used a particular thing before.
 \textit{
	\begin{itemize}
	\item Until he appeared in 'In the Line of Fire', the actor had been an action-movie virgin.
	\item He was a political virgin when he was appointed as Lord Advocate.
	\end{itemize}
}
\end{enumerate}

\section*{slit}
{\large \color{blue}  slits  slitting  }
\subsection*{Explain}
\begin{enumerate}
\item verb \\
If you \textbf{slit} something, you make a long narrow cut in it.
 \textit{
	\begin{itemize}
	\item They say somebody slit her throat.
	\item He began to slit open each envelope.
	\item She was wearing a white dress slit to the thigh.
	\end{itemize}
}
\item countable noun \\
A \textbf{slit} is a long narrow cut.
 \textit{
	\begin{itemize}
	\item Make a slit in the stem about half an inch long.
	\end{itemize}
}
\item countable noun \\
A \textbf{slit} is a long narrow opening in something.
 \textit{
	\begin{itemize}
	\item She watched them through a slit in the curtains.
	\end{itemize}
}
\end{enumerate}

\section*{winter}
{\large \color{blue}  winters  wintering  wintered  }
\subsection*{Explain}
\begin{enumerate}
\item variable noun \\
\textbf{Winter} is the season between autumn and spring when the weather is usually cold.
 \textit{
	\begin{itemize}
	\item In winter the nights are long and cold.
	\item Last winter's snowfall was heavier than usual.
	\item ...the winter months.
	\item ...the late winter of 1941.
	\end{itemize}
}
\item verb \\
If an animal or plant \textbf{winters}  somewhere or \textbf{is wintered} there, it spends the winter there.
 \textit{
	\begin{itemize}
	\item The birds will winter outside in an aviary.
	\item The young seedlings are usually wintered in a cold frame.
	\item ...one of the most important sites for wintering wildfowl.
	\end{itemize}
}
\item verb \\
If you \textbf{winter} somewhere, you spend the winter there.
 \textit{
	\begin{itemize}
	\item The family decided to winter in Nice again.
	\end{itemize}
}
\end{enumerate}

\section*{start}
{\large \color{blue}  starts  starting  started  }
\subsection*{Explain}
\begin{enumerate}
\item verb \\
If you \textbf{start}  \textbf{to} do something, you do something that you were not doing before and you continue doing it.
 \textbf{Start} is also a noun .
 \textit{
	\begin{itemize}
	\item John then unlocked the front door and I started to follow him up the stairs.
	\item It was 1956 when Susanna started the work on the garden.
	\item She started cleaning the kitchen.
	\item After several starts, she read the report properly.
	\end{itemize}
}
\item verb \\
When something \textbf{starts} , or if someone \textbf{starts} it, it takes place from a particular time.
 \textbf{Start} is also a noun.
 \textit{
	\begin{itemize}
	\item The fire is thought to have started in an upstairs room.
	\item The Great War started in August of that year.
	\item Trains start at 11.00 and an hourly service will operate until 16.00.
	\item All of the passengers started the day with a swim.
	\item ...1918, four years after the start of the Great War.
	\item She demanded to know why she had not been told from the start.
	\end{itemize}
}
\item verb \\
If you \textbf{start by} doing something, or if you \textbf{start with} something, you do that thing first in a series of actions.
 \textit{
	\begin{itemize}
	\item I started by asking about day-care centers.
	\item He started with a good holiday in Key West, Florida.
	\end{itemize}
}
\item verb \\
You use \textbf{start} to say what someone's first job was. For example , if their first job was that of a factory  worker , you can say that they \textbf{started as} a factory worker.
 \textbf{Start off}  means the same as start .
 \textit{
	\begin{itemize}
	\item Betty started as a shipping clerk at the clothes factory.
	\item Grace Robertson started as a photographer with Picture Post in 1947.
	\item Mr. Dambar had started off as an assistant to Mrs. Spear's husband.
	\end{itemize}
}
\item verb \\
When someone \textbf{starts} something such as a new  business , they create it or cause it to begin.
 \textbf{Start up} means the same as start .
 \textit{
	\begin{itemize}
	\item He has started a health centre and is looking for staff.
	\item Now is probably as good a time as any to start a business.
	\item The cost of starting up a day care center for children ranges from $150,000 to $300,000.
	\item He said what a good idea it would be to start a community magazine up.
	\end{itemize}
}
\item verb \\
If you \textbf{start} an engine , car , or machine , or if it \textbf{starts} , it begins to work.
 \textbf{Start up} means the same as start .
 \textit{
	\begin{itemize}
	\item He started the car, which hummed smoothly.
	\item We were just passing one of the parking bays when a car's engine started.
	\item He waited until they went inside the building before starting up the car and driving
off.
	\item Put the key in the ignition and turn it to start the car up.
	\item The engine of the seaplane started up.
	\end{itemize}
}
\item verb \\
If you \textbf{start} , your body suddenly moves  slightly as a result of surprise or fear .
 \textbf{Start} is also a noun.
 \textit{
	\begin{itemize}
	\item She put the bottle on the table, banging it down hard. He started at the sound.
	\item Rachel started forward on the sofa.–'You mean you've arrested Pete?'
	\item Sylvia woke with a start.
	\item He gave a start of surprise and astonishment.
	\end{itemize}
}
\item  \\
 for a start/to start with \textit{
	\begin{itemize}
	\end{itemize}
}
\item  \\
 get off to a good/bad start \textit{
	\begin{itemize}
	\end{itemize}
}
\item  \\
 to start with \textit{
	\begin{itemize}
	\end{itemize}
}
\end{enumerate}

\section*{agenda}
{\large \color{blue}  agendas  }
\subsection*{Explain}
\begin{enumerate}
\item countable noun \\
You can refer to the political  issues which are important at a particular time as an \textbf{agenda} .
 \textit{
	\begin{itemize}
	\item Does television set the agenda on foreign policy?
	\item Many of the coalition members could have their own political agendas.
	\item The Danish president will put environmental issues high on the agenda.
	\end{itemize}
}
\item countable noun \\
An \textbf{agenda} is a list of the items that have to be discussed at a meeting.
 \textit{
	\begin{itemize}
	\item This is sure to be an item on the agenda next week.
	\end{itemize}
}
\end{enumerate}

\section*{announce}
{\large \color{blue}  announces  announcing  announced  }
\subsection*{Explain}
\begin{enumerate}
\item verb \\
If you \textbf{announce} something, you tell people about it publicly or officially .
 \textit{
	\begin{itemize}
	\item He will announce tonight that he is resigning from office.
	\item When they announced their engagement, no one was surprised.
	\item It was announced that the groups have agreed to a cease-fire.
	\end{itemize}
}
\item verb \\
If you \textbf{announce} a piece of news or an intention, especially something that people may not like, you say it loudly and clearly , so that everyone you are with can hear it.
 \textit{
	\begin{itemize}
	\item Peter announced that he had no intention of wasting his time at any university.
	\item 'I'm having a bath and going to bed,' she announced, and left the room.
	\end{itemize}
}
\item verb \\
If an airport or railway  employee  \textbf{announces} something, they tell the public about it by means of a loudspeaker system.
 \textit{
	\begin{itemize}
	\item Station staff announced the arrival of the train over the tannoy.
	\item They announced his plane was delayed.
	\end{itemize}
}
\item verb \\
If a letter , sound, or sign  \textbf{announces} something, it informs people about it.
 \textit{
	\begin{itemize}
	\item The next letter announced the birth of another boy.
	\item His entrance was announced by a buzzer connected to the door.
	\end{itemize}
}
\item verb \\
If a meal or a guest  \textbf{is announced} by a servant at a formal party, the servant says clearly that the meal is ready or the guest has arrived .
 \textit{
	\begin{itemize}
	\item Dinner was announced, and served.
	\end{itemize}
}
\end{enumerate}

\section*{certainty}
{\large \color{blue}  certainties  }
\subsection*{Explain}
\begin{enumerate}
\item uncountable noun \\
\textbf{Certainty} is the state of being definite or of having no doubts at all about something.
 \textit{
	\begin{itemize}
	\item I have told them with absolute certainty there'll be no change of policy.
	\item If you buy from reputable dealers you have more certainty about what you're getting.

	\end{itemize}
}
\item uncountable noun \\
\textbf{Certainty} is the fact that something is certain to happen .
 \textit{
	\begin{itemize}
	\item A general election became a certainty three weeks ago.
	\item ...the certainty of more violence and bloodshed.
	\item I began to realize the certainty of freezing to death if I remained where I was.
	\end{itemize}
}
\item countable noun \\
\textbf{Certainties} are things that nobody has any doubts about.
 \textit{
	\begin{itemize}
	\item There are no certainties in modern Europe.
	\item The collapse of old certainties is reshaping the political parties.
	\end{itemize}
}
\end{enumerate}

\section*{arise}
{\large \color{blue}  arises  arising  arose  arisen  }
\subsection*{Explain}
\begin{enumerate}
\item verb \\
If a situation or problem  \textbf{arises} , it begins to exist or people start to become aware of it.
 \textit{
	\begin{itemize}
	\item ...if a problem arises later in the pregnancy.
	\item The birds also attack crops when the opportunity arises.
	\end{itemize}
}
\item verb \\
If something \textbf{arises from} a particular situation, or \textbf{arises out of} it, it is created or caused by the situation.
 \textit{
	\begin{itemize}
	\item ...an overwhelming sense of guilt arising from my actions.
	\item The charges arise out of a long-running fraud enquiry by Merseyside police.
	\end{itemize}
}
\item verb \\
If something such as a new  species , organization , or system \textbf{arises} , it begins to exist and develop .
 \textit{
	\begin{itemize}
	\item Heavy Metal music really arose in the late 60s.
	\end{itemize}
}
\item verb \\
When you \textbf{arise} , you get out of bed in the morning .
 \textit{
	\begin{itemize}
	\item He arose at 6:30 a.m. as usual.
	\end{itemize}
}
\item verb \\
When you \textbf{arise}  \textbf{from} a sitting or kneeling position, you stand up.
 \textit{
	\begin{itemize}
	\item When I arose from the chair, my father and Eleanor's father were in deep conversation.
	\item Arise, Sir William.
	\end{itemize}
}
\item verb \\
You can  say that something tall such as a building or mountain  \textbf{arises}  \textbf{from} the ground around it.
 \textit{
	\begin{itemize}
	\item ...the flat terrace, from which arises the cubic volume of the house.
	\end{itemize}
}
\end{enumerate}

\section*{chairman}
{\large \color{blue}  chairmen  }
\subsection*{Explain}
\begin{enumerate}
\item countable noun \\
The \textbf{chairman} of a committee, organization , or company is the head of it.
 \textit{
	\begin{itemize}
	\item Glyn Ford is chairman of the Committee which produced the report.
	\item I had done business with the company's chairman.
	\end{itemize}
}
\item countable noun \\
The \textbf{chairman} of a meeting or debate is the person in charge , who decides when each person is allowed to speak .
 \textit{
	\begin{itemize}
	\item The chairman declared the meeting open.
	\item I hear you, Mr. Chairman.
	\end{itemize}
}
\end{enumerate}

\section*{bleed}
{\large \color{blue}  bleeds  bleeding  bled  }
\subsection*{Explain}
\begin{enumerate}
\item verb \\
When you \textbf{bleed} , you lose blood from your body as a result of injury or illness .
 \textit{
	\begin{itemize}
	\item His head had struck the sink and was bleeding.
	\item He was bleeding profusely.
	\item She's going to bleed to death!
	\end{itemize}
}
\item verb \\
If the colour of one substance \textbf{bleeds}  \textbf{into} the colour of another substance that it is touching , it goes into the other thing so that its colour changes in an undesirable way.
 \textit{
	\begin{itemize}
	\item The colouring pigments from the skins are not allowed to bleed into the grape juice.
	\end{itemize}
}
\item verb \\
If someone \textbf{is being bled} , money or other resources are gradually being taken away from them.
 \textit{
	\begin{itemize}
	\item We have been gradually bled for twelve years.
	\item They mean to bleed the British to the utmost.
	\end{itemize}
}
\item  \\
 bleed sb dry \textit{
	\begin{itemize}
	\end{itemize}
}
\end{enumerate}

\section*{concession}
{\large \color{blue}  concessions  }
\subsection*{Explain}
\begin{enumerate}
\item countable noun \\
If you make a \textbf{concession}  \textbf{to} someone, you agree to let them do or have something, especially in order to end an argument or conflict .
 \textit{
	\begin{itemize}
	\item The King made major concessions to end the confrontation with his people.
	\end{itemize}
}
\item countable noun \\
A \textbf{concession} is a special right or privilege that is given to someone.
 \textit{
	\begin{itemize}
	\item The government has granted concessions to three private telephone companies.
	\item ...tax concessions for mothers who stay at home with their children.
	\end{itemize}
}
\item countable noun \\
A \textbf{concession} is a special price which is lower than the usual price and which is often given to old people, students , and the unemployed .
 \textit{
	\begin{itemize}
	\item Open daily; admission £1.10 with concessions for children and OAPs.
	\end{itemize}
}
\item countable noun \\
A \textbf{concession} is an arrangement where someone is given the right to sell a product or to run a business, especially in a building  belonging to another business.
 \textit{
	\begin{itemize}
	\end{itemize}
}
\end{enumerate}

\section*{calculate}
{\large \color{blue}  calculates  calculating  calculated  }
\subsection*{Explain}
\begin{enumerate}
\item verb \\
If you \textbf{calculate} a number or amount, you discover it from information that you already have, by using arithmetic , mathematics , or a special machine.
 \textit{
	\begin{itemize}
	\item From this you can calculate the total mass in the Galaxy.
	\item We calculate that the average size farm in Lancaster County is 65 acres.
	\item A computer calculates by switching currents on or off.
	\end{itemize}
}
\item verb \\
If you \textbf{calculate} the effects of something, especially a possible  course of action, you think about them in order to form an opinion or decide what to do.
 \textit{
	\begin{itemize}
	\item I believe I am capable of calculating the political consequences accurately.
	\item He is calculating that the property market will be back on its feet within two years.
	\end{itemize}
}
\end{enumerate}

\section*{corporation}
{\large \color{blue}  corporations  }
\subsection*{Explain}
\begin{enumerate}
\item countable noun \\
A \textbf{corporation} is a large business or company.
 \textit{
	\begin{itemize}
	\item ...multi-national corporations.
	\item ...the Seiko Corporation.
	\end{itemize}
}
\item countable noun \\
In some large British cities, the \textbf{corporation} is the local authority that is responsible for providing public services.
 \textit{
	\begin{itemize}
	\item ...the corporation's task of regenerating 900 acres of the inner city.
	\end{itemize}
}
\end{enumerate}

\section*{cite}
{\large \color{blue}  cites  citing  cited  }
\subsection*{Explain}
\begin{enumerate}
\item verb \\
If you \textbf{cite} something, you quote it or mention it, especially as an example or proof of what you are saying .
 \textit{
	\begin{itemize}
	\item She cites a favourite poem by George Herbert.
	\item He cites just one example.
	\item I am merely citing his reaction as typical of British industry.
	\item Spain was cited as the most popular holiday destination.
	\end{itemize}
}
\item verb \\
To \textbf{cite} a person means to officially name them in a legal  case . To \textbf{cite} a reason or cause means to state it as the official reason for your case.
 \textit{
	\begin{itemize}
	\item They cited Alex's refusal to return to the marital home.
	\item Three admirals and a top Navy civilian will be cited for failing to act on reports
of sexual assaults.
	\end{itemize}
}
\item verb \\
If someone \textbf{is cited} , they are officially ordered to appear before a court.
 \textit{
	\begin{itemize}
	\item The judge ruled a mistrial and cited the prosecutors for outrageous misconduct.
	\end{itemize}
}
\end{enumerate}

\section*{deal}
{\large \color{blue}  }
\subsection*{Explain}
\begin{enumerate}
\item quantifier \\
If you say that you need or have \textbf{a great deal of} or \textbf{a good deal of} a particular thing, you are emphasizing that you need or have a lot of it.
 \textbf{Deal} is also an adverb .
 \textbf{Deal} is also a pronoun .
 \textit{
	\begin{itemize}
	\item ...a great deal of money.
	\item I am in a position to save you a good deal of time.
	\item Their lives became a good deal more comfortable.
	\item He depended a great deal on his partner for support.
	\item Although he had never met the man, he knew a good deal about him.
	\end{itemize}
}
\item quantifier \\
A \textbf{deal of} something is a lot of it.
 \textit{
	\begin{itemize}
	\item He had a deal of work to do.
	\end{itemize}
}
\end{enumerate}

\section*{gift}
{\large \color{blue}  gifts  }
\subsection*{Explain}
\begin{enumerate}
\item countable noun \\
A \textbf{gift} is something that you give someone as a present.
 \textit{
	\begin{itemize}
	\item ...a gift of $50.00.
	\item They believed the unborn child was a gift from God.
	\item ...gift shops.
	\end{itemize}
}
\item countable noun \\
If someone has a \textbf{gift}  \textbf{for} doing something, they have a natural ability for doing it.
 \textit{
	\begin{itemize}
	\item As a youth he discovered a gift for teaching.
	\item She had the gift of making people happy.
	\end{itemize}
}
\end{enumerate}

\section*{deem}
{\large \color{blue}  deems  deeming  deemed  }
\subsection*{Explain}
\begin{enumerate}
\item verb \\
If something \textbf{is deemed}  \textbf{to} have a particular quality or \textbf{to} do a particular thing, it is considered to have that quality or do that thing.
 \textit{
	\begin{itemize}
	\item French and German were deemed essential.
	\item He says he would support the use of force if the U.N. deemed it necessary.
	\item I was deemed to be a competent shorthand typist.
	\end{itemize}
}
\end{enumerate}

\section*{discern}
{\large \color{blue}  discerns  discerning  discerned  }
\subsection*{Explain}
\begin{enumerate}
\item verb \\
If you can \textbf{discern} something, you are aware of it and know what it is.
 \textit{
	\begin{itemize}
	\item You need a long series of data to be able to discern such a trend.
	\item It was hard to discern why this was happening.
	\end{itemize}
}
\item verb \\
If you can \textbf{discern} something, you can just see it, but not clearly.
 \textit{
	\begin{itemize}
	\item Below the bridge we could just discern a narrow, weedy ditch.
	\end{itemize}
}
\end{enumerate}

\section*{hall}
{\large \color{blue}  halls  }
\subsection*{Explain}
\begin{enumerate}
\item countable noun \\
The \textbf{hall} in a house or flat is the area just inside the front door , into which some of the other rooms open.
 \textit{
	\begin{itemize}
	\item The lights were on in the hall and in the bedroom.
	\end{itemize}
}
\item countable noun \\
A \textbf{hall} in a building is a long passage with doors into rooms on both sides of it.
 \textit{
	\begin{itemize}
	\end{itemize}
}
\item countable noun \\
A \textbf{hall} is a large room or building which is used for public events such as concerts, exhibitions , and meetings.
 \textit{
	\begin{itemize}
	\item Its 300 inhabitants will be celebrating with a dance in the village hall.
	\item We picked up our conference materials and filed into the lecture hall.
	\item His five-night residency at London's Royal Albert Hall was a tour-de-force.
	\end{itemize}
}
\item countable noun \\
If students  live  \textbf{in}  \textbf{hall} in British English, or \textbf{in a}  \textbf{hall} in American English, they live in a university or college building called a hall of residence .
 \textit{
	\begin{itemize}
	\end{itemize}
}
\item countable noun \\
\textbf{Hall} is sometimes used as part of the name of a large house in the country.
 \textit{
	\begin{itemize}
	\item He died at Holly Hall, his wife's family home.
	\end{itemize}
}
\item noun, in names \\
\textbf{Hall} is sometimes used as part of the name of a large building, especially one where public events or concerts take place
 \textit{
	\begin{itemize}
	\item ...New York's Carnegie Hall.
	\end{itemize}
}
\end{enumerate}

\section*{discover}
{\large \color{blue}  discovers  discovering  discovered  }
\subsection*{Explain}
\begin{enumerate}
\item verb \\
If you \textbf{discover} something that you did not know about before, you become aware of it or learn of it.
 \textit{
	\begin{itemize}
	\item She discovered that they'd escaped.
	\item I discovered I was pregnant.
	\item As he discovered, she had a brilliant mind.
	\item It was difficult for the inspectors to discover which documents were important.
	\item Haskell did not live to discover the deception.
	\item It was discovered that the files were missing.
	\end{itemize}
}
\item verb \\
If a person or thing \textbf{is discovered} , someone finds them, either by accident or because they have been looking for them.
 \textit{
	\begin{itemize}
	\item A few days later his badly beaten body was discovered on a roadside outside the city.
	\end{itemize}
}
\item verb \\
When someone \textbf{discovers} a new place, substance, scientific  fact , or scientific technique , they are the first person to find it or become aware of it.
 \textit{
	\begin{itemize}
	\item Astronomers have discovered a new planet on the edge of the solar system.
	\item In the 19th century, gold was discovered in California.
	\item They discovered how to form the image in a thin layer on the surface.
	\end{itemize}
}
\item verb \\
If you say that someone \textbf{has discovered} a particular activity or subject, you mean that they have tried doing it or studying it for the first time and that they enjoyed it.
 \textit{
	\begin{itemize}
	\item I wish I'd discovered photography when I was younger.
	\item Discover the delights and luxury of a private yacht.
	\end{itemize}
}
\item verb \\
When a actor , musician , or other performer who is not well-known  \textbf{is discovered} , someone recognizes that they have talent and helps them in their career .
 \textit{
	\begin{itemize}
	\item The Beatles were discovered in the early 1960's.
	\end{itemize}
}
\end{enumerate}

\section*{hamburger}
{\large \color{blue}  hamburgers  }
\subsection*{Explain}
\begin{enumerate}
\item countable noun \\
A \textbf{hamburger} is minced meat which has been shaped into a flat circle . Hamburgers are fried or grilled and then eaten , often in a bread roll.
 \textit{
	\begin{itemize}
	\end{itemize}
}
\end{enumerate}

\section*{fail}
{\large \color{blue}  fails  failing  failed  }
\subsection*{Explain}
\begin{enumerate}
\item verb \\
If you \textbf{fail} to do something that you were trying to do, you are unable to do it or do not succeed in doing it.
 \textit{
	\begin{itemize}
	\item The Workers' Party failed to win a single governorship.
	\item He failed in his attempt to take control of the company.
	\item Many of us have tried to lose weight and failed miserably.
	\item The truth is, I'm a failed comedy writer really.
	\end{itemize}
}
\item verb \\
If an activity, attempt, or plan \textbf{fails} , it is not successful .
 \textit{
	\begin{itemize}
	\item We tried to develop plans for them to get along, which all failed miserably.
	\item He was afraid the revolution they had started would fail.
	\item Scotch-Irish emigration began with the failed harvest of 1717.
	\end{itemize}
}
\item verb \\
If someone or something \textbf{fails} to do a particular thing that they should have done, they do not do it.
 \textit{
	\begin{itemize}
	\item Some schools fail to set any homework.
	\item He failed to file tax returns for 1982.
	\item The bomb failed to explode.
	\end{itemize}
}
\item verb \\
If something \textbf{fails} , it stops working properly, or does not do what it is supposed to do.
 \textit{
	\begin{itemize}
	\item The lights mysteriously failed, and we stumbled around in complete darkness.
	\item In fact many food crops failed because of the drought.
	\end{itemize}
}
\item verb \\
If a business, organization, or system \textbf{fails} , it becomes unable to continue in operation or in existence .
 \textit{
	\begin{itemize}
	\item So far this year, 104 banks have failed.
	\item ...a failed hotel business.
	\item Who wants to buy a computer from a failing company?
	\end{itemize}
}
\item verb \\
If something such as your health or a physical quality \textbf{is failing} , it is becoming gradually weaker or less effective .
 \textit{
	\begin{itemize}
	\item He was 58, and his health was failing rapidly.
	\item Here in the hills, the light failed more quickly.
	\item An apparently failing memory is damaging for a national leader.
	\end{itemize}
}
\item verb \\
If someone \textbf{fails} you, they do not do what you had expected or trusted them to do.
 \textit{
	\begin{itemize}
	\item We waited twenty-one years, don't fail us now.
	\item ...communities who feel that the political system has failed them.
	\end{itemize}
}
\item verb \\
If someone \textbf{fails in} their duty or \textbf{fails in} their responsibilities , they do not do everything that they have a duty or a responsibility to do.
 \textit{
	\begin{itemize}
	\item Lawyers are accused of failing in their duties to advise clients of their rights.
	\item If we did not report what was happening in the country, we would be failing in our
duty.
	\end{itemize}
}
\item verb \\
If a quality or ability that you have \textbf{fails} you, or if it \textbf{fails} , it is not good enough in a particular situation to enable you to do what you want to do.
 \textit{
	\begin{itemize}
	\item For once, the artist's fertile imagination failed him.
	\item Their courage failed a few steps short and they came running back.
	\end{itemize}
}
\item verb \\
If someone \textbf{fails} a test, examination, or course, they perform badly in it and do not reach the standard that is required.
 \textbf{Fail} is also a noun .
 \textit{
	\begin{itemize}
	\item I lived in fear of failing my end-of-term exams.
	\item It's the difference between a pass and a fail.
	\end{itemize}
}
\item verb \\
If someone \textbf{fails} you in a test, examination, or course, they judge that you have not reached a high
enough standard in it.
 \textit{
	\begin{itemize}
	\item ...the two men who had failed him during his first year of law school.
	\end{itemize}
}
\item  \\
 if all else fails \textit{
	\begin{itemize}
	\end{itemize}
}
\item  \\
 I fail to see/I fail to understand \textit{
	\begin{itemize}
	\end{itemize}
}
\item  \\
 without fail \textit{
	\begin{itemize}
	\end{itemize}
}
\item  \\
 without fail \textit{
	\begin{itemize}
	\end{itemize}
}
\end{enumerate}

\section*{fell}
{\large \color{blue}  fells  felling  felled  }
\subsection*{Explain}
\begin{enumerate}
\item  \\
\textbf{Fell} is the past  tense of fall .
 \textit{
	\begin{itemize}
	\end{itemize}
}
\item verb \\
If trees \textbf{are felled} , they are cut down.
 \textit{
	\begin{itemize}
	\item Badly infected trees should be felled and burned.
	\end{itemize}
}
\item verb \\
If you \textbf{fell} someone, you knock them down, for example in a fight .
 \textit{
	\begin{itemize}
	\item ...a blow on the forehead which felled him to the ground.
	\end{itemize}
}
\end{enumerate}

\section*{headache}
{\large \color{blue}  headaches  }
\subsection*{Explain}
\begin{enumerate}
\item countable noun \\
If you have a \textbf{headache} , you have a pain in your head.
 \textit{
	\begin{itemize}
	\item I have had a terrible headache for the last two days.
	\end{itemize}
}
\item countable noun \\
If you say that something is a \textbf{headache} , you mean that it causes you difficulty or worry.
 \textit{
	\begin{itemize}
	\item The airline's biggest headache is the increase in the price of aviation fuel.
	\item This is a real headache for us.
	\end{itemize}
}
\end{enumerate}

\section*{go}
{\large \color{blue}  goes  going  went  gone  }
\subsection*{Explain}
\begin{enumerate}
\item verb \\
When you \textbf{go}  somewhere , you move or travel there.
 \textit{
	\begin{itemize}
	\item We went to Rome.
	\item Gladys had just gone into the kitchen.
	\item I went home at the weekend.
	\item Four of them had gone off to find help.
	\item It took us an hour to go three miles.
	\end{itemize}
}
\item verb \\
When you \textbf{go} , you leave the place where you are.
 \textit{
	\begin{itemize}
	\item Let's go.
	\item She's going tomorrow.
	\end{itemize}
}
\item verb \\
You use \textbf{go} to say that someone leaves the place where they are and does an activity, often a
 leisure activity.
 \textit{
	\begin{itemize}
	\item We went swimming very early.
	\item Maybe they've just gone shopping.
	\item He went for a walk.
	\end{itemize}
}
\item verb \\
When you \textbf{go}  \textbf{to} do something, you move to a place in order to do it and you do it. You can also \textbf{go and} do something, and in American English, you can \textbf{go} do something. However, you always say that someone \textbf{went and} did something.
 \textit{
	\begin{itemize}
	\item His second son, Paddy, had gone to live in Canada.
	\item I must go and see this film.
	\item Go ask whoever you want.
	\end{itemize}
}
\item verb \\
If you \textbf{go to} school, work, or church, you attend it regularly as part of your normal life.
 \textit{
	\begin{itemize}
	\item She will have to go to school.
	\item His son went to a top university in America.
	\end{itemize}
}
\item verb \\
When you say where a road or path  \textbf{goes} , you are saying where it begins or ends, or what places it is in.
 \textit{
	\begin{itemize}
	\item There's a mountain road that goes from Blairstown to Millbrook Village.
	\end{itemize}
}
\item verb \\
You can use \textbf{go} in expressions such as ' \textbf{don't go telling everybody} ', in order to express disapproval of the kind of behaviour you mention , or to tell someone not to behave in that way.
 \textit{
	\begin{itemize}
	\item You don't have to go running upstairs every time she rings.
	\item Don't you go thinking it was your fault.
	\end{itemize}
}
\item verb \\
You can use \textbf{go} with words like 'further' and 'beyond' to show the degree or extent of something.
 \textit{
	\begin{itemize}
	\item He went even further in his speech to the conference.
	\item Some physicists have gone so far as to suggest that the entire Universe is a sort
of gigantic computer.
	\end{itemize}
}
\item verb \\
If you say that a period of time \textbf{goes} quickly or slowly, you mean that it seems to pass quickly or slowly.
 \textit{
	\begin{itemize}
	\item The weeks go so quickly!
	\end{itemize}
}
\item verb \\
If you say where money \textbf{goes} , you are saying what it is spent on.
 \textit{
	\begin{itemize}
	\item Most of my money goes on bills.
	\item The money goes to projects chosen by the wider community.
	\end{itemize}
}
\item verb \\
If you say that something \textbf{goes to} someone, you mean that it is given to them.
 \textit{
	\begin{itemize}
	\item A lot of credit must go to the chairman and his father.
	\item All the most important ministerial jobs went to men.
	\end{itemize}
}
\item verb \\
If someone \textbf{goes on} television or radio, they take part in a television or radio programme.
 \textit{
	\begin{itemize}
	\item The president has gone on television to defend stringent new security measures.
	\item We went on the air, live, at 7.30.
	\end{itemize}
}
\item verb \\
If something \textbf{goes} , someone gets rid of it.
 \textit{
	\begin{itemize}
	\item There are fears that 100,000 jobs will go.
	\item If people stand firm against the tax, it is only a matter of time before it has to
go.
	\end{itemize}
}
\item verb \\
If someone \textbf{goes} , they leave their job, usually because they are forced to.
 \textit{
	\begin{itemize}
	\item He had made a humiliating tactical error and he had to go.
	\end{itemize}
}
\item verb \\
If something \textbf{goes into} something else, it is put in it as one of the parts or elements that form it.
 \textit{
	\begin{itemize}
	\item ...the really interesting ingredients that go into the dishes that we all love to
eat.
	\end{itemize}
}
\item verb \\
If something \textbf{goes} in a particular place, it fits in that place or should be put there because it is
the right size or shape.
 \textit{
	\begin{itemize}
	\item He was trying to push it through the hole and it wouldn't go.
	\item ...This knob goes here.
	\end{itemize}
}
\item verb \\
If something \textbf{goes} in a particular place, it belongs there or should be put there, because that is where
you normally keep it.
 \textit{
	\begin{itemize}
	\item The shoes go on the shoe shelf.
	\item 'Where does everything go?'
	\end{itemize}
}
\item verb \\
If you say that one number \textbf{goes into} another number a particular number of times, you are dividing the second number by
the first.
 \textit{
	\begin{itemize}
	\item Six goes into thirty five times.
	\end{itemize}
}
\item verb \\
If one of a person's senses, such as their sight or hearing , \textbf{is going} , it is getting weak and they may soon lose it completely.
 \textit{
	\begin{itemize}
	\item His eyes are going; he says he has glaucoma.
	\item Lately he'd been making mistakes; his nerve was beginning to go.
	\end{itemize}
}
\item verb \\
If something such as a light bulb or a part of an engine \textbf{is going} , it is no longer working properly and will soon need to be replaced.
 \textit{
	\begin{itemize}
	\item I thought it looked as though the battery was going.
	\end{itemize}
}
\item verb \\
If you say that someone \textbf{is going} or \textbf{has gone} , you are saying in an indirect way that they are dying or are dead.
 \textit{
	\begin{itemize}
	\item 'Any hope?'—'No, he's gone.'
	\end{itemize}
}
\end{enumerate}

\section*{helmet}
{\large \color{blue}  helmets  }
\subsection*{Explain}
\begin{enumerate}
\item countable noun \\
A \textbf{helmet} is a hat made of a strong material which you wear to protect your head.
 \textit{
	\begin{itemize}
	\end{itemize}
}
\end{enumerate}

\section*{grow}
{\large \color{blue}  grows  growing  grew  grown  }
\subsection*{Explain}
\begin{enumerate}
\item verb \\
When people, animals, and plants \textbf{grow} , they increase in size and change physically over a period of time.
 \textit{
	\begin{itemize}
	\item We stop growing at maturity.
	\end{itemize}
}
\item verb \\
If a plant or tree \textbf{grows} in a particular place, it is alive there.
 \textit{
	\begin{itemize}
	\item The station had roses growing at each end of the platform.
	\end{itemize}
}
\item verb \\
If you \textbf{grow} a particular type of plant, you put  seeds or young plants in the ground and look after them as they develop.
 \textit{
	\begin{itemize}
	\item I always grow a few red onions.
	\item Lettuce was grown by the Ancient Romans.
	\end{itemize}
}
\item verb \\
When someone's hair \textbf{grows} , it gradually becomes longer. Your nails  also  \textbf{grow} .
 \textit{
	\begin{itemize}
	\item Then the hair began to grow again and I felt terrific.
	\end{itemize}
}
\item verb \\
If someone \textbf{grows} their hair, or \textbf{grows} a beard or moustache , they stop  cutting their hair or shaving so that their hair becomes longer. You can also \textbf{grow} your nails.
 \textit{
	\begin{itemize}
	\item I'd better start growing my hair.
	\end{itemize}
}
\item verb \\
If someone \textbf{grows} mentally, they change and develop in character or attitude .
 \textit{
	\begin{itemize}
	\item They began to grow as persons.
	\end{itemize}
}
\item link verb \\
You use \textbf{grow} to say that someone or something gradually changes until they have a new quality, feeling , or attitude.
 \textit{
	\begin{itemize}
	\item I grew a little afraid of the guy next door.
	\item He's growing old.
	\item He grew to love his work.
	\end{itemize}
}
\item verb \\
If an amount, feeling, or problem  \textbf{grows} , it becomes greater or more intense .
 \textit{
	\begin{itemize}
	\item Fears were growing last night for four skaters trapped under the wreckage of an ice
rink.
	\item Opposition grew and the government agreed to negotiate.
	\item ...a growing number of children living in poverty.
	\end{itemize}
}
\item verb \\
If one thing \textbf{grows into} another, it develops or changes until it becomes that thing.
 \textit{
	\begin{itemize}
	\item The boys grew into men.
	\item This political row threatens to grow into a full blown crisis.
	\end{itemize}
}
\item verb \\
If something such as an idea or a plan  \textbf{grows out of} something else, it develops from it.
 \textit{
	\begin{itemize}
	\item The idea for this book grew out of conversations with Philippa Brewster.
	\end{itemize}
}
\item verb \\
If the economy or a business  \textbf{grows} , it increases in wealth , size, or importance .
 \textit{
	\begin{itemize}
	\item The economy continues to grow.
	\item ...a fast growing business.
	\end{itemize}
}
\item verb \\
If someone \textbf{grows} a business, they take actions that will cause it to increase in wealth, size, or importance.
 \textit{
	\begin{itemize}
	\item To grow the business, he needs to develop management expertise and innovation across
his team.
	\end{itemize}
}
\item verb \\
If a crystal  \textbf{grows} , or if a scientist  \textbf{grows} it, it forms from a solution .
 \textit{
	\begin{itemize}
	\item ...crystals that grow in cavities in the rock.
	\item We tried to grow some copper sulphate crystals with our children.
	\end{itemize}
}
\end{enumerate}

\section*{hemisphere}
{\large \color{blue}  hemispheres  }
\subsection*{Explain}
\begin{enumerate}
\item countable noun \\
A \textbf{hemisphere} is one half of the earth .
 \textit{
	\begin{itemize}
	\item ...the northern hemisphere.
	\end{itemize}
}
\item countable noun \\
A \textbf{hemisphere} is one half of the brain .
 \textit{
	\begin{itemize}
	\item In most people, the left hemisphere is bigger than the right.
	\end{itemize}
}
\end{enumerate}

\section*{hit}
{\large \color{blue}  hits  hitting  }
\subsection*{Explain}
\begin{enumerate}
\item verb \\
If you \textbf{hit} someone or something, you deliberately touch them with a lot of force, with your hand or an object held in your hand.
 \textit{
	\begin{itemize}
	\item Find the exact grip that allows you to hit the ball hard.
	\item She hit him hard across his left arm.
	\item Police at the scene said the victim had been hit several times in the head.
	\end{itemize}
}
\item verb \\
When one thing \textbf{hits} another, it touches it with a lot of force.
 \textit{
	\begin{itemize}
	\item The car had apparently hit a traffic sign before skidding out of control.
	\item She hit the last barrier and sprawled across the track.
	\end{itemize}
}
\item verb \\
If a bomb or missile \textbf{hits} its target , it reaches it.
 \textbf{Hit} is also a noun .
 \textit{
	\begin{itemize}
	\item ...multiple-warhead missiles that could hit many targets at a time.
	\item The hospital had been hit with heavy artillery fire.
	\item First a house took a direct hit and then the rocket exploded.
	\end{itemize}
}
\item verb \\
If something \textbf{hits} a person, place, or thing, it affects them very badly .
 \textit{
	\begin{itemize}
	\item The plan to charge motorists £75 a year to use the motorway is going to hit me hard.
	\item Spain has been hit by storms since the beginning of the week.
	\item Special schools were hardest hit.
	\end{itemize}
}
\item verb \\
When a feeling or an idea \textbf{hits} you, it suddenly affects you or comes into your mind .
 \textit{
	\begin{itemize}
	\item It hit me that I had a choice.
	\item Then the answer hit me. It had been staring me in the face.
	\end{itemize}
}
\item verb \\
If you \textbf{hit} a particular high or low point on a scale of something such as success or health , you reach it.
 \textit{
	\begin{itemize}
	\item He admits to having hit the lowest point in his life.
	\item Oil prices hit record levels yesterday.
	\end{itemize}
}
\item countable noun \\
If a CD, film, or play is a \textbf{hit} , it is very popular and successful .
 \textit{
	\begin{itemize}
	\item The song became a massive hit in 1945.
	\item ...the surprise hit video of the year.
	\end{itemize}
}
\item countable noun \\
A \textbf{hit} is a single visit to a website.
 \textit{
	\begin{itemize}
	\item Our small company has had 78,000 hits on its internet pages.
	\end{itemize}
}
\item countable noun \\
If someone who is searching for information on the internet  gets a \textbf{hit} , they find a website where there is that information.
 \textit{
	\begin{itemize}
	\end{itemize}
}
\item  \\
 hit it off \textit{
	\begin{itemize}
	\end{itemize}
}
\item  \\
 make a hit \textit{
	\begin{itemize}
	\end{itemize}
}
\end{enumerate}

\section*{journalist}
{\large \color{blue}  journalists  }
\subsection*{Explain}
\begin{enumerate}
\item countable noun \\
A \textbf{journalist} is a person whose job is to collect  news and write about it for newspapers , magazines , television, or radio .
 \textit{
	\begin{itemize}
	\end{itemize}
}
\end{enumerate}

\section*{include}
{\large \color{blue}  includes  including  included  }
\subsection*{Explain}
\begin{enumerate}
\item verb \\
If one thing \textbf{includes} another thing, it has the other thing as one of its parts.
 \textit{
	\begin{itemize}
	\item A good British breakfast always includes sausages.
	\item The trip has been extended to include a few other events.
	\item The list includes many British internationals.
	\end{itemize}
}
\item verb \\
If someone or something \textbf{is included}  \textbf{in} a large group, system, or area, they become a part of it or are considered a part
of it.
 \textit{
	\begin{itemize}
	\item I had worked hard to be included in a project like this.
	\item The President is expected to include this idea in his education plan.
	\end{itemize}
}
\end{enumerate}

\section*{lightning}
{\large \color{blue}  }
\subsection*{Explain}
\begin{enumerate}
\item uncountable noun \\
\textbf{Lightning} is the very bright flashes of light in the sky that happen during thunderstorms.
 \textit{
	\begin{itemize}
	\item One man died when he was struck by lightning.
	\item Another flash of lightning lit up the cave.
	\item ...thunder and lightning.
	\end{itemize}
}
\item adjective \\
\textbf{Lightning}  describes things that happen very quickly or last for only a short time.
 \textit{
	\begin{itemize}
	\item Driving today demands lightning reflexes.
	\end{itemize}
}
\end{enumerate}

\section*{induce}
{\large \color{blue}  induces  inducing  induced  }
\subsection*{Explain}
\begin{enumerate}
\item verb \\
To \textbf{induce} a state or condition means to cause it.
 \textit{
	\begin{itemize}
	\item Doctors said surgery could induce a heart attack.
	\item ...an economic crisis induced by high oil prices.
	\end{itemize}
}
\item verb \\
If you \textbf{induce} someone \textbf{to} do something, you persuade or influence them to do it.
 \textit{
	\begin{itemize}
	\item I would do anything to induce them to stay.
	\item More than 4,000 teachers were induced to take early retirement.
	\end{itemize}
}
\item verb \\
If a doctor or nurse  \textbf{induces} labour or birth , they cause a pregnant woman to start giving birth by using drugs or other medical means.
 \textit{
	\begin{itemize}
	\item He might decide that it is best to induce labour.
	\end{itemize}
}
\end{enumerate}

\section*{mayor}
{\large \color{blue}  mayors  }
\subsection*{Explain}
\begin{enumerate}
\item countable noun \\
The \textbf{mayor} of a town or city is the person who has been elected to represent it for a fixed period of time or, in some places, to run its government.
 \textit{
	\begin{itemize}
	\end{itemize}
}
\end{enumerate}

\section*{memory}
{\large \color{blue}  memories  }
\subsection*{Explain}
\begin{enumerate}
\item variable noun \\
Your \textbf{memory} is your ability to remember things.
 \textit{
	\begin{itemize}
	\item All the details of the meeting are fresh in my memory.
	\item He'd a good memory for faces, and he was sure he hadn't seen her before.
	\item But locals with long memories thought this was fair revenge for the injustice of
1961.
	\item Two major areas in which these children require help are memory and attention.
	\end{itemize}
}
\item countable noun \\
A \textbf{memory} is something that you remember from the past.
 \textit{
	\begin{itemize}
	\item She cannot bear to watch the film because of the bad memories it brings back.
	\item He had happy memories of his father.
	\item Her earliest memory is of singing at the age of four to wounded soldiers.
	\end{itemize}
}
\item countable noun \\
A computer's \textbf{memory} is the part of the computer where information is stored, especially for a short time before it is transferred to disks or magnetic  tapes .
 \textit{
	\begin{itemize}
	\item The data are stored in the computer's memory.
	\end{itemize}
}
\item singular noun \\
If you talk about the \textbf{memory} of someone who has died , especially someone who was loved or respected , you are referring to the thoughts, actions, and ceremonies by which they are remembered.
 \textit{
	\begin{itemize}
	\item She remained devoted to his memory.
	\item The congress opened with a minute's silence in memory of those who died in the struggle.
	\end{itemize}
}
\item  \\
 from memory \textit{
	\begin{itemize}
	\end{itemize}
}
\item  \\
 in living memory \textit{
	\begin{itemize}
	\end{itemize}
}
\item  \\
 lose your memory \textit{
	\begin{itemize}
	\end{itemize}
}
\end{enumerate}

\section*{occupy}
{\large \color{blue}  occupies  occupying  occupied  }
\subsection*{Explain}
\begin{enumerate}
\item verb \\
The people who \textbf{occupy} a building or a place are the people who live or work there.
 \textit{
	\begin{itemize}
	\item There were over 40 tenants, all occupying one wing of the hospital.
	\item Land is, in most instances, purchased by those who occupy it.
	\end{itemize}
}
\item passive verb \\
If a room or something such as a seat  \textbf{is occupied} , someone is using it, so that it is not available for anyone else.
 \textit{
	\begin{itemize}
	\item Two-thirds of hospital beds are occupied by elderly people.
	\item I saw three camp beds, two of which were occupied.
	\end{itemize}
}
\item verb \\
If a group of people or an army  \textbf{occupies} a place or country , they move into it, using force in order to gain  control of it.
 \textit{
	\begin{itemize}
	\item U.S. forces now occupy a part of the country.
	\item Alexandretta had been occupied by the French in 1918 after the defeat of Turkey.
	\item ...the occupied territories.
	\end{itemize}
}
\item verb \\
If someone or something \textbf{occupies} a particular place in a system, process, or plan , they have that place.
 \textit{
	\begin{itemize}
	\item We occupy a quality position in the market place.
	\item Men still occupy more positions of power than women.
	\end{itemize}
}
\item verb \\
If something \textbf{occupies} you, or if you \textbf{occupy} yourself, your time, or your mind with it, you are busy doing that thing or thinking about it.
 \textit{
	\begin{itemize}
	\item Her parliamentary career has occupied all of her time.
	\item He hurried to take the suitcases and occupy himself with packing the car.
	\item I would deserve to be pitied if I couldn't occupy myself.
	\end{itemize}
}
\item verb \\
If something \textbf{occupies} you, it requires your efforts , attention, or time.
 \textit{
	\begin{itemize}
	\item I had other matters to occupy me, during the day at least.
	\item This challenge will occupy Europe for a generation or more.
	\end{itemize}
}
\item verb \\
If something \textbf{occupies} a particular area or place, it fills or covers it, or exists there.
 \textit{
	\begin{itemize}
	\item Even quite small aircraft occupy a lot of space.
	\item Bookshelves occupied most of the living room walls.
	\end{itemize}
}
\item verb \\
If something such as a journey  \textbf{occupies} a particular period of time, it takes that amount of time to complete .
 \textit{
	\begin{itemize}
	\item She reached Karachi on Monday evening, the journey having occupied three days and
nine hours.
	\end{itemize}
}
\end{enumerate}

\section*{necessity}
{\large \color{blue}  necessities  }
\subsection*{Explain}
\begin{enumerate}
\item uncountable noun \\
The \textbf{necessity} of something is the fact that it must  happen or exist .
 \textit{
	\begin{itemize}
	\item There is agreement on the necessity of reforms.
	\item As soon as the necessity for action is over the troops must be withdrawn.
	\item Most women, like men, work from economic necessity.
	\item Some people have to lead stressful lifestyles out of necessity.
	\end{itemize}
}
\item countable noun \\
A \textbf{necessity} is something that you must have in order to live properly or do something.
 \textit{
	\begin{itemize}
	\item Water is a basic necessity of life.
	\item ...food, fuel and other daily necessities.
	\end{itemize}
}
\item countable noun \\
A situation or action that is a \textbf{necessity} is necessary and cannot be avoided .
 \textit{
	\begin{itemize}
	\item The President pleaded that strong rule from the centre was a regrettable, but temporary
necessity.
	\end{itemize}
}
\end{enumerate}

\section*{occur}
{\large \color{blue}  occurs  occurring  occurred  }
\subsection*{Explain}
\begin{enumerate}
\item verb \\
When something \textbf{occurs} , it happens.
 \textit{
	\begin{itemize}
	\item If headaches only occur at night, lack of fresh air and oxygen is often the cause.
	\item The crash occurred when the crew shut down the wrong engine.
	\item In March 1770, there occurred what became known as the Boston Massacre.
	\end{itemize}
}
\item verb \\
When something \textbf{occurs} in a particular place, it exists or is present there.
 \textit{
	\begin{itemize}
	\item The cattle disease occurs more or less anywhere in Africa where the fly occurs.
	\item These snails do not occur on low-lying coral islands or atolls.
	\end{itemize}
}
\item verb \\
If a thought or idea  \textbf{occurs}  \textbf{to} you, you suddenly  think of it or realize it.
 \textit{
	\begin{itemize}
	\item It did not occur to me to check my insurance policy.
	\item It occurred to me that I could have the book sent to me.
	\item The same idea had occurred to Elizabeth.
	\end{itemize}
}
\end{enumerate}

\section*{nuisance}
{\large \color{blue}  nuisances  }
\subsection*{Explain}
\begin{enumerate}
\item countable noun \\
If you say that someone or something is a \textbf{nuisance} , you mean that they annoy you or cause you a lot of problems .
 \textit{
	\begin{itemize}
	\item He could be a bit of a nuisance when he was drunk.
	\item Sorry to be a nuisance.
	\end{itemize}
}
\end{enumerate}

\section*{offend}
{\large \color{blue}  offends  offending  offended  }
\subsection*{Explain}
\begin{enumerate}
\item verb \\
If you \textbf{offend} someone, you say or do something rude which upsets or embarrasses them.
 \textit{
	\begin{itemize}
	\item He apologizes for his comments and says he had no intention of offending the community.
	\item The survey found almost 90 percent of people were offended by strong swearwords.
	\item Television censors are cutting out scenes which they claim may offend.
	\end{itemize}
}
\item verb \\
To \textbf{offend}  \textbf{against} a law, rule , or principle means to break it.
 \textit{
	\begin{itemize}
	\item This bill offends against good sense and against justice.
	\item The peak age for offending the law is between 15 and 25.
	\end{itemize}
}
\item verb \\
If someone \textbf{offends} , they commit a crime .
 \textit{
	\begin{itemize}
	\item In Western countries girls are far less likely to offend than boys.
	\end{itemize}
}
\end{enumerate}

\section*{peninsula}
{\large \color{blue}  peninsulas  }
\subsection*{Explain}
\begin{enumerate}
\item countable noun \\
A \textbf{peninsula} is a long narrow piece of land which sticks out from a larger piece of land and is almost completely surrounded by water.
 \textit{
	\begin{itemize}
	\item I had walked around the entire peninsula.
	\end{itemize}
}
\end{enumerate}

\section*{pay}
{\large \color{blue}  pays  paying  paid  }
\subsection*{Explain}
\begin{enumerate}
\item verb \\
When you \textbf{pay} an amount of money \textbf{to} someone, you give it to them because you are buying something from them or because
you owe it to them. When you \textbf{pay} something such as a bill or a debt, you pay the amount that you owe.
 \textit{
	\begin{itemize}
	\item Accommodation is free–all you pay for is breakfast and dinner.
	\item We paid £35 for each ticket.
	\item The wealthier may have to pay a little more in taxes.
	\item He proposes that businesses should pay taxes to the federal government.
	\item You can pay by credit card.
	\end{itemize}
}
\item verb \\
When you \textbf{are paid} , you get your wages or salary from your employer .
 \textit{
	\begin{itemize}
	\item The lawyer was paid a huge salary.
	\item I get paid monthly.
	\item They could wander where they wished and take jobs from who paid best.
	\end{itemize}
}
\item uncountable noun \\
Your \textbf{pay} is the money that you get from your employer as wages or salary.
 \textit{
	\begin{itemize}
	\item ...their complaints about their pay and conditions.
	\item ...the workers' demand for a twenty per cent pay rise.
	\end{itemize}
}
\item verb \\
If you \textbf{are paid}  \textbf{to} do something, someone gives you some money so that you will help them or perform
some service for them.
 \textit{
	\begin{itemize}
	\item Students were paid substantial sums of money to do nothing all day but lie in bed.
	\item If you help me, I'll pay you anything.
	\end{itemize}
}
\item verb \\
If a government or organization makes someone \textbf{pay}  \textbf{for} something, it makes them responsible for providing the money for it, for example
by increasing prices or taxes.
 \textit{
	\begin{itemize}
	\item ...an international treaty that establishes who must pay for environmental damage.
	\item If you don't subsidize ballet and opera, seat prices will have to go up to pay for
it.
	\end{itemize}
}
\item verb \\
If a job , deal , or investment  \textbf{pays} a particular amount, it brings you that amount of money.
 \textit{
	\begin{itemize}
	\item We're stuck in jobs that don't pay very well.
	\item The account does not pay interest on a credit balance.
	\end{itemize}
}
\item verb \\
If a job, deal, or investment \textbf{pays} , it brings you a profit or earns you some money.
 \textit{
	\begin{itemize}
	\item There are some agencies now specialising in helping older people to find jobs which
pay.
	\item They owned land; they made it pay.
	\end{itemize}
}
\item verb \\
When you \textbf{pay} money \textbf{into} a bank account, you put the money in the account.
 \textit{
	\begin{itemize}
	\item He paid £20 into his savings account.
	\item There is nothing more annoying than queueing when you only want to pay in a few cheques.
	\end{itemize}
}
\item verb \\
If a course of action \textbf{pays} , it results in some advantage or benefit for you.
 \textit{
	\begin{itemize}
	\item It pays to invest in protective clothing.
	\item He talked of defending small nations, of ensuring that aggression does not pay.
	\end{itemize}
}
\item verb \\
If you \textbf{pay}  \textbf{for} something that you do or have, you suffer as a result of it.
 \textit{
	\begin{itemize}
	\item Britain was to pay dearly for its lack of resolve.
	\item Why should I pay the penalty for somebody else's mistake?
	\item She feels it's a small price to pay for the pleasure of living in this delightful
house.
	\end{itemize}
}
\item verb \\
You use \textbf{pay} with some nouns , for example in the expressions \textbf{pay a visit} and \textbf{pay attention} , to indicate that something is given or done.
 \textit{
	\begin{itemize}
	\item Do pay us a visit next time you're in Birmingham.
	\item He felt a heavy bump, but paid no attention to it.
	\item He had nothing to do with arranging the funeral, but came along to pay his last respects.
	\end{itemize}
}
\item adjective \\
\textbf{Pay} television consists of programmes and channels which are not part of a public broadcasting system, and for which people have to pay.
 \textit{
	\begin{itemize}
	\item ...the top 100 programmes on pay television in Australia.
	\end{itemize}
}
\item  \\
 pay for itself \textit{
	\begin{itemize}
	\end{itemize}
}
\item  \\
 in the pay of \textit{
	\begin{itemize}
	\end{itemize}
}
\item  \\
 to pay your way \textit{
	\begin{itemize}
	\end{itemize}
}
\end{enumerate}

\section*{proposition}
{\large \color{blue}  propositions  propositioning  propositioned  }
\subsection*{Explain}
\begin{enumerate}
\item countable noun \\
If you describe something such as a task or an activity as, for example , a difficult  \textbf{proposition} or an attractive  \textbf{proposition} , you mean that it is difficult or pleasant to do.
 \textit{
	\begin{itemize}
	\item Making easy money has always been an attractive proposition.
	\item Even among seasoned mountaineers this peak is considered quite a tough proposition.
	\end{itemize}
}
\item countable noun \\
A \textbf{proposition} is a statement or an idea which people can consider or discuss to decide whether it is true.
 \textit{
	\begin{itemize}
	\item The proposition that democracies do not fight each other is based on a tiny historical
sample.
	\end{itemize}
}
\item countable noun \\
In the United  States , a \textbf{proposition} is a question or statement about an issue of public  policy which appears on a voting  paper so that people can vote for or against it.
 \textit{
	\begin{itemize}
	\item I voted 'yes' on proposition 136, but 'no' on propositions 129, 133 and 134.
	\end{itemize}
}
\item countable noun \\
A \textbf{proposition} is an offer or a suggestion that someone makes to you, usually concerning some work or business that you might be able to do together .
 \textit{
	\begin{itemize}
	\item You came to see me at my office the other day with a business proposition.
	\item I want to make you a proposition.
	\end{itemize}
}
\item verb \\
If someone who you do not know very well  \textbf{propositions} you, they suggest that you have sex with them.
 \textbf{Proposition} is also a noun .
 \textit{
	\begin{itemize}
	\item He had allegedly tried to proposition a colleague.
	\item ...unwanted sexual propositions.
	\end{itemize}
}
\end{enumerate}

\section*{possess}
{\large \color{blue}  possesses  possessing  possessed  }
\subsection*{Explain}
\begin{enumerate}
\item verb \\
If you \textbf{possess} something, you have it or own it.
 \textit{
	\begin{itemize}
	\item He was then arrested and charged with possessing an offensive weapon.
	\item He is said to possess a fortune of more than two-and-a-half-thousand million dollars.
	\end{itemize}
}
\item verb \\
If someone or something \textbf{possesses} a particular quality, ability , or feature , they have it.
 \textit{
	\begin{itemize}
	\item ...individuals who are deemed to possess the qualities of sense, loyalty and discretion.
	\item This figure has long been held to possess miraculous power.
	\end{itemize}
}
\item verb \\
If a feeling or belief  \textbf{possesses} you, it strongly influences your thinking or behaviour .
 \textit{
	\begin{itemize}
	\item Absolute terror possessed her.
	\item Tsvetayeva was possessed by a frenzied urge to get out of Moscow.
	\end{itemize}
}
\item  \\
 what possessed you? \textit{
	\begin{itemize}
	\end{itemize}
}
\end{enumerate}

\section*{radius}
{\large \color{blue}  radii  }
\subsection*{Explain}
\begin{enumerate}
\item singular noun \\
The \textbf{radius} around a particular point is the distance from it in any direction.
 \textit{
	\begin{itemize}
	\item Nigel has searched for work in a ten-mile radius around his home.
	\end{itemize}
}
\item countable noun \\
The \textbf{radius} of a circle is the distance from its centre to its outside edge .
 \textit{
	\begin{itemize}
	\item He indicated a semicircle with a radius of about thirty miles.
	\end{itemize}
}
\end{enumerate}

\section*{produce}
{\large \color{blue}  produces  producing  produced  }
\subsection*{Explain}
\begin{enumerate}
\item verb \\
To \textbf{produce} something means to cause it to happen .
 \textit{
	\begin{itemize}
	\item The drug is known to produce side-effects in women.
	\item Talks aimed at producing a new world trade treaty have been under way for six years.
	\end{itemize}
}
\item verb \\
If you \textbf{produce} something, you make or create it.
 \textit{
	\begin{itemize}
	\item The company produced circuitry for communications systems.
	\item I'm quite pleased that we do have the capacity to produce that much food.
	\end{itemize}
}
\item verb \\
When things or people \textbf{produce} something, it comes from them or slowly forms from them, especially as the result of a biological or chemical process.
 \textit{
	\begin{itemize}
	\item These plants are then pollinated and allowed to mature and produce seed.
	\item ...gases produced by burning coal and oil.
	\end{itemize}
}
\item verb \\
If you \textbf{produce}  evidence or an argument , you show it or explain it to people in order to make them agree with you.
 \textit{
	\begin{itemize}
	\item They challenged him to produce evidence to support his allegations.
	\item Scientists have produced powerful arguments against his ideas.
	\end{itemize}
}
\item verb \\
If you \textbf{produce} an object from somewhere , you show it or bring it out so that it can be seen .
 \textit{
	\begin{itemize}
	\item To hire a car you must produce a passport and a current driving licence.
	\end{itemize}
}
\item verb \\
If someone \textbf{produces} something such as a film, a magazine , or a CD, they organize it and decide how it should be done.
 \textit{
	\begin{itemize}
	\item He has produced his own sports magazine.
	\item He had four months to produce a film from a script he had already written.
	\end{itemize}
}
\item uncountable noun \\
\textbf{Produce} is food or other things that are grown in large quantities to be sold.
 \textit{
	\begin{itemize}
	\item We manage to get most of our produce in Britain.
	\item Winter produce will cost more for the next few weeks.
	\end{itemize}
}
\end{enumerate}

\section*{record}
{\large \color{blue}  records  recording  recorded  }
\subsection*{Explain}
\begin{enumerate}
\item countable noun \\
If you keep a \textbf{record}  \textbf{of} something, you keep a written account or photographs of it so that it can be referred to later.
 \textit{
	\begin{itemize}
	\item Keep a record of all the payments.
	\item There's no record of them having any children.
	\item The result will go on your medical records.
	\end{itemize}
}
\item verb \\
If you \textbf{record} a piece of information or an event, you write it down, photograph it, or put it into
a computer so that in the future people can refer to it.
 \textit{
	\begin{itemize}
	\item ...software packages which record the details of your photographs.
	\item ...a place which has rarely suffered a famine in its recorded history.
	\end{itemize}
}
\item verb \\
If you \textbf{record} something such as a speech or performance, you put it on tape or film so that it can be heard or seen again later.
 \textit{
	\begin{itemize}
	\item There is nothing to stop viewers recording the films at home.
	\item The call was answered by a recorded message saying the company had closed early.
	\end{itemize}
}
\item verb \\
If a musician or performer \textbf{records} a piece of music or a television or radio show, they perform it so that it can be
put online or onto CD or film.
 \textit{
	\begin{itemize}
	\item It took the musicians two and a half days to record their soundtrack for the film.
	\item She has recently recorded a programme for television.
	\end{itemize}
}
\item countable noun \\
A \textbf{record} is a round, flat piece of black plastic on which sound, especially music, is stored, and which can be played on a record player. You can also refer
to the music stored on this piece of plastic as a \textbf{record} .
 \textit{
	\begin{itemize}
	\item This is one of my favourite records.
	\item ...the biggest and best-known record company in England.
	\end{itemize}
}
\item verb \\
If a dial or other measuring device \textbf{records} a certain measurement or value, it shows that measurement or value.
 \textit{
	\begin{itemize}
	\item The test records the electrical activity of the brain.
	\item The index of the performance of leading shares recorded a 16 per cent fall.
	\end{itemize}
}
\item countable noun \\
A \textbf{record} is the best result that has ever been achieved in a particular sport or activity, for example the fastest time, the furthest distance, or the greatest number of victories .
 \textit{
	\begin{itemize}
	\item He set the world record of 12.92 seconds.
	\item The painting was sold for £665,000–a record for the artist.
	\item ...the 800 metres, where she is the world record holder.
	\end{itemize}
}
\item adjective \\
You use \textbf{record} to say that something is higher , lower, better , or worse than has ever been achieved before.
 \textit{
	\begin{itemize}
	\item Profits were at record levels.
	\item She won the race in record time.
	\end{itemize}
}
\item countable noun \\
Someone's \textbf{record} is the facts that are known about their achievements or character.
 \textit{
	\begin{itemize}
	\item His record reveals a tough streak.
	\item He had a distinguished record as a chaplain.
	\item His country is making a big effort to improve its human rights record.
	\end{itemize}
}
\item countable noun \\
If someone has a criminal  \textbf{record} , it is officially known that they have committed crimes in the past .
 \textit{
	\begin{itemize}
	\item ...a heroin addict with a criminal record going back 15 years.
	\item Where the accused has a record of violence, they should always be kept in custody.
	\end{itemize}
}
\item  \\
 for the record \textit{
	\begin{itemize}
	\end{itemize}
}
\item  \\
 for the record \textit{
	\begin{itemize}
	\end{itemize}
}
\item  \\
 off the record \textit{
	\begin{itemize}
	\end{itemize}
}
\item  \\
 on record \textit{
	\begin{itemize}
	\end{itemize}
}
\item  \\
 on record \textit{
	\begin{itemize}
	\end{itemize}
}
\item  \\
 on record \textit{
	\begin{itemize}
	\end{itemize}
}
\item  \\
 to set the record straight \textit{
	\begin{itemize}
	\end{itemize}
}
\end{enumerate}

\section*{promise}
{\large \color{blue}  promises  promising  promised  }
\subsection*{Explain}
\begin{enumerate}
\item verb \\
If you \textbf{promise}  \textbf{that} you will do something, you say to someone that you will definitely do it.
 \textit{
	\begin{itemize}
	\item The post office has promised to resume first class mail delivery to the area on Friday.
	\item He had promised that the rich and privileged would no longer get preferential treatment.
	\item Promise me you will not waste your time.
	\item 'We'll be back next year,' he promised.
	\item 'You promise?'—'All right, I promise.'
	\end{itemize}
}
\item verb \\
If you \textbf{promise} someone something, you tell them that you will definitely give it to them or make sure that they have it.
 \textit{
	\begin{itemize}
	\item In 1920 the great powers promised them an independent state.
	\item The officers promise a return to multiparty rule.
	\end{itemize}
}
\item countable noun \\
A \textbf{promise} is a statement which you make to a person in which you say that you will definitely
do something or give them something.
 \textit{
	\begin{itemize}
	\item If you make a promise, you should keep it.
	\item The program has lived up to its promise to promote family welfare.
	\end{itemize}
}
\item verb \\
If a situation or event  \textbf{promises}  \textbf{to} have a particular quality or \textbf{to} be a particular thing, it shows  signs that it will have that quality or be that thing.
 \textit{
	\begin{itemize}
	\item While it will be fun, the seminar also promises to be most instructive.
	\end{itemize}
}
\item uncountable noun \\
If someone or something shows \textbf{promise} , they seem likely to be very good or successful .
 \textit{
	\begin{itemize}
	\item The boy first showed promise as an athlete in grade school.
	\end{itemize}
}
\end{enumerate}

\section*{recorder}
{\large \color{blue}  recorders  }
\subsection*{Explain}
\begin{enumerate}
\item variable noun \\
A \textbf{recorder} is a wooden or plastic  musical instrument in the shape of a pipe . You play the recorder by blowing into the top of it and covering and uncovering the holes with your fingers .
 \textit{
	\begin{itemize}
	\end{itemize}
}
\item countable noun \\
In the legal system of England and Wales, a \textbf{recorder} is a lawyer who is appointed as a part-time judge in the Crown Court.
 \textit{
	\begin{itemize}
	\end{itemize}
}
\item countable noun \\
A \textbf{recorder} is a machine or instrument that keeps a record of something, for example in an experiment or on a vehicle .
 \textit{
	\begin{itemize}
	\item Data recorders also pin-point mechanical faults rapidly, reducing repair times.
	\end{itemize}
}
\item countable noun \\
In the past , you could refer to a cassette recorder, a tape recorder, or a video recorder as a \textbf{recorder} .
 \textit{
	\begin{itemize}
	\item Rodney put the recorder on the desk top and pushed the play button.
	\end{itemize}
}
\end{enumerate}

\section*{pronounce}
{\large \color{blue}  pronounces  pronouncing  pronounced  }
\subsection*{Explain}
\begin{enumerate}
\item verb \\
To \textbf{pronounce} a word means to say it using particular sounds.
 \textit{
	\begin{itemize}
	\item Have I pronounced your name correctly?
	\item He pronounced it Per-sha, the way the English do.
	\end{itemize}
}
\item verb \\
If you \textbf{pronounce} something to be true , you state that it is the case .
 \textit{
	\begin{itemize}
	\item A specialist has now pronounced him fully fit.
	\item I now pronounce you husband and wife.
	\end{itemize}
}
\item verb \\
If someone \textbf{pronounces} a verdict or opinion  \textbf{on} something, they give their verdict or opinion.
 \textit{
	\begin{itemize}
	\item The authorities took time to pronounce their verdicts.
	\item 'As for me,' he pronounced, 'I can recognize a good deal when I see one'.
	\item He walked around the garden and pronounced himself satisfied.
	\end{itemize}
}
\end{enumerate}

\section*{quote}
{\large \color{blue}  quotes  quoting  quoted  }
\subsection*{Explain}
\begin{enumerate}
\item verb \\
If you \textbf{quote} someone as saying something, you repeat what they have written or said .
 \textit{
	\begin{itemize}
	\item He quoted Mr Polay as saying that peace negotiations were already underway.
	\item She quoted a great line from a book by Romain Gary.
	\item The newspaper quoted a professor who said he witnessed the killings.
	\item I gave the letter to our local press and they quoted from it.
	\end{itemize}
}
\item countable noun \\
A \textbf{quote}  \textbf{from} a book, poem, play, or speech is a passage or phrase from it.
 \textit{
	\begin{itemize}
	\item The article starts with a quote from an unnamed member of the Cabinet.
	\end{itemize}
}
\item verb \\
If you \textbf{quote} something such as a law or a fact , you state it because it supports what you are saying.
 \textit{
	\begin{itemize}
	\item Mr Meacher quoted statistics saying that the standard of living of the poorest people
had fallen.
	\end{itemize}
}
\item verb \\
If someone \textbf{quotes} a price \textbf{for} doing something, they say how much money they would charge you for a service they are offering or a for a job that you want them to do.
 \textit{
	\begin{itemize}
	\item A travel agent quoted her £160 for the flight.
	\item He quoted a price for the repairs.
	\end{itemize}
}
\item countable noun \\
A \textbf{quote}  \textbf{for} a piece of work is the price that someone says they will charge you to do the work.
 \textit{
	\begin{itemize}
	\item Always get a written quote for any repairs needed.
	\end{itemize}
}
\item passive verb \\
If a company's shares , a substance, or a currency  \textbf{is quoted} at a particular price, that is its current market price.
 \textit{
	\begin{itemize}
	\item In early trading, gold was quoted at $368.20 an ounce.
	\item Heron is a private company and is not quoted on the Stock Market.
	\end{itemize}
}
\item plural noun \\
\textbf{Quotes} are the same as quotation marks .
 \textit{
	\begin{itemize}
	\item The word 'remembered' is in quotes.
	\end{itemize}
}
\item convention \\
You can say ' \textbf{quote} ' to show that you are about to quote someone's words.
 \textit{
	\begin{itemize}
	\item He predicts they will have, quote, 'an awful lot of explaining to do'.
	\end{itemize}
}
\end{enumerate}

\section*{standpoint}
{\large \color{blue}  standpoints  }
\subsection*{Explain}
\begin{enumerate}
\item countable noun \\
\textbf{From} a particular \textbf{standpoint} means looking at an event, situation , or idea in a particular way.
 \textit{
	\begin{itemize}
	\item He believes that from a military standpoint, the situation is under control.
	\item From my standpoint, you know, this thing is just ridiculous.
	\end{itemize}
}
\end{enumerate}

\section*{refrain}
{\large \color{blue}  refrains  refraining  refrained  }
\subsection*{Explain}
\begin{enumerate}
\item verb \\
If you \textbf{refrain}  \textbf{from} doing something, you deliberately do not do it.
 \textit{
	\begin{itemize}
	\item Mrs Hardie refrained from making any comment.
	\item He appealed to all factions to refrain from violence.
	\end{itemize}
}
\item countable noun \\
A \textbf{refrain} is a short, simple part of a song, which is repeated many times.
 \textit{
	\begin{itemize}
	\item ...a refrain from an old song.
	\end{itemize}
}
\item countable noun \\
A \textbf{refrain} is a comment or saying that people often repeat.
 \textit{
	\begin{itemize}
	\item Rosa's constant refrain is that she doesn't have a life.
	\end{itemize}
}
\end{enumerate}

\section*{staple}
{\large \color{blue}  staples  stapling  stapled  }
\subsection*{Explain}
\begin{enumerate}
\item adjective \\
A \textbf{staple} food, product, or activity is one that is basic and important in people's everyday  lives .
 \textbf{Staple} is also a noun .
 \textit{
	\begin{itemize}
	\item Rice is the staple food of more than half the world's population.
	\item The Chinese also eat a type of pasta as part of their staple diet.
	\item Staple goods are disappearing from the shops.
	\item Fish is a staple in the diet of many Africans.
	\item ...boutiques selling staples such as jeans and T-shirts.
	\end{itemize}
}
\item countable noun \\
A \textbf{staple} is something that forms an important part of something else.
 \textit{
	\begin{itemize}
	\item Political reporting has become a staple of American journalism.
	\end{itemize}
}
\item countable noun \\
\textbf{Staples} are small pieces of bent wire that are used mainly for holding sheets of paper together firmly. You put the staples into the paper using a device called a stapler.
 \textit{
	\begin{itemize}
	\end{itemize}
}
\item verb \\
If you \textbf{staple} something, you fasten it to something else or fix it in place using staples.
 \textit{
	\begin{itemize}
	\item Staple some sheets of paper together into a book.
	\item ...polythene bags stapled to an illustrated card.
	\end{itemize}
}
\end{enumerate}

\section*{remember}
{\large \color{blue}  remembers  remembering  remembered  }
\subsection*{Explain}
\begin{enumerate}
\item verb \\
If you \textbf{remember} people or events from the past , you still have an idea of them in your mind and you are able to think about them.
 \textit{
	\begin{itemize}
	\item You wouldn't remember me. I was in another group.
	\item I certainly don't remember talking to you at all.
	\item I remember her being a dominant figure.
	\item I remembered that we had drunk the last of the coffee the week before.
	\item I can remember where and when I bought each one.
	\item I used to do that when you were a little girl, remember?
	\end{itemize}
}
\item verb \\
If you \textbf{remember} that something is the case , you become aware of it again after a time when you did not think about it.
 \textit{
	\begin{itemize}
	\item She remembered that she was going to the social club that evening.
	\item Then I remembered the cheque, which cheered me up.
	\end{itemize}
}
\item verb \\
If you cannot \textbf{remember} something, you are not able to bring it back into your mind when you make an effort to do so.
 \textit{
	\begin{itemize}
	\item If you can't remember your number, write it in code in a diary.
	\item I couldn't remember ever having felt so safe and secure.
	\item I don't remember you asking me about that.
	\item I can't remember what I said.
	\item Don't tell me you can't remember.
	\end{itemize}
}
\item verb \\
If you \textbf{remember}  \textbf{to} do something, you do it when you intend to.
 \textit{
	\begin{itemize}
	\item I did remember to take the present.
	\item Please remember to enclose a stamped addressed envelope when writing.
	\end{itemize}
}
\item verb \\
You tell someone to \textbf{remember}  \textbf{that} something is the case when you want to emphasize its importance . It may be something that they already  know about or a new  piece of information .
 \textit{
	\begin{itemize}
	\item It is important to remember that each person reacts differently.
	\item It is worth remembering that children tend to copy their parents in this respect.
	\item It should be remembered that this loss of control can never be regained.
	\end{itemize}
}
\item verb \\
If you say that someone will \textbf{be remembered} for something that they have done , you mean that people will think of this whenever they think about the person.
 \textit{
	\begin{itemize}
	\item At his grammar school he is remembered for being bad at games.
	\item Lincoln is remembered as the man who abolished slavery in the US.
	\end{itemize}
}
\item verb \\
If you ask someone to \textbf{remember} you \textbf{to} a person who you have not seen for a long time, you are asking them to pass your greetings on to that person.
 \textit{
	\begin{itemize}
	\item 'Remember me to Lyle, won't you?' I said.
	\item She asked to be remembered to you.
	\end{itemize}
}
\item verb \\
If you make a celebration an occasion  \textbf{to}  \textbf{remember} , you make it very enjoyable for all the people involved .
 \textit{
	\begin{itemize}
	\item We'll give everyone a night to remember.
	\item I'll make it a birthday to remember.
	\end{itemize}
}
\end{enumerate}

\section*{repel}
{\large \color{blue}  repels  repelling  repelled  }
\subsection*{Explain}
\begin{enumerate}
\item verb \\
When an army  \textbf{repels} an attack , they successfully fight and drive back soldiers from another army who have attacked them.
 \textit{
	\begin{itemize}
	\item They have fifty thousand troops along the border ready to repel any attack.
	\end{itemize}
}
\item verb \\
When a magnetic  pole  \textbf{repels} another magnetic pole, it gives out a force that pushes the other pole away. You can also  say that two magnetic poles \textbf{repel} each other or that they \textbf{repel} .
 \textit{
	\begin{itemize}
	\item Like poles repel, unlike poles attract.
	\item As these electrons are negatively charged they will attempt to repel each other.
	\end{itemize}
}
\item verb \\
If something \textbf{repels} you, you find it horrible and disgusting.
 \textit{
	\begin{itemize}
	\item ...a violent excitement that frightened and repelled her.
	\end{itemize}
}
\end{enumerate}

\section*{subject}
{\large \color{blue}  subjects  subjecting  subjected  }
\subsection*{Explain}
\begin{enumerate}
\item countable noun \\
The \textbf{subject} of something such as a conversation , letter, or book is the thing that is being discussed or written about.
 \textit{
	\begin{itemize}
	\item It was I who first raised the subject of plastic surgery.
	\item ...the president's own views on the subject.
	\item ...steering the conversation round to his favourite subject.
	\end{itemize}
}
\item countable noun \\
Someone or something that is the \textbf{subject of}  criticism , study, or an investigation is being criticized , studied, or investigated .
 \textit{
	\begin{itemize}
	\item Over the past few years, some of the positions Mr. Meredith has adopted have made
him the subject of criticism.
	\item He's now the subject of an official inquiry.
	\end{itemize}
}
\item countable noun \\
A \textbf{subject} is an area of knowledge or study, especially one that you study at school, college , or university.
 \textit{
	\begin{itemize}
	\item Surprisingly, mathematics was voted their favourite subject.
	\item ...a tutor in maths and science subjects.
	\end{itemize}
}
\item countable noun \\
In an experiment or piece of research , the \textbf{subject} is the person or animal that is being tested or studied.
 \textit{
	\begin{itemize}
	\item 'White noise' was played into the subject's ears through headphones.
	\item Subjects in the study were asked to follow a modified diet.
	\end{itemize}
}
\item countable noun \\
An artist's \textbf{subjects} are the people, animals, or objects that he or she paints , models , or photographs .
 \textit{
	\begin{itemize}
	\item Her favourite subjects are shells spotted on beach walks.
	\end{itemize}
}
\item countable noun \\
In grammar , the \textbf{subject} of a clause is the noun group that refers to the person or thing that is doing the action expressed by the verb . For example, in 'My cat keeps catching birds', 'my cat' is the subject.
 \textit{
	\begin{itemize}
	\end{itemize}
}
\item adjective \\
To be \textbf{subject to} something means to be affected by it or to be likely to be affected by it.
 \textit{
	\begin{itemize}
	\item Prices may be subject to alteration.
	\item Foreign wine was subject to an import tax.
	\item ...a disorder in which the person's mood is subject to wild swings from mania to
depression.
	\end{itemize}
}
\item adjective \\
If someone is \textbf{subject to} a particular set of rules or laws, they have to obey those rules or laws.
 \textit{
	\begin{itemize}
	\item The tribunal is unique because Mr Jones is not subject to the normal police discipline
code.
	\item ...arguing that as a sovereign state it could not be subject to another country's
laws.
	\end{itemize}
}
\item verb \\
If you \textbf{subject} someone \textbf{to} something unpleasant , you make them experience it.
 \textit{
	\begin{itemize}
	\item ...the man who had subjected her to four years of beatings and abuse.
	\item Innocent civilians are being arrested and subjected to inhumane treatment.
	\end{itemize}
}
\item countable noun \\
The people who live in or belong to a particular country, usually one ruled by a monarch, are the \textbf{subjects} of that monarch or country.
 \textit{
	\begin{itemize}
	\item ...his subjects regarded him as a great and wise monarch.
	\item Roughly half of them are British subjects.
	\end{itemize}
}
\item adjective \\
\textbf{Subject} peoples and countries are ruled or controlled by the government of another country.
 \textit{
	\begin{itemize}
	\item The subject peoples of her empire were anxious for their own independence.
	\item ...colonies and other subject territories.
	\end{itemize}
}
\item  \\
 to change the subject \textit{
	\begin{itemize}
	\end{itemize}
}
\item  \\
 subject to sth \textit{
	\begin{itemize}
	\end{itemize}
}
\end{enumerate}

\section*{reproduce}
{\large \color{blue}  reproduces  reproducing  reproduced  }
\subsection*{Explain}
\begin{enumerate}
\item verb \\
If you try to \textbf{reproduce} something, you try to copy it.
 \textit{
	\begin{itemize}
	\item I shall not try to reproduce the policemen's English.
	\item The effect has proved hard to reproduce.
	\end{itemize}
}
\item verb \\
If you \textbf{reproduce} a picture , speech , or a piece of writing , you make a photograph or printed copy of it.
 \textit{
	\begin{itemize}
	\item We are grateful to you for permission to reproduce this article.
	\end{itemize}
}
\item verb \\
If you \textbf{reproduce} an action or an achievement , you repeat it.
 \textit{
	\begin{itemize}
	\item If we can reproduce the form we have shown in the last couple of months we will be
successful.
	\end{itemize}
}
\item verb \\
When people, animals, or plants \textbf{reproduce} , they produce young .
 \textit{
	\begin{itemize}
	\item ...a society where women are defined by their ability to reproduce.
	\item We are not reproducing ourselves fast enough to pay for the welfare of our older
citizens.
	\end{itemize}
}
\end{enumerate}

\section*{suburb}
{\large \color{blue}  suburbs  }
\subsection*{Explain}
\begin{enumerate}
\item countable noun \\
A \textbf{suburb}  \textbf{of} a city or large town is a smaller area which is part of the city or large town but
is outside its centre .
 \textit{
	\begin{itemize}
	\item Anna was born in 1923 in Ardwick, a suburb of Manchester.
	\item ...the north London suburbs of Harrow, Barnet and Enfield.
	\end{itemize}
}
\item plural noun \\
If you live  \textbf{in the}  \textbf{suburbs} , you live in an area of houses outside the centre of a large town or city.
 \textit{
	\begin{itemize}
	\item His family lived in the suburbs.
	\item ...Bombay's suburbs.
	\end{itemize}
}
\end{enumerate}

\section*{sparkle}
{\large \color{blue}  sparkles  sparkling  sparkled  }
\subsection*{Explain}
\begin{enumerate}
\item verb \\
If something \textbf{sparkles} , it is clear and bright and shines with a lot of very small points of light.
 \textbf{Sparkle} is also a noun .
 \textit{
	\begin{itemize}
	\item The jewels on her fingers sparkled.
	\item His bright eyes sparkled.
	\item ...the sparkling blue waters of the ocean.
	\item ...the sparkle of coloured glass.
	\end{itemize}
}
\item countable noun \\
\textbf{Sparkles} are small points of light caused by light reflecting off a clear bright surface.
 \textit{
	\begin{itemize}
	\item ...sparkles of light.
	\item There was a sparkle in her eye that could not be hidden.
	\end{itemize}
}
\item verb \\
Someone who \textbf{sparkles} is lively , intelligent , and witty.
 \textbf{Sparkle} is also a noun.
 \textit{
	\begin{itemize}
	\item She sparkles with wit and charm.
	\item They'd been dejected when he arrived, but now they sparkled with enthusiasm.
	\item There was little sparkle in their performance.
	\end{itemize}
}
\end{enumerate}

\section*{symposium}
{\large \color{blue}  symposia  symposiums  }
\subsection*{Explain}
\begin{enumerate}
\item countable noun \\
A \textbf{symposium} is a conference in which experts or academics discuss a particular subject.
 \textit{
	\begin{itemize}
	\item He had been taking part in an international symposium on population.
	\end{itemize}
}
\end{enumerate}

\section*{strengthen}
{\large \color{blue}  strengthens  strengthening  strengthened  }
\subsection*{Explain}
\begin{enumerate}
\item verb \\
If something \textbf{strengthens} a person or group or if they \textbf{strengthen} their position , they become more powerful and secure , or more likely to succeed .
 \textit{
	\begin{itemize}
	\item The new constitution strengthens the government.
	\item To strengthen his position in Parliament, he held talks with the opposition.
	\item He hoped to strengthen the position of the sciences in the leading universities.
	\end{itemize}
}
\item verb \\
If something \textbf{strengthens} a case or argument , it supports it by providing more reasons or evidence for it.
 \textit{
	\begin{itemize}
	\item He does not seem to be familiar with research which might have strengthened his own
arguments.
	\end{itemize}
}
\item verb \\
If a currency , economy , or industry  \textbf{strengthens} , or if something \textbf{strengthens} it, it increases in value or becomes more successful .
 \textit{
	\begin{itemize}
	\item The dollar strengthened against most other currencies.
	\item The Government should start by strengthening the economy.
	\end{itemize}
}
\item verb \\
If a government  \textbf{strengthens}  laws or measures or if they \textbf{strengthen} , they are made more severe .
 \textit{
	\begin{itemize}
	\item I am also looking urgently at how we can strengthen the law.
	\item Community leaders want to strengthen controls at external frontiers.
	\item Because of the war, security procedures have strengthened.
	\end{itemize}
}
\item verb \\
If something \textbf{strengthens} you or \textbf{strengthens} your resolve or character , it makes you more confident and determined .
 \textit{
	\begin{itemize}
	\item Any experience can teach and strengthen you, but particularly the more difficult
ones.
	\item This merely strengthens our resolve to win the league.
	\item She began to believe that Nick would survive, and every day that came and went strengthened
her conviction.
	\end{itemize}
}
\item verb \\
If something \textbf{strengthens} a relationship or link , or if a relationship or link \textbf{strengthens} , it makes it closer and more likely to last for a long time.
 \textit{
	\begin{itemize}
	\item It will draw you closer together, and it will strengthen the bond of your relationship.
	\item His visit is intended to strengthen ties between the two countries.
	\item In a strange way, his affair caused our relationship to strengthen.
	\end{itemize}
}
\item verb \\
If something \textbf{strengthens} an impression , feeling , or belief , or if it \textbf{strengthens} , it becomes greater or affects more people.
 \textit{
	\begin{itemize}
	\item His speech strengthens the impression he is the main power in the organization.
	\item Every day of sunshine strengthens the feelings of optimism.
	\item Amy's own Republican sympathies strengthened as the days passed.
	\end{itemize}
}
\item verb \\
If something \textbf{strengthens} your body or a part of your body, it makes it healthier , often in such a way that you can  move or carry  heavier things.
 \textit{
	\begin{itemize}
	\item Cycling is good exercise. It strengthens all the muscles of the body.
	\item Yoga can be used to strengthen the immune system.
	\end{itemize}
}
\item verb \\
If something \textbf{strengthens} an object or structure , it makes it able to be treated  roughly or able to support heavy weights , without being damaged or destroyed .
 \textit{
	\begin{itemize}
	\item The builders will have to strengthen the existing joists with additional timber.
	\end{itemize}
}
\item verb \\
If the wind , current , or other force  \textbf{strengthens} , it becomes faster or more powerful.
 \textit{
	\begin{itemize}
	\item As it strengthened the wind was veering southerly.
	\item There was a short sharp shower followed by a strengthening breeze.
	\end{itemize}
}
\end{enumerate}

\section*{town}
{\large \color{blue}  towns  }
\subsection*{Explain}
\begin{enumerate}
\item countable noun \\
A \textbf{town} is a place with many streets and buildings, where people live and work. Towns are larger than villages and smaller than cities. Many places that
are called towns in Britain would be called cities in the United  States .
 You can use \textbf{the town} to refer to the people of a town.
 \textit{
	\begin{itemize}
	\item ...Saturday night in the small town of Braintree, Essex.
	\item Parking can be tricky in the town centre.
	\item The town takes immense pride in recent achievements.
	\end{itemize}
}
\item uncountable noun \\
You use \textbf{town} in order to refer to the town where you live.
 \textit{
	\begin{itemize}
	\item He admits he doesn't even know when his brother is in town.
	\item She left town.
	\item ...attractive and fun loving Americans, new to town.
	\end{itemize}
}
\item uncountable noun \\
You use \textbf{town} in order to refer to the central area of a town where most of the shops and offices are.
 \textit{
	\begin{itemize}
	\item I walked around town.
	\item I caught a bus into town.
	\end{itemize}
}
\item singular noun \\
If you refer to \textbf{the town} , you are referring to town and city areas in general , as opposed to country areas.
 \textit{
	\begin{itemize}
	\item More people are going to want to escape from the town into the country.
	\item It had the advantages of town and country combined.
	\end{itemize}
}
\item  \\
 go to town \textit{
	\begin{itemize}
	\end{itemize}
}
\item  \\
 man about town \textit{
	\begin{itemize}
	\end{itemize}
}
\item  \\
 on the town \textit{
	\begin{itemize}
	\end{itemize}
}
\end{enumerate}

\section*{summon}
{\large \color{blue}  summons  summoning  summoned  }
\subsection*{Explain}
\begin{enumerate}
\item verb \\
If you \textbf{summon} someone, you order them to come to you.
 \textit{
	\begin{itemize}
	\item Howe summoned a doctor and hurried over.
	\item Suddenly we were summoned to the interview room.
	\item He has been summoned to appear in court on charges of incitement to law-breaking.
	\end{itemize}
}
\item verb \\
If you \textbf{summon} a quality, you make a great  effort to have it. For example , if you \textbf{summon} the courage or strength to do something, you make a great effort to be brave or strong , so that you will be able to do it.
 \textbf{Summon up} means the same as summon .
 \textit{
	\begin{itemize}
	\item It took her a full month to summon the courage to tell her mother.
	\item Painfully shy, he finally summoned up courage to ask her to a game.
	\item We couldn't even summon up the energy to open the envelope.
	\end{itemize}
}
\end{enumerate}

\section*{training}
{\large \color{blue}  }
\subsection*{Explain}
\begin{enumerate}
\item uncountable noun \\
\textbf{Training} is the process of learning the skills that you need for a particular job or activity.
 \textit{
	\begin{itemize}
	\item He called for much higher spending on education and training.
	\item Kennedy had no formal training as a decorator.
	\item ...a one-day training course.
	\end{itemize}
}
\item uncountable noun \\
\textbf{Training} is physical  exercise that you do regularly in order to keep fit or to prepare for an activity such as a race.
 \textit{
	\begin{itemize}
	\item The emphasis is on developing fitness through exercises and training.
	\item ...her busy training schedule.
	\end{itemize}
}
\end{enumerate}

\section*{tug}
{\large \color{blue}  tugs  tugging  tugged  }
\subsection*{Explain}
\begin{enumerate}
\item verb \\
If you \textbf{tug} something or \textbf{tug}  \textbf{at} it, you give it a quick and usually strong pull.
 \textbf{Tug} is also a noun .
 \textit{
	\begin{itemize}
	\item A little boy came running up and tugged at his sleeve excitedly.
	\item She kicked him, tugging his thick hair.
	\item Bobby gave her hair a tug.
	\item I felt a tug at my sleeve.
	\end{itemize}
}
\item countable noun \\
A \textbf{tug} or a \textbf{tug boat} is a small powerful boat which pulls large ships, usually when they come into a port.
 \textit{
	\begin{itemize}
	\end{itemize}
}
\end{enumerate}

\section*{trial}
{\large \color{blue}  trials  }
\subsection*{Explain}
\begin{enumerate}
\item variable noun \\
A \textbf{trial} is a formal  meeting in a law court, at which a judge and jury  listen to evidence and decide whether a person is guilty of a crime .
 \textit{
	\begin{itemize}
	\item New evidence showed the police lied at the trial.
	\item He's awaiting trial in a military court on charges of plotting against the state.
	\item They believed that his case would never come to trial.
	\end{itemize}
}
\item variable noun \\
A \textbf{trial} is an experiment in which you test something by using it or doing it for a period
of time to see how well it works. If something is \textbf{on trial} , it is being tested in this way.
 \textit{
	\begin{itemize}
	\item They have been treated with this drug in clinical trials.
	\item I took the car out for a trial on the roads.
	\item The robots have been on trial for the past year.
	\item We plan to release a prototype this autumn for trial in hospitals.
	\end{itemize}
}
\item countable noun \\
If someone gives you a \textbf{trial} for a job , or if you are \textbf{on trial} , you do the job for a short period of time to see if you are suitable for it.
 \textit{
	\begin{itemize}
	\item He had just given a trial to a young woman who said she had previous experience.
	\item The 26-year old fullback has been on trial at the club for ten days.
	\end{itemize}
}
\item countable noun \\
If you refer to the \textbf{trials}  \textbf{of} a situation , you mean the unpleasant things that you experience in it.
 \textit{
	\begin{itemize}
	\item ...the trials of adolescence.
	\end{itemize}
}
\item countable noun \\
In some sports or outdoor activities, \textbf{trials} are a series of contests that test a competitor's skill and ability .
 \textit{
	\begin{itemize}
	\item He has been riding in horse trials for less than a year.
	\item ...Dovedale Sheepdog Trials.
	\end{itemize}
}
\item  \\
 trial and error \textit{
	\begin{itemize}
	\end{itemize}
}
\item  \\
 on trial \textit{
	\begin{itemize}
	\end{itemize}
}
\item  \\
 on trial \textit{
	\begin{itemize}
	\end{itemize}
}
\item  \\
 to stand trial \textit{
	\begin{itemize}
	\end{itemize}
}
\end{enumerate}

\section*{write}
{\large \color{blue}  writes  writing  wrote  written  }
\subsection*{Explain}
\begin{enumerate}
\item verb \\
When you \textbf{write} something on a surface, you use something such as a pen or pencil to produce words,
letters, or numbers on the surface.
 \textit{
	\begin{itemize}
	\item If you'd like one, simply write your name and address on a postcard and send it to
us.
	\item They were still trying to teach her to read and write.
	\item He wrote the word 'pride' in huge letters on the blackboard.
	\end{itemize}
}
\item verb \\
If you \textbf{write} something such as a book, a poem , or a piece of music, you create it and record it on paper or perhaps on a computer.
 \textit{
	\begin{itemize}
	\item I had written quite a lot of orchestral music in my student days.
	\item Finding a volunteer to write the computer program isn't a problem.
	\item Thereafter she wrote articles for papers and magazines in Paris.
	\item Jung Lu wrote me a poem once.
	\end{itemize}
}
\item verb \\
Someone who \textbf{writes} creates books, stories , or articles , usually for publication .
 \textit{
	\begin{itemize}
	\item Jay wanted to write.
	\item She writes for many papers, including the Sunday Times.
	\item He now works in industry and writes on science in his spare time.
	\end{itemize}
}
\item verb \\
When you \textbf{write}  \textbf{to} someone or \textbf{write} them an email or a letter, you give them information, ask them something, or express your feelings in an email or letter. In American English, you can also  \textbf{write} someone.
 \textit{
	\begin{itemize}
	\item Many people have written to me on this subject.
	\item She had written him a note a couple of weeks earlier.
	\item I wrote a letter to the car rental agency, explaining what had happened.
	\item Why didn't you write, call, anything?
	\item He had written her in Italy but received no reply.
	\end{itemize}
}
\item verb \\
If someone \textbf{writes} that something is the case , they say it in a letter, book, or article.
 \textit{
	\begin{itemize}
	\item 'Some six months later,' Freud writes, 'Hans had got over his jealousy.'.
	\item A few days later he wrote that he had hopes of a staff job.
	\end{itemize}
}
\item verb \\
When someone \textbf{writes} something such as a receipt or a prescription , they put the necessary information on it and usually sign it.
 \textbf{Write out} means the same as write .
 \textit{
	\begin{itemize}
	\item Snape wrote a receipt with a gold fountain pen.
	\item He wrote me a prescription for an anti-anxiety medication.
	\item We went straight to the estate agent and wrote out a cheque.
	\item Get my assistant to write you out a receipt before you leave.
	\end{itemize}
}
\item verb \\
If you \textbf{write}  \textbf{to} a computer or a disk , you record data on it.
 \textit{
	\begin{itemize}
	\item You should write-protect all disks that you do not usually need to write to.
	\end{itemize}
}
\end{enumerate}

\section*{worship}
{\large \color{blue}  worships  worshipping  worshipped  }
\subsection*{Explain}
\begin{enumerate}
\item verb \\
If you \textbf{worship} a god, you show your respect to the god, for example by saying prayers.
 \textbf{Worship} is also a noun .
 \textit{
	\begin{itemize}
	\item I enjoy going to church and worshipping God.
	\item ...Jews worshipping at the Wailing Wall.
	\item ...the worship of the ancient Roman gods.
	\item St Jude's church is a public place of worship.
	\end{itemize}
}
\item verb \\
If you \textbf{worship} someone or something, you love them or admire them very much.
 \textit{
	\begin{itemize}
	\item She had worshipped him for years.
	\item They worship James Brown, Bob Marley and Jimi Hendrix.
	\end{itemize}
}
\end{enumerate}

\section*{yawn}
{\large \color{blue}  yawns  yawning  yawned  }
\subsection*{Explain}
\begin{enumerate}
\item verb \\
If you \textbf{yawn} , you open your mouth very wide and breathe in more air than usual , often when you are tired or when you are not interested in something.
 \textbf{Yawn} is also a noun .
 \textit{
	\begin{itemize}
	\item She yawned, and stretched lazily.
	\item They looked bored and yawned at the speeches.
	\item Rosanna stifled a huge yawn.
	\end{itemize}
}
\item singular noun \\
If you describe something such as a book or a film as \textbf{a yawn} , you think it is very boring .
 \textit{
	\begin{itemize}
	\item The debate was a mockery. A big yawn.
	\item The concert was a predictable yawn.
	\end{itemize}
}
\item verb \\
A gap or an opening that \textbf{yawns} is large and wide, and often frightening .
 \textit{
	\begin{itemize}
	\item The gulf between them yawned wider than ever.
	\item Liddie's doorway yawned blackly open at the end of the hall.
	\end{itemize}
}
\end{enumerate}

\section*{writing}
{\large \color{blue}  writings  }
\subsection*{Explain}
\begin{enumerate}
\item uncountable noun \\
\textbf{Writing} is something that has been written or printed .
 \textit{
	\begin{itemize}
	\item 'It's from a notebook,' the sheriff said, 'And there's writing on it.'
	\item If you have a complaint about your holiday, please inform us in writing.
	\end{itemize}
}
\item uncountable noun \\
You can refer to any piece of written work as \textbf{writing} , especially when you are considering the style of language used in it.
 \textit{
	\begin{itemize}
	\item The writing is brutally tough and savagely humorous.
	\item It was such a brilliant piece of writing.
	\end{itemize}
}
\item uncountable noun \\
\textbf{Writing} is the activity of writing, especially of writing books for money.
 \textit{
	\begin{itemize}
	\item She had begun to be a little bored with novel writing.
	\item ...activities to help prepare children for writing.
	\end{itemize}
}
\item uncountable noun \\
Your \textbf{writing} is the way that you write with a pen or pencil , which can usually be recognized as belonging to you.
 \textit{
	\begin{itemize}
	\item It was a little difficult to read your writing.
	\item I think it's due to being left handed that he's got terrible writing.
	\end{itemize}
}
\item plural noun \\
An author's \textbf{writings} are all the things that he or she has written, especially on a particular subject.
 \textit{
	\begin{itemize}
	\item Althusser's writings are focused mainly on France.
	\item The pieces he is reading are adapted from the writings of Michael Frayn.
	\end{itemize}
}
\item  \\
 the writing is on the wall \textit{
	\begin{itemize}
	\end{itemize}
}
\end{enumerate}

\section*{yield}
{\large \color{blue}  yields  yielding  yielded  }
\subsection*{Explain}
\begin{enumerate}
\item verb \\
If you \textbf{yield}  \textbf{to} someone or something, you stop  resisting them.
 \textit{
	\begin{itemize}
	\item Will she yield to growing pressure for her to retire?
	\item I yielded to an impulse.
	\item Men of courage faced down injustice and refused to yield.
	\end{itemize}
}
\item verb \\
If you \textbf{yield} something that you have control of or responsibility for, you allow someone else to have control or responsibility for it.
 \textbf{Yield up} means the same as yield .
 \textit{
	\begin{itemize}
	\item He may yield control.
	\item The Director is now under pressure to yield power to the shareholders.
	\item He yielded up the prime ministership last summer.
	\end{itemize}
}
\item verb \\
If one thing \textbf{yields}  \textbf{to} another thing, it is replaced by this other thing.
 \textit{
	\begin{itemize}
	\item Boston's traditional drab brick was slow to yield to the modern glass palaces of
so many American urban areas.
	\end{itemize}
}
\item verb \\
If a moving person or a vehicle \textbf{yields} , they slow down or stop in order to allow other people or vehicles to pass in front of them.
 \textit{
	\begin{itemize}
	\item When entering a trail or starting a descent, yield to other skiers.
	\item ...examples of common signs like No Smoking or Yield.
	\end{itemize}
}
\item verb \\
If something \textbf{yields} , it breaks or moves position because force or pressure has been put on it.
 \textit{
	\begin{itemize}
	\item The door yielded easily when he pushed it.
	\end{itemize}
}
\item verb \\
If an area of land \textbf{yields} a particular amount of a crop , this is the amount that is produced. You can also  say that a number of animals \textbf{yield} a particular amount of meat .
 \textbf{Yield up} means the same as yield .
 \textit{
	\begin{itemize}
	\item Last year 400,000 acres of land yielded a crop worth $1.75 billion.
	\item The shallow sea bed yields up an abundance of food.
	\end{itemize}
}
\item countable noun \\
A \textbf{yield} is the amount of food produced on an area of land or by a number of animals.
 \textit{
	\begin{itemize}
	\item ...improving the yield of the crop.
	\item Polluted water lessens crop yields.
	\end{itemize}
}
\item verb \\
If a tax or investment \textbf{yields} an amount of money or profit, this money or profit is obtained from it.
 \textit{
	\begin{itemize}
	\item It yielded a profit of at least $36 million.
	\end{itemize}
}
\item countable noun \\
A \textbf{yield} is the amount of money or profit produced by an investment.
 \textit{
	\begin{itemize}
	\item ...a yield of 4%.
	\item The high yields available on the dividend shares made them attractive to private
investors.
	\item ...the yield on a bank's investments.
	\end{itemize}
}
\item verb \\
If something \textbf{yields} a result or piece of information, it produces it.
 \textit{
	\begin{itemize}
	\item This research has yielded a great number of positive results.
	\item His trip to Melbourne had yielded a lot of information.
	\end{itemize}
}
\end{enumerate}

\section*{addition}
{\large \color{blue}  additions  }
\subsection*{Explain}
\begin{enumerate}
\item  \\
 in addition \textit{
	\begin{itemize}
	\end{itemize}
}
\item countable noun \\
An \textbf{addition}  \textbf{to} something is a thing which is added to it.
 \textit{
	\begin{itemize}
	\item Most would agree that this particular use of technology is a worthy addition to
the game.
	\item This plywood addition helps to strengthen the structure.
	\end{itemize}
}
\item uncountable noun \\
\textbf{The}  \textbf{addition}  \textbf{of} something is the fact that it is added to something else.
 \textit{
	\begin{itemize}
	\item It was completely refurbished in 1987, with the addition of a picnic site.
	\end{itemize}
}
\item uncountable noun \\
\textbf{Addition} is the process of calculating the total of two or more numbers.
 \textit{
	\begin{itemize}
	\item ...simple addition and subtraction problems.
	\end{itemize}
}
\end{enumerate}

\section*{acclaim}
{\large \color{blue}  acclaims  acclaiming  acclaimed  }
\subsection*{Explain}
\begin{enumerate}
\item verb \\
If someone or something \textbf{is acclaimed} , they are praised enthusiastically.
 \textit{
	\begin{itemize}
	\item She has been acclaimed for her leading roles in both theatre and film.
	\item He was acclaimed as England's greatest modern painter.
	\item The group's debut album was immediately acclaimed a hip hop classic.
	\end{itemize}
}
\item uncountable noun \\
\textbf{Acclaim} is public praise for someone or something.
 \textit{
	\begin{itemize}
	\item She has won critical acclaim for her excellent performance.
	\item All this equipment has received international acclaim from the specialist hi-fi press.
	\end{itemize}
}
\end{enumerate}

\section*{appearance}
{\large \color{blue}  appearances  }
\subsection*{Explain}
\begin{enumerate}
\item countable noun \\
When someone makes an \textbf{appearance} at a public event or in a broadcast , they take part in it.
 \textit{
	\begin{itemize}
	\item It was the president's second public appearance to date.
	\item He makes frequent television appearances.
	\end{itemize}
}
\item singular noun \\
Someone's or something's \textbf{appearance} is the way that they look .
 \textit{
	\begin{itemize}
	\item She used to be so fussy about her appearance.
	\item He had the appearance of a college student.
	\item A flat-roofed extension will add nothing to the value or appearance of the house.
	\end{itemize}
}
\item singular noun \\
The \textbf{appearance}  \textbf{of} someone or something in a place is their arrival there, especially when it is unexpected .
 \textit{
	\begin{itemize}
	\item The sudden appearance of a few bags of rice could start a riot.
	\item ...last Christmas, when there'd been the welcome appearance of Cousin Fred.
	\end{itemize}
}
\item singular noun \\
The \textbf{appearance}  \textbf{of} something new is its coming into existence or use.
 \textit{
	\begin{itemize}
	\item Flowering plants were making their first appearance, but were still a rarity.
	\item Fears are growing of a cholera outbreak following the appearance of a several cases
in the city.
	\end{itemize}
}
\item singular noun \\
If something has the \textbf{appearance}  \textbf{of} a quality, it seems to have that quality.
 \textit{
	\begin{itemize}
	\item We tried to meet both children's needs without the appearance of favoritism or unfairness.
	\item The U.S. president risked giving the appearance that the U.S. was taking sides.
	\end{itemize}
}
\item  \\
 to all appearances/from all appearances/by all appearances \textit{
	\begin{itemize}
	\end{itemize}
}
\item  \\
 keep up appearances \textit{
	\begin{itemize}
	\end{itemize}
}
\item  \\
 put in an appearance \textit{
	\begin{itemize}
	\end{itemize}
}
\end{enumerate}

\section*{beware}
{\large \color{blue}  }
\subsection*{Explain}
\begin{enumerate}
\item verb \\
If you tell someone to \textbf{beware}  \textbf{of} a person or thing, you are warning them that the person or thing may harm them or be dangerous .
 \textit{
	\begin{itemize}
	\item Beware of being too impatient with others.
	\item Motorists were warned to beware of slippery conditions.
	\item Beware, this recipe is not for slimmers.
	\end{itemize}
}
\end{enumerate}

\section*{attendance}
{\large \color{blue}  attendances  }
\subsection*{Explain}
\begin{enumerate}
\item uncountable noun \\
Someone's \textbf{attendance} at an event or an institution is the fact that they are present at the event or go regularly to the institution.
 \textit{
	\begin{itemize}
	\item Her attendance at school was sporadic.
	\end{itemize}
}
\item variable noun \\
The \textbf{attendance} at an event is the number of people who are present at it.
 \textit{
	\begin{itemize}
	\item Average weekly cinema attendance in February was 2.41 million.
	\item This year attendances were 28% lower than forecast.
	\item Some estimates put the attendance at 60,000.
	\end{itemize}
}
\item  \\
 in attendance \textit{
	\begin{itemize}
	\end{itemize}
}
\item  \\
 in attendance \textit{
	\begin{itemize}
	\end{itemize}
}
\end{enumerate}

\section*{birth}
{\large \color{blue}  births  }
\subsection*{Explain}
\begin{enumerate}
\item variable noun \\
When a baby is born, you refer to this event as his or her \textbf{birth} .
 \textit{
	\begin{itemize}
	\item It was the birth of his grandchildren which gave him greatest pleasure.
	\item She concealed her pregnancy right up to the moment of birth.
	\item She weighed 5lb 7oz at birth.
	\item ...premature births.
	\end{itemize}
}
\item uncountable noun \\
You can refer to the beginning or origin of something as its \textbf{birth} .
 \textit{
	\begin{itemize}
	\item ...the birth of popular democracy.
	\end{itemize}
}
\item uncountable noun \\
Some people talk about a person's \textbf{birth} when they are referring to the social position of the person's family.
 \textit{
	\begin{itemize}
	\item ...men of low birth.
	\item His birth, background and career show that you can make it in this country on merit
alone.
	\end{itemize}
}
\item  \\
 by birth \textit{
	\begin{itemize}
	\end{itemize}
}
\item  \\
 give birth \textit{
	\begin{itemize}
	\end{itemize}
}
\item  \\
 give birth to \textit{
	\begin{itemize}
	\end{itemize}
}
\item  \\
 of o's birth \textit{
	\begin{itemize}
	\end{itemize}
}
\end{enumerate}

\section*{consider}
{\large \color{blue}  considers  considering  considered  }
\subsection*{Explain}
\begin{enumerate}
\item verb \\
If you \textbf{consider} a person or thing \textbf{to} be something, you have the opinion that this is what they are.
 \textit{
	\begin{itemize}
	\item We don't consider our customers to be mere consumers; we consider them to be our
friends.
	\item I had always considered myself a strong, competent woman.
	\item I consider activities such as jogging and weightlifting as unnatural.
	\item Barbara considers that pet shops which sell customers these birds are very unfair.
	\end{itemize}
}
\item verb \\
If you \textbf{consider} something, you think about it carefully.
 \textit{
	\begin{itemize}
	\item The jury was asked to consider the credibility of his evidence.
	\item You do have to consider the feelings of those around you.
	\item Consider how much you can afford to pay for a course, and what is your upper limit.
	\end{itemize}
}
\item verb \\
If you \textbf{are considering} doing something, you intend to do it, but have not yet made a final decision whether to do it.
 \textit{
	\begin{itemize}
	\item I had seriously considered telling the story from the point of view of the wives.
	\item Watersports enthusiasts should consider hiring a wetsuit as well as a lifejacket.
	\item They are considering the launch of their own political party.
	\end{itemize}
}
\item  \\
 all things considered \textit{
	\begin{itemize}
	\end{itemize}
}
\end{enumerate}

\section*{border}
{\large \color{blue}  borders  bordering  bordered  }
\subsection*{Explain}
\begin{enumerate}
\item countable noun \\
The \textbf{border} between two countries or regions is the dividing line between them. Sometimes  \textbf{the border}  also  refers to the land  close to this line.
 \textit{
	\begin{itemize}
	\item They fled across the border.
	\item ...the isolated jungle area near the Panamanian border.
	\item Clifford is enjoying life north of the border.
	\item ...the Mexican border town of Tijuana.
	\item Soldiers had temporarily closed the border between the two countries.
	\end{itemize}
}
\item verb \\
A country that \textbf{borders} another country, a sea , or a river is next to it.
 \textbf{Border on}  means the same as border .
 \textit{
	\begin{itemize}
	\item ...the European and Arab countries bordering the Mediterranean.
	\item Both republics border on the Black Sea.
	\end{itemize}
}
\item countable noun \\
A \textbf{border} is a strip or band around the edge of something.
 \textit{
	\begin{itemize}
	\item ...pillowcases trimmed with a hand-crocheted border.
	\end{itemize}
}
\item countable noun \\
In a garden , a \textbf{border} is a long strip of ground along the edge planted with flowers.
 \textit{
	\begin{itemize}
	\item ...a lawn flanked by wide herbaceous borders.
	\item ...border plants.
	\end{itemize}
}
\item verb \\
If something \textbf{is bordered} by another thing, the other thing forms a line along the edge of it.
 \textit{
	\begin{itemize}
	\item ...the mile of white sand beach bordered by palm trees and tropical flowers.
	\item Caesar marched north into the forests that border the Danube River.
	\end{itemize}
}
\end{enumerate}

\section*{create}
{\large \color{blue}  creates  creating  created  }
\subsection*{Explain}
\begin{enumerate}
\item verb \\
To \textbf{create} something means to cause it to happen or exist .
 \textit{
	\begin{itemize}
	\item We set business free to create more jobs in Britain.
	\item She could create a fight out of anything.
	\item The lights create such a glare it's next to impossible to see anything behind them.
	\item Criticizing will only destroy a relationship and create feelings of failure.
	\end{itemize}
}
\item verb \\
When someone \textbf{creates} a new product or process, they invent it or design it.
 \textit{
	\begin{itemize}
	\item It is really great for a radio producer to create a show like this.
	\item He's creating a whole new language of painting.
	\end{itemize}
}
\end{enumerate}

\section*{brim}
{\large \color{blue}  brims  brimming  brimmed  }
\subsection*{Explain}
\begin{enumerate}
\item countable noun \\
The \textbf{brim} of a hat is the wide part that sticks  outwards at the bottom .
 \textit{
	\begin{itemize}
	\item Rain dripped from the brim of his baseball cap.
	\item ...a flat black hat with a wide brim.
	\end{itemize}
}
\item verb \\
If someone or something \textbf{is brimming with} a particular quality, they are full of that quality.
 \textbf{Brim over} means the same as brim .
 \textit{
	\begin{itemize}
	\item England are brimming with confidence after two straight wins in the tournament.
	\item I noticed Dorabella was brimming over with excitement.
	\item Her heart brimmed over with love and adoration for Charles.
	\end{itemize}
}
\item verb \\
When your eyes  \textbf{are brimming with}  tears , they are full of fluid because you are upset , although you are not actually  crying .
 \textbf{Brim over} means the same as brim .
 \textit{
	\begin{itemize}
	\item Michael looked at him imploringly, eyes brimming with tears.
	\item When she saw me, her eyes brimmed over with tears and she could not speak.
	\end{itemize}
}
\item verb \\
If something \textbf{brims}  \textbf{with} particular things, it is packed full of them.
 \textit{
	\begin{itemize}
	\item The flowerbeds brim with a mixture of lilies and roses.
	\end{itemize}
}
\item  \\
 full to the brim \textit{
	\begin{itemize}
	\end{itemize}
}
\item  \\
 full to the brim with something \textit{
	\begin{itemize}
	\end{itemize}
}
\end{enumerate}

\section*{deposit}
{\large \color{blue}  deposits  depositing  deposited  }
\subsection*{Explain}
\begin{enumerate}
\item countable noun \\
A \textbf{deposit} is a sum of money which is part of the full price of something, and which you pay when you
 agree to buy it.
 \textit{
	\begin{itemize}
	\item A £50 deposit is required when ordering, and the balance is due upon delivery.
	\end{itemize}
}
\item countable noun \\
A \textbf{deposit} is a sum of money which you pay when you start  renting something. The money is returned to you if you do not damage what you have rented.
 \textit{
	\begin{itemize}
	\item It is common to ask for the equivalent of a month's rent as a deposit.
	\end{itemize}
}
\item countable noun \\
A \textbf{deposit} is a sum of money which is in a bank account or savings account, especially a sum which will be left there for some time.
 \textit{
	\begin{itemize}
	\end{itemize}
}
\item countable noun \\
A \textbf{deposit} is a sum of money which you have to pay if you want to be a candidate in a parliamentary or European election . The money is returned to you if you receive more than a certain percentage of the votes .
 \textit{
	\begin{itemize}
	\item The Tory candidate lost his deposit.
	\end{itemize}
}
\item countable noun \\
A \textbf{deposit} is an amount of a substance that has been left somewhere as a result of a chemical or geological process.
 \textit{
	\begin{itemize}
	\item The surplus material is washed away and any remaining deposit examined.
	\item ...underground deposits of gold and diamonds.
	\item ...mineral deposits.
	\end{itemize}
}
\item verb \\
To \textbf{deposit} someone or something somewhere means to put them or leave them there.
 \textit{
	\begin{itemize}
	\item Someone was seen depositing a packet.
	\item Fritz deposited a glass and two bottles of beer in front of Wolfe.
	\end{itemize}
}
\item verb \\
If you \textbf{deposit} something somewhere, you put it where it will be safe until it is needed again.
 \textit{
	\begin{itemize}
	\item You are advised to deposit valuables in the hotel safe.
	\end{itemize}
}
\item verb \\
If you \textbf{deposit} a sum of money, you pay it into a bank account or savings account.
 \textit{
	\begin{itemize}
	\item The customer has to deposit a minimum of £100 monthly.
	\end{itemize}
}
\item verb \\
If a substance \textbf{is deposited} somewhere, it is left there as a result of a chemical or geological process.
 \textit{
	\begin{itemize}
	\item The phosphate was deposited by the decay of marine microorganisms.
	\end{itemize}
}
\end{enumerate}

\section*{cab}
{\large \color{blue}  cabs  }
\subsection*{Explain}
\begin{enumerate}
\item countable noun \\
A \textbf{cab} is a taxi.
 \textit{
	\begin{itemize}
	\end{itemize}
}
\item countable noun \\
The \textbf{cab} of a truck or train is the front part in which the driver  sits .
 \textit{
	\begin{itemize}
	\item A Luton van has additional load space over the driver's cab.
	\end{itemize}
}
\end{enumerate}

\section*{depress}
{\large \color{blue}  depresses  depressing  depressed  }
\subsection*{Explain}
\begin{enumerate}
\item verb \\
If someone or something \textbf{depresses} you, they make you feel  sad and disappointed .
 \textit{
	\begin{itemize}
	\item I must admit the state of the country depresses me.
	\item I know he is too optimistic but I don't want to depress him.
	\end{itemize}
}
\item verb \\
If something \textbf{depresses} prices, wages , or figures , it causes them to become less.
 \textit{
	\begin{itemize}
	\item The stronger U.S. dollar depressed sales.
	\end{itemize}
}
\end{enumerate}

\section*{cereal}
{\large \color{blue}  cereals  }
\subsection*{Explain}
\begin{enumerate}
\item variable noun \\
\textbf{Cereal} or \textbf{breakfast cereal} is a food made from grain. It is mixed with milk and eaten for breakfast.
 \textit{
	\begin{itemize}
	\item I have a bowl of cereal every morning.
	\end{itemize}
}
\item countable noun \\
\textbf{Cereals} are plants such as wheat, corn , or rice that produce grain.
 \textit{
	\begin{itemize}
	\item ...the rich cereal-growing districts of the Paris Basin.
	\end{itemize}
}
\end{enumerate}

\section*{develop}
{\large \color{blue}  develops  developing  developed  }
\subsection*{Explain}
\begin{enumerate}
\item verb \\
When something \textbf{develops} , it grows or changes over a period of time and usually becomes more advanced, complete , or severe .
 \textit{
	\begin{itemize}
	\item There have been huge advances in our understanding of how children develop.
	\item It's hard to say at this stage how the market will develop.
	\item These clashes could develop into open warfare.
	\item Society begins to have an impact on the developing child.
	\end{itemize}
}
\item verb \\
If a problem or difficulty  \textbf{develops} , it begins to occur.
 \textit{
	\begin{itemize}
	\item A huge row has developed about the pollution emanating from a chemical plant.
	\item ...blood clots in his lungs, a problem which developed from a leg injury.
	\end{itemize}
}
\item verb \\
If you say that a country \textbf{develops} , you mean that it changes from being a poor agricultural country to being a rich  industrial country.
 \textit{
	\begin{itemize}
	\item All countries, it was predicted, would develop and develop fast.
	\end{itemize}
}
\item verb \\
If you \textbf{develop} a business or industry , or if it \textbf{develops} , it becomes bigger and more successful .
 \textit{
	\begin{itemize}
	\item She won a grant to develop her own business.
	\item Over the last few years tourism here has developed considerably.
	\end{itemize}
}
\item verb \\
To \textbf{develop} land or property means to make it more profitable , by building houses or factories or by improving the existing buildings.
 \textit{
	\begin{itemize}
	\item Entrepreneurs developed fashionable restaurants and bars in the area.
	\item ...the cost of acquiring or developing property.
	\end{itemize}
}
\item verb \\
If you \textbf{develop} a habit , reputation , or belief , you start to have it and it then becomes stronger or more noticeable .
 \textit{
	\begin{itemize}
	\item She later developed a taste for expensive nightclubs.
	\item Mr Robinson has developed the reputation of a ruthless cost-cutter.
	\end{itemize}
}
\item verb \\
If you \textbf{develop} a skill , quality, or relationship , or if it \textbf{develops} , it becomes better or stronger.
 \textit{
	\begin{itemize}
	\item Now you have an opportunity to develop a greater understanding of each other.
	\item ...weekly workshops that are designed to develop acting and theatre skills.
	\item We must develop closer ties with Germany.
	\item Their friendship developed through their shared interest in the Arts.
	\end{itemize}
}
\item verb \\
If you \textbf{develop} an illness, or if it \textbf{develops} , you become affected by it.
 \textit{
	\begin{itemize}
	\item The test should identify which patients are most prone to develop the disease.
	\item A sharp ache developed in her back muscles.
	\end{itemize}
}
\item verb \\
If a piece of equipment  \textbf{develops} a fault , it starts to have the fault.
 \textit{
	\begin{itemize}
	\item The aircraft made an unscheduled landing after developing an electrical fault.
	\end{itemize}
}
\item verb \\
If someone \textbf{develops} a new product , they design it and produce it.
 \textit{
	\begin{itemize}
	\item He claims that several countries have developed nuclear weapons secretly.
	\item ...a computer system specially developed for the Coastguard service.
	\end{itemize}
}
\item verb \\
If you \textbf{develop} an idea , theory , story , or theme, or if it \textbf{develops} , it gradually becomes more detailed, advanced, or complex.
 \textit{
	\begin{itemize}
	\item I would like to thank them for allowing me to develop their original idea.
	\item This point is developed further at the end of this chapter.
	\item The idea of weather forecasting developed incredibly quickly.
	\end{itemize}
}
\item verb \\
To \textbf{develop}  photographs means to make negatives or prints from a photographic film.
 \textit{
	\begin{itemize}
	\item ...after developing one roll of film.
	\end{itemize}
}
\end{enumerate}

\section*{democracy}
{\large \color{blue}  democracies  }
\subsection*{Explain}
\begin{enumerate}
\item uncountable noun \\
\textbf{Democracy} is a system of government in which people choose their rulers by voting for them in elections .
 \textit{
	\begin{itemize}
	\item ...the spread of democracy in Eastern Europe.
	\item ...the pro-democracy movement.
	\end{itemize}
}
\item countable noun \\
A \textbf{democracy} is a country in which the people choose their government by voting for it.
 \textit{
	\begin{itemize}
	\item The new democracies face tough challenges.
	\end{itemize}
}
\item uncountable noun \\
\textbf{Democracy} is a system of running organizations, businesses, and groups in which each member is entitled to vote and take part in decisions .
 \textit{
	\begin{itemize}
	\item ...the union's emphasis on industrial democracy.
	\end{itemize}
}
\end{enumerate}

\section*{diffuse}
{\large \color{blue}  diffuses  diffusing  diffused  }
\subsection*{Explain}
\begin{enumerate}
\item verb \\
If something such as knowledge or information  \textbf{is diffused} , or if it \textbf{diffuses}  somewhere , it is made known over a wide area or to a lot of people.
 \textit{
	\begin{itemize}
	\item Over time, the technology is diffused and adopted by other countries.
	\item ...an attempt to diffuse new ideas.
	\item As agriculture developed, agricultural ideas diffused across Europe.
	\end{itemize}
}
\item verb \\
To \textbf{diffuse} a feeling , especially an undesirable one, means to cause it to weaken and lose its power to affect people.
 \textit{
	\begin{itemize}
	\item The arrival of letters from the president did nothing to diffuse the tension.
	\end{itemize}
}
\item verb \\
If something \textbf{diffuses}  light , it causes the light to spread weakly in different directions.
 \textit{
	\begin{itemize}
	\item Diffusing a light also reduces its power.
	\item The sun slid behind trees, its last light diffused by wintry branches.
	\end{itemize}
}
\item verb \\
To \textbf{diffuse} or \textbf{be diffused} through something means to move and spread through it.
 \textit{
	\begin{itemize}
	\item It allows nicotine to diffuse slowly and steadily into the bloodstream.
	\item It created a glowing centre of warmth that quickly diffused through my limbs.
	\item Speakers that diffuse music into each room are hidden in the ceiling.
	\end{itemize}
}
\item adjective \\
Something that is \textbf{diffuse} is not directed towards one place or concentrated in one place but spread out over a large area.
 \textit{
	\begin{itemize}
	\item ...a diffuse community.
	\item A cold, diffuse light filtered in through the skylight.
	\end{itemize}
}
\item adjective \\
If you describe something as \textbf{diffuse} , you mean that it is vague and difficult to understand or explain .
 \textit{
	\begin{itemize}
	\item His writing is diffuse and it is difficult to make out what he is trying to say.
	\end{itemize}
}
\end{enumerate}

\section*{detector}
{\large \color{blue}  detectors  }
\subsection*{Explain}
\begin{enumerate}
\item countable noun \\
A \textbf{detector} is an instrument which is used to discover that something is present somewhere , or to measure how much of something there is.
 \textit{
	\begin{itemize}
	\item ...a metal detector.
	\item ...fire alarms and smoke detectors.
	\end{itemize}
}
\end{enumerate}

\section*{dine}
{\large \color{blue}  dines  dining  dined  }
\subsection*{Explain}
\begin{enumerate}
\item verb \\
When you \textbf{dine} , you have dinner.
 \textit{
	\begin{itemize}
	\item He dines alone most nights.
	\item That night the two men dined at Wilson's club.
	\item They used to enjoy going out to dine.
	\end{itemize}
}
\end{enumerate}

\section*{discovery}
{\large \color{blue}  discoveries  }
\subsection*{Explain}
\begin{enumerate}
\item variable noun \\
If someone makes a \textbf{discovery} , they become aware of something that they did not know about before.
 \textit{
	\begin{itemize}
	\item I felt I'd made an incredible discovery.
	\item ...the discovery that both his wife and son are HIV positive.
	\end{itemize}
}
\item variable noun \\
If someone makes a \textbf{discovery} , they are the first person to find or become aware of a place, substance, or scientific  fact that no one knew about before.
 \textit{
	\begin{itemize}
	\item In that year, two momentous discoveries were made.
	\item ...the discovery of the ozone hole over the South Pole.
	\item The fascination of discovery has never left him.
	\end{itemize}
}
\item variable noun \\
If someone makes a \textbf{discovery} , they recognize that an actor , musician , or other performer who is not well-known has talent .
 \textit{
	\begin{itemize}
	\item His job is the discovery and promotion of new artists.
	\end{itemize}
}
\item variable noun \\
When the \textbf{discovery} of people or objects happens , someone finds them, either by accident or as a result of looking for them.
 \textit{
	\begin{itemize}
	\item ...the discovery and destruction by soldiers of millions of marijuana plants.
	\end{itemize}
}
\end{enumerate}

\section*{exaggerate}
{\large \color{blue}  exaggerates  exaggerating  exaggerated  }
\subsection*{Explain}
\begin{enumerate}
\item verb \\
If you \textbf{exaggerate} , you indicate that something is, for example , worse or more important than it really is.
 \textit{
	\begin{itemize}
	\item He thinks I'm exaggerating.
	\item Don't exaggerate.
	\item Sheila admitted that she did sometimes exaggerate the demands of her job.
	\end{itemize}
}
\item verb \\
If something \textbf{exaggerates} a situation , quality, or feature , it makes the situation, quality, or feature appear greater, more obvious , or more important than it really is.
 \textit{
	\begin{itemize}
	\item These figures exaggerate the loss of competitiveness.
	\end{itemize}
}
\end{enumerate}

\section*{edge}
{\large \color{blue}  edges  edging  edged  }
\subsection*{Explain}
\begin{enumerate}
\item countable noun \\
The \textbf{edge} of something is the place or line where it stops , or the part of it that is furthest from the middle.
 \textit{
	\begin{itemize}
	\item We were on a hill, right on the edge of town.
	\item She was standing at the water's edge.
	\item Daniel stepped in front of her desk and sat down on its edge.
	\end{itemize}
}
\item countable noun \\
The \textbf{edge} of something sharp such as a knife or an axe is its sharp or narrow side.
 \textit{
	\begin{itemize}
	\item ...the sharp edge of the sword.
	\end{itemize}
}
\item verb \\
If someone or something \textbf{edges}  somewhere , they move very slowly in that direction.
 \textit{
	\begin{itemize}
	\item He edged closer to the phone, ready to grab it.
	\item He is edging ahead in the opinion polls.
	\end{itemize}
}
\item singular noun \\
\textbf{The}  \textbf{edge}  \textbf{of} something, especially something bad , is the point at which it may start to happen .
 \textit{
	\begin{itemize}
	\item They have driven the rhino to the edge of extinction.
	\item She was on the edge of tears.
	\end{itemize}
}
\item singular noun \\
If someone or something has an \textbf{edge} , they have an advantage that makes them stronger or more likely to be successful than another thing or person.
 \textit{
	\begin{itemize}
	\item The three days France have to prepare could give them the edge over England.
	\item Through superior production techniques they were able to gain the competitive edge.
	\end{itemize}
}
\item singular noun \\
If you say that someone or something has \textbf{an edge} , you mean that they have a powerful quality.
 \textit{
	\begin{itemize}
	\item Featuring new bands gives the show an edge.
	\item Greene's stories had an edge of realism.
	\end{itemize}
}
\item singular noun \\
If someone's voice has an \textbf{edge}  \textbf{to} it, it has a sharp, bitter , or emotional quality.
 \textit{
	\begin{itemize}
	\item But underneath the humour is an edge of bitterness.
	\item There was a nervous edge to his voice.
	\end{itemize}
}
\item  \\
 on edge \textit{
	\begin{itemize}
	\end{itemize}
}
\item  \\
 on the edge of one's seat \textit{
	\begin{itemize}
	\end{itemize}
}
\item  \\
 rough edges \textit{
	\begin{itemize}
	\end{itemize}
}
\item  \\
 take the edge off \textit{
	\begin{itemize}
	\end{itemize}
}
\end{enumerate}

\section*{exist}
{\large \color{blue}  exists  existing  existed  }
\subsection*{Explain}
\begin{enumerate}
\item verb \\
If something \textbf{exists} , it is present in the world as a real thing.
 \textit{
	\begin{itemize}
	\item He thought that if he couldn't see something, it didn't exist.
	\item Research opportunities exist in a wide range of pure and applied areas of entomology.
	\end{itemize}
}
\item verb \\
To \textbf{exist} means to live, especially under difficult conditions or with very little food or money.
 \textit{
	\begin{itemize}
	\item I exist from one visit to the next.
	\item I was barely existing.
	\item ...the problems of having to exist on unemployment benefit.
	\end{itemize}
}
\end{enumerate}

\section*{engine}
{\large \color{blue}  engines  }
\subsection*{Explain}
\begin{enumerate}
\item countable noun \\
The \textbf{engine} of a car or other vehicle is the part that produces the power which makes the vehicle move.
 \textit{
	\begin{itemize}
	\item He got into the driving seat and started the engine.
	\item ...an engine failure that forced a jetliner to crash-land in a field.
	\end{itemize}
}
\item countable noun \\
An \textbf{engine} is also the large vehicle that pulls a railway train .
 \textit{
	\begin{itemize}
	\item In 1941, the train would have been pulled by a steam engine.
	\end{itemize}
}
\end{enumerate}

\section*{extend}
{\large \color{blue}  extends  extending  extended  }
\subsection*{Explain}
\begin{enumerate}
\item verb \\
If you say that something, usually something large, \textbf{extends}  \textbf{for} a particular distance or \textbf{extends}  \textbf{from} one place \textbf{to} another, you are indicating its size or position.
 \textit{
	\begin{itemize}
	\item The caves extend for some 18 kilometres.
	\item The main stem will extend to around 12ft, if left to develop naturally.
	\item Our personal space extends about 12 to 18 inches around us.
	\item Diyala extends from the suburbs of Baghdad to the Iranian border.
	\item The new territory would extend over one-fifth of Canada's land mass.
	\end{itemize}
}
\item verb \\
If an object \textbf{extends from} a surface or place, it sticks out from it.
 \textit{
	\begin{itemize}
	\item A shelf of land extended from the escarpment.
	\end{itemize}
}
\item verb \\
If an event or activity \textbf{extends}  \textbf{over} a period of time, it continues for that time.
 \textit{
	\begin{itemize}
	\item ...a playing career in first-class cricket that extended from 1894 to 1920.
	\item The courses are based on a weekly two-hour class, extending over a period of 25 weeks.
	\end{itemize}
}
\item verb \\
If something \textbf{extends}  \textbf{to} a group of people, things, or activities, it includes or affects them.
 \textit{
	\begin{itemize}
	\item The service also extends to wrapping and delivering gifts.
	\item The talks will extend to the church, human rights groups and other social organizations.
	\item His influence extends beyond the TV viewing audience.
	\end{itemize}
}
\item verb \\
If you \textbf{extend} something, you make it longer or bigger .
 \textit{
	\begin{itemize}
	\item This year they have introduced three new products to extend their range.
	\item The building was extended in 1500.
	\item ...an extended exhaust pipe.
	\end{itemize}
}
\item verb \\
If a piece of equipment or furniture  \textbf{extends} , its length can be increased.
 \textit{
	\begin{itemize}
	\item ... a table which extends to accommodate extra guests.
	\item The table extends to 220cm.
	\end{itemize}
}
\item verb \\
If you \textbf{extend} something, you make it last longer than before or end at a later date .
 \textit{
	\begin{itemize}
	\item They have extended the deadline by twenty-four hours.
	\item ...an extended contract.
	\end{itemize}
}
\item verb \\
If you \textbf{extend} something \textbf{to} other people or things, you make it include or affect more people or things.
 \textit{
	\begin{itemize}
	\item It might be possible to extend the technique to other crop plants.
	\end{itemize}
}
\item verb \\
If someone \textbf{extends} their hand , they stretch out their arm and hand to shake  hands with someone.
 \textit{
	\begin{itemize}
	\item The man extended his hand: 'I'm Chuck'.
	\end{itemize}
}
\end{enumerate}

\section*{eternal}
{\large \color{blue}  }
\subsection*{Explain}
\begin{enumerate}
\item adjective \\
Something that is \textbf{eternal}  lasts for ever.
 \textit{
	\begin{itemize}
	\item Whoever believes in Him shall have eternal life.
	\item ...the quest for eternal youth.
	\end{itemize}
}
\item adjective \\
If you describe something as \textbf{eternal} , you mean that it seems to last for ever, often because you think it is boring or annoying .
 \textit{
	\begin{itemize}
	\item In the background was that eternal hum.
	\end{itemize}
}
\item adjective \\
\textbf{Eternal}  truths , values, and questions  never change and are believed to be always true and to be relevant in all situations .
 \textit{
	\begin{itemize}
	\item ...the eternal truths of the Indian subcontinent.
	\item ...the eternal question.
	\end{itemize}
}
\end{enumerate}

\section*{glow}
{\large \color{blue}  glows  glowing  glowed  }
\subsection*{Explain}
\begin{enumerate}
\item countable noun \\
A \textbf{glow} is a dull , steady light, for example the light produced by a fire when there are no flames.
 \textit{
	\begin{itemize}
	\item ...the cigarette's red glow.
	\item The rising sun casts a golden glow over the fields.
	\end{itemize}
}
\item singular noun \\
A \textbf{glow} is a pink colour on a person's face , usually because they are healthy or have been exercising .
 \textit{
	\begin{itemize}
	\item The moisturiser gave my face a healthy glow that lasted all day.
	\end{itemize}
}
\item singular noun \\
If you feel a \textbf{glow}  \textbf{of} satisfaction or achievement , you have a strong feeling of pleasure because of something that you have done or that has happened .
 \textit{
	\begin{itemize}
	\item Exercise will give you a glow of satisfaction at having achieved something.
	\item He felt a glow of pride in what she had accomplished.
	\end{itemize}
}
\item verb \\
If something \textbf{glows} , it produces a dull, steady light.
 \textit{
	\begin{itemize}
	\item The night lantern glowed softly in the darkness.
	\item Even the mantel above the fire glowed white.
	\end{itemize}
}
\item verb \\
If a place \textbf{glows}  \textbf{with} a colour or a quality, it is bright, attractive , and colourful .
 \textit{
	\begin{itemize}
	\item Used together these colours will make your interiors glow with warmth and vitality.
	\item ...carved wood bathed in glowing colors and gold leaf.
	\end{itemize}
}
\item verb \\
If something \textbf{glows} , it looks bright because it is reflecting light.
 \textit{
	\begin{itemize}
	\item The instruments glowed in the bright orange light.
	\item The fall foliage glowed red and yellow in the morning sunlight.
	\end{itemize}
}
\item verb \\
If someone's skin  \textbf{glows} , it looks pink because they are healthy or excited , or have been doing physical exercise.
 \textit{
	\begin{itemize}
	\item Her freckled skin glowed with health again.
	\item ...a glowing complexion.
	\end{itemize}
}
\item verb \\
If someone \textbf{glows}  \textbf{with} an emotion such as pride or pleasure, the expression on their face shows how they feel.
 \textit{
	\begin{itemize}
	\item The expectant mothers that Amy had encountered positively glowed with pride.
	\end{itemize}
}
\end{enumerate}

\section*{exit}
{\large \color{blue}  exits  exiting  exited  }
\subsection*{Explain}
\begin{enumerate}
\item countable noun \\
The \textbf{exit} is the door through which you can leave a public  building .
 \textit{
	\begin{itemize}
	\item He picked up the case and walked towards the exit.
	\item There's a fire exit by the downstairs ladies room.
	\end{itemize}
}
\item countable noun \\
An \textbf{exit} on a motorway or highway is a place where traffic can leave it.
 \textit{
	\begin{itemize}
	\item Take the A422 exit at Old Stratford.
	\end{itemize}
}
\item countable noun \\
If you refer to someone's \textbf{exit} , you are referring to the way that they left a room or building, or the fact that they left it.
 \textit{
	\begin{itemize}
	\item I made a hasty exit and managed to open the gate.
	\end{itemize}
}
\item countable noun \\
If you refer to someone's \textbf{exit} , you are referring to the way that they left a situation or activity , or the fact that they left it.
 \textit{
	\begin{itemize}
	\item ...after England's exit from the European Championship.
	\item They suggested that she make a dignified exit in the interest of the party.
	\end{itemize}
}
\item verb \\
If you \textbf{exit} from a room or building, you leave it.
 \textit{
	\begin{itemize}
	\item She exits into the tropical storm.
	\item As I exited the final display, I entered a hexagonal room.
	\item She walked into the front door of a store and exited from the rear.
	\end{itemize}
}
\item verb \\
If you \textbf{exit} a computer program or system, you stop  running it.
 \textbf{Exit} is also a noun .
 \textit{
	\begin{itemize}
	\item I can open other applications without having to exit WordPerfect.
	\item Press Exit to return to your document.
	\end{itemize}
}
\end{enumerate}

\section*{grant}
{\large \color{blue}  grants  granting  granted  }
\subsection*{Explain}
\begin{enumerate}
\item countable noun \\
A \textbf{grant} is an amount of money that a government or other institution gives to an individual or to an organization for a particular purpose such as education or home  improvements .
 \textit{
	\begin{itemize}
	\item They'd got a special grant to encourage research.
	\item Unfortunately, my application for a grant was rejected.
	\end{itemize}
}
\item verb \\
If someone in authority \textbf{grants} you something, or if something \textbf{is granted}  \textbf{to} you, you are allowed to have it.
 \textit{
	\begin{itemize}
	\item France has agreed to grant him political asylum.
	\item It was a Labour government which granted independence to India and Pakistan.
	\item Permission was granted a few weeks ago.
	\end{itemize}
}
\item verb \\
If you \textbf{grant}  \textbf{that} something is true , you accept that it is true, even though your opinion about it does not change.
 \textit{
	\begin{itemize}
	\item The magistrates granted that the charity was justified in bringing the action.
	\end{itemize}
}
\item  \\
 take someone for granted \textit{
	\begin{itemize}
	\end{itemize}
}
\item  \\
 take something for granted \textit{
	\begin{itemize}
	\end{itemize}
}
\item  \\
 take it for granted \textit{
	\begin{itemize}
	\end{itemize}
}
\end{enumerate}

\section*{finding}
{\large \color{blue}  findings  }
\subsection*{Explain}
\begin{enumerate}
\item countable noun \\
Someone's \textbf{findings} are the information they get or the conclusions they come to as the result of an investigation or some research .
 \textit{
	\begin{itemize}
	\item One of the main findings of the survey was the confusion about the facilities already
in place.
	\item Manufacturers should take note of the findings and improve their products accordingly.
	\end{itemize}
}
\item countable noun \\
The \textbf{findings} of a court are the decisions that it reaches after a trial or an investigation.
 \textit{
	\begin{itemize}
	\item The government hopes the court will announce its findings before the end of the month.
	\end{itemize}
}
\end{enumerate}

\section*{have}
{\large \color{blue}  has  having  had  }
\subsection*{Explain}
\begin{enumerate}
\item auxiliary verb \\
You use the forms \textbf{have} and \textbf{has} with a past  participle to form the present perfect tense of verbs .
 \textit{
	\begin{itemize}
	\item Alex has already gone.
	\item I've just seen a play that I can highly recommend.
	\item My term hasn't finished yet.
	\item What have you found so far?
	\item This is something which you might have forgotten.
	\item Frankie hasn't been feeling well for a long time.
	\end{itemize}
}
\item auxiliary verb \\
You use the form \textbf{had} with a past participle to form the past perfect tense of verbs.
 \textit{
	\begin{itemize}
	\item When I met her, she had just returned from a job interview.
	\item By Friday at 5:30 p.m., I still hadn't heard from Lund.
	\item Miss Windham said she had spoken to them over the weekend.
	\end{itemize}
}
\item auxiliary verb \\
\textbf{Have} is used in question  tags .
 \textit{
	\begin{itemize}
	\item You haven't sent her away, have you?
	\item It's happened, hasn't it?
	\item They hadn't invented sequencers back in those days, had they?
	\end{itemize}
}
\item auxiliary verb \\
You use \textbf{have} when you are confirming or contradicting a statement containing 'have', 'has', or 'had', or answering a question.
 \textit{
	\begin{itemize}
	\item 'Have you been to York before?'—'Yes we have.'
	\end{itemize}
}
\item auxiliary verb \\
The form \textbf{having} with a past participle can be used to introduce a clause in which you mention an action which had already  happened before another action began .
 \textit{
	\begin{itemize}
	\item He arrived in San Francisco, having left New Jersey on January 19th.
	\item Having been told by his doctor that he was overweight, he's eating all the fibre
and fruit he can.
	\end{itemize}
}
\end{enumerate}

\section*{fringe}
{\large \color{blue}  fringes  }
\subsection*{Explain}
\begin{enumerate}
\item countable noun \\
A \textbf{fringe} is hair which is cut so that it hangs over your forehead.
 \textit{
	\begin{itemize}
	\end{itemize}
}
\item countable noun \\
A \textbf{fringe} is a decoration  attached to clothes , or other objects such as curtains , consisting of a row of hanging strips or threads.
 \textit{
	\begin{itemize}
	\item The jacket had leather fringes.
	\end{itemize}
}
\item countable noun \\
To be \textbf{on the}  \textbf{fringe} or \textbf{the}  \textbf{fringes}  \textbf{of} a place means to be on the outside edge of it, or to be in one of the parts that are farthest from its centre .
 \textit{
	\begin{itemize}
	\item ...black townships located on the fringes of the city.
	\item They lived together in a mixed household on the fringe of a campus.
	\end{itemize}
}
\item countable noun \\
\textbf{The}  \textbf{fringe} or \textbf{the}  \textbf{fringes}  \textbf{of} an activity or organization are its less important , least typical , or most extreme parts, rather than its main and central part.
 \textit{
	\begin{itemize}
	\item The party remained on the fringe of the political scene until last year.
	\end{itemize}
}
\item adjective \\
\textbf{Fringe} groups or events are less important or popular than other related groups or events.
 \textit{
	\begin{itemize}
	\item The monarchists are a small fringe group who quarrel fiercely among themselves.
	\item ...the numerous fringe meetings held during the party conference.
	\end{itemize}
}
\end{enumerate}

\section*{incorporate}
{\large \color{blue}  incorporates  incorporating  incorporated  }
\subsection*{Explain}
\begin{enumerate}
\item verb \\
If one thing \textbf{incorporates} another thing, it includes the other thing.
 \textit{
	\begin{itemize}
	\item The new cars will incorporate a number of major improvements.
	\end{itemize}
}
\item verb \\
If someone or something \textbf{is incorporated}  \textbf{into} a large group, system, or area, they become a part of it.
 \textit{
	\begin{itemize}
	\item The agreement would allow the rebels to be incorporated into a new national police
force.
	\item The party vowed to incorporate environmental considerations into all its policies.
	\end{itemize}
}
\end{enumerate}

\section*{gallon}
{\large \color{blue}  gallons  }
\subsection*{Explain}
\begin{enumerate}
\item countable noun \\
A \textbf{gallon} is a unit of measurement for liquids that is equal to eight  pints . In Britain , it is equal to 4.564 litres. In America , it is equal to 3.785 litres.
 \textit{
	\begin{itemize}
	\item ...80 million gallons of water a day.
	\item ...a gasoline tax of 4.3 cents a gallon.
	\end{itemize}
}
\end{enumerate}

\section*{inhale}
{\large \color{blue}  inhales  inhaling  inhaled  }
\subsection*{Explain}
\begin{enumerate}
\item verb \\
When you \textbf{inhale} , you breathe in. When you \textbf{inhale} something such as smoke , you take it into your lungs when you breathe in.
 \textit{
	\begin{itemize}
	\item He took a long slow breath, inhaling deeply.
	\item He was treated for the effects of inhaling smoke.
	\end{itemize}
}
\end{enumerate}

\section*{generator}
{\large \color{blue}  generators  }
\subsection*{Explain}
\begin{enumerate}
\item countable noun \\
A \textbf{generator} is a machine which produces electricity .
 \textit{
	\begin{itemize}
	\end{itemize}
}
\item countable noun \\
A \textbf{generator}  \textbf{of} something is a person, organization , product , or situation which produces it or causes it to happen .
 \textit{
	\begin{itemize}
	\item The U.S. economy is still an impressive generator of new jobs.
	\item The company has been a very good cash generator.
	\end{itemize}
}
\end{enumerate}

\section*{invent}
{\large \color{blue}  invents  inventing  invented  }
\subsection*{Explain}
\begin{enumerate}
\item verb \\
If you \textbf{invent} something such as a machine or process, you are the first person to think of it or make it.
 \textit{
	\begin{itemize}
	\item He invented the first electric clock.
	\item Writing had not been invented as yet.
	\end{itemize}
}
\item verb \\
If you \textbf{invent} a story or excuse , you try to make other people believe that it is true when in fact it is not.
 \textit{
	\begin{itemize}
	\item I stood still, trying to invent a plausible excuse.
	\end{itemize}
}
\end{enumerate}

\section*{launch}
{\large \color{blue}  launches  launching  launched  }
\subsection*{Explain}
\begin{enumerate}
\item verb \\
To \textbf{launch} a rocket , missile, or satellite means to send it into the air or into space.
 \textbf{Launch} is also a noun .
 \textit{
	\begin{itemize}
	\item NASA plans to launch a satellite to study cosmic rays.
	\item The rocket was launched early this morning.
	\item This morning's launch of the space shuttle has been delayed.
	\end{itemize}
}
\item verb \\
To \textbf{launch} a ship or a boat means to put it into water, often for the first time after it has
been built.
 \textbf{Launch} is also a noun.
 \textit{
	\begin{itemize}
	\item There was no time to launch the lifeboats because the ferry capsized with such alarming
speed.
	\item The launch of a ship was a big occasion.
	\end{itemize}
}
\item verb \\
To \textbf{launch} a large and important activity, for example a military attack , means to start it.
 \textbf{Launch} is also a noun.
 \textit{
	\begin{itemize}
	\item Heavy fighting has been going on after the guerrillas had launched their offensive.
	\item The police have launched an investigation into the incident.
	\item The President was on holiday when the coup was launched.
	\item ...the launch of a campaign to restore law and order.
	\end{itemize}
}
\item verb \\
If a company \textbf{launches} a new product, it makes it available to the public.
 \textbf{Launch} is also a noun.
 \textit{
	\begin{itemize}
	\item Crabtree & Evelyn has just launched a new jam, Worcesterberry Preserve.
	\item The company recently hired her as a model to launch its new range.
	\item The company's spending has also risen following the launch of a new Sunday magazine.
	\end{itemize}
}
\item countable noun \\
A \textbf{launch} is a large motorboat that is used for carrying people on rivers and lakes and in harbours .
 \textit{
	\begin{itemize}
	\item The captain was on the deck of the launch, steadying the boat for the pilot.
	\item We'll make a trip by launch to White Island.
	\end{itemize}
}
\end{enumerate}

\section*{invention}
{\large \color{blue}  inventions  }
\subsection*{Explain}
\begin{enumerate}
\item countable noun \\
An \textbf{invention} is a machine, device, or system that has been invented by someone.
 \textit{
	\begin{itemize}
	\item It's been a tricky business marketing his new invention.
	\item The spinning wheel was a Chinese invention.
	\end{itemize}
}
\item uncountable noun \\
\textbf{Invention} is the act of inventing something that has never been made or used before.
 \textit{
	\begin{itemize}
	\item In 1839, the invention of photography was announced to the world.
	\item ...the invention of the telephone.
	\end{itemize}
}
\item variable noun \\
If you refer to someone's account of something as an \textbf{invention} , you think that it is untrue and that they have made it up.
 \textit{
	\begin{itemize}
	\item In these and several other respects, there are many inventions and exaggerations.
	\item The story was certainly a favourite one, but it was undoubtedly pure invention.
	\end{itemize}
}
\item uncountable noun \\
\textbf{Invention} is the ability to invent things or to have clever and original  ideas .
 \textit{
	\begin{itemize}
	\item ...his great powers of invention.
	\end{itemize}
}
\end{enumerate}

\section*{lose}
{\large \color{blue}  loses  losing  lost  }
\subsection*{Explain}
\begin{enumerate}
\item verb \\
If you \textbf{lose} a contest, a fight , or an argument , you do not succeed because someone does better than you and defeats you.
 \textit{
	\begin{itemize}
	\item They lost the Italian Cup Final.
	\item The government lost the argument over the pace of reform.
	\item The Vietnam conflict ultimately was lost.
	\item No one likes to be on the losing side.
	\end{itemize}
}
\item verb \\
If you \textbf{lose} something, you do not know where it is, for example because you have forgotten where you put it.
 \textit{
	\begin{itemize}
	\item I lost my keys.
	\item I had to go back for my checkup; they'd lost my X-rays.
	\end{itemize}
}
\item verb \\
You say that you \textbf{lose} something when you no longer have it because it has been taken away from you or destroyed .
 \textit{
	\begin{itemize}
	\item I lost my job when the company moved to another state.
	\item He lost his licence for six months.
	\item She was terrified they'd lose their home.
	\end{itemize}
}
\item verb \\
If someone \textbf{loses} a quality, characteristic, attitude , or belief, they no longer have it.
 \textit{
	\begin{itemize}
	\item He lost all sense of reason.
	\item The government had lost all credibility.
	\item He had lost his desire to live.
	\end{itemize}
}
\item verb \\
If you \textbf{lose} an ability, you stop having that ability because of something such as an accident.
 \textit{
	\begin{itemize}
	\item They lost their ability to hear.
	\item He had lost the use of his legs.
	\end{itemize}
}
\item verb \\
If someone or something \textbf{loses} heat, their temperature becomes lower.
 \textit{
	\begin{itemize}
	\item Babies lose heat much faster than adults.
	\item A lot of body heat is lost through the scalp.
	\end{itemize}
}
\item verb \\
If you \textbf{lose} blood or fluid from your body, it leaves your body so that you have less of it.
 \textit{
	\begin{itemize}
	\item The victim suffered a dreadful injury and lost a lot of blood.
	\item During fever a large quantity of fluid is lost in perspiration.
	\end{itemize}
}
\item verb \\
If you \textbf{lose} weight, you become less heavy, and usually look  thinner .
 \textit{
	\begin{itemize}
	\item I have lost a lot of weight.
	\item Martha was able to lose 25 pounds.
	\end{itemize}
}
\item verb \\
If you \textbf{lose} a part of your body, it is cut off in an operation or in an accident.
 \textit{
	\begin{itemize}
	\item He lost a foot when he was struck by a train.
	\end{itemize}
}
\item verb \\
If someone \textbf{loses} their life, they die.
 \textit{
	\begin{itemize}
	\item ...the ferry disaster in 1987, in which 192 people lost their lives.
	\item Hundreds of lives were lost in fighting.
	\end{itemize}
}
\item verb \\
If you \textbf{lose} a close relative or friend , they die.
 \textit{
	\begin{itemize}
	\item My Grandma lost her brother in the war.
	\end{itemize}
}
\item verb \\
If things \textbf{are lost} , they are destroyed in a disaster .
 \textit{
	\begin{itemize}
	\item ...the famous Nankin pottery that was lost in a shipwreck off the coast of China.
	\end{itemize}
}
\item verb \\
If you \textbf{lose} time, something slows you down so that you do not make as much progress as you hoped .
 \textit{
	\begin{itemize}
	\item They claim that police lost valuable time in the early part of the investigation.
	\item Six hours were lost in all.
	\end{itemize}
}
\item verb \\
If you \textbf{lose} an opportunity , you do not take advantage of it.
 \textit{
	\begin{itemize}
	\item If you don't do it soon you're going to lose the opportunity.
	\item They did not lose the opportunity to say what they thought of events.
	\item ...a lost opportunity.
	\end{itemize}
}
\item verb \\
If you \textbf{lose}  \textbf{yourself in} something or if you \textbf{are lost}  \textbf{in} it, you give a lot of attention to it and do not think about anything else.
 \textit{
	\begin{itemize}
	\item Michael held on to her arm, losing himself in the music.
	\item He was lost in the contemplation of the landscape.
	\end{itemize}
}
\item verb \\
If a business \textbf{loses} money, it earns less money than it spends , and is therefore in debt .
 \textit{
	\begin{itemize}
	\item His shops stand to lose millions of pounds.
	\item $1 billion a year may be lost.
	\end{itemize}
}
\item verb \\
If something \textbf{loses} you a contest or \textbf{loses} you something that you had, it causes you to fail or to no longer have what you had.
 \textit{
	\begin{itemize}
	\item My own stupidity lost me the match.
	\item His economic mismanagement has lost him the support of the general public.
	\end{itemize}
}
\item  \\
 have nothing to lose/much to lose \textit{
	\begin{itemize}
	\end{itemize}
}
\item  \\
 lose it \textit{
	\begin{itemize}
	\end{itemize}
}
\item  \\
 lose it \textit{
	\begin{itemize}
	\end{itemize}
}
\item  \\
 lose no opportunity \textit{
	\begin{itemize}
	\end{itemize}
}
\item  \\
 lose no time \textit{
	\begin{itemize}
	\end{itemize}
}
\item  \\
 lose one's way \textit{
	\begin{itemize}
	\end{itemize}
}
\item  \\
 lose one's way \textit{
	\begin{itemize}
	\end{itemize}
}
\end{enumerate}

\section*{leather}
{\large \color{blue}  leathers  }
\subsection*{Explain}
\begin{enumerate}
\item variable noun \\
\textbf{Leather} is treated animal skin which is used for making shoes , clothes, bags , and furniture .
 \textit{
	\begin{itemize}
	\item He wore a leather jacket and dark trousers.
	\item ...an impressive range of upholstered furniture, in a choice of fabrics and leathers.
	\end{itemize}
}
\item plural noun \\
\textbf{Leathers} are leather clothes such as jackets and trousers , especially those worn by motorcyclists.
 \textit{
	\begin{itemize}
	\item ...a couple of youths in motorcyclists' black leathers.
	\end{itemize}
}
\end{enumerate}

\section*{migrate}
{\large \color{blue}  migrates  migrating  migrated  }
\subsection*{Explain}
\begin{enumerate}
\item verb \\
If people \textbf{migrate} , they move from one place to another, especially in order to find work or to live  somewhere for a short time.
 \textit{
	\begin{itemize}
	\item People migrate to cities like Jakarta in search of work.
	\item Farmers have learned that they have to migrate if they want to survive.
	\end{itemize}
}
\item verb \\
When birds, fish , or animals \textbf{migrate} , they move at a particular season from one part of the world or from one part of a country to another, usually in order
to breed or to find new feeding  grounds .
 \textit{
	\begin{itemize}
	\item Most birds have to fly long distances to migrate.
	\item ...a dam system that kills the fish as they migrate from streams to the ocean.
	\end{itemize}
}
\end{enumerate}

\section*{motor}
{\large \color{blue}  motors  motoring  motored  }
\subsection*{Explain}
\begin{enumerate}
\item countable noun \\
The \textbf{motor} in a machine, vehicle, or boat is the part that uses electricity or fuel to produce movement, so that the machine, vehicle, or boat can work.
 \textit{
	\begin{itemize}
	\item She got in and started the motor.
	\end{itemize}
}
\item adjective \\
\textbf{Motor} vehicles and boats have a petrol or diesel engine.
 \textit{
	\begin{itemize}
	\item Theft of motor vehicles is up by 15.9%.
	\end{itemize}
}
\item adjective \\
\textbf{Motor} is used to describe activities relating to vehicles such as cars and buses .
 \textit{
	\begin{itemize}
	\item ...the future of the British motor industry.
	\item He worked as a motor mechanic.
	\end{itemize}
}
\item countable noun \\
Some people refer to a car as a \textbf{motor} .
 \textit{
	\begin{itemize}
	\end{itemize}
}
\item verb \\
If you \textbf{motor}  somewhere , you travel there in a car, usually for pleasure .
 \textit{
	\begin{itemize}
	\item I had motored down from Cheshire.
	\end{itemize}
}
\item verb \\
If people in a small sailing boat \textbf{motor} somewhere, they use the boat's motor rather than the power of the wind to get there.
 \textit{
	\begin{itemize}
	\item Restarting the engine, we motored downriver.
	\end{itemize}
}
\end{enumerate}

\section*{oppress}
{\large \color{blue}  oppresses  oppressing  oppressed  }
\subsection*{Explain}
\begin{enumerate}
\item verb \\
To \textbf{oppress} people means to treat them cruelly, or to prevent them from having the same opportunities , freedom , and benefits as others.
 \textit{
	\begin{itemize}
	\item These people often are oppressed by the governments of the countries they find themselves
in.
	\item We are not normal like everybody else. If we were, they wouldn't be oppressing us.
	\end{itemize}
}
\item verb \\
If something \textbf{oppresses} you, it makes you feel  depressed , anxious , and uncomfortable .
 \textit{
	\begin{itemize}
	\item The place oppressed Aubrey even before his eyes adjusted to the dark.
	\item It was not just the weather which oppressed her.
	\end{itemize}
}
\end{enumerate}

\section*{nation}
{\large \color{blue}  nations  }
\subsection*{Explain}
\begin{enumerate}
\item countable noun \\
A \textbf{nation} is an individual country considered together with its social and political structures.
 \textit{
	\begin{itemize}
	\item Such policies would require unprecedented cooperation between nations.
	\item ...Nigeria, by far the most populous of African nations.
	\end{itemize}
}
\item singular noun \\
\textbf{The}  \textbf{nation} is sometimes used to refer to all the people who live in a particular country.
 \textit{
	\begin{itemize}
	\item It was a story that touched the nation's heart.
	\end{itemize}
}
\end{enumerate}

\section*{press}
{\large \color{blue}  presses  pressing  pressed  }
\subsection*{Explain}
\begin{enumerate}
\item verb \\
If you \textbf{press} something somewhere , you push it firmly against something else.
 \textit{
	\begin{itemize}
	\item He pressed his back against the door.
	\item They pressed the silver knife into the cake.
	\end{itemize}
}
\item verb \\
If you \textbf{press} a button or switch , you push it with your finger in order to make a machine or device work.
 \textbf{Press} is also a noun.
 \textit{
	\begin{itemize}
	\item Drago pressed a button and the door closed.
	\item There was no-one at the reception desk, so he pressed a bell for service.
	\item ...a TV which rises from a table at the press of a button.
	\end{itemize}
}
\item verb \\
If you \textbf{press} something or \textbf{press down on} it, you push hard against it with your foot or hand.
 \textit{
	\begin{itemize}
	\item The engine stalled. He pressed the accelerator hard.
	\item She stood up and leaned forward with her hands pressing down on the desk.
	\end{itemize}
}
\item verb \\
If you \textbf{press for} something, you try hard to persuade someone to give it to you or to agree to it.
 \textit{
	\begin{itemize}
	\item Police might now press for changes in the law.
	\item They had pressed for their children to be taught French.
	\end{itemize}
}
\item verb \\
If you \textbf{press} someone, you try hard to persuade them to do something.
 \textit{
	\begin{itemize}
	\item Trade unions are pressing him to stand firm.
	\item Mr King seems certain to be pressed for further details.
	\item She smiles coyly when pressed about her private life.
	\end{itemize}
}
\item verb \\
If someone \textbf{presses} their claim, demand, or point, they state it in a very forceful way.
 \textit{
	\begin{itemize}
	\item The protest campaign has used mass strikes and demonstrations to press its demands.
	\item His officials have visited Washington to press their case for economic aid.
	\end{itemize}
}
\item verb \\
If an unpleasant feeling or worry  \textbf{presses on} you, it affects you very much or you are always thinking about it.
 \textit{
	\begin{itemize}
	\item The weight of irrational guilt pressed on her.
	\item Right now, I've got other problems that are pressing on me.
	\end{itemize}
}
\item verb \\
If you \textbf{press} something \textbf{on} someone, you give it to them and insist that they take it.
 \textit{
	\begin{itemize}
	\item All I had was money, which I pressed on her reluctant mother.
	\item Food and drink were pressed on him.
	\end{itemize}
}
\item verb \\
If you \textbf{press} clothes, you iron them in order to get rid of the creases.
 \textit{
	\begin{itemize}
	\item Vera pressed his shirt.
	\item There's a couple of dresses to be pressed.
	\item ...clean, neatly pressed, conservative clothes.
	\end{itemize}
}
\item verb \\
If you \textbf{press} fruits or vegetables, you squeeze them or crush them, usually in order to extract the juice.
 \textit{
	\begin{itemize}
	\item The grapes are hand-picked and pressed.
	\item I pressed the juice of half a lemon into a glass of water.
	\item ...1 clove fresh garlic, pressed or diced.
	\end{itemize}
}
\item singular noun \\
Newspapers are referred to as \textbf{the press} .
 \textit{
	\begin{itemize}
	\item Today the press is full of articles on the new prime minister.
	\item ...freedom of the Press.
	\item Press reports revealed that ozone levels in the upper atmosphere fell during the
past month.
	\end{itemize}
}
\item singular noun \\
Journalists are referred to as \textbf{the press} .
 \textit{
	\begin{itemize}
	\item Christie looked relaxed and calm as he faced the press afterwards.
	\item A meeting was promised, but the Press was not admitted.
	\end{itemize}
}
\item countable noun \\
A \textbf{press} or a \textbf{printing press} is a machine used for printing things such as books and newspapers.
 \textit{
	\begin{itemize}
	\item ...the invention of the printing press.
	\item He was writing the book up to the moment the presses rolled.
	\end{itemize}
}
\item  \\
 to get a bad press \textit{
	\begin{itemize}
	\end{itemize}
}
\item  \\
 press charges \textit{
	\begin{itemize}
	\end{itemize}
}
\item  \\
 go to press \textit{
	\begin{itemize}
	\end{itemize}
}
\item  \\
 to be pressed into service \textit{
	\begin{itemize}
	\end{itemize}
}
\end{enumerate}

\section*{nylon}
{\large \color{blue}  nylons  }
\subsection*{Explain}
\begin{enumerate}
\item uncountable noun \\
\textbf{Nylon} is a strong, flexible  artificial fibre.
 \textit{
	\begin{itemize}
	\item Europe's largest producer of nylon is based in Belgium.
	\item I put on a new pair of nylon socks.
	\end{itemize}
}
\item plural noun \\
\textbf{Nylons} are stockings made of nylon.
 \textit{
	\begin{itemize}
	\item This woman wore seamed nylons and kept smoothing her skirt.
	\end{itemize}
}
\end{enumerate}

\section*{prosper}
{\large \color{blue}  prospers  prospering  prospered  }
\subsection*{Explain}
\begin{enumerate}
\item verb \\
If people or businesses  \textbf{prosper} , they are successful and do well .
 \textit{
	\begin{itemize}
	\item The high street banks continue to prosper.
	\item His teams have always prospered in cup competitions.
	\end{itemize}
}
\end{enumerate}

\section*{occurrence}
{\large \color{blue}  occurrences  }
\subsection*{Explain}
\begin{enumerate}
\item countable noun \\
An \textbf{occurrence} is something that happens .
 \textit{
	\begin{itemize}
	\item Complaints seemed to be an everyday occurrence.
	\item The food queues have become a daily occurrence across the country.
	\end{itemize}
}
\item countable noun \\
\textbf{The occurrence of} something is the fact that it happens or is present .
 \textit{
	\begin{itemize}
	\item The greatest occurrence of coronary heart disease is in those over 65.
	\end{itemize}
}
\end{enumerate}

\section*{quarrel}
{\large \color{blue}  quarrels  quarrelling  quarrelled  }
\subsection*{Explain}
\begin{enumerate}
\item countable noun \\
A \textbf{quarrel} is an angry argument between two or more friends or family members.
 \textit{
	\begin{itemize}
	\item I had a terrible quarrel with my other brothers.
	\item It could have happened during a quarrel between them over Naomi.
	\end{itemize}
}
\item countable noun \\
\textbf{Quarrels} between countries or groups of people are disagreements, which may be diplomatic or include fighting .
 \textit{
	\begin{itemize}
	\item Canning thought that Persia would probably not want to risk a quarrel with England.
	\end{itemize}
}
\item verb \\
When two or more people \textbf{quarrel} , they have an angry argument.
 \textit{
	\begin{itemize}
	\item At one point we quarrelled, over something silly.
	\item My brother quarrelled with my father.
	\end{itemize}
}
\item singular noun \\
If you say that you have no \textbf{quarrel} with someone or something, you mean that you do not disagree with them.
 \textit{
	\begin{itemize}
	\item We have no quarrel with the people of Spain or of any other country.
	\item She had no quarrel with much of what had been said at dinner.
	\end{itemize}
}
\item verb \\
If you say that you would \textbf{quarrel}  \textbf{with} someone or \textbf{with} something that they have said , you mean that you disagree with them.
 \textit{
	\begin{itemize}
	\item I would quarrel with you on that figure.
	\item While some of his peers might quarrel with the title, his credentials remain impressive.
	\end{itemize}
}
\end{enumerate}

\section*{outlet}
{\large \color{blue}  outlets  }
\subsection*{Explain}
\begin{enumerate}
\item countable noun \\
An \textbf{outlet} is a shop or organization which sells the goods made by a particular manufacturer .
 \textit{
	\begin{itemize}
	\item ...the largest retail outlet in the city.
	\end{itemize}
}
\item countable noun \\
An \textbf{outlet} or an \textbf{outlet store} is a place which sells slightly  damaged or outdated goods from a particular manufacturer, or goods that it made in greater  quantities than needed .
 \textit{
	\begin{itemize}
	\item ...the factory outlet store in Belmont.
	\end{itemize}
}
\item countable noun \\
If someone has an \textbf{outlet}  \textbf{for} their feelings or ideas , they have a means of expressing and releasing them.
 \textit{
	\begin{itemize}
	\item Her father had found an outlet for his ambition in his work.
	\end{itemize}
}
\item countable noun \\
An \textbf{outlet} is a hole or pipe through which liquid or air can flow  away .
 \textit{
	\begin{itemize}
	\item ...a warm air outlet.
	\item ...an underwater outlet pipe discharging waste into the sea.
	\end{itemize}
}
\item countable noun \\
An \textbf{outlet} is a place, usually in a wall , where you can connect electrical devices to the electricity supply.
 \textit{
	\begin{itemize}
	\end{itemize}
}
\end{enumerate}

\section*{recall}
{\large \color{blue}  recalls  recalling  recalled  }
\subsection*{Explain}
\begin{enumerate}
\item verb \\
When you \textbf{recall} something, you remember it and tell others about it.
 \textit{
	\begin{itemize}
	\item Henderson recalled that he first met Pollard during a business trip to Washington.
	\item Her teacher recalled: 'She was always on about modelling.'
	\item Colleagues today recall with humor how meetings would crawl into the early morning
hours.
	\item I recalled the way they had been dancing together.
	\item I have no idea what she said, something about airline travel, I seem to recall.
	\end{itemize}
}
\item verb \\
You can say  \textbf{as I recall} , \textbf{you might recall} , or \textbf{you will recall} to someone that you are talking to when you want to mention something that you are both already  aware of which is relevant to the discussion .
 \textit{
	\begin{itemize}
	\item As I recall, you're not on the board, Joe; you're only a minor shareholder.
	\item At the final a fortnight ago, you will recall that after eight minutes the club had
a two-goal lead.
	\end{itemize}
}
\item uncountable noun \\
\textbf{Recall} is the ability to remember something that has happened in the past or the act of remembering it.
 \textit{
	\begin{itemize}
	\item He had a good memory, and total recall of her spoken words.
	\end{itemize}
}
\item verb \\
If you \textbf{are}  \textbf{recalled} to your home , country, or the place where you work, you are ordered to return there.
 \textbf{Recall} is also a noun .
 \textit{
	\begin{itemize}
	\item The Ambassador was recalled after a row over refugees seeking asylum at the embassy.
	\item Parliament was recalled from its summer recess.
	\item The recall of ambassador Alan Green was a public signal of America's concern.
	\end{itemize}
}
\item verb \\
In sport , if a player is \textbf{recalled}  \textbf{to} a team , he or she is included in that team again after being left out.
 \textbf{Recall} is also a noun.
 \textit{
	\begin{itemize}
	\item He is still delighted at being recalled to the Argentina squad after a nine-year
absence.
	\item I had done enough after being recalled against Pakistan to have got on the tour to
India.
	\item It would be great to get a recall to the England squad for Sweden.
	\end{itemize}
}
\item verb \\
If a company  \textbf{recalls} a product , it asks the shops or the people who have bought that product to return it because there is something wrong with it.
 \textit{
	\begin{itemize}
	\item The toy company said it was recalling the building set.
	\item More than 3,000 cars were recalled yesterday because of a brake problem.
	\end{itemize}
}
\item  \\
 beyond recall \textit{
	\begin{itemize}
	\end{itemize}
}
\end{enumerate}

\section*{poll}
{\large \color{blue}  polls  polling  polled  }
\subsection*{Explain}
\begin{enumerate}
\item countable noun \\
A \textbf{poll} is a survey in which people are asked their opinions about something, usually in order to find out how popular something is or what people intend to do in the future .
 \textit{
	\begin{itemize}
	\item At least 60 per cent of the country wants the strikers to win, polls show.
	\item We are doing a weekly poll on the president, and clearly his popularity has declined.
	\item The Socialist Party, which won a convincing victory in elections in June, has been
losing support in the polls recently.
	\end{itemize}
}
\item verb \\
If you \textbf{are polled}  \textbf{on} something, you are asked what you think about it as part of a survey.
 \textit{
	\begin{itemize}
	\item More than 18,000 people were polled.
	\item Audiences were going to be polled on which of three pieces of music they liked best.
	\item More than 70 per cent of those polled said that they approved of his record as president.
	\end{itemize}
}
\item plural noun \\
\textbf{The polls} means an election for a country's government, or the place where people go to vote in an election.
 \textit{
	\begin{itemize}
	\item In 1945, Winston Churchill was defeated at the polls.
	\item Voters are due to go to the polls on Sunday to elect a new president.
	\item The polls have closed in the Pakistan parliamentary elections.
	\end{itemize}
}
\item verb \\
If a political party or a candidate  \textbf{polls} a particular number or percentage of votes, they get that number or percentage of votes in an election.
 \textit{
	\begin{itemize}
	\item It was a disappointing result for the Greens who polled three percent.
	\item The result showed he had polled enough votes to force a second ballot.
	\end{itemize}
}
\end{enumerate}

\section*{recollect}
{\large \color{blue}  recollects  recollecting  recollected  }
\subsection*{Explain}
\begin{enumerate}
\item verb \\
If you \textbf{recollect} something, you remember it.
 \textit{
	\begin{itemize}
	\item Ramona spoke with warmth when she recollected the doctor who used to be at the county
hospital.
	\item His efforts, she recollected many years later, were distinctly half-hearted.
	\end{itemize}
}
\end{enumerate}

\section*{presence}
{\large \color{blue}  presences  }
\subsection*{Explain}
\begin{enumerate}
\item singular noun \\
Someone's \textbf{presence} in a place is the fact that they are there.
 \textit{
	\begin{itemize}
	\item They argued that his presence in the village could only stir up trouble.
	\item She received hundreds of emails and phone calls requesting her presence at company
meetings.
	\end{itemize}
}
\item uncountable noun \\
If you say that someone has \textbf{presence} , you mean that they impress people by their appearance and manner.
 \textit{
	\begin{itemize}
	\item They do not seem to have the vast, authoritative presence of those great men.
	\item Hendrix's stage presence appealed to thousands of teenage rebels.
	\end{itemize}
}
\item countable noun \\
A \textbf{presence} is a person or creature that you cannot see , but that you are aware of.
 \textit{
	\begin{itemize}
	\item The forest was dark and silent, haunted by shadows and unseen presences.
	\item She started to be affected by the ghostly presence she could feel in the house.
	\end{itemize}
}
\item singular noun \\
If a country has a military  \textbf{presence} in another country, it has some of its armed forces there.
 \textit{
	\begin{itemize}
	\item London had intended to grant Aden independence in 1968 but retain a military presence.
	\end{itemize}
}
\item uncountable noun \\
If you refer to the \textbf{presence} of a substance in another thing, you mean that it is in that thing.
 \textit{
	\begin{itemize}
	\item The somewhat acid flavour is caused by the presence of lactic acid.
	\item ...the presence of a carcinogen in the water.
	\item Although the fluid presents no symptoms to the patient, its presence can be detected
by a test.
	\end{itemize}
}
\item  \\
 make one's presence felt \textit{
	\begin{itemize}
	\end{itemize}
}
\item  \\
 in someone's presence \textit{
	\begin{itemize}
	\end{itemize}
}
\item  \\
 presence of mind \textit{
	\begin{itemize}
	\end{itemize}
}
\end{enumerate}

\section*{rectify}
{\large \color{blue}  rectifies  rectifying  rectified  }
\subsection*{Explain}
\begin{enumerate}
\item verb \\
If you \textbf{rectify} something that is wrong , you change it so that it becomes correct or satisfactory .
 \textit{
	\begin{itemize}
	\item Only an act of Congress could rectify the situation.
	\item That mistake could have been rectified within 28 days.
	\end{itemize}
}
\end{enumerate}

\section*{resent}
{\large \color{blue}  resents  resenting  resented  }
\subsection*{Explain}
\begin{enumerate}
\item verb \\
If you \textbf{resent} someone or something, you feel bitter and angry about them.
 \textit{
	\begin{itemize}
	\item She resents her mother for being so tough on her.
	\item I resent being dependent on her.
	\end{itemize}
}
\end{enumerate}

\section*{publication}
{\large \color{blue}  publications  }
\subsection*{Explain}
\begin{enumerate}
\item uncountable noun \\
The \textbf{publication} of a book or magazine is the act of printing it and sending it to shops to be sold .
 \textit{
	\begin{itemize}
	\item The guide is being translated into several languages for publication at the end of
the year.
	\item The publication of his collected poems was approaching the status of an event.
	\end{itemize}
}
\item countable noun \\
A \textbf{publication} is a book or magazine that has been published.
 \textit{
	\begin{itemize}
	\item They have started legal proceedings against two publications which spoke of an affair.
	\end{itemize}
}
\item uncountable noun \\
The \textbf{publication of} something such as information is the act of making it known to the public, for example by informing  journalists or by publishing a government document .
 \textit{
	\begin{itemize}
	\item A spokesman said: 'We have no comment regarding the publication of these photographs.'
	\end{itemize}
}
\end{enumerate}

\section*{respond}
{\large \color{blue}  responds  responding  responded  }
\subsection*{Explain}
\begin{enumerate}
\item verb \\
When you \textbf{respond} to something that is done or said , you react to it by doing or saying something yourself.
 \textit{
	\begin{itemize}
	\item They are likely to respond positively to the President's request for aid.
	\item The army responded with gunfire and tear gas.
	\item 'Are you well enough to carry on?'—'Of course,' she responded scornfully.
	\item The Belgian Minister of Foreign Affairs responded that the protection of refugees
was a matter for an international organization.
	\end{itemize}
}
\item verb \\
When you \textbf{respond}  \textbf{to} a need , crisis , or challenge , you take the necessary or appropriate action.
 \textit{
	\begin{itemize}
	\item This modest group size allows our teachers to respond to the needs of each student.
	\end{itemize}
}
\item verb \\
If a patient or their injury or illness  \textbf{is responding}  \textbf{to}  treatment , the treatment is working and they are getting  better .
 \textit{
	\begin{itemize}
	\item I'm pleased to say that he is now doing well and responding to treatment.
	\end{itemize}
}
\end{enumerate}

\section*{sale}
{\large \color{blue}  sales  }
\subsection*{Explain}
\begin{enumerate}
\item singular noun \\
The \textbf{sale} of goods is the act of selling them for money.
 \textit{
	\begin{itemize}
	\item Efforts were made to limit the sale of sugary drinks.
	\item ...a proposed manufacturing sale to India.
	\item He decided to move house and set about trying to make a sale.
	\end{itemize}
}
\item plural noun \\
The \textbf{sales} of a product are the quantity of it that is sold.
 \textit{
	\begin{itemize}
	\item The newspaper has sales of 1.72 million.
	\item ...the huge Christmas sales of computer games.
	\item ...retail sales figures.
	\end{itemize}
}
\item plural noun \\
The part of a company that deals with \textbf{sales} deals with selling the company's products.
 \textit{
	\begin{itemize}
	\item Until 1983 he worked in sales and marketing.
	\item She was their Dusseldorf sales manager.
	\end{itemize}
}
\item countable noun \\
A \textbf{sale} is an occasion when a shop sells things at less than their normal price.
 \textit{
	\begin{itemize}
	\item ...a pair of jeans bought half-price in a sale.
	\item Many stores have started their January sales a month early.
	\end{itemize}
}
\item countable noun \\
A \textbf{sale} is an event when goods are sold to the person who offers the highest price.
 \textit{
	\begin{itemize}
	\item The painting was bought by dealers at the Christie's sale.
	\end{itemize}
}
\item  \\
 for sale \textit{
	\begin{itemize}
	\end{itemize}
}
\item  \\
 on sale \textit{
	\begin{itemize}
	\end{itemize}
}
\item  \\
 on sale \textit{
	\begin{itemize}
	\end{itemize}
}
\item  \\
 up for sale \textit{
	\begin{itemize}
	\end{itemize}
}
\end{enumerate}

\section*{review}
{\large \color{blue}  reviews  reviewing  reviewed  }
\subsection*{Explain}
\begin{enumerate}
\item countable noun \\
A \textbf{review}  \textbf{of} a situation or system is its formal examination by people in authority. This is usually done
in order to see whether it can be improved or corrected.
 \textit{
	\begin{itemize}
	\item The president ordered a review of U.S. economic aid to Jordan.
	\item The White House quickly announced that the policy is under review.
	\end{itemize}
}
\item verb \\
If you \textbf{review} a situation or system, you consider it carefully to see what is wrong with it or how it could be improved.
 \textit{
	\begin{itemize}
	\item The Prime Minister reviewed the situation with his Cabinet yesterday.
	\item The next day we reviewed the previous day's work.
	\end{itemize}
}
\item countable noun \\
A \textbf{review} is a report in the media in which someone gives their opinion of something such as a new book or film.
 \textit{
	\begin{itemize}
	\item Rave reviews and commercial success were accompanied by industry-wide adulation.
	\item We've never had a good review in the music press.
	\end{itemize}
}
\item verb \\
If someone \textbf{reviews} something such as a new book or film, they write a report or give a talk on television or radio in which they express their opinion of it.
 \textit{
	\begin{itemize}
	\item Richard Coles reviews all of the latest film releases.
	\item His book about Afghanistan is reviewed here by Anthony Hyman.
	\end{itemize}
}
\item verb \\
When a military or political leader  \textbf{reviews}  troops , they examine them, or watch them marching .
 \textit{
	\begin{itemize}
	\item When he reviewed the troops they cheered him as he smiled and raised his hat.
	\end{itemize}
}
\item verb \\
When you \textbf{review}  \textbf{for} an examination, you read things again and make notes in order to be prepared for the examination.
 \textbf{Review} is also a noun .
 \textit{
	\begin{itemize}
	\item Reviewing for exams lets you bring together all the individual parts of the course.
	\item Review all the notes you need to cover for each course.
	\item Begin by planning on three two-hour reviews with four chapters per session.
	\end{itemize}
}
\end{enumerate}

\section*{skin}
{\large \color{blue}  skins  skinning  skinned  }
\subsection*{Explain}
\begin{enumerate}
\item variable noun \\
Your \textbf{skin} is the natural covering of your body.
 \textit{
	\begin{itemize}
	\item His skin is clear and smooth.
	\item There are three major types of skin cancer.
	\item The only difference between us is the colour of our skins.
	\end{itemize}
}
\item variable noun \\
An animal \textbf{skin} is skin which has been removed from a dead animal. Skins are used to make things such as coats and rugs .
 \textit{
	\begin{itemize}
	\item That was real crocodile skin.
	\item ...a leopard skin coat.
	\end{itemize}
}
\item variable noun \\
The \textbf{skin} of a fruit or vegetable is its outer layer or covering.
 \textit{
	\begin{itemize}
	\item The outer skin of the orange is called the 'zest'.
	\item ...banana skins.
	\end{itemize}
}
\item singular noun \\
If a \textbf{skin} forms on the surface of a liquid, a thin , fairly  solid layer forms on it.
 \textit{
	\begin{itemize}
	\item Stir the custard occasionally to prevent a skin forming.
	\end{itemize}
}
\item verb \\
If you \textbf{skin} a dead animal, you remove its skin.
 \textit{
	\begin{itemize}
	\item ...with the expertise of a chef skinning a rabbit.
	\end{itemize}
}
\item  \\
 to jump out of one's skin \textit{
	\begin{itemize}
	\end{itemize}
}
\item  \\
 save one's own skin/save one's skin \textit{
	\begin{itemize}
	\end{itemize}
}
\item  \\
 by the skin of your teeth \textit{
	\begin{itemize}
	\end{itemize}
}
\item  \\
 a thick skin \textit{
	\begin{itemize}
	\end{itemize}
}
\end{enumerate}

\section*{shoot}
{\large \color{blue}  shoots  shooting  shot  }
\subsection*{Explain}
\begin{enumerate}
\item verb \\
If someone \textbf{shoots} a person or an animal, they kill them or injure them by firing a bullet or arrow at them.
 \textit{
	\begin{itemize}
	\item The police had orders to shoot anyone who attacked them.
	\item Namibian law permits ranchers to shoot cheetahs to protect their livestock.
	\item Gunmen shot dead the brother of the minister.
	\item The man was shot dead by the police during a raid on his house.
	\item Her father shot himself in the head with a shotgun.
	\end{itemize}
}
\item verb \\
To \textbf{shoot} means to fire a bullet from a weapon such as a gun.
 \textit{
	\begin{itemize}
	\item He taunted armed officers by pointing to his head, as if inviting them to shoot.
	\item The police came around the corner and they started shooting at us.
	\item She had never been able to shoot straight.
	\item Troops began shooting in all directions.
	\end{itemize}
}
\item verb \\
If someone or something \textbf{shoots} in a particular direction, they move in that direction quickly and suddenly .
 \textit{
	\begin{itemize}
	\item They had almost reached the boat when a figure shot past them.
	\item Another car shot out of a junction and smashed into the back of them.
	\end{itemize}
}
\item verb \\
If you \textbf{shoot} something somewhere or if it \textbf{shoots} somewhere, it moves there quickly and suddenly.
 \textit{
	\begin{itemize}
	\item Masters shot a hand across the table and gripped his wrist.
	\item As soon as she got close, the old woman's hand shot out.
	\item You'd turn on the water, and it would shoot straight up in the air.
	\end{itemize}
}
\item verb \\
If you \textbf{shoot} a look at someone, you look at them quickly and briefly, often in a way that expresses
your feelings.
 \textit{
	\begin{itemize}
	\item Mary Ann shot him a rueful look.
	\item The man in the black overcoat shot a penetrating look at the other man.
	\end{itemize}
}
\item verb \\
If someone \textbf{shoots}  \textbf{to}  fame , they become famous or successful very quickly.
 \textit{
	\begin{itemize}
	\item She shot to fame a few years ago with her extraordinary first novel.
	\item She shot to stardom on Broadway in a Noel Coward play.
	\end{itemize}
}
\item verb \\
When people \textbf{shoot} a film or \textbf{shoot} photographs, they make a film or take photographs using a camera .
 \textbf{Shoot} is also a noun .
 \textit{
	\begin{itemize}
	\item He'd love to shoot his film in Cuba.
	\item Three CBS cameramen were on site to shoot and edit taped reports.
	\item ...a barn presently being used for a video shoot.
	\end{itemize}
}
\item countable noun \\
\textbf{Shoots} are plants that are beginning to grow, or new parts growing from a plant or tree.
 \textit{
	\begin{itemize}
	\end{itemize}
}
\item verb \\
In sports such as football or basketball , when someone \textbf{shoots} , they try to score by kicking , throwing, or hitting the ball towards the goal.
 \textit{
	\begin{itemize}
	\item Spencer scuttled away from Young to shoot wide when he should have scored.
	\item A time limit was set for a team to shoot at the basket.
	\end{itemize}
}
\item verb \\
When someone \textbf{shoots}  pool or \textbf{shoots}  craps , they play a game of pool or the dice game called craps.
 \textit{
	\begin{itemize}
	\item People are still hanging out, maybe shooting some pool.
	\end{itemize}
}
\item  \\
 shoot the breeze/shoot the bull \textit{
	\begin{itemize}
	\end{itemize}
}
\item  \\
 to shoot yourself in the foot \textit{
	\begin{itemize}
	\end{itemize}
}
\end{enumerate}

\section*{slave}
{\large \color{blue}  slaves  slaving  slaved  }
\subsection*{Explain}
\begin{enumerate}
\item countable noun \\
A \textbf{slave} is someone who is the property of another person and has to work for that person.
 \textit{
	\begin{itemize}
	\item The state of Liberia was formed a century and a half ago by freed slaves from the
United States.
	\end{itemize}
}
\item countable noun \\
You can  describe someone as a \textbf{slave} when they are completely under the control of another person or of a powerful influence.
 \textit{
	\begin{itemize}
	\item Movie stars used to be slaves to the studio system.
	\end{itemize}
}
\item verb \\
If you say that a person \textbf{is slaving}  \textbf{over} something or \textbf{is slaving}  \textbf{for} someone, you mean that they are working very hard .
 \textbf{Slave away} means the same as slave .
 \textit{
	\begin{itemize}
	\item When you're busy all day the last thing you want to do is spend hours slaving over
a hot stove.
	\item He stares at the hundreds of workers slaving away in the intense sun.
	\end{itemize}
}
\end{enumerate}

\section*{spit}
{\large \color{blue}  spits  spitting  spat  }
\subsection*{Explain}
\begin{enumerate}
\item uncountable noun \\
\textbf{Spit} is the watery liquid produced in your mouth. You usually use \textbf{spit} to refer to an amount of it that has been forced out of someone's mouth.
 \textit{
	\begin{itemize}
	\end{itemize}
}
\item verb \\
If someone \textbf{spits} , they force an amount of liquid out of their mouth, often to show hatred or contempt .
 \textit{
	\begin{itemize}
	\item The gang thought of hitting him too, but decided just to spit.
	\item They spat at me and taunted me.
	\item She spit into the little tray of mascara and brushed it on her lashes.
	\end{itemize}
}
\item verb \\
If you \textbf{spit} liquid or food somewhere , you force a small amount of it out of your mouth.
 \textit{
	\begin{itemize}
	\item Spit out that gum and pay attention.
	\item He felt as if a serpent had spat venom into his eyes.
	\item I started spitting blood and my mother panicked.
	\end{itemize}
}
\item verb \\
If something such as a machine or food that is cooking  \textbf{spits} , it sends out small amounts of something, making a series of short, sharp noises .
 \textit{
	\begin{itemize}
	\item The engine spat and banged.
	\item ...the fire where kebabs were sizzling and spitting.
	\end{itemize}
}
\item verb \\
If someone \textbf{spits} an insult or comment , they say it in an angry or hostile way.
 \textbf{Spit out} means the same as spit .
 \textit{
	\begin{itemize}
	\item 'Wait a minute,' Mindy spat. 'You can't stay overnight.'.
	\item Cramer spat an obscenity.
	\item He spat out 'I don't like the way he looks at me.'.
	\item She spat the name out like an insult.
	\item He appeared to be angry, spitting out disconnected words.
	\end{itemize}
}
\item verb \\
If \textbf{it}  \textbf{is spitting} , it is raining very lightly.
 \textit{
	\begin{itemize}
	\item It will stop in a minute–it's only spitting.
	\end{itemize}
}
\item countable noun \\
A \textbf{spit} is a long rod which is pushed through a piece of meat and hung over an open fire to cook the meat.
 \textit{
	\begin{itemize}
	\item She roasted the meat on a spit.
	\end{itemize}
}
\item countable noun \\
A \textbf{spit of} land is a long, flat , narrow piece of land that sticks out into the sea.
 \textit{
	\begin{itemize}
	\end{itemize}
}
\item  \\
 within spitting distance \textit{
	\begin{itemize}
	\end{itemize}
}
\item  \\
 spitting image \textit{
	\begin{itemize}
	\end{itemize}
}
\end{enumerate}

\section*{spokesman}
{\large \color{blue}  spokesmen  }
\subsection*{Explain}
\begin{enumerate}
\item countable noun \\
A \textbf{spokesman} is a male spokesperson.
 \textit{
	\begin{itemize}
	\item A U.N. spokesman said that the mission will carry 20 tons of relief supplies.
	\end{itemize}
}
\end{enumerate}

\section*{squeeze}
{\large \color{blue}  squeezes  squeezing  squeezed  }
\subsection*{Explain}
\begin{enumerate}
\item verb \\
If you \textbf{squeeze} something, you press it firmly, usually with your hands .
 \textbf{Squeeze} is also a noun .
 \textit{
	\begin{itemize}
	\item He squeezed her arm reassuringly.
	\item Dip the bread briefly in water, then squeeze it dry.
	\item I liked her way of reassuring you with a squeeze of the hand.
	\end{itemize}
}
\item verb \\
If you \textbf{squeeze} a liquid or a soft substance out of an object, you get the liquid or substance out by pressing the object.
 \textit{
	\begin{itemize}
	\item Joe put the plug in the sink and squeezed some detergent over the dishes.
	\item ...freshly squeezed lemon juice.
	\end{itemize}
}
\item verb \\
If you \textbf{squeeze} your eyes  shut or if your eyes \textbf{squeeze} shut, you close them tightly, usually because you are frightened or to protect your eyes from something such as strong sunlight .
 \textit{
	\begin{itemize}
	\item Nancy squeezed her eyes shut and prayed.
	\item If you keep your eyes squeezed shut, you'll miss the show.
	\item My eyes were squeezed shut against the light.
	\end{itemize}
}
\item verb \\
If you \textbf{squeeze} a person or thing somewhere or if they \textbf{squeeze} there, they manage to get through or into a small space.
 \textit{
	\begin{itemize}
	\item Somehow they squeezed him into the cockpit, and strapped him in.
	\item Many break-ins are carried out by youngsters who can squeeze through tiny windows.
	\end{itemize}
}
\item singular noun \\
If you say that getting a number of people into a small space is \textbf{a squeeze} , you mean that it is only just possible for them all to get into it.
 \textit{
	\begin{itemize}
	\item It was a squeeze in the car with five of them.
	\item The lift holds six people at a squeeze.
	\end{itemize}
}
\item verb \\
If you \textbf{squeeze} something \textbf{out of} someone, you persuade them to give it to you, although they may be unwilling to do this.
 \textit{
	\begin{itemize}
	\item The investigators complained about the difficulties of squeezing information out
of residents.
	\item The company intends to squeeze further savings from its suppliers.
	\end{itemize}
}
\item verb \\
If a government \textbf{squeezes} the economy , they put strict controls on people's ability to borrow money or on their own departments ' freedom to spend money, in order to control the country's rate of inflation.
 \textbf{Squeeze} is also a noun.
 \textit{
	\begin{itemize}
	\item The government will squeeze the economy into a severe recession to force inflation
down.
	\item Defense experts say joint projects are increasingly squeezed by budget pressures.
	\item The CBI also says the squeeze is slowing down inflation.
	\end{itemize}
}
\item countable noun \\
Someone's \textbf{squeeze} is their boyfriend or girlfriend .
 \textit{
	\begin{itemize}
	\item Jack showed off his latest squeeze at the weekend.
	\end{itemize}
}
\end{enumerate}

\section*{strap}
{\large \color{blue}  straps  strapping  strapped  }
\subsection*{Explain}
\begin{enumerate}
\item countable noun \\
A \textbf{strap} is a narrow piece of leather, cloth , or other material. Straps are used to carry things, fasten things together, or to hold a piece of clothing in place.
 \textit{
	\begin{itemize}
	\item Nancy gripped the strap of her beach bag.
	\item Brian pulled the straps through the buckles of his suitcase.
	\item She pulled the strap of her nightgown onto her shoulder.
	\item I undid my watch strap.
	\end{itemize}
}
\item verb \\
If you \textbf{strap} something somewhere , you fasten it there with a strap.
 \textit{
	\begin{itemize}
	\item We strapped the skis onto the roof of the car.
	\item She strapped the gun belt around her middle.
	\item Through the basement window I saw him strap on his pink cycling helmet.
	\item The carer has to place the patient on the seat and strap him in.
	\end{itemize}
}
\end{enumerate}

\section*{suit}
{\large \color{blue}  suits  suiting  suited  }
\subsection*{Explain}
\begin{enumerate}
\item countable noun \\
A man's \textbf{suit} consists of a jacket, trousers, and sometimes a waistcoat , all made from the same fabric.
 \textit{
	\begin{itemize}
	\item ...a dark pin-striped business suit.
	\item ...a smart suit and tie.
	\end{itemize}
}
\item countable noun \\
A woman's \textbf{suit} consists of a jacket and skirt, or sometimes trousers, made from the same fabric.
 \textit{
	\begin{itemize}
	\item I was wearing my tweed suit.
	\end{itemize}
}
\item countable noun \\
A particular type of \textbf{suit} is a piece of clothing that you wear for a particular activity.
 \textit{
	\begin{itemize}
	\item ...a completely revolutionary atmospheric diving suit.
	\end{itemize}
}
\item verb \\
If something \textbf{suits} you, it is convenient for you or is the best thing for you in the circumstances .
 \textit{
	\begin{itemize}
	\item They will only release information if it suits them.
	\item They should be able to find you the best package to suit your needs.
	\end{itemize}
}
\item verb \\
If something \textbf{suits} you, you like it.
 \textit{
	\begin{itemize}
	\item I don't think a sedentary life would altogether suit me.
	\end{itemize}
}
\item verb \\
If a piece of clothing or a particular style or colour \textbf{suits} you, it makes you look  attractive .
 \textit{
	\begin{itemize}
	\item Green suits you.
	\end{itemize}
}
\item verb \\
If you \textbf{suit}  \textbf{yourself} , you do something just because you want to do it, without bothering to consider other people.
 \textit{
	\begin{itemize}
	\item These large institutions make–and change–the rules to suit themselves.
	\item He made a dismissive gesture. 'Suit yourself.'
	\end{itemize}
}
\item countable noun \\
In a court of law, a \textbf{suit} is a case in which someone tries to get a legal  decision against a person or company, often so that the person or company will have to pay
them money for having done something wrong to them.
 In American English, you can say that someone \textbf{files} or \textbf{brings}  \textbf{suit}  \textbf{against} another person.
 \textit{
	\begin{itemize}
	\item Up to 2,000 former employees have filed personal injury suits against the company.
	\item The judge dismissed the suit.
	\item The insurance company filed suit against the two girls.
	\end{itemize}
}
\item countable noun \\
A \textbf{suit} is one of the four types of card in a set of playing cards. These are hearts, diamonds,
clubs, and spades.
 \textit{
	\begin{itemize}
	\end{itemize}
}
\item  \\
 to follow suit \textit{
	\begin{itemize}
	\end{itemize}
}
\item  \\
 suit sb down to the ground \textit{
	\begin{itemize}
	\end{itemize}
}
\end{enumerate}

\section*{taxi}
{\large \color{blue}  taxis  taxiing  taxied  }
\subsection*{Explain}
\begin{enumerate}
\item countable noun \\
A \textbf{taxi} is a car driven by a person whose job is to take people where they want to go in return for money .
 \textit{
	\begin{itemize}
	\item The taxi drew up in front of the Riviera Club.
	\item He set off by taxi.
	\end{itemize}
}
\item verb \\
When an aircraft \textbf{taxis} along the ground, or when a pilot  \textbf{taxis} a plane  somewhere , it moves slowly along the ground.
 \textit{
	\begin{itemize}
	\item She gave permission to the plane to taxi into position and hold for takeoff.
	\item The pilot taxied the plane to the end of the runway.
	\end{itemize}
}
\end{enumerate}

\section*{sweep}
{\large \color{blue}  sweeps  sweeping  swept  }
\subsection*{Explain}
\begin{enumerate}
\item verb \\
If you \textbf{sweep} an area of floor or ground, you push dirt or rubbish off it using a brush with a long handle.
 \textit{
	\begin{itemize}
	\item The owner of the store was sweeping his floor when I walked in.
	\item She was in the kitchen sweeping crumbs into a dust pan.
	\item Norma picked up the broom and began sweeping.
	\end{itemize}
}
\item verb \\
If you \textbf{sweep} things off something, you push them off with a quick smooth movement of your arm.
 \textit{
	\begin{itemize}
	\item I swept rainwater off the flat top of a gravestone.
	\item With a gesture of frustration, she swept the cards from the table.
	\item 'Thanks friend,' he said, while sweeping the money into his pocket.
	\end{itemize}
}
\item verb \\
If someone with long hair \textbf{sweeps} their hair into a particular style, they put it into that style.
 \textit{
	\begin{itemize}
	\item ...stylish ways of sweeping your hair off your face.
	\item Her long, fine hair was swept back in a ponytail.
	\end{itemize}
}
\item verb \\
If your arm or hand \textbf{sweeps} in a particular direction, or if you \textbf{sweep} it there, it moves quickly and smoothly in that direction.
 \textbf{Sweep} is also a noun.
 \textit{
	\begin{itemize}
	\item His arm swept around the room.
	\item Daniels swept his arm over his friend's shoulder.
	\item ...the long sweeping arm movements of a violinist.
	\item With one sweep of her hand she threw back the sheets.
	\end{itemize}
}
\item verb \\
If wind, a stormy sea, or another strong force \textbf{sweeps} someone or something along, it moves them quickly along.
 \textit{
	\begin{itemize}
	\item ...landslides that buried homes and swept cars into the sea.
	\item Suddenly, she was swept along by the crowd.
	\end{itemize}
}
\item verb \\
If you \textbf{are swept}  somewhere , you are taken there very quickly.
 \textit{
	\begin{itemize}
	\item The visitors were swept past various monuments.
	\item A limousine swept her along the busy freeway to the airport.
	\end{itemize}
}
\item verb \\
If something \textbf{sweeps} from one place to another, it moves there extremely quickly.
 \textit{
	\begin{itemize}
	\item An icy wind swept through the streets.
	\item The car swept past the gate house.
	\end{itemize}
}
\item verb \\
If events, ideas, or beliefs \textbf{sweep} through a place, they spread quickly through it.
 \textit{
	\begin{itemize}
	\item A flu epidemic is sweeping through Moscow.
	\item ...the wave of patriotism sweeping the country.
	\end{itemize}
}
\item verb \\
If someone \textbf{sweeps} into a place, they walk into it in a proud, confident way, often when they are angry .
 \textit{
	\begin{itemize}
	\item She swept into the conference room.
	\item Scarlet with rage, she swept past her employer and stormed up the stairs.
	\item The Chief turned and swept out.
	\end{itemize}
}
\item verb \\
If a person or thing \textbf{sweeps} something \textbf{away} or \textbf{aside} , they remove it quickly and completely.
 \textit{
	\begin{itemize}
	\item The commission's conclusions sweep away a decade of denials and cover-ups.
	\item In times of war, governments often sweep human rights aside.
	\item He swept the names from his mind.
	\end{itemize}
}
\item verb \\
If lights or someone's eyes \textbf{sweep} an area, they move across the area from side to side.
 \textit{
	\begin{itemize}
	\item Helicopters with searchlights swept the park which was sealed off.
	\item Her gaze sweeps rapidly around the room.
	\end{itemize}
}
\item verb \\
If land or water \textbf{sweeps} somewhere, it stretches out in a long, wide, curved shape.
 \textit{
	\begin{itemize}
	\item The land sweeps away from long areas of greenery.
	\item ...the arc of countries that sweeps down from South Korea to Indonesia.
	\end{itemize}
}
\item countable noun \\
A \textbf{sweep}  \textbf{of} land or water forms a long, wide, curved shape.
 \textit{
	\begin{itemize}
	\item The ground fell away in a broad sweep down to the river.
	\item ...the great sweep of the bay.
	\end{itemize}
}
\item verb \\
If a person or group \textbf{sweeps} an election or \textbf{sweeps}  \textbf{to}  victory , they win the election easily.
 \textit{
	\begin{itemize}
	\item ...a man who's promised to make radical changes to benefit the poor has swept the
election.
	\item In both republics, centre-right parties swept to power.
	\item ...voters nostalgic for the free-spending policies of the 1980s swept his Socialists
back to power.
	\item ...a sweeping victory.
	\end{itemize}
}
\item countable noun \\
If someone makes a \textbf{sweep}  \textbf{of} a place, they search it, usually because they are looking for people who are hiding or for an illegal activity.
 \textit{
	\begin{itemize}
	\item Two of the soldiers swiftly began making a sweep of the premises.
	\item There may be periodic police 'sweeps' of crime in the area.
	\end{itemize}
}
\item singular noun \\
If you refer to the \textbf{sweep}  \textbf{of} something, you are indicating that it includes a large number of different events,
qualities, or opinions.
 \textit{
	\begin{itemize}
	\item The charter brought accountability to the whole sweep of public services.
	\end{itemize}
}
\item  \\
 to sweep something under the carpet \textit{
	\begin{itemize}
	\end{itemize}
}
\item  \\
 a clean sweep \textit{
	\begin{itemize}
	\end{itemize}
}
\item  \\
 to sweep someone off their feet \textit{
	\begin{itemize}
	\end{itemize}
}
\end{enumerate}

\section*{verge}
{\large \color{blue}  verges  verging  verged  }
\subsection*{Explain}
\begin{enumerate}
\item  \\
 on the verge of \textit{
	\begin{itemize}
	\end{itemize}
}
\item countable noun \\
The \textbf{verge} of a road is a narrow piece of ground by the side of a road, which is usually covered with grass or flowers.
 \textit{
	\begin{itemize}
	\end{itemize}
}
\end{enumerate}

\section*{treat}
{\large \color{blue}  treats  treating  treated  }
\subsection*{Explain}
\begin{enumerate}
\item verb \\
If you \textbf{treat} someone or something in a particular  way , you behave towards them or deal with them in that way.
 \textit{
	\begin{itemize}
	\item Artie treated most women with indifference.
	\item Police say they're treating it as a case of attempted murder.
	\item He felt the press had never treated him fairly.
	\item The issues should be treated separately.
	\end{itemize}
}
\item verb \\
When a doctor or nurse  \textbf{treats} a patient or an illness , he or she tries to make the patient well again.
 \textit{
	\begin{itemize}
	\item Doctors treated her with aspirin.
	\item The boy was treated for a minor head wound.
	\item An experienced nurse treats all minor injuries.
	\end{itemize}
}
\item verb \\
If something \textbf{is treated}  \textbf{with} a particular substance, the substance is put onto or into it in order to clean it, to protect it, or to give it special  properties .
 \textit{
	\begin{itemize}
	\item About 70% of the cocoa acreage is treated with insecticide.
	\item It was many years before the city began to treat its sewage.
	\end{itemize}
}
\item verb \\
If you \textbf{treat} someone \textbf{to} something special which they will  enjoy , you buy it or arrange it for them.
 \textit{
	\begin{itemize}
	\item She was always treating him to ice cream.
	\item Tomorrow I'll treat myself to a day's gardening.
	\item If you want to treat yourself, the Malta Hilton offers high international standards.
	\end{itemize}
}
\item countable noun \\
If you give someone a \textbf{treat} , you buy or arrange something special for them which they will enjoy.
 \textit{
	\begin{itemize}
	\item Lettie had never yet failed to return from town without some special treat for him.
	\end{itemize}
}
\item singular noun \\
If you say that something is your \textbf{treat} , you mean that you are paying for it as a treat for someone else.
 \textit{
	\begin{itemize}
	\end{itemize}
}
\item  \\
 a treat \textit{
	\begin{itemize}
	\end{itemize}
}
\end{enumerate}

\section*{wallet}
{\large \color{blue}  wallets  }
\subsection*{Explain}
\begin{enumerate}
\item countable noun \\
A \textbf{wallet} is a small flat folded case, usually made of leather or plastic , in which you can keep banknotes and credit  cards .
 \textit{
	\begin{itemize}
	\end{itemize}
}
\end{enumerate}

\section*{whisper}
{\large \color{blue}  whispers  whispering  whispered  }
\subsection*{Explain}
\begin{enumerate}
\item verb \\
When you \textbf{whisper} , you say something very quietly , using your breath  rather than your throat , so that only one person can  hear you.
 \textbf{Whisper} is also a noun .
 \textit{
	\begin{itemize}
	\item 'Keep your voice down,' I whispered.
	\item She sat on Rossi's knee as he whispered in her ear.
	\item He whispered the message to David.
	\item Somebody whispered that films like that were illegal.
	\item She whispered his name.
	\item Men were talking in whispers in every office.
	\end{itemize}
}
\item verb \\
If people \textbf{whisper} about a piece of information , they talk about it, although it might not be true or accurate , or might be a secret.
 \textbf{Whisper} is also a noun.
 \textit{
	\begin{itemize}
	\item We hit it off so well that everyone started whispering about us.
	\item It is whispered that he intended to resign.
	\item But don't whisper a word of that.
	\item I've heard a whisper that the Bishop intends to leave.
	\end{itemize}
}
\item verb \\
If something \textbf{whispers} , it makes a low quiet sound which can only just be heard.
 \textbf{Whisper} is also a noun.
 \textit{
	\begin{itemize}
	\item The trees sway and whisper in the wind.
	\item The car's tires whispered through the puddles.
	\item ...whispering ceiling fans.
	\item They heard the whisper of leaves.
	\end{itemize}
}
\end{enumerate}

\section*{ambition}
{\large \color{blue}  ambitions  }
\subsection*{Explain}
\begin{enumerate}
\item countable noun \\
If you have an \textbf{ambition}  \textbf{to} do or achieve something, you want very much to do it or achieve it.
 \textit{
	\begin{itemize}
	\item His ambition is to sail round the world.
	\item He harboured ambitions of becoming a Tory MP.
	\end{itemize}
}
\item uncountable noun \\
\textbf{Ambition} is the desire to be successful , rich , or powerful .
 \textit{
	\begin{itemize}
	\item Even when I was young I never had any ambition.
	\item ...a mixture of ambition and ruthlessness.
	\end{itemize}
}
\end{enumerate}

\section*{arrange}
{\large \color{blue}  arranges  arranging  arranged  }
\subsection*{Explain}
\begin{enumerate}
\item verb \\
If you \textbf{arrange} an event or meeting , you make plans for it to happen .
 \textit{
	\begin{itemize}
	\item She arranged an appointment for Friday afternoon at four-fifteen.
	\item This time it was a friend ringing to try to arrange a fishing trip in Scotland.
	\item The prime minister threw the carefully arranged welcome into chaos.
	\end{itemize}
}
\item verb \\
If you \textbf{arrange} with someone \textbf{to} do something, you make plans with them to do it.
 \textit{
	\begin{itemize}
	\item I've arranged to see him on Friday morning.
	\item It was arranged that the party would gather for lunch in the Royal Garden Hotel.
	\item He had arranged for the boxes to be stored until they could be collected.
	\end{itemize}
}
\item verb \\
If you \textbf{arrange} something \textbf{for} someone, you make it possible for them to have it or to do it.
 \textit{
	\begin{itemize}
	\item I will arrange for someone to take you round.
	\item The hotel manager will arrange for a baby-sitter.
	\item I've arranged your hotels for you.
	\item Transport is not included but can be arranged.
	\end{itemize}
}
\item verb \\
If you \textbf{arrange} things somewhere , you place them in a particular position, usually in order to make them look  attractive or tidy .
 \textit{
	\begin{itemize}
	\item When she has a little spare time she enjoys arranging dried flowers.
	\item He started to arrange the books in piles.
	\item A number of seats have been arranged in front of the painting.
	\end{itemize}
}
\item verb \\
If a piece of music \textbf{is arranged}  \textbf{by} someone, it is changed or adapted so that it is suitable for particular instruments or voices , or for a particular performance.
 \textit{
	\begin{itemize}
	\item The songs were arranged by another well-known bass player.
	\end{itemize}
}
\end{enumerate}

\section*{animal}
{\large \color{blue}  animals  }
\subsection*{Explain}
\begin{enumerate}
\item countable noun \\
An \textbf{animal} is a living creature such as a dog , lion , or rabbit , rather than a bird, fish, insect , or human being.
 \textit{
	\begin{itemize}
	\item He was attacked by wild animals.
	\item He had a real knowledge of animals, birds and flowers.
	\end{itemize}
}
\item countable noun \\
Any living creature other than a human being can be referred to as an \textbf{animal} .
 \textit{
	\begin{itemize}
	\item Language is something which fundamentally distinguishes humans from animals.
	\item ...a habitat for plants and animals.
	\end{itemize}
}
\item countable noun \\
Any living creature, including a human being, can be referred to as an \textbf{animal} .
 \textit{
	\begin{itemize}
	\item Watch any young human being, or any other young animal.
	\end{itemize}
}
\item adjective \\
\textbf{Animal} products come from animals rather than from plants.
 \textit{
	\begin{itemize}
	\item The illegal trade in animal products continues to flourish.
	\item Cut down on animal fats found in red meat, hard cheeses and so on.
	\end{itemize}
}
\item countable noun \\
If you say that someone is an \textbf{animal} , you find their behaviour disgusting or very unpleasant .
 \textit{
	\begin{itemize}
	\item This man is an animal, a beast.
	\item He was an animal in his younger days.
	\end{itemize}
}
\item adjective \\
\textbf{Animal} qualities, feelings , or abilities relate to someone's physical nature and instincts rather than to their mind .
 \textit{
	\begin{itemize}
	\item There was no doubting the animal magnetism of the man.
	\item You feel an animal panic to run and hide.
	\end{itemize}
}
\item countable noun \\
You can refer to someone as a particular type of \textbf{animal} in order to say what their interests are or what their typical behaviour is.
 \textit{
	\begin{itemize}
	\item You're quite a party animal aren't you, out there every night.
	\item The entrepreneur at twenty-five is a different animal from the entrepreneur at fifty.
	\end{itemize}
}
\end{enumerate}

\section*{avert}
{\large \color{blue}  averts  averting  averted  }
\subsection*{Explain}
\begin{enumerate}
\item verb \\
If you \textbf{avert} something unpleasant , you prevent it from happening .
 \textit{
	\begin{itemize}
	\item Talks with the teachers' union over the weekend have averted a strike.
	\item A fresh tragedy was narrowly averted yesterday.
	\end{itemize}
}
\item verb \\
If you \textbf{avert} your eyes or gaze  \textbf{from} someone or something, you look away from them.
 \textit{
	\begin{itemize}
	\item He avoids any eye contact, quickly averting his gaze when anyone approaches.
	\item He kept his eyes averted.
	\end{itemize}
}
\end{enumerate}

\section*{appraisal}
{\large \color{blue}  appraisals  }
\subsection*{Explain}
\begin{enumerate}
\item variable noun \\
If you make an \textbf{appraisal}  \textbf{of} something, you consider it carefully and form an opinion about it.
 \textit{
	\begin{itemize}
	\item What is needed in such cases is a calm appraisal of the situation.
	\item Self-appraisal is never easy.
	\end{itemize}
}
\item variable noun \\
\textbf{Appraisal} is the official or formal assessment of the strengths and weaknesses of someone or something. Appraisal often involves observation or some kind of testing .
 \textit{
	\begin{itemize}
	\item Staff problems should be addressed through training and appraisals.
	\item ...an appraisal of your financial standing.
	\end{itemize}
}
\end{enumerate}

\section*{boom}
{\large \color{blue}  booms  booming  boomed  }
\subsection*{Explain}
\begin{enumerate}
\item countable noun \\
If there is a \textbf{boom} in the economy , there is an increase in economic activity, for example in the amount of things that
are being bought and sold.
 \textit{
	\begin{itemize}
	\item An economic boom followed, especially in housing and construction.
	\item The 1980s were indeed boom years.
	\item ...the cycle of boom and bust which has damaged us for 40 years.
	\end{itemize}
}
\item countable noun \\
A \textbf{boom}  \textbf{in} something is an increase in its amount, frequency , or success .
 \textit{
	\begin{itemize}
	\item The boom in the sport's popularity has meant more calls for stricter safety regulations.
	\item Public transport has not been able to cope adequately with the travel boom.
	\end{itemize}
}
\item verb \\
If the economy or a business \textbf{is booming} , the amount of things being bought or sold is increasing.
 \textit{
	\begin{itemize}
	\item When the economy is booming, people buy new cars.
	\item Sales are booming.
	\item It has a booming tourist industry.
	\end{itemize}
}
\item countable noun \\
On a boat , \textbf{the}  \textbf{boom} is the long pole which is attached to the bottom of the sail and to the mast and which you move when you want to alter the direction in which you are sailing .
 \textit{
	\begin{itemize}
	\end{itemize}
}
\item countable noun \\
A \textbf{boom} is a large floating barrier that is used for stopping oil that has spilled from spreading .
 \textit{
	\begin{itemize}
	\end{itemize}
}
\item verb \\
When something such as someone's voice , a cannon , or a big  drum  \textbf{booms} , it makes a loud , deep sound that lasts for several seconds.
 \textbf{Boom out} means the same as boom .
 \textbf{Boom} is also a noun .
 \textit{
	\begin{itemize}
	\item 'Ladies,' boomed Helena, without a microphone, 'we all know why we're here tonight.'
	\item Thunder boomed like battlefield cannons over Crooked Mountain.
	\item Music boomed out from loudspeakers.
	\item A megaphone boomed out, 'This is the police.'
	\item He turned his sightless eyes their way and boomed out a greeting.
	\item The stillness of night was broken by the boom of a cannon.
	\end{itemize}
}
\end{enumerate}

\section*{artery}
{\large \color{blue}  arteries  }
\subsection*{Explain}
\begin{enumerate}
\item countable noun \\
\textbf{Arteries} are the tubes in your body that carry blood from your heart to the rest of your body. Compare  vein .
 \textit{
	\begin{itemize}
	\item ...patients suffering from blocked arteries.
	\end{itemize}
}
\item countable noun \\
You can  refer to an important  main  route within a complex road, railway , or river system as an \textbf{artery} .
 \textit{
	\begin{itemize}
	\item Clarence Street was one of the north-bound arteries of the central business district.
	\item ...the point where the Ohio River, itself a great artery, joins the Mississippi.
	\end{itemize}
}
\end{enumerate}

\section*{carry}
{\large \color{blue}  carries  carrying  carried  }
\subsection*{Explain}
\begin{enumerate}
\item verb \\
If you \textbf{carry} something, you take it with you, holding it so that it does not touch the ground.
 \textit{
	\begin{itemize}
	\item He was carrying a briefcase.
	\item He carried the plate through to the dining room.
	\item She carried her son to the car.
	\item If your job involves a lot of paperwork, you're going to need something to carry
it all in.
	\end{itemize}
}
\item verb \\
If you \textbf{carry} something, you have it with you wherever you go.
 \textit{
	\begin{itemize}
	\item You have to carry a bleeper so that they can call you in at any time.
	\end{itemize}
}
\item verb \\
If something \textbf{carries} a person or thing somewhere , it takes them there.
 \textit{
	\begin{itemize}
	\item Flowers are designed to attract insects which then carry the pollen from plant to
plant.
	\item The delegation was carrying a message of thanks to the president.
	\item The ship could carry seventy passengers.
	\end{itemize}
}
\item verb \\
If a person or animal \textbf{is carrying} a disease, they are infected with it and can pass it on to other people or animals.
 \textit{
	\begin{itemize}
	\item He has been carrying the HIV virus for 12 years.
	\item Frogs eat pests which destroy crops and carry diseases.
	\end{itemize}
}
\item verb \\
If an action or situation has a particular quality or consequence , you can say that it \textbf{carries} it.
 \textit{
	\begin{itemize}
	\item Check that any medication you're taking carries no risk for your developing baby.
	\item However, professionalism carries a price.
	\end{itemize}
}
\item verb \\
If a quality or advantage  \textbf{carries} someone into a particular position or through a difficult situation, it helps them to achieve that position or deal with that situation.
 \textit{
	\begin{itemize}
	\item He had the ruthless streak necessary to carry him into the Cabinet.
	\item The warmth and strength of their relationship carried them through difficult times.
	\end{itemize}
}
\item verb \\
If you \textbf{carry} an idea or a method to a particular extent , you use or develop it to that extent.
 \textit{
	\begin{itemize}
	\item It's not such a new idea, but I carried it to extremes.
	\item We could carry that one step further by taking the same genes and putting them into
another crop.
	\end{itemize}
}
\item verb \\
If a newspaper or poster  \textbf{carries} a picture or a piece of writing, it contains it or displays it.
 \textit{
	\begin{itemize}
	\item Several papers carry the photograph of Mr Anderson.
	\end{itemize}
}
\item verb \\
In a debate , if a proposal or motion \textbf{is carried} , a majority of people vote in favour of it.
 \textit{
	\begin{itemize}
	\item A motion backing its economic policy was carried by 322 votes to 296.
	\end{itemize}
}
\item verb \\
If a crime  \textbf{carries} a particular punishment , a person who is found guilty of that crime will receive that punishment.
 \textit{
	\begin{itemize}
	\item It was a crime of espionage and carried the death penalty.
	\end{itemize}
}
\item verb \\
If a sound \textbf{carries} , it can be heard a long way away.
 \textit{
	\begin{itemize}
	\item Even in this stillness Leaphorn doubted if the sound would carry far.
	\end{itemize}
}
\item verb \\
If a candidate or party \textbf{carries} a state or area, they win the election in that state or area.
 \textit{
	\begin{itemize}
	\item At that time George W. Bush carried the state with 56 percent of the vote.
	\end{itemize}
}
\item verb \\
If you \textbf{carry}  \textbf{yourself} in a particular way, you walk and move in that way.
 \textit{
	\begin{itemize}
	\item They carried themselves with great pride and dignity.
	\end{itemize}
}
\item verb \\
If a woman \textbf{is carrying} a child, she is pregnant.
 \textit{
	\begin{itemize}
	\item She claims to be able to predict whether you're carrying a boy or a girl.
	\end{itemize}
}
\item  \\
 to get/be carried away \textit{
	\begin{itemize}
	\end{itemize}
}
\item  \\
 to carry all before you \textit{
	\begin{itemize}
	\end{itemize}
}
\end{enumerate}

\section*{bible}
{\large \color{blue}  Bibles  }
\subsection*{Explain}
\begin{enumerate}
\item proper noun \\
\textbf{The}  \textbf{Bible} is the holy book on which the Jewish and Christian religions are based.
 \textit{
	\begin{itemize}
	\end{itemize}
}
\item countable noun \\
A \textbf{Bible} is a copy of the Bible.
 \textit{
	\begin{itemize}
	\end{itemize}
}
\item countable noun \\
If someone describes a book or magazine about their job or interest as their \textbf{bible} , they mean that it is the best and most useful book about the subject.
 \textit{
	\begin{itemize}
	\end{itemize}
}
\end{enumerate}

\section*{collect}
{\large \color{blue}  collects  collecting  collected  }
\subsection*{Explain}
\begin{enumerate}
\item verb \\
If you \textbf{collect} a number of things, you bring them together from several places or from several people.
 \textit{
	\begin{itemize}
	\item Two young girls were collecting firewood.
	\item Elizabeth had been collecting snails for a school project.
	\item 1.5 million signatures have been collected.
	\end{itemize}
}
\item verb \\
If you \textbf{collect} things, such as stamps or books, as a hobby, you get a large number of them over a period of time because they interest you.
 \textit{
	\begin{itemize}
	\item I used to collect stamps.
	\item One of Tony's hobbies was collecting rare coins.
	\end{itemize}
}
\item verb \\
When you \textbf{collect} someone or something, you go and get them from the place where they are waiting for you or have been left for you.
 \textit{
	\begin{itemize}
	\item David always collects Alistair from school on Wednesdays.
	\item She had just collected her pension from the post office.
	\item After collecting the cash, the kidnapper made his escape down the disused railway
line.
	\end{itemize}
}
\item verb \\
If a substance  \textbf{collects}  somewhere , or if something \textbf{collects} it, it keeps  arriving over a period of time and is held in that place or thing.
 \textit{
	\begin{itemize}
	\item Methane gas does collect in the mines around here.
	\item ...water tanks which collect rainwater from the house roof.
	\end{itemize}
}
\item verb \\
If something \textbf{collects}  light , energy , or heat , it attracts it.
 \textit{
	\begin{itemize}
	\item Like a telescope, it has a curved mirror to collect the sunlight.
	\end{itemize}
}
\item verb \\
If you \textbf{collect}  \textbf{for} a charity or \textbf{for} a present for someone, you ask people to give you money for it.
 \textit{
	\begin{itemize}
	\item Are you collecting for charity?
	\item They collected donations for a fund to help military families.
	\end{itemize}
}
\item verb \\
If you \textbf{collect}  \textbf{yourself} or \textbf{collect} your thoughts , you make an effort to calm yourself or prepare yourself mentally.
 \textit{
	\begin{itemize}
	\item She paused for a moment to collect herself.
	\item He was grateful for a chance to relax and collect his thoughts.
	\end{itemize}
}
\item adjective \\
A \textbf{collect call} is a telephone call that is paid for by the person receiving it, not the person making it.
 \textit{
	\begin{itemize}
	\item She received a collect phone call from Alaska.
	\end{itemize}
}
\end{enumerate}

\section*{boss}
{\large \color{blue}  bosses  bossing  bossed  }
\subsection*{Explain}
\begin{enumerate}
\item countable noun \\
Your \textbf{boss} is the person in charge of the organization or department where you work.
 \textit{
	\begin{itemize}
	\item He cannot stand his boss.
	\item Occasionally I have to go and ask the boss for a rise.
	\end{itemize}
}
\item countable noun \\
If you are \textbf{the}  \textbf{boss} in a group or relationship , you are the person who makes all the decisions .
 \textit{
	\begin{itemize}
	\item He thinks he's the boss.
	\end{itemize}
}
\item verb \\
If you say that someone \textbf{bosses} you, you mean that they keep telling you what to do in a way that is irritating .
 \textbf{Boss around} , or in British English \textbf{boss about} , means the same as boss .
 \textit{
	\begin{itemize}
	\item We cannot boss them into doing more.
	\item 'You are not to boss me!' she shouted.
	\item He started bossing people around and I didn't like what was happening.
	\end{itemize}
}
\item  \\
 be one's own boss \textit{
	\begin{itemize}
	\end{itemize}
}
\end{enumerate}

\section*{contaminate}
{\large \color{blue}  contaminates  contaminating  contaminated  }
\subsection*{Explain}
\begin{enumerate}
\item verb \\
If something \textbf{is contaminated}  \textbf{by} waste, dirt , chemicals, or radiation, it is made dirty or harmful .
 \textit{
	\begin{itemize}
	\item Have any fish been contaminated in the Arctic Ocean?
	\item ...vast tracts of empty land, much of it contaminated by years of army activity.
	\end{itemize}
}
\end{enumerate}

\section*{broom}
{\large \color{blue}  brooms  }
\subsection*{Explain}
\begin{enumerate}
\item countable noun \\
A \textbf{broom} is a kind of brush with a long handle. You use a broom for sweeping the floor .
 \textit{
	\begin{itemize}
	\end{itemize}
}
\item uncountable noun \\
\textbf{Broom} is a wild  bush with a lot of tiny  yellow flowers.
 \textit{
	\begin{itemize}
	\end{itemize}
}
\end{enumerate}

\section*{convey}
{\large \color{blue}  conveys  conveying  conveyed  }
\subsection*{Explain}
\begin{enumerate}
\item verb \\
To \textbf{convey} information or feelings means to cause them to be known or understood by someone.
 \textit{
	\begin{itemize}
	\item When I returned home, I tried to convey the wonder of this machine to my partner.
	\item In every one of her pictures she conveys a sense of immediacy.
	\item He also conveyed his views and the views of the bureaucracy.
	\end{itemize}
}
\item verb \\
To \textbf{convey} someone or something to a place means to carry or transport them there.
 \textit{
	\begin{itemize}
	\item The railway company extended a branch line to Brightlingsea to convey fish direct
to Billingsgate.
	\end{itemize}
}
\end{enumerate}

\section*{childhood}
{\large \color{blue}  childhoods  }
\subsection*{Explain}
\begin{enumerate}
\item variable noun \\
A person's \textbf{childhood} is the period of their life when they are a child.
 \textit{
	\begin{itemize}
	\item She had a happy childhood.
	\item He was remembering a story heard in childhood.
	\item ...childhood illnesses.
	\end{itemize}
}
\end{enumerate}

\section*{cruise}
{\large \color{blue}  cruises  cruising  cruised  }
\subsection*{Explain}
\begin{enumerate}
\item countable noun \\
A \textbf{cruise} is a holiday during which you travel on a ship or boat and visit a number of places.
 \textit{
	\begin{itemize}
	\item He and his wife were planning to go on a world cruise.
	\item The next stop on this cruise is likely to be in Cornwall.
	\end{itemize}
}
\item verb \\
If you \textbf{cruise} a sea, river, or canal , you travel around it or along it on a cruise.
 \textit{
	\begin{itemize}
	\item She wants to cruise the canals of France in a barge.
	\item Try cruising around the Greek islands in a traditional fishing boat.
	\end{itemize}
}
\item verb \\
If a car , ship, or aircraft \textbf{cruises}  somewhere , it moves there at a steady  comfortable speed.
 \textit{
	\begin{itemize}
	\item A black and white police car cruised past.
	\end{itemize}
}
\item verb \\
If a team or sports player \textbf{cruises to}  victory , they win  easily .
 \textit{
	\begin{itemize}
	\item Williams looked in awesome form as she cruised to an easy victory.
	\end{itemize}
}
\item verb \\
If a gay man \textbf{is cruising} , he is searching in public places for a sexual partner.
 \textit{
	\begin{itemize}
	\end{itemize}
}
\end{enumerate}

\section*{christmas}
{\large \color{blue}  Christmases  }
\subsection*{Explain}
\begin{enumerate}
\item variable noun \\
\textbf{Christmas} is a Christian festival when the birth of Jesus Christ is celebrated . Christmas is celebrated on the 25th of December .
 \textit{
	\begin{itemize}
	\item The day after Christmas is generally a busy one for retailers.
	\item Merry Christmas, Mom.
	\end{itemize}
}
\item variable noun \\
\textbf{Christmas} is the period of several days around and including Christmas Day.
 \textit{
	\begin{itemize}
	\item During the Christmas holidays there's a tremendous amount of traffic between the
Northeast and Florida.
	\item He'll be in the hospital over Christmas, so we'll be spending our Christmas Day there.
	\end{itemize}
}
\end{enumerate}

\section*{decide}
{\large \color{blue}  decides  deciding  decided  }
\subsection*{Explain}
\begin{enumerate}
\item verb \\
If you \textbf{decide} to do something, you choose to do it, usually after you have thought carefully about the other possibilities .
 \textit{
	\begin{itemize}
	\item She decided to do a secretarial course.
	\item He has decided that he doesn't want to embarrass the movement and will therefore
step down.
	\item The house needed totally rebuilding, so we decided against buying it.
	\item I had a cold and couldn't decide whether to go to work or not.
	\item Think about it very carefully before you decide.
	\end{itemize}
}
\item verb \\
If a person or group of people \textbf{decides} something, they choose what something should be like or how a particular problem should be solved .
 \textit{
	\begin{itemize}
	\item She was still young, he said, and that would be taken into account when deciding
her sentence.
	\item This is an issue that should be decided by local and metropolitan government.
	\end{itemize}
}
\item verb \\
If an event or fact  \textbf{decides} something, it makes it certain that a particular choice  will be made or that there will be a particular result.
 \textit{
	\begin{itemize}
	\item The goal that decided the match came just before the interval.
	\item The results will decide if he will win a place at a good university.
	\item Luck is certainly one deciding factor.
	\end{itemize}
}
\item verb \\
If you \textbf{decide} that something is true , you form that opinion about it after considering the facts.
 \textit{
	\begin{itemize}
	\item He decided Franklin must be suffering from a bad cold.
	\item I couldn't decide whether he was incredibly brave or just insane.
	\end{itemize}
}
\item verb \\
If something \textbf{decides} you to do something, it is the reason that causes you to choose to do it.
 \textit{
	\begin{itemize}
	\item The banning of his play decided him to write about censorship.
	\item I don't know what finally decided her, but she agreed.
	\end{itemize}
}
\end{enumerate}

\section*{clasp}
{\large \color{blue}  clasps  clasping  clasped  }
\subsection*{Explain}
\begin{enumerate}
\item verb \\
If you \textbf{clasp} someone or something, you hold them tightly in your hands or arms .
 \textbf{Clasp} is also a noun .
 \textit{
	\begin{itemize}
	\item She clasped the children to her.
	\item He paced the corridor, hands clasped behind his back.
	\item With one last clasp of his hand, she left him and went to her usual chair.
	\end{itemize}
}
\item countable noun \\
A \textbf{clasp} is a small device that fastens something.
 \textit{
	\begin{itemize}
	\item ...the clasp of her handbag.
	\end{itemize}
}
\end{enumerate}

\section*{defend}
{\large \color{blue}  defends  defending  defended  }
\subsection*{Explain}
\begin{enumerate}
\item verb \\
If you \textbf{defend} someone or something, you take action in order to protect them.
 \textit{
	\begin{itemize}
	\item Every man who could fight was now committed to defend the ridge.
	\item His courage in defending religious and civil rights inspired many outside the church.
	\item They would have killed him if he had not defended himself.
	\end{itemize}
}
\item verb \\
If you \textbf{defend} someone or something when they have been criticized , you argue in support of them.
 \textit{
	\begin{itemize}
	\item Matt defended all of Clarence's decisions, right or wrong.
	\item The author defended herself against charges of plagiarism.
	\item Police chiefs strongly defended police conduct against a wave of criticism.
	\end{itemize}
}
\item verb \\
When a lawyer  \textbf{defends} a person who has been accused of something, the lawyer argues on their behalf in a court of law that the charges are not true .
 \textit{
	\begin{itemize}
	\item ...a lawyer who defended political prisoners during the military regime.
	\item He has hired a lawyer to defend him against the allegations.
	\item Guy Powell, defending, told magistrates: 'It's a sad and disturbing case.'
	\end{itemize}
}
\item verb \\
When a sports  player  plays in the tournament which they won the previous time it was held , you can  say that they \textbf{are defending} their title.
 \textit{
	\begin{itemize}
	\item The reigning champion expects to defend her title successfully next year.
	\item India had to struggle to beat defending champions South Korea 2-0.
	\end{itemize}
}
\end{enumerate}

\section*{contrast}
{\large \color{blue}  contrasts  contrasting  contrasted  }
\subsection*{Explain}
\begin{enumerate}
\item variable noun \\
A \textbf{contrast} is a great difference between two or more things which is clear when you compare them.
 \textit{
	\begin{itemize}
	\item ...the contrast between town and country.
	\item The two visitors provided a startling contrast in appearance.
	\item Silk was used with wool for contrast.
	\end{itemize}
}
\item  \\
 by contrast/in contrast/in contrast to sth \textit{
	\begin{itemize}
	\end{itemize}
}
\item  \\
 in contrast \textit{
	\begin{itemize}
	\end{itemize}
}
\item countable noun \\
If one thing is a \textbf{contrast}  \textbf{to} another, it is very different from it.
 \textit{
	\begin{itemize}
	\item The boy's room is a complete contrast to the guest room.
	\item ...a country of great contrasts.
	\end{itemize}
}
\item verb \\
If you \textbf{contrast} one thing \textbf{with} another, you point out or consider the differences between those things.
 \textit{
	\begin{itemize}
	\item She contrasted the situation then with the present crisis.
	\item Contrast that approach with what goes on in most organizations.
	\item In this section we contrast four possible broad approaches.
	\end{itemize}
}
\item verb \\
If one thing \textbf{contrasts}  \textbf{with} another, it is very different from it.
 \textit{
	\begin{itemize}
	\item Johnson's easy charm contrasted sharply with the prickliness of his boss.
	\item Paint the wall in a contrasting colour.
	\end{itemize}
}
\item uncountable noun \\
\textbf{Contrast} is the degree of difference between the darker and lighter parts of a photograph , television picture , or painting.
 \textit{
	\begin{itemize}
	\item ...a television with brighter colours, better contrast, and digital sound.
	\end{itemize}
}
\end{enumerate}

\section*{deliver}
{\large \color{blue}  delivers  delivering  delivered  }
\subsection*{Explain}
\begin{enumerate}
\item verb \\
If you \textbf{deliver} something somewhere , you take it there.
 \textit{
	\begin{itemize}
	\item The Canadians plan to deliver more food to southern Somalia.
	\item The spy returned to deliver a second batch of classified documents.
	\item We were told the pizza would be delivered in 20 minutes.
	\end{itemize}
}
\item verb \\
If you \textbf{deliver} something that you have promised to do, make, or produce, you do, make, or produce it.
 \textit{
	\begin{itemize}
	\item They have yet to show that they can really deliver working technologies.
	\item ... proving they could deliver the vote in their areas.
	\item We don't promise what we can't deliver.
	\end{itemize}
}
\item verb \\
If you \textbf{deliver} a person or thing into someone's care , you give them responsibility for that person or thing.
 \textit{
	\begin{itemize}
	\item Mrs Montgomery was delivered into Mr Hinchcliffe's care.
	\item David delivered Holly gratefully into the woman's outstretched arms.
	\item He was led in in handcuffs and delivered over to me.
	\end{itemize}
}
\item verb \\
If you \textbf{deliver} a lecture or speech, you give it in public .
 \textit{
	\begin{itemize}
	\item The president will deliver a speech about schools.
	\item It is shocking that only one woman has delivered the lecture in 44 years.
	\end{itemize}
}
\item verb \\
When someone \textbf{delivers} a baby , they help the woman who is giving birth to the baby.
 \textit{
	\begin{itemize}
	\item Her husband had to deliver the baby himself.
	\end{itemize}
}
\item verb \\
If someone \textbf{delivers} a blow to someone else, they hit them.
 \textit{
	\begin{itemize}
	\item Those blows to the head could have been delivered by a woman.
	\end{itemize}
}
\item verb \\
If someone \textbf{delivers} you from something, they rescue or save you from it.
 \textit{
	\begin{itemize}
	\item I have given thanks to God for delivering me from that pain.
	\end{itemize}
}
\end{enumerate}

\section*{counterpart}
{\large \color{blue}  counterparts  }
\subsection*{Explain}
\begin{enumerate}
\item countable noun \\
Someone's or something's \textbf{counterpart} is another person or thing that has a similar function or position in a different
place.
 \textit{
	\begin{itemize}
	\item The Foreign Secretary telephoned his Italian counterpart to protest.
	\item The Finnish organization was very different from that of its counterparts in the
rest of the Nordic region.
	\end{itemize}
}
\end{enumerate}

\section*{demonstrate}
{\large \color{blue}  demonstrates  demonstrating  demonstrated  }
\subsection*{Explain}
\begin{enumerate}
\item verb \\
To \textbf{demonstrate} a fact  means to make it clear to people.
 \textit{
	\begin{itemize}
	\item The study also demonstrated a direct link between obesity and mortality.
	\item You have to demonstrate that you are reliable.
	\item They are anxious to demonstrate to the voters that they have practical policies.
	\item He's demonstrated how a campaign based on domestic issues can move votes.
	\end{itemize}
}
\item verb \\
If you \textbf{demonstrate} a particular skill , quality, or feeling , you show by your actions that you have it.
 \textit{
	\begin{itemize}
	\item Have they, for example, demonstrated a commitment to democracy?
	\item The government's going to great lengths to demonstrate its military might.
	\end{itemize}
}
\item verb \\
When people \textbf{demonstrate} , they march or gather  somewhere to show their opposition to something or their support for something.
 \textit{
	\begin{itemize}
	\item 30,000 angry farmers demonstrated against possible cuts in subsidies.
	\item In the cities vast crowds have been demonstrating for change.
	\item Thousands of people demonstrated outside the parliament building.
	\end{itemize}
}
\item verb \\
If you \textbf{demonstrate} something, you show people how it works or how to do it.
 \textit{
	\begin{itemize}
	\item The company demonstrated an app for surgeons that showed X-rays on the screen.
	\item He flew the prototype to West Raynham to demonstrate it to a group of senior officers.
	\item A style consultant will demonstrate how to dress to impress.
	\end{itemize}
}
\end{enumerate}

\section*{cow}
{\large \color{blue}  cows  cowing  cowed  }
\subsection*{Explain}
\begin{enumerate}
\item countable noun \\
A \textbf{cow} is a large female animal that is kept on farms for its milk . People sometimes  refer to male and female animals of this species as \textbf{cows} .
 \textit{
	\begin{itemize}
	\item He kept a few dairy cows.
	\item Dad went out to milk the cows.
	\item ...a herd of cows.
	\end{itemize}
}
\item countable noun \\
Some female animals, including elephants and whales, are called  \textbf{cows} .
 \textit{
	\begin{itemize}
	\item ...a cow elephant.
	\end{itemize}
}
\item countable noun \\
If someone describes a woman as a \textbf{cow} , they dislike her and think that she is unpleasant or stupid .
 \textit{
	\begin{itemize}
	\end{itemize}
}
\item verb \\
If someone \textbf{is cowed} , they are made afraid , or made to behave in a particular  way because they have been frightened or badly  treated .
 \textit{
	\begin{itemize}
	\item The government, far from being cowed by these threats, has vowed to continue its
policy.
	\item ...cowing them into submission.
	\end{itemize}
}
\item  \\
 (do sthg) until the cows come home \textit{
	\begin{itemize}
	\end{itemize}
}
\end{enumerate}

\section*{design}
{\large \color{blue}  designs  designing  designed  }
\subsection*{Explain}
\begin{enumerate}
\item verb \\
When someone \textbf{designs} a garment , building, machine, or other object, they plan it and make a detailed drawing of it from which it can be built or made.
 \textit{
	\begin{itemize}
	\item They wanted to design a machine that was both attractive and practical.
	\item ...men wearing specially designed boots.
	\end{itemize}
}
\item verb \\
When someone \textbf{designs} a survey , policy , or system, they plan and prepare it, and decide on all the details of it.
 \textit{
	\begin{itemize}
	\item We may be able to design a course to suit your particular needs.
	\item Computer security systems will be designed by independent technicians.
	\item A number of very well designed studies have been undertaken.
	\end{itemize}
}
\item uncountable noun \\
\textbf{Design} is the process and art of planning and making detailed drawings of something.
 \textit{
	\begin{itemize}
	\item He was a born mechanic with a flair for design.
	\item Most mobile robots are still in the design stage.
	\item She came to London in 1960 to study fashion design.
	\end{itemize}
}
\item uncountable noun \\
The \textbf{design} of something is the way in which it has been planned and made.
 \textit{
	\begin{itemize}
	\item These machines are constantly updated by improving the design of the computers.
	\item ...a new design of clock.
	\item The shoes were of good design and good quality.
	\item BMW is recalling 8,000 cars because of a design fault.
	\end{itemize}
}
\item countable noun \\
A \textbf{design} is a drawing which someone produces to show how they would like something to be built or made.
 \textit{
	\begin{itemize}
	\item They drew up the design for the house in a week.
	\end{itemize}
}
\item countable noun \\
A \textbf{design} is a pattern of lines, flowers, or shapes which is used to decorate something.
 \textit{
	\begin{itemize}
	\item Their range of tableware is decorated with a blackberry design.
	\item Many pictures have been based on simple geometric designs.
	\end{itemize}
}
\item countable noun \\
A \textbf{design} is a general plan or intention that someone has in their mind when they are doing
something.
 \textit{
	\begin{itemize}
	\item Is there some design in having him in the middle?
	\item His grand design of unifying the country was a battle against overwhelming odds.
	\end{itemize}
}
\item passive verb \\
If something \textbf{is designed} for a particular purpose, it is intended for that purpose.
 \textit{
	\begin{itemize}
	\item This project is designed to help landless people.
	\item It's not designed for anyone under age eighteen.
	\end{itemize}
}
\item  \\
 by design \textit{
	\begin{itemize}
	\end{itemize}
}
\item  \\
 have designs on sth \textit{
	\begin{itemize}
	\end{itemize}
}
\end{enumerate}

\section*{disturbance}
{\large \color{blue}  disturbances  }
\subsection*{Explain}
\begin{enumerate}
\item countable noun \\
A \textbf{disturbance} is an incident in which people behave violently in public.
 \textit{
	\begin{itemize}
	\item During the disturbance which followed, three Englishmen were hurt.
	\item ...the worst of last September's disturbances.
	\end{itemize}
}
\item uncountable noun \\
\textbf{Disturbance} means upsetting or disorganizing something which was previously in a calm and well-ordered state.
 \textit{
	\begin{itemize}
	\item The home would cause less disturbance to local residents than a school.
	\item The animals are very sensitive to disturbance and have never bred in captivity.
	\end{itemize}
}
\item variable noun \\
You can use \textbf{disturbance} to refer to a medical or psychological problem, when someone's body or mind is not working in the normal way.
 \textit{
	\begin{itemize}
	\item Poor educational performance is related to emotional disturbance.
	\item ...the treatment of certain heart rhythm disturbances.
	\end{itemize}
}
\end{enumerate}

\section*{determine}
{\large \color{blue}  determines  determining  determined  }
\subsection*{Explain}
\begin{enumerate}
\item verb \\
If a particular factor  \textbf{determines} the nature of a thing or event , it causes it to be of a particular kind .
 \textit{
	\begin{itemize}
	\item The size of the chicken pieces will determine the cooking time.
	\item Social status is largely determined by the occupation of the main breadwinner.
	\item What determines whether you are a career success or a failure?
	\end{itemize}
}
\item verb \\
To \textbf{determine} a fact  means to discover it as a result of investigation .
 \textit{
	\begin{itemize}
	\item The investigation will determine what really happened.
	\item Testing needs to be done to determine the long-term effects on humans.
	\item Science has determined that the risk is very small.
	\end{itemize}
}
\item verb \\
If you \textbf{determine} something, you decide it or settle it.
 \textit{
	\begin{itemize}
	\item The Baltic people have a right to determine their own future.
	\item The final wording had not yet been determined.
	\item My aim was first of all to determine what I should do next.
	\end{itemize}
}
\item verb \\
If you \textbf{determine}  \textbf{to} do something, you make a firm decision to do it.
 \textit{
	\begin{itemize}
	\item He determined to rescue his two countrymen.
	\item I determined that I would ask him outright.
	\end{itemize}
}
\end{enumerate}

\section*{expansion}
{\large \color{blue}  expansions  }
\subsection*{Explain}
\begin{enumerate}
\item variable noun \\
\textbf{Expansion} is the process of becoming  greater in size , number, or amount.
 \textit{
	\begin{itemize}
	\item ...the rapid expansion of private health insurance.
	\item ...a new period of economic expansion.
	\item The company has abandoned plans for further expansion.
	\end{itemize}
}
\end{enumerate}

\section*{distort}
{\large \color{blue}  distorts  distorting  distorted  }
\subsection*{Explain}
\begin{enumerate}
\item verb \\
If you \textbf{distort} a statement , fact, or idea , you report or represent it in an untrue way.
 \textit{
	\begin{itemize}
	\item The media distorts reality; categorises people as all good or all bad.
	\item The minister has said his remarks at the weekend have been distorted.
	\end{itemize}
}
\item verb \\
If something you can see or hear  \textbf{is distorted} or \textbf{distorts} , its appearance or sound is changed so that it seems  unclear .
 \textit{
	\begin{itemize}
	\item A painter may exaggerate or distort shapes and forms.
	\item His size was persistently distorted by the cartoonists.
	\item This caused the sound to distort.
	\end{itemize}
}
\end{enumerate}

\section*{fever}
{\large \color{blue}  fevers  }
\subsection*{Explain}
\begin{enumerate}
\item variable noun \\
If you have a \textbf{fever} when you are ill , your body temperature is higher than usual and your heart  beats faster.
 \textit{
	\begin{itemize}
	\item My Uncle Jim had a high fever.
	\item Symptoms of the disease include fever and weight loss.
	\end{itemize}
}
\item countable noun \\
A \textbf{fever} is extreme excitement or nervousness about something.
 \textit{
	\begin{itemize}
	\item Angie waited in a fever of excitement.
	\end{itemize}
}
\end{enumerate}

\section*{disturb}
{\large \color{blue}  disturbs  disturbing  disturbed  }
\subsection*{Explain}
\begin{enumerate}
\item verb \\
If you \textbf{disturb} someone, you interrupt what they are doing and upset them.
 \textit{
	\begin{itemize}
	\item I hope I'm not disturbing you.
	\item Find a quiet, warm, comfortable room where you won't be disturbed.
	\end{itemize}
}
\item verb \\
If something \textbf{disturbs} you, it makes you feel upset or worried .
 \textit{
	\begin{itemize}
	\item I dream about him, dreams so vivid that they disturb me for days.
	\end{itemize}
}
\item verb \\
If something \textbf{is disturbed} , its position or shape is changed.
 \textit{
	\begin{itemize}
	\item He'd placed his notes in the brown envelope. They hadn't been disturbed.
	\item She patted Mona, taking care not to disturb her costume.
	\end{itemize}
}
\item verb \\
If something \textbf{disturbs} a situation or atmosphere , it spoils it or causes trouble.
 \textit{
	\begin{itemize}
	\item Neither Baker nor Levy seemed eager to disturb the cordial atmosphere by discussing
more sensitive issues.
	\item What could possibly disturb such tranquility?
	\end{itemize}
}
\item  \\
 to disturb the peace \textit{
	\begin{itemize}
	\end{itemize}
}
\end{enumerate}

\section*{grease}
{\large \color{blue}  greases  greasing  greased  }
\subsection*{Explain}
\begin{enumerate}
\item uncountable noun \\
\textbf{Grease} is a thick, oily substance which is put on the moving parts of cars and other machines in order to make them work smoothly.
 \textit{
	\begin{itemize}
	\item ...grease-stained hands.
	\end{itemize}
}
\item verb \\
If you \textbf{grease} a part of a car, machine, or device , you put grease on it in order to make it work smoothly.
 \textit{
	\begin{itemize}
	\item I greased front and rear hubs and adjusted the brakes.
	\end{itemize}
}
\item uncountable noun \\
\textbf{Grease} is an oily substance that is produced by your skin.
 \textit{
	\begin{itemize}
	\item His hair is thick with grease.
	\end{itemize}
}
\item uncountable noun \\
\textbf{Grease} is animal fat that is produced by cooking  meat . You can use \textbf{grease} for cooking.
 \textit{
	\begin{itemize}
	\item He could smell the bacon grease.
	\end{itemize}
}
\item verb \\
If you \textbf{grease} a dish , you put a small amount of fat or oil around the inside of it in order to prevent  food  sticking to it during cooking.
 \textit{
	\begin{itemize}
	\item Grease two sturdy baking sheets and heat the oven to 400 degrees.
	\item Place the frozen rolls on a greased baking tray.
	\end{itemize}
}
\end{enumerate}

\section*{exclude}
{\large \color{blue}  excludes  excluding  excluded  }
\subsection*{Explain}
\begin{enumerate}
\item verb \\
If you \textbf{exclude} someone \textbf{from} a place or activity , you prevent them from entering it or taking part in it.
 \textit{
	\begin{itemize}
	\item The Academy excluded women from its classes.
	\item Many of the youngsters feel excluded.
	\end{itemize}
}
\item verb \\
If you \textbf{exclude} something that has some connection with what you are doing, you deliberately do not use it or consider it.
 \textit{
	\begin{itemize}
	\item They eat only plant foods, and exclude animal products from other areas of their
lives.
	\item ...plans to redraft and downgrade the role to exclude any involvement with the England
team.
	\end{itemize}
}
\item verb \\
To \textbf{exclude} a possibility  means to decide or prove that it is wrong and not worth considering.
 \textit{
	\begin{itemize}
	\item I cannot entirely exclude the possibility that some form of pressure was applied
to the neck.
	\item ...the pathological evidence, which does not exclude suicide.
	\end{itemize}
}
\item verb \\
To \textbf{exclude} something such as the sun's rays or harmful  germs means to prevent them physically from reaching or entering a particular place.
 \textit{
	\begin{itemize}
	\item This was intended to exclude the direct rays of the sun.
	\item They have spent $3 million building fences around the National Park to exclude such
pests.
	\end{itemize}
}
\end{enumerate}

\section*{guitar}
{\large \color{blue}  guitars  }
\subsection*{Explain}
\begin{enumerate}
\item variable noun \\
A \textbf{guitar} is a musical instrument with six strings and a long neck . You play the guitar by plucking or strumming the strings.
 \textit{
	\begin{itemize}
	\end{itemize}
}
\end{enumerate}

\section*{exert}
{\large \color{blue}  exerts  exerting  exerted  }
\subsection*{Explain}
\begin{enumerate}
\item verb \\
If someone or something \textbf{exerts} influence, authority, or pressure , they use it in a strong or determined way, especially in order to produce a particular effect.
 \textit{
	\begin{itemize}
	\item He exerted considerable influence on the thinking of the scientific community on
these issues.
	\item The cyst was causing swelling and exerting pressure on her brain.
	\end{itemize}
}
\item verb \\
If you \textbf{exert}  \textbf{yourself} , you make a great physical or mental effort, or work hard to do something.
 \textit{
	\begin{itemize}
	\item Youngsters get so absorbed that they don't realise how much they're exerting themselves.
	\item Do not exert yourself unnecessarily.
	\end{itemize}
}
\end{enumerate}

\section*{observe}
{\large \color{blue}  observes  observing  observed  }
\subsection*{Explain}
\begin{enumerate}
\item verb \\
If you \textbf{observe} a person or thing, you watch them carefully, especially in order to learn something about them.
 \textit{
	\begin{itemize}
	\item Stern also studies and observes the behaviour of babies.
	\item Are there any classes I could observe?
	\item Our sniper teams observed them manning an anti-aircraft gun.
	\end{itemize}
}
\item verb \\
If you \textbf{observe} someone or something, you see or notice them.
 \textit{
	\begin{itemize}
	\item In 1664 Hooke observed a reddish spot on the surface of the planet.
	\end{itemize}
}
\item verb \\
If you \textbf{observe} that something is the case , you make a remark or comment about it, especially when it is something you have
noticed and thought about a lot .
 \textit{
	\begin{itemize}
	\item When he spoke, it was to observe that the world was full of criminals.
	\item 'He is a fine young man,' observed Stephen.
	\end{itemize}
}
\item verb \\
If you \textbf{observe} something such as a law or custom, you obey it or follow it.
 \textit{
	\begin{itemize}
	\item Imposing speed restrictions is easy, but forcing motorists to observe them is trickier.
	\item The army was observing a ceasefire.
	\item American forces are observing Christmas quietly.
	\end{itemize}
}
\end{enumerate}

\section*{hen}
{\large \color{blue}  hens  }
\subsection*{Explain}
\begin{enumerate}
\item countable noun \\
A \textbf{hen} is a female chicken . People often keep hens in order to eat them or sell their eggs .
 \textit{
	\begin{itemize}
	\end{itemize}
}
\item countable noun \\
The female of any bird can be referred to as a \textbf{hen} .
 \textit{
	\begin{itemize}
	\item ...ostrich hens.
	\end{itemize}
}
\end{enumerate}

\section*{patrol}
{\large \color{blue}  patrols  patrolling  patrolled  }
\subsection*{Explain}
\begin{enumerate}
\item verb \\
When soldiers , police , or guards  \textbf{patrol} an area or building , they move around it in order to make sure that there is no trouble there.
 \textbf{Patrol} is also a noun .
 \textit{
	\begin{itemize}
	\item Prison officers continued to patrol the grounds within the jail.
	\item He failed to return from a patrol.
	\end{itemize}
}
\item  \\
 on patrol \textit{
	\begin{itemize}
	\end{itemize}
}
\item countable noun \\
A \textbf{patrol} is a group of soldiers or vehicles that are patrolling an area.
 \textit{
	\begin{itemize}
	\item Guerrillas attacked a patrol with hand grenades.
	\item ...a border patrol operating near the frontier.
	\end{itemize}
}
\end{enumerate}

\section*{kindergarten}
{\large \color{blue}  kindergartens  }
\subsection*{Explain}
\begin{enumerate}
\item countable noun \\
A \textbf{kindergarten} is an informal  kind of school for very young children, where they learn things by playing.
 \textit{
	\begin{itemize}
	\item She's in kindergarten now.
	\end{itemize}
}
\end{enumerate}

\section*{permit}
{\large \color{blue}  permits  permitting  permitted  }
\subsection*{Explain}
\begin{enumerate}
\item verb \\
If someone \textbf{permits} something, they allow it to happen . If they \textbf{permit} you \textbf{to} do something, they allow you to do it.
 \textit{
	\begin{itemize}
	\item He can let the court's decision stand and permit the execution.
	\item The guards permitted me to bring my camera.
	\item Employees are permitted to use the golf course during their free hours.
	\item No outside journalists have been permitted into the country.
	\item If they appear to be under 12, then the doorman is not allowed to permit them entry
to the film.
	\end{itemize}
}
\item countable noun \\
A \textbf{permit} is an official document which says that you may do something. For example you usually need a \textbf{permit} to work in a foreign country.
 \textit{
	\begin{itemize}
	\item The majority of foreign nationals working here have work permits.
	\item He has to apply for a permit, and we have to find him a job.
	\end{itemize}
}
\item verb \\
If a situation  \textbf{permits} something, it makes it possible for that thing to exist , happen, or be done or it provides the opportunity for it.
 \textit{
	\begin{itemize}
	\item He sets about creating an environment that doesn't just permit experiment, it encourages
it.
	\item Try to go out for a walk at lunchtime, if the weather permits.
	\item This method of cooking also permits heat to penetrate evenly from both sides.
	\end{itemize}
}
\item verb \\
If you \textbf{permit}  \textbf{yourself} something, you allow yourself to do something that you do not normally do or that you think you probably should not do.
 \textit{
	\begin{itemize}
	\item Captain Bowen permitted himself one cigar a day.
	\item Only once in his life had Douglas permitted himself to lose control of his emotions.
	\end{itemize}
}
\item  \\
 permit me \textit{
	\begin{itemize}
	\end{itemize}
}
\end{enumerate}

\section*{momentum}
{\large \color{blue}  }
\subsection*{Explain}
\begin{enumerate}
\item uncountable noun \\
If a process or movement  gains  \textbf{momentum} , it keeps  developing or happening more quickly and keeps becoming less likely to stop .
 \textit{
	\begin{itemize}
	\item This campaign is really gaining momentum.
	\item They are each anxious to maintain the momentum of the search for a solution.
	\end{itemize}
}
\item uncountable noun \\
In physics , \textbf{momentum} is the mass of a moving object multiplied by its speed in a particular direction .
 \textit{
	\begin{itemize}
	\end{itemize}
}
\end{enumerate}

\section*{pollute}
{\large \color{blue}  pollutes  polluting  polluted  }
\subsection*{Explain}
\begin{enumerate}
\item verb \\
To \textbf{pollute} water, air, or land means to make it dirty and dangerous to live in or to use, especially with poisonous chemicals or sewage .
 \textit{
	\begin{itemize}
	\item Heavy industry pollutes our rivers with noxious chemicals.
	\item A number of beaches in the region have been polluted by sewage pumped into the sea.
	\end{itemize}
}
\end{enumerate}

\section*{mother}
{\large \color{blue}  mothers  mothering  mothered  }
\subsection*{Explain}
\begin{enumerate}
\item countable noun \\
Your \textbf{mother} is the woman who gave birth to you. You can also  call someone your \textbf{mother} if she brings you up as if she was this woman. You can call your mother 'Mother'.
 \textit{
	\begin{itemize}
	\item She sat on the edge of her mother's bed.
	\item She's an English teacher and a mother of two children.
	\item Mother and child form a close attachment.
	\item I'm here, Mother.
	\end{itemize}
}
\item verb \\
If a woman \textbf{mothers} a child , she looks after it and brings it up, usually because she is its mother.
 \textit{
	\begin{itemize}
	\item Colleen had dreamed of mothering a large family.
	\end{itemize}
}
\item verb \\
If you \textbf{mother} someone, you treat them with great  care and affection, as if they were a small child.
 \textit{
	\begin{itemize}
	\item She felt a great need to mother him.
	\item Stop mothering me.
	\end{itemize}
}
\end{enumerate}

\section*{proceed}
{\large \color{blue}  proceeds  proceeding  proceeded  }
\subsection*{Explain}
\begin{enumerate}
\item verb \\
If you \textbf{proceed}  \textbf{to} do something, you do it, often after doing something else first.
 \textit{
	\begin{itemize}
	\item He proceeded to tell me of my birth.
	\item He picked up an orange and proceeded to eat it.
	\end{itemize}
}
\item verb \\
If you \textbf{proceed with} a course of action, you continue with it.
 \textit{
	\begin{itemize}
	\item The group proceeded with a march they knew would lead to bloodshed.
	\item The trial has been delayed until November because the defence is not ready to proceed.
	\end{itemize}
}
\item verb \\
If an activity, process, or event \textbf{proceeds} , it goes on and does not stop .
 \textit{
	\begin{itemize}
	\item The ideas were not new. Their development had proceeded steadily since the war.
	\item Efforts to reform the Interior Ministry have not yet proceeded very far.
	\end{itemize}
}
\item verb \\
If you \textbf{proceed} in a particular direction , you go in that direction.
 \textit{
	\begin{itemize}
	\item She climbed the steps and proceeded along the upstairs hallway.
	\item The freighter was allowed to proceed after satisfying them that it was not breaking
sanctions.
	\end{itemize}
}
\item plural noun \\
\textbf{The proceeds} of an event or activity are the money that has been obtained from it.
 \textit{
	\begin{itemize}
	\item They have not received any of the proceeds of the book's sales.
	\item The proceeds from the concert will go towards famine relief.
	\end{itemize}
}
\end{enumerate}

\section*{purchase}
{\large \color{blue}  purchases  purchasing  purchased  }
\subsection*{Explain}
\begin{enumerate}
\item verb \\
When you \textbf{purchase} something, you buy it.
 \textit{
	\begin{itemize}
	\item He purchased a ticket and went up on the top deck.
	\item Most of those shares were purchased from brokers.
	\end{itemize}
}
\item uncountable noun \\
The \textbf{purchase of} something is the act of buying it.
 \textit{
	\begin{itemize}
	\item Some of the receipts had been for the purchase of cars.
	\end{itemize}
}
\item countable noun \\
A \textbf{purchase} is something that you buy.
 \textit{
	\begin{itemize}
	\item She opened the tie box and looked at her purchase. It was silk, with maroon stripes.
	\end{itemize}
}
\item variable noun \\
If you get a \textbf{purchase}  \textbf{on} something, you manage to get a firm grip on it.
 \textit{
	\begin{itemize}
	\item I got a purchase on the rope and pulled.
	\item I couldn't get any purchase with the screwdriver on the damn screws.
	\end{itemize}
}
\end{enumerate}

\section*{nursery}
{\large \color{blue}  nurseries  }
\subsection*{Explain}
\begin{enumerate}
\item countable noun \\
A \textbf{nursery} is a place where children who are not old enough to go to school are looked after.
 \textit{
	\begin{itemize}
	\item This nursery will be able to cater for 29 children.
	\item Her company ran its own workplace nursery.
	\end{itemize}
}
\item variable noun \\
\textbf{Nursery} is a school for very young children.
 \textit{
	\begin{itemize}
	\item An affordable nursery education service is an essential basic amenity.
	\item ...a nursery teacher.
	\end{itemize}
}
\item countable noun \\
A \textbf{nursery} is a room in a family home in which the young children of the family sleep or play.
 \textit{
	\begin{itemize}
	\item He has painted murals in his children's nursery.
	\end{itemize}
}
\item countable noun \\
A \textbf{nursery} is a place where plants are grown in order to be sold .
 \textit{
	\begin{itemize}
	\item The garden, developed over the past 35 years, includes a nursery.
	\end{itemize}
}
\end{enumerate}

\section*{reap}
{\large \color{blue}  reaps  reaping  reaped  }
\subsection*{Explain}
\begin{enumerate}
\item verb \\
If you \textbf{reap} the benefits or the rewards of something, you enjoy the good things that happen as a result of it.
 \textit{
	\begin{itemize}
	\item You'll soon begin to reap the benefits of being fitter.
	\item We are not in this to reap immense financial rewards.
	\end{itemize}
}
\item verb \\
To \textbf{reap} crops means to cut them down and gather them.
 \textit{
	\begin{itemize}
	\item The painting depicted a group of peasants reaping a harvest of fruits and vegetables.
	\end{itemize}
}
\end{enumerate}

\section*{opponent}
{\large \color{blue}  opponents  }
\subsection*{Explain}
\begin{enumerate}
\item countable noun \\
A politician's \textbf{opponents} are other politicians who belong to a different party or who have different aims or policies .
 \textit{
	\begin{itemize}
	\item ...Mr Kennedy's opponent in the leadership contest.
	\item He described the detention without trial of political opponents as a cowardly act.
	\end{itemize}
}
\item countable noun \\
In a sporting contest, your \textbf{opponent} is the person who is playing against you.
 \textit{
	\begin{itemize}
	\item Norris twice knocked down his opponent in the early rounds of the fight.
	\item He's the best opponent I've come across this season, a great player.
	\end{itemize}
}
\item countable noun \\
The \textbf{opponents}  \textbf{of} an idea or policy do not agree with it and do not want it to be carried out.
 \textit{
	\begin{itemize}
	\item ...opponents of the spread of nuclear weapons.
	\item He became an outspoken opponent of the government.
	\end{itemize}
}
\end{enumerate}

\section*{receive}
{\large \color{blue}  receives  receiving  received  }
\subsection*{Explain}
\begin{enumerate}
\item verb \\
When you \textbf{receive} something, you get it after someone gives it to you or sends it to you.
 \textit{
	\begin{itemize}
	\item They will receive their awards at a ceremony in Stockholm.
	\item I received your letter of November 7.
	\end{itemize}
}
\item verb \\
You can use \textbf{receive} to say that certain kinds of thing happen to someone. For example if they are injured , you can say that they \textbf{received} an injury .
 \textit{
	\begin{itemize}
	\item He received more of the blame than anyone when the plan failed to work.
	\item She was suffering from whiplash injuries received in a car crash.
	\end{itemize}
}
\item verb \\
When you \textbf{receive} a visitor or a guest, you greet them.
 \textit{
	\begin{itemize}
	\item The following evening the duchess was again receiving guests.
	\item The shop assistant received me indifferently while leaning on a counter.
	\end{itemize}
}
\item verb \\
If you say that something \textbf{is received} in a particular way, you mean that people react to it in that way.
 \textit{
	\begin{itemize}
	\item The resolution had been received with great disappointment within the organization.
	\item The proposals have been well received by many deputies.
	\end{itemize}
}
\item verb \\
When a radio or television \textbf{receives} signals that are being transmitted , it picks them up and converts them into sound or pictures.
 \textit{
	\begin{itemize}
	\item The reception was a little faint but clear enough for him to receive the signal.
	\end{itemize}
}
\item verb \\
If someone \textbf{receives} stolen goods, they buy or are given things that have been stolen.
 \textit{
	\begin{itemize}
	\item He went to prison for receiving stolen scrap iron.
	\item He received the shoes when stolen, and then passed them on to the men who would sell
them.
	\end{itemize}
}
\item  \\
 to be on the receiving end \textit{
	\begin{itemize}
	\end{itemize}
}
\end{enumerate}

\section*{region}
{\large \color{blue}  regions  }
\subsection*{Explain}
\begin{enumerate}
\item countable noun \\
A \textbf{region} is a large area of land that is different from other areas of land, for example because it is one of the different parts of a country with its own customs and characteristics, or because it has a particular geographical feature .
 \textit{
	\begin{itemize}
	\item ...a remote mountain region.
	\end{itemize}
}
\item plural noun \\
\textbf{The regions} are the parts of a country that are not the capital city and its surrounding area.
 \textit{
	\begin{itemize}
	\item ...London and the regions.
	\item Tax incentives would be used to attract firms to the regions, away from the capital.
	\end{itemize}
}
\item countable noun \\
You can refer to a part of your body as a \textbf{region} .
 \textit{
	\begin{itemize}
	\item ...the pelvic region.
	\item ...the frontal region of the brain.
	\end{itemize}
}
\item  \\
 in the region of \textit{
	\begin{itemize}
	\end{itemize}
}
\end{enumerate}

\section*{recover}
{\large \color{blue}  recovers  recovering  recovered  }
\subsection*{Explain}
\begin{enumerate}
\item verb \\
When you \textbf{recover}  \textbf{from} an illness or an injury , you become well again.
 \textit{
	\begin{itemize}
	\item He is recovering from a knee injury.
	\item A policeman was recovering in hospital last night after being stabbed.
	\item He is fully recovered from the virus.
	\end{itemize}
}
\item verb \\
If you \textbf{recover}  \textbf{from} an unhappy or unpleasant  experience , you stop being upset by it.
 \textit{
	\begin{itemize}
	\item ...a tragedy from which he never fully recovered.
	\item There was no time to recover from the defeat.
	\end{itemize}
}
\item verb \\
If something \textbf{recovers}  \textbf{from} a period of weakness or difficulty , it improves or gets  stronger again.
 \textit{
	\begin{itemize}
	\item He recovered from a 4-2 deficit to reach the quarter-finals.
	\item The stockmarket index fell by 80% before it began to recover.
	\end{itemize}
}
\item verb \\
If you \textbf{recover} something that has been lost or stolen , you find it or get it back.
 \textit{
	\begin{itemize}
	\item Police raided five houses in south-east London and recovered stolen goods.
	\item Rescue teams recovered more bodies from the rubble.
	\end{itemize}
}
\item verb \\
If you \textbf{recover} a mental or physical state, it comes back again. For example , if you \textbf{recover} consciousness, you become conscious again.
 \textit{
	\begin{itemize}
	\item For a minute he looked uncertain, and then recovered his composure.
	\item She had a severe attack of asthma and it took an hour to recover her breath.
	\item She never recovered consciousness.
	\end{itemize}
}
\item verb \\
If you \textbf{recover} money that you have spent , invested , or lent to someone, you get the same amount back.
 \textit{
	\begin{itemize}
	\item Legal action is being taken to try to recover the money.
	\item The home market was not large enough to recover their costs of production.
	\end{itemize}
}
\end{enumerate}

\section*{saint}
{\large \color{blue}  saints  }
\subsection*{Explain}
\begin{enumerate}
\item countable noun \\
A \textbf{saint} is someone who has died and been officially recognized and honoured by the Christian church because his or her life was a perfect  example of the way Christians should live .
 \textit{
	\begin{itemize}
	\item Every parish was named after a saint.
	\item ...Saint John.
	\end{itemize}
}
\item countable noun \\
If you refer to a living person as a \textbf{saint} , you mean that they are extremely  kind , patient , and unselfish .
 \textit{
	\begin{itemize}
	\item My girlfriend is a saint to put up with me.
	\end{itemize}
}
\end{enumerate}

\section*{reject}
{\large \color{blue}  rejects  rejecting  rejected  }
\subsection*{Explain}
\begin{enumerate}
\item verb \\
If you \textbf{reject} something such as a proposal , a request , or an offer , you do not accept it or you do not agree to it.
 \textit{
	\begin{itemize}
	\item The British government is expected to reject the idea of state subsidy for a new
high speed railway.
	\item Seventeen publishers rejected the manuscript before Jenks saw its potential.
	\end{itemize}
}
\item verb \\
If you \textbf{reject} a belief or a political system, you refuse to believe in it or to live by its rules .
 \textit{
	\begin{itemize}
	\item ...the children of Eastern European immigrants who had rejected their parents' political
and religious beliefs.
	\end{itemize}
}
\item verb \\
If someone \textbf{is rejected} for a job or course of study , it is not offered to them.
 \textit{
	\begin{itemize}
	\item One of my most able students was rejected by another university.
	\end{itemize}
}
\item verb \\
If someone \textbf{rejects} another person who expects  affection from them, they are cold and unfriendly towards them.
 \textit{
	\begin{itemize}
	\item You make friends with people and then make unreasonable demands so that they reject
you.
	\item ...people who had been rejected by their lovers.
	\end{itemize}
}
\item verb \\
If a person's body \textbf{rejects} something such as a new heart that has been transplanted into it, it tries to attack and destroy it.
 \textit{
	\begin{itemize}
	\item It was feared his body was rejecting a kidney he received in a transplant four years
ago.
	\end{itemize}
}
\item verb \\
If a machine  \textbf{rejects} a coin that you put in it, the coin comes out and the machine does not work.
 \textit{
	\begin{itemize}
	\end{itemize}
}
\item countable noun \\
A \textbf{reject} is a product that has not been accepted for use or sale , because there is something wrong with it.
 \textit{
	\begin{itemize}
	\end{itemize}
}
\end{enumerate}

\section*{silk}
{\large \color{blue}  silks  }
\subsection*{Explain}
\begin{enumerate}
\item variable noun \\
\textbf{Silk} is a substance which is made into smooth fine cloth and sewing thread. You can also  refer to this cloth or thread as \textbf{silk} .
 \textit{
	\begin{itemize}
	\item They continued to get their silks from China.
	\item Pauline wore a silk dress with a strand of pearls.
	\item ...softer-looking shades in silk.
	\end{itemize}
}
\item uncountable noun \\
You can refer to the substance produced by some creatures such as spiders as \textbf{silk} .
 \textit{
	\begin{itemize}
	\item ...the silk threads of a spider's web.
	\end{itemize}
}
\end{enumerate}

\section*{resolve}
{\large \color{blue}  resolves  resolving  resolved  }
\subsection*{Explain}
\begin{enumerate}
\item verb \\
To \textbf{resolve} a problem , argument , or difficulty means to find a solution to it.
 \textit{
	\begin{itemize}
	\item We must find a way to resolve these problems before it's too late.
	\item They hoped the crisis could be resolved peacefully.
	\end{itemize}
}
\item verb \\
If you \textbf{resolve}  \textbf{to} do something, you make a firm  decision to do it.
 \textit{
	\begin{itemize}
	\item She resolved to report the matter to the hospital's nursing manager.
	\item She resolved that, if her sister forgot this promise, she would remind her.
	\end{itemize}
}
\item variable noun \\
\textbf{Resolve} is determination to do what you have decided to do.
 \textit{
	\begin{itemize}
	\item This will strengthen the American public's resolve to go to war.
	\end{itemize}
}
\item ergative verb \\
If you \textbf{resolve} something \textbf{into} a clearer form, or if it \textbf{resolves} into a clearer form, its shape or the different parts it contains become clear .
 \textit{
	\begin{itemize}
	\item ...like a musician resolving a confused mass of sound into melodic or harmonic order.
	\item Each of the spirals of light resolved into points.
	\end{itemize}
}
\end{enumerate}

\section*{spear}
{\large \color{blue}  spears  spearing  speared  }
\subsection*{Explain}
\begin{enumerate}
\item countable noun \\
A \textbf{spear} is a weapon consisting of a long pole with a sharp metal point attached to the end.
 \textit{
	\begin{itemize}
	\end{itemize}
}
\item verb \\
If you \textbf{spear} something, you push or throw a pointed object into it.
 \textit{
	\begin{itemize}
	\item Spear a piece of fish with a carving fork and dip it in the batter.
	\item A police officer was speared to death.
	\end{itemize}
}
\item countable noun \\
Asparagus or broccoli \textbf{spears} are individual stalks of asparagus or broccoli.
 \textit{
	\begin{itemize}
	\end{itemize}
}
\end{enumerate}

\section*{seem}
{\large \color{blue}  seems  seeming  seemed  }
\subsection*{Explain}
\begin{enumerate}
\item link verb \\
You use \textbf{seem} to say that someone or something gives the impression of having a particular quality, or
of happening in the way you describe .
 \textit{
	\begin{itemize}
	\item We heard a series of explosions. They seemed quite close by.
	\item Everyone seems busy except us.
	\item To everyone who knew them, they seemed an ideal couple.
	\item £50 seems a lot to pay.
	\item The calming effect seemed to last for about ten minutes.
	\item It was a record that seemed beyond reach.
	\item The proposal seems designed to break opposition to the government's economic programme.
	\item It seems that the attack this morning was very carefully planned to cause few casualties.
	\item It seems clear that he has no reasonable alternative.
	\item It seemed as if she'd been gone forever.
	\item There seems to be a lot of support in Congress for this move.
	\item There seems no possibility that such action can be averted.
	\item This phenomenon is not as outrageous as it seems.
	\end{itemize}
}
\item link verb \\
You use \textbf{seem} when you are describing your own feelings or thoughts , or describing something that has happened to you, in order to make your statement less forceful .
 \textit{
	\begin{itemize}
	\item I seem to have lost all my self-confidence.
	\item I seem to remember giving you very precise instructions.
	\item I seemed to have contracted the stomach problem.
	\item Excuse me, I seem to be a little bit lost.
	\end{itemize}
}
\item  \\
 cannot seem \textit{
	\begin{itemize}
	\end{itemize}
}
\end{enumerate}

\section*{sponsor}
{\large \color{blue}  sponsors  sponsoring  sponsored  }
\subsection*{Explain}
\begin{enumerate}
\item verb \\
If an organization or an individual \textbf{sponsors} something such as an event or someone's training, they pay some or all of the expenses connected with it, often in order to get  publicity for themselves.
 \textit{
	\begin{itemize}
	\item Mercury, in association with The Independent, is sponsoring Britain's first major
Pop Art exhibition for over 20 years.
	\item The competition was sponsored by Ruinart Champagne.
	\item Most students are sponsored by the National Department of Education.
	\end{itemize}
}
\item verb \\
In Britain, if you \textbf{sponsor} someone who is doing something to raise money for charity, for example  trying to walk a certain distance , you agree to give them a sum of money for the charity if they succeed in doing it.
 \textit{
	\begin{itemize}
	\item Please could you sponsor me for my school's campaign for Help the Aged?
	\end{itemize}
}
\item verb \\
If you \textbf{sponsor} a proposal or suggestion , you officially put it forward and support it.
 \textit{
	\begin{itemize}
	\item Eight senators sponsored legislation to stop the military funding.
	\end{itemize}
}
\item verb \\
When a country or an organization such as the United  Nations  \textbf{sponsors}  negotiations between countries, it suggests  holding the negotiations and organizes them.
 \textit{
	\begin{itemize}
	\item The superpowers may well have difficulties sponsoring negotiations.
	\item ...two days of talks in Addis Ababa, Ethiopia, sponsored by the United Nations.
	\end{itemize}
}
\item verb \\
If one country accuses another of \textbf{sponsoring}  attacks on it, they mean that the other country does not do anything to prevent the attacks,
and may even  encourage them.
 \textit{
	\begin{itemize}
	\item We have to make the states that sponsor terrorism pay a price.
	\end{itemize}
}
\item verb \\
If a company or organization \textbf{sponsors} a television programme, they pay to have a special  advertisement shown at the beginning and end of the programme, and at each commercial break.
 \textit{
	\begin{itemize}
	\item Companies will now be able to sponsor programmes on certain channels.
	\end{itemize}
}
\item countable noun \\
A \textbf{sponsor} is a person or organization that sponsors something or someone.
 \textit{
	\begin{itemize}
	\item I understand they are to be named as the new sponsors of the League Cup.
	\item The event has retained its chief sponsor In difficult financial times.
	\end{itemize}
}
\end{enumerate}

\section*{settle}
{\large \color{blue}  settles  settling  settled  }
\subsection*{Explain}
\begin{enumerate}
\item verb \\
If people \textbf{settle} an argument or problem , or if something \textbf{settles} it, they solve it, for example by making a decision about who is right or about what to do.
 \textit{
	\begin{itemize}
	\item They agreed to try to settle their dispute by negotiation.
	\item Both sides are looking for ways to settle their differences.
	\item Tomorrow's vote is unlikely to settle the question of who will replace their leader.
	\end{itemize}
}
\item verb \\
If people \textbf{settle} a legal dispute or if they \textbf{settle} , they agree to end the dispute without going to a court of law, for example by paying some money or by apologizing.
 \textit{
	\begin{itemize}
	\item In an attempt to settle the case, Molken has agreed to pay restitution.
	\item She got much less than she would have done if she had settled out of court.
	\item His company settled with the authorities by paying a $200 million fine.
	\end{itemize}
}
\item verb \\
If you \textbf{settle} a bill or debt, you pay the amount that you owe .
 \textit{
	\begin{itemize}
	\item I settled the bill for my coffee.
	\item They settled with Colin at the end of the evening.
	\end{itemize}
}
\item verb \\
If something \textbf{is settled} , it has all been decided and arranged.
 \textit{
	\begin{itemize}
	\item As far as we're concerned, the matter is settled.
	\item That's settled then. We'll exchange addresses tonight.
	\end{itemize}
}
\item verb \\
To \textbf{settle} money \textbf{on} someone means to formally give it to them, for example in a will.
 \textit{
	\begin{itemize}
	\item She offered to settle a legacy on Katharine.
	\end{itemize}
}
\item verb \\
When people \textbf{settle} a place or in a place, or when a government \textbf{settles} them there, they start living there permanently.
 \textit{
	\begin{itemize}
	\item Refugees settling in Britain suffer from a number of problems.
	\item He visited Paris and eventually settled there.
	\item This was one of the first areas to be settled by Europeans.
	\item Thirty-thousand-million dollars is needed to settle the refugees.
	\end{itemize}
}
\item verb \\
If you \textbf{settle}  \textbf{yourself}  somewhere or \textbf{settle} somewhere, you sit down or make yourself comfortable.
 \textit{
	\begin{itemize}
	\item Albert settled himself on the sofa.
	\item Jessica settled into her chair with a small sigh of relief.
	\end{itemize}
}
\item verb \\
If something \textbf{settles} or if you \textbf{settle} it, it sinks slowly down and becomes still .
 \textit{
	\begin{itemize}
	\item A black dust settled on the walls.
	\item Once its impurities had settled, the oil could be graded.
	\item Tap each one firmly on your work surface to settle the mixture.
	\end{itemize}
}
\item verb \\
If your eyes  \textbf{settle}  \textbf{on} or \textbf{upon} something, you stop  looking around and look at that thing for some time.
 \textit{
	\begin{itemize}
	\item The man let his eyes settle upon Cross's face.
	\end{itemize}
}
\item verb \\
When birds or insects  \textbf{settle on} something, they land on it from above.
 \textit{
	\begin{itemize}
	\item Moths flew in front of it, eventually settling on the rough painted metal.
	\end{itemize}
}
\end{enumerate}

\section*{symmetry}
{\large \color{blue}  symmetries  }
\subsection*{Explain}
\begin{enumerate}
\item variable noun \\
Something that has \textbf{symmetry} is symmetrical in shape , design , or structure .
 \textit{
	\begin{itemize}
	\item ...the incredible beauty and symmetry of a snowflake.
	\item I loved the house because it had perfect symmetry.
	\item Their own lives already seemed to possess the symmetries of narrative art.
	\end{itemize}
}
\item uncountable noun \\
\textbf{Symmetry} in a relationship or agreement is the fact of both sides giving and receiving an equal  amount .
 \textit{
	\begin{itemize}
	\item The superpowers pledged to maintain symmetry in their arms shipments.
	\end{itemize}
}
\item variable noun \\
You can refer to \textbf{symmetry} between countries , institutions , or situations if you think that there is a close similarity between them.
 \textit{
	\begin{itemize}
	\end{itemize}
}
\end{enumerate}

\section*{shut}
{\large \color{blue}  shuts  shutting  }
\subsection*{Explain}
\begin{enumerate}
\item verb \\
If you \textbf{shut} something such as a door or if it \textbf{shuts} , it moves so that it fills a hole or a space .
 \textbf{Shut} is also an adjective .
 \textit{
	\begin{itemize}
	\item Just make sure you shut the gate after you.
	\item The screen door shut gently.
	\item They have warned residents to stay inside and keep their doors and windows shut.
	\item The exit doors were locked shut.
	\end{itemize}
}
\item verb \\
If you \textbf{shut} your eyes , you lower your eyelids so that you cannot see anything.
 \textbf{Shut} is also an adjective.
 \textit{
	\begin{itemize}
	\item Lucy shut her eyes so she wouldn't see it happen.
	\item His eyes were shut and he seemed to have fallen asleep.
	\end{itemize}
}
\item verb \\
If your mouth  \textbf{shuts} or if you \textbf{shut} your mouth, you place your lips firmly together.
 \textbf{Shut} is also an adjective.
 \textit{
	\begin{itemize}
	\item Daniel's mouth opened, and then shut again.
	\item He opened and shut his mouth, unspeaking.
	\item She was silent for a moment, lips tight shut, eyes distant.
	\end{itemize}
}
\item verb \\
When a store , bar , or other public  building  \textbf{shuts} or when someone \textbf{shuts} it, it is closed and you cannot use it until it is open again.
 \textbf{Shut} is also an adjective.
 \textit{
	\begin{itemize}
	\item There is a tendency to shut museums or shops at a moment's notice.
	\item Shops usually shut from noon-3pm, and stay open late.
	\item What time do the pubs shut?
	\item Make sure you have food to tide you over when the local shop may be shut.
	\end{itemize}
}
\item  \\
 shut one's eyes to something \textit{
	\begin{itemize}
	\end{itemize}
}
\item  \\
 keep your mouth shut \textit{
	\begin{itemize}
	\end{itemize}
}
\item  \\
 keep your mouth shut \textit{
	\begin{itemize}
	\end{itemize}
}
\item  \\
 shut your mouth \textit{
	\begin{itemize}
	\end{itemize}
}
\end{enumerate}

\section*{tangle}
{\large \color{blue}  tangles  tangling  tangled  }
\subsection*{Explain}
\begin{enumerate}
\item countable noun \\
A \textbf{tangle}  \textbf{of} something is a mass of it twisted together in an untidy way.
 \textit{
	\begin{itemize}
	\item A tangle of wires is all that remains of the computer and phone systems.
	\item There he stood: hair in wild tangles, dark stubble shadowing his chin.
	\end{itemize}
}
\item verb \\
If something \textbf{is tangled} or \textbf{tangles} , it becomes twisted together in an untidy way.
 \textit{
	\begin{itemize}
	\item Animals get tangled in fishing nets and drown.
	\item She tried to kick the pajamas loose, but they were tangled in the satin sheet.
	\item Lee and I fell in a tangled heap.
	\item Her hair tends to tangle.
	\item He suggested that tangling fishing gear should be made a criminal offence.
	\end{itemize}
}
\item singular noun \\
You can refer to a confusing or complicated situation as a \textbf{tangle} .
 \textit{
	\begin{itemize}
	\item I was thinking what a tangle we had got ourselves into.
	\item ...the tangle of domestic politics.
	\end{itemize}
}
\item verb \\
If ideas or situations \textbf{are tangled} , they become confused and complicated.
 \textit{
	\begin{itemize}
	\item The themes get tangled in Mr Mahfouz's epic storytelling.
	\item You are currently in a muddle where financial and emotional concerns are tangled
together.
	\end{itemize}
}
\end{enumerate}

\section*{suffice}
{\large \color{blue}  suffices  sufficing  sufficed  }
\subsection*{Explain}
\begin{enumerate}
\item verb \\
If you say that something will  \textbf{suffice} , you mean it will be enough to achieve a purpose or to fulfil a need .
 \textit{
	\begin{itemize}
	\item A cover letter should never exceed one page; often a far shorter letter will suffice.
	\end{itemize}
}
\item  \\
 suffice it to say \textit{
	\begin{itemize}
	\end{itemize}
}
\end{enumerate}

\section*{typhoon}
{\large \color{blue}  typhoons  }
\subsection*{Explain}
\begin{enumerate}
\item countable noun \\
A \textbf{typhoon} is a very violent tropical storm.
 \textit{
	\begin{itemize}
	\end{itemize}
}
\end{enumerate}

\section*{tease}
{\large \color{blue}  teases  teasing  teased  }
\subsection*{Explain}
\begin{enumerate}
\item verb \\
To \textbf{tease} someone means to laugh at them or make jokes about them in order to embarrass , annoy, or upset them.
 \textbf{Tease} is also a noun .
 \textit{
	\begin{itemize}
	\item He told her how the boys in East Poldown had set on him, teasing him.
	\item He teased me mercilessly about going Hollywood.
	\item 'You must be expecting a young man,' she teased.
	\item Calling her by her real name had always been one of his teases.
	\end{itemize}
}
\item countable noun \\
If you refer to someone as a \textbf{tease} , you mean that they like laughing at people or making jokes about them.
 \textit{
	\begin{itemize}
	\item My brother's such a tease.
	\item The best way to deal with a tease is to ignore him.
	\end{itemize}
}
\item verb \\
If you say that someone \textbf{is teasing} , you mean that they are pretending to offer you something that you want , especially  sex , but then not giving it to you.
 \textit{
	\begin{itemize}
	\item I thought she was teasing, playing the innocent, but looking back, I'm not so sure.
	\item When did you last flirt with him or tease him?
	\end{itemize}
}
\item countable noun \\
If you refer to someone as a \textbf{tease} , you mean that they pretend to offer someone what they want, especially sex, but
then do not give it to them.
 \textit{
	\begin{itemize}
	\item Later she heard he had told one of her friends she was a tease.
	\end{itemize}
}
\end{enumerate}

\section*{velvet}
{\large \color{blue}  velvets  }
\subsection*{Explain}
\begin{enumerate}
\item variable noun \\
\textbf{Velvet} is soft material made from cotton, silk, or nylon, which has a thick layer of short cut  threads on one side.
 \textit{
	\begin{itemize}
	\item ...a charcoal-gray overcoat with a velvet collar.
	\item She looked pretty and rather fragile, dressed in black velvet.
	\end{itemize}
}
\end{enumerate}

\section*{thrive}
{\large \color{blue}  thrives  thriving  thrived  }
\subsection*{Explain}
\begin{enumerate}
\item verb \\
If someone or something \textbf{thrives} , they do well and are successful , healthy , or strong .
 \textit{
	\begin{itemize}
	\item Today his company continues to thrive.
	\item Lavender thrives in poor soil.
	\item ...the river's thriving population of kingfishers.
	\end{itemize}
}
\item verb \\
If you say that someone \textbf{thrives on} a particular situation , you mean that they enjoy it or that they can deal with it very well, especially when other people find it unpleasant or difficult .
 \textit{
	\begin{itemize}
	\item Many people thrive on a stressful lifestyle.
	\item Creative people are usually very determined and thrive on overcoming obstacles.
	\end{itemize}
}
\end{enumerate}

\section*{verb}
{\large \color{blue}  verbs  }
\subsection*{Explain}
\begin{enumerate}
\item countable noun \\
A \textbf{verb} is a word such as ' sing ', ' feel ', or ' die ' which is used with a subject to say what someone or something does or what happens to them, or to give information about them.
 \textit{
	\begin{itemize}
	\end{itemize}
}
\end{enumerate}

\section*{transport}
{\large \color{blue}  transports  transporting  transported  }
\subsection*{Explain}
\begin{enumerate}
\item uncountable noun \\
\textbf{Transport}  refers to any vehicle that you can travel in or carry goods in.
 \textit{
	\begin{itemize}
	\item Have you got your own transport?
	\item Which type of transport do you prefer?
	\end{itemize}
}
\item uncountable noun \\
\textbf{Transport} is a system for taking people or goods from one place to another, for example using buses or trains .
 \textit{
	\begin{itemize}
	\item The extra money could be spent on improving public transport.
	\item The sudden onset of winter caused havoc with rail and air transport.
	\item An efficient transport system is critical to the long-term future of London.
	\end{itemize}
}
\item uncountable noun \\
\textbf{Transport} is the activity of taking goods or people from one place to another in a vehicle.
 \textit{
	\begin{itemize}
	\item Local production virtually eliminates transport costs.
	\end{itemize}
}
\item verb \\
To \textbf{transport} people or goods somewhere is to take them from one place to another in a vehicle.
 \textit{
	\begin{itemize}
	\item There's no petrol, so it's very difficult to transport goods.
	\item They use tankers to transport the oil to Los Angeles.
	\item The troops were transported to Moscow.
	\end{itemize}
}
\item verb \\
If you say that you \textbf{are transported} to another place or time, you mean that something causes you to feel that you are living in the other place or at the other time.
 \textit{
	\begin{itemize}
	\item Dr Drummond felt that he had been transported into a world that rivalled the Arabian
Nights.
	\item In a dream you can be transported back in time.
	\item This delightful musical comedy transports the audience to the innocent days of 1950s
America.
	\end{itemize}
}
\item countable noun \\
A military or troop \textbf{transport} is a military vehicle, especially a plane , that is used to carry soldiers or equipment .
 \textit{
	\begin{itemize}
	\end{itemize}
}
\end{enumerate}

\section*{veteran}
{\large \color{blue}  veterans  }
\subsection*{Explain}
\begin{enumerate}
\item countable noun \\
A \textbf{veteran} is someone who has served in the armed forces of their country, especially during a war.
 \textit{
	\begin{itemize}
	\item The charity was formed in 1919 to care for veterans of the First World War.
	\end{itemize}
}
\item countable noun \\
You use \textbf{veteran} to refer to someone who has been involved in a particular activity for a long time.
 \textit{
	\begin{itemize}
	\item ...the veteran Labour MP and former Cabinet minister.
	\end{itemize}
}
\end{enumerate}

\section*{zone}
{\large \color{blue}  zones  zoning  zoned  }
\subsection*{Explain}
\begin{enumerate}
\item countable noun \\
A \textbf{zone} is an area that has particular features or characteristics.
 \textit{
	\begin{itemize}
	\item Many people have stayed behind in the potential war zone.
	\item The area has been declared a disaster zone.
	\item ...time zones.
	\end{itemize}
}
\item verb \\
If an area of land \textbf{is zoned} , it is formally set aside for a particular purpose .
 \textit{
	\begin{itemize}
	\item The land was not zoned for commercial purposes.
	\item Most of the private land in the park wasn't zoned or protected in any way.
	\end{itemize}
}
\end{enumerate}

\section*{withdraw}
{\large \color{blue}  withdraws  withdrawing  withdrew  withdrawn  }
\subsection*{Explain}
\begin{enumerate}
\item verb \\
If you \textbf{withdraw} something \textbf{from} a place, you remove it or take it away.
 \textit{
	\begin{itemize}
	\item He reached into his pocket and withdrew a sheet of notepaper.
	\item Cassandra withdrew her hand from Roger's.
	\end{itemize}
}
\item verb \\
When groups of people such as troops  \textbf{withdraw} or when someone \textbf{withdraws} them, they leave the place where they are fighting or where they are based and return nearer home .
 \textit{
	\begin{itemize}
	\item He stated that all foreign forces would withdraw as soon as the crisis ended.
	\item Unless Hitler withdrew his troops from Poland by 11 o'clock that morning, a state
of war would exist between Great Britain and Germany.
	\item Troops withdrew from the north east of the country last March.
	\end{itemize}
}
\item verb \\
If you \textbf{withdraw}  money  \textbf{from} a bank account , you take it out of that account.
 \textit{
	\begin{itemize}
	\item Open a savings account that does not charge ridiculous fees to withdraw money.
	\item They withdrew 100 dollars from a bank account after checking out of their hotel.
	\end{itemize}
}
\item verb \\
If you \textbf{withdraw}  \textbf{to} another room , you go there.
 \textit{
	\begin{itemize}
	\item He and the others withdrew to their rented rooms.
	\item He served the dinner and then withdrew again.
	\item Kenworthy withdrew into his bedroom, washed and shaved.
	\end{itemize}
}
\item verb \\
If you \textbf{withdraw}  \textbf{from} an activity or organization , you stop  taking part in it.
 \textit{
	\begin{itemize}
	\item The African National Congress threatened to withdraw from the talks.
	\end{itemize}
}
\item verb \\
If you \textbf{withdraw} a remark or statement that you have made, you say that you want people to ignore it.
 \textit{
	\begin{itemize}
	\item He withdrew his remarks and explained what he had meant to say.
	\end{itemize}
}
\end{enumerate}

\section*{airport}
{\large \color{blue}  airports  }
\subsection*{Explain}
\begin{enumerate}
\item countable noun \\
An \textbf{airport} is a place where aircraft land and take off, which has buildings and facilities for
passengers.
 \textit{
	\begin{itemize}
	\item ...Heathrow Airport, the busiest international airport in the world.
	\end{itemize}
}
\end{enumerate}

\section*{cast}
{\large \color{blue}  casts  casting  }
\subsection*{Explain}
\begin{enumerate}
\item countable noun \\
The \textbf{cast} of a play or film is all the people who act in it.
 \textit{
	\begin{itemize}
	\item The show is very amusing and the cast are very good.
	\end{itemize}
}
\item verb \\
To \textbf{cast} an actor \textbf{in} a play or film means to choose them to act a particular role in it.
 \textit{
	\begin{itemize}
	\item Casting three actresses in the film to play one role was very challenging.
	\item He was cast as a college professor.
	\item He had no trouble casting the movie.
	\end{itemize}
}
\item verb \\
To \textbf{cast} someone \textbf{in} a particular way or \textbf{as} a particular thing means to describe them in that way or suggest they are that thing.
 \textit{
	\begin{itemize}
	\item Democrats have been worried about being cast as the party of the poor.
	\item Holland would never dare cast himself as a virtuoso pianist.
	\end{itemize}
}
\item verb \\
If you \textbf{cast} your eyes or \textbf{cast} a look in a particular direction, you look quickly in that direction.
 \textit{
	\begin{itemize}
	\item He cast a stern glance at the two men.
	\item I cast my eyes down briefly.
	\item The maid, casting black looks, hurried out.
	\end{itemize}
}
\item verb \\
If something \textbf{casts} a light or shadow  somewhere , it causes it to appear there.
 \textit{
	\begin{itemize}
	\item The moon cast a bright light over the yard.
	\item They flew in over the beach, casting a huge shadow.
	\end{itemize}
}
\item verb \\
To \textbf{cast} doubt \textbf{on} something means to cause people to be unsure about it.
 \textit{
	\begin{itemize}
	\item Last night a top criminal psychologist cast doubt on the theory.
	\end{itemize}
}
\item verb \\
When you \textbf{cast} your vote in an election , you vote.
 \textit{
	\begin{itemize}
	\item About ninety-five per cent of those who cast their votes approve the new constitution.
	\item Gaviria had been widely expected to obtain well over half the votes cast.
	\end{itemize}
}
\item verb \\
To \textbf{cast} something or someone somewhere means to throw them there.
 \textit{
	\begin{itemize}
	\item He gathered up the twigs and cast them into the fire.
	\item John had Maude and her son cast into a dungeon.
	\end{itemize}
}
\item verb \\
If someone \textbf{casts} a fishing line or \textbf{casts} , they throw one end of the fishing line into the water.
 \textit{
	\begin{itemize}
	\end{itemize}
}
\item verb \\
To \textbf{cast} an object means to make it by pouring a liquid such as hot metal into a specially
shaped container and leaving it there until it becomes hard.
 \textit{
	\begin{itemize}
	\item ...sculptures cast in bronze.
	\end{itemize}
}
\item countable noun \\
A \textbf{cast} is a model that has been made by pouring a liquid such as plaster or hot metal onto
something or into something, so that when it hardens it has the same shape as that thing.
 \textit{
	\begin{itemize}
	\item An orthodontist took a cast of the inside of Billy's mouth.
	\end{itemize}
}
\item countable noun \\
A \textbf{cast} is the same as a plaster cast .
 \textit{
	\begin{itemize}
	\end{itemize}
}
\item countable noun \\
If someone has a particular \textbf{cast}  \textbf{of} mind or \textbf{cast}  \textbf{of} thought, they have that kind of character or way of thinking of things.
 \textit{
	\begin{itemize}
	\item There were some of a more cynical cast of mind, however.
	\item Hers was an essentially sunny cast of mind.
	\end{itemize}
}
\end{enumerate}

\section*{assistance}
{\large \color{blue}  }
\subsection*{Explain}
\begin{enumerate}
\item uncountable noun \\
If you give someone \textbf{assistance} , you help them do a job or task by doing part of the work for them.
 \textit{
	\begin{itemize}
	\item Since 1976 he has been operating the shop with the assistance of volunteers.
	\item She can still come downstairs with assistance but she's very weak.
	\end{itemize}
}
\item uncountable noun \\
If you give someone \textbf{assistance} , you give them information or advice .
 \textit{
	\begin{itemize}
	\item Any assistance you could give the police will be greatly appreciated.
	\item Employees are being offered assistance in finding new jobs.
	\end{itemize}
}
\item uncountable noun \\
If someone gives a person or country  \textbf{assistance} , they help them by giving them money .
 \textit{
	\begin{itemize}
	\item ...a viable programme of economic assistance.
	\item We shall offer you assistance with legal expenses up to $5,000.
	\end{itemize}
}
\item uncountable noun \\
If something is done  \textbf{with the}  \textbf{assistance}  \textbf{of} a particular thing, that thing is helpful or necessary for doing it.
 \textit{
	\begin{itemize}
	\item The translations were carried out with the assistance of a medical dictionary.
	\end{itemize}
}
\item  \\
 be of assistance \textit{
	\begin{itemize}
	\end{itemize}
}
\item  \\
 come to sb's assistance \textit{
	\begin{itemize}
	\end{itemize}
}
\end{enumerate}

\section*{clutch}
{\large \color{blue}  clutches  clutching  clutched  }
\subsection*{Explain}
\begin{enumerate}
\item verb \\
If you \textbf{clutch}  \textbf{at} something or \textbf{clutch} something, you hold it tightly, usually because you are afraid or anxious .
 \textit{
	\begin{itemize}
	\item I staggered and had to clutch at a chair for support.
	\item She was clutching a photograph.
	\end{itemize}
}
\item plural noun \\
If someone is in another person's \textbf{clutches} , that person has captured them or has power over them.
 \textit{
	\begin{itemize}
	\item Sophie had fallen into the clutches of a human trafficker.
	\item Stojanovic escaped their clutches by jumping from a moving vehicle.
	\end{itemize}
}
\item countable noun \\
In a vehicle, the \textbf{clutch} is the pedal that you press before you change gear .
 \textit{
	\begin{itemize}
	\item Laura let out the clutch and pulled slowly away down the drive.
	\end{itemize}
}
\item countable noun \\
A \textbf{clutch}  \textbf{of} eggs is a number of eggs laid by a bird at one time.
 \textit{
	\begin{itemize}
	\item ...the second clutch of eggs.
	\end{itemize}
}
\item countable noun \\
A \textbf{clutch}  \textbf{of} people or things is a small group of them.
 \textit{
	\begin{itemize}
	\item The party has attracted a clutch of young southern liberals.
	\item ...a clutch of songs about adolescent experiences.
	\end{itemize}
}
\end{enumerate}

\section*{authority}
{\large \color{blue}  authorities  }
\subsection*{Explain}
\begin{enumerate}
\item plural noun \\
\textbf{The}  \textbf{authorities} are the people who have the power to make decisions and to make sure that laws are obeyed .
 \textit{
	\begin{itemize}
	\item This provided a pretext for the authorities to cancel the elections.
	\item The prison authorities have been criticised for not ending the protest more quickly.
	\end{itemize}
}
\item countable noun \\
An \textbf{authority} is an official  organization or government department that has the power to make decisions.
 \textit{
	\begin{itemize}
	\item ...the Health Education Authority.
	\item Any alterations had to meet the approval of the local planning authority.
	\end{itemize}
}
\item uncountable noun \\
\textbf{Authority} is the right to command and control other people.
 \textit{
	\begin{itemize}
	\item Local police chiefs should re-emerge as figures of authority and reassurance in their
areas.
	\item The judge had no authority to order a second trial.
	\end{itemize}
}
\item uncountable noun \\
If someone has \textbf{authority} , they have a quality which makes other people take  notice of what they say .
 \textit{
	\begin{itemize}
	\item He had no natural authority and no capacity for imposing his will on others.
	\end{itemize}
}
\item uncountable noun \\
\textbf{Authority} is official permission to do something.
 \textit{
	\begin{itemize}
	\item The police were given authority to arrest anyone suspected of subversive thoughts.
	\end{itemize}
}
\item countable noun \\
Someone who is an \textbf{authority on} a particular subject  knows a lot about it.
 \textit{
	\begin{itemize}
	\item He's universally recognized as an authority on Russian affairs.
	\end{itemize}
}
\item  \\
 have it on good authority \textit{
	\begin{itemize}
	\end{itemize}
}
\end{enumerate}

\section*{come}
{\large \color{blue}  comes  coming  came  }
\subsection*{Explain}
\begin{enumerate}
\item verb \\
When a person or thing \textbf{comes} to a particular place, especially to a place where you are, they move there.
 \textit{
	\begin{itemize}
	\item Two police officers came into the hall.
	\item Come here, Tom.
	\item You'll have to come with us.
	\item We want you to come to lunch.
	\item I came over from Ireland to start a new life after my divorce.
	\item We heard the train coming.
	\item Can I come too?
	\item The impact blew out some of the windows and the sea came rushing in.
	\end{itemize}
}
\item verb \\
When someone \textbf{comes}  \textbf{to} do something, they move to the place where someone else is in order to do it, and
they do it. In British English, someone can also \textbf{come and} do something and in American English, someone can \textbf{come} do something. However, you always  say that someone \textbf{came and} did something.
 \textit{
	\begin{itemize}
	\item Eleanor had come to visit her.
	\item Come and meet Roger.
	\item A lot of our friends came and saw me.
	\item I want you to come visit me.
	\end{itemize}
}
\item verb \\
When you \textbf{come to} a place, you reach it.
 \textit{
	\begin{itemize}
	\item He came to a door that led into a passageway.
	\end{itemize}
}
\item verb \\
If something \textbf{comes up}  \textbf{to} a particular point or \textbf{down}  \textbf{to} it, it is tall enough, deep enough, or long enough to reach that point.
 \textit{
	\begin{itemize}
	\item The water came up to my chest.
	\item I wore a large shirt of Jamie's which came down over my hips.
	\end{itemize}
}
\item verb \\
If something \textbf{comes apart} or \textbf{comes to pieces} , it breaks into pieces. If something \textbf{comes off} or \textbf{comes away} , it becomes detached from something else.
 \textit{
	\begin{itemize}
	\item The pistol came to pieces, easily and quickly.
	\item The door knobs came off in our hands.
	\end{itemize}
}
\item link verb \\
You use \textbf{come} in expressions such as \textbf{come to an end} or \textbf{come into operation} to indicate that someone or something enters or reaches a particular state or situation.
 \textit{
	\begin{itemize}
	\item The summer came to an end.
	\item The Communists came to power in 1944.
	\item I came into contact with very bright Harvard and Yale students.
	\item ...new taxes which come into force next month.
	\item Their worst fears may be coming true.
	\end{itemize}
}
\item verb \\
If someone \textbf{comes}  \textbf{to} do something, they do it at the end of a long process or period of time.
 \textit{
	\begin{itemize}
	\item She said it so many times that she came to believe it.
	\item Although it was a secret wedding, the press did eventually come to hear about it.
	\end{itemize}
}
\item verb \\
You can ask how something \textbf{came}  \textbf{to} happen when you want to know what caused it to happen or made it possible .
 \textit{
	\begin{itemize}
	\item How did you come to meet him?
	\end{itemize}
}
\item verb \\
When a particular event or time \textbf{comes} , it arrives or happens.
 \textit{
	\begin{itemize}
	\item The announcement came after a meeting at the Home Office.
	\item The time has come for us to move on.
	\item There will come a time when the crisis will occur.
	\end{itemize}
}
\item preposition \\
You can use \textbf{come} before a date , time, or event to mean when that date, time, or event arrives. For example , you can say \textbf{come the spring} to mean 'when the spring arrives'.
 \textit{
	\begin{itemize}
	\item Come the election on the 20th of May, we will have to decide.
	\item He's going to be up there again come Sunday.
	\end{itemize}
}
\item verb \\
If a thought, idea, or memory  \textbf{comes to} you, you suddenly  think of it or remember it.
 \textit{
	\begin{itemize}
	\item He was about to shut the door when an idea came to him.
	\item Then it came to me that perhaps he did understand.
	\end{itemize}
}
\item verb \\
If money or property is going to \textbf{come to} you, you are going to inherit or receive it.
 \textit{
	\begin{itemize}
	\item The fortune will come to you.
	\item He did have pension money coming to him when the factory shut down.
	\end{itemize}
}
\item verb \\
If a case \textbf{comes before} a court or tribunal or \textbf{comes to} court, it is presented there so that the court or tribunal can examine it.
 \textit{
	\begin{itemize}
	\item They were ready to explain their case when it came before the planning committee.
	\item President Cristiani expected the case to come to court within ninety days.
	\end{itemize}
}
\item verb \\
If something \textbf{comes to} a particular number or amount, it adds up to it.
 \textit{
	\begin{itemize}
	\item Lunch came to $80.
	\end{itemize}
}
\item verb \\
If someone or something \textbf{comes from} a particular place or thing, that place or thing is their origin , source, or starting point.
 \textit{
	\begin{itemize}
	\item Nearly half the students come from abroad.
	\item Chocolate comes from the cacao tree.
	\item The term 'claret', used to describe Bordeaux wines, may come from the French word
'clairet'.
	\end{itemize}
}
\item verb \\
Something that \textbf{comes from} something else or \textbf{comes of} it is the result of it.
 \textit{
	\begin{itemize}
	\item There is a feeling of power that comes from driving fast.
	\item Some good might come of all this gloomy business.
	\item He asked to be transferred there some years ago, but nothing came of it.
	\end{itemize}
}
\item verb \\
If someone \textbf{comes of} a particular family or type of family, they are descended from them.
 \textit{
	\begin{itemize}
	\item She comes of a very good family.
	\end{itemize}
}
\item verb \\
If someone or something \textbf{comes} first, next , or last , they are first, next, or last in a series, list , or competition .
 \textit{
	\begin{itemize}
	\item The two countries have been unable to agree which step should come next.
	\item The alphabet might be more rational if all the vowels came first.
	\item The horse had already won at Lincolnshire and come second at Lowesby.
	\end{itemize}
}
\item verb \\
If a type of thing \textbf{comes}  \textbf{in} a particular range of colours, forms, styles, or sizes, it can have any of those
colours, forms, styles, or sizes.
 \textit{
	\begin{itemize}
	\item Bikes come in all shapes and sizes.
	\item The wallpaper comes in black and white only.
	\end{itemize}
}
\item verb \\
You use \textbf{come} in expressions such as \textbf{it came as a surprise} when indicating a person's reaction to something that happens.
 \textit{
	\begin{itemize}
	\item Major's reply came as a complete surprise to the House of Commons.
	\item The arrest has come as a terrible shock.
	\end{itemize}
}
\item verb \\
The next subject in a discussion that you \textbf{come to} is the one that you talk about next.
 \textit{
	\begin{itemize}
	\item Finally in the programme, we come to the news that the American composer and conductor,
Leonard Bernstein, has died.
	\item That is another matter altogether. And we shall come to that next.
	\end{itemize}
}
\item convention \\
People say ' \textbf{Come} ' to encourage or comfort someone.
 \textit{
	\begin{itemize}
	\item 'Come, eat!' the old woman urged.
	\end{itemize}
}
\item verb \\
To \textbf{come} means to have an orgasm.
 \textit{
	\begin{itemize}
	\end{itemize}
}
\item  \\
 come again \textit{
	\begin{itemize}
	\end{itemize}
}
\item  \\
 as good/stupid/quick etc as they come \textit{
	\begin{itemize}
	\end{itemize}
}
\item  \\
 come come \textit{
	\begin{itemize}
	\end{itemize}
}
\item  \\
 when you come/it comes down to it \textit{
	\begin{itemize}
	\end{itemize}
}
\item  \\
 to have it/get what's coming to you \textit{
	\begin{itemize}
	\end{itemize}
}
\item  \\
 come to think of it \textit{
	\begin{itemize}
	\end{itemize}
}
\item  \\
 to come \textit{
	\begin{itemize}
	\end{itemize}
}
\item  \\
 when it comes (down) to \textit{
	\begin{itemize}
	\end{itemize}
}
\item  \\
 where someone is coming from \textit{
	\begin{itemize}
	\end{itemize}
}
\end{enumerate}

\section*{carpet}
{\large \color{blue}  carpets  carpeting  carpeted  }
\subsection*{Explain}
\begin{enumerate}
\item variable noun \\
A \textbf{carpet} is a thick covering of soft material which is laid over a floor or a staircase .
 \textit{
	\begin{itemize}
	\item They put down wooden boards, and laid new carpets on top.
	\item ...the stain on our living-room carpet.
	\end{itemize}
}
\item verb \\
If a floor or a room  \textbf{is carpeted} , a carpet is laid on the floor.
 \textit{
	\begin{itemize}
	\item The room had been carpeted and the windows glazed with coloured glass.
	\item The main gaming room was thickly carpeted.
	\end{itemize}
}
\item countable noun \\
A \textbf{carpet}  \textbf{of} something such as leaves or plants is a layer of them which covers the ground .
 \textit{
	\begin{itemize}
	\item The carpet of leaves in my yard became more and more noticeable.
	\end{itemize}
}
\item verb \\
If the ground \textbf{is carpeted}  \textbf{with} something such as leaves or plants, it is completely covered by them.
 \textit{
	\begin{itemize}
	\item The ground was thickly carpeted with pine needles.
	\end{itemize}
}
\end{enumerate}

\section*{commit}
{\large \color{blue}  commits  committing  committed  }
\subsection*{Explain}
\begin{enumerate}
\item verb \\
If someone \textbf{commits} a crime or a sin , they do something illegal or bad .
 \textit{
	\begin{itemize}
	\item I have never committed any crime.
	\item This is a man who has committed murder.
	\item ...the temptation to commit adultery.
	\end{itemize}
}
\item verb \\
If someone \textbf{commits}  \textbf{suicide} , they deliberately kill themselves.
 \textit{
	\begin{itemize}
	\item There are unconfirmed reports he tried to commit suicide.
	\end{itemize}
}
\item verb \\
If you \textbf{commit}  money or resources  \textbf{to} something, you decide to use them for a particular purpose .
 \textit{
	\begin{itemize}
	\item They called on Western nations to commit more money to the poorest nations.
	\item The government had committed billions of pounds for a programme to reduce acid rain.
	\item He should not commit American troops without the full consent of Congress.
	\end{itemize}
}
\item verb \\
If you \textbf{commit}  \textbf{yourself to} something, you say that you will  definitely do it. If you \textbf{commit}  \textbf{yourself to} someone, you decide that you want to have a long-term  relationship with them.
 \textit{
	\begin{itemize}
	\item They could not commit themselves to any definite course of action.
	\item I'd like us to be closer but I don't want to commit myself too soon.
	\item You don't have to commit to anything over the phone.
	\end{itemize}
}
\item verb \\
If you do not want to \textbf{commit}  \textbf{yourself on} something, you do not want to say what you really  think about it or what you are going to do.
 \textit{
	\begin{itemize}
	\item It isn't their diplomatic style to commit themselves on such a delicate issue.
	\item She didn't want to commit herself one way or the other.
	\end{itemize}
}
\item verb \\
If someone \textbf{is committed}  \textbf{to} a hospital , prison , or other institution , they are officially sent there for a period of time.
 \textit{
	\begin{itemize}
	\item Offenders would be committed to these prisons by the local courts.
	\end{itemize}
}
\item verb \\
In the British legal system, if someone \textbf{is committed for trial} , they are sent by magistrates to stand trial in a crown  court .
 \textit{
	\begin{itemize}
	\item He is expected to be committed for trial at Liverpool Crown Court.
	\end{itemize}
}
\item  \\
 to commit something to memory \textit{
	\begin{itemize}
	\end{itemize}
}
\end{enumerate}

\section*{cellar}
{\large \color{blue}  cellars  }
\subsection*{Explain}
\begin{enumerate}
\item countable noun \\
A \textbf{cellar} is a room underneath a building, which is often used for storing things in.
 \textit{
	\begin{itemize}
	\item The box of papers had been stored in a cellar at the family home.
	\end{itemize}
}
\item countable noun \\
A person's or restaurant's \textbf{cellar} is the collection of different wines that they have.
 \textit{
	\begin{itemize}
	\item ...the restaurant's extensive wine cellar.
	\end{itemize}
}
\end{enumerate}

\section*{deny}
{\large \color{blue}  denies  denying  denied  }
\subsection*{Explain}
\begin{enumerate}
\item verb \\
When you \textbf{deny} something, you state that it is not true .
 \textit{
	\begin{itemize}
	\item She denied both accusations.
	\item The government has denied that there was a plot to assassinate the president.
	\item They all denied ever having seen her.
	\end{itemize}
}
\item verb \\
If you \textbf{deny} someone or something, you say that they have no connection with you or do not belong to you.
 \textit{
	\begin{itemize}
	\item I denied my father because I wanted to become someone else.
	\end{itemize}
}
\item verb \\
If you \textbf{deny} someone something that they need or want , you refuse to let them have it.
 \textit{
	\begin{itemize}
	\item If he is unlucky, he may find that his ex-partner denies him access to his children.
	\item Don't deny yourself pleasure.
	\item My mother denied herself for us.
	\end{itemize}
}
\end{enumerate}

\section*{chance}
{\large \color{blue}  chances  chancing  chanced  }
\subsection*{Explain}
\begin{enumerate}
\item variable noun \\
If there is a \textbf{chance}  \textbf{of} something happening , it is possible that it will happen.
 \textit{
	\begin{itemize}
	\item Do you think they have a chance of beating Australia?
	\item This partnership has a good chance of success.
	\item The specialist who carried out the brain scan thought Tim's chances of survival were
still slim.
	\item There was really very little chance that Ben would ever have led a normal life.
	\end{itemize}
}
\item countable noun \\
If you have a \textbf{chance}  \textbf{to} do something, you have the opportunity to do it.
 \textit{
	\begin{itemize}
	\item The electoral council announced that all eligible people would get a chance to vote.
	\item I felt I had to give him a chance.
	\end{itemize}
}
\item adjective \\
A \textbf{chance}  meeting or event is one that is not planned or expected .
 \textbf{Chance} is also a noun .
 \textit{
	\begin{itemize}
	\item ...a chance meeting.
	\item ...a victim of chance and circumstance.
	\end{itemize}
}
\item verb \\
If you \textbf{chance}  \textbf{to} do something or \textbf{chance}  \textbf{on} something, you do it or find it although you had not planned or tried to.
 \textit{
	\begin{itemize}
	\item A man I chanced to meet proved to be a most unusual character.
	\item It was just then that I chanced to look round.
	\item They once holidayed in Rome and chanced upon a bar called The Seamus Heaney.
	\end{itemize}
}
\item verb \\
If you \textbf{chance} something, you do it even though there is a risk that you may not succeed or that something bad may happen.
 \textit{
	\begin{itemize}
	\item Andy knew the risks. I cannot believe he would have chanced it.
	\item He decided no assassin would chance a shot from amongst that crowd.
	\end{itemize}
}
\item  \\
 by chance \textit{
	\begin{itemize}
	\end{itemize}
}
\item  \\
 by any chance \textit{
	\begin{itemize}
	\end{itemize}
}
\item  \\
 to stand a chance \textit{
	\begin{itemize}
	\end{itemize}
}
\item  \\
 take a chance \textit{
	\begin{itemize}
	\end{itemize}
}
\end{enumerate}

\section*{deserve}
{\large \color{blue}  deserves  deserving  deserved  }
\subsection*{Explain}
\begin{enumerate}
\item verb \\
If you say that a person or thing \textbf{deserves} something, you mean that they should have it or receive it because of their actions or qualities.
 \textit{
	\begin{itemize}
	\item Government officials clearly deserve some of the blame as well.
	\item They know the sport inside out, and we treat them with the respect they deserve.
	\item These people deserve to make more than the minimum wage.
	\item His children's books are classics that deserve to be much better known.
	\item I felt I deserved better than that.
	\item The Park Hotel has a well-deserved reputation.
	\end{itemize}
}
\item  \\
 got what one deserved \textit{
	\begin{itemize}
	\end{itemize}
}
\end{enumerate}

\section*{dwell}
{\large \color{blue}  dwells  dwelling  dwelt  dwelled  }
\subsection*{Explain}
\begin{enumerate}
\item verb \\
If you \textbf{dwell}  \textbf{on} something, especially something unpleasant , you think , speak, or write about it a lot or for quite a long time.
 \textit{
	\begin{itemize}
	\item I'd rather not dwell on the past.
	\end{itemize}
}
\item verb \\
If you \textbf{dwell}  somewhere , you live there.
 \textit{
	\begin{itemize}
	\item They are concerned for the fate of the forest and the Indians who dwell in it.
	\item Shiva is a dark god; he dwells in the mountains and deserts.
	\end{itemize}
}
\end{enumerate}

\section*{defect}
{\large \color{blue}  defects  defecting  defected  }
\subsection*{Explain}
\begin{enumerate}
\item countable noun \\
A \textbf{defect} is a fault or imperfection in a person or thing.
 \textit{
	\begin{itemize}
	\item He was born with a hearing defect.
	\item ...a defect in the aircraft caused the crash.
	\item A report has pointed out the defects of the present system.
	\end{itemize}
}
\item verb \\
If you \textbf{defect} , you leave your country, political party, or other group, and join an opposing country,
party, or group.
 \textit{
	\begin{itemize}
	\item 25 per cent of its listed client base defect to rival auditors.
	\item He defected from the party in the late 1970s.
	\item ...a KGB officer who defected in 1963.
	\end{itemize}
}
\end{enumerate}

\section*{enclose}
{\large \color{blue}  encloses  enclosing  enclosed  }
\subsection*{Explain}
\begin{enumerate}
\item verb \\
If a place or object \textbf{is enclosed} by something, the place or object is inside that thing or completely surrounded by it.
 \textit{
	\begin{itemize}
	\item The rules state that samples must be enclosed in two watertight containers.
	\item Enclose the pot in a clear polythene bag.
	\item The surrounding land was enclosed by an eight foot wire fence.
	\item ...the enclosed waters of the Baltic.
	\end{itemize}
}
\item verb \\
If you \textbf{enclose} something with a letter, you put it in the same envelope as the letter.
 \textit{
	\begin{itemize}
	\item I have enclosed a cheque for £10.
	\item He tore open the creamy envelope that had been enclosed in the letter.
	\item The enclosed leaflet shows how Service Care can ease all your worries.
	\end{itemize}
}
\end{enumerate}

\section*{district}
{\large \color{blue}  districts  }
\subsection*{Explain}
\begin{enumerate}
\item countable noun \\
A \textbf{district} is a particular area of a town or country.
 \textit{
	\begin{itemize}
	\item I drove around the business district.
	\item ...Nashville's shopping district.
	\item ...a summer holiday hike in the Lake District.
	\item Varieties of these crops have been collected from all around the district.
	\end{itemize}
}
\item countable noun \\
A \textbf{district} is an area of a town or country which has been given official boundaries for the purpose of administration .
 \textit{
	\begin{itemize}
	\item ...the home of the governor of the district.
	\item ...the continuing support of Glasgow District Council.
	\item ...the district health authority.
	\end{itemize}
}
\end{enumerate}

\section*{enlighten}
{\large \color{blue}  enlightens  enlightening  enlightened  }
\subsection*{Explain}
\begin{enumerate}
\item verb \\
To \textbf{enlighten} someone means to give them more knowledge and greater understanding about something.
 \textit{
	\begin{itemize}
	\item A few dedicated doctors have fought for years to enlighten the profession.
	\item If you know what is wrong with her, please enlighten me.
	\end{itemize}
}
\end{enumerate}

\section*{ear}
{\large \color{blue}  ears  }
\subsection*{Explain}
\begin{enumerate}
\item countable noun \\
Your \textbf{ears} are the two parts of your body, one on each side of your head, with which you hear sounds.
 \textit{
	\begin{itemize}
	\item He whispered something in her ear.
	\item I'm having my ears pierced.
	\end{itemize}
}
\item singular noun \\
If you have \textbf{an}  \textbf{ear}  \textbf{for}  music or language, you are able to hear its sounds accurately and to interpret them or reproduce them well .
 \textit{
	\begin{itemize}
	\item Moby certainly has a fine ear for a tune.
	\item An ear for foreign languages is advantageous.
	\end{itemize}
}
\item countable noun \\
\textbf{Ear} is often used to refer to people's willingness to listen to what someone is saying .
 \textit{
	\begin{itemize}
	\item What would cause the masses to give him a far more sympathetic ear?
	\item They had shut their eyes and ears to everything.
	\end{itemize}
}
\item countable noun \\
The \textbf{ears} of a cereal plant such as wheat or barley are the parts at the top of the stem , which contain the seeds or grains.
 \textit{
	\begin{itemize}
	\end{itemize}
}
\item  \\
 all ears \textit{
	\begin{itemize}
	\end{itemize}
}
\item  \\
 to bend someone's ear \textit{
	\begin{itemize}
	\end{itemize}
}
\item  \\
 to box someone's ears \textit{
	\begin{itemize}
	\end{itemize}
}
\item  \\
 to fall on deaf ears \textit{
	\begin{itemize}
	\end{itemize}
}
\item  \\
 keep an ear to the ground \textit{
	\begin{itemize}
	\end{itemize}
}
\item  \\
 half an ear \textit{
	\begin{itemize}
	\end{itemize}
}
\item  \\
 to lend an ear \textit{
	\begin{itemize}
	\end{itemize}
}
\item  \\
 in one ear and out the other \textit{
	\begin{itemize}
	\end{itemize}
}
\item  \\
 out on your ear \textit{
	\begin{itemize}
	\end{itemize}
}
\item  \\
 play (a piece of music) by ear \textit{
	\begin{itemize}
	\end{itemize}
}
\item  \\
 to play it by ear \textit{
	\begin{itemize}
	\end{itemize}
}
\item  \\
 up to your ears \textit{
	\begin{itemize}
	\end{itemize}
}
\end{enumerate}

\section*{evaluate}
{\large \color{blue}  evaluates  evaluating  evaluated  }
\subsection*{Explain}
\begin{enumerate}
\item verb \\
If you \textbf{evaluate} something or someone, you consider them in order to make a judgment about them, for example about how good or bad they are.
 \textit{
	\begin{itemize}
	\item They will first send in trained nurses to evaluate the needs of the individual situation.
	\item The market situation is difficult to evaluate.
	\end{itemize}
}
\end{enumerate}

\section*{earth}
{\large \color{blue}  earths  }
\subsection*{Explain}
\begin{enumerate}
\item proper noun \\
\textbf{Earth} or \textbf{the Earth} is the planet on which we live. People usually say  \textbf{Earth} when they are referring to the planet as part of the universe , and \textbf{the Earth} when they are talking about the planet as the place where we live.
 \textit{
	\begin{itemize}
	\item The space shuttle returned safely to Earth.
	\item ...a fault in the Earth's crust.
	\end{itemize}
}
\item singular noun \\
\textbf{The earth} is the land surface on which we live and move about.
 \textit{
	\begin{itemize}
	\item The earth shook and the walls of neighbouring houses fell around them.
	\end{itemize}
}
\item uncountable noun \\
\textbf{Earth} is the substance on the land surface of the earth, for example clay or sand , in which plants grow.
 \textit{
	\begin{itemize}
	\item The road winds for miles through parched earth, scrub and cactus.
	\item They will revert to tilling the earth in an old-fashioned way.
	\end{itemize}
}
\item countable noun \\
An \textbf{earth} is a hole in the ground in which an animal such as a fox lives.
 \textit{
	\begin{itemize}
	\end{itemize}
}
\item singular noun \\
The \textbf{earth} in an electric plug or piece of electrical equipment is the wire through which electricity can pass into the ground, which makes the equipment safe if something goes  wrong with it.
 \textit{
	\begin{itemize}
	\item The earth wire was not connected.
	\end{itemize}
}
\item  \\
 on earth \textit{
	\begin{itemize}
	\end{itemize}
}
\item  \\
 on earth \textit{
	\begin{itemize}
	\end{itemize}
}
\item  \\
 on earth \textit{
	\begin{itemize}
	\end{itemize}
}
\item  \\
 back/down to earth \textit{
	\begin{itemize}
	\end{itemize}
}
\item  \\
 to run someone to earth \textit{
	\begin{itemize}
	\end{itemize}
}
\item  \\
 cost the earth/pay the earth \textit{
	\begin{itemize}
	\end{itemize}
}
\end{enumerate}

\section*{expel}
{\large \color{blue}  expels  expelling  expelled  }
\subsection*{Explain}
\begin{enumerate}
\item verb \\
If someone \textbf{is expelled}  \textbf{from} a school or organization , they are officially  told to leave because they have behaved  badly .
 \textit{
	\begin{itemize}
	\item More than five-thousand secondary school students have been expelled for cheating.
	\item ...a boy expelled from school for making death threats to his teacher.
	\end{itemize}
}
\item verb \\
If people \textbf{are expelled}  \textbf{from} a place, they are made to leave it, often by force.
 \textit{
	\begin{itemize}
	\item An American academic was expelled from the country yesterday.
	\item They were told at first that they should simply expel the refugees.
	\end{itemize}
}
\item verb \\
To \textbf{expel} something means to force it out from a container or from your body.
 \textit{
	\begin{itemize}
	\item Daily brushing of the skin helps the skin expel toxins.
	\item As the lungs exhale this waste, gas is expelled into the atmosphere.
	\end{itemize}
}
\end{enumerate}

\section*{earthquake}
{\large \color{blue}  earthquakes  }
\subsection*{Explain}
\begin{enumerate}
\item countable noun \\
An \textbf{earthquake} is a shaking of the ground caused by movement of the Earth's crust.
 \textit{
	\begin{itemize}
	\end{itemize}
}
\end{enumerate}

\section*{fault}
{\large \color{blue}  faults  faulting  faulted  }
\subsection*{Explain}
\begin{enumerate}
\item singular noun \\
If a bad or undesirable  situation is your \textbf{fault} , you caused it or are responsible for it.
 \textit{
	\begin{itemize}
	\item There was no escaping the fact: it was all his fault.
	\item A few borrowers will be in trouble with their repayments through no fault of their
own.
	\end{itemize}
}
\item countable noun \\
A \textbf{fault} is a mistake in what someone is doing or in what they have done.
 \textit{
	\begin{itemize}
	\item It is a big fault to think that you can learn how to manage people in business school.
	\end{itemize}
}
\item countable noun \\
A \textbf{fault} in someone or something is a weakness in them or something that is not perfect .
 \textit{
	\begin{itemize}
	\item His manners had always made her blind to his faults.
	\item ...a short delay due to a minor technical fault.
	\item Pilots were trying to repair a fault in the plane when it crashed.
	\item For all its faults, the film presents a clear message.
	\end{itemize}
}
\item verb \\
If you \textbf{cannot}  \textbf{fault} someone, you cannot find any reason for criticizing them or the things that they are doing.
 \textit{
	\begin{itemize}
	\item You can't fault them for lack of invention.
	\item It is hard to fault the way he runs his own operation.
	\end{itemize}
}
\item countable noun \\
A \textbf{fault} is a large crack in the surface of the earth.
 \textit{
	\begin{itemize}
	\item ...the San Andreas Fault.
	\end{itemize}
}
\item countable noun \\
A \textbf{fault} in tennis is a service that is wrong  according to the rules.
 \textit{
	\begin{itemize}
	\end{itemize}
}
\item  \\
 at fault \textit{
	\begin{itemize}
	\end{itemize}
}
\item  \\
 to find fault with \textit{
	\begin{itemize}
	\end{itemize}
}
\item  \\
 to a fault \textit{
	\begin{itemize}
	\end{itemize}
}
\end{enumerate}

\section*{forget}
{\large \color{blue}  forgets  forgetting  forgot  forgotten  }
\subsection*{Explain}
\begin{enumerate}
\item verb \\
If you \textbf{forget} something or \textbf{forget} how to do something, you cannot think of it or think how to do it, although you knew it or knew how to do it in the past .
 \textit{
	\begin{itemize}
	\item Sometimes I improvise and change the words because I forget them.
	\item She forgot where she left the car and it took us two days to find it.
	\end{itemize}
}
\item verb \\
If you \textbf{forget} something or \textbf{forget} to do it, you fail to think about it or fail to remember to do it, for example because you are thinking about other things.
 \textit{
	\begin{itemize}
	\item She never forgets her daddy's birthday.
	\item She forgot to lock her door one day and two men got in.
	\item Don't forget that all dogs need a supply of fresh water to drink.
	\item She forgot about everything but the sun and the wind and the salt spray.
	\end{itemize}
}
\item verb \\
If you \textbf{forget} something that you had intended to bring with you, you do not bring it because you did not think about it at the right time.
 \textit{
	\begin{itemize}
	\item Once when we were going to Paris, I forgot my passport.
	\end{itemize}
}
\item verb \\
If you \textbf{forget} something or someone, you deliberately put them out of your mind and do not think about them any more.
 \textit{
	\begin{itemize}
	\item I hope you will forget the bad experience you had today.
	\item I can't forget what happened.
	\item I found it very easy to forget about Sumner.
	\item She tried to forget that sometimes she heard them quarrelling.
	\end{itemize}
}
\item verb \\
If you \textbf{forget}  \textbf{yourself} , you behave in an uncontrolled or unacceptable  way , which is not the way in which you usually behave.
 \textit{
	\begin{itemize}
	\item He was so fascinated by her beauty that he forgot himself and leaned across to touch
her.
	\end{itemize}
}
\item  \\
 forget it \textit{
	\begin{itemize}
	\end{itemize}
}
\item  \\
 not forgetting \textit{
	\begin{itemize}
	\end{itemize}
}
\end{enumerate}

\section*{floor}
{\large \color{blue}  floors  flooring  floored  }
\subsection*{Explain}
\begin{enumerate}
\item countable noun \\
\textbf{The}  \textbf{floor} of a room is the part of it that you walk on.
 \textit{
	\begin{itemize}
	\item Jack's sitting on the floor watching TV.
	\item We painted the wooden floor with a white stain.
	\end{itemize}
}
\item countable noun \\
A \textbf{floor} of a building is all the rooms that are on a particular level.
 \textit{
	\begin{itemize}
	\item It is on the fifth floor of the hospital.
	\item They occupied the first two floors of the tower.
	\end{itemize}
}
\item countable noun \\
The ocean  \textbf{floor} is the ground at the bottom of an ocean. The valley  \textbf{floor} is the ground at the bottom of a valley.
 \textit{
	\begin{itemize}
	\item They spend hours feeding on the ocean floor.
	\item ...a two-hour climb from the valley floor.
	\end{itemize}
}
\item countable noun \\
The place where official debates and discussions are held, especially between members of parliament , is referred to as \textbf{the}  \textbf{floor} .
 \textit{
	\begin{itemize}
	\item The issues were debated on the floor of the House.
	\end{itemize}
}
\item singular noun \\
In a debate or discussion, \textbf{the floor} is the people who are listening to the arguments being put forward but who are not among the main speakers .
 \textit{
	\begin{itemize}
	\item The president is taking questions from the floor.
	\end{itemize}
}
\item countable noun \\
\textbf{The}  \textbf{floor} of a stock exchange is the large open area where trading is done.
 \textit{
	\begin{itemize}
	\item ...the dealing floor at Standard Chartered Bank.
	\end{itemize}
}
\item countable noun \\
\textbf{The}  \textbf{floor} in a place such as a club or disco is the area where people dance .
 \textit{
	\begin{itemize}
	\end{itemize}
}
\item verb \\
If you \textbf{are floored}  \textbf{by} something, you are unable to respond to it because you are so surprised by it.
 \textit{
	\begin{itemize}
	\item He was floored by the announcement.
	\item He seemed floored by a string of scandals.
	\end{itemize}
}
\item verb \\
If someone \textbf{is floored} , especially in boxing , they are hit so hard that they fall over.
 \textit{
	\begin{itemize}
	\item He was floored twice in the second round.
	\end{itemize}
}
\item  \\
 take the floor \textit{
	\begin{itemize}
	\end{itemize}
}
\item  \\
 take to the floor \textit{
	\begin{itemize}
	\end{itemize}
}
\item  \\
 through the floor \textit{
	\begin{itemize}
	\end{itemize}
}
\item  \\
 to wipe the floor with someone \textit{
	\begin{itemize}
	\end{itemize}
}
\end{enumerate}

\section*{hear}
{\large \color{blue}  hears  hearing  heard  }
\subsection*{Explain}
\begin{enumerate}
\item verb \\
When you \textbf{hear} a sound, you become aware of it through your ears .
 \textit{
	\begin{itemize}
	\item She heard no further sounds.
	\item The trumpet can be heard all over their house.
	\item They heard the protesters shout: 'No more fascism!'.
	\item And then we heard the bells ringing out.
	\item I'm not hearing properly.
	\end{itemize}
}
\item verb \\
If you \textbf{hear} something such as a lecture or a piece of music , you listen to it.
 \textit{
	\begin{itemize}
	\item You can hear commentary on the match in about half an hour's time.
	\item I don't think you've ever heard Doris talking about her emotional life before.
	\item I'd love to hear it played by a professional orchestra.
	\end{itemize}
}
\item verb \\
If you say that you can \textbf{hear} someone saying something, you mean that you are able to imagine hearing it.
 \textit{
	\begin{itemize}
	\item Can't you just hear the clichés roll?
	\item 'I was hot,' I could still hear Charlotte say with her delicious French accent.
	\end{itemize}
}
\item verb \\
When a judge or a court of law  \textbf{hears} a case, or evidence in a case, they listen to it officially in order to make a decision about it.
 \textit{
	\begin{itemize}
	\item The jury have heard evidence from defence witnesses.
	\item He had to wait months before his case was heard.
	\end{itemize}
}
\item verb \\
If you \textbf{hear}  \textbf{from} someone, you receive a letter, email, or phone  call from them.
 \textit{
	\begin{itemize}
	\item Drop us a line, it's always great to hear from you.
	\item The police are anxious to hear from anyone who may know her.
	\end{itemize}
}
\item verb \\
In a debate or discussion , if you \textbf{hear from} someone, you listen to them giving their opinion or information.
 \textit{
	\begin{itemize}
	\item What are you hearing from people there?
	\end{itemize}
}
\item verb \\
If you \textbf{hear} some news or information about something, you find out about it by someone telling you, or from the radio or television .
 \textit{
	\begin{itemize}
	\item My mother heard of this school through Leslie.
	\item ...the rumours I've been hearing about for years.
	\item He had heard that the trophy had been sold.
	\item I had waited to hear the result.
	\end{itemize}
}
\item verb \\
If you \textbf{have heard}  \textbf{of} something or someone, you know about them, but not in great  detail .
 \textit{
	\begin{itemize}
	\item Many people haven't heard of reflexology.
	\item ...people who, maybe, had hardly heard the word till a year or two ago.
	\end{itemize}
}
\item  \\
 have heard sth before \textit{
	\begin{itemize}
	\end{itemize}
}
\item  \\
 do/did you hear (me)? \textit{
	\begin{itemize}
	\end{itemize}
}
\item  \\
 Hear,hear! \textit{
	\begin{itemize}
	\end{itemize}
}
\item  \\
 you can't hear yourself think \textit{
	\begin{itemize}
	\end{itemize}
}
\item  \\
 won't/wouldn't hear of sth \textit{
	\begin{itemize}
	\end{itemize}
}
\end{enumerate}

\section*{geography}
{\large \color{blue}  }
\subsection*{Explain}
\begin{enumerate}
\item uncountable noun \\
\textbf{Geography} is the study of the countries of the world and of such things as the land, seas, climate, towns, and population .
 \textit{
	\begin{itemize}
	\end{itemize}
}
\item uncountable noun \\
The \textbf{geography} of a place is the way that features such as rivers , mountains , towns, or streets are arranged within it.
 \textit{
	\begin{itemize}
	\item ...police officers who knew the local geography.
	\item ...a pictorial journey through the history, geography and culture of the Caribbean.
	\end{itemize}
}
\end{enumerate}

\section*{hesitate}
{\large \color{blue}  hesitates  hesitating  hesitated  }
\subsection*{Explain}
\begin{enumerate}
\item verb \\
If you \textbf{hesitate} , you do not speak or act for a short time, usually because you are uncertain, embarrassed , or worried about what you are going to say or do.
 \textit{
	\begin{itemize}
	\item The phone rang. Catherine hesitated, debating whether to answer it.
	\item She hesitated a long time and then she said 'Yes'.
	\end{itemize}
}
\item verb \\
If you \textbf{hesitate}  \textbf{to} do something, you delay doing it or are unwilling to do it, usually because you are not certain it would be right . If you do not \textbf{hesitate}  \textbf{to} do something, you do it immediately .
 \textit{
	\begin{itemize}
	\item Some parents hesitate to do this because they suspect their child is exaggerating.
	\item I hesitated to apply the word 'vulnerable' to him but it came into my mind.
	\item I will not hesitate to take unpopular decisions.
	\end{itemize}
}
\item verb \\
You can use \textbf{hesitate} in expressions such as ' \textbf{don't hesitate to call me} ' or ' \textbf{don't hesitate to contact us} ' when you are telling someone that they should do something as soon as it needs to be done and should not worry about disturbing other people.
 \textit{
	\begin{itemize}
	\item In the event of difficulties, please do not hesitate to contact our Customer Service
Department.
	\item Please don't hesitate to tell either Mr Schrader or myself should you feel ill again.
	\item Do not hesitate to laugh at anything you find amusing.
	\end{itemize}
}
\end{enumerate}

\section*{geology}
{\large \color{blue}  }
\subsection*{Explain}
\begin{enumerate}
\item uncountable noun \\
\textbf{Geology} is the study of the Earth's structure, surface, and origins.
 \textit{
	\begin{itemize}
	\item He was visiting professor of geology at the University of Jordan.
	\end{itemize}
}
\item uncountable noun \\
The \textbf{geology} of an area is the structure of its land, together with the types of rocks and minerals that exist within it.
 \textit{
	\begin{itemize}
	\item ...an expert on the geology of southeast Asia.
	\end{itemize}
}
\end{enumerate}

\section*{hijack}
{\large \color{blue}  hijacks  hijacking  hijacked  }
\subsection*{Explain}
\begin{enumerate}
\item verb \\
If someone \textbf{hijacks} a plane or other vehicle, they illegally take control of it by force while it is travelling from one place to another.
 \textbf{Hijack} is also a noun .
 \textit{
	\begin{itemize}
	\item ...a plot to hijack an airliner.
	\item A chemical tanker with 26 crew was hijacked by pirates on Monday.
	\item The hijacked plane exploded in a ball of fire.
	\item Every minute during the hijack seemed like a week.
	\end{itemize}
}
\item verb \\
If you say that someone \textbf{has hijacked} something, you disapprove of the way in which they have taken control of it when they had no right to do so.
 \textit{
	\begin{itemize}
	\item A peaceful demonstration had been hijacked by anarchists intent on causing trouble.
	\end{itemize}
}
\end{enumerate}

\section*{harmony}
{\large \color{blue}  harmonies  }
\subsection*{Explain}
\begin{enumerate}
\item uncountable noun \\
If people are living \textbf{in}  \textbf{harmony}  \textbf{with} each other, they are living together peacefully rather than fighting or arguing .
 \textit{
	\begin{itemize}
	\item We must try to live in peace and harmony with ourselves and those around us.
	\item He projected himself as the protector of national unity and harmony.
	\end{itemize}
}
\item variable noun \\
\textbf{Harmony} is the pleasant combination of different notes of music played at the same time.
 \textit{
	\begin{itemize}
	\item ...complex vocal harmonies.
	\item ...singing in harmony.
	\end{itemize}
}
\item uncountable noun \\
The \textbf{harmony} of something is the way in which its parts are combined into a pleasant arrangement .
 \textit{
	\begin{itemize}
	\item ...the ordered harmony of the universe.
	\item He looked more relaxed, as if some of the harmony from his surroundings had flowed
into him.
	\end{itemize}
}
\end{enumerate}

\section*{identify}
{\large \color{blue}  identifies  identifying  identified  }
\subsection*{Explain}
\begin{enumerate}
\item verb \\
If you can \textbf{identify} someone or something, you are able to recognize them or distinguish them from others.
 \textit{
	\begin{itemize}
	\item There are a number of distinguishing characteristics by which you can identify a
Hollywood epic.
	\item I tried to identify her perfume.
	\item A uniformed chauffeur identified me among the crowd.
	\end{itemize}
}
\item verb \\
If you \textbf{identify} someone or something, you name them or say who or what they are.
 \textit{
	\begin{itemize}
	\item Police have already identified around 10 murder suspects.
	\item The reporters identified one of the six Americans as an Army Specialist.
	\item They identified six plants as having potential for development into pharmaceutical
drugs.
	\end{itemize}
}
\item verb \\
If you \textbf{identify} something, you discover or notice its existence .
 \textit{
	\begin{itemize}
	\item Scientists claim to have identified natural substances with cancer-combating properties.
	\item Having identified the problem, the question arises of how to overcome it.
	\end{itemize}
}
\item verb \\
If a particular thing \textbf{identifies} someone or something, it makes them easy to recognize, by making them different in some way.
 \textit{
	\begin{itemize}
	\item She wore a little nurse's hat on her head to identify her.
	\item His boots and purple beret identify him as commanding the Scottish Paratroops.
	\end{itemize}
}
\item verb \\
If you \textbf{identify with} someone or something, you feel that you understand them or their feelings and ideas .
 \textit{
	\begin{itemize}
	\item She would only play a role if she could identify with the character.
	\item I could speak their language and identify with their problems because I had been
there myself.
	\end{itemize}
}
\item verb \\
If you \textbf{identify} one person or thing \textbf{with} another, you think that they are closely associated or involved in some way.
 \textit{
	\begin{itemize}
	\item She hates playing the types of women that audiences identify her with.
	\item The candidates all want to identify themselves with reform.
	\end{itemize}
}
\end{enumerate}

\section*{hell}
{\large \color{blue}  hells  }
\subsection*{Explain}
\begin{enumerate}
\item proper noun \\
In some religions, \textbf{hell} is the place where the Devil lives, and where wicked people are sent to be punished when they die . Hell is usually imagined as being under the ground and full of flames .
 \textit{
	\begin{itemize}
	\end{itemize}
}
\item variable noun \\
If you say that a particular situation or place is \textbf{hell} , you are emphasizing that it is extremely  unpleasant .
 \textit{
	\begin{itemize}
	\item ...the hell of the Siberian labor camps.
	\item Bullies can make your life hell.
	\item ...the hells of grief and shame and lost love.
	\end{itemize}
}
\item exclamation \\
\textbf{Hell} is used by some people when they are angry or excited , or when they want to emphasize what they are saying .
 \textit{
	\begin{itemize}
	\item 'Hell, no!' the doctor snapped.
	\end{itemize}
}
\item  \\
 as hell \textit{
	\begin{itemize}
	\end{itemize}
}
\item  \\
 hell on earth \textit{
	\begin{itemize}
	\end{itemize}
}
\item  \\
 for the hell of it \textit{
	\begin{itemize}
	\end{itemize}
}
\item  \\
 until hell freezes over \textit{
	\begin{itemize}
	\end{itemize}
}
\item  \\
 from hell \textit{
	\begin{itemize}
	\end{itemize}
}
\item  \\
 to give someone hell \textit{
	\begin{itemize}
	\end{itemize}
}
\item  \\
 give sb hell \textit{
	\begin{itemize}
	\end{itemize}
}
\item  \\
 go to hell \textit{
	\begin{itemize}
	\end{itemize}
}
\item  \\
 go to hell \textit{
	\begin{itemize}
	\end{itemize}
}
\item  \\
 hell for leather \textit{
	\begin{itemize}
	\end{itemize}
}
\item  \\
 like hell \textit{
	\begin{itemize}
	\end{itemize}
}
\item  \\
 like hell \textit{
	\begin{itemize}
	\end{itemize}
}
\item  \\
 living hell \textit{
	\begin{itemize}
	\end{itemize}
}
\item  \\
 all hell breaks loose \textit{
	\begin{itemize}
	\end{itemize}
}
\item  \\
 a/one hell of a lot \textit{
	\begin{itemize}
	\end{itemize}
}
\item  \\
 a/one hell of \textit{
	\begin{itemize}
	\end{itemize}
}
\item  \\
 get the hell out \textit{
	\begin{itemize}
	\end{itemize}
}
\item  \\
 the hell out of \textit{
	\begin{itemize}
	\end{itemize}
}
\item  \\
 there'll be hell to pay \textit{
	\begin{itemize}
	\end{itemize}
}
\item  \\
 play hell \textit{
	\begin{itemize}
	\end{itemize}
}
\item  \\
 to raise hell \textit{
	\begin{itemize}
	\end{itemize}
}
\item  \\
 the hell \textit{
	\begin{itemize}
	\end{itemize}
}
\item  \\
 go through hell \textit{
	\begin{itemize}
	\end{itemize}
}
\item  \\
 hope to hell/wish to hell \textit{
	\begin{itemize}
	\end{itemize}
}
\item  \\
 come hell or high water \textit{
	\begin{itemize}
	\end{itemize}
}
\item  \\
 what the hell \textit{
	\begin{itemize}
	\end{itemize}
}
\item  \\
 to hell with \textit{
	\begin{itemize}
	\end{itemize}
}
\end{enumerate}

\section*{improve}
{\large \color{blue}  improves  improving  improved  }
\subsection*{Explain}
\begin{enumerate}
\item verb \\
If something \textbf{improves} or if you \textbf{improve} it, it gets better.
 \textit{
	\begin{itemize}
	\item Both the texture and condition of your hair should improve.
	\item The weather is beginning to improve.
	\item Time won't improve the situation.
	\item He improved their house.
	\end{itemize}
}
\item verb \\
If a skill you have \textbf{improves} or you \textbf{improve} a skill, you get better at it.
 \textit{
	\begin{itemize}
	\item Their French has improved enormously.
	\item He said he was going to improve his football.
	\end{itemize}
}
\item verb \\
If you \textbf{improve} after an illness or an injury , your health gets better or you get stronger .
 \textit{
	\begin{itemize}
	\item He had improved so much the doctor had cut his dosage.
	\end{itemize}
}
\item verb \\
If you \textbf{improve}  \textbf{on} a previous  achievement of your own or of someone else, you achieve a better standard or result .
 \textit{
	\begin{itemize}
	\item We need to improve on our performance against France.
	\end{itemize}
}
\end{enumerate}

\section*{horizon}
{\large \color{blue}  horizons  }
\subsection*{Explain}
\begin{enumerate}
\item singular noun \\
\textbf{The}  \textbf{horizon} is the line in the far  distance where the sky seems to meet the land or the sea .
 \textit{
	\begin{itemize}
	\item A grey smudge appeared on the horizon. That must be Calais, thought Fay.
	\item The sun had already sunk below the horizon.
	\end{itemize}
}
\item countable noun \\
Your \textbf{horizons} are the limits of what you want to do or of what you are interested or involved in.
 \textit{
	\begin{itemize}
	\item As your horizons expand, these new ideas can give a whole new meaning to life.
	\item Gary has exceptional vision and is always looking to broaden his horizons.
	\end{itemize}
}
\item  \\
 on the horizon \textit{
	\begin{itemize}
	\end{itemize}
}
\end{enumerate}

\section*{justify}
{\large \color{blue}  justifies  justifying  justified  }
\subsection*{Explain}
\begin{enumerate}
\item verb \\
To \textbf{justify} a decision , action, or idea means to show or prove that it is reasonable or necessary .
 \textit{
	\begin{itemize}
	\item No argument can justify a war.
	\item Ministers agreed that this decision was fully justified by economic conditions.
	\end{itemize}
}
\item verb \\
To \textbf{justify}  printed  text means to adjust the spaces between the words so that each line of type is exactly
the same length.
 \textit{
	\begin{itemize}
	\item Click on this icon to align or justify text.
	\end{itemize}
}
\end{enumerate}

\section*{linen}
{\large \color{blue}  linens  }
\subsection*{Explain}
\begin{enumerate}
\item variable noun \\
\textbf{Linen} is a kind of cloth that is made from a plant called flax. It is used for making clothes and things such as tablecloths and sheets.
 \textit{
	\begin{itemize}
	\item ...a white linen suit.
	\item ...cottons, woolens, silks and linens.
	\end{itemize}
}
\item uncountable noun \\
\textbf{Linen} is tablecloths, sheets, pillowcases , and similar things made of cloth that are used in the home .
 \textit{
	\begin{itemize}
	\item ...embroidered bed linen.
	\item All linens and towels are provided.
	\end{itemize}
}
\end{enumerate}

\section*{link}
{\large \color{blue}  links  linking  linked  }
\subsection*{Explain}
\begin{enumerate}
\item countable noun \\
If there is a \textbf{link}  \textbf{between} two things or situations , there is a relationship between them, for example because one thing causes or affects the other.
 \textit{
	\begin{itemize}
	\item ...the link between smoking and lung cancer.
	\end{itemize}
}
\item verb \\
If someone or something \textbf{links} two things or situations, there is a relationship between them, for example because
one thing causes or affects the other.
 \textit{
	\begin{itemize}
	\item The U.N. Security Council has linked any lifting of sanctions to compliance with
the ceasefire terms.
	\item The study further strengthens the evidence linking smoking with early death.
	\item Liver cancer is linked to the hepatitis B virus.
	\item The detention raised two distinct but closely linked questions.
	\end{itemize}
}
\item countable noun \\
A \textbf{link}  \textbf{between} two things or places is a physical connection between them.
 \textit{
	\begin{itemize}
	\item ...the high-speed rail link between London and the Channel Tunnel.
	\item The new road schemes include a link between the two cities.
	\item Stalin insisted that the radio link with the German Foreign Ministry should remain
open.
	\end{itemize}
}
\item verb \\
If two places or objects \textbf{are linked} or something \textbf{links} them, there is a physical connection between them.
 \textit{
	\begin{itemize}
	\item ...the Rama Road, which links the capital, Managua, with the Caribbean coast.
	\item The campus is linked by regular bus services to Coventry.
	\item ...the Channel Tunnel linking Britain and France.
	\end{itemize}
}
\item countable noun \\
A \textbf{link} between two people, organizations, or places is a friendly or business connection between them.
 \textit{
	\begin{itemize}
	\item The university has close links with local employers.
	\item In 1984 the long link between AC Cars and the Hurlock family was severed.
	\item A cabinet minister came under investigation for links to the Mafia.
	\end{itemize}
}
\item countable noun \\
A \textbf{link} to another person or organization is something that allows you to communicate with them or have contact with them.
 \textit{
	\begin{itemize}
	\item She was my only link with the past.
	\item The Red Cross was created to provide a link between soldiers in battle and their
families at home.
	\item These projects will provide vital links between companies and universities.
	\end{itemize}
}
\item verb \\
If you \textbf{link} one person or thing to another, you claim that there is a relationship or connection between them.
 \textit{
	\begin{itemize}
	\item Criminologist Dr Ann Jones has linked the crime to social circumstances.
	\item The tabloids have linked him with various women.
	\item ...a report in The Sunday Times linking him with organized crime.
	\end{itemize}
}
\item countable noun \\
In computing , a \textbf{link} is a connection between different documents , or between different parts of the same document, using hypertext .
 \textbf{Link} is also a verb .
 \textit{
	\begin{itemize}
	\item Certainly, Andreessen didn't think up using hypertext to link Internet documents.
	\end{itemize}
}
\item countable noun \\
A \textbf{link} is one of the rings in a chain.
 \textit{
	\begin{itemize}
	\end{itemize}
}
\item verb \\
If you \textbf{link} one thing with another, you join them by putting one thing through the other.
 \textit{
	\begin{itemize}
	\item She linked her arm through his.
	\item He linked the fingers of his hands together on his stomach.
	\end{itemize}
}
\end{enumerate}

\section*{locomotive}
{\large \color{blue}  locomotives  }
\subsection*{Explain}
\begin{enumerate}
\item countable noun \\
A \textbf{locomotive} is a large vehicle that pulls a railway train.
 \textit{
	\begin{itemize}
	\item Steam locomotives pumped out clouds of white smoke.
	\end{itemize}
}
\end{enumerate}

\section*{mislead}
{\large \color{blue}  misleads  misleading  misled  }
\subsection*{Explain}
\begin{enumerate}
\item verb \\
If you say that someone \textbf{has misled} you, you mean that they have made you believe something which is not true , either by telling you a lie or by giving you a wrong idea or impression .
 \textit{
	\begin{itemize}
	\item Jack was furious with his London doctors for having misled him.
	\item Ministers must not knowingly mislead Parliament and the public.
	\end{itemize}
}
\end{enumerate}

\section*{loudspeaker}
{\large \color{blue}  loudspeakers  }
\subsection*{Explain}
\begin{enumerate}
\item countable noun \\
A \textbf{loudspeaker} is a piece of equipment , for example part of a radio or hi-fi system, through which sound comes out.
 \textit{
	\begin{itemize}
	\end{itemize}
}
\end{enumerate}

\section*{modify}
{\large \color{blue}  modifies  modifying  modified  }
\subsection*{Explain}
\begin{enumerate}
\item verb \\
If you \textbf{modify} something, you change it slightly , usually in order to improve it.
 \textit{
	\begin{itemize}
	\item The club members did agree to modify their recruitment policy.
	\item The plane was a modified version of the C-130.
	\end{itemize}
}
\item verb \\
A word or group of words that \textbf{modifies} another word describes or classifies something, or restricts the meaning of the word.
 \textit{
	\begin{itemize}
	\item It is a rule of English that adjectives generally precede the noun they modify: we
say 'a good cry', not 'a cry good'.
	\end{itemize}
}
\end{enumerate}

\section*{machine}
{\large \color{blue}  machines  machining  machined  }
\subsection*{Explain}
\begin{enumerate}
\item countable noun \\
A \textbf{machine} is a piece of equipment which uses electricity or an engine in order to do a particular kind of work.
 \textit{
	\begin{itemize}
	\item I put the coin in the machine and pulled the lever.
	\item The machine can be remotely operated and monitored.
	\item ...a color photograph of the sort taken by machine to be pasted in passports.
	\end{itemize}
}
\item verb \\
If you \textbf{machine} something, you make it or work on it using a machine.
 \textit{
	\begin{itemize}
	\item The material is machined in a factory.
	\item All parts are machined from top grade, high tensile aluminium.
	\item ...machined brass zinc alloy gears.
	\end{itemize}
}
\item countable noun \\
You can use \textbf{machine} to refer to a large and well-controlled system or organization.
 \textit{
	\begin{itemize}
	\item ...Nazi Germany's military machine.
	\item He has put the party publicity machine behind another candidate.
	\end{itemize}
}
\item countable noun \\
If you say that someone is a \textbf{machine} , you mean that they are so tired or bored that they do their work without thinking .
 \textit{
	\begin{itemize}
	\item I think I have got to stop or I might turn into a machine.
	\item He has dedicated himself to his work and become just a writing machine.
	\end{itemize}
}
\end{enumerate}

\section*{persevere}
{\large \color{blue}  perseveres  persevering  persevered  }
\subsection*{Explain}
\begin{enumerate}
\item verb \\
If you \textbf{persevere}  \textbf{with} something, you keep trying to do it and do not give up, even though it is difficult .
 \textit{
	\begin{itemize}
	\item ...his ability to persevere despite obstacles and setbacks.
	\item ...a school with a reputation for persevering with difficult and disruptive children.
	\item She persevered in her idea despite obvious objections raised by friends.
	\end{itemize}
}
\end{enumerate}

\section*{magistrate}
{\large \color{blue}  magistrates  }
\subsection*{Explain}
\begin{enumerate}
\item countable noun \\
A \textbf{magistrate} is an official who acts as a judge in law courts which deal with minor  crimes or disputes .
 \textit{
	\begin{itemize}
	\end{itemize}
}
\end{enumerate}

\section*{ponder}
{\large \color{blue}  ponders  pondering  pondered  }
\subsection*{Explain}
\begin{enumerate}
\item verb \\
If you \textbf{ponder} something, you think about it carefully.
 \textit{
	\begin{itemize}
	\item I found myself constantly pondering the question: 'How could anyone do these things?'
	\item The Prime Minister pondered on when to go to the polls.
	\item I'm continually pondering how to improve the team.
	\end{itemize}
}
\end{enumerate}

\section*{map}
{\large \color{blue}  maps  mapping  mapped  }
\subsection*{Explain}
\begin{enumerate}
\item countable noun \\
A \textbf{map} is a drawing of a particular area such as a city, a country, or a continent , showing its main features as they would appear if you looked at them from above.
 \textit{
	\begin{itemize}
	\item He unfolded the map and set it on the floor.
	\item Have you got a map of the city centre?
	\end{itemize}
}
\item countable noun \\
A \textbf{map} is a drawing that gives special  information about an area.
 \textit{
	\begin{itemize}
	\item ...geological maps, books and atlases.
	\item ...weather maps on television.
	\end{itemize}
}
\item verb \\
To \textbf{map} an area means to make a map of it.
 \textit{
	\begin{itemize}
	\item ...a spacecraft which is using radar to map the surface of Venus.
	\item ...better mapping of the ocean floor.
	\end{itemize}
}
\item  \\
 put sb/sth/swh on the map \textit{
	\begin{itemize}
	\end{itemize}
}
\end{enumerate}

\section*{prove}
{\large \color{blue}  proves  proving  proved  proved  proven  }
\subsection*{Explain}
\begin{enumerate}
\item link verb \\
If something \textbf{proves}  \textbf{to} be true or \textbf{to} have a particular quality, it becomes clear after a period of time that it is true or has that quality.
 \textit{
	\begin{itemize}
	\item None of the rumours has ever been proved to be true.
	\item In the past this process of transition has often proven difficult.
	\item ...an experiment which was to prove a source of inspiration for many years to come.
	\end{itemize}
}
\item verb \\
If you \textbf{prove}  \textbf{that} something is true, you show by means of argument or evidence that it is definitely true.
 \textit{
	\begin{itemize}
	\item You brought this charge. You prove it!
	\item The results prove that regulation of the salmon farming industry is inadequate.
	\item ...trying to prove how groups of animals have evolved.
	\item That made me hopping mad and determined to prove him wrong.
	\item History will prove him to have been right all along.
	\item ...a proven cause of cancer.
	\end{itemize}
}
\item verb \\
If you \textbf{prove}  \textbf{yourself to} have a certain good quality, you show by your actions that you have it.
 \textit{
	\begin{itemize}
	\item Margaret proved herself to be a good mother.
	\item As a composer he proved himself adept at large dramatic forms.
	\item A man needs time to prove himself.
	\item Few would argue that this team has experience and proven ability.
	\end{itemize}
}
\item  \\
 prove a point \textit{
	\begin{itemize}
	\end{itemize}
}
\end{enumerate}

\section*{mechanism}
{\large \color{blue}  mechanisms  }
\subsection*{Explain}
\begin{enumerate}
\item countable noun \\
In a machine or piece of equipment , a \textbf{mechanism} is a part, often consisting of a set of smaller parts, which performs a particular
function.
 \textit{
	\begin{itemize}
	\item ...the locking mechanism.
	\item A bomb has been detonated by a special mechanism.
	\end{itemize}
}
\item countable noun \\
A \textbf{mechanism} is a special way of getting something done within a particular system.
 \textit{
	\begin{itemize}
	\item There's no mechanism for punishing arms exporters who break the rules.
	\item ...the clumsy and ineffective mechanism of price controls.
	\end{itemize}
}
\item countable noun \\
A \textbf{mechanism} is a part of your behaviour that is automatic and that helps you to survive or to cope with a difficult  situation .
 \textit{
	\begin{itemize}
	\item ...a survival mechanism, a means of coping with intolerable stress.
	\end{itemize}
}
\end{enumerate}

\section*{refuse}
{\large \color{blue}  refuses  refusing  refused  }
\subsection*{Explain}
\begin{enumerate}
\item verb \\
If you \textbf{refuse}  \textbf{to} do something, you deliberately do not do it, or you say firmly that you will not do it.
 \textit{
	\begin{itemize}
	\item He refused to comment after the trial.
	\item He expects me to stay on here and I can hardly refuse.
	\end{itemize}
}
\item verb \\
If someone \textbf{refuses} you something, they do not give it to you or do not allow you to have it.
 \textit{
	\begin{itemize}
	\item The United States has refused him a visa.
	\item She was refused access to her children.
	\item The town council had refused permission for the march.
	\end{itemize}
}
\item verb \\
If you \textbf{refuse} something that is offered to you, you do not accept it.
 \textit{
	\begin{itemize}
	\item He offered me a second drink which I refused.
	\item The patient has the right to refuse treatment.
	\end{itemize}
}
\item uncountable noun \\
\textbf{Refuse} consists of the rubbish and all the things that are not wanted in a house , shop , or factory , and that are regularly thrown away; used mainly in official language.
 \textit{
	\begin{itemize}
	\item The District Council made a weekly collection of refuse.
	\end{itemize}
}
\end{enumerate}

\section*{northwest}
{\large \color{blue}  }
\subsection*{Explain}
\begin{enumerate}
\item noun \\
1.  2.  \textit{
	\begin{itemize}
	\end{itemize}
}
\item adjective \\
3.  \textit{
	\begin{itemize}
	\item northwest Greenland
	\end{itemize}
}
\item adjective \\
4.  \textit{
	\begin{itemize}
	\end{itemize}
}
\end{enumerate}

\section*{renovate}
{\large \color{blue}  renovates  renovating  renovated  }
\subsection*{Explain}
\begin{enumerate}
\item verb \\
If someone \textbf{renovates} an old building, they repair and improve it and get it back into good condition.
 \textit{
	\begin{itemize}
	\item The couple spent thousands renovating the house.
	\item She lives in a large, renovated farmhouse.
	\end{itemize}
}
\end{enumerate}

\section*{occasion}
{\large \color{blue}  occasions  occasioning  occasioned  }
\subsection*{Explain}
\begin{enumerate}
\item countable noun \\
An \textbf{occasion} is a time when something happens , or a case of it happening.
 \textit{
	\begin{itemize}
	\item I often think fondly of an occasion some years ago at Covent Garden.
	\item Mr Davis has been asked on a number of occasions.
	\end{itemize}
}
\item countable noun \\
An \textbf{occasion} is an important event, ceremony , or celebration.
 \textit{
	\begin{itemize}
	\item Taking her with me on official occasions has been a challenge.
	\item It will be a unique family occasion.
	\end{itemize}
}
\item countable noun \\
An \textbf{occasion for} doing something is an opportunity for doing it.
 \textit{
	\begin{itemize}
	\item It is an occasion for all the family to celebrate.
	\item It is always an important occasion for setting out government policy.
	\end{itemize}
}
\item verb \\
To \textbf{occasion} something means to cause it.
 \textit{
	\begin{itemize}
	\item He argued that the release of hostages should not occasion a change in policy.
	\end{itemize}
}
\item  \\
 have occasion \textit{
	\begin{itemize}
	\end{itemize}
}
\item  \\
 on occasion \textit{
	\begin{itemize}
	\end{itemize}
}
\item  \\
 to rise to the occasion \textit{
	\begin{itemize}
	\end{itemize}
}
\end{enumerate}

\section*{scratch}
{\large \color{blue}  scratches  scratching  scratched  }
\subsection*{Explain}
\begin{enumerate}
\item verb \\
If you \textbf{scratch}  \textbf{yourself} , you rub your fingernails against your skin because it is itching.
 \textit{
	\begin{itemize}
	\item He scratched himself under his arm.
	\item The old man lifted his cardigan to scratch his side.
	\item I had to wear long sleeves to stop myself scratching.
	\end{itemize}
}
\item verb \\
If a sharp object \textbf{scratches} someone or something, it makes small shallow cuts on their skin or surface.
 \textit{
	\begin{itemize}
	\item The branches tore at my jacket and scratched my hands and face.
	\item Knives will scratch the worktop.
	\end{itemize}
}
\item countable noun \\
\textbf{Scratches} on someone or something are small shallow cuts.
 \textit{
	\begin{itemize}
	\item The seven-year-old was found crying with scratches on his face and neck.
	\item I pointed to a number of scratches in the tile floor.
	\end{itemize}
}
\item  \\
 from scratch \textit{
	\begin{itemize}
	\end{itemize}
}
\item  \\
 scratch one's head \textit{
	\begin{itemize}
	\end{itemize}
}
\item  \\
 scratch the surface \textit{
	\begin{itemize}
	\end{itemize}
}
\item  \\
 up to scratch \textit{
	\begin{itemize}
	\end{itemize}
}
\end{enumerate}

\section*{opportunity}
{\large \color{blue}  opportunities  }
\subsection*{Explain}
\begin{enumerate}
\item variable noun \\
An \textbf{opportunity} is a situation in which it is possible for you to do something that you want to do.
 \textit{
	\begin{itemize}
	\item I had an opportunity to go to New York and study.
	\item The best reason for a trip to London is the super opportunity for shopping.
	\item I want to see more opportunities for young people.
	\item ...equal opportunities in employment.
	\end{itemize}
}
\end{enumerate}

\section*{seize}
{\large \color{blue}  seizes  seizing  seized  }
\subsection*{Explain}
\begin{enumerate}
\item verb \\
If you \textbf{seize} something, you take hold of it quickly, firmly, and forcefully.
 \textit{
	\begin{itemize}
	\item 'Leigh,' he said, seizing my arm to hold me back.
	\item ...an otter seizing a fish.
	\end{itemize}
}
\item verb \\
When a group of people \textbf{seize} a place or \textbf{seize}  control of it, they take control of it quickly and suddenly , using force.
 \textit{
	\begin{itemize}
	\item Troops have seized the airport and railroad terminals.
	\item Army officers plotted a failed attempt yesterday to seize power.
	\end{itemize}
}
\item verb \\
If a government or other authority  \textbf{seize} someone's property , they take it from them, often by force.
 \textit{
	\begin{itemize}
	\item Police were reported to have seized all copies of this morning's edition of the newspaper.
	\item Bailiffs need a certificate from the county court to seize goods for rent arrears.
	\end{itemize}
}
\item verb \\
When someone \textbf{is seized} , they are arrested or captured.
 \textit{
	\begin{itemize}
	\item Two military observers were seized by enemy troops yesterday.
	\item Men carrying sub-machine guns seized the five soldiers and drove them away.
	\end{itemize}
}
\item verb \\
When you \textbf{seize} an opportunity , you take advantage of it and do something that you want to do.
 \textit{
	\begin{itemize}
	\item During the riots hundreds of people seized the opportunity to steal property.
	\item The government now hopes to seize the initiative on education.
	\end{itemize}
}
\end{enumerate}

\section*{overpass}
{\large \color{blue}  overpasses  }
\subsection*{Explain}
\begin{enumerate}
\item countable noun \\
An \textbf{overpass} is a structure which carries one road over the top of another one.
 \textit{
	\begin{itemize}
	\item ...a $16 million highway overpass over Route 1.
	\end{itemize}
}
\end{enumerate}

\section*{smuggle}
{\large \color{blue}  smuggles  smuggling  smuggled  }
\subsection*{Explain}
\begin{enumerate}
\item verb \\
If someone \textbf{smuggles} things or people into a place or out of it, they take them there illegally or secretly.
 \textit{
	\begin{itemize}
	\item ...speculation that the Arctic Sea is being used for smuggling weapons.
	\item The gang was allegedly smuggling migrants from Calais to Britain.
	\item Had it really been impossible to find someone who could smuggle out a letter?
	\item Some of the trucks carried smuggled goods.
	\end{itemize}
}
\end{enumerate}

\section*{place}
{\large \color{blue}  places  placing  placed  }
\subsection*{Explain}
\begin{enumerate}
\item countable noun \\
A \textbf{place} is any point, building, area, town, or country.
 \textit{
	\begin{itemize}
	\item ...Temple Mount, the place where the Temple actually stood.
	\item ...a list of museums and places of interest.
	\item We're going to a place called Mont-St-Jean.
	\item ...the opportunity to visit new places.
	\item The best place to catch fish on a canal is close to a lock.
	\item The pain is always in the same place.
	\end{itemize}
}
\item singular noun \\
You can use \textbf{the place} to refer to the point, building, area, town, or country that you have already  mentioned .
 \textit{
	\begin{itemize}
	\item Except for the remarkably tidy kitchen, the place was a mess.
	\item For a ruin it was in good condition, as though the place was still being used.
	\end{itemize}
}
\item countable noun \\
You can refer to somewhere that provides a service, such as a hotel , restaurant , or institution, as a particular kind of \textbf{place} .
 \textit{
	\begin{itemize}
	\item He found a bed-and-breakfast place.
	\item We discovered some superb places to eat.
	\item My hospital is one of many places that benefited from the queen's support.
	\end{itemize}
}
\item  \\
 take place \textit{
	\begin{itemize}
	\end{itemize}
}
\item singular noun \\
\textbf{Place} can be used after 'any', 'no', 'some', or 'every' to mean ' anywhere ', ' nowhere ', 'somewhere', or ' everywhere '.
 \textit{
	\begin{itemize}
	\item The poor guy obviously didn't have any place to go for Easter.
	\item Why not go out and see if there's some place we can dance?
	\end{itemize}
}
\item adverb \\
If you go \textbf{places} , you visit  pleasant or interesting places.
 \textit{
	\begin{itemize}
	\item I don't have money to go places.
	\item People were talking to him, listening to him, taking him places.
	\end{itemize}
}
\item countable noun \\
You can refer to the position where something belongs, or where it is supposed to be, as its \textbf{place} .
 \textit{
	\begin{itemize}
	\item He returned the album to its place on the shelf.
	\item He returned to his place on the sofa.
	\end{itemize}
}
\item countable noun \\
A \textbf{place} is a seat or position that is available for someone to occupy.
 \textit{
	\begin{itemize}
	\item He walked back to the table and sat at the nearest of two empty places.
	\item I found a place to park beside a station wagon.
	\end{itemize}
}
\item countable noun \\
Someone's or something's \textbf{place} in a society, system, or situation is their position in relation to other people
or things.
 \textit{
	\begin{itemize}
	\item They want to see more women take their place higher up the corporate or professional
ladder.
	\item ...the important place of Christianity in our national culture.
	\end{itemize}
}
\item countable noun \\
Your \textbf{place} in a race or competition is your position in relation to the other competitors . If you are in first place, you are ahead of all the other competitors.
 \textit{
	\begin{itemize}
	\item He won third place in the Latin America's 50 Best Restaurants awards.
	\item He has risen to second place in the opinion polls.
	\end{itemize}
}
\item countable noun \\
If you get a \textbf{place} in a team, on a committee , or on a course of study, for example, you are accepted as a member of the team or
committee or as a student on the course.
 \textit{
	\begin{itemize}
	\item He has found a place in the first team.
	\item All the candidates won places on the ruling council.
	\item I eventually got a place at York University.
	\item They should be in residential care but there are no places available.
	\item To book your place fill in the coupon on page 187 and return it by 1st October.
	\end{itemize}
}
\item singular noun \\
A good \textbf{place}  \textbf{to} do something in a situation or activity is a good time or stage at which to do it.
 \textit{
	\begin{itemize}
	\item It seemed an appropriate place to end somehow.
	\item This is not the place for a lengthy discussion.
	\end{itemize}
}
\item countable noun \\
Your \textbf{place} is the house or flat where you live.
 \textit{
	\begin{itemize}
	\item Let's all go back to my place!
	\item He kept encouraging Rosie to find a place of her own.
	\end{itemize}
}
\item countable noun \\
Your \textbf{place}  \textbf{in} a book or speech is the point you have reached in reading the book or making the speech.
 \textit{
	\begin{itemize}
	\item ...her finger marking her place in the book.
	\item He lost his place in his notes.
	\end{itemize}
}
\item countable noun \\
If you say how many decimal  \textbf{places} there are in a number, you are saying how many numbers there are to the right of the decimal point.
 \textit{
	\begin{itemize}
	\item A pocket calculator only works to eight decimal places.
	\end{itemize}
}
\item verb \\
If you \textbf{place} something somewhere, you put it in a particular position, especially in a careful , firm , or deliberate way.
 \textit{
	\begin{itemize}
	\item Brand folded it in his handkerchief and placed it in the inside pocket of his jacket.
	\item Chairs were hastily placed in rows for the parents.
	\end{itemize}
}
\item verb \\
To \textbf{place} a person or thing in a particular state means to cause them to be in it.
 \textit{
	\begin{itemize}
	\item Widespread protests have placed the President under serious pressure.
	\item The crisis could well place the relationship at risk.
	\item The remaining 30 percent of each army will be placed under U.N. control.
	\end{itemize}
}
\item verb \\
You can use \textbf{place}  instead of 'put' or ' lay ' in certain expressions where the meaning is carried by the following noun. For example,
if you \textbf{place emphasis}  \textbf{on} something, you emphasize it, and if you \textbf{place the blame}  \textbf{on} someone, you blame them.
 \textit{
	\begin{itemize}
	\item We should teach the young by placing responsibility on them.
	\item He placed great emphasis on the importance of family life and ties.
	\item She seemed to be placing most of the blame on her mother.
	\item His government is placing its faith in international diplomacy.
	\end{itemize}
}
\item verb \\
If you \textbf{place} someone or something in a particular class or group, you label or judge them in that way.
 \textit{
	\begin{itemize}
	\item You take a simple written and verbal test so you can be placed in the appropriate
class.
	\item This logic places stupidity on the same level as intelligence.
	\end{itemize}
}
\item verb \\
If a competitor \textbf{is placed} first, second, or last, for example, that is their position at the end of a race
or competition. In American English, \textbf{be placed} often means 'finish in second position'.
 \textit{
	\begin{itemize}
	\item I had been placed 2nd and 3rd a few times but had never won.
	\item Second-placed Auxerre suffered a surprising 2-0 home defeat to Nantes.
	\end{itemize}
}
\item verb \\
If you \textbf{place} an order \textbf{for} a product or \textbf{for} a meal , you ask for it to be sent or brought to you.
 \textit{
	\begin{itemize}
	\item It is a good idea to place your order well in advance.
	\item Before placing your order for a meal, study the menu.
	\end{itemize}
}
\item verb \\
If you \textbf{place} an advertisement \textbf{in} a newspaper, you arrange for the advertisement to appear in the newspaper.
 \textit{
	\begin{itemize}
	\item They placed an advertisement in the local paper for a secretary.
	\end{itemize}
}
\item verb \\
If you \textbf{place} a phone call to a particular place, you give the telephone operator the number of the person you want to speak to and ask them to connect you.
 \textit{
	\begin{itemize}
	\item I'd like to place an overseas call.
	\end{itemize}
}
\item verb \\
If you \textbf{place} a bet, you bet money on something.
 \textit{
	\begin{itemize}
	\item For this race, though, he had already placed a bet on one of the horses.
	\end{itemize}
}
\item verb \\
If an agency or organization \textbf{places} someone, it finds them a job or somewhere to live.
 \textit{
	\begin{itemize}
	\item In 1861, they managed to place fourteen women in paid positions in the colonies.
	\item In cases where it proves difficult to place a child, the reception centre provides
long-term care.
	\end{itemize}
}
\item verb \\
If you say that you cannot \textbf{place} someone, you mean that you recognize them but cannot remember  exactly who they are or where you have met them before.
 \textit{
	\begin{itemize}
	\item Something about the man was familiar, although I could not immediately place him.
	\item It was a voice he recognized, though he could not immediately place it.
	\end{itemize}
}
\item  \\
 all over the place \textit{
	\begin{itemize}
	\end{itemize}
}
\item  \\
 all over the place \textit{
	\begin{itemize}
	\end{itemize}
}
\item  \\
 all over the place \textit{
	\begin{itemize}
	\end{itemize}
}
\item  \\
 to change places \textit{
	\begin{itemize}
	\end{itemize}
}
\item  \\
 to click into place \textit{
	\begin{itemize}
	\end{itemize}
}
\item  \\
 to fall into place \textit{
	\begin{itemize}
	\end{itemize}
}
\item  \\
 go places \textit{
	\begin{itemize}
	\end{itemize}
}
\item  \\
 in high places \textit{
	\begin{itemize}
	\end{itemize}
}
\item  \\
 in place/into place/out of place \textit{
	\begin{itemize}
	\end{itemize}
}
\item  \\
 in place \textit{
	\begin{itemize}
	\end{itemize}
}
\item  \\
 in place of sth/sb, in sth's/sb's place \textit{
	\begin{itemize}
	\end{itemize}
}
\item  \\
 in places \textit{
	\begin{itemize}
	\end{itemize}
}
\item  \\
 in sb's place \textit{
	\begin{itemize}
	\end{itemize}
}
\item  \\
 in the first place \textit{
	\begin{itemize}
	\end{itemize}
}
\item  \\
 in the first place \textit{
	\begin{itemize}
	\end{itemize}
}
\item  \\
 not sb's place to do sth \textit{
	\begin{itemize}
	\end{itemize}
}
\item  \\
 out of place \textit{
	\begin{itemize}
	\end{itemize}
}
\item  \\
 a place in the sun \textit{
	\begin{itemize}
	\end{itemize}
}
\item  \\
 place sth above/before/over sth \textit{
	\begin{itemize}
	\end{itemize}
}
\item  \\
 put sb in their place \textit{
	\begin{itemize}
	\end{itemize}
}
\item  \\
 show sb their place/keep sb in their place \textit{
	\begin{itemize}
	\end{itemize}
}
\item  \\
 take second place \textit{
	\begin{itemize}
	\end{itemize}
}
\item  \\
 take the place of/take sb's place \textit{
	\begin{itemize}
	\end{itemize}
}
\end{enumerate}

\section*{spend}
{\large \color{blue}  spends  spending  spent  }
\subsection*{Explain}
\begin{enumerate}
\item verb \\
When you \textbf{spend} money, you pay money for things that you want .
 \textit{
	\begin{itemize}
	\item By the end of the holiday I had spent all my money.
	\item Businessmen spend enormous amounts advertising their products.
	\item Juventus have spent £23m on new players.
	\item The survey may cost at least £100 but is money well spent.
	\end{itemize}
}
\item verb \\
If you \textbf{spend} time or energy doing something, you use your time or effort doing it.
 \textit{
	\begin{itemize}
	\item Engineers spend much time and energy developing brilliant solutions.
	\item This energy could be much better spent taking some positive action.
	\end{itemize}
}
\item verb \\
If you \textbf{spend} a period of time in a place, you stay there for a period of time.
 \textit{
	\begin{itemize}
	\item We spent the night in a hotel.
	\end{itemize}
}
\item countable noun \\
The \textbf{spend} on a particular thing is the amount of money that is spent on it, or will be spent.
 \textit{
	\begin{itemize}
	\item ...the marketing and advertising spend.
	\end{itemize}
}
\end{enumerate}

\section*{process}
{\large \color{blue}  processes  processing  processed  }
\subsection*{Explain}
\begin{enumerate}
\item countable noun \\
A \textbf{process} is a series of actions which are carried out in order to achieve a particular result.
 \textit{
	\begin{itemize}
	\item There was total agreement to start the peace process as soon as possible.
	\item They decided to spread the building process over three years.
	\item The best way to proceed is by a process of elimination.
	\end{itemize}
}
\item countable noun \\
A \textbf{process} is a series of things which happen  naturally and result in a biological or chemical change.
 \textit{
	\begin{itemize}
	\item It occurs in elderly men, apparently as part of the ageing process.
	\item The regularity with which this occurs suggests that the process is genetically determined.
	\end{itemize}
}
\item verb \\
When raw materials or foods \textbf{are processed} , they are prepared in factories before they are used or sold.
 \textbf{Process} is also a noun .
 \textit{
	\begin{itemize}
	\item ...fish which are processed by freezing, canning or smoking.
	\item The material will be processed into plastic pellets.
	\item ...diets high in refined and processed foods.
	\item ...the cost of re-engineering the production process.
	\end{itemize}
}
\item verb \\
When people \textbf{process} information, they put it through a system or into a computer in order to deal with it.
 \textit{
	\begin{itemize}
	\item ...facilities to process the data, and the right to publish the results.
	\item The information gathered by the telescopes will be processed by computers.
	\end{itemize}
}
\item verb \\
When people \textbf{are processed} by officials, their case is dealt with in stages and they pass from one stage of
the process to the next .
 \textit{
	\begin{itemize}
	\item Patients took more than two hours to be processed through the department.
	\end{itemize}
}
\item  \\
 in the process of \textit{
	\begin{itemize}
	\end{itemize}
}
\item  \\
 in the process \textit{
	\begin{itemize}
	\end{itemize}
}
\end{enumerate}

\section*{stretch}
{\large \color{blue}  stretches  stretching  stretched  }
\subsection*{Explain}
\begin{enumerate}
\item verb \\
Something that \textbf{stretches} over an area or distance covers or exists in the whole of that area or distance.
 \textit{
	\begin{itemize}
	\item The procession stretched for several miles.
	\item He had burns that stretched from his neck to his hips.
	\item ...an artificial reef stretching the length of the coast.
	\end{itemize}
}
\item countable noun \\
A \textbf{stretch}  \textbf{of} road, water, or land is a length or area of it.
 \textit{
	\begin{itemize}
	\item It's a very dangerous stretch of road.
	\item This stretch of Lost River was broader and deeper.
	\item ...a long stretch of beach with fine white sand.
	\end{itemize}
}
\item verb \\
When you \textbf{stretch} , you put your arms or legs out straight and tighten your muscles.
 \textbf{Stretch} is also a noun .
 \textit{
	\begin{itemize}
	\item He yawned and stretched.
	\item Try stretching your legs and pulling your toes upwards.
	\item She arched her back and stretched herself.
	\item At the end of a workout spend time cooling down with some slow stretches.
	\end{itemize}
}
\item countable noun \\
A \textbf{stretch}  \textbf{of} time is a period of time.
 \textit{
	\begin{itemize}
	\item He was fluent in French, having spent stretches of time in Southern France.
	\item ...after an 18-month stretch in the army.
	\item He would study for eight to ten hours at a stretch.
	\end{itemize}
}
\item ergative verb \\
If an event or activity \textbf{stretches} or \textbf{is stretched}  \textbf{into} a further period of time, it continues into that period, which is later than expected .
 \textit{
	\begin{itemize}
	\item ...as anti-abortion protests stretched into their second week.
	\item The talks could be stretched into the summer of next year.
	\end{itemize}
}
\item verb \\
If something \textbf{stretches}  \textbf{from} one time \textbf{to} another, it begins at the first time and ends at the second, which is longer than expected.
 \textit{
	\begin{itemize}
	\item ...a working day that stretches from seven in the morning to eight at night.
	\end{itemize}
}
\item verb \\
If a group of things \textbf{stretch}  \textbf{from} one type of thing \textbf{to} another, the group includes a wide range of things.
 \textit{
	\begin{itemize}
	\item ...a trading empire, with interests that stretched from chemicals to sugar.
	\end{itemize}
}
\item verb \\
When something soft or elastic  \textbf{stretches} or \textbf{is stretched} , it becomes longer or bigger as well as thinner , usually because it is pulled .
 \textit{
	\begin{itemize}
	\item The cables are designed not to stretch.
	\item Ease the pastry into the corners of the tin, making sure you don't stretch it.
	\end{itemize}
}
\item adjective \\
\textbf{Stretch} fabric is soft and elastic and stretches easily .
 \textit{
	\begin{itemize}
	\item ...stretch fabrics such as Lycra.
	\item ...stretch cotton swimsuits.
	\end{itemize}
}
\item verb \\
If you \textbf{stretch} an amount of something or if it \textbf{stretches} , you make it last longer than it usually would by being careful and not wasting any of it.
 \textit{
	\begin{itemize}
	\item They're used to stretching their budgets.
	\item During his senior year his earnings stretched far enough to buy an old car.
	\end{itemize}
}
\item verb \\
If your resources can \textbf{stretch to} something, you can just afford to do it.
 \textit{
	\begin{itemize}
	\item She suggested to me that I might like to start regular savings and I said 'Well,
I don't know whether I can stretch to that.'
	\end{itemize}
}
\item verb \\
If something \textbf{stretches} your money or resources, it uses them up so you have hardly enough for your needs.
 \textit{
	\begin{itemize}
	\item The drought there is stretching American resources.
	\item Public expenditure was being stretched to the limit.
	\end{itemize}
}
\item verb \\
If you say that a job or task  \textbf{stretches} you, you mean that you like it because it makes you work hard and use all your energy
and skills so that you do not become bored or achieve less than you should.
 \textit{
	\begin{itemize}
	\item I'm trying to move on and stretch myself with something different.
	\item They criticised the quality of teaching, claiming pupils were not stretched enough.
	\end{itemize}
}
\item  \\
 at full stretch \textit{
	\begin{itemize}
	\end{itemize}
}
\item  \\
 at full stretch \textit{
	\begin{itemize}
	\end{itemize}
}
\item  \\
 by any stretch of the imagination/by no stretch of the imagination \textit{
	\begin{itemize}
	\end{itemize}
}
\item  \\
 stretch one's legs \textit{
	\begin{itemize}
	\end{itemize}
}
\item  \\
 stretch a point \textit{
	\begin{itemize}
	\end{itemize}
}
\end{enumerate}

\section*{republic}
{\large \color{blue}  republics  }
\subsection*{Explain}
\begin{enumerate}
\item countable noun \\
A \textbf{republic} is a country where power is held by the people or the representatives that they elect.
Republics have presidents who are elected, rather than kings or queens .
 \textit{
	\begin{itemize}
	\item In 1918, Austria became a republic.
	\item ...the Baltic republics.
	\item ...the Republic of Ireland.
	\end{itemize}
}
\end{enumerate}

\section*{submerge}
{\large \color{blue}  submerges  submerging  submerged  }
\subsection*{Explain}
\begin{enumerate}
\item verb \\
If something \textbf{submerges} or if you \textbf{submerge} it, it goes below the surface of some water or another liquid.
 \textit{
	\begin{itemize}
	\item Hippos are unable to submerge in the few remaining water holes.
	\item The river burst its banks, submerging an entire village.
	\end{itemize}
}
\item verb \\
If you \textbf{submerge}  \textbf{yourself in} an activity, you give all your attention to it and do not think about anything else.
 \textit{
	\begin{itemize}
	\item He submerges himself in the world of his imagination.
	\end{itemize}
}
\end{enumerate}

\section*{robot}
{\large \color{blue}  robots  }
\subsection*{Explain}
\begin{enumerate}
\item countable noun \\
A \textbf{robot} is a machine which is programmed to move and perform certain tasks automatically.
 \textit{
	\begin{itemize}
	\item ...very lightweight robots that we could send to the moon for planetary exploration.
	\end{itemize}
}
\end{enumerate}

\section*{testify}
{\large \color{blue}  testifies  testifying  testified  }
\subsection*{Explain}
\begin{enumerate}
\item verb \\
When someone \textbf{testifies} in a court of law, they give a statement of what they saw someone do or what they know of a situation , after having promised to tell the truth .
 \textit{
	\begin{itemize}
	\item Several eyewitnesses testified that they saw him run from the scene.
	\item Eva testified to having seen Herndon with his gun on the stairs.
	\item He hopes to have his 12-year prison term reduced by testifying against his former
colleagues.
	\end{itemize}
}
\item verb \\
If one thing \textbf{testifies}  \textbf{to} another, it supports the belief that the second thing is true .
 \textit{
	\begin{itemize}
	\item Recent excavations testify to the presence of cultivated inhabitants on the hill
during the Arthurian period.
	\end{itemize}
}
\end{enumerate}

\section*{status}
{\large \color{blue}  }
\subsection*{Explain}
\begin{enumerate}
\item uncountable noun \\
Your \textbf{status} is your social or professional position.
 \textit{
	\begin{itemize}
	\item The fact that the burial involved an expensive coffin signifies that the person was
of high status.
	\item ...women and men of wealth and status.
	\item ... her former status as a vice-president of the Spanish Athletics Federation.
	\end{itemize}
}
\item uncountable noun \\
\textbf{Status} is the importance and respect that someone has among the public or a particular group.
 \textit{
	\begin{itemize}
	\item Nurses are undervalued, and they never enjoy the same status as doctors.
	\item He has risen to gain the status of a national hero.
	\end{itemize}
}
\item uncountable noun \\
The \textbf{status} of something is the importance that people give it.
 \textit{
	\begin{itemize}
	\item Those things that can be assessed by external tests are being given unduly high status.
	\item The fact that the most senior judge of the High Court's Family Division had taken
control of the case was proof of its urgency and status.
	\end{itemize}
}
\item uncountable noun \\
A particular \textbf{status} is an official  description that says what category a person, organization , or place belongs to, and gives them particular rights or advantages .
 \textit{
	\begin{itemize}
	\item Bristol regained its status as a city in the local government reorganisation.
	\item ...his status as a British citizen.
	\item The WHO recommendation has no legal status.
	\item The personal allowance depends on your age and marital status.
	\end{itemize}
}
\item uncountable noun \\
The \textbf{status} of something is its state of affairs at a particular time.
 \textit{
	\begin{itemize}
	\item She prepared a status report on the project.
	\item What is your current financial status?
	\item Please keep us informed of the status of this project.
	\end{itemize}
}
\end{enumerate}

\section*{update}
{\large \color{blue}  updates  updating  updated  }
\subsection*{Explain}
\begin{enumerate}
\item verb \\
If you \textbf{update} something, you make it more modern , usually by adding new parts to it or giving new information .
 \textit{
	\begin{itemize}
	\item He was back in the office, updating the work schedule on the computer.
	\item Airlines would prefer to update rather than retrain crews.
	\item ...an updated edition of the book.
	\end{itemize}
}
\item countable noun \\
An \textbf{update} is a news  item containing the latest information about a particular  situation .
 \textit{
	\begin{itemize}
	\item She had heard the news-flash on a TV channel's news update.
	\item ...a weather update.
	\item ...football results update.
	\end{itemize}
}
\item verb \\
If you \textbf{update} someone \textbf{on} a situation, you tell them the latest developments in that situation.
 \textit{
	\begin{itemize}
	\item We'll update you on the day's top news stories.
	\item I would just update them on any news we might have.
	\end{itemize}
}
\end{enumerate}

\section*{subway}
{\large \color{blue}  subways  }
\subsection*{Explain}
\begin{enumerate}
\item countable noun \\
A \textbf{subway} is an underground railway.
 \textit{
	\begin{itemize}
	\item ...the Bay Area Rapid Transit subway system.
	\item I don't ride the subway late at night.
	\end{itemize}
}
\item countable noun \\
A \textbf{subway} is a passage underneath a busy road or a railway track for people to walk through.
 \textit{
	\begin{itemize}
	\item The majority of us feel worried if we walk through a subway.
	\end{itemize}
}
\end{enumerate}

\section*{walk}
{\large \color{blue}  walks  walking  walked  }
\subsection*{Explain}
\begin{enumerate}
\item verb \\
When you \textbf{walk} , you move forward by putting one foot in front of the other in a regular way.
 \textit{
	\begin{itemize}
	\item Rosanna and Forbes walked in silence for some while.
	\item We walked into the foyer.
	\item She turned and walked away.
	\item They would stop the car and walk a few steps.
	\item When I was your age I walked five miles to school.
	\end{itemize}
}
\item countable noun \\
A \textbf{walk} is a journey that you make by walking, usually for pleasure .
 \textit{
	\begin{itemize}
	\item I went for a walk.
	\item He often took long walks in the hills.
	\end{itemize}
}
\item singular noun \\
A \textbf{walk} of a particular distance is the distance which a person has to walk to get  somewhere .
 \textit{
	\begin{itemize}
	\item It was only a three-mile walk to Kabul from there.
	\item The church is a short walk from Piazza Dante.
	\end{itemize}
}
\item countable noun \\
A \textbf{walk} is a route suitable for walking along for pleasure.
 \textit{
	\begin{itemize}
	\item There is a 2 mile coastal walk from Craster to Newton.
	\end{itemize}
}
\item singular noun \\
\textbf{A walk} is the action of walking rather than running .
 \textit{
	\begin{itemize}
	\item She slowed to a steady walk.
	\end{itemize}
}
\item singular noun \\
Someone's \textbf{walk} is the way that they walk.
 \textit{
	\begin{itemize}
	\item George, despite his great height and gangling walk, was a keen dancer.
	\end{itemize}
}
\item verb \\
If you \textbf{walk} someone somewhere, you walk there with them in order to show politeness or to make
 sure that they get there safely.
 \textit{
	\begin{itemize}
	\item She walked me to my car.
	\end{itemize}
}
\item verb \\
If you \textbf{walk} your dog, you take it for a walk in order to keep it healthy .
 \textit{
	\begin{itemize}
	\item I walk my dog each evening around my local streets.
	\end{itemize}
}
\end{enumerate}

\section*{surplus}
{\large \color{blue}  surpluses  }
\subsection*{Explain}
\begin{enumerate}
\item variable noun \\
If there is a \textbf{surplus}  \textbf{of} something, there is more than is needed .
 \textit{
	\begin{itemize}
	\item Germany suffers from a surplus of teachers.
	\end{itemize}
}
\item adjective \\
\textbf{Surplus} is used to describe something that is extra or that is more than is needed.
 \textit{
	\begin{itemize}
	\item Few people have large sums of surplus cash.
	\item I sell my surplus birds to a local pet shop.
	\item The houses are being sold because they are surplus to requirements.
	\end{itemize}
}
\item countable noun \\
If a country has a trade  \textbf{surplus} , it exports more than it imports .
 \textit{
	\begin{itemize}
	\item Japan's annual trade surplus is in the region of 100 billion dollars.
	\end{itemize}
}
\item countable noun \\
If a government has a budget  \textbf{surplus} , it has spent less than it received in taxes .
 \textit{
	\begin{itemize}
	\item The Government also runs a modest budget surplus.
	\end{itemize}
}
\end{enumerate}

\section*{wish}
{\large \color{blue}  wishes  wishing  wished  }
\subsection*{Explain}
\begin{enumerate}
\item countable noun \\
A \textbf{wish} is a desire or strong  feeling that you want to have something or do something.
 \textit{
	\begin{itemize}
	\item She was sincere and genuine in her wish to make amends for the past.
	\item Clearly she had no wish for conversation.
	\item She wanted to go everywhere in the world. She soon got her wish.
	\item The decision was made against the wishes of the party leader.
	\end{itemize}
}
\item verb \\
If you \textbf{wish}  \textbf{to} do something or \textbf{to} have it done for you, you want to do it or have it done.
 \textit{
	\begin{itemize}
	\item Older children may not wish to spend all their time in adult company.
	\item We can dress as we wish now.
	\item There were the collaborators, who wished for a German victory.
	\end{itemize}
}
\item verb \\
\textbf{Wish} is used in expressions such as \textbf{I don't wish}  \textbf{to} be rude or \textbf{without wishing}  \textbf{to} be rude as a way of apologizing or warning someone when you are going to say something which might  upset , annoy , or worry them.
 \textit{
	\begin{itemize}
	\item I don't wish to sound callous, but I am glad I wasn't here.
	\item Without wishing to be unkind, she's not the most interesting company.
	\end{itemize}
}
\item verb \\
If you \textbf{wish} something were true , you would like it to be true, even though you know that it is impossible or unlikely .
 \textit{
	\begin{itemize}
	\item I wish that I could do that.
	\item I wish it weren't true.
	\item Pa, I wish you wouldn't shout.
	\item The world is not always what we wish it to be.
	\end{itemize}
}
\item verb \\
If you \textbf{wish for} something, you express the desire for that thing silently to yourself. In fairy  stories , when a person wishes for something, the thing they wish for often happens by magic .
 \textbf{Wish} is also a noun .
 \textit{
	\begin{itemize}
	\item We have all wished for men who are more considerate.
	\item A philosopher once said, 'Be careful what you wish for; you might get it.'
	\item Blow out the candles and make a wish.
	\end{itemize}
}
\item verb \\
\textbf{Wish} is used in sentences such as \textbf{I could not wish for anything better} to indicate that you are very pleased with what you have and could not imagine anything better.
 \textit{
	\begin{itemize}
	\item I really could not have wished for a better teacher.
	\item Who could wish for a better opportunity?
	\end{itemize}
}
\item verb \\
If you say that you would not \textbf{wish} a particular thing \textbf{on} someone, you mean that the thing is so unpleasant that you would not want them to be forced to experience it.
 \textit{
	\begin{itemize}
	\item It's a horrid experience and I wouldn't wish it on my worst enemy.
	\end{itemize}
}
\item verb \\
If you \textbf{wish} someone something such as luck or happiness, you express the hope that they will be lucky or happy .
 \textit{
	\begin{itemize}
	\item I wish you both a very good journey.
	\item Goodbye, Hanu. I wish you well.
	\end{itemize}
}
\item plural noun \\
If you express your good \textbf{wishes} towards someone, you are politely expressing your friendly feelings towards them and your hope that they will be successful or happy.
 \textit{
	\begin{itemize}
	\item I found George's story very sad. Please give him my best wishes.
	\item Western leaders sent good wishes to the new American president.
	\end{itemize}
}
\end{enumerate}

\section*{adult}
{\large \color{blue}  adults  }
\subsection*{Explain}
\begin{enumerate}
\item countable noun \\
An \textbf{adult} is a mature, fully developed person. An adult has reached the age when they are legally
 responsible for their actions .
 \textit{
	\begin{itemize}
	\item Becoming a father signified that he was now an adult.
	\item Children under 14 must be accompanied by an adult.
	\end{itemize}
}
\item countable noun \\
An \textbf{adult} is a fully developed animal.
 \textit{
	\begin{itemize}
	\item ...a pair of adult birds.
	\end{itemize}
}
\item adjective \\
\textbf{Adult}  means relating to the time when you are an adult, or typical of adult people.
 \textit{
	\begin{itemize}
	\item I've lived most of my adult life in London.
	\end{itemize}
}
\item adjective \\
If you say that someone is \textbf{adult} about something, you think that they act in a mature, intelligent  way , especially when faced with a difficult  situation .
 \textit{
	\begin{itemize}
	\item We were very adult about it. We discussed it rationally over a drink.
	\item ...dealing with emerging sexuality in an adult way.
	\end{itemize}
}
\item adjective \\
You can describe things such as films or books as \textbf{adult} when they deal with sex in a very clear and open way.
 \textit{
	\begin{itemize}
	\item ...an adult movie.
	\end{itemize}
}
\end{enumerate}

\section*{act}
{\large \color{blue}  acts  acting  acted  }
\subsection*{Explain}
\begin{enumerate}
\item verb \\
When you \textbf{act} , you do something for a particular purpose.
 \textit{
	\begin{itemize}
	\item The deaths occurred when police acted to stop widespread looting and vandalism.
	\item I do not doubt that the bank acted properly.
	\end{itemize}
}
\item verb \\
If you \textbf{act on}  advice or information, you do what has been advised or suggested .
 \textit{
	\begin{itemize}
	\item A patient will usually listen to the doctor's advice and act on it.
	\end{itemize}
}
\item verb \\
If someone \textbf{acts} in a particular way, they behave in that way.
 \textit{
	\begin{itemize}
	\item ...a gang of youths who were acting suspiciously.
	\item He acted as if he hadn't heard any of it.
	\item Open wounds act like a magnet to flies.
	\end{itemize}
}
\item verb \\
If someone or something \textbf{acts as} a particular thing, they have that role or function.
 \textit{
	\begin{itemize}
	\item He acted as the ship's surgeon.
	\item A layer of warmer air acted like a lid that trapped any air pollution on the ground.
	\end{itemize}
}
\item verb \\
If someone \textbf{acts} in a particular way, they pretend to be something that they are not.
 \textit{
	\begin{itemize}
	\item Chris acted astonished as he examined the note.
	\item Kenworthy had tried not to act the policeman.
	\end{itemize}
}
\item verb \\
When professionals such as lawyers  \textbf{act for} you, or \textbf{act on} your \textbf{behalf} , they are employed by you to deal with a particular matter.
 \textit{
	\begin{itemize}
	\item Lawyers acting for the families of the victims ...
	\item Because we travelled so much, Sam and I asked a broker to act on our behalf.
	\end{itemize}
}
\item verb \\
If a force or substance \textbf{acts on} someone or something, it has a certain effect on them.
 \textit{
	\begin{itemize}
	\item The drug acts very fast on the central nervous system.
	\item A hypnotist can act upon the unconscious mind directly.
	\end{itemize}
}
\item verb \\
If you \textbf{act} , or \textbf{act} a part in a play or film, you have a part in it.
 \textit{
	\begin{itemize}
	\item She confessed to her parents her desire to act.
	\item She acted in her first film when she was 13 years old.
	\end{itemize}
}
\item countable noun \\
An \textbf{act} is a single thing that someone does.
 \textit{
	\begin{itemize}
	\item Language interpretation is the whole point of the act of reading.
	\item My insurance excludes acts of sabotage and damage done by weapons of war.
	\end{itemize}
}
\item singular noun \\
If you say that someone's behaviour is an \textbf{act} , you mean that it does not express their real  feelings .
 \textit{
	\begin{itemize}
	\item There were moments when I wondered: did she do this on purpose, was it all just a
game, an act?
	\item His anger was real. It wasn't an act.
	\end{itemize}
}
\item countable noun \\
An \textbf{Act} is a law passed by the government.
 \textit{
	\begin{itemize}
	\item ...an Act of Parliament.
	\end{itemize}
}
\item countable noun \\
An \textbf{act} in a play, opera , or ballet is one of the main parts into which it is divided.
 \textit{
	\begin{itemize}
	\item Act II contained one of the funniest scenes I have ever witnessed.
	\end{itemize}
}
\item countable noun \\
An \textbf{act} in a show is a short performance which is one of several in the show.
 \textit{
	\begin{itemize}
	\item This year numerous bands are playing, as well as comedy acts.
	\end{itemize}
}
\item  \\
 catch sb in the act \textit{
	\begin{itemize}
	\end{itemize}
}
\item  \\
 to clean up your act \textit{
	\begin{itemize}
	\end{itemize}
}
\item  \\
 get in on the act \textit{
	\begin{itemize}
	\end{itemize}
}
\item  \\
 in the act of \textit{
	\begin{itemize}
	\end{itemize}
}
\item  \\
 to get your act together \textit{
	\begin{itemize}
	\end{itemize}
}
\end{enumerate}

\section*{ash}
{\large \color{blue}  ashes  }
\subsection*{Explain}
\begin{enumerate}
\item uncountable noun \\
\textbf{Ash} is the grey or black powdery substance that is left after something is burnt. You
can also refer to this substance as \textbf{ashes} .
 \textit{
	\begin{itemize}
	\item A cloud of volcanic ash is spreading across wide areas of the Philippines.
	\item He brushed the cigarette ash from his sleeve.
	\item He ordered their villages burned to ashes.
	\end{itemize}
}
\item plural noun \\
A dead person's \textbf{ashes} are their remains after their body has been cremated .
 \textit{
	\begin{itemize}
	\end{itemize}
}
\item variable noun \\
An \textbf{ash} is a tree that has smooth grey bark and loses its leaves in winter .
 \textbf{Ash} is the wood from this tree.
 \textit{
	\begin{itemize}
	\item ...a high forest of oak and ash.
	\item The rafters are made from ash.
	\end{itemize}
}
\end{enumerate}

\section*{ban}
{\large \color{blue}  bans  banning  banned  }
\subsection*{Explain}
\begin{enumerate}
\item verb \\
To \textbf{ban} something means to state officially that it must not be done, shown , or used.
 \textit{
	\begin{itemize}
	\item It was decided to ban smoking in all offices later this year.
	\item Last year arms sales were banned.
	\item ...a banned substance.
	\end{itemize}
}
\item countable noun \\
A \textbf{ban} is an official ruling that something must not be done, shown, or used.
 \textit{
	\begin{itemize}
	\item The General also lifted a ban on political parties.
	\end{itemize}
}
\item verb \\
If you \textbf{are banned}  \textbf{from} doing something, you are officially prevented from doing it.
 \textit{
	\begin{itemize}
	\item He was banned from driving for three years.
	\end{itemize}
}
\end{enumerate}

\section*{breakfast}
{\large \color{blue}  breakfasts  breakfasting  breakfasted  }
\subsection*{Explain}
\begin{enumerate}
\item variable noun \\
\textbf{Breakfast} is the first meal of the day. It is usually eaten in the early part of the morning .
 \textit{
	\begin{itemize}
	\item What's for breakfast?
	\item ...breakfast cereal.
	\end{itemize}
}
\item countable noun \\
A cooked  \textbf{breakfast} or a hot  \textbf{breakfast} is a breakfast that consists of cooked food, such as bacon and eggs .
 \textit{
	\begin{itemize}
	\end{itemize}
}
\item verb \\
When you \textbf{breakfast} , you have breakfast.
 \textit{
	\begin{itemize}
	\item We breakfasted on the balcony.
	\end{itemize}
}
\end{enumerate}

\section*{blame}
{\large \color{blue}  blames  blaming  blamed  }
\subsection*{Explain}
\begin{enumerate}
\item verb \\
If you \textbf{blame} a person or thing \textbf{for} something bad , you believe or say that they are responsible for it or that they caused it.
 \textbf{Blame} is also a noun .
 \textit{
	\begin{itemize}
	\item The commission is expected to blame the army for many of the atrocities.
	\item The bank blamed the error on technological failings.
	\item If it wasn't Sam's fault, why was I blaming him?
	\item Nothing could relieve my terrible sense of blame.
	\end{itemize}
}
\item uncountable noun \\
The \textbf{blame}  \textbf{for} something bad that has happened is the responsibility for causing it or letting it happen.
 \textit{
	\begin{itemize}
	\item Some of the blame for the miscarriage of justice must be borne by the solicitors.
	\item The president put the blame squarely on his opponent.
	\end{itemize}
}
\item verb \\
If you say that you do not \textbf{blame} someone \textbf{for} doing something, you mean that you consider it was a reasonable thing to do in the circumstances .
 \textit{
	\begin{itemize}
	\item I do not blame them for trying to make some money.
	\item He slammed the door and stormed off. I could hardly blame him.
	\end{itemize}
}
\item  \\
 to blame \textit{
	\begin{itemize}
	\end{itemize}
}
\item  \\
 have only oneself to blame/have no-one but oneself to blame \textit{
	\begin{itemize}
	\end{itemize}
}
\end{enumerate}

\section*{cost}
{\large \color{blue}  costs  costing  }
\subsection*{Explain}
\begin{enumerate}
\item countable noun \\
The \textbf{cost}  \textbf{of} something is the amount of money that is needed in order to buy , do, or make it.
 \textit{
	\begin{itemize}
	\item The cost of a loaf of bread has increased five-fold.
	\item In 1989 the price of coffee fell so low that in many countries it did not even cover
the cost of production.
	\item Badges are also available at a cost of £2.50.
	\end{itemize}
}
\item verb \\
If something \textbf{costs} a particular amount of money, you can buy, do, or make it for that amount.
 \textit{
	\begin{itemize}
	\item This course is limited to 12 people and costs £50.
	\item Painted walls look much more interesting and don't cost much.
	\item It's going to cost me over $100,000 to buy new trucks.
	\end{itemize}
}
\item plural noun \\
Your \textbf{costs} are the total amount of money that you must  spend on running your home or business.
 \textit{
	\begin{itemize}
	\item Costs have been cut by 30 to 50 per cent.
	\item The company admits its costs are still too high.
	\end{itemize}
}
\item verb \\
When something that you plan to do or make \textbf{is costed} , the amount of money you need is calculated in advance .
 \textbf{Cost out} means the same as cost .
 \textit{
	\begin{itemize}
	\item Everything that goes into making a programme, staff, rent, lighting, is now costed.
	\item ...apartments, a restaurant and a hotel, costed at around 10 million pounds.
	\item ...training days for charity staff on how to draw up contracts and cost out proposals.
	\item It is always worth having a loft conversion costed out.
	\end{itemize}
}
\item plural noun \\
If someone is ordered by a court of law to pay \textbf{costs} , they have to pay a sum of money towards the expenses of a court case they are involved in.
 \textit{
	\begin{itemize}
	\item He was jailed for 18 months and ordered to pay £550 costs.
	\end{itemize}
}
\item uncountable noun \\
If something is sold \textbf{at}  \textbf{cost} , it is sold for the same price as it cost the seller to buy it.
 \textit{
	\begin{itemize}
	\item ...a store that provided soft drinks and candy bars at cost.
	\item They started selling below cost to drive competition out of business.
	\end{itemize}
}
\item singular noun \\
The \textbf{cost}  \textbf{of} something is the loss, damage , or injury that is involved in trying to achieve it.
 \textit{
	\begin{itemize}
	\item In March Mr Salinas shut down the city's oil refinery at a cost of $500 million and
5,000 jobs.
	\item He had to protect his family, whatever the cost to himself.
	\end{itemize}
}
\item verb \\
If an event or mistake  \textbf{costs} you something, you lose that thing as the result of it.
 \textit{
	\begin{itemize}
	\item ...a six-year-old boy whose life was saved by an operation that cost him his sight.
	\item The increase will hurt small business and cost many thousands of jobs.
	\end{itemize}
}
\item  \\
 at all costs \textit{
	\begin{itemize}
	\end{itemize}
}
\item  \\
 at any cost \textit{
	\begin{itemize}
	\end{itemize}
}
\item  \\
 count the cost \textit{
	\begin{itemize}
	\end{itemize}
}
\item  \\
 cost money \textit{
	\begin{itemize}
	\end{itemize}
}
\item  \\
 to one's cost \textit{
	\begin{itemize}
	\end{itemize}
}
\end{enumerate}

\section*{breathe}
{\large \color{blue}  breathes  breathing  breathed  }
\subsection*{Explain}
\begin{enumerate}
\item verb \\
When people or animals \textbf{breathe} , they take air into their lungs and let it out again. When they \textbf{breathe}  smoke or a particular kind of air, they take it into their lungs and let it out again as they breathe.
 \textit{
	\begin{itemize}
	\item He stood there breathing deeply and evenly.
	\item Always breathe through your nose.
	\item No American should have to drive out of town to breathe clean air.
	\item A thirteen year old girl is being treated after breathing in smoke.
	\end{itemize}
}
\item verb \\
If someone \textbf{breathes} something, they say it very quietly .
 \textit{
	\begin{itemize}
	\item 'You don't understand,' he breathed.
	\end{itemize}
}
\item verb \\
If you do not \textbf{breathe} a word about something, you say nothing about it, because it is a secret .
 \textit{
	\begin{itemize}
	\item He never breathed a word about our conversation.
	\end{itemize}
}
\item verb \\
If someone \textbf{breathes} life, confidence , or excitement  \textbf{into} something, they improve it by adding this quality.
 \textit{
	\begin{itemize}
	\item It is the readers who breathe life into a newspaper with their letters.
	\end{itemize}
}
\item verb \\
If you let wine  \textbf{breathe} , you open the bottle to allow the air to get in and improve its flavour before you drink it.
 \textit{
	\begin{itemize}
	\item Red wines should be allowed to 'breathe' if possible before drinking.
	\end{itemize}
}
\item  \\
 breathe one's last \textit{
	\begin{itemize}
	\end{itemize}
}
\end{enumerate}

\section*{curve}
{\large \color{blue}  curves  curving  curved  }
\subsection*{Explain}
\begin{enumerate}
\item countable noun \\
A \textbf{curve} is a smooth , gradually bending line, for example part of the edge of a circle .
 \textit{
	\begin{itemize}
	\item ...the curve of his lips.
	\item ...a curve in the road.
	\end{itemize}
}
\item verb \\
If something \textbf{curves} , or if someone or something \textbf{curves} it, it has the shape of a curve.
 \textit{
	\begin{itemize}
	\item Her spine curved.
	\item The track curved away below him.
	\item ...a knife with a slightly curving blade.
	\item A small, unobtrusive smile curved the cook's thin lips.
	\end{itemize}
}
\item verb \\
If something \textbf{curves} , it moves in a curve, for example through the air .
 \textit{
	\begin{itemize}
	\item The ball curved strangely in the air.
	\end{itemize}
}
\item countable noun \\
You can refer to a change in something as a particular \textbf{curve} , especially when it is represented on a graph.
 \textit{
	\begin{itemize}
	\item Each firm will face a downward-sloping demand curve.
	\item Was it just a temporary blip on an otherwise healthy growth curve?
	\end{itemize}
}
\item  \\
 throw someone a curve/throw someone a curve ball \textit{
	\begin{itemize}
	\end{itemize}
}
\item  \\
 ahead of the curve \textit{
	\begin{itemize}
	\end{itemize}
}
\end{enumerate}

\section*{cancel}
{\large \color{blue}  cancels  cancelling  cancelled  }
\subsection*{Explain}
\begin{enumerate}
\item verb \\
If you \textbf{cancel} something that has been arranged, you stop it from happening . If you \textbf{cancel} an order for goods or services, you tell the person or organization supplying them that you no longer wish to receive them.
 \textit{
	\begin{itemize}
	\item She cancelled her visit to Japan.
	\item Many trains have been cancelled and a limited service is operating on other lines.
	\item I called to arrange a delivery date and was told the order had been cancelled.
	\item There is normally no refund should a client choose to cancel.
	\end{itemize}
}
\item verb \\
If someone in authority \textbf{cancels} a document , an insurance  policy , or a debt, they officially  declare that it is no longer valid or no longer legally exists .
 \textit{
	\begin{itemize}
	\item He intends to try to leave the country, in spite of a government order cancelling
his passport.
	\item She learned her insurance had been canceled by Pacific Mutual Insurance Company.
	\item Under the agreement, Britain will cancel hundreds of millions of pounds in debts
owed to it by some of world's poorest countries.
	\end{itemize}
}
\item verb \\
To \textbf{cancel} a stamp or a cheque means to mark it to show that it has already been used and cannot be used again.
 \textit{
	\begin{itemize}
	\item The new device can also cancel the check after the transaction is complete.
	\item ...cancelled stamps.
	\end{itemize}
}
\end{enumerate}

\section*{death}
{\large \color{blue}  deaths  }
\subsection*{Explain}
\begin{enumerate}
\item variable noun \\
\textbf{Death} is the permanent end of the life of a person or animal.
 \textit{
	\begin{itemize}
	\item 1.5 million people are in immediate danger of death from starvation.
	\item ...the thirtieth anniversary of her death.
	\item The report mentions the death of 18 people in suspicious circumstances.
	\item There had been a death in the family.
	\end{itemize}
}
\item countable noun \\
A particular kind of \textbf{death} is a particular way of dying.
 \textit{
	\begin{itemize}
	\item They made sure that he died a horrible death.
	\item He would rather have a decent death which served some purpose than a meaningless
death.
	\end{itemize}
}
\item singular noun \\
\textbf{The}  \textbf{death}  \textbf{of} something is the permanent end of it.
 \textit{
	\begin{itemize}
	\item It meant the death of everything he had ever been or ever hoped to be.
	\item ...the death of pop music.
	\end{itemize}
}
\item  \\
 at death's door \textit{
	\begin{itemize}
	\end{itemize}
}
\item  \\
 fight to the death \textit{
	\begin{itemize}
	\end{itemize}
}
\item  \\
 a fight to the death \textit{
	\begin{itemize}
	\end{itemize}
}
\item  \\
 life and/or death \textit{
	\begin{itemize}
	\end{itemize}
}
\item  \\
 put sb to death \textit{
	\begin{itemize}
	\end{itemize}
}
\item  \\
 to death \textit{
	\begin{itemize}
	\end{itemize}
}
\item  \\
 to death \textit{
	\begin{itemize}
	\end{itemize}
}
\item  \\
 work sb/os to death \textit{
	\begin{itemize}
	\end{itemize}
}
\end{enumerate}

\section*{chew}
{\large \color{blue}  chews  chewing  chewed  }
\subsection*{Explain}
\begin{enumerate}
\item verb \\
When you \textbf{chew} food, you use your teeth to break it up in your mouth so that it becomes  easier to swallow .
 \textit{
	\begin{itemize}
	\item Be certain to eat slowly and chew your food extremely well.
	\item Daniel leaned back on the sofa, still chewing on his apple.
	\item ...the sound of his mother chewing and swallowing.
	\end{itemize}
}
\item verb \\
If you \textbf{chew}  gum or tobacco, you keep biting it and moving it around your mouth to taste the flavour of it. You do not swallow it.
 \textit{
	\begin{itemize}
	\item One girl was chewing gum.
	\item He chews tobacco constantly.
	\end{itemize}
}
\item verb \\
If you \textbf{chew} your lips or your fingernails , you keep biting them because you are nervous .
 \textit{
	\begin{itemize}
	\item He chewed his lower lip nervously.
	\end{itemize}
}
\item verb \\
If a person or animal \textbf{chews} an object , they bite it with their teeth.
 \textit{
	\begin{itemize}
	\item They pause and chew their pencils.
	\item One owner left his pet under the stairs where the animal chewed through electric
cables.
	\end{itemize}
}
\item countable noun \\
A \textbf{chew} is a sweet that you have to chew very hard before it becomes soft .
 \textit{
	\begin{itemize}
	\item ...a selection of penny chews.
	\end{itemize}
}
\item  \\
 to bite off more than one can chew \textit{
	\begin{itemize}
	\end{itemize}
}
\item  \\
 chew the fat \textit{
	\begin{itemize}
	\end{itemize}
}
\end{enumerate}

\section*{diploma}
{\large \color{blue}  diplomas  }
\subsection*{Explain}
\begin{enumerate}
\item countable noun \\
A \textbf{diploma} is a qualification which may be awarded to a student by a university or college , or by a high school in the United  States .
 \textit{
	\begin{itemize}
	\item ...a new two-year course leading to a diploma in social work.
	\end{itemize}
}
\end{enumerate}

\section*{clap}
{\large \color{blue}  claps  clapping  clapped  }
\subsection*{Explain}
\begin{enumerate}
\item verb \\
When you \textbf{clap} , you hit your hands together to show  appreciation or attract  attention .
 \textbf{Clap} is also a noun .
 \textit{
	\begin{itemize}
	\item The men danced and the women clapped.
	\item Midge clapped her hands, calling them back to order.
	\item Londoners came out on to the pavement to wave and clap the marchers.
	\item Let's give the children a big clap.
	\end{itemize}
}
\item verb \\
If you \textbf{clap} your hand or an object onto something, you put it there quickly and firmly.
 \textit{
	\begin{itemize}
	\item I clapped a hand over her mouth.
	\end{itemize}
}
\item verb \\
If you \textbf{clap} someone \textbf{on} the back or \textbf{on} the shoulder , you hit their back or shoulder with your hand in a friendly  way .
 \textit{
	\begin{itemize}
	\item People have been clapping us on the back and saying, 'Well done'.
	\end{itemize}
}
\item countable noun \\
A \textbf{clap of thunder} is a sudden and loud  noise of thunder.
 \textit{
	\begin{itemize}
	\end{itemize}
}
\end{enumerate}

\section*{disgust}
{\large \color{blue}  disgusts  disgusting  disgusted  }
\subsection*{Explain}
\begin{enumerate}
\item uncountable noun \\
\textbf{Disgust} is a feeling of very strong  dislike or disapproval .
 \textit{
	\begin{itemize}
	\item He spoke of his disgust at the incident.
	\item A look of disgust came over his face.
	\item I threw the book aside in disgust.
	\end{itemize}
}
\item verb \\
To \textbf{disgust} someone means to make them feel a strong sense of dislike and disapproval.
 \textit{
	\begin{itemize}
	\item He disgusted many with his boorish behaviour.
	\end{itemize}
}
\end{enumerate}

\section*{derive}
{\large \color{blue}  derives  deriving  derived  }
\subsection*{Explain}
\begin{enumerate}
\item verb \\
If you \textbf{derive} something such as pleasure or benefit  \textbf{from} a person or from something, you get it from them.
 \textit{
	\begin{itemize}
	\item Mr Ying is one of those happy people who derive pleasure from helping others.
	\end{itemize}
}
\item verb \\
If you say that something such as a word or feeling  \textbf{derives} or \textbf{is derived from} something else, you mean that it comes from that thing.
 \textit{
	\begin{itemize}
	\item Anna's strength is derived from her parents and her sisters.
	\item The word Easter derives from Eostre, the pagan goddess of spring.
	\end{itemize}
}
\end{enumerate}

\section*{dust}
{\large \color{blue}  dusts  dusting  dusted  }
\subsection*{Explain}
\begin{enumerate}
\item uncountable noun \\
\textbf{Dust} is very small dry particles of earth or sand .
 \textit{
	\begin{itemize}
	\item Tanks raise huge trails of dust when they move.
	\item He reversed into the stockade in a cloud of dust.
	\end{itemize}
}
\item uncountable noun \\
\textbf{Dust} is the very small pieces of dirt which you find  inside buildings, for example on furniture , floors , or lights.
 \textit{
	\begin{itemize}
	\item I could see a thick layer of dust on the stairs.
	\item The rooms were empty of furniture and dust lay everywhere.
	\end{itemize}
}
\item uncountable noun \\
\textbf{Dust} is a fine powder which consists of very small particles of a substance such as gold , wood, or coal .
 \textit{
	\begin{itemize}
	\item The air is so black with diesel fumes and coal dust, I can barely see.
	\end{itemize}
}
\item verb \\
When you \textbf{dust} something such as furniture, you remove dust from it, usually using a cloth .
 \textit{
	\begin{itemize}
	\item I vacuumed and dusted the living room.
	\item She dusted, she cleaned, and she did the washing-up.
	\end{itemize}
}
\item verb \\
If you \textbf{dust} something \textbf{with} a fine substance such as powder or if you \textbf{dust} a fine substance \textbf{onto} something, you cover it lightly with that substance.
 \textit{
	\begin{itemize}
	\item Lightly dust the fish with flour.
	\item Dust and blend blusher on the apples of your cheeks.
	\item Dry your feet well and then dust between the toes with baby powder.
	\end{itemize}
}
\item  \\
 to bite the dust \textit{
	\begin{itemize}
	\end{itemize}
}
\item  \\
 the dust settles \textit{
	\begin{itemize}
	\end{itemize}
}
\item  \\
 to gather dust \textit{
	\begin{itemize}
	\end{itemize}
}
\end{enumerate}

\section*{dissipate}
{\large \color{blue}  dissipates  dissipating  dissipated  }
\subsection*{Explain}
\begin{enumerate}
\item verb \\
When something \textbf{dissipates} or when you \textbf{dissipate} it, it becomes less or becomes less strong until it disappears or goes  away completely.
 \textit{
	\begin{itemize}
	\item The tension in the room had dissipated.
	\item He wound down the windows to dissipate the heat.
	\end{itemize}
}
\item verb \\
When someone \textbf{dissipates} money, time, or effort , they waste it in a foolish way.
 \textit{
	\begin{itemize}
	\item He is dissipating his time and energy on too many different things.
	\item Her father had dissipated her inheritance.
	\end{itemize}
}
\end{enumerate}

\section*{existence}
{\large \color{blue}  existences  }
\subsection*{Explain}
\begin{enumerate}
\item uncountable noun \\
The \textbf{existence} of something is the fact that it is present in the world as a real thing.
 \textit{
	\begin{itemize}
	\item ...the existence of other galaxies.
	\item Tuna are being fished out of existence.
	\item Public worries about accidents are threatening the very existence of the nuclear
power industry.
	\end{itemize}
}
\item countable noun \\
You can refer to someone's way of life as an \textbf{existence} , especially when they live under difficult conditions.
 \textit{
	\begin{itemize}
	\item You may be stuck with a miserable existence for the rest of your life.
	\item Their day-to-day existence was routine.
	\end{itemize}
}
\end{enumerate}

\section*{drag}
{\large \color{blue}  drags  dragging  dragged  }
\subsection*{Explain}
\begin{enumerate}
\item verb \\
If you \textbf{drag} something, you pull it along the ground, often with difficulty.
 \textit{
	\begin{itemize}
	\item He got up and dragged his chair towards the table.
	\end{itemize}
}
\item verb \\
To \textbf{drag} a computer image means to use the mouse to move the position of the image on the screen, or to change
its size or shape.
 \textit{
	\begin{itemize}
	\item Use your mouse to drag the pictures to their new size.
	\end{itemize}
}
\item verb \\
If someone \textbf{drags} you somewhere , they pull you there, or force you to go there by physically threatening you.
 \textit{
	\begin{itemize}
	\item The vigilantes dragged the men out of the vehicles.
	\end{itemize}
}
\item verb \\
If someone \textbf{drags} you somewhere you do not want to go, they make you go there.
 \textit{
	\begin{itemize}
	\item When you can drag him away from his work, he can also be a devoted father.
	\item I've been dragged back from Australia for no sufficient reason.
	\end{itemize}
}
\item verb \\
If you say that you \textbf{drag}  \textbf{yourself} somewhere, you are emphasizing that you have to make a very great effort to go there.
 \textit{
	\begin{itemize}
	\item I find it really hard to drag myself out and exercise regularly.
	\item ...if you manage to drag yourself away from the luxury of the hotel.
	\end{itemize}
}
\item verb \\
If you \textbf{drag} your foot or your leg behind you, you walk with great difficulty because your foot or leg is injured in some way.
 \textit{
	\begin{itemize}
	\item He was barely able to drag his poisoned leg behind him.
	\item He drags his leg, and he can hardly lift his arm.
	\end{itemize}
}
\item verb \\
If the police  \textbf{drag} a river or lake, they pull nets or hooks across the bottom of it in order to look for something.
 \textit{
	\begin{itemize}
	\item Yesterday police frogmen dragged a small pond on the Common.
	\end{itemize}
}
\item verb \\
If a period of time or an event \textbf{drags} , it is very boring and seems to last a long time.
 \textit{
	\begin{itemize}
	\item The minutes dragged past.
	\item The pacing was uneven, and the early second act dragged.
	\end{itemize}
}
\item singular noun \\
If something is \textbf{a drag on} the development or progress of something, it slows it down or makes it more difficult .
 \textit{
	\begin{itemize}
	\item The satellite acts as a drag on the shuttle.
	\item Spending cuts will put a drag on growth.
	\end{itemize}
}
\item singular noun \\
If you say that something is \textbf{a drag} , you mean that it is unpleasant or very dull .
 \textit{
	\begin{itemize}
	\item As far as shopping for clothes goes, it's a drag.
	\item A dry sandwich is a drag to eat.
	\end{itemize}
}
\item countable noun \\
If you take a \textbf{drag}  \textbf{on} a cigarette or pipe that you are smoking , you take in air through it.
 \textit{
	\begin{itemize}
	\item He took a drag on his cigarette, and exhaled the smoke.
	\end{itemize}
}
\item uncountable noun \\
\textbf{Drag} is the resistance to the movement that is experienced by something that is moving through air or through a fluid.
 \textit{
	\begin{itemize}
	\item The drag of those extra air molecules brought the satellite crashing to Earth.
	\end{itemize}
}
\item uncountable noun \\
\textbf{Drag} is the wearing of women's clothes by a male entertainer .
 \textit{
	\begin{itemize}
	\item The star wore drag and false eyelashes.
	\item The neighborhood is given over to performers, stilt walkers and drag queens.
	\end{itemize}
}
\item  \\
 drag your feet \textit{
	\begin{itemize}
	\end{itemize}
}
\end{enumerate}

\section*{glitter}
{\large \color{blue}  glitters  glittering  glittered  }
\subsection*{Explain}
\begin{enumerate}
\item verb \\
If something \textbf{glitters} , light comes from or is reflected off different parts of it.
 \textit{
	\begin{itemize}
	\item The bay glittered in the sunshine.
	\item The Palace glittered with lights.
	\end{itemize}
}
\item verb \\
If someone's eyes  \textbf{glitter} , they are bright and express a particular emotion such as excitement or interest .
 \textit{
	\begin{itemize}
	\item His eyes glittered with a tense amusement.
	\end{itemize}
}
\item uncountable noun \\
\textbf{Glitter} consists of tiny shining pieces of metal. It is glued to things for decoration .
 \textit{
	\begin{itemize}
	\item Decorate the tunic with sequins or glitter.
	\end{itemize}
}
\item uncountable noun \\
You can use \textbf{glitter} to refer to superficial attractiveness or to the excitement connected with something.
 \textit{
	\begin{itemize}
	\item She was blinded by the glitter and the glamour of her own life.
	\end{itemize}
}
\end{enumerate}

\section*{draw}
{\large \color{blue}  draws  drawing  drew  drawn  }
\subsection*{Explain}
\begin{enumerate}
\item verb \\
When you \textbf{draw} , or when you \textbf{draw} something, you use a pencil or pen to produce a picture, pattern, or diagram .
 \textit{
	\begin{itemize}
	\item She would sit there drawing with the pencil stub.
	\item Draw a rough design for a logo.
	\item He starts a painting by quickly drawing simplified shapes.
	\end{itemize}
}
\item verb \\
When a vehicle \textbf{draws}  somewhere , it moves there smoothly and steadily.
 \textit{
	\begin{itemize}
	\item Claire had seen the taxi drawing away.
	\end{itemize}
}
\item verb \\
If you \textbf{draw} somewhere, you move there slowly.
 \textit{
	\begin{itemize}
	\item She drew away and did not smile.
	\item When we drew level, he neither slowed down nor accelerated.
	\end{itemize}
}
\item verb \\
If you \textbf{draw} something or someone in a particular direction, you move them in that direction,
usually by pulling them gently.
 \textit{
	\begin{itemize}
	\item He drew his chair nearer the fire.
	\item He put his arm around Caroline's shoulders and drew her close to him.
	\item Wilson drew me aside after an interview.
	\end{itemize}
}
\item verb \\
When you \textbf{draw} a curtain or blind , you pull it across a window, either to cover or to uncover it.
 \textit{
	\begin{itemize}
	\item After drawing the curtains, she lit a candle.
	\item Mother was lying on her bed, with the blinds drawn.
	\end{itemize}
}
\item verb \\
If someone \textbf{draws} a gun, knife , or other weapon, they pull it out of its container and threaten you with it.
 \textit{
	\begin{itemize}
	\item He drew his dagger and turned to face his pursuers.
	\end{itemize}
}
\item verb \\
If an animal or vehicle \textbf{draws} something such as a cart , carriage , or another vehicle, it pulls it along.
 \textit{
	\begin{itemize}
	\item ...a slow-moving tractor, drawing a trailer.
	\item ...a chariot drawn by six black mules.
	\end{itemize}
}
\item verb \\
If you \textbf{draw} a deep breath , you breathe in deeply once.
 \textit{
	\begin{itemize}
	\item He paused, drawing a deep breath.
	\end{itemize}
}
\item verb \\
If you \textbf{draw}  \textbf{on} a cigarette , you breathe the smoke from it into your mouth or lungs .
 \textit{
	\begin{itemize}
	\item He drew on an American cigarette.
	\item Her cheeks hollowed as she drew smoke into her lungs.
	\end{itemize}
}
\item verb \\
To \textbf{draw} something such as water or energy \textbf{from} a particular source means to take it from that source.
 \textit{
	\begin{itemize}
	\item Villagers still have to draw their water from wells.
	\end{itemize}
}
\item verb \\
If something that hits you or presses part of your body \textbf{draws} blood, it cuts your skin so that it bleeds.
 \textit{
	\begin{itemize}
	\item Any practice that draws blood could increase the risk of getting the virus.
	\end{itemize}
}
\item verb \\
If you \textbf{draw} money out of a bank, building society, or savings account, you get it from the account so that you can use it.
 \textit{
	\begin{itemize}
	\item She was drawing out cash from a cash machine.
	\item Companies could not draw money from bank accounts as cash.
	\end{itemize}
}
\item verb \\
If you \textbf{draw} a salary or a sum of money, you receive a sum of money regularly.
 \textit{
	\begin{itemize}
	\item For the first few years I didn't draw any salary at all.
	\item He is moving ever closer to drawing his pension.
	\end{itemize}
}
\item verb \\
To \textbf{draw} something means to choose it or to be given it, as part of a competition, game, or
lottery.
 \textbf{Draw} is also a noun.
 \textit{
	\begin{itemize}
	\item We delved through a sackful of letters to draw the winning name.
	\item Aston Villa have drawn a Czech team in the first round of the UEFA Cup.
	\item ...the draw for the quarter-finals of the UEFA Cup.
	\end{itemize}
}
\item countable noun \\
A \textbf{draw} is a competition where people pay money for numbered or named tickets, then some
of those tickets are chosen, and the owners are given prizes.
 \textit{
	\begin{itemize}
	\end{itemize}
}
\item verb \\
To \textbf{draw} something \textbf{from} a particular thing or place means to take or get it from that thing or place.
 \textit{
	\begin{itemize}
	\item I draw strength from the millions of women who have faced this challenge successfully.
	\item The students are drawn from a cross-section of backgrounds.
	\end{itemize}
}
\item verb \\
If you \textbf{draw} a particular conclusion , you decide that that conclusion is true.
 \textit{
	\begin{itemize}
	\item He draws two conclusions from this.
	\item He says he cannot yet draw any conclusions about the murders.
	\end{itemize}
}
\item verb \\
If you \textbf{draw} a comparison , parallel, or distinction , you compare or contrast two different ideas, systems, or other things.
 \textit{
	\begin{itemize}
	\item ...literary critics drawing comparisons between George Sand and George Eliot.
	\item Interesting distinctions can be drawn between the two populations.
	\end{itemize}
}
\item verb \\
If you \textbf{draw} someone's attention to something, you make them aware of it or make them think about it.
 \textit{
	\begin{itemize}
	\item He was waving his arms to draw their attention.
	\item He just wants to draw attention to the plight of the unemployed.
	\end{itemize}
}
\item verb \\
If someone or something \textbf{draws} a particular reaction, people react to it in that way.
 \textit{
	\begin{itemize}
	\item Such a policy would inevitably draw fierce resistance from farmers.
	\item The club's summer signings have drawn criticism.
	\end{itemize}
}
\item verb \\
If something such as a film or an event \textbf{draws} a lot of people, it is so interesting or entertaining that a lot of people go to it.
 \textit{
	\begin{itemize}
	\item The game is currently drawing huge crowds.
	\end{itemize}
}
\item verb \\
If someone or something \textbf{draws} you, it attracts you very strongly.
 \textit{
	\begin{itemize}
	\item He drew and enthralled her.
	\item What drew him to the area was its proximity to central London.
	\end{itemize}
}
\item verb \\
If someone will not \textbf{be drawn} or refuses to \textbf{be drawn} , they will not reply to questions in the way that you want them to, or will not reveal information or their opinion.
 \textit{
	\begin{itemize}
	\item The ambassador would not be drawn on questions of a political nature.
	\item 'Did he say why?'—'No, he refuses to be drawn.'
	\end{itemize}
}
\item verb \\
In a game or competition, if one person or team \textbf{draws}  \textbf{with} another one, or if two people or teams \textbf{draw} , they have the same number of points or goals at the end of the game.
 \textbf{Draw} is also a noun.
 \textit{
	\begin{itemize}
	\item Holland and the Republic of Ireland drew one–one.
	\item We drew with Ireland in the first game.
	\item Egypt drew two of their matches in Italy.
	\item We were happy to come away with a draw against Sweden.
	\end{itemize}
}
\item  \\
 draw to an end/draw to a close \textit{
	\begin{itemize}
	\end{itemize}
}
\item  \\
 draw close/draw near \textit{
	\begin{itemize}
	\end{itemize}
}
\end{enumerate}

\section*{glory}
{\large \color{blue}  glories  glorying  gloried  }
\subsection*{Explain}
\begin{enumerate}
\item uncountable noun \\
\textbf{Glory} is the fame and admiration that you gain by doing something impressive .
 \textit{
	\begin{itemize}
	\item Walsham had his moment of glory when he won a 20km race.
	\item ...we were still basking in the glory of our Championship win.
	\end{itemize}
}
\item plural noun \\
A person's \textbf{glories} are the occasions when they have done something people greatly admire which makes them famous .
 \textit{
	\begin{itemize}
	\item The album sees them re-living past glories but not really breaking any new ground.
	\item ...the military glories of Frederick the Great.
	\end{itemize}
}
\item uncountable noun \\
\textbf{The}  \textbf{glory}  \textbf{of} something is its great beauty or impressive nature.
 \textit{
	\begin{itemize}
	\item The glory of the idea blossomed in his mind.
	\end{itemize}
}
\item countable noun \\
\textbf{The}  \textbf{glories}  \textbf{of} a culture or place are the things that people admire most about it.
 \textit{
	\begin{itemize}
	\item ...a tour of Florence, to enjoy the artistic glories of the Italian Renaissance.
	\item One of the glories of the island has always been its bird population.
	\end{itemize}
}
\item verb \\
If you \textbf{glory in} a situation or activity, you enjoy it very much.
 \textit{
	\begin{itemize}
	\item The workers were glorying in their new-found freedom.
	\item He does not glory in his past successes and looks forward to achieving more.
	\end{itemize}
}
\item  \\
 a blaze of glory \textit{
	\begin{itemize}
	\end{itemize}
}
\end{enumerate}

\section*{embrace}
{\large \color{blue}  embraces  embracing  embraced  }
\subsection*{Explain}
\begin{enumerate}
\item verb \\
If you \textbf{embrace} someone, you put your arms around them and hold them tightly, usually in order to show your love or affection for them. You can also  say that two people \textbf{embrace} .
 \textbf{Embrace} is also a noun .
 \textit{
	\begin{itemize}
	\item Penelope came forward and embraced her sister.
	\item At first people were sort of crying for joy and embracing each other.
	\item He threw his arms round her and they embraced passionately.
	\item ...a young couple locked in an embrace.
	\end{itemize}
}
\item verb \\
If you \textbf{embrace} a change , political system, or idea, you accept it and start  supporting it or believing in it.
 \textbf{Embrace} is also a noun.
 \textit{
	\begin{itemize}
	\item He embraces the new information age.
	\item The new rules have been embraced by government watchdog organizations.
	\item The marriage signalled James's embrace of the Catholic faith.
	\end{itemize}
}
\item verb \\
If something \textbf{embraces} a group of people, things, or ideas, it includes them in a larger group or category .
 \textit{
	\begin{itemize}
	\item ...a theory that would embrace the whole field of human endeavour.
	\end{itemize}
}
\end{enumerate}

\section*{kit}
{\large \color{blue}  kits  kitting  kitted  }
\subsection*{Explain}
\begin{enumerate}
\item countable noun \\
A \textbf{kit} is a group of items that are kept together, often in the same container, because they are all used for
similar purposes.
 \textit{
	\begin{itemize}
	\item ...a well-stocked first aid kit.
	\end{itemize}
}
\item uncountable noun \\
\textbf{Kit} is special clothing and equipment that you use when you take part in a particular activity,
 especially a sport .
 \textit{
	\begin{itemize}
	\item I forgot my gym kit.
	\end{itemize}
}
\item countable noun \\
A \textbf{kit} is a set of parts that can be put together in order to make something.
 \textit{
	\begin{itemize}
	\item Her popular pot holder is also available in do-it-yourself kits.
	\end{itemize}
}
\item  \\
 get your kit off/keep your kit on \textit{
	\begin{itemize}
	\end{itemize}
}
\end{enumerate}

\section*{endure}
{\large \color{blue}  endures  enduring  endured  }
\subsection*{Explain}
\begin{enumerate}
\item verb \\
If you \textbf{endure} a painful or difficult  situation , you experience it and do not avoid it or give up, usually because you cannot.
 \textit{
	\begin{itemize}
	\item The company endured heavy financial losses.
	\item They'd never allow their children to have the kind of life or experiences they had
to endure.
	\end{itemize}
}
\item verb \\
If something \textbf{endures} , it continues to exist without any loss in quality or importance .
 \textit{
	\begin{itemize}
	\item Somehow the language endures and continues to survive.
	\end{itemize}
}
\end{enumerate}

\section*{light}
{\large \color{blue}  lights  lighting  lit  lighted  lighter  lightest  }
\subsection*{Explain}
\begin{enumerate}
\item uncountable noun \\
\textbf{Light} is the brightness that lets you see things. Light comes from sources such as the sun, moon, lamps, and fire.
 \textit{
	\begin{itemize}
	\item Cracks of light filtered through the shutters.
	\item Light and water in embassy buildings were cut off.
	\item It was difficult to see in the dim light.
	\item ...ultraviolet light.
	\end{itemize}
}
\item countable noun \\
A \textbf{light} is something such as an electric lamp which produces light.
 \textit{
	\begin{itemize}
	\item The janitor comes round to turn the lights out.
	\item You get into the music, the lights and the people around you.
	\item ...street lights.
	\end{itemize}
}
\item plural noun \\
You can use \textbf{lights} to refer to a set of traffic lights.
 \textit{
	\begin{itemize}
	\item ...the heavy city traffic with its endless delays at lights and crossings.
	\end{itemize}
}
\item verb \\
If a place or object \textbf{is lit} by something, it has light shining on it.
 \textit{
	\begin{itemize}
	\item It was dark and a giant moon lit the road so brightly you could see the landscape
clearly.
	\item The room was lit by only the one light.
	\item The low sun lit the fortress walls with yellow light.
	\item ...the little lighted space at the bottom of the stairwell.
	\end{itemize}
}
\item adjective \\
If it is \textbf{light} , the sun is providing light at the beginning or end of the day.
 \textit{
	\begin{itemize}
	\item It was still light when we arrived at Lalong Creek.
	\item He would often rise as soon as it was light and go into the garden.
	\item ...light summer evenings.
	\end{itemize}
}
\item adjective \\
If a room or building is \textbf{light} , it has a lot of natural light in it, for example because it has large windows.
 \textit{
	\begin{itemize}
	\item It is a light room with tall windows.
	\item Her house is light and airy, crisp and clean.
	\end{itemize}
}
\item verb \\
If you \textbf{light} something such as a cigarette or fire, or if it \textbf{lights} , it starts burning.
 \textit{
	\begin{itemize}
	\item Stephen hunched down to light a cigarette.
	\item If the charcoal does fail to light, use a special liquid spray and light it with
a long taper.
	\item ...a lighted candle.
	\end{itemize}
}
\item singular noun \\
If someone asks you for \textbf{a light} , they want a match or cigarette lighter so they can start smoking.
 \textit{
	\begin{itemize}
	\item Have you got a light anybody?
	\end{itemize}
}
\item countable noun \\
If something is presented in a particular \textbf{light} , it is presented so that you think about it in a particular way or so that it appears
to be of a particular nature.
 \textit{
	\begin{itemize}
	\item He has worked hard in recent months to portray New York in a better light.
	\end{itemize}
}
\item singular noun \\
You can refer to the type of influence that something has on situations, people, or
things as \textbf{the light of} that situation, person, or thing.
 \textit{
	\begin{itemize}
	\item ...the harsh light of reality.
	\end{itemize}
}
\item plural noun \\
You say that something is done or is acceptable according to someone's \textbf{lights} when you mean that it is done or is acceptable according to their own ideas and standards.
 \textit{
	\begin{itemize}
	\item By his own lights, obeying orders is what a soldier is meant to do.
	\end{itemize}
}
\item singular noun \\
If there is a \textbf{light} in someone's eyes, there is an expression in their eyes that shows you the mood they are in or what they are thinking about.
 \textit{
	\begin{itemize}
	\item I remembered the curious expectant light in his eyes.
	\end{itemize}
}
\item  \\
 come to light/bring sth to light \textit{
	\begin{itemize}
	\end{itemize}
}
\item  \\
 light dawns on sb \textit{
	\begin{itemize}
	\end{itemize}
}
\item  \\
 first light \textit{
	\begin{itemize}
	\end{itemize}
}
\item  \\
 give sb a green light \textit{
	\begin{itemize}
	\end{itemize}
}
\item  \\
 in the light of sth \textit{
	\begin{itemize}
	\end{itemize}
}
\item  \\
 out like a light \textit{
	\begin{itemize}
	\end{itemize}
}
\item  \\
 to see the light of day/see the light \textit{
	\begin{itemize}
	\end{itemize}
}
\item  \\
 to see the light of day \textit{
	\begin{itemize}
	\end{itemize}
}
\item  \\
 to see the light \textit{
	\begin{itemize}
	\end{itemize}
}
\item  \\
 set light to sth \textit{
	\begin{itemize}
	\end{itemize}
}
\item  \\
 to shed light on something \textit{
	\begin{itemize}
	\end{itemize}
}
\item  \\
 light at the end of the tunnel \textit{
	\begin{itemize}
	\end{itemize}
}
\end{enumerate}

\section*{entertain}
{\large \color{blue}  entertains  entertaining  entertained  }
\subsection*{Explain}
\begin{enumerate}
\item verb \\
If a performer, performance , or activity  \textbf{entertains} you, it amuses you, interests you, or gives you pleasure .
 \textit{
	\begin{itemize}
	\item ...games and ideas to entertain children.
	\item They were entertained by top singers, dancers and celebrities.
	\item Children's television not only entertains but also teaches.
	\end{itemize}
}
\item verb \\
If you \textbf{entertain} people, you provide food and drink for them, for example when you have invited them to your house .
 \textit{
	\begin{itemize}
	\item I don't like to entertain guests anymore.
	\item You weren't allowed to entertain men in your rooms even with a chaperone.
	\item The Monroes continued to entertain extravagantly.
	\end{itemize}
}
\item verb \\
If you \textbf{entertain} an idea or suggestion , you allow yourself to consider it as possible or as worth  thinking about seriously .
 \textit{
	\begin{itemize}
	\item I feel how foolish I am to entertain doubts.
	\item I wouldn't entertain the idea of such an unsociable job.
	\end{itemize}
}
\end{enumerate}

\section*{literacy}
{\large \color{blue}  }
\subsection*{Explain}
\begin{enumerate}
\item uncountable noun \\
\textbf{Literacy} is the ability to read and write.
 \textit{
	\begin{itemize}
	\item Many adults have some problems with literacy and numeracy.
	\item The literacy rate there is the highest in Central America.
	\end{itemize}
}
\end{enumerate}

\section*{excite}
{\large \color{blue}  excites  exciting  excited  }
\subsection*{Explain}
\begin{enumerate}
\item verb \\
If something \textbf{excites} you, it makes you feel very happy , eager , or enthusiastic .
 \textit{
	\begin{itemize}
	\item I only take on work that excites me, even if it means turning down lots of money.
	\item We'd not been excited by anything for about three years.
	\item Where the show really excites is in the display of avant-garde photography.
	\end{itemize}
}
\item verb \\
If something \textbf{excites} a particular feeling, emotion, or reaction in someone, it causes them to experience it.
 \textit{
	\begin{itemize}
	\item He is a man who excites strong feelings.
	\item Daniel's early exposure to motor racing did not excite his interest.
	\item Reports of the plot of this unusual film tend to excite revulsion.
	\end{itemize}
}
\item verb \\
If something or someone \textbf{excites} you, they cause you to feel sexual  desire .
 \textit{
	\begin{itemize}
	\end{itemize}
}
\item verb \\
To \textbf{excite} a physical object such as an atomic  particle or an organ in your body means to increase the amount of energy, movement, or activity
in it.
 \textit{
	\begin{itemize}
	\item The amount of nicotine in these nicotine substitutes can be enough to excite the
heart.
	\end{itemize}
}
\end{enumerate}

\section*{margin}
{\large \color{blue}  margins  }
\subsection*{Explain}
\begin{enumerate}
\item countable noun \\
A \textbf{margin} is the difference between two amounts, especially the difference in the number of votes or points between the winner and the loser in an election or other contest .
 \textit{
	\begin{itemize}
	\item They could end up with a 50-point winning margin.
	\item The Sunday Times remains the brand leader by a huge margin.
	\item The margin in favor was 280-to-153.
	\end{itemize}
}
\item countable noun \\
The \textbf{margin} of a written or printed page is the empty space at the side of the page.
 \textit{
	\begin{itemize}
	\item She added her comments in the margin.
	\end{itemize}
}
\item variable noun \\
If there is a \textbf{margin} for something in a situation, there is some freedom to choose what to do or decide how to do it.
 \textit{
	\begin{itemize}
	\item The money is collected in a straightforward way with little margin for error.
	\item Courts and parliaments have a wide margin of discretion in enforcing convention rights.
	\item Out in front, Clarke had built up such a sizeable safety margin that he eased the
pace and started cruising.
	\end{itemize}
}
\item countable noun \\
The \textbf{margin} of a place or area is the extreme edge of it.
 \textit{
	\begin{itemize}
	\item ...the low coastal plain along the western margin.
	\item These islands are on the margins of human habitation.
	\end{itemize}
}
\item plural noun \\
To be \textbf{on the}  \textbf{margins} of a society, group, or activity means to be among the least typical or least important parts of it.
 \textit{
	\begin{itemize}
	\item Students have played an important role in the past, but for the moment, they're on
the margins.
	\item ...signs of the party's rapid retreat to the political margins.
	\end{itemize}
}
\end{enumerate}

\section*{express}
{\large \color{blue}  expresses  expressing  expressed  }
\subsection*{Explain}
\begin{enumerate}
\item verb \\
When you \textbf{express} an idea or feeling, or \textbf{express}  \textbf{yourself} , you show what you think or feel .
 \textit{
	\begin{itemize}
	\item He expressed grave concern at American attitudes.
	\item Sumner frowned at us, doing his best to express wordless disapproval.
	\item He expresses himself easily in English.
	\end{itemize}
}
\item verb \\
If an idea or feeling \textbf{expresses}  \textbf{itself} in some way, it can be clearly seen in someone's actions or in its effects on a situation.
 \textit{
	\begin{itemize}
	\item She is obviously unhappy, and misery often expresses itself as anger.
	\end{itemize}
}
\item verb \\
In mathematics , if you \textbf{express} a quantity or mathematical  problem in a particular way, you write it using particular symbols, figures, or equations .
 \textit{
	\begin{itemize}
	\item We can express that equation like that.
	\item It is expressed as a percentage.
	\end{itemize}
}
\item adjective \\
An \textbf{express}  command or order is one that is clearly and deliberately stated.
 \textit{
	\begin{itemize}
	\item The ship was sunk on express orders from the Prime Minister.
	\end{itemize}
}
\item adjective \\
If you refer to an \textbf{express}  intention or purpose, you are emphasizing that it is a deliberate and specific one that you have before you do something.
 \textit{
	\begin{itemize}
	\item I had obtained my first camera for the express purpose of taking railway photographs.
	\end{itemize}
}
\item adjective \\
\textbf{Express} is used to describe special services which are provided by companies or organizations such as the Post Office, in which things are sent or done faster than usual for a higher price.
 \textbf{Express} is also an adverb .
 \textit{
	\begin{itemize}
	\item I paid extra for the express service.
	\item It was sent to us by express mail.
	\item Send it express.
	\end{itemize}
}
\item countable noun \\
An \textbf{express} or an \textbf{express} train is a fast train which stops at very few stations.
 \textit{
	\begin{itemize}
	\item Punctually at 7.45, the express to Kuala Lumpur left Singapore station.
	\item He had boarded an express for Rome.
	\end{itemize}
}
\item countable noun \\
An \textbf{express} is a fast bus or coach which goes from one place to another directly or with very few stops.
 \textit{
	\begin{itemize}
	\end{itemize}
}
\end{enumerate}

\section*{member}
{\large \color{blue}  members  }
\subsection*{Explain}
\begin{enumerate}
\item countable noun \\
A \textbf{member} of a group is one of the people, animals, or things belonging to that group.
 \textit{
	\begin{itemize}
	\item He refused to name the members of staff involved.
	\item Their lack of training could put members of the public at risk.
	\item ...a sunflower or a similar member of the daisy family.
	\item ...the brightest members of a dense cluster of stars at the Galaxy's centre.
	\end{itemize}
}
\item countable noun \\
A \textbf{member} of an organization such as a club or a political party is a person who has officially  joined the organization.
 \textit{
	\begin{itemize}
	\item The support of our members is of great importance to the Association.
	\item Britain is a full member of NATO.
	\end{itemize}
}
\item adjective \\
A \textbf{member country} or \textbf{member state} is one of the countries that has joined an international organization or group.
 \textit{
	\begin{itemize}
	\item ...the 13 member countries of Opec.
	\item ...a co-ordinated approach, with each member state doing what it could.
	\end{itemize}
}
\item countable noun \\
A \textbf{member} or \textbf{Member} is a person who has been elected to a parliament or legislature .
 \textit{
	\begin{itemize}
	\item He was elected to Parliament as the Member for Leeds.
	\item He was elected to Parliament in 1959 as the member for Finchley.
	\item ...a member of the Italian Parliament.
	\end{itemize}
}
\end{enumerate}

\section*{fetch}
{\large \color{blue}  fetches  fetching  fetched  }
\subsection*{Explain}
\begin{enumerate}
\item verb \\
If you \textbf{fetch} something or someone, you go and get them from the place where they are.
 \textit{
	\begin{itemize}
	\item Sylvia fetched a towel from the bathroom.
	\item Fetch me a glass of water.
	\item The caddie ran over to fetch something for him.
	\end{itemize}
}
\item verb \\
If something \textbf{fetches} a particular sum of money, it is sold for that amount.
 \textit{
	\begin{itemize}
	\item The painting is expected to fetch between two and three million pounds.
	\end{itemize}
}
\item  \\
 fetch and carry \textit{
	\begin{itemize}
	\end{itemize}
}
\end{enumerate}

\section*{millionaire}
{\large \color{blue}  millionaires  }
\subsection*{Explain}
\begin{enumerate}
\item countable noun \\
A \textbf{millionaire} is a very rich person who has money or property worth at least a million pounds or dollars .
 \textit{
	\begin{itemize}
	\item By the time he died, he was a millionaire.
	\end{itemize}
}
\end{enumerate}

\section*{hamper}
{\large \color{blue}  hampers  hampering  hampered  }
\subsection*{Explain}
\begin{enumerate}
\item verb \\
If someone or something \textbf{hampers} you, they make it difficult for you to do what you are trying to do.
 \textit{
	\begin{itemize}
	\item The bad weather hampered rescue operations.
	\item I was hampered by a lack of information.
	\end{itemize}
}
\item countable noun \\
A \textbf{hamper} is a basket containing food of various kinds that is given to people as a present .
 \textit{
	\begin{itemize}
	\item ...a luxury food hamper.
	\end{itemize}
}
\item countable noun \\
A \textbf{hamper} is a large basket with a lid , used especially for carrying food in.
 \textit{
	\begin{itemize}
	\item ...a picnic hamper.
	\end{itemize}
}
\end{enumerate}

\section*{morning}
{\large \color{blue}  mornings  }
\subsection*{Explain}
\begin{enumerate}
\item variable noun \\
The \textbf{morning} is the part of each day between the time that people usually wake up and 12 o'clock noon or lunchtime .
 \textit{
	\begin{itemize}
	\item During the morning your guide will take you around the city.
	\item On Sunday morning Bill was woken by the telephone.
	\item He read about it in his morning paper.
	\end{itemize}
}
\item singular noun \\
If you refer to a particular time in \textbf{the}  \textbf{morning} , you mean a time between 12 o'clock midnight and 12 o'clock noon.
 \textit{
	\begin{itemize}
	\item I often stayed up until two or three in the morning.
	\item The attack happened in the early hours of the morning.
	\end{itemize}
}
\item  \\
 in the morning \textit{
	\begin{itemize}
	\end{itemize}
}
\item  \\
 morning, noon, and night \textit{
	\begin{itemize}
	\end{itemize}
}
\end{enumerate}

\section*{handicap}
{\large \color{blue}  handicaps  handicapping  handicapped  }
\subsection*{Explain}
\begin{enumerate}
\item countable noun \\
A \textbf{handicap} is a physical or mental disability.
 \textit{
	\begin{itemize}
	\item He lost his leg when he was ten, but learnt to overcome his handicap.
	\end{itemize}
}
\item countable noun \\
A \textbf{handicap} is an event or situation that places you at a disadvantage and makes it harder for you to do something.
 \textit{
	\begin{itemize}
	\item She was away from school for 15 weeks, a handicap she could have done without.
	\item Being a foreigner was not a handicap.
	\end{itemize}
}
\item verb \\
If an event or a situation \textbf{handicaps} someone or something, it places them at a disadvantage.
 \textit{
	\begin{itemize}
	\item Greater levels of stress may seriously handicap some students.
	\item We felt our system was handicapping some of the good players we have.
	\end{itemize}
}
\item countable noun \\
In golf , a \textbf{handicap} is an advantage given to someone who is not a good  player , in order to make the players more equal . As you improve , your handicap gets  lower .
 \textit{
	\begin{itemize}
	\item I see your handicap is down from 16 to 12.
	\end{itemize}
}
\item countable noun \\
In horse racing, a \textbf{handicap} is a race in which some competitors are given a disadvantage of extra weight in an attempt to give everyone an equal chance of winning.
 \textit{
	\begin{itemize}
	\end{itemize}
}
\end{enumerate}

\section*{orbit}
{\large \color{blue}  orbits  orbiting  orbited  }
\subsection*{Explain}
\begin{enumerate}
\item variable noun \\
An \textbf{orbit} is the curved path in space that is followed by an object  going  round and round a planet, moon , or star .
 \textit{
	\begin{itemize}
	\item Mars and Earth have orbits which change with time.
	\item The planet is probably in orbit around a small star.
	\end{itemize}
}
\item verb \\
If something such as a satellite \textbf{orbits} a planet, moon, or sun , it moves around it in a continuous , curving path.
 \textit{
	\begin{itemize}
	\item In 1957 the Soviet Union launched the first satellite to orbit the Earth.
	\end{itemize}
}
\item singular noun \\
The \textbf{orbit} of a particular person, group, or institution is the area over which they have influence.
 \textit{
	\begin{itemize}
	\item He is a man who still commands enormous respect within the orbit of football club
management.
	\end{itemize}
}
\end{enumerate}

\section*{haul}
{\large \color{blue}  hauls  hauling  hauled  }
\subsection*{Explain}
\begin{enumerate}
\item verb \\
If you \textbf{haul} something which is heavy or difficult to move, you move it using a lot of effort.
 \textit{
	\begin{itemize}
	\item A crane had to be used to haul the car out of the stream.
	\item He hauled himself to his feet.
	\item She hauled up her bedroom window and leaned out.
	\end{itemize}
}
\item verb \\
If someone \textbf{is hauled}  \textbf{before} a court or someone in authority , they are made to appear before them because they are accused of having done something wrong .
 \textbf{Haul up} means the same as haul .
 \textit{
	\begin{itemize}
	\item He was hauled before the managing director and fired.
	\item He was hauled up before the Board of Trustees.
	\item She was late for the interview after being hauled up for speeding.
	\end{itemize}
}
\item countable noun \\
A \textbf{haul} is a quantity of things that are stolen , or a quantity of stolen or illegal goods found by police or customs .
 \textit{
	\begin{itemize}
	\item The size of the drugs haul shows that the international trade in heroin is still
flourishing.
	\item Another break-in yielded a £4,000 haul of jewellery.
	\end{itemize}
}
\item  \\
 long haul \textit{
	\begin{itemize}
	\end{itemize}
}
\end{enumerate}

\section*{page}
{\large \color{blue}  pages  paging  paged  }
\subsection*{Explain}
\begin{enumerate}
\item countable noun \\
A \textbf{page} is one side of one of the pieces of paper in a book, magazine , or newspaper. Each page usually has a number printed at the top or bottom .
 \textit{
	\begin{itemize}
	\item Where's your book? Take it out and turn to page 4.
	\item ...the front page of the Guardian.
	\item ...1,400 pages of top-secret information.
	\end{itemize}
}
\item countable noun \\
The \textbf{pages} of a book, magazine, or newspaper are the pieces of paper it consists of.
 \textit{
	\begin{itemize}
	\item He turned the pages of his notebook.
	\item Over the page you can read all about the six great books on offer.
	\end{itemize}
}
\item countable noun \\
You can refer to an important event or period of time as a \textbf{page} of history .
 \textit{
	\begin{itemize}
	\item ...a new page in the country's political history.
	\end{itemize}
}
\item verb \\
If someone who is in a public place \textbf{is paged} , they receive a message, often over a speaker , telling them that someone is trying to contact them.
 \textit{
	\begin{itemize}
	\item He was paged repeatedly as the flight was boarding.
	\item I'll have them paged and tell them you're here.
	\end{itemize}
}
\item countable noun \\
A \textbf{page} is a small boy who accompanies the bride at a wedding.
 \textit{
	\begin{itemize}
	\end{itemize}
}
\item countable noun \\
A \textbf{page} is a young person who takes messages or does small jobs for members of the United States Congress or state legislatures .
 \textit{
	\begin{itemize}
	\end{itemize}
}
\item countable noun \\
In former times, a \textbf{page} was a young boy who was a knight's servant and was learning to be a knight.
 \textit{
	\begin{itemize}
	\end{itemize}
}
\end{enumerate}

\section*{incur}
{\large \color{blue}  incurs  incurring  incurred  }
\subsection*{Explain}
\begin{enumerate}
\item verb \\
If you \textbf{incur} something unpleasant , it happens to you because of something you have done.
 \textit{
	\begin{itemize}
	\item The government had also incurred huge debts.
	\item She falls in love and incurs the wrath of her father.
	\item ...the terrible damage incurred during the past decade.
	\end{itemize}
}
\end{enumerate}

\section*{percentage}
{\large \color{blue}  percentages  }
\subsection*{Explain}
\begin{enumerate}
\item countable noun \\
A \textbf{percentage} is a fraction of an amount expressed as a particular number of hundredths of that amount.
 \textit{
	\begin{itemize}
	\item Only a few vegetable-origin foods have such a high percentage of protein.
	\end{itemize}
}
\end{enumerate}

\section*{manifest}
{\large \color{blue}  manifests  manifesting  manifested  }
\subsection*{Explain}
\begin{enumerate}
\item adjective \\
If you say that something is \textbf{manifest} , you mean that it is clearly  true and that nobody would disagree with it if they saw it or considered it.
 \textit{
	\begin{itemize}
	\item ...the manifest failure of the policies.
	\item There may be unrecognised cases of manifest injustice of which we are unaware.
	\end{itemize}
}
\item verb \\
If you \textbf{manifest} a particular quality, feeling , or illness , or if it \textbf{manifests}  \textbf{itself} , it becomes visible or obvious.
 \textbf{Manifest} is also an adjective .
 \textit{
	\begin{itemize}
	\item He manifested a pleasing personality on stage.
	\item The virus needs two weeks to manifest itself.
	\item Their frustration and anger will manifest itself in crying and screaming.
	\item He's only convincing when that inner fury manifests itself.
	\item The same alarm is manifest everywhere.
	\item Some of her social aspirations were made manifest.
	\end{itemize}
}
\end{enumerate}

\section*{plaster}
{\large \color{blue}  plasters  plastering  plastered  }
\subsection*{Explain}
\begin{enumerate}
\item uncountable noun \\
\textbf{Plaster} is a smooth paste made of sand, lime, and water which goes  hard when it dries. Plaster is used to cover walls and ceilings and is also used to make sculptures .
 \textit{
	\begin{itemize}
	\item There were huge cracks in the plaster, and the green shutters were faded.
	\item In the Musée d'Orsay in Paris is a sculpture in plaster by Rodin.
	\end{itemize}
}
\item verb \\
If you \textbf{plaster} a wall or ceiling, you cover it with a layer of plaster.
 \textit{
	\begin{itemize}
	\item The ceiling he had just plastered fell in and knocked him off his ladder.
	\end{itemize}
}
\item verb \\
If you \textbf{plaster} a surface or a place \textbf{with}  posters or pictures , you stick a lot of them all over it.
 \textit{
	\begin{itemize}
	\item They plastered the city with posters condemning her election.
	\item His room is plastered with pictures of Porsches and Ferraris.
	\end{itemize}
}
\item verb \\
If you \textbf{plaster}  \textbf{yourself in} some kind of sticky substance, you cover yourself in it.
 \textit{
	\begin{itemize}
	\item She plastered herself from head to toe in high-factor sun lotion.
	\end{itemize}
}
\item countable noun \\
A \textbf{plaster} is a strip of sticky material used for covering small cuts or sores on your body.
 \textit{
	\begin{itemize}
	\end{itemize}
}
\item  \\
 in plaster \textit{
	\begin{itemize}
	\end{itemize}
}
\end{enumerate}

\section*{mean}
{\large \color{blue}  means  meaning  meant  }
\subsection*{Explain}
\begin{enumerate}
\item verb \\
If you want to know what a word, code , signal, or gesture  \textbf{means} , you want to know what it refers to or what its message is.
 \textit{
	\begin{itemize}
	\item In modern Welsh, 'glas' means 'blue'.
	\item What does 'evidence' mean?
	\item The red signal means you can shoot.
	\end{itemize}
}
\item verb \\
If you ask someone what they \textbf{mean} , you are asking them to explain  exactly what or who they are referring to or what they are intending to say.
 \textit{
	\begin{itemize}
	\item Do you mean me?
	\item Let me illustrate what I mean with an old story.
	\item What do you think he means by that?
	\item I think he means that he does not want this marriage to turn out like his friend's.
	\end{itemize}
}
\item verb \\
If something \textbf{means} something \textbf{to} you, it is important to you in some way.
 \textit{
	\begin{itemize}
	\item The idea that she witnessed this shameful incident meant nothing to him.
	\item It would mean a lot to them to win.
	\end{itemize}
}
\item verb \\
If one thing \textbf{means} another, it shows that the second thing exists or is true.
 \textit{
	\begin{itemize}
	\item An enlarged prostate does not necessarily mean cancer.
	\item Just because he has a beard doesn't necessarily mean he's a hippy.
	\end{itemize}
}
\item verb \\
If one thing \textbf{means} another, the first thing leads to the second thing happening .
 \textit{
	\begin{itemize}
	\item It would almost certainly mean the end of NATO.
	\item Trade and product discounts can also mean big savings.
	\item Failure to act now will mean that our society will change beyond recognition.
	\end{itemize}
}
\item verb \\
If doing one thing \textbf{means} doing another, it involves doing the second thing.
 \textit{
	\begin{itemize}
	\item Children prefer to live in peace, even if that means living with only one parent.
	\item Managing well means communicating well.
	\end{itemize}
}
\item verb \\
If you say that you \textbf{mean} what you are saying , you are telling someone that you are serious about it and are not joking , exaggerating , or just being polite .
 \textit{
	\begin{itemize}
	\item He says you're fired if you're not back at work on Friday. And I think he meant it.
	\item He could see I meant what I said. So he took his fur coat and left.
	\end{itemize}
}
\item verb \\
If you say that someone \textbf{meant}  \textbf{to} do something, you are saying that they did it deliberately.
 \textit{
	\begin{itemize}
	\item I didn't mean to hurt you.
	\item If that sounds harsh, it is meant to.
	\item Did you mean to leave your dog here?
	\item I can see why you believed my letters were threatening but I never meant them to
be.
	\end{itemize}
}
\item verb \\
If you say that someone \textbf{did not}  \textbf{mean}  \textbf{any}  harm , offence , or disrespect , you are saying that they did not intend to upset or offend people or to cause problems, even though they may in fact have done so.
 \textit{
	\begin{itemize}
	\item I'm sure he didn't mean any harm.
	\item I didn't mean any offence. It was a flippant, off-the-cuff remark.
	\end{itemize}
}
\item verb \\
If you \textbf{mean}  \textbf{to} do something, you intend or plan to do it.
 \textit{
	\begin{itemize}
	\item Summer is the perfect time to catch up on the new books you meant to read.
	\item You know very well what I meant to say.
	\item I mean to look after my body.
	\end{itemize}
}
\item verb \\
If you say that something \textbf{was meant}  \textbf{to}  happen , you believe that it was made to happen by God or fate , and did not just happen by chance .
 \textit{
	\begin{itemize}
	\item John was constantly reassuring me that we were meant to be together.
	\end{itemize}
}
\item  \\
 I mean \textit{
	\begin{itemize}
	\end{itemize}
}
\item  \\
 I mean \textit{
	\begin{itemize}
	\end{itemize}
}
\item  \\
 I mean \textit{
	\begin{itemize}
	\end{itemize}
}
\item  \\
 know what it means to do sth/know what sth means \textit{
	\begin{itemize}
	\end{itemize}
}
\item  \\
 mean something to sb \textit{
	\begin{itemize}
	\end{itemize}
}
\item  \\
 to mean well \textit{
	\begin{itemize}
	\end{itemize}
}
\item  \\
 you mean \textit{
	\begin{itemize}
	\end{itemize}
}
\end{enumerate}

\section*{point}
{\large \color{blue}  points  pointing  pointed  }
\subsection*{Explain}
\begin{enumerate}
\item countable noun \\
You use \textbf{point} to refer to something that someone has said or written.
 \textit{
	\begin{itemize}
	\item We disagree with every point the Minister makes.
	\item This article makes the right point about the report.
	\item The following tale will clearly illustrate this point.
	\end{itemize}
}
\item singular noun \\
If you say that someone \textbf{has a point} , or if you \textbf{take} their \textbf{point} , you mean that you accept that what they have said is important and should be considered.
 \textit{
	\begin{itemize}
	\item 'If he'd already killed once, surely he'd have killed Sarah?' She had a point there.
	\item Oh I take your point, John, about that.
	\end{itemize}
}
\item singular noun \\
\textbf{The point} of what you are saying or discussing is the most important part that provides a reason or explanation for the rest.
 \textit{
	\begin{itemize}
	\item 'Did I ask you to talk to me?'—'That's not the point.'
	\item The American Congress and media mostly missed the point about all this.
	\end{itemize}
}
\item singular noun \\
If you ask what \textbf{the}  \textbf{point}  \textbf{of} something is, or say that there is \textbf{no point}  \textbf{in} it, you are indicating that a particular action has no purpose or would not be useful .
 \textit{
	\begin{itemize}
	\item What was the point of thinking about him?
	\item There was no point in staying any longer.
	\end{itemize}
}
\item countable noun \\
A \textbf{point} is a detail, aspect, or quality of something or someone.
 \textit{
	\begin{itemize}
	\item Many of the points in the report are correct.
	\item The most interesting point about the village was its religion.
	\item Science was never my strong point at school.
	\end{itemize}
}
\item countable noun \\
A \textbf{point} is a particular place or position where something happens.
 \textit{
	\begin{itemize}
	\item As a mark of respect the emperor met him at a point several weeks' march from the
capital.
	\item The pain originated from a point in his right thigh.
	\end{itemize}
}
\item singular noun \\
You use \textbf{point} to refer to a particular time, or to a particular stage in the development of something.
 \textit{
	\begin{itemize}
	\item We're all going to die at some point.
	\item At one point, around 70,000 members had failed to pay.
	\item At this point Diana arrived.
	\item It got to the point where he had to leave.
	\end{itemize}
}
\item countable noun \\
The \textbf{point} of something such as a pin, needle, or knife is the thin, sharp end of it.
 \textit{
	\begin{itemize}
	\end{itemize}
}
\item  \\
In spoken English, you use \textbf{point} to refer to the dot or mark in a decimal number that separates the whole numbers
from the fractions .
 \textit{
	\begin{itemize}
	\item Inflation at nine point four percent is the worst for eight years.
	\end{itemize}
}
\item countable noun \\
In some sports, competitions, and games, a \textbf{point} is one of the single marks that are added together to give the total score.
 \textit{
	\begin{itemize}
	\item They lost the 1977 World Cup final to Australia by a single point.
	\item Chamberlain scored 50 or more points four times in the season.
	\end{itemize}
}
\item countable noun \\
The \textbf{points of the compass} are directions such as North, South, East, and West.
 \textit{
	\begin{itemize}
	\item Sightseers arrived from all points of the compass.
	\end{itemize}
}
\item plural noun \\
On a railway track, the \textbf{points} are the levers and rails at a place where two tracks join or separate. The points enable a train
to move from one track to another.
 \textit{
	\begin{itemize}
	\item ...the rattle of the wheels across the points.
	\end{itemize}
}
\item countable noun \\
A \textbf{point} is an electric socket .
 \textit{
	\begin{itemize}
	\item ...too far away from the nearest electrical point.
	\end{itemize}
}
\item verb \\
If you \textbf{point at} a person or thing, you hold out your finger towards them in order to make someone
 notice them.
 \textit{
	\begin{itemize}
	\item I pointed at the boy sitting nearest me.
	\item He pointed at me with the stem of his pipe.
	\item He pointed to a chair, signalling for her to sit.
	\end{itemize}
}
\item verb \\
If you \textbf{point} something \textbf{at} someone, you aim the tip or end of it towards them.
 \textit{
	\begin{itemize}
	\item David Khan pointed his finger at Mary.
	\item A man pointed a gun at them and pulled the trigger.
	\end{itemize}
}
\item verb \\
If something \textbf{points}  \textbf{to} a place or \textbf{points} in a particular direction, it shows where that place is or it faces in that direction.
 \textit{
	\begin{itemize}
	\item An arrow pointed to the toilets.
	\item You can go anywhere and still the compass points north or south.
	\item He controlled the car until it was pointing forwards again.
	\end{itemize}
}
\item verb \\
If something \textbf{points to} a particular situation, it suggests that the situation exists or is likely to occur.
 \textit{
	\begin{itemize}
	\item Earlier reports pointed to pupils working harder, more continuously, and with enthusiasm.
	\item Private polls and embassy reports pointed to a no vote.
	\end{itemize}
}
\item verb \\
If you \textbf{point to} something that has happened or that is happening , you are using it as proof that a particular situation exists.
 \textit{
	\begin{itemize}
	\item George Fodor points to other weaknesses in the way the campaign has progressed.
	\item Gooch last night pointed to their bowling as the key to World Cup success.
	\end{itemize}
}
\item verb \\
When builders  \textbf{point} a wall, they put a substance such as cement into the gaps between the bricks or stones in order to make the wall stronger and seal it.
 \textit{
	\begin{itemize}
	\end{itemize}
}
\item  \\
 beside the point \textit{
	\begin{itemize}
	\end{itemize}
}
\item  \\
 come/get to the point \textit{
	\begin{itemize}
	\end{itemize}
}
\item  \\
 make/prove one's point \textit{
	\begin{itemize}
	\end{itemize}
}
\item  \\
 make a point of \textit{
	\begin{itemize}
	\end{itemize}
}
\item  \\
 on the point of \textit{
	\begin{itemize}
	\end{itemize}
}
\item  \\
 to the point \textit{
	\begin{itemize}
	\end{itemize}
}
\item  \\
 up to a point \textit{
	\begin{itemize}
	\end{itemize}
}
\end{enumerate}

\section*{obstruct}
{\large \color{blue}  obstructs  obstructing  obstructed  }
\subsection*{Explain}
\begin{enumerate}
\item verb \\
If something \textbf{obstructs} a road or path , it blocks it, stopping people or vehicles getting  past .
 \textit{
	\begin{itemize}
	\item Tractors and container lorries have completely obstructed the road.
	\end{itemize}
}
\item verb \\
To \textbf{obstruct} someone or something means to make it difficult for them to move forward by blocking their path.
 \textit{
	\begin{itemize}
	\item A number of local people have been arrested for trying to obstruct lorries loaded
with logs.
	\item They were fined for obstructing traffic.
	\end{itemize}
}
\item verb \\
To \textbf{obstruct} progress or a process means to prevent it from happening properly.
 \textit{
	\begin{itemize}
	\item The authorities are obstructing a United Nations investigation.
	\end{itemize}
}
\item verb \\
If someone or something \textbf{obstructs} your view, they are positioned between you and the thing you are trying to look at, stopping you from seeing it properly.
 \textit{
	\begin{itemize}
	\item Claire positioned herself so as not to obstruct David's line of sight.
	\end{itemize}
}
\end{enumerate}

\section*{pressure}
{\large \color{blue}  pressures  pressuring  pressured  }
\subsection*{Explain}
\begin{enumerate}
\item uncountable noun \\
\textbf{Pressure} is force that you produce when you press hard on something.
 \textit{
	\begin{itemize}
	\item She kicked at the door with her foot, and the pressure was enough to open it.
	\item The pressure of his fingers had relaxed.
	\item The best way to treat such bleeding is to apply firm pressure.
	\end{itemize}
}
\item uncountable noun \\
The \textbf{pressure} in a place or container is the force produced by the quantity of gas or liquid in that place or container.
 \textit{
	\begin{itemize}
	\item The window in the cockpit had blown in and the pressure dropped dramatically.
	\item Warm air is being drawn in from a high pressure area.
	\end{itemize}
}
\item uncountable noun \\
If there is \textbf{pressure}  \textbf{on} a person, someone is trying to persuade or force them to do something.
 \textit{
	\begin{itemize}
	\item He may have put pressure on her to agree.
	\item Its government is under pressure from the European Commission.
	\item The political pressures to do something are pretty enormous.
	\end{itemize}
}
\item uncountable noun \\
If you are experiencing  \textbf{pressure} , you feel that you must do a lot of tasks or make a lot of decisions in very little time, or that people expect a lot from you.
 \textit{
	\begin{itemize}
	\item Can you work under pressure?
	\item Even if I had the talent to play tennis I couldn't stand the pressure.
	\item The pressures of modern life are great.
	\end{itemize}
}
\item verb \\
If you \textbf{pressure} someone \textbf{to} do something, you try forcefully to persuade them to do it.
 \textit{
	\begin{itemize}
	\item He will never pressure you to get married.
	\item The Government should not be pressured into making hasty decisions.
	\item Don't pressure me.
	\item His boss did not pressure him for results.
	\end{itemize}
}
\end{enumerate}

\section*{perish}
{\large \color{blue}  perishes  perishing  perished  }
\subsection*{Explain}
\begin{enumerate}
\item verb \\
If people or animals \textbf{perish} , they die as a result of very harsh conditions or as the result of an accident .
 \textit{
	\begin{itemize}
	\item Most of the butterflies perish in the first frosts of autumn.
	\item ...the ferry disaster in which 193 passengers perished.
	\end{itemize}
}
\item verb \\
If something \textbf{perishes} , it comes to an end or is destroyed for ever .
 \textit{
	\begin{itemize}
	\item Civilizations do eventually decline and perish.
	\end{itemize}
}
\item verb \\
If a substance or material \textbf{perishes} , it starts to fall to pieces and becomes useless .
 \textit{
	\begin{itemize}
	\item Obviously the plaster's just perished and all fallen off.
	\item Their tyres are slowly perishing.
	\end{itemize}
}
\item  \\
 perish the thought \textit{
	\begin{itemize}
	\end{itemize}
}
\end{enumerate}

\section*{quart}
{\large \color{blue}  quarts  }
\subsection*{Explain}
\begin{enumerate}
\item countable noun \\
A \textbf{quart} is a unit of volume that is equal to two pints.
 \textit{
	\begin{itemize}
	\item Pick up a quart of milk or a loaf of bread.
	\end{itemize}
}
\end{enumerate}

\section*{play}
{\large \color{blue}  plays  playing  played  }
\subsection*{Explain}
\begin{enumerate}
\item verb \\
When children, animals, or perhaps  adults  \textbf{play} , they spend time doing enjoyable things, such as using toys and taking part in games.
 \textbf{Play} is also a noun.
 \textit{
	\begin{itemize}
	\item ...invite the children round to play.
	\item They played in the little garden.
	\item Polly was playing with her teddy bear.
	\item ...a few hours of play until the baby-sitter takes them off to bed.
	\end{itemize}
}
\item verb \\
When you \textbf{play} a sport, game, or match, you take part in it.
 \textbf{Play} is also a noun.
 \textit{
	\begin{itemize}
	\item All they want to do is sit around playing computer games.
	\item Alain was playing cards with his friends.
	\item I used to play basketball.
	\item I want to play for my country.
	\item He captained the team but he didn't actually play.
	\item Both sides adopted the Continental style of play.
	\end{itemize}
}
\item verb \\
When one person or team \textbf{plays} another or \textbf{plays against} them, they compete against them in a sport or game.
 \textbf{Play} is also a noun.
 \textit{
	\begin{itemize}
	\item Northern Ireland will play Latvia.
	\item I've played against him a few times.
	\item Fischer won after 5 hours and 41 minutes of play.
	\end{itemize}
}
\item verb \\
When you \textbf{play} the ball or \textbf{play} a shot in a game or sport, you kick or hit the ball.
 \textit{
	\begin{itemize}
	\item Think first before playing the ball.
	\item He considered how to play a shot from the rough on his final hole.
	\item I played the ball back slightly.
	\end{itemize}
}
\item verb \\
If you \textbf{play} a joke or a trick  \textbf{on} someone, you deceive them or give them a surprise in a way that you think is funny , but that often causes problems for them or annoys them.
 \textit{
	\begin{itemize}
	\item Someone had played a trick on her, stretched a piece of string at the top of those
steps.
	\item I thought: 'This cannot be happening, somebody must be playing a joke'.
	\end{itemize}
}
\item verb \\
If you \textbf{play with} an object or with your hair, you keep moving it or touching it with your fingers , perhaps because you are bored or nervous .
 \textit{
	\begin{itemize}
	\item She stared at the floor, idly playing with the strap of her handbag.
	\end{itemize}
}
\item countable noun \\
A \textbf{play} is a piece of writing which is performed in a theatre, on the radio, or on television.
 \textit{
	\begin{itemize}
	\item The company put on a play about the homeless.
	\item It's my favourite Shakespeare play.
	\end{itemize}
}
\item verb \\
If an actor \textbf{plays} a role or character in a play or film, he or she performs the part of that character.
 \textit{
	\begin{itemize}
	\item ...Dr Jekyll and Mr Hyde, in which he played Hyde.
	\item His ambition is to play the part of Dracula.
	\end{itemize}
}
\item link verb \\
You can use \textbf{play} to describe how someone behaves, when they are deliberately behaving in a certain
way or like a certain type of person. For example, to \textbf{play the innocent} , means to pretend to be innocent, and to \textbf{play deaf} means to pretend not to hear something.
 \textit{
	\begin{itemize}
	\item Hill tried to play the peacemaker.
	\item She was just playing the devoted mother.
	\item So you want to play nervous today?
	\end{itemize}
}
\item verb \\
You can describe how someone deals with a situation by saying that they \textbf{play it} in a certain way. For example, if someone \textbf{plays it cool} , they keep calm and do not show much emotion, and if someone \textbf{plays it straight} , they behave in an honest and direct way.
 \textit{
	\begin{itemize}
	\item Investors are playing it cautious, and they're playing it smart.
	\end{itemize}
}
\item verb \\
If you \textbf{play} a musical instrument or \textbf{play} a tune on a musical instrument, or if a musical instrument \textbf{plays} , music is produced from it.
 \textit{
	\begin{itemize}
	\item Nina had been playing the piano.
	\item Two people played jazz on a piano.
	\item He played for me.
	\item Place your baby in her seat and play her a lullaby.
	\item The guitars played.
	\end{itemize}
}
\item verb \\
If you \textbf{play} a record or a CD, you put it into a machine and sound is produced. If a record or
a CD \textbf{is playing} , sound is being produced from it.
 \textit{
	\begin{itemize}
	\item She played her records too loudly.
	\item ...v.
	\item There is classical music playing in the background.
	\end{itemize}
}
\item verb \\
If a musician or group of musicians \textbf{plays} or \textbf{plays} a concert , they perform music for people to listen or dance to.
 \textit{
	\begin{itemize}
	\item A band was playing.
	\item He will play concerts in Amsterdam and Paris.
	\end{itemize}
}
\item verb \\
When light \textbf{plays}  somewhere , it moves about on a surface in an unsteady way.
 \textit{
	\begin{itemize}
	\item The sun played on the frosty roofs.
	\end{itemize}
}
\item  \\
 what are you playing at? \textit{
	\begin{itemize}
	\end{itemize}
}
\item  \\
 come into play/be brought into play \textit{
	\begin{itemize}
	\end{itemize}
}
\item  \\
 play a part/play a role \textit{
	\begin{itemize}
	\end{itemize}
}
\end{enumerate}

\section*{ray}
{\large \color{blue}  rays  }
\subsection*{Explain}
\begin{enumerate}
\item countable noun \\
\textbf{Rays} of light are narrow beams of light.
 \textit{
	\begin{itemize}
	\item ...the first rays of light spread over the horizon.
	\item It can be seen clearly in a ray of sunlight or under a lamp.
	\item The sun's rays can penetrate water up to 10 feet.
	\end{itemize}
}
\item countable noun \\
A \textbf{ray of} hope, comfort , or other positive quality is a small amount of it that you welcome because it makes a bad situation seem less bad.
 \textit{
	\begin{itemize}
	\item They could provide a ray of hope amid the general business and economic gloom.
	\item The one ray of sunlight in this depressing history is her meeting and falling in
love with Martin.
	\end{itemize}
}
\item countable noun \\
A \textbf{ray} is a fairly large sea fish which has a flat body, eyes on the top of its body, and a long tail.
 \textit{
	\begin{itemize}
	\end{itemize}
}
\end{enumerate}

\section*{prolong}
{\large \color{blue}  prolongs  prolonging  prolonged  }
\subsection*{Explain}
\begin{enumerate}
\item verb \\
To \textbf{prolong} something means to make it last longer.
 \textit{
	\begin{itemize}
	\item Mr Chesler said foreign military aid was prolonging the war.
	\item The actual action of the drug can be prolonged significantly.
	\end{itemize}
}
\end{enumerate}

\section*{repression}
{\large \color{blue}  repressions  }
\subsection*{Explain}
\begin{enumerate}
\item uncountable noun \\
\textbf{Repression} is the use of force to restrict and control a society or other group of people.
 \textit{
	\begin{itemize}
	\item ...a society conditioned by violence and repression.
	\item ...the repressions of the 1930s.
	\end{itemize}
}
\item uncountable noun \\
\textbf{Repression} of feelings , especially  sexual ones, is a person's unwillingness to allow themselves to have natural feelings and desires .
 \textit{
	\begin{itemize}
	\item ...the repression of his feelings about men.
	\end{itemize}
}
\end{enumerate}

\section*{pull}
{\large \color{blue}  pulls  pulling  pulled  }
\subsection*{Explain}
\begin{enumerate}
\item verb \\
When you \textbf{pull} something, you hold it firmly and use force in order to move it towards you or away
from its previous position.
 \textbf{Pull} is also a noun .
 \textit{
	\begin{itemize}
	\item They have pulled out patients' teeth unnecessarily.
	\item He pulled on a jersey.
	\item Erica was solemn, pulling at her blonde curls.
	\item I helped pull him out of the water.
	\item Someone pulled her hair.
	\item He knew he should pull the trigger, but he was suddenly paralysed by fear.
	\item Pull as hard as you can.
	\item I let myself out into the street and pulled the door shut.
	\item The feather must be removed with a straight, firm pull.
	\end{itemize}
}
\item verb \\
When you \textbf{pull} an object from a bag , pocket , or cupboard , you put your hand in and bring the object out.
 \textit{
	\begin{itemize}
	\item Jack pulled the slip of paper from his shirt pocket.
	\item Katie reached into her shopping bag and pulled out a loaf of bread.
	\end{itemize}
}
\item verb \\
When a vehicle, animal, or person \textbf{pulls} a cart or piece of machinery , they are attached to it or hold it, so that it moves along behind them when they
move forward.
 \textit{
	\begin{itemize}
	\item This is early-20th-century rural Sussex, when horses still pulled the plough.
	\item He pulls a rickshaw, probably the oldest form of human taxi service.
	\end{itemize}
}
\item verb \\
If you \textbf{pull yourself} or \textbf{pull} a part of your body in a particular direction, you move your body or a part of your
body with effort or force.
 \textit{
	\begin{itemize}
	\item Hughes pulled himself slowly to his feet.
	\item He pulled his arms out of the sleeves.
	\item She tried to pull her hand free.
	\item Lillian brushed his cheek with her fingertips. He pulled away and said, 'Don't!'
	\end{itemize}
}
\item verb \\
When a driver or vehicle \textbf{pulls to} a stop or a halt , the vehicle stops.
 \textit{
	\begin{itemize}
	\item He pulled to a stop behind a pickup truck.
	\item The train pulled to a halt at the platform.
	\end{itemize}
}
\item verb \\
In a race or contest , if you \textbf{pull ahead of} or \textbf{pull away from} an opponent , you gradually increase the amount by which you are ahead of them.
 \textit{
	\begin{itemize}
	\item He pulled away, extending his lead to 15 seconds.
	\item The six states he won in 1988 are the same states in which he has yet to pull ahead
of his opponent.
	\end{itemize}
}
\item verb \\
If you \textbf{pull} something \textbf{apart} , you break or divide it into small pieces, often in order to put them back together
again in a different way.
 \textit{
	\begin{itemize}
	\item If I wanted to improve the car significantly I would have to pull it apart and start
again.
	\end{itemize}
}
\item verb \\
If someone \textbf{pulls} a gun or a knife  \textbf{on} someone else, they take out a gun or knife and threaten the other person with it.
 \textit{
	\begin{itemize}
	\item They had a fight. One of them pulled a gun on the other.
	\item I pulled a knife and threatened her.
	\end{itemize}
}
\item verb \\
To \textbf{pull}  crowds , viewers , or voters means to attract them.
 \textbf{Pull in} means the same as pull .
 \textit{
	\begin{itemize}
	\item The organisers have to employ performers to pull a crowd.
	\item They provided a far better news service and pulled in many more viewers.
	\item The musical is popular with theatre-goers, continuing to pull the crowds in 10 years
after its debut.
	\end{itemize}
}
\item verb \\
If something \textbf{pulls} you or \textbf{pulls} your thoughts or feelings in a particular direction, it strongly attracts you or
influences you in a particular way.
 \textbf{Pull} is also a noun.
 \textit{
	\begin{itemize}
	\item He felt there was little he could do to help his friend, and his heart was pulling
him elsewhere.
	\item No matter how much you feel the pull of the past, make a determined effort to look
to the future.
	\end{itemize}
}
\item countable noun \\
A \textbf{pull} is a strong physical force which causes things to move in a particular direction.
 \textit{
	\begin{itemize}
	\item ...the pull of gravity.
	\end{itemize}
}
\item verb \\
If you \textbf{are pulling for} someone, you support and encourage them, especially in a competition .
 \textit{
	\begin{itemize}
	\item We're all pulling for each other because we're desperate to win the Cup back.
	\item You know I've been pulling for you.
	\end{itemize}
}
\item verb \\
If you \textbf{pull} a muscle, you injure it by straining it.
 \textit{
	\begin{itemize}
	\item Dave pulled a back muscle and could barely kick the ball.
	\item He suffered a pulled calf muscle.
	\end{itemize}
}
\item verb \\
If someone \textbf{pulls on} a cigarette , they take a deep breath with the cigarette in their mouth.
 \textbf{Pull} is also a noun.
 \textit{
	\begin{itemize}
	\item Jeff leaned back and pulled on his cigarette.
	\item He took a deep pull and exhaled the smoke.
	\end{itemize}
}
\item verb \\
To \textbf{pull} a stunt or a trick  \textbf{on} someone means to do something dramatic or silly in order to get their attention or trick them.
 \textit{
	\begin{itemize}
	\item Everyone saw the stunt you pulled on me.
	\end{itemize}
}
\item verb \\
If someone \textbf{pulls} someone else, they succeed in attracting them sexually and in spending the rest of the evening or night with them.
 \textit{
	\begin{itemize}
	\end{itemize}
}
\item  \\
 pull the other one (it's got bells on) \textit{
	\begin{itemize}
	\end{itemize}
}
\end{enumerate}

\section*{shutter}
{\large \color{blue}  shutters  }
\subsection*{Explain}
\begin{enumerate}
\item countable noun \\
The \textbf{shutter} in a camera is the part which opens to allow light through the lens when a photograph is taken.
 \textit{
	\begin{itemize}
	\item There are a few things you should check before pressing the shutter release.
	\item ...a slow shutter speed.
	\end{itemize}
}
\item countable noun \\
\textbf{Shutters} are wooden or metal covers fitted on the outside of a window. They can be opened to let in the light, or closed to keep out the sun or the cold .
 \textit{
	\begin{itemize}
	\item She opened the shutters and gazed out over village roofs.
	\end{itemize}
}
\end{enumerate}

\section*{reach}
{\large \color{blue}  reaches  reaching  reached  }
\subsection*{Explain}
\begin{enumerate}
\item verb \\
A leak was found when the train reached Ipswich .
 \textit{
	\begin{itemize}
	\item He did not stop until he reached the door.
	\item A leak was found when the train reached Ipswich.
	\item He reached Cambridge shortly before three o'clock.
	\end{itemize}
}
\item verb \\
If someone or something has \textbf{reached} a certain stage, level, or amount, they are at that stage, level, or amount.
 \textit{
	\begin{itemize}
	\item The process of political change has reached the stage where it is irreversible.
	\item He reached the final of the US Championships.
	\item We're told the figure could reach 100,000 next year.
	\end{itemize}
}
\item verb \\
If you \textbf{reach}  somewhere , you move your arm and hand to take or touch something.
 \textit{
	\begin{itemize}
	\item Judy reached into her handbag and handed me a small printed leaflet.
	\item I reached across the table and squeezed his hand.
	\item He reached up for an overhanging branch.
	\end{itemize}
}
\item verb \\
If you can \textbf{reach} something, you are able to touch it by stretching out your arm or leg .
 \textit{
	\begin{itemize}
	\item Can you reach your toes with your fingertips?
	\end{itemize}
}
\item verb \\
If you try to \textbf{reach} someone, you try to contact them, usually by phone .
 \textit{
	\begin{itemize}
	\item Has the doctor told you how to reach him or her in emergencies?
	\item If I see her, I'll tell her you've been trying to reach her.
	\end{itemize}
}
\item verb \\
If something \textbf{reaches} a place, point, or level, it extends as far as that place, point, or level.
 \textit{
	\begin{itemize}
	\item ...a nightshirt which reached to his knees.
	\item The water level in Lake Taihu has reached record levels.
	\item Eventually those ideas should reach the capital city.
	\end{itemize}
}
\item verb \\
When people \textbf{reach} an agreement or a decision , they succeed in achieving it.
 \textit{
	\begin{itemize}
	\item A meeting of agriculture ministers in Luxembourg today has so far failed to reach
agreement over farm subsidies.
	\item They are meeting in Lusaka in an attempt to reach a compromise.
	\end{itemize}
}
\item uncountable noun \\
Someone's or something's \textbf{reach} is the distance or limit to which they can stretch, extend, or travel .
 \textit{
	\begin{itemize}
	\item Isabelle placed a cup on the table within his reach.
	\item ...a heavyweight who possesses a longer reach and more strength.
	\item ...long-handled shears, secateurs and long-reach tree pruners.
	\end{itemize}
}
\item uncountable noun \\
If a place or thing is within \textbf{reach} , it is possible to have it or get to it. If it is out of \textbf{reach} , it is not possible to have it or get to it.
 \textit{
	\begin{itemize}
	\item It is located within reach of many important Norman towns, including Bayeux.
	\item The clothes they model for this mail-order catalogue are all within easy reach of
every woman.
	\item These products are normally bought and stored carefully out of reach of children.
	\item The price is ten times what it normally is and totally beyond the reach of ordinary
people.
	\end{itemize}
}
\end{enumerate}

\section*{statute}
{\large \color{blue}  statutes  }
\subsection*{Explain}
\begin{enumerate}
\item variable noun \\
A \textbf{statute} is a rule or law which has been made by a government or other organization and formally
written down.
 \textit{
	\begin{itemize}
	\item The new statute covers the care for, bringing up and protection of children.
	\item The independence of the judiciary in France is guaranteed by statute.
	\end{itemize}
}
\end{enumerate}

\section*{resist}
{\large \color{blue}  resists  resisting  resisted  }
\subsection*{Explain}
\begin{enumerate}
\item verb \\
If you \textbf{resist} something such as a change, you refuse to accept it and try to prevent it.
 \textit{
	\begin{itemize}
	\item The Chancellor warned employers to resist demands for high pay increases.
	\item They resisted our attempts to modernize the distribution of books.
	\end{itemize}
}
\item verb \\
If you \textbf{resist} someone or \textbf{resist} an attack by them, you fight back against them.
 \textit{
	\begin{itemize}
	\item The man was shot outside his house as he tried to resist arrest.
	\item When she had attempted to cut his nails he resisted.
	\end{itemize}
}
\item verb \\
If you \textbf{resist} doing something, or \textbf{resist} the temptation to do it, you stop yourself from doing it although you would like to do it.
 \textit{
	\begin{itemize}
	\item Students should resist the temptation to focus on exams alone.
	\item She cannot resist giving him advice.
	\end{itemize}
}
\item verb \\
If someone or something \textbf{resists}  damage of some kind , they are not damaged.
 \textit{
	\begin{itemize}
	\item ...bodies trained and toughened to resist the cold.
	\item Chemicals form a protective layer that resists both oil and water-based stains.
	\end{itemize}
}
\end{enumerate}

\section*{run}
{\large \color{blue}  runs  running  ran  }
\subsection*{Explain}
\begin{enumerate}
\item verb \\
When you \textbf{run} , you move more quickly than when you walk, for example because you are in a hurry to get somewhere , or for exercise.
 \textbf{Run} is also a noun.
 \textit{
	\begin{itemize}
	\item I excused myself and ran back to the phone.
	\item Police believe the gunmen ran off into the woods.
	\item Neighbouring shopkeepers ran after the man and caught him.
	\item He ran the last block to the White House with two cases of gear.
	\item Antonia ran to meet them.
	\item After a six-mile run, Jackie returns home for a substantial breakfast.
	\end{itemize}
}
\item verb \\
When someone \textbf{runs} in a race, they run in competition with other people.
 \textit{
	\begin{itemize}
	\item ...when I was running in the New York Marathon.
	\item The British sprinter ran a controlled race to qualify in 51.32 sec.
	\end{itemize}
}
\item verb \\
When a horse \textbf{runs} in a race or when its owner \textbf{runs} it, it competes in a race.
 \textit{
	\begin{itemize}
	\item The owner insisted on Cool Ground running in the Gold Cup.
	\item If we have a wet spell, Cecil could also run Armiger in the Derby.
	\end{itemize}
}
\item verb \\
If you say that something long, such as a road, \textbf{runs} in a particular direction, you are describing its course or position. You can also
say that something \textbf{runs} the length or width of something else.
 \textit{
	\begin{itemize}
	\item ...the sun-dappled trail which ran through the beech woods.
	\item ...a gas-filled glass tube with a thin wire running down the centre.
	\item The hallway ran the length of the villa.
	\end{itemize}
}
\item verb \\
If you \textbf{run} a wire or tube somewhere, you fix it or pull it from, to, or across a particular
place.
 \textit{
	\begin{itemize}
	\item Our host ran a long extension cord out from the house and set up a screen and a projector.
	\end{itemize}
}
\item verb \\
If you \textbf{run} your hand or an object \textbf{through} something, you move your hand or the object through it.
 \textit{
	\begin{itemize}
	\item He laughed and ran his fingers through his hair.
	\item I ran the brush through my hair and dashed out.
	\item Fumbling, he ran her card through the machine.
	\item It hurt to breathe, and he winced as he ran his hand over his ribs.
	\end{itemize}
}
\item verb \\
If you \textbf{run} something through a machine, process, or series of tests, you make it go through
the machine, process, or tests.
 \textit{
	\begin{itemize}
	\item They have gathered the best statistics they can find and run them through their own
computers.
	\end{itemize}
}
\item verb \\
If someone \textbf{runs}  \textbf{for} office in an election, they take part as a candidate.
 \textit{
	\begin{itemize}
	\item It was only last February that he announced he would run for president.
	\item In 1864, McClellan ran against Lincoln as the Democratic candidate for president.
	\item Women are running in nearly all the contested seats in Los Angeles.
	\end{itemize}
}
\item singular noun \\
A \textbf{run for} office is an attempt to be elected to office.
 \textit{
	\begin{itemize}
	\item He was already preparing his run for the presidency.
	\end{itemize}
}
\item verb \\
If you \textbf{run} something such as a business or an activity, you are in charge of it or you organize
it.
 \textit{
	\begin{itemize}
	\item His stepfather ran a prosperous paint business.
	\item Is this any way to run a country?
	\item Each teacher will run a different workshop that covers a specific area of the language.
	\item ...a well-run, profitable organisation.
	\end{itemize}
}
\item verb \\
If you talk about how a system, an organization, or someone's life \textbf{is running} , you are saying how well it is operating or progressing.
 \textit{
	\begin{itemize}
	\item Officials in charge of the camps say the system is now running extremely smoothly.
	\item ...the staff who have kept the bank running.
	\end{itemize}
}
\item verb \\
If you \textbf{run} an experiment, computer program, or other process, or start it \textbf{running} , you start it and let it continue.
 \textit{
	\begin{itemize}
	\item He ran a lot of tests and it turned out I had an infection called mycoplasma.
	\item You can check your program one command at a time while it's running.
	\end{itemize}
}
\item verb \\
When you \textbf{run} a cassette or video tape or when it \textbf{runs} , it moves through the machine as the machine operates.
 \textit{
	\begin{itemize}
	\item He pushed the play button again and ran the tape.
	\item The tape had run to the end but recorded nothing.
	\end{itemize}
}
\item verb \\
When a machine \textbf{is running} or when you \textbf{are running} it, it is switched on and is working.
 \textit{
	\begin{itemize}
	\item He had failed to realise that the camera was still running.
	\item We told him to wait out front with the engine running.
	\item ...with everybody running their appliances all at the same time.
	\end{itemize}
}
\item verb \\
A machine or equipment that \textbf{runs}  \textbf{on} or \textbf{off} a particular source of energy functions using that source of energy.
 \textit{
	\begin{itemize}
	\item Black cabs run on diesel.
	\item Rows of stalls are given over to restaurants running off gas cylinders.
	\end{itemize}
}
\item verb \\
If you \textbf{run} a car or a piece of equipment, you have it and use it.
 \textit{
	\begin{itemize}
	\item I ran a 1960 Rover 100 from 1977 until 1983.
	\end{itemize}
}
\item verb \\
When you say that vehicles such as trains and buses \textbf{run} from one place to another, you mean they regularly travel along that route.
 \textit{
	\begin{itemize}
	\item A shuttle bus runs frequently between the Inn and the Country Club.
	\item ...a government which can't make the trains run on time.
	\end{itemize}
}
\item verb \\
If you \textbf{run} someone somewhere in a car, you drive them there.
 \textit{
	\begin{itemize}
	\item Could you run me up to Baltimore?
	\end{itemize}
}
\item verb \\
If you \textbf{run} over or down to a place that is quite near, you drive there.
 \textit{
	\begin{itemize}
	\item I'll run over to Short Mountain and check on Mrs Adams.
	\end{itemize}
}
\item countable noun \\
A \textbf{run} is a journey somewhere.
 \textit{
	\begin{itemize}
	\item A run to Southampton showed the car was capable of a reasonable journey.
	\item ...doing the morning school run.
	\item ...after their bombing runs against ground troops.
	\end{itemize}
}
\item verb \\
If a liquid \textbf{runs} in a particular direction, it flows in that direction.
 \textit{
	\begin{itemize}
	\item Tears were running down her cheeks.
	\item There were cisterns to catch rainwater as it ran off the castle walls.
	\item Wash the rice in cold water until the water runs clear.
	\end{itemize}
}
\item verb \\
If you \textbf{run} water, or if you \textbf{run} a tap or a bath , you cause water to flow from a tap.
 \textit{
	\begin{itemize}
	\item She went to the sink and ran water into her empty glass.
	\item They heard him running the kitchen tap.
	\item I threw off my clothing quickly and ran a warm bath.
	\end{itemize}
}
\item verb \\
If a tap or a bath \textbf{is running} , water is coming out of a tap.
 \textit{
	\begin{itemize}
	\item You must have left a tap running in the bathroom.
	\item He came fully awake to hear the bath running.
	\end{itemize}
}
\item verb \\
If your nose \textbf{is running} , liquid is flowing out of it, usually because you have a cold.
 \textit{
	\begin{itemize}
	\item Timothy was crying, mostly from exhaustion, and his nose was running.
	\end{itemize}
}
\item verb \\
If a surface \textbf{is running}  \textbf{with} a liquid, that liquid is flowing down it.
 \textit{
	\begin{itemize}
	\item After an hour he realised he was completely running with sweat.
	\item The window panes were running with condensation.
	\end{itemize}
}
\item verb \\
If the dye in some cloth or the ink on some paper \textbf{runs} , it comes off or spreads when the cloth or paper gets wet.
 \textit{
	\begin{itemize}
	\item The ink had run on the wet paper.
	\end{itemize}
}
\item verb \\
If a feeling \textbf{runs}  \textbf{through} your body or a thought \textbf{runs}  \textbf{through} your mind, you experience it or think it quickly.
 \textit{
	\begin{itemize}
	\item She felt a surge of excitement run through her.
	\item All sorts of thoughts were running through my head.
	\end{itemize}
}
\item verb \\
If a feeling or noise \textbf{runs}  \textbf{through} a group of people, it spreads among them.
 \textit{
	\begin{itemize}
	\item A buzz of excitement ran through the crowd.
	\end{itemize}
}
\item verb \\
If a theme or feature \textbf{runs}  \textbf{through} something such as someone's actions or writing, it is present in all of it.
 \textit{
	\begin{itemize}
	\item Another thread running through this series is the role of doctors.
	\item ...the theme running through the book.
	\item There was something of this mood running throughout the Party's deliberations.
	\end{itemize}
}
\item verb \\
When newspapers or magazines \textbf{run} a particular item or story or if it \textbf{runs} , it is published or printed.
 \textit{
	\begin{itemize}
	\item The newspaper ran a series of four editorials entitled 'The Choice of Our Lives.'
	\item ...an editorial that ran this weekend entitled 'Mr. Cuomo Backs Out.'
	\end{itemize}
}
\item verb \\
You can use \textbf{run} to indicate that you are quoting someone else's words or ideas.
 \textit{
	\begin{itemize}
	\item 'Whoa, I'm goin' to Barbay-dos!' ran the jaunty lyrics of a 1970s hit song.
	\end{itemize}
}
\item verb \\
If an amount \textbf{is running} at a particular level, it is at that level.
 \textit{
	\begin{itemize}
	\item Today's RPI figure shows inflation running at 10.9 per cent.
	\item The deficit is now running at about 300 million dollars a year.
	\end{itemize}
}
\item verb \\
If a play, event, or legal contract \textbf{runs} for a particular period of time, it lasts for that period of time.
 \textit{
	\begin{itemize}
	\item It pleased critics but ran for only three years in the West End.
	\item The contract was to run from 1992 to 2020.
	\item I predict it will run and run.
	\end{itemize}
}
\item verb \\
If someone or something \textbf{is running} late, they have taken more time than had been planned. If they \textbf{are running} to time or ahead of time, they have taken the time planned or less than the time planned.
 \textit{
	\begin{itemize}
	\item Tell her I'll call her back later, I'm running late again.
	\item The steward will tell you whether the event is running to time.
	\end{itemize}
}
\item verb \\
If you \textbf{are running} a temperature or a fever, you have a high temperature because you are ill.
 \textit{
	\begin{itemize}
	\item The little girl is running a fever and she needs help.
	\end{itemize}
}
\item countable noun \\
A \textbf{run} of a play or television programme is the period of time during which performances
are given or programmes are shown.
 \textit{
	\begin{itemize}
	\item The Globe begins a two-month run of the Bard of Avon's most famous and enduring love
story.
	\item This excellent BBC series begins a run on Artsworld with a look at Edvard Munch's
The Scream.
	\end{itemize}
}
\item singular noun \\
A \textbf{run}  \textbf{of} successes or failures is a series of successes or failures.
 \textit{
	\begin{itemize}
	\item The England skipper is haunted by a run of low scores.
	\item The Scottish Tories' run of luck is holding.
	\end{itemize}
}
\item countable noun \\
A \textbf{run} of a product is the amount that a company or factory decides to produce at one time.
 \textit{
	\begin{itemize}
	\item Wayne plans to increase the print run to 1,000.
	\item Their defense markets are too small to sustain economically viable production runs.
	\end{itemize}
}
\item countable noun \\
In cricket or baseball , a \textbf{run} is a score of one, which is made by players running between marked places on the
field after hitting the ball.
 \textit{
	\begin{itemize}
	\item At 20 he became the youngest player to score 2,000 runs in a season.
	\end{itemize}
}
\item singular noun \\
If someone gives you \textbf{the run of} a place, they give you permission to go where you like in it and use it as you wish.
 \textit{
	\begin{itemize}
	\item He had the run of the house and the pool.
	\end{itemize}
}
\item singular noun \\
If you say that someone or something is different from the average \textbf{run} or common \textbf{run}  \textbf{of} people or things, you mean that they are different from ordinary people or things.
 \textit{
	\begin{itemize}
	\item ...a man who was outside the common run of professional athletes at the time.
	\item ...trying to accomplish the usual run of maintenance jobs and write a column too.
	\end{itemize}
}
\item singular noun \\
If there is a \textbf{run on} something, a lot of people want to buy it or get it at the same time.
 \textit{
	\begin{itemize}
	\item A run on sterling has killed off hopes of a rate cut.
	\end{itemize}
}
\item countable noun \\
A ski  \textbf{run} or bobsleigh \textbf{run} is a course or route that has been designed for skiing or for riding in a bobsleigh.
 \textit{
	\begin{itemize}
	\end{itemize}
}
\item  \\
 against the run of sth \textit{
	\begin{itemize}
	\end{itemize}
}
\item  \\
 run someone close/run someone a close second/run a close second \textit{
	\begin{itemize}
	\end{itemize}
}
\item  \\
 run dry \textit{
	\begin{itemize}
	\end{itemize}
}
\item  \\
 run dry \textit{
	\begin{itemize}
	\end{itemize}
}
\item  \\
 run in sb's family \textit{
	\begin{itemize}
	\end{itemize}
}
\item  \\
 make a run for it/run for it \textit{
	\begin{itemize}
	\end{itemize}
}
\item  \\
 run high \textit{
	\begin{itemize}
	\end{itemize}
}
\item  \\
 in the long run \textit{
	\begin{itemize}
	\end{itemize}
}
\item  \\
 run a mile \textit{
	\begin{itemize}
	\end{itemize}
}
\item  \\
 to give someone a run for their money \textit{
	\begin{itemize}
	\end{itemize}
}
\item  \\
 on the run \textit{
	\begin{itemize}
	\end{itemize}
}
\item  \\
 on the run \textit{
	\begin{itemize}
	\end{itemize}
}
\item  \\
 be running scared \textit{
	\begin{itemize}
	\end{itemize}
}
\item  \\
 run short/run low \textit{
	\begin{itemize}
	\end{itemize}
}
\end{enumerate}

\section*{success}
{\large \color{blue}  successes  }
\subsection*{Explain}
\begin{enumerate}
\item uncountable noun \\
\textbf{Success} is the achievement of something that you have been trying to do.
 \textit{
	\begin{itemize}
	\item It's important for the long-term success of any diet that you vary your meals.
	\item ...the success of European business in building a stronger partnership between management
and workers.
	\end{itemize}
}
\item uncountable noun \\
\textbf{Success} is the achievement of a high position in a particular field , for example in business or politics .
 \textit{
	\begin{itemize}
	\item Nearly all of the young people interviewed believed that work was the key to success.
	\end{itemize}
}
\item uncountable noun \\
The \textbf{success} of something is the fact that it works in a satisfactory way or has the result that is intended .
 \textit{
	\begin{itemize}
	\item Most of the cast was amazed by the play's success.
	\item They were enthused by the success of the first exhibition.
	\end{itemize}
}
\item countable noun \\
Someone or something that is a \textbf{success}  achieves a high position, makes a lot of money , or is admired a great  deal .
 \textit{
	\begin{itemize}
	\item The jewellery was a great success.
	\item We hope it will be a commercial success.
	\end{itemize}
}
\end{enumerate}

\section*{scold}
{\large \color{blue}  scolds  scolding  scolded  }
\subsection*{Explain}
\begin{enumerate}
\item verb \\
If you \textbf{scold} someone, you speak angrily to them because they have done something wrong .
 \textit{
	\begin{itemize}
	\item If he finds out, he'll scold me.
	\item Later she scolded her daughter for having talked to her father like that.
	\item 'You should be at school,' he scolded.
	\end{itemize}
}
\end{enumerate}

\section*{tip}
{\large \color{blue}  tips  tipping  tipped  }
\subsection*{Explain}
\begin{enumerate}
\item countable noun \\
The \textbf{tip} of something long and narrow is the end of it.
 \textit{
	\begin{itemize}
	\item The sleeves covered his hands to the tips of his fingers.
	\item She poked and shifted things with the tip of her walking stick.
	\item The city was concentrated into the southern tip of the island.
	\end{itemize}
}
\item verb \\
If you \textbf{tip} an object or part of your body or if it \textbf{tips} , it moves into a sloping position with one end or side higher than the other.
 \textit{
	\begin{itemize}
	\item He leaned away from her, and she had to tip her head back to see him.
	\item A young boy is standing on a stool, reaching for a cookie jar, and the stool is about
to tip.
	\item The north pole is slightly tipped towards the sun.
	\end{itemize}
}
\item verb \\
If you \textbf{tip} something somewhere , you pour it there.
 \textit{
	\begin{itemize}
	\item Tip the vegetables into a bowl.
	\item She took out the plate, stared blankly at the dried-up food on it, and tipped it
into the bin.
	\item Tip away the salt and wipe the pan.
	\end{itemize}
}
\item verb \\
To \textbf{tip} rubbish means to get  rid of it by leaving it somewhere.
 \textit{
	\begin{itemize}
	\item ...the costs of tipping rubbish in landfills.
	\item How do you stop people tipping?
	\item We live in a street off Soho Road and there's rubbish tipped everywhere.
	\end{itemize}
}
\item countable noun \\
A \textbf{tip} is a place where rubbish is left.
 \textit{
	\begin{itemize}
	\item Officers had found a large bread knife on the rubbish tip.
	\item I took a load of rubbish and grass cuttings to the tip.
	\end{itemize}
}
\item countable noun \\
If you describe a place as \textbf{a}  \textbf{tip} , you mean it is very untidy .
 \textit{
	\begin{itemize}
	\item The flat is an absolute tip.
	\end{itemize}
}
\item verb \\
If you \textbf{tip} someone such as a waiter in a restaurant , you give them some money in order to thank them for their services.
 \textit{
	\begin{itemize}
	\item Do you really think it's customary to tip the waiters?
	\item She tipped the barmen 10 dollars and bought drinks all round.
	\end{itemize}
}
\item countable noun \\
If you give a \textbf{tip} to someone such as a waiter in a restaurant, you give them some money to thank them
for their services.
 \textit{
	\begin{itemize}
	\item I gave the barber a tip.
	\item The Head Porter was keeping all the tips.
	\end{itemize}
}
\item countable noun \\
A \textbf{tip} is a useful piece of advice .
 \textit{
	\begin{itemize}
	\item It shows how to prepare a CV, and gives tips on applying for jobs.
	\item ...tips for busy managers.
	\item A good tip is to buy the most expensive lens you can afford.
	\end{itemize}
}
\item verb \\
If a person \textbf{is tipped to} do something or \textbf{is tipped for}  success at something, experts or journalists  believe that they will do that thing or achieve that success.
 \textit{
	\begin{itemize}
	\item He is tipped to be the country's next foreign minister.
	\item He was widely tipped for success.
	\end{itemize}
}
\item countable noun \\
Someone's \textbf{tip} for a race or competition is their advice on its likely result, especially to someone who wants to bet money on the result.
 \textit{
	\begin{itemize}
	\item I've a tip for the races.
	\item United are still my tip for the Title.
	\end{itemize}
}
\item  \\
 the tip of the iceberg \textit{
	\begin{itemize}
	\end{itemize}
}
\item  \\
 tip the scales/balance \textit{
	\begin{itemize}
	\end{itemize}
}
\item  \\
 on the tip of your tongue \textit{
	\begin{itemize}
	\end{itemize}
}
\end{enumerate}

\section*{sell}
{\large \color{blue}  sells  selling  sold  }
\subsection*{Explain}
\begin{enumerate}
\item verb \\
If you \textbf{sell} something that you own, you let someone have it in return for money.
 \textit{
	\begin{itemize}
	\item I sold everything I owned except for my car and my books.
	\item His heir sold the painting to the London art dealer Agnews.
	\item The directors sold the business for £14.8 million.
	\item It's not a very good time to sell at the moment.
	\end{itemize}
}
\item verb \\
If a shop  \textbf{sells} a particular thing, it is available for people to buy there.
 \textit{
	\begin{itemize}
	\item It sells everything from hair ribbons to oriental rugs.
	\item Bean sprouts are also sold in cans.
	\end{itemize}
}
\item verb \\
If something \textbf{sells}  \textbf{for} a particular price, that price is paid for it.
 \textit{
	\begin{itemize}
	\item Unmodernised property can sell for up to 40 per cent of its modernised market value.
	\item ... grain sells at 10 times usual prices.
	\end{itemize}
}
\item verb \\
If something \textbf{sells} , it is bought by the public , usually in fairly large quantities .
 \textit{
	\begin{itemize}
	\item Even if this album doesn't sell and the critics don't like it, we wouldn't ever change.
	\item The company believes the products will sell well in the run-up to Christmas.
	\end{itemize}
}
\item verb \\
Something that \textbf{sells} a product makes people want to buy the product.
 \textit{
	\begin{itemize}
	\item It is only the sensational that sells news magazines.
	\item ...car manufacturers' long-held maxim that safety doesn't sell.
	\end{itemize}
}
\item verb \\
If you \textbf{sell} someone an idea or proposal , or \textbf{sell} someone \textbf{on} an idea, you convince them that it is a good one.
 \textit{
	\begin{itemize}
	\item She tried to sell me the idea of buying my own paper shredder.
	\item She is hoping she can sell the idea to clients.
	\item An employee sold him on the notion that cable was the medium of the future.
	\item You know, I wasn't sold on this trip in the beginning.
	\end{itemize}
}
\item  \\
 sell one's body \textit{
	\begin{itemize}
	\end{itemize}
}
\item  \\
 sell sb down the river \textit{
	\begin{itemize}
	\end{itemize}
}
\item  \\
 sell oneself short \textit{
	\begin{itemize}
	\end{itemize}
}
\item  \\
 sell one's soul \textit{
	\begin{itemize}
	\end{itemize}
}
\end{enumerate}

\section*{tomato}
{\large \color{blue}  tomatoes  }
\subsection*{Explain}
\begin{enumerate}
\item variable noun \\
\textbf{Tomatoes} are small, soft , red fruit that you can eat raw in salads or cooked as a vegetable.
 \textit{
	\begin{itemize}
	\end{itemize}
}
\end{enumerate}

\section*{signify}
{\large \color{blue}  signifies  signifying  signified  }
\subsection*{Explain}
\begin{enumerate}
\item verb \\
If an event, a sign, or a symbol \textbf{signifies} something, it is a sign of that thing or represents that thing.
 \textit{
	\begin{itemize}
	\item Fever accompanied by a runny nose usually signifies a cold.
	\item The symbol displayed outside a restaurant signifies there's excellent cuisine inside.
	\end{itemize}
}
\item verb \\
If you \textbf{signify} something, you make a sign or gesture in order to communicate a particular meaning .
 \textit{
	\begin{itemize}
	\item Two jurors signified their dissent.
	\item The U.N. flag was raised at the airport to signify that control had passed into its
hands.
	\end{itemize}
}
\end{enumerate}

\section*{train}
{\large \color{blue}  trains  }
\subsection*{Explain}
\begin{enumerate}
\item countable noun \\
A \textbf{train} is a number of carriages , cars, or trucks which are all connected together and which are pulled by an engine along a railway. Trains carry people and goods from one place to another.
 \textit{
	\begin{itemize}
	\item The train pulled into a station.
	\item We can catch the early morning train.
	\item He arrived in Shenyang by train yesterday.
	\end{itemize}
}
\item countable noun \\
A \textbf{train}  \textbf{of} vehicles, people, or animals is a long line of them travelling slowly in the same
direction.
 \textit{
	\begin{itemize}
	\item In the old days this used to be done with a baggage train of camels.
	\item ...a long train of oil tankers.
	\end{itemize}
}
\item countable noun \\
A \textbf{train}  \textbf{of} thought or a \textbf{train}  \textbf{of} events is a connected sequence, in which each thought or event seems to occur naturally or logically as a result of the previous one.
 \textit{
	\begin{itemize}
	\item He lost his train of thought for a moment, then recovered it.
	\item Giles set in motion a train of events which would culminate in tragedy.
	\end{itemize}
}
\item countable noun \\
The \textbf{train} of a woman's formal dress or wedding dress is the long part at the back of it which flows along the floor behind her.
 \textit{
	\begin{itemize}
	\end{itemize}
}
\item  \\
 in train \textit{
	\begin{itemize}
	\end{itemize}
}
\item  \\
 in its train \textit{
	\begin{itemize}
	\end{itemize}
}
\end{enumerate}

\section*{stimulate}
{\large \color{blue}  stimulates  stimulating  stimulated  }
\subsection*{Explain}
\begin{enumerate}
\item verb \\
To \textbf{stimulate} something means to encourage it to begin or develop further.
 \textit{
	\begin{itemize}
	\item America's priority is rightly to stimulate its economy.
	\item The Commonwealth Games have stimulated public interest in doing sport.
	\end{itemize}
}
\item verb \\
If you \textbf{are}  \textbf{stimulated}  \textbf{by} something, it makes you feel  full of ideas and enthusiasm.
 \textit{
	\begin{itemize}
	\item Bill was stimulated by the challenge.
	\item I was stimulated to examine my deepest thoughts.
	\end{itemize}
}
\item verb \\
If something \textbf{stimulates} a part of a person's body, it causes it to move or start working .
 \textit{
	\begin{itemize}
	\item Exercise stimulates the digestive and excretory systems.
	\item The production of melanin in the skin is stimulated by exposure to the sun.
	\item The body is stimulated to build up resistance.
	\end{itemize}
}
\end{enumerate}

\section*{tram}
{\large \color{blue}  trams  }
\subsection*{Explain}
\begin{enumerate}
\item countable noun \\
A \textbf{tram} is a public transport vehicle, usually powered by electricity from wires above it, which travels along rails laid in the surface of a street .
 \textit{
	\begin{itemize}
	\item You can get to the beach easily from the centre of town by tram.
	\end{itemize}
}
\end{enumerate}

\section*{strain}
{\large \color{blue}  strains  straining  strained  }
\subsection*{Explain}
\begin{enumerate}
\item variable noun \\
If \textbf{strain} is put \textbf{on} an organization or system, it has to do more than it is able to do.
 \textit{
	\begin{itemize}
	\item The prison service is already under considerable strain.
	\item The vast expansion in secondary education is putting an enormous strain on the system.
	\item ...the credit crunch caused by strains on the banking system.
	\end{itemize}
}
\item verb \\
To \textbf{strain} something means to make it do more than it is able to do.
 \textit{
	\begin{itemize}
	\item The volume of scheduled flights is straining the air traffic control system.
	\item Resources will be further strained by new demands for housing.
	\end{itemize}
}
\item uncountable noun \\
\textbf{Strain} is a state of worry and tension caused by a difficult situation.
 \textit{
	\begin{itemize}
	\item She was tired and under great strain.
	\item ...the stresses and strains of a busy and demanding career.
	\end{itemize}
}
\item singular noun \\
If you say that a situation is \textbf{a strain} , you mean that it makes you worried and tense .
 \textit{
	\begin{itemize}
	\item I sometimes find it a strain to be responsible for the mortgage.
	\end{itemize}
}
\item uncountable noun \\
\textbf{Strain} is a force that pushes, pulls, or stretches something in a way that may damage it.
 \textit{
	\begin{itemize}
	\item Place your hands under your buttocks to take some of the strain off your back.
	\item The large door already places plenty of strain on the hinges.
	\end{itemize}
}
\item variable noun \\
\textbf{Strain} is an injury to a muscle in your body, caused by using the muscle too much or twisting it.
 \textit{
	\begin{itemize}
	\item Avoid muscle strain by warming up with slow jogging.
	\item ...a groin strain.
	\end{itemize}
}
\item verb \\
If you \textbf{strain} a muscle, you injure it by using it too much or twisting it.
 \textit{
	\begin{itemize}
	\item He strained his back during a practice session.
	\end{itemize}
}
\item verb \\
If you \textbf{strain}  \textbf{to} do something, you make a great effort to do it when it is difficult to do.
 \textit{
	\begin{itemize}
	\item I had to strain to hear.
	\item Several thousand supporters strained to catch a glimpse of the new president.
	\item They strained their eyes, but saw nothing.
	\end{itemize}
}
\item verb \\
When you \textbf{strain} food, you separate the liquid part of it from the solid parts.
 \textit{
	\begin{itemize}
	\item Strain the stock and put it back into the pan.
	\end{itemize}
}
\item singular noun \\
You can use \textbf{strain} to refer to a particular quality in someone's character, remarks , or work.
 \textit{
	\begin{itemize}
	\item There was a strain of bitterness in his voice.
	\item ...this cynical strain in the book.
	\end{itemize}
}
\item countable noun \\
A \textbf{strain}  \textbf{of} a germ , plant, or other organism is a particular type of it.
 \textit{
	\begin{itemize}
	\item Every year new strains of influenza develop.
	\item ...a particularly beautiful strain of Swiss pansies.
	\end{itemize}
}
\item plural noun \\
If you hear the \textbf{strains}  \textbf{of} music, you hear music being played.
 \textit{
	\begin{itemize}
	\item She could hear the tinny strains of a chamber orchestra.
	\end{itemize}
}
\end{enumerate}

\section*{van}
{\large \color{blue}  vans  }
\subsection*{Explain}
\begin{enumerate}
\item countable noun \\
A \textbf{van} is a small or medium-sized road vehicle with one row of seats at the front and a space for carrying goods behind.
 \textit{
	\begin{itemize}
	\end{itemize}
}
\item countable noun \\
A \textbf{van} is a railway carriage , often without windows , which is used to carry luggage , goods, or mail.
 \textit{
	\begin{itemize}
	\item In the guard's van lay my tin trunk.
	\end{itemize}
}
\end{enumerate}

\section*{strive}
{\large \color{blue}  strives  striving  }
\subsection*{Explain}
\begin{enumerate}
\item verb \\
If you \textbf{strive}  \textbf{to} do something or \textbf{strive}  \textbf{for} something, you make a great effort to do it or get it.
 \textit{
	\begin{itemize}
	\item He strives hard to keep himself very fit.
	\item She strove to read the name on the stone pillar.
	\item The region must now strive for economic development as well as peace.
	\end{itemize}
}
\end{enumerate}

\section*{survive}
{\large \color{blue}  survives  surviving  survived  }
\subsection*{Explain}
\begin{enumerate}
\item verb \\
If a person or living thing \textbf{survives} in a dangerous  situation such as an accident or an illness , they do not die .
 \textit{
	\begin{itemize}
	\item They battled to survive in icy seas for over four hours.
	\item Those organisms that are that are most suited to the environment will survive.
	\item Drugs that dissolve blood clots can help people survive heart attacks.
	\end{itemize}
}
\item verb \\
If you \textbf{survive} in difficult  circumstances , you manage to live or continue in spite of them and do not let them affect you very much.
 \textit{
	\begin{itemize}
	\item On my first day here I thought, 'Ooh, how will I survive?'
	\item ...people who are struggling to survive without jobs.
	\item Where once she had been totally self-sufficient, she now had to survive on income
support.
	\item ...a man who had survived his share of boardroom coups.
	\end{itemize}
}
\item verb \\
If something \textbf{survives} , it continues to exist  even after being in a dangerous situation or existing for a long time.
 \textit{
	\begin{itemize}
	\item When the market economy is introduced, many factories will not survive.
	\item The chances of a planet surviving a supernova always looked terribly slim.
	\item ...surviving examples of 19th-century architecture in the Mid-West.
	\end{itemize}
}
\item verb \\
If you \textbf{survive} someone, you continue to live after they have died.
 \textit{
	\begin{itemize}
	\item Most women will survive their spouses.
	\item She is survived by two daughters from her first marriage.
	\item ...William Shakespeare's last surviving descendant.
	\end{itemize}
}
\end{enumerate}

\section*{worm}
{\large \color{blue}  worms  worming  wormed  }
\subsection*{Explain}
\begin{enumerate}
\item countable noun \\
A \textbf{worm} is a small animal with a long thin body, no bones and no legs.
 \textit{
	\begin{itemize}
	\end{itemize}
}
\item plural noun \\
If animals or people have \textbf{worms} , worms are living in their intestines .
 \textit{
	\begin{itemize}
	\end{itemize}
}
\item verb \\
If you \textbf{worm} an animal, you give it medicine in order to kill the worms that are living in its
intestines.
 \textit{
	\begin{itemize}
	\item I worm all my birds in early spring.
	\item All adult dogs are routinely wormed at least every six months.
	\end{itemize}
}
\item verb \\
If you \textbf{worm} your \textbf{way}  somewhere , you move there with difficulty , twisting or bending your body or making it narrow.
 \textit{
	\begin{itemize}
	\item I had to worm my way out sideways from the bench in a ridiculous, undignified fashion.
	\item The kitten wormed its way through the just-open door.
	\end{itemize}
}
\item verb \\
If you say that someone \textbf{is worming} their \textbf{way} to success , or \textbf{is worming} their \textbf{way} into someone else's affection , you disapprove of the way that they are gradually making someone trust them or like them, often in order to deceive them or gain some advantage .
 \textit{
	\begin{itemize}
	\item She never misses a chance to worm her way into the public's hearts.
	\item Everyone knows people who have wormed their way up on old school connections.
	\end{itemize}
}
\item singular noun \\
If you call a person a \textbf{worm} , you are insulting them by saying that they have a very weak or unpleasant character and you have no respect for them.
 \textit{
	\begin{itemize}
	\end{itemize}
}
\item countable noun \\
A \textbf{worm} is a computer program that contains a virus which duplicates itself many times in
a network.
 \textit{
	\begin{itemize}
	\end{itemize}
}
\item  \\
 a can of worms \textit{
	\begin{itemize}
	\end{itemize}
}
\item  \\
 the worm turns \textit{
	\begin{itemize}
	\end{itemize}
}
\end{enumerate}

\section*{undo}
{\large \color{blue}  undoes  undoing  undid  undone  }
\subsection*{Explain}
\begin{enumerate}
\item verb \\
If you \textbf{undo} something that is closed , tied , or held  together , or if you \textbf{undo} the thing holding it, you loosen or remove the thing holding it.
 \textit{
	\begin{itemize}
	\item I managed secretly to undo a corner of the parcel.
	\item I undid the bottom two buttons of my yellow and grey shirt.
	\item Some clamps that had held the device together came undone.
	\end{itemize}
}
\item verb \\
To \textbf{undo} something that has been done  means to reverse its effect.
 \textit{
	\begin{itemize}
	\item A heavy-handed approach from the police could undo that good impression.
	\item She knew it would be difficult to undo the damage that had been done.
	\item If Michael won, he would undo everything I have fought for.
	\end{itemize}
}
\item verb \\
If a person, organization , or plan  \textbf{is undone by} something, that thing causes their failure .
 \textit{
	\begin{itemize}
	\item They were undone by a goal from Cooper.
	\item Macbeth is the story of a Scottish soldier who becomes king but is undone by his
own ambition.
	\end{itemize}
}
\end{enumerate}

\section*{accelerate}
{\large \color{blue}  accelerates  accelerating  accelerated  }
\subsection*{Explain}
\begin{enumerate}
\item verb \\
If the process or rate of something \textbf{accelerates} or if something \textbf{accelerates} it, it gets faster and faster.
 \textit{
	\begin{itemize}
	\item Growth will accelerate to 2.9% next year.
	\item The government is to accelerate its privatisation programme.
	\end{itemize}
}
\item verb \\
When a moving vehicle  \textbf{accelerates} , it goes faster and faster.
 \textit{
	\begin{itemize}
	\item Suddenly the car accelerated.
	\item She accelerated away from us.
	\end{itemize}
}
\end{enumerate}

\section*{alliance}
{\large \color{blue}  alliances  }
\subsection*{Explain}
\begin{enumerate}
\item countable noun \\
An \textbf{alliance} is a group of countries or political parties that are formally united and working together because they have similar aims.
 \textit{
	\begin{itemize}
	\item The two parties were still too much apart to form an alliance.
	\end{itemize}
}
\item countable noun \\
An \textbf{alliance} is a relationship in which two countries, political parties, or organizations work together for some
purpose.
 \textit{
	\begin{itemize}
	\item The trend has led to the formation of alliances between online-only retailers and
traditional shops.
	\item They are now in a position to govern the state in alliance with either the Free Democrats
or the Green Party.
	\end{itemize}
}
\end{enumerate}

\section*{addict}
{\large \color{blue}  addicts  }
\subsection*{Explain}
\begin{enumerate}
\item countable noun \\
An \textbf{addict} is someone who takes harmful drugs and cannot stop  taking them.
 \textit{
	\begin{itemize}
	\item He's only 24 years old and a drug addict.
	\end{itemize}
}
\item countable noun \\
If you say that someone is an \textbf{addict} , you mean that they like a particular activity very much and spend as much time doing it as they can.
 \textit{
	\begin{itemize}
	\item She is a TV addict and watches as much as she can.
	\end{itemize}
}
\end{enumerate}

\section*{affiliate}
{\large \color{blue}  affiliates  affiliating  affiliated  }
\subsection*{Explain}
\begin{enumerate}
\item countable noun \\
An \textbf{affiliate} is an organization which is officially  connected with another, larger organization or is a member of it.
 \textit{
	\begin{itemize}
	\item ...twelve companies, including three affiliates of a Texas oil firm.
	\item The World Chess Federation has affiliates in around 120 countries.
	\end{itemize}
}
\item verb \\
If an organization \textbf{affiliates to} or \textbf{with} another larger organization, it forms a close connection with the larger organization
or becomes a member of it.
 \textit{
	\begin{itemize}
	\item All youth groups will have to affiliate to the National Youth Agency.
	\item The Government will not allow the staff association to affiliate with outside unions.
	\end{itemize}
}
\item verb \\
If a professional person such as a lawyer or doctor  \textbf{affiliates}  \textbf{with} an organization, they become officially connected with that organization.
 \textit{
	\begin{itemize}
	\item He said he wanted to affiliate with a U.S. firm.
	\end{itemize}
}
\end{enumerate}

\section*{biography}
{\large \color{blue}  biographies  }
\subsection*{Explain}
\begin{enumerate}
\item countable noun \\
A \textbf{biography} of someone is an account of their life, written by someone else.
 \textit{
	\begin{itemize}
	\end{itemize}
}
\item uncountable noun \\
\textbf{Biography} is the branch of literature which deals with accounts of people's lives.
 \textit{
	\begin{itemize}
	\item ...a volume of biography and criticism.
	\end{itemize}
}
\end{enumerate}

\section*{appal}
{\large \color{blue}  appals  appalling  appalled  }
\subsection*{Explain}
\begin{enumerate}
\item verb \\
If something \textbf{appals} you, it disgusts you because it seems so bad or unpleasant.
 \textit{
	\begin{itemize}
	\item His ignorance appals me.
	\item The complete disregard for suffering would appal any decent person.
	\end{itemize}
}
\end{enumerate}

\section*{canvas}
{\large \color{blue}  canvases  }
\subsection*{Explain}
\begin{enumerate}
\item uncountable noun \\
\textbf{Canvas} is a strong , heavy cloth that is used for making things such as tents, sails, and bags .
 \textit{
	\begin{itemize}
	\item ...a canvas bag.
	\end{itemize}
}
\item variable noun \\
A \textbf{canvas} is a piece of canvas or similar material on which an oil painting can be done.
 \textit{
	\begin{itemize}
	\end{itemize}
}
\item countable noun \\
A \textbf{canvas} is a painting that has been done on canvas.
 \textit{
	\begin{itemize}
	\item The show includes canvases by masters like Carpaccio, Canaletto and Guardi.
	\end{itemize}
}
\item  \\
 under canvas \textit{
	\begin{itemize}
	\end{itemize}
}
\end{enumerate}

\section*{bewilder}
{\large \color{blue}  bewilders  bewildering  bewildered  }
\subsection*{Explain}
\begin{enumerate}
\item verb \\
If something \textbf{bewilders} you, it is so confusing or difficult that you cannot understand it.
 \textit{
	\begin{itemize}
	\item The silence from Alex had hurt and bewildered her.
	\end{itemize}
}
\end{enumerate}

\section*{charm}
{\large \color{blue}  charms  charming  charmed  }
\subsection*{Explain}
\begin{enumerate}
\item variable noun \\
\textbf{Charm} is the quality of being pleasant or attractive.
 \textit{
	\begin{itemize}
	\item 'Snow White and the Seven Dwarfs', the 1937 Disney classic, has lost none of its
original charm.
	\item The house had its charms, not the least of which was the furniture that came with
it.
	\end{itemize}
}
\item uncountable noun \\
Someone who has \textbf{charm}  behaves in a friendly , pleasant way that makes people like them.
 \textit{
	\begin{itemize}
	\item He was a man of great charm and distinction.
	\end{itemize}
}
\item verb \\
If you \textbf{charm} someone, you please them, especially by using your charm.
 \textit{
	\begin{itemize}
	\item He even charmed Mrs Prichard, carrying her shopping and flirting with her.
	\item At first we were charmed by his straight talk and eccentric antics.
	\item You can't force–or even charm–him into behaving differently if he doesn't want to.
	\end{itemize}
}
\item verb \\
If you \textbf{charm} your \textbf{way} into or out of a place or situation , you use your charm to get into or out of that place or situation.
 \textit{
	\begin{itemize}
	\item ...charming his way into the British Embassy in Teheran.
	\item He charmed his way out of trouble.
	\end{itemize}
}
\item verb \\
If you say that someone \textbf{charmed} something \textbf{out of} you or \textbf{from} you, you mean that they used their charm to persuade you to give it to them.
 \textit{
	\begin{itemize}
	\item He is good at charming money out of companies.
	\end{itemize}
}
\item countable noun \\
A \textbf{charm} is a small ornament that is fixed to a bracelet or necklace .
 \textit{
	\begin{itemize}
	\end{itemize}
}
\item countable noun \\
A \textbf{charm} is an act, saying , or object that is believed to have magic powers.
 \textit{
	\begin{itemize}
	\item They cross their fingers and spit over their shoulders as charms against the evil
eye.
	\item ...a good luck charm.
	\end{itemize}
}
\item  \\
 turn on the charm \textit{
	\begin{itemize}
	\end{itemize}
}
\item  \\
 work/go/run like a charm \textit{
	\begin{itemize}
	\end{itemize}
}
\end{enumerate}

\section*{brace}
{\large \color{blue}  braces  bracing  braced  }
\subsection*{Explain}
\begin{enumerate}
\item verb \\
If you \textbf{brace}  \textbf{yourself for} something unpleasant or difficult , you prepare yourself for it.
 \textit{
	\begin{itemize}
	\item He braced himself for the icy plunge into the black water.
	\item She braced herself, as if to meet a blow.
	\end{itemize}
}
\item verb \\
If you \textbf{brace}  \textbf{yourself}  \textbf{against} something or \textbf{brace} part of your body \textbf{against} it, you press against something in order to steady your body or to avoid falling.
 \textit{
	\begin{itemize}
	\item Elaine braced herself against the dresser and looked in the mirror.
	\item He braced his back against the wall.
	\end{itemize}
}
\item verb \\
If you \textbf{brace} your shoulders or knees , you keep them stiffly in a particular position.
 \textit{
	\begin{itemize}
	\item He braced his shoulders as the snow slashed across his face.
	\end{itemize}
}
\item verb \\
To \textbf{brace} something means to strengthen or support it with something else.
 \textit{
	\begin{itemize}
	\item Overhead, the lights showed the old timbers, used to brace the roof.
	\end{itemize}
}
\item countable noun \\
You can refer to two things of the same kind as a \textbf{brace}  \textbf{of} that thing. The plural form is also \textbf{brace} .
 \textit{
	\begin{itemize}
	\item ...a brace of bottles of Mercier Rose champagne.
	\item ...a few brace of grouse.
	\end{itemize}
}
\item countable noun \\
A \textbf{brace} is a device attached to a part of a person's body, for example to a weak leg, in order to strengthen or support it.
 \textit{
	\begin{itemize}
	\item He wore leg braces after polio in childhood.
	\item She wears a neck brace.
	\end{itemize}
}
\item countable noun \\
A \textbf{brace} is a metal device that can be fastened to a child's teeth in order to help them grow straight .
 \textit{
	\begin{itemize}
	\end{itemize}
}
\item plural noun \\
\textbf{Braces} are a pair of straps that pass over your shoulders and fasten to your trousers at the front and back in order to stop them from falling down.
 \textit{
	\begin{itemize}
	\end{itemize}
}
\item countable noun \\
\textbf{Braces} or \textbf{curly braces} are a pair of written marks that you place around words, numbers, or parts of a computer
 code , for example to indicate that they are connected in some way or are separate from
other parts of the writing or code.
 \textit{
	\begin{itemize}
	\end{itemize}
}
\end{enumerate}

\section*{colleague}
{\large \color{blue}  colleagues  }
\subsection*{Explain}
\begin{enumerate}
\item countable noun \\
Your \textbf{colleagues} are the people you work with, especially in a professional  job .
 \textit{
	\begin{itemize}
	\item Without consulting his colleagues he flew from Lisbon to Split.
	\item A colleague urged him to see a psychiatrist, but Faulkner refused.
	\end{itemize}
}
\end{enumerate}

\section*{complicate}
{\large \color{blue}  complicates  complicating  complicated  }
\subsection*{Explain}
\begin{enumerate}
\item verb \\
To \textbf{complicate} something means to make it more difficult to understand or deal with.
 \textit{
	\begin{itemize}
	\item What complicates the issue is the burden of history.
	\item The day's events, he said, would only complicate the task of the peacekeeping forces.
	\item To complicate matters further, everybody's vitamin requirements vary.
	\item Bad weather continues to complicate efforts to deal with oil spilling from the tanker.
	\end{itemize}
}
\end{enumerate}

\section*{community}
{\large \color{blue}  communities  }
\subsection*{Explain}
\begin{enumerate}
\item singular noun \\
\textbf{The}  \textbf{community} is all the people who live in a particular area or place.
 \textit{
	\begin{itemize}
	\item He's well liked by people in the community.
	\item The community has set up a campaign to save the park.
	\item The growth of such vigilante gangs has worried community leaders, police and politicians.
	\end{itemize}
}
\item countable noun \\
A particular \textbf{community} is a group of people who are similar in some way.
 \textit{
	\begin{itemize}
	\item The police haven't really done anything for the black community in particular.
	\item ...the business community.
	\end{itemize}
}
\item uncountable noun \\
\textbf{Community} is friendship between different people or groups, and a sense of having something in common.
 \textit{
	\begin{itemize}
	\item The retirement home provides a sense of community.
	\item Two of our greatest strengths are diversity and community.
	\end{itemize}
}
\end{enumerate}

\section*{confess}
{\large \color{blue}  confesses  confessing  confessed  }
\subsection*{Explain}
\begin{enumerate}
\item verb \\
If someone \textbf{confesses} to doing something wrong , they admit that they did it.
 \textit{
	\begin{itemize}
	\item He had confessed to seventeen murders.
	\item Her husband confessed to having had an affair.
	\item I had expected her to confess that she only wrote these books for the money.
	\item Most rape victims confess a feeling of helplessness.
	\item Ray changed his mind, claiming that he had been forced into confessing.
	\item 'I played a very bad match,' he confessed.
	\end{itemize}
}
\item verb \\
If someone \textbf{confesses} or \textbf{confesses} their sins, they tell God or a priest about their sins so that they can be forgiven .
 \textit{
	\begin{itemize}
	\item You just go to the church and confess your sins.
	\item Once we have confessed our failures and mistakes to God, we should stop feeling guilty.
	\end{itemize}
}
\item  \\
 I confess/I must confess/I have to confess \textit{
	\begin{itemize}
	\end{itemize}
}
\end{enumerate}

\section*{companion}
{\large \color{blue}  companions  }
\subsection*{Explain}
\begin{enumerate}
\item countable noun \\
A \textbf{companion} is someone who you spend time with or who you are travelling with.
 \textit{
	\begin{itemize}
	\item Fred had been her constant companion for the last six years of her life.
	\item I asked my travelling companion what he thought of the situation.
	\end{itemize}
}
\end{enumerate}

\section*{confirm}
{\large \color{blue}  confirms  confirming  confirmed  }
\subsection*{Explain}
\begin{enumerate}
\item verb \\
If something \textbf{confirms} what you believe , suspect , or fear , it shows that it is definitely true.
 \textit{
	\begin{itemize}
	\item X-rays have confirmed that he has not broken any bones.
	\item These new statistics confirm our worst fears about the depth of the recession.
	\item This confirms what I suspected all along.
	\end{itemize}
}
\item verb \\
If you \textbf{confirm} something that has been stated or suggested , you say that it is true because you know about it.
 \textit{
	\begin{itemize}
	\item The spokesman confirmed that the area was now in rebel hands.
	\item He confirmed what had long been feared.
	\item Can you confirm this?
	\end{itemize}
}
\item verb \\
If you \textbf{confirm} an arrangement or appointment , you say that it is definite, usually in a letter or on the telephone .
 \textit{
	\begin{itemize}
	\item You make the reservation, and I'll confirm it in writing.
	\end{itemize}
}
\item verb \\
If someone \textbf{is confirmed} , they are formally accepted as a member of a Christian church during a ceremony in which they say they believe what the church teaches .
 \textit{
	\begin{itemize}
	\item He was confirmed as a member of the Church of England.
	\end{itemize}
}
\item verb \\
If something \textbf{confirms} you \textbf{in} your decision , belief , or opinion , it makes you think that you are definitely right.
 \textit{
	\begin{itemize}
	\item It has confirmed me in my decision not to become a nurse.
	\end{itemize}
}
\item verb \\
If someone \textbf{confirms} their position, role , or power, they do something to make their power, position, or role stronger or more definite.
 \textit{
	\begin{itemize}
	\item Williams has confirmed his position as the world's number one snooker player.
	\end{itemize}
}
\item verb \\
If something \textbf{confirms} you \textbf{as} something, it shows that you definitely deserve a name, role, or position.
 \textit{
	\begin{itemize}
	\item His new role could confirm him as one of our leading actors.
	\end{itemize}
}
\end{enumerate}

\section*{compassion}
{\large \color{blue}  }
\subsection*{Explain}
\begin{enumerate}
\item uncountable noun \\
\textbf{Compassion} is a feeling of pity, sympathy , and understanding for someone who is suffering.
 \textit{
	\begin{itemize}
	\item Patients need time and compassion from their physicians.
	\end{itemize}
}
\end{enumerate}

\section*{confront}
{\large \color{blue}  confronts  confronting  confronted  }
\subsection*{Explain}
\begin{enumerate}
\item verb \\
If you \textbf{are confronted}  \textbf{with} a problem , task , or difficulty , you have to deal with it.
 \textit{
	\begin{itemize}
	\item She was confronted with severe money problems.
	\item Ministers underestimated the magnitude of the task confronting them.
	\end{itemize}
}
\item verb \\
If you \textbf{confront} a difficult  situation or issue , you accept the fact that it exists and try to deal with it.
 \textit{
	\begin{itemize}
	\item We are learning how to confront death.
	\item NATO countries have been forced to confront fundamental moral questions.
	\end{itemize}
}
\item verb \\
If you \textbf{are confronted} by something that you find  threatening or difficult to deal with, it is there in front of you.
 \textit{
	\begin{itemize}
	\item I was confronted with an array of knobs, levers, and switches.
	\end{itemize}
}
\item verb \\
If you \textbf{confront} someone, you stand or sit in front of them, especially when you are going to fight , argue , or compete with them.
 \textit{
	\begin{itemize}
	\item She pushed her way through the mob and confronted him face to face.
	\item They don't hesitate to open fire when confronted by police.
	\item The candidates confronted each other during a televised debate.
	\end{itemize}
}
\item verb \\
If you \textbf{confront} someone \textbf{with} something, you present facts or evidence to them in order to accuse them of something.
 \textit{
	\begin{itemize}
	\item She had decided to confront Kathryn with what she had learnt.
	\item I could not bring myself to confront him about it.
	\item His confronting me forced me to search for the answers.
	\end{itemize}
}
\end{enumerate}

\section*{comrade}
{\large \color{blue}  comrades  }
\subsection*{Explain}
\begin{enumerate}
\item countable noun \\
Your \textbf{comrades} are your friends , especially friends that you share a difficult or dangerous  situation with.
 \textit{
	\begin{itemize}
	\item Unlike so many of his comrades he survived the war.
	\end{itemize}
}
\item title noun \\
Socialists or communists sometimes  call each other \textbf{comrade} , especially in meetings .
 \textit{
	\begin{itemize}
	\item The Party's authority, comrades, will be put to a serious test.
	\end{itemize}
}
\end{enumerate}

\section*{confuse}
{\large \color{blue}  confuses  confusing  confused  }
\subsection*{Explain}
\begin{enumerate}
\item verb \\
If you \textbf{confuse} two things, you get them mixed up, so that you think one of them is the other one.
 \textit{
	\begin{itemize}
	\item Great care is taken to avoid confusing the two types of projects.
	\item I can't see how anyone could confuse you with another!
	\end{itemize}
}
\item verb \\
To \textbf{confuse} someone means to make it difficult for them to know  exactly what is happening or what to do.
 \textit{
	\begin{itemize}
	\item German politics surprised and confused him.
	\end{itemize}
}
\item verb \\
To \textbf{confuse} a situation means to make it complicated or difficult to understand .
 \textit{
	\begin{itemize}
	\item To confuse the issue, the amount of sleep people need varies enormously.
	\end{itemize}
}
\end{enumerate}

\section*{equality}
{\large \color{blue}  }
\subsection*{Explain}
\begin{enumerate}
\item uncountable noun \\
\textbf{Equality} is the same status , rights, and responsibilities for all the members of a society , group, or family.
 \textit{
	\begin{itemize}
	\item ...equality of the sexes.
	\end{itemize}
}
\end{enumerate}

\section*{consolidate}
{\large \color{blue}  consolidates  consolidating  consolidated  }
\subsection*{Explain}
\begin{enumerate}
\item verb \\
If you \textbf{consolidate} something that you have, for example  power or success , you strengthen it so that it becomes more effective or secure .
 \textit{
	\begin{itemize}
	\item The question is: will the junta consolidate its power by force?
	\item Many young singers started and consolidated their careers at Covent Garden.
	\end{itemize}
}
\item verb \\
To \textbf{consolidate} a number of small groups or firms  means to make them into one large organization .
 \textit{
	\begin{itemize}
	\item The state has 60 days to consolidate Louisiana's four higher-education boards.
	\item The company consolidated some operations last summer.
	\end{itemize}
}
\end{enumerate}

\section*{factor}
{\large \color{blue}  factors  factoring  factored  }
\subsection*{Explain}
\begin{enumerate}
\item countable noun \\
A \textbf{factor} is one of the things that affects an event, decision , or situation .
 \textit{
	\begin{itemize}
	\item Physical activity is an important factor in maintaining fitness.
	\end{itemize}
}
\item countable noun \\
If an amount increases by \textbf{a}  \textbf{factor}  \textbf{of} two, for example , or by \textbf{a}  \textbf{factor}  \textbf{of}  eight , then it becomes two times bigger or eight times bigger.
 \textit{
	\begin{itemize}
	\item The cost of butter quadrupled and bread prices increased by a factor of five.
	\end{itemize}
}
\item singular noun \\
You can use \textbf{factor} to refer to a particular level on a scale of measurement .
 \textit{
	\begin{itemize}
	\item ...suncream with a protection factor of 8.
	\end{itemize}
}
\item countable noun \\
A \textbf{factor} of a whole number is a smaller whole number which can be multiplied with another whole number to produce the first whole number.
 \textit{
	\begin{itemize}
	\end{itemize}
}
\end{enumerate}

\section*{dazzle}
{\large \color{blue}  dazzles  dazzling  dazzled  }
\subsection*{Explain}
\begin{enumerate}
\item verb \\
If someone or something \textbf{dazzles} you, you are extremely  impressed by their skill , qualities, or beauty .
 \textit{
	\begin{itemize}
	\item George dazzled her with his knowledge of the world.
	\item The movie's special effects fail to dazzle.
	\end{itemize}
}
\item singular noun \\
The \textbf{dazzle}  \textbf{of} something is a quality it has, such as beauty or skill, which is impressive and attractive .
 \textit{
	\begin{itemize}
	\item The dazzle of stardom and status attracts them.
	\end{itemize}
}
\item verb \\
If a bright light \textbf{dazzles} you, it makes you unable to see properly for a short time.
 \textit{
	\begin{itemize}
	\item The sun, glinting from the pool, dazzled me.
	\item Kelly was dazzled by the lights.
	\end{itemize}
}
\item uncountable noun \\
The \textbf{dazzle} of a light is its brightness , which makes it impossible for you to see properly for a short time.
 \textit{
	\begin{itemize}
	\item The sun's dazzle on the water hurts my eyes.
	\item ...a filter that can cut dazzle.
	\end{itemize}
}
\end{enumerate}

\section*{federation}
{\large \color{blue}  federations  }
\subsection*{Explain}
\begin{enumerate}
\item countable noun \\
A \textbf{federation} is a federal country.
 \textit{
	\begin{itemize}
	\item ...the Russian Federation.
	\end{itemize}
}
\item countable noun \\
A \textbf{federation} is a group of societies or other organizations which have joined together, usually because they share a common interest.
 \textit{
	\begin{itemize}
	\item ...the British Athletic Federation.
	\item The organization emerged from a federation of six national agencies.
	\end{itemize}
}
\end{enumerate}

\section*{discourage}
{\large \color{blue}  discourages  discouraging  discouraged  }
\subsection*{Explain}
\begin{enumerate}
\item verb \\
If someone or something \textbf{discourages} you, they cause you to lose your enthusiasm about your actions.
 \textit{
	\begin{itemize}
	\item It may be difficult to do at first. Don't let this discourage you.
	\end{itemize}
}
\item verb \\
To \textbf{discourage} an action or to \textbf{discourage} someone \textbf{from} doing it means to make them not want to do it.
 \textit{
	\begin{itemize}
	\item ...typhoons that discouraged shopping and leisure activities.
	\item ...a campaign to discourage children from smoking.
	\end{itemize}
}
\end{enumerate}

\section*{flesh}
{\large \color{blue}  fleshes  fleshing  fleshed  }
\subsection*{Explain}
\begin{enumerate}
\item uncountable noun \\
\textbf{Flesh} is the soft part of a person's or animal's body between the bones and the skin.
 \textit{
	\begin{itemize}
	\item ...maggots which eat away dead flesh.
	\item ...the pale pink flesh of trout and salmon.
	\end{itemize}
}
\item uncountable noun \\
You can use \textbf{flesh} to refer to human skin and the human body, especially when you are considering it in a sexual  way .
 \textit{
	\begin{itemize}
	\item ...the warmth of her flesh.
	\item ...the sins of the flesh.
	\end{itemize}
}
\item uncountable noun \\
The \textbf{flesh} of a fruit or vegetable is the soft inside part of it.
 \textit{
	\begin{itemize}
	\item Cut the flesh from the olives and discard the stones.
	\end{itemize}
}
\item  \\
 flesh and blood \textit{
	\begin{itemize}
	\end{itemize}
}
\item  \\
 own flesh and blood \textit{
	\begin{itemize}
	\end{itemize}
}
\item  \\
 to make someone's flesh creep \textit{
	\begin{itemize}
	\end{itemize}
}
\item  \\
 in the flesh \textit{
	\begin{itemize}
	\end{itemize}
}
\item  \\
 put flesh on \textit{
	\begin{itemize}
	\end{itemize}
}
\end{enumerate}

\section*{enable}
{\large \color{blue}  enables  enabling  enabled  }
\subsection*{Explain}
\begin{enumerate}
\item verb \\
If someone or something \textbf{enables} you \textbf{to} do a particular thing, they give you the opportunity to do it.
 \textit{
	\begin{itemize}
	\item The new test should enable doctors to detect the disease early.
	\item ...a new charter for training to enable young people to make the most of their potential.
	\end{itemize}
}
\item verb \\
To \textbf{enable} something \textbf{to}  happen means to make it possible for it to happen.
 \textit{
	\begin{itemize}
	\item The hot sun enables the grapes to reach optimum ripeness.
	\item A series of holes in the side panels enables the position of the shelves to be adjusted.
	\item The working class is still too small to enable a successful socialist revolution.
	\end{itemize}
}
\item verb \\
To \textbf{enable} someone \textbf{to} do something means to give them permission or the right to do it.
 \textit{
	\begin{itemize}
	\item The republic's legislation enables young people to do a form of alternative service.
	\end{itemize}
}
\end{enumerate}

\section*{league}
{\large \color{blue}  leagues  }
\subsection*{Explain}
\begin{enumerate}
\item countable noun \\
A \textbf{league} is a group of people, clubs, or countries that have joined together for a particular purpose , or because they share a common interest.
 \textit{
	\begin{itemize}
	\item ...the League of Nations.
	\item ...the World Muslim League.
	\end{itemize}
}
\item countable noun \\
A \textbf{league} is a group of teams that play the same sport or activity against each other.
 \textit{
	\begin{itemize}
	\item ...the American League series between the Boston Red Sox and World Champion Oakland
Athletics.
	\item The club are on the brink of promotion to the Premier League.
	\end{itemize}
}
\item countable noun \\
You use the word \textbf{league} to make comparisons between different people or things, especially in terms of their quality.
 \textit{
	\begin{itemize}
	\item Her success has taken her out of my league.
	\item Their record sales would put them in the same league as The Rolling Stones.
	\end{itemize}
}
\item  \\
 in league \textit{
	\begin{itemize}
	\end{itemize}
}
\end{enumerate}

\section*{exclaim}
{\large \color{blue}  exclaims  exclaiming  exclaimed  }
\subsection*{Explain}
\begin{enumerate}
\item verb \\
Writers  sometimes use \textbf{exclaim} to show that someone is speaking suddenly, loudly, or emphatically , often because they are excited , shocked , or angry .
 \textit{
	\begin{itemize}
	\item 'He went back to the lab,' Iris exclaimed impatiently.
	\item He exclaims that it must be a typing error.
	\end{itemize}
}
\end{enumerate}

\section*{legend}
{\large \color{blue}  legends  }
\subsection*{Explain}
\begin{enumerate}
\item variable noun \\
A \textbf{legend} is a very old and popular story that may be true .
 \textit{
	\begin{itemize}
	\item ...the legends of ancient Greece.
	\item ...the Robin Hood legend.
	\item The play was based on Irish legend.
	\end{itemize}
}
\item countable noun \\
If you refer to someone as a \textbf{legend} , you mean that they are very famous and admired by a lot of people.
 \textit{
	\begin{itemize}
	\item ...blues legends John Lee Hooker and B.B. King.
	\end{itemize}
}
\item variable noun \\
A \textbf{legend} is a story that people talk about, concerning people, places, or events that exist or are famous at the present time.
 \textit{
	\begin{itemize}
	\item The incident has since become a family legend.
	\item His frequent brushes with death are the stuff of legend among the press.
	\end{itemize}
}
\end{enumerate}

\section*{exhaust}
{\large \color{blue}  exhausts  exhausting  exhausted  }
\subsection*{Explain}
\begin{enumerate}
\item verb \\
If something \textbf{exhausts} you, it makes you so tired, either physically or mentally, that you have no energy
left.
 \textit{
	\begin{itemize}
	\item Don't exhaust him.
	\item He took to walking long distances in an attempt to physically exhaust himself.
	\end{itemize}
}
\item verb \\
If you \textbf{exhaust} something such as money or food, you use or finish it all.
 \textit{
	\begin{itemize}
	\item We have exhausted all our material resources.
	\item They said that food supplies were almost exhausted.
	\end{itemize}
}
\item verb \\
If you \textbf{have exhausted} a subject or topic , you have talked about it so much that there is nothing more to say about it.
 \textit{
	\begin{itemize}
	\item She and Chantal must have exhausted the subject of babies and clothes.
	\end{itemize}
}
\item countable noun \\
The \textbf{exhaust} or the \textbf{exhaust pipe} is the pipe which carries the gas out of the engine of a vehicle .
 \textit{
	\begin{itemize}
	\end{itemize}
}
\item uncountable noun \\
\textbf{Exhaust} is the gas or steam that is produced when the engine of a vehicle is running .
 \textit{
	\begin{itemize}
	\item ...the exhaust from a car engine.
	\item The city's streets are filthy and choked with exhaust fumes.
	\item ...the concentration of car exhausts in the Los Angeles area.
	\end{itemize}
}
\end{enumerate}

\section*{lump}
{\large \color{blue}  lumps  lumping  lumped  }
\subsection*{Explain}
\begin{enumerate}
\item countable noun \\
A \textbf{lump}  \textbf{of} something is a solid piece of it.
 \textit{
	\begin{itemize}
	\item The potter shaped and squeezed the lump of clay into a graceful shape.
	\item ...a lump of wood.
	\item They used to buy ten kilos of meat in one lump.
	\end{itemize}
}
\item countable noun \\
A \textbf{lump} on or in someone's body is a small, hard swelling that has been caused by an injury or an illness .
 \textit{
	\begin{itemize}
	\item I've got a lump on my shoulder.
	\item Howard had to have cancer surgery for a lump in his chest.
	\end{itemize}
}
\item countable noun \\
A \textbf{lump}  \textbf{of}  sugar is a small cube of it.
 \textit{
	\begin{itemize}
	\item ...a nugget of rough gold about the size of a lump of sugar.
	\item 'No sugar,' I said, and Jim asked for two lumps.
	\end{itemize}
}
\item  \\
 have to lump it \textit{
	\begin{itemize}
	\end{itemize}
}
\item  \\
 a lump in your throat \textit{
	\begin{itemize}
	\end{itemize}
}
\end{enumerate}

\section*{facilitate}
{\large \color{blue}  facilitates  facilitating  facilitated  }
\subsection*{Explain}
\begin{enumerate}
\item verb \\
To \textbf{facilitate} an action or process, especially one that you would like to happen , means to make it easier or more likely to happen.
 \textit{
	\begin{itemize}
	\item The new airport will facilitate the development of tourism.
	\item He argued that the economic recovery had been facilitated by his tough stance.
	\end{itemize}
}
\end{enumerate}

\section*{lure}
{\large \color{blue}  lures  luring  lured  }
\subsection*{Explain}
\begin{enumerate}
\item verb \\
To \textbf{lure} someone means to trick them into a particular place or to trick them into doing something that they should
not do.
 \textit{
	\begin{itemize}
	\item He lured her to his home and shot her with his father's gun.
	\item They did not realise that they were being lured into a trap.
	\item Supermarkets will try to lure customers back in with special offers.
	\end{itemize}
}
\item countable noun \\
A \textbf{lure} is an object which is used to attract animals so that they can be caught .
 \textit{
	\begin{itemize}
	\end{itemize}
}
\item countable noun \\
A \textbf{lure} is an attractive quality that something has, or something that you find attractive.
 \textit{
	\begin{itemize}
	\item Plans like these will without doubt act as a lure to potential investors.
	\item The lure of rural life is proving as strong as ever.
	\end{itemize}
}
\end{enumerate}

\section*{freeze}
{\large \color{blue}  freezes  freezing  froze  frozen  }
\subsection*{Explain}
\begin{enumerate}
\item verb \\
If a liquid or a substance containing a liquid \textbf{freezes} , or if something \textbf{freezes} it, it becomes solid because of low temperatures.
 \textit{
	\begin{itemize}
	\item If the temperature drops below 0°C, water freezes.
	\item The ground froze solid.
	\item ...the discovery of how to freeze water at higher temperatures.
	\item ...frozen puddles.
	\end{itemize}
}
\item verb \\
If you \textbf{freeze} something such as food, you preserve it by storing it at a temperature below freezing
point. You can also talk about how well food \textbf{freezes} .
 \textit{
	\begin{itemize}
	\item You can freeze the soup at this stage.
	\item Most fresh herbs will freeze successfully.
	\end{itemize}
}
\item verb \\
If something such as a pipe or machine \textbf{freezes} , it becomes blocked or stiff with ice or frozen liquid.
 \textit{
	\begin{itemize}
	\item The water pipes will freeze.
	\end{itemize}
}
\item verb \\
When \textbf{it freezes} outside, the temperature falls below freezing point.
 \textbf{Freeze} is also a noun .
 \textit{
	\begin{itemize}
	\item What if it rained and then froze all through those months?
	\item The trees were damaged by a freeze in December.
	\end{itemize}
}
\item verb \\
If you \textbf{freeze} , you feel extremely cold.
 \textit{
	\begin{itemize}
	\item The windows didn't fit at the bottom so for a while we froze even in the middle of
summer.
	\item Your hands will freeze doing this.
	\end{itemize}
}
\item verb \\
If someone who is moving \textbf{freezes} , they suddenly stop and become completely still and quiet .
 \textit{
	\begin{itemize}
	\item She froze when the beam of the flashlight struck her.
	\end{itemize}
}
\item verb \\
If the government or a company \textbf{freeze} things such as prices or wages , they state officially that they will not allow them to increase for a fixed period of time.
 \textbf{Freeze} is also a noun.
 \textit{
	\begin{itemize}
	\item They want the government to freeze prices.
	\item Wages have been frozen and workers laid off.
	\item A wage freeze was imposed on all staff earlier this month.
	\item ...a freeze on the prices of consumer goods.
	\end{itemize}
}
\item verb \\
If a government \textbf{freezes} a plan or process, they state officially that they will not allow it to continue for a period of time.
 \textbf{Freeze} is also a noun.
 \textit{
	\begin{itemize}
	\item Britain has already frozen its aid programme.
	\item Diplomatic relations were frozen until August this year.
	\item ...a freeze in nuclear weapons programs.
	\end{itemize}
}
\item verb \\
If someone in authority \textbf{freezes} something such as a bank account, fund , or property, they obtain a legal order which states that it cannot be used or sold
for a particular period of time.
 \textbf{Freeze} is also a noun.
 \textit{
	\begin{itemize}
	\item The governor's action freezes 300,000 accounts.
	\item Under these laws, he said, Mr. Rice's assets could have been frozen.
	\item ...a freeze on private savings.
	\end{itemize}
}
\end{enumerate}

\section*{net}
{\large \color{blue}  nets  netting  netted  }
\subsection*{Explain}
\begin{enumerate}
\item uncountable noun \\
\textbf{Net} is a kind of cloth that you can see through. It is made of very fine threads  woven together so that there are small equal spaces between them.
 \textit{
	\begin{itemize}
	\end{itemize}
}
\item countable noun \\
A \textbf{net} is a piece of netting which is used as a protective covering for something, for example to protect vegetables from birds.
 \textit{
	\begin{itemize}
	\item I threw aside my mosquito net and jumped out of bed.
	\end{itemize}
}
\item countable noun \\
A \textbf{net} is a piece of netting which is used for catching fish, insects, or animals.
 \textit{
	\begin{itemize}
	\item Several fishermen sat on wooden barrels, tending their nets.
	\end{itemize}
}
\item singular noun \\
\textbf{The Net} is the same as the internet .
 \textit{
	\begin{itemize}
	\end{itemize}
}
\item verb \\
If you \textbf{net} a fish or other animal, you catch it in a net.
 \textit{
	\begin{itemize}
	\item I'm quite happy to net a fish and then let it go.
	\item Poachers have been netting salmon to supply the black market.
	\end{itemize}
}
\item countable noun \\
In games such as tennis , \textbf{the}  \textbf{net} is the piece of netting across the centre of the court which the ball has to go over.
 \textit{
	\begin{itemize}
	\end{itemize}
}
\item countable noun \\
\textbf{The}  \textbf{net} on a football or hockey field is the framework with netting over it which is attached to the back of the goal.
 \textit{
	\begin{itemize}
	\item He let the ball slip through his grasp and into the net.
	\end{itemize}
}
\item countable noun \\
In basketball , \textbf{the}  \textbf{net} is the netting which hangs from the metal hoop . You score goals by throwing the ball through the hoop and netting.
 \textit{
	\begin{itemize}
	\end{itemize}
}
\item verb \\
In basketball or football, when you \textbf{net} a goal, you score a goal.
 \textit{
	\begin{itemize}
	\item Centre half Tiler netted his first goal for the club.
	\end{itemize}
}
\item verb \\
If you \textbf{net} something, you manage to get it, especially by using skill .
 \textit{
	\begin{itemize}
	\item They were intent on netting the £250,000 reward.
	\end{itemize}
}
\item verb \\
When a police operation \textbf{nets} a number of people or things, they catch those people or find those things.
 \textit{
	\begin{itemize}
	\item Secret investigations have netted ninety staff suspected of fraud and theft.
	\item The raids also netted weapons including knives.
	\end{itemize}
}
\item verb \\
If you \textbf{net} a particular amount of money, you gain it as profit after all expenses have been paid.
 \textit{
	\begin{itemize}
	\item Last year he netted a cool 3 million pounds by selling his holdings.
	\end{itemize}
}
\item  \\
 to cast your net wider \textit{
	\begin{itemize}
	\end{itemize}
}
\item  \\
 slip through the net \textit{
	\begin{itemize}
	\end{itemize}
}
\item  \\
 slip/fall through the net \textit{
	\begin{itemize}
	\end{itemize}
}
\end{enumerate}

\section*{frighten}
{\large \color{blue}  frightens  frightening  frightened  }
\subsection*{Explain}
\begin{enumerate}
\item verb \\
If something or someone \textbf{frightens} you, they cause you to suddenly  feel afraid, anxious , or nervous .
 \textit{
	\begin{itemize}
	\item He knew that Soli was trying to frighten him, so he smiled to hide his fear.
	\item Most children are frightened by the sight of blood.
	\end{itemize}
}
\item  \\
 frighten the life/wits out of sb \textit{
	\begin{itemize}
	\end{itemize}
}
\end{enumerate}

\section*{network}
{\large \color{blue}  networks  networking  networked  }
\subsection*{Explain}
\begin{enumerate}
\item countable noun \\
A \textbf{network}  \textbf{of} lines, roads, veins, or other long thin things is a large number of them which cross each other or meet at many points.
 \textit{
	\begin{itemize}
	\item ...Strasbourg, with its rambling network of medieval streets.
	\item The uterus is supplied with a rich network of blood vessels and nerves.
	\end{itemize}
}
\item countable noun \\
A \textbf{network}  \textbf{of} people or institutions is a large number of them that have a connection with each other and work together as a system.
 \textit{
	\begin{itemize}
	\item Distribution of the food is going ahead using a network of local volunteers.
	\item He is keen to point out the benefits which the family network can provide.
	\end{itemize}
}
\item countable noun \\
A particular \textbf{network} is a system of things which are connected and which operate together. For example , a \textbf{computer network} consists of a number of computers that are part of the same system.
 \textit{
	\begin{itemize}
	\item ...a computer network with 154 terminals.
	\item Huge sections of the rail network are out of action.
	\end{itemize}
}
\item countable noun \\
A radio or television  \textbf{network} is a company or group of companies that broadcasts radio or television programmes
throughout an area.
 \textit{
	\begin{itemize}
	\item An American network says it has obtained the recordings.
	\item ...Fuji Television Network, a highly successful commercial station.
	\end{itemize}
}
\item verb \\
When a television or radio programme \textbf{is networked} , it is broadcast at the same time by several different television companies.
 \textit{
	\begin{itemize}
	\item Lumsdon would like to see his programme sold and networked.
	\item He had once had his own networked chat show.
	\end{itemize}
}
\item verb \\
If you \textbf{network} , you try to meet new people who might be useful to you in your job .
 \textit{
	\begin{itemize}
	\item In business, it is important to network with as many people as possible.
	\end{itemize}
}
\end{enumerate}

\section*{furnish}
{\large \color{blue}  furnishes  furnishing  furnished  }
\subsection*{Explain}
\begin{enumerate}
\item verb \\
If you \textbf{furnish} a room or building, you put furniture and furnishings into it.
 \textit{
	\begin{itemize}
	\item Many proprietors try to furnish their hotels with antiques.
	\end{itemize}
}
\item verb \\
If you \textbf{furnish} someone \textbf{with} something, you provide or supply it.
 \textit{
	\begin{itemize}
	\item They'll be able to furnish you with the rest of the details.
	\end{itemize}
}
\end{enumerate}

\section*{peer}
{\large \color{blue}  peers  peering  peered  }
\subsection*{Explain}
\begin{enumerate}
\item verb \\
If you \textbf{peer}  \textbf{at} something, you look at it very hard , usually because it is difficult to see  clearly .
 \textit{
	\begin{itemize}
	\item I had been peering at a computer print-out that made no sense at all.
	\item He watched the Customs official peer into the driver's window.
	\end{itemize}
}
\item countable noun \\
In Britain, a \textbf{peer} is a member of the nobility who has or had the right to vote in the House of Lords .
 \textit{
	\begin{itemize}
	\item Lord Swan was made a life peer in 1981.
	\end{itemize}
}
\item countable noun \\
Your \textbf{peers} are the people who are the same age as you or who have the same status as you.
 \textit{
	\begin{itemize}
	\item ...children who are much cleverer than their peers.
	\item His engaging personality made him popular with his peers.
	\end{itemize}
}
\end{enumerate}

\section*{grieve}
{\large \color{blue}  grieves  grieving  grieved  }
\subsection*{Explain}
\begin{enumerate}
\item verb \\
If you \textbf{grieve}  \textbf{over} something, especially someone's death, you feel very sad about it.
 \textit{
	\begin{itemize}
	\item He's grieving over his dead wife and son.
	\item I didn't have any time to grieve.
	\item Margery's grieving family battled to come to terms with their loss.
	\end{itemize}
}
\item verb \\
If you \textbf{are grieved}  \textbf{by} something, it makes you unhappy or upset .
 \textit{
	\begin{itemize}
	\item He was deeply grieved by the sufferings of the common people.
	\item I was grieved to hear of the suicide of James.
	\item It grieved me to see the poor man in such distress.
	\end{itemize}
}
\end{enumerate}

\section*{pioneer}
{\large \color{blue}  pioneers  pioneering  pioneered  }
\subsection*{Explain}
\begin{enumerate}
\item countable noun \\
Someone who is referred to as a \textbf{pioneer} in a particular area of activity is one of the first people to be involved in it
and develop it.
 \textit{
	\begin{itemize}
	\item ...one of the leading pioneers of British photo journalism.
	\item ...Amelia Earhart, the pioneer pilot.
	\end{itemize}
}
\item verb \\
Someone who \textbf{pioneers} a new activity, invention , or process is one of the first people to do it.
 \textit{
	\begin{itemize}
	\item ...Professor Alec Jeffreys, who invented and pioneered DNA tests.
	\item Marine biologists have pioneered a technique to grow man-made coral gardens.
	\item ...the folk-tale writing style pioneered by Gabriel Garcia Marquez.
	\end{itemize}
}
\item countable noun \\
\textbf{Pioneers} are people who leave their own country or the place where they were living, and go and live in a place that has not been lived in before.
 \textit{
	\begin{itemize}
	\item ...abandoned settlements of early European pioneers.
	\end{itemize}
}
\end{enumerate}

\section*{highlight}
{\large \color{blue}  highlights  highlighting  highlighted  }
\subsection*{Explain}
\begin{enumerate}
\item verb \\
If someone or something \textbf{highlights} a point or problem , they emphasize it or make you think about it.
 \textit{
	\begin{itemize}
	\item ...a moving ballad which highlighted the plight of the homeless.
	\item Once again, the 'Free Press' prefers not to highlight these facts.
	\item Two events have highlighted the tensions in recent days.
	\end{itemize}
}
\item verb \\
To \textbf{highlight} a piece of text means to mark it in a different colour, either with a special type of pen or on a computer screen .
 \textit{
	\begin{itemize}
	\item Highlight the chosen area by clicking and holding down the left mouse button.
	\item ...the relevant maps with the route highlighted in yellow.
	\end{itemize}
}
\item countable noun \\
The \textbf{highlights}  \textbf{of} an event, activity, or period of time are the most interesting or exciting parts of it.
 \textit{
	\begin{itemize}
	\item ...a match that is likely to prove one of the highlights of the tournament.
	\item The highlight of my day used to be cooking Meg a meal when she came in from work.
	\item I don't want to watch the game now. I'll just wait till the highlights come on later
tonight.
	\end{itemize}
}
\item plural noun \\
\textbf{Highlights} in a person's hair are narrow  lighter areas made by dyeing or sunlight .
 \textit{
	\begin{itemize}
	\end{itemize}
}
\end{enumerate}

\section*{precedent}
{\large \color{blue}  precedents  }
\subsection*{Explain}
\begin{enumerate}
\item variable noun \\
If there is a \textbf{precedent}  \textbf{for} an action or event, it has happened before, and this can be regarded as an argument for doing it again.
 \textit{
	\begin{itemize}
	\item The trial could set an important precedent for dealing with large numbers of similar
cases.
	\item There are plenty of precedents in Hollywood for letting people out of contracts.
	\end{itemize}
}
\end{enumerate}

\section*{humiliate}
{\large \color{blue}  humiliates  humiliating  humiliated  }
\subsection*{Explain}
\begin{enumerate}
\item verb \\
To \textbf{humiliate} someone means to say or do something which makes them feel  ashamed or stupid .
 \textit{
	\begin{itemize}
	\item She had been beaten and humiliated by her husband.
	\item There are people out there who want to humiliate you.
	\end{itemize}
}
\end{enumerate}

\section*{priority}
{\large \color{blue}  priorities  }
\subsection*{Explain}
\begin{enumerate}
\item countable noun \\
If something is a \textbf{priority} , it is the most important thing you have to do or deal with, or must be done or dealt with before everything else you have to do.
 \textit{
	\begin{itemize}
	\item Being a parent is her first priority.
	\item The government's priority is to build more power plants.
	\item Getting your priorities in order is a good way to not waste energy on meaningless
pursuits.
	\end{itemize}
}
\item  \\
 give priority \textit{
	\begin{itemize}
	\end{itemize}
}
\item  \\
 take priority/has priority \textit{
	\begin{itemize}
	\end{itemize}
}
\end{enumerate}

\section*{immerse}
{\large \color{blue}  immerses  immersing  immersed  }
\subsection*{Explain}
\begin{enumerate}
\item verb \\
If you \textbf{immerse} yourself in something that you are doing, you become completely involved in it.
 \textit{
	\begin{itemize}
	\item Since then I've lived alone and immersed myself in my career.
	\end{itemize}
}
\item verb \\
If something \textbf{is immersed} in a liquid, someone puts it into the liquid so that it is completely covered.
 \textit{
	\begin{itemize}
	\item The electrodes are immersed in liquid.
	\end{itemize}
}
\end{enumerate}

\section*{response}
{\large \color{blue}  responses  }
\subsection*{Explain}
\begin{enumerate}
\item countable noun \\
Your \textbf{response} to an event or to something that is said is your reply or reaction to it.
 \textit{
	\begin{itemize}
	\item There has been no response to his remarks from the government.
	\item Your positive response will reinforce her actions.
	\item The meeting was called in response to a request from Venezuela.
	\end{itemize}
}
\end{enumerate}

\section*{manufacture}
{\large \color{blue}  manufactures  manufacturing  manufactured  }
\subsection*{Explain}
\begin{enumerate}
\item verb \\
To \textbf{manufacture} something means to make it in a factory , usually in large quantities .
 \textbf{Manufacture} is also a noun .
 \textit{
	\begin{itemize}
	\item They manufacture the class of plastics known as thermoplastic materials.
	\item The first three models are being manufactured at the factory in Ashton-under-Lyne.
	\item We import foreign manufactured goods.
	\item ...the manufacture of nuclear weapons.
	\item ...celebrating 90 years of car manufacture.
	\end{itemize}
}
\item countable noun \\
\textbf{Manufactures} are goods or products which have been made in a factory.
 \textit{
	\begin{itemize}
	\item ...a long-term rise in the share of manufactures in non-oil exports.
	\end{itemize}
}
\item verb \\
If you say that someone \textbf{manufactures} information, you are criticizing them because they invent information that is not true .
 \textit{
	\begin{itemize}
	\item According to the prosecution, the officers manufactured an elaborate story.
	\item He said the allegations were manufactured on the flimsiest evidence.
	\end{itemize}
}
\end{enumerate}

\section*{romance}
{\large \color{blue}  romances  romancing  romanced  }
\subsection*{Explain}
\begin{enumerate}
\item countable noun \\
A \textbf{romance} is a relationship between two people who are in love with each other but who are
not married to each other.
 \textit{
	\begin{itemize}
	\item After a whirlwind romance the couple announced their engagement in July.
	\item ...a holiday romance.
	\end{itemize}
}
\item uncountable noun \\
\textbf{Romance}  refers to the actions and feelings of people who are in love, especially behaviour which is very caring or affectionate .
 \textit{
	\begin{itemize}
	\item He still finds time for romance by cooking candlelit dinners for his girlfriend.
	\item He takes a rather sceptical view of love and romance.
	\end{itemize}
}
\item uncountable noun \\
You can refer to the pleasure and excitement of doing something new or exciting as \textbf{romance} .
 \textit{
	\begin{itemize}
	\item We want to recreate the romance and excitement that used to be part of rail journeys.
	\end{itemize}
}
\item countable noun \\
A \textbf{romance} is a novel or film about a love affair.
 \textit{
	\begin{itemize}
	\item Her taste in fiction was for chunky historical romances.
	\end{itemize}
}
\item uncountable noun \\
\textbf{Romance} is used to refer to novels about love affairs.
 \textit{
	\begin{itemize}
	\item Since taking up writing romance in 1967 she has brought out over fifty books.
	\end{itemize}
}
\item variable noun \\
A medieval  \textbf{romance} is a story about adventures such as battles and long journeys .
 \textit{
	\begin{itemize}
	\item ...Arthurian Romances.
	\end{itemize}
}
\item verb \\
When a man \textbf{romances} a woman, he has a love affair with her.
 \textit{
	\begin{itemize}
	\item He has romanced some of the world's most eligible women.
	\end{itemize}
}
\item adjective \\
\textbf{Romance} languages are languages such as French, Spanish, and Italian, which come from Latin.
 \textit{
	\begin{itemize}
	\end{itemize}
}
\end{enumerate}

\section*{mix}
{\large \color{blue}  mixes  mixing  mixed  }
\subsection*{Explain}
\begin{enumerate}
\item verb \\
If two substances \textbf{mix} or if you \textbf{mix} one substance \textbf{with} another, you stir or shake them together, or combine them in some other way, so that they become a single substance.
 \textit{
	\begin{itemize}
	\item Oil and water don't mix.
	\item It mixes easily with cold or hot water to make a tasty, filling drink.
	\item A quick stir will mix them thoroughly.
	\item Mix the cinnamon with the rest of the sugar.
	\item Mix the ingredients together slowly.
	\end{itemize}
}
\item verb \\
If you \textbf{mix} something, you prepare it by mixing other things together.
 \textit{
	\begin{itemize}
	\item He had spent several hours mixing cement.
	\item Are you sure I can't mix you a drink?
	\end{itemize}
}
\item variable noun \\
A \textbf{mix} is a powder containing all the substances that you need in order to make something such as a cake or a sauce . When you want to use it, you add liquid.
 \textit{
	\begin{itemize}
	\item ...packets of pizza dough mix.
	\item It was a packet mix.
	\end{itemize}
}
\item countable noun \\
A \textbf{mix}  \textbf{of} different things or people is two or more of them together.
 \textit{
	\begin{itemize}
	\item The story is a magical mix of fantasy and reality.
	\item We get a very representative mix of people.
	\end{itemize}
}
\item verb \\
If two things or activities do not \textbf{mix} or if one thing does not \textbf{mix}  \textbf{with} another, it is not a good idea to have them or do them together, because the result would be unpleasant or dangerous .
 \textit{
	\begin{itemize}
	\item Politics and sport don't mix.
	\item ...some of these pills that don't mix with drink.
	\item Ted managed to mix business with pleasure.
	\item The military has accused the clergy of mixing religion and politics.
	\end{itemize}
}
\item verb \\
If you \textbf{mix}  \textbf{with} other people, you meet them and talk to them. You can also  say that people \textbf{mix} .
 \textit{
	\begin{itemize}
	\item I ventured the idea that the secret of staying young was to mix with older people.
	\item People are supposed to mix, do you understand?
	\item When you came away you made a definite effort to mix.
	\end{itemize}
}
\item verb \\
When a record producer  \textbf{mixes} a piece of music, he or she puts together the various sounds that have been recorded
in order to make the finished record.
 \textit{
	\begin{itemize}
	\item They've been mixing tracks for a new album due out later this year.
	\end{itemize}
}
\item  \\
 mix it \textit{
	\begin{itemize}
	\end{itemize}
}
\end{enumerate}

\section*{sail}
{\large \color{blue}  sails  sailing  sailed  }
\subsection*{Explain}
\begin{enumerate}
\item countable noun \\
\textbf{Sails} are large pieces of material  attached to the mast of a ship. The wind blows against the sails and pushes the ship along.
 \textit{
	\begin{itemize}
	\item The white sails billow with the breezes they catch.
	\end{itemize}
}
\item verb \\
You say a ship \textbf{sails} when it moves over the sea .
 \textit{
	\begin{itemize}
	\item The trawler had sailed from the port of Zeebrugge.
	\item The Kruzenshtern is expected to sail for Boston this week.
	\end{itemize}
}
\item verb \\
If you \textbf{sail} a boat or if a boat \textbf{sails} , it moves across water using its sails.
 \textit{
	\begin{itemize}
	\item I shall get myself a little boat and sail her around the world.
	\item For nearly two hundred miles she sailed on, her sails hard with ice.
	\item She sails beautifully in winds over 60 knots.
	\end{itemize}
}
\item countable noun \\
The \textbf{sails} on a windmill are the long flat parts that are turned by the wind.
 \textit{
	\begin{itemize}
	\item ...a windmill, its sails turning in the breeze.
	\end{itemize}
}
\item verb \\
If a person or thing \textbf{sails}  somewhere , they move there smoothly and fairly quickly.
 \textit{
	\begin{itemize}
	\item We got into the lift and sailed to the top floor.
	\item The cabs sailed past.
	\end{itemize}
}
\item  \\
 to set sail \textit{
	\begin{itemize}
	\end{itemize}
}
\item  \\
 under sail \textit{
	\begin{itemize}
	\end{itemize}
}
\end{enumerate}

\section*{necessitate}
{\large \color{blue}  necessitates  necessitating  necessitated  }
\subsection*{Explain}
\begin{enumerate}
\item verb \\
If something \textbf{necessitates} an event , action , or situation , it makes it necessary.
 \textit{
	\begin{itemize}
	\item A prolonged drought had necessitated the introduction of water rationing.
	\item Frank was carrying out fuel-system tests which necessitated turning the booster pumps
off.
	\end{itemize}
}
\end{enumerate}

\section*{sequence}
{\large \color{blue}  sequences  }
\subsection*{Explain}
\begin{enumerate}
\item countable noun \\
A \textbf{sequence}  \textbf{of} events or things is a number of events or things that come one after another in a particular order.
 \textit{
	\begin{itemize}
	\item ...the sequence of events which led to the murder.
	\item ...a dazzling sequence of novels by John Updike.
	\end{itemize}
}
\item countable noun \\
A particular \textbf{sequence} is a particular order in which things happen or are arranged.
 \textit{
	\begin{itemize}
	\item ...the colour sequence yellow, orange, purple, blue, green and white.
	\item The chronological sequence gives the book an element of structure.
	\end{itemize}
}
\item countable noun \\
A film \textbf{sequence} is a part of a film that shows a single set of actions.
 \textit{
	\begin{itemize}
	\item The best sequence in the film occurs when Roth stops at a house he used to live in.
	\end{itemize}
}
\item countable noun \\
A gene  \textbf{sequence} or a DNA \textbf{sequence} is the order in which the elements making up a particular gene are combined .
 \textit{
	\begin{itemize}
	\item The project is nothing less than mapping every gene sequence in the human body.
	\item ...the complete DNA sequence of the human genome.
	\end{itemize}
}
\item verb \\
To \textbf{sequence} genes is to determine the order in which the elements that make them up are combined.
 \textit{
	\begin{itemize}
	\item The technique offers a means of sequencing the human genome much more quickly.
	\end{itemize}
}
\end{enumerate}

\section*{perplex}
{\large \color{blue}  perplexes  perplexing  perplexed  }
\subsection*{Explain}
\begin{enumerate}
\item verb \\
If something \textbf{perplexes} you, it confuses and worries you because you do not understand it or because it causes you difficulty .
 \textit{
	\begin{itemize}
	\item It perplexed him because he was tackling it the wrong way.
	\end{itemize}
}
\end{enumerate}

\section*{shilling}
{\large \color{blue}  shillings  }
\subsection*{Explain}
\begin{enumerate}
\item countable noun \\
A \textbf{shilling} was a unit of money that was used in Britain until 1971 which was the equivalent of 5 p . There were twenty shillings in a pound.
 \textit{
	\begin{itemize}
	\end{itemize}
}
\end{enumerate}

\section*{provide}
{\large \color{blue}  provides  providing  provided  }
\subsection*{Explain}
\begin{enumerate}
\item verb \\
If you \textbf{provide} something that someone needs or wants , or if you \textbf{provide} them \textbf{with} it, you give it to them or make it available to them.
 \textit{
	\begin{itemize}
	\item I'll be glad to provide a copy of this.
	\item They would not provide any details.
	\item The government was not in a position to provide them with food.
	\end{itemize}
}
\item verb \\
If a law or agreement  \textbf{provides}  \textbf{that} something will happen, it states that it will happen.
 \textit{
	\begin{itemize}
	\item The treaty provides that, by the end of the century, the United States must have
removed its bases.
	\item The Act provides that only the parents of a child have a responsibility for that
child's financial support.
	\end{itemize}
}
\end{enumerate}

\section*{sir}
{\large \color{blue}  sirs  }
\subsection*{Explain}
\begin{enumerate}
\item countable noun \\
People sometimes  say  \textbf{sir} as a very formal and polite way of addressing a man whose name they do not know or a man of superior  rank . For example , a shop  assistant  might address a male  customer as \textbf{sir} .
 \textit{
	\begin{itemize}
	\item Excuse me sir, but would you mind telling me what sort of car that is?
	\item Good afternoon to you, sir.
	\end{itemize}
}
\item title noun \\
\textbf{Sir} is the title used in front of the name of a knight or baronet.
 \textit{
	\begin{itemize}
	\item She introduced me to Sir Tobias and Lady Clarke.
	\end{itemize}
}
\item convention \\
You use the expression  \textbf{Dear sir} at the beginning of a formal letter or a business letter when you are writing to a man. You use \textbf{Dear sirs} when you are writing to an organization .
 \textit{
	\begin{itemize}
	\item Dear Sir, Your letter of the 9th October has been referred to us.
	\end{itemize}
}
\end{enumerate}

\section*{purify}
{\large \color{blue}  purifies  purifying  purified  }
\subsection*{Explain}
\begin{enumerate}
\item verb \\
If you \textbf{purify} a substance, you make it pure by removing any harmful , dirty , or inferior substances from it.
 \textit{
	\begin{itemize}
	\item I take wheat and yeast tablets daily to purify the blood.
	\item Only purified water is used.
	\end{itemize}
}
\end{enumerate}

\section*{solidarity}
{\large \color{blue}  }
\subsection*{Explain}
\begin{enumerate}
\item uncountable noun \\
If a group of people show  \textbf{solidarity} , they show support for each other or for another group, especially in political or international  affairs .
 \textit{
	\begin{itemize}
	\item Supporters want to march tomorrow to show solidarity with their leaders.
	\end{itemize}
}
\end{enumerate}

\section*{reassure}
{\large \color{blue}  reassures  reassuring  reassured  }
\subsection*{Explain}
\begin{enumerate}
\item verb \\
If you \textbf{reassure} someone, you say or do things to make them stop worrying about something.
 \textit{
	\begin{itemize}
	\item I tried to reassure her, 'Don't worry about it. We won't let it happen again.'
	\item She just reassured me that everything was fine.
	\end{itemize}
}
\end{enumerate}

\section*{sympathy}
{\large \color{blue}  sympathies  }
\subsection*{Explain}
\begin{enumerate}
\item uncountable noun \\
If you have \textbf{sympathy} for someone who is in a bad  situation , you are sorry for them, and show this in the way you behave towards them.
 \textit{
	\begin{itemize}
	\item We expressed our sympathy for her loss.
	\item I have had very little help from doctors and no sympathy whatsoever.
	\item I wanted to express my sympathies on your resignation.
	\end{itemize}
}
\item uncountable noun \\
If you have \textbf{sympathy} with someone's ideas or opinions , you agree with them.
 \textit{
	\begin{itemize}
	\item I have some sympathy with this point of view.
	\item Lithuania still commands considerable international sympathy for its cause.
	\item He had strong left-wing sympathies.
	\end{itemize}
}
\item uncountable noun \\
If you take some action \textbf{in}  \textbf{sympathy}  \textbf{with} someone else, you do it in order to show that you support them.
 \textit{
	\begin{itemize}
	\item Several hundred workers struck in sympathy with their colleagues.
	\item Milne resigned in sympathy because of the way Donald had been treated.
	\item ...calls for sympathy strikes.
	\end{itemize}
}
\end{enumerate}

\section*{reconcile}
{\large \color{blue}  reconciles  reconciling  reconciled  }
\subsection*{Explain}
\begin{enumerate}
\item verb \\
If you \textbf{reconcile} two beliefs , facts , or demands that seem to be opposed or completely different , you find a way in which they can both be true or both be successful .
 \textit{
	\begin{itemize}
	\item It's difficult to reconcile the demands of my job and the desire to be a good father.
	\item We suggest that it is possible to reconcile these apparently opposing perspectives.
	\item She struggles to reconcile the demands and dangers of her work with her role as a
mother and wife.
	\end{itemize}
}
\item passive verb \\
If you \textbf{are reconciled}  \textbf{with} someone, you become friendly with them again after a quarrel or disagreement .
 \textit{
	\begin{itemize}
	\item He never believed he and Susan would be reconciled.
	\item Devlin was reconciled with the Catholic Church in his last few days.
	\end{itemize}
}
\item verb \\
If you \textbf{reconcile} two people, you make them become friends again after a quarrel or disagreement.
 \textit{
	\begin{itemize}
	\item ...my attempt to reconcile him with Toby.
	\end{itemize}
}
\item verb \\
If you \textbf{reconcile}  \textbf{yourself}  \textbf{to} an unpleasant situation , you accept it, although it does not make you happy to do so.
 \textit{
	\begin{itemize}
	\item She had reconciled herself to never seeing him again.
	\end{itemize}
}
\end{enumerate}

\section*{tennis}
{\large \color{blue}  }
\subsection*{Explain}
\begin{enumerate}
\item uncountable noun \\
\textbf{Tennis} is a game played by two or four players on a rectangular court. The players use an oval racket with strings across it to hit a ball over a net across the middle of the court.
 \textit{
	\begin{itemize}
	\end{itemize}
}
\end{enumerate}

\section*{render}
{\large \color{blue}  renders  rendering  rendered  }
\subsection*{Explain}
\begin{enumerate}
\item verb \\
You can use \textbf{render} with an adjective that describes a particular state to say that someone or something is changed into that state. For example , if someone or something makes a thing harmless , you can say that they \textbf{render} it harmless.
 \textit{
	\begin{itemize}
	\item It contained so many errors as to render it worthless.
	\item Many factories are rendered obsolete by the competitive pressures of the world market.
	\end{itemize}
}
\item verb \\
If you \textbf{render} someone help or service, you help them.
 \textit{
	\begin{itemize}
	\item He had a chance to render some service to his country.
	\item Any assistance you can render him will be appreciated.
	\item The money was in fact payment by the CIA for services rendered.
	\end{itemize}
}
\item verb \\
When a jury or authority \textbf{renders} a verdict, decision , or response , they announce it.
 \textit{
	\begin{itemize}
	\item The Board had been slow to render its verdict.
	\end{itemize}
}
\item verb \\
To \textbf{render} something in a particular language or in a particular way means to translate it into
that language or in that way.
 \textit{
	\begin{itemize}
	\item ...'Zensho shimasu,' which the translator rendered literally as, 'I will do my best.'.
	\item All the signs and announcements were rendered in English and Spanish.
	\end{itemize}
}
\item verb \\
To \textbf{render} a wall means to cover it with a layer of plaster or cement , usually in order to protect it.
 \textit{
	\begin{itemize}
	\end{itemize}
}
\end{enumerate}

\section*{tongue}
{\large \color{blue}  tongues  }
\subsection*{Explain}
\begin{enumerate}
\item countable noun \\
Your \textbf{tongue} is the soft movable part inside your mouth which you use for tasting , eating, and speaking.
 \textit{
	\begin{itemize}
	\item I walked over to the mirror and stuck my tongue out.
	\item She ran her tongue around her lips.
	\end{itemize}
}
\item countable noun \\
You can use \textbf{tongue} to refer to the kind of things that a person says .
 \textit{
	\begin{itemize}
	\item ...her sharp wit and quick tongue.
	\item She had a nasty tongue, but I liked her.
	\end{itemize}
}
\item countable noun \\
A \textbf{tongue} is a language.
 \textit{
	\begin{itemize}
	\item The French feel passionately about their native tongue.
	\end{itemize}
}
\item variable noun \\
\textbf{Tongue} is the cooked tongue of an ox or sheep . It is usually eaten cold .
 \textit{
	\begin{itemize}
	\end{itemize}
}
\item countable noun \\
The \textbf{tongue} of a shoe or boot is the piece of leather which is underneath the laces.
 \textit{
	\begin{itemize}
	\end{itemize}
}
\item countable noun \\
A \textbf{tongue of} something such as fire or land is a long thin piece of it.
 \textit{
	\begin{itemize}
	\item A yellow tongue of flame shot upwards.
	\item ...a silver, frozen tongue of water.
	\end{itemize}
}
\item  \\
 tongue in cheek \textit{
	\begin{itemize}
	\end{itemize}
}
\item  \\
 to hold your tongue \textit{
	\begin{itemize}
	\end{itemize}
}
\item  \\
 to get your tongue around something \textit{
	\begin{itemize}
	\end{itemize}
}
\item  \\
 slip of the tongue \textit{
	\begin{itemize}
	\end{itemize}
}
\end{enumerate}

\section*{scrape}
{\large \color{blue}  scrapes  scraping  scraped  }
\subsection*{Explain}
\begin{enumerate}
\item verb \\
If you \textbf{scrape} something from a surface, you remove it, especially by pulling a sharp object over the surface.
 \textit{
	\begin{itemize}
	\item She went round the car scraping the frost off the windows.
	\item Young children were trying to scrape up some of the rice that spilled from the sacks.
	\end{itemize}
}
\item verb \\
If something \textbf{scrapes} against something else or if someone or something \textbf{scrapes} something else, it rubs against it, making a noise or causing slight damage.
 \textbf{Scrape} is also a noun .
 \textit{
	\begin{itemize}
	\item The only sound is that of knives and forks scraping against china.
	\item The cab driver struggled with her luggage, scraping a bag against the door as they
came in.
	\item The car hurtled past us, scraping the wall and screeching to a halt.
	\item There was a scraping sound as she dragged the heels of her shoes along the pavement.
	\item From the other side of the door came the scrape of a guard's boot.
	\end{itemize}
}
\item verb \\
If you \textbf{scrape} a part of your body, you accidentally rub it against something hard and rough, and damage it slightly .
 \textit{
	\begin{itemize}
	\item She stumbled and fell, scraping her palms and knees.
	\end{itemize}
}
\item countable noun \\
If you are \textbf{in} a \textbf{scrape} , you are in a difficult  situation which you have caused yourself.
 \textit{
	\begin{itemize}
	\item We got into terrible scrapes.
	\end{itemize}
}
\end{enumerate}

\section*{surprise}
{\large \color{blue}  surprises  surprising  surprised  }
\subsection*{Explain}
\begin{enumerate}
\item countable noun \\
A \textbf{surprise} is an unexpected event, fact , or piece of news .
 \textbf{Surprise} is also an adjective .
 \textit{
	\begin{itemize}
	\item I have a surprise for you: We are moving to Switzerland!
	\item It may come as a surprise to some that a child is born with many skills.
	\item It is perhaps no surprise to see another 80s singing star attempting a comeback.
	\item Baxter arrived here this afternoon, on a surprise visit.
	\item German intelligence expected Japan to launch a surprise attack on the U.S..
	\end{itemize}
}
\item uncountable noun \\
\textbf{Surprise} is the feeling that you have when something unexpected happens .
 \textit{
	\begin{itemize}
	\item The Foreign Office in London has expressed surprise at these allegations.
	\item 'You mean he's going to vote against her?' Scobie asked in surprise.
	\item I started working hard for the first time in my life. To my surprise, I liked it.
	\end{itemize}
}
\item verb \\
If something \textbf{surprises} you, it gives you a feeling of surprise.
 \textit{
	\begin{itemize}
	\item We'll solve the case ourselves and surprise everyone.
	\item It surprised me that someone of her experience should make those mistakes.
	\item It wouldn't surprise me if there was such chaos after this election that another
had to be held.
	\item They were served lamb and she surprised herself by eating greedily.
	\end{itemize}
}
\item verb \\
If you \textbf{surprise} someone, you give them, tell them, or do something pleasant that they are not expecting .
 \textit{
	\begin{itemize}
	\item Surprise a new neighbour with one of your favourite home-made dishes.
	\end{itemize}
}
\item countable noun \\
If you describe someone or something as a \textbf{surprise} , you mean that they are very good or pleasant although you were not expecting this.
 \textit{
	\begin{itemize}
	\item She was one of the surprises of the World Championships three months ago.
	\item My father decided to slip a little extra spending money into my purse as a surprise.
	\end{itemize}
}
\item verb \\
If you \textbf{surprise} someone, you attack , capture, or find them when they are not expecting it.
 \textit{
	\begin{itemize}
	\item Marlborough surprised the French and Bavarian armies near the village of Blenheim.
	\end{itemize}
}
\item  \\
 surprise, surprise \textit{
	\begin{itemize}
	\end{itemize}
}
\item  \\
 surprise, surprise \textit{
	\begin{itemize}
	\end{itemize}
}
\item  \\
 to take someone by surprise \textit{
	\begin{itemize}
	\end{itemize}
}
\end{enumerate}

\section*{unity}
{\large \color{blue}  }
\subsection*{Explain}
\begin{enumerate}
\item uncountable noun \\
\textbf{Unity} is the state of different areas or groups being joined  together to form a single country or organization .
 \textit{
	\begin{itemize}
	\item There is support for economic unity in trade and industry to promote growth and prosperity.
	\item ...German unity.
	\end{itemize}
}
\item uncountable noun \\
When there is \textbf{unity} , people are in agreement and act together for a particular purpose.
 \textit{
	\begin{itemize}
	\item ...a renewed unity of purpose.
	\item Speakers at the rally mouthed sentiments of unity.
	\item The choice was meant to create an impression of party unity.
	\end{itemize}
}
\end{enumerate}

\section*{terrify}
{\large \color{blue}  terrifies  terrifying  terrified  }
\subsection*{Explain}
\begin{enumerate}
\item verb \\
If something \textbf{terrifies} you, it makes you feel  extremely frightened.
 \textit{
	\begin{itemize}
	\item Flying terrifies him.
	\item The thought of dying slowly and painfully terrified me.
	\end{itemize}
}
\end{enumerate}

\section*{web}
{\large \color{blue}  webs  }
\subsection*{Explain}
\begin{enumerate}
\item countable noun \\
A \textbf{web} is the thin net made by a spider from a sticky substance which it produces in its body.
 \textit{
	\begin{itemize}
	\item ...the spider's web in the window.
	\end{itemize}
}
\item countable noun \\
A \textbf{web} is a complicated pattern of connections or relationships , sometimes considered as an obstacle or a danger .
 \textit{
	\begin{itemize}
	\item He's forced to untangle a complex web of financial dealings.
	\item They accused him of weaving a web of lies and deceit.
	\item ...the complex web of life on this planet.
	\end{itemize}
}
\item proper noun \\
\textbf{The Web} is the same as the World Wide Web .
 \textit{
	\begin{itemize}
	\end{itemize}
}
\end{enumerate}

\section*{unify}
{\large \color{blue}  unifies  unifying  unified  }
\subsection*{Explain}
\begin{enumerate}
\item verb \\
If someone \textbf{unifies}  different things or parts, or if the things or parts \textbf{unify} , they are brought  together to form one thing.
 \textit{
	\begin{itemize}
	\item A flexible retirement age is being considered by Ministers to unify men's and women's
pension rights.
	\item He said he would seek to unify the party and win the next general election.
	\item The plan has been for the rival armies to unify, and then to hold elections.
	\item Both sides say they want to re-unify their country, which has been divided since
the end of the Second World War.
	\item The former British colony unified with the north after the British withdrawal.
	\end{itemize}
}
\end{enumerate}

\section*{year}
{\large \color{blue}  years  }
\subsection*{Explain}
\begin{enumerate}
\item countable noun \\
A \textbf{year} is a period of twelve months or 365 or 366 days, beginning on the first of January and ending on the thirty-first of December.
 \textit{
	\begin{itemize}
	\item The year was 1840.
	\item We had an election last year.
	\item ...the number of people on the planet by the year 2050.
	\end{itemize}
}
\item countable noun \\
A \textbf{year} is any period of twelve months.
 \textit{
	\begin{itemize}
	\item The museums attract more than two and a half million visitors a year.
	\item She's done quite a bit of work this past year.
	\item The school has been empty for ten years.
	\end{itemize}
}
\item countable noun \\
\textbf{Year} is used to refer to the age of a person. For example , if someone or something is twenty  \textbf{years} old or twenty \textbf{years} of age, they have lived or existed for twenty years.
 \textit{
	\begin{itemize}
	\item He's 58 years old.
	\item I've been in trouble since I was eleven years of age.
	\item This column is ten years old today.
	\end{itemize}
}
\item countable noun \\
A school \textbf{year} or academic  \textbf{year} is the period of time in each twelve months when schools or universities are open
and students are studying there. In Britain and the United  States , the school year starts in September .
 \textit{
	\begin{itemize}
	\item ...the 1990/91 academic year.
	\item The twins didn't have to repeat their second year at school.
	\end{itemize}
}
\item countable noun \\
You can refer to someone who is, for example, in their first year at school or university
as a first \textbf{year} .
 \textit{
	\begin{itemize}
	\item The first years and second years got a choice of French, German and Spanish.
	\end{itemize}
}
\item countable noun \\
A financial or business \textbf{year} is an exact period of twelve months which businesses or institutions use as a basis for organizing their finances .
 \textit{
	\begin{itemize}
	\item He announced big tax increases for the next two financial years.
	\item The company admits it will make a loss for the year ending September.
	\end{itemize}
}
\item plural noun \\
You can use \textbf{years} to emphasize that you are referring to a long time.
 \textit{
	\begin{itemize}
	\item I haven't laughed so much in years.
	\item It took me years to fully recover.
	\item People hold onto letters for years and years.
	\end{itemize}
}
\item plural noun \\
You can refer to the time you spend in a place or doing an activity as your \textbf{years} there or your \textbf{years} of doing that activity.
 \textit{
	\begin{itemize}
	\item The joy turned to tragedy during his years in Cyprus.
	\item ...his years as Director of the Manchester City Art Gallery.
	\end{itemize}
}
\item  \\
 year after year \textit{
	\begin{itemize}
	\end{itemize}
}
\item  \\
 year by year \textit{
	\begin{itemize}
	\end{itemize}
}
\item  \\
 year in, year out \textit{
	\begin{itemize}
	\end{itemize}
}
\item  \\
 a man of his years/a woman of her years \textit{
	\begin{itemize}
	\end{itemize}
}
\item  \\
 put years on sb \textit{
	\begin{itemize}
	\end{itemize}
}
\item  \\
 all year round \textit{
	\begin{itemize}
	\end{itemize}
}
\item  \\
 take years off sb \textit{
	\begin{itemize}
	\end{itemize}
}
\end{enumerate}

\section*{upset}
{\large \color{blue}  upsets  upsetting  upset  }
\subsection*{Explain}
\begin{enumerate}
\item adjective \\
If you are \textbf{upset} , you are unhappy or disappointed because something unpleasant has happened to you.
 \textbf{Upset} is also a noun .
 \textit{
	\begin{itemize}
	\item After she died I felt very, very upset.
	\item Marta looked upset.
	\item She sounded upset when I said you couldn't give her an appointment.
	\item They are terribly upset by the break-up of their parents' marriage.
	\item ...stress and other emotional upsets.
	\end{itemize}
}
\item verb \\
If something \textbf{upsets} you, it makes you feel  worried or unhappy.
 \textit{
	\begin{itemize}
	\item The whole incident had upset me and my fiancee terribly.
	\item She warned me not to say anything to upset him.
	\item Don't upset yourself, Ida.
	\end{itemize}
}
\item verb \\
If events  \textbf{upset} something such as a procedure or a state of affairs , they cause it to go  wrong .
 \textbf{Upset} is also a noun.
 \textit{
	\begin{itemize}
	\item ...a deal that would upset the balance of power in the world's gold markets.
	\item House prices are easily upset by factors which have nothing to do with property.
	\item Markets are very sensitive to any upsets in the economic machine.
	\end{itemize}
}
\item verb \\
If you \textbf{upset} an object, you accidentally knock or push it over so that it scatters over a large area.
 \textit{
	\begin{itemize}
	\item Don't upset the piles of sheets under the box.
	\item ...bumping into him, and almost upsetting the ginger ale.
	\end{itemize}
}
\item countable noun \\
A stomach  \textbf{upset} is a slight  illness in your stomach caused by an infection or by something that you have eaten .
 \textbf{Upset} is also an adjective .
 \textit{
	\begin{itemize}
	\item Paul was unwell last night with a stomach upset.
	\item It wasn't anything serious. A mild stomach upset, that's all.
	\item Larry is suffering from an upset stomach.
	\end{itemize}
}
\end{enumerate}

\section*{action}
{\large \color{blue}  actions  actioning  actioned  }
\subsection*{Explain}
\begin{enumerate}
\item uncountable noun \\
\textbf{Action} is doing something for a particular purpose.
 \textit{
	\begin{itemize}
	\item The government is taking emergency action to deal with a housing crisis.
	\item What was needed, he said, was decisive action to halt what he called these savage
crimes.
	\end{itemize}
}
\item countable noun \\
An \textbf{action} is something that you do on a particular occasion .
 \textit{
	\begin{itemize}
	\item As always, Peter had a reason for his action.
	\item Jack was the sort of man who did not like his actions questioned.
	\end{itemize}
}
\item countable noun \\
To bring a legal \textbf{action} against someone means to bring a case against them in a court of law.
 \textit{
	\begin{itemize}
	\item Two leading law firms are to prepare legal actions against tobacco companies.
	\item ...a libel action brought by one of France's bureaucrats.
	\end{itemize}
}
\item uncountable noun \\
The \textbf{action} of a chemical is the way in which it works, or the effects that it has.
 \textit{
	\begin{itemize}
	\item Her description of the nature and action of poisons is amazingly accurate.
	\end{itemize}
}
\item singular noun \\
\textbf{The action} is all the important and exciting things that are happening in a situation.
 \textit{
	\begin{itemize}
	\item Hollywood is where the action is now.
	\end{itemize}
}
\item uncountable noun \\
The fighting which takes place in a war can be referred to as \textbf{action} .
 \textit{
	\begin{itemize}
	\item ...a murderous war that would see millions die, as a result of direct military action.
	\item 13 soldiers were killed and 10 wounded in action.
	\end{itemize}
}
\item adjective \\
An \textbf{action}  movie is a film in which a lot of dangerous and exciting things happen . An \textbf{action}  hero is the main character in one of these films.
 \textit{
	\begin{itemize}
	\end{itemize}
}
\item verb \\
If you \textbf{action} something that needs to be done, you deal with it.
 \textit{
	\begin{itemize}
	\item Documents can be actioned, or filed immediately.
	\end{itemize}
}
\item  \\
 out of action \textit{
	\begin{itemize}
	\end{itemize}
}
\item  \\
 a piece of the action \textit{
	\begin{itemize}
	\end{itemize}
}
\item  \\
 put sth into action \textit{
	\begin{itemize}
	\end{itemize}
}
\end{enumerate}

\section*{accomplish}
{\large \color{blue}  accomplishes  accomplishing  accomplished  }
\subsection*{Explain}
\begin{enumerate}
\item verb \\
If you \textbf{accomplish} something, you succeed in doing it.
 \textit{
	\begin{itemize}
	\item If we'd all work together, I think we could accomplish our goal.
	\item They are sceptical about how much will be accomplished by legislation.
	\end{itemize}
}
\end{enumerate}

\section*{advantage}
{\large \color{blue}  advantages  }
\subsection*{Explain}
\begin{enumerate}
\item countable noun \\
An \textbf{advantage} is something that puts you in a better position than other people.
 \textit{
	\begin{itemize}
	\item They are deliberately flouting the law in order to obtain an advantage over their
competitors.
	\item A good crowd will be a definite advantage to me and the rest of the team.
	\end{itemize}
}
\item uncountable noun \\
\textbf{Advantage} is the state of being in a better position than others who are competing against you.
 \textit{
	\begin{itemize}
	\item The family hold a position of social and economic advantage in the region.
	\end{itemize}
}
\item countable noun \\
An \textbf{advantage} is a way in which one thing is better than another.
 \textit{
	\begin{itemize}
	\item The great advantage of home-grown oranges is their magnificent flavour.
	\item This custom-built kitchen has many advantages over a standard one.
	\end{itemize}
}
\item  \\
 take advantage of something \textit{
	\begin{itemize}
	\end{itemize}
}
\item  \\
 take advantage of someone \textit{
	\begin{itemize}
	\end{itemize}
}
\item  \\
 to one's advantage \textit{
	\begin{itemize}
	\end{itemize}
}
\item  \\
 to best advantage \textit{
	\begin{itemize}
	\end{itemize}
}
\end{enumerate}

\section*{audit}
{\large \color{blue}  audits  auditing  audited  }
\subsection*{Explain}
\begin{enumerate}
\item verb \\
When an accountant \textbf{audits} an organization's accounts, he or she examines the accounts officially in order to make sure that they have been done correctly.
 \textbf{Audit} is also a noun .
 \textit{
	\begin{itemize}
	\item Each year they audit our accounts and certify them as being true and fair.
	\item The bank first learned of the problem when it carried out an internal audit.
	\end{itemize}
}
\end{enumerate}

\section*{automation}
{\large \color{blue}  }
\subsection*{Explain}
\begin{enumerate}
\item noun \\
1.  2.  \textit{
	\begin{itemize}
	\end{itemize}
}
\end{enumerate}

\section*{coil}
{\large \color{blue}  coils  coiling  coiled  }
\subsection*{Explain}
\begin{enumerate}
\item countable noun \\
A \textbf{coil}  \textbf{of} rope or wire is a length of it that has been wound into a series of loops.
 \textit{
	\begin{itemize}
	\item Tod shook his head angrily and slung the coil of rope over his shoulder.
	\item The steel arrives at the factory in coils.
	\end{itemize}
}
\item countable noun \\
A \textbf{coil} is one loop in a series of loops.
 \textit{
	\begin{itemize}
	\item Pythons kill by tightening their coils so that their victim cannot breathe.
	\end{itemize}
}
\item countable noun \\
A \textbf{coil} is a thick spiral of wire through which an electrical current  passes .
 \textit{
	\begin{itemize}
	\end{itemize}
}
\item countable noun \\
In a vehicle , the \textbf{coil} is the part on a petrol engine that sends  electricity to the spark plugs.
 \textit{
	\begin{itemize}
	\end{itemize}
}
\item countable noun \\
The \textbf{coil} is a contraceptive device used by women. It is fitted  inside a woman's womb , usually for several months or years .
 \textit{
	\begin{itemize}
	\end{itemize}
}
\item verb \\
If you \textbf{coil} something, you wind it into a series of loops or into the shape of a ring . If it \textbf{coils}  \textbf{around} something, it forms loops or a ring.
 \textbf{Coil up}  means the same as coil .
 \textit{
	\begin{itemize}
	\item He turned off the water and began to coil the hose.
	\item Louisa was dancing, spinning by herself, her skirt flying out and coiling around
her feet.
	\item A huge rattlesnake lay coiled on the blanket.
	\item Once we have the wire, we can coil it up into the shape of a spring.
	\item Her hair was coiled up on top of her head.
	\end{itemize}
}
\end{enumerate}

\section*{autonomy}
{\large \color{blue}  }
\subsection*{Explain}
\begin{enumerate}
\item uncountable noun \\
\textbf{Autonomy} is the control or government of a country, organization, or group by itself rather
than by others.
 \textit{
	\begin{itemize}
	\item Activists stepped up their demands for local autonomy last month.
	\end{itemize}
}
\item uncountable noun \\
\textbf{Autonomy} is the ability to make your own decisions about what to do rather than being influenced by someone else or told what to do.
 \textit{
	\begin{itemize}
	\item Each of the area managers enjoys considerable autonomy in the running of his or her
own area.
	\end{itemize}
}
\end{enumerate}

\section*{compose}
{\large \color{blue}  composes  composing  composed  }
\subsection*{Explain}
\begin{enumerate}
\item verb \\
The things that something \textbf{is composed}  \textbf{of} are its parts or members. The separate things that \textbf{compose} something are the parts or members that form it.
 \textit{
	\begin{itemize}
	\item The force would be composed of troops from NATO countries.
	\item Protein molecules compose all the complex working parts of living cells.
	\item They agreed to form a council composed of leaders of the rival factions.
	\end{itemize}
}
\item verb \\
When someone \textbf{composes} a piece of music, they write it.
 \textit{
	\begin{itemize}
	\item Vivaldi composed a large number of very fine concertos.
	\item Cale also uses electronic keyboards to compose.
	\end{itemize}
}
\item verb \\
If you \textbf{compose} something such as a letter , poem , or speech , you write it, often using a lot of concentration or skill .
 \textit{
	\begin{itemize}
	\item He started at once to compose a reply to Anna.
	\item The document composed in Philadelphia transformed the confederation of sovereign
states into a national government.
	\end{itemize}
}
\item verb \\
If you \textbf{compose} a picture or image , you arrange it in an attractive and artistic  way .
 \textit{
	\begin{itemize}
	\item Anthony dismounted with his camera and walked away from the walls to compose a shot.
	\item The drawing is beautifully composed.
	\end{itemize}
}
\item verb \\
If you \textbf{compose}  \textbf{yourself} or if you \textbf{compose} your features , you succeed in becoming calm after you have been angry , excited , or upset .
 \textit{
	\begin{itemize}
	\item She quickly composed herself as the car started off.
	\item Then he composed his features, took Godwin's hand awkwardly and began to usher him
from the office.
	\end{itemize}
}
\end{enumerate}

\section*{blood}
{\large \color{blue}  }
\subsection*{Explain}
\begin{enumerate}
\item uncountable noun \\
\textbf{Blood} is the red liquid that flows inside your body, which you can see if you cut yourself.
 \textit{
	\begin{itemize}
	\end{itemize}
}
\item uncountable noun \\
You can use \textbf{blood} to refer to the race or social class of someone's parents or ancestors .
 \textit{
	\begin{itemize}
	\item There was Greek blood in his veins.
	\item He was of noble blood, and an officer.
	\end{itemize}
}
\item  \\
 bad blood \textit{
	\begin{itemize}
	\end{itemize}
}
\item  \\
 bay for blood \textit{
	\begin{itemize}
	\end{itemize}
}
\item  \\
 blue blood \textit{
	\begin{itemize}
	\end{itemize}
}
\item  \\
 to make someone's blood boil \textit{
	\begin{itemize}
	\end{itemize}
}
\item  \\
 in cold blood \textit{
	\begin{itemize}
	\end{itemize}
}
\item  \\
 to make your blood run cold \textit{
	\begin{itemize}
	\end{itemize}
}
\item  \\
 blood on one's hands \textit{
	\begin{itemize}
	\end{itemize}
}
\item  \\
 in one's blood \textit{
	\begin{itemize}
	\end{itemize}
}
\item  \\
 new blood/fresh blood/young blood \textit{
	\begin{itemize}
	\end{itemize}
}
\item  \\
 get blood out of a stone/get blood from a stone \textit{
	\begin{itemize}
	\end{itemize}
}
\item  \\
 to sweat blood \textit{
	\begin{itemize}
	\end{itemize}
}
\item phrase \\
If you say that someone \textbf{draws first blood} , you mean that they have had a success at the beginning of a competition or conflict .
 \textit{
	\begin{itemize}
	\item The home side drew first blood with a penalty from Murray Strang.
	\end{itemize}
}
\item  \\
 blood, sweat, and tears \textit{
	\begin{itemize}
	\end{itemize}
}
\item  \\
 blood is thicker than water \textit{
	\begin{itemize}
	\end{itemize}
}
\end{enumerate}

\section*{conquer}
{\large \color{blue}  conquers  conquering  conquered  }
\subsection*{Explain}
\begin{enumerate}
\item verb \\
If one country or group of people \textbf{conquers} another, they take complete control of their land.
 \textit{
	\begin{itemize}
	\item During 1936, Mussolini conquered Abyssinia.
	\item Early in the eleventh century the whole of England was again conquered by the Vikings.
	\end{itemize}
}
\item verb \\
If you \textbf{conquer} something such as a problem , you succeed in ending it or dealing with it successfully.
 \textit{
	\begin{itemize}
	\item I was certain that love was quite enough to conquer our differences.
	\item He has never conquered his addiction to smoking.
	\item ...the first man in history to conquer Everest.
	\end{itemize}
}
\end{enumerate}

\section*{buffet}
{\large \color{blue}  buffets  buffeting  buffeted  }
\subsection*{Explain}
\begin{enumerate}
\item countable noun \\
A \textbf{buffet} is a meal of cold food that is displayed on a long table at a party or public  occasion . Guests usually serve themselves from the table.
 \textit{
	\begin{itemize}
	\item ...a buffet lunch.
	\item A cold buffet had been laid out in the dining-room.
	\end{itemize}
}
\item countable noun \\
A \textbf{buffet} is a café, usually in a hotel or station .
 \textit{
	\begin{itemize}
	\item We sat in the station buffet sipping tea.
	\end{itemize}
}
\item countable noun \\
On a train , the \textbf{buffet} or the \textbf{buffet car} is the carriage or car where meals and snacks are sold .
 \textit{
	\begin{itemize}
	\end{itemize}
}
\item verb \\
If something \textbf{is buffeted} by strong  winds or by stormy seas, it is repeatedly struck or blown around by them.
 \textit{
	\begin{itemize}
	\item Their plane had been severely buffeted by storms.
	\item Storms swept the country, closing roads, buffeting ferries and killing as many as
30 people.
	\end{itemize}
}
\item verb \\
If an economy or government  \textbf{is buffeted}  \textbf{by}  difficult or unpleasant  situations , it experiences many of them.
 \textit{
	\begin{itemize}
	\item The whole of Africa had been buffeted by social and political upheavals.
	\end{itemize}
}
\end{enumerate}

\section*{construct}
{\large \color{blue}  constructs  constructing  constructed  }
\subsection*{Explain}
\begin{enumerate}
\item verb \\
If you \textbf{construct} something such as a building, road , or machine , you build it or make it.
 \textit{
	\begin{itemize}
	\item The company is constructing 70 homes and a 130-room hotel on the land.
	\item The boxes should be constructed from rough-sawn timber.
	\item They thought he had escaped through a specially-constructed tunnel.
	\end{itemize}
}
\item verb \\
If you \textbf{construct} something such as an idea, a piece of writing , or a system, you create it by putting  different parts together.
 \textit{
	\begin{itemize}
	\item You will find it difficult to construct a spending plan without first recording your
spending.
	\item He eventually constructed a business empire which ran to Thailand and Singapore.
	\item The novel is constructed from a series of on-the-spot reports.
	\item ...using carefully-constructed tests.
	\end{itemize}
}
\item countable noun \\
A \textbf{construct} is a complex idea.
 \textit{
	\begin{itemize}
	\item ...the underlying constructs (beliefs, philosophy, etc.) which influence action and
behaviour.
	\item It was a re-enactment of the same mental construct under which slavery was justified.
	\end{itemize}
}
\item countable noun \\
A \textbf{construct} is something that is built, made, or created.
 \textit{
	\begin{itemize}
	\item The kites were flimsy constructs but soared to over a thousand feet.
	\item The country was an artificial construct held together by force.
	\end{itemize}
}
\end{enumerate}

\section*{cafeteria}
{\large \color{blue}  cafeterias  }
\subsection*{Explain}
\begin{enumerate}
\item countable noun \\
A \textbf{cafeteria} is a restaurant where you choose your food from a counter and take it to your table after paying for it. Cafeterias are usually found in public buildings such as hospitals and stores .
 \textit{
	\begin{itemize}
	\end{itemize}
}
\end{enumerate}

\section*{convert}
{\large \color{blue}  converts  converting  converted  }
\subsection*{Explain}
\begin{enumerate}
\item verb \\
If one thing \textbf{is converted} or \textbf{converts}  \textbf{into} another, it is changed into a different form.
 \textit{
	\begin{itemize}
	\item The signal will be converted into digital code.
	\item ...naturally occurring substances which the body can convert into vitamins.
	\item ...a table that converts into an ironing board.
	\end{itemize}
}
\item verb \\
If someone \textbf{converts} a room or building, they alter it in order to use it for a different purpose .
 \textit{
	\begin{itemize}
	\item By converting the loft, they were able to have two extra bedrooms.
	\item ...the entrepreneur who wants to convert County Hall into an hotel.
	\item He is living in a converted barn.
	\end{itemize}
}
\item verb \\
If you \textbf{convert} a vehicle or piece of equipment , you change it so that it can use a different fuel .
 \textit{
	\begin{itemize}
	\item Save money by converting your car to unleaded.
	\item The programme to convert every gas burner in Britain took 10 years.
	\end{itemize}
}
\item verb \\
If you \textbf{convert} a quantity  \textbf{from} one system of measurement \textbf{to} another, you calculate what the quantity is in the second system.
 \textit{
	\begin{itemize}
	\item Converting metric measurements to U.S. equivalents is easy.
	\end{itemize}
}
\item verb \\
If someone \textbf{converts} you, they persuade you to change your religious or political beliefs. You can also  say that someone \textbf{converts}  \textbf{to} a different religion.
 \textit{
	\begin{itemize}
	\item If you try to convert him, you could find he just walks away.
	\item He was a major influence in converting Godwin to political radicalism.
	\item He converted to Catholicism in 1917.
	\end{itemize}
}
\item countable noun \\
A \textbf{convert} is someone who has changed their religious or political beliefs.
 \textit{
	\begin{itemize}
	\item She, too, was a convert to Roman Catholicism.
	\item I took to these new pursuits with the enthusiasm of a convert who has just found
religion.
	\end{itemize}
}
\item verb \\
If someone \textbf{converts} you \textbf{to} something, they make you very enthusiastic about it.
 \textit{
	\begin{itemize}
	\item He quickly converted me to the joys of cross-country skiing.
	\end{itemize}
}
\item countable noun \\
If you describe someone as a \textbf{convert}  \textbf{to} something, you mean that they have recently become very enthusiastic about it.
 \textit{
	\begin{itemize}
	\item ...recent converts to vegetarianism.
	\end{itemize}
}
\end{enumerate}

\section*{casualty}
{\large \color{blue}  casualties  }
\subsection*{Explain}
\begin{enumerate}
\item countable noun \\
A \textbf{casualty} is a person who is injured or killed in a war or in an accident.
 \textit{
	\begin{itemize}
	\item Troops fired on demonstrators near the Royal Palace causing many casualties.
	\end{itemize}
}
\item countable noun \\
A \textbf{casualty}  \textbf{of} a particular  event or situation is a person or a thing that has suffered  badly as a result of that event or situation.
 \textit{
	\begin{itemize}
	\item The car manufacturer has been one of the greatest casualties of the recession.
	\end{itemize}
}
\item uncountable noun \\
\textbf{Casualty} is the part of a hospital where people who have severe  injuries or sudden  illnesses are taken for emergency  treatment .
 \textit{
	\begin{itemize}
	\item I was taken to casualty at St Thomas's Hospital.
	\end{itemize}
}
\end{enumerate}

\section*{creep}
{\large \color{blue}  creeps  creeping  crept  }
\subsection*{Explain}
\begin{enumerate}
\item verb \\
When people or animals \textbf{creep}  somewhere , they move quietly and slowly.
 \textit{
	\begin{itemize}
	\item Back I go to the hotel and creep up to my room.
	\item The rabbit creeps away and hides in a hole.
	\end{itemize}
}
\item verb \\
If something \textbf{creeps} somewhere, it moves very slowly.
 \textit{
	\begin{itemize}
	\item Mist had crept in again from the sea.
	\end{itemize}
}
\item verb \\
If something \textbf{creeps} in or \textbf{creeps}  back , it begins to occur or becomes part of something without people realizing or without them wanting it.
 \textit{
	\begin{itemize}
	\item Insecurity might creep in.
	\item An increasing ratio of mistakes, perhaps induced by tiredness, crept into her game.
	\item ...a proposal that crept through unnoticed at the National Council in December.
	\item Now his other major works are creeping back into concert programmes.
	\item The public are concerned at the creeping privatisation of core police functions.
	\end{itemize}
}
\item verb \\
If a rate or number \textbf{creeps}  \textbf{up} to a higher  level , it gradually reaches that level.
 \textit{
	\begin{itemize}
	\item The inflation rate has been creeping up to 9.5 per cent.
	\item The average number of students in each class is creeping up from three to four.
	\end{itemize}
}
\item countable noun \\
If you describe someone as a \textbf{creep} , you mean that you dislike them a great  deal , especially because they are insincere and flatter people.
 \textit{
	\begin{itemize}
	\end{itemize}
}
\item  \\
 give sb the creeps \textit{
	\begin{itemize}
	\end{itemize}
}
\end{enumerate}

\section*{conduct}
{\large \color{blue}  conducts  conducting  conducted  }
\subsection*{Explain}
\begin{enumerate}
\item verb \\
When you \textbf{conduct} an activity or task , you organize it and carry it out.
 \textit{
	\begin{itemize}
	\item I decided to conduct an experiment.
	\item He said they were conducting a campaign against democrats across the country.
	\item The council conducted a survey of the uses to which farm buildings are put.
	\end{itemize}
}
\item singular noun \\
The \textbf{conduct}  \textbf{of} a task or activity is the way in which it is organized and carried out.
 \textit{
	\begin{itemize}
	\item Also up for discussion will be the conduct of free and fair elections.
	\item There is emerging opposition to his conduct of foreign policy.
	\end{itemize}
}
\item verb \\
If you \textbf{conduct} yourself in a particular way, you behave in that way.
 \textit{
	\begin{itemize}
	\item The way he conducts himself reflects on the party and will increase criticisms against
him.
	\item Most people believe they conduct their private and public lives in accordance with
Christian morality.
	\end{itemize}
}
\item uncountable noun \\
Someone's \textbf{conduct} is the way they behave in particular situations .
 \textit{
	\begin{itemize}
	\item For Europeans, the law is a statement of basic principles of civilised conduct.
	\item He has trouble understanding that other people judge him by his conduct.
	\end{itemize}
}
\item verb \\
When someone \textbf{conducts} an orchestra or choir, they stand in front of it and direct its performance .
 \textit{
	\begin{itemize}
	\item Dennis had recently begun a successful career conducting opera in Europe.
	\item The choral director will continue to conduct here and abroad.
	\item At the Curtis Institute he studied conducting with Fritz Reiner.
	\end{itemize}
}
\item verb \\
If something \textbf{conducts} heat or electricity, it allows heat or electricity to pass through it or along it.
 \textit{
	\begin{itemize}
	\item Water conducts heat faster than air.
	\end{itemize}
}
\item verb \\
If you \textbf{conduct} someone to a place, you go there with them.
 \textit{
	\begin{itemize}
	\item He asked if he might conduct us to the ball which was to bring the proceedings to
an end.
	\end{itemize}
}
\end{enumerate}

\section*{drill}
{\large \color{blue}  drills  drilling  drilled  }
\subsection*{Explain}
\begin{enumerate}
\item countable noun \\
A \textbf{drill} is a tool or machine that you use for making holes.
 \textit{
	\begin{itemize}
	\item ...pneumatic drills.
	\item ...a dentist's drill.
	\end{itemize}
}
\item verb \\
When you \textbf{drill}  \textbf{into} something or \textbf{drill} a hole in something, you make a hole in it using a drill.
 \textit{
	\begin{itemize}
	\item He drilled into the wall of Lili's bedroom.
	\item I drilled five holes at equal distance.
	\end{itemize}
}
\item verb \\
When people \textbf{drill}  \textbf{for} oil or water, they search for it by drilling deep holes in the ground or in the bottom of the sea.
 \textit{
	\begin{itemize}
	\item There have been proposals to drill for more oil.
	\item The team is still drilling.
	\end{itemize}
}
\item countable noun \\
A \textbf{drill} is a way that teachers teach their students something by making them repeat it many times.
 \textit{
	\begin{itemize}
	\item The teacher runs them through a drill–the days of the week, the weather and some
counting.
	\end{itemize}
}
\item verb \\
If you \textbf{drill} people, you teach them to do something by making them repeat it many times.
 \textit{
	\begin{itemize}
	\item He drills the choir to a high standard.
	\end{itemize}
}
\item variable noun \\
A \textbf{drill} is repeated training for a group of people, especially soldiers, so that they can do something quickly and efficiently.
 \textit{
	\begin{itemize}
	\item The Marines carried out a drill that included 18 ships and 90 aircraft.
	\item His hands were clasped behind him like a drill sergeant.
	\end{itemize}
}
\item countable noun \\
A \textbf{drill} is a routine exercise or activity, in which people practise what they should do in dangerous situations.
 \textit{
	\begin{itemize}
	\item ...a fire drill.
	\item ...air-raid drills.
	\end{itemize}
}
\item uncountable noun \\
\textbf{Drill} is thick cotton material which is used for making uniforms and trousers .
 \textit{
	\begin{itemize}
	\item ...cotton drill.
	\end{itemize}
}
\item countable noun \\
A \textbf{drill} is a long line in the earth, a few centimetres deep, which a farmer or gardener makes to plant seeds in.
 \textit{
	\begin{itemize}
	\item Sow the seeds in drills about 1/2in. deep and 12in. apart.
	\end{itemize}
}
\end{enumerate}

\section*{cycle}
{\large \color{blue}  cycles  cycling  cycled  }
\subsection*{Explain}
\begin{enumerate}
\item verb \\
If you \textbf{cycle} , you ride a bicycle.
 \textit{
	\begin{itemize}
	\item He cycled to Ingwold.
	\item Britain could save £4.6 billion a year in road transport costs if more people cycled.
	\item Over 1000 riders cycled 100 miles around the Vale of York.
	\end{itemize}
}
\item countable noun \\
A \textbf{cycle} is a bicycle.
 \textit{
	\begin{itemize}
	\item ...an eight-mile cycle ride.
	\end{itemize}
}
\item countable noun \\
A \textbf{cycle} is a motorcycle .
 \textit{
	\begin{itemize}
	\end{itemize}
}
\item countable noun \\
A \textbf{cycle} is a series of events or processes that is repeated again and again, always in the same order.
 \textit{
	\begin{itemize}
	\item ...the life cycle of the plant.
	\item The figures marked the final low point of the present economic cycle.
	\item They must break out of the cycle of violence.
	\end{itemize}
}
\item countable noun \\
A \textbf{cycle} is a single complete series of movements in an electrical , electronic , or mechanical process.
 \textit{
	\begin{itemize}
	\item ...10 cycles per second.
	\end{itemize}
}
\item countable noun \\
A \textbf{cycle} is a series of songs or poems that are intended to be performed or read one after the other.
 \textit{
	\begin{itemize}
	\item ...Wagner's Ring cycle.
	\end{itemize}
}
\end{enumerate}

\section*{enforce}
{\large \color{blue}  enforces  enforcing  enforced  }
\subsection*{Explain}
\begin{enumerate}
\item verb \\
If people in authority  \textbf{enforce} a law or a rule , they make sure that it is obeyed , usually by punishing people who do not obey it.
 \textit{
	\begin{itemize}
	\item One of the beat officer's duties was to help the council to enforce the ban.
	\item The measures are being enforced by Interior Ministry troops.
	\end{itemize}
}
\item verb \\
To \textbf{enforce} something means to force or cause it to be done or to happen .
 \textit{
	\begin{itemize}
	\item They struggled to limit the cost by enforcing a low-tech specification.
	\item David is now living in Beirut again after an enforced absence.
	\end{itemize}
}
\end{enumerate}

\section*{deed}
{\large \color{blue}  deeds  }
\subsection*{Explain}
\begin{enumerate}
\item countable noun \\
A \textbf{deed} is something that is done, especially something that is very good or very bad .
 \textit{
	\begin{itemize}
	\item His heroic deeds were celebrated in every corner of India.
	\item ...the warm feeling one gets from doing a good deed.
	\item The perpetrators of this evil deed must be brought to justice.
	\end{itemize}
}
\item countable noun \\
A \textbf{deed} is a document containing the terms of an agreement , especially an agreement concerning the ownership of land or a building .
 \textit{
	\begin{itemize}
	\item He asked if I had the deeds to his father's property.
	\end{itemize}
}
\end{enumerate}

\section*{enrich}
{\large \color{blue}  enriches  enriching  enriched  }
\subsection*{Explain}
\begin{enumerate}
\item verb \\
To \textbf{enrich} something means to improve its quality, usually by adding something to it.
 \textit{
	\begin{itemize}
	\item An extended family enriches life in many ways.
	\item It is important to enrich the soil prior to planting.
	\end{itemize}
}
\item verb \\
To \textbf{enrich} someone means to increase the amount of money that they have.
 \textit{
	\begin{itemize}
	\item He will drain, rather than enrich, the country.
	\end{itemize}
}
\item verb \\
To \textbf{enrich} a nuclear  fuel such as uranium means to increase the number of atoms of a particular kind in it, so that it can be used to produce more energy or a greater  explosion .
 \textit{
	\begin{itemize}
	\item It was actually used for enriching uranium to weapons-grade levels.
	\end{itemize}
}
\end{enumerate}

\section*{ego}
{\large \color{blue}  egos  }
\subsection*{Explain}
\begin{enumerate}
\item variable noun \\
Someone's \textbf{ego} is their sense of their own worth . For example , if someone has a large \textbf{ego} , they think they are very important and valuable .
 \textit{
	\begin{itemize}
	\item He had a massive ego; never would he admit he was wrong.
	\end{itemize}
}
\end{enumerate}

\section*{establish}
{\large \color{blue}  establishes  establishing  established  }
\subsection*{Explain}
\begin{enumerate}
\item verb \\
If someone \textbf{establishes} something such as an organization, a type of activity, or a set of rules , they create it or introduce it in such a way that it is likely to last for a long time.
 \textit{
	\begin{itemize}
	\item The U.N. has established detailed criteria for who should be allowed to vote.
	\item The School was established in 1989 by an Italian professor.
	\end{itemize}
}
\item verb \\
If you \textbf{establish}  contact with someone, you start to have contact with them. You can also  say that two people, groups, or countries \textbf{establish} contact.
 \textit{
	\begin{itemize}
	\item We had already established contact with the museum.
	\item The two countries have established diplomatic relations.
	\end{itemize}
}
\item verb \\
If you \textbf{establish}  \textbf{that} something is true , you discover  facts that show that it is definitely true.
 \textit{
	\begin{itemize}
	\item Medical tests established that she was not their own child.
	\item It will be essential to establish how the money is being spent.
	\item An autopsy was being done to establish the cause of death.
	\item It was established that the missile had landed on a test range in Australia.
	\end{itemize}
}
\item verb \\
If you \textbf{establish}  \textbf{yourself} , your reputation , or a good quality that you have, you succeed in doing something, and achieve  respect or a secure position as a result of this.
 \textit{
	\begin{itemize}
	\item This is going to be the show where up-and-coming comedians will establish themselves.
	\item He has established himself as a pivotal figure in U.S. politics.
	\item We shall fight to establish our innocence.
	\end{itemize}
}
\end{enumerate}

\section*{extension}
{\large \color{blue}  extensions  }
\subsection*{Explain}
\begin{enumerate}
\item countable noun \\
An \textbf{extension} is a new room or building which is added to an existing building or group of buildings.
 \textit{
	\begin{itemize}
	\end{itemize}
}
\item countable noun \\
An \textbf{extension} is a new section of a road or rail line that is added to an existing road or line.
 \textit{
	\begin{itemize}
	\item ...the Jubilee Line extension.
	\end{itemize}
}
\item countable noun \\
An \textbf{extension} is an extra  period of time for which something lasts or is valid , usually as a result of official  permission .
 \textit{
	\begin{itemize}
	\item He first entered Britain on a six-month visa, and was given a further extension of
six months.
	\end{itemize}
}
\item countable noun \\
Something that is an \textbf{extension}  \textbf{of} something else is a development of it that includes or affects more people, things, or activities .
 \textit{
	\begin{itemize}
	\item He saw his civil rights activity as an extension of his ministry.
	\item That's the logical extension of my approach.
	\end{itemize}
}
\item countable noun \\
An \textbf{extension} is a phone line that is connected to the switchboard of a company or institution , and that has its own number. The written  abbreviation  ext. is also used.
 \textit{
	\begin{itemize}
	\item She can get me on extension 308.
	\item For further information, please contact 414 3925, extension 2253.
	\end{itemize}
}
\item countable noun \\
An \textbf{extension} is a part which is connected to a piece of equipment in order to make it reach something further away .
 \textit{
	\begin{itemize}
	\item ...a 30-foot extension cord.
	\item Some of the best extensions are made from sections of rod tube or drainpipe.
	\end{itemize}
}
\end{enumerate}

\section*{execute}
{\large \color{blue}  executes  executing  executed  }
\subsection*{Explain}
\begin{enumerate}
\item verb \\
To \textbf{execute} someone means to kill them as a punishment for a serious  crime .
 \textit{
	\begin{itemize}
	\item He was executed by lethal injection earlier today.
	\item One group claimed to have executed the American hostage.
	\item This boy's father had been executed for conspiring against the throne.
	\end{itemize}
}
\item verb \\
If you \textbf{execute} a plan , you carry it out.
 \textit{
	\begin{itemize}
	\item We are going to execute our campaign plan to the letter.
	\end{itemize}
}
\item verb \\
If you \textbf{execute} a difficult  action or movement , you successfully perform it.
 \textit{
	\begin{itemize}
	\item I executed the hairpin turn high on the sheer western face of the mountains.
	\item The landing was skilfully executed.
	\end{itemize}
}
\item verb \\
When someone \textbf{executes} a work of art , they make or produce it, using an idea as a basis .
 \textit{
	\begin{itemize}
	\item Morris executed a suite of twelve drawings in 1978.
	\item A well-executed shot of a tall ship is a joy to behold.
	\end{itemize}
}
\end{enumerate}

\section*{fellowship}
{\large \color{blue}  fellowships  }
\subsection*{Explain}
\begin{enumerate}
\item countable noun \\
A \textbf{fellowship} is a group of people that join  together for a common  purpose or interest.
 \textit{
	\begin{itemize}
	\item ...the National Schizophrenia Fellowship.
	\item At Merlin's instigation, Arthur founds the Fellowship of the Round Table.
	\end{itemize}
}
\item countable noun \\
A \textbf{fellowship} at a university is a post which involves research work.
 \textit{
	\begin{itemize}
	\item He was offered a research fellowship at Clare College.
	\end{itemize}
}
\item uncountable noun \\
\textbf{Fellowship} is a feeling of friendship that people have when they are talking or doing something together and sharing their experiences.
 \textit{
	\begin{itemize}
	\item ...a sense of community and fellowship.
	\end{itemize}
}
\end{enumerate}

\section*{fry}
{\large \color{blue}  fries  frying  fried  }
\subsection*{Explain}
\begin{enumerate}
\item verb \\
When you \textbf{fry} food, you cook it in a pan that contains hot fat or oil.
 \textit{
	\begin{itemize}
	\item Fry the breadcrumbs until golden brown.
	\item ...fried rice.
	\end{itemize}
}
\item plural noun \\
\textbf{Fry} are very small, young fish.
 \textit{
	\begin{itemize}
	\end{itemize}
}
\item plural noun \\
\textbf{Fries} are the same as French fries .
 \textit{
	\begin{itemize}
	\end{itemize}
}
\end{enumerate}

\section*{freedom}
{\large \color{blue}  freedoms  }
\subsection*{Explain}
\begin{enumerate}
\item uncountable noun \\
\textbf{Freedom} is the state of being allowed to do what you want to do. \textbf{Freedoms} are instances of this.
 \textit{
	\begin{itemize}
	\item ...freedom of speech.
	\item They want greater political freedom.
	\item Today we have the freedom to decide our own futures.
	\item The United Nations Secretary-General has spoken of the need for individual freedoms
and human rights.
	\end{itemize}
}
\item uncountable noun \\
When prisoners or slaves are set free or escape , they gain their \textbf{freedom} .
 \textit{
	\begin{itemize}
	\item ...an agreement under which all hostages and detainees would gain their freedom.
	\end{itemize}
}
\item uncountable noun \\
\textbf{Freedom from} something you do not want means not being affected by it.
 \textit{
	\begin{itemize}
	\item ...all the freedom from pain that medicine could provide.
	\item ...freedom from government control.
	\end{itemize}
}
\item singular noun \\
\textbf{The freedom of} a particular city is a special  honour which is given to a famous person who is connected with that city, or to someone who has performed some special service for the city.
 \textit{
	\begin{itemize}
	\item He was given the Freedom of the City of Dublin by the Lord Mayor.
	\end{itemize}
}
\end{enumerate}

\section*{harness}
{\large \color{blue}  harnesses  harnessing  harnessed  }
\subsection*{Explain}
\begin{enumerate}
\item verb \\
If you \textbf{harness} something such as an emotion or natural source of energy, you bring it under your control and use it.
 \textit{
	\begin{itemize}
	\item We need to find new ways of harnessing that enthusiasm and commitment.
	\item Turkey plans to harness the waters of the Tigris and Euphrates rivers for big hydro-electric
power projects.
	\end{itemize}
}
\item countable noun \\
A \textbf{harness} is a set of straps which fit under a person's arms and fasten round their body in order to keep a piece of equipment in place or to prevent the person moving from a place.
 \textit{
	\begin{itemize}
	\end{itemize}
}
\item countable noun \\
A \textbf{harness} is a set of leather straps and metal links fastened round a horse's head or body so that the horse can have a carriage , cart, or plough fastened to it.
 \textit{
	\begin{itemize}
	\end{itemize}
}
\item verb \\
If a horse or other animal \textbf{is harnessed} , a harness is put on it, especially so that it can pull a carriage, cart, or plough.
 \textit{
	\begin{itemize}
	\item On Sunday the horses were harnessed to a heavy wagon for a day-long ride over the
Border.
	\end{itemize}
}
\item  \\
 in harness \textit{
	\begin{itemize}
	\end{itemize}
}
\item  \\
 in harness \textit{
	\begin{itemize}
	\end{itemize}
}
\end{enumerate}

\section*{grace}
{\large \color{blue}  graces  gracing  graced  }
\subsection*{Explain}
\begin{enumerate}
\item uncountable noun \\
If someone moves with \textbf{grace} , they move in a smooth , controlled , and attractive  way .
 \textit{
	\begin{itemize}
	\item He moved with the grace of a trained boxer.
	\item Ballet classes are important for poise and grace.
	\end{itemize}
}
\item uncountable noun \\
If someone behaves with \textbf{grace} , they behave in a pleasant , polite , and dignified way, even when they are upset or being treated unfairly.
 \textit{
	\begin{itemize}
	\item The new King seemed to be carrying out his duties with grace and due decorum.
	\item The young woman had grace beyond her years.
	\end{itemize}
}
\item plural noun \\
The \textbf{graces} are the ways of behaving and doing things which are considered polite and well-mannered .
 \textit{
	\begin{itemize}
	\item She didn't fit in and she had few social graces.
	\end{itemize}
}
\item uncountable noun \\
\textbf{Grace} is used in expressions such as \textbf{a day's grace} and \textbf{a month's grace} to say that you are allowed that amount of extra time before you have to finish something.
 \textit{
	\begin{itemize}
	\item She wanted a couple of days' grace to get the maisonette cleaned before she moved
in.
	\item We have only a few hours' grace before the soldiers come.
	\end{itemize}
}
\item verb \\
If you say that something \textbf{graces} a place or a person, you mean that it makes them more attractive.
 \textit{
	\begin{itemize}
	\item He went to the beautiful old Welsh dresser that graced this homely room.
	\item Her shoulders were graced with mink and her fingers sparkled with diamonds.
	\end{itemize}
}
\item verb \\
If you say that someone important  \textbf{will grace} an event or an organization , you mean that they have agreed to be present at the event or to be part of the organization.
 \textit{
	\begin{itemize}
	\item He had been invited to grace a function at the evening college.
	\end{itemize}
}
\item uncountable noun \\
In Christianity and some other religions , \textbf{grace} is the kindness that God shows to people because he loves them.
 \textit{
	\begin{itemize}
	\item It was only by the grace of God that no one died.
	\end{itemize}
}
\item variable noun \\
When someone says \textbf{grace} before or after a meal, they say a prayer in which they thank God for the food and
 ask Him to bless it.
 \textit{
	\begin{itemize}
	\item Leo, will you say grace?
	\item ...a Latin grace.
	\end{itemize}
}
\item countable noun \\
You use expressions such as \textbf{Your Grace} and \textbf{His Grace} when you are addressing or referring to a duke, duchess, or archbishop.
 \textit{
	\begin{itemize}
	\item Your Grace, I have a great favour to ask of you.
	\end{itemize}
}
\item  \\
 to fall from grace \textit{
	\begin{itemize}
	\end{itemize}
}
\item  \\
 have the grace to do sth/have the good grace to do sth \textit{
	\begin{itemize}
	\end{itemize}
}
\item  \\
 there but for the grace of God go I \textit{
	\begin{itemize}
	\end{itemize}
}
\item  \\
 with good grace/with a good grace/with bad grace/with a bad grace \textit{
	\begin{itemize}
	\end{itemize}
}
\item  \\
 good graces \textit{
	\begin{itemize}
	\end{itemize}
}
\end{enumerate}

\section*{heal}
{\large \color{blue}  heals  healing  healed  }
\subsection*{Explain}
\begin{enumerate}
\item verb \\
When a broken  bone or other injury  \textbf{heals} or when something \textbf{heals} it, it becomes  healthy and normal again.
 \textit{
	\begin{itemize}
	\item Within six weeks the bruising had gone, but it was six months before it all healed.
	\item If they'd operated and pinned her arm at once, she might have healed by now.
	\item No doctor has ever healed a broken bone: he or she sets them.
	\item Therapies like acupuncture do work and many people have been healed by them.
	\end{itemize}
}
\item ergative verb \\
When someone's emotional  pain  \textbf{heals} , they feel normal and happy again.
 \textit{
	\begin{itemize}
	\item A year later, she had healed to the point of at least being able to consider a romantic
relationship.
	\item Only by fully experiencing the depth of our pain can we be healed from it and be
done with it.
	\end{itemize}
}
\item verb \\
If you \textbf{heal} something such as a rift or a wound, or if it \textbf{heals} , the situation is put  right so that people are friendly or happy again.
 \textit{
	\begin{itemize}
	\item Today Sophie and her sister have healed the family rift and visit their family every
weekend.
	\item The psychological effects on the United States were immense and in Washington the
wounds have still not fully healed.
	\end{itemize}
}
\end{enumerate}

\section*{hurt}
{\large \color{blue}  hurts  hurting  hurt  }
\subsection*{Explain}
\begin{enumerate}
\item verb \\
If you \textbf{hurt}  \textbf{yourself} or \textbf{hurt} a part of your body, you feel pain because you have injured yourself.
 \textit{
	\begin{itemize}
	\item Yasin had seriously hurt himself while trying to escape from the police.
	\item He had hurt his back in an accident.
	\end{itemize}
}
\item verb \\
If a part of your body \textbf{hurts} , you feel pain there.
 \textit{
	\begin{itemize}
	\item His collar bone only hurt when he lifted his arm.
	\end{itemize}
}
\item adjective \\
If you are \textbf{hurt} , you have been injured.
 \textit{
	\begin{itemize}
	\item His comrades asked him if he was hurt.
	\item They were dazed but did not seem to be badly hurt.
	\end{itemize}
}
\item verb \\
If you \textbf{hurt} someone, you cause them to feel pain.
 \textit{
	\begin{itemize}
	\item I didn't mean to hurt her, only to keep her still.
	\item You're hurting my arm.
	\item Ouch. That hurt.
	\end{itemize}
}
\item verb \\
If someone \textbf{hurts} you, they say or do something that makes you unhappy .
 \textit{
	\begin{itemize}
	\item He is afraid of hurting Bessy's feelings.
	\item She's afraid she's going to be hurt and that she'll never fall in love again.
	\item What hurts most is the betrayal, the waste.
	\end{itemize}
}
\item adjective \\
If you are \textbf{hurt} , you are upset because of something that someone has said or done .
 \textit{
	\begin{itemize}
	\item Yes, I was hurt, jealous.
	\item He gave me a slightly hurt look.
	\end{itemize}
}
\item verb \\
If you say that you \textbf{are hurting} , you mean that you are experiencing emotional pain.
 \textit{
	\begin{itemize}
	\item I am lonely and I am hurting.
	\end{itemize}
}
\item verb \\
To \textbf{hurt} someone or something means to have a bad  effect on them or prevent them from succeeding .
 \textit{
	\begin{itemize}
	\item The combination of hot weather and decreased water supplies is hurting many industries.
	\item The planned closures will really hurt the local economies.
	\end{itemize}
}
\item variable noun \\
A feeling of \textbf{hurt} is a feeling that you have when you think that you have been treated  badly or judged unfairly.
 \textit{
	\begin{itemize}
	\item ...feelings of hurt and anger, fear and despair.
	\item I was full of jealousy and hurt.
	\item There would be a hurt in her heart for a while, but in the end she would get over
it.
	\end{itemize}
}
\item  \\
 it won't hurt/it never hurts \textit{
	\begin{itemize}
	\end{itemize}
}
\end{enumerate}

\section*{help}
{\large \color{blue}  helps  helping  helped  }
\subsection*{Explain}
\begin{enumerate}
\item verb \\
If you \textbf{help} someone, you make it easier for them to do something, for example by doing part of the work for them or by giving
them advice or money.
 \textbf{Help} is also a noun .
 \textit{
	\begin{itemize}
	\item He has helped to raise a lot of money.
	\item My mum used to help cook the meals for the children.
	\item America's priority is to help nations defend themselves.
	\item You can of course help by giving them a donation directly.
	\item I was only trying to help.
	\item If you're not willing to help me, I'll find somebody who will.
	\item Thanks very much for your help.
	\item Always ask the pharmacist for help.
	\item Some of them have qualified for help with monthly payments.
	\end{itemize}
}
\item verb \\
If you say that something \textbf{helps} , you mean that it makes something easier to do or get , or that it improves a situation to some extent .
 \textit{
	\begin{itemize}
	\item The right style of swimsuit can help to hide, minimise or emphasise what you want
it to.
	\item By using less energy we are also helping the environment by reducing the threat of
global warming.
	\item Understanding these rare molecules will help chemists to find out what is achievable.
	\item I could cook your supper, though, if that would help.
	\end{itemize}
}
\item verb \\
If you \textbf{help} someone go somewhere or move in some way, you give them support so that they can move more easily.
 \textit{
	\begin{itemize}
	\item Martin helped Tanya over the rail.
	\item I allowed her to help me to my feet.
	\item Come and help me up!
	\item She helped her sit up in bed so she could hold her baby.
	\end{itemize}
}
\item verb \\
If you \textbf{help}  \textbf{yourself} , you try to get yourself out of a difficult situation rather than accept it and think you can do nothing to change it.
 \textit{
	\begin{itemize}
	\item He firmly believes they should do more to help themselves.
	\end{itemize}
}
\item singular noun \\
If you say that someone or something has been \textbf{a help} or has been some \textbf{help} , you mean that they have helped you to solve a problem .
 \textit{
	\begin{itemize}
	\item Thank you. You've been a great help already.
	\item ...a quality which will be a help rather than a hindrance to them.
	\item She's been a lot of help.
	\item The books were not much help.
	\end{itemize}
}
\item uncountable noun \\
\textbf{Help} is action taken to rescue a person who is in danger . You shout ' \textbf{help!} ' when you are in danger in order to attract someone's attention so that they can come and rescue you.
 \textit{
	\begin{itemize}
	\item He was screaming for help.
	\item 'Help!' I screamed, turning to run.
	\end{itemize}
}
\item uncountable noun \\
In computing , \textbf{help} , or the \textbf{help}  menu , is a file that gives you information and advice, for example about how to use a particular
 program .
 \textit{
	\begin{itemize}
	\item If you get stuck, click on Help.
	\end{itemize}
}
\item verb \\
If you \textbf{help}  \textbf{yourself}  \textbf{to} something, you serve yourself or you take it for yourself. If someone tells you to \textbf{help}  \textbf{yourself} , they are telling you politely to serve yourself anything you want or to take anything you want.
 \textit{
	\begin{itemize}
	\item There's bread on the table. Help yourself.
	\item Just help yourself to leaflets.
	\end{itemize}
}
\item verb \\
If someone \textbf{helps}  \textbf{themselves}  \textbf{to} something, they steal it.
 \textit{
	\begin{itemize}
	\item Has somebody helped himself to some film star's diamonds?
	\end{itemize}
}
\item  \\
 can't help sth \textit{
	\begin{itemize}
	\end{itemize}
}
\item  \\
 can't help \textit{
	\begin{itemize}
	\end{itemize}
}
\item  \\
 be of help \textit{
	\begin{itemize}
	\end{itemize}
}
\end{enumerate}

\section*{hypocrisy}
{\large \color{blue}  hypocrisies  }
\subsection*{Explain}
\begin{enumerate}
\item variable noun \\
If you accuse someone of \textbf{hypocrisy} , you mean that they pretend to have qualities, beliefs, or feelings that they do not really have.
 \textit{
	\begin{itemize}
	\item He accused newspapers of hypocrisy in their treatment of the story.
	\item You'll have little patience with the hypocrisy and double standards you encounter.
	\end{itemize}
}
\end{enumerate}

\section*{impose}
{\large \color{blue}  imposes  imposing  imposed  }
\subsection*{Explain}
\begin{enumerate}
\item verb \\
If you \textbf{impose} something \textbf{on} people, you use your authority to force them to accept it.
 \textit{
	\begin{itemize}
	\item Britain imposed fines on airlines which bring in passengers without proper papers.
	\item Many companies have imposed a pay freeze.
	\item The conditions imposed on volunteers were stringent.
	\end{itemize}
}
\item verb \\
If you \textbf{impose} your opinions or beliefs  \textbf{on} other people, you try and make people accept them as a rule or as a model to copy .
 \textit{
	\begin{itemize}
	\item Parents of either sex should beware of imposing their own tastes on their children.
	\end{itemize}
}
\item verb \\
If something \textbf{imposes}  strain , pressure , or suffering  \textbf{on} someone, it causes them to experience it.
 \textit{
	\begin{itemize}
	\item The filming imposed an additional strain on her.
	\item ...the pressures imposed upon teachers by ceaseless curriculum reforms.
	\end{itemize}
}
\item verb \\
If someone \textbf{imposes}  \textbf{on} you, they unreasonably expect you to do something for them which you do not want to do.
 \textit{
	\begin{itemize}
	\item I was afraid you'd simply feel we were imposing on you.
	\item 'Mum thinks I should stop imposing on your hospitality, Leo,' said Grace.
	\end{itemize}
}
\item verb \\
If someone \textbf{imposes}  \textbf{themselves on} you, they force you to accept their company although you may not want to.
 \textit{
	\begin{itemize}
	\item I didn't want to impose myself on my married friends.
	\end{itemize}
}
\end{enumerate}

\section*{injury}
{\large \color{blue}  injuries  }
\subsection*{Explain}
\begin{enumerate}
\item variable noun \\
An \textbf{injury} is damage done to a person's or an animal's body.
 \textit{
	\begin{itemize}
	\item Four police officers sustained serious injuries in the explosion.
	\item The two other passengers escaped serious injury.
	\end{itemize}
}
\item variable noun \\
If someone suffers  \textbf{injury}  \textbf{to} their feelings , they are badly  upset by something. If they suffer \textbf{injury}  \textbf{to} their reputation, their reputation is seriously harmed.
 \textit{
	\begin{itemize}
	\item She was awarded £3,500 for injury to her feelings.
	\end{itemize}
}
\end{enumerate}

\section*{indulge}
{\large \color{blue}  indulges  indulging  indulged  }
\subsection*{Explain}
\begin{enumerate}
\item verb \\
If you \textbf{indulge}  \textbf{in} something or if you \textbf{indulge}  \textbf{yourself} , you allow yourself to have or do something that you know you will  enjoy .
 \textit{
	\begin{itemize}
	\item We had enough time to indulge in a bit of window shopping.
	\item He returned to Britain so that he could indulge his passion for football.
	\item You can indulge yourself without spending a fortune.
	\end{itemize}
}
\item verb \\
If you \textbf{indulge} someone, you let them have or do what they want , even if this is not good for them.
 \textit{
	\begin{itemize}
	\item He did not agree with indulging children.
	\end{itemize}
}
\end{enumerate}

\section*{liberty}
{\large \color{blue}  liberties  }
\subsection*{Explain}
\begin{enumerate}
\item variable noun \\
\textbf{Liberty} is the freedom to live your life in the way that you want , without interference from other people or the authorities .
 \textit{
	\begin{itemize}
	\item Wit Wolzek claimed the legislation could impinge on privacy, self determination and
respect for religious liberty.
	\item Such a system would be a fundamental blow to the rights and liberties of the English
people.
	\end{itemize}
}
\item uncountable noun \\
\textbf{Liberty} is the freedom to go wherever you want, which you lose when you are a prisoner .
 \textit{
	\begin{itemize}
	\item Drug addicts need help, not loss of liberty.
	\item He was at liberty, but under 24-hour surveillance.
	\end{itemize}
}
\item  \\
 at liberty to do \textit{
	\begin{itemize}
	\end{itemize}
}
\item  \\
 take the liberty of doing something \textit{
	\begin{itemize}
	\end{itemize}
}
\item  \\
 take liberties \textit{
	\begin{itemize}
	\end{itemize}
}
\end{enumerate}

\section*{involve}
{\large \color{blue}  involves  involving  involved  }
\subsection*{Explain}
\begin{enumerate}
\item verb \\
If a situation or activity \textbf{involves} something, that thing is a necessary part or consequence of it.
 \textit{
	\begin{itemize}
	\item Running a kitchen involves a great deal of discipline and speed.
	\item Nicky's job involves spending quite a lot of time with other people.
	\end{itemize}
}
\item verb \\
If a situation or activity \textbf{involves} someone, they are taking part in it.
 \textit{
	\begin{itemize}
	\item If there was a cover-up, it involved people at the very highest levels of government.
	\item ...a riot involving a hundred inmates.
	\item Detectives launched an operation involving Interpol and Nigerian police.
	\end{itemize}
}
\item verb \\
If you say that someone \textbf{involves} themselves \textbf{in} something, you mean that they take part in it, often in a way that is unnecessary or unwanted .
 \textit{
	\begin{itemize}
	\item I seem to have involved myself in something I don't understand.
	\end{itemize}
}
\item verb \\
If you \textbf{involve} someone else \textbf{in} something, you get them to take part in it.
 \textit{
	\begin{itemize}
	\item Noel and I do everything together, he involves me in everything.
	\item Before too long he started involving me in the more confidential aspects of the job.
	\end{itemize}
}
\item verb \\
If one thing \textbf{involves} you \textbf{in} another thing, especially something unpleasant or inconvenient , the first thing causes you to do or deal with the second .
 \textit{
	\begin{itemize}
	\item A late booking may involve you in extra cost.
	\item This involved me in a round trip of over 400 miles.
	\end{itemize}
}
\end{enumerate}

\section*{lounge}
{\large \color{blue}  lounges  lounging  lounged  }
\subsection*{Explain}
\begin{enumerate}
\item countable noun \\
In a house, a \textbf{lounge} is a room where people sit and relax.
 \textit{
	\begin{itemize}
	\item The Holmbergs were sitting before a roaring fire in the lounge, sipping their cocoa.
	\end{itemize}
}
\item countable noun \\
In a hotel, club , or other public place, a \textbf{lounge} is a room where people can sit and relax.
 \textit{
	\begin{itemize}
	\item I spoke to her in the lounge of a big Johannesburg hotel where she was attending
a union meeting.
	\end{itemize}
}
\item countable noun \\
In an airport , a \textbf{lounge} is a very large room where people can sit and wait for aircraft to arrive or leave.
 \textit{
	\begin{itemize}
	\item Instead of taking me to the departure lounge they took me right to my seat on the
plane.
	\end{itemize}
}
\item verb \\
If you \textbf{lounge}  somewhere , you sit or lie there in a relaxed or lazy way.
 \textit{
	\begin{itemize}
	\item They ate and drank and lounged in the shade.
	\item If you don't want to lounge on the beach, you can go on a guided walk along the nature
trail.
	\end{itemize}
}
\end{enumerate}

\section*{join}
{\large \color{blue}  joins  joining  joined  }
\subsection*{Explain}
\begin{enumerate}
\item verb \\
If one person or vehicle  \textbf{joins} another, they move or go to the same place, for example so that both of them can do something together.
 \textit{
	\begin{itemize}
	\item His wife and children moved to join him in their new home.
	\end{itemize}
}
\item verb \\
If you \textbf{join} an organization, you become a member of it or start work as an employee of it.
 \textit{
	\begin{itemize}
	\item He joined the Army five years ago.
	\item She joined a dance company which took her around the world.
	\end{itemize}
}
\item verb \\
If you \textbf{join} an activity that other people are doing, you take part in it or become involved with it.
 \textit{
	\begin{itemize}
	\item Telephone operators joined the strike.
	\item The pastor requested the women present to join him in prayer.
	\item Private contractors joined in condemning the Government's stance.
	\end{itemize}
}
\item verb \\
If you \textbf{join} a queue , you stand at the end of it so that you are part of it.
 \textit{
	\begin{itemize}
	\item Make sure you join the queue inside the bank.
	\end{itemize}
}
\item verb \\
To \textbf{join} two things means to fix or fasten them together.
 \textit{
	\begin{itemize}
	\item The opened link is used to join the two ends of the chain.
	\item ...the conjunctiva, the skin which joins the eye to the lid.
	\item ...two springs that are joined together by a string.
	\end{itemize}
}
\item verb \\
If something such as a line or path  \textbf{joins} two things, it connects them.
 \textit{
	\begin{itemize}
	\item It has a dormer roof joining both gable ends.
	\item The car parks are joined by a footpath.
	\item ...a global highway of cables joining all the continents together.
	\end{itemize}
}
\item verb \\
If two roads or rivers  \textbf{join} , they meet or come together at a particular point.
 \textit{
	\begin{itemize}
	\item Do you know the highway to Tulsa? The airport road joins it.
	\item ...Allahabad, where the Ganges and the Yamuna rivers join.
	\end{itemize}
}
\item countable noun \\
A \textbf{join} is a place where two things are fastened or fixed together.
 \textit{
	\begin{itemize}
	\end{itemize}
}
\end{enumerate}

\section*{mate}
{\large \color{blue}  mates  mating  mated  }
\subsection*{Explain}
\begin{enumerate}
\item countable noun \\
You can  refer to someone's friends as their \textbf{mates} , especially when you are talking about a man and his male friends.
 \textit{
	\begin{itemize}
	\item He's off drinking with his mates.
	\item A mate of mine used to play soccer for Liverpool.
	\end{itemize}
}
\item countable noun \\
Some men use \textbf{mate} as a way of addressing other men when they are talking to them.
 \textit{
	\begin{itemize}
	\item Come on mate, things aren't that bad.
	\end{itemize}
}
\item countable noun \\
Someone's wife , husband , or sexual partner can be referred to as their \textbf{mate} .
 \textit{
	\begin{itemize}
	\item He has found his ideal mate.
	\end{itemize}
}
\item countable noun \\
An animal's \textbf{mate} is its sexual partner.
 \textit{
	\begin{itemize}
	\item The males guard their mates zealously.
	\end{itemize}
}
\item verb \\
When animals \textbf{mate} , a male and a female have sex in order to produce young .
 \textit{
	\begin{itemize}
	\item This allows the pair to mate properly and stops the hen staying in the nest-box.
	\item They want the males to mate with wild females.
	\item It is easy to tell when a female is ready to mate.
	\item ...the mating season.
	\end{itemize}
}
\item countable noun \\
On a commercial ship, \textbf{the mate} or \textbf{the first mate} is the most important officer except for the captain . Officers of lower  rank are also  called  \textbf{mates} .
 \textit{
	\begin{itemize}
	\item ...the mate of a fishing trawler.
	\end{itemize}
}
\item countable noun \\
In the United  States  Navy , a \textbf{mate} is a petty officer who has a particular  set of skills and who assists a warrant officer.
 \textit{
	\begin{itemize}
	\end{itemize}
}
\item uncountable noun \\
In chess , \textbf{mate} is the same as checkmate .
 \textit{
	\begin{itemize}
	\end{itemize}
}
\end{enumerate}

\section*{lay}
{\large \color{blue}  lays  laying  laid  }
\subsection*{Explain}
\begin{enumerate}
\item verb \\
If you \textbf{lay} something somewhere , you put it there in a careful , gentle , or neat way.
 \textit{
	\begin{itemize}
	\item Lay a sheet of newspaper on the floor.
	\item My father's working bench was covered with a cloth and his coffin was laid there.
	\item Mothers routinely lay babies on their backs to sleep.
	\end{itemize}
}
\item verb \\
If you \textbf{lay} the table or \textbf{lay} the places at a table, you arrange the knives , forks , and other things that people need on the table before a meal.
 \textit{
	\begin{itemize}
	\item The butler always laid the table.
	\end{itemize}
}
\item verb \\
If you \textbf{lay} something such as carpets , cables, or foundations , you put them into their permanent position.
 \textit{
	\begin{itemize}
	\item A man came to lay the saloon carpet.
	\item Public utilities dig up roads to lay pipes.
	\end{itemize}
}
\item verb \\
To \textbf{lay} a trap means to prepare it in order to catch someone or something.
 \textit{
	\begin{itemize}
	\item They were laying a trap for the kidnapper.
	\end{itemize}
}
\item verb \\
When a female bird \textbf{lays} an egg, it produces an egg by pushing it out of its body.
 \textit{
	\begin{itemize}
	\item My canary has laid an egg.
	\item Freezing weather in spring hampered the hens' ability to lay.
	\end{itemize}
}
\item verb \\
\textbf{Lay} is used with some nouns to talk about making official preparations for something. For example, if you \textbf{lay the basis} for something or \textbf{lay plans} for it, you prepare it carefully.
 \textit{
	\begin{itemize}
	\item Diplomats meeting in Chile have laid the groundwork for far-reaching environmental
regulations.
	\item The organisers meet in March to lay plans.
	\end{itemize}
}
\item verb \\
\textbf{Lay} is used with some nouns in expressions about accusing or blaming someone. For example, if you \textbf{lay the blame} for a mistake on someone, you say it is their fault , or if the police  \textbf{lay charges} against someone, they officially accuse that person of a crime .
 \textit{
	\begin{itemize}
	\item She refused to lay the blame on any one party.
	\item He could not bear to lay responsibility for the unhappiness of his later years on
his own shoulders.
	\item Police have decided not to lay charges over allegations of a phone tapping operation.
	\end{itemize}
}
\item verb \\
If you say that you would \textbf{lay} bets, odds , or money on something happening , you mean that you are very confident that it will happen .
 \textit{
	\begin{itemize}
	\item I wouldn't lay bets on his still remaining manager after the spring.
	\item I'll lay odds that Dean is at your office right now.
	\end{itemize}
}
\item verb \\
To \textbf{lay} someone means to have sex with them.
 \textit{
	\begin{itemize}
	\end{itemize}
}
\item countable noun \\
\textbf{Lay} is used in expressions such a \textbf{good lay} or an \textbf{easy lay} to describe what someone is like as a sexual partner.
 \textit{
	\begin{itemize}
	\end{itemize}
}
\item  \\
 to lay it on thick \textit{
	\begin{itemize}
	\end{itemize}
}
\item  \\
 to lay oneself open to sth \textit{
	\begin{itemize}
	\end{itemize}
}
\end{enumerate}

\section*{merit}
{\large \color{blue}  merits  meriting  merited  }
\subsection*{Explain}
\begin{enumerate}
\item uncountable noun \\
If something has \textbf{merit} , it has good or worthwhile qualities.
 \textit{
	\begin{itemize}
	\item The argument seemed to have considerable merit.
	\item Box-office success mattered more than artistic merit.
	\item Your feature has the merit of simply stating what has been achieved.
	\end{itemize}
}
\item plural noun \\
The \textbf{merits} of something are its advantages or other good points.
 \textit{
	\begin{itemize}
	\item They have been persuaded of the merits of peace.
	\item ...the technical merits of a film.
	\item It was obvious that, whatever its merits, their work would never be used.
	\end{itemize}
}
\item verb \\
If someone or something \textbf{merits} a particular action or treatment , they deserve it.
 \textit{
	\begin{itemize}
	\item He said he had done nothing wrong to merit a criminal investigation.
	\item Such ideas merit careful consideration.
	\end{itemize}
}
\item  \\
 on merit \textit{
	\begin{itemize}
	\end{itemize}
}
\end{enumerate}

\section*{learn}
{\large \color{blue}  learns  learning  learned  learnt  }
\subsection*{Explain}
\begin{enumerate}
\item verb \\
If you \textbf{learn} something, you obtain knowledge or a skill through studying or training .
 \textit{
	\begin{itemize}
	\item Their children were going to learn English.
	\item He is learning to play the piano.
	\item ...learning how to use new computer systems.
	\item Experienced teachers help you learn quickly.
	\end{itemize}
}
\item verb \\
If you \textbf{learn} of something, you find out about it.
 \textit{
	\begin{itemize}
	\item It was only after his death that she learned of his affair with Betty.
	\item Later I learnt that the house was to be sold.
	\item ...the Admiral, who, on learning who I was, wanted to meet me.
	\end{itemize}
}
\item verb \\
If people \textbf{learn}  \textbf{to}  behave or react in a particular  way , they gradually start to behave in that way as a result of a change in attitudes .
 \textit{
	\begin{itemize}
	\item You have to learn to face your problem.
	\item We are learning how to confront death instead of avoiding its reality.
	\end{itemize}
}
\item verb \\
If you \textbf{learn}  \textbf{from} an unpleasant experience, you change the way you behave so that it does not happen again or so that, if it happens again, you can  deal with it better .
 \textit{
	\begin{itemize}
	\item I am convinced that he has learned from his mistakes.
	\item The company failed to learn any lessons from this experience.
	\end{itemize}
}
\item verb \\
If you \textbf{learn} something such as a poem or a role in a play , you study or repeat the words so that you can remember them.
 \textit{
	\begin{itemize}
	\item He learned this song as an inmate at a Texas prison.
	\end{itemize}
}
\end{enumerate}

\section*{nature}
{\large \color{blue}  natures  }
\subsection*{Explain}
\begin{enumerate}
\item uncountable noun \\
\textbf{Nature} is all the animals, plants, and other things in the world that are not made by people,
and all the events and processes that are not caused by people.
 \textit{
	\begin{itemize}
	\item The most amazing thing about nature is its infinite variety.
	\item ...grasses that grow wild in nature.
	\item ...the ecological balance of nature.
	\end{itemize}
}
\item singular noun \\
The \textbf{nature} of something is its basic quality or character.
 \textit{
	\begin{itemize}
	\item Mr Sharp would not comment on the nature of the issues being investigated.
	\item ...the ambitious nature of the programme.
	\item The protests had been non-political by nature.
	\item The rise of a major power is both economic and military in nature.
	\end{itemize}
}
\item singular noun \\
Someone's \textbf{nature} is their character, which they show by the way they behave .
 \textit{
	\begin{itemize}
	\item Jeya feels that her ambitious nature made her unsuitable for an arranged marriage.
	\item She trusted people. That was her nature.
	\item He was by nature affectionate.
	\end{itemize}
}
\item  \\
 against nature \textit{
	\begin{itemize}
	\end{itemize}
}
\item  \\
 back to nature \textit{
	\begin{itemize}
	\end{itemize}
}
\item  \\
 by its nature \textit{
	\begin{itemize}
	\end{itemize}
}
\item  \\
 call of nature \textit{
	\begin{itemize}
	\end{itemize}
}
\item  \\
 in the nature of things \textit{
	\begin{itemize}
	\end{itemize}
}
\item  \\
 in the nature of sth \textit{
	\begin{itemize}
	\end{itemize}
}
\item  \\
 second nature \textit{
	\begin{itemize}
	\end{itemize}
}
\end{enumerate}

\section*{manage}
{\large \color{blue}  manages  managing  managed  }
\subsection*{Explain}
\begin{enumerate}
\item verb \\
If you \textbf{manage} an organization , business , or system, or the people who work in it, you are responsible for controlling them.
 \textit{
	\begin{itemize}
	\item Within two years he was managing the store.
	\item Most factories in the area are obsolete and badly managed.
	\item There is a lack of confidence in the government's ability to manage the economy.
	\end{itemize}
}
\item verb \\
If you \textbf{manage} time, money, or other resources , you deal with them carefully and do not waste them.
 \textit{
	\begin{itemize}
	\item In a busy world, managing your time is increasingly important.
	\item Josh expects me to manage all the household expenses on very little.
	\end{itemize}
}
\item verb \\
If you \textbf{manage}  \textbf{to} do something, especially something difficult , you succeed in doing it.
 \textit{
	\begin{itemize}
	\item Somehow, he'd managed to persuade Kay to buy one for him.
	\item I managed to pull myself up onto a wet, sloping ledge.
	\item Over the past 12 months the company has managed a 10 per cent improvement.
	\end{itemize}
}
\item verb \\
If you \textbf{manage} , you succeed in coping with a difficult situation .
 \textit{
	\begin{itemize}
	\item She had managed perfectly well without medication for three years.
	\item I am managing, but I could not possibly give up work.
	\item How did your mother manage when your father left?
	\end{itemize}
}
\item verb \\
If you say that you can \textbf{manage} an amount of time or money for something, you mean that you can afford to spend that time or money on it.
 \textit{
	\begin{itemize}
	\item This makes it ideal for those who can only manage a few hours in the morning or evening.
	\item 'All right, I can manage a fiver,' McMinn said with reluctance.
	\end{itemize}
}
\item verb \\
If you say that someone \textbf{managed} a particular response , such as a laugh or a greeting , you mean that it was difficult for them to do it because they were feeling  sad or upset .
 \textit{
	\begin{itemize}
	\item He looked dazed as he spoke to reporters, managing only a weak smile.
	\item He managed a few sentences about his visit to the prison.
	\item Now is the time to forge ahead with all the enthusiasm and optimism that you can
manage.
	\end{itemize}
}
\item  \\
 I will/can manage \textit{
	\begin{itemize}
	\end{itemize}
}
\end{enumerate}

\section*{paradox}
{\large \color{blue}  paradoxes  }
\subsection*{Explain}
\begin{enumerate}
\item countable noun \\
You describe a situation as a \textbf{paradox} when it involves two or more facts or qualities which seem to contradict each other.
 \textit{
	\begin{itemize}
	\item The paradox is that the region's most dynamic economies have the most primitive financial
systems.
	\item The paradox of exercise is that while using a lot of energy it seems to generate
more.
	\item Death itself is a paradox, the end yet the beginning.
	\end{itemize}
}
\item variable noun \\
A \textbf{paradox} is a statement in which it seems that if one part of it is true, the other part of
it cannot be true.
 \textit{
	\begin{itemize}
	\item The story contains many levels of paradox.
	\item Although I'm so successful I'm really rather a failure. That's a paradox, isn't it?
	\end{itemize}
}
\end{enumerate}

\section*{obey}
{\large \color{blue}  obeys  obeying  obeyed  }
\subsection*{Explain}
\begin{enumerate}
\item verb \\
If you \textbf{obey} a person, a command , or an instruction, you do what you are told to do.
 \textit{
	\begin{itemize}
	\item Cissie obeyed her mother without question.
	\item Most people obey the law.
	\item It was still Baker's duty to obey.
	\end{itemize}
}
\end{enumerate}

\section*{price}
{\large \color{blue}  prices  pricing  priced  }
\subsection*{Explain}
\begin{enumerate}
\item countable noun \\
The \textbf{price} of something is the amount of money that you have to pay in order to buy it.
 \textit{
	\begin{itemize}
	\item ...a sharp increase in the price of petrol.
	\item They expected house prices to rise.
	\item Computers haven't come down in price.
	\end{itemize}
}
\item singular noun \\
The \textbf{price} that you pay for something that you want is an unpleasant thing that you have to do or suffer in order to get it.
 \textit{
	\begin{itemize}
	\item We will have to pay a high price for independence.
	\item There may be a price to pay for such relentless activity, perhaps ill health.
	\item He's paying the price for working his body so hard.
	\end{itemize}
}
\item verb \\
If something \textbf{is priced at} a particular amount, the price is set at that amount.
 \textit{
	\begin{itemize}
	\item The shares are expected to be priced at about 330p.
	\item Digital priced the new line at less than half the cost of comparable mainframes.
	\item There is a very reasonably priced menu.
	\end{itemize}
}
\item  \\
 at any price \textit{
	\begin{itemize}
	\end{itemize}
}
\item  \\
 at a price \textit{
	\begin{itemize}
	\end{itemize}
}
\item  \\
 at a price \textit{
	\begin{itemize}
	\end{itemize}
}
\item  \\
 price on sb's head \textit{
	\begin{itemize}
	\end{itemize}
}
\item  \\
 (can't) put a price on sthg \textit{
	\begin{itemize}
	\end{itemize}
}
\item  \\
 what price \textit{
	\begin{itemize}
	\end{itemize}
}
\item  \\
 what price \textit{
	\begin{itemize}
	\end{itemize}
}
\end{enumerate}

\section*{pass}
{\large \color{blue}  passes  passing  passed  }
\subsection*{Explain}
\begin{enumerate}
\item verb \\
To \textbf{pass} someone or something means to go past them without stopping .
 \textit{
	\begin{itemize}
	\item As she passed the library door, the phone began to ring.
	\item Jane stood aside to let her pass.
	\item I sat in the garden and watched the passing cars.
	\end{itemize}
}
\item verb \\
When someone or something \textbf{passes} in a particular direction, they move in that direction.
 \textit{
	\begin{itemize}
	\item He passed through the doorway into Ward B.
	\item He passed down the tunnel.
	\item As the car passed by, I saw them point at me and laugh.
	\end{itemize}
}
\item verb \\
If something such as a road or pipe \textbf{passes} along a particular route, it goes along that route.
 \textit{
	\begin{itemize}
	\item After going over the Col de Vars, the route passes through St-Paul-sur-Ubaye.
	\item The road passes a farmyard.
	\end{itemize}
}
\item verb \\
If you \textbf{pass} something through, over, or round something else, you move or push it through, over,
or round that thing.
 \textit{
	\begin{itemize}
	\item She passed the needle through the rough cloth, back and forth.
	\item 'I don't understand,' the Inspector mumbled, passing a hand through his hair.
	\item He passed a hand wearily over his eyes.
	\end{itemize}
}
\item verb \\
If you \textbf{pass} something \textbf{to} someone, you take it in your hand and give it to them.
 \textit{
	\begin{itemize}
	\item Ken passed the books to Sergeant Parrott.
	\item Pass me that bottle.
	\end{itemize}
}
\item verb \\
If something \textbf{passes} or \textbf{is passed}  \textbf{from} one person \textbf{to} another, the second person then has it instead of the first.
 \textit{
	\begin{itemize}
	\item His mother's small estate had passed to him after her death.
	\item These powers were eventually passed to municipalities.
	\item ...a genetic trait, which can be passed from one generation to the next.
	\end{itemize}
}
\item verb \\
If you \textbf{pass} information \textbf{to} someone, you give it to them because it concerns them.
 \textbf{Pass on} means the same as pass .
 \textit{
	\begin{itemize}
	\item Officials failed to pass vital information to their superiors.
	\item He passed the letters to the Department of Trade and Industry.
	\item I do not know what to do with the information if I cannot pass it on.
	\item From time to time he passed on confidential information to him.
	\item He has written a note asking me to pass on his thanks.
	\end{itemize}
}
\item verb \\
If you \textbf{pass} the ball \textbf{to} someone in your team in a game such as football, basketball , hockey , or rugby , you kick, hit, or throw it to them.
 \textbf{Pass} is also a noun.
 \textit{
	\begin{itemize}
	\item Your partner should then pass the ball back to you.
	\item I passed back to Brendan.
	\item She rolled a short pass to Ashleigh.
	\end{itemize}
}
\item verb \\
When a period of time \textbf{passes} , it happens and finishes.
 \textit{
	\begin{itemize}
	\item He couldn't imagine why he had let so much time pass without contacting her.
	\item As the years passed, her condition worsened.
	\item Several minutes passed before the girls were noticed.
	\end{itemize}
}
\item verb \\
If you \textbf{pass} a period of time in a particular way, you spend it in that way.
 \textit{
	\begin{itemize}
	\item The children passed the time playing in the streets.
	\item To pass the time they sang songs and played cards.
	\end{itemize}
}
\item verb \\
If you \textbf{pass through} a stage of development or a period of time, you experience it.
 \textit{
	\begin{itemize}
	\item The country was passing through a grave crisis.
	\item 'Have you ever been at all religious?'—'No. I never passed through that phase.'
	\end{itemize}
}
\item verb \\
If an amount \textbf{passes} a particular total or level, it becomes greater than that total or level.
 \textit{
	\begin{itemize}
	\item They became the first company in their field to pass the £2 billion turn-over mark.
	\end{itemize}
}
\item verb \\
If someone or something \textbf{passes} a test, they are considered to be of an acceptable standard.
 \textit{
	\begin{itemize}
	\item Kevin has just passed his driving test.
	\item Additives used in processed foods have passed safety tests.
	\item I didn't pass.
	\end{itemize}
}
\item countable noun \\
A \textbf{pass} in an examination, test, or course is a successful result in it.
 \textit{
	\begin{itemize}
	\item An A-level pass in Biology is preferred for all courses.
	\item Passes are graded from 'A' down to 'E'.
	\end{itemize}
}
\item verb \\
If someone in authority \textbf{passes} a person or thing, they declare that they are of an acceptable standard or have reached
an acceptable standard.
 \textit{
	\begin{itemize}
	\item Several popular beaches were found unfit for bathing although the government passed
them.
	\item The medical board would not pass him fit for General Service.
	\end{itemize}
}
\item verb \\
When people in authority \textbf{pass} a new law or a proposal , they formally agree to it or approve it.
 \textit{
	\begin{itemize}
	\item They passed a resolution declaring the republic fully independent.
	\item Throughout the 1580s laws were passed to control new building.
	\end{itemize}
}
\item verb \\
When a judge \textbf{passes} sentence on someone, he or she says what their punishment will be.
 \textit{
	\begin{itemize}
	\item Passing sentence, the judge said it all had the appearance of a con trick.
	\item Before sentence was passed, Mr Mills escaped from jail.
	\end{itemize}
}
\item verb \\
If you \textbf{pass} comment or \textbf{pass} a comment, you say something.
 \textit{
	\begin{itemize}
	\item I don't really know so I could not pass comment on that.
	\item We passed a few remarks about the weather.
	\end{itemize}
}
\item verb \\
If something \textbf{passes}  \textbf{without} comment, or \textbf{passes}  unnoticed , nobody comments on it, reacts to it, or notices it.
 \textit{
	\begin{itemize}
	\item This practice embarrassed Luther, but he let it pass without comment.
	\item The cocktails were so sweet that the strength of them might pass unnoticed.
	\end{itemize}
}
\item verb \\
If someone or something \textbf{passes for} or \textbf{passes as} something that they are not, they are accepted as that thing or mistaken for that
thing.
 \textit{
	\begin{itemize}
	\item Children's toy guns now look so realistic that they can often pass for the real thing.
	\item It is doubtful whether Ted, even with his fluent French, passed for one of the locals.
	\item ...a woman passing as a man.
	\end{itemize}
}
\item verb \\
If someone makes you an offer or asks you a question and you say that you will \textbf{pass}  \textbf{on} it, you mean that you do not want to accept or answer it now.
 \textit{
	\begin{itemize}
	\item I think I'll pass on the hiking next time.
	\item 'You can join us if you like.' Brad shook his head. 'I'll pass, thanks.'
	\end{itemize}
}
\item verb \\
If someone \textbf{passes} water or \textbf{passes} urine, they urinate .
 \textit{
	\begin{itemize}
	\item A sensitive bladder can make you feel the need to pass water frequently.
	\end{itemize}
}
\item countable noun \\
A \textbf{pass} is a document that allows you to do something.
 \textit{
	\begin{itemize}
	\item I got myself a pass into the barracks.
	\end{itemize}
}
\item countable noun \\
A \textbf{pass} is a narrow path or route between mountains.
 \textit{
	\begin{itemize}
	\item The monastery is in a remote mountain pass.
	\end{itemize}
}
\item  \\
 make a pass \textit{
	\begin{itemize}
	\end{itemize}
}
\end{enumerate}

\section*{pride}
{\large \color{blue}  prides  priding  prided  }
\subsection*{Explain}
\begin{enumerate}
\item uncountable noun \\
\textbf{Pride} is a feeling of satisfaction which you have because you or people close to you have
done something good or possess something good.
 \textit{
	\begin{itemize}
	\item ...the sense of pride in a job well done.
	\item We take pride in offering you the highest standards.
	\item They can look back on their endeavours with pride.
	\end{itemize}
}
\item uncountable noun \\
\textbf{Pride} is a sense of the respect that other people have for you, and that you have for yourself.
 \textit{
	\begin{itemize}
	\item Davis had to salvage his pride.
	\item It was a severe blow to Kendall's pride.
	\end{itemize}
}
\item uncountable noun \\
Someone's \textbf{pride} is the feeling that they have that they are better or more important than other people.
 \textit{
	\begin{itemize}
	\item His pride may still be his downfall.
	\end{itemize}
}
\item verb \\
If you \textbf{pride}  \textbf{yourself on} a quality or skill that you have, you are very proud of it.
 \textit{
	\begin{itemize}
	\item Smith prides himself on being able to organise his own life.
	\item Doyle prides himself on his accuracy.
	\end{itemize}
}
\item countable noun \\
A \textbf{pride} of lions is a group of lions that live together.
 \textit{
	\begin{itemize}
	\end{itemize}
}
\item  \\
 one's pride and joy \textit{
	\begin{itemize}
	\end{itemize}
}
\item  \\
 pride of place \textit{
	\begin{itemize}
	\end{itemize}
}
\item  \\
 to swallow one's pride \textit{
	\begin{itemize}
	\end{itemize}
}
\end{enumerate}

\section*{permeate}
{\large \color{blue}  permeates  permeating  permeated  }
\subsection*{Explain}
\begin{enumerate}
\item verb \\
If an idea , feeling , or attitude  \textbf{permeates} a system or \textbf{permeates}  society , it affects every part of it or is present throughout it.
 \textit{
	\begin{itemize}
	\item Bias against women permeates every level of the judicial system.
	\item An obvious change of attitude at the top will permeate through the system.
	\end{itemize}
}
\item verb \\
If something \textbf{permeates} a place, it spreads throughout it.
 \textit{
	\begin{itemize}
	\item The smell of roast chicken permeated the air.
	\item Eventually, the water will permeate through the surrounding concrete.
	\end{itemize}
}
\end{enumerate}

\section*{rear}
{\large \color{blue}  rears  rearing  reared  }
\subsection*{Explain}
\begin{enumerate}
\item singular noun \\
The \textbf{rear} of something such as a building or vehicle is the back part of it.
 \textbf{Rear} is also an adjective .
 \textit{
	\begin{itemize}
	\item He settled back in the rear of the taxi.
	\item ...a stairway in the rear of the building.
	\item Manufacturers have been obliged to fit rear seat belts in all new cars.
	\end{itemize}
}
\item singular noun \\
If you are at the \textbf{rear} of a moving line of people, you are the last person in it.
 \textit{
	\begin{itemize}
	\item Musicians played at the front and rear of the procession.
	\item The Lord Mayor follows at the rear in his gilded coach.
	\end{itemize}
}
\item countable noun \\
Your \textbf{rear} is the part of your body that you sit on.
 \textit{
	\begin{itemize}
	\item He plans to have a dragon tattooed on his rear.
	\end{itemize}
}
\item verb \\
If you \textbf{rear} children, you look after them until they are old enough to look after themselves.
 \textit{
	\begin{itemize}
	\item She reared six children.
	\item I was reared in east Texas.
	\end{itemize}
}
\item verb \\
If you \textbf{rear} a young animal, you keep and look after it until it is old enough to be used for
work or food, or until it can look after itself.
 \textit{
	\begin{itemize}
	\item She spends a lot of time rearing animals.
	\end{itemize}
}
\item verb \\
When a horse \textbf{rears} , it moves the front part of its body upwards , so that its front legs are high in the air and it is standing on its back legs.
 \textbf{Rear up} means the same as rear .
 \textit{
	\begin{itemize}
	\item The horse reared and threw off its rider.
	\item ...an army pony that didn't rear up at the sound of gunfire.
	\end{itemize}
}
\item verb \\
If you say that something such as a building or mountain  \textbf{rears} above you, you mean that is very tall and close to you.
 \textit{
	\begin{itemize}
	\item The exhibition hall reared above me behind a high fence.
	\item The mountains reared up on each side, steep and white.
	\end{itemize}
}
\item  \\
 to bring up the rear \textit{
	\begin{itemize}
	\end{itemize}
}
\item  \\
 to rear its ugly head \textit{
	\begin{itemize}
	\end{itemize}
}
\end{enumerate}

\section*{persecute}
{\large \color{blue}  persecutes  persecuting  persecuted  }
\subsection*{Explain}
\begin{enumerate}
\item verb \\
If someone \textbf{is}  \textbf{persecuted} , they are treated cruelly and unfairly, often because of their race or beliefs.
 \textit{
	\begin{itemize}
	\item Mr Weaver and his family have been persecuted by the authorities for their beliefs.
	\item They began by brutally persecuting the Catholic Church.
	\item ...a persecuted minority.
	\end{itemize}
}
\item verb \\
If you say that someone \textbf{is}  \textbf{persecuting} you, you mean that they are deliberately making your life difficult .
 \textit{
	\begin{itemize}
	\item Local boys constantly persecuted him, throwing stones at his windows.
	\item Vic was bullied by his father and persecuted by his sisters.
	\end{itemize}
}
\end{enumerate}

\section*{rest}
{\large \color{blue}  }
\subsection*{Explain}
\begin{enumerate}
\item quantifier \\
\textbf{The rest} is used to refer to all the parts of something or all the things in a group that remain or that you
have not already  mentioned .
 \textbf{Rest} is also a pronoun .
 \textit{
	\begin{itemize}
	\item It was an experience I will treasure for the rest of my life.
	\item I'm going to throw a party, then invest the rest of the money.
	\item He was unable to travel to Barcelona with the rest of the team.
	\item Only 55 per cent of the raw material is canned. The rest is thrown away.
	\end{itemize}
}
\item  \\
 and the rest/all the rest of it \textit{
	\begin{itemize}
	\end{itemize}
}
\end{enumerate}

\section*{practise}
{\large \color{blue}  practises  practising  practised  }
\subsection*{Explain}
\begin{enumerate}
\item verb \\
If you \textbf{practise} something, you keep doing it regularly in order to be able to do it better .
 \textit{
	\begin{itemize}
	\item Lauren practises the piano every day.
	\item When she wanted to get something right, she would practise and practise and practise.
	\end{itemize}
}
\item verb \\
When people \textbf{practise} something such as a custom , craft , or religion, they take part in the activities associated with it.
 \textit{
	\begin{itemize}
	\item ...countries which practise multi-party politics.
	\item Acupuncture was practised in China as long ago as the third millennium BC.
	\item He was brought up in a family which practised traditional Judaism.
	\end{itemize}
}
\item verb \\
If something cruel is regularly done to people, you can say that it \textbf{is practised on} them.
 \textit{
	\begin{itemize}
	\item There are consistent reports of electrical torture being practised on inmates.
	\end{itemize}
}
\item verb \\
Someone who \textbf{practises}  medicine or law works as a doctor or a lawyer .
 \textit{
	\begin{itemize}
	\item In Belgium only qualified doctors may practise alternative medicine.
	\item He practised as a lawyer there until his retirement.
	\item The ways in which solicitors practise are varied.
	\item An art historian and collector, he was also a practising architect.
	\end{itemize}
}
\end{enumerate}

\section*{radiate}
{\large \color{blue}  radiates  radiating  radiated  }
\subsection*{Explain}
\begin{enumerate}
\item verb \\
If things \textbf{radiate} out \textbf{from} a place, they form a pattern that is like lines drawn from the centre of a circle to various points on its edge .
 \textit{
	\begin{itemize}
	\item ...the various walks which radiate from the Heritage Centre.
	\item From here, contaminated air radiates out to the open countryside.
	\end{itemize}
}
\item verb \\
If you \textbf{radiate} an emotion or quality or if it \textbf{radiates}  \textbf{from} you, people can  see it very clearly in your face and in your behaviour .
 \textit{
	\begin{itemize}
	\item She radiates happiness and health.
	\item Her voice hadn't changed but I felt the anger that radiated from her.
	\end{itemize}
}
\item verb \\
If something \textbf{radiates} heat or light, heat or light comes from it.
 \textit{
	\begin{itemize}
	\item Stoves are meant to radiate heat.
	\end{itemize}
}
\end{enumerate}

\section*{suicide}
{\large \color{blue}  suicides  }
\subsection*{Explain}
\begin{enumerate}
\item variable noun \\
People who commit  \textbf{suicide} deliberately kill themselves because they do not want to continue  living .
 \textit{
	\begin{itemize}
	\item She tried to commit suicide on several occasions.
	\item ...a case of attempted suicide.
	\item ...a growing number of suicides in the community.
	\end{itemize}
}
\item uncountable noun \\
You say that people commit \textbf{suicide} when they deliberately do something which ruins their career or position in society .
 \textit{
	\begin{itemize}
	\item Quite a few have committed social suicide by writing their boring memoirs.
	\item They say it would be political suicide for the party to abstain.
	\end{itemize}
}
\item adjective \\
The people involved in a \textbf{suicide}  attack , mission , or bombing do not expect to survive .
 \textit{
	\begin{itemize}
	\item According to the army, the teenager said he was on a 'suicide mission' for the movement.
	\item ...a suicide bomber.
	\end{itemize}
}
\end{enumerate}

\section*{recruit}
{\large \color{blue}  recruits  recruiting  recruited  }
\subsection*{Explain}
\begin{enumerate}
\item verb \\
If you \textbf{recruit} people for an organization, you select them and persuade them to join it or work for it.
 \textit{
	\begin{itemize}
	\item The police are trying to recruit more black and Asian officers.
	\item In recruiting students to Computer Systems Engineering, the University looks for
evidence of all-round ability.
	\item The museum is now hoping to recruit volunteers to help with its running.
	\end{itemize}
}
\item countable noun \\
A \textbf{recruit} is a person who has recently joined an organization or an army.
 \textit{
	\begin{itemize}
	\end{itemize}
}
\end{enumerate}

\section*{superiority}
{\large \color{blue}  }
\subsection*{Explain}
\begin{enumerate}
\item uncountable noun \\
If one side in a war or conflict has \textbf{superiority} , it has an advantage over its enemy , for example because it has more soldiers or better equipment.
 \textit{
	\begin{itemize}
	\item The U.S. will need a three-to-one superiority in forces to be sure of a successful
attack.
	\item We have air superiority.
	\end{itemize}
}
\end{enumerate}

\section*{refer}
{\large \color{blue}  refers  referring  referred  }
\subsection*{Explain}
\begin{enumerate}
\item verb \\
If you \textbf{refer}  \textbf{to} a particular subject or person, you talk about them or mention them.
 \textit{
	\begin{itemize}
	\item In his speech, he referred to a recent trip to Canada.
	\end{itemize}
}
\item verb \\
If you \textbf{refer}  \textbf{to} someone or something \textbf{as} a particular thing, you use a particular word, expression , or name to mention or describe them.
 \textit{
	\begin{itemize}
	\item Marcia had referred to him as a dear friend.
	\item He simply referred to him as Ronnie.
	\item Our economy is referred to as a free market.
	\end{itemize}
}
\item verb \\
If a word \textbf{refers}  \textbf{to} a particular thing, situation , or idea , it describes it in some way.
 \textit{
	\begin{itemize}
	\item The term electronics refers to electrically-induced action.
	\end{itemize}
}
\item verb \\
If a person who is ill  \textbf{is referred}  \textbf{to} a hospital or a specialist, they are sent there by a doctor in order to be treated .
 \textit{
	\begin{itemize}
	\item Patients are mostly referred to hospital by their general practitioners.
	\item The patient should be referred for tests immediately.
	\end{itemize}
}
\item verb \\
If you \textbf{refer} a task or a problem  \textbf{to} a person or an organization, you formally tell them about it, so that they can deal with it.
 \textit{
	\begin{itemize}
	\item He could refer the matter to the high court.
	\end{itemize}
}
\item verb \\
If you \textbf{refer} someone \textbf{to} a person or organization, you send them there for the help they need .
 \textit{
	\begin{itemize}
	\item Now and then I referred a client to him.
	\end{itemize}
}
\item verb \\
If you \textbf{refer}  \textbf{to} a book or other source of information, you look at it in order to find something out.
 \textit{
	\begin{itemize}
	\item He referred briefly to his notebook.
	\end{itemize}
}
\item verb \\
If you \textbf{refer} someone \textbf{to} a source of information, you tell them the place where they will find the information which they need or which you think will interest them.
 \textit{
	\begin{itemize}
	\item Mr Bryan also referred me to a book by the American journalist Anthony Scaduto.
	\end{itemize}
}
\end{enumerate}

\section*{task}
{\large \color{blue}  tasks  tasking  tasked  }
\subsection*{Explain}
\begin{enumerate}
\item countable noun \\
A \textbf{task} is an activity or piece of work which you have to do, usually as part of a larger
 project .
 \textit{
	\begin{itemize}
	\item Walker had the unenviable task of breaking the bad news to Hill.
	\item She used the day to catch up with administrative tasks.
	\end{itemize}
}
\item verb \\
If you \textbf{are tasked}  \textbf{with} doing a particular activity or piece of work, someone in authority  asks you to do it.
 \textit{
	\begin{itemize}
	\item The minister was tasked with checking that aid money was being spent wisely.
	\end{itemize}
}
\item  \\
 take sb to task \textit{
	\begin{itemize}
	\end{itemize}
}
\end{enumerate}

\section*{resort}
{\large \color{blue}  resorts  resorting  resorted  }
\subsection*{Explain}
\begin{enumerate}
\item verb \\
If you \textbf{resort}  \textbf{to} a course of action that you do not really  approve of, you adopt it because you cannot see any other way of achieving what you want .
 \textit{
	\begin{itemize}
	\item When all else failed, he resorted to violence.
	\item Some schools have resorted to recruiting teachers from overseas.
	\end{itemize}
}
\item uncountable noun \\
If you achieve something without \textbf{resort}  \textbf{to} a particular course of action, you succeed without carrying out that action. To have \textbf{resort}  \textbf{to} a particular course of action means to have to do that action in order to achieve
something.
 \textit{
	\begin{itemize}
	\item Congress must ensure that all peaceful options are exhausted before resort to war.
	\end{itemize}
}
\item countable noun \\
A \textbf{resort} is a place where a lot of people spend their holidays .
 \textit{
	\begin{itemize}
	\item ...the ski resorts.
	\end{itemize}
}
\item  \\
 as a last resort \textit{
	\begin{itemize}
	\end{itemize}
}
\item  \\
 in the last resort \textit{
	\begin{itemize}
	\end{itemize}
}
\end{enumerate}

\section*{tradition}
{\large \color{blue}  traditions  }
\subsection*{Explain}
\begin{enumerate}
\item variable noun \\
A \textbf{tradition} is a custom or belief that has existed for a long time.
 \textit{
	\begin{itemize}
	\item ...the rich traditions of Afro-Cuban music, and dance.
	\item Mary has carried on the family tradition of giving away plants.
	\item The story of King Arthur became part of oral tradition.
	\end{itemize}
}
\item  \\
 in the tradition of \textit{
	\begin{itemize}
	\end{itemize}
}
\end{enumerate}

\section*{restrict}
{\large \color{blue}  restricts  restricting  restricted  }
\subsection*{Explain}
\begin{enumerate}
\item verb \\
If you \textbf{restrict} something, you put a limit on it in order to reduce it or prevent it becoming too great .
 \textit{
	\begin{itemize}
	\item There is talk of restricting the number of students on campus.
	\item ...restricting imports to 0.6 billion pounds of sugar per year.
	\end{itemize}
}
\item verb \\
To \textbf{restrict} the movement or actions of someone or something means to prevent them from moving or acting freely.
 \textit{
	\begin{itemize}
	\item Villagers say the fence would restrict public access to the hills.
	\item The government imprisoned dissidents, forbade travel, and restricted the press.
	\item These dams have restricted the flow of the river downstream.
	\end{itemize}
}
\item verb \\
If you \textbf{restrict} someone or their activities  \textbf{to} one thing, they can only do, have, or deal with that thing. If you \textbf{restrict} them \textbf{to} one place, they cannot go  anywhere else.
 \textit{
	\begin{itemize}
	\item Two knee injuries restricted him to only nine more league appearances.
	\item The women were put on a diet that restricted them to 1,400 calories a day.
	\item For the first two weeks patients are restricted to the grounds.
	\end{itemize}
}
\item verb \\
If you \textbf{restrict} something \textbf{to} a particular group, only that group can do it or have it. If you \textbf{restrict} something \textbf{to} a particular place, it is allowed only in that place.
 \textit{
	\begin{itemize}
	\item The hospital may restrict bookings to people living locally.
	\item Camping is restricted to five designated campgrounds.
	\end{itemize}
}
\end{enumerate}

\section*{vacation}
{\large \color{blue}  vacations  vacationing  vacationed  }
\subsection*{Explain}
\begin{enumerate}
\item countable noun \\
A \textbf{vacation} is a period of the year when universities and colleges , and in the United  States  also  schools , are officially closed.
 \textit{
	\begin{itemize}
	\item During his summer vacation, he visited Russia.
	\item Did you have a lot of reading during the vacation?
	\end{itemize}
}
\item countable noun \\
A \textbf{vacation} is a period of time during which you relax and enjoy yourself away from home .
 \textit{
	\begin{itemize}
	\item They planned a late summer vacation in Europe.
	\item We went on vacation to Puerto Rico.
	\end{itemize}
}
\item uncountable noun \\
If you have a particular  number of days ' or weeks ' \textbf{vacation} , you do not have to go to work for that number of days or weeks.
 \textit{
	\begin{itemize}
	\item They get five to six weeks' vacation a year.
	\end{itemize}
}
\item verb \\
If you \textbf{are vacationing} in a place away from home, you are on vacation there.
 \textit{
	\begin{itemize}
	\item Myles vacationed in Jamaica.
	\item He was vacationing and couldn't be reached for comment.
	\end{itemize}
}
\end{enumerate}

\section*{scan}
{\large \color{blue}  scans  scanning  scanned  }
\subsection*{Explain}
\begin{enumerate}
\item verb \\
When you \textbf{scan} written material, you look through it quickly in order to find important or interesting information.
 \textbf{Scan} is also a noun .
 \textit{
	\begin{itemize}
	\item She scanned the advertisement pages of the newspapers.
	\item I haven't read much into it as yet. I've only just scanned through it.
	\item I just had a quick scan through your book again.
	\end{itemize}
}
\item verb \\
When you \textbf{scan} a place or group of people, you look at it carefully, usually because you are looking
for something or someone.
 \textit{
	\begin{itemize}
	\item The officer scanned the room.
	\item She was nervous and kept scanning the crowd for Paul.
	\item He raised the binoculars to his eye again, scanning across the scene.
	\end{itemize}
}
\item verb \\
If people \textbf{scan} something such as luggage , they examine it using a machine that can show or find things inside it that cannot be seen from the outside.
 \textit{
	\begin{itemize}
	\item Their approach is to scan every checked-in bag with a bomb detector.
	\end{itemize}
}
\item verb \\
If a computer disk  \textbf{is scanned} , a program on the computer checks the disk to make sure that it does not contain a virus .
 \textit{
	\begin{itemize}
	\item The disk has no viruses–I've scanned it already.
	\end{itemize}
}
\item verb \\
If a picture or document  \textbf{is scanned} into a computer, a machine passes a beam of light over it to make a copy of it in the computer.
 \textit{
	\begin{itemize}
	\item The entire paper contents of all libraries will eventually be scanned into computers.
	\item Designs can also be scanned in from paper.
	\end{itemize}
}
\item verb \\
If a radar or sonar machine \textbf{scans} an area, it examines or searches it by sending radar or sonar beams over it.
 \textit{
	\begin{itemize}
	\item The ship's radar scanned the sea ahead.
	\end{itemize}
}
\item countable noun \\
A \textbf{scan} is a medical  test in which a machine sends a beam of X-rays over a part of your body in order to check that it is healthy .
 \textit{
	\begin{itemize}
	\item He was rushed to hospital for a brain scan.
	\item ...a breast scan to check for cancer.
	\end{itemize}
}
\item countable noun \\
If a pregnant woman has a \textbf{scan} , a machine using sound waves produces an image of her womb on a screen so that a doctor can see if her baby is developing normally .
 \textit{
	\begin{itemize}
	\end{itemize}
}
\item verb \\
If a line of a poem does not \textbf{scan} , it is not the right length or does not have emphasis in the right places to match the rest of the poem.
 \textit{
	\begin{itemize}
	\item He had written a few poems. Sid told him they didn't scan.
	\end{itemize}
}
\end{enumerate}

\section*{value}
{\large \color{blue}  values  valuing  valued  }
\subsection*{Explain}
\begin{enumerate}
\item uncountable noun \\
The \textbf{value} of something such as a quality, attitude , or method is its importance or usefulness. If you place a particular \textbf{value} on something, that is the importance or usefulness you think it has.
 \textit{
	\begin{itemize}
	\item The value of this work experience should not be underestimated.
	\item Further studies will be needed to see if these therapies have any value.
	\item Ronnie put a high value on his appearance.
	\end{itemize}
}
\item verb \\
If you \textbf{value} something or someone, you think that they are important and you appreciate them.
 \textit{
	\begin{itemize}
	\item I've done business with Mr Weston before. I value the work he gives me.
	\item If you value your health then you'll start being a little kinder to yourself.
	\end{itemize}
}
\item variable noun \\
The \textbf{value} of something is how much money it is worth.
 \textit{
	\begin{itemize}
	\item The value of his investment has risen by more than $50,000.
	\item The company's market value rose to $5.5 billion.
	\item The country's currency went down in value by 3.5 per cent.
	\item That cup is priceless. You can't put a value on it.
	\end{itemize}
}
\item verb \\
When experts  \textbf{value} something, they decide how much money it is worth.
 \textit{
	\begin{itemize}
	\item Your lender will then send their own surveyor to value the property.
	\item I asked him if he would have my jewellery valued for insurance purposes.
	\item He has been selling properties valued at £700 million.
	\end{itemize}
}
\item uncountable noun \\
You use \textbf{value} in certain expressions to say whether something is worth the money that it costs . For example , if something is or gives \textbf{good value} , it is worth the money that it costs.
 \textit{
	\begin{itemize}
	\item The restaurant is informal, stylish and extremely good value.
	\item Both offer excellent value at around £50 for a double room.
	\item Courses which are offered through these local colleges are fantastic value for money.
	\end{itemize}
}
\item plural noun \\
The \textbf{values} of a person or group are the moral principles and beliefs that they think are important.
 \textit{
	\begin{itemize}
	\item The countries of South Asia also share many common values.
	\item The Health Secretary called for a return to traditional family values.
	\item ...young people's rejection of the values of an older generation.
	\end{itemize}
}
\item uncountable noun \\
\textbf{Value} is used after another noun when mentioning an important or noticeable  feature about something.
 \textit{
	\begin{itemize}
	\item The script has lost all of its shock value over the intervening 24 years.
	\item Having a mid-morning party certainly adds novelty value.
	\end{itemize}
}
\item countable noun \\
A \textbf{value} is a particular number or quantity that can replace a symbol such as ' x ' or ' y ' in a mathematical expression.
 \textit{
	\begin{itemize}
	\end{itemize}
}
\end{enumerate}

\section*{serve}
{\large \color{blue}  serves  serving  served  }
\subsection*{Explain}
\begin{enumerate}
\item verb \\
If you \textbf{serve} your country , an organization , or a person, you do useful work for them.
 \textit{
	\begin{itemize}
	\item It is unfair to soldiers who have served their country well for many years.
	\item I have always said that I would serve the Party in any way it felt appropriate.
	\end{itemize}
}
\item verb \\
If you \textbf{serve} in a particular place or as a particular official , you perform official duties , especially in the armed forces, as a civil  servant , or as a politician .
 \textit{
	\begin{itemize}
	\item During the second world war he served with RAF Coastal Command.
	\item He also served on the National Front's national executive committee.
	\item For seven years until 1991 he served as a district councillor in Solihull.
	\end{itemize}
}
\item verb \\
If something \textbf{serves}  \textbf{as} a particular thing or \textbf{serves} a particular purpose, it performs a particular function, which is often not its intended function.
 \textit{
	\begin{itemize}
	\item She ushered me into the front room, which served as her office.
	\item I really do not think that an inquiry would serve any useful purpose.
	\item Their brief visit has served to underline the deep differences between the two countries.
	\item The old drawing room serves her as both sitting room and study.
	\end{itemize}
}
\item verb \\
If something \textbf{serves} people or an area, it provides them with something that they need .
 \textit{
	\begin{itemize}
	\item This could mean the closure of thousands of small businesses which serve the community.
	\item ...improvements in the public water-supply system serving the Nairobi area.
	\item Cuba is well served by motorways.
	\item ...a desire to make education serve the needs of politicians and business.
	\end{itemize}
}
\item verb \\
Something that \textbf{serves} someone's interests  benefits them.
 \textit{
	\begin{itemize}
	\item The economy should be organized to serve the interests of all the people.
	\item They may well decide that their interests would be best served by joining in.
	\end{itemize}
}
\item verb \\
When you \textbf{serve} food and drink, you give people food and drink.
 \textbf{Serve up}  means the same as serve .
 \textit{
	\begin{itemize}
	\item Serve it with French bread.
	\item Serve the cakes warm.
	\item Prepare the garnishes shortly before you are ready to serve the soup.
	\item ...the pleasure of having someone serve you champagne and caviar in bed.
	\item They are expected to baby-sit, run errands, and help serve at cocktail parties.
	\item After all, it is no use serving up TV dinners if the kids won't eat them.
	\item He served it up on delicate white plates.
	\end{itemize}
}
\item verb \\
\textbf{Serve} is used to indicate how much food a recipe produces. For example , a recipe that \textbf{serves}  six provides enough food for six people.
 \textit{
	\begin{itemize}
	\item Garnish with fresh herbs. Serves 4.
	\end{itemize}
}
\item verb \\
Someone who \textbf{serves} customers in a shop or a bar helps them and provides them with what they want to buy .
 \textit{
	\begin{itemize}
	\item They wouldn't serve me in any pubs 'cos I looked too young.
	\item Auntie and Uncle suggested she serve in the shop.
	\end{itemize}
}
\item verb \\
When the police or other officials \textbf{serve} someone \textbf{with} a legal order or \textbf{serve} an order \textbf{on} them, they give or send the legal order to them.
 \textit{
	\begin{itemize}
	\item Immigration officers tried to serve her with a deportation order.
	\item Police said they had been unable to serve a summons on 25-year-old Lee Jones.
	\end{itemize}
}
\item verb \\
If you \textbf{serve} something such as a prison  sentence or an apprenticeship, you spend a period of time doing it.
 \textit{
	\begin{itemize}
	\item ...Leo, who is currently serving a life sentence for murder.
	\item He was able to serve his apprenticeship as a trainer with Eddie Futch.
	\end{itemize}
}
\item verb \\
When you \textbf{serve} in games such as tennis and badminton , you throw up the ball or shuttlecock and hit it to start play.
 \textbf{Serve} is also a noun .
 \textit{
	\begin{itemize}
	\item He served 17 double faults.
	\item If you serve like this nobody can beat you.
	\item His second serve clipped the net.
	\end{itemize}
}
\item countable noun \\
When you describe someone's \textbf{serve} , you are indicating how well or how fast they serve a ball or shuttlecock.
 \textit{
	\begin{itemize}
	\item His powerful serve was too much for the defending champion.
	\end{itemize}
}
\item  \\
 serve sb right \textit{
	\begin{itemize}
	\end{itemize}
}
\end{enumerate}

\section*{vein}
{\large \color{blue}  veins  }
\subsection*{Explain}
\begin{enumerate}
\item countable noun \\
Your \textbf{veins} are the thin tubes in your body through which your blood flows towards your heart. Compare  artery .
 \textit{
	\begin{itemize}
	\item Many veins are found just under the skin.
	\end{itemize}
}
\item countable noun \\
Something that is written or spoken  \textbf{in} a particular \textbf{vein} is written or spoken in that style or mood.
 \textit{
	\begin{itemize}
	\item It is one of his finest works in a lighter vein.
	\item The girl now replies in similar vein.
	\end{itemize}
}
\item countable noun \\
A \textbf{vein}  \textbf{of} a particular quality is evidence of that quality which someone often shows in their behaviour or work.
 \textit{
	\begin{itemize}
	\item The striker's rich vein of form this season has seen him net 32 goals.
	\item This Spanish drama has a vein of black humour running through it.
	\end{itemize}
}
\item countable noun \\
A \textbf{vein}  \textbf{of} a particular metal or mineral is a layer of it lying in rock.
 \textit{
	\begin{itemize}
	\item ...a vein of copper.
	\item ...a rich and deep vein of limestone.
	\end{itemize}
}
\item countable noun \\
The \textbf{veins} on a leaf are the thin lines on it.
 \textit{
	\begin{itemize}
	\item ...the serrated edges and veins of the feathery leaves.
	\end{itemize}
}
\end{enumerate}

\section*{suggest}
{\large \color{blue}  suggests  suggesting  suggested  }
\subsection*{Explain}
\begin{enumerate}
\item verb \\
If you \textbf{suggest} something, you put forward a plan or idea for someone to think about.
 \textit{
	\begin{itemize}
	\item He suggested a link between class size and test results of seven-year-olds.
	\item I suggest you ask him some specific questions about his past.
	\item I suggested to Mike that we go out for a meal with his colleagues.
	\item No one has suggested how this might occur.
	\item 'Could he be suffering from amnesia?' I suggested.
	\item So instead I suggested taking her out to dinner for a change.
	\end{itemize}
}
\item verb \\
If you \textbf{suggest} the name of a person or place, you recommend them to someone.
 \textit{
	\begin{itemize}
	\item Could you suggest someone to advise me how to do this?
	\item They can suggest where to buy one.
	\end{itemize}
}
\item verb \\
If you \textbf{suggest}  \textbf{that} something is the case , you say something which you believe is the case.
 \textit{
	\begin{itemize}
	\item I'm not suggesting that is what is happening.
	\item It is wrong to suggest that there are easy alternatives.
	\item Their success is conditional, I suggest, on this restriction.
	\end{itemize}
}
\item verb \\
If one thing \textbf{suggests} another, it implies it or makes you think that it might be the case.
 \textit{
	\begin{itemize}
	\item Earlier reports suggested that a meeting would take place on Sunday.
	\item Its hairy body suggests a mammal.
	\end{itemize}
}
\item verb \\
If one thing \textbf{suggests} another, it brings it to your mind through an association of ideas.
 \textit{
	\begin{itemize}
	\item This onomatopoeic word suggests to me the sound a mousetrap makes when it snaps shut.
	\end{itemize}
}
\end{enumerate}

\section*{volt}
{\large \color{blue}  volts  }
\subsection*{Explain}
\begin{enumerate}
\item countable noun \\
A \textbf{volt} is a unit used to measure the force of an electric current.
 \textit{
	\begin{itemize}
	\end{itemize}
}
\end{enumerate}

\section*{tuck}
{\large \color{blue}  tucks  tucking  tucked  }
\subsection*{Explain}
\begin{enumerate}
\item verb \\
If you \textbf{tuck} something somewhere , you put it there so that it is safe , comfortable , or neat.
 \textit{
	\begin{itemize}
	\item He tried to tuck his flapping shirt inside his trousers.
	\item She found a rose tucked under the windscreen wiper of her car one morning.
	\end{itemize}
}
\item uncountable noun \\
\textbf{Tuck} is food that children eat as a snack at school.
 \textit{
	\begin{itemize}
	\item He stole a Mars bar from the school tuck shop.
	\end{itemize}
}
\item countable noun \\
You can use \textbf{tuck} to refer to a form of plastic  surgery which involves reducing the size of a part of someone's body.
 \textit{
	\begin{itemize}
	\item She'd undergone 13 operations, including a tummy tuck.
	\end{itemize}
}
\end{enumerate}

\section*{worth}
{\large \color{blue}  }
\subsection*{Explain}
\begin{enumerate}
\item adjective \\
If something is \textbf{worth} a particular amount of money , it can be sold for that amount or is considered to have that value.
 \textit{
	\begin{itemize}
	\item These books might be worth £80 or £90 or more to a collector.
	\item His mother inherited a farm worth 15,000 dollars a year.
	\item The contract was worth £25 million a year.
	\end{itemize}
}
\item countable noun \\
\textbf{Worth}  combines with amounts of money, so that when you talk about a particular amount of money \textbf{'s}  \textbf{worth of} something, you mean the quantity of it that you can buy for that amount of money.
 \textbf{Worth} is also a pronoun .
 \textit{
	\begin{itemize}
	\item I went and bought about six dollars' worth of potato chips.
	\item The prepaid sim card gives you ten euros-worth of calls.
	\item 'How many do you want?'—'I'll have a pound's worth.'
	\end{itemize}
}
\item countable noun \\
\textbf{Worth} combines with time expressions , so you can use \textbf{worth} when you are saying how long an amount of something will  last . For example , a week's \textbf{worth of}  food is the amount of food that will last you for a week .
 \textbf{Worth} is also a pronoun.
 \textit{
	\begin{itemize}
	\item You've got three years' worth of research money to do what you want with.
	\item The film is his own compilation of more than 50 hours-worth of footage.
	\item There's really not very much food down there. About two weeks' worth.
	\end{itemize}
}
\item adjective \\
If you say that something is \textbf{worth} having, you mean that it is pleasant or useful , and therefore a good thing to have.
 \textit{
	\begin{itemize}
	\item He's decided to get a look at the house and see if it might be worth buying.
	\item If this was what his job required, then the job wasn't really worth having.
	\item Most things worth having never come easy.
	\end{itemize}
}
\item adjective \\
If something is \textbf{worth} a particular action , or if an action is \textbf{worth} doing, it is considered to be important enough for that action.
 \textit{
	\begin{itemize}
	\item No one is worth a great deal of sacrifice.
	\item I am spending a lot of money and time on this boat, but it is worth it.
	\item This restaurant is well worth a visit.
	\item It is worth pausing to consider these statements from Mr Davies.
	\end{itemize}
}
\item uncountable noun \\
Someone's \textbf{worth} is the value, usefulness, or importance that they are considered to have.
 \textit{
	\begin{itemize}
	\item He had never had a woman of her worth as a friend.
	\item The team would have need of a driver of his worth.
	\end{itemize}
}
\item  \\
 for all sb is worth \textit{
	\begin{itemize}
	\end{itemize}
}
\item  \\
 for all it is worth \textit{
	\begin{itemize}
	\end{itemize}
}
\item  \\
 for what it's worth \textit{
	\begin{itemize}
	\end{itemize}
}
\item  \\
 worth your while \textit{
	\begin{itemize}
	\end{itemize}
}
\end{enumerate}

\section*{undertake}
{\large \color{blue}  undertakes  undertaking  undertook  undertaken  }
\subsection*{Explain}
\begin{enumerate}
\item verb \\
When you \textbf{undertake} a task or job , you start doing it and accept  responsibility for it.
 \textit{
	\begin{itemize}
	\item She undertook the arduous task of monitoring the elections.
	\end{itemize}
}
\item verb \\
If you \textbf{undertake}  \textbf{to} do something, you promise that you will do it.
 \textit{
	\begin{itemize}
	\item He undertook to edit the text himself.
	\end{itemize}
}
\end{enumerate}

\section*{accountant}
{\large \color{blue}  accountants  }
\subsection*{Explain}
\begin{enumerate}
\item countable noun \\
An \textbf{accountant} is a person whose job is to keep financial accounts.
 \textit{
	\begin{itemize}
	\end{itemize}
}
\end{enumerate}

\section*{adjoin}
{\large \color{blue}  adjoins  adjoining  adjoined  }
\subsection*{Explain}
\begin{enumerate}
\item verb \\
If one room , place, or object \textbf{adjoins} another, they are next to each other.
 \textit{
	\begin{itemize}
	\item Fields adjoined the garden and there were no neighbours.
	\item We waited in an adjoining office.
	\end{itemize}
}
\end{enumerate}

\section*{alloy}
{\large \color{blue}  alloys  }
\subsection*{Explain}
\begin{enumerate}
\item variable noun \\
An \textbf{alloy} is a metal that is made by mixing two or more types of metal together .
 \textit{
	\begin{itemize}
	\item Bronze is an alloy of copper and tin.
	\item The company produces titanium alloy.
	\end{itemize}
}
\end{enumerate}

\section*{betray}
{\large \color{blue}  betrays  betraying  betrayed  }
\subsection*{Explain}
\begin{enumerate}
\item verb \\
If you \textbf{betray} someone who loves or trusts you, your actions hurt and disappoint them.
 \textit{
	\begin{itemize}
	\item When I tell someone I will not betray his confidence, I keep my word.
	\item The President betrayed them when he went back on his promise not to raise taxes.
	\end{itemize}
}
\item verb \\
If someone \textbf{betrays} their country or their friends, they give information to an enemy, putting their country's security or their friends' safety at risk .
 \textit{
	\begin{itemize}
	\item They offered me money if I would betray my associates.
	\item The group were informers, and they betrayed the plan to the Germans.
	\end{itemize}
}
\item verb \\
If you \textbf{betray} an ideal or your principles , you say or do something which goes against those beliefs .
 \textit{
	\begin{itemize}
	\item We betray the ideals of our country when we support capital punishment.
	\end{itemize}
}
\item verb \\
If you \textbf{betray} a feeling or quality, you show it without intending to.
 \textit{
	\begin{itemize}
	\item She studied his face, but it betrayed nothing.
	\item He nodded his head instead of saying anything where his voice might betray him.
	\end{itemize}
}
\end{enumerate}

\section*{argument}
{\large \color{blue}  arguments  }
\subsection*{Explain}
\begin{enumerate}
\item variable noun \\
An \textbf{argument} is a statement or set of statements that you use in order to try to convince people that your opinion about something is correct .
 \textit{
	\begin{itemize}
	\item There's a strong argument for lowering the price.
	\item The doctors have set out their arguments against the proposals.
	\item It is better to convince by argument than seduce by example.
	\end{itemize}
}
\item variable noun \\
An \textbf{argument} is a discussion or debate in which a number of people put forward different or opposing opinions.
 \textit{
	\begin{itemize}
	\item The incident has triggered fresh arguments about public spending.
	\item The issue has caused heated political argument.
	\end{itemize}
}
\item countable noun \\
An \textbf{argument} is a conversation in which people disagree with each other angrily or noisily.
 \textit{
	\begin{itemize}
	\item Anny described how she got into an argument with one of the marchers.
	\item ...a heated argument.
	\end{itemize}
}
\item uncountable noun \\
If you accept something without \textbf{argument} , you do not question it or disagree with it.
 \textit{
	\begin{itemize}
	\item He complied without argument.
	\item It should of course be given back. There is no argument about that.
	\end{itemize}
}
\end{enumerate}

\section*{choose}
{\large \color{blue}  chooses  choosing  chose  chosen  }
\subsection*{Explain}
\begin{enumerate}
\item verb \\
If you \textbf{choose} someone or something \textbf{from} several people or things that are available , you decide which person or thing you want to have.
 \textit{
	\begin{itemize}
	\item They will be able to choose their own leaders in democratic elections.
	\item ...citizens who had chosen that weekend to begin their holiday.
	\item There are several patchwork cushions to choose from.
	\item Houston was chosen as the site for the convention.
	\item He did well in his chosen profession.
	\end{itemize}
}
\item verb \\
If you \textbf{choose}  \textbf{to} do something, you do it because you want to or because you feel that it is right .
 \textit{
	\begin{itemize}
	\item They knew that discrimination was going on, but chose to ignore it.
	\item You can just take out the interest each year, if you choose.
	\end{itemize}
}
\item  \\
 little to choose between/nothing to choose between \textit{
	\begin{itemize}
	\end{itemize}
}
\item  \\
 the chosen few/a chosen few/someone's chosen \textit{
	\begin{itemize}
	\end{itemize}
}
\end{enumerate}

\section*{baggage}
{\large \color{blue}  }
\subsection*{Explain}
\begin{enumerate}
\item uncountable noun \\
Your \textbf{baggage} consists of the bags that you take with you when you travel .
 \textit{
	\begin{itemize}
	\item The passengers went through immigration control and collected their baggage.
	\item ...excess baggage.
	\end{itemize}
}
\item uncountable noun \\
You can use \textbf{baggage} to refer to someone's emotional  problems , fixed  ideas , or prejudices .
 \textit{
	\begin{itemize}
	\item How much emotional baggage is he bringing with him into the relationship?
	\item Slowly, she shed the ideological baggage of her upbringing.
	\end{itemize}
}
\end{enumerate}

\section*{chop}
{\large \color{blue}  chops  chopping  chopped  }
\subsection*{Explain}
\begin{enumerate}
\item verb \\
If you \textbf{chop} something, you cut it into pieces with strong downward movements of a knife or an axe.
 \textit{
	\begin{itemize}
	\item Chop the butter into small pieces.
	\item Chop the onions very finely.
	\item Visitors were set to work chopping wood.
	\item ...chopped tomatoes.
	\end{itemize}
}
\item countable noun \\
A \textbf{chop} is a small piece of meat cut from the ribs of a sheep or pig .
 \textit{
	\begin{itemize}
	\item ...grilled lamb chops.
	\end{itemize}
}
\item  \\
 chop and change \textit{
	\begin{itemize}
	\end{itemize}
}
\item  \\
 be for the chop/get the chop \textit{
	\begin{itemize}
	\end{itemize}
}
\end{enumerate}

\section*{celebrity}
{\large \color{blue}  celebrities  }
\subsection*{Explain}
\begin{enumerate}
\item countable noun \\
A \textbf{celebrity} is someone who is famous, especially in areas of entertainment such as films, music, writing , or sport .
 \textit{
	\begin{itemize}
	\item He signed his first contract with Universal, changed his name and became a celebrity
almost overnight.
	\item ...a host of celebrities.
	\end{itemize}
}
\item uncountable noun \\
If a person or thing achieves  \textbf{celebrity} , they become famous, especially in areas of entertainment such as films, music, writing,
or sport.
 \textit{
	\begin{itemize}
	\item After 25 years in acting, Joanna is experiencing a level of celebrity for the first
time.
	\end{itemize}
}
\end{enumerate}

\section*{claim}
{\large \color{blue}  claims  claiming  claimed  }
\subsection*{Explain}
\begin{enumerate}
\item verb \\
If you say that someone \textbf{claims}  \textbf{that} something is true, you mean they say that it is true but you are not sure whether or not they are telling the truth .
 \textit{
	\begin{itemize}
	\item He claimed that it was all a conspiracy against him.
	\item A man claiming to be a journalist threatened to reveal details about her private
life.
	\item 'I had never received one single complaint against me,' claimed the humiliated doctor.
	\item He claims a 70 to 80 per cent success rate.
	\end{itemize}
}
\item countable noun \\
A \textbf{claim} is something which someone says which they cannot prove and which may be false .
 \textit{
	\begin{itemize}
	\item He repeated his claim that the people backed his action.
	\item He rejected claims that he had affairs with six women.
	\end{itemize}
}
\item verb \\
If you say that someone \textbf{claims}  responsibility or credit for something, you mean they say that they are responsible for it, but you are not sure whether or not they are telling the truth.
 \textit{
	\begin{itemize}
	\item An underground organisation has claimed responsibility for the bomb explosion.
	\item He was too modest to claim the credit.
	\end{itemize}
}
\item verb \\
If you \textbf{claim} something, you try to get it because you think you have a right to it.
 \textit{
	\begin{itemize}
	\item Now they are returning to claim what was theirs.
	\end{itemize}
}
\item countable noun \\
A \textbf{claim} is a demand for something that you think you have a right to.
 \textit{
	\begin{itemize}
	\item Rival claims to Macedonian territory caused conflict in the Balkans.
	\end{itemize}
}
\item verb \\
If someone \textbf{claims} a record, title, or prize , they gain or win it.
 \textit{
	\begin{itemize}
	\item Zhuang claimed the record in 54.64 seconds.
	\item ...the first time a British man has claimed a world title in the sport.
	\end{itemize}
}
\item countable noun \\
If you have a \textbf{claim on} someone or their attention , you have the right to demand things from them or to demand their attention.
 \textit{
	\begin{itemize}
	\item She'd no claims on him now.
	\item He was surrounded by people, all with claims on his attention.
	\end{itemize}
}
\item verb \\
If something or someone \textbf{claims} your attention, they need you to spend your time and effort on them.
 \textit{
	\begin{itemize}
	\item There is already a long list of people claiming her attention.
	\end{itemize}
}
\item verb \\
If you \textbf{claim} money from the government, an insurance company, or another organization, you officially apply to them for it, because you think you are entitled to it according to their rules.
 \textbf{Claim} is also a noun .
 \textit{
	\begin{itemize}
	\item Some 25 per cent of the people who are entitled to claim State benefits do not do
so.
	\item John had taken out redundancy insurance but when he tried to claim, he was refused
payment.
	\item They intend to claim for damages against the three doctors.
	\item ...the office which has been dealing with their claim for benefit.
	\item Last time we made a claim on our insurance they paid up really quickly.
	\end{itemize}
}
\item verb \\
If you \textbf{claim} money or other benefits from your employers , you demand them because you think you deserve or need them.
 \textbf{Claim} is also a noun.
 \textit{
	\begin{itemize}
	\item The union claimed a pay rise worth four times the rate of inflation.
	\item They are making substantial claims for improved working conditions.
	\item Electricity workers have voted for industrial action in pursuit of a pay claim.
	\end{itemize}
}
\item verb \\
If you say that a war, disease, or accident  \textbf{claims} someone's life, you mean that they are killed in it or by it.
 \textit{
	\begin{itemize}
	\item The civil war claimed the life of a U.N. interpreter yesterday.
	\item Heart disease is the biggest killer, claiming 180,000 lives a year.
	\end{itemize}
}
\item  \\
 claim to fame \textit{
	\begin{itemize}
	\end{itemize}
}
\item  \\
 to lay claim to something \textit{
	\begin{itemize}
	\end{itemize}
}
\end{enumerate}

\section*{chorus}
{\large \color{blue}  choruses  chorusing  chorused  }
\subsection*{Explain}
\begin{enumerate}
\item countable noun \\
A \textbf{chorus} is a part of a song which is repeated after each verse.
 \textit{
	\begin{itemize}
	\item Caroline sang two verses and the chorus of her song.
	\item Everyone joined in the chorus.
	\end{itemize}
}
\item countable noun \\
A \textbf{chorus} is a large group of people who sing together.
 \textit{
	\begin{itemize}
	\item The chorus was singing 'The Ode to Joy'.
	\end{itemize}
}
\item countable noun \\
A \textbf{chorus} is a piece of music written to be sung by a large group of people.
 \textit{
	\begin{itemize}
	\item ...the Hallelujah Chorus.
	\end{itemize}
}
\item countable noun \\
A \textbf{chorus} is a group of singers or dancers who perform together in a show , in contrast to the soloists.
 \textit{
	\begin{itemize}
	\item Students played the lesser parts and sang in the chorus.
	\end{itemize}
}
\item countable noun \\
When there is a \textbf{chorus}  \textbf{of}  criticism , disapproval , or praise , that attitude is expressed by a lot of people at the same time.
 \textit{
	\begin{itemize}
	\item The government is defending its economic policies against a growing chorus of criticism.
	\end{itemize}
}
\item verb \\
When people \textbf{chorus} something, they say it or sing it together.
 \textbf{Chorus} is also a noun .
 \textit{
	\begin{itemize}
	\item 'Hi,' they chorused.
	\item He was greeted with a rousing chorus of Happy Birthday.
	\item 'All the best,' called the other typists in chorus.
	\end{itemize}
}
\end{enumerate}

\section*{designate}
{\large \color{blue}  designates  designating  designated  }
\subsection*{Explain}
\begin{enumerate}
\item verb \\
When you \textbf{designate} someone or something \textbf{as} a particular thing, you formally give them that description or name.
 \textit{
	\begin{itemize}
	\item ...a man interviewed in one of our studies whom we shall designate as E.
	\item There are efforts under way to designate the bridge a historic landmark.
	\item I live in Exmoor, which is designated as a national park.
	\end{itemize}
}
\item verb \\
If something \textbf{is designated}  \textbf{for} a particular purpose , it is set  aside for that purpose.
 \textit{
	\begin{itemize}
	\item Some of the rooms were designated as offices.
	\item ...scholarships designated for minorities.
	\item Smoking is allowed in designated areas.
	\end{itemize}
}
\item verb \\
When you \textbf{designate} someone \textbf{as} something, you formally choose them to do that particular job .
 \textit{
	\begin{itemize}
	\item Designate someone as the spokesperson.
	\item The President's designated successor is his son.
	\end{itemize}
}
\item adjective \\
\textbf{Designate} is used to describe someone who has been formally chosen to do a particular job, but has not yet started doing it.
 \textit{
	\begin{itemize}
	\item Japan's Prime Minister-designate is completing his Cabinet today.
	\end{itemize}
}
\end{enumerate}

\section*{consultant}
{\large \color{blue}  consultants  }
\subsection*{Explain}
\begin{enumerate}
\item countable noun \\
A \textbf{consultant} is an experienced  doctor with a high position, who specializes in one area of medicine.
 \textit{
	\begin{itemize}
	\item Shirley's brother is now a consultant heart surgeon in Sweden.
	\end{itemize}
}
\item countable noun \\
A \textbf{consultant} is a person who gives expert advice to a person or organization on a particular subject.
 \textit{
	\begin{itemize}
	\item He was a consultant to the Swedish government.
	\item ...a team of management consultants sent in to reorganise the department.
	\end{itemize}
}
\end{enumerate}

\section*{despise}
{\large \color{blue}  despises  despising  despised  }
\subsection*{Explain}
\begin{enumerate}
\item verb \\
If you \textbf{despise} something or someone, you dislike them and have a very low  opinion of them.
 \textit{
	\begin{itemize}
	\item I can never, ever forgive him. I despise him.
	\item She secretly despises his work.
	\item How I despised myself for my cowardice!
	\end{itemize}
}
\end{enumerate}

\section*{conversation}
{\large \color{blue}  conversations  }
\subsection*{Explain}
\begin{enumerate}
\item countable noun \\
If you have a \textbf{conversation}  \textbf{with} someone, you talk with them, usually in an informal situation.
 \textit{
	\begin{itemize}
	\item He's a talkative guy, and I struck up a conversation with him.
	\item I waited for her to finish a telephone conversation.
	\end{itemize}
}
\item  \\
 in conversation \textit{
	\begin{itemize}
	\end{itemize}
}
\item  \\
 make conversation \textit{
	\begin{itemize}
	\end{itemize}
}
\end{enumerate}

\section*{dig}
{\large \color{blue}  digs  digging  dug  }
\subsection*{Explain}
\begin{enumerate}
\item verb \\
If people or animals \textbf{dig} , they make a hole in the ground or in a pile of earth, stones, or rubbish .
 \textit{
	\begin{itemize}
	\item They tried digging in a patch just below the cave.
	\item Dig a largish hole and bang the stake in first.
	\item Rescue workers are digging through the rubble in search of other victims.
	\item They dug for shellfish at low tide.
	\item Two men were standing by the freshly dug grave.
	\end{itemize}
}
\item verb \\
If you \textbf{dig}  \textbf{into} something such as a deep  container , you put your hand in it to search for something.
 \textit{
	\begin{itemize}
	\item He dug into his coat pocket for his keys.
	\end{itemize}
}
\item verb \\
If you \textbf{dig} one thing \textbf{into} another or if one thing \textbf{digs}  \textbf{into} another, the first thing is pushed hard into the second, or presses hard into it.
 \textit{
	\begin{itemize}
	\item She digs the serving spoon into the moussaka.
	\item I grab George's arm and dig my nails into his flesh.
	\item He could feel the beads digging into his palm.
	\item Graham was standing there, his hands dug into the pockets of his baggy white trousers.
	\end{itemize}
}
\item verb \\
If you \textbf{dig into} a subject or a store of information, you study it very carefully in order to discover
or check facts.
 \textit{
	\begin{itemize}
	\item The enquiry dug deeper into the alleged financial misdeeds of his government.
	\item He has been digging into the local archives.
	\item With so many books on the subject, one must dig hard for reliable new material.
	\end{itemize}
}
\item verb \\
If you \textbf{dig}  \textbf{yourself out of} a difficult or unpleasant situation, especially one which you caused yourself, you manage to get out of it.
 \textit{
	\begin{itemize}
	\item He's taken these measures to try and dig himself out of a hole.
	\end{itemize}
}
\item verb \\
If you say that you \textbf{dig} something, you mean that you like it and understand it.
 \textit{
	\begin{itemize}
	\item 'They play classic rock'n'roll,' states her boyfriend, 'My dad digs them too.'.
	\item I can dig it. I don't expect a band always to be innovative.
	\end{itemize}
}
\item countable noun \\
A \textbf{dig} is an organized activity in which people dig into the ground in order to discover ancient historical objects.
 \textit{
	\begin{itemize}
	\item He's an archaeologist and has been on a dig in Crete for the past year.
	\end{itemize}
}
\item countable noun \\
If you have a \textbf{dig}  \textbf{at} someone, you say something which is intended to make fun of them or upset them.
 \textit{
	\begin{itemize}
	\item She couldn't resist a dig at Dave after his unfortunate performance.
	\end{itemize}
}
\item countable noun \\
If you give someone a \textbf{dig} in a part of their body, you push them with your finger or your elbow , usually as a warning or as a joke .
 \textit{
	\begin{itemize}
	\item Cassandra silenced him with a sharp dig in the small of the back.
	\end{itemize}
}
\item plural noun \\
If you live  \textbf{in}  \textbf{digs} , you live in a room in someone else's house and pay them rent .
 \textit{
	\begin{itemize}
	\item He went to London and lived in digs in Gloucester Road.
	\end{itemize}
}
\item  \\
 dig deep \textit{
	\begin{itemize}
	\end{itemize}
}
\item  \\
 dig into one's pockets/purse \textit{
	\begin{itemize}
	\end{itemize}
}
\end{enumerate}

\section*{crew}
{\large \color{blue}  crews  crewing  crewed  }
\subsection*{Explain}
\begin{enumerate}
\item countable noun \\
The \textbf{crew} of a ship, an aircraft, or a spacecraft is the people who work on and operate it.
 \textit{
	\begin{itemize}
	\item The mission for the crew of the space shuttle is essentially over.
	\item Despite their size, these vessels carry small crews, usually of around twenty men.
	\item The surviving crew members were ferried ashore.
	\end{itemize}
}
\item countable noun \\
A \textbf{crew} is a group of people with special  technical  skills who work together on a task or project .
 \textit{
	\begin{itemize}
	\item ...a two-man film crew making a documentary.
	\item A paramedic ambulance crew went to the accident scene.
	\end{itemize}
}
\item verb \\
If you \textbf{crew} a boat, you work on it as part of the crew.
 \textit{
	\begin{itemize}
	\item She took part in ocean races and crewed on yachts.
	\item There were to be five teams of three crewing the boat.
	\item ...a fully-crewed yacht.
	\end{itemize}
}
\item singular noun \\
You can use \textbf{crew} to refer to a group of people you disapprove of.
 \textit{
	\begin{itemize}
	\item ...the motley crew of failed and aspiring actors who comprised the 'distinguished
guests'.
	\item ...one of the youngest members of a criminal crew.
	\end{itemize}
}
\end{enumerate}

\section*{elect}
{\large \color{blue}  elects  electing  elected  }
\subsection*{Explain}
\begin{enumerate}
\item verb \\
When people \textbf{elect} someone, they choose that person to represent them, by voting for them.
 \textit{
	\begin{itemize}
	\item The people of the Philippines have voted to elect a new president.
	\item Manchester College elected him Principal in 1956.
	\item The country is about to take a radical departure by electing a woman as its new president.
	\end{itemize}
}
\item verb \\
If you \textbf{elect}  \textbf{to} do something, you choose to do it.
 \textit{
	\begin{itemize}
	\item After six months he elected to take early retirement
	\end{itemize}
}
\item adjective \\
\textbf{Elect} is added after words such as ' president ' or ' governor ' to indicate that a person has been elected to the post but has not officially  started to carry out the duties  involved .
 \textit{
	\begin{itemize}
	\item ...the date when the president-elect takes office.
	\end{itemize}
}
\end{enumerate}

\section*{danger}
{\large \color{blue}  dangers  }
\subsection*{Explain}
\begin{enumerate}
\item uncountable noun \\
\textbf{Danger} is the possibility that someone may be harmed or killed .
 \textit{
	\begin{itemize}
	\item My friends endured tremendous danger in order to help me.
	\item His life could be in danger.
	\end{itemize}
}
\item countable noun \\
A \textbf{danger} is something or someone that can hurt or harm you.
 \textit{
	\begin{itemize}
	\item ...the dangers of smoking.
	\item ...the danger of open conflict.
	\item Britain's roads are a danger to cyclists.
	\item Public health physicians say there are other dangers, too.
	\end{itemize}
}
\item singular noun \\
If there is a \textbf{danger}  \textbf{that} something unpleasant  will  happen , it is possible that it will happen.
 \textit{
	\begin{itemize}
	\item There is a real danger that some people will no longer be able to afford insurance.
	\item There was no danger that any of these groups would be elected to power.
	\item If there is a danger of famine, we should help.
	\end{itemize}
}
\item  \\
 out of danger \textit{
	\begin{itemize}
	\end{itemize}
}
\end{enumerate}

\section*{erect}
{\large \color{blue}  erects  erecting  erected  }
\subsection*{Explain}
\begin{enumerate}
\item verb \\
If people \textbf{erect} something such as a building, bridge , or barrier , they build it or create it.
 \textit{
	\begin{itemize}
	\item Opposition demonstrators have erected barricades in roads leading to the parliament
building.
	\item The building was erected in 1900–1901.
	\item We all unconsciously erect barriers against intimacy.
	\end{itemize}
}
\item verb \\
If you \textbf{erect} a system, a theory , or an institution , you create it.
 \textit{
	\begin{itemize}
	\item Officials have stressed the importance of erecting a solid regional infrastructure
to facilitate trade.
	\item He erected a new doctrine of precedent.
	\item A sham legal and organisational structure appears to have been erected solely to
keep the debts off the Government's books.
	\end{itemize}
}
\item adjective \\
People or things that are \textbf{erect} are straight and upright.
 \textit{
	\begin{itemize}
	\item Stand reasonably erect, your arms hanging naturally.
	\item Her head was erect and her back was straight.
	\item ...the short, stiff, erect stems of almost bead-like blue flowers.
	\end{itemize}
}
\end{enumerate}

\section*{dispute}
{\large \color{blue}  disputes  disputing  disputed  }
\subsection*{Explain}
\begin{enumerate}
\item variable noun \\
A \textbf{dispute} is an argument or disagreement between people or groups.
 \textit{
	\begin{itemize}
	\item They have won previous pay disputes with the government.
	\item Negotiators failed to resolve the bitter dispute between the European Community and
the United States over cutting subsides to farmers.
	\end{itemize}
}
\item verb \\
If you \textbf{dispute} a fact , statement , or theory , you say that it is incorrect or untrue .
 \textit{
	\begin{itemize}
	\item He disputed the allegations.
	\item Nobody disputed that Davey was clever.
	\item Some economists disputed whether consumer spending is as strong as the figures suggest.
	\end{itemize}
}
\item verb \\
When people \textbf{dispute} something, they fight for control or ownership of it. You can also say that one group of people \textbf{dispute} something with another group.
 \textit{
	\begin{itemize}
	\item Russia and Ukraine have been disputing the ownership of the fleet.
	\item Fishermen from Bristol disputed fishing rights with the Danes.
	\item ...a disputed border region.
	\end{itemize}
}
\item  \\
 in dispute \textit{
	\begin{itemize}
	\end{itemize}
}
\item  \\
 in dispute \textit{
	\begin{itemize}
	\end{itemize}
}
\end{enumerate}

\section*{ignite}
{\large \color{blue}  ignites  igniting  ignited  }
\subsection*{Explain}
\begin{enumerate}
\item verb \\
When you \textbf{ignite} something or when it \textbf{ignites} , it starts  burning or explodes .
 \textit{
	\begin{itemize}
	\item The bombs ignited a fire which destroyed some 60 houses.
	\item The blasts were caused by pockets of methane gas that ignited.
	\end{itemize}
}
\item verb \\
If something or someone \textbf{ignites} your feelings , they cause you to have very strong feelings about something.
 \textit{
	\begin{itemize}
	\item There was one teacher who really ignited my interest in words.
	\item The recent fighting in the area could ignite regional passions far beyond the borders.
	\end{itemize}
}
\end{enumerate}

\section*{fame}
{\large \color{blue}  }
\subsection*{Explain}
\begin{enumerate}
\item uncountable noun \\
If you achieve  \textbf{fame} , you become very well-known .
 \textit{
	\begin{itemize}
	\item At the height of his fame, his every word was valued.
	\item The film earned him international fame.
	\item ...her rise to fame and fortune as a dramatist.
	\end{itemize}
}
\end{enumerate}

\section*{indicate}
{\large \color{blue}  indicates  indicating  indicated  }
\subsection*{Explain}
\begin{enumerate}
\item verb \\
If one thing \textbf{indicates} another, the first thing shows that the second is true or exists .
 \textit{
	\begin{itemize}
	\item A survey of retired people has indicated that most are independent and enjoying life.
	\item Our vote today indicates a change in United States policy.
	\item This indicates whether remedies are suitable for children.
	\end{itemize}
}
\item verb \\
If you \textbf{indicate} an opinion , an intention , or a fact , you mention it in an indirect  way .
 \textit{
	\begin{itemize}
	\item Mr Rivers has indicated that he may resign.
	\item U.S. authorities have not yet indicated their monetary policy plans.
	\end{itemize}
}
\item verb \\
If you \textbf{indicate} something to someone, you show them where it is, especially by pointing to it.
 \textit{
	\begin{itemize}
	\item He indicated a chair. 'Sit down.'
	\item Pelham moved across to indicate a wall chart.
	\end{itemize}
}
\item verb \\
If one thing \textbf{indicates} something else, it is a sign of that thing.
 \textit{
	\begin{itemize}
	\item Dreams can help indicate your true feelings.
	\item His language indicates a poor education.
	\end{itemize}
}
\item verb \\
If a technical instrument \textbf{indicates} something, it shows a measurement or reading.
 \textit{
	\begin{itemize}
	\item The needles that indicate your height are at the top right-hand corner.
	\item The temperature gauge indicated that it was boiling.
	\end{itemize}
}
\item verb \\
When drivers  \textbf{indicate} , they make lights  flash on one side of their vehicle to show that they are going to turn in that direction .
 \textit{
	\begin{itemize}
	\item He told us when to indicate and when to change gear.
	\end{itemize}
}
\end{enumerate}

\section*{hazard}
{\large \color{blue}  hazards  hazarding  hazarded  }
\subsection*{Explain}
\begin{enumerate}
\item countable noun \\
A \textbf{hazard} is something which could be dangerous to you, your health or safety , or your plans or reputation .
 \textit{
	\begin{itemize}
	\item A new report suggests that chewing-gum may be a health hazard.
	\item Oil leaking from a barge in the Mississippi River poses a hazard to the drinking
water of New Orleans.
	\end{itemize}
}
\item verb \\
If you \textbf{hazard} someone or something, you put them into a situation which might be dangerous for them.
 \textit{
	\begin{itemize}
	\item He could not believe that, had the Englishman known how much he was at risk, he would
have hazarded his grandson.
	\end{itemize}
}
\item verb \\
If you \textbf{hazard} or if you \textbf{hazard} a \textbf{guess} , you make a suggestion about something which is only a guess and which you know might be wrong .
 \textit{
	\begin{itemize}
	\item I would hazard a guess that they'll do fairly well in the next election.
	\item 'Fifteen or sixteen?' Mrs Dearden hazarded.
	\end{itemize}
}
\end{enumerate}

\section*{locate}
{\large \color{blue}  locates  locating  located  }
\subsection*{Explain}
\begin{enumerate}
\item verb \\
If you \textbf{locate} something or someone, you find out where they are.
 \textit{
	\begin{itemize}
	\item The scientists want to locate the position of the gene on a chromosome.
	\item We've simply been unable to locate him.
	\end{itemize}
}
\item verb \\
If you \textbf{locate} something in a particular place, you put it there or build it there.
 \textit{
	\begin{itemize}
	\item Atlanta was voted the best city in which to locate a business by more than 400 chief
executives.
	\item Tudor Court represents your opportunity to locate at the heart of the new Birmingham.
	\end{itemize}
}
\item verb \\
If you \textbf{locate} in a particular place, you move there or open a business there.
 \textit{
	\begin{itemize}
	\item ...tax breaks for businesses that locate in run-down neighborhoods.
	\end{itemize}
}
\end{enumerate}

\section*{hinge}
{\large \color{blue}  hinges  hinging  hinged  }
\subsection*{Explain}
\begin{enumerate}
\item countable noun \\
A \textbf{hinge} is a piece of metal, wood, or plastic that is used to join a door to its frame or to join two things together so that one of them can swing freely.
 \textit{
	\begin{itemize}
	\item The top swung open on well-oiled hinges.
	\end{itemize}
}
\end{enumerate}

\section*{lodge}
{\large \color{blue}  lodges  lodging  lodged  }
\subsection*{Explain}
\begin{enumerate}
\item countable noun \\
A \textbf{lodge} is a house or hut in the country or in the mountains where people stay on holiday , especially when they want to shoot or fish.
 \textit{
	\begin{itemize}
	\item ...a Victorian hunting lodge.
	\item ...a ski lodge.
	\end{itemize}
}
\item countable noun \\
A \textbf{lodge} is a small house at the entrance to the grounds of a large house.
 \textit{
	\begin{itemize}
	\item I drove out of the gates, past the keeper's lodge.
	\end{itemize}
}
\item countable noun \\
In some organizations, a \textbf{lodge} is a local branch or meeting place of the organization.
 \textit{
	\begin{itemize}
	\item My father would occasionally go to his Masonic lodge.
	\end{itemize}
}
\item verb \\
If you \textbf{lodge} a complaint , protest , accusation, or claim , you officially make it.
 \textit{
	\begin{itemize}
	\item He has four weeks in which to lodge an appeal.
	\end{itemize}
}
\item verb \\
If you \textbf{lodge}  somewhere , such as in someone else's house or if you \textbf{are lodged} there, you live there, usually paying rent .
 \textit{
	\begin{itemize}
	\item ...the story of the farming family she lodged with as a young teacher.
	\item The building he was lodged in turned out to be a church.
	\end{itemize}
}
\item verb \\
If someone \textbf{lodges} you somewhere, they give you a place to stay, for example because they are responsible for your safety or comfort .
 \textit{
	\begin{itemize}
	\item They took me into custody, questioned me, then lodged me in a children's home.
	\item They lodged the delegates in different hotels.
	\end{itemize}
}
\item verb \\
If an object \textbf{lodges} somewhere, it becomes stuck there.
 \textit{
	\begin{itemize}
	\item The bullet lodged in the sergeant's leg, shattering his thigh bone.
	\item His car has a bullet lodged in the passenger door.
	\end{itemize}
}
\item verb \\
If a fact or feeling \textbf{lodges}  \textbf{in} your mind or \textbf{is lodged} there, you remember it for a long time.
 \textit{
	\begin{itemize}
	\item It just lodged in my mind as a very sentimental song.
	\item If you've got something to say it's got to be lodged in their brains at the end.
	\item The festival has lodged itself in the public mind.
	\end{itemize}
}
\end{enumerate}

\section*{jargon}
{\large \color{blue}  }
\subsection*{Explain}
\begin{enumerate}
\item uncountable noun \\
You use \textbf{jargon} to refer to words and expressions that are used in special or technical ways by particular groups of people, often making the language difficult to understand .
 \textit{
	\begin{itemize}
	\item The manual is full of the jargon and slang of self-improvement courses.
	\item ...the reading habits of 600,000 C2 males (marketing jargon for skilled manual workers).
	\end{itemize}
}
\end{enumerate}

\section*{mend}
{\large \color{blue}  mends  mending  mended  }
\subsection*{Explain}
\begin{enumerate}
\item verb \\
If you \textbf{mend} something that is broken or not working , you repair it, so that it works properly or can be used.
 \textit{
	\begin{itemize}
	\item They took a long time to mend the roof.
	\item Somebody else lent me a pump and helped me mend the puncture.
	\item I should have had the catch mended, but never got round to it.
	\end{itemize}
}
\item verb \\
If a person or a part of their body \textbf{mends} or \textbf{is mended} , they get better after they have been ill or have had an injury .
 \textit{
	\begin{itemize}
	\item You'll mend. The X-rays show that your arm will heal all right.
	\item I'm feeling a good bit better. The cut aches, but it's mending.
	\item He must have a major operation on his knee to mend severed ligaments.
	\end{itemize}
}
\item verb \\
If you try to \textbf{mend}  divisions between people, you try to end the disagreements or quarrels between them.
 \textit{
	\begin{itemize}
	\item He sent Evans as his personal envoy to discuss ways to mend relations between the
two countries.
	\item I felt that might well mend the rift between them.
	\end{itemize}
}
\item  \\
 on the mend \textit{
	\begin{itemize}
	\end{itemize}
}
\item  \\
 on the mend \textit{
	\begin{itemize}
	\end{itemize}
}
\item  \\
 mend o's ways \textit{
	\begin{itemize}
	\end{itemize}
}
\end{enumerate}

\section*{luggage}
{\large \color{blue}  }
\subsection*{Explain}
\begin{enumerate}
\item uncountable noun \\
\textbf{Luggage} is the suitcases and bags that you take with you when travel .
 \textit{
	\begin{itemize}
	\item Leave your luggage in the hotel.
	\item Each passenger was allowed two 30-kg pieces of luggage.
	\end{itemize}
}
\end{enumerate}

\section*{nod}
{\large \color{blue}  nods  nodding  nodded  }
\subsection*{Explain}
\begin{enumerate}
\item verb \\
If you \textbf{nod} , you move your head downwards and upwards to show that you are answering 'yes' to a question , or to show agreement, understanding , or approval .
 \textbf{Nod} is also a noun .
 \textit{
	\begin{itemize}
	\item 'Are you okay?' I asked. She nodded and smiled.
	\item David said nothing, but simply nodded, as if understanding perfectly.
	\item Jacques tasted one and nodded his approval.
	\item 'Oh, yes,' she nodded. 'I understand you very well.'
	\item She gave a nod and said, 'I see'.
	\item 'Probably,' agreed Hunter, with a slow nod of his head.
	\item He gave Sabrina a quick nod of acknowledgement.
	\end{itemize}
}
\item verb \\
If you \textbf{nod} in a particular direction , you bend your head once in that direction in order to indicate something or to give
someone a signal .
 \textit{
	\begin{itemize}
	\item 'Does it work?' he asked, nodding at the piano.
	\item She nodded towards the drawing room. 'He's in there.'.
	\item He lifted the end of the canoe, nodding to me to take up mine.
	\end{itemize}
}
\item verb \\
If you \textbf{nod} , you bend your head once, as a way of saying hello or goodbye.
 \textit{
	\begin{itemize}
	\item All the girls nodded and said 'Hi'.
	\item Tom nodded a greeting but didn't say anything.
	\item Both of them smiled and nodded at friends.
	\item They nodded goodnight to the security man.
	\end{itemize}
}
\item verb \\
In football , if a player  \textbf{nods} the ball in a particular direction, they hit the ball there with their head.
 \textit{
	\begin{itemize}
	\item Taylor leapt up to nod the ball home.
	\item He nodded in his twenty-third goal of the season.
	\end{itemize}
}
\item  \\
 give the nod/give sb the nod \textit{
	\begin{itemize}
	\end{itemize}
}
\item  \\
 on the nod \textit{
	\begin{itemize}
	\end{itemize}
}
\end{enumerate}

\section*{magazine}
{\large \color{blue}  magazines  }
\subsection*{Explain}
\begin{enumerate}
\item countable noun \\
A \textbf{magazine} is a publication with a paper cover which is issued regularly, usually every week or every month , and which contains articles, stories , photographs, and advertisements .
 \textit{
	\begin{itemize}
	\item Her face is on the cover of a dozen or more magazines.
	\end{itemize}
}
\item countable noun \\
On radio or television, a \textbf{magazine} or a \textbf{magazine programme} is a programme consisting of several items about different topics , people, and events.
 \textit{
	\begin{itemize}
	\item ...a live arts magazine.
	\item ...'Science In Action', a weekly science magazine programme.
	\end{itemize}
}
\item countable noun \\
In an automatic gun , the \textbf{magazine} is the part that contains the bullets .
 \textit{
	\begin{itemize}
	\end{itemize}
}
\item countable noun \\
A \textbf{magazine} is a building in which things such as guns , bullets, and explosives are kept.
 \textit{
	\begin{itemize}
	\end{itemize}
}
\end{enumerate}

\section*{omit}
{\large \color{blue}  omits  omitting  omitted  }
\subsection*{Explain}
\begin{enumerate}
\item verb \\
If you \textbf{omit} something, you do not include it in an activity or piece of work, deliberately or accidentally.
 \textit{
	\begin{itemize}
	\item Omit the salt in this recipe.
	\item Our apologies to David Pannick for omitting his name from last week's article.
	\end{itemize}
}
\item verb \\
If you \textbf{omit}  \textbf{to} do something, you do not do it.
 \textit{
	\begin{itemize}
	\item His new girlfriend had omitted to tell him she was married.
	\end{itemize}
}
\end{enumerate}

\section*{majority}
{\large \color{blue}  majorities  }
\subsection*{Explain}
\begin{enumerate}
\item singular noun \\
The \textbf{majority} of people or things in a group is more than half of them.
 \textit{
	\begin{itemize}
	\item The majority of my patients come to me from out of town.
	\item The vast majority of our cheeses are made with pasteurised milk.
	\item As a fuel it is preferred by top chefs and is used in the majority of British homes.
	\item Still, a majority continue to support the treaty.
	\end{itemize}
}
\item countable noun \\
A \textbf{majority} is the difference between the number of votes or seats in parliament or legislature that the winner  gets in an election, and the number of votes or seats that the next person or party gets.
 \textit{
	\begin{itemize}
	\item Members of Parliament approved the move by a majority of ninety-nine.
	\item According to most opinion polls, he is set to win a clear majority.
	\item The Trust's annual meeting has decided by a narrow majority to ban deer hunting.
	\end{itemize}
}
\item adjective \\
\textbf{Majority} is used to describe  opinions , decisions , and systems of government that are supported by more than half the people involved.
 \textit{
	\begin{itemize}
	\item ...her continuing disagreement with the majority view.
	\item A majority vote of 75% is required from shareholders for the plan to go ahead.
	\end{itemize}
}
\item uncountable noun \\
\textbf{Majority} is the state of legally being an adult . In Britain and most states in the United States, people reach their majority at the age of eighteen .
 \textit{
	\begin{itemize}
	\item The age of majority in Romania is eighteen.
	\item Once you reach your majority, you may do what you damned well please.
	\end{itemize}
}
\end{enumerate}

\section*{overcome}
{\large \color{blue}  overcomes  overcoming  overcame  }
\subsection*{Explain}
\begin{enumerate}
\item verb \\
If you \textbf{overcome} a problem or a feeling , you successfully deal with it and control it.
 \textit{
	\begin{itemize}
	\item Molly had fought and overcome her fear of flying.
	\item Find a way to overcome your difficulties.
	\end{itemize}
}
\item verb \\
If you \textbf{are overcome}  \textbf{by} a feeling or event, it is so strong or has such a strong effect that you cannot think  clearly .
 \textit{
	\begin{itemize}
	\item The night before the test I was overcome by fear and despair.
	\item A dizziness overcame him, blurring his vision.
	\end{itemize}
}
\item verb \\
If you \textbf{are overcome}  \textbf{by}  smoke or a poisonous  gas , you become very ill or die from breathing it in.
 \textit{
	\begin{itemize}
	\item The residents were trying to escape from the fire but were overcome by smoke.
	\end{itemize}
}
\end{enumerate}

\section*{meet}
{\large \color{blue}  meets  meeting  met  }
\subsection*{Explain}
\begin{enumerate}
\item verb \\
If you \textbf{meet} someone, you happen to be in the same place as them and start  talking to them. You may know the other person, but be surprised to see them, or you may not know them at all.
 \textbf{Meet up} means the same as meet .
 \textit{
	\begin{itemize}
	\item I have just met the man I want to spend the rest of my life with.
	\item He's the kindest and sincerest person I've ever met.
	\item We met by chance.
	\item When he was in the supermarket, he met up with a buddy he had at Oxford.
	\item They met up in 1956, when they were both young schoolboys.
	\end{itemize}
}
\item verb \\
If two or more people \textbf{meet} , they go to the same place, which they have earlier arranged to do, so that they
can talk or do something together.
 \textbf{Meet up} means the same as meet .
 \textit{
	\begin{itemize}
	\item We could meet for a drink after work.
	\item Meet me down at the beach tomorrow, at 6am sharp.
	\item We tend to meet up for lunch once a week.
	\item My intention was to have a holiday and meet up with old friends.
	\end{itemize}
}
\item verb \\
If you \textbf{meet} someone, you are introduced to them and begin talking to them and getting to know them.
 \textit{
	\begin{itemize}
	\item Hey, Terry, come and meet my Dad.
	\end{itemize}
}
\item verb \\
You use \textbf{meet} in expressions such as ' \textbf{Pleased to meet you} ' and ' \textbf{Nice to have met you} ' when you want to politely say hello or goodbye to someone you have just met for the first time.
 \textit{
	\begin{itemize}
	\item 'Jennifer,' Miss Mallory said, 'this is Leigh Van-Voreen.'—'Pleased to meet you,'
Jennifer said.
	\item I have to leave. Nice to have met you.
	\end{itemize}
}
\item verb \\
If you \textbf{meet} someone off their train, plane, or bus , you go to the station , airport , or bus stop in order to be there when they arrive .
 \textit{
	\begin{itemize}
	\item Mama met me at the station.
	\item Lili and my father met me off the boat.
	\item Kurt's parents weren't able to meet our plane so we took a taxi.
	\end{itemize}
}
\item verb \\
When a group of people such as a committee  \textbf{meet} , they gather together for a particular purpose.
 \textit{
	\begin{itemize}
	\item Officials from the two countries will meet again soon to resume negotiations.
	\item The commission met 14 times between 1988 and 1991.
	\end{itemize}
}
\item verb \\
If you \textbf{meet with} someone, you have a meeting with them.
 \textit{
	\begin{itemize}
	\item Most of the lawmakers who met with the president yesterday said they backed the mission.
	\end{itemize}
}
\item verb \\
If something such as a suggestion , proposal , or new book \textbf{meets with} or \textbf{is met with} a particular reaction , it gets that reaction from people.
 \textit{
	\begin{itemize}
	\item The idea met with a cool response from various quarters.
	\item We hope today's offer will meet with your approval too.
	\item Reagan's speech was met with incredulity in the U.S.
	\end{itemize}
}
\item verb \\
If something \textbf{meets} a need , requirement , or condition, it is good enough to do what is required.
 \textit{
	\begin{itemize}
	\item It is up to parents to decide what form of health care best meets their needs.
	\item Out of the original 23,000 applications, 16,000 candidates meet the entry requirements.
	\end{itemize}
}
\item verb \\
If you \textbf{meet} something such as a problem or challenge , you deal with it satisfactorily or do what is required.
 \textit{
	\begin{itemize}
	\item British manufacturing failed to meet the crisis of the 1970s.
	\item It is an enormous challenge but we hope to meet it within a year or 18 months.
	\item They had worked heroically to meet the deadline.
	\end{itemize}
}
\item verb \\
If you \textbf{meet} the cost of something, you provide the money that is needed for it.
 \textit{
	\begin{itemize}
	\item The government said it will help meet some of the cost of the damage.
	\item You need to find the money to meet your monthly repayments.
	\end{itemize}
}
\item verb \\
If you \textbf{meet} a situation, attitude , or problem, you experience it or become aware of it.
 \textit{
	\begin{itemize}
	\item I honestly don't know how I will react the next time I meet a potentially dangerous
situation.
	\item Never had she met such spite and pettiness.
	\end{itemize}
}
\item verb \\
You can say that someone \textbf{meets with}  success or failure when they are successful or unsuccessful .
 \textit{
	\begin{itemize}
	\item Attempts to find civilian volunteers have met with embarrassing failure.
	\item The policy met with little success.
	\end{itemize}
}
\item verb \\
When a moving object \textbf{meets} another object, it hits or touches it.
 \textit{
	\begin{itemize}
	\item You sense the stresses in the hull each time the keel meets the ground.
	\item Nick's head bent slowly over hers until their mouths met.
	\end{itemize}
}
\item verb \\
If your eyes \textbf{meet} someone else's, you both look at each other at the same time.
 \textit{
	\begin{itemize}
	\item Nina's eyes met her sisters' across the table.
	\item I found myself smiling back instinctively when our eyes met.
	\end{itemize}
}
\item verb \\
If two areas \textbf{meet} , especially two areas of land or sea, they are next to one another.
 \textit{
	\begin{itemize}
	\item It is one of the rare places in the world where the desert meets the sea.
	\item ...the southernmost point of South America where the Pacific and Atlantic oceans
meet.
	\end{itemize}
}
\item verb \\
The place where two lines \textbf{meet} is the place where they join together.
 \textit{
	\begin{itemize}
	\item Parallel lines will never meet no matter how far extended.
	\item The track widened as it met the road.
	\end{itemize}
}
\item verb \\
If two sportsmen , teams, or armies  \textbf{meet} , they compete or fight against one another.
 \textit{
	\begin{itemize}
	\item The two women will meet tomorrow in the final.
	\item The unevenly matched armies met at Guilford on 15 March 1781.
	\item ...when England last met the Aussies in a cricket Test match.
	\end{itemize}
}
\item countable noun \\
A \textbf{meet} is an event in which athletes come to a particular place in order to take part in a race or races.
 \textit{
	\begin{itemize}
	\item I was competing at a meet in the National Stadium, Kingston.
	\end{itemize}
}
\item countable noun \\
A \textbf{meet} is when riders and dogs gather somewhere before they set off on a fox hunt.
 \textit{
	\begin{itemize}
	\end{itemize}
}
\item  \\
 meet sb's eyes/gaze \textit{
	\begin{itemize}
	\end{itemize}
}
\item  \\
 meet one's death/meet one's end \textit{
	\begin{itemize}
	\end{itemize}
}
\end{enumerate}

\section*{overlap}
{\large \color{blue}  overlaps  overlapping  overlapped  }
\subsection*{Explain}
\begin{enumerate}
\item verb \\
If one thing \textbf{overlaps} another, or if you \textbf{overlap} them, a part of the first thing occupies the same area as a part of the other thing. You can also  say that two things \textbf{overlap} .
 \textit{
	\begin{itemize}
	\item When the bag is folded flat, the bag bottom overlaps one side of the bag.
	\item Overlap the slices carefully so there are no gaps.
	\item Use vinyl seam adhesive where vinyls overlap.
	\item The edges must overlap each other or weeds will push through the gaps.
	\item ...neat overlapping circles.
	\end{itemize}
}
\item verb \\
If one idea or activity  \textbf{overlaps} another, or \textbf{overlaps} with another, they involve some of the same subjects, people, or periods of time.
 \textbf{Overlap} is also a noun .
 \textit{
	\begin{itemize}
	\item Elizabeth met other Oxford intellectuals, some of whom overlapped Naomi's world.
	\item Christian Holy Week overlaps with the beginning of the Jewish holiday of Passover.
	\item The needs of patients invariably overlap.
	\item Their life-spans overlapped by six years.
	\item ...the overlap between civil and military technology.
	\item We may begin to discover overlaps.
	\end{itemize}
}
\end{enumerate}

\section*{meeting}
{\large \color{blue}  meetings  }
\subsection*{Explain}
\begin{enumerate}
\item countable noun \\
A \textbf{meeting} is an event in which a group of people come together to discuss things or make decisions .
 You can also  refer to the people at a meeting as \textbf{the meeting} .
 \textit{
	\begin{itemize}
	\item Can we have a meeting to discuss that?
	\item ...business meetings.
	\item The meeting decided that further efforts were needed.
	\end{itemize}
}
\item countable noun \\
When you meet someone, either by chance or by arrangement , you can refer to this event as a \textbf{meeting} .
 \textit{
	\begin{itemize}
	\item In January, 37 years after our first meeting, I was back in the studio with Denis.
	\item Her life was changed by a chance meeting with her former art master a few years ago.
	\end{itemize}
}
\end{enumerate}

\section*{overlook}
{\large \color{blue}  overlooks  overlooking  overlooked  }
\subsection*{Explain}
\begin{enumerate}
\item verb \\
If a building or window  \textbf{overlooks} a place, you can see the place clearly from the building or window.
 \textit{
	\begin{itemize}
	\item Pretty and comfortable rooms overlook a flower-filled garden.
	\item Jack Aldwych lived in a huge, old two-storeyed house overlooking Harbord.
	\end{itemize}
}
\item verb \\
If you \textbf{overlook} a fact or problem , you do not notice it, or do not realize how important it is.
 \textit{
	\begin{itemize}
	\item We overlook all sorts of warning signals about our own health.
	\item ...a fact that we all tend to overlook.
	\end{itemize}
}
\item verb \\
If you \textbf{overlook} someone's faults or bad  behaviour , you forgive them and take no action.
 \textit{
	\begin{itemize}
	\item ...satisfying relationships that enable them to overlook each other's faults.
	\end{itemize}
}
\end{enumerate}

\section*{membership}
{\large \color{blue}  memberships  }
\subsection*{Explain}
\begin{enumerate}
\item uncountable noun \\
\textbf{Membership} of an organization is the state of being a member of it.
 \textit{
	\begin{itemize}
	\item The country has also been granted membership of the World Trade Organisation.
	\item He sent me a membership form.
	\item Membership in her church youth group helped develop her political ideas.
	\end{itemize}
}
\item variable noun \\
The \textbf{membership} of an organization is the people who belong to it, or the number of people who belong
to it.
 \textit{
	\begin{itemize}
	\item The European Builders Confederation has a membership of over 350,000 building companies.
	\item ...organizations with huge memberships.
	\item ...the recent fall in party membership.
	\end{itemize}
}
\end{enumerate}

\section*{peel}
{\large \color{blue}  peels  peeling  peeled  }
\subsection*{Explain}
\begin{enumerate}
\item uncountable noun \\
The \textbf{peel} of a fruit such as a lemon or an apple is its skin.
 You can also  refer to a \textbf{peel} .
 \textit{
	\begin{itemize}
	\item ...grated lemon peel.
	\item ...a banana peel.
	\end{itemize}
}
\item verb \\
When you \textbf{peel} fruit or vegetables , you remove their skins.
 \textit{
	\begin{itemize}
	\item She sat down and began peeling potatoes.
	\end{itemize}
}
\item verb \\
If you \textbf{peel}  \textbf{off} something that has been sticking to a surface or if it \textbf{peels}  \textbf{off} , it comes away from the surface.
 \textit{
	\begin{itemize}
	\item One of the kids was peeling plaster off the wall.
	\item It took me two days to peel off the labels.
	\item Paint was peeling off the walls.
	\item The wallpaper was peeling away close to the ceiling.
	\item ...an unrenovated bungalow with slightly peeling blue paint.
	\end{itemize}
}
\item verb \\
If a surface \textbf{is peeling} , the paint on it is coming away.
 \textit{
	\begin{itemize}
	\item Its once-elegant white pillars are peeling.
	\end{itemize}
}
\item verb \\
If you \textbf{are peeling} or if your skin \textbf{is peeling} , small pieces of skin are coming off, usually because you have been burned by the sun .
 \textit{
	\begin{itemize}
	\item His face was peeling from sunburn.
	\end{itemize}
}
\end{enumerate}

\section*{multitude}
{\large \color{blue}  multitudes  }
\subsection*{Explain}
\begin{enumerate}
\item quantifier \\
A \textbf{multitude of} things or people is a very large number of them.
 \textit{
	\begin{itemize}
	\item There are a multitude of small quiet roads to cycle along.
	\item Being inactive can lead to a multitude of health problems.
	\end{itemize}
}
\item countable noun \\
You can refer to a very large number of people as a \textbf{multitude} .
 \textit{
	\begin{itemize}
	\item ...surrounded by a noisy multitude.
	\item ...the multitudes that throng around the Pope.
	\end{itemize}
}
\item countable noun \\
You can refer to the great  majority of people in a particular country or situation as \textbf{the multitude} or \textbf{the multitudes} .
 \textit{
	\begin{itemize}
	\item The hideous truth was hidden from the multitude.
	\item It is our task to convince the multitudes that we are pursuing a lawful hobby in
a lawful way.
	\end{itemize}
}
\end{enumerate}

\section*{pick}
{\large \color{blue}  picks  picking  picked  }
\subsection*{Explain}
\begin{enumerate}
\item verb \\
If you \textbf{pick} a particular person or thing, you choose that one.
 \textit{
	\begin{itemize}
	\item Mr Nowell had picked ten people to interview for six sales jobs in London.
	\item I had deliberately picked a city with a tropical climate.
	\end{itemize}
}
\item singular noun \\
You can refer to the best things or people in a particular group as \textbf{the pick}  \textbf{of} that group.
 \textit{
	\begin{itemize}
	\item The boys here are the pick of the under-15 cricketers in the country.
	\item We had the pick of suits from the shop.
	\end{itemize}
}
\item verb \\
When you \textbf{pick} flowers, fruit, or leaves, you break them off the plant or tree and collect them.
 \textit{
	\begin{itemize}
	\item She used to pick flowers in the Cromwell Road.
	\item He helps his mother pick fruit.
	\end{itemize}
}
\item verb \\
If you \textbf{pick} something from a place, you remove it from there with your fingers or your hand.
 \textit{
	\begin{itemize}
	\item He picked the napkin from his lap and placed it alongside his plate.
	\item He picked the telephone off the wall bracket.
	\end{itemize}
}
\item verb \\
If you \textbf{pick} your \textbf{nose} or \textbf{teeth} , you remove substances from inside your nose or between your teeth.
 \textit{
	\begin{itemize}
	\item Edgar, don't pick your nose, dear.
	\item He had just had a meal and was picking his teeth after it.
	\end{itemize}
}
\item verb \\
If you \textbf{pick} a fight or quarrel  \textbf{with} someone, you deliberately cause one.
 \textit{
	\begin{itemize}
	\item He picked a fight with a waiter and landed in jail.
	\item He was clearly in a mood to pick a quarrel with anybody.
	\end{itemize}
}
\item verb \\
If someone such as a thief  \textbf{picks} a lock, they open it without a key, for example by using a piece of wire .
 \textit{
	\begin{itemize}
	\item He picked each lock deftly, and rifled the papers within each drawer.
	\end{itemize}
}
\item countable noun \\
A \textbf{pick} is the same as a pickaxe .
 \textit{
	\begin{itemize}
	\end{itemize}
}
\item  \\
 to pick and choose \textit{
	\begin{itemize}
	\end{itemize}
}
\item  \\
 have one's pick \textit{
	\begin{itemize}
	\end{itemize}
}
\item  \\
 take one's pick \textit{
	\begin{itemize}
	\end{itemize}
}
\item  \\
 pick one's way \textit{
	\begin{itemize}
	\end{itemize}
}
\end{enumerate}

\section*{muscle}
{\large \color{blue}  muscles  muscling  muscled  }
\subsection*{Explain}
\begin{enumerate}
\item variable noun \\
A \textbf{muscle} is a piece of tissue inside your body which connects two bones and which you use when you make a movement.
 \textit{
	\begin{itemize}
	\item Keeping your muscles strong and in tone helps you to avoid back problems.
	\item He is suffering from a strained thigh muscle.
	\item There are three types of muscle in the body.
	\end{itemize}
}
\item uncountable noun \\
If you say that someone has \textbf{muscle} , you mean that they have power and influence , which enables them to do difficult things.
 \textit{
	\begin{itemize}
	\item Eisenhower used his muscle to persuade Congress to change the law.
	\item The group lacks the financial muscle of its larger rivals.
	\end{itemize}
}
\item  \\
 to flex your muscles \textit{
	\begin{itemize}
	\end{itemize}
}
\item  \\
 to move a muscle \textit{
	\begin{itemize}
	\end{itemize}
}
\end{enumerate}

\section*{pierce}
{\large \color{blue}  pierces  piercing  pierced  }
\subsection*{Explain}
\begin{enumerate}
\item verb \\
If a sharp object \textbf{pierces} something, or if you \textbf{pierce} something \textbf{with} a sharp object, the object goes into it and makes a hole in it.
 \textit{
	\begin{itemize}
	\item One bullet pierced the left side of his chest.
	\item Pierce the skin of the potato with a fork.
	\end{itemize}
}
\item verb \\
If you \textbf{have} your ears or some other part of your body \textbf{pierced} , you have a small hole made through them so that you can wear a piece of jewellery in them.
 \textit{
	\begin{itemize}
	\item I'm having my ears pierced on Saturday.
	\item ...her pierced ears with their tiny gold studs.
	\end{itemize}
}
\item verb \\
If a light or sound \textbf{pierces} something or \textbf{pierces through} it, it is suddenly seen or heard very clearly .
 \textit{
	\begin{itemize}
	\item A spotlight pierced the darkness.
	\item Then he spoke, in a voice that pierced the thick air.
	\item The clock striking the hour pierced through his thoughts.
	\end{itemize}
}
\item verb \\
If a thought , feeling, or sound \textbf{pierces} someone's heart , it makes them experience a feeling, especially sadness, very strongly.
 \textit{
	\begin{itemize}
	\item This sound, like all music, pierced my heart like a dagger.
	\end{itemize}
}
\item verb \\
If someone \textbf{pierces} something that acts as a barrier , they manage to get through it.
 \textit{
	\begin{itemize}
	\end{itemize}
}
\end{enumerate}

\section*{name}
{\large \color{blue}  names  naming  named  }
\subsection*{Explain}
\begin{enumerate}
\item countable noun \\
The \textbf{name} of a person, place, or thing is the word or group of words that is used to identify them.
 \textit{
	\begin{itemize}
	\item 'What's his name?'—'Peter.'
	\item I don't even know if Sullivan's his real name.
	\item They changed the name of the street.
	\end{itemize}
}
\item verb \\
When you \textbf{name} someone or something, you give them a name, usually at the beginning of their life.
 \textit{
	\begin{itemize}
	\item My mother insisted on naming me Horace.
	\item ...a man named John T. Benson.
	\item He won his first race on the aptly named 'Never Say Die'.
	\end{itemize}
}
\item verb \\
If you \textbf{name} someone or something \textbf{after} another person or thing, you give them the same name as that person or thing.
 \textit{
	\begin{itemize}
	\item Why have you not named any of your sons after yourself?
	\end{itemize}
}
\item verb \\
If you \textbf{name} someone, you identify them by stating their name.
 \textit{
	\begin{itemize}
	\item It's nearly thirty years since a journalist was jailed for refusing to name a source.
	\item One of the victims of the weekend's snowstorm has been named as twenty-year-old John
Barr.
	\end{itemize}
}
\item verb \\
If you \textbf{name} something such as a price , time, or place, you say what you want it to be.
 \textit{
	\begin{itemize}
	\item Call Marty, tell him to name his price.
	\end{itemize}
}
\item verb \\
If you \textbf{name} the person for a particular job , you say who you want to have the job.
 \textit{
	\begin{itemize}
	\item The England manager will be naming a new captain.
	\item When the chairman retired, McGovern was named as his successor.
	\item Early in 1941 he was named commander of the Afrika Korps.
	\end{itemize}
}
\item countable noun \\
You can refer to the reputation of a person or thing as their \textbf{name} .
 \textit{
	\begin{itemize}
	\item He had a name for good judgement.
	\item She's never had any drug problems or done anything to give jazz a bad name.
	\end{itemize}
}
\item countable noun \\
You can refer to someone as, for example , a famous \textbf{name} or a great \textbf{name} when they are well-known .
 \textit{
	\begin{itemize}
	\item ...some of the most famous names in modelling and show business.
	\item ...top names such as Nike, Levi's, Calvin Klein and Tommy Hilfiger.
	\end{itemize}
}
\item  \\
 in sb's name/in the name of sb \textit{
	\begin{itemize}
	\end{itemize}
}
\item  \\
 in the name of sb/in sb's name \textit{
	\begin{itemize}
	\end{itemize}
}
\item  \\
 in the name of sth \textit{
	\begin{itemize}
	\end{itemize}
}
\item  \\
 in the name of sb/in the name of sth \textit{
	\begin{itemize}
	\end{itemize}
}
\item  \\
 in all but name \textit{
	\begin{itemize}
	\end{itemize}
}
\item  \\
 by name \textit{
	\begin{itemize}
	\end{itemize}
}
\item  \\
 by name/by the name of something \textit{
	\begin{itemize}
	\end{itemize}
}
\item  \\
 call sb names \textit{
	\begin{itemize}
	\end{itemize}
}
\item  \\
 the name of the game \textit{
	\begin{itemize}
	\end{itemize}
}
\item  \\
 to lend your name to something \textit{
	\begin{itemize}
	\end{itemize}
}
\item  \\
 make a name for oneself/make one's name \textit{
	\begin{itemize}
	\end{itemize}
}
\item  \\
 name names \textit{
	\begin{itemize}
	\end{itemize}
}
\item  \\
 name and shame \textit{
	\begin{itemize}
	\end{itemize}
}
\item  \\
 in name only \textit{
	\begin{itemize}
	\end{itemize}
}
\item  \\
 you name it \textit{
	\begin{itemize}
	\end{itemize}
}
\end{enumerate}

\section*{noun}
{\large \color{blue}  nouns  }
\subsection*{Explain}
\begin{enumerate}
\item countable noun \\
A \textbf{noun} is a word such as ' car ', ' love ', or ' Anne ' which is used to refer to a person or thing.
 \textit{
	\begin{itemize}
	\end{itemize}
}
\end{enumerate}

\section*{prescribe}
{\large \color{blue}  prescribes  prescribing  prescribed  }
\subsection*{Explain}
\begin{enumerate}
\item verb \\
If a doctor  \textbf{prescribes}  medicine or treatment for you, he or she tells you what medicine or treatment to have.
 \textit{
	\begin{itemize}
	\item Our doctor diagnosed a throat infection and prescribed antibiotics and junior aspirin.
	\item She took twice the prescribed dose of sleeping tablets.
	\item The law allows doctors to prescribe contraception to the under 16s.
	\end{itemize}
}
\item verb \\
If a person or set of laws or rules \textbf{prescribes} an action or duty , they state that it must be carried out.
 \textit{
	\begin{itemize}
	\item ...article II of the constitution, which prescribes the method of electing a president.
	\item Alliott told Singleton he was passing the sentence prescribed by law.
	\end{itemize}
}
\end{enumerate}

\section*{pact}
{\large \color{blue}  pacts  }
\subsection*{Explain}
\begin{enumerate}
\item countable noun \\
A \textbf{pact} is a formal agreement between two or more people, organizations, or governments to do a particular
thing or to help each other.
 \textit{
	\begin{itemize}
	\item Stalin signed a non-aggression pact with Nazi Germany in 1939.
	\item The other two opposition parties cannot agree on an electoral pact between themselves.
	\end{itemize}
}
\end{enumerate}

\section*{provoke}
{\large \color{blue}  provokes  provoking  provoked  }
\subsection*{Explain}
\begin{enumerate}
\item verb \\
If you \textbf{provoke} someone, you deliberately annoy them and try to make them behave aggressively.
 \textit{
	\begin{itemize}
	\item He started beating me when I was about fifteen but I didn't do anything to provoke
him.
	\item I provoked him into doing something really stupid.
	\end{itemize}
}
\item verb \\
If something \textbf{provokes} a reaction , it causes it.
 \textit{
	\begin{itemize}
	\item His election success has provoked a shocked reaction.
	\end{itemize}
}
\end{enumerate}

\section*{panorama}
{\large \color{blue}  panoramas  }
\subsection*{Explain}
\begin{enumerate}
\item countable noun \\
A \textbf{panorama} is a view in which you can  see a long way over a wide area of land , usually because you are on high  ground .
 \textit{
	\begin{itemize}
	\item Horton looked out over a panorama of fertile valleys and gentle hills.
	\end{itemize}
}
\item countable noun \\
A \textbf{panorama} is a broad view of a state of affairs or of a constantly changing series of events .
 \textit{
	\begin{itemize}
	\item The play presents a panorama of the history of communism.
	\end{itemize}
}
\end{enumerate}

\section*{rally}
{\large \color{blue}  rallies  rallying  rallied  }
\subsection*{Explain}
\begin{enumerate}
\item countable noun \\
A \textbf{rally} is a large public meeting that is held in order to show support for something such as a political party.
 \textit{
	\begin{itemize}
	\item About three thousand people held a rally to mark international human rights day.
	\item Supporters of the policy are reported to be gathering for a mass rally.
	\end{itemize}
}
\item verb \\
When people \textbf{rally}  \textbf{to} something or when something \textbf{rallies} them, they unite to support it.
 \textit{
	\begin{itemize}
	\item His supporters have rallied to his defence.
	\item He rallied his own supporters for a fight.
	\end{itemize}
}
\item verb \\
When someone or something \textbf{rallies} , they begin to recover or improve after having been weak .
 \textbf{Rally} is also a noun .
 \textit{
	\begin{itemize}
	\item He rallied enough to thank his doctors.
	\item Markets began to rally worldwide.
	\item After a brief rally the shares returned to 126p.
	\end{itemize}
}
\item countable noun \\
A \textbf{rally} is a competition in which vehicles are driven over public roads.
 \textit{
	\begin{itemize}
	\item Between them the pair won the women's section of the Monte Carlo Rally three times.
	\item ...an accomplished rally driver.
	\end{itemize}
}
\item countable noun \\
A \textbf{rally} in tennis , badminton , or squash is a continuous series of shots that the players exchange without stopping .
 \textit{
	\begin{itemize}
	\item ...a long rally.
	\end{itemize}
}
\end{enumerate}

\section*{partner}
{\large \color{blue}  partners  partnering  partnered  }
\subsection*{Explain}
\begin{enumerate}
\item countable noun \\
Your \textbf{partner} is the person you are married to or are having a romantic or sexual relationship with.
 \textit{
	\begin{itemize}
	\item Wanting other friends doesn't mean you don't love your partner.
	\item ...his choice of marriage partner.
	\end{itemize}
}
\item countable noun \\
Your \textbf{partner} is the person you are doing something with, for example  dancing with or playing with in a game against two other people.
 \textit{
	\begin{itemize}
	\item ...to dance with a partner.
	\item My partner for the event was the marvellous American player.
	\item ...a partner in crime.
	\end{itemize}
}
\item countable noun \\
The \textbf{partners} in a firm or business are the people who share the ownership of it.
 \textit{
	\begin{itemize}
	\item He's a partner in a Chicago law firm.
	\item ...her business partner Max Hampshire.
	\end{itemize}
}
\item countable noun \\
The \textbf{partner} of a country or organization is another country or organization with which they work
or do business.
 \textit{
	\begin{itemize}
	\item Spain has been one of the country's major trading partners.
	\item The party will have to find a coalition partner in order to form a government.
	\end{itemize}
}
\item verb \\
If you \textbf{partner} someone, you are their partner in a game or in a dance.
 \textit{
	\begin{itemize}
	\item He had partnered the famous Russian ballerina.
	\item He will be partnered by the defending champion.
	\item She partnered him to a 6-1 first-set success.
	\end{itemize}
}
\end{enumerate}

\section*{recite}
{\large \color{blue}  recites  reciting  recited  }
\subsection*{Explain}
\begin{enumerate}
\item verb \\
When someone \textbf{recites} a poem or other piece of writing , they say it aloud after they have learned it.
 \textit{
	\begin{itemize}
	\item They recited poetry to one another.
	\end{itemize}
}
\item verb \\
If you \textbf{recite} something such as a list , you say it aloud.
 \textit{
	\begin{itemize}
	\item All he could do was recite a list of Government failings.
	\item She suddenly realized that Wim was reciting Kirk's telephone number.
	\end{itemize}
}
\end{enumerate}

\section*{personnel}
{\large \color{blue}  }
\subsection*{Explain}
\begin{enumerate}
\item plural noun \\
The \textbf{personnel} of an organization are the people who work for it.
 \textit{
	\begin{itemize}
	\item Since 1954 Japan has never dispatched military personnel abroad.
	\item There has been very little renewal of personnel in higher education.
	\item He learnt a lot about personnel management, budgeting and account-keeping.
	\end{itemize}
}
\item uncountable noun \\
\textbf{Personnel} is the department in a large company or organization that deals with employees, keeps their records, and helps with any problems they might have.
 \textit{
	\begin{itemize}
	\item Her first job was in personnel.
	\end{itemize}
}
\end{enumerate}

\section*{reclaim}
{\large \color{blue}  reclaims  reclaiming  reclaimed  }
\subsection*{Explain}
\begin{enumerate}
\item verb \\
If you \textbf{reclaim} something that you have lost or that has been taken away from you, you succeed in getting it back.
 \textit{
	\begin{itemize}
	\item In 1986, they got the right to reclaim South African citizenship.
	\item 'I've come to reclaim my property,' she announced to the desk clerk.
	\end{itemize}
}
\item verb \\
If you \textbf{reclaim} an amount of money, for example  tax that you have paid , you claim it back.
 \textit{
	\begin{itemize}
	\item There are eight million people currently eligible to reclaim income tax.
	\end{itemize}
}
\item verb \\
When people \textbf{reclaim} land, they make it suitable for a purpose such as farming or building, for example by draining it or by building a barrier against the sea.
 \textit{
	\begin{itemize}
	\item The Netherlands has been reclaiming farmland from water.
	\item ...a scheme to build a residential development on reclaimed land.
	\end{itemize}
}
\item verb \\
If a piece of land that was used for farming or building \textbf{is reclaimed}  \textbf{by} a desert, forest , or the sea, it turns back into desert, forest, or sea.
 \textit{
	\begin{itemize}
	\item The diamond towns are gradually being reclaimed by the desert.
	\item This method of spraying would allow the land to be reclaimed by the rain forests.
	\end{itemize}
}
\item verb \\
If you \textbf{reclaim} a person who has been involved in bad or criminal  behaviour , you cause them to stop acting in that way.
 \textit{
	\begin{itemize}
	\item He set out to fight crime by reclaiming youths from local gangs.
	\end{itemize}
}
\end{enumerate}

\section*{planet}
{\large \color{blue}  planets  }
\subsection*{Explain}
\begin{enumerate}
\item countable noun \\
A \textbf{planet} is a large, round object in space that moves around a star. The Earth is a planet.
 \textit{
	\begin{itemize}
	\item The picture shows six of the nine planets in the solar system.
	\end{itemize}
}
\end{enumerate}

\section*{repeat}
{\large \color{blue}  repeats  repeating  repeated  }
\subsection*{Explain}
\begin{enumerate}
\item verb \\
If you \textbf{repeat} something, you say or write it again. You can say \textbf{I repeat} to show that you feel strongly about what you are repeating.
 \textit{
	\begin{itemize}
	\item He repeated that he had been mis-quoted.
	\item I repeat that medicine is on the brink of a revolution.
	\item The President repeated his call for the release of hostages.
	\item 'You fool,' she kept repeating.
	\end{itemize}
}
\item verb \\
If you \textbf{repeat} something that someone else has said or written, you say or write the same thing, or tell it to another person.
 \textit{
	\begin{itemize}
	\item She had an irritating habit of repeating everything I said to her.
	\item I trust you not to repeat that to anyone else.
	\item Repeat after me: 'Tomorrow is just another day.'
	\end{itemize}
}
\item verb \\
If you \textbf{repeat}  \textbf{yourself} , you say something which you have said before, usually by mistake .
 \textit{
	\begin{itemize}
	\item Then he started rambling and repeating himself.
	\end{itemize}
}
\item convention \\
People sometimes say \textbf{repeat} before saying again a word they have just said, in order to emphasize it or to make sure that people hear it.
 \textit{
	\begin{itemize}
	\item We are not, I repeat, not actually in the negotiating process.
	\item Find and destroy, repeat destroy, these units.
	\end{itemize}
}
\item verb \\
If you \textbf{repeat} an action, you do it again.
 \textit{
	\begin{itemize}
	\item The next day I repeated the procedure.
	\item He said Japan would never repeat its mistakes.
	\item Hold this position for 30 seconds, release and repeat on the other side.
	\end{itemize}
}
\item verb \\
If an event or series of events \textbf{repeats}  \textbf{itself} , it happens again.
 \textit{
	\begin{itemize}
	\item The U.N. will have to work hard to stop history repeating itself.
	\item The cycle then repeats itself.
	\end{itemize}
}
\item countable noun \\
If there is a \textbf{repeat}  \textbf{of} an event, usually an undesirable event, it happens again.
 \textit{
	\begin{itemize}
	\item There were fears that there might be a repeat of last year's campaign of strikes.
	\item ...in order to prevent a repeat tragedy.
	\end{itemize}
}
\item adjective \\
If a company  gets  \textbf{repeat}  business or \textbf{repeat}  customers , people who have bought their goods or services before buy them again.
 \textit{
	\begin{itemize}
	\item Nearly 60% of our bookings come from repeat business and personal recommendation.
	\end{itemize}
}
\item countable noun \\
A \textbf{repeat} is a television or radio programme that has been broadcast before.
 \textit{
	\begin{itemize}
	\item There's nothing except sport and repeats on TV.
	\end{itemize}
}
\item  \\
 a repeat performance \textit{
	\begin{itemize}
	\end{itemize}
}
\end{enumerate}

\section*{proceeding}
{\large \color{blue}  proceedings  }
\subsection*{Explain}
\begin{enumerate}
\item countable noun \\
Legal \textbf{proceedings} are legal action taken against someone.
 \textit{
	\begin{itemize}
	\item ...criminal proceedings against the former prime minister.
	\item The Council had brought proceedings to stop the store from trading on Sundays.
	\end{itemize}
}
\item countable noun \\
\textbf{The proceedings} are an organized  series of events that take place in a particular place.
 \textit{
	\begin{itemize}
	\item The proceedings of the enquiry will take place in private.
	\item He viewed the proceedings with doubt and alarm.
	\end{itemize}
}
\item plural noun \\
You can refer to a written  record of the discussions at a meeting or conference as \textbf{the proceedings} .
 \textit{
	\begin{itemize}
	\item The Department of Transport is to publish the conference proceedings.
	\end{itemize}
}
\end{enumerate}

\section*{revise}
{\large \color{blue}  revises  revising  revised  }
\subsection*{Explain}
\begin{enumerate}
\item verb \\
If you \textbf{revise} the way you think about something, you adjust your thoughts , usually in order to make them better or more suited to how things are.
 \textit{
	\begin{itemize}
	\item He soon came to revise his opinion of the profession.
	\end{itemize}
}
\item verb \\
If you \textbf{revise} a price , amount, or estimate , you change it to make it more fair , realistic , or accurate .
 \textit{
	\begin{itemize}
	\item Some of their prices were higher than their competitors' so they revised their prices
accordingly.
	\item It was right that estimates were revised when new information became available.
	\end{itemize}
}
\item verb \\
When you \textbf{revise} an article , a book, a law, or a piece of music, you change it in order to improve it, make it more modern , or make it more suitable for a particular purpose .
 \textit{
	\begin{itemize}
	\item Three editors handled the work of revising the articles for publication.
	\item The staff should work together to revise the school curriculum.
	\end{itemize}
}
\item verb \\
When you \textbf{revise}  \textbf{for} an examination, you read things again and make notes in order to be prepared for the examination.
 \textit{
	\begin{itemize}
	\item I have to revise for maths.
	\item I'd better skip the party and stay at home to revise.
	\end{itemize}
}
\end{enumerate}

\section*{queue}
{\large \color{blue}  queues  queuing  queued  }
\subsection*{Explain}
\begin{enumerate}
\item countable noun \\
A \textbf{queue} is a line of people or vehicles that are waiting for something.
 \textit{
	\begin{itemize}
	\item I watched as he got a tray and joined the queue.
	\item She waited in the bus queue.
	\item There was still a queue for tickets on the night.
	\item Behind him was a long queue of angry motorists.
	\end{itemize}
}
\item countable noun \\
If you say there is a \textbf{queue}  \textbf{of} people who want to do or have something, you mean that a lot of people are waiting for an opportunity to do it or have it.
 \textit{
	\begin{itemize}
	\item Manchester United would be at the front of a queue of potential buyers.
	\item Single parents got priority in the housing queue.
	\item The queue for places at the school has never been longer.
	\end{itemize}
}
\item verb \\
When people \textbf{queue} , they stand in a line waiting for something.
 \textbf{Queue up} means the same as queue .
 \textit{
	\begin{itemize}
	\item I had to queue for quite a while.
	\item ...a line of women queueing for bread.
	\item A mob of journalists are queuing up at the gate to photograph him.
	\item We all had to queue up for our ration books.
	\end{itemize}
}
\item countable noun \\
A \textbf{queue} is a list of computer tasks which will be done in order.
 \textit{
	\begin{itemize}
	\item Your print job has been sent to the network print queue.
	\end{itemize}
}
\item verb \\
To \textbf{queue} a number of computer tasks means to arrange them to be done in order.
 \textit{
	\begin{itemize}
	\end{itemize}
}
\end{enumerate}

\section*{risk}
{\large \color{blue}  risks  risking  risked  }
\subsection*{Explain}
\begin{enumerate}
\item variable noun \\
If there is a \textbf{risk}  \textbf{of} something unpleasant , there is a possibility that it will  happen .
 \textit{
	\begin{itemize}
	\item There is a small risk of brain damage from the procedure.
	\item In all the confusion, there's a serious risk that the main issues will be forgotten.
	\item ...mentally disordered women who pose a serious risk to the public.
	\item I suppose people do it because there is that element of danger and risk.
	\item Obesity is a major risk factor in many diseases.
	\end{itemize}
}
\item countable noun \\
If something that you do is a \textbf{risk} , it might have unpleasant or undesirable results.
 \textit{
	\begin{itemize}
	\item You're taking a big risk showing this to Kravis.
	\item This was one risk that paid off.
	\end{itemize}
}
\item countable noun \\
If you say that something or someone is a \textbf{risk} , you mean they are likely to cause harm .
 \textit{
	\begin{itemize}
	\item It's being overfat that constitutes a health risk.
	\item The restaurant has been refurbished–it was found to be a fire risk.
	\item He was not seen as a risk to national security.
	\end{itemize}
}
\item countable noun \\
If you are considered a good \textbf{risk} , a bank or shop  thinks that it is safe to lend you money or let you have goods without paying for them at the time.
 \textit{
	\begin{itemize}
	\item Before providing the cash, they will have to decide whether you are a good or bad
risk.
	\item If you are considered a credit risk, a secured loan might be your only alternative.
	\end{itemize}
}
\item verb \\
If you \textbf{risk} something unpleasant, you do something which might result in that thing happening or affecting you.
 \textit{
	\begin{itemize}
	\item Those who fail to register risk severe penalties.
	\end{itemize}
}
\item verb \\
If you \textbf{risk} doing something, you do it, even though you know that it might have undesirable consequences .
 \textit{
	\begin{itemize}
	\item The captain was not willing to risk taking his ship through the straits in such bad
weather.
	\item At the top, I risked a glance back.
	\item Don't risk it. It isn't worth it.
	\end{itemize}
}
\item verb \\
If you \textbf{risk} your life or something else important , you behave in a way that might result in it being lost or harmed.
 \textit{
	\begin{itemize}
	\item She risked her own life to help a disabled woman.
	\item Why should he have risked all that to become an agent of a foreign power?
	\end{itemize}
}
\item  \\
 at risk \textit{
	\begin{itemize}
	\end{itemize}
}
\item  \\
 at the risk of sth \textit{
	\begin{itemize}
	\end{itemize}
}
\item  \\
 at one's own risk \textit{
	\begin{itemize}
	\end{itemize}
}
\item  \\
 to run a risk \textit{
	\begin{itemize}
	\end{itemize}
}
\end{enumerate}

\section*{reputation}
{\large \color{blue}  reputations  }
\subsection*{Explain}
\begin{enumerate}
\item countable noun \\
To have a \textbf{reputation} for something means to be known or remembered for it.
 \textit{
	\begin{itemize}
	\item She has a reputation for being a very depressing writer.
	\item ...Barcelona's reputation as a design-conscious, artistic city.
	\end{itemize}
}
\item countable noun \\
Something's or someone's \textbf{reputation} is the opinion that people have about how good they are. If they have a good reputation,
people think they are good.
 \textit{
	\begin{itemize}
	\item This college has a good academic reputation.
	\item The stories ruined his reputation.
	\end{itemize}
}
\item  \\
 by reputation \textit{
	\begin{itemize}
	\end{itemize}
}
\end{enumerate}

\section*{select}
{\large \color{blue}  selects  selecting  selected  }
\subsection*{Explain}
\begin{enumerate}
\item verb \\
If you \textbf{select} something, you choose it from a number of things of the same kind .
 \textit{
	\begin{itemize}
	\item Voters are selecting candidates for both U.S. Senate seats and for 52 congressional
seats.
	\item With a difficult tee shot, select a club which will keep you short of the trouble.
	\item The movie is being shown in selected cities.
	\end{itemize}
}
\item verb \\
If you \textbf{select} a file or a piece of text on a computer  screen , you click on it so that it is marked in a different  colour , usually in order for you to give the computer an instruction relating to that file or piece of text.
 \textit{
	\begin{itemize}
	\item I selected a file and pressed the Delete key.
	\end{itemize}
}
\item adjective \\
A \textbf{select} group is a small group of some of the best people or things of their kind.
 \textit{
	\begin{itemize}
	\item He was one of the small select group assembled by Penney, at the High Explosive Research
centre.
	\item ...a select group of French cheeses.
	\item He will join a select band of quarterbacks to win the Super Bowl three times.
	\end{itemize}
}
\item adjective \\
If you describe something as \textbf{select} , you mean it has many desirable  features , but is available only to people who have a lot of money or who belong to a high  social  class .
 \textit{
	\begin{itemize}
	\item The couturier is throwing a very lavish and very select party.
	\item ...a meeting of a very select club.
	\end{itemize}
}
\end{enumerate}

\section*{rib}
{\large \color{blue}  ribs  ribbing  ribbed  }
\subsection*{Explain}
\begin{enumerate}
\item countable noun \\
Your \textbf{ribs} are the 12 pairs of curved bones that surround your chest.
 \textit{
	\begin{itemize}
	\item Her heart was thumping against her ribs.
	\item My face was covered with bruises and I had a broken rib.
	\end{itemize}
}
\item countable noun \\
A \textbf{rib}  \textbf{of} meat such as beef or pork is a piece that has been cut to include one of the animal's ribs.
 \textit{
	\begin{itemize}
	\item ...a rib of beef.
	\item ...pork ribs in sweet sauce.
	\end{itemize}
}
\item uncountable noun \\
\textbf{Rib} is a method of knitting that makes a raised pattern of parallel lines. You use rib, for example , round the edge of sweaters so that the material can stretch without losing its shape.
 \textit{
	\begin{itemize}
	\end{itemize}
}
\item verb \\
If you \textbf{rib} someone \textbf{about} something, you tease them about it in a friendly way.
 \textit{
	\begin{itemize}
	\item The guys in my local pub used to rib me about drinking 'girly' drinks.
	\end{itemize}
}
\end{enumerate}

\section*{specify}
{\large \color{blue}  specifies  specifying  specified  }
\subsection*{Explain}
\begin{enumerate}
\item verb \\
If you \textbf{specify} something, you give information about what is required or should happen in a certain situation .
 \textit{
	\begin{itemize}
	\item They specified a spacious entrance hall.
	\item He has not specified what action he would like them to take.
	\end{itemize}
}
\item verb \\
If you \textbf{specify} what should happen or be done, you explain it in an exact and detailed way.
 \textit{
	\begin{itemize}
	\item Each recipe specifies the size of egg to be used.
	\item One rule specifies that learner drivers must be supervised by adults.
	\item Patients eat together at a specified time.
	\end{itemize}
}
\end{enumerate}

\section*{staff}
{\large \color{blue}  staffs  staffing  staffed  }
\subsection*{Explain}
\begin{enumerate}
\item countable noun \\
The \textbf{staff} of an organization are the people who work for it.
 \textit{
	\begin{itemize}
	\item The staff were very good.
	\item The outpatient program has a staff of six people.
	\item He thanked his staff.
	\item ...members of staff.
	\item Many employers seek diversity in their staffs.
	\end{itemize}
}
\item plural noun \\
People who are part of a particular staff are often referred to as \textbf{staff} .
 \textit{
	\begin{itemize}
	\item 10 staff were allocated to the task.
	\item He had the complete support of hospital staff.
	\end{itemize}
}
\item verb \\
If an organization \textbf{is staffed}  \textbf{by} particular people, they are the people who work for it.
 \textit{
	\begin{itemize}
	\item They are staffed by volunteers.
	\item The center is staffed with highly trained physicians.
	\item The centre is staffed at all times.
	\item Some have regular clinics staffed by nursing officers.
	\end{itemize}
}
\item countable noun \\
A \textbf{staff} is a strong stick or pole.
 \textit{
	\begin{itemize}
	\end{itemize}
}
\item  \\
A \textbf{staff} is the five lines that music is written on.
 \textit{
	\begin{itemize}
	\end{itemize}
}
\end{enumerate}

\section*{spell}
{\large \color{blue}  spells  spelling  spelled  spelt  }
\subsection*{Explain}
\begin{enumerate}
\item verb \\
When you \textbf{spell} a word, you write or speak each letter in the word in the correct order.
 \textbf{Spell out}  means the same as spell .
 \textit{
	\begin{itemize}
	\item He gave his name and then helpfully spelt it.
	\item How do you spell 'potato'?
	\item 'Tang' is 'Gnat' spelt backwards.
	\item If I don't know a word, I ask them to spell it out for me.
	\item I never have to spell out my first name.
	\end{itemize}
}
\item verb \\
Someone who can \textbf{spell}  knows the correct order of letters in words.
 \textit{
	\begin{itemize}
	\item It's shocking how many students can't spell these days.
	\item You accused me of inaccuracy yet you can't spell 'Middlesex'.
	\end{itemize}
}
\item verb \\
If something \textbf{spells} a particular result , often an unpleasant one, it suggests that this will be the result.
 \textit{
	\begin{itemize}
	\item If the irrigation plan goes ahead, it could spell disaster for the birds.
	\item A report has just arrived on government desks which spells more trouble.
	\end{itemize}
}
\item countable noun \\
A \textbf{spell}  \textbf{of} a particular type of weather or a particular activity is a short period of time during which this type of weather or activity occurs.
 \textit{
	\begin{itemize}
	\item There has been a long spell of dry weather.
	\item You join a barrister for two six-month spells of practical experience.
	\item ...sunny spells.
	\end{itemize}
}
\item countable noun \\
A \textbf{spell} is a situation in which events are controlled by a magical power .
 \textit{
	\begin{itemize}
	\item They say she died after a witch cast a spell on her.
	\item ...the kiss that will break the spell.
	\end{itemize}
}
\item  \\
 to cast a/its spell \textit{
	\begin{itemize}
	\end{itemize}
}
\item  \\
 under sb's spell \textit{
	\begin{itemize}
	\end{itemize}
}
\end{enumerate}

\section*{umbrella}
{\large \color{blue}  umbrellas  }
\subsection*{Explain}
\begin{enumerate}
\item countable noun \\
An \textbf{umbrella} is an object which you use to protect yourself from the rain or hot  sun . It consists of a long stick with a folding frame covered in cloth .
 \textit{
	\begin{itemize}
	\item Harry held an umbrella over Dawn.
	\end{itemize}
}
\item singular noun \\
\textbf{Umbrella} is used to refer to a single group or description that includes a lot of different organizations or ideas .
 \textit{
	\begin{itemize}
	\item Does coincidence come under the umbrella of the paranormal?
	\item Within the umbrella term 'dementia' there are many different kinds of disease.
	\item ...Socialist International, an umbrella group comprising almost a hundred Social
Democrat parties.
	\end{itemize}
}
\item singular noun \\
\textbf{Umbrella} is used to refer to a system or agreement which protects a country or group of people.
 \textit{
	\begin{itemize}
	\item As regulated investments, they come under the umbrella of the country's financial
compensation scheme.
	\item Britain cannot avoid being under the U.S. nuclear umbrella, whether it wants to or
not.
	\end{itemize}
}
\end{enumerate}

\section*{taste}
{\large \color{blue}  tastes  tasting  tasted  }
\subsection*{Explain}
\begin{enumerate}
\item uncountable noun \\
\textbf{Taste} is one of the five senses that people have. When you have food or drink in your mouth, your sense of
taste makes it possible for you to recognize what it is.
 \textit{
	\begin{itemize}
	\item ...a keen sense of taste.
	\end{itemize}
}
\item countable noun \\
The \textbf{taste} of something is the individual quality which it has when you put it in your mouth
and which distinguishes it from other things. For example , something may have a sweet , bitter , sour , or salty taste.
 \textit{
	\begin{itemize}
	\item I like the taste of fast food too much to give it up.
	\item The taste of blood in her throat made her want to vomit.
	\item Nettles are surprisingly good–much like spinach but with a sweetish taste.
	\end{itemize}
}
\item singular noun \\
If you have a \textbf{taste} of some food or drink, you try a small amount of it in order to see what the flavour is like.
 \textit{
	\begin{itemize}
	\item Let them have a taste of cold food but I prefer mine hot.
	\end{itemize}
}
\item verb \\
If food or drink \textbf{tastes}  \textbf{of} something, it has that particular flavour, which you notice when you eat or drink it.
 \textit{
	\begin{itemize}
	\item I drank a cup of tea that tasted of diesel.
	\item It tastes like chocolate.
	\item The pizza tastes delicious without any cheese at all.
	\end{itemize}
}
\item verb \\
If you \textbf{taste} some food or drink, you eat or drink a small amount of it in order to try its flavour,
for example to see if you like it or not.
 \textit{
	\begin{itemize}
	\item We tasted the water just to see how salty it was.
	\item Before proceeding any further, cut off a small bit of the meat and taste it.
	\end{itemize}
}
\item verb \\
If you can \textbf{taste} something that you are eating or drinking, you are aware of its flavour.
 \textit{
	\begin{itemize}
	\item You can taste the chilli in the dish but it is a little sweet.
	\end{itemize}
}
\item singular noun \\
If you have a \textbf{taste of} a particular way of life or activity, you have a brief experience of it.
 \textit{
	\begin{itemize}
	\item But having had a taste of the big time, he won't want to go back to playing in the
reserves.
	\item This voyage was his first taste of freedom.
	\end{itemize}
}
\item verb \\
If you \textbf{taste} something such as a way of life or a pleasure , you experience it for a short period of time.
 \textit{
	\begin{itemize}
	\item Anyone who has tasted this life wants it to carry on for as long as possible.
	\end{itemize}
}
\item singular noun \\
If you have a \textbf{taste for} something, you have a liking or preference for it.
 \textit{
	\begin{itemize}
	\item She developed a taste for journeys to isolated hazardous regions in North America.
	\item That gave me a taste for reading.
	\end{itemize}
}
\item uncountable noun \\
A person's \textbf{taste} is their choice  \textbf{in} the things that they like or buy , for example their clothes, possessions , or music. If you say that someone has good \textbf{taste} , you mean that you approve of their choices. If you say that they have poor  \textbf{taste} , you disapprove of their choices.
 \textit{
	\begin{itemize}
	\item His taste in clothes is extremely good.
	\item Oxford's social circle was far too liberal for her taste.
	\item ...a large family with different tastes and preferences.
	\item How could so many people have such bad taste in music?
	\end{itemize}
}
\item  \\
 in bad/good/etc taste \textit{
	\begin{itemize}
	\end{itemize}
}
\item  \\
 to taste \textit{
	\begin{itemize}
	\end{itemize}
}
\end{enumerate}

\section*{weed}
{\large \color{blue}  weeds  weeding  weeded  }
\subsection*{Explain}
\begin{enumerate}
\item countable noun \\
A \textbf{weed} is a wild plant that grows in gardens or fields of crops and prevents the plants that you want from growing properly.
 \textit{
	\begin{itemize}
	\item ...a garden overgrown with weeds.
	\end{itemize}
}
\item variable noun \\
\textbf{Weed} is a plant that grows in water and usually forms a thick  floating  mass . There are many different kinds of weed.
 \textit{
	\begin{itemize}
	\item Large, clogging banks of weed are the only problem.
	\end{itemize}
}
\item verb \\
If you \textbf{weed} an area, you remove the weeds from it.
 \textit{
	\begin{itemize}
	\item Caspar was weeding the garden.
	\item Try not to walk on the flower beds when weeding or hoeing.
	\end{itemize}
}
\item uncountable noun \\
People sometimes  refer to tobacco or marijuana as \textbf{weed} .
 \textit{
	\begin{itemize}
	\item Two and a half years ago I gave up the evil weed.
	\end{itemize}
}
\end{enumerate}

\section*{torture}
{\large \color{blue}  tortures  torturing  tortured  }
\subsection*{Explain}
\begin{enumerate}
\item verb \\
If someone \textbf{is tortured} , another person deliberately causes them great pain over a period of time, in order to punish them or to make them reveal information.
 \textbf{Torture} is also a noun .
 \textit{
	\begin{itemize}
	\item French police are convinced that she was tortured and killed.
	\item Three members of the group had been tortured to death.
	\item They never again tortured a prisoner in his presence.
	\item ...alleged cases of torture and murder by the security forces.
	\item Many died under torture, others committed suicide.
	\item I had thought this was a medieval torture that had mercifully disappeared.
	\end{itemize}
}
\item verb \\
To \textbf{torture} someone means to cause them to suffer mental pain or anxiety .
 \textit{
	\begin{itemize}
	\item He would not torture her further by trying to argue with her.
	\item She tortured herself with fantasies of Bob and his new girlfriend.
	\end{itemize}
}
\item uncountable noun \\
If you say that something is \textbf{torture} or a \textbf{torture} , you mean that it causes you great mental or physical suffering .
 \textit{
	\begin{itemize}
	\item Waiting for the result was torture.
	\item The friction of the sheets against his skin was torture.
	\item Learning–something she had always loved–became a torture.
	\end{itemize}
}
\end{enumerate}

\section*{whole}
{\large \color{blue}  wholes  }
\subsection*{Explain}
\begin{enumerate}
\item quantifier \\
If you refer to \textbf{the whole of} something, you mean all of it.
 \textbf{Whole} is also an adjective .
 \textit{
	\begin{itemize}
	\item Early in the eleventh century the whole of England was conquered by the Vikings.
	\item I was cold throughout the whole of my body.
	\item ...the whole of August.
	\item He'd been observing her the whole trip.
	\item We spent the whole summer in Italy that year.
	\end{itemize}
}
\item countable noun \\
A \textbf{whole} is a single thing which contains several different parts.
 \textit{
	\begin{itemize}
	\item An atom itself is a complete whole, with its electrons, protons and neutrons.
	\item Taken as a percentage of the whole, the mouth has to be a fairly minor body part.
	\end{itemize}
}
\item adjective \\
If something is \textbf{whole} , it is in one piece and is not broken or damaged .
 \textit{
	\begin{itemize}
	\item Much of the temple was ruined, but the front was whole, as well as a large hall behind
it.
	\item I struck the glass with my fist with all my might; yet it remained whole.
	\item Small bones should be avoided as the dog may swallow them whole and risk internal
injury.
	\end{itemize}
}
\item adverb \\
You use \textbf{whole} to emphasize what you are saying .
 \textbf{Whole} is also an adjective.
 \textit{
	\begin{itemize}
	\item It was like seeing a whole different side of somebody.
	\item His father had helped invent a whole new way of doing business.
	\item That saved me a whole bunch of money.
	\item There's a whole group of friends he doesn't want you to meet.
	\end{itemize}
}
\item  \\
 as a whole \textit{
	\begin{itemize}
	\end{itemize}
}
\item  \\
 on the whole \textit{
	\begin{itemize}
	\end{itemize}
}
\end{enumerate}

\section*{try}
{\large \color{blue}  tries  trying  tried  }
\subsection*{Explain}
\begin{enumerate}
\item verb \\
If you \textbf{try} to do something, you want to do it, and you take action which you hope  will  help you to do it.
 \textbf{Try} is also a noun .
 \textit{
	\begin{itemize}
	\item He secretly tried to block her advancement in the Party.
	\item Try to make the effort to work your way through all of your tasks one at a time.
	\item Does it annoy you if others don't seem to try hard enough?
	\item I tried calling him when I got here but he wasn't at home.
	\item No matter how bad you feel, keep trying.
	\item She didn't really expect to get any money out of him, but it seemed worth a try.
	\item After a few tries Patrick had given up any attempt to reform his brother.
	\end{itemize}
}
\item verb \\
To \textbf{try}  \textbf{and} do something means to try to do it.
 \textit{
	\begin{itemize}
	\item He has started a privatisation programme to try and win support from the business
community.
	\item I must try and see him.
	\end{itemize}
}
\item verb \\
If you \textbf{try for} something, you make an effort to get it or achieve it.
 \textit{
	\begin{itemize}
	\item My partner and I have been trying for a baby for two years.
	\item He said he was going to try for first place next year.
	\end{itemize}
}
\item verb \\
If you \textbf{try} something new or different, you use it, do it, or experience it in order to discover its qualities or effects.
 \textbf{Try} is also a noun.
 \textit{
	\begin{itemize}
	\item It's best not to try a new recipe for the first time on such an important occasion.
	\item I tried everything, from nutritionists to acupuncture, but nothing worked.
	\item I have tried painting the young shoots with weed poisoner, but this does not kill
them off.
	\item If you're still sceptical about exercising, we can only ask you to trust us and give
it a try.
	\end{itemize}
}
\item verb \\
If you \textbf{try} a particular place or person, you go to that place or person because you think that they may be able to provide you with what you want.
 \textit{
	\begin{itemize}
	\item Have you tried the local music shops?
	\end{itemize}
}
\item verb \\
If you \textbf{try} a door or window , you try to open it.
 \textit{
	\begin{itemize}
	\item Bob tried the door. To his surprise, it opened.
	\end{itemize}
}
\item verb \\
When a person \textbf{is tried} , he or she has to appear in a law court and is found innocent or guilty after the judge and jury have heard the evidence. When a legal  case  \textbf{is tried} , it is considered in a court of law.
 \textit{
	\begin{itemize}
	\item He suggested that those responsible should be tried for crimes against humanity.
	\item Whether he is guilty is a decision that will be made when the case is tried in court.
	\item The military court which tried him excluded two of his lawyers.
	\item Why does it take 253 days to try a case of fraud?
	\end{itemize}
}
\item countable noun \\
In the game of rugby, a \textbf{try} is the action of scoring by putting the ball down behind the goal line of the opposing team .
 \textit{
	\begin{itemize}
	\item The French, who led 21-3 at half time, scored eight tries.
	\end{itemize}
}
\item  \\
 for want of trying/for lack of trying \textit{
	\begin{itemize}
	\end{itemize}
}
\end{enumerate}

\section*{wound}
{\large \color{blue}  }
\subsection*{Explain}
\begin{enumerate}
\item  \\
\textbf{Wound} is the past  tense and past participle of wind2 2.
 \textit{
	\begin{itemize}
	\end{itemize}
}
\end{enumerate}

\section*{venture}
{\large \color{blue}  ventures  venturing  ventured  }
\subsection*{Explain}
\begin{enumerate}
\item countable noun \\
A \textbf{venture} is a project or activity which is new, exciting , and difficult because it involves the risk of failure .
 \textit{
	\begin{itemize}
	\item ...his latest writing venture.
	\item Both parties sounded full of high hopes for their joint venture.
	\end{itemize}
}
\item verb \\
If you \textbf{venture}  somewhere , you go somewhere that might be dangerous .
 \textit{
	\begin{itemize}
	\item People are afraid to venture out for fear of sniper attacks.
	\item Few Europeans who had ventured beyond the Himalayas had returned to tell the tale.
	\end{itemize}
}
\item verb \\
If you \textbf{venture} a question or statement , you say it in an uncertain way because you are afraid it might be stupid or wrong .
 \textit{
	\begin{itemize}
	\item 'So you're Leo's girlfriend?' he ventured.
	\item He ventured that plants draw part of their nourishment from the air.
	\item Stephen ventured a few more sentences in halting Welsh.
	\end{itemize}
}
\item verb \\
If you \textbf{venture}  \textbf{to} do something that requires  courage or is risky, you do it.
 \textit{
	\begin{itemize}
	\item 'Don't ask,' he said, whenever Ginny ventured to raise the subject.
	\end{itemize}
}
\item verb \\
If you \textbf{venture into} an activity, you do something that involves the risk of failure because it is new
and different .
 \textit{
	\begin{itemize}
	\item He enjoyed little success when he ventured into business.
	\end{itemize}
}
\end{enumerate}

\section*{bulb}
{\large \color{blue}  bulbs  }
\subsection*{Explain}
\begin{enumerate}
\item countable noun \\
A \textbf{bulb} is the glass part of an electric  lamp , which gives out light when electricity  passes through it.
 \textit{
	\begin{itemize}
	\item The stairwell was lit by a single bulb.
	\end{itemize}
}
\item countable noun \\
A \textbf{bulb} is a root shaped like an onion that grows into a flower or plant.
 \textit{
	\begin{itemize}
	\item ...tulip bulbs.
	\end{itemize}
}
\end{enumerate}

\section*{administer}
{\large \color{blue}  administers  administering  administered  }
\subsection*{Explain}
\begin{enumerate}
\item verb \\
If someone \textbf{administers} something such as a country, the law , or a test , they take responsibility for organizing and supervising it.
 \textit{
	\begin{itemize}
	\item The plan calls for the U.N. to administer the country until elections can be held.
	\item We hope that they're going to administer justice impartially.
	\item Next summer's exams would be straightforward to administer and mark.
	\end{itemize}
}
\item verb \\
If a doctor or a nurse  \textbf{administers} a drug, they give it to a patient .
 \textit{
	\begin{itemize}
	\item Paramedics are trained to administer certain drugs.
	\end{itemize}
}
\item verb \\
If someone \textbf{administers} a punch or a kick , they punch or kick someone.
 \textit{
	\begin{itemize}
	\item He is shown in the video of the beating as administering most of the blows.
	\end{itemize}
}
\end{enumerate}

\section*{clothes}
{\large \color{blue}  }
\subsection*{Explain}
\begin{enumerate}
\item plural noun \\
\textbf{Clothes} are the things that people wear, such as shirts , coats , trousers , and dresses.
 \textit{
	\begin{itemize}
	\item Moira walked upstairs to change her clothes.
	\item He dressed quickly in casual clothes.
	\end{itemize}
}
\end{enumerate}

\section*{behave}
{\large \color{blue}  behaves  behaving  behaved  }
\subsection*{Explain}
\begin{enumerate}
\item verb \\
The way that you \textbf{behave} is the way that you do and say things, and the things that you do and say.
 \textit{
	\begin{itemize}
	\item I couldn't believe these people were behaving in this way.
	\item He'd behaved badly.
	\end{itemize}
}
\item verb \\
If you \textbf{behave} or \textbf{behave}  \textbf{yourself} , you act in the way that people think is correct and proper .
 \textit{
	\begin{itemize}
	\item You have to behave.
	\item They were expected to behave themselves.
	\end{itemize}
}
\item verb \\
In science , the way that something \textbf{behaves} is the things that it does.
 \textit{
	\begin{itemize}
	\item Under certain conditions, electrons can behave like waves rather than particles.
	\end{itemize}
}
\end{enumerate}

\section*{collar}
{\large \color{blue}  collars  collaring  collared  }
\subsection*{Explain}
\begin{enumerate}
\item countable noun \\
The \textbf{collar} of a shirt or coat is the part which fits round the neck and is usually folded over.
 \textit{
	\begin{itemize}
	\item His tie was pulled loose and his collar hung open.
	\item ...a coat with a huge fake fur collar.
	\end{itemize}
}
\item countable noun \\
A \textbf{collar} is a band of leather or plastic which is put round the neck of a dog or cat .
 \textit{
	\begin{itemize}
	\end{itemize}
}
\item verb \\
If you \textbf{collar} someone who has done something wrong or who is running away, you catch them and hold them so that they cannot escape .
 \textit{
	\begin{itemize}
	\item As Kerr fled towards the exit, Boycott collared him at the ticket barrier.
	\end{itemize}
}
\item verb \\
If you \textbf{collar} someone, you stop them and make them listen to you.
 \textit{
	\begin{itemize}
	\item Beattie managed to collar Atkins in a hallway.
	\item Bernard was once collared by an aggressive stranger in Soho.
	\end{itemize}
}
\item  \\
 get hot under the collar \textit{
	\begin{itemize}
	\end{itemize}
}
\end{enumerate}

\section*{bid}
{\large \color{blue}  bids  bidding  }
\subsection*{Explain}
\begin{enumerate}
\item countable noun \\
A \textbf{bid}  \textbf{for} something or a \textbf{bid}  \textbf{to} do something is an attempt to obtain it or do it.
 \textit{
	\begin{itemize}
	\item ...the city's successful bid for European City of Culture.
	\item The company said that it might cut 2,232 jobs in a bid to reduce costs.
	\end{itemize}
}
\item countable noun \\
A \textbf{bid} is an offer to pay a particular amount of money for something that is being sold .
 \textit{
	\begin{itemize}
	\item Hanson made an agreed takeover bid of £351 million.
	\end{itemize}
}
\item verb \\
If you \textbf{bid}  \textbf{for} something or \textbf{bid}  \textbf{to} do something, you try to obtain it or do it.
 \textit{
	\begin{itemize}
	\item The German private equity group reiterated its interest in bidding for the company.
	\item I don't think she is bidding to be Prime Minister again.
	\end{itemize}
}
\item verb \\
If you \textbf{bid}  \textbf{for} something that is being sold, you offer to pay a particular amount of money for it.
 \textit{
	\begin{itemize}
	\item She decided to bid for a Georgian dressing table.
	\item The bank announced its intention to bid.
	\item He certainly wasn't going to bid $18 billion for this company.
	\end{itemize}
}
\end{enumerate}

\section*{controversy}
{\large \color{blue}  controversies  }
\subsection*{Explain}
\begin{enumerate}
\item variable noun \\
\textbf{Controversy} is a lot of discussion and argument about something, often involving strong feelings of anger or disapproval .
 \textit{
	\begin{itemize}
	\item The proposed cuts have caused considerable controversy.
	\item ...a fierce political controversy over human rights abuses.
	\end{itemize}
}
\end{enumerate}

\section*{bless}
{\large \color{blue}  blesses  blessing  blessed  }
\subsection*{Explain}
\begin{enumerate}
\item verb \\
When someone such as a priest  \textbf{blesses} people or things, he asks for God's favour and protection for them.
 \textit{
	\begin{itemize}
	\item ...asking for all present to bless this couple and their loving commitment to one
another.
	\end{itemize}
}
\item convention \\
\textbf{Bless} is used in expressions such as ' \textbf{God bless} ' or ' \textbf{bless you} ' to express  affection , thanks , or good wishes .
 \textit{
	\begin{itemize}
	\item 'Bless you, Eva,' he whispered.
	\item God bless and thank you all so much.
	\end{itemize}
}
\item  \\
 bless you \textit{
	\begin{itemize}
	\end{itemize}
}
\end{enumerate}

\section*{decision}
{\large \color{blue}  decisions  }
\subsection*{Explain}
\begin{enumerate}
\item countable noun \\
When you make a \textbf{decision} , you choose what should be done or which is the best of various possible actions.
 \textit{
	\begin{itemize}
	\item A decision was taken to discipline Marshall.
	\item The president said he'd made no firm decision on whether he would run for a second
term in office.
	\item I don't want to make the wrong decision and regret it later.
	\item Who makes the financial decisions in your household?
	\end{itemize}
}
\item uncountable noun \\
\textbf{Decision} is the act of deciding something or the need to decide something.
 \textit{
	\begin{itemize}
	\item The moment of decision cannot be delayed.
	\item This was a matter for decision by the individual.
	\end{itemize}
}
\item uncountable noun \\
\textbf{Decision} is the ability to decide quickly and definitely what to do.
 \textit{
	\begin{itemize}
	\item He is very much a man of decision and action.
	\end{itemize}
}
\end{enumerate}

\section*{conclude}
{\large \color{blue}  concludes  concluding  concluded  }
\subsection*{Explain}
\begin{enumerate}
\item verb \\
If you \textbf{conclude}  \textbf{that} something is true , you decide that it is true using the facts you know as a basis .
 \textit{
	\begin{itemize}
	\item Larry had concluded that he had no choice but to accept Paul's words as the truth.
	\item So what can we conclude from this debate?
	\item 'The situation in the inner cities is bad and getting worse,' she concluded.
	\end{itemize}
}
\item verb \\
When you \textbf{conclude} , you say the last thing that you are going to say.
 \textit{
	\begin{itemize}
	\item 'It's a waste of time,' he concluded.
	\item I would like to conclude by saying that I do enjoy your magazine.
	\end{itemize}
}
\item verb \\
When something \textbf{concludes} , or when you \textbf{conclude} it, you end it.
 \textit{
	\begin{itemize}
	\item The evening concluded with dinner and speeches.
	\item The Group of Seven major industrial countries concluded its annual summit meeting
today.
	\end{itemize}
}
\item verb \\
If one person or group \textbf{concludes} an agreement , such as a treaty or business  deal , \textbf{with} another, they arrange it. You can  also say that two people or groups \textbf{conclude} an agreement.
 \textit{
	\begin{itemize}
	\item Iceland concluded agreements with several other countries.
	\item If the clubs cannot conclude a deal, an independent tribunal will decide.
	\end{itemize}
}
\end{enumerate}

\section*{dirt}
{\large \color{blue}  }
\subsection*{Explain}
\begin{enumerate}
\item uncountable noun \\
If there is \textbf{dirt} on something, there is dust, mud, or a stain on it.
 \textit{
	\begin{itemize}
	\item I started to scrub off the dirt.
	\end{itemize}
}
\item uncountable noun \\
You can refer to the earth on the ground as \textbf{dirt} , especially when it is dusty .
 \textit{
	\begin{itemize}
	\item They all sit on the dirt in the dappled shade of a tree.
	\end{itemize}
}
\item adjective \\
A \textbf{dirt}  road or track is made from hard earth. A \textbf{dirt}  floor is made from earth without any cement , stone , or wood laid on it.
 \textit{
	\begin{itemize}
	\item I drove along the dirt road.
	\item The rooster chased me across the dirt floor of the barn.
	\end{itemize}
}
\item singular noun \\
If you say that you have \textbf{the}  \textbf{dirt}  \textbf{on} someone, you mean that you have information that could harm their reputation or career .
 \textit{
	\begin{itemize}
	\item Steve was keen to get all the dirt he could on her.
	\item Both parties use computers to dig up dirt on their opponents.
	\end{itemize}
}
\item  \\
 to dish the dirt \textit{
	\begin{itemize}
	\end{itemize}
}
\item  \\
 to treat someone like dirt \textit{
	\begin{itemize}
	\end{itemize}
}
\end{enumerate}

\section*{declare}
{\large \color{blue}  declares  declaring  declared  }
\subsection*{Explain}
\begin{enumerate}
\item verb \\
If you \textbf{declare} that something is true , you say that it is true in a firm , deliberate way. You can also  \textbf{declare} an attitude or intention .
 \textit{
	\begin{itemize}
	\item Speaking outside Ten Downing Street, she declared that she would fight on.
	\item 'I'm absolutely thrilled to have done what I've done,' he declared.
	\item He declared his intention to become the best golfer in the world.
	\item Glasses of Madeira wine were brought to us. We declared it delicious.
	\item He turned up in northern Cyprus, declaring himself happy to be home.
	\end{itemize}
}
\item verb \\
If you \textbf{declare} something, you state officially and formally that it exists or is the case .
 \textit{
	\begin{itemize}
	\item The government is ready to declare a permanent ceasefire.
	\item His lawyers are confident that the judges will declare Mr Stevens innocent.
	\item The U.N. has declared it to be a safe zone.
	\item On striking his sword on the stone, he declared himself Lord of the City.
	\item You may have to declare that you have had an HIV test.
	\end{itemize}
}
\item verb \\
If you \textbf{declare} goods that you have bought in another country or money that you have earned , you say how much you have bought or earned so that you can pay tax on it.
 \textit{
	\begin{itemize}
	\item Your income must be declared on this form.
	\item She had nothing to declare, and was starting to go through the 'Green' channel when
she was stopped.
	\end{itemize}
}
\end{enumerate}

\section*{generation}
{\large \color{blue}  generations  }
\subsection*{Explain}
\begin{enumerate}
\item countable noun \\
A \textbf{generation} is all the people in a group or country who are of a similar age, especially when they are considered as having the same experiences or attitudes.
 \textit{
	\begin{itemize}
	\item ...the younger generation of Party members.
	\item He has long been considered the leading American playwright of his generation.
	\end{itemize}
}
\item countable noun \\
A \textbf{generation} is the period of time, usually considered to be about thirty years, that it takes for children to grow up and become adults and have children of their own.
 \textit{
	\begin{itemize}
	\item Within a generation flight has become the method used by many travellers.
	\end{itemize}
}
\item countable noun \\
You can use \textbf{generation} to refer to a stage of development in the design and manufacture of machines or equipment .
 \textit{
	\begin{itemize}
	\item ...a new generation of IBM/Apple computers.
	\end{itemize}
}
\item adjective \\
\textbf{Generation} is used to indicate how long members of your family have had a particular nationality . For example , second generation means that you were born in the country you live in, but your parents were not.
 \textit{
	\begin{itemize}
	\item ...a second-generation 'immigrant' of Italian and Irish descent.
	\item She is a first generation American.
	\end{itemize}
}
\item uncountable noun \\
\textbf{Generation} is the production of a form of energy or power from fuel or another source of power such as water.
 \textit{
	\begin{itemize}
	\item Japan has announced plans for a sharp rise in its nuclear power generation.
	\end{itemize}
}
\end{enumerate}

\section*{exemplify}
{\large \color{blue}  exemplifies  exemplifying  exemplified  }
\subsection*{Explain}
\begin{enumerate}
\item verb \\
If a person or thing \textbf{exemplifies} something such as a situation , quality, or class of things, they are a typical example of it.
 \textit{
	\begin{itemize}
	\item The room's style exemplifies Conran's ideal of 'beauty and practicality'.
	\item ...the art of canon-writing as exemplified by Bach.
	\end{itemize}
}
\end{enumerate}

\section*{greeting}
{\large \color{blue}  greetings  }
\subsection*{Explain}
\begin{enumerate}
\item variable noun \\
A \textbf{greeting} is something friendly that you say or do when you meet someone.
 \textit{
	\begin{itemize}
	\item His greeting was familiar and friendly.
	\item They exchanged greetings.
	\item He raised a hand in greeting.
	\end{itemize}
}
\item convention \\
' \textbf{Greetings} ' is an old-fashioned greeting.
 \textit{
	\begin{itemize}
	\end{itemize}
}
\end{enumerate}

\section*{fascinate}
{\large \color{blue}  fascinates  fascinating  fascinated  }
\subsection*{Explain}
\begin{enumerate}
\item verb \\
If something \textbf{fascinates} you, it interests and delights you so much that your thoughts tend to concentrate on it.
 \textit{
	\begin{itemize}
	\item Politics fascinated Franklin's father.
	\item She fascinated him, both on and off stage.
	\end{itemize}
}
\end{enumerate}

\section*{flee}
{\large \color{blue}  flees  fleeing  fled  }
\subsection*{Explain}
\begin{enumerate}
\item verb \\
If you \textbf{flee}  \textbf{from} something or someone, or \textbf{flee} a person or thing, you escape from them.
 \textit{
	\begin{itemize}
	\item He slammed the bedroom door behind him and fled.
	\item He fled to Costa Rica to avoid military service.
	\item ...refugees who have fled from wars, famine and persecution.
	\item ...refugees fleeing persecution or torture.
	\item Thousands have been compelled to flee the country in makeshift boats.
	\end{itemize}
}
\end{enumerate}

\section*{cream}
{\large \color{blue}  creams  creaming  creamed  }
\subsection*{Explain}
\begin{enumerate}
\item uncountable noun \\
\textbf{Cream} is a thick yellowish-white liquid taken from milk. You can use it in cooking or put it on fruit
or desserts .
 \textit{
	\begin{itemize}
	\item ...strawberries and cream.
	\end{itemize}
}
\item uncountable noun \\
\textbf{Cream} is used in the names of soups that contain cream or milk.
 \textit{
	\begin{itemize}
	\item ...cream of mushroom soup.
	\end{itemize}
}
\item variable noun \\
A \textbf{cream} is a substance that you rub into your skin, for example to keep it soft or to heal or protect it.
 \textit{
	\begin{itemize}
	\item Gently apply the cream to the affected areas.
	\item ...sun protection creams.
	\end{itemize}
}
\item colour \\
Something that is \textbf{cream} is yellowish-white in colour.
 \textit{
	\begin{itemize}
	\item ...cream silk stockings.
	\item ...a cream-coloured Persian cat.
	\end{itemize}
}
\item singular noun \\
\textbf{Cream} is used in expressions such as \textbf{the cream of society} and \textbf{the cream of British athletes} to refer to the best people or things of a particular kind .
 \textit{
	\begin{itemize}
	\item The Ball was attended by the cream of Hollywood society.
	\item ...the cream of Chicago's 200 jazz and blues clubs.
	\end{itemize}
}
\end{enumerate}

\section*{hate}
{\large \color{blue}  hates  hating  hated  }
\subsection*{Explain}
\begin{enumerate}
\item verb \\
If you \textbf{hate} someone or something, you have an extremely  strong feeling of dislike for them.
 \textbf{Hate} is also a noun .
 \textit{
	\begin{itemize}
	\item Most people hate him, but they don't dare to say so, because he still rules the country.
	\item I hated myself for writing that letter.
	\item I was 17 and filled with a lot of hate.
	\item It is difficult to bear the agony of our loved ones' anger and hate.
	\item ...eyes that held a look of chronic hate.
	\end{itemize}
}
\item verb \\
If you say that you \textbf{hate} something such as a particular  activity , you mean that you find it very unpleasant .
 \textit{
	\begin{itemize}
	\item Ted hated parties, even gatherings of people he liked individually.
	\item She hated hospitals and didn't like the idea of having an operation.
	\item He hates to be interrupted during training.
	\item He hated coming home to the empty house.
	\item I hate it when people accuse us of that.
	\item I would hate him to think I'm trying to trap him.
	\item She hates me having any fun and is quite jealous and spoiled.
	\end{itemize}
}
\item verb \\
You can use \textbf{hate} in expressions such as ' \textbf{I hate to trouble you} ' or ' \textbf{I hate to bother you} ' when you are apologizing to someone for interrupting them or asking them to do something.
 \textit{
	\begin{itemize}
	\item I hate to rush you but I have another appointment later on.
	\end{itemize}
}
\item verb \\
You can use \textbf{hate} in expressions such as ' \textbf{I hate to say it} ' or ' \textbf{I hate to tell you} ' when you want to express regret about what you are about to say, because you think it is unpleasant or should not be the case .
 \textit{
	\begin{itemize}
	\item I hate to tell you this, but tomorrow's your last day.
	\item I hate to admit it, but you were right.
	\end{itemize}
}
\item verb \\
You can use \textbf{hate} in expressions such as ' \textbf{I hate to see} ' or ' \textbf{I hate to think} ' when you are emphasizing that you find a situation or an idea unpleasant.
 \textit{
	\begin{itemize}
	\item I just hate to see you doing this to yourself.
	\end{itemize}
}
\item verb \\
You can use \textbf{hate} in expressions such as ' \textbf{I'd hate to think} ' when you hope that something is not true or that something will not happen .
 \textit{
	\begin{itemize}
	\item I'd hate to think my job would not be secure if I left it temporarily.
	\end{itemize}
}
\end{enumerate}

\section*{impact}
{\large \color{blue}  impacts  impacting  impacted  }
\subsection*{Explain}
\begin{enumerate}
\item countable noun \\
The \textbf{impact} that something has \textbf{on} a situation , process, or person is a sudden and powerful effect that it has on them.
 \textit{
	\begin{itemize}
	\item They say they expect the meeting to have a marked impact on the future of the country.
	\item The major impact of this epidemic worldwide is yet to come.
	\item When an executive comes into a new job, he wants to quickly make an impact.
	\end{itemize}
}
\item variable noun \\
An \textbf{impact} is the action of one object hitting another, or the force with which one object hits
another.
 \textit{
	\begin{itemize}
	\item The plane is destroyed, a complete wreck: the pilot must have died on impact.
	\item A running track should be capable of absorbing the impact of a runner's foot landing
on it.
	\end{itemize}
}
\item verb \\
To \textbf{impact}  \textbf{on} a situation, process, or person means to affect them.
 \textit{
	\begin{itemize}
	\item Such schemes mean little unless they impact on people.
	\item The reduction in the number of days that Parliament sat would impact on the quality
of its work.
	\item ...the potential for women to impact the political process.
	\end{itemize}
}
\item verb \\
If one object \textbf{impacts}  \textbf{on} another, it hits it with great force.
 \textit{
	\begin{itemize}
	\item ...the sharp tinkle of metal impacting on stone.
	\item According to the air force, the missile merely impacted with the ground prematurely.
	\item When a large object impacts the Earth, it makes a crater.
	\end{itemize}
}
\end{enumerate}

\section*{hoist}
{\large \color{blue}  hoists  hoisting  hoisted  }
\subsection*{Explain}
\begin{enumerate}
\item verb \\
If you \textbf{hoist} something heavy  somewhere , you lift it or pull it up there.
 \textit{
	\begin{itemize}
	\item Hoisting my suitcase on to my shoulder, I turned and headed toward my hotel.
	\item Grabbing the side of the bunk, he hoisted himself to a sitting position.
	\end{itemize}
}
\item verb \\
If something heavy \textbf{is hoisted} somewhere, it is lifted there using a machine such as a crane .
 \textit{
	\begin{itemize}
	\item A twenty-foot steel pyramid is to be hoisted into position on top of the tower.
	\item Then a crane hoisted him on to the platform.
	\end{itemize}
}
\item countable noun \\
A \textbf{hoist} is a machine for lifting heavy things.
 \textit{
	\begin{itemize}
	\end{itemize}
}
\item verb \\
If you \textbf{hoist} a flag or a sail, you pull it up to its correct position by using ropes .
 \textit{
	\begin{itemize}
	\item A group of youths hoisted their flag on top of the disputed monument.
	\end{itemize}
}
\end{enumerate}

\section*{intercourse}
{\large \color{blue}  }
\subsection*{Explain}
\begin{enumerate}
\item uncountable noun \\
\textbf{Intercourse} is the act of having sex .
 \textit{
	\begin{itemize}
	\item ...sexual intercourse.
	\item We didn't have intercourse.
	\end{itemize}
}
\item uncountable noun \\
Social  \textbf{intercourse} is communication between people as they spend time together.
 \textit{
	\begin{itemize}
	\item There was social intercourse between the old and the young.
	\end{itemize}
}
\end{enumerate}

\section*{illustrate}
{\large \color{blue}  illustrates  illustrating  illustrated  }
\subsection*{Explain}
\begin{enumerate}
\item verb \\
If you say that something \textbf{illustrates} a situation that you are drawing  attention to, you mean that it shows that the situation exists .
 \textit{
	\begin{itemize}
	\item The example of the United States illustrates this point.
	\item This change is neatly illustrated by what has happened to the Arab League.
	\item The incident graphically illustrates how parlous their position is.
	\item The case also illustrates that some women are now trying to fight back.
	\end{itemize}
}
\item verb \\
If you use an example, story , or diagram to \textbf{illustrate} a point, you use it show that what you are saying is true or to make your meaning clearer.
 \textit{
	\begin{itemize}
	\item Let me give another example to illustrate this difficult point.
	\item Throughout, she illustrates her analysis with excerpts from discussions.
	\end{itemize}
}
\item verb \\
If you \textbf{illustrate} a book, you put pictures, photographs or diagrams into it.
 \textit{
	\begin{itemize}
	\item She went on to art school and is now illustrating a book.
	\item He has illustrated the book with black-and-white photographs.
	\end{itemize}
}
\end{enumerate}

\section*{issue}
{\large \color{blue}  issues  issuing  issued  }
\subsection*{Explain}
\begin{enumerate}
\item countable noun \\
An \textbf{issue} is an important subject that people are arguing about or discussing .
 \textit{
	\begin{itemize}
	\item Agents will raise the issue of prize-money for next year's world championships.
	\item Is it right for the Church to express a view on political issues?
	\end{itemize}
}
\item singular noun \\
If something is \textbf{the issue} , it is the thing you consider to be the most important part of a situation or discussion.
 \textit{
	\begin{itemize}
	\item I was earning a lot of money, but that was not the issue.
	\item She avoided the issue by ordering a turkey sandwich.
	\item Do not draw it on the chart, however, as this will confuse the issue.
	\item The real issue was never addressed.
	\end{itemize}
}
\item countable noun \\
An \textbf{issue} of something such as a magazine or newspaper is the version of it that is published, for example, in a particular month or on a particular day.
 \textit{
	\begin{itemize}
	\item The growing problem is underlined in the latest issue of the Lancet.
	\end{itemize}
}
\item verb \\
If you \textbf{issue} a statement or a warning , you make it known formally or publicly .
 \textit{
	\begin{itemize}
	\item Last night he issued a statement denying the allegations.
	\item The government issued a warning that the strikers should end their action or face
dismissal.
	\item Yesterday his kidnappers issued a second threat to kill him.
	\end{itemize}
}
\item verb \\
If you \textbf{are issued with} something, it is officially given to you.
 \textbf{Issue} is also a noun .
 \textit{
	\begin{itemize}
	\item On your appointment you will be issued with a written statement of particulars of
employment.
	\item Staff will be issued with new grey-and-yellow designer uniforms.
	\item ...a standard army issue rifle.
	\end{itemize}
}
\item verb \\
When something such as a liquid, sound, or smell  \textbf{issues from} something, it comes out of that thing.
 \textit{
	\begin{itemize}
	\item A tinny voice issued from a speaker.
	\end{itemize}
}
\item  \\
 at issue \textit{
	\begin{itemize}
	\end{itemize}
}
\item  \\
 make an issue of \textit{
	\begin{itemize}
	\end{itemize}
}
\item  \\
 take issue with \textit{
	\begin{itemize}
	\end{itemize}
}
\item  \\
 have issues \textit{
	\begin{itemize}
	\end{itemize}
}
\end{enumerate}

\section*{keep}
{\large \color{blue}  keeps  keeping  kept  }
\subsection*{Explain}
\begin{enumerate}
\item link verb \\
If someone \textbf{keeps} or \textbf{is kept} in a particular state, they remain in it.
 \textit{
	\begin{itemize}
	\item The noise kept him awake.
	\item Reggie was being kept busy behind the bar.
	\item To keep warm they burnt wood in a rusty oil barrel.
	\item For several years I kept in touch with her.
	\end{itemize}
}
\item verb \\
If you \textbf{keep} or you \textbf{are kept} in a particular position or place, you remain in it.
 \textit{
	\begin{itemize}
	\item Keep away from the doors while the train is moving.
	\item He kept his head down, hiding his features.
	\item It was against all orders to smoke, but a cigarette kept away mosquitoes.
	\item Doctors will keep her in hospital for at least another week.
	\end{itemize}
}
\item verb \\
If you \textbf{keep}  \textbf{off} something or \textbf{keep}  \textbf{away from} it, you avoid it. If you \textbf{keep}  \textbf{out of} something, you avoid getting involved in it. You can also say that you \textbf{keep} someone \textbf{off} , \textbf{away from} or \textbf{out of} something.
 \textit{
	\begin{itemize}
	\item I managed to stick to the diet and keep off sweet foods.
	\item The best way to keep babies off sugar is to go back to the natural diet.
	\end{itemize}
}
\item verb \\
If someone or something \textbf{keeps} you \textbf{from} a particular action, they prevent you from doing it.
 \textit{
	\begin{itemize}
	\item Embarrassment has kept me from doing all sorts of things.
	\item He kept her from being lonely.
	\item What can you do to keep it from happening again?
	\end{itemize}
}
\item verb \\
If you try to \textbf{keep}  \textbf{from} doing something, you try to stop yourself from doing it.
 \textit{
	\begin{itemize}
	\item She bit her lip to keep from crying.
	\item He had to lean on Dan to keep from falling.
	\end{itemize}
}
\item verb \\
If you \textbf{keep} something \textbf{from} someone, you do not tell them about it.
 \textit{
	\begin{itemize}
	\item She knew that Gabriel was keeping something from her.
	\end{itemize}
}
\item verb \\
If you \textbf{keep} doing something, you do it repeatedly or continue to do it.
 \textbf{Keep on} means the same as keep .
 \textit{
	\begin{itemize}
	\item I keep forgetting it's December.
	\item I turned back after a while, but he kept walking.
	\item Did he give up or keep on trying?
	\item My wife keeps on saying that I work too hard.
	\end{itemize}
}
\item verb \\
\textbf{Keep} is used with some nouns to indicate that someone does something for a period of time or continues to do it.
For example, if you \textbf{keep a grip}  \textbf{on} something, you continue to hold or control it.
 \textit{
	\begin{itemize}
	\item Until last year, the regime kept a tight grip on the country.
	\item One of them would keep a look-out on the road behind to warn us of approaching vehicles.
	\item His parents kept a vigil by his bedside as he was given brain and body scans.
	\end{itemize}
}
\item verb \\
If you \textbf{keep} something, you continue to have it in your possession and do not throw it away, give it away, or sell it.
 \textit{
	\begin{itemize}
	\item 'I like this dress,' she said. 'Keep it. You can have it,' said Daphne.
	\item Lathan had to choose between marrying her and keeping his job.
	\end{itemize}
}
\item verb \\
If you \textbf{keep} something in a particular place, you always have it or store it in that place so that you can use it whenever you need it.
 \textit{
	\begin{itemize}
	\item She kept her money under the mattress.
	\item She remembered where she kept the gun.
	\item To make it easier to contact us, keep this card handy.
	\end{itemize}
}
\item verb \\
When you \textbf{keep} something such as a promise or an appointment , you do what you said you would do.
 \textit{
	\begin{itemize}
	\item I'm hoping you'll keep your promise to come for a long visit.
	\item He had again failed to keep his word.
	\end{itemize}
}
\item verb \\
If you \textbf{keep} a record of a series of events, you write down details of it so that they can be referred to later.
 \textit{
	\begin{itemize}
	\item Eleanor began to keep a diary.
	\item The volunteers kept a record of everything they ate for a week.
	\end{itemize}
}
\item verb \\
If you \textbf{keep} yourself or \textbf{keep} someone else, you support yourself or the other person by earning enough money to provide food, clothing, money, and other necessary things.
 \textit{
	\begin{itemize}
	\item She could just about afford to keep her five kids.
	\item I just cannot afford to keep myself.
	\item That should earn her enough to keep her in cookies for a while.
	\end{itemize}
}
\item singular noun \\
Someone's \textbf{keep} is the cost of food and other things that they need in their daily life.
 \textit{
	\begin{itemize}
	\item Ray will earn his keep on local farms while studying.
	\item I need to give my parents money for my keep.
	\end{itemize}
}
\item verb \\
If you \textbf{keep} animals, you own them and take care of them.
 \textit{
	\begin{itemize}
	\item I've brought you some eggs. We keep chickens.
	\item This mad writer kept a lobster as a pet.
	\end{itemize}
}
\item verb \\
If you \textbf{keep} a business such as a small shop or hotel , you own it and manage it.
 \textit{
	\begin{itemize}
	\item His father kept a village shop.
	\end{itemize}
}
\item verb \\
If someone or something \textbf{keeps} you, they delay you and make you late.
 \textit{
	\begin{itemize}
	\item Sorry to keep you, Jack.
	\item 'What kept you?'—'I went in the wrong direction.'
	\end{itemize}
}
\item verb \\
If food \textbf{keeps} for a certain length of time, it stays fresh and suitable to eat for that time.
 \textit{
	\begin{itemize}
	\item Whatever is left over may be put into the refrigerator, where it will keep for 2-3
weeks.
	\end{itemize}
}
\item verb \\
You can say or ask how someone \textbf{is keeping} as a way of saying or asking whether they are well .
 \textit{
	\begin{itemize}
	\item She hasn't been keeping too well lately.
	\item How are you keeping these days?
	\end{itemize}
}
\item countable noun \\
A \textbf{keep} is the main tower of a medieval castle, in which people lived .
 \textit{
	\begin{itemize}
	\end{itemize}
}
\item  \\
 to keep at it \textit{
	\begin{itemize}
	\end{itemize}
}
\item  \\
 for keeps \textit{
	\begin{itemize}
	\end{itemize}
}
\item  \\
 to keep going \textit{
	\begin{itemize}
	\end{itemize}
}
\item  \\
 in/out of keeping \textit{
	\begin{itemize}
	\end{itemize}
}
\item  \\
 to keep it up \textit{
	\begin{itemize}
	\end{itemize}
}
\item  \\
 to keep sth to yourself \textit{
	\begin{itemize}
	\end{itemize}
}
\item  \\
 to keep (yourself) to yourself \textit{
	\begin{itemize}
	\end{itemize}
}
\end{enumerate}

\section*{lamp}
{\large \color{blue}  lamps  }
\subsection*{Explain}
\begin{enumerate}
\item countable noun \\
A \textbf{lamp} is a light that works by using electricity or by burning  oil or gas .
 \textit{
	\begin{itemize}
	\item She switched on the bedside lamp.
	\item In the evenings we eat by the light of an oil lamp.
	\end{itemize}
}
\item countable noun \\
A \textbf{lamp} is an electrical device which produces a special type of light or heat , used especially in medical or beauty  treatment .
 \textit{
	\begin{itemize}
	\item ...a sun lamp.
	\item ...the use of infra-red lamps.
	\end{itemize}
}
\end{enumerate}

\section*{live}
{\large \color{blue}  lives  living  lived  }
\subsection*{Explain}
\begin{enumerate}
\item verb \\
If someone \textbf{lives} in a particular place or with a particular person, their home is in that place or
with that person.
 \textit{
	\begin{itemize}
	\item She has lived here for 10 years.
	\item She always said I ought to live alone.
	\item Where do you live?
	\item He still lives with his parents.
	\end{itemize}
}
\item verb \\
If you say that someone \textbf{lives} in particular circumstances or that they \textbf{live} a particular kind of life, you mean that they are in those circumstances or that
they have that kind of life.
 \textit{
	\begin{itemize}
	\item We lived quite grandly.
	\item We live in an age of rapid technological change.
	\item We can start living a normal life again now.
	\item ...the local support group for people living with HIV and AIDS.
	\end{itemize}
}
\item verb \\
If you say that someone \textbf{lives for} a particular thing, you mean that it is the most important thing in their life.
 \textit{
	\begin{itemize}
	\item He lived for his work.
	\end{itemize}
}
\item verb \\
To \textbf{live} means to be alive. If someone \textbf{lives}  \textbf{to} a particular age, they stay alive until they are that age.
 \textit{
	\begin{itemize}
	\item He's got a terrible disease and will not live long.
	\item A perennial is a plant that lives indefinitely.
	\item He lived to be 103.
	\item My father died nigh on ten years ago, but he lived to see his first grandson.
	\item Matilda was born in northern Italy in 1046 and apparently lived to a ripe old age.
	\item The blue whale is the largest living thing on the planet.
	\item Ian was her only living relative.
	\end{itemize}
}
\item verb \\
If people \textbf{live by} doing a particular activity, they get the money, food, or clothing they need by doing that activity.
 \textit{
	\begin{itemize}
	\item ...the last indigenous people to live by hunting.
	\item These crimes were committed largely by professional criminals who lived by crime.
	\end{itemize}
}
\item verb \\
If you \textbf{live by} a particular rule, belief, or ideal , you behave in the way in which it says you should behave.
 \textit{
	\begin{itemize}
	\item They live by the principle that we are here to add what we can to life.
	\end{itemize}
}
\item verb \\
If a person or occasion  \textbf{lives} in someone's mind or in history , they are remembered for a long time.
 \textbf{Live on} means the same as live1 .
 \textit{
	\begin{itemize}
	\item The memory of that will live with me for many years to come.
	\item His name will live in history as one of the greatest bowlers of all time.
	\item Lenin lives on in the minds and hearts of millions of people.
	\end{itemize}
}
\item  \\
 live and breathe sth \textit{
	\begin{itemize}
	\end{itemize}
}
\item  \\
 haven't lived \textit{
	\begin{itemize}
	\end{itemize}
}
\item  \\
 live in fear \textit{
	\begin{itemize}
	\end{itemize}
}
\item  \\
 to live and let live \textit{
	\begin{itemize}
	\end{itemize}
}
\item  \\
 live it up \textit{
	\begin{itemize}
	\end{itemize}
}
\end{enumerate}

\section*{lantern}
{\large \color{blue}  lanterns  }
\subsection*{Explain}
\begin{enumerate}
\item countable noun \\
A \textbf{lantern} is a lamp in a metal frame with glass sides and with a handle on top so you can carry it.
 \textit{
	\begin{itemize}
	\end{itemize}
}
\end{enumerate}

\section*{marry}
{\large \color{blue}  marries  marrying  married  }
\subsection*{Explain}
\begin{enumerate}
\item verb \\
When two people \textbf{get married} or \textbf{marry} , they legally become partners in a special  ceremony . \textbf{Get married} is less formal and more commonly used than \textbf{marry} .
 \textit{
	\begin{itemize}
	\item I thought he would change after we got married.
	\item They married a month after they met.
	\item He wants to marry her.
	\item He got married to wife Beryl when he was 19.
	\item I am getting married on Monday.
	\item She ought to marry again, don't you think?
	\end{itemize}
}
\item verb \\
When a priest or official  \textbf{marries} two people, he or she conducts the ceremony in which the two people legally become partners.
 \textit{
	\begin{itemize}
	\item The local vicar has agreed to marry us in the chapel on the estate.
	\item In July 1957, we were married in New York.
	\end{itemize}
}
\item verb \\
If a parent  \textbf{marries} their child  \textbf{to} someone, the parent chooses who their child will marry and arranges it.
 \textit{
	\begin{itemize}
	\end{itemize}
}
\end{enumerate}

\section*{mutton}
{\large \color{blue}  }
\subsection*{Explain}
\begin{enumerate}
\item uncountable noun \\
\textbf{Mutton} is meat from an adult sheep that is eaten as food.
 \textit{
	\begin{itemize}
	\item ...a leg of mutton.
	\item ...mutton stew.
	\end{itemize}
}
\item  \\
 mutton dressed as lamb \textit{
	\begin{itemize}
	\end{itemize}
}
\end{enumerate}

\section*{measure}
{\large \color{blue}  measures  measuring  measured  }
\subsection*{Explain}
\begin{enumerate}
\item verb \\
If you \textbf{measure} the quality, value, or effect of something, you discover or judge how great it is.
 \textit{
	\begin{itemize}
	\item I continued to measure his progress against the charts in the doctor's office.
	\item A school's success was measured in terms of the number of pupils who got into university.
	\item It was difficult to measure the precise impact of the labor action.
	\end{itemize}
}
\item verb \\
If you \textbf{measure} a quantity that can be expressed in numbers , such as the length of something, you discover it using a particular instrument or
device, for example a ruler .
 \textit{
	\begin{itemize}
	\item Measure the length and width of the gap.
	\item He measured the speed at which ultrasonic waves travel along the bone.
	\end{itemize}
}
\item verb \\
If something \textbf{measures} a particular length, width, or amount, that is its size or intensity , expressed in numbers.
 \textit{
	\begin{itemize}
	\item The house is more than twenty metres long and measures six metres in width.
	\item This hand-decorated plate measures 30cm across.
	\item Their paddock measures 24 metres square.
	\end{itemize}
}
\item singular noun \\
A \textbf{measure of} a particular quality, feeling, or activity is a fairly large amount of it.
 \textit{
	\begin{itemize}
	\item With the exception of Juan, each attained a measure of success.
	\item The colonies were claiming a larger measure of self-government.
	\end{itemize}
}
\item singular noun \\
If you say that one aspect of a situation is \textbf{a}  \textbf{measure of} that situation, you mean that it shows that the situation is very serious or has developed to a very great extent.
 \textit{
	\begin{itemize}
	\item It is a measure of their plight that few of them have anywhere to go to.
	\item That is a measure of how bad things have become at the bank.
	\end{itemize}
}
\item countable noun \\
When someone, usually a government or other authority, takes \textbf{measures} to do something, they carry out particular actions in order to achieve a particular
result.
 \textit{
	\begin{itemize}
	\item The government warned that police would take tougher measures to contain the trouble.
	\item He said stern measures would be taken against the killers.
	\item As a precautionary measure repeat the medication.
	\end{itemize}
}
\item countable noun \\
A \textbf{measure}  \textbf{of} a strong alcoholic drink such as brandy or whisky is an amount of it in a glass. In pubs and bars, a \textbf{measure} is an official standard amount.
 \textit{
	\begin{itemize}
	\item He poured himself another generous measure of malt.
	\item ...a pub measure of spirits.
	\end{itemize}
}
\item countable noun \\
In music, a \textbf{measure} is one of the several short parts of the same length into which a piece of music
is divided.
 \textit{
	\begin{itemize}
	\end{itemize}
}
\item  \\
 beyond measure \textit{
	\begin{itemize}
	\end{itemize}
}
\item  \\
 for good measure \textit{
	\begin{itemize}
	\end{itemize}
}
\item  \\
 get/take/have the measure of sb/sthg \textit{
	\begin{itemize}
	\end{itemize}
}
\item  \\
 in large measure \textit{
	\begin{itemize}
	\end{itemize}
}
\end{enumerate}

\section*{pocket}
{\large \color{blue}  pockets  pocketing  pocketed  }
\subsection*{Explain}
\begin{enumerate}
\item countable noun \\
A \textbf{pocket} is a kind of small bag which forms part of a piece of clothing, and which is used
for carrying small things such as money or a handkerchief .
 \textit{
	\begin{itemize}
	\item He took his flashlight from his jacket pocket and switched it on.
	\item The man stood with his hands in his pockets.
	\end{itemize}
}
\item countable noun \\
You can use \textbf{pocket} in a lot of different ways to refer to money that people have, get , or spend . For example, if someone gives or pays a lot of money, you can say that they \textbf{dig deep into} their \textbf{pocket} . If you approve of something because it is very cheap to buy , you can say that it \textbf{suits people's pockets} .
 \textit{
	\begin{itemize}
	\item It really is worth digging deep into your pocket for the best you can afford.
	\item ...ladies' fashions to suit all shapes, sizes and pockets.
	\item You would be buying a piece of history as well as a boat, if you put your hand in
your pocket for this one.
	\item We don't believe that they have the economic reforms in place which would justify
putting huge sums of Western money into their pockets.
	\end{itemize}
}
\item adjective \\
You use \textbf{pocket} to describe something that is small enough to fit into a pocket, often something that is a smaller version of a larger item.
 \textit{
	\begin{itemize}
	\item ...a pocket calculator.
	\item ...my pocket edition of the Collins Spanish Dictionary.
	\end{itemize}
}
\item countable noun \\
A \textbf{pocket}  \textbf{of} something is a small area where something is happening , or a small area which has a particular quality, and which is different from the
other areas around it.
 \textit{
	\begin{itemize}
	\item He survived the earthquake after spending 3 days in an air pocket.
	\item The army controls the city apart from a few pockets of resistance.
	\end{itemize}
}
\item verb \\
If someone who is in possession of something valuable such as a sum of money \textbf{pockets} it, they steal it or take it for themselves, even though it does not belong to them.
 \textit{
	\begin{itemize}
	\item Dishonest importers would be able to pocket the VAT collected from customers.
	\end{itemize}
}
\item verb \\
If you say that someone \textbf{pockets} something such as a prize or sum of money, you mean that they win or obtain it, often
without needing to make much effort or in a way that seems  unfair .
 \textit{
	\begin{itemize}
	\item He pocketed more money from this tournament than in his entire three years as a professional.
	\end{itemize}
}
\item verb \\
If someone \textbf{pockets} something, they put it in their pocket, for example because they want to steal it or hide it.
 \textit{
	\begin{itemize}
	\item Anthony snatched his letters and pocketed them.
	\item He pocketed a wallet containing £40 cash from the bedside of a dead man.
	\end{itemize}
}
\item  \\
 burn a hole in someone's pocket \textit{
	\begin{itemize}
	\end{itemize}
}
\item  \\
 in someone's pocket \textit{
	\begin{itemize}
	\end{itemize}
}
\item  \\
 to line your pockets \textit{
	\begin{itemize}
	\end{itemize}
}
\item  \\
 out of pocket \textit{
	\begin{itemize}
	\end{itemize}
}
\item  \\
 to pick someone's pocket \textit{
	\begin{itemize}
	\end{itemize}
}
\end{enumerate}

\section*{pollution}
{\large \color{blue}  }
\subsection*{Explain}
\begin{enumerate}
\item uncountable noun \\
\textbf{Pollution} is the process of polluting water, air, or land, especially with poisonous chemicals .
 \textit{
	\begin{itemize}
	\item The fine was for the company's pollution of the air near its plants.
	\item Recycling also helps control environmental pollution by reducing the need for waste
dumps.
	\end{itemize}
}
\item uncountable noun \\
\textbf{Pollution} is poisonous or dirty substances that are polluting the water, air, or land somewhere .
 \textit{
	\begin{itemize}
	\item The level of pollution in the river was falling.
	\end{itemize}
}
\end{enumerate}

\section*{overtake}
{\large \color{blue}  overtakes  overtaking  overtook  overtaken  }
\subsection*{Explain}
\begin{enumerate}
\item verb \\
If you \textbf{overtake} a vehicle or a person that is ahead of you and moving in the same direction, you pass them.
 \textit{
	\begin{itemize}
	\item When he eventually overtook the last truck, he pulled over to the inside lane.
	\item The red car was pulling out ready to overtake.
	\end{itemize}
}
\item verb \\
If someone or something \textbf{overtakes} a competitor , they become more successful than them.
 \textit{
	\begin{itemize}
	\item Leeds, who could have overtaken us, lost their last game to Wolves.
	\item The coffee chain has overtaken its US rival.
	\end{itemize}
}
\item verb \\
If an event \textbf{overtakes} you, it happens unexpectedly or suddenly.
 \textit{
	\begin{itemize}
	\item Tragedy was shortly to overtake him, however.
	\end{itemize}
}
\item verb \\
If a feeling  \textbf{overtakes} you, it affects you very strongly.
 \textit{
	\begin{itemize}
	\item Something like panic overtook me.
	\item From the moment Edward had told her of the escape attempt, she had been overtaken
by a sense of impending doom.
	\end{itemize}
}
\end{enumerate}

\section*{pond}
{\large \color{blue}  ponds  }
\subsection*{Explain}
\begin{enumerate}
\item countable noun \\
A \textbf{pond} is a small area of water that is smaller than a lake . Ponds are often made artificially.
 \textit{
	\begin{itemize}
	\item She chose a bench beside the duck pond and sat down.
	\item ...a garden pond.
	\end{itemize}
}
\item singular noun \\
People sometimes  refer to the Atlantic  Ocean as \textbf{the pond} .
 \textit{
	\begin{itemize}
	\item Usually, the presentation is made on the other side of the pond.
	\item Tourist numbers from across the pond have dropped dramatically.
	\end{itemize}
}
\end{enumerate}

\section*{penetrate}
{\large \color{blue}  penetrates  penetrating  penetrated  }
\subsection*{Explain}
\begin{enumerate}
\item verb \\
If something or someone \textbf{penetrates} a physical  object or an area, they succeed in getting into it or passing through it.
 \textit{
	\begin{itemize}
	\item X-rays can penetrate many objects.
	\item His men had been ordered to shoot on sight anyone trying to penetrate the area.
	\end{itemize}
}
\item verb \\
If someone \textbf{penetrates} an organization , a group, or a profession , they succeed in entering it although it is difficult to do so.
 \textit{
	\begin{itemize}
	\item ...the continuing failure of women to penetrate the higher levels of engineering.
	\item They opened an office in Tokyo in an effort to penetrate the Japanese market.
	\end{itemize}
}
\item verb \\
If someone \textbf{penetrates} an enemy group or a rival organization, they succeed in joining it in order to get information or cause trouble .
 \textit{
	\begin{itemize}
	\item The CIA were trying to penetrate a crime ring operating out of Munich.
	\item The army was one of the few institutions the secret police were not encouraged to
penetrate.
	\end{itemize}
}
\item verb \\
If a company or country  \textbf{penetrates} a market or area, they succeed in selling their products there.
 \textit{
	\begin{itemize}
	\item There have been around 15 attempts from outside France to penetrate the market.
	\end{itemize}
}
\item verb \\
If you \textbf{penetrate} something that is difficult to understand, you succeed in understanding it.
 \textit{
	\begin{itemize}
	\item ...long answers that were often difficult to penetrate.
	\end{itemize}
}
\end{enumerate}

\section*{problem}
{\large \color{blue}  problems  }
\subsection*{Explain}
\begin{enumerate}
\item countable noun \\
A \textbf{problem} is a situation that is unsatisfactory and causes difficulties for people.
 \textit{
	\begin{itemize}
	\item ...the economic problems of the inner city.
	\item The main problem is unemployment.
	\item He told Americans that solving the energy problem was very important.
	\item ...people with mental health problems.
	\end{itemize}
}
\item countable noun \\
A \textbf{problem} is a puzzle that requires logical  thought or mathematics to solve it.
 \textit{
	\begin{itemize}
	\item With mathematical problems, you can save time by approximating.
	\end{itemize}
}
\item adjective \\
\textbf{Problem} children or \textbf{problem} families have serious problems or cause serious problems for other people.
 \textit{
	\begin{itemize}
	\item In some cases a problem child is placed in a special school.
	\item She is afraid to contact the social services in case they are labelled a problem
family.
	\end{itemize}
}
\item  \\
 no problem \textit{
	\begin{itemize}
	\end{itemize}
}
\item  \\
 no problem \textit{
	\begin{itemize}
	\end{itemize}
}
\end{enumerate}

\section*{persuade}
{\large \color{blue}  persuades  persuading  persuaded  }
\subsection*{Explain}
\begin{enumerate}
\item verb \\
If you \textbf{persuade} someone \textbf{to} do something, you cause them to do it by giving them good reasons for doing it.
 \textit{
	\begin{itemize}
	\item My partner persuaded me to come.
	\item We're trying to persuade manufacturers to sell them here.
	\item They were eventually persuaded by the police to give themselves up.
	\end{itemize}
}
\item verb \\
If something \textbf{persuades} someone \textbf{to} take a particular course of action, it causes them to take that course of action because it is a good reason
for doing so.
 \textit{
	\begin{itemize}
	\item The Conservative Party's victory in April's general election persuaded him to run
for President again.
	\item It was the lack of privacy that eventually persuaded us to move after Ben was born.
	\end{itemize}
}
\item verb \\
If you \textbf{persuade} someone \textbf{that} something is true , you say things that eventually make them believe that it is true.
 \textit{
	\begin{itemize}
	\item I've persuaded Mrs Tennant that it's time she retired.
	\item We had managed to persuade them that it was worth working with us.
	\item Derek persuaded me of the feasibility of the idea.
	\end{itemize}
}
\end{enumerate}

\section*{product}
{\large \color{blue}  products  }
\subsection*{Explain}
\begin{enumerate}
\item countable noun \\
A \textbf{product} is something that is produced and sold in large quantities, often as a result of
a manufacturing process.
 \textit{
	\begin{itemize}
	\item Try to get the best product at the lowest price.
	\item ...the company's ability to produce cheap electronic consumer products.
	\end{itemize}
}
\item countable noun \\
If you say that someone or something is a \textbf{product of} a situation or process, you mean that the situation or process has had a significant effect in making them what they are.
 \textit{
	\begin{itemize}
	\item We are all products of our time.
	\item The bank is the product of a 1971 merger of two Japanese banks.
	\end{itemize}
}
\end{enumerate}

\section*{preach}
{\large \color{blue}  preaches  preaching  preached  }
\subsection*{Explain}
\begin{enumerate}
\item verb \\
When a member of the clergy  \textbf{preaches} a sermon, he or she gives a talk on a religious or moral subject during a religious service .
 \textit{
	\begin{itemize}
	\item At High Mass the priest preached a sermon on the devil.
	\item The bishop preached to a crowd of several hundred local people.
	\item He denounced the decision to invite his fellow archbishop to preach.
	\end{itemize}
}
\item verb \\
When people \textbf{preach} a belief or a course of action, they try to persuade other people to accept the belief or to take the course of action.
 \textit{
	\begin{itemize}
	\item He tried to preach peace and tolerance to his people.
	\item Health experts are now preaching that even a little exercise is far better than none
at all.
	\item For many years I have preached against war.
	\end{itemize}
}
\item verb \\
If someone gives you advice in a very serious , boring way, you can  say that they \textbf{are preaching at} you.
 \textit{
	\begin{itemize}
	\item 'Don't preach at me,' he shouted.
	\end{itemize}
}
\item  \\
 to practise what you preach \textit{
	\begin{itemize}
	\end{itemize}
}
\item  \\
 to preach to the converted \textit{
	\begin{itemize}
	\end{itemize}
}
\end{enumerate}

\section*{punch}
{\large \color{blue}  punches  punching  punched  }
\subsection*{Explain}
\begin{enumerate}
\item verb \\
If you \textbf{punch} someone or something, you hit them hard with your fist.
 In American English, \textbf{punch out} means the same as punch .
 \textbf{Punch} is also a noun .
 \textit{
	\begin{itemize}
	\item After punching him on the chin she wound up hitting him over the head.
	\item He punched the wall angrily, then spun round to face her.
	\item 'I almost lost my job today.'—'What happened?'—'Oh, I punched out this guy.'.
	\item In the past, many kids would settle disputes by punching each other out.
	\item He was hurting Johansson with body punches in the fourth round.
	\end{itemize}
}
\item verb \\
If you \textbf{punch the air} , you put one or both of your fists forcefully above your shoulders as a gesture of delight or victory .
 \textit{
	\begin{itemize}
	\item At the end, Graf punched the air in delight, a huge grin on her face.
	\end{itemize}
}
\item verb \\
If you \textbf{punch} something such as the buttons on a keyboard , you touch them in order to store information on a machine such as a computer or to give the machine a command to do something.
 \textit{
	\begin{itemize}
	\item Mrs. Baylor strode to the elevator and punched the button.
	\end{itemize}
}
\item verb \\
If you \textbf{punch} holes \textbf{in} something, you make holes in it by pushing or pressing it with something sharp .
 \textit{
	\begin{itemize}
	\item I took a ballpoint pen and punched a hole in the carton.
	\end{itemize}
}
\item countable noun \\
A \textbf{punch} is a tool that you use for making holes in something.
 \textit{
	\begin{itemize}
	\item Make two holes with a hole punch.
	\end{itemize}
}
\item uncountable noun \\
If you say that something has \textbf{punch} , you mean that it has force or effectiveness.
 \textit{
	\begin{itemize}
	\item My nervousness made me deliver the vital points of my address without sufficient
punch.
	\item Hurricane Andrew may be slowly losing its punch, but its winds are still around 100
miles an hour.
	\end{itemize}
}
\item variable noun \\
\textbf{Punch} is a drink made from wine or spirits mixed with things such as sugar , lemons , and spices.
 \textit{
	\begin{itemize}
	\end{itemize}
}
\item  \\
 to pull your punches \textit{
	\begin{itemize}
	\end{itemize}
}
\end{enumerate}

\section*{preserve}
{\large \color{blue}  preserves  preserving  preserved  }
\subsection*{Explain}
\begin{enumerate}
\item verb \\
If you \textbf{preserve} a situation or condition, you make sure that it remains as it is, and does not change or end.
 \textit{
	\begin{itemize}
	\item We will do everything to preserve peace.
	\item ...an effort to fit in more students while preserving standards.
	\end{itemize}
}
\item verb \\
If you \textbf{preserve} something, you take action to save it or protect it from damage or decay.
 \textit{
	\begin{itemize}
	\item We need to preserve the forest.
	\item Why are we so obsessed with preserving buildings that are no longer fit for purpose?

	\item ...perfectly preserved medieval houses.
	\end{itemize}
}
\item verb \\
If you \textbf{preserve} food, you treat it in order to prevent it from decaying so that you can store it for a long time.
 \textit{
	\begin{itemize}
	\item I like to make puree, using only enough sugar to preserve the plums.
	\item ...preserved ginger in syrup.
	\end{itemize}
}
\item plural noun \\
\textbf{Preserves} are foods such as jam that are made by cooking fruit with a large amount of sugar so that they can be stored
for a long time.
 \textit{
	\begin{itemize}
	\end{itemize}
}
\item countable noun \\
If you say that a job or activity is the \textbf{preserve of} a particular person or group of people, you mean that they are the only ones who
take part in it.
 \textit{
	\begin{itemize}
	\item The conduct of foreign policy is largely the preserve of the president.
	\item Many areas of the film industry remain almost entirely male preserves.
	\end{itemize}
}
\item countable noun \\
A nature  \textbf{preserve} is an area of land or water where animals are protected from hunters .
 \textit{
	\begin{itemize}
	\item ...Pantanal, one of the world's great wildlife preserves.
	\end{itemize}
}
\end{enumerate}

\section*{question}
{\large \color{blue}  questions  questioning  questioned  }
\subsection*{Explain}
\begin{enumerate}
\item countable noun \\
A \textbf{question} is something that you say or write in order to ask a person about something.
 \textit{
	\begin{itemize}
	\item They asked a great many questions about England.
	\item The President refused to answer further questions on the subject.
	\item Right, next question.
	\end{itemize}
}
\item verb \\
If you \textbf{question} someone, you ask them a lot of questions about something.
 \textit{
	\begin{itemize}
	\item This led the therapist to question Jim about his parents.
	\end{itemize}
}
\item verb \\
If you \textbf{question} something, you have or express doubts about whether it is true , reasonable , or worthwhile .
 \textit{
	\begin{itemize}
	\item It never occurs to them to question the doctor's decisions.
	\item Weber is challenging his audience to question their own beliefs.
	\end{itemize}
}
\item  \\
 to call something into question \textit{
	\begin{itemize}
	\end{itemize}
}
\item countable noun \\
A \textbf{question} is a problem, matter, or point which needs to be considered .
 \textit{
	\begin{itemize}
	\item But the whole question of aid is a tricky political one.
	\item That decision raised questions about the fairness of his procedure.
	\item The question is: Is this what we really want?
	\item ...if the security question is not resolved.
	\item It was just a question of having the time to re-adjust.
	\end{itemize}
}
\item countable noun \\
The \textbf{questions} in an examination are the problems which are set in order to test your knowledge or ability .
 \textit{
	\begin{itemize}
	\item The style of exam questions had changed over the decades.
	\item That question did come up in the examination.
	\end{itemize}
}
\item  \\
 good question \textit{
	\begin{itemize}
	\end{itemize}
}
\item  \\
 in question \textit{
	\begin{itemize}
	\end{itemize}
}
\item  \\
 out of the question \textit{
	\begin{itemize}
	\end{itemize}
}
\item  \\
 to pop the question \textit{
	\begin{itemize}
	\end{itemize}
}
\item  \\
 there's no question of doing sth \textit{
	\begin{itemize}
	\end{itemize}
}
\item  \\
 without question \textit{
	\begin{itemize}
	\end{itemize}
}
\item  \\
 without question \textit{
	\begin{itemize}
	\end{itemize}
}
\end{enumerate}

\section*{proclaim}
{\large \color{blue}  proclaims  proclaiming  proclaimed  }
\subsection*{Explain}
\begin{enumerate}
\item verb \\
If people \textbf{proclaim} something, they formally make it known to the public.
 \textit{
	\begin{itemize}
	\item He has sent me to proclaim liberty to the captives.
	\item Britain proudly proclaims that it is a nation of animal lovers.
	\item He still proclaims himself a believer in the Revolution.
	\end{itemize}
}
\item verb \\
If you \textbf{proclaim} something, you state it in an emphatic way.
 \textit{
	\begin{itemize}
	\item 'I think we have been heard today,' he proclaimed.
	\item He confidently proclaims that he is offering the best value in the market.
	\end{itemize}
}
\end{enumerate}

\section*{relation}
{\large \color{blue}  relations  }
\subsection*{Explain}
\begin{enumerate}
\item countable noun \\
\textbf{Relations} between people, groups, or countries are contacts between them and the way in which they behave towards each other.
 \textit{
	\begin{itemize}
	\item Greece has established full diplomatic relations with Israel.
	\item Relations between the two men became strained.
	\item The company has a track record of good employee relations.
	\end{itemize}
}
\item countable noun \\
If you talk about the \textbf{relation}  \textbf{of} one thing \textbf{to} another, you are talking about the ways in which they are connected.
 \textit{
	\begin{itemize}
	\item It is a question of the relation of ethics to economics.
	\item ...the relation between power and knowledge.
	\item This theory bears no relation to reality.
	\end{itemize}
}
\item countable noun \\
Your \textbf{relations} are the members of your family.
 \textit{
	\begin{itemize}
	\item ...visits to friends and relations.
	\item Jansher, another Khan from Peshawar, though no relation, dominated squash in the
Nineties.
	\end{itemize}
}
\item  \\
 in relation to \textit{
	\begin{itemize}
	\end{itemize}
}
\item  \\
 in relation to \textit{
	\begin{itemize}
	\end{itemize}
}
\end{enumerate}

\section*{promote}
{\large \color{blue}  promotes  promoting  promoted  }
\subsection*{Explain}
\begin{enumerate}
\item verb \\
If people \textbf{promote} something, they help or encourage it to happen , increase , or spread .
 \textit{
	\begin{itemize}
	\item You don't have to sacrifice environmental protection to promote economic growth.
	\item We actively promote the use of alternative transport methods.
	\end{itemize}
}
\item verb \\
If a firm  \textbf{promotes} a product, it tries to increase the sales or popularity of that product.
 \textit{
	\begin{itemize}
	\item The singer has announced a full British tour to promote his second solo album.
	\item ...a special St Lucia week where the island could be promoted as a tourist destination.
	\end{itemize}
}
\item verb \\
If someone \textbf{is promoted} , they are given a more important  job or rank in the organization that they work for.
 \textit{
	\begin{itemize}
	\item I was promoted to editor and then editorial director.
	\item In fact, those people have been promoted.
	\end{itemize}
}
\item verb \\
If a team that competes in a league  \textbf{is promoted} , it starts competing in a higher division in the next  season because it was one of the most successful teams in the lower division.
 \textit{
	\begin{itemize}
	\item They won the Second Division title and were promoted to the First Division.
	\end{itemize}
}
\end{enumerate}

\section*{relationship}
{\large \color{blue}  relationships  }
\subsection*{Explain}
\begin{enumerate}
\item countable noun \\
The \textbf{relationship} between two people or groups is the way in which they feel and behave towards each other.
 \textit{
	\begin{itemize}
	\item ...the friendly relationship between France and Britain.
	\item ...family relationships.
	\end{itemize}
}
\item countable noun \\
A \textbf{relationship} is a close friendship between two people, especially one involving romantic or sexual feelings.
 \textit{
	\begin{itemize}
	\item Both of us felt the relationship wasn't really going anywhere.
	\end{itemize}
}
\item countable noun \\
The \textbf{relationship} between two things is the way in which they are connected.
 \textit{
	\begin{itemize}
	\item There is a relationship between diet and cancer.
	\item ...market mechanisms and their relationship to state capitalism and political freedom.
	\end{itemize}
}
\end{enumerate}

\section*{pursue}
{\large \color{blue}  pursues  pursuing  pursued  }
\subsection*{Explain}
\begin{enumerate}
\item verb \\
If you \textbf{pursue} an activity, interest , or plan, you carry it out or follow it.
 \textit{
	\begin{itemize}
	\item He said his country would continue to pursue the policies laid down at the summit.
	\item She had come to England to pursue an acting career.
	\end{itemize}
}
\item verb \\
If you \textbf{pursue} a particular aim or result, you make efforts to achieve it, often over a long period of time.
 \textit{
	\begin{itemize}
	\item The implication is that it is impossible to pursue economic reform and democracy
simultaneously.
	\item Mr. Menendez has aggressively pursued new business.
	\end{itemize}
}
\item verb \\
If you \textbf{pursue} a particular topic , you try to find out more about it by asking  questions .
 \textit{
	\begin{itemize}
	\item If your original request is denied, don't be afraid to pursue the matter.
	\end{itemize}
}
\item verb \\
If you \textbf{pursue} a person, vehicle, or animal, you follow them, usually in order to catch them.
 \textit{
	\begin{itemize}
	\item She pursued the man who had stolen a woman's bag.
	\end{itemize}
}
\end{enumerate}

\section*{salad}
{\large \color{blue}  salads  }
\subsection*{Explain}
\begin{enumerate}
\item variable noun \\
A \textbf{salad} is a mixture of raw or cold foods such as lettuce, cucumber , and tomatoes. It is often served with other food as part of a meal .
 \textit{
	\begin{itemize}
	\item ...a salad of tomato, onion and cucumber.
	\item ...potato salad.
	\end{itemize}
}
\item  \\
 one's salad days \textit{
	\begin{itemize}
	\end{itemize}
}
\end{enumerate}

\section*{raise}
{\large \color{blue}  raises  raising  raised  }
\subsection*{Explain}
\begin{enumerate}
\item verb \\
If you \textbf{raise} something, you move it so that it is in a higher position.
 \textit{
	\begin{itemize}
	\item He raised his hand to wave.
	\item She went to the window and raised the blinds.
	\item Milton raised the glass to his lips.
	\item ...a small raised platform.
	\end{itemize}
}
\item verb \\
If you \textbf{raise} a flag , you display it by moving it up a pole or into a high place where it can be seen.
 \textit{
	\begin{itemize}
	\item They had raised the white flag in surrender.
	\item At midnight, the German flag will be raised over the Reichstag.
	\end{itemize}
}
\item verb \\
If you \textbf{raise}  \textbf{yourself} , you lift your body so that you are standing up straight , or so that you are no longer lying flat.
 \textit{
	\begin{itemize}
	\item He raised himself into a sitting position.
	\item She raised herself on one elbow.
	\end{itemize}
}
\item verb \\
If you \textbf{raise} the rate or level of something, you increase it.
 \textit{
	\begin{itemize}
	\item The Republic of Ireland is expected to raise interest rates.
	\item Two incidents in recent days have raised the level of concern.
	\item ...a raised body temperature.
	\end{itemize}
}
\item verb \\
To \textbf{raise} the standard of something means to improve it.
 \textit{
	\begin{itemize}
	\item ...a new drive to raise standards of literacy in Britain's schools.
	\end{itemize}
}
\item verb \\
If you \textbf{raise} your \textbf{voice} , you speak more loudly, usually because you are angry .
 \textit{
	\begin{itemize}
	\item Don't you raise your voice to me, Henry Rollins!
	\item Anne raised her voice in order to be heard.
	\end{itemize}
}
\item countable noun \\
A \textbf{raise} is an increase in your wages or salary.
 \textit{
	\begin{itemize}
	\item Within two months Kelly got a raise.
	\end{itemize}
}
\item verb \\
If you \textbf{raise} money \textbf{for} a charity or an institution, you ask people for money which you collect on its behalf .
 \textit{
	\begin{itemize}
	\item ...events held to raise money for Help the Aged.
	\item All funds raised will be used by Children With Leukaemia.
	\end{itemize}
}
\item verb \\
If a person or company \textbf{raises} money that they need, they manage to get it, for example by selling their property or by borrowing .
 \textit{
	\begin{itemize}
	\item They raised the money to buy the house and two hundred acres of grounds.
	\end{itemize}
}
\item verb \\
If an event \textbf{raises} a particular emotion or question, it makes people feel the emotion or consider the
question.
 \textit{
	\begin{itemize}
	\item The agreement has raised hopes that the war may end soon.
	\item The accident again raises questions about the safety of the plant.
	\item ...a joke that raised a smile on everyone's lips.
	\end{itemize}
}
\item verb \\
If you \textbf{raise} a subject, an objection , or a question, you mention it or bring it to someone's attention.
 \textit{
	\begin{itemize}
	\item The debate was not long but all the issues were raised.
	\item He had been consulted and had raised no objections.
	\end{itemize}
}
\item verb \\
Someone who \textbf{raises} a child looks after it until it is grown up.
 \textit{
	\begin{itemize}
	\item My mother was an amazing woman. She raised four of us kids virtually singlehandedly.
	\item ...the house where she was raised.
	\end{itemize}
}
\item verb \\
If someone \textbf{raises} a particular type of animal or crop , they breed that type of animal or grow that type of crop.
 \textit{
	\begin{itemize}
	\item He raises 2,000 acres of wheat and hay.
	\item ...a perfectly cooked farm-raised chicken.
	\end{itemize}
}
\end{enumerate}

\section*{soup}
{\large \color{blue}  soups  souping  souped  }
\subsection*{Explain}
\begin{enumerate}
\item variable noun \\
\textbf{Soup} is liquid food made by boiling meat, fish, or vegetables in water.
 \textit{
	\begin{itemize}
	\item ...home-made chicken soup.
	\end{itemize}
}
\item  \\
 in the soup \textit{
	\begin{itemize}
	\end{itemize}
}
\end{enumerate}

\section*{recede}
{\large \color{blue}  recedes  receding  receded  }
\subsection*{Explain}
\begin{enumerate}
\item verb \\
If something \textbf{recedes} from you, it moves away .
 \textit{
	\begin{itemize}
	\item Luke's footsteps receded into the night.
	\item As she receded he waved goodbye.
	\item ...the receding lights of the car.
	\end{itemize}
}
\item verb \\
When something such as a quality, problem , or illness  \textbf{recedes} , it becomes weaker , smaller, or less intense .
 \textit{
	\begin{itemize}
	\item Just as I started to think that I was never going to get well, the illness began
to recede.
	\item Dealers grew concerned over the sliding dollar and receding prospects for economic
recovery.
	\end{itemize}
}
\item verb \\
If a man's hair starts to \textbf{recede} , it no longer grows on the front of his head.
 \textit{
	\begin{itemize}
	\item ...a youngish man with dark hair just beginning to recede.
	\end{itemize}
}
\item verb \\
If your gums start to \textbf{recede} , they begin to cover less of your teeth , usually as the result of an infection .
 \textit{
	\begin{itemize}
	\item If untreated, the gums recede, become swollen and bleed.
	\item Receding gums can be the result of disease or simply incorrect brushing.
	\end{itemize}
}
\end{enumerate}

\section*{spacecraft}
{\large \color{blue}  spacecraft  }
\subsection*{Explain}
\begin{enumerate}
\item countable noun \\
A \textbf{spacecraft} is a rocket or other vehicle that can travel in space .
 \textit{
	\begin{itemize}
	\end{itemize}
}
\end{enumerate}

\section*{relate}
{\large \color{blue}  relates  relating  related  }
\subsection*{Explain}
\begin{enumerate}
\item verb \\
If something \textbf{relates to} a particular subject , it concerns that subject.
 \textit{
	\begin{itemize}
	\item Other recommendations relate to the details of how such data is stored.
	\item I had papers relating to the children which we had to sign.
	\end{itemize}
}
\item verb \\
The way that two things \textbf{relate} , or the way that one thing \textbf{relates}  \textbf{to} another, is the sort of connection that exists between them.
 \textit{
	\begin{itemize}
	\item He understands building and cities, the spaces between them and how they relate to
each other.
	\item The course investigates how language relates to particular cultural codes.
	\item He felt the need to relate his experience to that of people from different cultures.
	\item ...a paper in which the writer tries to relate his linguistic and political views.
	\item At the end, we have a sense of names, dates, and events but no sense of how they
relate.
	\end{itemize}
}
\item verb \\
If you can \textbf{relate}  \textbf{to} someone, you can understand how they feel or behave so that you are able to communicate with them or deal with them easily .
 \textit{
	\begin{itemize}
	\item He is unable to relate to other people.
	\item When people are cut off from contact with others, they lose all ability to relate.
	\end{itemize}
}
\item verb \\
If you \textbf{relate} a story, you tell it.
 \textit{
	\begin{itemize}
	\item There were officials to whom he could relate the whole story.
	\item She related her tale of living rough.
	\end{itemize}
}
\end{enumerate}

\section*{spaceship}
{\large \color{blue}  spaceships  }
\subsection*{Explain}
\begin{enumerate}
\item countable noun \\
A \textbf{spaceship} is a spacecraft that carries people through space .
 \textit{
	\begin{itemize}
	\end{itemize}
}
\end{enumerate}

\section*{restore}
{\large \color{blue}  restores  restoring  restored  }
\subsection*{Explain}
\begin{enumerate}
\item verb \\
To \textbf{restore} a situation or practice  means to cause it to exist again.
 \textit{
	\begin{itemize}
	\item The army has recently been brought in to restore order.
	\item As they smiled at each other, harmony was restored again.
	\item The death penalty was never restored.
	\end{itemize}
}
\item verb \\
To \textbf{restore} someone or something \textbf{to} a previous condition means to cause them to be in that condition once again.
 \textit{
	\begin{itemize}
	\item We will restore her to health but it may take time.
	\item He said the ousted president must be restored to power.
	\item His country desperately needs Western aid to restore its ailing economy.
	\end{itemize}
}
\item verb \\
When someone \textbf{restores} something such as an old building, painting , or piece of furniture , they repair and clean it, so that it looks like it did when it was new.
 \textit{
	\begin{itemize}
	\item ...experts who specialise in examining and restoring ancient parchments.
	\item ...the beautifully restored old town square.
	\end{itemize}
}
\item verb \\
If something that was lost or stolen \textbf{is restored}  \textbf{to} its owner, it is returned to them.
 \textit{
	\begin{itemize}
	\item The following day their horses and goods were restored to them.
	\item The burglars were arrested and my stolen property was restored.
	\end{itemize}
}
\end{enumerate}

\section*{stain}
{\large \color{blue}  stains  staining  stained  }
\subsection*{Explain}
\begin{enumerate}
\item countable noun \\
A \textbf{stain} is a mark on something that is difficult to remove .
 \textit{
	\begin{itemize}
	\item Remove stains by soaking in a mild solution of bleach.
	\item ...a black stain.
	\end{itemize}
}
\item verb \\
If a liquid \textbf{stains} something, the thing becomes coloured or marked by the liquid.
 \textit{
	\begin{itemize}
	\item Some foods can stain the teeth, as of course can smoking.
	\end{itemize}
}
\end{enumerate}

\section*{retain}
{\large \color{blue}  retains  retaining  retained  }
\subsection*{Explain}
\begin{enumerate}
\item verb \\
To \textbf{retain} something means to continue to have that thing.
 \textit{
	\begin{itemize}
	\item The interior of the shop still retains a nineteenth-century atmosphere.
	\item He retains a deep respect for the profession.
	\item Other countries retained their traditional and habitual ways of doing things.
	\item If left covered in a warm place, this rice will retain its heat for a good hour.
	\end{itemize}
}
\item verb \\
If you \textbf{retain} a lawyer , you pay him or her a fee to make sure that he or she will  represent you when your case  comes before the court.
 \textit{
	\begin{itemize}
	\item He decided to retain him for the trial.
	\end{itemize}
}
\end{enumerate}

\section*{strife}
{\large \color{blue}  }
\subsection*{Explain}
\begin{enumerate}
\item uncountable noun \\
\textbf{Strife} is strong  disagreement or fighting .
 \textit{
	\begin{itemize}
	\item Money is a major cause of strife in many marriages.
	\item The boardroom strife at the company is far from over.
	\item It remains a highly unstable and strife-torn country.
	\end{itemize}
}
\end{enumerate}

\section*{retire}
{\large \color{blue}  retires  retiring  retired  }
\subsection*{Explain}
\begin{enumerate}
\item verb \\
When older people \textbf{retire} , they leave their job and usually stop  working completely.
 \textit{
	\begin{itemize}
	\item At the age when most people retire, he is ready to face a new career.
	\item Although their careers are important many said they plan to retire at 50.
	\item In 1974 he retired from the museum.
	\end{itemize}
}
\item verb \\
When a sports player  \textbf{retires}  \textbf{from} their sport, they stop playing in competitions . When they \textbf{retire}  \textbf{from} a race or a match , they stop competing in it.
 \textit{
	\begin{itemize}
	\item I have decided to retire from Formula One racing at the end of the season.
	\item One of the most serious injuries was to Simon Littlejohn, who was forced to retire
from the race with a leg injury.
	\end{itemize}
}
\item verb \\
If you \textbf{retire}  \textbf{to} another room or place, you go there.
 \textit{
	\begin{itemize}
	\item Eisenhower left the White House and retired to his farm in Gettysburg.
	\end{itemize}
}
\item verb \\
When a jury in a court of law  \textbf{retires} , the members of it leave the court in order to decide whether someone is guilty or innocent .
 \textit{
	\begin{itemize}
	\item The jury will retire to consider its verdict today.
	\end{itemize}
}
\item verb \\
When you \textbf{retire} , you go to bed.
 \textit{
	\begin{itemize}
	\item She retires early most nights, exhausted.
	\item Some time after midnight, he retired to bed.
	\end{itemize}
}
\end{enumerate}

\section*{sweat}
{\large \color{blue}  sweats  sweating  sweated  }
\subsection*{Explain}
\begin{enumerate}
\item uncountable noun \\
\textbf{Sweat} is the salty  colourless liquid which comes through your skin when you are hot , ill , or afraid .
 \textit{
	\begin{itemize}
	\item Both horse and rider were dripping with sweat within five minutes.
	\item He wiped the sweat off his face and looked around.
	\item Her sweat-stained clothing clung to her body.
	\end{itemize}
}
\item verb \\
When you \textbf{sweat} , sweat comes through your skin.
 \textit{
	\begin{itemize}
	\item Already they were sweating as the sun beat down upon them.
	\end{itemize}
}
\item countable noun \\
If someone is \textbf{in} a \textbf{sweat} , they are sweating a lot .
 \textit{
	\begin{itemize}
	\item Every morning I would break out in a sweat.
	\item Cool down very gradually after working up a sweat.
	\item I really don't feel a bit sick, no night sweats, no fevers.
	\end{itemize}
}
\item plural noun \\
\textbf{Sweats} are the same as a sweatsuit or , sweatpants .
 \textit{
	\begin{itemize}
	\end{itemize}
}
\item  \\
 in a cold sweat/in a sweat \textit{
	\begin{itemize}
	\end{itemize}
}
\item  \\
 sweat it out \textit{
	\begin{itemize}
	\end{itemize}
}
\item  \\
 no sweat \textit{
	\begin{itemize}
	\end{itemize}
}
\end{enumerate}

\section*{revive}
{\large \color{blue}  revives  reviving  revived  }
\subsection*{Explain}
\begin{enumerate}
\item verb \\
When something such as the economy , a business , a trend , or a feeling  \textbf{is revived} or when it \textbf{revives} , it becomes active, popular , or successful again.
 \textit{
	\begin{itemize}
	\item ...an attempt to revive the British economy.
	\item His trial revived memories of French suffering during the war.
	\item There is no doubt that grades have improved and interest in education has revived.
	\end{itemize}
}
\item verb \\
When someone \textbf{revives} a play, opera , or ballet , they present a new production of it.
 \textit{
	\begin{itemize}
	\item The Gaiety is reviving John B. Kean's comedy 'The Man from Clare'.
	\end{itemize}
}
\item verb \\
If you manage to \textbf{revive} someone who has fainted or if they \textbf{revive} , they become conscious again.
 \textit{
	\begin{itemize}
	\item She and a neighbour tried in vain to revive him.
	\item With a glazed stare she revived for one last instant.
	\end{itemize}
}
\end{enumerate}

\section*{symphony}
{\large \color{blue}  symphonies  }
\subsection*{Explain}
\begin{enumerate}
\item countable noun \\
A \textbf{symphony} is a piece of music written to be played by an orchestra . Symphonies are usually made up of four  separate  sections  called movements.
 \textit{
	\begin{itemize}
	\end{itemize}
}
\end{enumerate}

\section*{shave}
{\large \color{blue}  shaves  shaving  shaved  }
\subsection*{Explain}
\begin{enumerate}
\item verb \\
When a man  \textbf{shaves} , he removes the hair from his face using a razor or shaver so that his face is smooth .
 \textbf{Shave} is also a noun .
 \textit{
	\begin{itemize}
	\item He took a bath and shaved before dinner.
	\item He had shaved his face until it was smooth.
	\item It's a pity you shaved your moustache off.
	\item He never seemed to need a shave.
	\end{itemize}
}
\item verb \\
If someone \textbf{shaves} a part of their body, they remove the hair from it so that it is smooth.
 \textit{
	\begin{itemize}
	\item Some women shave their legs.
	\item If you have long curly hair, don't shave it off.
	\end{itemize}
}
\item verb \\
If you \textbf{shave} someone, you remove the hair from their face or another part of their body so that
it is smooth.
 \textit{
	\begin{itemize}
	\item The doctors shaved his head.
	\item She had to call a barber to shave him.
	\end{itemize}
}
\item verb \\
If you \textbf{shave off} part of a piece of wood or other material , you cut very thin pieces from it.
 \textit{
	\begin{itemize}
	\item I set the log on the ground and shaved off the bark.
	\item She was shaving thin slices off a courgette.
	\end{itemize}
}
\item verb \\
If you \textbf{shave} a small amount \textbf{off} something such as a record , cost , or price, you reduce it by that amount.
 \textit{
	\begin{itemize}
	\item She's already shaved four seconds off the national record for the mile.
	\item Supermarket chains have shaved prices.
	\end{itemize}
}
\item  \\
 a close shave \textit{
	\begin{itemize}
	\end{itemize}
}
\end{enumerate}

\section*{tariff}
{\large \color{blue}  tariffs  }
\subsection*{Explain}
\begin{enumerate}
\item countable noun \\
A \textbf{tariff} is a tax that a government collects on goods coming into a country.
 \textit{
	\begin{itemize}
	\item America wants to eliminate tariffs on items such as electronics.
	\end{itemize}
}
\item countable noun \\
A \textbf{tariff} is the rate at which you are charged for public services such as gas and electricity, or for
 accommodation and services in a hotel .
 \textit{
	\begin{itemize}
	\item The daily tariff includes accommodation and unlimited use of the pool and gymnasium.
	\item ...electricity tariffs and telephone charges.
	\end{itemize}
}
\end{enumerate}

\section*{traffic}
{\large \color{blue}  traffics  trafficking  trafficked  }
\subsection*{Explain}
\begin{enumerate}
\item uncountable noun \\
\textbf{Traffic}  refers to all the vehicles that are moving along the roads in a particular area.
 \textit{
	\begin{itemize}
	\item There was heavy traffic on the roads.
	\item Traffic was unusually light for that time of day.
	\item ...the problems of city life, such as traffic congestion.
	\end{itemize}
}
\item uncountable noun \\
\textbf{Traffic} refers to the movement of ships, trains , or aircraft between one place and another. \textbf{Traffic}  also refers to the people and goods that are being transported.
 \textit{
	\begin{itemize}
	\item Air traffic had returned to normal.
	\item The railways will carry a far higher proportion of freight traffic.
	\item The ferries can cope with the traffic of both goods and passengers.
	\end{itemize}
}
\item uncountable noun \\
\textbf{Traffic}  \textbf{in} something such as drugs or stolen goods is an illegal trade in them.
 \textit{
	\begin{itemize}
	\item He condemned the ruthless illegal traffic in endangered animals.
	\end{itemize}
}
\item verb \\
Someone who \textbf{traffics}  \textbf{in} something such as drugs or stolen goods buys and sells them even though it is illegal to do so.
 \textit{
	\begin{itemize}
	\item The president said anyone who traffics in illegal drugs should be brought to justice.
	\end{itemize}
}
\item uncountable noun \\
The amount of \textbf{traffic} that a website  gets is the number of visitors to that website.
 \textit{
	\begin{itemize}
	\item Traffic to the site had increased threefold.
	\end{itemize}
}
\end{enumerate}

\section*{stoop}
{\large \color{blue}  stoops  stooping  stooped  }
\subsection*{Explain}
\begin{enumerate}
\item verb \\
If you \textbf{stoop} , you stand or walk with your shoulders bent forwards.
 \textbf{Stoop} is also a noun .
 \textit{
	\begin{itemize}
	\item She was taller than he was and stooped slightly.
	\item He was a tall, thin fellow with a slight stoop.
	\end{itemize}
}
\item verb \\
If you \textbf{stoop} , you bend your body forwards and downwards .
 \textit{
	\begin{itemize}
	\item He stooped to pick up the carrier bag of groceries.
	\item Two men in shirt sleeves stooped over the car.
	\item Stooping down, he picked up a big stone and hurled it.
	\end{itemize}
}
\item verb \\
If you say that a person \textbf{stoops}  \textbf{to} doing something, you are criticizing them because they do something wrong or immoral that they would not normally do.
 \textit{
	\begin{itemize}
	\item He had not, until recently, stooped to personal abuse.
	\item They've stooped to using any and every weapon at their disposal.
	\item How could anyone stoop so low?
	\end{itemize}
}
\item countable noun \\
A \textbf{stoop} is a small platform at the door of a building, with steps leading up to it.
 \textit{
	\begin{itemize}
	\item They stood together on the stoop and rang the bell.
	\end{itemize}
}
\end{enumerate}

\section*{universe}
{\large \color{blue}  universes  }
\subsection*{Explain}
\begin{enumerate}
\item countable noun \\
\textbf{The}  \textbf{universe} is the whole of space and all the stars , planets , and other forms of matter and energy in it.
 \textit{
	\begin{itemize}
	\item Einstein's equations showed the Universe to be expanding.
	\item Early astronomers thought that our planet was the centre of the universe.
	\end{itemize}
}
\item countable noun \\
If you talk about someone's \textbf{universe} , you are referring to the whole of their experience or an important part of it.
 \textit{
	\begin{itemize}
	\item Good writers suck in what they see of the world, re-creating their own universe on
the page.
	\item They marked out the boundaries of our visual universe.
	\item Behind his eyes was a whole universe of pain.
	\end{itemize}
}
\item  \\
 in the universe \textit{
	\begin{itemize}
	\end{itemize}
}
\end{enumerate}

\section*{tempt}
{\large \color{blue}  tempts  tempting  tempted  }
\subsection*{Explain}
\begin{enumerate}
\item verb \\
Something that \textbf{tempts} you attracts you and makes you want it, even though it may be wrong or harmful .
 \textit{
	\begin{itemize}
	\item Reducing income could tempt an offender into further crime.
	\item It is the fresh fruit that tempts me at this time of year.
	\item Can I tempt you with a little puff pastry?
	\item The fact that she had become wealthy did not tempt her to alter her frugal way of
life.
	\end{itemize}
}
\item verb \\
If you \textbf{tempt} someone, you offer them something they want in order to encourage them to do what you want them to do.
 \textit{
	\begin{itemize}
	\item ...a million dollar marketing campaign to tempt American tourists back to Britain.
	\item Having spent so long at a great club like Rangers, no other Scottish team could tempt
him away.
	\item Don't let credit tempt you to buy something you can't afford.
	\item The bank will offer a current account and try to tempt customers into switching.
	\end{itemize}
}
\item  \\
 to tempt fate \textit{
	\begin{itemize}
	\end{itemize}
}
\end{enumerate}

\section*{wardrobe}
{\large \color{blue}  wardrobes  }
\subsection*{Explain}
\begin{enumerate}
\item countable noun \\
A \textbf{wardrobe} is a tall cupboard or cabinet in which you can hang your clothes.
 \textit{
	\begin{itemize}
	\end{itemize}
}
\item countable noun \\
Someone's \textbf{wardrobe} is the total collection of clothes that they have.
 \textit{
	\begin{itemize}
	\item Her wardrobe consists primarily of huge cashmere sweaters and tiny Italian sandals.
	\end{itemize}
}
\item uncountable noun \\
The \textbf{wardrobe} in a theatre company is the actors ' and actresses ' costumes.
 \textit{
	\begin{itemize}
	\item In the wardrobe department were rows of costumes.
	\end{itemize}
}
\end{enumerate}

\section*{utter}
{\large \color{blue}  utters  uttering  uttered  }
\subsection*{Explain}
\begin{enumerate}
\item verb \\
If someone \textbf{utters} sounds or words, they say them.
 \textit{
	\begin{itemize}
	\item He uttered a snorting laugh.
	\item They departed without uttering a word.
	\end{itemize}
}
\item adjective \\
You use \textbf{utter} to emphasize that something is great in extent , degree , or amount.
 \textit{
	\begin{itemize}
	\item This, of course, is utter nonsense.
	\item ...this utter lack of responsibility.
	\item A look of utter confusion swept across his handsome face.
	\end{itemize}
}
\end{enumerate}

\section*{wool}
{\large \color{blue}  wools  }
\subsection*{Explain}
\begin{enumerate}
\item uncountable noun \\
\textbf{Wool} is the hair that grows on sheep and on some other animals.
 \textit{
	\begin{itemize}
	\end{itemize}
}
\item variable noun \\
\textbf{Wool} is a material made from animal's wool that is used to make things such as clothes , blankets , and carpets .
 \textit{
	\begin{itemize}
	\item ...a wool overcoat.
	\item The carpets are made in wool and nylon.
	\end{itemize}
}
\item  \\
 to pull the wool over someone's eyes \textit{
	\begin{itemize}
	\end{itemize}
}
\end{enumerate}

\section*{wear}
{\large \color{blue}  wears  wearing  wore  worn  }
\subsection*{Explain}
\begin{enumerate}
\item verb \\
When you \textbf{wear} something such as clothes, shoes , or jewellery , you have them on your body or on part of your body.
 \textit{
	\begin{itemize}
	\item He was wearing a brown uniform.
	\item I sometimes wear contact lenses.
	\item She can't make her mind up what to wear.
	\end{itemize}
}
\item verb \\
If you \textbf{wear} your hair or beard in a particular way, you have it cut or styled in that way.
 \textit{
	\begin{itemize}
	\item She wore her hair in a long braid.
	\item He wore a full moustache.
	\end{itemize}
}
\item verb \\
If you \textbf{wear} a particular expression , that expression is on your face and shows the emotions that you are feeling .
 \textit{
	\begin{itemize}
	\item When we drove through the gates, she wore a look of amazement.
	\item Millson's face wore a satisfied expression.
	\end{itemize}
}
\item uncountable noun \\
You use \textbf{wear} to refer to clothes that are suitable for a certain time or place. For example , \textbf{evening wear} is clothes suitable for the evening.
 \textit{
	\begin{itemize}
	\item The shop stocks an extensive range of beach wear.
	\item Bring informal casual wear.
	\end{itemize}
}
\item uncountable noun \\
\textbf{Wear} is the amount or type of use that something has over a period of time.
 \textit{
	\begin{itemize}
	\item You'll get more wear out of a hat if you choose one in a neutral colour.
	\item Rugs in the bedrooms got much less wear.
	\end{itemize}
}
\item uncountable noun \\
\textbf{Wear} is the damage or change that is caused by something being used a lot or for a long time.
 \textit{
	\begin{itemize}
	\item ...a large, well-upholstered armchair which showed signs of wear.
	\end{itemize}
}
\item verb \\
If something \textbf{wears} , it becomes thinner or weaker because it is constantly being used over a long period of time.
 \textit{
	\begin{itemize}
	\item The stone steps, dating back to 1855, are beginning to wear.
	\item Your horse needs new shoes if the shoe has worn thin or smooth.
	\end{itemize}
}
\item verb \\
You can use \textbf{wear} to talk about how well something lasts over a period of time. For example, if something \textbf{wears well} , it still  seems  quite new or useful after a long time or a lot of use.
 \textit{
	\begin{itemize}
	\item Casual shoes need to wear well.
	\item Ten years on, the original concept was wearing well.
	\end{itemize}
}
\item  \\
 to wear the trousers \textit{
	\begin{itemize}
	\end{itemize}
}
\item  \\
 wear thin \textit{
	\begin{itemize}
	\end{itemize}
}
\item  \\
 wear thin \textit{
	\begin{itemize}
	\end{itemize}
}
\item  \\
 the worse for wear \textit{
	\begin{itemize}
	\end{itemize}
}
\end{enumerate}

\section*{agriculture}
{\large \color{blue}  }
\subsection*{Explain}
\begin{enumerate}
\item uncountable noun \\
\textbf{Agriculture} is farming and the methods that are used to raise and look after crops and animals.
 \textit{
	\begin{itemize}
	\item The Ukraine is strong both in industry and agriculture.
	\end{itemize}
}
\end{enumerate}

\section*{acquire}
{\large \color{blue}  acquires  acquiring  acquired  }
\subsection*{Explain}
\begin{enumerate}
\item verb \\
If you \textbf{acquire} something, you buy or obtain it for yourself, or someone gives it to you.
 \textit{
	\begin{itemize}
	\item He yesterday revealed he had acquired a 2.98 per cent stake in the company.
	\item I recently acquired some wood from a holly tree.
	\item She was sitting in her newly-acquired wheelchair.
	\end{itemize}
}
\item verb \\
If you \textbf{acquire} something such as a skill or a habit , you learn it, or develop it through your daily life or experience .
 \textit{
	\begin{itemize}
	\item I've never acquired a taste for spicy food.
	\item Having read the book, she will be able to pass on the acquired knowledge to trainee
teachers.
	\end{itemize}
}
\item verb \\
If someone or something \textbf{acquires} a certain reputation , they start to have that reputation.
 \textit{
	\begin{itemize}
	\item He has acquired a reputation as this country's premier solo violinist.
	\end{itemize}
}
\item  \\
 acquired taste \textit{
	\begin{itemize}
	\end{itemize}
}
\end{enumerate}

\section*{alphabet}
{\large \color{blue}  alphabets  }
\subsection*{Explain}
\begin{enumerate}
\item countable noun \\
An \textbf{alphabet} is a set of letters usually presented in a fixed order which is used for writing the words of a particular language or
group of languages.
 \textit{
	\begin{itemize}
	\item The modern Russian alphabet has 31 letters.
	\item By two and a half he knew the alphabet.
	\end{itemize}
}
\end{enumerate}

\section*{balance}
{\large \color{blue}  balances  balancing  balanced  }
\subsection*{Explain}
\begin{enumerate}
\item verb \\
If you \textbf{balance} something somewhere , or if it \textbf{balances} there, it remains steady and does not fall.
 \textit{
	\begin{itemize}
	\item I balanced on the ledge.
	\item He balanced a football on his head.
	\end{itemize}
}
\item uncountable noun \\
\textbf{Balance} is the ability to remain steady when you are standing up.
 \textit{
	\begin{itemize}
	\item The medicines you are currently taking could be affecting your balance.
	\end{itemize}
}
\item verb \\
If you \textbf{balance} one thing \textbf{with} something different, each of the things has the same strength or importance.
 \textit{
	\begin{itemize}
	\item Balance spicy dishes with mild ones.
	\item The state has got to find some way to balance these two needs.
	\item Supply and demand on the currency market will generally balance.
	\end{itemize}
}
\item singular noun \\
A \textbf{balance} is a situation in which all the different parts are equal in strength or importance.
 \textit{
	\begin{itemize}
	\item Their marriage is a delicate balance between traditional and contemporary values.
	\item There was no other way to ensure that people would get the right balance of foods.
	\item ...the ecological balance of the forest.
	\end{itemize}
}
\item singular noun \\
If you say that \textbf{the}  \textbf{balance}  tips in your favour , you start winning or succeeding , especially in a conflict or contest .
 \textit{
	\begin{itemize}
	\item ...a powerful new gun which could tip the balance of the war in their favour.
	\item The balance continues to swing away from final examinations to continuous assessment.
	\end{itemize}
}
\item verb \\
If you \textbf{balance} one thing \textbf{against} another, you consider its importance in relation to the other one.
 \textit{
	\begin{itemize}
	\item She carefully tried to balance religious sensitivities against democratic freedom.
	\end{itemize}
}
\item verb \\
If someone \textbf{balances} their budget or if a government \textbf{balances} the economy of a country, they make sure that the amount of money that is spent is not greater than the amount that is received.
 \textit{
	\begin{itemize}
	\item He balanced his budgets by rigid control over public expenditure.
	\end{itemize}
}
\item verb \\
If you \textbf{balance} your books or make them \textbf{balance} , you prove by calculation that the amount of money you have received is equal to the amount that you have spent.
 \textit{
	\begin{itemize}
	\item ...teaching them to balance the books.
	\item To make the books balance, spending must fall and taxes must rise.
	\end{itemize}
}
\item countable noun \\
The \textbf{balance} in your bank account is the amount of money you have in it.
 \textit{
	\begin{itemize}
	\item I'd like to check the balance in my account please.
	\end{itemize}
}
\item singular noun \\
\textbf{The}  \textbf{balance} of an amount of money is what remains to be paid for something or what remains when
part of the amount has been spent.
 \textit{
	\begin{itemize}
	\item They were due to pay the balance on delivery.
	\end{itemize}
}
\item  \\
 in the balance \textit{
	\begin{itemize}
	\end{itemize}
}
\item  \\
 keep your balance \textit{
	\begin{itemize}
	\end{itemize}
}
\item  \\
 off balance \textit{
	\begin{itemize}
	\end{itemize}
}
\item  \\
 off balance \textit{
	\begin{itemize}
	\end{itemize}
}
\item  \\
 on balance \textit{
	\begin{itemize}
	\end{itemize}
}
\end{enumerate}

\section*{army}
{\large \color{blue}  armies  }
\subsection*{Explain}
\begin{enumerate}
\item countable noun \\
An \textbf{army} is a large organized group of people who are armed and trained to fight on land in a war. Most armies are organized and controlled by governments.
 \textit{
	\begin{itemize}
	\item After returning from France, he joined the army.
	\item The army is about to launch a major offensive.
	\item ...a top-ranking army officer.
	\end{itemize}
}
\item countable noun \\
An \textbf{army}  \textbf{of} people, animals, or things is a large number of them, especially when they are regarded as a force of some kind .
 \textit{
	\begin{itemize}
	\item ...data collected by an army of volunteers.
	\item ...armies of shoppers looking for bargains.
	\item ...the army of television cameras outside his house.
	\end{itemize}
}
\end{enumerate}

\section*{bind}
{\large \color{blue}  binds  binding  bound  }
\subsection*{Explain}
\begin{enumerate}
\item verb \\
If something \textbf{binds} people \textbf{together} , it makes them feel as if they are all part of the same group or have something in common.
 \textit{
	\begin{itemize}
	\item It is the memory and threat of persecution that binds them together.
	\item ...the social and political ties that bind the U.S.A. to Britain.
	\item ...a group of people bound together by shared language, culture, and beliefs.
	\end{itemize}
}
\item verb \\
If you \textbf{are bound} by something such as a rule, agreement, or restriction , you are forced or required to act in a certain way.
 \textit{
	\begin{itemize}
	\item Employers are not bound by law to conduct equal pay reviews.
	\item The authorities will be legally bound to arrest any suspects.
	\item There is a bottom deck though, so you're not bound to sit on top.
	\item The treaty binds them to respect their neighbour's independence.
	\end{itemize}
}
\item verb \\
If you \textbf{bind} something or someone, you tie rope , string , tape , or other material around them so that they are held firmly.
 \textit{
	\begin{itemize}
	\item Bind the ends of the cord together with thread.
	\item ...the red tape which was used to bind the files.
	\item He said there were cases where prisoners were tightly bound, often for several days.
	\end{itemize}
}
\item verb \\
When a book \textbf{is bound} , the pages are joined together and the cover is put on.
 \textit{
	\begin{itemize}
	\item Each volume is bound in bright-coloured cloth.
	\item Their business came from a few big publishers, all of whose books they bound.
	\item ...four immaculately bound hardbacks.
	\end{itemize}
}
\item ergative verb \\
If one chemical or particle \textbf{is bound} to another, it becomes attached to it or reacts with it to form a single particle or substance.
 \textit{
	\begin{itemize}
	\item Nobody understands why these three quarks in the proton are bound together.
	\item These may bind to receptor molecules on the surfaces of cells.
	\item These compounds bind with genetic material in the liver.
	\end{itemize}
}
\item verb \\
In cookery , if you \textbf{bind} a mixture of food, you form it into a mass by mixing it with a sticky substance.
 \textit{
	\begin{itemize}
	\item Bind the mixture with the raw minced liver and cook for 3 minutes more.
	\item ...a divine mixture of vegetarian cheeses bound with egg.
	\end{itemize}
}
\item singular noun \\
If you are in \textbf{a bind} , you are in a difficult situation, usually because you have to make a decision or a choice and whatever decision or choice you make will have unpleasant  consequences .
 \textit{
	\begin{itemize}
	\item This puts the politicians in a bind as to what course to take.
	\item I'll advance you the money for it, here and now, just to help you out of a bind.
	\end{itemize}
}
\item singular noun \\
If you say that something is \textbf{a bind} , you mean that it is unpleasant and boring to do.
 \textit{
	\begin{itemize}
	\item It is expensive to buy and a bind to carry home.
	\end{itemize}
}
\end{enumerate}

\section*{balcony}
{\large \color{blue}  balconies  }
\subsection*{Explain}
\begin{enumerate}
\item countable noun \\
A \textbf{balcony} is a platform on the outside of a building, above ground  level , with a wall or railing around it.
 \textit{
	\begin{itemize}
	\end{itemize}
}
\item singular noun \\
\textbf{The}  \textbf{balcony} in a theatre or cinema is an area of seats above the main seating area.
 \textit{
	\begin{itemize}
	\end{itemize}
}
\end{enumerate}

\section*{blackmail}
{\large \color{blue}  blackmails  blackmailing  blackmailed  }
\subsection*{Explain}
\begin{enumerate}
\item uncountable noun \\
\textbf{Blackmail} is the action of threatening to reveal a secret about someone, unless they do something you tell them to do, such as giving you money.
 \textit{
	\begin{itemize}
	\item It looks like the pictures were being used for blackmail.
	\item Opponents accused him of blackmail and extortion.
	\end{itemize}
}
\item uncountable noun \\
If you describe an action as emotional or moral  \textbf{blackmail} , you disapprove of it because someone is using a person's emotions or moral values to persuade them to do something against their will .
 \textit{
	\begin{itemize}
	\item The tactics employed can range from overt bullying to subtle emotional blackmail.
	\end{itemize}
}
\item verb \\
If one person \textbf{blackmails} another person, they use blackmail against them.
 \textit{
	\begin{itemize}
	\item He alleged that she was blackmailing him.
	\item The government insisted that it would not be blackmailed by violence.
	\item I thought he was trying to blackmail me into saying whatever he wanted.
	\end{itemize}
}
\end{enumerate}

\section*{bury}
{\large \color{blue}  buries  burying  buried  }
\subsection*{Explain}
\begin{enumerate}
\item verb \\
To \textbf{bury} something means to put it into a hole in the ground and cover it up with earth.
 \textit{
	\begin{itemize}
	\item They make the charcoal by burying wood in the ground and then slowly burning it.
	\item ...squirrels who bury nuts and seeds.
	\item ...buried treasure.
	\end{itemize}
}
\item verb \\
To \textbf{bury} a dead person means to put their body into a grave and cover it with earth.
 \textit{
	\begin{itemize}
	\item They buried the dead in a communal grave.
	\item I was horrified that people would think I was dead and bury me alive.
	\item More than 9,000 men lie buried here.
	\end{itemize}
}
\item verb \\
If someone says they \textbf{have buried} one of their relatives , they mean that one of their relatives has died .
 \textit{
	\begin{itemize}
	\item He had buried his wife some two years before he retired.
	\end{itemize}
}
\item verb \\
If you \textbf{bury} something under a large quantity of things, you put it there, often in order to hide it.
 \textit{
	\begin{itemize}
	\item She buried it under some leaves.
	\item I was looking for my handbag, which was buried under a pile of old newspapers.
	\end{itemize}
}
\item verb \\
If something \textbf{buries} a place or person, it falls on top of them so that it completely covers them and often harms them in some way .
 \textit{
	\begin{itemize}
	\item Latest reports say that mud slides buried entire villages.
	\item Their house was buried by a landslide.
	\item He was buried under the debris for several hours.
	\end{itemize}
}
\item verb \\
If you \textbf{bury} your head or face in something, you press your head or face against it, often because you are unhappy .
 \textit{
	\begin{itemize}
	\item She buried her face in the pillows.
	\item He held her closely, burying his head against her shoulder.
	\end{itemize}
}
\item verb \\
If something \textbf{buries}  \textbf{itself}  somewhere , or if you \textbf{bury} it there, it is pushed very deeply in there.
 \textit{
	\begin{itemize}
	\item The missile buried itself deep in the grassy hillside.
	\item He stood on the sidewalk with his hands buried in the pockets of his dark overcoat.
	\end{itemize}
}
\item verb \\
If you \textbf{bury} a feeling , you try not to show it. If you \textbf{bury} a memory , you try to forget it.
 \textit{
	\begin{itemize}
	\item When we feel anger, we bury the emotion and feel guilty instead.
	\item It is time to bury our past misunderstandings.
	\item ...deeply-buried memories.
	\end{itemize}
}
\item verb \\
If you \textbf{bury}  \textbf{yourself}  \textbf{in} a place or in an activity such as your work, you spend all your time in that place or doing that activity, usually because you want to forget about things.
 \textit{
	\begin{itemize}
	\item His reaction was to withdraw, to bury himself in work.
	\item ...the popular image of writers burying themselves in the country in order to write.
	\end{itemize}
}
\item verb \\
If you \textbf{bury} your head \textbf{in} something such as a book or newspaper , or \textbf{bury}  \textbf{yourself in} it, you look at it closely and concentrate very hard on it.
 \textit{
	\begin{itemize}
	\item My father buried his head in his newspaper.
	\item He buried himself in his detective story again.
	\end{itemize}
}
\end{enumerate}

\section*{cashier}
{\large \color{blue}  cashiers  cashiering  cashiered  }
\subsection*{Explain}
\begin{enumerate}
\item countable noun \\
A \textbf{cashier} is a person who customers pay money to or get money from in places such as shops or banks .
 \textit{
	\begin{itemize}
	\end{itemize}
}
\item verb \\
If a person in the armed forces \textbf{is cashiered} , he or she is forced to leave because they have done something seriously  wrong .
 \textit{
	\begin{itemize}
	\item The government had to recall many officers who had been cashiered on political grounds.
	\item ...a cashiered army colonel.
	\end{itemize}
}
\end{enumerate}

\section*{catch}
{\large \color{blue}  catches  catching  caught  }
\subsection*{Explain}
\begin{enumerate}
\item verb \\
If you \textbf{catch} a person or animal, you capture them after chasing them, or by using a trap , net , or other device.
 \textit{
	\begin{itemize}
	\item Police say they are confident of catching the gunman.
	\item Where did you catch the fish?
	\item I wondered if it was an animal caught in a trap.
	\end{itemize}
}
\item verb \\
If you \textbf{catch} an object that is moving through the air, you seize it with your hands .
 \textbf{Catch} is also a noun.
 \textit{
	\begin{itemize}
	\item I jumped up to catch a ball and fell over.
	\item He missed the catch and the match was lost.
	\end{itemize}
}
\item verb \\
If you \textbf{catch} a part of someone's body, you take or seize it with your hand, often in order to
stop them going  somewhere .
 \textit{
	\begin{itemize}
	\item Liz caught his arm.
	\item He knelt beside her and caught her hand in both of his.
	\item Garrido caught her by the wrist.
	\end{itemize}
}
\item verb \\
If one thing \textbf{catches} another, it hits it accidentally or manages to hit it.
 \textit{
	\begin{itemize}
	\item The stinging slap almost caught his face.
	\item I may have caught him with my elbow but it was just an accident.
	\item He caught her on the side of her head with his other fist.
	\end{itemize}
}
\item verb \\
If something \textbf{catches}  \textbf{on} or \textbf{in} an object or if an object \textbf{catches} something, it accidentally becomes attached to the object or stuck in it.
 \textit{
	\begin{itemize}
	\item Her ankle caught on a root, and she almost lost her balance.
	\item A man caught his foot in the lawnmower.
	\end{itemize}
}
\item verb \\
When you \textbf{catch} a bus , train, or plane, you get on it in order to travel somewhere.
 \textit{
	\begin{itemize}
	\item We were in plenty of time for Anthony to catch the ferry.
	\item He caught a taxi to Harrods.
	\end{itemize}
}
\item verb \\
If you \textbf{catch} someone doing something wrong , you see or find them doing it.
 \textit{
	\begin{itemize}
	\item He caught a youth breaking into a car.
	\item I don't want to catch you pushing yourself into the picture to get some personal
publicity.
	\item They caught him on camera doing it more than once.
	\end{itemize}
}
\item verb \\
If you \textbf{catch}  \textbf{yourself} doing something, especially something surprising, you suddenly become aware that you are doing it.
 \textit{
	\begin{itemize}
	\item I caught myself feeling almost sorry for poor Mr Laurence.
	\end{itemize}
}
\item verb \\
If you \textbf{catch} something or \textbf{catch} a glimpse of it, you notice it or manage to see it briefly.
 \textit{
	\begin{itemize}
	\item As she turned back, she caught the puzzled look on her mother's face.
	\item He caught a glimpse of the man's face in a shop window.
	\end{itemize}
}
\item verb \\
If you \textbf{catch} something that someone has said , you manage to hear it.
 \textit{
	\begin{itemize}
	\item His ears caught a faint cry.
	\item I do not believe I caught your name.
	\item The men out in the corridor were trying to catch what they said.
	\end{itemize}
}
\item verb \\
If you \textbf{catch} a TV or radio programme or an event, you manage to see or listen to it.
 \textit{
	\begin{itemize}
	\item Bill turns on the radio to catch the local news.
	\item The exhibition is on at Droitwich until May 24. You can also catch it at Leominster
from June 5.
	\end{itemize}
}
\item verb \\
If you \textbf{catch} someone, you manage to contact or meet them to talk to them, especially when they are just about to go somewhere else.
 \textit{
	\begin{itemize}
	\item I dialled Elizabeth's number thinking I might catch her before she left for work.
	\item Hello, Dolph. Glad I caught you.
	\end{itemize}
}
\item verb \\
If something or someone \textbf{catches} you by surprise or at a bad time, you were not expecting them or do not feel able to deal with them.
 \textit{
	\begin{itemize}
	\item She looked as if the photographer had caught her by surprise.
	\item I'm sorry but I just cannot say anything. You've caught me at a bad time.
	\item The sheer number of spectators has caught everyone unprepared.
	\end{itemize}
}
\item verb \\
If something \textbf{catches} your attention or your eye, you notice it or become interested in it.
 \textit{
	\begin{itemize}
	\item My shoes caught his attention.
	\item A quick movement across the aisle caught his eye.
	\end{itemize}
}
\item verb \\
If someone or something \textbf{catches} a mood or an atmosphere , they successfully represent it or reflect it.
 \textit{
	\begin{itemize}
	\end{itemize}
}
\item passive verb \\
If you \textbf{are caught} in a storm or other unpleasant situation, it happens when you cannot avoid its effects.
 \textit{
	\begin{itemize}
	\item When he was fishing off the island he was caught in a storm and almost drowned.
	\item Visitors to the area were caught between police and the rioters.
	\end{itemize}
}
\item passive verb \\
If you \textbf{are caught between} two alternatives or two people, you do not know which one to choose or follow.
 \textit{
	\begin{itemize}
	\item The Jordanian leader is caught between both sides in the dispute.
	\item She was caught between envy and admiration.
	\end{itemize}
}
\item verb \\
If you \textbf{catch} a cold or a disease, you become ill with it.
 \textit{
	\begin{itemize}
	\item The more stress you are under, the more likely you are to catch a cold.
	\end{itemize}
}
\item verb \\
To \textbf{catch} liquids or small pieces that fall from somewhere means to collect them in a container.
 \textit{
	\begin{itemize}
	\item The fish is laid out on a large serving plate to catch the juices.
	\item ...a specially designed breadboard with a tray to catch the crumbs.
	\end{itemize}
}
\item verb \\
If something \textbf{catches} the light or if the light \textbf{catches} it, it reflects the light and looks bright or shiny .
 \textit{
	\begin{itemize}
	\item They saw the ship's guns, catching the light of the moon.
	\item Often a fox goes across the road in front of me and I just catch it in the headlights.
	\end{itemize}
}
\item verb \\
If the wind or water \textbf{catches} something, it carries or pushes it along.
 \textit{
	\begin{itemize}
	\item A gust of wind caught the parachute.
	\end{itemize}
}
\item countable noun \\
A \textbf{catch} on a window, door, or container is a device that fastens it.
 \textit{
	\begin{itemize}
	\item She fiddled with the catch of her bag.
	\item Fit windows with safety locks or catches.
	\end{itemize}
}
\item countable noun \\
A \textbf{catch} is a hidden problem or difficulty in a plan or an offer that seems surprisingly good.
 \textit{
	\begin{itemize}
	\item The catch is that you work for your supper, and the food and accommodation can be
very basic.
	\item 'It's your money. You deserve it.'—'What's the catch?'
	\end{itemize}
}
\item countable noun \\
When people have been fishing, their \textbf{catch} is the total number of fish that they have caught.
 \textit{
	\begin{itemize}
	\item The catch included one fish over 18 pounds.
	\end{itemize}
}
\item singular noun \\
If you describe someone as a good \textbf{catch} , you mean that they have lots of good qualities and you think their partner or employer is very lucky to have found them.
 \textit{
	\begin{itemize}
	\item I was so in love with him and all my friends said what a good catch he was.
	\end{itemize}
}
\item uncountable noun \\
\textbf{Catch} is a game in which children throw a ball to each other.
 \textit{
	\begin{itemize}
	\end{itemize}
}
\item uncountable noun \\
\textbf{Catch} is a game in which one child chases other children and tries to touch or catch one
of them.
 \textit{
	\begin{itemize}
	\end{itemize}
}
\item  \\
 you wouldn't/won't catch me \textit{
	\begin{itemize}
	\end{itemize}
}
\item  \\
 to catch sb with their trousers/pants down \textit{
	\begin{itemize}
	\end{itemize}
}
\end{enumerate}

\section*{collection}
{\large \color{blue}  collections  }
\subsection*{Explain}
\begin{enumerate}
\item countable noun \\
A \textbf{collection}  \textbf{of} things is a group of similar things that you have deliberately acquired , usually over a period of time.
 \textit{
	\begin{itemize}
	\item Robert's collection of prints and paintings has been bought over the years.
	\item The Art Gallery of Ontario has the world's largest collection of sculptures by Henry
Moore.
	\item He made the mistake of leaving his valuable record collection with a former girlfriend.
	\end{itemize}
}
\item countable noun \\
A \textbf{collection}  \textbf{of}  stories , poems , or articles is a number of them published in one book.
 \textit{
	\begin{itemize}
	\item He published a collection of short stories called 'Facing The Music'.
	\item The institute has assembled a collection of essays from foreign affairs experts.
	\end{itemize}
}
\item countable noun \\
A \textbf{collection}  \textbf{of} things is a group of things.
 \textit{
	\begin{itemize}
	\item Wye Lea is a collection of farm buildings that have been converted into an attractive
complex.
	\end{itemize}
}
\item countable noun \\
A fashion designer's new \textbf{collection} consists of the new clothes they have designed for the next season.
 \textit{
	\begin{itemize}
	\end{itemize}
}
\item uncountable noun \\
\textbf{Collection} is the act of collecting something from a place or from people.
 \textit{
	\begin{itemize}
	\item Money can be sent to any one of 22,000 agents worldwide for collection.
	\item ...computer systems to speed up collection of information.
	\item ...public services including mail delivery and garbage collection.
	\end{itemize}
}
\item countable noun \\
If you organize a \textbf{collection} for charity , you collect money from people to give to charity.
 \textit{
	\begin{itemize}
	\item I asked my headmaster if he could arrange a collection for a refugee charity.
	\end{itemize}
}
\item countable noun \\
A \textbf{collection} is money that is given by people in church during some Christian services.
 \textit{
	\begin{itemize}
	\end{itemize}
}
\end{enumerate}

\section*{charge}
{\large \color{blue}  charges  charging  charged  }
\subsection*{Explain}
\begin{enumerate}
\item verb \\
If you \textbf{charge} someone an amount of money, you ask them to pay that amount for something that you have sold to them or done for them.
 \textit{
	\begin{itemize}
	\item Even local nurseries charge £100 a week.
	\item The majority of stalls charged a fair price.
	\item The hospitals charge the patients for every aspirin.
	\item Some banks charge if you access your account to determine your balance.
	\item ...the architect who charged us a fee of seven hundred and fifty pounds.
	\end{itemize}
}
\item verb \\
To \textbf{charge} something \textbf{to} a person or organization means to tell the people providing it to send the bill to that person or organization. To \textbf{charge} something \textbf{to} someone's account means to add it to their account so they can pay for it later.
 \textit{
	\begin{itemize}
	\item Go out and buy a pair of glasses, and charge it to us.
	\item All transactions have been charged to your account.
	\end{itemize}
}
\item countable noun \\
A \textbf{charge} is an amount of money that you have to pay for a service.
 \textit{
	\begin{itemize}
	\item We can arrange this for a small charge.
	\item Customers who arrange overdrafts will face a monthly charge of £5.
	\end{itemize}
}
\item countable noun \\
A \textbf{charge} is a formal accusation that someone has committed a crime.
 \textit{
	\begin{itemize}
	\item He may still face criminal charges.
	\item They appeared at court yesterday to deny charges of murder.
	\end{itemize}
}
\item verb \\
When the police \textbf{charge} someone, they formally accuse them of having done something illegal .
 \textit{
	\begin{itemize}
	\item They have the evidence to charge him.
	\item Police have charged Mr Bell with murder.
	\end{itemize}
}
\item verb \\
If you \textbf{charge} someone \textbf{with} doing something wrong or unpleasant , you publicly  say that they have done it.
 \textit{
	\begin{itemize}
	\item He charged the minister with lying about the economy.
	\end{itemize}
}
\item uncountable noun \\
If you take \textbf{charge}  \textbf{of} someone or something, you make yourself responsible for them and take control over
them. If someone or something is \textbf{in} your \textbf{charge} , you are responsible for them.
 \textit{
	\begin{itemize}
	\item A few years ago Bacryl took charge of the company.
	\item I have been given charge of this class.
	\item They would never forget their time in his charge.
	\end{itemize}
}
\item  \\
 in charge \textit{
	\begin{itemize}
	\end{itemize}
}
\item countable noun \\
If you describe someone as your \textbf{charge} , they have been given to you to be looked after and you are responsible for them.
 \textit{
	\begin{itemize}
	\item The coach tried to get his charges motivated.
	\end{itemize}
}
\item verb \\
If you \textbf{charge} towards someone or something, you move quickly and aggressively towards them.
 \textbf{Charge} is also a noun.
 \textit{
	\begin{itemize}
	\item He charged through the door to my mother's office.
	\item He ordered us to charge.
	\item ...a charging bull.
	\item ...a bayonet charge.
	\end{itemize}
}
\item verb \\
To \textbf{charge} a battery means to pass an electrical current through it in order to make it more powerful or to make it last longer.
 \textbf{Charge up} means the same as charge .
 \textit{
	\begin{itemize}
	\item Alex had forgotten to charge the battery.
	\item The car recovers energy to charge up the batteries while driving.
	\end{itemize}
}
\item countable noun \\
An electrical \textbf{charge} is an amount of electricity that is held in or carried by something.
 \textit{
	\begin{itemize}
	\end{itemize}
}
\item countable noun \\
The \textbf{charge} in a cartridge or shell is the explosive inside it. You can also refer to the cartridge or shell itself as a charge .
 \textit{
	\begin{itemize}
	\end{itemize}
}
\item  \\
 free of charge \textit{
	\begin{itemize}
	\end{itemize}
}
\end{enumerate}

\section*{conspiracy}
{\large \color{blue}  conspiracies  }
\subsection*{Explain}
\begin{enumerate}
\item variable noun \\
\textbf{Conspiracy} is the secret planning by a group of people to do something illegal.
 \textit{
	\begin{itemize}
	\item Seven men, all from Bristol, admitted conspiracy to commit arson.
	\item He believes there was a conspiracy to kill the president.
	\end{itemize}
}
\item countable noun \\
A \textbf{conspiracy} is an agreement between a group of people which other people think is wrong or is likely to be harmful.
 \textit{
	\begin{itemize}
	\item It's all part of a conspiracy to move everything out of the town centre.
	\item He persuaded himself that they had formed some kind of conspiracy against him.
	\end{itemize}
}
\item  \\
 conspiracy of silence \textit{
	\begin{itemize}
	\end{itemize}
}
\end{enumerate}

\section*{charter}
{\large \color{blue}  charters  chartering  chartered  }
\subsection*{Explain}
\begin{enumerate}
\item countable noun \\
A \textbf{charter} is a formal document describing the rights, aims , or principles of an organization or group of people.
 \textit{
	\begin{itemize}
	\item ...Article 50 of the United Nations Charter.
	\item ...the Social Charter of workers' rights.
	\end{itemize}
}
\item adjective \\
A \textbf{charter}  plane or boat is one which is hired for use by a particular person or group and which is not part
of a regular service.
 \textit{
	\begin{itemize}
	\item ...the last charter plane carrying out foreign nationals.
	\item ...frequent charter flights to Spain.
	\end{itemize}
}
\item verb \\
If a person or organization \textbf{charters} a plane, boat, or other vehicle, they hire it for their own use.
 \textit{
	\begin{itemize}
	\item He chartered a jet to fly her home from California to Switzerland.
	\item They arrived in a yacht chartered by the sports management company.
	\end{itemize}
}
\item  \\
 a charter for sth \textit{
	\begin{itemize}
	\end{itemize}
}
\end{enumerate}

\section*{curiosity}
{\large \color{blue}  curiosities  }
\subsection*{Explain}
\begin{enumerate}
\item uncountable noun \\
\textbf{Curiosity} is a desire to know about something.
 \textit{
	\begin{itemize}
	\item Ryle accepted more out of curiosity than anything else.
	\item ...an enthusiasm and genuine curiosity about the past.
	\item To satisfy our own curiosity we traveled to Baltimore.
	\end{itemize}
}
\item countable noun \\
A \textbf{curiosity} is something that is unusual , interesting , and fairly rare.
 \textit{
	\begin{itemize}
	\item There is much to see in the way of castles, curiosities, and museums.
	\end{itemize}
}
\end{enumerate}

\section*{commend}
{\large \color{blue}  commends  commending  commended  }
\subsection*{Explain}
\begin{enumerate}
\item verb \\
If you \textbf{commend} someone or something, you praise them formally.
 \textit{
	\begin{itemize}
	\item I commended her for that action.
	\item I commend Ms. Orth on writing such an informative article.
	\item The book was widely commended for its candour.
	\item The reports commend her bravery.
	\item His actions were commended by the Jury.
	\end{itemize}
}
\item verb \\
If someone \textbf{commends} a person or thing \textbf{to} you, they tell you that you will  find them good or useful .
 \textit{
	\begin{itemize}
	\item I can commend it to him as a realistic course of action.
	\end{itemize}
}
\item verb \\
If something \textbf{commends}  \textbf{itself}  \textbf{to} you, you approve of it.
 \textit{
	\begin{itemize}
	\item This was not a view that commended itself to all.
	\end{itemize}
}
\item  \\
 much/little to commend it \textit{
	\begin{itemize}
	\end{itemize}
}
\end{enumerate}

\section*{discourse}
{\large \color{blue}  discourses  discoursing  discoursed  }
\subsection*{Explain}
\begin{enumerate}
\item uncountable noun \\
\textbf{Discourse} is spoken or written communication between people, especially  serious discussion of a particular subject.
 \textit{
	\begin{itemize}
	\item ...a tradition of political discourse.
	\end{itemize}
}
\item uncountable noun \\
In linguistics , \textbf{discourse} is natural spoken or written language in context , especially when complete texts are being considered .
 \textit{
	\begin{itemize}
	\item The Centre has a strong record of research in discourse analysis.
	\item ...our work on discourse and the way people talk to each other.
	\end{itemize}
}
\item countable noun \\
A \textbf{discourse} is a serious talk or piece of writing which is intended to teach or explain something.
 \textit{
	\begin{itemize}
	\item Hastings responds with a lengthy discourse on marketing strategies.
	\end{itemize}
}
\item verb \\
If someone \textbf{discourses}  \textbf{on} something, they talk for a long time about it in a confident way.
 \textit{
	\begin{itemize}
	\item He discoursed for several hours on French and English prose.
	\end{itemize}
}
\end{enumerate}

\section*{commute}
{\large \color{blue}  commutes  commuting  commuted  }
\subsection*{Explain}
\begin{enumerate}
\item verb \\
If you \textbf{commute} , you travel a long distance every day between your home and your place of work.
 \textit{
	\begin{itemize}
	\item Mike commutes to London every day.
	\item McLaren began commuting between Paris and London.
	\item He's going to commute.
	\end{itemize}
}
\item countable noun \\
A \textbf{commute} is the journey that you make when you commute.
 \textit{
	\begin{itemize}
	\item The average Los Angeles commute is over 60 miles a day.
	\end{itemize}
}
\item verb \\
If a death sentence or prison sentence \textbf{is commuted}  \textbf{to} a less serious  punishment , it is changed to that punishment.
 \textit{
	\begin{itemize}
	\item His death sentence was commuted to life imprisonment.
	\item Prison sentences have been commuted.
	\end{itemize}
}
\end{enumerate}

\section*{equipment}
{\large \color{blue}  }
\subsection*{Explain}
\begin{enumerate}
\item uncountable noun \\
\textbf{Equipment} consists of the things which are used for a particular purpose, for example a hobby or job .
 \textit{
	\begin{itemize}
	\item ...computers, electronic equipment and machine tools.
	\item ...outdoor playing equipment.
	\end{itemize}
}
\end{enumerate}

\section*{corrupt}
{\large \color{blue}  corrupts  corrupting  corrupted  }
\subsection*{Explain}
\begin{enumerate}
\item adjective \\
Someone who is \textbf{corrupt}  behaves in a way that is morally wrong , especially by doing dishonest or illegal things in return for money or power.
 \textit{
	\begin{itemize}
	\item ...to save the nation from corrupt politicians of both parties.
	\item He had accused three opposition members of corrupt practices.
	\end{itemize}
}
\item verb \\
If someone \textbf{is corrupted}  \textbf{by} something, it causes them to become dishonest and unjust and unable to be trusted .
 \textit{
	\begin{itemize}
	\item It is sad to see a man so corrupted by the desire for money and power.
	\end{itemize}
}
\item verb \\
To \textbf{corrupt} someone means to cause them to stop  caring about moral  standards .
 \textit{
	\begin{itemize}
	\item ...warning that television will corrupt us all.
	\item Cruelty depraves and corrupts.
	\end{itemize}
}
\item verb \\
If something \textbf{is corrupted} , it becomes damaged or spoiled in some way.
 \textit{
	\begin{itemize}
	\item Some of the finer type-faces are corrupted by cheap, popular computer printers.
	\item They can ensure that traditional cuisines are not totally corrupted by commercial
practices.
	\item ...corrupted data.
	\end{itemize}
}
\end{enumerate}

\section*{farm}
{\large \color{blue}  farms  farming  farmed  }
\subsection*{Explain}
\begin{enumerate}
\item countable noun \\
A \textbf{farm} is an area of land, together with the buildings on it, that is used for growing crops or raising animals, usually in order to sell them.
 \textit{
	\begin{itemize}
	\item Farms in France are much smaller than those in the United States or even Britain.
	\end{itemize}
}
\item verb \\
If you \textbf{farm} an area of land, you grow crops or keep animals on it.
 \textit{
	\begin{itemize}
	\item They farmed some of the best land in Scotland.
	\item He has lived and farmed in the area for 46 years.
	\end{itemize}
}
\item countable noun \\
A mink  \textbf{farm} or a fish \textbf{farm} , for example , is a place where a particular kind of animal or fish is bred and kept in large quantities in order to be sold.
 \textit{
	\begin{itemize}
	\item ...trout fresh from a local trout farm.
	\end{itemize}
}
\end{enumerate}

\section*{cultivate}
{\large \color{blue}  cultivates  cultivating  cultivated  }
\subsection*{Explain}
\begin{enumerate}
\item verb \\
If you \textbf{cultivate} land or crops, you prepare land and grow crops on it.
 \textit{
	\begin{itemize}
	\item She also cultivated a small garden of her own.
	\item ...the few patches of cultivated land.
	\end{itemize}
}
\item verb \\
If you \textbf{cultivate} an attitude , image , or skill, you try  hard to develop it and make it stronger or better .
 \textit{
	\begin{itemize}
	\item He has written eight books and has cultivated the image of an elder statesman.
	\item Cultivating a positive mental attitude towards yourself can reap tremendous benefits.
	\end{itemize}
}
\item verb \\
If you \textbf{cultivate} someone or \textbf{cultivate} a friendship with them, you try hard to develop a friendship with them.
 \textit{
	\begin{itemize}
	\item He may be monarch one day so we must cultivate him.
	\item The President has carefully cultivated relationships with at least five influential
ministers.
	\end{itemize}
}
\end{enumerate}

\section*{farmer}
{\large \color{blue}  farmers  }
\subsection*{Explain}
\begin{enumerate}
\item countable noun \\
A \textbf{farmer} is a person who owns or manages a farm.
 \textit{
	\begin{itemize}
	\end{itemize}
}
\end{enumerate}

\section*{define}
{\large \color{blue}  defines  defining  defined  }
\subsection*{Explain}
\begin{enumerate}
\item verb \\
If you \textbf{define} something, you show , describe, or state clearly what it is and what its limits are, or what it is like.
 \textit{
	\begin{itemize}
	\item We were unable to define what exactly was wrong with him.
	\item He was asked to define his concept of cool.
	\end{itemize}
}
\item verb \\
If you \textbf{define} a word or expression , you explain its meaning, for example in a dictionary .
 \textit{
	\begin{itemize}
	\item I don't know how to define intelligence, but I can recognize it when I see it.
	\item Collins English Dictionary defines a workaholic as 'a person obsessively addicted
to work'.
	\end{itemize}
}
\end{enumerate}

\section*{feather}
{\large \color{blue}  feathers  }
\subsection*{Explain}
\begin{enumerate}
\item countable noun \\
A bird's \textbf{feathers} are the soft covering on its body. Each \textbf{feather} consists of a lot of smooth hairs on each side of a thin stiff centre.
 \textit{
	\begin{itemize}
	\item ...a hat that she had made herself from black ostrich feathers.
	\item ...a feather bed.
	\end{itemize}
}
\item  \\
 birds of a feather \textit{
	\begin{itemize}
	\end{itemize}
}
\item  \\
 feather in one's cap \textit{
	\begin{itemize}
	\end{itemize}
}
\end{enumerate}

\section*{destroy}
{\large \color{blue}  destroys  destroying  destroyed  }
\subsection*{Explain}
\begin{enumerate}
\item verb \\
To \textbf{destroy} something means to cause so much damage to it that it is completely ruined or does not exist any more.
 \textit{
	\begin{itemize}
	\item That's a sure recipe for destroying the economy and creating chaos.
	\item No one was injured in the explosion, but the building was completely destroyed.
	\item Even the most gifted can have confidence destroyed by the wrong instructor.
	\end{itemize}
}
\item verb \\
To \textbf{destroy} someone means to ruin their life or to make their situation  impossible to bear.
 \textit{
	\begin{itemize}
	\item If I was younger or more naive, the criticism would have destroyed me.
	\end{itemize}
}
\item verb \\
If an animal \textbf{is destroyed} , it is killed, either because it is ill or because it is dangerous .
 \textit{
	\begin{itemize}
	\item Lindsay was unhurt but the horse had to be destroyed.
	\end{itemize}
}
\end{enumerate}

\section*{forum}
{\large \color{blue}  forums  }
\subsection*{Explain}
\begin{enumerate}
\item countable noun \\
A \textbf{forum} is a place, situation , or group in which people exchange  ideas and discuss  issues , especially  important public issues.
 \textit{
	\begin{itemize}
	\item Members of the council agreed that it still had an important role as a forum for
discussion.
	\item The organisation would provide a forum where problems could be discussed.
	\end{itemize}
}
\item countable noun \\
In ancient Roman towns, the \textbf{forum} was a square where people met to discuss business and political  matters .
 \textit{
	\begin{itemize}
	\end{itemize}
}
\end{enumerate}

\section*{detain}
{\large \color{blue}  detains  detaining  detained  }
\subsection*{Explain}
\begin{enumerate}
\item verb \\
When people such as the police  \textbf{detain} someone, they keep them in a place under their control.
 \textit{
	\begin{itemize}
	\item The act allows police to detain a suspect for up to 48 hours.
	\item He was arrested and detained for questioning.
	\end{itemize}
}
\item verb \\
To \textbf{detain} someone means to delay them, for example by talking to them.
 \textit{
	\begin{itemize}
	\item Thank you. We won't detain you any further.
	\end{itemize}
}
\end{enumerate}

\section*{harvest}
{\large \color{blue}  harvests  harvesting  harvested  }
\subsection*{Explain}
\begin{enumerate}
\item singular noun \\
\textbf{The harvest} is the gathering of a crop.
 \textit{
	\begin{itemize}
	\item There were about 300 million tons of grain in the fields at the start of the harvest.
	\end{itemize}
}
\item countable noun \\
A \textbf{harvest} is the crop that is gathered in.
 \textit{
	\begin{itemize}
	\item ...a bumper potato harvest.
	\item Millions of people are threatened with starvation as a result of drought and poor
harvests.
	\end{itemize}
}
\item verb \\
When you \textbf{harvest} a crop, you gather it in.
 \textit{
	\begin{itemize}
	\item Many farmers are refusing to harvest the cane.
	\item ...freshly harvested beetroot.
	\end{itemize}
}
\item verb \\
If you \textbf{harvest} a large number of things, you collect them, often by making great efforts.
 \textit{
	\begin{itemize}
	\item In his new career as a restaurateur he has blossomed and harvested many awards.
	\end{itemize}
}
\item  \\
 reap the harvest \textit{
	\begin{itemize}
	\end{itemize}
}
\end{enumerate}

\section*{entitle}
{\large \color{blue}  entitles  entitling  entitled  }
\subsection*{Explain}
\begin{enumerate}
\item verb \\
If you \textbf{are entitled}  \textbf{to} something, you have the right to have it or do it.
 \textit{
	\begin{itemize}
	\item If the warranty is limited, the terms may entitle you to a replacement or refund.
	\item They are entitled to first class travel.
	\item There are 23 Clubs throughout the U.S., and your membership entitles you to enjoy
all of them.
	\end{itemize}
}
\item verb \\
If the title of something such as a book , film , or painting is, for example , ' Sunrise ', you can  say that it \textbf{is entitled} 'Sunrise'.
 \textit{
	\begin{itemize}
	\item Chomsky's review is entitled 'Psychology and Ideology'.
	\item ...a performance entitled 'United States'.
	\end{itemize}
}
\end{enumerate}

\section*{hospitality}
{\large \color{blue}  }
\subsection*{Explain}
\begin{enumerate}
\item uncountable noun \\
\textbf{Hospitality} is friendly , welcoming behaviour towards guests or people you have just met .
 \textit{
	\begin{itemize}
	\item Every visitor to Georgia is overwhelmed by the kindness, charm and hospitality of
the people.
	\end{itemize}
}
\item uncountable noun \\
\textbf{Hospitality} is the food, drink, and other privileges which some companies provide for their visitors or clients at major  sporting or other public events.
 \textit{
	\begin{itemize}
	\item A few of us in the press were there, goggle-eyed at the lavish hospitality.
	\item ...corporate hospitality tents.
	\end{itemize}
}
\end{enumerate}

\section*{excuse}
{\large \color{blue}  excuses  excusing  excused  }
\subsection*{Explain}
\begin{enumerate}
\item countable noun \\
An \textbf{excuse} is a reason which you give in order to explain why something has been done or has not been done, or in order to avoid doing something.
 \textit{
	\begin{itemize}
	\item It is easy to find excuses for his indecisiveness.
	\item Once I had had a baby I had the perfect excuse to stay at home.
	\item If you stop making excuses and do it you'll wonder what took you so long.
	\end{itemize}
}
\item verb \\
To \textbf{excuse} someone or \textbf{excuse} their behaviour means to provide reasons for their actions , especially when other people disapprove of these actions.
 \textit{
	\begin{itemize}
	\item He excused himself by saying he was 'forced to rob to maintain my wife and cat'.
	\item That doesn't excuse my mother's behaviour.
	\end{itemize}
}
\item verb \\
If you \textbf{excuse} someone \textbf{for} something wrong that they have done, you forgive them for it.
 \textit{
	\begin{itemize}
	\item Many people might have excused them for shirking some of their responsibilities.
	\end{itemize}
}
\item verb \\
If someone \textbf{is excused}  \textbf{from} a duty or responsibility , they are told that they do not have to carry it out.
 \textit{
	\begin{itemize}
	\item She is usually excused from her duties during the school holidays.
	\item She was excused duties on Saturday.
	\end{itemize}
}
\item verb \\
If you \textbf{excuse}  \textbf{yourself} , you use a phrase such as 'Excuse me' as a polite  way of saying that you are about to leave.
 \textit{
	\begin{itemize}
	\item He excused himself and went up to his room.
	\end{itemize}
}
\item  \\
 excuse me \textit{
	\begin{itemize}
	\end{itemize}
}
\item  \\
 excuse me \textit{
	\begin{itemize}
	\end{itemize}
}
\item  \\
 excuse me \textit{
	\begin{itemize}
	\end{itemize}
}
\item  \\
 excuse me \textit{
	\begin{itemize}
	\end{itemize}
}
\item  \\
 excuse me \textit{
	\begin{itemize}
	\end{itemize}
}
\item  \\
 excuse me \textit{
	\begin{itemize}
	\end{itemize}
}
\item  \\
 excuse me \textit{
	\begin{itemize}
	\end{itemize}
}
\end{enumerate}

\section*{income}
{\large \color{blue}  incomes  }
\subsection*{Explain}
\begin{enumerate}
\item variable noun \\
A person's or organization's \textbf{income} is the money that they earn or receive , as opposed to the money that they have to spend or pay out.
 \textit{
	\begin{itemize}
	\item Many families on low incomes will be unable to afford to buy their own home.
	\item To cover its costs, the company will need an annual income of £15 million.
	\item Over a third of their income comes from comedy videos.
	\end{itemize}
}
\end{enumerate}

\section*{fabricate}
{\large \color{blue}  fabricates  fabricating  fabricated  }
\subsection*{Explain}
\begin{enumerate}
\item verb \\
If someone \textbf{fabricates}  information , they invent it in order to deceive people.
 \textit{
	\begin{itemize}
	\item All four claim that officers fabricated evidence against them.
	\item Eleven key officials were hanged on fabricated charges.
	\end{itemize}
}
\item verb \\
If something \textbf{is fabricated}  \textbf{from} different materials or substances, it is made out of those materials or substances.
 \textit{
	\begin{itemize}
	\item All the tools are fabricated from high quality steel.
	\item ...a plant which fabricates airplane components.
	\end{itemize}
}
\end{enumerate}

\section*{installation}
{\large \color{blue}  installations  }
\subsection*{Explain}
\begin{enumerate}
\item countable noun \\
An \textbf{installation} is a place that contains equipment and machinery which are being used for a particular purpose.
 \textit{
	\begin{itemize}
	\item The building was turned into a secret military installation.
	\item ...a nuclear installation.
	\end{itemize}
}
\end{enumerate}

\section*{forgive}
{\large \color{blue}  forgives  forgiving  forgave  forgiven  }
\subsection*{Explain}
\begin{enumerate}
\item verb \\
If you \textbf{forgive} someone who has done something bad or wrong , you stop being angry with them and no longer want to punish them.
 \textit{
	\begin{itemize}
	\item Hopefully she'll understand and forgive you, if she really loves you.
	\item She'd find a way to forgive him for the theft of the money.
	\item Still, for those flashes of genius, you can forgive him anything.
	\end{itemize}
}
\item passive verb \\
If you say that someone could \textbf{be forgiven for} doing something, you mean that they were wrong or mistaken, but not seriously , because many people would have done the same thing in those circumstances .
 \textit{
	\begin{itemize}
	\item Looking at the figures, you could be forgiven for thinking the recession is already
over.
	\item If the research which enticed them to Britain is removed, they can be forgiven for
feeling betrayed.
	\end{itemize}
}
\item verb \\
\textbf{Forgive} is used in polite  expressions and apologies like ' \textbf{forgive me} ' and ' \textbf{forgive my ignorance} ' when you are saying or doing something that might  seem  rude , silly , or complicated .
 \textit{
	\begin{itemize}
	\item Forgive me, I don't mean to insult you.
	\item I do hope you'll forgive me but I've got to leave.
	\item 'Forgive my manners,' she said calmly. 'I neglected to introduce myself.'
	\end{itemize}
}
\item verb \\
If an organization such as a bank  \textbf{forgives} someone's debt, they agree not to ask for that money to be repaid .
 \textit{
	\begin{itemize}
	\item This man has just been forgiven a debt of $10 million.
	\end{itemize}
}
\end{enumerate}

\section*{intrigue}
{\large \color{blue}  intrigues  intriguing  intrigued  }
\subsection*{Explain}
\begin{enumerate}
\item variable noun \\
\textbf{Intrigue} is the making of secret plans to harm or deceive people.
 \textit{
	\begin{itemize}
	\item ...political intrigue.
	\item ...a powerful story of intrigue, passion and betrayal.
	\item ...the plots and intrigues in the novel.
	\end{itemize}
}
\item verb \\
If something, especially something strange , \textbf{intrigues} you, it interests you and you want to know more about it.
 \textit{
	\begin{itemize}
	\item The novelty of the situation intrigued him.
	\end{itemize}
}
\end{enumerate}

\section*{gain}
{\large \color{blue}  gains  gaining  gained  }
\subsection*{Explain}
\begin{enumerate}
\item verb \\
If a person or place \textbf{gains} something such as an ability or quality, they gradually get more of it.
 \textit{
	\begin{itemize}
	\item Students can gain valuable experience by working on the campus radio or magazine.
	\item While it has lost its tranquility, the area has gained in liveliness.
	\end{itemize}
}
\item verb \\
If you \textbf{gain}  \textbf{from} something such as an event or situation , you get some advantage or benefit from it.
 \textit{
	\begin{itemize}
	\item The company didn't disclose how much it expects to gain from the two deals.
	\item There is absolutely nothing to be gained by feeling bitter.
	\item It is sad that a major company should try to gain from other people's suffering.
	\end{itemize}
}
\item verb \\
To \textbf{gain} something such as weight or speed means to have an increase in that particular thing.
 \textbf{Gain} is also a noun .
 \textit{
	\begin{itemize}
	\item Some people do gain weight after they stop smoking.
	\item The BMW started coming forward, passing the other cars and gaining speed as it approached.
	\item She gained some 25lb in weight during her pregnancy.
	\item News on new home sales is brighter, showing a gain of nearly 8% in June.
	\item Excessive weight gain doesn't do you any good.
	\end{itemize}
}
\item verb \\
If you \textbf{gain} something, you obtain it, especially after a lot of hard work or effort .
 \textit{
	\begin{itemize}
	\item They realise that passing exams is no longer enough to gain a place at university.
	\item Their efforts helped the hostages gain their freedom.
	\end{itemize}
}
\item  \\
 for gain \textit{
	\begin{itemize}
	\end{itemize}
}
\item  \\
 gain ground \textit{
	\begin{itemize}
	\end{itemize}
}
\item  \\
 gain time \textit{
	\begin{itemize}
	\end{itemize}
}
\end{enumerate}

\section*{layout}
{\large \color{blue}  layouts  }
\subsection*{Explain}
\begin{enumerate}
\item countable noun \\
The \textbf{layout} of a garden , building, or piece of writing is the way in which the parts of it are arranged .
 \textit{
	\begin{itemize}
	\item He tried to recall the layout of the farmhouse.
	\item This boat has a good deck layout making everything easy to operate.
	\end{itemize}
}
\end{enumerate}

\section*{get}
{\large \color{blue}  gets  getting  got  gotten  }
\subsection*{Explain}
\begin{enumerate}
\item link verb \\
You use \textbf{get} with adjectives to mean 'become'. For example, if someone \textbf{gets cold} , they become cold, and if they \textbf{get angry} , they become angry.
 \textit{
	\begin{itemize}
	\item The boys were getting bored.
	\item There's no point in getting upset.
	\item From here on, it can only get better.
	\end{itemize}
}
\item link verb \\
\textbf{Get} is used with expressions referring to states or situations. For example, to \textbf{get into trouble} means to start being in trouble.
 \textit{
	\begin{itemize}
	\item Half the pleasure of an evening out is getting ready.
	\item Perhaps I shouldn't say that–I might get into trouble.
	\item How did we get into this recession, and what can we do to get out of it?
	\end{itemize}
}
\item verb \\
To \textbf{get} someone or something into a particular state or situation means to cause them to
be in it.
 \textit{
	\begin{itemize}
	\item I don't know if I can get it clean.
	\item What got me interested was looking at an old New York Times.
	\item Brian will get them out of trouble.
	\end{itemize}
}
\item verb \\
If you \textbf{get} someone \textbf{to} do something, you cause them to do it by asking , persuading, or telling them to do it.
 \textit{
	\begin{itemize}
	\item Get your partner to massage your abdomen.
	\item How did you get him to pose for this picture?
	\end{itemize}
}
\item verb \\
If you \textbf{get} something done, you cause it to be done.
 \textit{
	\begin{itemize}
	\item I might benefit from getting my teeth fixed.
	\item It was best to get things done quickly.
	\end{itemize}
}
\item verb \\
To \textbf{get}  somewhere means to move there.
 \textit{
	\begin{itemize}
	\item I got off the bed and opened the door.
	\item How can I get past her without her seeing me?
	\item I heard David yelling and telling them to get back.
	\end{itemize}
}
\item verb \\
When you \textbf{get} to a place, you arrive there.
 \textit{
	\begin{itemize}
	\item Generally I get to work at 9.30am.
	\item It was dark by the time she got home.
	\end{itemize}
}
\item verb \\
To \textbf{get} something or someone into a place or position means to cause them to move there.
 \textit{
	\begin{itemize}
	\item Mack got his wallet out.
	\item Go and get your coat on.
	\item The U.N. was supposed to be getting aid to where it was most needed.
	\end{itemize}
}
\item auxiliary verb \\
\textbf{Get} is often used in place of 'be' as an auxiliary  verb to form passives.
 \textit{
	\begin{itemize}
	\item Does she ever get asked for her autograph?
	\item A pane of glass got broken.
	\end{itemize}
}
\item verb \\
If you \textbf{get}  \textbf{to} do something, you eventually or gradually reach a stage at which you do it.
 \textit{
	\begin{itemize}
	\item Miller and Ferlinghetti got to be friends.
	\item No one could figure out how he got to be so wealthy.
	\end{itemize}
}
\item verb \\
If you \textbf{get}  \textbf{to} do something, you manage to do it or have the opportunity to do it.
 \textit{
	\begin{itemize}
	\item How do these people get to be the bosses of major companies?
	\item Do you get to see him often?
	\item They get to stay in nice hotels.
	\end{itemize}
}
\item verb \\
You can use \textbf{get} in expressions like \textbf{get moving} , \textbf{get going} , and \textbf{get working} when you want to tell people to begin moving, going, or working quickly.
 \textit{
	\begin{itemize}
	\item I aim to be off the lake before dawn, so let's get moving.
	\item We need to get thinking, talking and acting on this before it is too late.
	\end{itemize}
}
\item verb \\
If you \textbf{get}  \textbf{to} a particular stage in your life or in something you are doing, you reach that stage.
 \textit{
	\begin{itemize}
	\item We haven't got to the stage of a full-scale military conflict.
	\item If she gets that far, Jane may get legal aid to take her case to court.
	\item It got to the point where I was so ill I was waiting to die.
	\end{itemize}
}
\item verb \\
You can use \textbf{get} to talk about the progress that you are making. For example, if you say that you \textbf{are}  \textbf{getting somewhere} , you mean that you are making progress, and if you say that something \textbf{won't get} you \textbf{anywhere} , you mean it will not help you to progress at all.
 \textit{
	\begin{itemize}
	\item Radical factions say the talks are getting nowhere and they want to withdraw.
	\item My perseverance was getting me somewhere.
	\end{itemize}
}
\item link verb \\
When \textbf{it}  \textbf{gets}  \textbf{to} a particular time, it is that time. If \textbf{it}  \textbf{is getting}  \textbf{towards} a particular time, it is approaching that time.
 \textit{
	\begin{itemize}
	\item It got to after 1am and I was exhausted.
	\item It was getting towards evening when we got back.
	\item It's getting late.
	\end{itemize}
}
\item verb \\
If something that has continued for some time \textbf{gets to} you, it starts causing you to suffer .
 \textit{
	\begin{itemize}
	\item That's the first time I lost my cool in 20 years in this job. This whole thing's
getting to me.
	\end{itemize}
}
\item verb \\
If something \textbf{gets} you, it annoys you.
 \textit{
	\begin{itemize}
	\item What gets me is the attitude of so many of the people.
	\end{itemize}
}
\end{enumerate}

\section*{lot}
{\large \color{blue}  lots  }
\subsection*{Explain}
\begin{enumerate}
\item quantifier \\
\textbf{A lot of} something or \textbf{lots of} it is a large amount of it. \textbf{A lot of} people or things, or \textbf{lots of} them, is a large number of them.
 \textbf{Lot} is also a pronoun .
 \textit{
	\begin{itemize}
	\item A lot of our land is used to grow crops for export.
	\item I remember a lot of things.
	\item 'You'll find that everybody will try and help their colleague.'—'Yeah. There's a
lot of that.'
	\item Lots of pubs like to deck themselves out with flowers in summer.
	\item He drank lots of milk.
	\item A lot of the play is very funny.
	\item There's lots going on at Selfridges this month.
	\item I learned a lot from him about how to run a band.
	\item I know a lot has been said about my sister's role in my career.
	\end{itemize}
}
\item adverb \\
\textbf{A lot}  means to a great extent or degree.
 \textit{
	\begin{itemize}
	\item Matthew's out quite a lot doing his research.
	\item I like you, a lot.
	\item If I went out and accepted a job at a lot less money, I'd jeopardize a good career.
	\end{itemize}
}
\item adverb \\
If you do something \textbf{a lot} , you do it often or for a long time.
 \textit{
	\begin{itemize}
	\item They went out a lot, to the Cafe Royal or The Ivy.
	\item He talks a lot about his own children.
	\end{itemize}
}
\item countable noun \\
You can use \textbf{lot} to refer to a set or group of things or people.
 \textit{
	\begin{itemize}
	\item He bought two lots of 1,000 shares in the company during August and September.
	\item We've just sacked one lot of builders.
	\end{itemize}
}
\item singular noun \\
You can refer to a specific group of people as a particular  \textbf{lot} .
 \textit{
	\begin{itemize}
	\item Future generations are going to think that we were a pretty boring lot.
	\end{itemize}
}
\item singular noun \\
You can use \textbf{the lot} to refer to the whole of an amount that you have just mentioned .
 \textit{
	\begin{itemize}
	\item She was given £20 and by Monday morning had spent the lot.
	\end{itemize}
}
\item singular noun \\
Your \textbf{lot} is the kind of life you have or the things that you have or experience .
 \textit{
	\begin{itemize}
	\item Sometimes you just have to accept your lot in life .
	\end{itemize}
}
\item countable noun \\
A \textbf{lot} is a small area of land that belongs to a person or company .
 \textit{
	\begin{itemize}
	\item If oil or gold are discovered under your lot, you can sell the mineral rights.
	\end{itemize}
}
\item countable noun \\
A \textbf{lot} in an auction is one of the objects or groups of objects that are being sold .
 \textit{
	\begin{itemize}
	\item The receivers are keen to sell the stores as one lot.
	\item The two lots have made just over £5 million.
	\end{itemize}
}
\item  \\
 to draw lots \textit{
	\begin{itemize}
	\end{itemize}
}
\item  \\
 to throw in your lot with someone \textit{
	\begin{itemize}
	\end{itemize}
}
\end{enumerate}

\section*{give}
{\large \color{blue}  gives  giving  gave  given  }
\subsection*{Explain}
\begin{enumerate}
\item verb \\
You can use \textbf{give} with nouns that refer to physical actions. The whole expression refers to the performing of the action. For example, \textbf{She gave a smile} means almost the same as 'She smiled'.
 \textit{
	\begin{itemize}
	\item She stretched her arms out and gave a great yawn.
	\item Giving a sigh, she fell to her knees at my feet.
	\item He gave her a fond smile.
	\item He reached for her hand and gave it a reassuring squeeze.
	\end{itemize}
}
\item verb \\
You use \textbf{give} to say that a person does something for another person. For example, if you \textbf{give} someone a lift , you take them somewhere in your car.
 \textit{
	\begin{itemize}
	\item I gave her a lift back out to her house.
	\item He was given mouth-to-mouth resuscitation.
	\item Sophie asked her if she would like to come and give art lessons.
	\end{itemize}
}
\item verb \\
You use \textbf{give} with nouns that refer to information, opinions , or greetings to indicate that something is communicated. For example, if you \textbf{give} someone some news , you tell it to them.
 \textit{
	\begin{itemize}
	\item He gave no details.
	\item Would you like to give me your name?
	\item He asked me to give his regards to all of you.
	\item He gave the cause of death as multiple injuries.
	\end{itemize}
}
\item verb \\
You use \textbf{give} to say how long you think something will last or how much you think something will be.
 \textit{
	\begin{itemize}
	\item A poll last week gave the opposition a narrow six-point lead.
	\item Ted and his lawyers gave the company 11 months to sell off everything.
	\end{itemize}
}
\item verb \\
People use \textbf{give} in expressions such as \textbf{I don't give a damn} to show that they do not care about something.
 \textit{
	\begin{itemize}
	\item They don't give a damn about the country.
	\end{itemize}
}
\item verb \\
If someone or something \textbf{gives} you a particular idea or impression , it causes you to have that idea or impression.
 \textit{
	\begin{itemize}
	\item They gave me the impression that they were doing exactly what they wanted in life.
	\item The examiner's final report does not give an accurate picture.
	\end{itemize}
}
\item verb \\
If someone or something \textbf{gives} you a particular physical or emotional feeling, it makes you experience it.
 \textit{
	\begin{itemize}
	\item He gave me a shock.
	\item It will give great pleasure to the many thousands of children who visit the hospital
each year.
	\end{itemize}
}
\item verb \\
If you \textbf{give} a performance or speech, you perform or speak in public.
 \textit{
	\begin{itemize}
	\item Kotto gives a stupendous performance.
	\item I am sure you remember Mrs Butler who gave us such an interesting talk last year.
	\end{itemize}
}
\item verb \\
If you \textbf{give} something thought or attention , you think about it, concentrate on it, or deal with it.
 \textit{
	\begin{itemize}
	\item I've been giving it some thought.
	\item Priority will be given to those who apply early.
	\end{itemize}
}
\item verb \\
If you \textbf{give} a party or other social event, you organize it.
 \textit{
	\begin{itemize}
	\item That evening, I gave a dinner party for a few close friends.
	\end{itemize}
}
\end{enumerate}

\section*{missile}
{\large \color{blue}  missiles  }
\subsection*{Explain}
\begin{enumerate}
\item countable noun \\
A \textbf{missile} is a tube-shaped weapon that travels long distances through the air and explodes when it reaches its target.
 \textit{
	\begin{itemize}
	\item Helicopters fired missiles at the camp.
	\item ...nuclear missiles.
	\end{itemize}
}
\item countable noun \\
Anything that is thrown as a weapon can be called a \textbf{missile} .
 \textit{
	\begin{itemize}
	\item The football supporters began throwing missiles, one of which hit the referee.
	\end{itemize}
}
\end{enumerate}

\section*{govern}
{\large \color{blue}  governs  governing  governed  }
\subsection*{Explain}
\begin{enumerate}
\item verb \\
To \textbf{govern} a place such as a country , or its people, means to be officially in charge of the place, and to have responsibility for making laws , managing the economy , and controlling public  services .
 \textit{
	\begin{itemize}
	\item They go to the polls on Friday to choose the people they want to govern their country.
	\item Their citizens are very thankful they are not governed by a dictator.
	\end{itemize}
}
\item verb \\
If a situation or activity  \textbf{is governed}  \textbf{by} a particular  factor , rule, or force , it is controlled by that factor, rule, or force.
 \textit{
	\begin{itemize}
	\item Marine insurance is governed by a strict series of rules and regulations.
	\item The government has altered the rules governing eligibility for unemployment benefit.
	\end{itemize}
}
\end{enumerate}

\section*{perfection}
{\large \color{blue}  }
\subsection*{Explain}
\begin{enumerate}
\item uncountable noun \\
\textbf{Perfection} is the quality of being as good as it is possible for something of a particular kind to be.
 \textit{
	\begin{itemize}
	\item His quest for perfection is relentless.
	\item Physical perfection in a human being is exceedingly rare.
	\end{itemize}
}
\item uncountable noun \\
If you say that something is \textbf{perfection} , you mean that you think it is as good as it could possibly be.
 \textit{
	\begin{itemize}
	\item The house and garden were perfection.
	\end{itemize}
}
\item uncountable noun \\
\textbf{The perfection of} something such as a skill , system, or product  involves making it as good as it could possibly be.
 \textit{
	\begin{itemize}
	\item Madame Clicquot is credited with the perfection of this technique.
	\end{itemize}
}
\item  \\
 to perfection \textit{
	\begin{itemize}
	\end{itemize}
}
\end{enumerate}

\section*{greet}
{\large \color{blue}  greets  greeting  greeted  }
\subsection*{Explain}
\begin{enumerate}
\item verb \\
When you \textbf{greet} someone, you say 'Hello' or shake  hands with them.
 \textit{
	\begin{itemize}
	\item She liked to be home to greet Steve when he came in from school.
	\end{itemize}
}
\item verb \\
If something \textbf{is greeted} in a particular way, people react to it in that way.
 \textit{
	\begin{itemize}
	\item The court's decision has been greeted with dismay by fishermen.
	\item It is unlikely that this suggestion will be greeted enthusiastically in the Baltic
States.
	\end{itemize}
}
\item verb \\
If you \textbf{are greeted}  \textbf{by} something, it is the first thing you notice in a particular place.
 \textit{
	\begin{itemize}
	\item I was greeted by a shocking sight.
	\item The savoury smell greeted them as they went through the door.
	\end{itemize}
}
\end{enumerate}

\section*{phase}
{\large \color{blue}  phases  phasing  phased  }
\subsection*{Explain}
\begin{enumerate}
\item countable noun \\
A \textbf{phase} is a particular stage in a process or in the gradual development of something.
 \textit{
	\begin{itemize}
	\item This autumn, 6000 residents will participate in the first phase of the project.
	\item The crisis is entering a crucial, critical phase.
	\item Most kids will go through a phase of being faddy about what they eat.
	\end{itemize}
}
\item verb \\
If an action or change \textbf{is phased}  \textbf{over} a period of time, it is done in stages.
 \textit{
	\begin{itemize}
	\item The redundancies will be phased over two years.
	\item He wants military commanders to plan a phased withdrawal starting at the end of the
year.
	\end{itemize}
}
\item  \\
 in phase/out of phase \textit{
	\begin{itemize}
	\end{itemize}
}
\end{enumerate}

\section*{hum}
{\large \color{blue}  hums  humming  hummed  }
\subsection*{Explain}
\begin{enumerate}
\item verb \\
If something \textbf{hums} , it makes a low continuous noise.
 \textbf{Hum} is also a noun .
 \textit{
	\begin{itemize}
	\item The birds sang, the bees hummed.
	\item Within five hours, the equipment will be humming away again.
	\item There was a low humming sound in the sky.
	\item ...the hum of traffic.
	\item There was a general hum of conversation around them.
	\end{itemize}
}
\item verb \\
When you \textbf{hum} a tune , you sing it with your lips closed.
 \textit{
	\begin{itemize}
	\item She was humming a merry little tune.
	\item He hummed to himself as he opened the trunk.
	\end{itemize}
}
\item verb \\
If you say that a place \textbf{hums} , you mean that it is full of activity.
 \textit{
	\begin{itemize}
	\item The place is really beginning to hum.
	\item On Saturday morning, the town hums with activity and life.
	\end{itemize}
}
\item convention \\
\textbf{Hum} is sometimes used to represent the sound people make when they are not sure what to say.
 \textit{
	\begin{itemize}
	\item Hum, I am sorry but I thought you were French.
	\end{itemize}
}
\end{enumerate}

\section*{porter}
{\large \color{blue}  porters  }
\subsection*{Explain}
\begin{enumerate}
\item countable noun \\
A \textbf{porter} is a person whose job is to be in charge of the entrance of a building such as a hotel.
 \textit{
	\begin{itemize}
	\end{itemize}
}
\item countable noun \\
A \textbf{porter} is a person whose job is to carry things, for example people's luggage at a railway station or in a hotel.
 \textit{
	\begin{itemize}
	\end{itemize}
}
\item countable noun \\
A \textbf{porter} on a train is a person whose job is to make up beds in the sleeping  car and to help passengers.
 \textit{
	\begin{itemize}
	\end{itemize}
}
\item countable noun \\
In a hospital, a \textbf{porter} is someone whose job is to move patients from place to place.
 \textit{
	\begin{itemize}
	\end{itemize}
}
\end{enumerate}

\section*{impair}
{\large \color{blue}  impairs  impairing  impaired  }
\subsection*{Explain}
\begin{enumerate}
\item verb \\
If something \textbf{impairs} something such as an ability or the way something works, it damages it or makes it worse .
 \textit{
	\begin{itemize}
	\item Consumption of alcohol impairs your ability to drive a car or operate machinery.
	\item His movements were painfully impaired by arthritis.
	\end{itemize}
}
\end{enumerate}

\section*{injure}
{\large \color{blue}  injures  injuring  injured  }
\subsection*{Explain}
\begin{enumerate}
\item verb \\
If you \textbf{injure} a person or animal, you damage some part of their body.
 \textit{
	\begin{itemize}
	\item A number of bombs have exploded, seriously injuring at least five people.
	\item ...stiff penalties for motorists who kill, maim, and injure.
	\end{itemize}
}
\end{enumerate}

\section*{rank}
{\large \color{blue}  ranker  rankest  ranks  ranking  ranked  }
\subsection*{Explain}
\begin{enumerate}
\item variable noun \\
Someone's \textbf{rank} is the position or grade that they have in an organization.
 \textit{
	\begin{itemize}
	\item He eventually rose to the rank of captain.
	\item The former head of counter-intelligence had been stripped of his rank and privileges.
	\item ...officers of equivalent rank in the other branches.
	\end{itemize}
}
\item variable noun \\
Someone's \textbf{rank} is the social class, especially the high social class, that they belong to.
 \textit{
	\begin{itemize}
	\item Each rank of the peerage was represented.
	\item He must be treated as a hostage of high rank, not as a common prisoner.
	\end{itemize}
}
\item verb \\
If an official organization \textbf{ranks} someone or something 1st, 5th, or 50th, for example, they calculate that the person or thing has that position on a scale. You can also say that someone or something \textbf{ranks} 1st, 5th, or 50th, for example.
 \textit{
	\begin{itemize}
	\item The report ranks the U.K. 20th out of 22 advanced nations.
	\item He was at the time ranked 10th in the world and had a regular place in the Swedish
Davis Cup team.
	\item She was ranked 12th in a recent Top 50 Models poll.
	\item She was ranked in the top 50 of the women's world rankings.
	\item Mr Short does not even rank in the world's top ten.
	\end{itemize}
}
\item verb \\
If you say that someone or something \textbf{ranks} high or low on a scale or if you \textbf{rank} them high or low, you are saying how good or important you think they are.
 \textit{
	\begin{itemize}
	\item His prices rank high among those of other contemporary photographers.
	\item Investors ranked South Korea high among Asian nations.
	\item St Petersburg's night life ranks as more exciting than the capital's.
	\item 18 per cent of women ranked sex as very important in their lives.
	\item The Ritz-Carlton in Aspen has to rank as one of the most extraordinary hotels I have
ever been to.
	\item Since the 1930s, cancer has always been ranked as the disease people are most concerned
about.
	\end{itemize}
}
\item verb \\
If you say that someone or something \textbf{ranks with} a group of famous people or things, you mean that they are extremely good and should be included in
that group.
 \textit{
	\begin{itemize}
	\item ...a performance of heroic calibre that must rank with the most memorable.
	\item We found his Hot soufflé in cinnamon spice with Drambuie cream to rank with the best
English sweets.
	\end{itemize}
}
\item plural noun \\
The \textbf{ranks} of a group or organization are the people who belong to it.
 \textit{
	\begin{itemize}
	\item There were some misgivings within the ranks of the media too.
	\item The General Assembly welcomed five new members to its ranks.
	\end{itemize}
}
\item plural noun \\
\textbf{The ranks} are the ordinary members of an organization, especially of the armed forces.
 \textit{
	\begin{itemize}
	\item Top military leaders say there have been reports of demoralization in the ranks.
	\item Most store managers have worked their way up through the ranks.
	\end{itemize}
}
\item countable noun \\
A \textbf{rank}  \textbf{of} people or things is a row of them.
 \textit{
	\begin{itemize}
	\item Ranks of police in riot gear stood nervously by.
	\item She continued to smile at the ranks of cameras on their doorstep.
	\end{itemize}
}
\item countable noun \\
A \textbf{taxi}  \textbf{rank} is a place on a city street where taxis park when they are available for hire.
 \textit{
	\begin{itemize}
	\item The man led the way to the taxi rank.
	\item He walked towards the first taxi on the rank.
	\end{itemize}
}
\item adjective \\
You can use \textbf{rank} to emphasize a bad or undesirable quality that exists in an extreme form.
 \textit{
	\begin{itemize}
	\item He called it 'rank hypocrisy' that the government was now promoting equal rights.
	\end{itemize}
}
\item adjective \\
You can describe something as \textbf{rank} when it has a strong and unpleasant smell.
 \textit{
	\begin{itemize}
	\item The kitchen was rank with the smell of drying uniforms.
	\item ...the rank smell of unwashed clothes.
	\end{itemize}
}
\item  \\
 to break ranks \textit{
	\begin{itemize}
	\end{itemize}
}
\item  \\
 to close ranks \textit{
	\begin{itemize}
	\end{itemize}
}
\item  \\
 join the ranks of X, join sb's ranks \textit{
	\begin{itemize}
	\end{itemize}
}
\item  \\
 rank outsider/outsiders \textit{
	\begin{itemize}
	\end{itemize}
}
\item  \\
 to pull rank \textit{
	\begin{itemize}
	\end{itemize}
}
\end{enumerate}

\section*{laugh}
{\large \color{blue}  laughs  laughing  laughed  }
\subsection*{Explain}
\begin{enumerate}
\item verb \\
When you \textbf{laugh} , you make a sound with your throat while smiling and show that you are happy or amused . People also  sometimes laugh when they feel  nervous or are being unfriendly .
 \textbf{Laugh} is also a noun .
 \textit{
	\begin{itemize}
	\item He was about to offer an explanation, but she was beginning to laugh.
	\item He laughed with pleasure when people said he looked like his dad.
	\item The British don't laugh at the same jokes as the French.
	\item 'I'll be astonished if I win on Sunday,' laughed Lyle.
	\item Lysenko gave a deep rumbling laugh at his own joke.
	\end{itemize}
}
\item verb \\
If people \textbf{laugh at} someone or something, they mock them or make jokes about them.
 \textit{
	\begin{itemize}
	\item I thought they were laughing at me because I was ugly.
	\item She wanted to laugh at the melodramatic way he was acting.
	\end{itemize}
}
\item  \\
 for a laugh/for laughs \textit{
	\begin{itemize}
	\end{itemize}
}
\item  \\
 get a laugh \textit{
	\begin{itemize}
	\end{itemize}
}
\item  \\
 a good laugh/a bit of a laugh \textit{
	\begin{itemize}
	\end{itemize}
}
\item  \\
 a good laugh \textit{
	\begin{itemize}
	\end{itemize}
}
\item  \\
 have a good laugh about something \textit{
	\begin{itemize}
	\end{itemize}
}
\item  \\
 to have the last laugh \textit{
	\begin{itemize}
	\end{itemize}
}
\item  \\
 don't make me laugh \textit{
	\begin{itemize}
	\end{itemize}
}
\item  \\
 you've got to laugh/you have to laugh \textit{
	\begin{itemize}
	\end{itemize}
}
\end{enumerate}

\section*{receipt}
{\large \color{blue}  receipts  }
\subsection*{Explain}
\begin{enumerate}
\item countable noun \\
A \textbf{receipt} is a piece of paper that you get from someone as proof that they have received money or goods from you. In British English a \textbf{receipt} is a piece of paper that you get in a shop when you buy something, but in American English the more usual  term for this is sales slip .
 \textit{
	\begin{itemize}
	\item I wrote her a receipt for the money.
	\end{itemize}
}
\item plural noun \\
\textbf{Receipts} are the amount of money received during a particular period, for example by a shop or theatre .
 \textit{
	\begin{itemize}
	\item The film opened to healthy box office receipts before rapidly falling off.
	\item He was tallying the day's receipts.
	\end{itemize}
}
\item uncountable noun \\
The \textbf{receipt} of something is the act of receiving it.
 \textit{
	\begin{itemize}
	\item Goods should be supplied within 28 days after the receipt of your order.
	\end{itemize}
}
\item  \\
 in receipt of \textit{
	\begin{itemize}
	\end{itemize}
}
\end{enumerate}

\section*{linger}
{\large \color{blue}  lingers  lingering  lingered  }
\subsection*{Explain}
\begin{enumerate}
\item verb \\
When something such as an idea, feeling, or illness  \textbf{lingers} , it continues to exist for a long time, often much longer than expected .
 \textit{
	\begin{itemize}
	\item The scent of her perfume lingered on in the room.
	\item A guerrilla war has lingered into its fourth decade.
	\item He was ashamed. That feeling lingered, and he was never comfortable in church after
that.
	\item He would rather be killed in a race than die a lingering death in hospital.
	\end{itemize}
}
\item verb \\
If you \textbf{linger}  somewhere , you stay there for a longer time than is necessary , for example because you are enjoying yourself.
 \textit{
	\begin{itemize}
	\item Customers are welcome to linger over coffee until around midnight.
	\item I lingered on in Atlanta for a few days, spending much of my time with an artist
friend.
	\item It is a dreary little town where few would choose to linger.
	\end{itemize}
}
\end{enumerate}

\section*{safety}
{\large \color{blue}  }
\subsection*{Explain}
\begin{enumerate}
\item uncountable noun \\
\textbf{Safety} is the state of being safe from harm or danger.
 \textit{
	\begin{itemize}
	\item The report goes on to make a number of recommendations to improve safety on aircraft.
	\end{itemize}
}
\item uncountable noun \\
If you reach  \textbf{safety} , you reach a place where you are safe from danger.
 \textit{
	\begin{itemize}
	\item He stumbled through smoke and fumes given off from her burning sofa to pull her to
safety.
	\item Guests ran for safety as the device went off in a ground-floor men's toilet.
	\item The refugees were groping their way through the dark, trying to reach safety.
	\item ...the safety of one's own home.
	\end{itemize}
}
\item singular noun \\
If you are concerned about the \textbf{safety} of something, you are concerned that it might be harmful or dangerous .
 \textit{
	\begin{itemize}
	\item ...consumers are showing growing concern about the safety of the food they buy.
	\item There is concern about the safety of the new treatment as it has not yet been proven.
	\end{itemize}
}
\item singular noun \\
If you are concerned for someone's \textbf{safety} , you are concerned that they might be in danger.
 \textit{
	\begin{itemize}
	\item There is grave concern for the safety of witnesses.
	\item The two youths today declined to testify because they said they feared for their
safety.
	\end{itemize}
}
\item adjective \\
\textbf{Safety}  features or measures are intended to make something less dangerous.
 \textit{
	\begin{itemize}
	\item The built-in safety device compensates for a fall in water pressure.
	\item ...safety glasses.
	\end{itemize}
}
\item  \\
 safety in numbers \textit{
	\begin{itemize}
	\end{itemize}
}
\end{enumerate}

\section*{obtain}
{\large \color{blue}  obtains  obtaining  obtained  }
\subsection*{Explain}
\begin{enumerate}
\item verb \\
To \textbf{obtain} something means to get it or achieve it.
 \textit{
	\begin{itemize}
	\item Evans was trying to obtain a false passport and other documents.
	\item The perfect body has always been difficult to obtain.
	\end{itemize}
}
\item verb \\
If a situation  \textbf{obtains} , it exists.
 \textit{
	\begin{itemize}
	\item The longer this situation obtains, the more extensive the problems become .
	\end{itemize}
}
\end{enumerate}

\section*{satire}
{\large \color{blue}  satires  }
\subsection*{Explain}
\begin{enumerate}
\item uncountable noun \\
\textbf{Satire} is the use of humour or exaggeration in order to show how foolish or wicked some people's behaviour or ideas are.
 \textit{
	\begin{itemize}
	\item The commercial side of the Christmas season is an easy target for satire.
	\end{itemize}
}
\item countable noun \\
A \textbf{satire} is a play, film, or novel in which humour or exaggeration is used to criticize something.
 \textit{
	\begin{itemize}
	\item ...a sharp satire on the American political process.
	\end{itemize}
}
\end{enumerate}

\section*{security}
{\large \color{blue}  securities  }
\subsection*{Explain}
\begin{enumerate}
\item uncountable noun \\
\textbf{Security}  refers to all the measures that are taken to protect a place, or to ensure that only people with permission  enter it or leave it.
 \textit{
	\begin{itemize}
	\item They are now under a great deal of pressure to tighten their airport security.
	\item Strict security measures are in force in the capital.
	\item ...a top security jail.
	\end{itemize}
}
\item uncountable noun \\
A feeling of \textbf{security} is a feeling of being safe and free from worry .
 \textit{
	\begin{itemize}
	\item He loves the security of a happy home life.
	\item If an alarm gives you that feeling of security, then it's worth carrying.
	\end{itemize}
}
\item uncountable noun \\
If something is \textbf{security} for a loan , you promise to give that thing to the person who lends you money, if you fail to pay the money back.
 \textit{
	\begin{itemize}
	\item The central bank will provide special loans, and the banks will pledge the land as
security.
	\end{itemize}
}
\item plural noun \\
\textbf{Securities} are stocks , shares, bonds, or other certificates that you buy in order to earn  regular interest from them or to sell them later for a profit .
 \textit{
	\begin{itemize}
	\item National banks can package their own mortgages and underwrite them as securities.
	\item ...U.S. government securities and bonds.
	\end{itemize}
}
\end{enumerate}

\section*{pinch}
{\large \color{blue}  pinches  pinching  pinched  }
\subsection*{Explain}
\begin{enumerate}
\item verb \\
If you \textbf{pinch} a part of someone's body, you take a piece of their skin between your thumb and first finger and give it a short squeeze.
 \textbf{Pinch} is also a noun .
 \textit{
	\begin{itemize}
	\item She pinched his arm as hard as she could.
	\item We both kept pinching ourselves to prove that it wasn't all a dream.
	\item She gave him a little pinch.
	\end{itemize}
}
\item countable noun \\
A \textbf{pinch of} an ingredient such as salt is the amount of it that you can hold between your thumb and your first
finger.
 \textit{
	\begin{itemize}
	\item Put all the ingredients, including a pinch of salt, into a food processor.
	\item ...a pinch of nutmeg.
	\end{itemize}
}
\item verb \\
To \textbf{pinch} something, especially something of little value, means to steal it.
 \textit{
	\begin{itemize}
	\item Do you remember when I pinched your glasses?
	\item ...pickpockets who pinched his wallet.
	\end{itemize}
}
\item  \\
 at a pinch \textit{
	\begin{itemize}
	\end{itemize}
}
\item  \\
 feel the pinch \textit{
	\begin{itemize}
	\end{itemize}
}
\item  \\
 in a pinch \textit{
	\begin{itemize}
	\end{itemize}
}
\end{enumerate}

\section*{setting}
{\large \color{blue}  settings  }
\subsection*{Explain}
\begin{enumerate}
\item countable noun \\
A particular \textbf{setting} is a particular place or type of surroundings where something is or takes place.
 \textit{
	\begin{itemize}
	\item Rome is the perfect setting for romance.
	\item Perth was the setting for the SNP's conference this year.
	\item The house is in a lovely setting in the Malvern hills.
	\end{itemize}
}
\item countable noun \\
A \textbf{setting} is one of the positions to which the controls of a device such as a cooker , stove , or heater can be adjusted .
 \textit{
	\begin{itemize}
	\item You can boil the fish fillets on a high setting.
	\end{itemize}
}
\item countable noun \\
A table \textbf{setting} is the complete set of equipment that one person needs to eat a meal , including knives , forks , spoons , and glasses .
 \textit{
	\begin{itemize}
	\end{itemize}
}
\end{enumerate}

\section*{rent}
{\large \color{blue}  rents  renting  rented  }
\subsection*{Explain}
\begin{enumerate}
\item verb \\
If you \textbf{rent} something, you regularly pay its owner a sum of money in order to be able to have it and use it yourself.
 \textit{
	\begin{itemize}
	\item She rents a house with three other girls.
	\item He left his hotel in a rented car.
	\end{itemize}
}
\item verb \\
If you \textbf{rent} something \textbf{to} someone, you let them have it and use it in exchange for a sum of money which they pay you regularly.
 \textbf{Rent out} means the same as rent .
 \textit{
	\begin{itemize}
	\item She rented rooms to university students.
	\item He rented out his house while he worked abroad.
	\item He repaired the boat, and rented it out for $150.
	\end{itemize}
}
\item variable noun \\
\textbf{Rent} is the amount of money that you pay regularly to use a house, flat, or piece of land.
 \textit{
	\begin{itemize}
	\item She worked to pay the rent while I went to college.
	\item Traders in Marble Arch are facing huge rent increases.
	\end{itemize}
}
\item  \\
\textbf{Rent} is the past  tense and past participle of rend .
 \textit{
	\begin{itemize}
	\end{itemize}
}
\end{enumerate}

\section*{shadow}
{\large \color{blue}  shadows  shadowing  shadowed  }
\subsection*{Explain}
\begin{enumerate}
\item countable noun \\
A \textbf{shadow} is a dark shape on a surface that is made when something stands between a light and the surface.
 \textit{
	\begin{itemize}
	\item An oak tree cast its shadow over a tiny round pool.
	\item Nothing would grow in the shadow of the grey wall.
	\item All he could see was his shadow.
	\end{itemize}
}
\item uncountable noun \\
\textbf{Shadow} is darkness in a place caused by something preventing light from reaching it.
 \textit{
	\begin{itemize}
	\item Most of the lake was in shadow.
	\item ...a combination of light and shadow.
	\end{itemize}
}
\item verb \\
If something \textbf{shadows} a thing or place, it covers it with a shadow.
 \textit{
	\begin{itemize}
	\item The hood shadowed her face.
	\end{itemize}
}
\item verb \\
If someone \textbf{shadows} you, they follow you very closely wherever you go .
 \textit{
	\begin{itemize}
	\item The supporters are being shadowed by a large and highly visible body of police.
	\end{itemize}
}
\item adjective \\
A British Member of Parliament who is a member of the \textbf{shadow}  cabinet or who is a \textbf{shadow} cabinet minister belongs to the main opposition party and takes a special interest in matters which are the responsibility of a particular government minister.
 \textbf{Shadow} is also a noun .
 \textit{
	\begin{itemize}
	\item ...the shadow chancellor.
	\item Clarke swung at his shadow the accusation that he was 'a tabloid politician'.
	\end{itemize}
}
\item  \\
 a shadow of a doubt \textit{
	\begin{itemize}
	\end{itemize}
}
\item  \\
 in sb's shadow/ in the shadow of sb \textit{
	\begin{itemize}
	\end{itemize}
}
\item  \\
 be a shadow of one's former self \textit{
	\begin{itemize}
	\end{itemize}
}
\end{enumerate}

\section*{rule}
{\large \color{blue}  rules  ruling  ruled  }
\subsection*{Explain}
\begin{enumerate}
\item countable noun \\
\textbf{Rules} are instructions that tell you what you are allowed to do and what you are not allowed to do.
 \textit{
	\begin{itemize}
	\item ...a thirty-two-page pamphlet explaining the rules of basketball.
	\item Sikhs were expected to adhere strictly to the religious rules concerning appearance.
	\item Strictly speaking, this was against the rules.
	\item ...the amendment to Rule 22.
	\end{itemize}
}
\item countable noun \\
A \textbf{rule} is a statement telling people what they should do in order to achieve  success or a benefit of some kind .
 \textit{
	\begin{itemize}
	\item An important rule is to drink plenty of water during any flight.
	\item By and large, the rules for healthy eating are the same during pregnancy as at any
other time.
	\end{itemize}
}
\item countable noun \\
The \textbf{rules}  \textbf{of} something such as a language or a science are statements that describe the way that things usually happen in a particular situation .
 \textit{
	\begin{itemize}
	\item It is a rule of English that adjectives generally precede the noun they modify.
	\item ...according to the rules of quantum theory.
	\end{itemize}
}
\item singular noun \\
If something is \textbf{the rule} , it is the normal state of affairs .
 \textit{
	\begin{itemize}
	\item However, for many Americans today, weekend work has unfortunately become the rule
rather than the exception.
	\end{itemize}
}
\item verb \\
The person or group that \textbf{rules} a country controls its affairs.
 \textbf{Rule} is also a noun .
 \textit{
	\begin{itemize}
	\item Mongan ruled Ulster until his death in AD 625.
	\item He ruled for eight months.
	\item ...the long line of feudal lords who had ruled over this land.
	\item ...demands for an end to one-party rule.
	\end{itemize}
}
\item verb \\
If something \textbf{rules} your life, it influences or restricts your actions in a way that is not good for you.
 \textit{
	\begin{itemize}
	\item Scientists have always been aware of how fear can rule our lives and make us ill.
	\end{itemize}
}
\item verb \\
When someone in authority \textbf{rules} that something is true or should happen, they state that they have officially decided that it is true or should happen.
 \textit{
	\begin{itemize}
	\item The court ruled that laws passed by the assembly remained valid.
	\item The court has not yet ruled on the case.
	\item A provincial magistrates' court last week ruled it unconstitutional.
	\item The committee ruled against all-night opening mainly on safety grounds.
	\end{itemize}
}
\item verb \\
If you \textbf{rule} a straight line, you draw it using something that has a straight edge.
 \textit{
	\begin{itemize}
	\item ...a ruled grid of horizontal and vertical lines.
	\end{itemize}
}
\item  \\
 as a rule \textit{
	\begin{itemize}
	\end{itemize}
}
\item  \\
 bend the rules/stretch the rules \textit{
	\begin{itemize}
	\end{itemize}
}
\item  \\
 rule of thumb \textit{
	\begin{itemize}
	\end{itemize}
}
\item  \\
 work to rule \textit{
	\begin{itemize}
	\end{itemize}
}
\end{enumerate}

\section*{shield}
{\large \color{blue}  shields  shielding  shielded  }
\subsection*{Explain}
\begin{enumerate}
\item countable noun \\
Something or someone which is a \textbf{shield} against a particular danger or risk provides protection from it.
 \textit{
	\begin{itemize}
	\item He used his left hand as a shield against the reflecting sunlight.
	\item ...asbestos heat shields.
	\end{itemize}
}
\item verb \\
If something or someone \textbf{shields} you \textbf{from} a danger or risk, they protect you from it.
 \textit{
	\begin{itemize}
	\item He shielded his head from the sun with an old sack.
	\item The company does not bet its own money on equities, and so is shielded from market
risk.
	\end{itemize}
}
\item verb \\
If you \textbf{shield} your eyes , you put your hand above your eyes to protect them from direct  sunlight .
 \textit{
	\begin{itemize}
	\item He squinted and shielded his eyes.
	\end{itemize}
}
\item countable noun \\
A \textbf{shield} is a large piece of metal or leather which soldiers used to carry to protect their bodies while they were fighting .
 \textit{
	\begin{itemize}
	\end{itemize}
}
\item countable noun \\
A \textbf{shield} is a sports prize or badge that is shaped like a shield.
 \textit{
	\begin{itemize}
	\end{itemize}
}
\end{enumerate}

\section*{shrink}
{\large \color{blue}  shrinks  shrinking  shrank  shrunk  }
\subsection*{Explain}
\begin{enumerate}
\item verb \\
If cloth or clothing  \textbf{shrinks} , it becomes smaller in size, usually as a result of being washed .
 \textit{
	\begin{itemize}
	\item All my jumpers have shrunk.
	\end{itemize}
}
\item verb \\
If something \textbf{shrinks} or something else \textbf{shrinks} it, it becomes smaller.
 \textit{
	\begin{itemize}
	\item The vast forests of West Africa have shrunk.
	\item Hungary may have to lower its hopes of shrinking its state sector.
	\end{itemize}
}
\item verb \\
If you \textbf{shrink}  \textbf{away from} someone or something, you move away from them because you are frightened , shocked , or disgusted by them.
 \textit{
	\begin{itemize}
	\item One child shrinks away from me when I try to talk to him.
	\item Siegfried cringed and shrank against the wall.
	\item She shrank back with an involuntary gasp.
	\end{itemize}
}
\item verb \\
If you do not \textbf{shrink}  \textbf{from} a task or duty , you do it even though it is unpleasant or dangerous .
 \textit{
	\begin{itemize}
	\item We must not shrink from the legitimate use of force if we are to remain credible.
	\item They didn't shrink from danger.
	\end{itemize}
}
\item countable noun \\
A \textbf{shrink} is a psychiatrist.
 \textit{
	\begin{itemize}
	\item I've seen a shrink already.
	\end{itemize}
}
\end{enumerate}

\section*{shower}
{\large \color{blue}  showers  showering  showered  }
\subsection*{Explain}
\begin{enumerate}
\item countable noun \\
A \textbf{shower} is a device for washing yourself. It consists of a pipe which ends in a flat cover with a lot of holes in it so that water comes out in a spray.
 \textit{
	\begin{itemize}
	\item She heard him turn on the shower.
	\end{itemize}
}
\item countable noun \\
A \textbf{shower} is a small enclosed area containing a shower.
 \textit{
	\begin{itemize}
	\end{itemize}
}
\item countable noun \\
\textbf{The}  \textbf{showers} or \textbf{the}  \textbf{shower} in a place such as a sports centre is the area containing showers.
 \textit{
	\begin{itemize}
	\item The showers are a mess.
	\item We all stood in the women's shower.
	\end{itemize}
}
\item countable noun \\
If you have a \textbf{shower} , you wash yourself by standing under a spray of water from a shower.
 \textit{
	\begin{itemize}
	\item I think I'll have a shower before dinner.
	\item She took two showers a day.
	\end{itemize}
}
\item verb \\
If you \textbf{shower} , you wash yourself by standing under a spray of water from a shower.
 \textit{
	\begin{itemize}
	\item There wasn't time to shower or change clothes.
	\end{itemize}
}
\item countable noun \\
A \textbf{shower} is a short period of rain, especially light rain.
 \textit{
	\begin{itemize}
	\item There'll be bright or sunny spells and scattered showers this afternoon.
	\end{itemize}
}
\item countable noun \\
You can refer to a lot of things that are falling as a \textbf{shower}  \textbf{of} them.
 \textit{
	\begin{itemize}
	\item Showers of sparks flew in all directions.
	\item ...a shower of meteorites.
	\end{itemize}
}
\item verb \\
If you \textbf{are showered with} a lot of small objects or pieces, they are scattered over you.
 \textit{
	\begin{itemize}
	\item They were showered with rice in the traditional manner.
	\item The President was showered with glass.
	\end{itemize}
}
\item verb \\
If you \textbf{shower} a person \textbf{with} presents or kisses , you give them a lot of presents or kisses in a very generous and extravagant way.
 \textit{
	\begin{itemize}
	\item He showered her with emeralds and furs.
	\item Her parents showered her with kisses.
	\end{itemize}
}
\item countable noun \\
A \textbf{shower} is a party or celebration at which the guests bring gifts.
 \textit{
	\begin{itemize}
	\item ...a bridal shower.
	\item ...a baby shower.
	\end{itemize}
}
\item singular noun \\
If you refer to a group of people as a particular kind of \textbf{shower} , you disapprove of them.
 \textit{
	\begin{itemize}
	\item ...a shower of wasters.
	\end{itemize}
}
\end{enumerate}

\section*{spoil}
{\large \color{blue}  spoils  spoiling  spoiled  spoilt  }
\subsection*{Explain}
\begin{enumerate}
\item verb \\
If you \textbf{spoil} something, you prevent it from being successful or satisfactory .
 \textit{
	\begin{itemize}
	\item It's important not to let mistakes spoil your life.
	\item Peaceful summer evenings can be spoilt by mosquitoes.
	\end{itemize}
}
\item verb \\
If you \textbf{spoil} children, you give them everything they want or ask for. This is considered to have a bad  effect on a child's character.
 \textit{
	\begin{itemize}
	\item Grandparents are often tempted to spoil their grandchildren whenever they come to
visit.
	\end{itemize}
}
\item verb \\
If you \textbf{spoil}  \textbf{yourself} or \textbf{spoil} another person, you give yourself or them something nice as a treat or do something special for them.
 \textit{
	\begin{itemize}
	\item Spoil yourself with a new perfume this summer.
	\item Perhaps I could employ someone to iron her shirts, but I wanted to spoil her.
	\end{itemize}
}
\item verb \\
If food \textbf{spoils} or if it \textbf{is spoilt} , it is no longer fit to be eaten .
 \textit{
	\begin{itemize}
	\item We all know that fats spoil by becoming rancid.
	\item Some organisms are responsible for spoiling food and cause food poisoning.
	\item Some of my apples were spoilt last year by grubs inside the fruit.
	\item ...the potential health problems from spoiled food.
	\end{itemize}
}
\item verb \\
If someone \textbf{spoils} their vote , they write something illegal on their voting paper , usually as a protest about the election , and their vote is not accepted .
 \textit{
	\begin{itemize}
	\item They had broadcast calls for voters to spoil their ballot papers.
	\item The results showed that 7.2% of the voters cast blank or spoiled ballots.
	\end{itemize}
}
\item plural noun \\
The \textbf{spoils}  \textbf{of} something are things that people get as a result of winning a battle or of doing something successfully.
 \textit{
	\begin{itemize}
	\item True to military tradition, the victors are now treating themselves to the spoils
of war.
	\item Competing warlords and foreign powers scrambled for political spoils.
	\end{itemize}
}
\item  \\
 spoilt for choice/spoiled for choice \textit{
	\begin{itemize}
	\end{itemize}
}
\end{enumerate}

\section*{tutor}
{\large \color{blue}  tutors  tutoring  tutored  }
\subsection*{Explain}
\begin{enumerate}
\item countable noun \\
A \textbf{tutor} is a teacher at a British university or college. In some American universities or colleges, a \textbf{tutor} is a teacher of the lowest  rank .
 \textit{
	\begin{itemize}
	\item He is course tutor in archaeology at the University of Southampton.
	\item Liam surprised his tutors by twice failing a second year exam.
	\end{itemize}
}
\item countable noun \\
A \textbf{tutor} is someone who gives  private  lessons to one pupil or a very small group of pupils.
 \textit{
	\begin{itemize}
	\end{itemize}
}
\item verb \\
If someone \textbf{tutors} a person or a subject , they teach that person or subject.
 \textit{
	\begin{itemize}
	\item The old man was tutoring her in the stringed instruments.
	\item ...at the college where I tutored a two-day Introduction to Chairmaking course.
	\item I tutored in economics.
	\end{itemize}
}
\end{enumerate}

\section*{threaten}
{\large \color{blue}  threatens  threatening  threatened  }
\subsection*{Explain}
\begin{enumerate}
\item verb \\
If a person \textbf{threatens}  \textbf{to} do something unpleasant to you, or if they \textbf{threaten} you, they say or imply that they will do something unpleasant to you, especially if you do not do what they want .
 \textit{
	\begin{itemize}
	\item He said army officers had threatened to destroy the town.
	\item He tied her up and threatened her with a six-inch knife.
	\item If you threaten me or use any force, I shall inform the police.
	\end{itemize}
}
\item verb \\
If something or someone \textbf{threatens} a person or thing, they are likely to harm that person or thing.
 \textit{
	\begin{itemize}
	\item The newcomers directly threaten the livelihood of the established workers.
	\item The unity of our society is threatened by troublesome and restless minorities.
	\item 30 percent of reptiles, birds, and fish are currently threatened with extinction.
	\end{itemize}
}
\item verb \\
If something unpleasant \textbf{threatens}  \textbf{to}  happen , it seems likely to happen.
 \textit{
	\begin{itemize}
	\item The fighting is threatening to turn into full-scale war.
	\item Plants must be covered with a leaf-mould or similarly protected if frost threatens.
	\end{itemize}
}
\end{enumerate}

\section*{visit}
{\large \color{blue}  visits  visiting  visited  }
\subsection*{Explain}
\begin{enumerate}
\item verb \\
If you \textbf{visit} someone, you go to see them and spend time with them.
 \textbf{Visit} is also a noun .
 \textit{
	\begin{itemize}
	\item He wanted to visit his brother in Worcester.
	\item He was visited by an old friend from Iraq.
	\item Bill would visit on weekends.
	\item Helen had recently paid him a visit.
	\end{itemize}
}
\item verb \\
If you \textbf{visit} a place, you go there for a short time.
 \textbf{Visit} is also a noun.
 \textit{
	\begin{itemize}
	\item He'll be visiting four cities including Cagliari in Sardinia.
	\item Caroline visited all the big stores.
	\item ...a visiting truck driver.
	\item ...the Pope's visit to Canada.
	\item I paid a visit to my local print shop.
	\end{itemize}
}
\item verb \\
If you \textbf{visit} a website, you look at it.
 \textit{
	\begin{itemize}
	\item For details visit our website at www.cobuild.collins.co.uk.
	\end{itemize}
}
\item verb \\
If you \textbf{visit} a professional person such as a doctor or lawyer , you go and see them in order to get professional advice . If they \textbf{visit} you, they come to see you in order to give you professional advice.
 \textbf{Visit} is also a noun.
 \textit{
	\begin{itemize}
	\item If necessary, the patient can then visit his doctor for further advice.
	\item A doctor will visit you in your apartment.
	\item You may have regular home visits from a neonatal nurse.
	\end{itemize}
}
\item passive verb \\
If something very unpleasant  \textbf{is visited}  \textbf{upon} you, it happens to you.
 \textit{
	\begin{itemize}
	\item Violence is visited upon us every day.
	\item Death and suffering had been visited on thousands of innocents.
	\end{itemize}
}
\end{enumerate}

\section*{weep}
{\large \color{blue}  weeps  weeping  wept  }
\subsection*{Explain}
\begin{enumerate}
\item verb \\
If someone \textbf{weeps} , they cry .
 \textbf{Weep} is also a noun .
 \textit{
	\begin{itemize}
	\item She wanted to laugh and weep all at once.
	\item The weeping family hugged and comforted each other.
	\item She wept tears of joy.
	\item There are times when I sit down and have a good weep.
	\end{itemize}
}
\item verb \\
If a wound \textbf{weeps} , liquid or blood comes from it because it is not healing properly.
 \textit{
	\begin{itemize}
	\item In severe cases, the skin can crack and weep.
	\item ...little blisters which develop into weeping sores.
	\end{itemize}
}
\end{enumerate}

\section*{visitor}
{\large \color{blue}  visitors  }
\subsection*{Explain}
\begin{enumerate}
\item countable noun \\
A \textbf{visitor} is someone who is visiting a person or place.
 \textit{
	\begin{itemize}
	\item The other day we had some visitors from Switzerland.
	\item As a student I lived in Oxford, but was a frequent visitor to Belfast.
	\end{itemize}
}
\end{enumerate}

\section*{weigh}
{\large \color{blue}  weighs  weighing  weighed  }
\subsection*{Explain}
\begin{enumerate}
\item verb \\
If someone or something \textbf{weighs} a particular amount, this amount is how heavy they are.
 \textit{
	\begin{itemize}
	\item It weighs nearly 27 kilos (about 65 pounds).
	\item This little ball of gold weighs a quarter of an ounce.
	\item You always weigh less in the morning.
	\end{itemize}
}
\item verb \\
If you \textbf{weigh} something or someone, you measure how heavy they are.
 \textit{
	\begin{itemize}
	\item The scales can be used to weigh other items such as parcels.
	\end{itemize}
}
\item verb \\
If you \textbf{weigh} the facts about a situation , you consider them very carefully before you make a decision , especially by comparing the various facts involved.
 \textbf{Weigh up} means the same as weigh .
 \textit{
	\begin{itemize}
	\item She weighed her options.
	\item He is weighing the possibility of filing criminal charges against the doctor.
	\item She spoke very slowly, weighing what she would say.
	\item The company will be able to weigh up the environmental pros and cons of each site.
	\item You have to weigh up whether a human life is more important than an animal's life.
	\end{itemize}
}
\item verb \\
If you \textbf{weigh} your words, you think very carefully before you say something.
 \textit{
	\begin{itemize}
	\item He said the words very slowly, as if weighing each one of them.
	\end{itemize}
}
\item verb \\
If a problem  \textbf{weighs}  \textbf{on} you, it makes you worried or unhappy .
 \textit{
	\begin{itemize}
	\item The separation weighed on both of them.
	\item She knows how your brother's disappearance weighs upon you.
	\end{itemize}
}
\item verb \\
Something that \textbf{weighs} heavily in a situation has a strong  influence or important  effect on it.
 \textit{
	\begin{itemize}
	\item Current economic hardships weigh heavily in young women's decisions to find salaried
work.
	\item Human life weighed more with him than purity of policy.
	\item There are many factors weighing against the meeting happening.
	\end{itemize}
}
\end{enumerate}

\section*{absorb}
{\large \color{blue}  absorbs  absorbing  absorbed  }
\subsection*{Explain}
\begin{enumerate}
\item verb \\
If something \textbf{absorbs} a liquid, gas, or other substance, it soaks it up or takes it in.
 \textit{
	\begin{itemize}
	\item Plants absorb carbon dioxide from the air and moisture from the soil.
	\item Refined sugars are absorbed into the bloodstream very quickly.
	\end{itemize}
}
\item verb \\
If something \textbf{absorbs} light, heat , or another form of energy, it takes it in.
 \textit{
	\begin{itemize}
	\item The dark material absorbs light and warms up.
	\end{itemize}
}
\item verb \\
If a group \textbf{is absorbed}  \textbf{into} a larger group, it becomes part of the larger group.
 \textit{
	\begin{itemize}
	\item The Colonial Office was absorbed into the Foreign Office.
	\item ...an economy capable of absorbing thousands of immigrants.
	\end{itemize}
}
\item verb \\
If something \textbf{absorbs} a force or shock , it reduces its effect .
 \textit{
	\begin{itemize}
	\item ...footwear which does not absorb the impact of the foot striking the ground.
	\end{itemize}
}
\item verb \\
If a system or society  \textbf{absorbs} changes, effects, or costs , it is able to deal with them.
 \textit{
	\begin{itemize}
	\item The banks would be forced to absorb large losses.
	\item We can't absorb those costs.
	\end{itemize}
}
\item verb \\
If something \textbf{absorbs} something valuable such as money, space , or time, it uses up a great deal of it.
 \textit{
	\begin{itemize}
	\item It absorbed vast amounts of capital that could have been used for investment.
	\item It might help if campaigning didn't absorb so much time and money.
	\end{itemize}
}
\item verb \\
If you \textbf{absorb}  information , you learn and understand it.
 \textit{
	\begin{itemize}
	\item Too often he only absorbs half the information in the manual.
	\item We closed our offices at 2:00 p.m. to give employees time to absorb the bad news.
	\end{itemize}
}
\item verb \\
If something \textbf{absorbs} you, it interests you a great deal and takes up all your attention and energy.
 \textit{
	\begin{itemize}
	\item ...a second career which absorbed her more completely than her acting ever had.
	\end{itemize}
}
\end{enumerate}

\section*{accompany}
{\large \color{blue}  accompanies  accompanying  accompanied  }
\subsection*{Explain}
\begin{enumerate}
\item verb \\
If you \textbf{accompany} someone, you go somewhere with them.
 \textit{
	\begin{itemize}
	\item Ken agreed to accompany me on a trip to Africa.
	\item She was accompanied by her younger brother.
	\item The Prime Minister, accompanied by the governor, led the President up to the house.
	\end{itemize}
}
\item verb \\
If one thing \textbf{accompanies} another, it happens or exists at the same time, or as a result of it.
 \textit{
	\begin{itemize}
	\item This volume of essays was designed to accompany an exhibition in Cologne.
	\item The proposal was instantly voted through, accompanied by enthusiastic applause.
	\item Perhaps the accompanying illustration will explain it.
	\end{itemize}
}
\item verb \\
If you \textbf{accompany} a singer or a musician , you play one part of a piece of music while they sing or play the main  tune .
 \textit{
	\begin{itemize}
	\item He sang and Alice accompanied him on the piano.
	\item She was accompanied on the guitar by her brother Massimo.
	\end{itemize}
}
\end{enumerate}

\section*{bake}
{\large \color{blue}  bakes  baking  baked  }
\subsection*{Explain}
\begin{enumerate}
\item verb \\
If you \textbf{bake} , you spend some time preparing and mixing  together  ingredients to make bread, cakes, pies , or other food which is cooked in the oven.
 \textit{
	\begin{itemize}
	\item I love to bake.
	\end{itemize}
}
\item verb \\
When a cake or bread \textbf{bakes} or when you \textbf{bake} it, it cooks in the oven without any extra  liquid or fat .
 \textit{
	\begin{itemize}
	\item Bake the cake for 35 to 50 minutes.
	\item The batter rises as it bakes.
	\item ...freshly baked bread.
	\end{itemize}
}
\item verb \\
If places or people become extremely hot because the sun is shining very strongly, you can say that they \textbf{bake} .
 \textit{
	\begin{itemize}
	\item If you closed the windows, you baked.
	\item Britain bakes in a Mediterranean heatwave.
	\end{itemize}
}
\item countable noun \\
A vegetable or fish  \textbf{bake} is a dish that is made by chopping up and mixing together a number of ingredients and cooking them in the oven so that
they form a fairly dry solid  mass .
 \textit{
	\begin{itemize}
	\item ...an aubergine bake.
	\end{itemize}
}
\end{enumerate}

\section*{add}
{\large \color{blue}  adds  adding  added  }
\subsection*{Explain}
\begin{enumerate}
\item verb \\
If you \textbf{add} one thing \textbf{to} another, you put it in or on the other thing, to increase, complete , or improve it.
 \textit{
	\begin{itemize}
	\item Add the grated cheese to the sauce.
	\item Since 1908, chlorine has been added to drinking water.
	\item He wants to add a huge sports complex to Binfield Manor.
	\end{itemize}
}
\item verb \\
If you \textbf{add} numbers or amounts \textbf{together} , you calculate their total .
 \textbf{Add up} means the same as add .
 \textit{
	\begin{itemize}
	\item Banks add all the interest and other charges together.
	\item Two and three added together are five.
	\item More than a quarter of seven year-olds cannot add up properly.
	\item We just added all the numbers up and divided one by the other.
	\item He said the numbers simply did not add up.
	\end{itemize}
}
\item verb \\
If one thing \textbf{adds}  \textbf{to} another, it makes the other thing greater in degree or amount.
 \textit{
	\begin{itemize}
	\item This latest incident will add to the pressure on the government.
	\item Smiles, nods, and cheerful faces added to the general gaiety.
	\end{itemize}
}
\item verb \\
To \textbf{add} a particular quality \textbf{to} something means to cause it to have that quality.
 \textit{
	\begin{itemize}
	\item The generous amount of garlic adds flavour.
	\item Pictures add interest to plain painted walls.
	\end{itemize}
}
\item verb \\
If you \textbf{add} something when you are speaking, you say something more.
 \textit{
	\begin{itemize}
	\item 'You can tell that he is extremely embarrassed,' Mr Brigden added.
	\item The President agreed, adding that he hoped for a peaceful solution.
	\item Hunt added his congratulations, saying 'Nigel has made a cracking job of it'.
	\end{itemize}
}
\item  \\
 added to this/added to that \textit{
	\begin{itemize}
	\end{itemize}
}
\end{enumerate}

\section*{bring}
{\large \color{blue}  brings  bringing  brought  }
\subsection*{Explain}
\begin{enumerate}
\item verb \\
If you \textbf{bring} someone or something with you when you come to a place, they come with you or you
have them with you.
 \textit{
	\begin{itemize}
	\item Remember to bring an apron or an old shirt to protect your clothes.
	\item Come to my party and bring a friend with you.
	\item Someone went upstairs and brought down a huge kettle.
	\item My father brought home a book for me.
	\end{itemize}
}
\item verb \\
If you \textbf{bring} something somewhere , you move it there.
 \textit{
	\begin{itemize}
	\item Reaching into her pocket, she brought out a key.
	\item Her mother brought her hands up to her face.
	\end{itemize}
}
\item verb \\
If you \textbf{bring} something that someone wants or needs , you get it for them or carry it to them.
 \textit{
	\begin{itemize}
	\item He went and poured a brandy for Dena and brought it to her.
	\item The stewardess kindly brought me a blanket.
	\end{itemize}
}
\item verb \\
To \textbf{bring} something or someone to a place or position means to cause them to come to the place or move into that position.
 \textit{
	\begin{itemize}
	\item I told you about what brought me here.
	\item ... an emotional acceptance speech that brought the crowd to its feet.
	\item She survived a gas blast which brought her home crashing down on top of her.
	\end{itemize}
}
\item verb \\
If you \textbf{bring} something new  \textbf{to} a place or group of people, you introduce it to that place or cause those people to hear or know about it.
 \textit{
	\begin{itemize}
	\item ...a brave reporter who had risked death to bring the story to the world.
	\item ...the drive to bring art to the public.
	\end{itemize}
}
\item verb \\
To \textbf{bring} someone or something into a particular state or condition means to cause them to be in that state or condition.
 \textit{
	\begin{itemize}
	\item He brought the car to a stop in front of the square.
	\item His work as a historian brought him into conflict with the political establishment.
	\item The incident brings the total of people killed to fifteen.
	\item They have brought down income taxes.
	\end{itemize}
}
\item verb \\
If something \textbf{brings} a particular feeling , situation , or quality, it makes people experience it or have it.
 \textit{
	\begin{itemize}
	\item We should be deeply proud of their efforts to bring peace to these warzones.
	\item Banks have brought trouble on themselves by lending rashly.
	\item He brought to the job not just considerable experience but passionate enthusiasm.
	\item Her three children brought her joy.
	\end{itemize}
}
\item verb \\
If a period of time \textbf{brings} a particular thing, it happens during that time.
 \textit{
	\begin{itemize}
	\item For Sandro, the new year brought disaster.
	\item We don't know what the future will bring.
	\end{itemize}
}
\item verb \\
If you \textbf{bring} a legal  action  \textbf{against} someone or \textbf{bring} them \textbf{to}  trial , you officially  accuse them of doing something illegal .
 \textit{
	\begin{itemize}
	\item He campaigned relentlessly to bring charges of corruption against members of the
party.
	\item The ship's captain and crew may be brought to trial and even sent to prison.
	\end{itemize}
}
\item verb \\
If a television or radio  programme  \textbf{is brought}  \textbf{to} you \textbf{by} an organization , they make it, broadcast it, or pay for it to be made or broadcast.
 \textit{
	\begin{itemize}
	\item You're listening to Science in Action, brought to you by the BBC World Service.
	\item We'll be bringing you all the details of the day's events.
	\end{itemize}
}
\item verb \\
When you are talking , you can say that something \textbf{brings} you \textbf{to} a particular point in order to indicate that you have now  reached that point and are going to talk about a new subject .
 \textit{
	\begin{itemize}
	\item Which brings me to a delicate matter I should like to raise.
	\item And that brings us to the end of this special report from Germany.
	\end{itemize}
}
\item verb \\
If you cannot \textbf{bring}  \textbf{yourself to} do something, you cannot do it because you find it too upsetting , embarrassing , or disgusting .
 \textit{
	\begin{itemize}
	\item It is very tragic and I am afraid I just cannot bring myself to talk about it.
	\end{itemize}
}
\end{enumerate}

\section*{aggravate}
{\large \color{blue}  aggravates  aggravating  aggravated  }
\subsection*{Explain}
\begin{enumerate}
\item verb \\
If someone or something \textbf{aggravates} a situation, they make it worse.
 \textit{
	\begin{itemize}
	\item Stress and lack of sleep can aggravate the situation.
	\item He would only aggravate the injury by rubbing it.
	\end{itemize}
}
\item verb \\
If someone or something \textbf{aggravates} you, they make you annoyed.
 \textit{
	\begin{itemize}
	\item What aggravates you most about this country?
	\end{itemize}
}
\end{enumerate}

\section*{concern}
{\large \color{blue}  concerns  concerning  concerned  }
\subsection*{Explain}
\begin{enumerate}
\item uncountable noun \\
\textbf{Concern} is worry about a situation .
 \textit{
	\begin{itemize}
	\item The group has expressed concern about reports of political violence.
	\item The move follows growing public concern over the spread of the disease.
	\item As the militants gather, there is concern that the protest might again run out of
control.
	\item There is no cause for concern.
	\end{itemize}
}
\item verb \\
If something \textbf{concerns} you, it worries you.
 \textit{
	\begin{itemize}
	\item The growing number of people seeking refuge in Thailand is beginning to concern Western
aid agencies.
	\item It concerned her that Bess was developing a crush on Max.
	\end{itemize}
}
\item countable noun \\
A \textbf{concern} is a fact or situation that worries you.
 \textit{
	\begin{itemize}
	\item His concern was that people would know that he was responsible.
	\item Unemployment was the electorate's main concern.
	\end{itemize}
}
\item variable noun \\
Someone's \textbf{concern}  \textbf{with} something is their feeling that it is important.
 \textit{
	\begin{itemize}
	\item ...a story that illustrates how dangerous excessive concern with safety can be.
	\end{itemize}
}
\item countable noun \\
Someone's \textbf{concerns} are the things that they consider to be important.
 \textit{
	\begin{itemize}
	\item Feminism must address issues beyond the concerns of middle-class whites.
	\end{itemize}
}
\item variable noun \\
\textbf{Concern}  \textbf{for} someone is a feeling that you want them to be happy , safe , and well . If you do something out of \textbf{concern}  \textbf{for} someone, you do it because you want them to be happy, safe, and well.
 \textit{
	\begin{itemize}
	\item Without her care and concern, he had no chance at all.
	\item He had only gone along out of concern for his two grandsons.
	\end{itemize}
}
\item verb \\
If you \textbf{concern}  \textbf{yourself with} something, you give it attention because you think that it is important.
 \textit{
	\begin{itemize}
	\item I didn't concern myself with politics.
	\item He would concern himself solely with the plight of the hostages.
	\end{itemize}
}
\item verb \\
If something such as a book or a piece of information  \textbf{concerns} a particular subject, it is about that subject.
 \textit{
	\begin{itemize}
	\item The bulk of the book concerns Sandy's two middle-aged children.
	\item Chapter 2 concerns itself with the methodological difficulties.
	\end{itemize}
}
\item verb \\
If a situation, event, or activity \textbf{concerns} you, it affects or involves you.
 \textit{
	\begin{itemize}
	\item It was just a little unfinished business from my past, and it doesn't concern you
at all.
	\end{itemize}
}
\item singular noun \\
If a situation or problem is your \textbf{concern} , it is something that you have a duty or responsibility to be involved with.
 \textit{
	\begin{itemize}
	\item The technical aspects were the concern of the Army.
	\item I would be glad to get rid of them myself. But that is not our concern.
	\end{itemize}
}
\item countable noun \\
You can refer to a company or business as a \textbf{concern} , usually when you are describing what type of company or business it is.
 \textit{
	\begin{itemize}
	\item If not a large concern, Queensbury Nursery was at least a successful one.
	\end{itemize}
}
\item  \\
 as far as I am concerned \textit{
	\begin{itemize}
	\end{itemize}
}
\item  \\
 as far as sth is concerned/where sth is concerned \textit{
	\begin{itemize}
	\end{itemize}
}
\item  \\
 going concern \textit{
	\begin{itemize}
	\end{itemize}
}
\item  \\
 of concern \textit{
	\begin{itemize}
	\end{itemize}
}
\item  \\
 of concern \textit{
	\begin{itemize}
	\end{itemize}
}
\end{enumerate}

\section*{amuse}
{\large \color{blue}  amuses  amusing  amused  }
\subsection*{Explain}
\begin{enumerate}
\item verb \\
If something \textbf{amuses} you, it makes you want to laugh or smile.
 \textit{
	\begin{itemize}
	\item The thought seemed to amuse him.
	\item Their antics never fail to amuse.
	\end{itemize}
}
\item verb \\
If you \textbf{amuse}  \textbf{yourself} , you do something in order to pass the time and not become bored .
 \textit{
	\begin{itemize}
	\item I need distractions. I need to amuse myself so I won't keep thinking about things.
	\item Put a selection of baby toys in his cot to amuse him if he wakes early.
	\end{itemize}
}
\end{enumerate}

\section*{consult}
{\large \color{blue}  consults  consulting  consulted  }
\subsection*{Explain}
\begin{enumerate}
\item verb \\
If you \textbf{consult} an expert or someone senior to you or \textbf{consult}  \textbf{with} them, you ask them for their opinion and advice about what you should do or their permission to do something.
 \textit{
	\begin{itemize}
	\item Consult your doctor about how much exercise you should attempt.
	\item He needed to consult with an attorney.
	\item If you are in any doubt, consult a financial adviser.
	\end{itemize}
}
\item verb \\
If a person or group of people \textbf{consults}  \textbf{with} other people or \textbf{consults} them, they talk and exchange  ideas and opinions about what they might  decide to do.
 \textit{
	\begin{itemize}
	\item After consulting with her daughter and manager she decided to take on the part, on
her terms.
	\item The two countries will have to consult their allies.
	\item The umpires consulted quickly.
	\end{itemize}
}
\item verb \\
If you \textbf{consult} a book or a map , you look in it or look at it in order to find some information.
 \textit{
	\begin{itemize}
	\item Consult the chart on page 44 for the correct cooking times.
	\item He had to consult a pocket dictionary.
	\end{itemize}
}
\end{enumerate}

\section*{answer}
{\large \color{blue}  answers  answering  answered  }
\subsection*{Explain}
\begin{enumerate}
\item verb \\
When you \textbf{answer} someone who has asked you something, you say something back to them.
 \textit{
	\begin{itemize}
	\item I knew Ben was lying when he answered me.
	\item Just answer the question.
	\item He paused before answering.
	\item 'When?' asked Alba, 'Tonight', answered Tom.
	\item Williams answered that he had no specific proposals yet.
	\end{itemize}
}
\item countable noun \\
An \textbf{answer} is something that you say when you answer someone.
 \textit{
	\begin{itemize}
	\item Without waiting for an answer, he turned and went in through the door.
	\item I don't quite know what to say in answer to your question.
	\end{itemize}
}
\item  \\
 not to take no for an answer \textit{
	\begin{itemize}
	\end{itemize}
}
\item verb \\
If you \textbf{answer} a letter or advertisement , you write to the person who wrote it.
 \textit{
	\begin{itemize}
	\item Did he answer your letter?
	\item She answered an advert for a job as a cook.
	\end{itemize}
}
\item countable noun \\
An \textbf{answer} is a letter that you write to someone who has written to you.
 \textit{
	\begin{itemize}
	\item I wrote to him but I never had an answer back.
	\item She wrote to his secretary in answer to his letter of the day before.
	\end{itemize}
}
\item verb \\
When you \textbf{answer} the telephone , you pick it up when it rings . When you \textbf{answer} the door , you open it when you hear a knock or the bell .
 \textbf{Answer} is also a noun .
 \textit{
	\begin{itemize}
	\item She answered her phone on the first ring.
	\item A middle-aged woman answered the door.
	\item I knocked at the front door and there was no answer.
	\end{itemize}
}
\item countable noun \\
An \textbf{answer}  \textbf{to} a problem is a solution to it.
 \textit{
	\begin{itemize}
	\item There are no easy answers to the problems facing the economy.
	\item Prison is not the answer for most young offenders.
	\item Legislation is only part of the answer.
	\end{itemize}
}
\item countable noun \\
Someone's \textbf{answer} to a question in a test or quiz is what they write or say in an attempt to give the facts that are asked for. The \textbf{answer} to a question is the fact that was asked for.
 \textit{
	\begin{itemize}
	\item Simply marking an answer wrong will not help the pupil to get future examples correct.
	\item Below are printed the answers to the Brain of Soccer quiz.
	\end{itemize}
}
\item verb \\
When you \textbf{answer} a question in a test or quiz, you write or say something in an attempt to give the
facts that are asked for.
 \textit{
	\begin{itemize}
	\item To obtain her degree, she answered 81 questions over 10 papers.
	\end{itemize}
}
\item countable noun \\
Your \textbf{answer}  \textbf{to} something that someone has said or done is what you say or do in response to it or in defence of yourself.
 \textit{
	\begin{itemize}
	\item In answer to speculation that she wouldn't finish the race, she boldly declared her
intention of winning it.
	\end{itemize}
}
\item verb \\
If you \textbf{answer} something that someone has said or done, you respond to it.
 \textit{
	\begin{itemize}
	\item He answered her smile with one of his own.
	\item That statement seemed designed to answer criticism of allied bombing missions.
	\end{itemize}
}
\item singular noun \\
If you say that something is a place's \textbf{answer}  \textbf{to} a famous thing, you mean that the first thing is the equivalent of the second in that place.
 \textit{
	\begin{itemize}
	\item Cachaca is Brazil's answer to tequila.
	\end{itemize}
}
\item verb \\
If something \textbf{answers} a need or purpose, it satisfies it, because it has the right qualities.
 \textit{
	\begin{itemize}
	\item We provide specially designed shopping trolleys to answer the needs of parents with
young children.
	\end{itemize}
}
\item verb \\
If someone or something \textbf{answers} a particular description or \textbf{answers to} it, they have the characteristics described .
 \textit{
	\begin{itemize}
	\item Two men answering the description of the suspects tried to enter Switzerland.
	\item The Japanese never built any aircraft remotely answering to this description.
	\end{itemize}
}
\end{enumerate}

\section*{contend}
{\large \color{blue}  contends  contending  contended  }
\subsection*{Explain}
\begin{enumerate}
\item verb \\
If you have to \textbf{contend with} a problem or difficulty , you have to deal with it or overcome it.
 \textit{
	\begin{itemize}
	\item It is time, once again, to contend with racism.
	\item American businesses could soon have a new kind of lawsuit to contend with.
	\end{itemize}
}
\item verb \\
If you \textbf{contend}  \textbf{that} something is true , you state or argue that it is true.
 \textit{
	\begin{itemize}
	\item The government contends that he is fundamentalist.
	\item 'You were just looking,' contends Samantha. 'I was the one doing all the work.'
	\end{itemize}
}
\item verb \\
If you \textbf{contend}  \textbf{with} someone \textbf{for} something such as power, you compete with them to try to get it.
 \textit{
	\begin{itemize}
	\item ...the two main groups contending for power.
	\item ...with 10 U.K. construction yards contending with rivals from Norway, Holland, Italy
and Spain.
	\item ...a binding political settlement between the contending parties.
	\end{itemize}
}
\end{enumerate}

\section*{appoint}
{\large \color{blue}  appoints  appointing  appointed  }
\subsection*{Explain}
\begin{enumerate}
\item verb \\
If you \textbf{appoint} someone \textbf{to} a job or official position, you formally choose them for it.
 \textit{
	\begin{itemize}
	\item It made sense to appoint a banker to this job.
	\item The commission appointed a special investigator to conduct its own inquiry.
	\item The Prime Minister has appointed a civilian as defence minister.
	\item She was appointed a U.S. delegate to the United Nations.
	\end{itemize}
}
\end{enumerate}

\section*{continue}
{\large \color{blue}  continues  continuing  continued  }
\subsection*{Explain}
\begin{enumerate}
\item verb \\
If someone or something \textbf{continues}  \textbf{to} do something, they keep doing it and do not stop .
 \textit{
	\begin{itemize}
	\item I hope they continue to fight for equal justice after I'm gone.
	\item Interest rates continue to fall.
	\item They are determined to continue working when they reach retirement age.
	\item There is no reason why you should not continue with any sport or activity you already
enjoy.
	\end{itemize}
}
\item verb \\
If something \textbf{continues} or if you \textbf{continue} it, it does not stop happening .
 \textit{
	\begin{itemize}
	\item He insisted that the conflict would continue until conditions were met for a ceasefire.
	\item But as the investigation continued, the plot began to thicken.
	\item Outside the building people continue their vigil, huddling around bonfires.
	\item ...the continued existence of a species.
	\end{itemize}
}
\item verb \\
If you \textbf{continue} with something, you start doing it again after a break or interruption.
 \textit{
	\begin{itemize}
	\item I went up to my room to continue with my packing.
	\item She looked up for a moment, then continued drawing.
	\end{itemize}
}
\item verb \\
If something \textbf{continues} or if you \textbf{continue} it, it starts again after a break or interruption.
 \textit{
	\begin{itemize}
	\item He denies 18 charges. The trial continues today.
	\item Once, he did dive for cover but he soon reappeared and continued his activities.
	\end{itemize}
}
\item verb \\
If you \textbf{continue} , you begin  speaking again after a pause or interruption.
 \textit{
	\begin{itemize}
	\item 'You have no right to intimidate this man,' Alison continued.
	\item Tony drank some coffee before he continued.
	\item Please continue.
	\end{itemize}
}
\item verb \\
If you \textbf{continue}  \textbf{as} something or \textbf{continue} in a particular state, you remain in a particular job or state.
 \textit{
	\begin{itemize}
	\item He had hoped to continue as a full-time career officer.
	\item For ten days I continued in this state.
	\end{itemize}
}
\item verb \\
If you \textbf{continue} in a particular direction , you keep walking or travelling in that direction.
 \textit{
	\begin{itemize}
	\item He continued rapidly up the path, not pausing until he neared the house.
	\end{itemize}
}
\item verb \\
If a road or path  \textbf{continues}  somewhere , it goes there after the place you have mentioned .
 \textit{
	\begin{itemize}
	\item The main road continues towards Viterbo before turning right to Bolsena.
	\end{itemize}
}
\end{enumerate}

\section*{ask}
{\large \color{blue}  asks  asking  asked  }
\subsection*{Explain}
\begin{enumerate}
\item verb \\
If you \textbf{ask} someone something, you say something to them in the form of a question because you want to know the answer.
 \textit{
	\begin{itemize}
	\item 'How is Frank?' he asked.
	\item I asked him his name.
	\item I wasn't the only one asking questions.
	\item She asked me if I'd enjoyed my dinner.
	\item If Daniel asks what happened in court we will tell him.
	\item You will have to ask David about that.
	\item 'I'm afraid to ask what it cost.'—'Then don't ask.'
	\end{itemize}
}
\item verb \\
If you \textbf{ask} someone \textbf{to} do something, you tell them that you want them to do it.
 \textit{
	\begin{itemize}
	\item We had to ask him to leave.
	\end{itemize}
}
\item verb \\
If you \textbf{ask}  \textbf{to} do something, you tell someone that you want to do it.
 \textit{
	\begin{itemize}
	\item I asked to see the Director.
	\end{itemize}
}
\item verb \\
If you \textbf{ask for} something, you say that you would like it.
 \textit{
	\begin{itemize}
	\item I decided to go to the next house and ask for food.
	\item Who asked for your opinion?
	\end{itemize}
}
\item verb \\
If you \textbf{ask for} someone, you say that you would like to speak to them.
 \textit{
	\begin{itemize}
	\item There's a man at the gate asking for you.
	\end{itemize}
}
\item verb \\
If you \textbf{ask} someone's permission , opinion , or forgiveness , you try to obtain it by putting a request to them.
 \textit{
	\begin{itemize}
	\item Please ask permission from whoever pays the phone bill before making your call.
	\end{itemize}
}
\item verb \\
If you \textbf{ask} someone \textbf{to} an event or place, you invite them to go there.
 \textit{
	\begin{itemize}
	\item Couldn't you ask Jon to the party?
	\item She asked me back to her house.
	\end{itemize}
}
\item verb \\
If someone \textbf{is asking} a particular  price  \textbf{for} something, they are selling it for that price.
 \textit{
	\begin{itemize}
	\item Mr Pantelaras was asking £6,000 for his collection.
	\end{itemize}
}
\item  \\
 don't ask me \textit{
	\begin{itemize}
	\end{itemize}
}
\item  \\
 for the asking \textit{
	\begin{itemize}
	\end{itemize}
}
\item  \\
 I ask you \textit{
	\begin{itemize}
	\end{itemize}
}
\item  \\
 may I ask \textit{
	\begin{itemize}
	\end{itemize}
}
\item  \\
 if you ask me \textit{
	\begin{itemize}
	\end{itemize}
}
\item  \\
 be asking for trouble/be asking for it \textit{
	\begin{itemize}
	\end{itemize}
}
\end{enumerate}

\section*{correspond}
{\large \color{blue}  corresponds  corresponding  corresponded  }
\subsection*{Explain}
\begin{enumerate}
\item verb \\
If one thing \textbf{corresponds}  \textbf{to} another, there is a close similarity or connection between them. You can  also  say that two things \textbf{correspond} .
 \textit{
	\begin{itemize}
	\item Racegoers will be given a number which will correspond to a horse running in a race.
	\item An increase in car travel corresponds with a drop in cycle mileage.
	\item The two maps of London correspond closely.
	\item Her expression is concerned but her body language does not correspond.
	\end{itemize}
}
\item verb \\
If you \textbf{correspond}  \textbf{with} someone, you write letters to them. You can also say that two people \textbf{correspond} .
 \textit{
	\begin{itemize}
	\item She still corresponds with American friends she met in Majorca nine years ago.
	\item We corresponded regularly.
	\end{itemize}
}
\end{enumerate}

\section*{assimilate}
{\large \color{blue}  assimilates  assimilating  assimilated  }
\subsection*{Explain}
\begin{enumerate}
\item verb \\
When people such as immigrants  \textbf{assimilate}  \textbf{into} a community or when that community \textbf{assimilates} them, they become an accepted part of it.
 \textit{
	\begin{itemize}
	\item There is every sign that new Asian-Americans are just as willing to assimilate.
	\item His family tried to assimilate into the white and Hispanic communities.
	\item The Vietnamese are trying to assimilate themselves and become Americans.
	\item French Jews generally had been assimilated into the nation's culture.
	\end{itemize}
}
\item verb \\
If you \textbf{assimilate}  new  ideas , techniques , or information, you learn them or adopt them.
 \textit{
	\begin{itemize}
	\item I was speechless, still trying to assimilate the enormity of what he'd told me.
	\end{itemize}
}
\end{enumerate}

\section*{degenerate}
{\large \color{blue}  degenerates  degenerating  degenerated  }
\subsection*{Explain}
\begin{enumerate}
\item verb \\
If you say that someone or something \textbf{degenerates} , you mean that they become worse in some way, for example  weaker , lower in quality, or more dangerous .
 \textit{
	\begin{itemize}
	\item Inactivity can make your joints stiff, and the bones may begin to degenerate.
	\item From then on the whole tone of the campaign began to degenerate.
	\item ...a very serious humanitarian crisis which could degenerate into a catastrophe.
	\end{itemize}
}
\item adjective \\
If you describe a person or their behaviour as \textbf{degenerate} , you disapprove of them because you think they have low  standards of behaviour or morality .
 \textit{
	\begin{itemize}
	\item ...a group of degenerate computer hackers.
	\item ...the degenerate attitudes he found among some of his fellow officers.
	\end{itemize}
}
\item countable noun \\
If you refer to someone as a \textbf{degenerate} , you disapprove of them because you think they have low standards of behaviour or
morality.
 \textit{
	\begin{itemize}
	\end{itemize}
}
\end{enumerate}

\section*{assist}
{\large \color{blue}  assists  assisting  assisted  }
\subsection*{Explain}
\begin{enumerate}
\item verb \\
If you \textbf{assist} someone, you help them to do a job or task by doing part of the work for them.
 \textit{
	\begin{itemize}
	\item Julia was assisting him to prepare his speech.
	\item The family decided to assist me with my chores.
	\item Dr Amid was assisted by a young Asian nurse.
	\end{itemize}
}
\item verb \\
If you \textbf{assist} someone, you give them information , advice , or money .
 \textit{
	\begin{itemize}
	\item The public is urgently requested to assist police in tracing this man.
	\item Foreign Office officials assisted with transport and finance problems.
	\item The Authority will provide a welfare worker to assist you.
	\end{itemize}
}
\item verb \\
If something \textbf{assists}  \textbf{in} doing a task, it makes the task easier to do.
 \textit{
	\begin{itemize}
	\item ...a chemical that assists in the manufacture of proteins.
	\item Here are some good sources of information to assist you in making the best selection.
	\item Salvage operations have been greatly assisted by the good weather conditions.
	\end{itemize}
}
\end{enumerate}

\section*{dump}
{\large \color{blue}  dumps  dumping  dumped  }
\subsection*{Explain}
\begin{enumerate}
\item verb \\
If you \textbf{dump} something somewhere , you put it or unload it there quickly and carelessly.
 \textit{
	\begin{itemize}
	\item We dumped our bags at the nearby Grand Hotel and hurried towards the market.
	\item He got my haversack from the cab and dumped it at my feet.
	\end{itemize}
}
\item verb \\
If something \textbf{is dumped} somewhere, it is put or left there because it is no longer wanted or needed .
 \textit{
	\begin{itemize}
	\item The getaway car was dumped near a motorway tunnel.
	\item A million tonnes of untreated sewage is dumped into the sea.
	\item The government declared that it did not dump radioactive waste at sea.
	\end{itemize}
}
\item countable noun \\
A \textbf{dump} is a place where rubbish is left, for example on open ground outside a town.
 \textit{
	\begin{itemize}
	\item ...companies that bring their rubbish straight to the dump.
	\item The walled garden was used as a dump.
	\end{itemize}
}
\item countable noun \\
If you say that a place is a \textbf{dump} , you think it is ugly and unpleasant to live in or visit .
 \textit{
	\begin{itemize}
	\item 'What a dump!' Christabel said, standing in the doorway of the youth hostel.
	\end{itemize}
}
\item countable noun \\
A \textbf{dump} is a place where an army stores food and weapons temporarily while it is stationed in a particular place.
 \textit{
	\begin{itemize}
	\end{itemize}
}
\item verb \\
To \textbf{dump} something such as an idea, policy , or practice means to stop supporting or using it.
 \textit{
	\begin{itemize}
	\item Ministers believed it was vital to dump the poll tax before the election.
	\end{itemize}
}
\item verb \\
If a firm or company \textbf{dumps} goods, it sells large quantities of them at prices far below their real value, usually in another country, in order to gain a bigger market share or to keep prices high in the home market.
 \textit{
	\begin{itemize}
	\item It produces more than it needs, then dumps its surplus onto the world market.
	\end{itemize}
}
\item verb \\
If you \textbf{dump} someone, you end your relationship with them.
 \textit{
	\begin{itemize}
	\item I thought he was going to dump me for another girl.
	\item She was dumped by her long-term lover after five years.
	\item ...parents who dump the kids in the play area and go off elsewhere.
	\end{itemize}
}
\item verb \\
If you say that a parent  \textbf{dumps} a child with someone, you are criticizing the parent for leaving the child to be looked after by that person.
 \textit{
	\begin{itemize}
	\item I was sometimes dumped with my grandmother or left with highly unsuitable au pairs.
	\item He can't cope and dumps his two teenage boys on them to be looked after.
	\end{itemize}
}
\item verb \\
To \textbf{dump} computer data or memory means to copy it from one storage system onto another, such as from disk to magnetic tape.
 \textit{
	\begin{itemize}
	\item All the data is then dumped into the main computer.
	\end{itemize}
}
\item countable noun \\
A \textbf{dump} is a list of the data that is stored in a computer's memory at a particular time. \textbf{Dumps} are often used by computer programmers to find out what is causing a problem with a program .
 \textit{
	\begin{itemize}
	\item ...a screen dump.
	\end{itemize}
}
\item verb \\
If someone \textbf{dumps on} you, they treat you very badly and unfairly.
 \textit{
	\begin{itemize}
	\item He was a nice guy, Mona. He didn't dump on me.
	\end{itemize}
}
\item  \\
 down in the dumps \textit{
	\begin{itemize}
	\end{itemize}
}
\end{enumerate}

\section*{attain}
{\large \color{blue}  attains  attaining  attained  }
\subsection*{Explain}
\begin{enumerate}
\item verb \\
If you \textbf{attain} something, you gain it or achieve it, often after a lot of effort.
 \textit{
	\begin{itemize}
	\item Jim is halfway to attaining his pilot's licence.
	\end{itemize}
}
\item verb \\
If you \textbf{attain} a particular state or condition, you may reach it as a result of natural  development or work hard to attain this state.
 \textit{
	\begin{itemize}
	\item ...attaining a state of calmness and confidence.
	\end{itemize}
}
\end{enumerate}

\section*{exhibit}
{\large \color{blue}  exhibits  exhibiting  exhibited  }
\subsection*{Explain}
\begin{enumerate}
\item verb \\
If someone or something shows a particular  quality , feeling , or type of behaviour , you can  say that they \textbf{exhibit} it.
 \textit{
	\begin{itemize}
	\item He has exhibited symptoms of anxiety and overwhelming worry.
	\item Two cats or more in one house will also exhibit territorial behaviour.
	\item The economy continued to exhibit signs of decline in September.
	\end{itemize}
}
\item verb \\
When a painting , sculpture , or object of interest \textbf{is exhibited} , it is put in a public place such as a museum or art  gallery so that people can come to look at it. You can also say that animals \textbf{are exhibited} in a zoo .
 \textit{
	\begin{itemize}
	\item His work was exhibited in the best galleries in America, Europe and Asia.
	\item ...a massive elephant exhibited by London Zoo in the late 19th Century.
	\end{itemize}
}
\item verb \\
When artists  \textbf{exhibit} , they show their work in public.
 \textit{
	\begin{itemize}
	\item By 1936 she was exhibiting at the Royal Academy.
	\end{itemize}
}
\item countable noun \\
An \textbf{exhibit} is a painting, sculpture, or object of interest that is displayed to the public in
a museum or art gallery.
 \textit{
	\begin{itemize}
	\item Shona showed me round the exhibits.
	\end{itemize}
}
\item countable noun \\
An \textbf{exhibit} is a public display of paintings, sculpture, or objects of interest, for example in a museum or art gallery.
 \textit{
	\begin{itemize}
	\item ...an exhibit at the Metropolitan Museum of Art.
	\end{itemize}
}
\item countable noun \\
An \textbf{exhibit} is an object that a lawyer shows in court as evidence in a legal  case .
 \textit{
	\begin{itemize}
	\end{itemize}
}
\item verb \\
If you say that someone \textbf{exhibits} something, you mean that they are showing it openly or publicly in order to be admired , noticed , or believed .
 \textit{
	\begin{itemize}
	\item He seems to want to exhibit his shame.
	\end{itemize}
}
\end{enumerate}

\section*{attempt}
{\large \color{blue}  attempts  attempting  attempted  }
\subsection*{Explain}
\begin{enumerate}
\item verb \\
If you \textbf{attempt}  \textbf{to} do something, especially something difficult , you try to do it.
 \textit{
	\begin{itemize}
	\item The only time that we attempted to do something like that was in the city of Philadelphia.
	\item Before I could attempt a reply he added over his shoulder: 'Wait there.'
	\end{itemize}
}
\item countable noun \\
If you make an \textbf{attempt}  \textbf{to} do something, you try to do it, often without success .
 \textit{
	\begin{itemize}
	\item ...a deliberate attempt to destabilise the defence.
	\item It was one of his rare attempts at humour.
	\item ...their involvement in a coup attempt in September.
	\end{itemize}
}
\item countable noun \\
An \textbf{attempt on} someone's life is an attempt to kill them.
 \textit{
	\begin{itemize}
	\item ...an attempt on the life of the former Iranian Prime Minister.
	\end{itemize}
}
\item countable noun \\
In British English, if a sportsman or sportswoman makes an \textbf{attempt}  \textbf{on} a sporting  record , they try to beat it. In American English you say that they make an \textbf{attempt}  \textbf{to}  break it.
 \textit{
	\begin{itemize}
	\item Everything is almost ready for me to make another attempt on the record.
	\end{itemize}
}
\end{enumerate}

\section*{expect}
{\large \color{blue}  expects  expecting  expected  }
\subsection*{Explain}
\begin{enumerate}
\item verb \\
If you \textbf{expect} something \textbf{to}  happen , you believe that it will happen.
 \textit{
	\begin{itemize}
	\item ...a council workman who expects to lose his job in the next few weeks.
	\item They no longer expect corporate profits to improve.
	\item The talks are expected to continue until tomorrow.
	\item Few expected that he would declare his candidacy for the Democratic nomination for
the presidency.
	\item It is expected that the new owner will change the yacht's name.
	\item They expect a gradual improvement in sales of new cars.
	\end{itemize}
}
\item verb \\
If you \textbf{are expecting} something or someone, you believe that they will be delivered to you or come to you soon , often because this has been arranged  earlier .
 \textit{
	\begin{itemize}
	\item I am expecting several important letters but nothing has arrived.
	\item I wasn't expecting a visitor.
	\item We were expecting him home again any day now.
	\end{itemize}
}
\item verb \\
If you \textbf{expect} something, or \textbf{expect} a person \textbf{to} do something, you believe that it is your right to have that thing, or the person's duty to do it for you.
 \textit{
	\begin{itemize}
	\item He wasn't expecting our hospitality.
	\item I do expect to have some time to myself in the evenings.
	\item I wasn't expecting you to help.
	\item Is this a rational thing to expect of your partner, or not?
	\item She realizes now she expected too much of Helen.
	\end{itemize}
}
\item verb \\
If you tell someone not to \textbf{expect} something, you mean that the thing is unlikely to happen as they have planned or imagined , and they should not hope that it will.
 \textit{
	\begin{itemize}
	\item Don't expect an instant cure.
	\item You cannot expect to like all the people you will work with.
	\item Don't expect me to come and visit you there.
	\end{itemize}
}
\item verb \\
If you say that a woman  \textbf{is expecting} a baby , or that she \textbf{is expecting} , you mean that she is pregnant .
 \textit{
	\begin{itemize}
	\item She was expecting another baby.
	\item I hear Dawn's expecting again.
	\end{itemize}
}
\item  \\
 I expect \textit{
	\begin{itemize}
	\end{itemize}
}
\item  \\
 what can/do you expect \textit{
	\begin{itemize}
	\end{itemize}
}
\end{enumerate}

\section*{attend}
{\large \color{blue}  attends  attending  attended  }
\subsection*{Explain}
\begin{enumerate}
\item verb \\
If you \textbf{attend} a meeting or other event, you are present at it.
 \textit{
	\begin{itemize}
	\item Thousands of people attended the funeral.
	\item The meeting will be attended by finance ministers from many countries.
	\item We want the maximum number of people to attend to help us cover our costs.
	\end{itemize}
}
\item verb \\
If you \textbf{attend} an institution such as a school , college , or church, you go there regularly.
 \textit{
	\begin{itemize}
	\item They attended college together at the University of Pennsylvania.
	\end{itemize}
}
\item verb \\
If you \textbf{attend}  \textbf{to} something, you deal with it. If you \textbf{attend}  \textbf{to} someone who is hurt or injured , you care for them.
 \textit{
	\begin{itemize}
	\item The staff will helpfully attend to your needs.
	\item There are more pressing matters to be attended to today.
	\item The main thing is to attend to the injured.
	\end{itemize}
}
\end{enumerate}

\section*{expend}
{\large \color{blue}  expends  expending  expended  }
\subsection*{Explain}
\begin{enumerate}
\item verb \\
To \textbf{expend} something, especially  energy , time, or money, means to use it or spend it.
 \textit{
	\begin{itemize}
	\item Children expend a lot of energy and may need more high-energy food than adults.
	\end{itemize}
}
\end{enumerate}

\section*{auditorium}
{\large \color{blue}  auditoriums  auditoria  }
\subsection*{Explain}
\begin{enumerate}
\item countable noun \\
An \textbf{auditorium} is the part of a theatre or concert hall where the audience sits.
 \textit{
	\begin{itemize}
	\item The Albert Hall is a huge auditorium.
	\end{itemize}
}
\item countable noun \\
An \textbf{auditorium} is a large room , hall, or building which is used for events such as meetings and concerts.
 \textit{
	\begin{itemize}
	\end{itemize}
}
\end{enumerate}

\section*{flow}
{\large \color{blue}  flows  flowing  flowed  }
\subsection*{Explain}
\begin{enumerate}
\item verb \\
If a liquid, gas , or electrical  current  \textbf{flows}  somewhere , it moves there steadily and continuously.
 \textbf{Flow} is also a noun .
 \textit{
	\begin{itemize}
	\item A stream flowed gently down into the valley.
	\item The current flows into electric motors that drive the wheels.
	\item ...compressor stations that keep the gas flowing.
	\item It works only in the veins, where the blood flow is slower.
	\end{itemize}
}
\item verb \\
If a number of people or things \textbf{flow} from one place to another, they move there steadily in large groups, usually without
 stopping .
 \textbf{Flow} is also a noun.
 \textit{
	\begin{itemize}
	\item Large numbers of refugees continue to flow from the troubled region into the no-man's
land.
	\item Troops would patrol major roads to ensure that traffic flows freely throughout the
country.
	\item She watched the frantic flow of cars and buses along the street.
	\item It would monitor traffic flows and provide feedback to motorists.
	\end{itemize}
}
\item verb \\
If information or money \textbf{flows} somewhere, it moves freely between people or organizations.
 \textbf{Flow} is also a noun.
 \textit{
	\begin{itemize}
	\item A lot of this information flowed through other police departments.
	\item An interest rate reduction is needed to get more money flowing and create jobs.
	\item ...the opportunity to control the flow of information.
	\end{itemize}
}
\item verb \\
If an emotion  \textbf{flows} through someone, they feel it very intensely, often so that other people notice it.
 \textit{
	\begin{itemize}
	\item In that moment a surge of hatred flowed through my blood.
	\item Waves of emotion flowed across his huge face.
	\end{itemize}
}
\item verb \\
If a quality or situation  \textbf{flows}  \textbf{from} something, it comes from it or results naturally from it.
 \textit{
	\begin{itemize}
	\item Undesirable consequences flow from these misconceptions.
	\end{itemize}
}
\item verb \\
If someone's words \textbf{flow} , they are spoken smoothly and continuously.
 \textit{
	\begin{itemize}
	\item His words flowed more readily.
	\end{itemize}
}
\item verb \\
If someone's hair or clothing  \textbf{flows} about them, it hangs freely and loosely.
 \textit{
	\begin{itemize}
	\item ...a long white dress which flowed over her body.
	\item ...long black flowing hair.
	\end{itemize}
}
\item verb \\
If you say that something \textbf{flows} or that a place \textbf{flows}  \textbf{with} it, you are indicating that there is a great deal of that thing in the place.
 \textit{
	\begin{itemize}
	\item The wine flowed and we danced the night away.
	\item The square was packed, and the cobbled streets flowed with coloured petals.
	\end{itemize}
}
\item  \\
 in full flow \textit{
	\begin{itemize}
	\end{itemize}
}
\item  \\
 in full flow \textit{
	\begin{itemize}
	\end{itemize}
}
\item  \\
 go with the flow \textit{
	\begin{itemize}
	\end{itemize}
}
\end{enumerate}

\section*{avoid}
{\large \color{blue}  avoids  avoiding  avoided  }
\subsection*{Explain}
\begin{enumerate}
\item verb \\
If you \textbf{avoid} something unpleasant that might  happen , you take action in order to prevent it from happening.
 \textit{
	\begin{itemize}
	\item The pilots had to take emergency action to avoid a disaster.
	\item She took a different route to work to avoid getting stuck in traffic.
	\end{itemize}
}
\item verb \\
If you \textbf{avoid} doing something, you choose not to do it, or you put yourself in a situation where you do not have to do it.
 \textit{
	\begin{itemize}
	\item Swann managed to avoid learning that lesson for a long time.
	\item He was always careful to avoid embarrassment.
	\end{itemize}
}
\item verb \\
If you \textbf{avoid} a person or thing, you keep away from them. When talking to someone, if you \textbf{avoid} the subject, you keep the conversation away from a particular topic .
 \textit{
	\begin{itemize}
	\item She eventually had to lock herself in the toilets to avoid him.
	\item All through lunch he had carefully avoided the subject of the house.
	\end{itemize}
}
\item verb \\
If a person or vehicle \textbf{avoids} someone or something, they change the direction they are moving in, so that they
do not hit them.
 \textit{
	\begin{itemize}
	\item The driver had ample time to brake or swerve and avoid the woman.
	\end{itemize}
}
\end{enumerate}

\section*{foresee}
{\large \color{blue}  foresees  foreseeing  foresaw  foreseen  }
\subsection*{Explain}
\begin{enumerate}
\item verb \\
If you \textbf{foresee} something, you expect and believe that it will  happen .
 \textit{
	\begin{itemize}
	\item He did not foresee any problems.
	\item ...a dangerous situation which could have been foreseen.
	\item He could never have foreseen that one day his books would sell in millions.
	\end{itemize}
}
\end{enumerate}

\section*{badminton}
{\large \color{blue}  }
\subsection*{Explain}
\begin{enumerate}
\item uncountable noun \\
\textbf{Badminton} is a game played by two or four  players on a rectangular  court with a high net across the middle . The players try to score points by hitting a small object called a shuttlecock across the net using a racket.
 \textit{
	\begin{itemize}
	\end{itemize}
}
\end{enumerate}

\section*{frown}
{\large \color{blue}  frowns  frowning  frowned  }
\subsection*{Explain}
\begin{enumerate}
\item verb \\
When someone \textbf{frowns} , their eyebrows become drawn together, because they are annoyed , worried, or puzzled , or because they are concentrating .
 \textbf{Frown} is also a noun .
 \textit{
	\begin{itemize}
	\item Nancy shook her head, frowning.
	\item He frowned at her anxiously.
	\item ...a frowning man.
	\item There was a deep frown on the boy's face.
	\end{itemize}
}
\end{enumerate}

\section*{call}
{\large \color{blue}  calls  calling  called  }
\subsection*{Explain}
\begin{enumerate}
\item verb \\
If you \textbf{call} someone or something \textbf{by} a particular name or title, you give them that name or title.
 \textit{
	\begin{itemize}
	\item I always wanted to call the dog Mufty for some reason.
	\item 'Doctor...'—'Will you please call me Sarah?'
	\item Everybody called each other by their surnames.
	\end{itemize}
}
\item verb \\
If you \textbf{call} someone or something a particular thing, you suggest they are that thing or describe them as that thing.
 \textit{
	\begin{itemize}
	\item The speech was interrupted by members of the Conservative Party, who called him a
traitor.
	\item I wouldn't call it a burden; I call it a responsibility.
	\item She calls me lazy and selfish.
	\item He called it particularly cynical to begin releasing the hostages on Christmas Day.
	\item Anyone can call themselves a psychotherapist.
	\end{itemize}
}
\item verb \\
If you \textbf{call} something, you say it in a loud voice, because you are trying to attract someone's attention.
 \textbf{Call out} means the same as call .
 \textit{
	\begin{itemize}
	\item He could hear the others downstairs in different parts of the house calling his name.
	\item 'Boys!' she called again.
	\item The butcher's son called out a greeting.
	\item The train stopped and a porter called out, 'Middlesbrough!'
	\end{itemize}
}
\item verb \\
If you \textbf{call} someone, you telephone them.
 \textit{
	\begin{itemize}
	\item Would you call me as soon as you find out?
	\item A friend of mine gave me this number to call.
	\item 'May I speak with Mr Coyne, please?'—'May I ask who's calling?'
	\end{itemize}
}
\item verb \\
If you \textbf{call} someone such as a doctor or the police, you ask them to come to you, usually by phoning them.
 \textit{
	\begin{itemize}
	\item He screamed for someone to call an ambulance.
	\item One night he was called to see a woman with tuberculosis.
	\end{itemize}
}
\item verb \\
If you \textbf{call} someone, you ask them to come to you by shouting to them.
 \textit{
	\begin{itemize}
	\item She called her young son: 'Here, Stephen, come and look at this!'
	\item He called me over the Tannoy.
	\end{itemize}
}
\item countable noun \\
When you make a telephone \textbf{call} , you telephone someone.
 \textit{
	\begin{itemize}
	\item I made a phone call to the United States to talk to a friend.
	\item I've had hundreds of calls from other victims.
	\item I got a call from him late yesterday evening.
	\end{itemize}
}
\item verb \\
If someone in authority \textbf{calls} something such as a meeting, rehearsal, or election , they arrange for it to take place at a particular time.
 \textit{
	\begin{itemize}
	\item The Committee decided to call a meeting of the All India Congress.
	\item The 79-year-old Swiss called a press conference in Zurich to announce his objections
to the decision.
	\item The strike was called by the Lebanese Forces militia.
	\item A meeting has been called for Monday.
	\end{itemize}
}
\item verb \\
If someone \textbf{is called} before a court or committee , they are ordered to appear there, usually to give evidence .
 \textit{
	\begin{itemize}
	\item The child waited two hours before she was called to give evidence.
	\item I was called as an expert witness.
	\end{itemize}
}
\item verb \\
If you \textbf{call}  somewhere , you make a short visit there.
 \textbf{Call} is also a noun.
 \textit{
	\begin{itemize}
	\item A market researcher called at the house where he was living.
	\item Andrew now came almost weekly to call.
	\item He decided to pay a call on Tommy Cummings.
	\end{itemize}
}
\item verb \\
When a train, bus , or ship \textbf{calls} somewhere, it stops there for a short time to allow people to get on or off.
 \textit{
	\begin{itemize}
	\item The steamer calls at several ports along the way.
	\end{itemize}
}
\item verb \\
To \textbf{call} a game or sporting event means to cancel it, for example because of rain or bad light.
 \textit{
	\begin{itemize}
	\item The game was called on account of darkness.
	\item We called the next game.
	\end{itemize}
}
\item countable noun \\
If there is a \textbf{call}  \textbf{for} something, someone demands that it should happen .
 \textit{
	\begin{itemize}
	\item There have been calls for a new kind of security arrangement.
	\item Almost all workers heeded a call to stay at home during the strike.
	\end{itemize}
}
\item uncountable noun \\
If there is little or no \textbf{call for} something, very few people want it to be done or provided.
 \textit{
	\begin{itemize}
	\item 'Have you got just plain chocolate?'—'No, I'm afraid there's not much call for that.'
	\end{itemize}
}
\item singular noun \\
The \textbf{call}  \textbf{of} something such as a place is the way it attracts or interests you strongly.
 \textit{
	\begin{itemize}
	\item You must be feeling exhilarated by the call of the new.
	\end{itemize}
}
\item countable noun \\
The \textbf{call} of a particular bird or animal is the characteristic sound that it makes.
 \textit{
	\begin{itemize}
	\item ...the plaintive call of a whale.
	\item ...a wide range of animal noises and bird calls.
	\end{itemize}
}
\item  \\
 first call on sth \textit{
	\begin{itemize}
	\end{itemize}
}
\item  \\
 there is no call for \textit{
	\begin{itemize}
	\end{itemize}
}
\item  \\
 on call \textit{
	\begin{itemize}
	\end{itemize}
}
\item  \\
 to call in sick \textit{
	\begin{itemize}
	\end{itemize}
}
\end{enumerate}

\section*{hold}
{\large \color{blue}  holds  holding  held  }
\subsection*{Explain}
\begin{enumerate}
\item verb \\
When you \textbf{hold} something, you carry or support it, using your hands or your arms.
 \textbf{Hold} is also a noun.
 \textit{
	\begin{itemize}
	\item Hold the knife at an angle.
	\item She is holding her journal and a pen.
	\item He held the pistol in his right hand.
	\item Hold the baby while I load the car.
	\item He released his hold on the camera.
	\end{itemize}
}
\item  \\
 to catch hold of something \textit{
	\begin{itemize}
	\end{itemize}
}
\item verb \\
When you \textbf{hold} someone, you put your arms round them, usually because you want to show them how much you like them or because you want to comfort them.
 \textit{
	\begin{itemize}
	\item If only he would hold her close to him.
	\end{itemize}
}
\item verb \\
If you \textbf{hold} someone in a particular position, you use force to keep them in that position and
stop them from moving.
 \textit{
	\begin{itemize}
	\item He then held the man in an armlock until police arrived.
	\item I'd got two nurses holding me down.
	\end{itemize}
}
\item countable noun \\
A \textbf{hold} is a particular way of keeping someone in a position using your own hands, arms,
or legs.
 \textit{
	\begin{itemize}
	\item ...use of an unauthorized hold on a handcuffed suspect.
	\end{itemize}
}
\item verb \\
When you \textbf{hold} a part of your body, you put your hand on or against it, often because it hurts .
 \textit{
	\begin{itemize}
	\item Soon she was crying bitterly about the pain and was holding her throat.
	\end{itemize}
}
\item verb \\
When you \textbf{hold} a part of your body in a particular position, you put it into that position and keep
it there.
 \textit{
	\begin{itemize}
	\item Hold your hands in front of your face.
	\item He walked at a rapid pace with his back straight and his head held erect.
	\end{itemize}
}
\item verb \\
If one thing \textbf{holds} another in a particular position, it keeps it in that position.
 \textit{
	\begin{itemize}
	\item ...the wooden wedge which held the heavy door open.
	\item They used steel pins to hold everything in place.
	\end{itemize}
}
\item verb \\
If one thing is used to \textbf{hold} another, it is used to store it.
 \textit{
	\begin{itemize}
	\item Two knife racks hold her favourite knives.
	\item ...the large cardboard and wooden cases used to hold my new appliances.
	\end{itemize}
}
\item countable noun \\
In a ship or aeroplane , a \textbf{hold} is a place where cargo or luggage is stored.
 \textit{
	\begin{itemize}
	\item A fire had been reported in the cargo hold.
	\end{itemize}
}
\item verb \\
If a place \textbf{holds} something, it keeps it available for reference or for future use.
 \textit{
	\begin{itemize}
	\item The Small Firms Service holds an enormous amount of information on any business problem.
	\item We have reviewed the data that we hold for the area.
	\end{itemize}
}
\item verb \\
If something \textbf{holds} a particular amount of something, it can contain that amount.
 \textit{
	\begin{itemize}
	\item The small bottles don't seem to hold much.
	\item ...a stadium that holds over 30,000 people.
	\end{itemize}
}
\item verb \\
If you can \textbf{hold} your drink, you are able to drink large quantities of alcohol without becoming  ill or getting drunk.
 \textit{
	\begin{itemize}
	\item ...but you had to be able to hold your liquor.
	\end{itemize}
}
\item verb \\
If a vehicle \textbf{holds} the road well, it remains in close contact with the road and can be controlled safely
and easily.
 \textit{
	\begin{itemize}
	\item I thought the car held the road really well.
	\end{itemize}
}
\end{enumerate}

\section*{celebrate}
{\large \color{blue}  celebrates  celebrating  celebrated  }
\subsection*{Explain}
\begin{enumerate}
\item verb \\
If you \textbf{celebrate} , you do something enjoyable because of a special occasion or to mark someone's success .
 \textit{
	\begin{itemize}
	\item I was in a mood to celebrate.
	\item Tom celebrated his 24th birthday two days ago.
	\end{itemize}
}
\item verb \\
If an organization or country  \textbf{is celebrating} an anniversary, it has existed for that length of time and is doing something special because of it.
 \textit{
	\begin{itemize}
	\item The Society is celebrating its tenth anniversary this year.
	\end{itemize}
}
\item verb \\
When priests  \textbf{celebrate}  Holy  Communion or Mass, they officially perform the actions and ceremonies that are involved .
 \textit{
	\begin{itemize}
	\item The Pope will celebrate Mass in Westminster Cathedral.
	\end{itemize}
}
\end{enumerate}

\section*{invert}
{\large \color{blue}  inverts  inverting  inverted  }
\subsection*{Explain}
\begin{enumerate}
\item verb \\
If you \textbf{invert} something, you turn it the other way up or back to front .
 \textit{
	\begin{itemize}
	\item Invert the cake onto a cooling rack.
	\item ...a black inverted triangle.
	\end{itemize}
}
\item verb \\
If you \textbf{invert} something, you change it to its opposite.
 \textit{
	\begin{itemize}
	\item They may be hoping to invert the presumption that a defendant is innocent until proved
guilty.
	\item ...a telling illustration of inverted moral values.
	\end{itemize}
}
\end{enumerate}

\section*{compress}
{\large \color{blue}  compresses  compressing  compressed  }
\subsection*{Explain}
\begin{enumerate}
\item verb \\
When you \textbf{compress} something or when it \textbf{compresses} , it is pressed or squeezed so that it takes up less space.
 \textit{
	\begin{itemize}
	\item Poor posture, sitting or walking slouched over, compresses the body's organs.
	\item Air will compress but the brake fluid won't.
	\end{itemize}
}
\item verb \\
If you \textbf{compress} something such as a piece of writing or a description , you make it shorter .
 \textit{
	\begin{itemize}
	\item He never understood how to organize or compress large masses of material.
	\item All those three books are compacted and compressed into one book.
	\end{itemize}
}
\item verb \\
If an event  \textbf{is compressed}  \textbf{into} a short space of time, it is given less time to happen than normal or previously.
 \textit{
	\begin{itemize}
	\item The four debates will be compressed into an unprecedentedly short eight-day period.
	\item Some courses such as engineering had to be compressed.
	\end{itemize}
}
\item countable noun \\
A \textbf{compress} is a pad of wet or dry cloth pressed on part of a patient's body to reduce fever.
 \textit{
	\begin{itemize}
	\item Sore throats may be relieved by cold compresses.
	\end{itemize}
}
\end{enumerate}

\section*{investigate}
{\large \color{blue}  investigates  investigating  investigated  }
\subsection*{Explain}
\begin{enumerate}
\item verb \\
If someone, especially an official, \textbf{investigates} an event, situation, or claim , they try to find out what happened or what is the truth.
 \textit{
	\begin{itemize}
	\item Gas officials are investigating the cause of an explosion which badly damaged a house
in Hampshire.
	\item The two officers were being investigated by the director of public prosecutions.
	\item Police are still investigating how the accident happened.
	\end{itemize}
}
\end{enumerate}

\section*{comprise}
{\large \color{blue}  comprises  comprising  comprised  }
\subsection*{Explain}
\begin{enumerate}
\item verb \\
If you say that something \textbf{comprises} or \textbf{is comprised of} a number of things or people, you mean it has them as its parts or members.
 \textit{
	\begin{itemize}
	\item MCC's main committee comprises 18 members.
	\item The exhibition comprises 50 oils and watercolours.
	\item The task force is comprised of congressional leaders and cabinet heads.
	\item A crowd comprised of the wives and children of scientists staged a demonstration.
	\end{itemize}
}
\item verb \\
The things or people that \textbf{comprise} something are the parts or members that form it.
 \textit{
	\begin{itemize}
	\item The proposals exclude three of the four nations comprising the UK..
	\item Women comprise 44% of hospital medical staff.
	\end{itemize}
}
\end{enumerate}

\section*{lean}
{\large \color{blue}  leans  leaning  leaned  leant  leaner  leanest  }
\subsection*{Explain}
\begin{enumerate}
\item verb \\
When you \textbf{lean} in a particular direction, you bend your body in that direction.
 \textit{
	\begin{itemize}
	\item Eileen leaned across and opened the passenger door.
	\item He leaned forward to give her a kiss.
	\item They stopped to lean over a gate.
	\end{itemize}
}
\item verb \\
If you \textbf{lean}  \textbf{on} or \textbf{against} someone or something, you rest against them so that they partly support your weight . If you \textbf{lean} an object \textbf{on} or \textbf{against} something, you place the object so that it is partly supported by that thing.
 \textit{
	\begin{itemize}
	\item She was feeling tired and was glad to lean against him.
	\item Lean the plants against a wall and cover the roots with peat.
	\item The table lurched as a young man leant his weight on it.
	\end{itemize}
}
\item adjective \\
If you describe someone as \textbf{lean} , you mean that they are thin but look strong and healthy .
 \textit{
	\begin{itemize}
	\item Like most athletes, she was lean and muscular.
	\item She watched the tall, lean figure step into the car.
	\end{itemize}
}
\item adjective \\
If meat is \textbf{lean} , it does not have very much fat.
 \textit{
	\begin{itemize}
	\item It is a beautiful meat, very lean and tender.
	\item Lean steak with vegetables.
	\end{itemize}
}
\item adjective \\
If you describe an organization as \textbf{lean} , you mean that it has become more efficient and less wasteful by getting  rid of staff , or by dropping  projects which were unprofitable .
 \textit{
	\begin{itemize}
	\item The value of the pound will force British companies to be leaner and fitter.
	\item ...cutting corporate flab and building leaner companies.
	\end{itemize}
}
\item adjective \\
If you describe periods of time as \textbf{lean} , you mean that people have less of something such as money or are less successful than they used to be.
 \textit{
	\begin{itemize}
	\item ...the lean years of the 1930s.
	\item With fewer tourists in town, the taxi trade is going through its leanest patch for
30 years.
	\end{itemize}
}
\end{enumerate}

\section*{deduct}
{\large \color{blue}  deducts  deducting  deducted  }
\subsection*{Explain}
\begin{enumerate}
\item verb \\
When you \textbf{deduct} an amount from a total , you subtract it from the total.
 \textit{
	\begin{itemize}
	\item The company deducted this payment from his compensation.
	\item Up to 5% of marks in the exams will be deducted for spelling mistakes.
	\end{itemize}
}
\end{enumerate}

\section*{leave}
{\large \color{blue}  leaves  leaving  left  }
\subsection*{Explain}
\begin{enumerate}
\item verb \\
If you \textbf{leave} a place or person, you go away from that place or person.
 \textit{
	\begin{itemize}
	\item He would not be allowed to leave the country.
	\item I simply couldn't bear to leave my little girl.
	\item My flight leaves in less than an hour.
	\item The last of the older children had left for school.
	\end{itemize}
}
\item verb \\
If you \textbf{leave} an institution , group, or job , you permanently stop  attending that institution, being a member of that group, or doing that job.
 \textit{
	\begin{itemize}
	\item He left school with no qualifications.
	\item I am leaving to concentrate on writing fiction.
	\item ...a leaving present.
	\end{itemize}
}
\item verb \\
If you \textbf{leave} your husband , wife , or some other person with whom you have had a close relationship , you stop living with them or you finish the relationship.
 \textit{
	\begin{itemize}
	\item He'll never leave you. You need have no worry.
	\item I would be insanely jealous if Bill left me for another woman.
	\end{itemize}
}
\item verb \\
If you \textbf{leave} something or someone in a particular place, you let them remain there when you go
away. If you \textbf{leave} something or someone with a person, you let them remain with that person so they
are safe while you are away.
 \textit{
	\begin{itemize}
	\item I left my bags in the car.
	\item Don't leave your truck there.
	\item From the moment that Philippe had left her in the bedroom at the hotel, she had heard
nothing of him.
	\item Leave your key with a neighbour in case you lock yourself out one day.
	\end{itemize}
}
\item verb \\
If you \textbf{leave} a message or an answer , you write it, record it, or give it to someone so that it can be found or passed
on.
 \textit{
	\begin{itemize}
	\item You can leave a message on our answering machine.
	\item Decide whether the ball is in square A, B, C, or D, then call and leave your answer.
	\item I left my phone number with several people.
	\end{itemize}
}
\item verb \\
If you \textbf{leave} someone doing something, they are doing that thing when you go away from them.
 \textit{
	\begin{itemize}
	\item Salter drove off, leaving Callendar surveying the scene.
	\end{itemize}
}
\item verb \\
If you \textbf{leave} someone \textbf{to} do something, you go away from them so that they do it on their own. If you \textbf{leave} someone \textbf{to} himself or herself, you go away from them and allow them to be alone .
 \textit{
	\begin{itemize}
	\item I'd better leave you to get on with it, then.
	\item Diana took the hint and left them to it.
	\item One of the advantages of a department store is that you are left to yourself to try
things on.
	\item He quietly slipped away and left me to my tears.
	\end{itemize}
}
\item verb \\
To \textbf{leave} an amount of something means to keep it available after the rest has been used or taken away.
 \textit{
	\begin{itemize}
	\item He always left a little food for the next day.
	\item Double rooms at any of the following hotels should leave you some change from £150.
	\end{itemize}
}
\item verb \\
If you take one number away from another, you can say that it \textbf{leaves} the number that remains. For example , five take away two leaves three.
 \textit{
	\begin{itemize}
	\end{itemize}
}
\item verb \\
To \textbf{leave} someone \textbf{with} something, especially when that thing is unpleasant or difficult to deal with, means to make them have it or make them responsible for it.
 \textit{
	\begin{itemize}
	\item ...a crash which left him with a broken collar-bone.
	\item He left me with a child to support.
	\end{itemize}
}
\item verb \\
If an event \textbf{leaves} people or things in a particular state, they are in that state when the event has
finished.
 \textit{
	\begin{itemize}
	\item ...violent disturbances which have left at least ten people dead.
	\item The documentary left me in a state of shock.
	\item So where does that leave me?
	\end{itemize}
}
\item verb \\
If you \textbf{leave} food or drink, you do not eat or drink it, often because you do not like it.
 \textit{
	\begin{itemize}
	\item If you don't like the cocktail you ordered, just leave it and try a different one.
	\end{itemize}
}
\item verb \\
If something \textbf{leaves} a mark, effect, or sign , it causes that mark, effect, or sign to remain as a result.
 \textit{
	\begin{itemize}
	\item A muscle tear will leave a scar after healing.
	\item She left a lasting impression on him.
	\end{itemize}
}
\item verb \\
If you \textbf{leave} something in a particular state, position, or condition, you let it remain in that
state, position, or condition.
 \textit{
	\begin{itemize}
	\item He left the album open on the table.
	\item I've left the car lights on.
	\item I left the engine running.
	\end{itemize}
}
\item verb \\
If you \textbf{leave} a space or gap in something, you deliberately make that space or gap.
 \textit{
	\begin{itemize}
	\item Leave a gap at the top and bottom so air can circulate.
	\end{itemize}
}
\item verb \\
If you \textbf{leave} a job, decision , or choice  \textbf{to} someone, you give them the responsibility for dealing with it or making it.
 \textit{
	\begin{itemize}
	\item Affix the blue airmail label and leave the rest to us.
	\item The judge should not have left it to the jury to decide.
	\item For the moment, I leave you to take all decisions.
	\end{itemize}
}
\item verb \\
If you say that something such as an arrangement or an agreement  \textbf{leaves} a lot  \textbf{to} another thing or person, you are critical of it because it is not adequate and its success  depends on the other thing or person.
 \textit{
	\begin{itemize}
	\item The ceasefire leaves a lot to the goodwill of the forces involved.
	\item It's a vague formulation that leaves much to the discretion of local authorities.
	\end{itemize}
}
\item verb \\
To \textbf{leave} someone \textbf{with} a particular course of action or the opportunity to do something means to let it be available to them, while restricting them in other ways.
 \textit{
	\begin{itemize}
	\item This left me only one possible course of action.
	\item He was left with no option but to resign.
	\end{itemize}
}
\item verb \\
If you \textbf{leave} something \textbf{until} a particular time, you delay doing it or dealing with it until then.
 \textit{
	\begin{itemize}
	\item Don't leave it all until the last minute.
	\end{itemize}
}
\item verb \\
If you \textbf{leave} a particular subject, you stop talking about it and start  discussing something else.
 \textit{
	\begin{itemize}
	\item I think we'd better leave the subject of Nationalism.
	\item He suggested we get together for a drink sometime. I said I'd like that, and we left
it there.
	\end{itemize}
}
\item verb \\
If you \textbf{leave} property or money \textbf{to} someone, you arrange for it to be given to them after you have died .
 \textit{
	\begin{itemize}
	\item He died two and a half years later, leaving everything to his wife.
	\end{itemize}
}
\item verb \\
If you say that someone \textbf{leaves} a wife, husband, or a particular number of children, you mean that the wife, husband,
or children remain alive after that person has died.
 \textit{
	\begin{itemize}
	\item Mr Sharp, who leaves a wife and two children, had been suffering from cancer.
	\end{itemize}
}
\item uncountable noun \\
\textbf{Leave} is a period of time when you are not working at your job, because you are on holiday or vacation , or for some other reason . If you are \textbf{on leave} , you are not working at your job.
 \textit{
	\begin{itemize}
	\item Why don't you take a few days' leave?
	\item ...maternity leave.
	\item He is home on leave from the Navy.
	\end{itemize}
}
\item uncountable noun \\
If you ask for \textbf{leave}  \textbf{to} do something, you ask for permission to do it.
 \textit{
	\begin{itemize}
	\item ...an application for leave to appeal against the judge's order.
	\end{itemize}
}
\item  \\
 to leave someone or something alone \textit{
	\begin{itemize}
	\end{itemize}
}
\item  \\
 leaving aside/leaving to one side \textit{
	\begin{itemize}
	\end{itemize}
}
\item  \\
 take one's leave/take leave of sb \textit{
	\begin{itemize}
	\end{itemize}
}
\item  \\
 to leave well alone \textit{
	\begin{itemize}
	\end{itemize}
}
\item  \\
 where you left off \textit{
	\begin{itemize}
	\end{itemize}
}
\end{enumerate}

\section*{devise}
{\large \color{blue}  devises  devising  devised  }
\subsection*{Explain}
\begin{enumerate}
\item verb \\
If you \textbf{devise} a plan, system, or machine, you have the idea for it and design it.
 \textit{
	\begin{itemize}
	\item We devised a scheme to help him.
	\item New long-range objectives must be devised.
	\end{itemize}
}
\end{enumerate}

\section*{lubricate}
{\large \color{blue}  lubricates  lubricating  lubricated  }
\subsection*{Explain}
\begin{enumerate}
\item verb \\
If you \textbf{lubricate} something such as a part of a machine, you put a substance such as oil on it so that it moves smoothly.
 \textit{
	\begin{itemize}
	\item Mineral oils are used to lubricate machinery.
	\item ...lubricating oil.
	\end{itemize}
}
\item verb \\
If you say that something \textbf{lubricates} a particular situation , you mean that it helps things to happen without any problems .
 \textit{
	\begin{itemize}
	\item Franklin's task was to lubricate the discussions with the French.
	\end{itemize}
}
\end{enumerate}

\section*{disappoint}
{\large \color{blue}  disappoints  disappointing  disappointed  }
\subsection*{Explain}
\begin{enumerate}
\item verb \\
If things or people \textbf{disappoint} you, they are not as good as you had hoped, or do not do what you hoped they would
do.
 \textit{
	\begin{itemize}
	\item She knew that she would disappoint him.
	\end{itemize}
}
\end{enumerate}

\section*{notify}
{\large \color{blue}  notifies  notifying  notified  }
\subsection*{Explain}
\begin{enumerate}
\item verb \\
If you \textbf{notify} someone of something, you officially inform them about it.
 \textit{
	\begin{itemize}
	\item The skipper notified the coastguard of the tragedy.
	\item Earlier this year they were notified that their homes were to be cleared away.
	\item She confirmed that she would notify the police and the hospital.
	\end{itemize}
}
\end{enumerate}

\section*{discard}
{\large \color{blue}  discards  discarding  discarded  }
\subsection*{Explain}
\begin{enumerate}
\item verb \\
If you \textbf{discard} something, you get rid of it because you no longer want it or need it.
 \textit{
	\begin{itemize}
	\item Read the manufacturer's guidelines before discarding the box.
	\item ...discarded cigarette butts.
	\end{itemize}
}
\end{enumerate}

\section*{plead}
{\large \color{blue}  pleads  pleading  pleaded  }
\subsection*{Explain}
\begin{enumerate}
\item verb \\
If you \textbf{plead with} someone \textbf{to} do something, you ask them in an intense , emotional  way to do it.
 \textit{
	\begin{itemize}
	\item The woman pleaded with her daughter to come back home.
	\item He was kneeling on the floor pleading for mercy.
	\item 'Do not say that,' she pleaded.
	\item I pleaded to be allowed to go.
	\end{itemize}
}
\item verb \\
When someone charged with a crime  \textbf{pleads guilty} or \textbf{not guilty} in a court of law, they officially state that they are guilty or not guilty of the crime.
 \textit{
	\begin{itemize}
	\item Morris had pleaded guilty to robbery.
	\end{itemize}
}
\item verb \\
If you \textbf{plead the case} or \textbf{cause} of someone or something, you speak out in their support or defence .
 \textit{
	\begin{itemize}
	\item He appeared before the Committee to plead his case.
	\item He pled the cause of the afflicted and the needy.
	\end{itemize}
}
\item verb \\
If you \textbf{plead} a particular thing as the reason for doing or not doing something, you give it as your excuse.
 \textit{
	\begin{itemize}
	\item Mr Burke, pleading poverty, changed his mind.
	\item Mr Giles pleads ignorance as his excuse.
	\item It was no defence to plead that they were only obeying orders.
	\end{itemize}
}
\end{enumerate}

\section*{discuss}
{\large \color{blue}  discusses  discussing  discussed  }
\subsection*{Explain}
\begin{enumerate}
\item verb \\
If people \textbf{discuss} something, they talk about it, often in order to reach a decision .
 \textit{
	\begin{itemize}
	\item I will be discussing the situation with colleagues tomorrow.
	\item The cabinet met today to discuss how to respond to the ultimatum.
	\end{itemize}
}
\item verb \\
If you \textbf{discuss} something, you write or talk about it in detail .
 \textit{
	\begin{itemize}
	\item I will discuss the role of diet in cancer prevention in Chapter 7.
	\end{itemize}
}
\end{enumerate}

\section*{please}
{\large \color{blue}  pleases  pleasing  pleased  }
\subsection*{Explain}
\begin{enumerate}
\item adverb \\
You say  \textbf{please} when you are politely asking or inviting someone to do something.
 \textit{
	\begin{itemize}
	\item Can you help us please?
	\item Would you please open the door?
	\item Please come in.
	\item 'May I sit here?'—'Please do.'
	\item Can we have the bill please?
	\end{itemize}
}
\item adverb \\
You say \textbf{please} when you are accepting something politely.
 \textit{
	\begin{itemize}
	\item 'Tea?'—'Yes, please.'
	\item 'You want an apple with your cheese?'—'Please.'
	\end{itemize}
}
\item convention \\
You can say \textbf{please} to indicate that you want someone to stop doing something or stop speaking . You would say this if, for example , what they are doing or saying makes you angry or upset .
 \textit{
	\begin{itemize}
	\item Please, Mary, this is all so unnecessary.
	\item Isabella. Please. I don't have time for this.
	\end{itemize}
}
\item convention \\
You can say \textbf{please} in order to attract someone's attention politely. Children in particular say ' \textbf{please} ' to attract the attention of a teacher or other adult .
 \textit{
	\begin{itemize}
	\item Please sir, can we have some more?
	\item Please, Miss Smith, a moment.
	\end{itemize}
}
\item verb \\
If someone or something \textbf{pleases} you, they make you feel  happy and satisfied .
 \textit{
	\begin{itemize}
	\item More than anything, I want to please you.
	\item Much of the food pleases rather than excites.
	\item It pleased him to talk to her.
	\end{itemize}
}
\item  \\
 as you please/whatever you please \textit{
	\begin{itemize}
	\end{itemize}
}
\item  \\
 as you please \textit{
	\begin{itemize}
	\end{itemize}
}
\item  \\
 if you please \textit{
	\begin{itemize}
	\end{itemize}
}
\item  \\
 if you please \textit{
	\begin{itemize}
	\end{itemize}
}
\item  \\
 please yourself \textit{
	\begin{itemize}
	\end{itemize}
}
\end{enumerate}

\section*{eat}
{\large \color{blue}  eats  eating  ate  eaten  }
\subsection*{Explain}
\begin{enumerate}
\item verb \\
When you \textbf{eat} something, you put it into your mouth, chew it, and swallow it.
 \textit{
	\begin{itemize}
	\item She was eating a sandwich.
	\item The bananas should be eaten within two days.
	\item We took our time and ate slowly.
	\end{itemize}
}
\item verb \\
If you \textbf{eat} sensibly or healthily, you eat food that is good for you.
 \textit{
	\begin{itemize}
	\item ...a campaign to persuade people to eat more healthily.
	\end{itemize}
}
\item verb \\
If you \textbf{eat} , you have a meal.
 \textit{
	\begin{itemize}
	\item Let's go out to eat.
	\item We ate lunch together a few times.
	\end{itemize}
}
\item verb \\
If something \textbf{is eating} you, it is annoying or worrying you.
 \textit{
	\begin{itemize}
	\item 'What the hell's eating you?' he demanded.
	\end{itemize}
}
\item  \\
 have sb eating out of one's hand \textit{
	\begin{itemize}
	\end{itemize}
}
\item  \\
 eat your heart out \textit{
	\begin{itemize}
	\end{itemize}
}
\item  \\
 eat sb out of house and home \textit{
	\begin{itemize}
	\end{itemize}
}
\end{enumerate}

\section*{pray}
{\large \color{blue}  prays  praying  prayed  }
\subsection*{Explain}
\begin{enumerate}
\item verb \\
When people \textbf{pray} , they speak to God in order to give thanks or to ask for his help .
 \textit{
	\begin{itemize}
	\item He spent his time in prison praying and studying.
	\item Now all we have to do is help ourselves and pray to God.
	\item ...all those who work and pray for peace.
	\item Kelly prayed that God would judge her with mercy.
	\end{itemize}
}
\item verb \\
When someone is hoping very much that something will  happen , you can say that they \textbf{are praying}  \textbf{that} it will happen.
 \textit{
	\begin{itemize}
	\item I'm just praying that the authorities will do something before it's too late.
	\item One can only pray that the team's manager learns something from it.
	\item Many were secretly praying for a compromise.
	\end{itemize}
}
\item adverb \\
\textbf{Pray} is used when asking a question in a rather  unfriendly way or in an angry but calm way.
 \textit{
	\begin{itemize}
	\item And what, pray, do you buy and sell, Major?
	\end{itemize}
}
\item adverb \\
\textbf{Pray} was used to add politeness to a command .
 \textit{
	\begin{itemize}
	\item I beg your pardon, pray continue.
	\end{itemize}
}
\end{enumerate}

\section*{emit}
{\large \color{blue}  emits  emitting  emitted  }
\subsection*{Explain}
\begin{enumerate}
\item verb \\
If something \textbf{emits}  heat , light, gas , or a smell , it produces it and sends it out by means of a physical or chemical process.
 \textit{
	\begin{itemize}
	\item The new device emits a powerful circular column of light.
	\item ...the amount of carbon dioxide emitted.
	\end{itemize}
}
\item verb \\
To \textbf{emit} a sound or noise means to produce it.
 \textit{
	\begin{itemize}
	\item Polly blinked and emitted a long, low whistle.
	\end{itemize}
}
\end{enumerate}

\section*{predict}
{\large \color{blue}  predicts  predicting  predicted  }
\subsection*{Explain}
\begin{enumerate}
\item verb \\
If you \textbf{predict} an event , you say that it will  happen .
 \textit{
	\begin{itemize}
	\item The latest opinion polls are predicting a very close contest.
	\item He predicted that my hair would grow back 'in no time'.
	\item It's hard to predict how a jury will react.
	\item 'The war will continue another two or three years,' he predicted.
	\end{itemize}
}
\end{enumerate}

\section*{enlarge}
{\large \color{blue}  enlarges  enlarging  enlarged  }
\subsection*{Explain}
\begin{enumerate}
\item verb \\
When you \textbf{enlarge} something or when it \textbf{enlarges} , it becomes bigger .
 \textit{
	\begin{itemize}
	\item ...the plan to enlarge Ewood Park into a 30,000 all-seater stadium.
	\item The glands in the neck may enlarge.
	\end{itemize}
}
\item verb \\
To \textbf{enlarge} a photograph  means to develop a bigger print of it.
 \textit{
	\begin{itemize}
	\item ...newly-weds wishing to enlarge snaps of their big day.
	\end{itemize}
}
\item verb \\
If you \textbf{enlarge}  \textbf{on} something that has been mentioned , you give more details about it.
 \textit{
	\begin{itemize}
	\item He didn't enlarge on the form that the interim government and assembly would take.
	\item I wish to enlarge upon a statement made by Gary Docking.
	\end{itemize}
}
\end{enumerate}

\section*{prepare}
{\large \color{blue}  prepares  preparing  prepared  }
\subsection*{Explain}
\begin{enumerate}
\item verb \\
If you \textbf{prepare} something, you make it ready for something that is going to happen .
 \textit{
	\begin{itemize}
	\item The most important task was to prepare a list of missing items.
	\item On average each report requires 1,000 hours to prepare.
	\item The crew of the Iowa has been preparing the ship for storage.
	\end{itemize}
}
\item verb \\
If you \textbf{prepare}  \textbf{for} an event or action that will happen soon , you get yourself ready for it or make the necessary  arrangements .
 \textit{
	\begin{itemize}
	\item The Party leadership is using management consultants to help prepare for the next
election.
	\item British books.
	\item He had to go back to his hotel and prepare to catch a train for New York.
	\item His doctor had told him to prepare himself for surgery.
	\end{itemize}
}
\item verb \\
When you \textbf{prepare} food, you get it ready to be eaten , for example by cooking it.
 \textit{
	\begin{itemize}
	\item She made her way to the kitchen, hoping to find someone preparing dinner.
	\item The best way of preparing the nuts is to rehydrate them by soaking overnight.
	\end{itemize}
}
\end{enumerate}

\section*{generate}
{\large \color{blue}  generates  generating  generated  }
\subsection*{Explain}
\begin{enumerate}
\item verb \\
To \textbf{generate} something means to cause it to begin and develop.
 \textit{
	\begin{itemize}
	\item The Employment Minister said the reforms would generate new jobs.
	\item ...the passion and emotion generated by football.
	\end{itemize}
}
\item verb \\
To \textbf{generate} a form of energy or power means to produce it.
 \textit{
	\begin{itemize}
	\item The estate uses solar panels and wind turbines to generate power.
	\end{itemize}
}
\end{enumerate}

\section*{quit}
{\large \color{blue}  quits  quitting  }
\subsection*{Explain}
\begin{enumerate}
\item verb \\
If you \textbf{quit} your job, you choose to leave it.
 \textit{
	\begin{itemize}
	\item He quit his job as an office boy in Athens.
	\item He figured he would quit before Johnson fired him.
	\end{itemize}
}
\item verb \\
If you \textbf{quit} an activity or \textbf{quit} doing something, you stop doing it.
 \textit{
	\begin{itemize}
	\item A nicotine spray can help smokers quit the habit.
	\item I was trying to quit smoking at the time.
	\end{itemize}
}
\item verb \\
If you \textbf{quit} a place, you leave it completely and do not go back to it.
 \textit{
	\begin{itemize}
	\item ...the idea that humans might one day quit the earth to colonise other planets.
	\item Police were called when he refused to quit the building.
	\end{itemize}
}
\item  \\
 to call it quits \textit{
	\begin{itemize}
	\end{itemize}
}
\end{enumerate}

\section*{impart}
{\large \color{blue}  imparts  imparting  imparted  }
\subsection*{Explain}
\begin{enumerate}
\item verb \\
If you \textbf{impart} information \textbf{to} people, you tell it to them.
 \textit{
	\begin{itemize}
	\item The ability to impart knowledge is the essential qualification for teachers.
	\item I am about to impart knowledge to you that you will never forget.
	\end{itemize}
}
\item verb \\
To \textbf{impart} a particular quality to something means to give it that quality.
 \textit{
	\begin{itemize}
	\item She managed to impart great elegance to the unpretentious dress she was wearing.
	\item His production of Harold Pinter's play fails to impart a sense of excitement or danger.
	\end{itemize}
}
\end{enumerate}

\section*{read}
{\large \color{blue}  reads  reading  }
\subsection*{Explain}
\begin{enumerate}
\item verb \\
When you \textbf{read} something such as a book or article , you look at and understand the words that are written there.
 \textbf{Read} is also a noun .
 \textit{
	\begin{itemize}
	\item Have you read this book?
	\item I read about it in the paper.
	\item He read through the pages slowly and carefully.
	\item It is nice to read that Dylan Thomas venerated the Welsh language.
	\item She spends her days reading and watching television.
	\item I settled down to have a good read.
	\end{itemize}
}
\item verb \\
When you \textbf{read} a piece of writing to someone, you say the words aloud.
 \textit{
	\begin{itemize}
	\item Jay reads poetry so beautifully.
	\item I like it when she reads to us.
	\item I sing to the boys or read them a story before tucking them in.
	\end{itemize}
}
\item verb \\
People who can \textbf{read} have the ability to look at and understand written words.
 \textit{
	\begin{itemize}
	\item He couldn't read or write.
	\item He could read words at 18 months.
	\end{itemize}
}
\item verb \\
If you can \textbf{read} music, you have the ability to look at and understand the symbols that are used in
written music to represent musical sounds.
 \textit{
	\begin{itemize}
	\item Later on I learned how to read music.
	\end{itemize}
}
\item verb \\
When a computer \textbf{reads} a file or a document, it takes information from a disk or tape.
 \textit{
	\begin{itemize}
	\item An update has left the program unable to read the files.
	\end{itemize}
}
\item verb \\
You can use \textbf{read} when saying what is written on something or in something. For example, if a notice  \textbf{reads} ' Entrance ', the word 'Entrance' is written on it.
 \textit{
	\begin{itemize}
	\item The sign on the bus read 'Private: Not In Service'.
	\end{itemize}
}
\item verb \\
If you refer to how a piece of writing \textbf{reads} , you are referring to its style.
 \textit{
	\begin{itemize}
	\item The book reads like a ballad.
	\item It reads very awkwardly.
	\end{itemize}
}
\item countable noun \\
If you say that a book or magazine is a good \textbf{read} , you mean that it is very enjoyable to read.
 \textit{
	\begin{itemize}
	\item His latest novel is a good read.
	\end{itemize}
}
\item verb \\
If something \textbf{is read} in a particular way, it is understood or interpreted in that way.
 \textit{
	\begin{itemize}
	\item The play is being widely read as an allegory of imperialist conquest.
	\item South Africans were praying last night that he has read the situation correctly.
	\item Now how do you read his remarks on that subject?
	\end{itemize}
}
\item verb \\
If you \textbf{read} someone's mind or thoughts, you know  exactly what they are thinking without them telling you.
 \textit{
	\begin{itemize}
	\item As if he could read her thoughts, Benny said, 'You're free to go any time you like.'
	\end{itemize}
}
\item verb \\
If you can \textbf{read} someone or you can \textbf{read} their gestures , you can understand what they are thinking or feeling by the way they behave or the things they say.
 \textit{
	\begin{itemize}
	\item If you have to work in a team you must learn to read people.
	\item Under the shaded light her expression was difficult to read.
	\end{itemize}
}
\item verb \\
If someone who is trying to talk to you with a radio transmitter says, 'Do you \textbf{read} me?', they are asking you if you can hear them.
 \textit{
	\begin{itemize}
	\item Alpha-Bravo-Zulu 643 to Saltezar, do you read me? Over.
	\item We read you loud and clear. Over.
	\end{itemize}
}
\item verb \\
When you \textbf{read} a measuring device, you look at it to see what the figure or measurement on it is.
 \textit{
	\begin{itemize}
	\item It is essential that you are able to read a thermometer.
	\end{itemize}
}
\item verb \\
If a measuring device \textbf{reads} a particular amount, it shows that amount.
 \textit{
	\begin{itemize}
	\item Cook for 1-1¼ hours for medium, or until the meat thermometer reads 55°C.
	\item The fuel gauge reads below zero.
	\end{itemize}
}
\item verb \\
If you \textbf{read} a subject at university, you study it.
 \textit{
	\begin{itemize}
	\item She read French and German at Cambridge University.
	\item He is now reading for a maths degree at Surrey University.
	\end{itemize}
}
\item  \\
 to take something as read \textit{
	\begin{itemize}
	\end{itemize}
}
\end{enumerate}

\section*{infect}
{\large \color{blue}  infects  infecting  infected  }
\subsection*{Explain}
\begin{enumerate}
\item verb \\
To \textbf{infect} people, animals, or plants means to cause them to have a disease or illness .
 \textit{
	\begin{itemize}
	\item A single mosquito can infect a large number of people.
	\item ...objects used by an infected person.
	\item ...people infected with HIV.
	\end{itemize}
}
\item verb \\
To \textbf{infect} a substance or area means to cause it to contain harmful  germs or bacteria .
 \textit{
	\begin{itemize}
	\item The birds infect the milk.
	\item ...a virus which is spread mainly by infected blood.
	\end{itemize}
}
\item verb \\
When people, places, or things \textbf{are infected} by a feeling or influence , it spreads to them.
 \textit{
	\begin{itemize}
	\item For an instant I was infected by her fear.
	\item He thought they might infect others with their bourgeois ideas.
	\item His urge for revenge would never infect her.
	\end{itemize}
}
\item verb \\
If a virus \textbf{infects} a computer, it affects the computer by damaging or destroying  programs .
 \textit{
	\begin{itemize}
	\item This virus infected thousands of computers within days.
	\end{itemize}
}
\end{enumerate}

\section*{reveal}
{\large \color{blue}  reveals  revealing  revealed  }
\subsection*{Explain}
\begin{enumerate}
\item verb \\
To \textbf{reveal} something means to make people aware of it.
 \textit{
	\begin{itemize}
	\item She has refused to reveal the whereabouts of her daughter.
	\item A survey of the British diet has revealed that a growing number of people are overweight.
	\item After the fire, it was revealed that North Carolina officials had never inspected
the factory.
	\item The X-rays reveal how the arrangement of atoms changes.
	\end{itemize}
}
\item verb \\
If you \textbf{reveal} something that has been out of sight , you uncover it so that people can see it.
 \textit{
	\begin{itemize}
	\item A grey carpet was removed to reveal the original pine floor.
	\end{itemize}
}
\end{enumerate}

\section*{kill}
{\large \color{blue}  kills  killing  killed  }
\subsection*{Explain}
\begin{enumerate}
\item verb \\
If a person, animal, or other living thing \textbf{is killed} , something or someone causes them to die .
 \textit{
	\begin{itemize}
	\item More than 1,000 people have been killed by the armed forces.
	\item He had attempted to kill himself on several occasions.
	\item Cattle should be killed cleanly and humanely.
	\item The earthquake killed 62 people.
	\item Heroin can kill.
	\end{itemize}
}
\item countable noun \\
The act of killing an animal after hunting it is referred to as \textbf{the}  \textbf{kill} .
 \textit{
	\begin{itemize}
	\item After the kill the men and old women collect in an open space and eat a meal of whale
meat.
	\end{itemize}
}
\item verb \\
If someone or something \textbf{kills} a project , activity, or idea , they completely destroy or end it.
 \textbf{Kill off} means the same as kill .
 \textit{
	\begin{itemize}
	\item His objective was to kill the space station project altogether.
	\item Public opinion may yet kill the proposal.
	\item He would soon launch a second offensive, killing off the peace process.
	\item The Government's financial squeeze had killed the scheme off.
	\end{itemize}
}
\item verb \\
If something \textbf{kills} pain, it weakens it so that it is no longer as strong as it was.
 \textit{
	\begin{itemize}
	\item He was forced to take opium to kill the pain.
	\end{itemize}
}
\item verb \\
If you say that something \textbf{is killing} you, you mean that it is causing you physical or emotional pain.
 \textit{
	\begin{itemize}
	\item My feet are killing me.
	\end{itemize}
}
\item verb \\
If you say that you \textbf{kill}  \textbf{yourself to} do something, you are emphasizing that you make a great effort to do it, even though it causes you a lot of trouble or suffering .
 \textit{
	\begin{itemize}
	\item You shouldn't always have to kill yourself to do well.
	\end{itemize}
}
\item verb \\
If you say that you will \textbf{kill} someone for something they have done, you are emphasizing that you are extremely  angry with them.
 \textit{
	\begin{itemize}
	\item Tell Richard I'm going to kill him when I get hold of him.
	\end{itemize}
}
\item verb \\
If you say that something will not \textbf{kill} you, you mean that it is not really as difficult or unpleasant as it might  seem .
 \textit{
	\begin{itemize}
	\item Three or four more weeks won't kill me!
	\end{itemize}
}
\item verb \\
If you \textbf{are killing} time, you are doing something because you have some time available , not because you really want to do it.
 \textit{
	\begin{itemize}
	\item I'm just killing time until I can talk to the other witnesses.
	\item To kill the hours while she waited, Ann worked in the garden.
	\end{itemize}
}
\item  \\
 if it kills me \textit{
	\begin{itemize}
	\end{itemize}
}
\item  \\
 kill yourself laughing \textit{
	\begin{itemize}
	\end{itemize}
}
\item  \\
 move in for the kill/close in for the kill \textit{
	\begin{itemize}
	\end{itemize}
}
\end{enumerate}

\section*{roast}
{\large \color{blue}  roasts  roasting  roasted  }
\subsection*{Explain}
\begin{enumerate}
\item verb \\
When you \textbf{roast} meat or other food, you cook it by dry heat in an oven or over a fire.
 \textit{
	\begin{itemize}
	\item I personally would rather roast a chicken whole.
	\end{itemize}
}
\item adjective \\
\textbf{Roast} meat has been cooked by roasting .
 \textit{
	\begin{itemize}
	\item ...roast chicken.
	\end{itemize}
}
\item countable noun \\
A \textbf{roast} is a piece of meat that is cooked by roasting.
 \textit{
	\begin{itemize}
	\item Come into the kitchen. I've got to put the roast in.
	\end{itemize}
}
\end{enumerate}

\section*{lease}
{\large \color{blue}  leases  leasing  leased  }
\subsection*{Explain}
\begin{enumerate}
\item countable noun \\
A \textbf{lease} is a legal  agreement by which the owner of a building, a piece of land, or something such as a car  allows someone else to use it for a period of time in return for money .
 \textit{
	\begin{itemize}
	\item He took up a 10 year lease on the house at Rossie Priory.
	\end{itemize}
}
\item verb \\
If you \textbf{lease} property or something such as a car from someone or if they \textbf{lease} it \textbf{to} you, they allow you to use it in return for regular  payments of money.
 \textit{
	\begin{itemize}
	\item He went to Toronto, where he leased an apartment.
	\item She hopes to lease the building to students.
	\item He will need more grazing land and perhaps La Prade could lease him a few acres.
	\end{itemize}
}
\item  \\
 a new lease of life \textit{
	\begin{itemize}
	\end{itemize}
}
\end{enumerate}

\section*{shed}
{\large \color{blue}  sheds  shedding  }
\subsection*{Explain}
\begin{enumerate}
\item countable noun \\
A \textbf{shed} is a small building that is used for storing things such as garden  tools .
 \textit{
	\begin{itemize}
	\item ...a garden shed.
	\end{itemize}
}
\item countable noun \\
A \textbf{shed} is a large shelter or building, for example at a railway  station , port, or factory .
 \textit{
	\begin{itemize}
	\item ...disused railway sheds.
	\end{itemize}
}
\item verb \\
When a tree \textbf{sheds} its leaves, its leaves fall off in the autumn . When an animal \textbf{sheds} hair or skin, some of its hair or skin drops off.
 \textit{
	\begin{itemize}
	\item Some of the trees were already beginning to shed their leaves.
	\item ...a snake who has shed its skin.
	\end{itemize}
}
\item verb \\
To \textbf{shed} something means to get rid of it.
 \textit{
	\begin{itemize}
	\item The firm is to shed 700 jobs.
	\item He had maintained a rigid diet, shedding some twenty pounds.
	\item ...a city trying to shed its rough image.
	\end{itemize}
}
\item verb \\
If a lorry \textbf{sheds} its load, the goods that it is carrying accidentally fall onto the road.
 \textit{
	\begin{itemize}
	\item A lorry piled with scrap metal had shed its load.
	\end{itemize}
}
\item verb \\
If you \textbf{shed}  tears , you cry .
 \textit{
	\begin{itemize}
	\item They will shed a few tears at their daughter's wedding.
	\end{itemize}
}
\item verb \\
To \textbf{shed} blood means to kill people in a violent way. If someone \textbf{sheds} their blood, they are killed in a violent way, usually when they are fighting in a war.
 \textit{
	\begin{itemize}
	\item They bear responsibility for shedding the blood of innocent civilians.
	\item Others promised to 'shed our blood and sacrifice our lives to oppose the invaders'.
	\end{itemize}
}
\end{enumerate}

\section*{sightseeing}
{\large \color{blue}  }
\subsection*{Explain}
\begin{enumerate}
\item uncountable noun \\
If you go \textbf{sightseeing} or do some \textbf{sightseeing} , you travel around visiting the interesting places that tourists usually visit.
 \textit{
	\begin{itemize}
	\item ...a day's sight-seeing in Venice.
	\item ...a sightseeing tour.
	\end{itemize}
}
\end{enumerate}

\section*{soak}
{\large \color{blue}  soaks  soaking  soaked  }
\subsection*{Explain}
\begin{enumerate}
\item verb \\
If you \textbf{soak} something or leave it \textbf{to}  \textbf{soak} , you put it into a liquid and leave it there.
 \textit{
	\begin{itemize}
	\item Soak the beans for 2 hours.
	\item He turned off the water and left the dishes to soak.
	\end{itemize}
}
\item verb \\
If a liquid \textbf{soaks} something or if you \textbf{soak} something \textbf{with} a liquid, the liquid makes the thing very wet.
 \textit{
	\begin{itemize}
	\item The water had soaked his jacket and shirt.
	\item Soak the soil around each bush with at least 4 gallons of water.
	\end{itemize}
}
\item verb \\
If a liquid \textbf{soaks}  \textbf{through} something, it passes through it.
 \textit{
	\begin{itemize}
	\item There was so much blood it had soaked through my boxer shorts.
	\item Rain had soaked into the sand.
	\end{itemize}
}
\item verb \\
If someone \textbf{soaks} , they spend a long time in a hot  bath , because they enjoy it.
 \textbf{Soak} is also a noun .
 \textit{
	\begin{itemize}
	\item What I need is to soak in a hot tub.
	\item I was having a long soak in the bath.
	\end{itemize}
}
\end{enumerate}

\section*{spectator}
{\large \color{blue}  spectators  }
\subsection*{Explain}
\begin{enumerate}
\item countable noun \\
A \textbf{spectator} is someone who watches something, especially a sporting event.
 \textit{
	\begin{itemize}
	\item Thirty thousand spectators watched the final game.
	\end{itemize}
}
\end{enumerate}

\section*{strip}
{\large \color{blue}  strips  stripping  stripped  }
\subsection*{Explain}
\begin{enumerate}
\item countable noun \\
A \textbf{strip}  \textbf{of} something such as paper, cloth, or food is a long, narrow piece of it.
 \textit{
	\begin{itemize}
	\item ...a new kind of manufactured wood made by pressing strips of wood together and baking
them.
	\item The simplest rag-rugs are made with strips of fabric plaited together.
	\item Serve dish with strips of fresh raw vegetables.
	\end{itemize}
}
\item countable noun \\
A \textbf{strip}  \textbf{of} land or water is a long narrow area of it.
 \textit{
	\begin{itemize}
	\item The coastal cities of Liguria sit on narrow strips of land lying under steep mountains.
	\item ...a short boat ride across a narrow strip of water.
	\end{itemize}
}
\item countable noun \\
A \textbf{strip} is a long street in a city or town, where there are a lot of stores, restaurants , and hotels .
 \textit{
	\begin{itemize}
	\item She owns a hotel-restaurant in the commercial strip on the mainland.
	\end{itemize}
}
\item verb \\
If you \textbf{strip} , you take off your clothes.
 \textbf{Strip off} means the same as strip .
 \textit{
	\begin{itemize}
	\item They stripped completely, and lay in the damp grass.
	\item The residents stripped naked in protest.
	\item The children were brazenly stripping off and leaping into the sea.
	\end{itemize}
}
\item verb \\
If someone \textbf{is stripped} , their clothes are taken off by another person, for example in order to search for hidden or illegal things.
 \textit{
	\begin{itemize}
	\item One prisoner claimed he'd been dragged to a cell, stripped and beaten.
	\end{itemize}
}
\item verb \\
To \textbf{strip} something means to remove everything that covers it.
 \textit{
	\begin{itemize}
	\item After Mike left for work I stripped the beds and vacuumed the carpets.
	\item The floorboards in both this room and the dining room have been stripped, sanded
and sealed.
	\end{itemize}
}
\item verb \\
If you \textbf{strip} an engine or a piece of equipment , you take it to pieces so that it can be cleaned or repaired .
 \textbf{Strip down} means the same as strip .
 \textit{
	\begin{itemize}
	\item Volvo's three-man team stripped the car and restored it.
	\item In five years I had to strip the water pump down four times.
	\item I stripped down the carburettors, cleaned and polished the pieces and rebuilt the
units.
	\end{itemize}
}
\item verb \\
To \textbf{strip} someone \textbf{of} their property, rights, or titles means to take those things away from them.
 \textit{
	\begin{itemize}
	\item The soldiers have stripped the civilians of their passports, and every other type
of document.
	\item A senior official was stripped of all his privileges for publicly criticising his
employer.
	\end{itemize}
}
\item countable noun \\
In a newspaper or magazine , a \textbf{strip} is a series of drawings which tell a story . The words spoken by the characters are often written on the drawings.
 \textit{
	\begin{itemize}
	\item ...the Doonesbury strip.
	\end{itemize}
}
\item  \\
 to tear a strip off \textit{
	\begin{itemize}
	\end{itemize}
}
\end{enumerate}

\section*{surround}
{\large \color{blue}  surrounds  surrounding  surrounded  }
\subsection*{Explain}
\begin{enumerate}
\item verb \\
If a person or thing \textbf{is surrounded} by something, that thing is situated all around them.
 \textit{
	\begin{itemize}
	\item The small churchyard was surrounded by a rusted wrought-iron fence.
	\item The shell surrounding the egg has many important functions.
	\item ...the snipers and artillerymen in the surrounding hills.
	\end{itemize}
}
\item verb \\
If you \textbf{are surrounded} by soldiers or police , they spread out so that they are in positions all the way around you.
 \textit{
	\begin{itemize}
	\item When the car stopped in the town square it was surrounded by soldiers and militiamen.
	\item He tried to run away but gave up when he found himself surrounded.
	\item Shooting broke out after the guards surrounded a villa in the city.
	\end{itemize}
}
\item verb \\
The circumstances , feelings , or ideas which \textbf{surround} something are those that are closely associated with it.
 \textit{
	\begin{itemize}
	\item The decision had been agreed in principle, but some controversy surrounded it.
	\item Once the euphoria surrounding this victory subsides, reality must return.
	\end{itemize}
}
\item verb \\
If you \textbf{surround}  \textbf{yourself}  \textbf{with} certain people or things, you make sure that you have a lot of them near you all the time.
 \textit{
	\begin{itemize}
	\item He surrounded himself with a hand-picked group of bright young officers.
	\item They love being surrounded by familiar possessions.
	\end{itemize}
}
\item countable noun \\
The \textbf{surround} of something such as a fireplace is the border, wall, or shelves around it.
 \textit{
	\begin{itemize}
	\item ...a small fireplace with a cast-iron surround.
	\end{itemize}
}
\item plural noun \\
Your \textbf{surrounds} are your surroundings .
 \textit{
	\begin{itemize}
	\item The entire team enjoyed hot showers in the spacious surrounds of a new, modern village
hall.
	\end{itemize}
}
\end{enumerate}

\section*{suppose}
{\large \color{blue}  supposes  supposing  supposed  }
\subsection*{Explain}
\begin{enumerate}
\item verb \\
You can use \textbf{suppose} or \textbf{supposing} before mentioning a possible situation or action. You usually then go on to consider the effects that this situation or action might have.
 \textit{
	\begin{itemize}
	\item Suppose someone gave you an egg and asked you to describe exactly what was inside.
	\item Supposing he's right and I do die tomorrow? Maybe I should take out an extra insurance
policy.
	\end{itemize}
}
\item verb \\
If you \textbf{suppose}  \textbf{that} something is true, you believe that it is probably true, because of other things that you know .
 \textit{
	\begin{itemize}
	\item The policy is perfectly clear and I see no reason to suppose that it isn't working.
	\item I knew very well that the problem was more complex than he supposed.
	\item It had been supposed that by then Peter would be married.
	\end{itemize}
}
\item  \\
 I suppose \textit{
	\begin{itemize}
	\end{itemize}
}
\item  \\
 I suppose \textit{
	\begin{itemize}
	\end{itemize}
}
\item  \\
 I don't suppose \textit{
	\begin{itemize}
	\end{itemize}
}
\item  \\
 do you suppose \textit{
	\begin{itemize}
	\end{itemize}
}
\item  \\
 do you suppose \textit{
	\begin{itemize}
	\end{itemize}
}
\end{enumerate}

\section*{toss}
{\large \color{blue}  tosses  tossing  tossed  }
\subsection*{Explain}
\begin{enumerate}
\item verb \\
If you \textbf{toss} something somewhere , you throw it there lightly, often in a rather  careless way.
 \textit{
	\begin{itemize}
	\item He screwed the paper into a ball and tossed it into the fire.
	\item He tossed his blanket aside and got up.
	\item He tossed Malone a bottle of water, and took one himself.
	\end{itemize}
}
\item verb \\
If you \textbf{toss} your head or \textbf{toss} your hair , you move your head backwards , quickly and suddenly , often as a way of expressing an emotion such as anger or contempt .
 \textbf{Toss} is also a noun .
 \textit{
	\begin{itemize}
	\item 'I'm sure I don't know.' Cook tossed her head.
	\item Gasping, she tossed her hair out of her face.
	\item With a toss of his head and a few hard gulps, Bob finished the last of his coffee.
	\end{itemize}
}
\item verb \\
In sports and informal situations , if you decide something by \textbf{tossing} a coin, you spin a coin into the air and guess which side of the coin will  face upwards when it lands .
 \textbf{Toss} is also a noun.
 \textit{
	\begin{itemize}
	\item We tossed a coin to decide who would go out and buy the buns.
	\item It would be better to decide it on the toss of a coin.
	\end{itemize}
}
\item singular noun \\
\textbf{The toss} is a way of deciding something, such as who is going to go  first in a game , that consists of spinning a coin into the air and guessing which side of the coin will face upwards when it
lands.
 \textit{
	\begin{itemize}
	\item Bangladesh won the toss and decided to bat first.
	\end{itemize}
}
\item verb \\
If something such as the wind or sea  \textbf{tosses} an object , it causes it to move from side to side or up and down.
 \textit{
	\begin{itemize}
	\item The seas grew turbulent, tossing the small boat like a cork.
	\item As the plane was tossed up and down, the pilot tried to stabilise it.
	\end{itemize}
}
\item verb \\
If you \textbf{toss}  food while preparing it, you put pieces of it into a liquid and lightly shake them so that they become  covered with the liquid.
 \textit{
	\begin{itemize}
	\item Do not toss the salad until you're ready to serve.
	\item Add the grated orange rind and toss the apple slices in the mixture.
	\item Serve straight from the dish with a tossed green salad.
	\end{itemize}
}
\item  \\
 to argue the toss \textit{
	\begin{itemize}
	\end{itemize}
}
\item  \\
 give a toss \textit{
	\begin{itemize}
	\end{itemize}
}
\item  \\
 toss and turn \textit{
	\begin{itemize}
	\end{itemize}
}
\end{enumerate}

\section*{take}
{\large \color{blue}  takes  taking  took  taken  }
\subsection*{Explain}
\begin{enumerate}
\item verb \\
You can use \textbf{take} followed by a noun to talk about an action or event, when it would also be possible
to use the verb that is related to that noun. For example, you can say ' \textbf{she took a shower} ' instead of 'she showered'.
 \textit{
	\begin{itemize}
	\item She was too tired to take a shower.
	\item Betty took a photograph of us.
	\item I've never taken a holiday since starting this job.
	\item There's not enough people willing to take the risk.
	\item Walk around the property and take a good look at it from the outside.
	\item We took a long walk through the pines.
	\end{itemize}
}
\item verb \\
In ordinary spoken or written English, people use \textbf{take} with a range of nouns instead of using a more specific verb. For example people often
say ' \textbf{he took control} ' or ' \textbf{she took a positive attitude} ' instead of 'he assumed control' or 'she adopted a positive attitude'.
 \textit{
	\begin{itemize}
	\item They took power after a three-month civil war.
	\item I felt it was important for women to join and take a leading role.
	\item The constitution requires members of parliament to take an oath of allegiance.
	\item In Asia the crisis took a different form.
	\end{itemize}
}
\end{enumerate}

\section*{transmit}
{\large \color{blue}  transmits  transmitting  transmitted  }
\subsection*{Explain}
\begin{enumerate}
\item verb \\
When radio and television programmes, computer  data , or other electronic  messages  \textbf{are transmitted} , they are sent from one place to another, using wires , radio waves, or satellites .
 \textit{
	\begin{itemize}
	\item The game was transmitted live in Spain and Italy.
	\item The information is electronically transmitted to schools and colleges.
	\item This is currently the most efficient way to transmit certain types of data.
	\item The device is not designed to transmit to satellites.
	\end{itemize}
}
\item verb \\
If one person or animal \textbf{transmits} a disease to another, they have the disease and cause the other person or animal
to have it.
 \textit{
	\begin{itemize}
	\item ...mosquitoes that transmit disease to humans.
	\item There was no danger of transmitting the infection through operations.
	\end{itemize}
}
\item verb \\
If you \textbf{transmit} an idea or feeling to someone else, you make them understand and share the idea or feeling.
 \textit{
	\begin{itemize}
	\item She looked into my eyes like a mother attempting to transmit a painful message.
	\item He transmitted his keen enjoyment of singing to the audience.
	\end{itemize}
}
\item verb \\
If an object or substance \textbf{transmits} something such as sound or electrical signals, the sound or signals are able to pass through it.
 \textit{
	\begin{itemize}
	\item These thin crystals transmit much of the power.
	\item There was no vibration transmitted to the handles and the machine wasn't noisy either.
	\end{itemize}
}
\end{enumerate}

\section*{waste}
{\large \color{blue}  wastes  wasting  wasted  }
\subsection*{Explain}
\begin{enumerate}
\item verb \\
If you \textbf{waste} something such as time, money, or energy, you use too much of it doing something
that is not important or necessary , or is unlikely to succeed .
 \textbf{Waste} is also a noun.
 \textit{
	\begin{itemize}
	\item There could be many reasons and he was not going to waste time speculating on them.
	\item I resolved not to waste money on a hotel.
	\item The system wastes a large amount of water.
	\item It is a waste of time going to the doctor with most mild complaints.
	\item I think that is a total waste of money.
	\end{itemize}
}
\item uncountable noun \\
\textbf{Waste} is the use of money or other resources on things that do not need it.
 \textit{
	\begin{itemize}
	\item The packets are measured to reduce waste.
	\item I hate waste.
	\end{itemize}
}
\item uncountable noun \\
\textbf{Waste} is material which has been used and is no longer wanted , for example because the valuable or useful part of it has been taken out.
 \textit{
	\begin{itemize}
	\item Congress passed a law that regulates the disposal of waste.
	\item Up to 10 million tonnes of toxic wastes are produced every year in the U.K..
	\item ...the process of eliminating body waste.
	\end{itemize}
}
\item verb \\
If you \textbf{waste} an opportunity for something, you do not take advantage of it when it is available.
 \textit{
	\begin{itemize}
	\item Let's not waste an opportunity to see the children.
	\item It was a wasted opportunity.
	\end{itemize}
}
\item verb \\
If you say that something \textbf{is wasted on} someone, you mean that there is no point giving it or telling it to them as they will not appreciate , understand , or pay any attention to it.
 \textit{
	\begin{itemize}
	\item All the well-meant, sincere advice is largely wasted on him.
	\end{itemize}
}
\item adjective \\
\textbf{Waste} land is land, especially in or near a city, which is not used or looked after by anyone, and so is covered
by wild plants and rubbish.
 \textit{
	\begin{itemize}
	\item There was a patch of waste land behind the church.
	\item Yarrow can be found growing wild in fields and on waste ground.
	\end{itemize}
}
\item plural noun \\
\textbf{Wastes} are a large area of land, for example a desert , in which there are very few people, plants, or animals.
 \textit{
	\begin{itemize}
	\item ...the barren wastes of the Sahara.
	\end{itemize}
}
\item  \\
 go to waste \textit{
	\begin{itemize}
	\end{itemize}
}
\item  \\
 to lay waste \textit{
	\begin{itemize}
	\end{itemize}
}
\item  \\
 waste not, want not \textit{
	\begin{itemize}
	\end{itemize}
}
\end{enumerate}

\section*{underline}
{\large \color{blue}  underlines  underlining  underlined  }
\subsection*{Explain}
\begin{enumerate}
\item verb \\
If one thing, for example an action or an event, \textbf{underlines} another, it draws  attention to it and emphasizes its importance .
 \textit{
	\begin{itemize}
	\item The report underlined his concern that standards were at risk.
	\item The decision to keep him in hospital underlines the seriousness of his injury.
	\item But the incident underlines how easily things can go wrong.
	\end{itemize}
}
\item verb \\
If you \textbf{underline} something such as a word or a sentence , you draw a line underneath it in order to make people notice it or to give it extra importance.
 \textit{
	\begin{itemize}
	\item Underline the following that apply to you.
	\item Take two coloured pens and underline the positive and negative words.
	\end{itemize}
}
\end{enumerate}

\section*{worry}
{\large \color{blue}  worries  worrying  worried  }
\subsection*{Explain}
\begin{enumerate}
\item verb \\
If you \textbf{worry} , you keep  thinking about problems that you have or about unpleasant things that might  happen .
 \textit{
	\begin{itemize}
	\item Don't worry, your luggage will come on afterwards by taxi.
	\item I worry about her constantly.
	\item I work in a school so I don't have to worry about finding someone to look after my
little boy.
	\item They worry that extremists might gain control.
	\end{itemize}
}
\item verb \\
If someone or something \textbf{worries} you, they make you anxious because you keep thinking about problems or unpleasant
things that might be connected with them.
 \textit{
	\begin{itemize}
	\item I'm still in the early days of my recovery and that worries me.
	\item 'Why didn't you tell us?'—'I didn't want to worry you.'
	\item The English, worried by the growing power of Prince Henry, sent a raiding party to
Scotland to kill him.
	\item Does it worry you that the Americans are discussing this?
	\end{itemize}
}
\item verb \\
If someone or something does not \textbf{worry} you, you do not dislike them or you are not annoyed by them.
 \textit{
	\begin{itemize}
	\item The cold doesn't worry me.
	\item It wouldn't worry me if I never returned to any of those places.
	\end{itemize}
}
\item uncountable noun \\
\textbf{Worry} is the state or feeling of anxiety and unhappiness caused by the problems that you
have or by thinking about unpleasant things that might happen.
 \textit{
	\begin{itemize}
	\item The admission shows the depth of worry among the Tories over the state of the economy.
	\item His last years were overshadowed by financial worry.
	\end{itemize}
}
\item countable noun \\
A \textbf{worry} is a problem that you keep thinking about and that makes you unhappy .
 \textit{
	\begin{itemize}
	\item My main worry was that Madeleine Johnson would still be there.
	\item The worry is that the use of force could make life impossible for the U.N. peacekeepers.
	\item His wife Cheryl said she had no worries about his health.
	\end{itemize}
}
\item  \\
 not to worry \textit{
	\begin{itemize}
	\end{itemize}
}
\end{enumerate}

\section*{activate}
{\large \color{blue}  activates  activating  activated  }
\subsection*{Explain}
\begin{enumerate}
\item verb \\
If a device or process \textbf{is}  \textbf{activated} , something causes it to start  working .
 \textit{
	\begin{itemize}
	\item Video cameras with night vision can be activated by movement.
	\item ...a voice-activated computer.
	\end{itemize}
}
\end{enumerate}

\section*{alleviate}
{\large \color{blue}  alleviates  alleviating  alleviated  }
\subsection*{Explain}
\begin{enumerate}
\item verb \\
If you \textbf{alleviate} pain, suffering , or an unpleasant condition, you make it less intense or severe .
 \textit{
	\begin{itemize}
	\item Nowadays, a great deal can be done to alleviate back pain.
	\item ...the problem of alleviating mass poverty.
	\end{itemize}
}
\end{enumerate}

\section*{adapt}
{\large \color{blue}  adapts  adapting  adapted  }
\subsection*{Explain}
\begin{enumerate}
\item verb \\
If you \textbf{adapt}  \textbf{to} a new situation or \textbf{adapt}  \textbf{yourself to} it, you change your ideas or behaviour in order to deal with it successfully.
 \textit{
	\begin{itemize}
	\item The world will be different, and we will have to be prepared to adapt to the change.
	\item They have had to adapt themselves to a war economy.
	\end{itemize}
}
\item verb \\
If you \textbf{adapt} something, you change it to make it suitable for a new purpose or situation.
 \textit{
	\begin{itemize}
	\item Shelves were built to adapt the library for use as an office.
	\end{itemize}
}
\item verb \\
If you \textbf{adapt} a book or play , you change it so that it can be made into a film or a television  programme .
 \textit{
	\begin{itemize}
	\item The scriptwriter helped him to adapt his novel for the screen.
	\item The film has been adapted from a play of the same title.
	\end{itemize}
}
\end{enumerate}

\section*{assault}
{\large \color{blue}  assaults  assaulting  assaulted  }
\subsection*{Explain}
\begin{enumerate}
\item countable noun \\
An \textbf{assault} by an army is a strong attack made on an area held by the enemy .
 \textit{
	\begin{itemize}
	\item The rebels are poised for a new assault on the government garrisons.
	\item Most U.S. soldiers welcomed the ground assault when the order was finally given.
	\end{itemize}
}
\item adjective \\
\textbf{Assault}  weapons such as rifles are intended for soldiers to use in battle  rather than for purposes such as hunting .
 \textit{
	\begin{itemize}
	\item ...a high-powered assault rifle.
	\end{itemize}
}
\item variable noun \\
An \textbf{assault}  \textbf{on} a person is a physical attack on them.
 \textit{
	\begin{itemize}
	\item The attack is one of a series of savage sexual assaults on women in the university
area.
	\item At the police station, I was charged with assault.
	\end{itemize}
}
\item verb \\
To \textbf{assault} someone means to physically attack them.
 \textit{
	\begin{itemize}
	\item The gang assaulted him with iron bars.
	\item She may have been sexually assaulted by her killer.
	\end{itemize}
}
\item countable noun \\
An \textbf{assault}  \textbf{on} someone's beliefs is a strong criticism of them.
 \textit{
	\begin{itemize}
	\item He leveled a verbal assault against his Democratic opponents.
	\end{itemize}
}
\end{enumerate}

\section*{advise}
{\large \color{blue}  advises  advising  advised  }
\subsection*{Explain}
\begin{enumerate}
\item verb \\
If you \textbf{advise} someone \textbf{to} do something, you tell them what you think they should do.
 \textit{
	\begin{itemize}
	\item The minister advised him to leave as soon as possible.
	\item Herbert would surely advise her how to approach the bank.
	\item I would strongly advise against it.
	\item Doctors advised that he should be transferred to a private room.
	\end{itemize}
}
\item verb \\
If an expert  \textbf{advises} people \textbf{on} a particular subject, he or she gives them help and information on that subject.
 \textit{
	\begin{itemize}
	\item ...an officer who advises undergraduates on money matters.
	\item A family doctor will be able to advise on suitable birth control.
	\end{itemize}
}
\item verb \\
If you \textbf{advise} someone \textbf{of} a fact or situation , you tell them the fact or explain what the situation is.
 \textit{
	\begin{itemize}
	\item ...the decision requiring police to advise suspects of their rights.
	\item I think it best that I advise you of my decision to retire.
	\end{itemize}
}
\item passive verb \\
If an official document  states that you \textbf{are advised}  \textbf{to} do something, it is telling you the correct or appropriate thing to do.
 \textit{
	\begin{itemize}
	\item Candidates in India are advised to submit their applications through the overseas
student office in London.
	\item Residents are advised not to put their rubbish bags on the pavement outside their
houses.
	\end{itemize}
}
\end{enumerate}

\section*{depict}
{\large \color{blue}  depicts  depicting  depicted  }
\subsection*{Explain}
\begin{enumerate}
\item verb \\
To \textbf{depict} someone or something means to show or represent them in a work of art such as a drawing or painting.
 \textit{
	\begin{itemize}
	\item ...a gallery of pictures depicting Nelson's most famous battles.
	\end{itemize}
}
\item verb \\
To \textbf{depict} someone or something means to describe them or give an impression of them in writing .
 \textit{
	\begin{itemize}
	\item Margaret Atwood's novel depicts a gloomy, futuristic America.
	\item Children's books often depict farmyard animals as gentle, lovable creatures.
	\end{itemize}
}
\end{enumerate}

\section*{afford}
{\large \color{blue}  affords  affording  afforded  }
\subsection*{Explain}
\begin{enumerate}
\item verb \\
If you \textbf{cannot afford} something, you do not have enough money to pay for it.
 \textit{
	\begin{itemize}
	\item My parents can't even afford a new refrigerator.
	\item The arts should be available to more people at prices they can afford.
	\item We couldn't afford to buy a new rug.
	\end{itemize}
}
\item verb \\
If you say that you cannot \textbf{afford}  \textbf{to} do something or allow it to happen , you mean that you must not do it or must prevent it from happening because it would be harmful or embarrassing to you.
 \textit{
	\begin{itemize}
	\item We can't afford to wait.
	\item The country could not afford the luxury of an election.
	\end{itemize}
}
\item verb \\
If someone or something \textbf{affords} you an opportunity or protection , they give it to you.
 \textit{
	\begin{itemize}
	\item This affords us the opportunity to ask questions about how the systems might change.
	\item It was a cold room, but it afforded a fine view of the Old City.
	\item ...the protection afforded by the police.
	\end{itemize}
}
\end{enumerate}

\section*{describe}
{\large \color{blue}  describes  describing  described  }
\subsection*{Explain}
\begin{enumerate}
\item verb \\
If you \textbf{describe} a person, object, event, or situation , you say what they are like or what happened .
 \textit{
	\begin{itemize}
	\item We asked her to describe what kind of things she did in her spare time.
	\item She read a poem by Carver which describes their life together.
	\item The myth of Narcissus is described in Ovid's work.
	\item Just before his death he described seeing their son in a beautiful garden.
	\end{itemize}
}
\item verb \\
If a person \textbf{describes} someone or something \textbf{as} a particular thing, he or she believes that they are that thing and says so.
 \textit{
	\begin{itemize}
	\item He described it as an extraordinarily tangled and complicated tale.
	\item Eriksson described him as 'the best player on the pitch'.
	\item Even his closest allies describe him as forceful, aggressive and determined.
	\item He described the meeting as marking a new stage in the peace process.
	\end{itemize}
}
\item verb \\
If something \textbf{describes} a particular shape, it forms that shape or makes a movement that follows the line of that shape.
 \textit{
	\begin{itemize}
	\item His pass described a perfect arc through the leaden sky.
	\end{itemize}
}
\end{enumerate}

\section*{alter}
{\large \color{blue}  alters  altering  altered  }
\subsection*{Explain}
\begin{enumerate}
\item verb \\
If something \textbf{alters} or if you \textbf{alter} it, it changes.
 \textit{
	\begin{itemize}
	\item Little had altered in the village.
	\item They have never altered their programmes by a single day.
	\end{itemize}
}
\end{enumerate}

\section*{displace}
{\large \color{blue}  displaces  displacing  displaced  }
\subsection*{Explain}
\begin{enumerate}
\item verb \\
If one thing \textbf{displaces} another, it forces the other thing out of its place, position, or role , and then occupies that place, position, or role itself.
 \textit{
	\begin{itemize}
	\item These factories have displaced tourism as the country's largest source of foreign
exchange.
	\item Coal is to be displaced by natural gas and nuclear power.
	\end{itemize}
}
\item verb \\
If a person or group of people \textbf{is displaced} , they are forced to moved away from the area where they live .
 \textit{
	\begin{itemize}
	\item In Europe alone thirty million people were displaced.
	\item Most of the civilians displaced by the war will be unable to return to their homes.
	\item ...the task of resettling refugees and displaced persons.
	\end{itemize}
}
\end{enumerate}

\section*{buy}
{\large \color{blue}  buys  buying  bought  }
\subsection*{Explain}
\begin{enumerate}
\item verb \\
If you \textbf{buy} something, you obtain it by paying money for it.
 \textit{
	\begin{itemize}
	\item He could not afford to buy a house.
	\item They can now be bought fresh in supermarkets.
	\item Lizzie bought herself a mountain bike.
	\item I'd like to buy him lunch.
	\end{itemize}
}
\item verb \\
If you talk about the quantity or standard of goods an amount of money \textbf{buys} , you are referring to the price of the goods or the value of the money.
 \textit{
	\begin{itemize}
	\item About £70,000 buys a habitable house around here.
	\item If the pound's value is high, British investors will spend their money abroad because
the pound will buy them more.
	\end{itemize}
}
\item verb \\
If you \textbf{buy} something like time, freedom , or victory , you obtain it but only by offering or giving up something in return .
 \textit{
	\begin{itemize}
	\item It was a risky operation, but might buy more time.
	\item For them, affluence was bought at the price of less freedom in their work environment.
	\end{itemize}
}
\item verb \\
If you say that a person can  \textbf{be bought} , you are criticizing the fact that they will give their help or loyalty to someone in return for money.
 \textit{
	\begin{itemize}
	\item Once he shows he can be bought, they settle down to a regular payment.
	\end{itemize}
}
\item verb \\
If you \textbf{buy} an idea or a theory , you believe and accept it.
 \textbf{Buy into}  means the same as buy .
 \textit{
	\begin{itemize}
	\item I'm not buying any of that nonsense.
	\item I bought into the popular myth that when I got the next house, I'd finally be happy.
	\end{itemize}
}
\item countable noun \\
If something is a good \textbf{buy} , it is of good quality and not very expensive .
 \textit{
	\begin{itemize}
	\item This was still a good buy even at the higher price.
	\end{itemize}
}
\end{enumerate}

\section*{educate}
{\large \color{blue}  educates  educating  educated  }
\subsection*{Explain}
\begin{enumerate}
\item verb \\
When someone, especially a child, \textbf{is educated} , he or she is taught at a school or college .
 \textit{
	\begin{itemize}
	\item He was educated at Haslingden Grammar School.
	\end{itemize}
}
\item verb \\
To \textbf{educate} people means to teach them better ways of doing something or a better way of living .
 \textit{
	\begin{itemize}
	\item The charity dedicates itself to educating people about the dangers of high-fat diets.

	\end{itemize}
}
\end{enumerate}

\section*{certify}
{\large \color{blue}  certifies  certifying  certified  }
\subsection*{Explain}
\begin{enumerate}
\item verb \\
If someone in an official position \textbf{certifies} something, they officially state that it is true .
 \textit{
	\begin{itemize}
	\item The president certified that the project would receive $650m from overseas sources.
	\item The National Election Council is supposed to certify the results of the election.
	\item It has been certified as genuine.
	\item Mrs Simpson was certified dead.
	\end{itemize}
}
\item verb \\
If someone \textbf{is certified}  \textbf{as} a particular kind of worker , they are given a certificate stating that they have successfully completed a course of training in their profession .
 \textit{
	\begin{itemize}
	\item They wanted to get certified as divers.
	\item ...a certified accountant.
	\item All three doctors are certified as addictions specialists.
	\end{itemize}
}
\end{enumerate}

\section*{examine}
{\large \color{blue}  examines  examining  examined  }
\subsection*{Explain}
\begin{enumerate}
\item verb \\
If you \textbf{examine} something, you look at it carefully.
 \textit{
	\begin{itemize}
	\item He examined her passport and stamped it.
	\end{itemize}
}
\item verb \\
If a doctor  \textbf{examines} you, he or she looks at your body, feels it, or does simple tests in order to check how healthy you are.
 \textit{
	\begin{itemize}
	\item Another doctor examined her and could still find nothing wrong.
	\item He was examined again and then prescribed a different herbal medicine.
	\end{itemize}
}
\item verb \\
If an idea , proposal , or plan  \textbf{is examined} , it is considered very carefully.
 \textit{
	\begin{itemize}
	\item I have given the matter much thought, examining all the possible alternatives.
	\item The plans will be examined by E.U. environment ministers.
	\end{itemize}
}
\item verb \\
If you \textbf{are examined} , you are given a formal test in order to show your knowledge of a subject.
 \textit{
	\begin{itemize}
	\item ...learning to cope with the pressures of being judged and examined by our teachers.
	\end{itemize}
}
\end{enumerate}

\section*{cherish}
{\large \color{blue}  cherishes  cherishing  cherished  }
\subsection*{Explain}
\begin{enumerate}
\item verb \\
If you \textbf{cherish} something such as a hope or a pleasant  memory , you keep it in your mind for a long period of time.
 \textit{
	\begin{itemize}
	\item The president will cherish the memory of this visit to Ohio.
	\item It was a wonderful occasion which we will cherish for many years to come.
	\end{itemize}
}
\item verb \\
If you \textbf{cherish} someone or something, you take good  care of them because you love them.
 \textit{
	\begin{itemize}
	\item He genuinely loved and cherished her.
	\item The previous owners had cherished the house.
	\end{itemize}
}
\item verb \\
If you \textbf{cherish} a right , a privilege , or a principle , you regard it as important and try  hard to keep it.
 \textit{
	\begin{itemize}
	\item These people cherish their independence and sovereignty.
	\end{itemize}
}
\end{enumerate}

\section*{explore}
{\large \color{blue}  explores  exploring  explored  }
\subsection*{Explain}
\begin{enumerate}
\item verb \\
If you \textbf{explore} a place, you travel around it to find out what it is like.
 \textit{
	\begin{itemize}
	\item I just wanted to explore Paris, read Sartre, listen to Sidney Bechet.
	\item After exploring the old part of town there is a guided tour of the cathedral.
	\item We've come to this country, let's explore!
	\end{itemize}
}
\item verb \\
If you \textbf{explore} an idea or suggestion , you think about it or comment on it in detail , in order to assess it carefully.
 \textit{
	\begin{itemize}
	\item The secretary is expected to explore ideas for post-war reconstruction of the area.
	\item The film explores the relationship between artist and instrument.
	\end{itemize}
}
\item verb \\
If people \textbf{explore} an area \textbf{for} a substance such as oil or minerals , they study the area and do tests on the land to see whether they can find it.
 \textit{
	\begin{itemize}
	\item Central to the operation is a mile-deep well, dug originally to explore for oil.
	\item The government is allowing the areas of inshore coastal waters to be explored for
oil and gas.
	\end{itemize}
}
\item verb \\
If you \textbf{explore} something with your hands or fingers , you touch it to find out what it feels like.
 \textit{
	\begin{itemize}
	\item He explored the wound with his finger, trying to establish its extent.
	\end{itemize}
}
\end{enumerate}

\section*{impulse}
{\large \color{blue}  impulses  }
\subsection*{Explain}
\begin{enumerate}
\item variable noun \\
An \textbf{impulse} is a sudden desire to do something.
 \textit{
	\begin{itemize}
	\item Unable to resist the impulse, he glanced at the sea again.
	\item He still couldn't understand the impulse that had made him confide in Cassandra.
	\item Wade resisted an impulse to smile.
	\end{itemize}
}
\item countable noun \\
An \textbf{impulse} is a short electrical  signal that is sent along a wire or nerve or through the air, usually as one of a series .
 \textit{
	\begin{itemize}
	\end{itemize}
}
\item adjective \\
An \textbf{impulse}  buy or \textbf{impulse}  purchase is something that you decide to buy when you see it, although you had not planned to buy it.
 \textit{
	\begin{itemize}
	\item The curtains were an impulse buy.
	\end{itemize}
}
\item  \\
 on impulse \textit{
	\begin{itemize}
	\end{itemize}
}
\end{enumerate}

\section*{dedicate}
{\large \color{blue}  dedicates  dedicating  dedicated  }
\subsection*{Explain}
\begin{enumerate}
\item verb \\
If you say that someone \textbf{has dedicated} themselves \textbf{to} something, you approve of the fact that they have decided to give a lot of time and effort to it because they think that it is important .
 \textit{
	\begin{itemize}
	\item Back on the island, he dedicated himself to politics.
	\item Bessie has dedicated her life to caring for others.
	\end{itemize}
}
\item verb \\
If someone \textbf{dedicates} something such as a book, play, or piece of music  \textbf{to} you, they mention your name, for example in the front of a book or when a piece of music is performed , as a way of showing affection or respect for you.
 \textit{
	\begin{itemize}
	\item She dedicated her first album to Woody Allen.
	\item This book is dedicated to the memory of my mother.
	\end{itemize}
}
\item verb \\
If a building or church  \textbf{is dedicated}  \textbf{to} someone, a formal  ceremony is held to show that the building will  always be associated with them.
 \textit{
	\begin{itemize}
	\item The church is dedicated to St Mary of Bec.
	\end{itemize}
}
\end{enumerate}

\section*{infer}
{\large \color{blue}  infers  inferring  inferred  }
\subsection*{Explain}
\begin{enumerate}
\item verb \\
If you \textbf{infer} that something is the case , you decide that it is true on the basis of information that you already have.
 \textit{
	\begin{itemize}
	\item I inferred from what she said that you have not been well.
	\item By measuring the motion of the galaxies in a cluster, astronomers can infer the cluster's
mass.
	\end{itemize}
}
\item verb \\
Some people use \textbf{infer} to mean 'imply', but many people consider this use to be incorrect .
 \textit{
	\begin{itemize}
	\item The police inferred that they found her behaviour rather suspicious.
	\end{itemize}
}
\end{enumerate}

\section*{delete}
{\large \color{blue}  deletes  deleting  deleted  }
\subsection*{Explain}
\begin{enumerate}
\item verb \\
If you \textbf{delete} something that has been written down or stored in a computer , you cross it out or remove it.
 \textit{
	\begin{itemize}
	\item He also deleted files from the computer system.
	\item The word 'exploded' had been deleted.
	\end{itemize}
}
\end{enumerate}

\section*{instruct}
{\large \color{blue}  instructs  instructing  instructed  }
\subsection*{Explain}
\begin{enumerate}
\item verb \\
If you \textbf{instruct} someone to do something, you formally tell them to do it.
 \textit{
	\begin{itemize}
	\item The family has instructed solicitors to sue Thomson for compensation.
	\item 'Go and have a word with her, Ken,' Webb instructed.
	\item I want you to instruct them that they've got three months to get the details sorted
out.
	\end{itemize}
}
\item verb \\
Someone who \textbf{instructs} people \textbf{in} a subject or skill teaches it to them.
 \textit{
	\begin{itemize}
	\item He instructed family members in nursing techniques.
	\end{itemize}
}
\end{enumerate}

\section*{denote}
{\large \color{blue}  denotes  denoting  denoted  }
\subsection*{Explain}
\begin{enumerate}
\item verb \\
If one thing \textbf{denotes} another, it is a sign or indication of it.
 \textit{
	\begin{itemize}
	\item Red eyes denote strain and fatigue.
	\item There was a message waiting, denoting that someone had been here ahead of her.
	\end{itemize}
}
\item verb \\
What a symbol \textbf{denotes} is what it represents .
 \textit{
	\begin{itemize}
	\item In figure 24 'D' denotes quantity demanded and 'S' denotes quantity supplied.
	\end{itemize}
}
\item verb \\
What a word or name \textbf{denotes} is what it means or refers to.
 \textit{
	\begin{itemize}
	\item In the Middle Ages the term 'drab' denoted a very simple type of woollen cloth which
was used by peasants to make their clothes.
	\end{itemize}
}
\end{enumerate}

\section*{lead}
{\large \color{blue}  leads  leading  led  }
\subsection*{Explain}
\begin{enumerate}
\item verb \\
If you \textbf{lead} a group of people, you walk or ride in front of them.
 \textit{
	\begin{itemize}
	\item John Major and the Duke of Edinburgh led the mourners.
	\item He walks with a stick but still leads his soldiers into battle.
	\item Tom was leading, a rifle slung over his back.
	\end{itemize}
}
\item verb \\
If you \textbf{lead} someone to a particular place or thing, you take them there.
 \textit{
	\begin{itemize}
	\item He took Dickon by the hand to lead him into the house.
	\item She confessed to the killing and led police to his remains.
	\item Leading the horse, Evandar walked to the door.
	\end{itemize}
}
\item verb \\
If a road, gate , or door \textbf{leads}  somewhere , you can get there by following the road or going through the gate or door.
 \textit{
	\begin{itemize}
	\item ...the doors that led to the yard.
	\item ...a short roadway leading to the car park.
	\item Hundreds of people are said to have blocked a main highway leading north.
	\end{itemize}
}
\item verb \\
If you \textbf{are leading} at a particular point in a race or competition, you are winning at that point.
 \textit{
	\begin{itemize}
	\item He's leading in the presidential race.
	\item So far he leads by five wins to two.
	\item Aston Villa last led the League in March 1990.
	\end{itemize}
}
\item singular noun \\
If you have \textbf{the}  \textbf{lead} or are \textbf{in the}  \textbf{lead} in a race or competition, you are winning.
 \textit{
	\begin{itemize}
	\item England took the lead after 31 minutes.
	\item Labour are still in the lead in the opinion polls.
	\end{itemize}
}
\item singular noun \\
Someone's \textbf{lead}  \textbf{over} a competitor at a particular point in a race or competition is the distance, amount of time, or
number of points by which they are ahead of them.
 \textit{
	\begin{itemize}
	\item ...a commanding lead for the opposition is clearly emerging throughout the country.
	\item His goal gave Forest a two-goal lead against Southampton.
	\item Sainz now has a lead of 28 points.
	\end{itemize}
}
\item verb \\
If one company or country \textbf{leads} others in a particular activity such as scientific research or business, it is more
 successful or advanced than they are in that activity.
 \textit{
	\begin{itemize}
	\item When it comes to pop music we not only lead Europe, we lead the world.
	\item ...foodstores such as Marks & Spencer, which led the market in microwaveable meals.
	\end{itemize}
}
\item verb \\
If you \textbf{lead} a group of people, an organization, or an activity, you are in control or in charge
of the people or the activity.
 \textit{
	\begin{itemize}
	\item He led the country between 1949 and 1984.
	\item Mr Mendes was leading a campaign to save Brazil's rainforest from exploitation.
	\end{itemize}
}
\item countable noun \\
If you give a \textbf{lead} , you do something new or develop new ideas or methods that other people consider
to be a good example or model to follow.
 \textit{
	\begin{itemize}
	\item ...the need for the president to give a moral lead.
	\item The American and Japanese navies took the lead in the development of naval aviation.
	\item Over the next 150 years, many others followed his lead.
	\end{itemize}
}
\item verb \\
You can use \textbf{lead} when you are saying what kind of life someone has. For example, if you \textbf{lead} a busy life, your life is busy.
 \textit{
	\begin{itemize}
	\item She led a normal, happy life with her sister and brother.
	\item Most of the women in here are not people who have led a life of crime.
	\end{itemize}
}
\item verb \\
If something \textbf{leads}  \textbf{to} a situation or event, usually an unpleasant one, it begins a process which causes
that situation or event to happen .
 \textit{
	\begin{itemize}
	\item Ethnic tensions among the republics could lead to civil war.
	\item He warned yesterday that a pay rise for teachers would lead to job cuts.
	\end{itemize}
}
\item verb \\
If something \textbf{leads} you \textbf{to} do something, it influences or affects you in such a way that you do it.
 \textit{
	\begin{itemize}
	\item His abhorrence of racism led him to write The Algiers Motel Incident.
	\item What was it ultimately that led you to leave Sarajevo for Zagreb?
	\end{itemize}
}
\item verb \\
If you say that someone or something \textbf{led} you \textbf{to} think something, you mean that they caused you to think it, although it was not true
or did not happen.
 \textit{
	\begin{itemize}
	\item Mother had led me to believe the new baby was a kind of present for me.
	\item It was not as straightforward as we were led to believe.
	\end{itemize}
}
\item verb \\
If you \textbf{lead} a conversation or discussion , you control the way that it develops so that you can introduce a particular subject.
 \textit{
	\begin{itemize}
	\item After a while I led the conversation around to her job.
	\item He planned to lead the conversation and keep Matt from changing the subject.
	\end{itemize}
}
\item verb \\
You can say that one point or topic in a discussion or piece of writing \textbf{leads} you \textbf{to} another in order to introduce a new point or topic that is linked with the previous one.
 \textit{
	\begin{itemize}
	\item Well, I think that leads me to the real point.
	\end{itemize}
}
\item countable noun \\
A \textbf{lead} is a piece of information or an idea which may help people to discover the facts
in a situation where many facts are not known, for example in the investigation of a crime or in a scientific experiment .
 \textit{
	\begin{itemize}
	\item The inquiry team is also following up possible leads after receiving 400 calls from
the public.
	\end{itemize}
}
\item countable noun \\
\textbf{The}  \textbf{lead} in a play, film, or show is the most important part in it. The person who plays this
part can also be called the \textbf{lead} .
 \textit{
	\begin{itemize}
	\item Performers from the Bolshoi Ballet dance the leads.
	\item Both the leads in the play are impressive.
	\end{itemize}
}
\item countable noun \\
A dog's \textbf{lead} is a long, thin chain or piece of leather which you attach to the dog's collar so that you can control the dog.
 \textit{
	\begin{itemize}
	\item An older man came out with a little dog on a lead.
	\end{itemize}
}
\item countable noun \\
A \textbf{lead} in a piece of equipment is a piece of wire covered in plastic which supplies electricity to the equipment or carries it from one part of the equipment to another.
 \textit{
	\begin{itemize}
	\end{itemize}
}
\item singular noun \\
The \textbf{lead story} or \textbf{lead} in a newspaper or on the television or radio news is the most important story.
 \textit{
	\begin{itemize}
	\item The economic situation makes the lead in tomorrow's Guardian.
	\item Cossiga's reaction is the lead story in the Italian press.
	\end{itemize}
}
\end{enumerate}

\section*{diagnose}
{\large \color{blue}  diagnoses  diagnosing  diagnosed  }
\subsection*{Explain}
\begin{enumerate}
\item verb \\
If someone or something \textbf{is diagnosed}  \textbf{as} having a particular illness or problem , their illness or problem is identified . If an illness or problem \textbf{is diagnosed} , it is identified.
 \textit{
	\begin{itemize}
	\item The soldiers were diagnosed as having flu.
	\item Susan had a mental breakdown and was diagnosed with schizophrenia.
	\item In 1894 her illness was diagnosed as cancer.
	\item He could diagnose an engine problem simply by listening.
	\item This disorder is easily diagnosed but not so easily treated.
	\end{itemize}
}
\end{enumerate}

\section*{move}
{\large \color{blue}  moves  moving  moved  }
\subsection*{Explain}
\begin{enumerate}
\item verb \\
When you \textbf{move} something or when it \textbf{moves} , its position changes and it does not remain  still .
 \textit{
	\begin{itemize}
	\item She moved the sheaf of papers into position.
	\item You can move the camera both vertically and horizontally.
	\item A traffic warden asked him to move his car.
	\item I could see the branches of the trees moving back and forth.
	\item The train began to move.
	\end{itemize}
}
\item verb \\
When you \textbf{move} , you change your position or go to a different place.
 \textbf{Move} is also a noun .
 \textit{
	\begin{itemize}
	\item She waited for him to get up, but he didn't move.
	\item There was so much furniture you could hardly move without bumping into something.
	\item He moved around the room, putting his possessions together.
	\item She moved away from the window.
	\item The doctor made a move towards the door.
	\item Daniel's eyes followed her every move.
	\end{itemize}
}
\item verb \\
If you \textbf{move} , you act or you begin to do something.
 \textit{
	\begin{itemize}
	\item Industrialists must move fast to take advantage of these new opportunities.
	\end{itemize}
}
\item countable noun \\
A \textbf{move} is an action that you take in order to achieve something.
 \textit{
	\begin{itemize}
	\item The one point cut in interest rates was a wise move.
	\item It may also be a good move to suggest she talks things over.
	\item The thirty-five member nations agreed to the move.
	\item Her latest disappearing act may be no more than a stunt, or a smart career move.
	\end{itemize}
}
\item verb \\
If a person or company \textbf{moves} , they leave the building where they have been living or working , and they go to live or work in a different place, taking their possessions with them.
 \textbf{Move} is also a noun.
 \textit{
	\begin{itemize}
	\item My family home is in Yorkshire and they don't want to move.
	\item She had often considered moving to London.
	\item They move house fairly frequently.
	\item The London Evening Standard moved offices a few years ago.
	\item Modigliani announced his move to Montparnasse in 1909.
	\end{itemize}
}
\item verb \\
If people in authority \textbf{move} someone, they make that person go from one place or job to another one.
 \textit{
	\begin{itemize}
	\item His superiors moved him to another parish.
	\item Ms Clark is still in position and there are no plans to move her.
	\item The family had to be moved because of an attack on their home.
	\end{itemize}
}
\item verb \\
If you \textbf{move}  \textbf{from} one job or interest \textbf{to} another, you change to it.
 \textbf{Move} is also a noun.
 \textit{
	\begin{itemize}
	\item He moved from being an extramural tutor to being a lecturer in social history.
	\item In the early days Christina moved jobs to get experience.
	\item His move to the chairmanship means he will take a less active role in day-to-day
management.
	\end{itemize}
}
\item verb \\
If you \textbf{move}  \textbf{to} a new topic in a conversation , you start  talking about something different.
 \textit{
	\begin{itemize}
	\item Let's move to another subject, Dan.
	\end{itemize}
}
\item verb \\
If you \textbf{move} an event or the date of an event, you change the time at which it happens .
 \textit{
	\begin{itemize}
	\item The club has moved its meeting to Saturday, January 22nd.
	\item The band have moved forward their Leeds date to October 27.
	\end{itemize}
}
\item verb \\
If you \textbf{move} towards a particular state, activity, or opinion , you start to be in that state, do that activity, or have that opinion.
 \textbf{Move} is also a noun.
 \textit{
	\begin{itemize}
	\item There is no doubt that the party has moved to the right.
	\item It is already possible to start moving toward the elimination of nuclear weapons.
	\item Since the Convention was drawn up, international opinion has begun to move against
it.
	\item His move to the left was not a sudden leap but a natural working out of ideas.
	\end{itemize}
}
\item verb \\
If a situation or process \textbf{is moving} , it is developing or progressing, rather than staying still.
 \textit{
	\begin{itemize}
	\item Events are moving fast.
	\item Someone has got to get things moving.
	\end{itemize}
}
\item verb \\
If you say that you will not \textbf{be moved} , you mean that you have come to a decision and nothing will change your mind .
 \textit{
	\begin{itemize}
	\item Everyone thought I was mad to go back, but I wouldn't be moved.
	\end{itemize}
}
\item verb \\
If something \textbf{moves} you \textbf{to} do something, it influences you and causes you to do it.
 \textit{
	\begin{itemize}
	\item It was punk that first moved him to join a band seriously.
	\item The president was moved to come up with these suggestions after the hearings.
	\end{itemize}
}
\item verb \\
If something \textbf{moves} you, it has an effect on your emotions and causes you to feel sadness or sympathy for another person.
 \textit{
	\begin{itemize}
	\item These stories surprised and moved me.
	\item His prayer moved me to tears.
	\end{itemize}
}
\item verb \\
If you say that someone \textbf{moves in} a particular society, circle, or world, you mean that they know people in a particular social class or group and spend most of their time with them.
 \textit{
	\begin{itemize}
	\item She moves in high-society circles in London.
	\end{itemize}
}
\item verb \\
At a meeting , if you \textbf{move} a motion, you formally suggest it so that everyone present can vote on it.
 \textit{
	\begin{itemize}
	\item Labour quickly moved a closure motion to end the debate.
	\item I move that the case be dismissed.
	\end{itemize}
}
\item countable noun \\
A \textbf{move} is an act of putting a chess piece or other counter in a different position on a board when it is your turn to do so in a game.
 \textit{
	\begin{itemize}
	\item With no idea of what to do for my next move, my hand hovered over the board.
	\end{itemize}
}
\item  \\
 one/a false move \textit{
	\begin{itemize}
	\end{itemize}
}
\item  \\
 get a move on \textit{
	\begin{itemize}
	\end{itemize}
}
\item  \\
 to make a move \textit{
	\begin{itemize}
	\end{itemize}
}
\item  \\
 to make a move \textit{
	\begin{itemize}
	\end{itemize}
}
\item  \\
 on the move \textit{
	\begin{itemize}
	\end{itemize}
}
\end{enumerate}

\section*{disable}
{\large \color{blue}  disables  disabling  disabled  }
\subsection*{Explain}
\begin{enumerate}
\item verb \\
If an injury or illness  \textbf{disables} someone, it affects them so badly that it restricts the way that they can live their life.
 \textit{
	\begin{itemize}
	\item She did all this tendon damage and it really disabled her.
	\item One in five men will be disabled by a stroke.
	\end{itemize}
}
\item verb \\
If someone or something \textbf{disables} a system or mechanism , they stop it working , usually temporarily.
 \textit{
	\begin{itemize}
	\item ...if you need to disable a car alarm.
	\end{itemize}
}
\end{enumerate}

\section*{overhear}
{\large \color{blue}  overhears  overhearing  overheard  }
\subsection*{Explain}
\begin{enumerate}
\item verb \\
If you \textbf{overhear} someone, you hear what they are saying when they are not talking to you and they do not know that you are listening .
 \textit{
	\begin{itemize}
	\item I overheard two doctors discussing my case.
	\item ...snatches of overheard conversation.
	\end{itemize}
}
\end{enumerate}

\section*{divert}
{\large \color{blue}  diverts  diverting  diverted  }
\subsection*{Explain}
\begin{enumerate}
\item verb \\
To \textbf{divert}  vehicles or travellers  means to make them follow a different  route or go to a different destination than they originally  intended . You can also  say that someone or something \textbf{diverts}  \textbf{from} a particular route or \textbf{to} a particular place.
 \textit{
	\begin{itemize}
	\item ...Rainham Marshes, east London, where a new bypass will divert traffic from the
A13.
	\item During the strike, ambulances will be diverted to private hospitals.
	\item We diverted a plane to rescue 100 passengers.
	\item She insists on diverting to a village close to the airport.
	\item The capital remained jammed with diverted traffic.
	\end{itemize}
}
\item verb \\
To \textbf{divert}  money or resources means to cause them to be used for a different purpose .
 \textit{
	\begin{itemize}
	\item The government is trying to divert more public funds from west to east.
	\item ...government departments involved in diverting resources into community care.
	\end{itemize}
}
\item verb \\
To \textbf{divert} a phone  call means to send it to a different number or place from the one that was dialled by the person making the call.
 \textit{
	\begin{itemize}
	\item He instructed switchboard staff to divert all Laura's calls to him.
	\item Customers will only incur additional call charges if the call is diverted outside
the U.K..
	\end{itemize}
}
\item verb \\
If you say that someone \textbf{diverts} your attention from something important or serious , you disapprove of them behaving or talking in a way that stops you thinking about it.
 \textit{
	\begin{itemize}
	\item They want to divert the attention of the people from the real issues.
	\item The President needed to divert attention away from his own economic record.
	\end{itemize}
}
\end{enumerate}

\section*{peep}
{\large \color{blue}  peeps  peeping  peeped  }
\subsection*{Explain}
\begin{enumerate}
\item verb \\
If you \textbf{peep} , or \textbf{peep at} something, you have a quick look at it, often secretly and quietly .
 \textbf{Peep} is also a noun .
 \textit{
	\begin{itemize}
	\item Children came to peep at him round the doorway.
	\item Now and then she peeped to see if he was noticing her.
	\item 'Fourteen minutes,' Chris said, taking a peep at his watch.
	\end{itemize}
}
\item verb \\
If something \textbf{peeps} out from behind or under something, a small part of it is visible or becomes visible.
 \textit{
	\begin{itemize}
	\item Purple and yellow flowers peeped up between rocks.
	\item Here and there a face peeped out from the shop doorway.
	\end{itemize}
}
\item  \\
 hear a peep \textit{
	\begin{itemize}
	\end{itemize}
}
\end{enumerate}

\section*{embarrass}
{\large \color{blue}  embarrasses  embarrassing  embarrassed  }
\subsection*{Explain}
\begin{enumerate}
\item verb \\
If something or someone \textbf{embarrasses} you, they make you feel shy or ashamed .
 \textit{
	\begin{itemize}
	\item His clumsiness embarrassed him.
	\item It embarrassed him that he had no idea of what was going on.
	\end{itemize}
}
\item verb \\
If something \textbf{embarrasses} a public figure such as a politician or an organization such as a political party, it causes problems for them.
 \textit{
	\begin{itemize}
	\item ...an attempt to find out who had leaked information that embarrassed the president.
	\item The Government has been embarrassed by the affair.
	\end{itemize}
}
\end{enumerate}

\section*{portray}
{\large \color{blue}  portrays  portraying  portrayed  }
\subsection*{Explain}
\begin{enumerate}
\item verb \\
When an actor or actress  \textbf{portrays} someone, he or she plays that person in a play or film.
 \textit{
	\begin{itemize}
	\item In 1975 he portrayed the king in a Los Angeles revival of 'Camelot'.
	\item At first Glenn Miller, brilliantly portrayed by Jimmy Stewart, is sunk into gloom.
	\end{itemize}
}
\item verb \\
When a writer or artist  \textbf{portrays} something, he or she writes a description or produces a painting of it.
 \textit{
	\begin{itemize}
	\item ...this northern novelist, who accurately portrays provincial domestic life.
	\item ...the landscape as portrayed by painters such as Claude and Poussin.
	\end{itemize}
}
\item verb \\
If a film, book , or television  programme  \textbf{portrays} someone in a certain way , it represents them in that way.
 \textit{
	\begin{itemize}
	\item She says the programme portrayed her as a 'lady of easy virtue'.
	\item ...complaints about the way women are portrayed in adverts.
	\end{itemize}
}
\end{enumerate}

\section*{embed}
{\large \color{blue}  embeds  embedding  embedded  }
\subsection*{Explain}
\begin{enumerate}
\item verb \\
If an object \textbf{embeds}  \textbf{itself} in a substance or thing, it becomes fixed there firmly and deeply.
 \textit{
	\begin{itemize}
	\item The bullet blew off the tip of his forefinger before embedding itself in the wall
beside Williams' head.
	\end{itemize}
}
\item verb \\
If something such as an attitude or feeling  \textbf{is embedded}  \textbf{in} a society or system, or in someone's personality , it becomes a permanent and noticeable  feature of it.
 \textit{
	\begin{itemize}
	\item This agreement will be embedded in a state treaty to be signed soon.
	\end{itemize}
}
\end{enumerate}

\section*{postpone}
{\large \color{blue}  postpones  postponing  postponed  }
\subsection*{Explain}
\begin{enumerate}
\item verb \\
If you \textbf{postpone} an event , you delay it or arrange for it to take place at a later time than was originally  planned .
 \textit{
	\begin{itemize}
	\item He decided to postpone the expedition until the following day.
	\item The visit has now been postponed indefinitely.
	\end{itemize}
}
\end{enumerate}

\section*{embody}
{\large \color{blue}  embodies  embodying  embodied  }
\subsection*{Explain}
\begin{enumerate}
\item verb \\
To \textbf{embody} an idea or quality means to be a symbol or expression of that idea or quality.
 \textit{
	\begin{itemize}
	\item Jack Kennedy embodied all the hopes of the 1960s.
	\item For twenty-nine years, Checkpoint Charlie embodied the Cold War.
	\item That stability was embodied in the Gandhi family.
	\end{itemize}
}
\item verb \\
If something \textbf{is embodied}  \textbf{in} a particular thing, the second thing contains or consists of the first.
 \textit{
	\begin{itemize}
	\item The proposal has been embodied in a draft resolution.
	\item U.K. employment law embodies arbitration mechanisms to settle industrial disputes.
	\item ...In the British system the executive is supposedly embodied by the Crown and the
legislative by Parliament
	\end{itemize}
}
\end{enumerate}

\section*{precede}
{\large \color{blue}  precedes  preceding  preceded  }
\subsection*{Explain}
\begin{enumerate}
\item verb \\
If one event or period of time \textbf{precedes} another, it happens before it.
 \textit{
	\begin{itemize}
	\item Intensive negotiations between the main parties preceded the vote.
	\item The earthquake was preceded by a loud roar and lasted 20 seconds.
	\item Industrial orders had already fallen in the preceding months.
	\end{itemize}
}
\item verb \\
If you \textbf{precede} someone somewhere , you go in front of them.
 \textit{
	\begin{itemize}
	\item He gestured to Alice to precede them from the room.
	\item They were preceded by mounted cowboys.
	\end{itemize}
}
\item verb \\
A sentence , paragraph , or chapter that \textbf{precedes} another one comes just before it.
 \textit{
	\begin{itemize}
	\item Look at the information that precedes the paragraph in question.
	\item Repeat the exercises described in the preceding section.
	\end{itemize}
}
\end{enumerate}

\section*{entail}
{\large \color{blue}  entails  entailing  entailed  }
\subsection*{Explain}
\begin{enumerate}
\item verb \\
If one thing \textbf{entails} another, it involves it or causes it.
 \textit{
	\begin{itemize}
	\item Such a decision would entail a huge political risk.
	\item The changed outlook entails higher economic growth than was previously assumed.
	\item The job of a choreologist entails teaching the performance of dance movements.
	\item I'll never accept parole because that entails me accepting guilt.
	\end{itemize}
}
\end{enumerate}

\section*{preclude}
{\large \color{blue}  precludes  precluding  precluded  }
\subsection*{Explain}
\begin{enumerate}
\item verb \\
If something \textbf{precludes} an event or action, it prevents the event or action from happening .
 \textit{
	\begin{itemize}
	\item At 84, John feels his age precludes too much travel.
	\item He would rebuff enquiries in such a way as to preclude any further discussion.
	\end{itemize}
}
\item verb \\
If something \textbf{precludes} you \textbf{from} doing something or going  somewhere , it prevents you from doing it or going there.
 \textit{
	\begin{itemize}
	\item A constitutional amendment precludes any president from serving more than two terms.
	\item In some cases poor English precluded them from ever finding a job.
	\end{itemize}
}
\end{enumerate}

\section*{enter}
{\large \color{blue}  enters  entering  entered  }
\subsection*{Explain}
\begin{enumerate}
\item verb \\
When you \textbf{enter} a place such as a room or building , you go into it or come into it.
 \textit{
	\begin{itemize}
	\item He entered the room briskly and stood near the door.
	\item Before entering the bathroom, he emptied his dirty laundry into the hamper.
	\item As soon as I entered, they stopped and turned my way.
	\end{itemize}
}
\item verb \\
If you \textbf{enter} an organization or institution , you start to work there or become a member of it.
 \textit{
	\begin{itemize}
	\item He entered the BBC as a general trainee.
	\item She entered a convent.
	\end{itemize}
}
\item verb \\
If something new  \textbf{enters} your mind , you suddenly  think about it.
 \textit{
	\begin{itemize}
	\item Whenever thoughts of his baby daughter enter his mind a smile appears on Jeremy's
face.
	\item Dreadful doubts began to enter my mind.
	\end{itemize}
}
\item verb \\
If it does not \textbf{enter} your head  \textbf{to} do, think or say something, you do not think of doing that thing although you should have done .
 \textit{
	\begin{itemize}
	\item It never enters his mind that anyone is better than him.
	\item Though she enjoyed flirting with Matt, it had not entered her head to have an affair
with him.
	\end{itemize}
}
\item verb \\
If someone or something \textbf{enters} a particular  situation or period of time, they start to be in it or part of it.
 \textit{
	\begin{itemize}
	\item China enters a new five-year plan period next year.
	\item The war has entered its second month.
	\item A million young people enter the labour market each year.
	\item The phrase has already entered the language.
	\end{itemize}
}
\item verb \\
If you \textbf{enter} a competition , race , or examination , you officially state that you will  compete or take part in it.
 \textit{
	\begin{itemize}
	\item I run so well I'm planning to enter some races.
	\item He entered for many competitions, winning several gold medals.
	\item To enter, simply complete the coupon on page 150.
	\end{itemize}
}
\item verb \\
If you \textbf{enter} someone \textbf{for} a race or competition, you officially state that they will compete or take part in
it.
 \textit{
	\begin{itemize}
	\item Marie secretly entered him for the Championship.
	\item ...some of the 150 projects entered for the awards.
	\end{itemize}
}
\item verb \\
If you \textbf{enter} something in a notebook , register, or financial account, you write it down.
 \textit{
	\begin{itemize}
	\item Each week she meticulously entered in her notebooks all sums received.
	\item Prue entered the passage in her notebook, then read it aloud again.
	\end{itemize}
}
\item verb \\
To \textbf{enter}  information  \textbf{into} a computer or database  means to record it there, for example by typing it on a keyboard .
 \textit{
	\begin{itemize}
	\item When a baby is born, they enter that baby's name into the computer.
	\item Postcodes will be entered into the statisticians' computers.
	\item A lot less time is now spent entering the data.
	\end{itemize}
}
\end{enumerate}

\section*{presume}
{\large \color{blue}  presumes  presuming  presumed  }
\subsection*{Explain}
\begin{enumerate}
\item verb \\
If you \textbf{presume}  \textbf{that} something is the case , you think that it is the case, although you are not certain .
 \textit{
	\begin{itemize}
	\item I presume you're here on business.
	\item Dido's told you the whole sad story, I presume?
	\item 'Had he been home all week?'—'I presume so.'
	\item ...areas that have been presumed to be safe.
	\item The missing person is presumed dead.
	\end{itemize}
}
\item verb \\
If you say that someone \textbf{presumes}  \textbf{to} do something, you mean that they do it even though they have no right to do it.
 \textit{
	\begin{itemize}
	\item They're resentful that outsiders presume to meddle in their affairs.
	\item I wouldn't presume to question your judgment.
	\end{itemize}
}
\item verb \\
If an idea , theory , or plan  \textbf{presumes} certain facts , it regards them as true so that they can be used as a basis for further ideas and theories.
 \textit{
	\begin{itemize}
	\item The legal definition of 'know' often presumes mental control.
	\item The arrangement presumes that both lenders and borrowers are rational.
	\end{itemize}
}
\end{enumerate}

\section*{escort}
{\large \color{blue}  escorts  escorting  escorted  }
\subsection*{Explain}
\begin{enumerate}
\item countable noun \\
An \textbf{escort} is a person who travels with someone in order to protect or guard them.
 \textit{
	\begin{itemize}
	\item He arrived with a police escort shortly before half past nine.
	\end{itemize}
}
\item countable noun \\
An \textbf{escort} is a person who accompanies another person of the opposite  sex to a social event. Sometimes people are paid to be escorts.
 \textit{
	\begin{itemize}
	\item My sister needed an escort for a company dinner.
	\end{itemize}
}
\item verb \\
If you \textbf{escort} someone somewhere , you accompany them there, usually in order to make sure that they leave a place or get to their destination .
 \textit{
	\begin{itemize}
	\item I escorted him to the door.
	\item The vessel was escorted to an undisclosed port.
	\end{itemize}
}
\end{enumerate}

\section*{pretend}
{\large \color{blue}  pretends  pretending  pretended  }
\subsection*{Explain}
\begin{enumerate}
\item verb \\
If you \textbf{pretend}  \textbf{that} something is the case , you act in a way that is intended to make people believe that it is the case, although in fact it is not.
 \textit{
	\begin{itemize}
	\item I pretend that things are really okay when they're not.
	\item Sometimes the boy pretended to be asleep.
	\item I had no option but to pretend ignorance.
	\end{itemize}
}
\item verb \\
If children or adults  \textbf{pretend}  \textbf{that} they are doing something, they imagine that they are doing it, for example as part of a game .
 \textit{
	\begin{itemize}
	\item She can sunbathe and pretend she's in Spain.
	\item The children pretend to be different animals dancing to the music.
	\end{itemize}
}
\item verb \\
If you do not \textbf{pretend}  \textbf{that} something is the case, you do not claim that it is the case.
 \textit{
	\begin{itemize}
	\item We do not pretend that the past six years have been without problems for us.
	\item Within this lecture I cannot pretend to deal adequately with dreams.
	\end{itemize}
}
\end{enumerate}

\section*{flatter}
{\large \color{blue}  flatters  flattering  flattered  }
\subsection*{Explain}
\begin{enumerate}
\item verb \\
If someone \textbf{flatters} you, they praise you in an exaggerated way that is not sincere , because they want to please you or to persuade you to do something.
 \textit{
	\begin{itemize}
	\item I knew she was just flattering me.
	\item ...a story of how the president flattered and feted him into taking his side.
	\end{itemize}
}
\item verb \\
If you \textbf{flatter}  \textbf{yourself that} something good is the case , you believe that it is true , although others may  disagree . If someone says to you ' \textbf{you're flattering yourself} ' or ' \textbf{don't flatter yourself} ', they mean that they disagree with your good opinion of yourself.
 \textit{
	\begin{itemize}
	\item I flatter myself that this campaign will put an end to the war.
	\item I flatter myself I've done it all rather well.
	\item You flatter yourself. Why would we go to such ludicrous lengths?
	\end{itemize}
}
\item verb \\
If something \textbf{flatters} you, it makes you appear more attractive.
 \textit{
	\begin{itemize}
	\item Orange and khaki flatter those with golden skin tones.
	\item My philosophy of fashion is that I like to make clothes that flatter.
	\end{itemize}
}
\end{enumerate}

\section*{push}
{\large \color{blue}  pushes  pushing  pushed  }
\subsection*{Explain}
\begin{enumerate}
\item verb \\
When you \textbf{push} something, you use force to make it move away from you or away from its previous position.
 \textbf{Push} is also a noun .
 \textit{
	\begin{itemize}
	\item The woman pushed back her chair and stood up.
	\item They pushed him into the car.
	\item ...a woman pushing a pushchair.
	\item He put both hands flat on the door and pushed as hard as he could.
	\item When there was no reply, he pushed the door open.
	\item He gave me a sharp push.
	\item Information is called up at the push of a button.
	\end{itemize}
}
\item verb \\
If you \textbf{push through} things that are blocking your way or \textbf{push} your \textbf{way through} them, you use force in order to move past them.
 \textit{
	\begin{itemize}
	\item I pushed through the crowds and on to the escalator.
	\item Dix pushed forward carrying a glass.
	\item He pushed his way towards her, laughing.
	\end{itemize}
}
\item verb \\
If an army \textbf{pushes into} a country or area that it is attacking or invading , it moves further into it.
 \textbf{Push} is also a noun.
 \textit{
	\begin{itemize}
	\item One detachment pushed into the eastern suburbs towards the airfield.
	\item The army may push southwards into the Kurdish areas.
	\item All that was needed was one final push, and the enemy would be vanquished once and
for all.
	\end{itemize}
}
\item verb \\
To \textbf{push} a value or amount \textbf{up} or \textbf{down} means to cause it to increase or decrease .
 \textit{
	\begin{itemize}
	\item Any shortage could push up grain prices.
	\item The government had done everything it could to push down inflation.
	\item Interest had pushed the loan up to $27,000.
	\end{itemize}
}
\item verb \\
If someone or something \textbf{pushes} an idea or project in a particular direction, they cause it to develop or progress in a particular way.
 \textit{
	\begin{itemize}
	\item We are continuing to push the business forward.
	\item The government seemed intent on pushing local and central government in opposite
directions.
	\end{itemize}
}
\item verb \\
If you \textbf{push} someone \textbf{to} do something or \textbf{push} them \textbf{into} doing it, you encourage or force them to do it.
 \textbf{Push} is also a noun.
 \textit{
	\begin{itemize}
	\item She thanks her parents for keeping her in school and pushing her to study.
	\item James did not push her into stealing the money.
	\item I knew he was pushing himself to the limit.
	\item There is no point in pushing them unless they are talented.
	\item We need a push to take the first step.
	\end{itemize}
}
\item verb \\
If you \textbf{push for} something, you try very hard to achieve it or to persuade someone to do it.
 \textbf{Push} is also a noun.
 \textit{
	\begin{itemize}
	\item Campaigners are pushing for more information and better treatments.
	\item Germany is pushing for direct flights to be established.
	\item In its push for economic growth it has ignored projects that would improve living
standards.
	\item They urged negotiators to make a final push to arrive at an agreement.
	\end{itemize}
}
\item verb \\
If someone \textbf{pushes} an idea, a point, or a product, they try in a forceful way to convince people to accept it or buy it.
 \textit{
	\begin{itemize}
	\item Ministers will push the case for opening the plant.
	\item She knew they could push a hundred thousand copies into the bookshops.
	\end{itemize}
}
\item verb \\
When someone \textbf{pushes} drugs, they sell them illegally.
 \textit{
	\begin{itemize}
	\item She was sent for trial yesterday accused of pushing drugs.
	\end{itemize}
}
\item verb \\
If you say that someone \textbf{is pushing it} , you mean that their actions or claims are rather excessive or risky .
 \textit{
	\begin{itemize}
	\item I think that he was pushing it a bit when he said it was the best stadium in the
world.
	\end{itemize}
}
\item  \\
 give sb/get the push \textit{
	\begin{itemize}
	\end{itemize}
}
\end{enumerate}

\section*{inhibit}
{\large \color{blue}  inhibits  inhibiting  inhibited  }
\subsection*{Explain}
\begin{enumerate}
\item verb \\
If something \textbf{inhibits} an event or process, it prevents it or slows it down.
 \textit{
	\begin{itemize}
	\item Sugary drinks inhibit digestion.
	\item The high cost of borrowing is inhibiting investment by industry in new equipment.
	\end{itemize}
}
\item verb \\
To \textbf{inhibit} someone \textbf{from} doing something means to prevent them from doing it, although they want to do it or should be able to do it.
 \textit{
	\begin{itemize}
	\item It could inhibit the poor from getting the medical care they need.
	\item Officers will be inhibited from doing their duty.
	\end{itemize}
}
\end{enumerate}

\section*{recommend}
{\large \color{blue}  recommends  recommending  recommended  }
\subsection*{Explain}
\begin{enumerate}
\item verb \\
If someone \textbf{recommends} a person or thing to you, they suggest that you would find that person or thing good or useful .
 \textit{
	\begin{itemize}
	\item I have just spent a holiday there and would recommend it to anyone.
	\item 'You're a good worker, boy,' he told him. 'I'll recommend you for a promotion.'.
	\item Ask your doctor to recommend a suitable therapist.
	\end{itemize}
}
\item verb \\
If you \textbf{recommend} that something is done, you suggest that it should be done.
 \textit{
	\begin{itemize}
	\item The judge recommended that he serve 20 years in prison.
	\item We strongly recommend reporting the incident to the police.
	\item It is recommended that you should consult your doctor.
	\item The recommended daily dose is 12 to 24 grams.
	\item Many financial planners now recommend against ever fully paying off your home loan.
	\end{itemize}
}
\item verb \\
If something or someone has a particular quality to \textbf{recommend} them, that quality makes them attractive or gives them an advantage over similar things or people.
 \textit{
	\begin{itemize}
	\item The restaurant has much to recommend it.
	\item He had little but his enthusiasm to recommend him.
	\item These qualities recommended him to Olivier.
	\end{itemize}
}
\end{enumerate}

\section*{invest}
{\large \color{blue}  invests  investing  invested  }
\subsection*{Explain}
\begin{enumerate}
\item verb \\
If you \textbf{invest}  \textbf{in} something, or if you \textbf{invest} a sum of money, you use your money in a way that you hope  will  increase its value, for example by paying it into a bank , or buying shares or property.
 \textit{
	\begin{itemize}
	\item They intend to invest directly in shares.
	\item He invested all our profits in gold shares.
	\item When people buy houses they're investing a lot of money.
	\end{itemize}
}
\item verb \\
When a government or organization \textbf{invests}  \textbf{in} something, it gives or lends money for a purpose that it considers  useful or profitable .
 \textit{
	\begin{itemize}
	\item ...the British government's failure to invest in an integrated transport system.
	\item ...the European Investment Bank, which invested £100 million in Canary Wharf.
	\item Why does Japan invest, on average, twice as much capital per worker per year than
the United States?
	\end{itemize}
}
\item verb \\
If you \textbf{invest in} something useful, you buy it, because it will help you to do something more efficiently or more cheaply.
 \textit{
	\begin{itemize}
	\item The company invested thousands in an electronic order-control system.
	\item The easiest way to make ice cream yourself is to invest in an ice cream machine.
	\end{itemize}
}
\item verb \\
If you \textbf{invest} time or energy  \textbf{in} something, you spend a lot of time or energy on something that you consider to be useful or likely to be successful .
 \textit{
	\begin{itemize}
	\item I would rather invest time in Rebecca than in the kitchen.
	\end{itemize}
}
\item verb \\
If you say that someone or something \textbf{is invested with} a particular quality, you mean that they seem to have that quality.
 \textit{
	\begin{itemize}
	\item The buildings are invested with a nation's history.
	\item A tsar was a living icon, invested with deep historical and religious significance.
	\end{itemize}
}
\item verb \\
To \textbf{invest} someone \textbf{with} rights or responsibilities means to give them those rights or responsibilities legally or officially .
 \textit{
	\begin{itemize}
	\item The constitution had invested him with certain powers.
	\end{itemize}
}
\end{enumerate}

\section*{reduce}
{\large \color{blue}  reduces  reducing  reduced  }
\subsection*{Explain}
\begin{enumerate}
\item verb \\
If you \textbf{reduce} something, you make it smaller in size or amount, or less in degree.
 \textit{
	\begin{itemize}
	\item It reduces the risks of heart disease.
	\item Consumption is being reduced by 25 per cent.
	\item The reduced consumer demand is also affecting company profits.
	\end{itemize}
}
\item verb \\
If someone \textbf{is reduced}  \textbf{to} a weaker or inferior state, they become weaker or inferior as a result of something that happens to them.
 \textit{
	\begin{itemize}
	\item They were reduced to extreme poverty.
	\item They wanted the army reduced to a police force.
	\end{itemize}
}
\item verb \\
If you say that someone \textbf{is reduced}  \textbf{to} doing something, you mean that they have to do it, although it is unpleasant or embarrassing .
 \textit{
	\begin{itemize}
	\item He was reduced to begging for a living.
	\end{itemize}
}
\item verb \\
If something is changed to a different or less complicated form, you can say that it \textbf{is reduced}  \textbf{to} that form.
 \textit{
	\begin{itemize}
	\item All the buildings in the town have been reduced to rubble.
	\item Politics has been reduced to class struggle.
	\end{itemize}
}
\item verb \\
If you \textbf{reduce} liquid when you are cooking, or if it \textbf{reduces} , it is boiled in order to make it less in quantity and thicker.
 \textit{
	\begin{itemize}
	\item Boil the liquid in a small saucepan to reduce it by half.
	\item Simmer until mixture reduces.
	\end{itemize}
}
\item  \\
 reduced circumstances \textit{
	\begin{itemize}
	\end{itemize}
}
\item  \\
 reduce to silence \textit{
	\begin{itemize}
	\end{itemize}
}
\item  \\
 reduce to tears \textit{
	\begin{itemize}
	\end{itemize}
}
\end{enumerate}

\section*{know}
{\large \color{blue}  knows  knowing  knew  known  }
\subsection*{Explain}
\begin{enumerate}
\item verb \\
If you \textbf{know} a fact, a piece of information , or an answer , you have it correctly in your mind .
 \textit{
	\begin{itemize}
	\item I don't know the name of the place.
	\item I know that you led a rifle platoon during the Second World War.
	\item 'People like doing things for nothing.'—'I know they do.'
	\item I don't know what happened to her husband.
	\item 'How did he meet your mother?'—'I don't know.'
	\item We all know about his early experiments in flying.
	\item They looked younger than I knew them to be.
	\item Radon is known to be harmful to humans in large quantities.
	\item It is not known whether the bomb was originally intended for the capital itself.
	\item It's always been known that key figures in the government do very well for themselves.
	\end{itemize}
}
\item verb \\
If you \textbf{know} someone, you are familiar with them because you have met them and talked to them before.
 \textit{
	\begin{itemize}
	\item Gifford was a friend. I'd known him for nine years.
	\item Do you two know each other?
	\item He doesn't know anybody in London.
	\end{itemize}
}
\item verb \\
If you say that you \textbf{know of} something, you mean that you have heard about it but you do not necessarily have a lot of information about it.
 \textit{
	\begin{itemize}
	\item We know of the incident but have no further details.
	\item The president admitted that he did not know of any rebels having surrendered so far.
	\item I know of no one who would want to murder Albert.
	\end{itemize}
}
\item verb \\
If you \textbf{know}  \textbf{about} a subject , you have studied it or taken an interest in it, and understand part or all of it.
 \textit{
	\begin{itemize}
	\item Hire someone with experience, someone who knows about real estate.
	\item She didn't know anything about music but she liked to sing.
	\end{itemize}
}
\item verb \\
If you \textbf{know} a language, you have learned it and can understand it.
 \textit{
	\begin{itemize}
	\item It helps to know French and Creole if you want to understand some of the lyrics.
	\item Rachel already knows as many words in German as she does in English.
	\item Her new classmates knew no Latin.
	\end{itemize}
}
\item verb \\
If you \textbf{know} something such as a place, a work of art , or an idea , you have visited it, seen it, read it, or heard about it, and so you are familiar with it.
 \textit{
	\begin{itemize}
	\item No matter how well you know Paris, it is easy to get lost.
	\item I don't know the play, I've just come to see it.
	\end{itemize}
}
\item verb \\
If you \textbf{know}  \textbf{how to} do something, you have the necessary  skills and knowledge to do it.
 \textit{
	\begin{itemize}
	\item The health authorities now know how to deal with the disease.
	\item We know what to do to make it work.
	\end{itemize}
}
\item verb \\
You can say that someone \textbf{knows}  \textbf{that} something is happening when they become aware of it.
 \textit{
	\begin{itemize}
	\item Then I saw a gun under the hall table so I knew that something was wrong.
	\item The first I knew about it was when I woke up in the ambulance.
	\end{itemize}
}
\item verb \\
If you \textbf{know} something or someone, you recognize them when you see them or hear them.
 \textit{
	\begin{itemize}
	\item Would she know you if she saw you on the street?
	\item I thought I knew the voice.
	\end{itemize}
}
\item verb \\
If someone or something \textbf{is known}  \textbf{as} a particular  name , they are called by that name.
 \textit{
	\begin{itemize}
	\item The disease is more commonly known as Mad Cow Disease.
	\item He was born as John Birks Gillespie, but everyone knew him as Dizzy.
	\item He was the only boy in the school who was known by his Christian name and not his
surname.
	\item ...British Nuclear Fuels, otherwise known as BNFL.
	\end{itemize}
}
\item verb \\
If you \textbf{know} someone or something \textbf{as} a person or thing that has particular qualities , you consider that they have those qualities.
 \textit{
	\begin{itemize}
	\item Lots of people know her as a very kind woman.
	\item We know them as inaccurate and misleading property descriptions.
	\item Kemp knew him for a meticulous officer.
	\end{itemize}
}
\item verb \\
If you \textbf{know} someone \textbf{as} a person with a particular job or role , you are familiar with them in that job or role, rather than in any other.
 \textit{
	\begin{itemize}
	\item Most of us know her as the woman who used to present the television news.
	\item The soldiers–all of whom we knew as neighbours–stood around pointing guns at us.
	\end{itemize}
}
\item  \\
 as we know it \textit{
	\begin{itemize}
	\end{itemize}
}
\item  \\
 to get to know sb \textit{
	\begin{itemize}
	\end{itemize}
}
\item  \\
 heaven/god/lord/christ etc knows \textit{
	\begin{itemize}
	\end{itemize}
}
\item  \\
 I know \textit{
	\begin{itemize}
	\end{itemize}
}
\item  \\
 I know \textit{
	\begin{itemize}
	\end{itemize}
}
\item  \\
 I know (how you feel, etc) \textit{
	\begin{itemize}
	\end{itemize}
}
\item  \\
 I don't know (about that) \textit{
	\begin{itemize}
	\end{itemize}
}
\item  \\
 I don't know about you \textit{
	\begin{itemize}
	\end{itemize}
}
\item  \\
 I don't know how/what \textit{
	\begin{itemize}
	\end{itemize}
}
\item  \\
 (I'm) blessed/damned/buggered if I know \textit{
	\begin{itemize}
	\end{itemize}
}
\item  \\
 in the know \textit{
	\begin{itemize}
	\end{itemize}
}
\item  \\
 you know what I mean \textit{
	\begin{itemize}
	\end{itemize}
}
\item  \\
 you never know \textit{
	\begin{itemize}
	\end{itemize}
}
\item  \\
 not that I know of \textit{
	\begin{itemize}
	\end{itemize}
}
\item  \\
 sb wasn't to know/how was sb to know \textit{
	\begin{itemize}
	\end{itemize}
}
\item  \\
 what does sb know \textit{
	\begin{itemize}
	\end{itemize}
}
\item  \\
 what do you know \textit{
	\begin{itemize}
	\end{itemize}
}
\item  \\
 you know \textit{
	\begin{itemize}
	\end{itemize}
}
\item  \\
 you know \textit{
	\begin{itemize}
	\end{itemize}
}
\item  \\
 you know \textit{
	\begin{itemize}
	\end{itemize}
}
\item  \\
 you don't know \textit{
	\begin{itemize}
	\end{itemize}
}
\end{enumerate}

\section*{relieve}
{\large \color{blue}  relieves  relieving  relieved  }
\subsection*{Explain}
\begin{enumerate}
\item verb \\
If something \textbf{relieves} an unpleasant feeling or situation , it makes it less unpleasant or causes it to disappear completely.
 \textit{
	\begin{itemize}
	\item Meditation can relieve stress.
	\item This should relieve the pressure on you to keep her entertained.
	\end{itemize}
}
\item verb \\
If someone or something \textbf{relieves} you \textbf{of} an unpleasant feeling or difficult  task , they take it from you.
 \textit{
	\begin{itemize}
	\item A bookkeeper will relieve you of the burden of chasing unpaid invoices.
	\end{itemize}
}
\item verb \\
If someone \textbf{relieves} you \textbf{of} something, they take it away from you.
 \textit{
	\begin{itemize}
	\item A porter relieved her of the three large cases.
	\end{itemize}
}
\item verb \\
If you \textbf{relieve} someone, you take their place and continue to do the job or duty that they have been doing.
 \textit{
	\begin{itemize}
	\item At seven o'clock the night nurse came in to relieve her.
	\end{itemize}
}
\item verb \\
If someone \textbf{is relieved}  \textbf{of} their duties or \textbf{is relieved}  \textbf{of} their post , they are told that they are no longer required to continue in their job.
 \textit{
	\begin{itemize}
	\item The officer involved was relieved of his duties because he had violated strict guidelines.
	\item The party leader has been relieved of his post.
	\end{itemize}
}
\item verb \\
If an army  \textbf{relieves} a town or another place which has been surrounded by enemy forces, it frees it.
 \textit{
	\begin{itemize}
	\item The offensive began several days ago as an attempt to relieve the town.
	\end{itemize}
}
\item verb \\
If people or animals \textbf{relieve}  \textbf{themselves} , they urinate or defecate .
 \textit{
	\begin{itemize}
	\item It is not difficult to train your dog to relieve itself on command.
	\end{itemize}
}
\end{enumerate}

\section*{replace}
{\large \color{blue}  replaces  replacing  replaced  }
\subsection*{Explain}
\begin{enumerate}
\item verb \\
If one thing or person \textbf{replaces} another, the first is used or acts instead of the second .
 \textit{
	\begin{itemize}
	\item The council tax replaced the poll tax in 1993.
	\item ...the city lawyer who replaced Bob as chairman of the company.
	\item The smile disappeared to be replaced by a doleful frown.
	\end{itemize}
}
\item verb \\
If you \textbf{replace} one thing or person \textbf{with} another, you put something or someone else in their place to do their job .
 \textit{
	\begin{itemize}
	\item I clean out all the grease and replace it with oil so it works better in very low
temperatures.
	\item The BBC decided it could not replace her.
	\end{itemize}
}
\item verb \\
If you \textbf{replace} something that is broken , damaged , or lost , you get a new one to use instead.
 \textit{
	\begin{itemize}
	\item The shower that we put in a few years back has broken and we cannot afford to replace
it.
	\end{itemize}
}
\item verb \\
If you \textbf{replace} something, you put it back where it was before.
 \textit{
	\begin{itemize}
	\item The line went dead. Whitlock replaced the receiver.
	\item Replace the caps on the bottles.
	\end{itemize}
}
\end{enumerate}

\section*{remove}
{\large \color{blue}  removes  removing  removed  }
\subsection*{Explain}
\begin{enumerate}
\item verb \\
If you \textbf{remove} something from a place, you take it away.
 \textit{
	\begin{itemize}
	\item As soon as the cake is done, remove it from the oven.
	\item At least three bullets were removed from his wounds.
	\item Often, the simplest answer is just to remove yourself from the situation.
	\item He went to the refrigerator and removed a bottle of milk.
	\end{itemize}
}
\item verb \\
If you \textbf{remove}  clothing , you take it off.
 \textit{
	\begin{itemize}
	\item He removed his jacket.
	\end{itemize}
}
\item verb \\
If you \textbf{remove} a stain from something, you make the stain disappear by treating it with a chemical or by washing it.
 \textit{
	\begin{itemize}
	\item This treatment removes the most stubborn stains.
	\item Try using lemon juice to remove tobacco stains from your fingers.
	\end{itemize}
}
\item verb \\
If people \textbf{remove} someone \textbf{from} power or \textbf{from} something such as a committee , they stop them being in power or being a member of the committee.
 \textit{
	\begin{itemize}
	\item The student senate voted to remove Fuller from office.
	\item The president could only be removed from power once free elections were organised.
	\item All senior officers involved in the coup will have to be removed.
	\end{itemize}
}
\item verb \\
If you \textbf{remove} an obstacle , a restriction , or a problem , you get rid of it.
 \textit{
	\begin{itemize}
	\item The agreement removes the last serious obstacle to the signing of the arms treaty.
	\item Most of her fears had been removed.
	\end{itemize}
}
\item  \\
 at one remove/ at several removes \textit{
	\begin{itemize}
	\end{itemize}
}
\end{enumerate}

\section*{segregate}
{\large \color{blue}  segregates  segregating  segregated  }
\subsection*{Explain}
\begin{enumerate}
\item verb \\
To \textbf{segregate} two groups of people or things means to keep them physically apart from each other.
 \textit{
	\begin{itemize}
	\item Police segregated the two rival camps of protesters.
	\item They segregate you from the rest of the community.
	\end{itemize}
}
\end{enumerate}

\section*{repay}
{\large \color{blue}  repays  repaying  repaid  }
\subsection*{Explain}
\begin{enumerate}
\item verb \\
If you \textbf{repay} a loan or a debt , you pay back the money that you owe to the person who you borrowed or took it from.
 \textit{
	\begin{itemize}
	\item It will take 30 years to repay the loan.
	\item He advanced funds of his own to his company, which was unable to repay him.
	\end{itemize}
}
\item verb \\
If you \textbf{repay} a favour that someone did for you, you do something for them in return.
 \textit{
	\begin{itemize}
	\item It was very kind. I don't know how I can ever repay you.
	\item I owe them a debt that cannot easily be repaid.
	\end{itemize}
}
\end{enumerate}

\section*{shove}
{\large \color{blue}  shoves  shoving  shoved  }
\subsection*{Explain}
\begin{enumerate}
\item verb \\
If you \textbf{shove} someone or something, you push them with a quick , violent movement .
 \textbf{Shove} is also a noun .
 \textit{
	\begin{itemize}
	\item He shoved her out of the way.
	\item He was then shoved face down on the pavement.
	\item He's the one who shoved me.
	\item She shoved as hard as she could.
	\item She gave Gracie a shove towards the house.
	\end{itemize}
}
\item verb \\
If you \textbf{shove} something somewhere, you push it there quickly and carelessly.
 \textit{
	\begin{itemize}
	\item We shoved a copy of the newsletter beneath their door.
	\item He shoved a cloth in my hand.
	\end{itemize}
}
\item  \\
 if push comes to shove \textit{
	\begin{itemize}
	\end{itemize}
}
\end{enumerate}

\section*{represent}
{\large \color{blue}  represents  representing  represented  }
\subsection*{Explain}
\begin{enumerate}
\item verb \\
If someone such as a lawyer or a politician  \textbf{represents} a person or group of people, they act on behalf of that person or group.
 \textit{
	\begin{itemize}
	\item ...the politicians we elect to represent us.
	\item The offer has yet to be accepted by the lawyers representing the victims.
	\end{itemize}
}
\item verb \\
If you \textbf{represent} a person or group at an official event, you go there on their behalf.
 \textit{
	\begin{itemize}
	\item The general secretary may represent the president at official ceremonies.
	\end{itemize}
}
\item verb \\
If you \textbf{represent} your country or town in a competition or sports event, you take part in it on behalf of the country or town where you live .
 \textit{
	\begin{itemize}
	\item My only aim is to represent Britain at the Commonwealth Games.
	\end{itemize}
}
\item passive verb \\
If a group of people or things \textbf{is}  well  \textbf{represented} in a particular activity or in a particular place, a lot of them can be found there.
 \textit{
	\begin{itemize}
	\item Sadly, women leaders are not well represented in our churches.
	\item In New Mexico all kinds of cuisines are represented.
	\end{itemize}
}
\item link verb \\
If you say that something \textbf{represents} a change, achievement , or victory , you mean that it is a change, achievement, or victory.
 \textit{
	\begin{itemize}
	\item The pieces on view represent superb examples from various periods.
	\item These developments represented a major change in the established order.
	\end{itemize}
}
\item verb \\
If a sign or symbol  \textbf{represents} something, it is accepted as meaning that thing.
 \textit{
	\begin{itemize}
	\item A black dot in the middle of the circle is supposed to represent the source of the
radiation.
	\end{itemize}
}
\item verb \\
To \textbf{represent} an idea or quality means to be a symbol or an expression of that idea or quality.
 \textit{
	\begin{itemize}
	\item We believe you represent everything British racing needs.
	\end{itemize}
}
\item verb \\
If you \textbf{represent} a person or thing \textbf{as} a particular thing, you describe them as being that thing.
 \textit{
	\begin{itemize}
	\item The popular press tends to represent him as an environmental guru.
	\end{itemize}
}
\end{enumerate}

\section*{stun}
{\large \color{blue}  stuns  stunning  stunned  }
\subsection*{Explain}
\begin{enumerate}
\item verb \\
If you \textbf{are stunned} by something, you are extremely shocked or surprised by it and are therefore unable to speak or do anything.
 \textit{
	\begin{itemize}
	\item Many cinema-goers were stunned by the film's violent and tragic end.
	\end{itemize}
}
\item verb \\
If something such as a blow on the head \textbf{stuns} you, it makes you unconscious or confused and unsteady .
 \textit{
	\begin{itemize}
	\item Sam stood his ground and got a blow that stunned him.
	\end{itemize}
}
\end{enumerate}

\section*{save}
{\large \color{blue}  saves  saving  saved  }
\subsection*{Explain}
\begin{enumerate}
\item verb \\
If you \textbf{save} someone or something, you help them to avoid harm or to escape from a dangerous or unpleasant  situation .
 \textit{
	\begin{itemize}
	\item ...a final attempt to save 40,000 jobs in the troubled aero industry.
	\item One man was still missing last night after the Belgian trawler Lucky capsized off
the Dutch coast. Three other men were saved.
	\item A new machine no bigger than a 10p piece could help save babies from cot death.
	\item The performance may have saved him from being eliminated.
	\end{itemize}
}
\item verb \\
If you \textbf{save} , you gradually collect money by spending less than you get , usually in order to buy something that you want .
 \textbf{Save up} means the same as save .
 \textit{
	\begin{itemize}
	\item Most people intend to save, but find that by the end of the month there is nothing
left.
	\item Tim and Barbara are now saving for a house in the suburbs.
	\item They could not find any way to save money.
	\item Julie wanted to save up something for a holiday.
	\item People often put money aside to save up enough to make one major expenditure.
	\end{itemize}
}
\item verb \\
If you \textbf{save} something such as time or money, you prevent the loss or waste of it.
 \textit{
	\begin{itemize}
	\item It saves time in the kitchen to have things you use a lot within reach.
	\item More cash will be saved by shutting studios and selling outside-broadcast vehicles.
	\item I'll try to save him the expense of a flight from Perth.
	\item I got the fishmonger to skin the fish which helped save on the preparation time.
	\end{itemize}
}
\item verb \\
If you \textbf{save} something, you keep it because it will be needed  later .
 \textit{
	\begin{itemize}
	\item Drain the beans thoroughly and save the stock for soup.
	\item Scraps of material were saved, cut up and pieced together for quilts.
	\end{itemize}
}
\item verb \\
If someone or something \textbf{saves} you \textbf{from} an unpleasant action or experience , they change the situation so that you do not have to do it or experience it.
 \textit{
	\begin{itemize}
	\item The scanner will save risk and pain for patients.
	\item She was hoping that something might save her from having to make a decision.
	\item He arranges to collect the payment from the customer, thus saving the client the
paperwork.
	\end{itemize}
}
\item verb \\
If you \textbf{save}  data in a computer, you give the computer an instruction to store the data on a tape or disk.
 \textit{
	\begin{itemize}
	\item Try to get into the habit of saving your work regularly.
	\item Import your scanned images from the scanner and save as a JPG file.
	\end{itemize}
}
\item verb \\
If a goalkeeper  \textbf{saves} , or \textbf{saves} a shot , they succeed in preventing the ball from going into the goal.
 \textbf{Save} is also a noun .
 \textit{
	\begin{itemize}
	\item He saved one shot when the ball hit him on the head.
	\item Their keeper made an unbelievable save at the end.
	\end{itemize}
}
\item preposition \\
You can use \textbf{save} to introduce the only things, people, or ideas that your main  statement does not apply to.
 \textit{
	\begin{itemize}
	\item There is almost no water at all in the area save that brought up from bore holes.
	\end{itemize}
}
\end{enumerate}

\section*{tell}
{\large \color{blue}  tells  telling  told  }
\subsection*{Explain}
\begin{enumerate}
\item verb \\
If you \textbf{tell} someone something, you give them information.
 \textit{
	\begin{itemize}
	\item In the evening I returned to tell Phyllis our relationship was over.
	\item I called Andie to tell her how spectacular the stuff looked.
	\item Claire had made me promise to tell her the truth.
	\item I only told the truth to the press when the single was released.
	\item Tell us about your moment on the summit.
	\item Her voice breaking with emotion, she told him: 'It doesn't seem fair.'
	\end{itemize}
}
\item verb \\
If you \textbf{tell} something such as a joke , a story , or your personal experiences, you communicate it to other people using speech.
 \textit{
	\begin{itemize}
	\item His friends say he was always quick to tell a joke.
	\item He told his story to The Sunday Times and produced photographs.
	\item Will you tell me a story?
	\end{itemize}
}
\item verb \\
If you \textbf{tell} someone \textbf{to} do something, you order or advise them to do it.
 \textit{
	\begin{itemize}
	\item A passer-by told the driver to move his car so that it was not causing an obstruction.
	\item She told me on the telephone to come help clean the house.
	\end{itemize}
}
\item verb \\
If you \textbf{tell}  \textbf{yourself} something, you put it into words in your own mind because you need to encourage or persuade yourself about something.
 \textit{
	\begin{itemize}
	\item 'Come on,' she told herself.
	\item I told myself that I would be satisfied with whatever I could get.
	\end{itemize}
}
\item verb \\
If you can \textbf{tell} what is happening or what is true , you are able to judge correctly what is happening or what is true.
 \textit{
	\begin{itemize}
	\item It was already impossible to tell where the bullet had entered.
	\item I couldn't tell if he had been in a fight or had just fallen down.
	\item You can tell he's joking.
	\end{itemize}
}
\item verb \\
If you can \textbf{tell} one thing \textbf{from} another, you are able to recognize the difference between it and other similar things.
 \textit{
	\begin{itemize}
	\item I can't really tell the difference between their policies and ours.
	\item How do you tell one from another?
	\item I had to look twice to tell which was Martin; the twins were almost identical.
	\end{itemize}
}
\item verb \\
If you \textbf{tell} , you reveal or give away a secret.
 \textit{
	\begin{itemize}
	\item Many of the children know who they are but are not telling.
	\end{itemize}
}
\item verb \\
If facts or events \textbf{tell} you something, they reveal certain information to you through ways other than speech.
 \textit{
	\begin{itemize}
	\item The facts tell us that this is not true.
	\item I don't think the unemployment rate ever tells us much about the future.
	\item The evidence of our eyes tells us a different story.
	\item While most of us feel complacent about our diets, the facts tell a very different
story.
	\end{itemize}
}
\item verb \\
If an unpleasant or tiring experience begins to \textbf{tell} , it begins to have a serious effect.
 \textit{
	\begin{itemize}
	\item The pressure began to tell as rain closed in after 20 laps.
	\item The strains of office are beginning to tell on the prime minister.
	\end{itemize}
}
\item  \\
 as far as one can tell/so far as one can tell \textit{
	\begin{itemize}
	\end{itemize}
}
\item  \\
 I tell you/I can tell you/I can't tell you \textit{
	\begin{itemize}
	\end{itemize}
}
\item  \\
 you never can tell \textit{
	\begin{itemize}
	\end{itemize}
}
\item  \\
 I told you so \textit{
	\begin{itemize}
	\end{itemize}
}
\item  \\
 I'll tell you what/I tell you what \textit{
	\begin{itemize}
	\end{itemize}
}
\end{enumerate}

\section*{sing}
{\large \color{blue}  sings  singing  sang  sung  }
\subsection*{Explain}
\begin{enumerate}
\item verb \\
When you \textbf{sing} , you make musical sounds with your voice , usually producing words that fit a tune .
 \textit{
	\begin{itemize}
	\item I can't sing.
	\item I sing about love most of the time.
	\item They were all singing the same song.
	\item Go on, then, sing us a song!
	\item 'You're getting to be a habit with me,' sang Eddie.
	\item ...an operatic aria sung by Luciano Pavarotti.
	\end{itemize}
}
\item verb \\
When birds or insects \textbf{sing} , they make pleasant  high-pitched sounds.
 \textit{
	\begin{itemize}
	\item Birds were already singing in the garden.
	\end{itemize}
}
\item  \\
 sing from the same hymn sheet \textit{
	\begin{itemize}
	\end{itemize}
}
\end{enumerate}

\section*{transform}
{\large \color{blue}  transforms  transforming  transformed  }
\subsection*{Explain}
\begin{enumerate}
\item verb \\
To \textbf{transform} something \textbf{into} something else means to change or convert it into that thing.
 \textit{
	\begin{itemize}
	\item Your metabolic rate is the speed at which your body transforms food into energy.
	\item Delegates also discussed transforming them from a guerrilla force into a regular
army.
	\end{itemize}
}
\item verb \\
To \textbf{transform} something or someone means to change them completely and suddenly so that they are much better or more attractive .
 \textit{
	\begin{itemize}
	\item The high-speed rail link is transforming the area.
	\item A cheap table can be transformed by an interesting cover.
	\item A love of rugby transformed him from a podgy child into a trophy winner.
	\end{itemize}
}
\end{enumerate}

\section*{snap}
{\large \color{blue}  snaps  snapping  snapped  }
\subsection*{Explain}
\begin{enumerate}
\item verb \\
If something \textbf{snaps} or if you \textbf{snap} it, it breaks suddenly, usually with a sharp cracking noise .
 \textbf{Snap} is also a noun .
 \textit{
	\begin{itemize}
	\item He shifted his weight and a twig snapped.
	\item The brake pedal had just snapped off.
	\item She gripped the pipe with both hands, trying to snap it in half.
	\item Every minute or so I could hear a snap, a crack and a crash as another tree went
down.
	\end{itemize}
}
\item verb \\
If you \textbf{snap} something into a particular position, or if it \textbf{snaps} into that position, it moves quickly into that position, with a sharp sound.
 \textbf{Snap} is also a noun.
 \textit{
	\begin{itemize}
	\item He snapped the notebook shut.
	\item He snapped the cap on his ballpoint.
	\item The bag snapped open.
	\item He shut the book with a snap and stood up.
	\end{itemize}
}
\item verb \\
If you \textbf{snap} your \textbf{fingers} , you make a sharp sound by moving your middle finger quickly across your thumb, for
example in order to accompany music or to order someone to do something.
 \textbf{Snap} is also a noun.
 \textit{
	\begin{itemize}
	\item She had millions of listeners snapping their fingers to her first single.
	\item He snapped his fingers, and Wilson produced a sheet of paper.
	\item She snapped her fingers at a passing waiter.
	\item I could obtain with the snap of my fingers anything I chose.
	\end{itemize}
}
\item verb \\
If someone \textbf{snaps}  \textbf{at} you, they speak to you in a sharp, unfriendly way.
 \textit{
	\begin{itemize}
	\item 'Of course I don't know her,' Roger snapped.
	\item I'm sorry, Casey, I didn't mean to snap at you like that.
	\end{itemize}
}
\item verb \\
If someone \textbf{snaps} , or if something \textbf{snaps}  inside them, they suddenly stop being calm and become very angry because the situation has become too tense or too difficult for them.
 \textit{
	\begin{itemize}
	\item He finally snapped when she prevented their children from visiting him one weekend.
	\item For the first and only time Grant's self-control snapped.
	\item Then something seemed to snap in me. I couldn't endure any more.
	\end{itemize}
}
\item verb \\
If an animal such as a dog  \textbf{snaps}  \textbf{at} you, it opens and shuts its jaws quickly near you, as if it were going to bite you.
 \textit{
	\begin{itemize}
	\item His teeth clicked as he snapped at my ankle.
	\item The poodle yapped and snapped.
	\end{itemize}
}
\item adjective \\
A \textbf{snap}  decision or action is one that is taken suddenly, often without careful thought.
 \textit{
	\begin{itemize}
	\item I think this is too important for a snap decision.
	\item It's important not to make snap judgments.
	\item The opposition is worried that a snap election will be held before they can get organised.
	\end{itemize}
}
\item countable noun \\
A \textbf{snap} is a photograph .
 \textit{
	\begin{itemize}
	\item ...a snap my mother took last year.
	\end{itemize}
}
\item verb \\
If you \textbf{snap} someone or something, you take a photograph of them.
 \textit{
	\begin{itemize}
	\item He was the first ever non-British photographer to be invited to snap a royal.
	\end{itemize}
}
\item uncountable noun \\
\textbf{Snap} is a simple British card game in which the players take turns to put cards down on a pile, and
 try to be the first to shout 'snap' when two cards with the same number or picture are put down.
 \textit{
	\begin{itemize}
	\end{itemize}
}
\item exclamation \\
You can say ' \textbf{Snap!} ' as an expression of surprise when you realize that two things are the same or very similar, for example if you meet a friend wearing the same shirt as you.
 \textit{
	\begin{itemize}
	\end{itemize}
}
\item countable noun \\
A \textbf{snap} is the same as a snap fastener .
 \textit{
	\begin{itemize}
	\end{itemize}
}
\end{enumerate}

\section*{stop}
{\large \color{blue}  stops  stopping  stopped  }
\subsection*{Explain}
\begin{enumerate}
\item verb \\
If you have been doing something and then you \textbf{stop} doing it, you no longer do it.
 \textit{
	\begin{itemize}
	\item Stop throwing those stones!
	\item He can't stop thinking about it.
	\item I've been told to lose weight and stop smoking.
	\item I stopped working last year to have a baby.
	\item Does either of the parties want to stop the fighting?
	\item She stopped in mid-sentence.
	\end{itemize}
}
\item verb \\
If you \textbf{stop} something happening , you prevent it from happening or prevent it from continuing.
 \textit{
	\begin{itemize}
	\item He proposed a new diplomatic initiative to try to stop the war.
	\item If the fire isn't stopped, it could spread to 25,000 acres.
	\item I think she really would have liked to stop us seeing each other.
	\item He put the radio on loud to stop himself thinking about it.
	\item Motherhood won't stop me from pursuing my acting career.
	\item There's nothing to stop you from doing a bit of exploring further afield.
	\end{itemize}
}
\item verb \\
If an activity or process \textbf{stops} , it is no longer happening.
 \textit{
	\begin{itemize}
	\item The rain had stopped and a star or two was visible over the mountains.
	\item The system overheated and filming had to stop.
	\item The music stopped and the lights were turned up.
	\item They're treating it like a game, a novelty. That's got to stop.
	\end{itemize}
}
\item verb \\
If something such as machine \textbf{stops} or \textbf{is stopped} , it is no longer moving or working.
 \textit{
	\begin{itemize}
	\item The clock had stopped at 2.12 a.m.
	\item His heart stopped three times.
	\item Arnold stopped the engine and got out of the car.
	\item He stopped the machine and replayed the message.
	\end{itemize}
}
\item verb \\
When a moving person or vehicle \textbf{stops} or \textbf{is stopped} , they no longer move and they remain in the same place.
 \textit{
	\begin{itemize}
	\item The car failed to stop at an army checkpoint.
	\item He stopped and let her catch up with him.
	\item The event literally stopped the traffic.
	\item The van was stopped at customs in Harwich.
	\end{itemize}
}
\item singular noun \\
If something that is moving comes \textbf{to a stop} or is brought \textbf{to a stop} , it slows down and no longer moves.
 \textit{
	\begin{itemize}
	\item People often wrongly open doors before the train has come to a stop.
	\item He slowed the car almost to a stop.
	\end{itemize}
}
\item verb \\
If someone does not \textbf{stop}  \textbf{to}  think or \textbf{to}  explain , they continue with what they are doing without taking any time to think about or
explain it.
 \textit{
	\begin{itemize}
	\item She doesn't stop to think about what she's saying.
	\item There is something rather strange about all this if one stops to consider it.
	\item People who lead busy lives have no time to stop and reflect.
	\end{itemize}
}
\item verb \\
If you say that a quality or state \textbf{stops}  somewhere , you mean that it exists or is true up to that point, but no further.
 \textit{
	\begin{itemize}
	\item The cafe owner has put up 'no smoking' signs, but thinks his responsibility stops
there.
	\item The good news did not stop there.
	\item Once you cross over the thin line to acts of lawlessness, who knows where it stops?
	\end{itemize}
}
\item countable noun \\
A \textbf{stop} is a place where buses or trains regularly stop so that people can get on and off.
 \textit{
	\begin{itemize}
	\item There was an Underground map above one of the windows and I counted the stops to
West Hampstead.
	\item They waited at a bus stop.
	\end{itemize}
}
\item verb \\
If you \textbf{stop} somewhere on a journey, you stay there for a short while.
 \textit{
	\begin{itemize}
	\item He insisted we stop at a small restaurant just outside of Atlanta.
	\item It would be a crime to travel all the way to Australia and not stop in Sydney.
	\end{itemize}
}
\item countable noun \\
A \textbf{stop} is a time or place at which you stop during a journey.
 \textit{
	\begin{itemize}
	\item The last stop in Mr Cook's lengthy tour was Paris.
	\item Mack was driving down from Vermont, with a stop in Boston to pick Sarah up.
	\end{itemize}
}
\item countable noun \\
In music, organ \textbf{stops} are the knobs at the side of the organ, which you pull or push in order to control the type of sound that comes out of the pipes.
 \textit{
	\begin{itemize}
	\end{itemize}
}
\item  \\
 to stop at nothing \textit{
	\begin{itemize}
	\end{itemize}
}
\item  \\
 to pull out all the stops \textit{
	\begin{itemize}
	\end{itemize}
}
\item  \\
 put a stop to sth \textit{
	\begin{itemize}
	\end{itemize}
}
\item  \\
 know when to stop \textit{
	\begin{itemize}
	\end{itemize}
}
\end{enumerate}

\section*{utilize}
{\large \color{blue}  utilizes  utilizing  utilized  }
\subsection*{Explain}
\begin{enumerate}
\item verb \\
If you \textbf{utilize} something, you use it.
 \textit{
	\begin{itemize}
	\item Sound engineers utilize a range of techniques to enhance the quality of the recordings.
	\item Minerals can be absorbed and utilized by the body in a variety of different forms.
	\end{itemize}
}
\end{enumerate}

\section*{sue}
{\large \color{blue}  sues  suing  sued  }
\subsection*{Explain}
\begin{enumerate}
\item verb \\
If you \textbf{sue} someone, you start a legal case against them, usually in order to claim money from them because they have harmed you in some way.
 \textit{
	\begin{itemize}
	\item She sued him for libel over the remarks.
	\item The company could be sued for damages.
	\item One former patient has already indicated his intention to sue.
	\end{itemize}
}
\end{enumerate}

\section*{vary}
{\large \color{blue}  varies  varying  varied  }
\subsection*{Explain}
\begin{enumerate}
\item verb \\
If things \textbf{vary} , they are different from each other in size , amount , or degree .
 \textit{
	\begin{itemize}
	\item As they're handmade, each one varies slightly.
	\item The text varies from the earlier versions.
	\item Different writers will prepare to varying degrees.
	\end{itemize}
}
\item verb \\
If something \textbf{varies} or if you \textbf{vary} it, it becomes different or changed.
 \textit{
	\begin{itemize}
	\item Ferry times vary according to seasons.
	\item You are welcome to vary the diet.
	\end{itemize}
}
\end{enumerate}

\section*{suspect}
{\large \color{blue}  suspects  suspecting  suspected  }
\subsection*{Explain}
\begin{enumerate}
\item verb \\
You use \textbf{suspect} when you are stating something that you believe is probably  true , in order to make it sound less strong or direct .
 \textit{
	\begin{itemize}
	\item I suspect they were right.
	\item The above complaints are, I suspect, just the tip of the iceberg.
	\item Do women really share such stupid jokes? We suspect not.
	\end{itemize}
}
\item verb \\
If you \textbf{suspect} that something dishonest or unpleasant has been done, you believe that it has probably been done. If you \textbf{suspect} someone \textbf{of} doing an action of this kind , you believe that they probably did it.
 \textit{
	\begin{itemize}
	\item He suspected that the woman staying in the flat above was using heroin.
	\item Police said they suspected that Sobhraj had accomplices.
	\item It was perfectly all right, he said, because the police had not suspected him of
anything.
	\item You don't really think Webb suspects you?
	\item Frears was rushed to hospital with a suspected heart attack.
	\end{itemize}
}
\item countable noun \\
A \textbf{suspect} is a person who the police or authorities think may be guilty of a crime .
 \textit{
	\begin{itemize}
	\item Police have arrested a suspect in a series of killings and sexual assaults in the
city.
	\end{itemize}
}
\item adjective \\
\textbf{Suspect} things or people are ones that you think may be dangerous or may be less good or genuine than they appear .
 \textit{
	\begin{itemize}
	\item Delegates evacuated the building when a suspect package was found.
	\item The firm has taken out adverts urging customers to return suspect products.
	\item The whole affair has been highly suspect.
	\end{itemize}
}
\end{enumerate}

\section*{verify}
{\large \color{blue}  verifies  verifying  verified  }
\subsection*{Explain}
\begin{enumerate}
\item verb \\
If you \textbf{verify} something, you check that it is true by careful  examination or investigation.
 \textit{
	\begin{itemize}
	\item I verified the source from which I had that information.
	\item A clerk simply verifies that the payment and invoice amount match.
	\end{itemize}
}
\item verb \\
If you \textbf{verify} something, you state or confirm that it is true.
 \textit{
	\begin{itemize}
	\item The government has not verified any of those reports.
	\item I can verify that it takes about thirty seconds.
	\end{itemize}
}
\end{enumerate}

\section*{suspend}
{\large \color{blue}  suspends  suspending  suspended  }
\subsection*{Explain}
\begin{enumerate}
\item verb \\
If you \textbf{suspend} something, you delay it or stop it from happening for a while or until a decision is made about it.
 \textit{
	\begin{itemize}
	\item The union suspended strike action this week.
	\item Aid programs will be suspended until there's adequate protection for relief convoys.
	\end{itemize}
}
\item verb \\
If someone \textbf{is suspended} , they are prevented from holding a particular job or position for a fixed length of time or until a decision is made about them.
 \textit{
	\begin{itemize}
	\item Julie was suspended from her job shortly after the incident.
	\item Buchanan was suspended for a year from Georgetown University after brawling with
police.
	\item The Lawn Tennis Association suspended him from the British team.
	\end{itemize}
}
\item verb \\
If something \textbf{is suspended} from a high place, it is hanging from that place.
 \textit{
	\begin{itemize}
	\item ...a mobile of birds or nursery rhyme characters which could be suspended over the
cot.
	\item ...chandeliers suspended on heavy chains from the ceiling.
	\end{itemize}
}
\end{enumerate}

\section*{violate}
{\large \color{blue}  violates  violating  violated  }
\subsection*{Explain}
\begin{enumerate}
\item verb \\
If someone \textbf{violates} an agreement, law, or promise , they break it.
 \textit{
	\begin{itemize}
	\item They went to prison because they violated the law.
	\item They violated the ceasefire agreement.
	\end{itemize}
}
\item verb \\
If you \textbf{violate} someone's privacy or peace , you disturb it.
 \textit{
	\begin{itemize}
	\item These men were violating her family's privacy.
	\end{itemize}
}
\item verb \\
If someone \textbf{violates} a special place, for example a grave , they damage it or treat it with disrespect .
 \textit{
	\begin{itemize}
	\item Detectives are still searching for those who violated the graveyard.
	\end{itemize}
}
\end{enumerate}

\section*{twist}
{\large \color{blue}  twists  twisting  twisted  }
\subsection*{Explain}
\begin{enumerate}
\item verb \\
If you \textbf{twist} something, you turn it to make a spiral shape, for example by turning the two ends
of it in opposite directions.
 \textit{
	\begin{itemize}
	\item Her hands began to twist the handles of the bag she carried.
	\item Twist the string carefully around the second stem with the other hand.
	\item She twisted her hair into a bun and pinned it at the back of her head.
	\end{itemize}
}
\item verb \\
If you \textbf{twist} something, especially a part of your body, or if it \textbf{twists} , it moves into an unusual , uncomfortable , or bent position, for example because of being hit or pushed , or because you are upset .
 \textit{
	\begin{itemize}
	\item He twisted her arms behind her back and clipped a pair of handcuffs on her wrists.
	\item Sophia's face twisted in pain.
	\item Her hands twisted in her lap.
	\item The body was twisted, its legs at an awkward angle.
	\item The car was left a mess of twisted metal.
	\end{itemize}
}
\item verb \\
If you \textbf{twist} part of your body such as your head or your shoulders , you turn that part while keeping the rest of your body still .
 \textit{
	\begin{itemize}
	\item She twisted her head sideways and looked towards the door.
	\item Susan twisted round in her seat until she could see Graham and Sabrina behind her.
	\item Holding your arms straight out in front of you, twist to the right and left.
	\end{itemize}
}
\item verb \\
If you \textbf{twist} a part of your body such as your ankle or wrist , you injure it by turning it too sharply, or in an unusual direction.
 \textit{
	\begin{itemize}
	\item He fell and twisted his ankle.
	\item Rupert is out of today's session with a twisted knee.
	\end{itemize}
}
\item verb \\
If you \textbf{twist} something, you turn it so that it moves around in a circular direction.
 \textbf{Twist} is also a noun .
 \textit{
	\begin{itemize}
	\item She was staring down at her hands, twisting the ring on her finger.
	\item She twisted the handle and opened the door.
	\item Reaching up to a cupboard he takes out a jar and twists the lid off.
	\item The bag is resealed with a simple twist of the valve.
	\end{itemize}
}
\item verb \\
If a road or river \textbf{twists} , it has a lot of sudden changes of direction in it.
 \textbf{Twist} is also a noun.
 \textit{
	\begin{itemize}
	\item The roads twist round hairpin bends.
	\item The lane twists and turns between pleasant but unspectacular cottages.
	\item The train maintains a constant speed through the twists and turns of track.
	\end{itemize}
}
\item verb \\
If you say that someone \textbf{has twisted} something that you have said , you disapprove of them because they have repeated it in a way that changes its meaning, in order to harm you or benefit themselves.
 \textit{
	\begin{itemize}
	\item It's a shame the way that the media can twist your words and misrepresent you.
	\item Even remarks that were quite innocent could be twisted to produce an unintended effect.
	\end{itemize}
}
\item countable noun \\
A \textbf{twist} in something is an unexpected and significant development.
 \textit{
	\begin{itemize}
	\item ...the twists and turns of economic policy.
	\item The battle of the sexes also took a new twist.
	\item The letter was the latest twist in the long-running fight.
	\item As so often happens, this little story has a twist in the tail.
	\end{itemize}
}
\item countable noun \\
A \textbf{twist} is the shape that something has when it has been twisted.
 \textit{
	\begin{itemize}
	\item ...bunches of violets in twists of paper.
	\item A thin twist of smoke curled from the cottage's single chimney.
	\end{itemize}
}
\item singular noun \\
\textbf{The twist} is a dance that was popular in the 1960's, in which you twist your body and move
your hips in an energetic way.
 \textit{
	\begin{itemize}
	\end{itemize}
}
\item  \\
 twist of fate \textit{
	\begin{itemize}
	\end{itemize}
}
\end{enumerate}

\section*{withstand}
{\large \color{blue}  withstands  withstanding  withstood  }
\subsection*{Explain}
\begin{enumerate}
\item verb \\
If something or someone \textbf{withstands} a force or action, they survive it or do not give in to it.
 \textit{
	\begin{itemize}
	\item ...armoured vehicles designed to withstand chemical attack.
	\item Exercise really can help you withstand stresses and strains more easily.
	\end{itemize}
}
\end{enumerate}

\section*{wrench}
{\large \color{blue}  wrenches  wrenching  wrenched  }
\subsection*{Explain}
\begin{enumerate}
\item verb \\
If you \textbf{wrench} something that is fixed in a particular  position , you pull or twist it violently, in order to move or remove it.
 \textit{
	\begin{itemize}
	\item He felt two men wrench the suitcase from his hand.
	\item He wrenched off his sneakers.
	\item They wrenched open the passenger doors and jumped into her car.
	\end{itemize}
}
\item verb \\
If you \textbf{wrench} yourself free from someone who is holding you, you get  away from them by suddenly twisting the part of your body that is being held .
 \textit{
	\begin{itemize}
	\item She wrenched herself from his grasp.
	\item He wrenched his arm free.
	\item She tore at one man's face as she tried to wrench free.
	\item I wrenched my hand away from my attacker.
	\end{itemize}
}
\item verb \\
If you \textbf{wrench} one of your joints , you twist it and injure it.
 \textit{
	\begin{itemize}
	\item He had wrenched his ankle badly from the force of the fall.
	\end{itemize}
}
\item singular noun \\
If you say that leaving someone or something is \textbf{a}  \textbf{wrench} , you feel very sad about it.
 \textit{
	\begin{itemize}
	\item I always knew it would be a wrench to leave Essex after all these years.
	\item Although it would be a wrench, we would all accept the challenge of moving abroad.
	\end{itemize}
}
\item countable noun \\
A \textbf{wrench} or a \textbf{monkey wrench} is an adjustable metal  tool used for tightening or loosening metal nuts of different  sizes .
 \textit{
	\begin{itemize}
	\end{itemize}
}
\item  \\
 to throw a wrench \textit{
	\begin{itemize}
	\end{itemize}
}
\end{enumerate}

\section*{abolish}
{\large \color{blue}  abolishes  abolishing  abolished  }
\subsection*{Explain}
\begin{enumerate}
\item verb \\
If someone in authority \textbf{abolishes} a system or practice, they formally put an end to it.
 \textit{
	\begin{itemize}
	\item The following year Parliament voted to abolish the death penalty for murder.
	\item The whole system should be abolished.
	\end{itemize}
}
\end{enumerate}

\section*{allege}
{\large \color{blue}  alleges  alleging  alleged  }
\subsection*{Explain}
\begin{enumerate}
\item verb \\
If you \textbf{allege}  \textbf{that} something bad is true , you say it but do not prove it.
 \textit{
	\begin{itemize}
	\item They alleged that the fires were caused by defective machinery.
	\item The accused is alleged to have killed a man.
	\item It was alleged that the restaurant discriminated against black customers.
	\end{itemize}
}
\end{enumerate}

\section*{achieve}
{\large \color{blue}  achieves  achieving  achieved  }
\subsection*{Explain}
\begin{enumerate}
\item verb \\
If you \textbf{achieve} a particular aim or effect , you succeed in doing it or causing it to happen , usually after a lot of effort.
 \textit{
	\begin{itemize}
	\item There are many who will work hard to achieve these goals.
	\item We have achieved what we set out to do.
	\end{itemize}
}
\end{enumerate}

\section*{cheat}
{\large \color{blue}  cheats  cheating  cheated  }
\subsection*{Explain}
\begin{enumerate}
\item verb \\
When someone \textbf{cheats} , they do not obey a set of rules which they should be obeying, for example in a game or exam .
 \textit{
	\begin{itemize}
	\item Students may be tempted to cheat in order to get into top schools.
	\end{itemize}
}
\item countable noun \\
Someone who is a \textbf{cheat} does not obey a set of rules which they should be obeying.
 \textit{
	\begin{itemize}
	\item Cheats will be disqualified.
	\end{itemize}
}
\item verb \\
If someone \textbf{cheats} you \textbf{out of} something, they get it from you by behaving dishonestly.
 \textit{
	\begin{itemize}
	\item The company engaged in a deliberate effort to cheat them out of their pensions.
	\item Many brokers were charged with cheating customers in commodity trades.
	\end{itemize}
}
\item  \\
 cheat death \textit{
	\begin{itemize}
	\end{itemize}
}
\item  \\
 feel cheated \textit{
	\begin{itemize}
	\end{itemize}
}
\end{enumerate}

\section*{choke}
{\large \color{blue}  chokes  choking  choked  }
\subsection*{Explain}
\begin{enumerate}
\item verb \\
When you \textbf{choke} or when something \textbf{chokes} you, you cannot breathe properly or get enough air into your lungs .
 \textit{
	\begin{itemize}
	\item The coffee was almost too hot to swallow and made him choke for a moment.
	\item A small child could choke on the doll's hair.
	\item Dense smoke swirled and billowed, its rank fumes choking her.
	\item The girl choked to death after breathing in smoke.
	\item Within minutes the hall was full of choking smoke.
	\end{itemize}
}
\item verb \\
To \textbf{choke} someone means to squeeze their neck until they are dead .
 \textit{
	\begin{itemize}
	\item The men pushed him into the entrance of a nearby building where they choked him with
his tie.
	\end{itemize}
}
\item verb \\
If a place \textbf{is choked}  \textbf{with} things or people, it is full of them and they prevent movement in it.
 \textit{
	\begin{itemize}
	\item The village's roads are choked with traffic.
	\item His pond has been choked by the fast-growing weed.
	\end{itemize}
}
\item countable noun \\
The \textbf{choke} in a car , truck , or other vehicle is a device that reduces the amount of air going into the engine and makes it easier to start .
 \textit{
	\begin{itemize}
	\end{itemize}
}
\end{enumerate}

\section*{admire}
{\large \color{blue}  admires  admiring  admired  }
\subsection*{Explain}
\begin{enumerate}
\item verb \\
If you \textbf{admire} someone or something, you like and respect them very much.
 \textit{
	\begin{itemize}
	\item I admired her when I first met her and I still think she's marvellous.
	\item He admired the way she had coped with life.
	\item All those who knew him will admire him for his work.
	\end{itemize}
}
\item verb \\
If you \textbf{admire} someone or something, you look at them with pleasure .
 \textit{
	\begin{itemize}
	\item We took time to stop and admire the view.
	\end{itemize}
}
\end{enumerate}

\section*{clip}
{\large \color{blue}  clips  clipping  clipped  }
\subsection*{Explain}
\begin{enumerate}
\item countable noun \\
A \textbf{clip} is a small device, usually made of metal or plastic , that is specially shaped for holding things together.
 \textit{
	\begin{itemize}
	\item She took the clip out of her hair.
	\end{itemize}
}
\item verb \\
When you \textbf{clip} things together or when things \textbf{clip} together, you fasten them together using a clip or clips.
 \textit{
	\begin{itemize}
	\item He clipped his safety belt to a fitting on the deck.
	\item He clipped his cufflinks neatly in place.
	\item ...an electronic pen which clips to the casing.
	\item His flashlight was still clipped to his belt.
	\end{itemize}
}
\item countable noun \\
A \textbf{clip} from a film or a radio or television programme is a short piece of it that is broadcast separately.
 \textit{
	\begin{itemize}
	\item ...an historical film clip of Lenin speaking.
	\item ...a clip from the movie 'Shane'.
	\end{itemize}
}
\item verb \\
If you \textbf{clip} something, you cut small pieces from it, especially in order to shape it.
 \textbf{Clip} is also a noun .
 \textit{
	\begin{itemize}
	\item I saw an old man out clipping his hedge.
	\item He had already clipped his hair close to the skull.
	\item Give hedges a last clip.
	\end{itemize}
}
\item verb \\
If you \textbf{clip} something out of a newspaper or magazine , you cut it out.
 \textit{
	\begin{itemize}
	\item Kids in his neighborhood clipped his picture from the newspaper and carried it around.
	\end{itemize}
}
\item verb \\
If something \textbf{clips} something else, it hits it accidentally at an angle before moving off in a different direction.
 \textit{
	\begin{itemize}
	\item The lorry clipped the rear of a tanker and then crashed into a second truck.
	\end{itemize}
}
\item countable noun \\
If you give someone a \textbf{clip} round the ear , you hit their head fairly lightly with the palm of your hand, usually as a punishment .
 \textit{
	\begin{itemize}
	\item The boy was later given a clip round the ear by his father.
	\end{itemize}
}
\item verb \\
If you \textbf{clip} a small amount \textbf{off} the time taken to do something, you reduce it by that amount.
 \textit{
	\begin{itemize}
	\end{itemize}
}
\item countable noun \\
An ammunition  \textbf{clip} is a metal container on an automatic  weapon which holds ammunition.
 \textit{
	\begin{itemize}
	\end{itemize}
}
\item  \\
 at a clip \textit{
	\begin{itemize}
	\end{itemize}
}
\end{enumerate}

\section*{admit}
{\large \color{blue}  admits  admitting  admitted  }
\subsection*{Explain}
\begin{enumerate}
\item verb \\
If you \textbf{admit} that something bad , unpleasant , or embarrassing is true , you agree , often unwillingly, that it is true.
 \textit{
	\begin{itemize}
	\item I am willing to admit that I do make mistakes.
	\item Up to two-thirds of drivers admit to driving while feeling tired.
	\item I'd be ashamed to admit feeling jealous.
	\item None of these people will admit responsibility for their actions.
	\item 'Actually, most of my tennis is at club level,' he admitted.
	\end{itemize}
}
\item verb \\
If someone \textbf{is admitted}  \textbf{to}  hospital , they are taken into hospital for treatment and kept there until they are well enough to go  home .
 \textit{
	\begin{itemize}
	\item She was admitted to hospital with a soaring temperature.
	\item He was admitted yesterday for treatment of blood clots in his lungs.
	\end{itemize}
}
\item verb \\
If someone \textbf{is admitted}  \textbf{to} an organization or group, they are allowed to join it.
 \textit{
	\begin{itemize}
	\item He was admitted to the Académie Culinaire de France.
	\item ...the continued survival of men's clubs where there is often great resistance to
admitting women.
	\end{itemize}
}
\item verb \\
To \textbf{admit} someone \textbf{to} a place means to allow them to enter it.
 \textit{
	\begin{itemize}
	\item Embassy security personnel refused to admit him or his wife.
	\item Journalists are rarely admitted to the region.
	\end{itemize}
}
\end{enumerate}

\section*{combine}
{\large \color{blue}  combines  combining  combined  }
\subsection*{Explain}
\begin{enumerate}
\item verb \\
If you \textbf{combine} two or more things or if they \textbf{combine} , they exist together.
 \textit{
	\begin{itemize}
	\item The Church has something to say on how to combine freedom with responsibility.
	\item If improved education is combined with other factors dramatic results can be achieved.
	\item Relief workers say it's worse than ever as disease and starvation combine to kill
thousands.
	\item This technique combined with any other therapy is perfectly safe.
	\end{itemize}
}
\item verb \\
If you \textbf{combine} two or more things or if they \textbf{combine} , they join together to make a single thing.
 \textit{
	\begin{itemize}
	\item David Jacobs was given the job of combining the data from these 19 studies into one
giant study.
	\item Combine the flour with 3 tablespoons water to make a paste.
	\item Carbon, hydrogen and oxygen combine chemically to form carbohydrates and fats.
	\item Combined with other compounds, they created a massive dynamite-type bomb.
	\end{itemize}
}
\item verb \\
If someone or something \textbf{combines} two qualities or features , they have both those qualities or features at the same time.
 \textit{
	\begin{itemize}
	\item Their system combines strong government and proportional representation.
	\item ...a clever, far-sighted lawyer who combines legal expertise with social concern.
	\item Her tale has a consciously youthful tone and storyline, combined with a sly humour.
	\end{itemize}
}
\item verb \\
If someone \textbf{combines} two activities, they do them both at the same time.
 \textit{
	\begin{itemize}
	\item It is possible to combine a career with being a mother.
	\item He will combine the two jobs over the next three years.
	\end{itemize}
}
\item verb \\
If two or more groups or organizations  \textbf{combine} or if someone \textbf{combines} them, they join to form a single group or organization.
 \textit{
	\begin{itemize}
	\item ...an announcement by Steetley and Tarmac of a joint venture that would combine their
operations.
	\item Different states or groups can combine to enlarge their markets.
	\end{itemize}
}
\item countable noun \\
A \textbf{combine} is a group of people or organizations that are working or acting together.
 \textit{
	\begin{itemize}
	\item ...an energy and chemicals combine that is Germany's fourth-biggest company.
	\end{itemize}
}
\end{enumerate}

\section*{adopt}
{\large \color{blue}  adopts  adopting  adopted  }
\subsection*{Explain}
\begin{enumerate}
\item verb \\
If you \textbf{adopt} a new attitude , plan, or way of behaving , you begin to have it.
 \textit{
	\begin{itemize}
	\item Parliament adopted a resolution calling for the complete withdrawal of troops.
	\item Pupils should be helped to adopt a positive approach to the environment.
	\end{itemize}
}
\item verb \\
If you \textbf{adopt} someone else's child, you take it into your own family and make it legally your son or daughter .
 \textit{
	\begin{itemize}
	\item There are hundreds of people desperate to adopt a child.
	\item The adopted child has the right to see his birth certificate.
	\end{itemize}
}
\item verb \\
If you \textbf{adopt} a physical position, you move yourself into it.
 \textit{
	\begin{itemize}
	\item I tried to adopt a curled-up position to avoid damaging my limbs.
	\end{itemize}
}
\item verb \\
If you \textbf{adopt} a country, you choose it as a place to live .
 \textit{
	\begin{itemize}
	\item Podulski had joined the U.S. Navy as an aviator, adopting a new country and a new
profession.
	\item ...their adopted home in England.
	\end{itemize}
}
\item verb \\
If you \textbf{adopt} an accent or a particular tone of voice , you speak differently from normal , especially to create an effect in a particular situation .
 \textit{
	\begin{itemize}
	\item Adult actors in American productions were expected to adopt English accents.
	\item The girl was uncertain what to do, or what tone of voice to adopt.
	\end{itemize}
}
\end{enumerate}

\section*{conceal}
{\large \color{blue}  conceals  concealing  concealed  }
\subsection*{Explain}
\begin{enumerate}
\item verb \\
If you \textbf{conceal} something, you cover it or hide it carefully.
 \textit{
	\begin{itemize}
	\item Frances decided to conceal the machine behind a hinged panel.
	\item Five people were arrested for carrying concealed weapons.
	\end{itemize}
}
\item verb \\
If you \textbf{conceal} a piece of information or a feeling, you do not let other people know about it.
 \textit{
	\begin{itemize}
	\item Robert could not conceal his relief.
	\item She knew at once that he was concealing something from her.
	\end{itemize}
}
\item verb \\
If something \textbf{conceals} something else, it covers it and prevents it from being seen .
 \textit{
	\begin{itemize}
	\item ...a pair of carved Indian doors which conceal a built-in cupboard.
	\item The hat concealed her hair.
	\end{itemize}
}
\end{enumerate}

\section*{amaze}
{\large \color{blue}  amazes  amazing  amazed  }
\subsection*{Explain}
\begin{enumerate}
\item verb \\
If something \textbf{amazes} you, it surprises you very much.
 \textit{
	\begin{itemize}
	\item He amazed us by his knowledge of Welsh history.
	\item The Riverside Restaurant promises a variety of food that never ceases to amaze!
	\end{itemize}
}
\end{enumerate}

\section*{confer}
{\large \color{blue}  confers  conferring  conferred  }
\subsection*{Explain}
\begin{enumerate}
\item verb \\
When you \textbf{confer}  \textbf{with} someone, you discuss something with them in order to make a decision . You can also  say that two people \textbf{confer} .
 \textit{
	\begin{itemize}
	\item He conferred with Hill and the others in his office.
	\item His doctors conferred by telephone and agreed that he must get away from his family
for a time.
	\end{itemize}
}
\item verb \\
To \textbf{confer} something such as power or an honour \textbf{on} someone means to give it to them.
 \textit{
	\begin{itemize}
	\item The constitution also confers large powers on Brazil's 25 constituent states.
	\item An honorary doctorate of law was conferred on him by Newcastle University.
	\item Never imagine that rank confers genuine authority.
	\end{itemize}
}
\end{enumerate}

\section*{amend}
{\large \color{blue}  amends  amending  amended  }
\subsection*{Explain}
\begin{enumerate}
\item verb \\
If you \textbf{amend} something that has been written such as a law, or something that is said , you change it in order to improve it or make it more accurate .
 \textit{
	\begin{itemize}
	\item The president agreed to amend the constitution and allow multi-party elections.
	\item 'You must admit that the man has got charm,' said Nicolson. 'Glamour,' amended Wells.
	\item ...the amended version of the Act.
	\end{itemize}
}
\item  \\
 make amends \textit{
	\begin{itemize}
	\end{itemize}
}
\end{enumerate}

\section*{dare}
{\large \color{blue}  dares  daring  dared  }
\subsection*{Explain}
\begin{enumerate}
\item verb \\
If you do not \textbf{dare}  \textbf{to} do something, you do not have enough courage to do it, or you do not want to do it because you fear the consequences . If you \textbf{dare}  \textbf{to} do something, you do something which requires a lot of courage.
 \textbf{Dare} is also a modal .
 \textit{
	\begin{itemize}
	\item Since he was stuck in a lift a year ago he hasn't dared to get back into one.
	\item Most people hate Harry but they don't dare to say so.
	\item He has also dared to take unpopular, but principled stands at times.
	\item We have had problems in our family that I didn't dare tell Uncle.
	\item Dare she risk staying where she was?
	\item The government dare not raise interest rates again.
	\item 'Are you coming with me?'—'I can't, Alice. I daren't.'
	\end{itemize}
}
\item verb \\
If you \textbf{dare} someone \textbf{to} do something, you challenge them to prove that they are not frightened of doing it.
 \textit{
	\begin{itemize}
	\item She looked at him, her eyes daring him to comment.
	\item I dare you to sit through forty-five minutes of that!
	\end{itemize}
}
\item countable noun \\
A \textbf{dare} is a challenge which one person gives to another to do something dangerous or frightening.
 \textit{
	\begin{itemize}
	\item He'd do pretty much anything on a dare.
	\item When found, the children said they'd run away for a dare.
	\end{itemize}
}
\item  \\
 don't you dare \textit{
	\begin{itemize}
	\end{itemize}
}
\item  \\
 how dare you \textit{
	\begin{itemize}
	\end{itemize}
}
\item  \\
 dare I say it \textit{
	\begin{itemize}
	\end{itemize}
}
\item  \\
 I dare say/I daresay \textit{
	\begin{itemize}
	\end{itemize}
}
\end{enumerate}

\section*{amplify}
{\large \color{blue}  amplifies  amplifying  amplified  }
\subsection*{Explain}
\begin{enumerate}
\item verb \\
If you \textbf{amplify} a sound, you make it louder , usually by using electronic  equipment .
 \textit{
	\begin{itemize}
	\item This landscape seemed to trap and amplify sounds.
	\item The music was amplified with microphones.
	\item 'This is the police,' came the amplified voice from the helicopter.
	\end{itemize}
}
\item verb \\
To \textbf{amplify} something means to increase its strength or intensity .
 \textit{
	\begin{itemize}
	\item The mist had been replaced by a kind of haze that seemed to amplify the heat.
	\item Her anxiety about the world was amplifying her personal fears about her future.
	\end{itemize}
}
\end{enumerate}

\section*{deceive}
{\large \color{blue}  deceives  deceiving  deceived  }
\subsection*{Explain}
\begin{enumerate}
\item verb \\
If you \textbf{deceive} someone, you make them believe something that is not true , usually in order to get some advantage for yourself.
 \textit{
	\begin{itemize}
	\item He has deceived and disillusioned us all.
	\item She deceived her father into thinking she was going to school.
	\end{itemize}
}
\item verb \\
If you \textbf{deceive}  \textbf{yourself} , you do not admit to yourself something that you know is true.
 \textit{
	\begin{itemize}
	\item Alcoholics are notorious for their ability to deceive themselves about the extent
of their problem.
	\end{itemize}
}
\item verb \\
If something \textbf{deceives} you, it gives you a wrong  impression and makes you believe something that is not true.
 \textit{
	\begin{itemize}
	\item His gentle, kindly appearance did not deceive me.
	\item The boys, if my eyes did not deceive me, were praying.
	\end{itemize}
}
\end{enumerate}

\section*{annoy}
{\large \color{blue}  annoys  annoying  annoyed  }
\subsection*{Explain}
\begin{enumerate}
\item verb \\
If someone or something \textbf{annoys} you, it makes you fairly  angry and impatient .
 \textit{
	\begin{itemize}
	\item Try making a note of the things which annoy you.
	\item It annoyed me that I didn't have time to do more reading.
	\item It just annoyed me to hear him going on.
	\end{itemize}
}
\end{enumerate}

\section*{disclose}
{\large \color{blue}  discloses  disclosing  disclosed  }
\subsection*{Explain}
\begin{enumerate}
\item verb \\
If you \textbf{disclose} new or secret information, you tell people about it.
 \textit{
	\begin{itemize}
	\item Neither side would disclose details of the transaction.
	\item The company disclosed that he will retire in May.
	\end{itemize}
}
\end{enumerate}

\section*{assure}
{\large \color{blue}  assures  assuring  assured  }
\subsection*{Explain}
\begin{enumerate}
\item verb \\
If you \textbf{assure} someone \textbf{that} something is true or will  happen , you tell them that it is definitely true or will definitely happen, often in order to make them less worried .
 \textit{
	\begin{itemize}
	\item He hastened to assure me that there was nothing traumatic to report.
	\item 'Are you sure the raft is safe?' she asked anxiously. 'Couldn't be safer,' Max assured
her confidently.
	\item Government officials recently assured Hindus of protection.
	\end{itemize}
}
\item verb \\
To \textbf{assure} someone \textbf{of} something means to make certain that they will get it.
 \textit{
	\begin{itemize}
	\item Real Madrid's 2-1 victory has all but assured them of the title.
	\item Ways must be found to assure our children a decent start in life.
	\end{itemize}
}
\item  \\
 I can assure you/let me assure you \textit{
	\begin{itemize}
	\end{itemize}
}
\end{enumerate}

\section*{ease}
{\large \color{blue}  eases  easing  eased  }
\subsection*{Explain}
\begin{enumerate}
\item  \\
 with ease \textit{
	\begin{itemize}
	\end{itemize}
}
\item uncountable noun \\
If you talk about the \textbf{ease of} a particular activity, you are referring to the way that it has been made easier to do, or to the fact that it is already easy to do.
 \textit{
	\begin{itemize}
	\item For ease of reference, only the relevant extracts of the regulations are included.
	\item ...the camera's ease of use in manual mode.
	\end{itemize}
}
\item uncountable noun \\
\textbf{Ease} is the state of being very comfortable and able to live as you want , without any worries or problems .
 \textit{
	\begin{itemize}
	\item She lived a life of ease.
	\end{itemize}
}
\item verb \\
If something unpleasant  \textbf{eases} or if you \textbf{ease} it, it is reduced in degree , speed , or intensity .
 \textit{
	\begin{itemize}
	\item Tensions had eased.
	\item The heavily falling snow had eased.
	\item I gave him some brandy to ease the pain.
	\item ...editorials calling for the easing of sanctions.
	\end{itemize}
}
\item verb \\
If you \textbf{ease} your \textbf{way}  somewhere or \textbf{ease} somewhere, you move there slowly, carefully, and gently. If you \textbf{ease} something somewhere, you move it there slowly, carefully, and gently.
 \textit{
	\begin{itemize}
	\item I eased my way towards the door.
	\item She eased back into the chair and nodded.
	\item He eased his foot off the accelerator.
	\item Leaphorn eased himself silently upward.
	\item I eased open the door.
	\end{itemize}
}
\item  \\
 at ease \textit{
	\begin{itemize}
	\end{itemize}
}
\item  \\
 at ease \textit{
	\begin{itemize}
	\end{itemize}
}
\item  \\
 ill at ease \textit{
	\begin{itemize}
	\end{itemize}
}
\end{enumerate}

\section*{astonish}
{\large \color{blue}  astonishes  astonishing  astonished  }
\subsection*{Explain}
\begin{enumerate}
\item verb \\
If something or someone \textbf{astonishes} you, they surprise you very much.
 \textit{
	\begin{itemize}
	\item My news will astonish you.
	\item Her dedication constantly astonishes me.
	\end{itemize}
}
\end{enumerate}

\section*{enhance}
{\large \color{blue}  enhances  enhancing  enhanced  }
\subsection*{Explain}
\begin{enumerate}
\item verb \\
To \textbf{enhance} something means to improve its value, quality, or attractiveness.
 \textit{
	\begin{itemize}
	\item They'll be keen to enhance their reputation abroad.
	\item The beiges and creams of its suites are enhanced by splashes of warm colour in the
cushions and throws.
	\end{itemize}
}
\end{enumerate}

\section*{believe}
{\large \color{blue}  believes  believing  believed  }
\subsection*{Explain}
\begin{enumerate}
\item verb \\
If you \textbf{believe} that something is true, you think that it is true, but you are not sure .
 \textit{
	\begin{itemize}
	\item Experts believe that the coming drought will be extensive.
	\item I believe you have something of mine.
	\item The main problem, I believe, lies elsewhere.
	\item We believe them to be hidden here in this apartment.
	\item 'You've never heard of him?'—'I don't believe so.'
	\end{itemize}
}
\item verb \\
If you \textbf{believe} someone or if you \textbf{believe} what they say or write , you accept that they are telling the truth.
 \textit{
	\begin{itemize}
	\item He did not sound as if he believed her.
	\item They no longer believe anything they are told.
	\item Don't believe what you read in the papers.
	\end{itemize}
}
\item verb \\
If you \textbf{believe}  \textbf{in}  fairies , ghosts , or miracles , you are sure that they exist or happen . If you \textbf{believe}  \textbf{in} a god , you are sure of the existence of that god.
 \textit{
	\begin{itemize}
	\item I don't believe in ghosts.
	\item Do you believe in magic?
	\end{itemize}
}
\item verb \\
If you \textbf{believe in} a way of life or an idea , you are in favour of it because you think it is good or right .
 \textit{
	\begin{itemize}
	\item He believed in marital fidelity.
	\item ...decent candidates who believed in democracy.
	\end{itemize}
}
\item verb \\
If you \textbf{believe in} someone or what they are doing, you have confidence in them and think that they will be successful .
 \textit{
	\begin{itemize}
	\item If you believe in yourself you can succeed.
	\end{itemize}
}
\item verb \\
\textbf{Believe} is used in expressions such as \textbf{I can't believe how} or \textbf{it's hard to believe that} in order to express  surprise , for example because something bad has happened or something very difficult has been achieved .
 \textit{
	\begin{itemize}
	\item Many officers I spoke to found it hard to believe what was happening around them.
	\item I was a physical and emotional wreck–I still can't believe how I ever got any work
done.
	\end{itemize}
}
\item  \\
 believe your eyes/believe your ears \textit{
	\begin{itemize}
	\end{itemize}
}
\item  \\
 believe it or not \textit{
	\begin{itemize}
	\end{itemize}
}
\item  \\
 would you believe it \textit{
	\begin{itemize}
	\end{itemize}
}
\item  \\
 believe you me \textit{
	\begin{itemize}
	\end{itemize}
}
\end{enumerate}

\section*{ensure}
{\large \color{blue}  ensures  ensuring  ensured  }
\subsection*{Explain}
\begin{enumerate}
\item verb \\
To \textbf{ensure} something, or to \textbf{ensure}  \textbf{that} something happens , means to make certain that it happens.
 \textit{
	\begin{itemize}
	\item Negotiators ensured that the treaty was a significant change in direction.
	\item Ensure that it is written into your contract.
	\item ...the President's Council, which ensures the supremacy of the National Party.
	\end{itemize}
}
\end{enumerate}

\section*{build}
{\large \color{blue}  builds  building  built  }
\subsection*{Explain}
\begin{enumerate}
\item verb \\
If you \textbf{build} something, you make it by joining things together .
 \textit{
	\begin{itemize}
	\item Developers are now proposing to build a hotel on the site.
	\item The house was built in the early 19th century.
	\item Workers at the plant build the F-16 jet fighter.
	\end{itemize}
}
\item verb \\
If you \textbf{build} something \textbf{into} a wall or object, you make it in such a way that it is in the wall or object, or is part
of it.
 \textit{
	\begin{itemize}
	\item If the TV was built into the ceiling, you could lie there while watching your favourite
programme.
	\end{itemize}
}
\item verb \\
If people \textbf{build} an organization , a society , or a relationship , they gradually form it.
 \textit{
	\begin{itemize}
	\item He and a partner set up on their own and built a successful fashion company.
	\item Their purpose is to build a fair society and a strong economy.
	\item I wanted to build a relationship with my team.
	\end{itemize}
}
\item verb \\
If you \textbf{build} an organization, system, or product  \textbf{on} something, you base it on it.
 \textit{
	\begin{itemize}
	\item We will then have a firmer foundation of fact on which to build theories.
	\item The town's nineteenth-century prosperity was built on steel.
	\end{itemize}
}
\item verb \\
If you \textbf{build} something \textbf{into} a policy , system, or product, you make it part of it.
 \textit{
	\begin{itemize}
	\item We have to build computers into the school curriculum.
	\item How much delay should we build into the plan?
	\end{itemize}
}
\item verb \\
To \textbf{build} someone's confidence or trust  means to increase it gradually. If someone's confidence or trust \textbf{builds} , it increases gradually.
 \textbf{Build up} means the same as build .
 \textit{
	\begin{itemize}
	\item The encouragement that young boys receive builds a greater self-confidence.
	\item Diplomats hope the meetings will build mutual trust.
	\item Usually when we're six months or so into a recovery, confidence begins to build.
	\item The delegations had begun to build up some trust in one another.
	\item We will start to see the confidence in the housing market building up again.
	\end{itemize}
}
\item verb \\
If you \textbf{build}  \textbf{on} the success of something, you take advantage of this success in order to make further progress .
 \textit{
	\begin{itemize}
	\item Build on the qualities you are satisfied with and work to change those you are unhappy
with.
	\item The new regime has no successful economic reforms on which to build.
	\end{itemize}
}
\item verb \\
If pressure , speed , sound, or excitement  \textbf{builds} , it gradually becomes greater .
 \textbf{Build up} means the same as build .
 \textit{
	\begin{itemize}
	\item Pressure built yesterday for postponement of the ceremony.
	\item The last chords of the suite build to a crescendo.
	\item We can build up the speed gradually and safely.
	\item Economists warn that enormous pressures could build up, forcing people to emigrate
westwards.
	\end{itemize}
}
\item variable noun \\
Someone's \textbf{build} is the shape that their bones and muscles give to their body.
 \textit{
	\begin{itemize}
	\item He's described as around thirty years old, six feet tall and of medium build.
	\item The authority of his voice is undermined by the smallness of his build.
	\end{itemize}
}
\end{enumerate}

\section*{exceed}
{\large \color{blue}  exceeds  exceeding  exceeded  }
\subsection*{Explain}
\begin{enumerate}
\item verb \\
If something \textbf{exceeds} a particular amount or number, it is greater or larger than that amount or number.
 \textit{
	\begin{itemize}
	\item Its research budget exceeds $700 million a year.
	\item The demand for places at some schools exceeds the supply.
	\item His performance exceeded all expectations.
	\end{itemize}
}
\item verb \\
If you \textbf{exceed} a limit or rule , you go beyond it, even though you are not supposed to or it is against the law .
 \textit{
	\begin{itemize}
	\item He accepts he was exceeding the speed limit.
	\item I would be exceeding my powers if I ordered the march to be halted.
	\end{itemize}
}
\end{enumerate}

\section*{concede}
{\large \color{blue}  concedes  conceding  conceded  }
\subsection*{Explain}
\begin{enumerate}
\item verb \\
If you \textbf{concede} something, you admit, often unwillingly, that it is true or correct.
 \textit{
	\begin{itemize}
	\item Bess finally conceded that Nancy was right.
	\item 'Well,' he conceded, 'I do sometimes mumble a bit.'
	\item Mr. Chapman conceded the need for Nomura's U.S. unit to improve its trading skills.
	\end{itemize}
}
\item verb \\
If you \textbf{concede} something \textbf{to} someone, you allow them to have it as a right or privilege .
 \textit{
	\begin{itemize}
	\item The government conceded the right to establish independent trade unions.
	\item The French subsequently conceded full independence to Laos.
	\end{itemize}
}
\item verb \\
If you \textbf{concede} something, you give it to the person who has been trying to get it from you.
 \textit{
	\begin{itemize}
	\item The strike ended after the government conceded some of their demands.
	\end{itemize}
}
\item verb \\
In sport , if you \textbf{concede}  goals or points, you are unable to prevent your opponent from scoring them.
 \textit{
	\begin{itemize}
	\item They conceded four goals to Leeds United.
	\item Luton conceded a free kick on the edge of the penalty area.
	\end{itemize}
}
\item verb \\
If you \textbf{concede} a game , contest , or argument , you end it by admitting that you can no longer win .
 \textit{
	\begin{itemize}
	\item Reiner, 56, has all but conceded the race to his rival.
	\item Alain Prost finished third and virtually conceded the world championship.
	\end{itemize}
}
\item verb \\
If you \textbf{concede}  defeat , you accept that you have lost a struggle .
 \textit{
	\begin{itemize}
	\item The company conceded defeat in its attempt to take control of its holiday industry
rival.
	\item He happily conceded the election.
	\end{itemize}
}
\end{enumerate}

\section*{expire}
{\large \color{blue}  expires  expiring  expired  }
\subsection*{Explain}
\begin{enumerate}
\item verb \\
When something such as a contract , deadline , or visa  \textbf{expires} , it comes to an end or is no longer valid .
 \textit{
	\begin{itemize}
	\item He had lived illegally in the United States for five years after his visitor's visa
expired.
	\end{itemize}
}
\item verb \\
When someone \textbf{expires} , they die.
 \textit{
	\begin{itemize}
	\item He endured excruciating agonies before he finally expired.
	\end{itemize}
}
\end{enumerate}

\section*{confine}
{\large \color{blue}  confines  confining  confined  }
\subsection*{Explain}
\begin{enumerate}
\item verb \\
To \textbf{confine} something \textbf{to} a particular place or group means to prevent it from spreading beyond that place or group.
 \textit{
	\begin{itemize}
	\item Health officials have successfully confined the epidemic to the Tabatinga area.
	\item The U.S. will soon be taking steps to confine the conflict.
	\end{itemize}
}
\item verb \\
If you \textbf{confine}  \textbf{yourself} or your activities \textbf{to} something, you do only that thing and are involved with nothing else.
 \textit{
	\begin{itemize}
	\item He did not confine himself to the one language.
	\item Yoko had largely confined her activities to the world of big business.
	\item His genius was not confined to the decoration of buildings.
	\end{itemize}
}
\item verb \\
If someone \textbf{is confined to} a mental  institution , prison , or other place, they are sent there and are not allowed to leave for a period of time.
 \textit{
	\begin{itemize}
	\item The woman will be confined to a mental institution.
	\item He announced that the army and police had been confined to barracks.
	\end{itemize}
}
\item plural noun \\
Something that is within the \textbf{confines}  \textbf{of} an area or place is within the boundaries enclosing it.
 \textit{
	\begin{itemize}
	\item The movie is set entirely within the confines of the abandoned factory.
	\item ...the wild grass and weeds that grew in the confines of the grandstand.
	\end{itemize}
}
\item plural noun \\
The \textbf{confines}  \textbf{of} a situation , system, or activity are the limits or restrictions it involves.
 \textit{
	\begin{itemize}
	\item ...away from the confines of the British class system.
	\item I can't stand the confines of this marriage.
	\end{itemize}
}
\end{enumerate}

\section*{heighten}
{\large \color{blue}  heightens  heightening  heightened  }
\subsection*{Explain}
\begin{enumerate}
\item verb \\
If something \textbf{heightens} a feeling or if the feeling \textbf{heightens} , the feeling increases in degree or intensity .
 \textit{
	\begin{itemize}
	\item The move has heightened tension in the state.
	\item These latest murders have heightened fears of further attacks.
	\item Cross's interest heightened.
	\item ...a heightened awareness of the dangers that they now face.
	\end{itemize}
}
\end{enumerate}

\section*{constitute}
{\large \color{blue}  constitutes  constituting  constituted  }
\subsection*{Explain}
\begin{enumerate}
\item link verb \\
If something \textbf{constitutes} a particular thing, it can be regarded as being that thing.
 \textit{
	\begin{itemize}
	\item Testing patients without their consent would constitute a professional and legal
offence.
	\item The vote hardly constitutes a victory.
	\item What constitutes abuse?
	\end{itemize}
}
\item link verb \\
If a number of things or people \textbf{constitute} something, they are the parts or members that form it.
 \textit{
	\begin{itemize}
	\item Volunteers constitute more than 95% of The Center's work force.
	\end{itemize}
}
\item verb \\
When something such as a committee or government \textbf{is constituted} , it is formally established and given authority to operate .
 \textit{
	\begin{itemize}
	\item On 6 July, a People's Revolutionary Government was constituted.
	\item The accused will appear before a specially-constituted military tribunal.
	\end{itemize}
}
\end{enumerate}

\section*{hide}
{\large \color{blue}  hides  hiding  hid  hidden  }
\subsection*{Explain}
\begin{enumerate}
\item verb \\
If you \textbf{hide} something or someone, you put them in a place where they cannot easily be seen or found.
 \textit{
	\begin{itemize}
	\item He hid the bicycle in the hawthorn hedge.
	\item They could see that I was terrified, and hid me until the coast was clear.
	\end{itemize}
}
\item verb \\
If you \textbf{hide} or if you \textbf{hide}  \textbf{yourself} , you go  somewhere where you cannot easily be seen or found.
 \textit{
	\begin{itemize}
	\item At their approach the little boy scurried away and hid.
	\item They hid themselves behind a tree.
	\end{itemize}
}
\item verb \\
If you \textbf{hide} your face , you press your face against something or cover your face with something, so that people cannot see it.
 \textit{
	\begin{itemize}
	\item She hid her face under the collar of his jacket and she started to cry.
	\item He hid his face in his hands again, lost in his own thoughts.
	\end{itemize}
}
\item verb \\
If you \textbf{hide} what you feel or know , you keep it a secret, so that no one knows about it.
 \textit{
	\begin{itemize}
	\item Lee tried to hide his excitement.
	\item I have absolutely nothing to hide, I have done nothing wrong.
	\item Alison was not the sort of person to hide anything from her dad.
	\end{itemize}
}
\item verb \\
If something \textbf{hides} an object, it covers it and prevents it from being seen.
 \textit{
	\begin{itemize}
	\item The man's heavy moustache hid his upper lip completely.
	\item The compound was hidden by trees and shrubs.
	\end{itemize}
}
\item countable noun \\
A \textbf{hide} is a place which is built to look  like its surroundings. Hides are used by people who want to watch or photograph animals and birds without being seen by them.
 \textit{
	\begin{itemize}
	\end{itemize}
}
\item variable noun \\
A \textbf{hide} is the skin of a large animal such as a cow , horse , or elephant , which can be used for making leather .
 \textit{
	\begin{itemize}
	\item ...the process of tanning animal hides.
	\item ...kangaroo hide.
	\end{itemize}
}
\end{enumerate}

\section*{constrain}
{\large \color{blue}  constrains  constraining  constrained  }
\subsection*{Explain}
\begin{enumerate}
\item verb \\
To \textbf{constrain} someone or something means to limit their development or force them to behave in a particular way .
 \textit{
	\begin{itemize}
	\item Many working parents are too often constrained by inflexible working hours.
	\item The company is constrained to offer salaries that can only attract mediocre staff.
	\item The capacity of those roads will constrain the amount of car travel.
	\end{itemize}
}
\item  \\
 feel constrained \textit{
	\begin{itemize}
	\end{itemize}
}
\end{enumerate}

\section*{insert}
{\large \color{blue}  inserts  inserting  inserted  }
\subsection*{Explain}
\begin{enumerate}
\item verb \\
If you \textbf{insert} an object  \textbf{into} something, you put the object inside it.
 \textit{
	\begin{itemize}
	\item He took a small key from his pocket and slowly inserted it into the lock.
	\item Wait for a couple of minutes with your mouth closed before inserting the thermometer.
	\end{itemize}
}
\item verb \\
If you \textbf{insert} a comment into a piece of writing or a speech , you include it.
 \textit{
	\begin{itemize}
	\item They joined with the monarchists to insert a clause calling for a popular vote on
the issue.
	\end{itemize}
}
\item countable noun \\
An \textbf{insert} is something that is inserted somewhere , especially an advertisement on a piece of paper that is placed between the pages of a book or magazine .
 \textit{
	\begin{itemize}
	\end{itemize}
}
\end{enumerate}

\section*{disrupt}
{\large \color{blue}  disrupts  disrupting  disrupted  }
\subsection*{Explain}
\begin{enumerate}
\item verb \\
If someone or something \textbf{disrupts} an event, system, or process, they cause difficulties that prevent it from continuing or operating in a normal way.
 \textit{
	\begin{itemize}
	\item Anti-war protesters disrupted the debate.
	\item The drought has severely disrupted agricultural production.
	\end{itemize}
}
\end{enumerate}

\section*{emerge}
{\large \color{blue}  emerges  emerging  emerged  }
\subsection*{Explain}
\begin{enumerate}
\item verb \\
To \textbf{emerge} means to come out from an enclosed or dark  space such as a room or a vehicle , or from a position where you could not be seen .
 \textit{
	\begin{itemize}
	\item Richard was waiting outside the door as she emerged.
	\item The postman emerged from his van soaked to the skin.
	\item ...holes made by the emerging adult beetle.
	\end{itemize}
}
\item verb \\
If you \textbf{emerge from} a difficult or bad experience, you come to the end of it.
 \textit{
	\begin{itemize}
	\item There is growing evidence that the economy is at last emerging from recession.
	\item ...their plans to emerge from bankruptcy by February of next year.
	\end{itemize}
}
\item verb \\
If a fact or result \textbf{emerges} from a period of thought , discussion , or investigation , it becomes known as a result of it.
 \textit{
	\begin{itemize}
	\item ...the growing corruption that has emerged in the past few years.
	\item It soon emerged that neither the July nor August mortgage repayment had been collected.
	\item The emerging caution over numbers is perhaps only to be expected.
	\end{itemize}
}
\item verb \\
If someone or something \textbf{emerges}  \textbf{as} a particular thing, they become recognized as that thing.
 \textit{
	\begin{itemize}
	\item He emerged as a major figure in the reform movement.
	\item Vietnam has emerged as the world's third-biggest rice exporter.
	\item New leaders have emerged.
	\end{itemize}
}
\item verb \\
When something such as an organization or an industry  \textbf{emerges} , it comes into existence .
 \textit{
	\begin{itemize}
	\item ...the new republic that emerged in October 1917.
	\item New skills are demanded for emerging industries.
	\end{itemize}
}
\end{enumerate}

\section*{maintain}
{\large \color{blue}  maintains  maintaining  maintained  }
\subsection*{Explain}
\begin{enumerate}
\item verb \\
If you \textbf{maintain} something, you continue to have it, and do not let it stop or grow  weaker .
 \textit{
	\begin{itemize}
	\item The Department maintains many close contacts with the chemical industry.
	\item Push yourself to make friends and to maintain the friendships.
	\item ...the emergency powers to try to maintain law and order.
	\end{itemize}
}
\item verb \\
If you say that someone \textbf{maintains}  \textbf{that} something is true , you mean that they have stated their opinion strongly but not everyone agrees with them or believes them.
 \textit{
	\begin{itemize}
	\item He has maintained that the money was donated for international purposes.
	\item Prosecutors maintain no deal was made.
	\item 'Not all feminism has to be like this,' Jo maintains.
	\item He had always maintained his innocence.
	\end{itemize}
}
\item verb \\
If you \textbf{maintain} something \textbf{at} a particular  rate or level , you keep it at that rate or level.
 \textit{
	\begin{itemize}
	\item The government was right to maintain interest rates at a high level.
	\item ...action is required to ensure standards are maintained at as high a level as possible.
	\end{itemize}
}
\item verb \\
If you \textbf{maintain} a road , building , vehicle , or machine , you keep it in good condition by regularly checking it and repairing it when necessary .
 \textit{
	\begin{itemize}
	\item ...a tough campaign to force authorities to maintain roads properly.
	\item The house costs a fortune to maintain.
	\item The cars are getting older and less well-maintained.
	\end{itemize}
}
\item verb \\
If you \textbf{maintain} someone, you provide them with money and other things that they need .
 \textit{
	\begin{itemize}
	\item ...the basic costs of maintaining a child.
	\end{itemize}
}
\end{enumerate}

\section*{enjoy}
{\large \color{blue}  enjoys  enjoying  enjoyed  }
\subsection*{Explain}
\begin{enumerate}
\item verb \\
If you \textbf{enjoy} something, you find pleasure and satisfaction in doing it or experiencing it.
 \textit{
	\begin{itemize}
	\item Ross had always enjoyed the company of women.
	\item He was a guy who enjoyed life to the full.
	\item I enjoyed playing cricket.
	\end{itemize}
}
\item verb \\
If you \textbf{enjoy}  \textbf{yourself} , you do something that you like doing or you take pleasure in the situation that you are in.
 \textit{
	\begin{itemize}
	\item I must say I am really enjoying myself at the moment.
	\end{itemize}
}
\item verb \\
If you \textbf{enjoy} something such as a right , benefit, or privilege , you have it.
 \textit{
	\begin{itemize}
	\item The average German will enjoy 40 days' paid holiday this year.
	\item He enjoys a reputation for honesty.
	\end{itemize}
}
\end{enumerate}

\section*{negotiate}
{\large \color{blue}  negotiates  negotiating  negotiated  }
\subsection*{Explain}
\begin{enumerate}
\item verb \\
If people \textbf{negotiate}  \textbf{with} each other or \textbf{negotiate} an agreement, they talk about a problem or a situation such as a business arrangement in order to solve the problem or complete the arrangement.
 \textit{
	\begin{itemize}
	\item It is not clear whether the president is willing to negotiate with the democrats.
	\item When you have two adversaries negotiating, you need to be on neutral territory.
	\item The local government and the army negotiated a truce.
	\item Western governments have this week urged him to negotiate and avoid force.
	\item He has bravely negotiated an end to the country's civil war.
	\item His publishing house had just begun negotiating for her next books.
	\item There were reports that three companies were negotiating to market the drug.
	\end{itemize}
}
\item verb \\
If you \textbf{negotiate} an area of land, a place, or an obstacle , you successfully travel across it or around it.
 \textit{
	\begin{itemize}
	\item Frank Mariano negotiates the desert terrain in his battered pickup.
	\item I negotiated the corner on my motorbike and pulled to a stop.
	\item I negotiated my way out of the airport and joined the flow of cars.
	\end{itemize}
}
\end{enumerate}

\section*{found}
{\large \color{blue}  founds  founding  founded  }
\subsection*{Explain}
\begin{enumerate}
\item  \\
\textbf{Found} is the past  tense and past participle of find .
 \textit{
	\begin{itemize}
	\end{itemize}
}
\item verb \\
When an institution, company , or organization  \textbf{is founded} by someone or by a group of people, they get it started , often by providing the necessary  money .
 \textit{
	\begin{itemize}
	\item The Independent Labour Party was founded in Bradford on January 13, 1893.
	\item He founded the Centre for Journalism Studies at University College Cardiff.
	\item The business, founded by Dawn and Nigel, suffered financial setbacks.
	\end{itemize}
}
\item verb \\
When a town, important building, or other place \textbf{is founded} by someone or by a group of people, they cause it to be built.
 \textit{
	\begin{itemize}
	\item The town was founded in 1610.
	\end{itemize}
}
\end{enumerate}

\section*{nominate}
{\large \color{blue}  nominates  nominating  nominated  }
\subsection*{Explain}
\begin{enumerate}
\item verb \\
If someone \textbf{is nominated} for a job or position, their name is formally suggested as a candidate for it.
 \textit{
	\begin{itemize}
	\item Under party rules each candidate has to be nominated by 55 Labour MPs.
	\item The public will be able to nominate candidates for awards such as the MBE.
	\item ...a presidential decree nominating him as sports ambassador.
	\end{itemize}
}
\item verb \\
If you \textbf{nominate} someone to a job or position, you formally choose them to hold that job or position.
 \textit{
	\begin{itemize}
	\item Voters will choose fifty of the seventy five deputies. The Emir will nominate the
rest.
	\item The E.U. would nominate two members to the committee.
	\item He was nominated by the African National Congress as one of its team at the Groote
Sehuur talks.
	\item An elderly person can nominate someone to act for them.
	\end{itemize}
}
\item verb \\
If someone or something such as an actor or a film  \textbf{is nominated} for an award , someone formally suggests that they should be given that award.
 \textit{
	\begin{itemize}
	\item Practically every movie he made was nominated for an Oscar.
	\item ...a campaign to nominate the twice World Champion as Sports Personality of the Year.
	\end{itemize}
}
\end{enumerate}

\section*{inhabit}
{\large \color{blue}  inhabits  inhabiting  inhabited  }
\subsection*{Explain}
\begin{enumerate}
\item verb \\
If a place or region \textbf{is inhabited} by a group of people or a species of animal, those people or animals live there.
 \textit{
	\begin{itemize}
	\item The valley is inhabited by the Dani tribe.
	\item ...the people who inhabit these islands.
	\item ...the beautifully coloured fish that inhabit the Egyptian reefs.
	\item ...a land primarily inhabited by nomads.
	\end{itemize}
}
\end{enumerate}

\section*{nourish}
{\large \color{blue}  nourishes  nourishing  nourished  }
\subsection*{Explain}
\begin{enumerate}
\item verb \\
To \textbf{nourish} a person, animal, or plant means to provide them with the food that is necessary for life, growth, and good health .
 \textit{
	\begin{itemize}
	\item The food she eats nourishes both her and the baby.
	\item ...microbes in the soil which nourish the plant.
	\end{itemize}
}
\item verb \\
To \textbf{nourish} something such as a feeling or belief means to allow or encourage it to grow .
 \textit{
	\begin{itemize}
	\item Journalists on the whole don't create public opinion. They can help to nourish it.
	\item ...a current of thought which has been nourished by historical tradition.
	\end{itemize}
}
\end{enumerate}

\section*{inject}
{\large \color{blue}  injects  injecting  injected  }
\subsection*{Explain}
\begin{enumerate}
\item verb \\
To \textbf{inject} someone with a substance such as a medicine means to put it into their body using a device with a needle  called a syringe.
 \textit{
	\begin{itemize}
	\item His son was injected with strong drugs.
	\item The technique consists of injecting healthy cells into the weakened muscles.
	\item He needs to inject himself once a month.
	\end{itemize}
}
\item verb \\
If you \textbf{inject} a new, exciting , or interesting quality \textbf{into} a situation , you add it.
 \textit{
	\begin{itemize}
	\item She kept trying to inject a little fun into their relationship.
	\item The result might inject more sense into future bargaining.
	\end{itemize}
}
\item verb \\
If you \textbf{inject} money or resources  \textbf{into} a business or organization, you provide more money or resources for it.
 \textit{
	\begin{itemize}
	\item He has injected £5.6 billion into the health service.
	\end{itemize}
}
\end{enumerate}

\section*{offer}
{\large \color{blue}  offers  offering  offered  }
\subsection*{Explain}
\begin{enumerate}
\item verb \\
If you \textbf{offer} something to someone, you ask them if they would like to have it or use it.
 \textit{
	\begin{itemize}
	\item He has offered seats at the conference table to the Russian leader and the president
of Kazakhstan.
	\item The number of companies offering them work increased.
	\item Rhys offered him an apple.
	\item Western governments have offered aid.
	\end{itemize}
}
\item verb \\
If you \textbf{offer}  \textbf{to} do something, you say that you are willing to do it.
 \textit{
	\begin{itemize}
	\item Peter offered to teach them water-skiing.
	\item 'Can I get you a drink?' she offered.
	\end{itemize}
}
\item countable noun \\
An \textbf{offer} is something that someone says they will give you or do for you.
 \textit{
	\begin{itemize}
	\item The offer of talks marks a significant change from their previous position.
	\item 'I ought to reconsider her offer to move in,' he mused.
	\item He had refused several excellent job offers.
	\end{itemize}
}
\item verb \\
If you \textbf{offer} someone information , advice , or praise , you give it to them, usually because you feel that they need it or deserve it.
 \textit{
	\begin{itemize}
	\item They manage a company offering advice on mergers and acquisitions.
	\item She offered him emotional and practical support in countless ways.
	\item They are offered very little counselling or support.
	\end{itemize}
}
\item verb \\
If you \textbf{offer} someone something such as love or friendship, you show them that you feel that way towards them.
 \textit{
	\begin{itemize}
	\item The Prime Minister offered his sympathy to the families of the victims.
	\item It must be better to be able to offer them love and security.
	\item John's mother and sister rallied round offering comfort.
	\end{itemize}
}
\item verb \\
If people \textbf{offer} prayers, praise, or a sacrifice to God or a god, they speak to or give something to their god.
 \textbf{Offer up}  means the same as offer .
 \textit{
	\begin{itemize}
	\item Church leaders offered prayers and condemned the bloodshed.
	\item He will offer the first harvest of rice to the sun goddess.
	\item He should consider offering up a prayer to St Lambert.
	\end{itemize}
}
\item verb \\
If an organization  \textbf{offers} something such as a service or product , it provides it.
 \textit{
	\begin{itemize}
	\item We have been successful because we are offering a quality service.
	\item Sainsbury's is offering customers 1p for each shopping bag re-used.
	\item The insurance company offers a 10% discount to the over-55s.
	\end{itemize}
}
\item countable noun \\
An \textbf{offer} in a shop is a specially low  price for a specific product or something extra that you get if you buy a certain product.
 \textit{
	\begin{itemize}
	\item This month's offers include a shirt, trousers and bed covers.
	\item Today's special offer gives you a choice of three destinations.
	\item Over 40 new books are on offer at 25 per cent off their normal retail price.
	\end{itemize}
}
\item verb \\
If you \textbf{offer} a particular amount of money for something, you say that you will pay that much to buy it.
 \textit{
	\begin{itemize}
	\item Whitney has offered $21.50 a share in cash.
	\item They are offering farmers $2.15 a bushel for corn.
	\item He will write Rachel a note and offer her a fair price for the land.
	\item It was his custom in buying real estate to offer a rather low price.
	\end{itemize}
}
\item countable noun \\
An \textbf{offer} is the amount of money that someone says they will pay to buy something or give to
someone because they have harmed them in some way.
 \textit{
	\begin{itemize}
	\item The lawyers say no one else will make me an offer.
	\item He has dismissed an offer of compensation.
	\end{itemize}
}
\item  \\
 have sth to offer \textit{
	\begin{itemize}
	\end{itemize}
}
\item  \\
 on offer \textit{
	\begin{itemize}
	\end{itemize}
}
\item  \\
 open to offers \textit{
	\begin{itemize}
	\end{itemize}
}
\end{enumerate}

\section*{insure}
{\large \color{blue}  insures  insuring  insured  }
\subsection*{Explain}
\begin{enumerate}
\item verb \\
If you \textbf{insure} yourself or your property , you pay  money to an insurance company so that, if you become  ill or if your property is damaged or stolen , the company will pay you a sum of money.
 \textit{
	\begin{itemize}
	\item For protection against unforeseen emergencies, you insure your house and your car.
	\item Think carefully before you insure against accident, sickness and redundancy.
	\item We automatically insure your belongings against fire and theft.
	\end{itemize}
}
\item verb \\
If you \textbf{insure}  \textbf{yourself against} something unpleasant that might  happen in the future , you do something to protect yourself in case it happens, or to prevent it happening .
 \textit{
	\begin{itemize}
	\item He insured himself against failure by treating only people he was sure he could cure.
	\item All the electronics in the world cannot insure against accidents, though.
	\end{itemize}
}
\end{enumerate}

\section*{overwhelm}
{\large \color{blue}  overwhelms  overwhelming  overwhelmed  }
\subsection*{Explain}
\begin{enumerate}
\item verb \\
If you \textbf{are overwhelmed}  \textbf{by} a feeling or event, it affects you very strongly, and you do not know how to deal with it.
 \textit{
	\begin{itemize}
	\item He was overwhelmed by a longing for times past.
	\item The need to talk to someone, anyone, overwhelmed her.
	\end{itemize}
}
\item verb \\
If a group of people \textbf{overwhelm} a place or another group, they gain  complete control or victory over them.
 \textit{
	\begin{itemize}
	\item It was clear that one massive Allied offensive would overwhelm the weakened enemy.
	\end{itemize}
}
\end{enumerate}

\section*{intimidate}
{\large \color{blue}  intimidates  intimidating  intimidated  }
\subsection*{Explain}
\begin{enumerate}
\item verb \\
If you \textbf{intimidate} someone, you deliberately make them frightened enough to do what you want them to do.
 \textit{
	\begin{itemize}
	\item Jones had set out to intimidate and dominate Paul.
	\item Attempts to intimidate people into voting for the governing party did not work.
	\end{itemize}
}
\end{enumerate}

\section*{propose}
{\large \color{blue}  proposes  proposing  proposed  }
\subsection*{Explain}
\begin{enumerate}
\item verb \\
If you \textbf{propose} something such as a plan or an idea , you suggest it for people to think about and decide upon.
 \textit{
	\begin{itemize}
	\item Britain is about to propose changes to some institutions.
	\item It was George who first proposed that we dry clothes in that locker.
	\end{itemize}
}
\item verb \\
If you \textbf{propose}  \textbf{to} do something, you intend to do it.
 \textit{
	\begin{itemize}
	\item It's still far from clear what action the government proposes to take over the affair.
	\item And where do you propose building such a huge thing?
	\end{itemize}
}
\item verb \\
If you \textbf{propose} a theory or an explanation , you state that it is possibly or probably  true , because it fits in with the evidence that you have considered .
 \textit{
	\begin{itemize}
	\item This highlights a problem faced by people proposing theories of ball lightning.
	\item Newton proposed that heavenly and terrestrial motion could be unified with the idea
of gravity.
	\end{itemize}
}
\item verb \\
If you \textbf{propose} a motion for debate , or a candidate for election , you begin the debate or the election procedure by formally stating your support for that motion or candidate.
 \textit{
	\begin{itemize}
	\item She was a pioneer in proposing that women should be able to control their own fertility.
	\item I asked Robin Balfour and Derek Haig to propose and second me.
	\end{itemize}
}
\item verb \\
If you \textbf{propose} a toast to someone or something, you ask people to drink a toast to them.
 \textit{
	\begin{itemize}
	\item Usually the bride's father proposes a toast to the health of the bride and groom.
	\end{itemize}
}
\item verb \\
If you \textbf{propose to} someone, or \textbf{propose marriage}  \textbf{to} them, you ask them to marry you.
 \textit{
	\begin{itemize}
	\item He had proposed to Isabel the day after taking his seat in Parliament.
	\end{itemize}
}
\end{enumerate}

\section*{magnify}
{\large \color{blue}  magnifies  magnifying  magnified  }
\subsection*{Explain}
\begin{enumerate}
\item verb \\
To \textbf{magnify} an object  means to make it appear larger than it really is, by means of a special lens or mirror .
 \textit{
	\begin{itemize}
	\item This version of the Digges telescope magnifies images 11 times.
	\item A lens would magnify the picture so it would be like looking at a large TV screen.
	\item ...magnifying lenses.
	\end{itemize}
}
\item verb \\
To \textbf{magnify} something means to increase its effect , size, loudness, or intensity .
 \textit{
	\begin{itemize}
	\item Poverty and human folly magnify natural disasters.
	\item Their noises were magnified in the still, wet air.
	\item ...using bank loans to magnify his buying power.
	\end{itemize}
}
\item verb \\
If you \textbf{magnify} something, you make it seem more important or serious than it really is.
 \textit{
	\begin{itemize}
	\item They do not grasp the broad situation and spend their time magnifying ridiculous
details.
	\item Any signs of discontent tend to be magnified and overanalyzed.
	\end{itemize}
}
\end{enumerate}

\section*{quantify}
{\large \color{blue}  quantifies  quantifying  quantified  }
\subsection*{Explain}
\begin{enumerate}
\item verb \\
If you try to \textbf{quantify} something, you try to calculate how much of it there is.
 \textit{
	\begin{itemize}
	\item It is difficult to quantify an exact figure as firms are reluctant to declare their
losses.
	\end{itemize}
}
\end{enumerate}

\section*{massacre}
{\large \color{blue}  massacres  massacring  massacred  }
\subsection*{Explain}
\begin{enumerate}
\item variable noun \\
A \textbf{massacre} is the killing of a large number of people at the same time in a violent and cruel  way .
 \textit{
	\begin{itemize}
	\item Maria lost her 62-year-old mother in the massacre.
	\item ...reports of massacre, torture and starvation.
	\end{itemize}
}
\item verb \\
If people \textbf{are massacred} , a large number of them are attacked and killed in a violent and cruel way.
 \textit{
	\begin{itemize}
	\item 300 civilians are believed to have been massacred by the rebels.
	\item Troops indiscriminately massacred the defenceless population.
	\end{itemize}
}
\end{enumerate}

\section*{safeguard}
{\large \color{blue}  safeguards  safeguarding  safeguarded  }
\subsection*{Explain}
\begin{enumerate}
\item verb \\
To \textbf{safeguard} something or someone means to protect them from being harmed , lost , or badly treated.
 \textit{
	\begin{itemize}
	\item They will press for international action to safeguard the ozone layer.
	\item The interests of minorities will have to be safeguarded under a new constitution.
	\item ...new guidelines to safeguard bill payers from future price rises.
	\end{itemize}
}
\item countable noun \\
A \textbf{safeguard} is a law, rule, or measure intended to prevent someone or something from being harmed.
 \textit{
	\begin{itemize}
	\item Many people took second jobs as a safeguard against unemployment.
	\item A system like ours lacks adequate safeguards for civil liberties.
	\end{itemize}
}
\end{enumerate}

\section*{motivate}
{\large \color{blue}  motivates  motivating  motivated  }
\subsection*{Explain}
\begin{enumerate}
\item verb \\
If you \textbf{are motivated} by something, especially an emotion , it causes you to behave in a particular way.
 \textit{
	\begin{itemize}
	\item They are motivated by a need to achieve.
	\item The crime was not politically motivated.
	\item I don't want to be missing out. And that motivates me to get up and do something
every day.
	\end{itemize}
}
\item verb \\
If someone \textbf{motivates} you to do something, they make you feel  determined to do it.
 \textit{
	\begin{itemize}
	\item How do you motivate people to work hard and efficiently?
	\item Never let it be said that the manager doesn't know how to motivate his players.
	\end{itemize}
}
\end{enumerate}

\section*{scatter}
{\large \color{blue}  scatters  scattering  scattered  }
\subsection*{Explain}
\begin{enumerate}
\item verb \\
If you \textbf{scatter} things over an area, you throw or drop them so that they spread all over the area.
 \textit{
	\begin{itemize}
	\item She tore the rose apart and scattered the petals over the grave.
	\item They've been scattering toys everywhere.
	\item He began by scattering seed and putting in plants.
	\end{itemize}
}
\item verb \\
If a group of people \textbf{scatter} or if you \textbf{scatter} them, they suddenly separate and move in different directions.
 \textit{
	\begin{itemize}
	\item After dinner, everyone scattered.
	\item The cavalry scattered them and chased them off the field.
	\end{itemize}
}
\item singular noun \\
A \textbf{scatter}  \textbf{of} things is a number of them spread over an area in an irregular way.
 \textit{
	\begin{itemize}
	\item On the table was a pile of books and a scatter of papers.
	\end{itemize}
}
\end{enumerate}

\section*{prosecute}
{\large \color{blue}  prosecutes  prosecuting  prosecuted  }
\subsection*{Explain}
\begin{enumerate}
\item verb \\
If the authorities \textbf{prosecute} someone, they charge them with a crime and put them on trial .
 \textit{
	\begin{itemize}
	\item The police have decided not to prosecute because the evidence is not strong enough.
	\item Photographs taken by roadside cameras will soon be enough to prosecute drivers for
speeding.
	\item He is being prosecuted for two criminal offences.
	\end{itemize}
}
\item verb \\
When a lawyer  \textbf{prosecutes} a case , he or she tries to prove that the person who is on trial is guilty .
 \textit{
	\begin{itemize}
	\item The attorney who will prosecute the case says he cannot reveal how much money is
involved.
	\item ...the prosecuting attorney.
	\end{itemize}
}
\end{enumerate}

\section*{shear}
{\large \color{blue}  shears  shearing  sheared  shorn  }
\subsection*{Explain}
\begin{enumerate}
\item verb \\
To \textbf{shear} a sheep means to cut its wool off.
 \textit{
	\begin{itemize}
	\item In the Hebrides they shear their sheep later than anywhere else.
	\end{itemize}
}
\item plural noun \\
A pair of \textbf{shears} is a garden  tool  like a very large pair of scissors. Shears are used especially for cutting hedges .
 \textit{
	\begin{itemize}
	\item Trim the shrubs with shears.
	\end{itemize}
}
\end{enumerate}

\section*{put}
{\large \color{blue}  puts  putting  }
\subsection*{Explain}
\begin{enumerate}
\item verb \\
When you \textbf{put} something in a particular place or position, you move it into that place or position.
 \textit{
	\begin{itemize}
	\item Leaphorn put the photograph on the desk.
	\item She hesitated, then put her hand on Grace's arm.
	\item Mishka put down a heavy shopping bag.
	\end{itemize}
}
\item verb \\
If you \textbf{put} someone somewhere , you cause them to go there and to stay there for a period of time.
 \textit{
	\begin{itemize}
	\item Rather than put him in the hospital, she had been caring for him at home.
	\item I'd put the children to bed.
	\end{itemize}
}
\item verb \\
To \textbf{put} someone or something in a particular state or situation means to cause them to be in that state or situation.
 \textit{
	\begin{itemize}
	\item This is going to put them out of business.
	\item He was putting himself at risk.
	\item My doctor put me in touch with a psychiatrist.
	\item The British people put us back in power.
	\end{itemize}
}
\item verb \\
To \textbf{put} something \textbf{on} people or things means to cause them to have it, or to cause them to be affected
by it.
 \textit{
	\begin{itemize}
	\item The ruling will put extra pressure on health authorities.
	\item Be aware of the terrible strain it can put on a child when you expect the best reports.
	\item They will also force schools to put more emphasis on teaching basic subjects.
	\end{itemize}
}
\item verb \\
If you \textbf{put} your trust , faith , or confidence  \textbf{in} someone or something, you trust them or have faith or confidence in them.
 \textit{
	\begin{itemize}
	\item He had decided long ago that he would put his trust in socialism when the time came.
	\item How much faith should we put in anti-ageing products?
	\end{itemize}
}
\item verb \\
If you \textbf{put} time, strength , or energy  \textbf{into} an activity, you use it in doing that activity.
 \textit{
	\begin{itemize}
	\item We're not saying that activists should put all their effort and time into party politics.
	\item Eleanor did not put much energy into the discussion.
	\end{itemize}
}
\item verb \\
If you \textbf{put} money \textbf{into} a business or project , you invest money in it.
 \textit{
	\begin{itemize}
	\item Investors should consider putting some money into an annuity.
	\item Put $10,000 into this investment and in 10 years, you'll have almost $18,000.
	\end{itemize}
}
\item verb \\
When you \textbf{put} an idea or remark in a particular way, you express it in that way. You can use expressions like \textbf{to put it simply} and \textbf{to put it bluntly} before saying something when you want to explain how you are going to express it.
 \textit{
	\begin{itemize}
	\item I had already met him a couple of times through–how should I put it–friends in low
places.
	\item He doesn't, to put it very bluntly, give a damn about the woman or the baby.
	\item If I was auditioning for a vocalist, let me put it this way, he wouldn't get to sing
in my band.
	\item He admitted the security forces might have made some mistakes, as he put it.
	\item You can't put that sort of fear into words.
	\end{itemize}
}
\item verb \\
When you \textbf{put a question}  \textbf{to} someone, you ask them the question.
 \textit{
	\begin{itemize}
	\item Is this fair? Well, I put that question today to the deputy counsel.
	\item He thinks that some workers may be afraid to put questions publicly.
	\end{itemize}
}
\item verb \\
If you \textbf{put} a case , opinion , or proposal , you explain it and list the reasons why you support or believe it.
 \textit{
	\begin{itemize}
	\item He always put his point of view with clarity and with courage.
	\item He put the case to the Saudi Foreign Minister.
	\item He sat there listening as we put suggestions to him.
	\end{itemize}
}
\item verb \\
If you \textbf{put} something \textbf{at} a particular value or \textbf{in} a particular category , you consider that it has that value or that it belongs in that category.
 \textit{
	\begin{itemize}
	\item I would put her age at about 50 or so.
	\item All the more technically advanced countries put a high value on science.
	\item It is not easy to put the guilty and innocent into clear-cut categories.
	\end{itemize}
}
\item verb \\
If you \textbf{put} written information somewhere, you write, type, or print it there.
 \textit{
	\begin{itemize}
	\item Mary's family were so pleased that they put an announcement in the local paper to
thank them.
	\item I think what I put in that book is now pretty much the agenda for this country.
	\item He crossed out 'Screenplay' and put 'Written by' instead.
	\end{itemize}
}
\item  \\
 to put one over on sb \textit{
	\begin{itemize}
	\end{itemize}
}
\item  \\
 to put it to sb that \textit{
	\begin{itemize}
	\end{itemize}
}
\item  \\
 put together \textit{
	\begin{itemize}
	\end{itemize}
}
\item  \\
 put it there \textit{
	\begin{itemize}
	\end{itemize}
}
\end{enumerate}

\section*{shift}
{\large \color{blue}  shifts  shifting  shifted  }
\subsection*{Explain}
\begin{enumerate}
\item verb \\
If you \textbf{shift} something or if it \textbf{shifts} , it moves slightly .
 \textit{
	\begin{itemize}
	\item He stopped, shifting his cane to his left hand.
	\item He shifted from foot to foot.
	\item The entire pile shifted and slid, thumping onto the floor.
	\item ...the squeak of his boots in the snow as he shifted his weight.
	\end{itemize}
}
\item verb \\
If someone's opinion , a situation , or a policy  \textbf{shifts} or \textbf{is shifted} , it changes slightly.
 \textbf{Shift} is also a noun .
 \textit{
	\begin{itemize}
	\item Attitudes to mental illness have shifted in recent years.
	\item The emphasis should be shifted more towards Parliament.
	\item ...a shift in government policy.
	\item ...the shift in opinion away from the Prime Minister.
	\end{itemize}
}
\item verb \\
If someone \textbf{shifts} the responsibility or blame for something onto you, they unfairly make you responsible or make people blame you for it, instead of them.
 \textit{
	\begin{itemize}
	\item It was a vain attempt to shift the responsibility for the murder to somebody else.
	\end{itemize}
}
\item verb \\
If a shop or company  \textbf{shifts}  goods , they sell goods that are difficult to sell.
 \textit{
	\begin{itemize}
	\item Some suppliers were selling at a loss to shift stock.
	\end{itemize}
}
\item verb \\
If you \textbf{shift} gears in a car , you put the car into a different gear.
 \textit{
	\begin{itemize}
	\end{itemize}
}
\item countable noun \\
If a group of factory workers, nurses , or other people work \textbf{shifts} , they work for a set period before being replaced by another group, so that there is always a group working . Each of these set periods is called a \textbf{shift} . You can also use \textbf{shift} to refer to a group of workers who work together on a particular shift.
 \textit{
	\begin{itemize}
	\item His father worked shifts in a steel mill.
	\item ...workers coming home from the afternoon shift.
	\item The night shift should have been safely down the mine long ago.
	\end{itemize}
}
\end{enumerate}

\section*{retrieve}
{\large \color{blue}  retrieves  retrieving  retrieved  }
\subsection*{Explain}
\begin{enumerate}
\item verb \\
If you \textbf{retrieve} something, you get it back from the place where you left it.
 \textit{
	\begin{itemize}
	\item He reached over and retrieved his jacket from the back seat.
	\item The men were trying to retrieve weapons left when the army abandoned the island.
	\end{itemize}
}
\item verb \\
If you manage to \textbf{retrieve} a situation , you succeed in bringing it back into a more acceptable state.
 \textit{
	\begin{itemize}
	\item He is the one man who could retrieve that situation.
	\end{itemize}
}
\item verb \\
To \textbf{retrieve} information from a computer or from your memory  means to get it back.
 \textit{
	\begin{itemize}
	\item Computers can instantly retrieve millions of information bits.
	\item As children older, their strategies for storing and retrieving information improve.
	\end{itemize}
}
\end{enumerate}

\section*{spin}
{\large \color{blue}  spins  spinning  spun  }
\subsection*{Explain}
\begin{enumerate}
\item verb \\
If something \textbf{spins} or if you \textbf{spin} it, it turns quickly around a central point.
 \textbf{Spin} is also a noun .
 \textit{
	\begin{itemize}
	\item The latest discs, used for small portable computers, spin 3600 times a minute.
	\item The Earth spins on its own axis.
	\item He spun the wheel sharply and made a U turn in the middle of the road.
	\item He spun his car round and went after them.
	\item This driving mode allows you to move off in third gear to reduce wheel-spin in icy
conditions.
	\end{itemize}
}
\item verb \\
When you \textbf{spin}  washing , it is turned round and round quickly in a spin drier or a washing machine to get the water out.
 \textbf{Spin} is also a noun.
 \textit{
	\begin{itemize}
	\item Just spin the washing and it's nearly dry.
	\item Set on a cool wash and finish with a short spin.
	\end{itemize}
}
\item verb \\
If your head \textbf{is spinning} , you feel  unsteady or confused , for example because you are drunk, ill , or excited .
 \textit{
	\begin{itemize}
	\item His head was spinning and he could barely stand.
	\item All those figures make my poor head spin.
	\end{itemize}
}
\item singular noun \\
If someone puts a certain \textbf{spin} on an event or situation, they interpret it and try to present it in a particular way.
 \textit{
	\begin{itemize}
	\item He interpreted the vote as support and that is the spin his supporters are putting
on the results today.
	\item ...the wholly improper political spin given to the report by sections of the press.
	\end{itemize}
}
\item uncountable noun \\
In politics , \textbf{spin} is the way in which political parties try to present everything they do in a positive way to the public and the media.
 \textit{
	\begin{itemize}
	\item The public is sick of spin and tired of promises. It's time for politicians to act.
	\end{itemize}
}
\item singular noun \\
If you go for \textbf{a spin} or take a car for \textbf{a spin} , you make a short journey in a car just to enjoy yourself.
 \textit{
	\begin{itemize}
	\end{itemize}
}
\item verb \\
If someone \textbf{spins} a story, they give you an account of something that is untrue or only partly true.
 \textit{
	\begin{itemize}
	\item She had spun a story which was too good to be condemned as a simple lie.
	\end{itemize}
}
\item verb \\
When people \textbf{spin} , they make thread by twisting together pieces of a fibre such as wool or cotton using a device or machine.
 \textit{
	\begin{itemize}
	\item Michelle will also spin a customer's wool fleece to specification at a cost of $2.25
an ounce.
	\end{itemize}
}
\item singular noun \\
If a plane goes into \textbf{a spin} , it falls very rapidly towards the ground in a spiral movement.
 \textit{
	\begin{itemize}
	\end{itemize}
}
\item uncountable noun \\
In a game such as tennis or cricket , if you put \textbf{spin} on a ball, you deliberately make it spin rapidly when you hit it or throw it.
 \textit{
	\begin{itemize}
	\end{itemize}
}
\item  \\
 in a spin \textit{
	\begin{itemize}
	\end{itemize}
}
\end{enumerate}

\section*{saturate}
{\large \color{blue}  saturates  saturating  saturated  }
\subsection*{Explain}
\begin{enumerate}
\item verb \\
If people or things \textbf{saturate} a place or object, they fill it completely so that no more can be added .
 \textit{
	\begin{itemize}
	\item In the last days before the vote, both sides are saturating the airwaves.
	\item As the market was saturated with goods and the economy became more balanced, inflation
went down.
	\end{itemize}
}
\item verb \\
If someone or something \textbf{is saturated} , they become extremely  wet .
 \textit{
	\begin{itemize}
	\item If the filter has been saturated with motor oil, it should be discarded and replaced.
	\end{itemize}
}
\end{enumerate}

\section*{split}
{\large \color{blue}  splits  splitting  }
\subsection*{Explain}
\begin{enumerate}
\item verb \\
If something \textbf{splits} or if you \textbf{split} it, it is divided into two or more parts.
 \textit{
	\begin{itemize}
	\item In a severe gale the ship split in two.
	\item If the chicken is fairly small, you may simply split it in half.
	\item We split the boards down the middle to use them for the back of the shelves.
	\item ...uniting families split by the war.
	\end{itemize}
}
\item verb \\
If an organization \textbf{splits} or \textbf{is split} , one group of members disagrees strongly with the other members, and may form a group of their own.
 \textbf{Split} is also an adjective .
 \textit{
	\begin{itemize}
	\item Yet it is feared the Republican leadership could split over the agreement.
	\item A leadership contest now would split the party.
	\item These organizations are really split by personal rivalries as much as by politics.
	\item The Kremlin is deeply split in its approach to foreign policy.
	\end{itemize}
}
\item countable noun \\
A \textbf{split}  \textbf{in} an organization is a disagreement between its members.
 \textit{
	\begin{itemize}
	\item They accused both radicals and conservatives of trying to provoke a split in the
party.
	\end{itemize}
}
\item singular noun \\
A \textbf{split}  \textbf{between} two things is a division or difference between them.
 \textit{
	\begin{itemize}
	\item ...a split between what is thought and what is felt.
	\end{itemize}
}
\item verb \\
If something such as wood or a piece of clothing \textbf{splits} or \textbf{is split} , a long crack or tear appears in it.
 \textit{
	\begin{itemize}
	\item The seat of his short grey trousers split.
	\item Twist the mixture into individual sausages without splitting the skins.
	\item He had a split lip and an eye that wouldn't open properly.
	\end{itemize}
}
\item countable noun \\
A \textbf{split} is a long crack or tear.
 \textit{
	\begin{itemize}
	\item The plastic-covered seat has a few small splits around the corners.
	\end{itemize}
}
\item verb \\
If two or more people \textbf{split} something, they share it between them.
 \textit{
	\begin{itemize}
	\item I would rather pay for a meal than watch nine friends pick over and split a bill.
	\item Split the wages between you.
	\item All exhibits are for sale, the proceeds being split between a charity and the artist.
	\end{itemize}
}
\end{enumerate}

\section*{see}
{\large \color{blue}  sees  seeing  saw  seen  }
\subsection*{Explain}
\begin{enumerate}
\item verb \\
When you \textbf{see} something, you notice it using your eyes.
 \textit{
	\begin{itemize}
	\item You can't see colours at night.
	\item I saw a man making his way towards me.
	\item She can see, hear, touch, smell, and taste.
	\item As he neared the farm, he saw that a police car was parked outside it.
	\item Did you see what happened?
	\end{itemize}
}
\item verb \\
If you \textbf{see} someone, you visit them or meet them.
 \textit{
	\begin{itemize}
	\item I saw him yesterday.
	\item Mick wants to see you in his office right away.
	\item You need to see a doctor.
	\end{itemize}
}
\item verb \\
If you \textbf{see} an entertainment such as a play, film, concert , or sports game, you watch it.
 \textit{
	\begin{itemize}
	\item He had been to see a Semi-Final of the FA Cup.
	\item It was one of the most amazing films I've ever seen.
	\end{itemize}
}
\item verb \\
If you \textbf{see} that something is true or exists, you realize by observing it that it is true or exists.
 \textit{
	\begin{itemize}
	\item I could see she was lonely.
	\item A lot of people saw what was happening but did nothing about it.
	\item You see young people going to school inadequately dressed for the weather.
	\item My taste has changed a bit over the years as you can see.
	\item You've just been cleaning it, I see.
	\item The army must be seen to be taking firm action.
	\end{itemize}
}
\item verb \\
If you \textbf{see} what someone means or \textbf{see} why something happened , you understand what they mean or understand why it happened.
 \textit{
	\begin{itemize}
	\item Oh, I see what you're saying.
	\item I don't see why you're complaining.
	\item I really don't see any reason for changing it.
	\item Now I see that I was wrong.
	\end{itemize}
}
\item verb \\
If you \textbf{see} someone or something \textbf{as} a certain thing, you have the opinion that they are that thing.
 \textit{
	\begin{itemize}
	\item She saw him as a visionary, but her father saw him as a man who couldn't make a living.
	\item They have a normal body weight but see themselves as being fat.
	\item Others saw it as a betrayal.
	\item I don't see it as my duty to take sides.
	\item As I see it, Llewelyn has three choices open to him.
	\item Some men are seen to be less effective as managers.
	\end{itemize}
}
\item verb \\
If you \textbf{see} a particular quality \textbf{in} someone, you believe they have that quality. If you ask what someone \textbf{sees}  \textbf{in} a particular person or thing, you want to know what they find attractive about that person or thing.
 \textit{
	\begin{itemize}
	\item Frankly, I don't know what Paul sees in her.
	\item Young and old saw in him an implacable opponent of apartheid.
	\end{itemize}
}
\item verb \\
If you \textbf{see} something happening in the future , you imagine it, or predict that it will happen.
 \textit{
	\begin{itemize}
	\item A good idea, but can you see Taylor trying it?
	\item We can see a day where all people live side by side.
	\end{itemize}
}
\item verb \\
If a period of time or a person \textbf{sees} a particular change or event, it takes place during that period of time or while
that person is alive .
 \textit{
	\begin{itemize}
	\item Yesterday saw the resignation of the acting Interior Minister.
	\item He had worked with the consultant for three years and was sorry to see him go.
	\item Mr Frank has seen the economy of his town slashed by the uprising.
	\end{itemize}
}
\item verb \\
You can use \textbf{see} in expressions to do with finding out information. For example, if you say ' \textbf{I'll see what's happening} ', you mean that you intend to find out what is happening.
 \textit{
	\begin{itemize}
	\item Let me just see what the next song is.
	\item Shake him gently to see if he responds.
	\end{itemize}
}
\item verb \\
You can use \textbf{see} to promise to try and help someone. For example, if you say ' \textbf{I'll see if I can do it} ', you mean that you will try to do the thing concerned.
 \textit{
	\begin{itemize}
	\item I'll see if I can call her for you.
	\item We'll see what we can do, miss.
	\end{itemize}
}
\item verb \\
If you \textbf{see}  \textbf{that} something is done or if you \textbf{see}  \textbf{to it that} it is done, you make sure that it is done.
 \textit{
	\begin{itemize}
	\item See that you take care of him.
	\item Catherine saw to it that the information went directly to Walter.
	\end{itemize}
}
\item verb \\
If you \textbf{see} someone to a particular place, you accompany them to make sure that they get there safely, or to show politeness.
 \textit{
	\begin{itemize}
	\item He didn't offer to see her to her car.
	\item 'Goodnight.'—'I'll see you out.'
	\end{itemize}
}
\item verb \\
If you \textbf{see} a lot  \textbf{of} someone, you often meet each other or visit each other.
 \textit{
	\begin{itemize}
	\item We used to see quite a lot of his wife, Carolyn.
	\item We didn't see much of each other after that because he was touring.
	\end{itemize}
}
\item verb \\
If you \textbf{are seeing} someone, you spend time with them socially, and are having a romantic or sexual relationship .
 \textit{
	\begin{itemize}
	\item I was seeing her but I wasn't her committed boyfriend.
	\end{itemize}
}
\item verb \\
Some writers use \textbf{see} in expressions such as \textbf{we saw} and \textbf{as we have seen} to refer to something that has already been explained or described .
 \textit{
	\begin{itemize}
	\item We saw in Chapter 16 how annual cash budgets are produced.
	\item Using the figures given above, it can be seen that machine A pays back the initial
investment in two years.
	\item As we have seen in previous chapters, visualization methods are varied.
	\end{itemize}
}
\item verb \\
\textbf{See} is used in books to indicate to readers that they should look at another part of the book, or at another book, because more
information is given there.
 \textit{
	\begin{itemize}
	\item Surveys consistently find that men report feeling safe on the street after dark.
See, for example, Hindelang and Garofalo (1978).
	\item See Chapter 7 below for further comments on the textile industry.
	\end{itemize}
}
\item  \\
 seeing as/that \textit{
	\begin{itemize}
	\end{itemize}
}
\item  \\
 I see \textit{
	\begin{itemize}
	\end{itemize}
}
\item  \\
 I'll/we'll see \textit{
	\begin{itemize}
	\end{itemize}
}
\item  \\
 let me/let's see \textit{
	\begin{itemize}
	\end{itemize}
}
\item  \\
 to see sense \textit{
	\begin{itemize}
	\end{itemize}
}
\item  \\
 you see \textit{
	\begin{itemize}
	\end{itemize}
}
\item  \\
 see you \textit{
	\begin{itemize}
	\end{itemize}
}
\item  \\
 you'll see \textit{
	\begin{itemize}
	\end{itemize}
}
\end{enumerate}

\section*{startle}
{\large \color{blue}  startles  startling  startled  }
\subsection*{Explain}
\begin{enumerate}
\item verb \\
If something sudden and unexpected  \textbf{startles} you, it surprises and frightens you slightly .
 \textit{
	\begin{itemize}
	\item The telephone startled him.
	\item Sorry, I didn't mean to startle you.
	\item The news will startle the City.
	\end{itemize}
}
\end{enumerate}

\section*{steer}
{\large \color{blue}  steers  steering  steered  }
\subsection*{Explain}
\begin{enumerate}
\item verb \\
When you \textbf{steer} a car , boat , or plane , you control it so that it goes in the direction that you want .
 \textit{
	\begin{itemize}
	\item What is it like to steer a ship this size?
	\item When I was a kid, about six or seven, she would often let me steer the car along
our driveway.
	\end{itemize}
}
\item verb \\
If you \textbf{steer} people towards a particular course of action or attitude , you try to lead them gently in that direction.
 \textit{
	\begin{itemize}
	\item The new government is seen as one that will steer the country in the right direction.
	\item You are trying to steer your mother towards increased independence.
	\end{itemize}
}
\item verb \\
If you \textbf{steer} someone in a particular direction, you guide them there.
 \textit{
	\begin{itemize}
	\item Nick steered them into the nearest seats.
	\end{itemize}
}
\item verb \\
If you \textbf{steer} a particular \textbf{course} , you take a particular line of action.
 \textit{
	\begin{itemize}
	\item The Prime Minister has sought to steer a course between the two groups.
	\item In nearly all these issues the British steered a middle course.
	\end{itemize}
}
\item countable noun \\
A \textbf{steer} is a bull that has been castrated.
 \textit{
	\begin{itemize}
	\end{itemize}
}
\item  \\
 steer clear of sb/sth \textit{
	\begin{itemize}
	\end{itemize}
}
\end{enumerate}

\section*{stir}
{\large \color{blue}  stirs  stirring  stirred  }
\subsection*{Explain}
\begin{enumerate}
\item verb \\
If you \textbf{stir} a liquid or other substance, you move it around or mix it in a container using something such as a spoon.
 \textit{
	\begin{itemize}
	\item Stir the soup for a few seconds.
	\item There was Mrs Bellingham, stirring sugar into her tea.
	\item You don't add the peanut butter until after you've stirred in the honey.
	\end{itemize}
}
\item verb \\
If you \textbf{stir} , you move slightly , for example because you are uncomfortable or beginning to wake up.
 \textit{
	\begin{itemize}
	\item Eileen shook him, and he started to stir.
	\item The two women lay on their backs, not stirring.
	\end{itemize}
}
\item verb \\
If you do not \textbf{stir}  \textbf{from} a place, you do not move from it.
 \textit{
	\begin{itemize}
	\item She had not stirred from the house that evening.
	\item There's something you could study without stirring from this room.
	\end{itemize}
}
\item verb \\
If something \textbf{stirs} or if the wind  \textbf{stirs} it, it moves gently in the wind.
 \textit{
	\begin{itemize}
	\item Palm trees stir in the soft Pacific breeze.
	\item Not a breath of fresh air stirred the long white curtains.
	\end{itemize}
}
\item verb \\
If you \textbf{stir}  \textbf{yourself} , or if something \textbf{stirs} you \textbf{into} action, you move in order to start doing something.
 \textit{
	\begin{itemize}
	\item Stir yourself! We've got a visitor.
	\item You can't even stir yourself to have a drink with them.
	\item The sight of them stirred him into action.
	\end{itemize}
}
\item verb \\
If something \textbf{stirs} you, it makes you react with a strong emotion .
 \textit{
	\begin{itemize}
	\item The voice, less coarse now, stirred her as it had then.
	\item I was intrigued by him, stirred by his intellect.
	\end{itemize}
}
\item verb \\
If a particular memory , feeling , or mood  \textbf{stirs} or \textbf{is stirred}  \textbf{in} you, you begin to think about it or feel it.
 \textit{
	\begin{itemize}
	\item Then a memory stirs in you and you start feeling anxious.
	\item Amy remembered the anger he had stirred in her.
	\item Deep inside the awareness was stirring that something was about to happen.
	\end{itemize}
}
\item singular noun \\
If an event causes a \textbf{stir} , it causes great excitement, shock , or anger among people.
 \textit{
	\begin{itemize}
	\item His film has caused a stir in America.
	\end{itemize}
}
\item  \\
 shaken but not stirred \textit{
	\begin{itemize}
	\end{itemize}
}
\end{enumerate}

\section*{study}
{\large \color{blue}  studies  studying  studied  }
\subsection*{Explain}
\begin{enumerate}
\item verb \\
If you \textbf{study} , you spend time learning about a particular subject or subjects.
 \textit{
	\begin{itemize}
	\item ...a relaxed and happy atmosphere that will allow you to study to your full potential.
	\item He went to Hull University, where he studied History and Economics.
	\item The rehearsals make it difficult for her to study for law school exams.
	\end{itemize}
}
\item uncountable noun \\
\textbf{Study} is the activity of studying.
 \textit{
	\begin{itemize}
	\item ...the use of maps and visual evidence in the study of local history.
	\item She gave up her studies to take a job with the company.
	\end{itemize}
}
\item countable noun \\
A \textbf{study} of a subject is a piece of research on it.
 \textit{
	\begin{itemize}
	\item Recent studies suggest that as many as 5 in 1000 new mothers are likely to have this
problem.
	\item ...the first study of English children's attitudes.
	\end{itemize}
}
\item plural noun \\
You can refer to educational subjects or courses that contain several elements as \textbf{studies} of a particular kind .
 \textit{
	\begin{itemize}
	\item ...a new centre for Islamic studies.
	\item She is currently doing a business studies course at Leeds.
	\end{itemize}
}
\item verb \\
If you \textbf{study} something, you look at it or watch it very carefully, in order to find something out.
 \textit{
	\begin{itemize}
	\item Debbie studied her friend's face for a moment.
	\end{itemize}
}
\item verb \\
If you \textbf{study} something, you consider it or observe it carefully in order to be able to understand it fully .
 \textit{
	\begin{itemize}
	\item I know that you've been studying chimpanzees for thirty years now.
	\item I invite every citizen to carefully study the document.
	\end{itemize}
}
\item countable noun \\
A \textbf{study} by an artist is a drawing which is done in preparation for a larger picture .
 \textit{
	\begin{itemize}
	\end{itemize}
}
\item countable noun \\
A \textbf{study} is a room in a house which is used for reading, writing, and studying.
 \textit{
	\begin{itemize}
	\end{itemize}
}
\end{enumerate}

\section*{tend}
{\large \color{blue}  tends  tending  tended  }
\subsection*{Explain}
\begin{enumerate}
\item verb \\
If something \textbf{tends}  \textbf{to}  happen , it usually happens or it often happens.
 \textit{
	\begin{itemize}
	\item A problem for manufacturers is that lighter cars tend to be noisy.
	\item In older age groups women predominate because men tend to die younger.
	\item They tend to buy cheap processed foods like canned chicken and macaroni.
	\end{itemize}
}
\item verb \\
If you \textbf{tend}  \textbf{towards} a particular characteristic, you often display that characteristic.
 \textit{
	\begin{itemize}
	\item Artistic and intellectual people allegedly tend towards left-wing views.
	\end{itemize}
}
\item verb \\
You can say that you \textbf{tend}  \textbf{to}  think something when you want to give your opinion , but do not want it to seem too forceful or definite .
 \textit{
	\begin{itemize}
	\item I tend to think that Members of Parliament by and large do a good job.
	\end{itemize}
}
\item verb \\
If you \textbf{tend} someone or something, you do what is necessary to keep them in a good condition or to improve their condition.
 \textit{
	\begin{itemize}
	\item For years he tended her in her painful illness.
	\item He tends the flower beds and evergreens that he has planted in the driveway.
	\end{itemize}
}
\item verb \\
If you \textbf{tend to} someone or something, you pay attention to them and deal with their problems and needs .
 \textit{
	\begin{itemize}
	\item In our culture, girls are brought up to tend to the needs of others.
	\item She hurried away to pour more coffee and tend to the grill.
	\end{itemize}
}
\end{enumerate}

\section*{trust}
{\large \color{blue}  trusts  trusting  trusted  }
\subsection*{Explain}
\begin{enumerate}
\item verb \\
If you \textbf{trust} someone, you believe that they are honest and sincere and will not deliberately do anything to harm you.
 \textit{
	\begin{itemize}
	\item 'I trust you completely,' he said.
	\item He did argue in a general way that the president can't be trusted.
	\end{itemize}
}
\item uncountable noun \\
Your \textbf{trust}  \textbf{in} someone is your belief that they are honest and sincere and will not deliberately
do anything to harm you.
 \textit{
	\begin{itemize}
	\item He destroyed me and my trust in men.
	\item You've betrayed their trust.
	\item There's a feeling of warmth and trust here.
	\end{itemize}
}
\item verb \\
If you \textbf{trust} someone \textbf{to} do something, you believe that they will do it.
 \textit{
	\begin{itemize}
	\item That's why I must trust you to keep this secret.
	\item They argued that the ruling party could not be trusted to oversee its own removal
from power.
	\end{itemize}
}
\item verb \\
If you \textbf{trust} someone \textbf{with} something important or valuable , you allow them to look after it or deal with it.
 \textbf{Trust} is also a noun .
 \textit{
	\begin{itemize}
	\item This could make your superiors hesitate to trust you with major responsibilities.
	\item I'd trust him with my life.
	\item ...a care home where you were working in a position of trust.
	\item Although I didn't betray a trust, I feel I behaved shabbily.
	\end{itemize}
}
\item verb \\
If you do not \textbf{trust} something, you feel that it is not safe or reliable .
 \textit{
	\begin{itemize}
	\item She nodded, not trusting her own voice.
	\item For one thing, he didn't trust his legs to hold him up.
	\item I still can't trust myself to remain composed in their presence.
	\end{itemize}
}
\item verb \\
If you \textbf{trust} someone's judgment or advice , you believe that it is good or right.
 \textit{
	\begin{itemize}
	\item Jake has raised two smashing kids and I trust his judgement.
	\item I blame myself and will never be able to trust my instinct again.
	\end{itemize}
}
\item verb \\
If you say you \textbf{trust that} something is true , you mean you hope and expect that it is true.
 \textit{
	\begin{itemize}
	\item I trust you will take the earliest opportunity to make a full apology.
	\item We trust that he and his department are considering our suggestion.
	\end{itemize}
}
\item verb \\
If you \textbf{trust in} someone or something, you believe strongly in them, and do not doubt their powers or their good intentions .
 \textit{
	\begin{itemize}
	\item He was a pastor who trusted in the Lord and who lived to preach.
	\item Don't blindly trust in the good faith of any government official.
	\end{itemize}
}
\item countable noun \\
A \textbf{trust} is a financial arrangement in which a group of people or an organization keeps and invests money for someone.
 \textit{
	\begin{itemize}
	\item You could set up a trust so the children can't spend their inheritance.
	\item The money will be put in trust until she is 18.
	\end{itemize}
}
\item countable noun \\
A \textbf{trust} is a group of people or an organization that has control of an amount of money or
property and invests it on behalf of other people or as a charity .
 \textit{
	\begin{itemize}
	\item He had set up two charitable trusts.
	\item Over the past 18 months, the trust has opened four more cafés in the area.
	\end{itemize}
}
\item countable noun \\
In Britain, a \textbf{trust} or a \textbf{trust hospital} is a public hospital that receives its funding  directly from the national government. It has its own board of governors and is not controlled by the local health authority.
 \textit{
	\begin{itemize}
	\item The hospital became a self-governing trust this week.
	\end{itemize}
}
\item  \\
 in trust \textit{
	\begin{itemize}
	\end{itemize}
}
\item  \\
 take sth on trust \textit{
	\begin{itemize}
	\end{itemize}
}
\end{enumerate}

\section*{uncover}
{\large \color{blue}  uncovers  uncovering  uncovered  }
\subsection*{Explain}
\begin{enumerate}
\item verb \\
If you \textbf{uncover} something, especially something that has been kept secret , you discover or find out about it.
 \textit{
	\begin{itemize}
	\item Auditors said they had uncovered evidence of fraud.
	\item A specific plot to kill him was uncovered in the past couple of weeks.
	\end{itemize}
}
\item verb \\
When people who are digging  somewhere  \textbf{uncover} something, they find a thing or a place that has been under the ground for a long
time.
 \textit{
	\begin{itemize}
	\item Archaeologists have uncovered an 11,700-year-old hunting camp in Alaska.
	\end{itemize}
}
\item verb \\
To \textbf{uncover} something means to remove something that is covering it.
 \textit{
	\begin{itemize}
	\item When the seedlings sprout, uncover the tray.
	\end{itemize}
}
\end{enumerate}

\section*{ventilate}
{\large \color{blue}  ventilates  ventilating  ventilated  }
\subsection*{Explain}
\begin{enumerate}
\item verb \\
If you \textbf{ventilate} a room or building, you allow  fresh air to get into it.
 \textit{
	\begin{itemize}
	\item Ventilate the room properly when paint stripping.
	\item The pit is ventilated by a steel fan.
	\item ...badly ventilated rooms.
	\end{itemize}
}
\item verb \\
If you \textbf{ventilate} your ideas or feelings , you talk about them or express them freely in front of other people.
 \textit{
	\begin{itemize}
	\item He did not think it the job of officials to ventilate their doubts or daydreams.
	\end{itemize}
}
\end{enumerate}

\section*{upgrade}
{\large \color{blue}  upgrades  upgrading  upgraded  }
\subsection*{Explain}
\begin{enumerate}
\item verb \\
If equipment or services \textbf{are upgraded} , they are improved or made more efficient .
 \textbf{Upgrade} is also a noun .
 \textit{
	\begin{itemize}
	\item Helicopters have been upgraded and modernized.
	\item Medical facilities are being reorganized and upgraded.
	\item ...upgraded catering facilities.
	\item ...equipment which needs expensive upgrades.
	\item ...upgrades in the level of security.
	\end{itemize}
}
\item verb \\
If someone \textbf{is upgraded} , their job or status is changed so that they become more important or receive more money.
 \textit{
	\begin{itemize}
	\item He was upgraded to security guard.
	\end{itemize}
}
\item verb \\
If you \textbf{upgrade} or \textbf{are upgraded} , you change something such as your air ticket or your hotel  room to one that is more expensive .
 \textit{
	\begin{itemize}
	\item You can upgrade from self-catering accommodation to a hotel.
	\end{itemize}
}
\end{enumerate}

\section*{withhold}
{\large \color{blue}  withholds  withholding  withheld  }
\subsection*{Explain}
\begin{enumerate}
\item verb \\
If you \textbf{withhold} something that someone wants , you do not let them have it.
 \textit{
	\begin{itemize}
	\item Police withheld the dead boy's name yesterday until relatives could be told.
	\item Financial aid for Britain has been withheld.
	\item The captain decided to withhold the terrible news even from his officers.
	\end{itemize}
}
\end{enumerate}

\section*{weld}
{\large \color{blue}  welds  welding  welded  }
\subsection*{Explain}
\begin{enumerate}
\item verb \\
To \textbf{weld} one piece of metal to another means to join them by heating the edges and putting them together so that they cool and harden into one piece.
 \textit{
	\begin{itemize}
	\item It's possible to weld stainless steel to ordinary steel.
	\item They will also be used on factory floors to weld things together.
	\item Where did you learn to weld?
	\end{itemize}
}
\item countable noun \\
A \textbf{weld} is a join where two pieces of metal have been welded together.
 \textit{
	\begin{itemize}
	\end{itemize}
}
\item verb \\
If you \textbf{weld} people together, you join them together to form a united organization .
 \textit{
	\begin{itemize}
	\item She has both the authority and the personality to weld the party together.
	\item The miracle was that Rose had welded them into a team.
	\end{itemize}
}
\end{enumerate}

\section*{abandon}
{\large \color{blue}  abandons  abandoning  abandoned  }
\subsection*{Explain}
\begin{enumerate}
\item verb \\
If you \textbf{abandon} a place, thing, or person, you leave the place, thing, or person permanently or for
a long time, especially when you should not do so.
 \textit{
	\begin{itemize}
	\item He claimed that his parents had abandoned him.
	\item The road is strewn with abandoned vehicles.
	\end{itemize}
}
\item verb \\
If you \textbf{abandon} an activity or piece of work, you stop doing it before it is finished .
 \textit{
	\begin{itemize}
	\item The authorities have abandoned any attempt to distribute food.
	\item The scheme's investors, fearful of bankruptcy, decided to abandon the project.
	\end{itemize}
}
\item verb \\
If you \textbf{abandon} an idea or way of thinking , you stop having that idea or thinking in that way.
 \textit{
	\begin{itemize}
	\item Logic had prevailed and he had abandoned the idea.
	\end{itemize}
}
\item verb \\
If you \textbf{abandon}  \textbf{yourself to} an emotion, you think about it a lot and feel it strongly, especially when other people might think you are wrong to do so.
 \textit{
	\begin{itemize}
	\item We are scared to abandon ourselves to our feelings in case we seem weak or out of
control.
	\end{itemize}
}
\item uncountable noun \\
If you say that someone does something \textbf{with}  \textbf{abandon} , you mean that they behave in a wild , uncontrolled way and do not think or care about how they should behave.
 \textit{
	\begin{itemize}
	\item He has spent money with gay abandon.
	\item Their permissiveness toward their children reflects the wild abandon of their own
lives.
	\end{itemize}
}
\item  \\
 abandon ship \textit{
	\begin{itemize}
	\end{itemize}
}
\end{enumerate}

\section*{agitate}
{\large \color{blue}  agitates  agitating  agitated  }
\subsection*{Explain}
\begin{enumerate}
\item verb \\
If people \textbf{agitate}  \textbf{for} something, they protest or take part in political  activity in order to get it.
 \textit{
	\begin{itemize}
	\item The women who worked in these mills had begun to agitate for better conditions.
	\end{itemize}
}
\item verb \\
If you \textbf{agitate} something, you shake it so that it moves about.
 \textit{
	\begin{itemize}
	\item All you need to do is gently agitate the water with a finger or paintbrush.
	\item Its molecules can be agitated by microwave energy.
	\end{itemize}
}
\item verb \\
If something \textbf{agitates} you, it worries you and makes you unable to think  clearly or calmly.
 \textit{
	\begin{itemize}
	\item The thought of them getting her possessions when she dies agitates her.
	\end{itemize}
}
\end{enumerate}

\section*{accept}
{\large \color{blue}  accepts  accepting  accepted  }
\subsection*{Explain}
\begin{enumerate}
\item verb \\
If you \textbf{accept} something that you have been offered, you say yes to it or agree to take it.
 \textit{
	\begin{itemize}
	\item Eventually Sam persuaded her to accept an offer of marriage.
	\item Your old clothes will be gratefully accepted by jumble sale organisers.
	\item All those invited to next week's peace conference have accepted.
	\end{itemize}
}
\item verb \\
If you \textbf{accept} an idea , statement , or fact , you believe that it is true or valid.
 \textit{
	\begin{itemize}
	\item I do not accept that there is any kind of crisis in British science.
	\item I don't think they would accept that view.
	\item He did not accept this reply as valid.
	\item ...a workforce generally accepted to have the best conditions in Europe.
	\end{itemize}
}
\item verb \\
If you \textbf{accept} a plan or an intended action, you agree to it and allow it to happen .
 \textit{
	\begin{itemize}
	\item Accepting the report's proposals would mean a change in church law.
	\item The Council will meet to decide if it should accept his resignation.
	\end{itemize}
}
\item verb \\
If you \textbf{accept} an unpleasant fact or situation , you get used to it or recognize that it is necessary or cannot be changed.
 \textit{
	\begin{itemize}
	\item Some people can accept suffering that can be shown to lead to a greater good.
	\item Urban dwellers often accept noise as part of city life.
	\item I wasn't willing to accept that her leaving was a possibility.
	\end{itemize}
}
\item verb \\
If a person, company, or organization \textbf{accepts} something such as a document, they recognize that it is genuine , correct , or satisfactory and agree to consider it or handle it.
 \textit{
	\begin{itemize}
	\item We advised newspapers not to accept the advertising.
	\item Cheques can only be accepted up to the value guaranteed on the card.
	\item Proof of postage will not be accepted as proof of receipt.
	\end{itemize}
}
\item verb \\
If an organization or person \textbf{accepts} you, you are allowed to join the organization or use the services that are offered.
 \textit{
	\begin{itemize}
	\item All-male groups will not be accepted.
	\item ...incentives to private landlords to accept young people as tenants.
	\end{itemize}
}
\item verb \\
If a person or a group of people \textbf{accepts} you, they begin to be friendly towards you and are happy with who you are or what you do.
 \textit{
	\begin{itemize}
	\item My grandparents have never had a problem accepting me.
	\item Many men still have difficulty accepting a woman as a business partner.
	\item Stephen Smith was accepted into the family like an adopted brother.
	\end{itemize}
}
\item verb \\
If you \textbf{accept} the responsibility or blame for something, you recognize that you are responsible for it.
 \textit{
	\begin{itemize}
	\item The company cannot accept responsibility for loss or damage.
	\end{itemize}
}
\item verb \\
If you \textbf{accept} someone's advice or suggestion , you agree to do what they say.
 \textit{
	\begin{itemize}
	\item The army refused to accept orders from the political leadership.
	\item Don't automatically accept the solicitor recommended by the broker.
	\end{itemize}
}
\item verb \\
If someone's body \textbf{accepts} a transplanted organ, the organ becomes part of the body and starts to function normally .
 \textit{
	\begin{itemize}
	\item ...drugs which will fool the body into accepting transplants.
	\end{itemize}
}
\item verb \\
If a machine \textbf{accepts} a particular kind of thing, it is designed to take it and deal with it or process it.
 \textit{
	\begin{itemize}
	\item The telephone booths accept 10 and 20 pence coins.
	\end{itemize}
}
\end{enumerate}

\section*{bathe}
{\large \color{blue}  bathes  bathing  bathed  }
\subsection*{Explain}
\begin{enumerate}
\item verb \\
If you \textbf{bathe} in a sea, river, or lake , you swim, play , or wash yourself in it. Birds and animals can  also  \textbf{bathe} .
 \textbf{Bathe} is also a noun .
 \textit{
	\begin{itemize}
	\item The police have warned the city's inhabitants not to bathe in the polluted river.
	\item ...small ponds for the birds to bathe in.
	\item Fifty soldiers were taking an early morning bathe in a nearby lake.
	\end{itemize}
}
\item verb \\
When you \textbf{bathe} , you have a bath.
 \textit{
	\begin{itemize}
	\item At least 60% of us now bathe or shower once a day.
	\end{itemize}
}
\item verb \\
If you \textbf{bathe} someone, especially a child , you wash them in a bath.
 \textit{
	\begin{itemize}
	\item Back home, Shirley plays with, feeds and bathes the baby.
	\end{itemize}
}
\item verb \\
If you \textbf{bathe} a part of your body or a wound, you wash it gently or soak it in a liquid.
 \textit{
	\begin{itemize}
	\item Bathe the infected area in a salt solution.
	\item She paused long enough to bathe her blistered feet.
	\end{itemize}
}
\item verb \\
If a place \textbf{is bathed}  \textbf{in}  light , it is covered with light, especially a gentle , pleasant light.
 \textit{
	\begin{itemize}
	\item The arena was bathed in warm sunshine.
	\item I was led to a small room bathed in soft red light.
	\item The lamp behind him seems to bathe him in warmth.
	\end{itemize}
}
\end{enumerate}

\section*{accord}
{\large \color{blue}  accords  according  accorded  }
\subsection*{Explain}
\begin{enumerate}
\item countable noun \\
An \textbf{accord} between countries or groups of people is a formal agreement, for example to end a war.
 \textit{
	\begin{itemize}
	\item ...a fitting way to celebrate the peace accord.
	\end{itemize}
}
\item verb \\
If you \textbf{are accorded} a particular kind of treatment , people act towards you or treat you in that way.
 \textit{
	\begin{itemize}
	\item His predecessor was accorded an equally tumultuous welcome.
	\item The government accorded him the rank of Colonel.
	\item The treatment accorded to a United Nations official was little short of insulting.
	\end{itemize}
}
\item verb \\
If one fact , idea , or condition \textbf{accords with} another, they are in agreement and there is no conflict between them.
 \textit{
	\begin{itemize}
	\item Such an approach accords with the principles of socialist ideology.
	\item ...scientific evidence that did not fully accord with the facts uncovered by the
police.
	\end{itemize}
}
\item  \\
 in accord \textit{
	\begin{itemize}
	\end{itemize}
}
\item  \\
 of it own accord \textit{
	\begin{itemize}
	\end{itemize}
}
\item  \\
 of one's own accord \textit{
	\begin{itemize}
	\end{itemize}
}
\item  \\
 with one accord \textit{
	\begin{itemize}
	\end{itemize}
}
\end{enumerate}

\section*{bet}
{\large \color{blue}  bets  betting  }
\subsection*{Explain}
\begin{enumerate}
\item verb \\
If you \textbf{bet}  \textbf{on} the result of a horse  race , football  game , or other event, you give someone a sum of money which they give you back with extra money if the result is what you predicted, or which they keep if it is not.
 \textbf{Bet} is also a noun .
 \textit{
	\begin{itemize}
	\item Jockeys are forbidden to bet on the outcome of races.
	\item I bet £10 on a horse called Premonition.
	\item He bet them £500 they would lose.
	\item Do you always have a bet on the Grand National?
	\end{itemize}
}
\item countable noun \\
A \textbf{bet} is a sum of money which you give to someone when you bet.
 \textit{
	\begin{itemize}
	\item You can put a bet on almost anything these days.
	\end{itemize}
}
\item verb \\
If someone \textbf{is betting} that something will happen , they are hoping or expecting that it will happen.
 \textit{
	\begin{itemize}
	\item The party is betting that the presidential race will turn into a battle for younger
voters.
	\item People were betting on a further easing of credit conditions.
	\end{itemize}
}
\item  \\
 I bet/I'll bet/you can bet \textit{
	\begin{itemize}
	\end{itemize}
}
\item  \\
 a good bet \textit{
	\begin{itemize}
	\end{itemize}
}
\item  \\
 a good bet/a safe bet \textit{
	\begin{itemize}
	\end{itemize}
}
\item  \\
 hedge your bets \textit{
	\begin{itemize}
	\end{itemize}
}
\item  \\
 I bet/I'll bet \textit{
	\begin{itemize}
	\end{itemize}
}
\item  \\
 my bet is/it's my bet that \textit{
	\begin{itemize}
	\end{itemize}
}
\item  \\
 don't bet on sth/I wouldn't bet on sth \textit{
	\begin{itemize}
	\end{itemize}
}
\item  \\
 do you want to bet?/want a bet? \textit{
	\begin{itemize}
	\end{itemize}
}
\item  \\
 you bet \textit{
	\begin{itemize}
	\end{itemize}
}
\end{enumerate}

\section*{accumulate}
{\large \color{blue}  accumulates  accumulating  accumulated  }
\subsection*{Explain}
\begin{enumerate}
\item verb \\
When you \textbf{accumulate} things or when they \textbf{accumulate} , they collect or are gathered over a period of time.
 \textit{
	\begin{itemize}
	\item Households accumulate wealth across a broad spectrum of assets.
	\item Lead can accumulate in the body until toxic levels are reached.
	\end{itemize}
}
\end{enumerate}

\section*{compel}
{\large \color{blue}  compels  compelling  compelled  }
\subsection*{Explain}
\begin{enumerate}
\item verb \\
If a situation , a rule, or a person \textbf{compels} you \textbf{to} do something, they force you to do it.
 \textit{
	\begin{itemize}
	\item ...the introduction of legislation to compel cyclists to wear a helmet.
	\item Leonie's mother was compelled to take in washing to help support her family.
	\item Drivers are compelled by law to have insurance.
	\end{itemize}
}
\item  \\
 feel compelled \textit{
	\begin{itemize}
	\end{itemize}
}
\end{enumerate}

\section*{accuse}
{\large \color{blue}  accuses  accusing  accused  }
\subsection*{Explain}
\begin{enumerate}
\item verb \\
If you \textbf{accuse} someone \textbf{of} doing something wrong or dishonest , you say or tell them that you believe that they did it.
 \textit{
	\begin{itemize}
	\item He was accusing my mum of having an affair with another man.
	\item Talk things through in stages. Do not accuse or apportion blame.
	\end{itemize}
}
\item verb \\
If you \textbf{are accused}  \textbf{of} a crime, a witness or someone in authority  states or claims that you did it, and you may be formally charged with it and put on trial .
 \textit{
	\begin{itemize}
	\item Her assistant was accused of theft and fraud by the police.
	\item All seven charges accused him of lying in his testimony.
	\item The accused men have been given relatively light sentences.
	\end{itemize}
}
\item  \\
 stand accused \textit{
	\begin{itemize}
	\end{itemize}
}
\end{enumerate}

\section*{decorate}
{\large \color{blue}  decorates  decorating  decorated  }
\subsection*{Explain}
\begin{enumerate}
\item verb \\
If you \textbf{decorate} something, you make it more attractive by adding things to it.
 \textit{
	\begin{itemize}
	\item He decorated his room with pictures of all his favorite sports figures.
	\item Use shells to decorate boxes, trays, mirrors or even pots.
	\end{itemize}
}
\item verb \\
If you \textbf{decorate} a room or the inside of a building , you put new paint or wallpaper on the walls and ceiling , and paint the woodwork .
 \textit{
	\begin{itemize}
	\item We decorated the guest bedroom in shades of white and cream.
	\item The boys are planning to decorate when they get the time.
	\item I had the flat decorated quickly so that Philippa could move in.
	\item ...a small, badly decorated office.
	\end{itemize}
}
\item verb \\
If something \textbf{decorates} a place or an object, it makes it look more attractive.
 \textit{
	\begin{itemize}
	\item Posters decorate the walls.
	\end{itemize}
}
\item verb \\
If someone \textbf{is decorated} , they are given a medal or other honour as an official  reward for something that they have done .
 \textit{
	\begin{itemize}
	\item He was decorated for bravery in battle.
	\end{itemize}
}
\end{enumerate}

\section*{adjust}
{\large \color{blue}  adjusts  adjusting  adjusted  }
\subsection*{Explain}
\begin{enumerate}
\item verb \\
When you \textbf{adjust}  \textbf{to} a new situation , you get used to it by changing your behaviour or your ideas .
 \textit{
	\begin{itemize}
	\item We have been preparing our fighters to adjust themselves to civil society.
	\item I felt I had adjusted to the idea of being a mother very well.
	\item It has been hard to adjust but now I'm getting satisfaction from my work.
	\end{itemize}
}
\item verb \\
If you \textbf{adjust} something, you change it so that it is more effective or appropriate .
 \textit{
	\begin{itemize}
	\item To attract investors, the government has adjusted its tax and labour laws.
	\end{itemize}
}
\item verb \\
If you \textbf{adjust} something such as your clothing or a machine , you correct or alter its position or setting .
 \textit{
	\begin{itemize}
	\item She adjusted her head scarf fussily.
	\item Liz adjusted her mirror and then edged the car out of its parking bay.
	\end{itemize}
}
\item verb \\
If you \textbf{adjust} your vision or if your vision \textbf{adjusts} , the muscles of your eye or the pupils alter to cope with changes in light or distance .
 \textit{
	\begin{itemize}
	\item He stopped to try to adjust his vision to the faint starlight.
	\item We stood in the doorway until our eyes adjusted.
	\item It was a few moments before his eyes became adjusted to the bright glare of the sun.
	\end{itemize}
}
\end{enumerate}

\section*{edit}
{\large \color{blue}  edits  editing  edited  }
\subsection*{Explain}
\begin{enumerate}
\item verb \\
If you \textbf{edit} a text such as an article or a book , you correct and adapt it so that it is suitable for publishing .
 \textit{
	\begin{itemize}
	\item The majority of contracts give the publisher the right to edit a book after it's
done.
	\item ...an edited version of the speech.
	\end{itemize}
}
\item verb \\
If you \textbf{edit} a book or a series of books, you collect several pieces of writing by different  authors and prepare them for publishing.
 \textit{
	\begin{itemize}
	\item This collection of essays is edited by Ellen Knight.
	\item She has edited the media studies quarterly, Screen.
	\item ...the Real Sandwich Book, edited by Miriam Polunin.
	\end{itemize}
}
\item verb \\
If you \textbf{edit} a film or a television or radio  programme , you choose some of what has been filmed or recorded and arrange it in a particular order.
 \textit{
	\begin{itemize}
	\item He taught me to edit and splice film.
	\item He is editing together excerpts of some of his films.
	\end{itemize}
}
\item verb \\
Someone who \textbf{edits} a newspaper , magazine , or journal is in charge of it.
 \textit{
	\begin{itemize}
	\item I used to edit the college paper in the old days.
	\end{itemize}
}
\item countable noun \\
An \textbf{edit} is the process of examining and correcting a text so that it is suitable for publishing.
 \textit{
	\begin{itemize}
	\item The purpose of the edit is fairly simple – to chop out the boring bits from the original.
	\end{itemize}
}
\end{enumerate}

\section*{adore}
{\large \color{blue}  adores  adoring  adored  }
\subsection*{Explain}
\begin{enumerate}
\item verb \\
If you \textbf{adore} someone, you feel great love and admiration for them.
 \textit{
	\begin{itemize}
	\item She adored her parents and would do anything to please them.
	\end{itemize}
}
\item verb \\
If you \textbf{adore} something, you like it very much.
 \textit{
	\begin{itemize}
	\item My mother adores bananas and eats two a day.
	\item I adore good books and the theatre.
	\end{itemize}
}
\end{enumerate}

\section*{eject}
{\large \color{blue}  ejects  ejecting  ejected  }
\subsection*{Explain}
\begin{enumerate}
\item verb \\
If you \textbf{eject} someone \textbf{from} a place, you force them to leave.
 \textit{
	\begin{itemize}
	\item Officials used guard dogs to eject the protesters.
	\item He was ejected from a restaurant.
	\end{itemize}
}
\item verb \\
To \textbf{eject} something means to remove it or push it out forcefully.
 \textit{
	\begin{itemize}
	\item He aimed his rifle, fired a single shot, then ejected the spent cartridge.
	\end{itemize}
}
\item verb \\
When a pilot  \textbf{ejects}  \textbf{from} an aircraft, he or she leaves the aircraft quickly using an ejector seat, usually because the plane is about to crash .
 \textit{
	\begin{itemize}
	\item The pilot ejected from the plane and escaped injury.
	\end{itemize}
}
\end{enumerate}

\section*{affirm}
{\large \color{blue}  affirms  affirming  affirmed  }
\subsection*{Explain}
\begin{enumerate}
\item verb \\
If you \textbf{affirm} that something is true or that something exists , you state firmly and publicly that it is true or exists.
 \textit{
	\begin{itemize}
	\item The courts have affirmed that the act can be applied to social media.
	\item ...a speech in which he affirmed a commitment to lower taxes.
	\item 'This place is a dump,' affirmed Miss T.
	\end{itemize}
}
\item verb \\
If an event  \textbf{affirms} something, it shows that it is true or exists.
 \textit{
	\begin{itemize}
	\item Everything I had accomplished seemed to affirm that opinion.
	\end{itemize}
}
\end{enumerate}

\section*{encourage}
{\large \color{blue}  encourages  encouraging  encouraged  }
\subsection*{Explain}
\begin{enumerate}
\item verb \\
If you \textbf{encourage} someone, you give them confidence, for example by letting them know that what they are doing is good and telling them that they should continue to do it.
 \textit{
	\begin{itemize}
	\item When things aren't going well, he encourages me, telling me not to give up.
	\end{itemize}
}
\item verb \\
If someone \textbf{is encouraged by} something that happens , it gives them hope or confidence.
 \textit{
	\begin{itemize}
	\item Investors were encouraged by the news.
	\item He is encouraged by a dry weather forecast.
	\end{itemize}
}
\item verb \\
If you \textbf{encourage} someone \textbf{to} do something, you try to persuade them to do it, for example by telling them that it would be a pleasant thing to do, or by trying to make it easier for them to do it. You can also  \textbf{encourage} an activity .
 \textit{
	\begin{itemize}
	\item We want to encourage people to go fishing, not put them off.
	\item He was encouraged by his family to learn music at a young age.
	\item Their task is to help encourage private investment in Russia.
	\item Participation is encouraged at all levels.
	\end{itemize}
}
\item verb \\
If something \textbf{encourages} a particular activity or state, it causes it to happen or increase .
 \textit{
	\begin{itemize}
	\item ...a natural substance that encourages cell growth.
	\item Such secrecy breeds and encourages fear and suspicion.
	\item Slow music encourages supermarket-shoppers to browse longer but spend more.
	\end{itemize}
}
\end{enumerate}

\section*{anticipate}
{\large \color{blue}  anticipates  anticipating  anticipated  }
\subsection*{Explain}
\begin{enumerate}
\item verb \\
If you \textbf{anticipate} an event, you realize in advance that it may happen and you are prepared for it.
 \textit{
	\begin{itemize}
	\item At the time we couldn't have anticipated the result of our campaigning.
	\item It is anticipated that the equivalent of 192 full-time jobs will be lost.
	\item I hadn't anticipated that Rob's team would advance that far.
	\end{itemize}
}
\item verb \\
If you \textbf{anticipate} a question , request , or need , you do what is necessary or required before the question, request, or need occurs.
 \textit{
	\begin{itemize}
	\item What Jeff did was to anticipate my next question.
	\item Do you expect your partner to anticipate your needs?
	\end{itemize}
}
\item verb \\
If you \textbf{anticipate} something, you do it, think it, or say it before someone else does.
 \textit{
	\begin{itemize}
	\item In the 50s, Rauschenberg anticipated the conceptual art movement of the 80s.
	\end{itemize}
}
\end{enumerate}

\section*{envisage}
{\large \color{blue}  envisages  envisaging  envisaged  }
\subsection*{Explain}
\begin{enumerate}
\item verb \\
If you \textbf{envisage} something, you imagine that it is true , real , or likely to happen .
 \textit{
	\begin{itemize}
	\item He envisages the possibility of establishing direct diplomatic relations in the future.
	\item He had never envisaged spending the whole of his working life in that particular
job.
	\item Personally, I envisage them staying together.
	\end{itemize}
}
\end{enumerate}

\section*{arouse}
{\large \color{blue}  arouses  arousing  aroused  }
\subsection*{Explain}
\begin{enumerate}
\item verb \\
If something \textbf{arouses} a particular reaction or attitude in people, it causes them to have that reaction or attitude.
 \textit{
	\begin{itemize}
	\item We left in the daytime so as not to arouse suspicion.
	\item ...the deep public anger you have aroused.
	\end{itemize}
}
\item verb \\
If something \textbf{arouses} a particular feeling or instinct that exists in someone, it causes them to experience that feeling or instinct strongly.
 \textit{
	\begin{itemize}
	\item There is nothing like a long walk to arouse the appetite.
	\end{itemize}
}
\item verb \\
If you \textbf{are aroused} by something, it makes you feel sexually excited .
 \textit{
	\begin{itemize}
	\item Some men are aroused when their partner says erotic words to them.
	\end{itemize}
}
\item verb \\
If something \textbf{arouses} you, it makes you feel angry .
 \textit{
	\begin{itemize}
	\item He apologized, saying this subject always aroused him.
	\end{itemize}
}
\item verb \\
If something \textbf{arouses} you \textbf{from} sleep, it wakes you up.
 \textit{
	\begin{itemize}
	\item About two o'clock, we were aroused from our sleep by a knocking at the door.
	\end{itemize}
}
\end{enumerate}

\section*{evaporate}
{\large \color{blue}  evaporates  evaporating  evaporated  }
\subsection*{Explain}
\begin{enumerate}
\item verb \\
When a liquid \textbf{evaporates} , or \textbf{is evaporated} , it changes from a liquid state to a gas , because its temperature has increased .
 \textit{
	\begin{itemize}
	\item Moisture is drawn to the surface of the fabric so that it evaporates.
	\item The water is evaporated by the sun.
	\end{itemize}
}
\item verb \\
If a feeling , plan , or activity  \textbf{evaporates} , it gradually becomes weaker and eventually disappears completely.
 \textit{
	\begin{itemize}
	\item My anger evaporated and I wanted to cry.
	\item Your dreams always seem to evaporate, and nothing ever quite matches expectations.
	\item The project evaporated and Harry was left high and dry.
	\end{itemize}
}
\end{enumerate}

\section*{assert}
{\large \color{blue}  asserts  asserting  asserted  }
\subsection*{Explain}
\begin{enumerate}
\item verb \\
If someone \textbf{asserts} a fact or belief , they state it firmly.
 \textit{
	\begin{itemize}
	\item Mr. Helm plans to assert that the bill violates the First Amendment.
	\item The defendants, who continue to assert their innocence, are expected to appeal.
	\item Altman asserted, 'We were making a political statement about western civilisation
and greed.'
	\end{itemize}
}
\item verb \\
If you \textbf{assert} your authority, you make it clear by your behaviour that you have authority.
 \textit{
	\begin{itemize}
	\item After the war, the army made an attempt to assert its authority in the south of the
country.
	\item The people have asserted their power and that will be very difficult to reverse.
	\end{itemize}
}
\item verb \\
If you \textbf{assert} your right or claim to something, you insist that you have the right to it.
 \textit{
	\begin{itemize}
	\item The republics began asserting their right to govern themselves.
	\end{itemize}
}
\item verb \\
If you \textbf{assert}  \textbf{yourself} , you speak and act in a forceful way, so that people take notice of you.
 \textit{
	\begin{itemize}
	\item He's speaking up and asserting himself confidently.
	\end{itemize}
}
\end{enumerate}

\section*{fall}
{\large \color{blue}  falls  falling  fell  fallen  }
\subsection*{Explain}
\begin{enumerate}
\item verb \\
If someone or something \textbf{falls} , they move quickly downwards onto or towards the ground, by accident or because of a natural force.
 \textbf{Fall} is also a noun.
 \textit{
	\begin{itemize}
	\item Her father fell into the sea after a massive heart attack.
	\item The prince has again fallen from his horse.
	\item Bombs fell in the town.
	\item I ought to seal the boxes up. I don't want the books falling out.
	\item Twenty people were injured by falling masonry.
	\item The helmets are designed to withstand impacts equivalent to a fall from a bicycle.
	\end{itemize}
}
\item verb \\
If a person or structure that is standing  somewhere  \textbf{falls} , they move from their upright position, so that they are then lying on the ground.
 \textbf{Fall} is also a noun.
 \textbf{Fall down} means the same as fall .
 \textit{
	\begin{itemize}
	\item The woman gripped the shoulders of her man to stop herself from falling.
	\item We watched buildings fall on top of people and pets.
	\item He lost his balance and fell backwards.
	\item Mrs Briscoe had a bad fall last week.
	\item I hit him so hard he fell down.
	\item Children jumped from upper floors as the building fell down around them.
	\end{itemize}
}
\item verb \\
When rain or snow  \textbf{falls} , it comes down from the sky .
 \textbf{Fall} is also a noun.
 \textit{
	\begin{itemize}
	\item Winds reached up to 100mph in some places with an inch of rain falling within 15
minutes.
	\item One night there was a heavy fall of snow.
	\end{itemize}
}
\item verb \\
If you \textbf{fall} somewhere, you allow yourself to drop there in a hurried or disorganized way, often because you are very tired .
 \textit{
	\begin{itemize}
	\item Totally exhausted, he tore his clothes off and fell into bed.
	\item In the morning I got as far as the sofa and fell on to it.
	\end{itemize}
}
\item verb \\
If something \textbf{falls} , it decreases in amount, value, or strength.
 \textbf{Fall} is also a noun.
 \textit{
	\begin{itemize}
	\item Output will fall by 6%.
	\item Her weight fell to under seven stones.
	\item Between July and August, oil product prices fell 0.2 per cent.
	\item The number of prosecutions has stayed static and the rate of convictions has fallen.
	\item ...a time of falling living standards and emerging mass unemployment.
	\item There was a sharp fall in the value of the pound.
	\end{itemize}
}
\item verb \\
If a powerful or successful person \textbf{falls} , they suddenly lose their power or position.
 \textbf{Fall} is also a noun.
 \textit{
	\begin{itemize}
	\item There's a danger of the government falling because it will lose its majority.
	\item When Cromwell fell from power, the king took everything.
	\item Since the fall of the military dictator, the country has had a civilian government.
	\item Her rise has mirrored his fall.
	\end{itemize}
}
\item verb \\
If a place \textbf{falls} in a war or election , an enemy army or a different political party takes control of it.
 \textbf{Fall} is also a noun.
 \textit{
	\begin{itemize}
	\item Paris fell to the allies in August 1944.
	\item With the announcement 'Paphos has fallen!' a cheer went up from the assembled soldiers.
	\item ...the fall of Rome.
	\end{itemize}
}
\item verb \\
If someone \textbf{falls} in battle, they are killed.
 \textit{
	\begin{itemize}
	\item Another wave of troops followed the first, running past those who had fallen.
	\end{itemize}
}
\item link verb \\
You can use \textbf{fall} to show that someone or something passes into another state. For example, if someone
 \textbf{falls ill} , they become ill, and if something \textbf{falls into disrepair} , it is then in a state of disrepair.
 \textit{
	\begin{itemize}
	\item It is almost impossible to visit Florida without falling in love with the state.
	\item A quarter of those surveyed had fallen into debt as a result of childcare costs.

	\item I took Moira to the cinema, where she fell asleep.
	\item Almost without exception these women fall victim to exploitation.
	\end{itemize}
}
\item verb \\
If you say that something or someone \textbf{falls into} a particular group or category , you mean that they belong in that group or category.
 \textit{
	\begin{itemize}
	\item The problems generally fall into two categories.
	\item Both women fall into the highest-risk group.
	\end{itemize}
}
\item verb \\
If the responsibility or blame for something \textbf{falls on} someone, they have to take the responsibility or the blame for it.
 \textit{
	\begin{itemize}
	\item That responsibility falls on the local office of the United Nations High Commissioner
for Refugees.
	\item A vastly disproportionate burden falls on women for child care.
	\item A lot of suspicion fell on her.
	\end{itemize}
}
\item verb \\
If silence or a feeling of sadness or tiredness \textbf{falls} on a group of people, they become silent , sad , or tired.
 \textit{
	\begin{itemize}
	\item The bus was stopped and silence fell on the passengers as the police checked identity
cards.
	\end{itemize}
}
\item verb \\
If a celebration or other special event \textbf{falls on} a particular day or date, it happens to be on that day or date.
 \textit{
	\begin{itemize}
	\item Easter falls on April 10 next year.
	\end{itemize}
}
\item verb \\
When light or shadow  \textbf{falls} on something, it covers it.
 \textit{
	\begin{itemize}
	\item Nancy, out of the corner of her eye, saw the shadow that suddenly fell across the
doorway.
	\end{itemize}
}
\item verb \\
If someone's hair or a garment \textbf{falls} in a certain way, it hangs downwards in that way.
 \textit{
	\begin{itemize}
	\item ...a slender boy with black hair falling across his forehead.
	\end{itemize}
}
\item verb \\
If you say that someone's eyes \textbf{fell}  \textbf{on} something, you mean they suddenly noticed it.
 \textit{
	\begin{itemize}
	\item As he laid the flowers on the table, his eye fell upon a note in Grace's handwriting.
	\end{itemize}
}
\item verb \\
When night or darkness \textbf{falls} , night begins and it becomes dark.
 \textit{
	\begin{itemize}
	\item As darkness fell outside, they sat down to eat at long tables.
	\end{itemize}
}
\item plural noun \\
You can refer to a waterfall as \textbf{the}  \textbf{falls} .
 \textit{
	\begin{itemize}
	\item ...panoramic views of the falls.
	\item ...Niagara Falls.
	\end{itemize}
}
\item variable noun \\
\textbf{Fall} is the season between summer and winter when the weather becomes cooler and the leaves fall off the trees.
 \textit{
	\begin{itemize}
	\item He was elected judge in the fall.
	\item The Supreme Court will not hear the case until next fall.
	\item The program was launched in the fall of 1990.
	\item The policy will take effect after the fall election.
	\end{itemize}
}
\item proper noun \\
In the Christian religion, \textbf{the Fall} was the occasion when Adam and Eve sinned and God made them leave the Garden of Eden .
 \textit{
	\begin{itemize}
	\end{itemize}
}
\item countable noun \\
In some sports such as judo and wrestling , a \textbf{fall} is the act of throwing or forcing your opponent to the floor.
 \textit{
	\begin{itemize}
	\end{itemize}
}
\item verb \\
In cricket , when a wicket \textbf{falls} , the team who are fielding get one of the batsmen out.
 \textit{
	\begin{itemize}
	\item The last seven wickets fell for ten runs.
	\end{itemize}
}
\item  \\
 to fall open \textit{
	\begin{itemize}
	\end{itemize}
}
\item  \\
 to fall over yourself to do \textit{
	\begin{itemize}
	\end{itemize}
}
\item  \\
 to fall to bits/pieces \textit{
	\begin{itemize}
	\end{itemize}
}
\end{enumerate}

\section*{assign}
{\large \color{blue}  assigns  assigning  assigned  }
\subsection*{Explain}
\begin{enumerate}
\item verb \\
If you \textbf{assign} a piece of work \textbf{to} someone, you give them the work to do.
 \textit{
	\begin{itemize}
	\item When I taught, I would assign a topic to children which they would write about.
	\item Later in the year, she'll assign them research papers.
	\item When teachers assign homework, students usually feel an obligation to do it.
	\end{itemize}
}
\item verb \\
If you \textbf{assign} something \textbf{to} someone, you say that it is for their use.
 \textit{
	\begin{itemize}
	\item The selling broker is then required to assign a portion of the commission to the
buyer broker.
	\item He assigned her all his land in Ireland.
	\end{itemize}
}
\item verb \\
If someone \textbf{is assigned}  \textbf{to} a particular place, group, or person, they are sent there, usually in order to work at that place or for that person.
 \textit{
	\begin{itemize}
	\item I was assigned to Troop A of the 10th Cavalry.
	\item Did you choose Russia or were you simply assigned there?
	\item Each of us was assigned a minder, someone who looked after us.
	\end{itemize}
}
\item verb \\
If you \textbf{assign} a particular function or value \textbf{to} someone or something, you say they have it.
 \textit{
	\begin{itemize}
	\item Under Mr. Harel's system, each business must assign a value to each job.
	\item Assign the letters of the alphabet their numerical values–A equals 1, B equals 2,
etc.
	\end{itemize}
}
\end{enumerate}

\section*{follow}
{\large \color{blue}  follows  following  followed  }
\subsection*{Explain}
\begin{enumerate}
\item verb \\
If you \textbf{follow} someone who is going  somewhere , you move along behind them because you want to go to the same place.
 \textit{
	\begin{itemize}
	\item We followed him up the steps into a large hall.
	\item Please follow me, madam.
	\item They took him into a small room and I followed.
	\end{itemize}
}
\item verb \\
If you \textbf{follow} someone who is going somewhere, you move along behind them without their knowledge , in order to catch them or find out where they are going.
 \textit{
	\begin{itemize}
	\item She realized that the Mercedes was following her.
	\item I think we're being followed.
	\end{itemize}
}
\item verb \\
If you \textbf{follow} someone to a place where they have recently gone and where they are now, you go to
join them there.
 \textit{
	\begin{itemize}
	\item He followed Janice to New York, where she was preparing an exhibition.
	\end{itemize}
}
\item verb \\
An event, activity, or period of time that \textbf{follows} a particular thing happens or comes after that thing, at a later time.
 \textit{
	\begin{itemize}
	\item ...the rioting and looting that followed the verdict.
	\item He was arrested in the confusion which followed.
	\item Other problems may follow.
	\item Eye witnesses spoke of a noise followed by a huge red light.
	\end{itemize}
}
\item verb \\
If you \textbf{follow} one thing \textbf{with} another, you do or say the second thing after you have done or said the first thing.
 \textbf{Follow up} means the same as follow .
 \textit{
	\begin{itemize}
	\item She broadcast on radio, and followed this with a series of credits in films and on
TV.
	\item Most comics who score a hit at Edinburgh follow it up with a big tour.
	\end{itemize}
}
\item verb \\
If it \textbf{follows} that a particular thing is the case, that thing is a logical result of something
else being true or being the case.
 \textit{
	\begin{itemize}
	\item Just because a bird does not breed one year, it does not follow that it will fail
the next.
	\item If the explanation is right, two things follow.
	\item It is easy to see the conclusions described in the text follow from this equation.
	\end{itemize}
}
\item verb \\
If you refer to the words that \textbf{follow} or \textbf{followed} , you are referring to the words that come next or came next in a piece of writing or speech.
 \textit{
	\begin{itemize}
	\item What follows is an eye-witness account.
	\item There followed a list of places where Hans intended to visit.
	\item General analysis is followed by five case studies.
	\end{itemize}
}
\item verb \\
If you \textbf{follow} a path , route , or set of signs, you go somewhere using the path, route, or signs to direct you.
 \textit{
	\begin{itemize}
	\item If they followed the road, they would be certain to reach a village.
	\item All we had to do was follow the map.
	\item I followed the signs to Metrocity.
	\end{itemize}
}
\item verb \\
If something such as a path or river \textbf{follows} a particular route or line, it goes along that route or line.
 \textit{
	\begin{itemize}
	\item Our route follows the Pacific coast through densely populated neighbourhoods.
	\item The Lot river follows a winding and tortuous course.
	\end{itemize}
}
\item verb \\
If you \textbf{follow} something with your eyes, or if your eyes \textbf{follow} it, you watch it as it moves or you look along its route or course.
 \textit{
	\begin{itemize}
	\item Ann's eyes followed a police car as it drove slowly past.
	\end{itemize}
}
\item verb \\
Something that \textbf{follows} a particular course of development happens or develops in that way.
 \textit{
	\begin{itemize}
	\item His release turned out to follow the pattern set by that of the other six hostages.
	\end{itemize}
}
\item verb \\
If you \textbf{follow}  advice , an instruction , or a recipe , you act or do something in the way that it indicates.
 \textit{
	\begin{itemize}
	\item Take care to follow the instructions carefully.
	\item No two chefs follow the same recipe.
	\end{itemize}
}
\item verb \\
If you \textbf{follow} what someone else has done, you do it too because you think it is a good thing or because you want to copy them.
 \textit{
	\begin{itemize}
	\item If I do my bit, others will follow my example and join in to improve things.
	\item Where rich celebrities lead, the great British public will surely follow.
	\end{itemize}
}
\item verb \\
If you \textbf{follow} someone in what you do, you do the same thing or job as they did previously.
 \textit{
	\begin{itemize}
	\item He followed his father and became a surgeon.
	\item Anni-Frid's son has followed her into the music business.
	\end{itemize}
}
\item verb \\
If you are able to \textbf{follow} something such as an explanation or the story of a film, you understand it as it continues and develops.
 \textit{
	\begin{itemize}
	\item Can you follow the plot so far?
	\item I'm afraid I don't follow.
	\end{itemize}
}
\item verb \\
If you \textbf{follow} something, you take an interest in it and keep informed about what happens.
 \textit{
	\begin{itemize}
	\item ...the millions of people who follow football because they genuinely love it.
	\item She was following Laura's progress closely.
	\end{itemize}
}
\item verb \\
If you \textbf{follow} someone on a social media  website , you choose to see messages and pictures that they post there.
 \textit{
	\begin{itemize}
	\item Camille has a blog, and you can follow her on Twitter.
	\item You can also browse Instagram's galleries and follow favourite photographers.
	\end{itemize}
}
\item verb \\
A story, film, or television programme that \textbf{follows} someone or something is about their experiences over a particular period of time.
 \textit{
	\begin{itemize}
	\item The film follows the fortunes of two women.
	\end{itemize}
}
\item verb \\
If you \textbf{follow} a score or written copy of a play, you read it while you listen to it being performed.
 \textit{
	\begin{itemize}
	\item ...an annotated version of Mozart's opera that allows the listener to follow the
score.
	\end{itemize}
}
\item verb \\
If you \textbf{follow} a particular religion or political belief, you have that religion or belief.
 \textit{
	\begin{itemize}
	\item 'Do you follow any particular religion?'—'Yes, we're all Hindus.'
	\end{itemize}
}
\item  \\
 as follows \textit{
	\begin{itemize}
	\end{itemize}
}
\item  \\
 followed by \textit{
	\begin{itemize}
	\end{itemize}
}
\item  \\
 to follow \textit{
	\begin{itemize}
	\end{itemize}
}
\end{enumerate}

\section*{assume}
{\large \color{blue}  assumes  assuming  assumed  }
\subsection*{Explain}
\begin{enumerate}
\item verb \\
If you \textbf{assume}  \textbf{that} something is true , you imagine that it is true, sometimes wrongly.
 \textit{
	\begin{itemize}
	\item It is a misconception to assume that the two continents are similar.
	\item If the package is wrapped well, we assume the contents are also wonderful.
	\item If mistakes occurred, they were assumed to be the fault of the commander on the spot.
	\item 'Today?'—'I'd assume so, yeah.'
	\end{itemize}
}
\item verb \\
If someone \textbf{assumes} power or responsibility, they take power or responsibility.
 \textit{
	\begin{itemize}
	\item Mr Cross will assume the role of Chief Executive with a team of four directors.
	\item If there is no president, power will be assumed by the most extremist forces.
	\end{itemize}
}
\item verb \\
If something \textbf{assumes} a particular  quality , it begins to have that quality.
 \textit{
	\begin{itemize}
	\item In his dreams, the mountains assumed enormous importance.
	\end{itemize}
}
\item verb \\
If you \textbf{assume} a particular expression or way of behaving , you start to look or behave in this way.
 \textit{
	\begin{itemize}
	\item He contented himself by assuming an air of superiority.
	\item Prue assumed a placatory tone of voice.
	\end{itemize}
}
\item  \\
 let us assume \textit{
	\begin{itemize}
	\end{itemize}
}
\end{enumerate}

\section*{forbid}
{\large \color{blue}  forbids  forbidding  forbade  forbidden  }
\subsection*{Explain}
\begin{enumerate}
\item verb \\
If you \textbf{forbid} someone \textbf{to} do something, or if you \textbf{forbid} an activity , you order that it must not be done .
 \textit{
	\begin{itemize}
	\item They'll forbid you to marry.
	\item She was shut away and forbidden to read.
	\item Brazil's constitution forbids the military use of nuclear energy.
	\end{itemize}
}
\item verb \\
If something \textbf{forbids} a particular course of action or state of affairs , it makes it impossible for the course of action or state of affairs to happen .
 \textit{
	\begin{itemize}
	\item His own pride forbids him to ask Arthur's help.
	\item Custom forbids any modernisation.
	\end{itemize}
}
\end{enumerate}

\section*{award}
{\large \color{blue}  awards  awarding  awarded  }
\subsection*{Explain}
\begin{enumerate}
\item countable noun \\
An \textbf{award} is a prize or certificate that a person is given for doing something well .
 \textit{
	\begin{itemize}
	\item ...the Booker prize, Britain's top award for fiction.
	\item She presented a bravery award to schoolgirl Caroline Tucker.
	\end{itemize}
}
\item countable noun \\
In law, an \textbf{award} is a sum of money that a court decides should be given to someone.
 \textit{
	\begin{itemize}
	\item ...workmen's compensation awards.
	\end{itemize}
}
\item countable noun \\
A pay  \textbf{award} is an increase in pay for a particular group of workers .
 \textit{
	\begin{itemize}
	\item ...this year's average pay award for teachers of just under 8%.
	\end{itemize}
}
\item verb \\
If someone \textbf{is awarded} something such as a prize or an examination  mark , it is given to them.
 \textit{
	\begin{itemize}
	\item She was awarded the prize for both films.
	\item For his dedication the Mayor awarded him a medal of merit.
	\end{itemize}
}
\item verb \\
To \textbf{award} something \textbf{to} someone means to decide that it will be given to that person.
 \textit{
	\begin{itemize}
	\item We have awarded the contract to a British shipyard.
	\item A High Court judge had awarded him £6 million damages.
	\end{itemize}
}
\end{enumerate}

\section*{imagine}
{\large \color{blue}  imagines  imagining  imagined  }
\subsection*{Explain}
\begin{enumerate}
\item verb \\
If you \textbf{imagine} something, you think about it and your mind forms a picture or idea of it.
 \textit{
	\begin{itemize}
	\item He could not imagine a more peaceful scene.
	\item She couldn't imagine living in a place like that.
	\item Can you imagine how she must have felt when Mary Brent turned up with me in tow?
	\item Imagine you're lying on a beach, listening to the steady rhythm of waves lapping
the shore.
	\item I can't imagine you being unfair to anyone, Leigh.
	\end{itemize}
}
\item verb \\
If you \textbf{imagine} that something is the case , you think that it is the case.
 \textit{
	\begin{itemize}
	\item I imagine you're referring to Jean-Paul Sartre.
	\item We tend to imagine that the Victorians were very prim and proper.
	\item 'Was he meeting someone?'—'I imagine so.'
	\end{itemize}
}
\item verb \\
If you \textbf{imagine} something, you think that you have seen , heard , or experienced that thing, although actually you have not.
 \textit{
	\begin{itemize}
	\item I realised that I must have imagined the whole thing.
	\end{itemize}
}
\end{enumerate}

\section*{beg}
{\large \color{blue}  begs  begging  begged  }
\subsection*{Explain}
\begin{enumerate}
\item verb \\
If you \textbf{beg} someone \textbf{to} do something, you ask them very anxiously or eagerly to do it.
 \textit{
	\begin{itemize}
	\item I begged him to come back to England with me.
	\item I begged to be allowed to leave.
	\item We are not going to beg for help any more.
	\item They dropped to their knees and begged forgiveness.
	\end{itemize}
}
\item verb \\
If someone who is poor  \textbf{is begging} , they are asking people to give them food or money.
 \textit{
	\begin{itemize}
	\item I was surrounded by people begging for food.
	\item There are thousands like him, begging on the streets and sleeping rough.
	\item She was living alone, begging food from neighbors.
	\end{itemize}
}
\item  \\
 I beg to differ \textit{
	\begin{itemize}
	\end{itemize}
}
\item  \\
 going begging \textit{
	\begin{itemize}
	\end{itemize}
}
\item  \\
 beg the question \textit{
	\begin{itemize}
	\end{itemize}
}
\item  \\
 beg the question \textit{
	\begin{itemize}
	\end{itemize}
}
\end{enumerate}

\section*{kneel}
{\large \color{blue}  kneels  kneeling  kneeled  knelt  }
\subsection*{Explain}
\begin{enumerate}
\item verb \\
When you \textbf{kneel} , you bend your legs so that your knees are touching the ground .
 \textbf{Kneel down} means the same as kneel .
 \textit{
	\begin{itemize}
	\item She knelt by the bed and prayed.
	\item Other people were kneeling, but she just sat.
	\item ...a kneeling position.
	\item She kneeled down beside him.
	\end{itemize}
}
\end{enumerate}

\section*{borrow}
{\large \color{blue}  borrows  borrowing  borrowed  }
\subsection*{Explain}
\begin{enumerate}
\item verb \\
If you \textbf{borrow} something that belongs to someone else, you take it or use it for a period of time,
usually with their permission .
 \textit{
	\begin{itemize}
	\item Can I borrow a pen please?
	\item He wouldn't let me borrow his clothes.
	\end{itemize}
}
\item verb \\
If you \textbf{borrow} money \textbf{from} someone or \textbf{from} a bank , they give it to you and you agree to pay it back at some time in the future .
 \textit{
	\begin{itemize}
	\item Morgan borrowed £5,000 from his father to form the company 20 years ago.
	\item It's so expensive to borrow from finance companies.
	\item He borrowed heavily to get the money together.
	\end{itemize}
}
\item verb \\
If you \textbf{borrow} a book \textbf{from} a library , you take it away for a fixed period of time.
 \textit{
	\begin{itemize}
	\item I couldn't afford to buy any, so I borrowed them from the library.
	\end{itemize}
}
\item verb \\
If you \textbf{borrow} something such as a word or an idea from another language or from another person's
work, you use it in your own language or work.
 \textit{
	\begin{itemize}
	\item I borrowed his words for my book's title.
	\item Their engineers are happier borrowing other people's ideas than developing their
own.
	\end{itemize}
}
\item  \\
 be/be living on borrowed time \textit{
	\begin{itemize}
	\end{itemize}
}
\end{enumerate}

\section*{knit}
{\large \color{blue}  knits  knitting  knitted  }
\subsection*{Explain}
\begin{enumerate}
\item verb \\
If you \textbf{knit} something, especially an article of clothing , you make it from wool or a similar thread by using two knitting needles or a machine.
 \textbf{Knit} is also a combining form.
 \textit{
	\begin{itemize}
	\item I had endless hours to knit and sew.
	\item I have already started knitting baby clothes.
	\item She knitted him 10 pairs of socks to take with him.
	\item During the war, Joan helped her mother knit scarves for soldiers.
	\item She pushed up the sleeves of her grey knitted cardigan and got to work.
	\item Ferris wore a heavy knit sweater.
	\item ...a cotton-knit sweater.
	\item ...hand-knit garments.
	\end{itemize}
}
\item verb \\
If someone or something \textbf{knits} things or people \textbf{together} , they make them fit or work together closely and successfully.
 \textit{
	\begin{itemize}
	\item The best thing about sport is that it knits the whole family close together.
	\item People have reservations about their president's drive to knit them so closely to
their neighbors.
	\end{itemize}
}
\item combining form \\
\textbf{Knit} is also a combining form.
 \textit{
	\begin{itemize}
	\item ...a closer-knit family.
	\item ...a tightly knit society.
	\end{itemize}
}
\item verb \\
When broken bones \textbf{knit} , the broken pieces grow together again.
 \textit{
	\begin{itemize}
	\item The bone hasn't knitted together properly.
	\item ...broken bones that have failed to knit.
	\end{itemize}
}
\item  \\
 to knit your brow \textit{
	\begin{itemize}
	\end{itemize}
}
\end{enumerate}

\section*{comprehend}
{\large \color{blue}  comprehends  comprehending  comprehended  }
\subsection*{Explain}
\begin{enumerate}
\item verb \\
If you cannot \textbf{comprehend} something, you cannot understand it.
 \textit{
	\begin{itemize}
	\item I just cannot comprehend your attitude.
	\item Whenever she failed to comprehend she invariably laughed.
	\end{itemize}
}
\end{enumerate}

\section*{leap}
{\large \color{blue}  leaps  leaping  leaped  leapt  }
\subsection*{Explain}
\begin{enumerate}
\item verb \\
If you \textbf{leap} , you jump high in the air or jump a long distance.
 \textbf{Leap} is also a noun .
 \textit{
	\begin{itemize}
	\item He had leapt from a window in the building and escaped.
	\item The newsreels show him leaping into the air.
	\item The man threw his arms out as he leapt.
	\item He won the championship with a leap of 2.37 metres.
	\end{itemize}
}
\item verb \\
If you \textbf{leap}  somewhere , you move there suddenly and quickly.
 \textit{
	\begin{itemize}
	\item The two men leaped into the jeep and roared off.
	\item With a terrible howl, he leapt forward and threw himself into the water.
	\end{itemize}
}
\item verb \\
If a vehicle  \textbf{leaps} somewhere, it moves there in a short  sudden  movement .
 \textit{
	\begin{itemize}
	\item The car leapt forward.
	\end{itemize}
}
\item countable noun \\
A \textbf{leap} is a large and important change, increase, or advance .
 \textit{
	\begin{itemize}
	\item The result has been a giant leap in productivity.
	\item ...the leap in the unemployed from 35,000 to 75,000.
	\item Contemporary art has taken a huge leap forward in the last five or six years.
	\end{itemize}
}
\item verb \\
If you \textbf{leap}  \textbf{to} a particular place or position, you make a large and important change, increase,
or advance.
 \textit{
	\begin{itemize}
	\item The team leapt to 12th in the table.
	\end{itemize}
}
\item verb \\
If you say that your heart  \textbf{leaps} , you mean that you experience a sudden, very strong  feeling of surprise , fear , or happiness.
 \textit{
	\begin{itemize}
	\item My heart leaped at the sight of her.
	\end{itemize}
}
\item verb \\
If you \textbf{leap}  \textbf{at} a chance or opportunity , you accept it quickly and eagerly.
 \textit{
	\begin{itemize}
	\item The post of principal of the theatre school became vacant and he leapt at the chance.
	\end{itemize}
}
\item  \\
 leaps and bounds \textit{
	\begin{itemize}
	\end{itemize}
}
\item  \\
 leap in the dark \textit{
	\begin{itemize}
	\end{itemize}
}
\end{enumerate}

\section*{consume}
{\large \color{blue}  consumes  consuming  consumed  }
\subsection*{Explain}
\begin{enumerate}
\item verb \\
If you \textbf{consume} something, you eat or drink it.
 \textit{
	\begin{itemize}
	\item Many people experienced a drop in their cholesterol levels when they consumed oat
bran.
	\item ...serving chocolate ice-creams for the children to consume in the kitchen.
	\end{itemize}
}
\item verb \\
To \textbf{consume} an amount of fuel , energy , or time means to use it up.
 \textit{
	\begin{itemize}
	\item New refrigerators consume 70 percent less electricity than older models.
	\item ...plans which will consume hours of time and deplete your cash reserves.
	\end{itemize}
}
\item verb \\
If a fire  \textbf{consumes} a building, it completely destroys it.
 \textit{
	\begin{itemize}
	\item ...the fire which consumed the dwelling.
	\end{itemize}
}
\item verb \\
If a feeling or idea  \textbf{consumes} you, it affects you very strongly indeed.
 \textit{
	\begin{itemize}
	\item The memories consumed him.
	\end{itemize}
}
\end{enumerate}

\section*{contemplate}
{\large \color{blue}  contemplates  contemplating  contemplated  }
\subsection*{Explain}
\begin{enumerate}
\item verb \\
If you \textbf{contemplate} an action, you think about whether to do it or not.
 \textit{
	\begin{itemize}
	\item For a time he contemplated a career as an army medical doctor.
	\item She contemplates leaving for the sake of the kids.
	\end{itemize}
}
\item verb \\
If you \textbf{contemplate} an idea or subject , you think about it carefully for a long time.
 \textit{
	\begin{itemize}
	\item As he lay in his hospital bed that night, he cried as he contemplated his future.
	\item That makes it difficult to contemplate the idea that the present policy may not be
sustainable.
	\end{itemize}
}
\item verb \\
If you \textbf{contemplate} something or someone, you look at them for a long time.
 \textit{
	\begin{itemize}
	\item He contemplated his hands, still frowning.
	\end{itemize}
}
\end{enumerate}

\section*{meditate}
{\large \color{blue}  meditates  meditating  meditated  }
\subsection*{Explain}
\begin{enumerate}
\item verb \\
If you \textbf{meditate on} something, you think about it very carefully and deeply for a long time.
 \textit{
	\begin{itemize}
	\item He meditated on the problem.
	\item On the day her son began school, she meditated on the uncertainties of his future.
	\end{itemize}
}
\item verb \\
If you \textbf{meditate} you remain in a silent and calm state for a period of time, as part of a religious training or so that you are more able to deal with the problems and difficulties of everyday life.
 \textit{
	\begin{itemize}
	\item I was meditating, and reached a higher state of consciousness.
	\end{itemize}
}
\end{enumerate}

\section*{contrive}
{\large \color{blue}  contrives  contriving  contrived  }
\subsection*{Explain}
\begin{enumerate}
\item verb \\
If you \textbf{contrive} an event or situation , you succeed in making it happen , often by tricking someone.
 \textit{
	\begin{itemize}
	\item The oil companies were accused of contriving a shortage of gasoline to justify price
increases.
	\end{itemize}
}
\item verb \\
If you \textbf{contrive} something such as a device or piece of equipment , you invent and construct it in a clever or unusual way.
 \textit{
	\begin{itemize}
	\item We therefore had to contrive a very large black-out curtain.
	\end{itemize}
}
\item verb \\
If you \textbf{contrive}  \textbf{to} do something difficult , you succeed in doing it.
 \textit{
	\begin{itemize}
	\item The orchestra contrived to produce some of its best playing for years.
	\end{itemize}
}
\item verb \\
When someone has done something dishonestly, you can say that they \textbf{have contrived}  \textbf{to} do it.
 \textit{
	\begin{itemize}
	\item They somehow contrived to lose tens of thousands of applications.
	\end{itemize}
}
\end{enumerate}

\section*{mount}
{\large \color{blue}  mounts  mounting  mounted  }
\subsection*{Explain}
\begin{enumerate}
\item verb \\
If you \textbf{mount} a campaign or event, you organize it and make it take place.
 \textit{
	\begin{itemize}
	\item ...a security operation mounted by the army.
	\end{itemize}
}
\item verb \\
If something \textbf{mounts} , it increases in intensity .
 \textit{
	\begin{itemize}
	\item For several hours, tension mounted.
	\item The decibel level was mounting.
	\item There was mounting concern in her voice.
	\item ...the mounting heat of the stadium.
	\end{itemize}
}
\item verb \\
If something \textbf{mounts} , it increases in quantity.
 To \textbf{mount up} means the same as to mount .
 \textit{
	\begin{itemize}
	\item The uncollected garbage mounts in city streets.
	\item He ignored his mounting debts.
	\item If you pretend your problems don't exist they will just continue to mount up.
	\item Her medical bills mounted up.
	\end{itemize}
}
\item verb \\
If you \textbf{mount} the stairs or a platform, you go up the stairs or go up onto the platform.
 \textit{
	\begin{itemize}
	\item Llewelyn was mounting the stairs up into the keep.
	\item The vehicle mounted the pavement.
	\end{itemize}
}
\item verb \\
If you \textbf{mount} a horse or cycle , you climb on to it so that you can ride it.
 \textit{
	\begin{itemize}
	\item He mounted his horse and rode away.
	\item A man in a crash helmet was mounting a motorbike.
	\item He harnessed his horse, mounted, and rode out to the beach.
	\end{itemize}
}
\item countable noun \\
A \textbf{mount} is a horse.
 \textit{
	\begin{itemize}
	\item ...the number of owners who care for older mounts.
	\end{itemize}
}
\item verb \\
If you \textbf{mount} an object \textbf{on} something, you fix it there firmly.
 \textit{
	\begin{itemize}
	\item Ella mounts the work on velour paper and makes the frame.
	\item The support for the fence is mounted on an extension to the table.
	\item ...a specially mounted horse shoe.
	\end{itemize}
}
\item verb \\
If you \textbf{mount} an exhibition or display, you organize and present it.
 \textit{
	\begin{itemize}
	\item The gallery has mounted an exhibition of art by Irish women painters.
	\end{itemize}
}
\item countable noun \\
\textbf{Mount} is used as part of the name of a mountain.
 \textit{
	\begin{itemize}
	\item ...Mount Everest.
	\end{itemize}
}
\end{enumerate}

\section*{deprive}
{\large \color{blue}  deprives  depriving  deprived  }
\subsection*{Explain}
\begin{enumerate}
\item verb \\
If you \textbf{deprive} someone \textbf{of} something that they want or need , you take it away from them, or you prevent them from having it.
 \textit{
	\begin{itemize}
	\item The disintegration of the Soviet Union deprived western intelligence agencies of
their main enemies.
	\item They've been deprived of the fuel necessary to heat their homes.
	\end{itemize}
}
\end{enumerate}

\section*{oblige}
{\large \color{blue}  obliges  obliging  obliged  }
\subsection*{Explain}
\begin{enumerate}
\item verb \\
If you \textbf{are obliged}  \textbf{to} do something, a situation , rule , or law makes it necessary for you to do that thing.
 \textit{
	\begin{itemize}
	\item The storm got worse and worse. Finally, I was obliged to abandon the car and continue
on foot.
	\item This decree obliges unions to delay strikes.
	\end{itemize}
}
\item verb \\
To \textbf{oblige} someone means to be helpful to them by doing what they have asked you to do.
 \textit{
	\begin{itemize}
	\item If you ever need help with the babysitting, I'd be glad to oblige.
	\item The gracious star was more than happy to oblige with an autograph.
	\item Mr Oakley always has been ready to oblige journalists with information.
	\end{itemize}
}
\item  \\
 much obliged/I am obliged to you/etc \textit{
	\begin{itemize}
	\end{itemize}
}
\item  \\
 would/should be obliged \textit{
	\begin{itemize}
	\end{itemize}
}
\end{enumerate}

\section*{digest}
{\large \color{blue}  digests  digesting  digested  }
\subsection*{Explain}
\begin{enumerate}
\item verb \\
When food \textbf{digests} or when you \textbf{digest} it, it passes through your body to your stomach . Your stomach removes the substances that your body needs and gets  rid of the rest .
 \textit{
	\begin{itemize}
	\item Do not undertake strenuous exercise for a few hours after a meal to allow food to
digest.
	\item She couldn't digest food properly.
	\item Nutrients from the digested food can be absorbed into the blood.
	\end{itemize}
}
\item verb \\
If you \textbf{digest} information, you think about it carefully so that you understand it.
 \textit{
	\begin{itemize}
	\item They learn well but seem to need time to digest information.
	\item She read everything, digesting every fragment of news.
	\end{itemize}
}
\item verb \\
If you \textbf{digest} some unpleasant news, you think about it until you are able to accept it and know how to deal with it.
 \textit{
	\begin{itemize}
	\item All this has upset me. I need time to digest it all.
	\end{itemize}
}
\item countable noun \\
A \textbf{digest} is a collection of pieces of writing. They are published  together in a shorter form than they were originally published.
 \textit{
	\begin{itemize}
	\item The organization publishes a regular digest of environmental statistics.
	\item ...the Middle East Economic Digest.
	\end{itemize}
}
\end{enumerate}

\section*{ornament}
{\large \color{blue}  ornaments  }
\subsection*{Explain}
\begin{enumerate}
\item countable noun \\
An \textbf{ornament} is an attractive object that you display in your home or in your garden .
 \textit{
	\begin{itemize}
	\item ...a shelf containing a few photographs and ornaments.
	\item ...Christmas tree ornaments.
	\end{itemize}
}
\item countable noun \\
Pieces of jewellery are sometimes  referred to as \textbf{ornaments} .
 \textit{
	\begin{itemize}
	\item I guessed he was the chief because he wore more gold ornaments than the others.
	\end{itemize}
}
\item uncountable noun \\
Decorations and patterns on a building or a piece of furniture can be referred to as \textbf{ornament} .
 \textit{
	\begin{itemize}
	\item ...walls of glass overlaid with ornament.
	\end{itemize}
}
\end{enumerate}

\section*{donate}
{\large \color{blue}  donates  donating  donated  }
\subsection*{Explain}
\begin{enumerate}
\item verb \\
If you \textbf{donate} something \textbf{to} a charity or other organization, you give it to them.
 \textit{
	\begin{itemize}
	\item He frequently donates large sums to charity.
	\item Others donated secondhand clothes.
	\end{itemize}
}
\item verb \\
If you \textbf{donate} your blood or a part of your body, you allow  doctors to use it to help someone who is ill .
 \textit{
	\begin{itemize}
	\item ...people who are willing to donate their organs for use after death.
	\item All donated blood is screened for HIV.
	\end{itemize}
}
\end{enumerate}

\section*{pave}
{\large \color{blue}  paves  paving  paved  }
\subsection*{Explain}
\begin{enumerate}
\item verb \\
If a road or an area of ground \textbf{has been paved} , it has been covered with flat  blocks of stone or concrete, so that it is suitable for walking or driving on.
 \textit{
	\begin{itemize}
	\item The avenue had never been paved, and deep mud made it impassable in winter.
	\end{itemize}
}
\item  \\
 pave the way for sth \textit{
	\begin{itemize}
	\end{itemize}
}
\end{enumerate}

\section*{elevate}
{\large \color{blue}  elevates  elevating  elevated  }
\subsection*{Explain}
\begin{enumerate}
\item verb \\
When someone or something achieves a more important rank or status, you can say that they \textbf{are elevated}  \textbf{to} it.
 \textit{
	\begin{itemize}
	\item He was elevated to the post of prime minister.
	\end{itemize}
}
\item verb \\
If you \textbf{elevate} something \textbf{to} a higher status, you consider it to be better or more important than it really is.
 \textit{
	\begin{itemize}
	\item Don't elevate your superiors to superstar status.
	\end{itemize}
}
\item verb \\
To \textbf{elevate} something means to increase it in amount or intensity.
 \textit{
	\begin{itemize}
	\item Emotional stress can elevate blood pressure.
	\item ...individuals who have elevated cholesterol levels.
	\end{itemize}
}
\item verb \\
If you \textbf{elevate} something, you raise it above a horizontal  level .
 \textit{
	\begin{itemize}
	\item Jack elevated the gun at the sky.
	\end{itemize}
}
\end{enumerate}

\section*{prohibit}
{\large \color{blue}  prohibits  prohibiting  prohibited  }
\subsection*{Explain}
\begin{enumerate}
\item verb \\
If a law or someone in authority \textbf{prohibits} something, they forbid it or make it illegal .
 \textit{
	\begin{itemize}
	\item ...a law that prohibits tobacco advertising in newspapers and magazines.
	\item Fishing is prohibited.
	\item Federal law prohibits foreign airlines from owning more than 25% of any U.S. airline.
	\end{itemize}
}
\end{enumerate}

\section*{eliminate}
{\large \color{blue}  eliminates  eliminating  eliminated  }
\subsection*{Explain}
\begin{enumerate}
\item verb \\
To \textbf{eliminate} something, especially something you do not want or need , means to remove it completely.
 \textit{
	\begin{itemize}
	\item The priority should be to eliminate child poverty.
	\item Academic departments are being eliminated.
	\item If you think you may be allergic to a food or drink, eliminate it from your diet.
	\end{itemize}
}
\item passive verb \\
When a person or team \textbf{is eliminated}  \textbf{from} a competition , they are defeated and so take no further part in the competition.
 \textit{
	\begin{itemize}
	\item I was eliminated from the 400 metres in the semi-finals.
	\item If you are eliminated in the show-jumping then you are out of the complete competition.
	\end{itemize}
}
\item verb \\
If someone says that they \textbf{have eliminated} an enemy , they mean that they have killed them. By using the word 'eliminate', they are trying to make the action  sound more positive than if they used the word 'kill'.
 \textit{
	\begin{itemize}
	\item He declared war on the government and urged right-wingers to eliminate their opponents.
	\item The radio station claimed that 87,000 'reactionaries' had been eliminated.
	\end{itemize}
}
\end{enumerate}

\section*{punish}
{\large \color{blue}  punishes  punishing  punished  }
\subsection*{Explain}
\begin{enumerate}
\item verb \\
To \textbf{punish} someone means to make them suffer in some way because they have done something wrong .
 \textit{
	\begin{itemize}
	\item I don't believe that George ever had to punish the children.
	\item According to present law, the authorities can only punish smugglers with small fines.
	\item Don't punish your child for being honest.
	\end{itemize}
}
\item verb \\
To \textbf{punish} a crime means to punish anyone who commits that crime.
 \textit{
	\begin{itemize}
	\item The government voted to punish corruption in sport with up to four years in jail.
	\item Such behaviour is unacceptable and will be punished.
	\end{itemize}
}
\end{enumerate}

\section*{endow}
{\large \color{blue}  endows  endowing  endowed  }
\subsection*{Explain}
\begin{enumerate}
\item verb \\
You say that someone \textbf{is endowed}  \textbf{with} a particular  desirable  ability , characteristic, or possession when they have it by chance or by birth .
 \textit{
	\begin{itemize}
	\item You are endowed with wealth, good health and a lively intellect.
	\end{itemize}
}
\item verb \\
If you \textbf{endow} something \textbf{with} a particular feature or quality, you provide it with that feature or quality.
 \textit{
	\begin{itemize}
	\item Herbs have been used for centuries to endow a whole range of foods with subtle flavours.
	\end{itemize}
}
\item verb \\
If someone \textbf{endows} an institution , scholarship , or project , they provide a large amount of money which will produce the income needed to pay for it.
 \textit{
	\begin{itemize}
	\item The ambassador has endowed a $1 million public-service fellowships program.
	\end{itemize}
}
\end{enumerate}

\section*{release}
{\large \color{blue}  releases  releasing  released  }
\subsection*{Explain}
\begin{enumerate}
\item verb \\
If a person or animal \textbf{is released} from somewhere where they have been locked up or looked after, they are set free or allowed to go.
 \textit{
	\begin{itemize}
	\item He was released from custody the next day.
	\item He is expected to be released from hospital today.
	\item Fifty-five foxes were released from a fur farm by animal rights activists.
	\item He was released on bail.
	\end{itemize}
}
\item countable noun \\
When someone is released, you refer to their \textbf{release} .
 \textit{
	\begin{itemize}
	\item He called for the immediate release of all political prisoners.
	\item ...the secret negotiations necessary to secure hostage releases.
	\item Serious complications have delayed his release from hospital.
	\end{itemize}
}
\item verb \\
If someone or something \textbf{releases} you \textbf{from} a duty, task , or feeling, they free you from it.
 \textbf{Release} is also a noun .
 \textit{
	\begin{itemize}
	\item The document released Mr Jackson from his obligations under the contract.
	\item This releases the teacher to work with individuals who are having extreme difficulty.
	\item ...release from stored tensions, traumas and grief.
	\item They look on life at college as a blessed release from the obligation to work.
	\end{itemize}
}
\item verb \\
To \textbf{release} feelings or abilities means to allow them to be expressed.
 \textbf{Release} is also a noun.
 \textit{
	\begin{itemize}
	\item Becoming your own person releases your creativity.
	\item I personally don't want to release my anger on anyone else.
	\item Humour is wonderful for releasing tension.
	\item She felt the sudden sweet release of her own tears.
	\end{itemize}
}
\item verb \\
If someone in authority \textbf{releases} something such as a document or information, they make it available.
 \textbf{Release} is also a noun.
 \textit{
	\begin{itemize}
	\item They're not releasing any more details yet.
	\item Figures released yesterday show retail sales were down in March.
	\item Action had been taken to speed up the release of cheques.
	\end{itemize}
}
\item verb \\
If you \textbf{release} someone or something, you stop holding them.
 \textit{
	\begin{itemize}
	\item He stopped and faced her, releasing her wrist.
	\item ...the twisting action before a bowler releases the ball.
	\end{itemize}
}
\item verb \\
If you \textbf{release} a device, you move it so that it stops holding something.
 \textit{
	\begin{itemize}
	\item Wade released the hand brake and pulled away from the curb.
	\end{itemize}
}
\item verb \\
If something \textbf{releases} gas, heat, or a substance, it causes it to leave its container or the substance that
it was part of and enter the surrounding atmosphere or area.
 \textbf{Release} is also a noun.
 \textit{
	\begin{itemize}
	\item ...a weapon which releases toxic nerve gas.
	\item The contraction of muscles uses energy and releases heat.
	\item A ceramic water holder gradually releases water into the plants.
	\item Under the agreement, releases of cancer-causing chemicals will be cut by about 80
per cent.
	\end{itemize}
}
\item verb \\
When an entertainer or company \textbf{releases} a new CD, video , or film, it becomes available so that people can buy it or see it.
 \textit{
	\begin{itemize}
	\item He is releasing an album of love songs.
	\end{itemize}
}
\item countable noun \\
A new \textbf{release} is a new CD, video, or film that has just become available for people to buy or see.
 \textit{
	\begin{itemize}
	\item Which of the new releases do you think are really good?
	\end{itemize}
}
\item uncountable noun \\
If a film or video is \textbf{on release} or \textbf{on general release} , it is available for people to see in public cinemas or for people to buy.
 \textit{
	\begin{itemize}
	\item The film goes on release on February 1st.
	\end{itemize}
}
\end{enumerate}

\section*{evade}
{\large \color{blue}  evades  evading  evaded  }
\subsection*{Explain}
\begin{enumerate}
\item verb \\
If you \textbf{evade} something, you find a way of not doing something that you really ought to do.
 \textit{
	\begin{itemize}
	\item By his own admission, he evaded taxes as a Florida real-estate speculator.
	\item Delegates accused them of evading responsibility for recent failures.
	\end{itemize}
}
\item verb \\
If you \textbf{evade} a question or a topic , you avoid talking about it or dealing with it.
 \textit{
	\begin{itemize}
	\item The Minister denied he was evading the question.
	\item Too many companies, she says, are evading the issue.
	\end{itemize}
}
\item verb \\
If you \textbf{evade} someone or something, you move so that you can avoid meeting them or avoid being touched or hit .
 \textit{
	\begin{itemize}
	\item Under the pretence of lighting a candle, she evades him and disappears.
	\item She turned and gazed at the river, evading his eyes.
	\item He managed to evade capture because of the breakdown of a police computer.
	\end{itemize}
}
\item verb \\
If something such as success , glory , or love  \textbf{evades} you, you do not manage to have it.
 \textit{
	\begin{itemize}
	\item Happiness, which had been so elusive in Henry's life, still evaded him.
	\item When she sat down at her desk she found that the words evaded her.
	\end{itemize}
}
\end{enumerate}

\section*{remain}
{\large \color{blue}  remains  remaining  remained  }
\subsection*{Explain}
\begin{enumerate}
\item link verb \\
If someone or something \textbf{remains} in a particular state or condition, they stay in that state or condition and do not
change.
 \textit{
	\begin{itemize}
	\item The three men remained silent.
	\item The situation remains tense.
	\item The government remained in control.
	\item He remained a formidable opponent.
	\item It remains possible that bad weather could tear more holes in the tanker's hull.
	\end{itemize}
}
\item verb \\
If you \textbf{remain} in a place, you stay there and do not move away .
 \textit{
	\begin{itemize}
	\item He will have to remain in hospital for at least 10 days.
	\item From time to time, James remained at home with his family.
	\end{itemize}
}
\item verb \\
You can say that something \textbf{remains} when it still  exists .
 \textit{
	\begin{itemize}
	\item Many of the differences in everyday life remain.
	\item The wider problem remains.
	\item There remains deep mistrust of his government.
	\end{itemize}
}
\item link verb \\
If something \textbf{remains}  \textbf{to be} done, it has not yet been done and still needs to be done.
 \textit{
	\begin{itemize}
	\item Major questions remain to be answered about his work.
	\item Huge amounts of weapons remain to be collected.
	\end{itemize}
}
\item plural noun \\
\textbf{The}  \textbf{remains}  \textbf{of} something are the parts of it that are left after most of it has been taken away
or destroyed .
 \textit{
	\begin{itemize}
	\item They were tidying up the remains of their picnic.
	\item ...the charred remains of a tank.
	\item ...the remains of an ancient mosque.
	\end{itemize}
}
\item plural noun \\
The \textbf{remains} of a person or animal are the parts of their body that are left after they have died , sometimes after they have been dead for a long time.
 \textit{
	\begin{itemize}
	\item The unrecognizable remains of a man had been found.
	\item More human remains have been unearthed in the north of the country.
	\end{itemize}
}
\item plural noun \\
Historical  \textbf{remains} are things that have been found from an earlier period of history , usually buried in the ground , for example parts of buildings and pieces of pottery .
 \textit{
	\begin{itemize}
	\item There are Roman remains all around us.
	\end{itemize}
}
\item link verb \\
You can use \textbf{remain} in expressions such as \textbf{the fact remains that} or \textbf{the question remains whether} to introduce and emphasize something that you want to talk about.
 \textit{
	\begin{itemize}
	\item The fact remains that inflation is unacceptably high.
	\item The question remains whether he was fully aware of the claims.
	\end{itemize}
}
\item  \\
 remains to be seen \textit{
	\begin{itemize}
	\end{itemize}
}
\end{enumerate}

\section*{evoke}
{\large \color{blue}  evokes  evoking  evoked  }
\subsection*{Explain}
\begin{enumerate}
\item verb \\
To \textbf{evoke} a particular memory, idea , emotion , or response means to cause it to occur.
 \textit{
	\begin{itemize}
	\item ...the scene evoking memories of those old movies.
	\item A sense of period was evoked by complementing pictures with appropriate furniture.
	\end{itemize}
}
\end{enumerate}

\section*{resemble}
{\large \color{blue}  resembles  resembling  resembled  }
\subsection*{Explain}
\begin{enumerate}
\item verb \\
If one thing or person \textbf{resembles} another, they are similar to each other.
 \textit{
	\begin{itemize}
	\item The fish had white, firm flesh that resembled chicken.
	\item She so resembles her mother.
	\end{itemize}
}
\end{enumerate}

\section*{foster}
{\large \color{blue}  fosters  fostering  fostered  }
\subsection*{Explain}
\begin{enumerate}
\item  \\
 foster parent \textit{
	\begin{itemize}
	\end{itemize}
}
\item verb \\
If you \textbf{foster} a child, you take it into your family for a period of time, without becoming its legal parent.
 \textit{
	\begin{itemize}
	\item She has since gone on to find happiness by fostering more than 100 children.
	\end{itemize}
}
\item verb \\
To \textbf{foster} something such as an activity or idea means to help it to develop.
 \textit{
	\begin{itemize}
	\item Developed countries should foster global economic growth to help new democracies.
	\item Its cash crisis has been fostered by declining property values.
	\end{itemize}
}
\end{enumerate}

\section*{resign}
{\large \color{blue}  resigns  resigning  resigned  }
\subsection*{Explain}
\begin{enumerate}
\item verb \\
If you \textbf{resign} from a job or position, you formally announce that you are leaving it.
 \textit{
	\begin{itemize}
	\item A hospital administrator has resigned over claims he lied to get the job.
	\item Mr Robb resigned his position last month.
	\end{itemize}
}
\item verb \\
If you \textbf{resign}  \textbf{yourself to} an unpleasant  situation or fact , you accept it because you realize that you cannot change it.
 \textit{
	\begin{itemize}
	\item Pat and I resigned ourselves to yet another summer without a boat.
	\item He had resigned himself to watching the European Championships on television.
	\end{itemize}
}
\end{enumerate}

\section*{frustrate}
{\large \color{blue}  frustrates  frustrating  frustrated  }
\subsection*{Explain}
\begin{enumerate}
\item verb \\
If something \textbf{frustrates} you, it upsets or angers you because you are unable to do anything about the problems it creates .
 \textit{
	\begin{itemize}
	\item These questions frustrated me.
	\item Doesn't it frustrate you that audiences in the theatre are so restricted?
	\end{itemize}
}
\item verb \\
If someone or something \textbf{frustrates} a plan or attempt to do something, they prevent it from succeeding .
 \textit{
	\begin{itemize}
	\item The government has frustrated his efforts to gain work permits for his foreign staff.
	\item ...her frustrated attempt to become governor.
	\end{itemize}
}
\end{enumerate}

\section*{ruin}
{\large \color{blue}  ruins  ruining  ruined  }
\subsection*{Explain}
\begin{enumerate}
\item verb \\
To \textbf{ruin} something means to severely harm , damage, or spoil it.
 \textit{
	\begin{itemize}
	\item Olivia was ruining her health through worry.
	\item Entire villages have been washed away. Roads and bridges have been destroyed and
crops ruined.
	\end{itemize}
}
\item verb \\
To \textbf{ruin} someone means to cause them to no longer have any money .
 \textit{
	\begin{itemize}
	\item She accused him of ruining her financially with his taste for the high life.
	\end{itemize}
}
\item uncountable noun \\
\textbf{Ruin} is the state of no longer having any money.
 \textit{
	\begin{itemize}
	\item The farmers say recent inflation has driven them to the brink of ruin.
	\end{itemize}
}
\item uncountable noun \\
\textbf{Ruin} is the state of being severely damaged or spoiled, or the process of reaching this state.
 \textit{
	\begin{itemize}
	\item The vineyards were falling into ruin.
	\item She wasn't going to let her plans go to ruin.
	\end{itemize}
}
\item plural noun \\
\textbf{The ruins of} something are the parts of it that remain after it has been severely damaged or weakened .
 \textit{
	\begin{itemize}
	\item The new republic he helped to build emerged from the ruins of a great empire.
	\item He stood very still, staring in at the ruins of his work.
	\end{itemize}
}
\item countable noun \\
\textbf{The}  \textbf{ruins} of a building are the parts of it that remain after the rest has fallen down or been destroyed.
 \textit{
	\begin{itemize}
	\item One dead child was found in the ruins almost two hours after the explosion.
	\item There's only the mountain in this direction, and higher up an old ruin, an abandoned
castle.
	\end{itemize}
}
\item  \\
 in ruins \textit{
	\begin{itemize}
	\end{itemize}
}
\item  \\
 in ruins \textit{
	\begin{itemize}
	\end{itemize}
}
\end{enumerate}

\section*{hurl}
{\large \color{blue}  hurls  hurling  hurled  }
\subsection*{Explain}
\begin{enumerate}
\item verb \\
If you \textbf{hurl} something, you throw it violently and with a lot of force.
 \textit{
	\begin{itemize}
	\item Groups of angry youths hurled stones at police.
	\item One prisoner set fire to rags and hurled them into the courtyard.
	\item Simon caught the grenade and hurled it back.
	\item Gangs rioted last night, breaking storefront windows and hurling rocks and bottles.
	\end{itemize}
}
\item verb \\
If you \textbf{hurl}  abuse or insults  \textbf{at} someone, you shout insults at them aggressively.
 \textit{
	\begin{itemize}
	\item How would you handle being locked in the back of a cab while the driver hurled abuse
at you?
	\end{itemize}
}
\end{enumerate}

\section*{shine}
{\large \color{blue}  shines  shining  shined  shone  }
\subsection*{Explain}
\begin{enumerate}
\item verb \\
When the sun or a light \textbf{shines} , it gives out bright light.
 \textit{
	\begin{itemize}
	\item It is a mild morning and the sun is shining.
	\item A few scattered lights shone on the horizon.
	\end{itemize}
}
\item verb \\
If you \textbf{shine} a torch or other light somewhere , you point it there, so that you can see something when it is dark .
 \textit{
	\begin{itemize}
	\item One of the men shone a torch in his face.
	\item The container is invisible until you shine an ultraviolet light on it.
	\item The man walked slowly towards her, shining the flashlight.
	\end{itemize}
}
\item verb \\
Something that \textbf{shines} is very bright and clear because it is reflecting light.
 \textit{
	\begin{itemize}
	\item Her blue eyes shone and caught the light.
	\item ...a pair of patent shoes that shone like mirrors.
	\item ...shining aluminum machines.
	\end{itemize}
}
\item singular noun \\
Something that has a \textbf{shine} is bright and clear because it is reflecting light.
 \textit{
	\begin{itemize}
	\item This gel gives a beautiful shine to the hair.
	\item The wood had been recently polished to bring back the shine.
	\end{itemize}
}
\item verb \\
If you \textbf{shine} a wooden , leather , or metal  object , you make it bright by rubbing or polishing it.
 \textit{
	\begin{itemize}
	\item Let him dust and shine the furniture.
	\item His high black boots had been shined to a gleaming finish.
	\end{itemize}
}
\item verb \\
Someone who \textbf{shines} at a skill or activity does it extremely  well .
 \textit{
	\begin{itemize}
	\item Did you shine at school?
	\item He failed to shine academically.
	\end{itemize}
}
\item  \\
 take a shine to sb \textit{
	\begin{itemize}
	\end{itemize}
}
\end{enumerate}

\section*{inherit}
{\large \color{blue}  inherits  inheriting  inherited  }
\subsection*{Explain}
\begin{enumerate}
\item verb \\
If you \textbf{inherit}  money or property, you receive it from someone who has died .
 \textit{
	\begin{itemize}
	\item He has no son to inherit his land.
	\item ...paintings that he inherited from his father.
	\item ...people with inherited wealth.
	\end{itemize}
}
\item verb \\
If you \textbf{inherit} something such as a task , problem , or attitude, you get it from the people who used to have it, for example because you have taken over their job or been influenced by them.
 \textit{
	\begin{itemize}
	\item The government inherited an impossible situation from its predecessors.
	\item Our legal system inherited laws from the English system.
	\item ...the inherited wisdoms contained in its social hierarchy.
	\end{itemize}
}
\item verb \\
If you \textbf{inherit} a characteristic or quality, you are born with it, because your parents or ancestors  also had it.
 \textit{
	\begin{itemize}
	\item We inherit many of our physical characteristics from our parents.
	\item Her children have inherited her love of sport.
	\item Stammering is probably an inherited defect.
	\end{itemize}
}
\end{enumerate}

\section*{simplify}
{\large \color{blue}  simplifies  simplifying  simplified  }
\subsection*{Explain}
\begin{enumerate}
\item verb \\
If you \textbf{simplify} something, you make it easier to understand or you remove the things which make it complex .
 \textit{
	\begin{itemize}
	\item ...a plan to simplify the complex social security system.
	\item He reduced his needs to the minimum by simplifying his life.
	\end{itemize}
}
\end{enumerate}

\section*{inspect}
{\large \color{blue}  inspects  inspecting  inspected  }
\subsection*{Explain}
\begin{enumerate}
\item verb \\
If you \textbf{inspect} something, you look at every part of it carefully in order to find out about it or check that it is all right.
 \textit{
	\begin{itemize}
	\item Elaine went outside to inspect the playing field.
	\item I got out of the car to inspect the damage.
	\end{itemize}
}
\item verb \\
When an official  \textbf{inspects} a place or a group of people, they visit it and check it carefully, for example in order to find out whether regulations are being obeyed .
 \textit{
	\begin{itemize}
	\item The Public Utilities Commission inspects us once a year.
	\item Each hotel is inspected and, if it fulfils certain criteria, is recommended.
	\end{itemize}
}
\end{enumerate}

\section*{slip}
{\large \color{blue}  slips  slipping  slipped  }
\subsection*{Explain}
\begin{enumerate}
\item verb \\
If you \textbf{slip} , you accidentally slide and lose your balance.
 \textit{
	\begin{itemize}
	\item He had slipped on an icy pavement.
	\item Be careful not to slip.
	\end{itemize}
}
\item verb \\
If something \textbf{slips} , it slides out of place or out of your hand.
 \textit{
	\begin{itemize}
	\item His glasses had slipped.
	\item The hammer slipped out of her grasp.
	\end{itemize}
}
\item verb \\
If you \textbf{slip}  somewhere , you go there quickly and quietly .
 \textit{
	\begin{itemize}
	\item Amy slipped downstairs and out of the house.
	\item She slipped into the driving seat and closed the door.
	\end{itemize}
}
\item verb \\
If you \textbf{slip} something somewhere, you put it there quickly in a way that does not attract attention.
 \textit{
	\begin{itemize}
	\item I slipped a note under Louise's door.
	\item He found a coin in his pocket and slipped it into her collecting tin.
	\item Just slip in a piece of paper.
	\end{itemize}
}
\item verb \\
If you \textbf{slip} something \textbf{to} someone, you give it to them secretly.
 \textit{
	\begin{itemize}
	\item Robert had slipped her a note in school.
	\item She looked round before pulling out a package and slipping it to the man.
	\end{itemize}
}
\item verb \\
To \textbf{slip into} a particular state or situation means to pass gradually into it, in a way that is
 hardly  noticed .
 \textit{
	\begin{itemize}
	\item It amazed him how easily one could slip into a routine.
	\item There was a 50-50 chance that the economy could slip back into recession.
	\end{itemize}
}
\item verb \\
If something \textbf{slips}  \textbf{to} a lower level or standard, it falls to that level or standard.
 \textbf{Slip} is also a noun.
 \textit{
	\begin{itemize}
	\item Shares slipped to 117p.
	\item The club had slipped to the bottom of Division Four.
	\item In June, producer prices slipped 0.1% from May.
	\item Overall business activity is slipping.
	\item ...a slip in consumer confidence.
	\end{itemize}
}
\item verb \\
If you \textbf{slip}  \textbf{into} or \textbf{out of} clothes or shoes, you put them on or take them off quickly and easily.
 \textit{
	\begin{itemize}
	\item She slipped out of the jacket and tossed it on the couch.
	\item I slipped off my woollen gloves.
	\end{itemize}
}
\item countable noun \\
A \textbf{slip} is a small or unimportant mistake.
 \textit{
	\begin{itemize}
	\item We must be well prepared, there must be no slips.
	\end{itemize}
}
\item countable noun \\
A \textbf{slip}  \textbf{of} paper is a small piece of paper.
 \textit{
	\begin{itemize}
	\item ...little slips of paper he had torn from a notebook.
	\item I put her name on the slip.
	\item ...credit card slips.
	\end{itemize}
}
\item countable noun \\
A \textbf{slip} is a thin piece of clothing that a woman wears under her dress or skirt .
 \textit{
	\begin{itemize}
	\end{itemize}
}
\item countable noun \\
If you refer to someone as a \textbf{slip of a} girl or a \textbf{slip of a}  boy , you mean they are small, thin, and young.
 \textit{
	\begin{itemize}
	\item He's a mere slip of a lad compared to his brother.
	\item She was just a slip of a thing.
	\end{itemize}
}
\item  \\
 give sb the slip \textit{
	\begin{itemize}
	\end{itemize}
}
\item  \\
 let slip \textit{
	\begin{itemize}
	\end{itemize}
}
\item  \\
 slip your mind \textit{
	\begin{itemize}
	\end{itemize}
}
\end{enumerate}

\section*{kidnap}
{\large \color{blue}  kidnaps  kidnapping  kidnapped  }
\subsection*{Explain}
\begin{enumerate}
\item verb \\
To \textbf{kidnap} someone is to take them away illegally and by force , and usually to hold them prisoner in order to demand something from their family, employer , or government .
 \textit{
	\begin{itemize}
	\item Police in Brazil uncovered a plot to kidnap him.
	\item They were intelligent and educated, yet they chose to kidnap and kill.
	\item The kidnapped man was said to have been seized by five people.
	\end{itemize}
}
\item variable noun \\
\textbf{Kidnap} or a \textbf{kidnap} is the crime of taking someone away by force.
 \textit{
	\begin{itemize}
	\item Stewart denies attempted murder and kidnap.
	\item He was charged with the kidnap of a 25 year-old woman.
	\end{itemize}
}
\end{enumerate}

\section*{splash}
{\large \color{blue}  splashes  splashing  splashed  }
\subsection*{Explain}
\begin{enumerate}
\item verb \\
If you \textbf{splash} about or \textbf{splash} around in water, you hit or disturb the water in a noisy way, causing some of it to fly up into the air .
 \textit{
	\begin{itemize}
	\item A lot of people were in the water, swimming or simply splashing about.
	\item She could hear the voices of her friends as they splashed in a nearby rock pool.
	\item The gliders and their pilots splashed into the lake and had to be fished out.
	\end{itemize}
}
\item verb \\
If you \textbf{splash} a liquid somewhere or if it \textbf{splashes} , it hits someone or something and scatters in a lot of small drops .
 \textit{
	\begin{itemize}
	\item He closed his eyes tight, and splashed the water on his face.
	\item A little wave, the first of many, splashed in my face.
	\item Coffee splashed the carpet.
	\item Lorries rumbled past them, splashing them with filthy water from the potholes in
the road.
	\item He heard the sounds of splashing water and glanced at the door to the bathroom.
	\end{itemize}
}
\item singular noun \\
A \textbf{splash} is the sound made when something hits water or falls into it.
 \textit{
	\begin{itemize}
	\item There was a splash and something fell clumsily into the water.
	\end{itemize}
}
\item countable noun \\
A \textbf{splash} of a liquid is a small quantity of it that falls on something or is added to something.
 \textit{
	\begin{itemize}
	\item Wallcoverings and floors should be able to withstand steam and splashes.
	\item Add a splash of lemon juice to flavor the butter.
	\end{itemize}
}
\item countable noun \\
A \textbf{splash}  \textbf{of}  colour is an area of a bright colour which contrasts strongly with the colours around it.
 \textit{
	\begin{itemize}
	\item Anne left the walls white, but added splashes of colour with the paintings.
	\end{itemize}
}
\item verb \\
If a magazine or newspaper \textbf{splashes} a story, it prints it in such a way that it is very noticeable .
 \textit{
	\begin{itemize}
	\item The newspapers splashed the story all over their front pages.
	\item A picture of his girlfriend Sheryl had been splashed in the previous weekend's tabloids.
	\end{itemize}
}
\item  \\
 make a splash \textit{
	\begin{itemize}
	\end{itemize}
}
\end{enumerate}

\section*{lend}
{\large \color{blue}  lends  lending  lent  }
\subsection*{Explain}
\begin{enumerate}
\item verb \\
When people or organizations such as banks  \textbf{lend} you money, they give it to you and you agree to pay it back at a future  date , often with an extra  amount as interest.
 \textit{
	\begin{itemize}
	\item The bank is reassessing its criteria for lending money.
	\item I had to lend him ten pounds to take his children to the pictures.
	\item ...financial de-regulation that led to institutions being more willing to lend.
	\end{itemize}
}
\item verb \\
If you \textbf{lend} something that you own, you allow someone to have it or use it for a period of time.
 \textit{
	\begin{itemize}
	\item Will you lend me your jacket for a little while?
	\item He had lent the bungalow to the Conrads for a couple of weeks.
	\end{itemize}
}
\item verb \\
If you \textbf{lend} your support \textbf{to} someone or something, you help them with what they are doing or with a problem that they have.
 \textit{
	\begin{itemize}
	\item He was approached by the organisers to lend support to a benefit concert.
	\item Stipe attended yesterday's news conference to lend his support.
	\end{itemize}
}
\item verb \\
If something \textbf{lends}  \textbf{itself to} a particular  activity or result , it is easy for it to be used for that activity or to achieve that result.
 \textit{
	\begin{itemize}
	\item The room lends itself well to summer eating with its light, airy atmosphere.
	\end{itemize}
}
\item verb \\
If something \textbf{lends} a particular quality \textbf{to} something else, it adds that quality to it.
 \textit{
	\begin{itemize}
	\item Enthusiastic applause lent a sense of occasion to the proceedings.
	\item A more relaxed regime and regular work lends the inmates a dignity not seen in other
prisons.
	\end{itemize}
}
\end{enumerate}

\section*{swing}
{\large \color{blue}  swings  swinging  swung  }
\subsection*{Explain}
\begin{enumerate}
\item verb \\
If something \textbf{swings} or if you \textbf{swing} it, it moves repeatedly backwards and forwards or from side to side from a fixed point.
 \textbf{Swing} is also a noun .
 \textit{
	\begin{itemize}
	\item The sail of the little boat swung crazily from one side to the other.
	\item She was swinging a bag containing a new dress.
	\item Ian sat on the end of the table, one leg swinging.
	\item ...a woman walking with a slight swing to her hips.
	\end{itemize}
}
\item verb \\
If something \textbf{swings} in a particular direction or if you \textbf{swing} it in that direction, it moves in that direction with a smooth , curving movement.
 \textbf{Swing} is also a noun.
 \textit{
	\begin{itemize}
	\item The torchlight swung across the little beach and out over the water, searching.
	\item The canoe found the current and swung around.
	\item Roy swung his legs carefully off the couch and sat up.
	\item When he's not on the tennis court, you'll find him practising his golf swing.
	\end{itemize}
}
\item verb \\
If a vehicle \textbf{swings} in a particular direction, or if the driver  \textbf{swings} it in a particular direction, they turn suddenly in that direction.
 \textit{
	\begin{itemize}
	\item Joanna swung back on to the main approach and headed for the airport.
	\item The tyres dug into the grit as he swung the car off the road.
	\end{itemize}
}
\item verb \\
If someone \textbf{swings}  \textbf{around} , they turn around quickly, usually because they are surprised .
 \textit{
	\begin{itemize}
	\item She swung around to him, spilling her tea without noticing it.
	\end{itemize}
}
\item verb \\
If you \textbf{swing}  \textbf{at} a person or thing, you try to hit them with your arm or with something that you are holding.
 \textbf{Swing} is also a noun.
 \textit{
	\begin{itemize}
	\item Blanche swung at her but she moved her head back and Blanche missed.
	\item I picked up his baseball bat and swung at the man's head.
	\item I often want to take a swing at someone to relieve my feelings.
	\end{itemize}
}
\item countable noun \\
A \textbf{swing} is a seat hanging by two ropes or chains from a metal frame or from the branch of a tree. You can sit on the seat and move forwards and backwards
through the air.
 \textit{
	\begin{itemize}
	\end{itemize}
}
\item uncountable noun \\
\textbf{Swing} is a style of jazz dance music that was popular in the 1930's. It was played by big
bands.
 \textit{
	\begin{itemize}
	\end{itemize}
}
\item countable noun \\
A \textbf{swing} in people's opinions, attitudes , or feelings is a change in them, especially a sudden or big change.
 \textit{
	\begin{itemize}
	\item There was a massive swing away from the governing party in the election.
	\item Educational practice is liable to sudden swings and changes.
	\item They suffer from violent mood swings.
	\end{itemize}
}
\item verb \\
If people's opinions, attitudes, or feelings \textbf{swing} , they change, especially in a sudden or extreme way.
 \textit{
	\begin{itemize}
	\item In two years' time there is a presidential election, and the voters could swing again.
	\item The mood amongst Tory MPs seems to be swinging away from their leader.
	\end{itemize}
}
\item countable noun \\
If someone such as a politician makes a \textbf{swing}  \textbf{through} a particular country or area, they go on a quick  trip through it, visiting a number of different places.
 \textit{
	\begin{itemize}
	\item ...a campaign swing through South Dakota and Texas.
	\end{itemize}
}
\item  \\
 in full swing \textit{
	\begin{itemize}
	\end{itemize}
}
\item  \\
 get into the swing of \textit{
	\begin{itemize}
	\end{itemize}
}
\item  \\
 go with a swing \textit{
	\begin{itemize}
	\end{itemize}
}
\item  \\
 swings and roundabouts \textit{
	\begin{itemize}
	\end{itemize}
}
\end{enumerate}

\section*{propel}
{\large \color{blue}  propels  propelling  propelled  }
\subsection*{Explain}
\begin{enumerate}
\item verb \\
To \textbf{propel} something in a particular direction means to cause it to move in that direction.
 \textbf{-propelled}  combines with nouns to form adjectives which indicate how something, especially a weapon , is propelled.
 \textit{
	\begin{itemize}
	\item Floor the accelerator pedal and you are propelled forward in a wave of power.
	\item ...rocket-propelled grenades.
	\item ...the first jet-propelled aeroplane.
	\end{itemize}
}
\item verb \\
If something \textbf{propels} you \textbf{into} a particular activity, it causes you to do it.
 \textit{
	\begin{itemize}
	\item It was a shooting star that propelled me into astronomy in the first place.
	\item He is propelled by both guilt and the need to avenge his father.
	\end{itemize}
}
\end{enumerate}

\section*{say}
{\large \color{blue}  says  saying  said  }
\subsection*{Explain}
\begin{enumerate}
\item verb \\
When you \textbf{say} something, you speak words.
 \textit{
	\begin{itemize}
	\item 'I'm sorry,' he said.
	\item She said they were very impressed.
	\item Forty-one people are said to have been seriously hurt.
	\item I packed and said goodbye to Charlie.
	\item I hope you didn't say anything about Gretchen.
	\item You didn't say much when you telephoned.
	\item Did he say where he was going?
	\item It doesn't sound exactly orthodox, if I may say so.
	\end{itemize}
}
\item verb \\
You use \textbf{say} in expressions such as \textbf{I would just like to say} to introduce what you are actually  saying , or to indicate that you are expressing an opinion or admitting a fact. If you state that you \textbf{can't say} something or you \textbf{wouldn't say} something, you are indicating in a polite or indirect  way that it is not the case.
 \textit{
	\begin{itemize}
	\item I would just like to say that this is the most hypocritical thing I have ever heard
in my life.
	\item I have to say I didn't even know Fox Lane Police Station existed till about four
or five years ago.
	\item I must say that rather shocked me, too.
	\item Well, I can't say I'm sorry to hear that.
	\end{itemize}
}
\item verb \\
You can  mention the contents of a piece of writing by mentioning what it \textbf{says} or what someone \textbf{says} in it.
 \textit{
	\begin{itemize}
	\item The report says there is widespread and routine torture of political prisoners in
the country.
	\item Auntie Winnie wrote back saying Mam wasn't well enough to write.
	\item You can't have one without the other, as the song says.
	\item 'Highly inflammable,' it says on the spare canister.
	\item It is a pervasively religious school and believes whatever the Bible says is so.
	\end{itemize}
}
\item verb \\
If you \textbf{say} something \textbf{to yourself} , you think it.
 \textit{
	\begin{itemize}
	\item Perhaps I'm still dreaming, I said to myself.
	\item 'Keep your temper,' he said to himself.
	\end{itemize}
}
\item singular noun \\
If you have \textbf{a}  \textbf{say}  \textbf{in} something, you have the right to give your opinion and influence decisions relating to it.
 \textit{
	\begin{itemize}
	\item You can get married at sixteen, and yet you haven't got a say in the running of the
country.
	\item The students wanted more say in the government of the university.
	\end{itemize}
}
\item verb \\
You indicate the information  given by something such as a clock , dial , or map by mentioning what it \textbf{says} .
 \textit{
	\begin{itemize}
	\item The clock said four minutes past eleven.
	\item The map says there's six of them.
	\end{itemize}
}
\item verb \\
If something \textbf{says} something \textbf{about} a person, situation , or thing, it gives important information about them.
 \textit{
	\begin{itemize}
	\item I think that says a lot about how well she is playing.
	\item The appearance of the place says something about the importance of the project.
	\end{itemize}
}
\item verb \\
If something \textbf{says} a lot  \textbf{for} a person or thing, it shows that this person or thing is very good or has a lot of good qualities .
 \textit{
	\begin{itemize}
	\item It says a lot for him that he has raised his game to the level required.
	\item It says much for the author's skill that the book is sad, but never depressing.
	\end{itemize}
}
\item verb \\
You use \textbf{say} in expressions such as \textbf{I'll say that for them} and \textbf{you can say this for them} after or before you mention a good quality that someone has, usually when you think
they do not have many good qualities.
 \textit{
	\begin{itemize}
	\item He's usually smartly-dressed, I'll say that for him.
	\item At the very least, he is devastatingly sure of himself, you can say that.
	\end{itemize}
}
\item verb \\
You can use \textbf{say} when you want to discuss something that might possibly happen or be true .
 \textit{
	\begin{itemize}
	\item Say you could change anything about the world we live in, what would it be?
	\end{itemize}
}
\item phrase \\
You can use \textbf{say} or \textbf{let's say} when you mention something as an example.
 \textit{
	\begin{itemize}
	\item If funds start arriving in January, construction can begin in, say, June.
	\item Someone with, say, between 300 and 500 acres could be losing thousands of pounds
a year.
	\end{itemize}
}
\item exclamation \\
\textbf{Say} is used to attract someone's attention or to express surprise, pleasure , or admiration .
 \textit{
	\begin{itemize}
	\item Say, how would you like to have dinner one night, just you and me?
	\end{itemize}
}
\item  \\
 say it all \textit{
	\begin{itemize}
	\end{itemize}
}
\item  \\
 you don't say \textit{
	\begin{itemize}
	\end{itemize}
}
\item  \\
 to be said for sth \textit{
	\begin{itemize}
	\end{itemize}
}
\item  \\
 not have much to say for oneself \textit{
	\begin{itemize}
	\end{itemize}
}
\item  \\
 what does someone have to say for themselves \textit{
	\begin{itemize}
	\end{itemize}
}
\item  \\
 goes without saying \textit{
	\begin{itemize}
	\end{itemize}
}
\item  \\
 have one's say \textit{
	\begin{itemize}
	\end{itemize}
}
\item  \\
 say what you like about sth \textit{
	\begin{itemize}
	\end{itemize}
}
\item  \\
 I wouldn't say no \textit{
	\begin{itemize}
	\end{itemize}
}
\item  \\
 not to say \textit{
	\begin{itemize}
	\end{itemize}
}
\item  \\
 to say nothing of \textit{
	\begin{itemize}
	\end{itemize}
}
\item  \\
 shall I say \textit{
	\begin{itemize}
	\end{itemize}
}
\item  \\
 that is to say \textit{
	\begin{itemize}
	\end{itemize}
}
\item  \\
 you can say that again \textit{
	\begin{itemize}
	\end{itemize}
}
\end{enumerate}

\section*{touch}
{\large \color{blue}  touches  touching  touched  }
\subsection*{Explain}
\begin{enumerate}
\item verb \\
If you \textbf{touch} something, you put your hand onto it in order to feel it or to make contact with
it.
 \textbf{Touch} is also a noun.
 \textit{
	\begin{itemize}
	\item Her tiny hands gently touched my face.
	\item Don't touch that dial.
	\item She reached down, touching her toes with opposite hands.
	\item The virus is not passed on through touching or shaking hands.
	\item Sometimes even a light touch on the face is enough to trigger off this pain.
	\end{itemize}
}
\item verb \\
If two things \textbf{are touching} , or if one thing \textbf{touches} another, or if you \textbf{touch} two things, their surfaces come into contact with each other.
 \textit{
	\begin{itemize}
	\item Their knees were touching.
	\item A cyclist crashed when he touched wheels with another rider.
	\item If my arm touches the wall, it has to be washed again.
	\item In some countries people stand close enough to touch elbows.
	\item He touched the cow's side with his stick.
	\end{itemize}
}
\item uncountable noun \\
Your sense of \textbf{touch} is your ability to tell what something is like when you feel it with your hands.
 \textit{
	\begin{itemize}
	\item The evidence suggests that our sense of touch is programmed to diminish with age.
	\item ...boys and girls who are blind and who want to be able to read and write by touch.
	\end{itemize}
}
\item verb \\
To \textbf{touch} something means to strike it, usually quite gently.
 \textit{
	\begin{itemize}
	\item He scored the first time he touched the ball.
	\item As the aeroplane went down the runway, the wing touched a pile of rubble.
	\end{itemize}
}
\item verb \\
If something \textbf{has} not \textbf{been touched} , nobody has dealt with it or taken care of it.
 \textit{
	\begin{itemize}
	\item When John began to restore the house, nothing had been touched for 40 years.
	\end{itemize}
}
\item verb \\
If you say that you did not \textbf{touch} someone or something, you are emphasizing that you did not attack, harm , or destroy them, especially when you have been accused of doing so.
 \textit{
	\begin{itemize}
	\item Pearce remained adamant, saying 'I didn't touch him'.
	\item I was in the garden. I never touched the sandwiches.
	\end{itemize}
}
\item verb \\
You say that you never  \textbf{touch} something or that you have not \textbf{touched} something for a long time to emphasize that you never use it, or you have not used
it for a long time.
 \textit{
	\begin{itemize}
	\item I never touch chocolate, it gives me spots.
	\item His diet is vegetarian, and he hasn't touched meat for six years.
	\item Jones hasn't touched a trumpet in 10 years.
	\end{itemize}
}
\item verb \\
If you \textbf{touch on} a particular subject or problem, you mention it or write briefly about it.
 \textit{
	\begin{itemize}
	\item The film touches on these issues, but only superficially.
	\item We will touch briefly on this aspect at the end of the chapter.
	\end{itemize}
}
\item verb \\
If something \textbf{touches} you, it affects you in some way for a short time.
 \textit{
	\begin{itemize}
	\item ...a guilt that in some sense touches everyone.
	\item Nor had the benefits of the war years touched all sectors of the population.
	\end{itemize}
}
\item verb \\
If something that someone says or does \textbf{touches} you, it affects you emotionally, often because you see that they are suffering a lot or that they are being very kind.
 \textit{
	\begin{itemize}
	\item It has touched me deeply to see how these people live.
	\item Her enthusiasm touched me.
	\end{itemize}
}
\item verb \\
If something \textbf{is touched with} a particular quality, it has a certain amount of that quality.
 \textit{
	\begin{itemize}
	\item His crinkly hair was touched with grey.
	\item The boy was touched with genius.
	\end{itemize}
}
\item verb \\
If you say about someone that nobody can \textbf{touch} him or her \textbf{for} a particular thing, you mean that he or she is much better at it than anyone else.
 \textit{
	\begin{itemize}
	\item No one can touch these girls for professionalism.
	\end{itemize}
}
\item verb \\
To \textbf{touch} a particular level, amount, or score, especially a high one, means to reach it.
 \textit{
	\begin{itemize}
	\item By the third lap he had touched 289 m.p.h.
	\item The winds had touched storm-force the day before.
	\end{itemize}
}
\item verb \\
If you \textbf{touch} someone \textbf{for} money, you ask them to give it to you.
 \textit{
	\begin{itemize}
	\item Now is the time to touch him for a loan.
	\end{itemize}
}
\item countable noun \\
A \textbf{touch} is a detail which is added to something to improve it.
 \textit{
	\begin{itemize}
	\item They called the event 'a tribute to heroes', which was a nice touch.
	\item Small touches to a room such as flowers can be what gives a house its vitality.
	\end{itemize}
}
\item singular noun \\
If someone has a particular kind of \textbf{touch} , they have a particular way of doing something.
 \textit{
	\begin{itemize}
	\item The dishes he produces all have a personal touch.
	\item The striker was unable to find his scoring touch.
	\end{itemize}
}
\item quantifier \\
\textbf{A touch of} something is a very small amount of it.
 \textit{
	\begin{itemize}
	\item She thought she just had a touch of flu.
	\item At university he wrote a bit, did a touch of acting, and indulged in internal college
politics.
	\end{itemize}
}
\item  \\
 a touch \textit{
	\begin{itemize}
	\end{itemize}
}
\item  \\
 at the touch of \textit{
	\begin{itemize}
	\end{itemize}
}
\item  \\
 the common touch \textit{
	\begin{itemize}
	\end{itemize}
}
\item  \\
 in touch \textit{
	\begin{itemize}
	\end{itemize}
}
\item  \\
 in touch/out of touch \textit{
	\begin{itemize}
	\end{itemize}
}
\item  \\
 to lose touch \textit{
	\begin{itemize}
	\end{itemize}
}
\item  \\
 to lose touch \textit{
	\begin{itemize}
	\end{itemize}
}
\item  \\
 touch and go \textit{
	\begin{itemize}
	\end{itemize}
}
\item  \\
 a soft touch \textit{
	\begin{itemize}
	\end{itemize}
}
\end{enumerate}

\section*{teach}
{\large \color{blue}  teaches  teaching  taught  }
\subsection*{Explain}
\begin{enumerate}
\item verb \\
If you \textbf{teach} someone something, you give them instructions so that they know about it or how to do it.
 \textit{
	\begin{itemize}
	\item The trainers have a programme to teach them vocational skills.
	\item George had taught him how to ride a horse.
	\item She taught Julie to read.
	\item The computer has simplified the difficult task of teaching reading to deaf people.
	\end{itemize}
}
\item verb \\
To \textbf{teach} someone something means to make them think , feel , or act in a new or different  way .
 \textit{
	\begin{itemize}
	\item Their daughter's death had taught him humility.
	\item He taught his followers that they could all be members of the kingdom of God.
	\item Teach them to voice their feelings.
	\end{itemize}
}
\item verb \\
If you \textbf{teach} or \textbf{teach} a subject, you help students to learn about it by explaining it or showing them how to do it, usually as a job at a school , college , or university .
 \textit{
	\begin{itemize}
	\item Ingrid is currently teaching Mathematics at Shimla Public School.
	\item The topic is not taught in degree courses.
	\item She taught English to Japanese business people.
	\item She has taught for 34 years.
	\item She taught children French.
	\item ...this twelve month taught course.
	\end{itemize}
}
\end{enumerate}

\section*{undermine}
{\large \color{blue}  undermines  undermining  undermined  }
\subsection*{Explain}
\begin{enumerate}
\item verb \\
If you \textbf{undermine} something such as a feeling or a system, you make it less strong or less secure than it was before, often by a gradual process or by repeated  efforts .
 \textit{
	\begin{itemize}
	\item Offering advice on each and every problem will undermine her feeling of being adult.
	\item Western intelligence agencies are accused of trying to undermine the government.
	\end{itemize}
}
\item verb \\
If you \textbf{undermine} someone or \textbf{undermine} their position or authority , you make their authority or position less secure, often by indirect  methods .
 \textit{
	\begin{itemize}
	\item She undermined him and destroyed his confidence in his own talent.
	\item The conversations were designed to undermine her authority.
	\end{itemize}
}
\item verb \\
If you \textbf{undermine} someone's efforts or \textbf{undermine} their chances of achieving something, you behave in a way that makes them less likely to succeed .
 \textit{
	\begin{itemize}
	\item The continued fighting threatens to undermine efforts to negotiate an agreement.
	\item I don't want to do something that would undermine the chances of success.
	\end{itemize}
}
\end{enumerate}

\section*{tolerate}
{\large \color{blue}  tolerates  tolerating  tolerated  }
\subsection*{Explain}
\begin{enumerate}
\item verb \\
If you \textbf{tolerate} a situation or person, you accept them although you do not particularly like them.
 \textit{
	\begin{itemize}
	\item She can no longer tolerate the position that she's in.
	\item The cousins tolerated each other, but did not really get on well together.
	\end{itemize}
}
\item verb \\
If you can \textbf{tolerate} something unpleasant or painful , you are able to bear it.
 \textit{
	\begin{itemize}
	\item The ability to tolerate pain varies from person to person.
	\end{itemize}
}
\end{enumerate}

\section*{weave}
{\large \color{blue}  weaves  weaving  wove  woven  }
\subsection*{Explain}
\begin{enumerate}
\item verb \\
If you \textbf{weave}  cloth or a carpet , you make it by crossing  threads over and under each other using a frame or machine  called a loom.
 \textit{
	\begin{itemize}
	\item They would spin and weave cloth, cook and attend to the domestic side of life.
	\item In one room, young mothers weave while babies doze in their laps.
	\end{itemize}
}
\item countable noun \\
A particular \textbf{weave} is the way in which the threads are arranged in a cloth or carpet.
 \textit{
	\begin{itemize}
	\item Fabrics with a close weave are ideal for painting.
	\end{itemize}
}
\item verb \\
If you \textbf{weave} something such as a basket, you make it by crossing long plant stems or fibres over and under each other.
 \textit{
	\begin{itemize}
	\item Jenny weaves baskets from willow she grows herself.
	\end{itemize}
}
\item verb \\
If you \textbf{weave} your \textbf{way}  somewhere , you move between and around things as you go there.
 \textit{
	\begin{itemize}
	\item The cars then weaved in and out of traffic at top speed.
	\item He weaved around the tables to where she sat with Bob.
	\item He weaves his way through a crowd.
	\end{itemize}
}
\item verb \\
If you \textbf{weave} a story , you invent a complicated story.
 \textit{
	\begin{itemize}
	\item Jan Roberts weaves a compelling tale which traps a young woman in a world run by
the Mafia.
	\end{itemize}
}
\item verb \\
If you \textbf{weave}  details into a story or design, you include them, so that they are closely linked  together or become an important part of the story or design.
 \textit{
	\begin{itemize}
	\item She weaves imaginative elements into her poems.
	\item Bragg weaves together the histories of his main characters.
	\end{itemize}
}
\end{enumerate}

\section*{abuse}
{\large \color{blue}  abuses  abusing  abused  }
\subsection*{Explain}
\begin{enumerate}
\item uncountable noun \\
\textbf{Abuse} of someone is cruel and violent  treatment of them.
 \textit{
	\begin{itemize}
	\item ...investigation of alleged child abuse.
	\item ...victims of sexual and physical abuse.
	\item ...controversy over human rights abuses.
	\end{itemize}
}
\item uncountable noun \\
\textbf{Abuse} is extremely  rude and insulting things that people say when they are angry .
 \textit{
	\begin{itemize}
	\item I was left shouting abuse as the car sped off.
	\item Raft repeatedly hurled verbal abuse at his co-star.
	\end{itemize}
}
\item variable noun \\
\textbf{Abuse} of something is the use of it in a wrong way or for a bad  purpose .
 \textit{
	\begin{itemize}
	\item What went on here was an abuse of power.
	\item ...drug and alcohol abuse.
	\end{itemize}
}
\item verb \\
If someone \textbf{is abused} , they are treated cruelly and violently.
 \textit{
	\begin{itemize}
	\item Janet had been abused by her father since she was eleven.
	\item ...parents who feel they cannot cope or might abuse their children.
	\item ...those who work with abused children.
	\end{itemize}
}
\item verb \\
You can say that someone \textbf{is abused} if extremely rude and insulting things are said to them.
 \textit{
	\begin{itemize}
	\item He alleged that he was verbally abused by other soldiers.
	\end{itemize}
}
\item verb \\
If you \textbf{abuse} something, you use it in a wrong way or for a bad purpose.
 \textit{
	\begin{itemize}
	\item He showed how the rich and powerful can abuse their position.
	\end{itemize}
}
\end{enumerate}

\section*{count}
{\large \color{blue}  counts  counting  counted  }
\subsection*{Explain}
\begin{enumerate}
\item verb \\
When you \textbf{count} , you say all the numbers one after another up to a particular number.
 \textit{
	\begin{itemize}
	\item He was counting slowly under his breath.
	\item Brian counted to twenty and lifted his binoculars.
	\end{itemize}
}
\item verb \\
If you \textbf{count} all the things in a group, you add them up in order to find how many there are.
 \textbf{Count up} means the same as count .
 \textit{
	\begin{itemize}
	\item At the last family wedding, George counted the total number in the family.
	\item I counted the money. It was more than five hundred pounds.
	\item I counted 34 wild goats grazing.
	\item With more than 90 percent of the votes counted, the Liberals should win nearly a
third of the seats.
	\item Couldn't we just count up our ballots and bring them to the courthouse?
	\end{itemize}
}
\item countable noun \\
A \textbf{count} is the action of counting a particular set of things, or the number that you get when you have counted them.
 \textit{
	\begin{itemize}
	\item The final count in last month's referendum showed 56.7 per cent in favour.
	\item At the last count the police in the Rimini area had 247 people in custody.
	\end{itemize}
}
\item countable noun \\
You use \textbf{count} when referring to the level or amount of something that someone or something has.
 \textit{
	\begin{itemize}
	\item He cut his daily calorie count from 3,000 to 2,000.
	\item My husband had a very low sperm count.
	\end{itemize}
}
\item singular noun \\
You use \textbf{count} in expressions such as \textbf{a count of three} or \textbf{a count of ten} when you are measuring a length of time by counting slowly up to a certain number.
 \textit{
	\begin{itemize}
	\item Hold your breath for a count of five, then slowly breathe out.
	\item The fight ended when Palacios went down for a count of eight.
	\end{itemize}
}
\item verb \\
If something or someone \textbf{counts}  \textbf{for} something or \textbf{counts} , they are important or valuable .
 \textit{
	\begin{itemize}
	\item It doesn't matter what you've said; what counts is how you act.
	\item It's as if your opinions, your likes and dislikes just don't count.
	\item When I first came to college I realised that brainpower didn't count for much.
	\item Experience counts for a lot in poker.
	\end{itemize}
}
\item verb \\
If something \textbf{counts} or \textbf{is counted}  \textbf{as} a particular thing, it is regarded as being that thing, especially in particular circumstances or under particular rules.
 \textit{
	\begin{itemize}
	\item No one agrees on what counts as a desert.
	\item Any word that's not legible will be counted as wrong.
	\item When you were a child, your wishes didn't always count.
	\item It can be counted a success, in that it has built up substantial sales.
	\item They can count it as a success.
	\end{itemize}
}
\item verb \\
If you \textbf{count} something when you are making a calculation , you include it in that calculation.
 \textit{
	\begin{itemize}
	\item Statistics don't count the people who aren't qualified to be in the work force.
	\item The years before their arrival in prison are not counted as part of their sentence.
	\end{itemize}
}
\item countable noun \\
You can use \textbf{count} to refer to one or more points that you are considering. For example , if someone is wrong  \textbf{on two counts} , they are wrong in two ways.
 \textit{
	\begin{itemize}
	\item 'You drink Scotch,' she said. 'All Republicans drink Scotch.'—'Wrong on both counts.
I'm a Democrat, and I drink bourbon.'
	\end{itemize}
}
\item countable noun \\
In law, a \textbf{count} is one of a number of charges brought against someone in court.
 \textit{
	\begin{itemize}
	\item He was indicted by a grand jury on two counts of murder.
	\end{itemize}
}
\item  \\
 keep count/lose count \textit{
	\begin{itemize}
	\end{itemize}
}
\item  \\
 out for the count \textit{
	\begin{itemize}
	\end{itemize}
}
\item  \\
 to stand up and be counted \textit{
	\begin{itemize}
	\end{itemize}
}
\end{enumerate}

\section*{ache}
{\large \color{blue}  aches  aching  ached  }
\subsection*{Explain}
\begin{enumerate}
\item verb \\
If you \textbf{ache} or a part of your body \textbf{aches} , you feel a steady , fairly  strong pain.
 \textit{
	\begin{itemize}
	\item Her head was throbbing and she ached all over.
	\item My leg still aches when I sit down.
	\item The weary walkers soothed their aching feet in the sea.
	\end{itemize}
}
\item countable noun \\
An \textbf{ache} is a steady, fairly strong pain in a part of your body.
 \textit{
	\begin{itemize}
	\item You feel nausea and aches in your muscles.
	\item Poor posture can cause neck ache, headaches and breathing problems.
	\end{itemize}
}
\item verb \\
If you \textbf{ache}  \textbf{for} something or your heart  \textbf{aches} , you want something very much, and feel very unhappy because you cannot have it.
 \textbf{Ache} is also a noun .
 \textit{
	\begin{itemize}
	\item ... couples aching for a child
	\item But Spain was a country aching to get away from its past.
	\item It was quite an achievement to keep smiling when his heart must have been aching.
	\item You also feel an overwhelming ache for support from others which you cannot put into
words.
	\end{itemize}
}
\item  \\
 aches and pains \textit{
	\begin{itemize}
	\end{itemize}
}
\end{enumerate}

\section*{detect}
{\large \color{blue}  detects  detecting  detected  }
\subsection*{Explain}
\begin{enumerate}
\item verb \\
To \textbf{detect} something means to find it or discover that it is present  somewhere by using equipment or making an investigation .
 \textit{
	\begin{itemize}
	\item ...a sensitive piece of equipment used to detect radiation.
	\item Most skin cancers can be cured if detected and treated early.
	\item ...a device which can detect who is more at risk of a heart attack.
	\end{itemize}
}
\item verb \\
If you \textbf{detect} something, you notice it or sense it, even though it is not very obvious .
 \textit{
	\begin{itemize}
	\item Arnold could detect a certain sadness in the old man's face.
	\end{itemize}
}
\end{enumerate}

\section*{affect}
{\large \color{blue}  affects  affecting  affected  }
\subsection*{Explain}
\begin{enumerate}
\item verb \\
If something \textbf{affects} a person or thing, it influences them or causes them to change in some way.
 \textit{
	\begin{itemize}
	\item Nicotine adversely affects the functioning of the heart and arteries.
	\item More than seven million people have been affected by drought.
	\item ...the worst-affected areas of Somalia.
	\end{itemize}
}
\item verb \\
If a disease \textbf{affects} someone, it causes them to become ill .
 \textit{
	\begin{itemize}
	\item Arthritis is a crippling disease which affects people all over the world.
	\end{itemize}
}
\item verb \\
If something or someone \textbf{affects} you, they make you feel a strong emotion, especially sadness or pity .
 \textit{
	\begin{itemize}
	\item Her experience of the last few hours has deeply affected her.
	\item Gregor loved his sister, and her loss clearly still affects him.
	\end{itemize}
}
\item verb \\
If you \textbf{affect} a particular characteristic or way of behaving , you pretend that it is genuine , or natural for you.
 \textit{
	\begin{itemize}
	\item He listened to them, affecting an amused interest.
	\item Ms. Redgrave affects a heavy Italian accent.
	\end{itemize}
}
\end{enumerate}

\section*{diminish}
{\large \color{blue}  diminishes  diminishing  diminished  }
\subsection*{Explain}
\begin{enumerate}
\item verb \\
When something \textbf{diminishes} , or when something \textbf{diminishes} it, it becomes reduced in size , importance , or intensity .
 \textit{
	\begin{itemize}
	\item The threat of nuclear war has diminished.
	\item Federalism is intended to diminish the power of the central state.
	\item Universities are facing grave problems because of diminishing resources.
	\item This could mean diminished public support for the war.
	\end{itemize}
}
\item verb \\
If you \textbf{diminish} someone or something, you talk about them or treat them in a way that makes them appear less important than they really are.
 \textit{
	\begin{itemize}
	\item He never put her down or diminished her.
	\item He could no longer cope; he relied on me, and felt diminished by it.
	\end{itemize}
}
\end{enumerate}

\section*{applaud}
{\large \color{blue}  applauds  applauding  applauded  }
\subsection*{Explain}
\begin{enumerate}
\item verb \\
When a group of people \textbf{applaud} , they clap their hands in order to show approval, for example when they have enjoyed a play or concert .
 \textit{
	\begin{itemize}
	\item The audience laughed and applauded.
	\item Every person stood to applaud his unforgettable act of courage.
	\end{itemize}
}
\item verb \\
When an attitude or action \textbf{is applauded} , people praise it.
 \textit{
	\begin{itemize}
	\item He should be applauded for his courage.
	\item This last move can only be applauded.
	\item She applauds the fact that they are promoting new ideas.
	\end{itemize}
}
\end{enumerate}

\section*{drip}
{\large \color{blue}  drips  dripping  dripped  }
\subsection*{Explain}
\begin{enumerate}
\item verb \\
When liquid \textbf{drips}  somewhere , or you \textbf{drip} it somewhere, it falls in individual small drops.
 \textit{
	\begin{itemize}
	\item Sit your child forward and let the blood drip into a tissue or on to the floor.
	\item Amid the trees the sea mist was dripping.
	\item The children kept dripping Coke on the carpets.
	\end{itemize}
}
\item verb \\
When something \textbf{drips} , drops of liquid fall from it.
 \textit{
	\begin{itemize}
	\item A tap in the kitchen was dripping.
	\item Lou was dripping with perspiration.
	\item He was holding a cloth that dripped pink drops upon the floor.
	\end{itemize}
}
\item countable noun \\
A \textbf{drip} is a small individual drop of a liquid.
 \textit{
	\begin{itemize}
	\item Drips of water rolled down the trousers of his uniform.
	\end{itemize}
}
\item countable noun \\
A \textbf{drip} is a piece of medical equipment by which a liquid is slowly passed through a tube into a patient's blood .
 \textit{
	\begin{itemize}
	\item I had a bad attack of pneumonia and spent two days in hospital on a drip.
	\end{itemize}
}
\item verb \\
If you say that something \textbf{is dripping with} a particular thing, you mean that it contains a lot of that thing.
 \textit{
	\begin{itemize}
	\item They were dazed by window displays dripping with diamonds and furs.
	\item His voice was dripping with sarcasm.
	\end{itemize}
}
\item countable noun \\
If you call someone a \textbf{drip} , you mean that they are rather  stupid and lacking in enthusiasm or energy .
 \textit{
	\begin{itemize}
	\end{itemize}
}
\end{enumerate}

\section*{ascertain}
{\large \color{blue}  ascertains  ascertaining  ascertained  }
\subsection*{Explain}
\begin{enumerate}
\item verb \\
If you \textbf{ascertain} the truth about something, you find out what it is, especially by making a deliberate  effort to do so.
 \textit{
	\begin{itemize}
	\item It is always vital to ascertain the cause of a continual headache.
	\item Once they had ascertained that he was not a spy, they agreed to release him.
	\item Take time to ascertain what services your bank is providing, and at what cost.
	\end{itemize}
}
\end{enumerate}

\section*{drown}
{\large \color{blue}  drowns  drowning  drowned  }
\subsection*{Explain}
\begin{enumerate}
\item verb \\
When someone \textbf{drowns} or \textbf{is drowned} , they die because they have gone or been pushed under water and cannot breathe .
 \textit{
	\begin{itemize}
	\item Forty-eight people have drowned after their boat capsized during a storm.
	\item A child can drown in only a few inches of water.
	\item Last night a boy was drowned in the river.
	\item He walked into the sea and drowned himself.
	\item Dolphins have sometimes been known to save drowning swimmers.
	\end{itemize}
}
\item verb \\
If you say that a person or thing \textbf{is drowning}  \textbf{in} something, you are emphasizing that they have a very large amount of it, or are completely covered in it.
 \textit{
	\begin{itemize}
	\item We were drowning in data but starved of information.
	\item ...people who gradually find themselves drowning in debt.
	\item The potatoes were drowned in chilli.
	\end{itemize}
}
\item verb \\
If something \textbf{drowns} a sound, it is so loud that you cannot hear that sound properly.
 \textbf{Drown out}  means the same as \textbf{drown} .
 \textit{
	\begin{itemize}
	\item Clapping drowned the speaker's words for a moment.
	\item The conversation was drowned by the arrival of the taxi.
	\item Their cheers drowned out the protests of demonstrators.
	\item Her voice was drowned out by a loud crash.
	\end{itemize}
}
\item  \\
 to drown one's sorrows \textit{
	\begin{itemize}
	\end{itemize}
}
\end{enumerate}

\section*{assassinate}
{\large \color{blue}  assassinates  assassinating  assassinated  }
\subsection*{Explain}
\begin{enumerate}
\item verb \\
When someone important  \textbf{is assassinated} , they are murdered as a political act.
 \textit{
	\begin{itemize}
	\item Would the U.S.A. be radically different today if Kennedy had not been assassinated?
	\item The plot to assassinate Martin Luther King had started long before he was actually
killed.
	\end{itemize}
}
\end{enumerate}

\section*{earn}
{\large \color{blue}  earns  earning  earned  }
\subsection*{Explain}
\begin{enumerate}
\item verb \\
If you \textbf{earn} money, you receive money in return for work that you do.
 \textit{
	\begin{itemize}
	\item Charlie was earning eight pounds, I was earning five.
	\item What a lovely way to earn a living.
	\item The dancers can earn anything between £80 and £130 for each session.
	\item She was always out earning.
	\end{itemize}
}
\item verb \\
If something \textbf{earns} money, it produces money as profit or interest.
 \textit{
	\begin{itemize}
	\item ...a current account which earns little or no interest.
	\item We buy everything abroad with the money earned from oil imports.
	\end{itemize}
}
\item verb \\
If you \textbf{earn} something such as praise , you get it because you deserve it.
 \textit{
	\begin{itemize}
	\item Companies must earn a reputation for honesty.
	\item I think that's earned him very high admiration.
	\end{itemize}
}
\end{enumerate}

\section*{assemble}
{\large \color{blue}  assembles  assembling  assembled  }
\subsection*{Explain}
\begin{enumerate}
\item verb \\
When people \textbf{assemble} or when someone \textbf{assembles} them, they come together in a group, usually for a particular purpose such as a meeting .
 \textit{
	\begin{itemize}
	\item There wasn't even a convenient place for students to assemble between classes.
	\item Thousands of people, mainly Zulus, assembled in a stadium in Thokoza.
	\item He has assembled a team of experts to handle queries.
	\item The assembled multitude cheered and whistled as the political leaders arrived.
	\end{itemize}
}
\item verb \\
To \textbf{assemble} something means to collect it together or to fit the different parts of it together.
 \textit{
	\begin{itemize}
	\item Greenpeace managed to assemble enough boats to waylay the ship at sea.
	\item She had been trying to assemble the bomb when it went off in her arms.
	\item He is assembling evidence concerning a murder.
	\end{itemize}
}
\end{enumerate}

\section*{explain}
{\large \color{blue}  explains  explaining  explained  }
\subsection*{Explain}
\begin{enumerate}
\item verb \\
If you \textbf{explain} something, you give details about it or describe it so that it can be understood .
 \textit{
	\begin{itemize}
	\item Not every judge, however, has the ability to explain the law in simple terms.
	\item Don't sign anything until your solicitor has explained the contract to you.
	\item Professor Griffiths explained how the drug appears to work.
	\item 'He and Mrs Stein have a plan,' she explained.
	\item I explained that each person has different ideas of what freedom is.
	\end{itemize}
}
\item verb \\
If you \textbf{explain} something that has happened , you give people reasons for it, especially in an attempt to justify it.
 \textit{
	\begin{itemize}
	\item 'Let me explain, sir.'—'Don't tell me about it. I don't want to know.'.
	\item Before she ran away, she left a note explaining her actions.
	\item Hospital discipline was broken. Amy would have to explain herself.
	\item Explain why you didn't phone.
	\item The receptionist apologized for the delay, explaining that it had been a hectic day.
	\end{itemize}
}
\end{enumerate}

\section*{associate}
{\large \color{blue}  associates  associating  associated  }
\subsection*{Explain}
\begin{enumerate}
\item verb \\
If you \textbf{associate} someone or something \textbf{with} another thing, the two are connected in your mind.
 \textit{
	\begin{itemize}
	\item Through science we've got the idea of associating progress with the future.
	\item Groups have been barred from events because they are associated with vandalism.
	\end{itemize}
}
\item verb \\
If you \textbf{are associated with} a particular  organization , cause, or point of view , or if you \textbf{associate}  \textbf{yourself with} it, you support it publicly .
 \textit{
	\begin{itemize}
	\item I haven't been associated with the project over the last year.
	\item The party feels the need to associate itself with the green movement.
	\end{itemize}
}
\item verb \\
If you say that someone \textbf{is associating with} another person or group of people, you mean they are spending a lot of time in the company of people you do not approve of.
 \textit{
	\begin{itemize}
	\item What would they think if they knew that they were associating with a murderer?
	\end{itemize}
}
\item countable noun \\
Your \textbf{associates} are the people you are closely connected with, especially at work.
 \textit{
	\begin{itemize}
	\item ...the restaurant owner's business associates.
	\end{itemize}
}
\item adjective \\
\textbf{Associate} is used before a rank or title to indicate a slightly  different or lower rank or title.
 \textit{
	\begin{itemize}
	\item Mr Lin is associate director of the Institute.
	\item She applied for associate membership last year.
	\item ...an associate professor of political science.
	\end{itemize}
}
\end{enumerate}

\section*{extinguish}
{\large \color{blue}  extinguishes  extinguishing  extinguished  }
\subsection*{Explain}
\begin{enumerate}
\item verb \\
If you \textbf{extinguish} a fire or a light, you stop it burning or shining .
 \textit{
	\begin{itemize}
	\item It took about 50 minutes to extinguish the fire.
	\item The lights are extinguished as soon as the news conference is over.
	\end{itemize}
}
\item verb \\
If something \textbf{extinguishes} a feeling or idea , it destroys it.
 \textit{
	\begin{itemize}
	\item The message extinguished her hopes of Richard's return.
	\end{itemize}
}
\end{enumerate}

\section*{await}
{\large \color{blue}  awaits  awaiting  awaited  }
\subsection*{Explain}
\begin{enumerate}
\item verb \\
If you \textbf{await} someone or something, you wait for them.
 \textit{
	\begin{itemize}
	\item He's awaiting trial, which is expected to begin early next year.
	\end{itemize}
}
\item verb \\
Something that \textbf{awaits} you is going to happen or come to you in the future .
 \textit{
	\begin{itemize}
	\item A nasty surprise awaited them in Rosemary Lane.
	\end{itemize}
}
\end{enumerate}

\section*{fade}
{\large \color{blue}  fades  fading  faded  }
\subsection*{Explain}
\begin{enumerate}
\item verb \\
When a coloured object \textbf{fades} or when the light \textbf{fades} it, it gradually becomes paler .
 \textit{
	\begin{itemize}
	\item All colour fades–especially under the impact of direct sunlight.
	\item No matter how soft the light is, it still fades curtains in every room.
	\item ...fading portraits of the Queen and Prince Philip.
	\end{itemize}
}
\item verb \\
When light \textbf{fades} , it slowly becomes less bright . When a sound \textbf{fades} , it slowly becomes less loud .
 \textit{
	\begin{itemize}
	\item Seaton lay on his bed and gazed at the ceiling as the light faded.
	\item The sound of the last bomber's engines faded into the distance.
	\end{itemize}
}
\item verb \\
When something that you are looking at \textbf{fades} , it slowly becomes less bright or clear until it disappears .
 \textbf{Fade away} means the same as fade .
 \textit{
	\begin{itemize}
	\item They observed the comet for 70 days before it faded from sight.
	\item They watched the familiar mountains fade into the darkness.
	\item We watched the harbour and then the coastline fade away into the morning mist.
	\end{itemize}
}
\item verb \\
If someone or something \textbf{fades} , for example , into the background , they become hardly  noticeable or very unimportant .
 \textbf{Fade away} means the same as fade .
 \textit{
	\begin{itemize}
	\item She had a way of fading into the background when things got rough.
	\item This is one Briton with potential who will not quickly fade from the scene.
	\item The sound comes up and slowly fades away into the distance.
	\end{itemize}
}
\item verb \\
If memories , feelings , or possibilities  \textbf{fade} , they slowly become less intense or less strong .
 \textit{
	\begin{itemize}
	\item Sympathy for the rebels, the government claims, is beginning to fade.
	\item Prospects for peace had already started to fade.
	\item ...fading memories of better days.
	\end{itemize}
}
\item verb \\
If someone's smile  \textbf{fades} , they slowly stop smiling.
 \textit{
	\begin{itemize}
	\item Jay nodded, his smile fading.
	\end{itemize}
}
\end{enumerate}

\section*{compile}
{\large \color{blue}  compiles  compiling  compiled  }
\subsection*{Explain}
\begin{enumerate}
\item verb \\
When you \textbf{compile} something such as a report , book, or programme, you produce it by collecting and putting together many pieces of information .
 \textit{
	\begin{itemize}
	\item Councils were required to compile a register of all adults living in their areas.
	\item The book took 10 years to compile.
	\item A report compiled by the Fed's Philadelphia branch described the economy as weak.
	\end{itemize}
}
\end{enumerate}

\section*{feel}
{\large \color{blue}  feels  feeling  felt  }
\subsection*{Explain}
\begin{enumerate}
\item link verb \\
If you \textbf{feel} a particular emotion or physical sensation, you experience it.
 \textit{
	\begin{itemize}
	\item I am feeling very depressed.
	\item I will always feel grateful to that little guy.
	\item I remember feeling sick.
	\item ...soldiers who once felt proud to wear their uniforms.
	\item Suddenly I felt a sharp pain in my shoulder.
	\item You won't feel a thing.
	\item I felt as if all my strength had gone.
	\item I felt like I was being kicked in the teeth every day.
	\end{itemize}
}
\item link verb \\
If you talk about how an experience or event \textbf{feels} , you talk about the emotions and sensations connected with it.
 \textit{
	\begin{itemize}
	\item It feels good to have finished a piece of work.
	\item The speed at which everything moved felt strange.
	\item Within five minutes of arriving back from holiday, it feels as if I've never been
away.
	\item It felt like I'd had two babies instead of one.
	\item Preparing for that first trial felt like learning the rules of a new game.
	\end{itemize}
}
\item link verb \\
If you talk about how an object \textbf{feels} , you talk about the physical quality that you notice when you touch or hold it. For example , if something \textbf{feels}  soft , you notice that it is soft when you touch it.
 \textbf{Feel} is also a noun .
 \textit{
	\begin{itemize}
	\item The metal felt smooth and cold.
	\item The ten-foot oars felt heavy and awkward.
	\item When the clay feels like putty, it is ready to use.
	\item He remembered the feel of her skin.
	\item Linen raincoats have a crisp, papery feel.
	\end{itemize}
}
\item link verb \\
If you talk about how the weather  \textbf{feels} , you describe the weather, especially the temperature or whether or not you think it is going to rain or snow .
 \textit{
	\begin{itemize}
	\item It felt wintry cold that day.
	\end{itemize}
}
\item verb \\
If you \textbf{feel} an object, you touch it deliberately with your hand , so that you learn what it is like, for example what shape it is or whether it is rough or smooth .
 \textit{
	\begin{itemize}
	\item The doctor felt his head.
	\item When dry, feel the surface and it will no longer be smooth.
	\item Feel how soft the skin is in the small of the back.
	\item Her eyes squeezed shut, she felt inside the tin, expecting it to be bare.
	\end{itemize}
}
\item verb \\
If you can \textbf{feel} something, you are aware of it because it is touching you.
 \textit{
	\begin{itemize}
	\item Through several layers of clothes I could feel his muscles.
	\item He felt her leg against his.
	\end{itemize}
}
\item verb \\
If you \textbf{feel} something happening , you become aware of it because of the effect it has on your body.
 \textit{
	\begin{itemize}
	\item She felt something being pressed into her hands.
	\item He felt something move beside him.
	\item She felt herself lifted from her feet.
	\item Tremors were felt 250 miles away.
	\end{itemize}
}
\item verb \\
If you \textbf{feel}  \textbf{yourself} doing something or being in a particular state, you are aware that something is happening
to you which you are unable to control.
 \textit{
	\begin{itemize}
	\item I felt myself blush.
	\item If at any point you feel yourself becoming tense, make a conscious effort to relax.
	\item I actually felt my heart quicken.
	\end{itemize}
}
\item verb \\
If you \textbf{feel} the presence of someone or something, you become aware of them, even though you cannot see or hear them.
 \textit{
	\begin{itemize}
	\item He felt her eyes on him.
	\item Suddenly, I felt a presence behind me.
	\item I could feel that a man was watching me very intensely.
	\item He almost felt her wincing at the other end of the telephone.
	\end{itemize}
}
\item verb \\
If you \textbf{feel} that something is the case , you have a strong idea in your mind that it is the case.
 \textit{
	\begin{itemize}
	\item I feel that not enough is being done to protect the local animal life.
	\item I feel certain that it will all turn out well.
	\item She felt herself to be part of a large business empire.
	\item I never felt myself a real child of the sixties.
	\end{itemize}
}
\item verb \\
If you \textbf{feel} that you should do something, you think that you should do it.
 \textit{
	\begin{itemize}
	\item I feel I should resign.
	\item He felt that he had to do it.
	\item You need not feel obliged to contribute.
	\item They felt under no obligation to maintain their employees.
	\end{itemize}
}
\item verb \\
If you talk about how you \textbf{feel}  \textbf{about} something, you talk about your opinion, attitude , or reaction to it.
 \textit{
	\begin{itemize}
	\item We'd like to know what you feel about abortion.
	\item How do you feel about going back to the neighborhood?
	\item She feels guilty about spending less time lately with her two kids.
	\item He feels deep regret about his friend's death.
	\end{itemize}
}
\item verb \\
If you \textbf{feel like} doing something or having something, you want to do it or have it because you are in the right mood for it and think you would enjoy it.
 \textit{
	\begin{itemize}
	\item Neither of them felt like going back to sleep.
	\item Could we take a walk? I feel like a little exercise.
	\end{itemize}
}
\item verb \\
If you \textbf{feel} the effect or result of something, you experience it.
 \textit{
	\begin{itemize}
	\item The charity is still feeling the effects of revelations about its one-time president.
	\item The real impact will be felt in the developing world.
	\end{itemize}
}
\item singular noun \\
The \textbf{feel} of something, for example a place, is the general impression that it gives you.
 \textit{
	\begin{itemize}
	\item The room has a warm, cosy feel.
	\item ...a book that takes on the feel of an epic.
	\end{itemize}
}
\end{enumerate}

\section*{cue}
{\large \color{blue}  cues  cueing  cued  }
\subsection*{Explain}
\begin{enumerate}
\item countable noun \\
In the theatre or in a musical performance, a performer's \textbf{cue} is something another performer says or does that is a signal for them to begin  speaking , playing, or doing something.
 \textit{
	\begin{itemize}
	\item The actors not performing sit at the side of the stage in full view, waiting for
their cues.
	\item I had never known him miss a cue.
	\end{itemize}
}
\item verb \\
If one performer \textbf{cues} another, they say or do something which is a signal for the second performer to begin speaking, playing, or doing something.
 \textit{
	\begin{itemize}
	\item He read the scene, with Seaton cueing him.
	\end{itemize}
}
\item countable noun \\
If you say that something that happens is a \textbf{cue}  \textbf{for} an action, you mean that people start doing that action when it happens.
 \textit{
	\begin{itemize}
	\item That was the cue for several months of intense bargaining.
	\item That was Nicholas's cue to ask for another chocolate chip cookie.
	\end{itemize}
}
\item countable noun \\
A \textbf{cue} is a long, thin  wooden  stick that is used to hit the ball in games such as snooker , billiards , and pool .
 \textit{
	\begin{itemize}
	\item Their youngest brother was nine when he picked up a cue for the first time.
	\end{itemize}
}
\item  \\
 on cue/as if on cue \textit{
	\begin{itemize}
	\end{itemize}
}
\item  \\
 take one's cue from \textit{
	\begin{itemize}
	\end{itemize}
}
\end{enumerate}

\section*{forge}
{\large \color{blue}  forges  forging  forged  }
\subsection*{Explain}
\begin{enumerate}
\item verb \\
If one person or institution  \textbf{forges} an agreement or relationship with another, they create it with a lot of hard work, hoping that it will be strong or lasting .
 \textit{
	\begin{itemize}
	\item The Prime Minister is determined to forge a good relationship with America's new
leader.
	\item They agreed to forge closer economic ties.
	\item The programme aims to forge links between higher education and small businesses.
	\item The pair forged a formidable alliance.
	\end{itemize}
}
\item verb \\
If you say that a person \textbf{has forged} something that you approve of, you mean that you admire them for having done something difficult .
 \textit{
	\begin{itemize}
	\item The project will help inmates forge new careers.
	\end{itemize}
}
\item verb \\
If someone \textbf{forges} something such as a banknote, a document , or a painting , they copy it or make it so that it looks  genuine , in order to deceive people.
 \textit{
	\begin{itemize}
	\item He admitted seven charges including forging passports.
	\item She alleged that Taylor had forged her signature on the form.
	\item They used forged documents to leave the country.
	\end{itemize}
}
\item countable noun \\
A \textbf{forge} is a place where someone makes metal goods and equipment by heating pieces of metal and then shaping them.
 \textit{
	\begin{itemize}
	\item ...the blacksmith's forge.
	\item ...Woodbury Blacksmith & Forge Co.
	\end{itemize}
}
\item verb \\
If someone \textbf{forges} an object out of metal, they heat the metal and then hammer and bend it into the required shape.
 \textit{
	\begin{itemize}
	\item To forge a blade takes great skill.
	\end{itemize}
}
\end{enumerate}

\section*{dilute}
{\large \color{blue}  dilutes  diluting  diluted  }
\subsection*{Explain}
\begin{enumerate}
\item verb \\
If a liquid  \textbf{is diluted} or \textbf{dilutes} , it is added to or mixes with water or another liquid, and becomes weaker.
 \textit{
	\begin{itemize}
	\item If you give your baby juice, dilute it well with cooled, boiled water.
	\item The liquid is then diluted.
	\item The poisons seeping from contaminated land quickly dilute in the water.
	\end{itemize}
}
\item adjective \\
A \textbf{dilute} liquid is very thin and weak, usually because it has had water added to it.
 \textit{
	\begin{itemize}
	\item ...a dilute solution of bleach.
	\end{itemize}
}
\item verb \\
If someone or something \textbf{dilutes} a belief , quality, or value, they make it weaker and less effective .
 \textit{
	\begin{itemize}
	\item There was a clear intention to dilute black voting power.
	\item Serious attention is being given to diluting the value of personal tax allowances.
	\end{itemize}
}
\end{enumerate}

\section*{gather}
{\large \color{blue}  gathers  gathering  gathered  }
\subsection*{Explain}
\begin{enumerate}
\item verb \\
If people \textbf{gather}  somewhere or if someone \textbf{gathers} people somewhere, they come together in a group.
 \textit{
	\begin{itemize}
	\item In the evenings, we gathered around the fireplace and talked.
	\item The man signalled for me to gather the children together.
	\end{itemize}
}
\item verb \\
If you \textbf{gather} things, you collect them together so that you can use them.
 \textbf{Gather up}  means the same as gather .
 \textit{
	\begin{itemize}
	\item I suggest we gather enough firewood to last the night.
	\item She stood up and started gathering her things together.
	\item When Sutcliffe had gathered up his papers, he went out.
	\item He gathered the leaves up off the ground.
	\end{itemize}
}
\item verb \\
If you \textbf{gather} information or evidence , you collect it, especially over a period of time and after a lot of hard work.
 \textit{
	\begin{itemize}
	\item ...a private detective using a hidden recording device to gather information.
	\item This would help the prosecutor gather evidence against him which could be used in
court.
	\end{itemize}
}
\item verb \\
If something \textbf{gathers} speed, momentum , or force, it gradually becomes faster or more powerful .
 \textit{
	\begin{itemize}
	\item Demands for his dismissal have gathered momentum in recent weeks.
	\item The raft gathered speed as the current dragged it toward the falls.
	\end{itemize}
}
\item verb \\
When you \textbf{gather} something such as your strength , courage , or thoughts , you make an effort to prepare yourself to do something.
 \textbf{Gather up} means the same as gather .
 \textit{
	\begin{itemize}
	\item You must gather your strength for the journey.
	\item She was gathering up her courage to approach him when he called to her.
	\end{itemize}
}
\item verb \\
You use \textbf{gather} in expressions such as ' \textbf{I gather} ' and ' \textbf{as far as I can gather} ' to introduce information that you have found out, especially when you have found it out in an indirect  way .
 \textit{
	\begin{itemize}
	\item I gather his report is highly critical of the trial judge.
	\item 'He speaks English,' she said to Graham. 'I gathered that.'
	\item From what I could gather, he was trying to raise money by organising festivals.
	\end{itemize}
}
\item verb \\
If you \textbf{gather}  fabric or cloth , you make a row of very small folds in it by sewing a thread through it and then pulling the thread tight.
 \textit{
	\begin{itemize}
	\item Gather the skirt at the waist.
	\end{itemize}
}
\end{enumerate}

\section*{distill}
{\large \color{blue}  }
\subsection*{Explain}
\begin{enumerate}
\item verb intransitive \\
1.  2.  3.  \textit{
	\begin{itemize}
	\end{itemize}
}
\item verb transitive \\
4.  5.  6.  7.  8.  \textit{
	\begin{itemize}
	\item to distill water
	\item to distill whiskey
	\item to distill one's style
	\end{itemize}
}
\end{enumerate}

\section*{hatch}
{\large \color{blue}  hatches  hatching  hatched  }
\subsection*{Explain}
\begin{enumerate}
\item verb \\
When a baby bird, insect , or other animal \textbf{hatches} , or when it \textbf{is hatched} , it comes out of its egg by breaking the shell .
 \textit{
	\begin{itemize}
	\item As soon as the two chicks hatch, they leave the nest burrow.
	\item The young disappeared soon after they were hatched.
	\end{itemize}
}
\item verb \\
When an egg \textbf{hatches} or when a bird, insect, or other animal \textbf{hatches} an egg, the egg breaks open and a baby comes out.
 \textbf{Hatch out} means the same as hatch .
 \textit{
	\begin{itemize}
	\item The eggs hatch after a week or ten days.
	\item During these periods the birds will lie on the cage floor as if trying to lay or
hatch eggs.
	\item Seeing the eggs hatch out for the first time is a moment that I will never forget.
	\end{itemize}
}
\item verb \\
If you \textbf{hatch} a plot or a scheme, you think of it and work it out.
 \textit{
	\begin{itemize}
	\item They hatched a plot to set fire to the house.
	\end{itemize}
}
\item countable noun \\
A \textbf{hatch} is an opening in the deck of a ship, through which people or cargo can go . You can also  refer to the door of this opening as a \textbf{hatch} .
 \textit{
	\begin{itemize}
	\item He stuck his head up through the hatch.
	\item All deck fittings, windows, hatches and doors had been fastened.
	\end{itemize}
}
\item countable noun \\
A \textbf{hatch} is an opening in a ceiling or a wall, especially between a kitchen and a dining room , which you can pass something such as food through.
 \textit{
	\begin{itemize}
	\end{itemize}
}
\item  \\
 to batten down the hatches \textit{
	\begin{itemize}
	\end{itemize}
}
\end{enumerate}

\section*{evacuate}
{\large \color{blue}  evacuates  evacuating  evacuated  }
\subsection*{Explain}
\begin{enumerate}
\item verb \\
To \textbf{evacuate} someone means to send them to a place of safety, away from a dangerous building, town, or area.
 \textit{
	\begin{itemize}
	\item They were planning to evacuate the seventy American officials still in the country.
	\item Since 1951, 18,000 people have been evacuated from the area.
	\end{itemize}
}
\item verb \\
If people \textbf{evacuate} a place, they move out of it for a period of time, especially because it is dangerous.
 \textit{
	\begin{itemize}
	\item The fire is threatening about sixty homes, and residents have evacuated the area.
	\item Officials ordered the residents to evacuate.
	\end{itemize}
}
\end{enumerate}

\section*{imitate}
{\large \color{blue}  imitates  imitating  imitated  }
\subsection*{Explain}
\begin{enumerate}
\item verb \\
If you \textbf{imitate} someone, you copy what they do or produce.
 \textit{
	\begin{itemize}
	\item ...a genuine German musical which does not try to imitate the American model.
	\item ...a precedent which may be imitated by other activists in the future.
	\end{itemize}
}
\item verb \\
If you \textbf{imitate} a person or animal, you copy the way they speak or behave , usually because you are trying to be funny .
 \textit{
	\begin{itemize}
	\item Clarence screws up his face and imitates the Colonel again.
	\end{itemize}
}
\end{enumerate}

\section*{illuminate}
{\large \color{blue}  illuminates  illuminating  illuminated  }
\subsection*{Explain}
\begin{enumerate}
\item verb \\
To \textbf{illuminate} something means to shine light on it and to make it brighter and more visible .
 \textit{
	\begin{itemize}
	\item No streetlights illuminated the street.
	\item The black sky was illuminated by forked lightning.
	\end{itemize}
}
\item verb \\
If you \textbf{illuminate} something that is unclear or difficult to understand, you make it clearer by explaining it carefully or giving information about it.
 \textit{
	\begin{itemize}
	\item They use games and drawings to illuminate their subject.
	\end{itemize}
}
\end{enumerate}

\section*{imply}
{\large \color{blue}  implies  implying  implied  }
\subsection*{Explain}
\begin{enumerate}
\item verb \\
If you \textbf{imply}  \textbf{that} something is the case , you say something which indicates that it is the case in an indirect way.
 \textit{
	\begin{itemize}
	\item 'Are you implying that I have something to do with those attacks?' she asked coldly.
	\item She felt undermined by the implied criticism.
	\end{itemize}
}
\item verb \\
If an event or situation  \textbf{implies} that something is the case, it makes you think it likely that it is the case.
 \textit{
	\begin{itemize}
	\item Exports in June rose 1.5%, implying that the economy was stronger than many investors
thought.
	\item A high fat intake nearly always implies a low fibre intake.
	\end{itemize}
}
\end{enumerate}

\section*{inspire}
{\large \color{blue}  inspires  inspiring  inspired  }
\subsection*{Explain}
\begin{enumerate}
\item verb \\
If someone or something \textbf{inspires} you \textbf{to} do something new or unusual , they make you want to do it.
 \textit{
	\begin{itemize}
	\item These herbs will inspire you to try out all sorts of exotic-flavoured dishes!
	\item Our challenge is to motivate those voters and inspire them to join our cause.
	\item And what inspired you to change your name?
	\end{itemize}
}
\item verb \\
If someone or something \textbf{inspires} you, they give you new ideas and a strong  feeling of enthusiasm .
 \textit{
	\begin{itemize}
	\item Jimi Hendrix inspired a generation of guitarists.
	\end{itemize}
}
\item verb \\
If a book, work of art , or action \textbf{is inspired}  \textbf{by} something, that thing is the source of the idea for it.
 \textit{
	\begin{itemize}
	\item The book was inspired by a real person, namely Tamara de Treaux.
	\item ...a political murder inspired by the same nationalist conflicts now wrecking the
country.
	\end{itemize}
}
\item verb \\
Someone or something that \textbf{inspires} a particular emotion or reaction in people makes them feel this emotion or reaction.
 \textit{
	\begin{itemize}
	\item The car's performance quickly inspires confidence.
	\end{itemize}
}
\end{enumerate}

\section*{jog}
{\large \color{blue}  jogs  jogging  jogged  }
\subsection*{Explain}
\begin{enumerate}
\item verb \\
If you \textbf{jog} , you run slowly, often as a form of exercise.
 \textbf{Jog} is also a noun .
 \textit{
	\begin{itemize}
	\item I got up early the next morning to jog.
	\item He could scarcely jog around the block that first day.
	\item He went for another early morning jog.
	\end{itemize}
}
\item verb \\
If you \textbf{jog} something, you push or bump it slightly so that it moves.
 \textit{
	\begin{itemize}
	\item Avoid jogging the camera.
	\end{itemize}
}
\item  \\
 jog sb's memory \textit{
	\begin{itemize}
	\end{itemize}
}
\end{enumerate}

\section*{insulate}
{\large \color{blue}  insulates  insulating  insulated  }
\subsection*{Explain}
\begin{enumerate}
\item verb \\
If a person or group \textbf{is insulated}  \textbf{from} the rest of society or from outside influences , they are protected from them.
 \textit{
	\begin{itemize}
	\item They wonder if their community is no longer insulated from big city problems.
	\item Their wealth had insulated them from reality.
	\end{itemize}
}
\item verb \\
To \textbf{insulate} something such as a building  means to protect it from cold or noise by covering it or surrounding it in a thick  layer .
 \textit{
	\begin{itemize}
	\item ...a scheme to insulate the homes of pensioners and other low-income households.
	\item Is there any way we can insulate our home from the noise?
	\item Are your hot and cold water pipes well insulated?
	\item ...a garment lined with a light insulating material.
	\end{itemize}
}
\item verb \\
If a piece of equipment  \textbf{is insulated} , it is covered with rubber or plastic to prevent electricity passing through it and giving the person using it an electric  shock .
 \textit{
	\begin{itemize}
	\item In order to make it safe, the element is electrically insulated.
	\item ...electrical insulating tape.
	\end{itemize}
}
\end{enumerate}

\section*{knock}
{\large \color{blue}  knocks  knocking  knocked  }
\subsection*{Explain}
\begin{enumerate}
\item verb \\
If you \textbf{knock}  \textbf{on} something such as a door or window , you hit it, usually several times, to attract someone's attention.
 \textbf{Knock} is also a noun .
 \textit{
	\begin{itemize}
	\item She went directly to Simon's apartment and knocked on the door.
	\item Knock at my window at eight o'clock and I'll be ready.
	\item He knocked before going in.
	\item They heard a knock at the front door.
	\end{itemize}
}
\item verb \\
If you \textbf{knock} something, you touch or hit it roughly , especially so that it falls or moves.
 \textbf{Knock} is also a noun.
 \textit{
	\begin{itemize}
	\item She accidentally knocked the tea tin off the shelf.
	\item The baby was knocked from his father's arms.
	\item Isabel rose so abruptly that she knocked down her chair.
	\item Buckets of roses had been knocked over.
	\item The bags have tough exterior materials to protect against knocks, rain and dust.
	\end{itemize}
}
\item verb \\
If someone \textbf{knocks} two rooms or buildings  \textbf{into} one, or \textbf{knocks} them \textbf{together} , they make them form one room or building by removing a wall .
 \textit{
	\begin{itemize}
	\item They decided to knock the two rooms into one.
	\item The spacious kitchen was achieved by knocking together three small rooms.
	\end{itemize}
}
\item verb \\
To \textbf{knock} someone into a particular position or condition means to hit them very hard so that they fall over or become unconscious .
 \textit{
	\begin{itemize}
	\item The third wave was so strong it knocked me backwards.
	\item They were knocked to the ground and robbed of their wallets.
	\item Someone had knocked him unconscious.
	\end{itemize}
}
\item verb \\
To \textbf{knock} a particular quality or characteristic that someone has, or to \textbf{knock} it \textbf{out of} them means to make them lose it.
 \textit{
	\begin{itemize}
	\item Those people hurt me and knocked my confidence.
	\item The stories of his links with the actress had knocked the fun out of him.
	\item When they first joined for training many were starry eyed. We soon knocked that out
of them.
	\end{itemize}
}
\item verb \\
If something \textbf{knocks} , it makes a repeated  sharp  banging noise.
 \textit{
	\begin{itemize}
	\item His old truck, knocking and smoking, pulled down the road and out of sight.
	\end{itemize}
}
\item verb \\
If you \textbf{knock} something or someone, you criticize them and say  unpleasant things about them.
 \textit{
	\begin{itemize}
	\item I'm not knocking them: if they want to do it, it's up to them.
	\item Never knock charter flights; they are opening up the world for budget-conscious travellers.
	\end{itemize}
}
\item countable noun \\
If someone receives a \textbf{knock} , they have an unpleasant experience which prevents them from achieving something or which causes them to change their attitudes or plans .
 \textit{
	\begin{itemize}
	\item What they said was a real knock to my self-confidence.
	\item The art market has suffered some severe knocks during the past two years.
	\end{itemize}
}
\item  \\
 knock them/'em dead \textit{
	\begin{itemize}
	\end{itemize}
}
\item  \\
 knock it off \textit{
	\begin{itemize}
	\end{itemize}
}
\end{enumerate}

\section*{intend}
{\large \color{blue}  intends  intending  intended  }
\subsection*{Explain}
\begin{enumerate}
\item verb \\
If you \textbf{intend} to do something, you have decided or planned to do it.
 \textit{
	\begin{itemize}
	\item She intends to do A levels and go to university.
	\item I didn't intend coming to Germany to work.
	\item We had always intended that the new series would be live.
	\end{itemize}
}
\item verb \\
If something \textbf{is intended} for a particular purpose, it has been planned to fulfil that purpose. If something \textbf{is intended} for a particular person, it has been planned to be used by that person or to affect them in some way .
 \textit{
	\begin{itemize}
	\item This money is intended for the development of the tourist industry.
	\item Columns are usually intended in architecture to add grandeur and status.
	\item Originally, Hatfield had been intended as a leisure complex.
	\end{itemize}
}
\item verb \\
If you \textbf{intend} a particular idea or feeling in something that you say or do, you want to express it or want it to be understood .
 \textit{
	\begin{itemize}
	\item He didn't intend any sarcasm.
	\item His response seemed a little patronizing, though he undoubtedly hadn't intended it
that way.
	\item This sounds like a barrage of accusation–I don't intend it to be.
	\item I think he intended it as a put-down comment.
	\end{itemize}
}
\end{enumerate}

\section*{leak}
{\large \color{blue}  leaks  leaking  leaked  }
\subsection*{Explain}
\begin{enumerate}
\item verb \\
If a container  \textbf{leaks} , there is a hole or crack in it which lets a substance such as liquid or gas escape. You can also  say that a container \textbf{leaks} a substance such as liquid or gas.
 \textbf{Leak} is also a noun .
 \textit{
	\begin{itemize}
	\item The roof leaked.
	\item The gas had apparently leaked from a cylinder.
	\item The pool's fiberglass sides had cracked and the water had leaked out.
	\item A large diesel tank mysteriously leaked its contents into the river.
	\item It's thought a gas leak may have caused the blast.
	\end{itemize}
}
\item countable noun \\
A \textbf{leak} is a crack, hole, or other gap that a substance such as a liquid or gas can pass through.
 \textit{
	\begin{itemize}
	\item ...a leak in the radiator.
	\item In May engineers found a leak in a hydrogen fuel line.
	\end{itemize}
}
\item verb \\
If a secret document or piece of information \textbf{leaks} or \textbf{is leaked} , someone lets the public know about it.
 \textbf{Leak} is also a noun.
 \textbf{Leak out} means the same as leak .
 \textit{
	\begin{itemize}
	\item Last year, a civil servant was imprisoned for leaking a document to the press.
	\item He revealed who leaked a confidential police report.
	\item We don't know how the transcript leaked.
	\item ...a leaked report.
	\item Leaks involving national security will be investigated by the police.
	\item More details are now beginning to leak out.
	\item He said it would leak out to the newspapers and cause a scandal.
	\end{itemize}
}
\end{enumerate}

\section*{interpret}
{\large \color{blue}  interprets  interpreting  interpreted  }
\subsection*{Explain}
\begin{enumerate}
\item verb \\
If you \textbf{interpret} something in a particular way, you decide that this is its meaning or significance.
 \textit{
	\begin{itemize}
	\item The whole speech might well be interpreted as a coded message to the Americans.
	\item The judge quite rightly says that he has to interpret the law as it's been passed.
	\item Both approaches agree on what is depicted in the poem, but not on how it should be
interpreted.
	\end{itemize}
}
\item verb \\
If you \textbf{interpret} what someone is saying , you translate it immediately into another language.
 \textit{
	\begin{itemize}
	\item The chambermaid spoke little English, so her husband came with her to interpret.
	\item Interpreters found they could not interpret half of what he said.
	\end{itemize}
}
\end{enumerate}

\section*{loosen}
{\large \color{blue}  loosens  loosening  loosened  }
\subsection*{Explain}
\begin{enumerate}
\item verb \\
If someone \textbf{loosens}  restrictions or laws , for example , they make them less strict or severe.
 \textit{
	\begin{itemize}
	\item Many business groups have been pressing the Federal Reserve to loosen interest rates.
	\item Drilling regulations, too, have been loosened to speed the development of the fields.
	\end{itemize}
}
\item verb \\
If someone or something \textbf{loosens} the ties between people or groups of people, or if the ties \textbf{loosen} , they become weaker .
 \textit{
	\begin{itemize}
	\item The Federal Republic must loosen its ties with the United States.
	\item The deputy leader is cautious about loosening the links with the unions.
	\item The ties that bind them together are loosening.
	\end{itemize}
}
\item verb \\
If you \textbf{loosen} your clothing or something that is tied or fastened or if it \textbf{loosens} , you undo it slightly so that it is less tight or less firmly held in place.
 \textit{
	\begin{itemize}
	\item He reached up to loosen the scarf around his neck.
	\item Loosen the bolt so the bars can be turned.
	\item Her hair had loosened and was tangled around her shoulders.
	\end{itemize}
}
\item verb \\
If you \textbf{loosen} something that is stretched across something else, you make it less stretched or tight.
 \textit{
	\begin{itemize}
	\item Insert a small knife into the top of the chicken breast to loosen the skin.
	\end{itemize}
}
\item verb \\
If you \textbf{loosen} your grip on something, or if your grip \textbf{loosens} , you hold it less tightly.
 \textit{
	\begin{itemize}
	\item Harry loosened his grip momentarily and Anna wriggled free.
	\item When his grip loosened she eased herself away.
	\end{itemize}
}
\item verb \\
If a government or organization  \textbf{loosens} its grip on a group of people or an activity , or if its grip \textbf{loosens} , it begins to have less control over it.
 \textit{
	\begin{itemize}
	\item There is no sign that the Party will loosen its tight grip on the country.
	\item The President's own grip on power has loosened.
	\end{itemize}
}
\item  \\
 loosen someone's tongue \textit{
	\begin{itemize}
	\end{itemize}
}
\end{enumerate}

\section*{invite}
{\large \color{blue}  invites  inviting  invited  }
\subsection*{Explain}
\begin{enumerate}
\item verb \\
If you \textbf{invite} someone to something such as a party or a meal , you ask them to come to it.
 \textit{
	\begin{itemize}
	\item She invited him to her 26th birthday party in New Jersey.
	\item I invited her in for a coffee.
	\item Neighbours have invited us out, given us clothes, and taken us on excursions.
	\item Barron invited her to accompany him to the races.
	\item Sometimes it seems right to invite an entire class of children so no one will feel
left out.
	\item I haven't been invited.
	\item ...an invited audience of children from inner-city schools.
	\end{itemize}
}
\item verb \\
If you \textbf{are invited}  \textbf{to} do something, you are formally asked or given permission to do it.
 \textit{
	\begin{itemize}
	\item At a future date, managers will be invited to apply for a management buy-out.
	\item The person concerned would be shown the evidence in private and invited to stand
down.
	\item If a new leader emerged, it would then be for the Queen to invite him to form a government.
	\item The Department is inviting applications from groups within the Borough.
	\end{itemize}
}
\item verb \\
If something you say or do \textbf{invites}  trouble or criticism , it makes trouble or criticism more likely .
 \textit{
	\begin{itemize}
	\item Their refusal to compromise will inevitably invite more criticism from the U.N.
	\end{itemize}
}
\item countable noun \\
An \textbf{invite} is an invitation to something such as a party or a meal.
 \textit{
	\begin{itemize}
	\item They haven't got an invite to the wedding.
	\end{itemize}
}
\end{enumerate}

\section*{mock}
{\large \color{blue}  mocks  mocking  mocked  }
\subsection*{Explain}
\begin{enumerate}
\item verb \\
If someone \textbf{mocks} you, they show or pretend that they think you are foolish or inferior , for example by saying something funny about you, or by imitating your behaviour.
 \textit{
	\begin{itemize}
	\item I thought you were mocking me.
	\item I distinctly remember mocking the idea.
	\item 'I'm astonished, Benjamin,' she mocked.
	\end{itemize}
}
\item adjective \\
You use \textbf{mock} to describe something which is not real or genuine , but which is intended to be very similar to the real thing.
 \textit{
	\begin{itemize}
	\item 'It's tragic!' swoons Jeffrey in mock horror.
	\item ...a mock Tudor mansion.
	\end{itemize}
}
\item countable noun \\
\textbf{Mocks} are practice exams that you take as part of your preparation for real exams.
 \textit{
	\begin{itemize}
	\item She went from a D in her mocks to a B in the real thing.
	\end{itemize}
}
\end{enumerate}

\section*{irrigate}
{\large \color{blue}  irrigates  irrigating  irrigated  }
\subsection*{Explain}
\begin{enumerate}
\item verb \\
To \textbf{irrigate} land means to supply it with water in order to help crops grow .
 \textit{
	\begin{itemize}
	\item None of the water from Lake Powell is used to irrigate the area.
	\item ...strips of cultivated land irrigated by a maze of interconnected canals.
	\end{itemize}
}
\end{enumerate}

\section*{perceive}
{\large \color{blue}  perceives  perceiving  perceived  }
\subsection*{Explain}
\begin{enumerate}
\item verb \\
If you \textbf{perceive} something, you see , notice , or realize it, especially when it is not obvious .
 \textit{
	\begin{itemize}
	\item Many young people do not perceive the need to consider pensions at all.
	\item 'Precisely what other problems do you perceive?' she asked.
	\end{itemize}
}
\item verb \\
If you \textbf{perceive} someone or something \textbf{as} doing or being a particular thing, it is your opinion that they do this thing or that they are that thing.
 \textit{
	\begin{itemize}
	\item Stress is widely perceived as contributing to coronary heart disease.
	\item They strangely perceive television as entertainment.
	\end{itemize}
}
\end{enumerate}

\section*{irritate}
{\large \color{blue}  irritates  irritating  irritated  }
\subsection*{Explain}
\begin{enumerate}
\item verb \\
If something \textbf{irritates} you, it keeps annoying you.
 \textit{
	\begin{itemize}
	\item Their attitude irritates me.
	\item Perhaps they were irritated by the sound of crying.
	\end{itemize}
}
\item verb \\
If something \textbf{irritates} a part of your body, it causes it to itch or become sore .
 \textit{
	\begin{itemize}
	\item Wear rubber gloves while chopping chillies as they can irritate the skin.
	\end{itemize}
}
\end{enumerate}

\section*{quench}
{\large \color{blue}  quenches  quenching  quenched  }
\subsection*{Explain}
\begin{enumerate}
\item verb \\
If someone who is thirsty  \textbf{quenches} their \textbf{thirst} , they lose their thirst by having a drink.
 \textit{
	\begin{itemize}
	\item He stopped to quench his thirst at a stream.
	\end{itemize}
}
\end{enumerate}

\section*{isolate}
{\large \color{blue}  isolates  isolating  isolated  }
\subsection*{Explain}
\begin{enumerate}
\item verb \\
To \textbf{isolate} a person or organization  means to cause them to lose their friends or supporters .
 \textit{
	\begin{itemize}
	\item This policy could isolate the country from the other permanent members of the United
Nations Security Council.
	\item Political influence is being used to shape public opinion and isolate critics.
	\end{itemize}
}
\item verb \\
If you \textbf{isolate}  \textbf{yourself} , or if something \textbf{isolates} you, you become physically or socially separated from other people.
 \textit{
	\begin{itemize}
	\item When he was thinking out a problem Tweed's habit was never to isolate himself in
his room.
	\item His radicalism and refusal to compromise isolated him.
	\item Police officers had a siege mentality that isolated them from the people they served.
	\item But of course no one lives totally alone, isolated from the society around them.
	\end{itemize}
}
\item verb \\
If you \textbf{isolate} something such as an idea or a problem , you separate it from others that it is connected with, so that you can  concentrate on it or consider it on its own.
 \textit{
	\begin{itemize}
	\item Our anxieties can also be controlled by isolating thoughts, feelings and memories.
	\item Gandhi said that those who isolate religion from politics don't understand the nature
of either.
	\end{itemize}
}
\item verb \\
To \textbf{isolate} a substance means to obtain it by separating it from other substances using scientific processes.
 \textit{
	\begin{itemize}
	\item We can use genetic engineering techniques to isolate the gene that is responsible.
	\item Researchers have isolated a new protein from the seeds of poppies.
	\item ...the chemical isolated from brain tissue.
	\end{itemize}
}
\item verb \\
To \textbf{isolate} a sick person or animal means to keep them apart from other people or animals, so that their illness does not spread .
 \textit{
	\begin{itemize}
	\item She had swine flu and was isolated from her children.
	\item You don't have to isolate them from the community.
	\end{itemize}
}
\end{enumerate}

\section*{refine}
{\large \color{blue}  refines  refining  refined  }
\subsection*{Explain}
\begin{enumerate}
\item verb \\
When a substance \textbf{is refined} , it is made pure by having all other substances removed from it.
 \textit{
	\begin{itemize}
	\item Oil is refined to remove naturally occurring impurities.
	\end{itemize}
}
\item verb \\
If something such as a process, theory , or machine  \textbf{is refined} , it is improved by having small changes made to it.
 \textit{
	\begin{itemize}
	\item Surgical techniques are constantly being refined.
	\end{itemize}
}
\end{enumerate}

\section*{liberate}
{\large \color{blue}  liberates  liberating  liberated  }
\subsection*{Explain}
\begin{enumerate}
\item verb \\
To \textbf{liberate} a place or the people in it means to free them from the political or military  control of another country , area, or group of people.
 \textit{
	\begin{itemize}
	\item They planned to march on and liberate the city.
	\item They made a triumphal march into their liberated city.
	\end{itemize}
}
\item verb \\
To \textbf{liberate} someone \textbf{from} something means to help them escape from it or overcome it, and lead a better  way of life .
 \textit{
	\begin{itemize}
	\item He asked how committed the leadership was to liberating its people from poverty.
	\end{itemize}
}
\item verb \\
To \textbf{liberate} a prisoner means to set them free.
 \textit{
	\begin{itemize}
	\item The government is devising a plan to liberate prisoners held in detention camps.
	\end{itemize}
}
\end{enumerate}

\section*{regulate}
{\large \color{blue}  regulates  regulating  regulated  }
\subsection*{Explain}
\begin{enumerate}
\item verb \\
To \textbf{regulate} an activity or process means to control it, especially by means of rules.
 \textit{
	\begin{itemize}
	\item Serious reform is needed to improve institutions that regulate competition.
	\item As we get older the temperature-regulating mechanisms in the body become less efficient.
	\end{itemize}
}
\end{enumerate}

\section*{lick}
{\large \color{blue}  licks  licking  licked  }
\subsection*{Explain}
\begin{enumerate}
\item verb \\
When people or animals \textbf{lick} something, they move their tongue across its surface.
 \textbf{Lick} is also a noun .
 \textit{
	\begin{itemize}
	\item She folded up her letter, licking the envelope flap with relish.
	\item The dog rose awkwardly to his feet and licked the man's hand excitedly.
	\item Kevin wanted a lick of Sarah's lollipop.
	\end{itemize}
}
\item verb \\
If you \textbf{lick} someone or something, you easily defeat them in a fight or competition .
 \textit{
	\begin{itemize}
	\item He might be able to lick us all in a fair fight.
	\item The Chancellor's upbeat message that the Government had licked inflation for good
was marred by more job losses.
	\end{itemize}
}
\item verb \\
When flames of a large fire  \textbf{lick}  somewhere or something, the fire begins to reach that place or thing and the flames touch it lightly and briefly.
 \textit{
	\begin{itemize}
	\item The fire sent its red tongues licking into the entrance hall.
	\item The apex of the flames licked the crimson sky.
	\end{itemize}
}
\item countable noun \\
A \textbf{lick}  \textbf{of} something is a small amount of it.
 \textit{
	\begin{itemize}
	\item It could do with a lick of paint to brighten up its premises.
	\end{itemize}
}
\item countable noun \\
A \textbf{lick} is a short piece of music which is part of a song and is played on a guitar . A \textbf{lick} is also a short section in a piece of jazz , which the musician  invents while they are playing.
 \textit{
	\begin{itemize}
	\item ...the screeching licks of heavy metal guitar.
	\end{itemize}
}
\end{enumerate}

\section*{reinforce}
{\large \color{blue}  reinforces  reinforcing  reinforced  }
\subsection*{Explain}
\begin{enumerate}
\item verb \\
If something \textbf{reinforces} a feeling , situation , or process, it makes it stronger or more intense .
 \textit{
	\begin{itemize}
	\item A stronger European Parliament would, they fear, only reinforce the power of the
larger countries.
	\item This sense of privilege tends to be reinforced by the outside world.
	\end{itemize}
}
\item verb \\
If something \textbf{reinforces} an idea or point of view , it provides more evidence or support for it.
 \textit{
	\begin{itemize}
	\item The delegation hopes to reinforce the idea that human rights are not purely internal
matters.
	\end{itemize}
}
\item verb \\
To \textbf{reinforce} an object means to make it stronger or harder .
 \textit{
	\begin{itemize}
	\item Eventually, they had to reinforce the walls with exterior beams.
	\end{itemize}
}
\item verb \\
To \textbf{reinforce} an army or a police force means to make it stronger by increasing its size or providing it with more weapons . To \textbf{reinforce} a position or place means to make it stronger by sending more soldiers or weapons.
 \textit{
	\begin{itemize}
	\item Both sides have been reinforcing their positions after yesterday's fierce fighting.
	\item Troops and police have been reinforced in the city.
	\end{itemize}
}
\end{enumerate}

\section*{skim}
{\large \color{blue}  skims  skimming  skimmed  }
\subsection*{Explain}
\begin{enumerate}
\item verb \\
If you \textbf{skim} something \textbf{from} the surface of a liquid, you remove it.
 \textit{
	\begin{itemize}
	\item Rough seas today prevented specially equipped ships from skimming oil off the water's
surface.
	\item Skim off the fat.
	\end{itemize}
}
\item verb \\
If something \textbf{skims} a surface, it moves quickly along just above it.
 \textit{
	\begin{itemize}
	\item ...seagulls skimming the waves.
	\item The little boat was skimming across the sunlit surface of the bay.
	\end{itemize}
}
\item verb \\
If you \textbf{skim} a piece of writing , you read through it quickly.
 \textit{
	\begin{itemize}
	\item He skimmed the pages quickly, then read them again more carefully.
	\item I only had time to skim through the script before I flew over here.
	\end{itemize}
}
\end{enumerate}

\section*{require}
{\large \color{blue}  requires  requiring  required  }
\subsection*{Explain}
\begin{enumerate}
\item verb \\
If you \textbf{require} something or if something \textbf{is required} , you need it or it is necessary.
 \textit{
	\begin{itemize}
	\item If you require further information, you should consult the registrar.
	\item This isn't the kind of crisis that requires us to drop everything else.
	\item Some of the materials required for this technique may be difficult to obtain.
	\end{itemize}
}
\item verb \\
If a law or rule \textbf{requires} you \textbf{to} do something, you have to do it.
 \textit{
	\begin{itemize}
	\item The rules also require employers to provide safety training.
	\item At least 35 manufacturers have flouted a law requiring prompt reporting of such malfunctions.
	\item The law requires that employees are given the opportunity to improve their performance
before they are dismissed.
	\item Then he'll know exactly what's required of him.
	\end{itemize}
}
\item  \\
 required reading \textit{
	\begin{itemize}
	\end{itemize}
}
\end{enumerate}

\section*{sniff}
{\large \color{blue}  sniffs  sniffing  sniffed  }
\subsection*{Explain}
\begin{enumerate}
\item verb \\
When you \textbf{sniff} , you breathe in air through your nose hard enough to make a sound, for example when you are trying not to cry , or in order to show  disapproval .
 \textbf{Sniff} is also a noun .
 \textit{
	\begin{itemize}
	\item She wiped her face and sniffed loudly.
	\item Moira looked around and sniffed. 'This place badly needs a decorator.'.
	\item Then he sniffed. There was a smell of burning.
	\item He sniffed back the tears.
	\item At last the sobs ceased, to be replaced by sniffs.
	\end{itemize}
}
\item verb \\
If you \textbf{sniff} something or \textbf{sniff at} it, you smell it by sniffing.
 \textit{
	\begin{itemize}
	\item Suddenly, he stopped and sniffed the air.
	\item She sniffed at it suspiciously.
	\end{itemize}
}
\item verb \\
You can use \textbf{sniff} to indicate that someone says something in a way that shows their disapproval or contempt .
 \textit{
	\begin{itemize}
	\item 'Tourists!' she sniffed.
	\end{itemize}
}
\item verb \\
If you say that something is \textbf{not to be sniffed at} , you think it is very good or worth having. If someone \textbf{sniffs at} something, they do not think it is good enough, or they express their contempt for it.
 \textit{
	\begin{itemize}
	\item The salary was not to be sniffed at either.
	\item Foreign Office sources sniffed at reports that British troops might be sent.
	\end{itemize}
}
\item verb \\
If someone \textbf{sniffs} a substance such as glue , they deliberately breathe in the substance or the gases from it as a drug.
 \textit{
	\begin{itemize}
	\item He felt light-headed, as if he'd sniffed glue.
	\end{itemize}
}
\item singular noun \\
If you get a \textbf{sniff}  \textbf{of} something, you learn or guess that it might be happening or might be near.
 \textit{
	\begin{itemize}
	\item You know what they'll be like if they get a sniff of a murder investigation.
	\item Have the Press got a sniff yet?
	\item Then, at the first sniff of danger, he was back at his post.
	\end{itemize}
}
\item singular noun \\
If you say that someone has not had \textbf{a}  \textbf{sniff}  \textbf{of} something, you mean that they have not had even a small chance of getting it.
 \textit{
	\begin{itemize}
	\item The club hasn't had a sniff of winning a title for twenty years.
	\end{itemize}
}
\end{enumerate}

\section*{retreat}
{\large \color{blue}  retreats  retreating  retreated  }
\subsection*{Explain}
\begin{enumerate}
\item verb \\
If you \textbf{retreat} , you move away from something or someone.
 \textit{
	\begin{itemize}
	\item 'I've already got a job,' I said quickly, and retreated from the room.
	\item The young nurse pulled a face at the Matron's retreating figure.
	\end{itemize}
}
\item verb \\
When an army  \textbf{retreats} , it moves away from enemy forces in order to avoid  fighting them.
 \textbf{Retreat} is also a noun .
 \textit{
	\begin{itemize}
	\item The French, suddenly outnumbered, were forced to retreat.
	\item Retreating soldiers were dousing homes and shops with petrol and setting them on
fire.
	\item In June 1942, the British 8th Army was in full retreat.
	\end{itemize}
}
\item verb \\
If you \textbf{retreat}  \textbf{from} something such as a plan or a way of life, you give it up, usually in order to do something safer or less extreme .
 \textbf{Retreat} is also a noun.
 \textit{
	\begin{itemize}
	\item To save yourself, you sometimes need to retreat from the world.
	\item From bouncing confidence she had retreated into self-pity.
	\item Downing Street insisted that there would be no retreat from the £26,000 cap on the
amount of benefits that any family could claim.
	\item It's a retreat into the adolescence they never really had.
	\end{itemize}
}
\item countable noun \\
A \textbf{retreat} is a quiet, isolated place that you go to in order to rest or to do things in private.
 \textit{
	\begin{itemize}
	\item He spent yesterday hidden away in his country retreat.
	\end{itemize}
}
\item  \\
 to beat a retreat \textit{
	\begin{itemize}
	\end{itemize}
}
\end{enumerate}

\section*{stagger}
{\large \color{blue}  staggers  staggering  staggered  }
\subsection*{Explain}
\begin{enumerate}
\item verb \\
If you \textbf{stagger} , you walk very unsteadily, for example because you are ill or drunk.
 \textit{
	\begin{itemize}
	\item He lost his balance, staggered back against the rail and toppled over.
	\item He was staggering and had to lean on the bar.
	\end{itemize}
}
\item verb \\
If you say that someone or something \textbf{staggers}  \textbf{on} , you mean that it is only just succeeds in continuing .
 \textit{
	\begin{itemize}
	\item Truman allowed him to stagger on for nearly another two years.
	\item ...a government that staggered from crisis to crisis.
	\end{itemize}
}
\item verb \\
If something \textbf{staggers} you, it surprises you very much.
 \textit{
	\begin{itemize}
	\item The whole thing staggers me.
	\end{itemize}
}
\item verb \\
To \textbf{stagger} things such as people's holidays or hours of work means to arrange them so that they do not all happen at the same time.
 \textit{
	\begin{itemize}
	\item During the past few years the government has staggered summer vacation periods.
	\end{itemize}
}
\end{enumerate}

\section*{rip}
{\large \color{blue}  rips  ripping  ripped  }
\subsection*{Explain}
\begin{enumerate}
\item verb \\
When something \textbf{rips} or when you \textbf{rip} it, you tear it forcefully with your hands or with a tool such as a knife .
 \textit{
	\begin{itemize}
	\item I felt the banner rip as we were pushed in opposite directions.
	\item I tried not to rip the paper as I unwrapped it.
	\end{itemize}
}
\item countable noun \\
A \textbf{rip} is a long cut or split in something made of cloth or paper.
 \textit{
	\begin{itemize}
	\item Looking at the rip in her new dress, she flew into a rage.
	\end{itemize}
}
\item verb \\
If you \textbf{rip} something away , you remove it quickly and forcefully.
 \textit{
	\begin{itemize}
	\item He ripped away a wire that led to the alarm button.
	\item He ripped the phone from her hand.
	\end{itemize}
}
\item verb \\
If something \textbf{rips} into someone or something or \textbf{rips} through them, it enters that person or thing so quickly and forcefully that it often goes completely through them.
 \textit{
	\begin{itemize}
	\item A volley of bullets ripped into the facing wall.
	\item The fire ripped through the living room.
	\item A violent streak of pain ripped through her whole body.
	\end{itemize}
}
\item  \\
 let rip \textit{
	\begin{itemize}
	\end{itemize}
}
\item  \\
 let something rip \textit{
	\begin{itemize}
	\end{itemize}
}
\end{enumerate}

\section*{surpass}
{\large \color{blue}  surpasses  surpassing  surpassed  }
\subsection*{Explain}
\begin{enumerate}
\item verb \\
If one person or thing \textbf{surpasses} another, the first is better than, or has more of a particular quality than, the second .
 \textit{
	\begin{itemize}
	\item He was determined to surpass the achievements of his older brothers.
	\item Warwick Arts Centre is the second largest Arts Centre in Britain, surpassed in size
only by London's Barbican.
	\end{itemize}
}
\item verb \\
If something \textbf{surpasses}  expectations , it is much better than it was expected to be.
 \textit{
	\begin{itemize}
	\item Conrad Black gave an excellent party that surpassed expectations.
	\end{itemize}
}
\item verb \\
If something \textbf{surpasses}  understanding , it is too difficult to understand .
 \textit{
	\begin{itemize}
	\item ...a clever system, the complexity of which surpasses our understanding.
	\end{itemize}
}
\end{enumerate}

\section*{roll}
{\large \color{blue}  rolls  rolling  rolled  }
\subsection*{Explain}
\begin{enumerate}
\item verb \\
When something \textbf{rolls} or when you \textbf{roll} it, it moves along a surface, turning over many times.
 \textit{
	\begin{itemize}
	\item The ball rolled into the net.
	\item Their car went off the road and rolled over.
	\item I rolled a ball across the carpet.
	\item Roll the meat in coarsely ground black pepper to season it.
	\end{itemize}
}
\item verb \\
If you \textbf{roll}  somewhere , you move on a surface while lying down, turning your body over and over, so that
you are sometimes on your back, sometimes on your side, and sometimes on your front.
 \textit{
	\begin{itemize}
	\item When I was a little kid I rolled down a hill and broke my leg.
	\item They just rolled about on the floor punching each other like schoolboys.
	\item She rolled over and propped herself up on her elbows.
	\end{itemize}
}
\item verb \\
When vehicles \textbf{roll} along, they move along slowly.
 \textit{
	\begin{itemize}
	\item The lorry quietly rolled forward.
	\end{itemize}
}
\item verb \\
If a machine \textbf{rolls} , it is operating.
 \textit{
	\begin{itemize}
	\item He slipped and fell on an airplane gangway as the cameras rolled.
	\item The newspaper presses are rolling in Pittsburgh again today.
	\end{itemize}
}
\item verb \\
If drops of liquid \textbf{roll} down a surface, they move quickly down it.
 \textit{
	\begin{itemize}
	\item She looked at Ginny and tears rolled down her cheeks.
	\end{itemize}
}
\item verb \\
If you \textbf{roll} something flexible  \textbf{into} a cylinder or a ball, you form it into a cylinder or a ball by wrapping it several times around itself or by shaping it between your hands.
 \textbf{Roll up} means the same as roll .
 \textit{
	\begin{itemize}
	\item He took off his sweater, rolled it into a pillow and lay down on the grass.
	\item He rolled a cigarette.
	\item Stein rolled up the paper bag with the money inside.
	\end{itemize}
}
\item countable noun \\
A \textbf{roll}  \textbf{of} paper, plastic, cloth, or wire is a long piece of it that has been wrapped many times around itself or around a tube.
 \textit{
	\begin{itemize}
	\item The photographers had already shot a dozen rolls of film.
	\item ...a roll of blue insulated wire.
	\end{itemize}
}
\item verb \\
If you \textbf{roll}  \textbf{up} something such as a car window or a blind , you cause it to move upwards by turning a handle. If you \textbf{roll} it \textbf{down} , you cause it to move downwards by turning a handle.
 \textit{
	\begin{itemize}
	\item In mid-afternoon, shopkeepers began to roll down their shutters.
	\item She rolled up the window and drove on.
	\item He rolled his window down and gave the man the money.
	\end{itemize}
}
\item verb \\
If you \textbf{roll} your eyes or if your eyes \textbf{roll} , they move round and upwards. People sometimes roll their eyes when they are frightened , bored , or annoyed .
 \textit{
	\begin{itemize}
	\item People may roll their eyes and talk about overprotective, interfering grandmothers.
	\item His eyes rolled and he sobbed.
	\end{itemize}
}
\item countable noun \\
A \textbf{roll} is a small piece of bread that is round or long and is made to be eaten by one person.
Rolls can be eaten plain, with butter, or with a filling.
 \textit{
	\begin{itemize}
	\item He spread butter on a roll.
	\end{itemize}
}
\item countable noun \\
A \textbf{roll}  \textbf{of} drums is a long, low, fairly loud sound made by drums.
 \textit{
	\begin{itemize}
	\item As the town clock struck two, they heard the roll of drums.
	\end{itemize}
}
\item countable noun \\
A \textbf{roll} is an official list of people's names.
 \textit{
	\begin{itemize}
	\item ...the electoral roll.
	\end{itemize}
}
\item  \\
 on a roll \textit{
	\begin{itemize}
	\end{itemize}
}
\item  \\
 roll on sth \textit{
	\begin{itemize}
	\end{itemize}
}
\item  \\
 rolled into one \textit{
	\begin{itemize}
	\end{itemize}
}
\end{enumerate}

\section*{sway}
{\large \color{blue}  sways  swaying  swayed  }
\subsection*{Explain}
\begin{enumerate}
\item verb \\
When people or things \textbf{sway} , they lean or swing slowly from one side to the other.
 \textit{
	\begin{itemize}
	\item The people swayed back and forth with arms linked.
	\item The whole boat swayed and tipped.
	\item ...a coastal highway lined with tall, swaying palm trees.
	\end{itemize}
}
\item verb \\
If you \textbf{are swayed}  \textbf{by} someone or something, you are influenced by them.
 \textit{
	\begin{itemize}
	\item Don't ever be swayed by fashion.
	\item ...last minute efforts to sway the voters in tomorrow's local elections.
	\end{itemize}
}
\item  \\
 to hold sway \textit{
	\begin{itemize}
	\end{itemize}
}
\item  \\
 under the sway of sb/sth \textit{
	\begin{itemize}
	\end{itemize}
}
\end{enumerate}

\section*{satisfy}
{\large \color{blue}  satisfies  satisfying  satisfied  }
\subsection*{Explain}
\begin{enumerate}
\item verb \\
If someone or something \textbf{satisfies} you, they give you enough of what you want or need to make you pleased or contented .
 \textit{
	\begin{itemize}
	\item The pace of change has not been quick enough to satisfy everyone.
	\item We just can't find enough good second-hand cars to satisfy demand.
	\item The scandal stories satisfy people's curiosity for a few hours.
	\end{itemize}
}
\item verb \\
To \textbf{satisfy} someone \textbf{that} something is true or has been done properly means to convince them by giving them more information or by showing them what has been done.
 \textit{
	\begin{itemize}
	\item He has to satisfy the environmental lobby that real progress will be made to cut
emissions.
	\item He wanted to satisfy himself that he had given his best performance.
	\end{itemize}
}
\item verb \\
If you \textbf{satisfy} the requirements for something, you are good enough or have the right qualities to fulfil these requirements.
 \textit{
	\begin{itemize}
	\item The procedures should satisfy certain basic requirements.
	\end{itemize}
}
\end{enumerate}

\section*{swim}
{\large \color{blue}  swims  swimming  swam  swum  }
\subsection*{Explain}
\begin{enumerate}
\item verb \\
When you \textbf{swim} , you move through water by making movements with your arms and legs.
 \textbf{Swim} is also a noun .
 \textit{
	\begin{itemize}
	\item She learned to swim when she was really tiny.
	\item I went round to Jonathan's to see if he wanted to go swimming.
	\item He was rescued only when an exhausted friend swam ashore.
	\item I swim a mile a day.
	\item When can we go for a swim, Mam?
	\end{itemize}
}
\item verb \\
If you \textbf{swim} a race, you take part in a swimming race.
 \textit{
	\begin{itemize}
	\item She swam the 200 metres semi-finals and came second.
	\end{itemize}
}
\item verb \\
If you \textbf{swim} a stretch of water, you keep swimming until you have crossed it.
 \textit{
	\begin{itemize}
	\item In 1875, Captain Matthew Webb became the first man to swim the English Channel.
	\end{itemize}
}
\item verb \\
When a fish \textbf{swims} , it moves through water by moving its body.
 \textit{
	\begin{itemize}
	\item The barriers are lethal to fish trying to swim upstream.
	\end{itemize}
}
\item verb \\
If objects  \textbf{swim} , they seem to be moving backwards and forwards , usually because you are ill .
 \textit{
	\begin{itemize}
	\item Alexis suddenly could take no more: he felt too hot, he couldn't breathe, the room
swam.
	\end{itemize}
}
\item verb \\
If your head  \textbf{is swimming} , you feel  unsteady and slightly ill.
 \textit{
	\begin{itemize}
	\item The musty aroma of incense made her head swim.
	\end{itemize}
}
\item verb \\
If something \textbf{is swimming}  \textbf{in} liquid or \textbf{is swimming}  \textbf{with} liquid, it is surrounded by and covered with it.
 \textit{
	\begin{itemize}
	\item He polished off a large steak and broccoli swimming in thick sauce.
	\end{itemize}
}
\end{enumerate}

\section*{sew}
{\large \color{blue}  sews  sewing  sewed  sewn  }
\subsection*{Explain}
\begin{enumerate}
\item verb \\
When you \textbf{sew} something such as clothes , you make them or repair them by joining pieces of cloth together by passing thread through them with a needle.
 \textit{
	\begin{itemize}
	\item She sewed the dresses on the sewing machine.
	\item Anyone can sew on a button, including you.
	\item Mrs Roberts was a dressmaker, and she taught her daughter to sew.
	\end{itemize}
}
\item verb \\
When something such as a hand or finger  \textbf{is sewn}  \textbf{back} by a doctor , it is joined with the patient's body using a needle and thread.
 \textit{
	\begin{itemize}
	\item The hand was preserved in ice by neighbours and sewn back on in hospital.
	\item Surgeons at Odstock Hospital, Wilts, sewed the thumb on.
	\end{itemize}
}
\end{enumerate}

\section*{thank}
{\large \color{blue}  thanks  thanking  thanked  }
\subsection*{Explain}
\begin{enumerate}
\item convention \\
You use \textbf{thank you} or, in more informal  English , \textbf{thanks} to express your gratitude when someone does something for you or gives you what you want .
 \textit{
	\begin{itemize}
	\item Thank you very much for your call.
	\item Thanks for the information.
	\item Oh thank you so much! They're so pretty!
	\item Thanks a lot, Suzie. You've been great.
	\end{itemize}
}
\item convention \\
You use \textbf{thank you} or, in more informal English, \textbf{thanks} to politely accept or refuse something that has just been offered to you.
 \textit{
	\begin{itemize}
	\item 'You'd like a cup as well, would you, Mr Secombe?'—'Thank you, Jane, I'd love one.'
	\item 'Would you like a biscuit?'—'No thank you.'
	\item 'A coffee?'—'I'd better not, thanks.'
	\end{itemize}
}
\item convention \\
You use \textbf{thank you} or, in more informal English, \textbf{thanks} to politely acknowledge what someone has said to you, especially when they have answered your question or said something nice to you.
 \textit{
	\begin{itemize}
	\item The policeman smiled at her. 'Pretty dog.'—'Oh well, thank you.'
	\item 'His eyes were glassy?'—'And dilated. They were watery.'—'Thank you.'
	\item 'It's great to see you.'—'Thanks. Same to you.'
	\end{itemize}
}
\item convention \\
You use \textbf{thank you} or \textbf{thank you very much} in order to say firmly that you do not want someone's help or to tell them that you do not like the way that they are behaving towards you.
 \textit{
	\begin{itemize}
	\item I can stir my own tea, thank you.
	\item We know where we can get it, thank you very much.
	\end{itemize}
}
\item verb \\
When you \textbf{thank} someone \textbf{for} something, you express your gratitude to them for it.
 \textit{
	\begin{itemize}
	\item I thanked them for their long and loyal service.
	\item When the decision was read out Mrs Gardner thanked the judges.
	\end{itemize}
}
\item plural noun \\
When you express your \textbf{thanks} to someone, you express your gratitude to them for something.
 \textit{
	\begin{itemize}
	\item They accepted their certificates with words of thanks.
	\end{itemize}
}
\item  \\
 give thanks \textit{
	\begin{itemize}
	\end{itemize}
}
\item  \\
 thank God \textit{
	\begin{itemize}
	\end{itemize}
}
\item  \\
 have sb to thank \textit{
	\begin{itemize}
	\end{itemize}
}
\item  \\
 thanks to \textit{
	\begin{itemize}
	\end{itemize}
}
\item  \\
 no thanks to \textit{
	\begin{itemize}
	\end{itemize}
}
\end{enumerate}

\section*{simulate}
{\large \color{blue}  simulates  simulating  simulated  }
\subsection*{Explain}
\begin{enumerate}
\item verb \\
If you \textbf{simulate} an action or a feeling , you pretend that you are doing it or feeling it.
 \textit{
	\begin{itemize}
	\item They rolled about on the Gilligan Road, simulating a bloodthirsty fight.
	\item He performed a simulated striptease.
	\end{itemize}
}
\item verb \\
If you \textbf{simulate} an object, a substance, or a noise , you produce something that looks or sounds like it.
 \textit{
	\begin{itemize}
	\item The wood had been painted to simulate stone.
	\item Smoke was used to simulate steam coming from a smashed radiator.
	\item ...recordings to simulate the noise of the 250mph trains along the route.
	\end{itemize}
}
\item verb \\
If you \textbf{simulate} a set of conditions, you create them artificially, for example in order to conduct an experiment.
 \textit{
	\begin{itemize}
	\item The scientist developed one model to simulate a full year of the globe's climate.
	\item Cars are tested to see how much damage they suffer in simulated crashes.
	\end{itemize}
}
\end{enumerate}

\section*{thrust}
{\large \color{blue}  thrusts  thrusting  thrust  }
\subsection*{Explain}
\begin{enumerate}
\item verb \\
If you \textbf{thrust} something or someone somewhere , you push or move them there quickly with a lot of force.
 \textbf{Thrust} is also a noun .
 \textit{
	\begin{itemize}
	\item They thrust him into the back of a jeep.
	\item She grabs a stack of baby photos and thrusts them into my hands.
	\item Two of the knife thrusts were fatal.
	\end{itemize}
}
\item verb \\
If you \textbf{thrust} your \textbf{way} somewhere, you move there, pushing between people or things which are in your way.
 \textit{
	\begin{itemize}
	\item She thrust her way into the crowd.
	\item He reached the garden gate and thrust his way through it.
	\end{itemize}
}
\item verb \\
If something \textbf{thrusts} up or out of something else, it sticks up or sticks out in a noticeable way.
 \textit{
	\begin{itemize}
	\item An aerial thrust up from the grass verge.
	\item A ray of sunlight thrust out through the clouds.
	\end{itemize}
}
\item uncountable noun \\
\textbf{Thrust} is the power or force that is required to make a vehicle move in a particular direction.
 \textit{
	\begin{itemize}
	\item It provides the thrust that makes the craft move forward.
	\end{itemize}
}
\item singular noun \\
The \textbf{thrust} of an activity or of an idea is the main or essential things it expresses .
 \textit{
	\begin{itemize}
	\item The real thrust of the film is its examination of New York's Hasidic community.
	\item The main thrust of the research will be the study of the early Universe and galaxy
formation.
	\item The conductor brought home the full thrust of the work's emotional resolution.
	\end{itemize}
}
\end{enumerate}

\section*{solve}
{\large \color{blue}  solves  solving  solved  }
\subsection*{Explain}
\begin{enumerate}
\item verb \\
If you \textbf{solve} a problem or a question , you find a solution or an answer to it.
 \textit{
	\begin{itemize}
	\item Their domestic reforms did nothing to solve the problem of unemployment.
	\item That did not solve the question of who was to succeed him.
	\end{itemize}
}
\end{enumerate}

\section*{transcend}
{\large \color{blue}  transcends  transcending  transcended  }
\subsection*{Explain}
\begin{enumerate}
\item verb \\
Something that \textbf{transcends}  normal limits or boundaries  goes beyond them, because it is more significant than them.
 \textit{
	\begin{itemize}
	\item ...issues like humanitarian aid that transcend party loyalty.
	\end{itemize}
}
\end{enumerate}

\section*{sow}
{\large \color{blue}  sows  sowing  sowed  sown  }
\subsection*{Explain}
\begin{enumerate}
\item verb \\
If you \textbf{sow} seeds or \textbf{sow} an area of land  \textbf{with} seeds, you plant the seeds in the ground.
 \textit{
	\begin{itemize}
	\item Sow the seed in a warm place in February/March.
	\item Yesterday the field opposite was sown with maize.
	\end{itemize}
}
\item verb \\
If someone \textbf{sows} an undesirable  feeling or situation , they cause it to begin and develop .
 \textit{
	\begin{itemize}
	\item He cleverly sowed doubts into the minds of his rivals.
	\item Instead, the session has sowed confusion.
	\end{itemize}
}
\item  \\
 sow the seeds of sth/sow the seeds for sth \textit{
	\begin{itemize}
	\end{itemize}
}
\end{enumerate}

\section*{undergo}
{\large \color{blue}  undergoes  undergoing  underwent  undergone  }
\subsection*{Explain}
\begin{enumerate}
\item verb \\
If you \textbf{undergo} something necessary or unpleasant , it happens to you.
 \textit{
	\begin{itemize}
	\item New recruits have been undergoing training in recent weeks.
	\item He underwent an agonising 48-hour wait for the results of tests.
	\end{itemize}
}
\end{enumerate}

\section*{spill}
{\large \color{blue}  spills  spilling  spilled  spilt  }
\subsection*{Explain}
\begin{enumerate}
\item verb \\
If a liquid \textbf{spills} or if you \textbf{spill} it, it accidentally flows over the edge of a container.
 \textit{
	\begin{itemize}
	\item 70,000 tonnes of oil spilled from the tanker.
	\item ...water behind a dam, getting ready to spill over.
	\item He always spilled the drinks.
	\item Don't spill water on your suit.
	\end{itemize}
}
\item countable noun \\
A \textbf{spill} is an amount of liquid that has spilled from a container.
 \textit{
	\begin{itemize}
	\item She wiped a spill of milkshake off the counter.
	\item An oil spill could be devastating for wildlife.
	\end{itemize}
}
\item verb \\
If the contents of a bag , box , or other container \textbf{spill} or \textbf{are spilled} , they come out of the container onto a surface.
 \textit{
	\begin{itemize}
	\item A number of bags had split and were spilling their contents.
	\item He carefully balanced the satchel so that its contents would not spill out onto the
floor.
	\end{itemize}
}
\item verb \\
If people or things \textbf{spill} out of a place, they come out of it in large numbers .
 \textit{
	\begin{itemize}
	\item Tears began to spill out of the boy's eyes.
	\item When the bell rings, more than 1,000 children spill from classrooms.
	\end{itemize}
}
\item ergative verb \\
If light \textbf{spills} or \textbf{is spilled} into a place, it shines brightly into it, usually through a gap .
 \textit{
	\begin{itemize}
	\item She noticed the light spilling under Brian's door.
	\item The door swung open again, spilling light into the cell.
	\end{itemize}
}
\item  \\
 spill blood/spill sb's blood \textit{
	\begin{itemize}
	\end{itemize}
}
\end{enumerate}

\section*{urge}
{\large \color{blue}  urges  urging  urged  }
\subsection*{Explain}
\begin{enumerate}
\item verb \\
If you \textbf{urge} someone \textbf{to} do something, you try  hard to persuade them to do it.
 \textit{
	\begin{itemize}
	\item They urged parliament to approve plans for their reform programme.
	\item Firemen urged them to go to the shelter.
	\end{itemize}
}
\item verb \\
If you \textbf{urge} someone somewhere , you make them go there by touching them or talking to them.
 \textit{
	\begin{itemize}
	\item He slipped his arm around her waist and urged her away from the window.
	\item 'Come on, Grace,' he was urging her, 'don't wait, hurry up.'
	\end{itemize}
}
\item verb \\
If you \textbf{urge} a course of action, you strongly advise that it should be taken.
 \textit{
	\begin{itemize}
	\item He urged restraint on the security forces.
	\item We urge vigorous action to be taken immediately.
	\end{itemize}
}
\item countable noun \\
If you have an \textbf{urge}  \textbf{to} do or have something, you have a strong wish to do or have it.
 \textit{
	\begin{itemize}
	\item He had an urge to open a shop of his own.
	\item Resist the urge to nap during the day.
	\end{itemize}
}
\end{enumerate}

\section*{supervise}
{\large \color{blue}  supervises  supervising  supervised  }
\subsection*{Explain}
\begin{enumerate}
\item verb \\
If you \textbf{supervise} an activity or a person, you make sure that the activity is done correctly or that the person is doing a task or behaving correctly.
 \textit{
	\begin{itemize}
	\item University teachers have refused to supervise students' examinations.
	\item He supervised and trained more than 400 volunteers.
	\end{itemize}
}
\item verb \\
If you \textbf{supervise} a place where work is done, you ensure that the work there is done properly.
 \textit{
	\begin{itemize}
	\item He will be supervising the site.
	\item One of his jobs was supervising the dining room.
	\end{itemize}
}
\end{enumerate}

\section*{want}
{\large \color{blue}  wants  wanting  wanted  }
\subsection*{Explain}
\begin{enumerate}
\item verb \\
If you \textbf{want} something, you feel a desire or a need for it.
 \textit{
	\begin{itemize}
	\item I want a drink.
	\item Ian knows exactly what he wants in life.
	\item People wanted to know who this talented designer was.
	\item They began to want their father to be the same as other daddies.
	\item They didn't want people staring at them as they sat on the lawn, so they put up high
walls.
	\item He wanted his power recognised.
	\item I want my car this colour.
	\item And remember, we want him alive.
	\end{itemize}
}
\item verb \\
You can say that you \textbf{want}  \textbf{to} say something to indicate that you are about to say it.
 \textit{
	\begin{itemize}
	\item I want to say how really delighted I am that you're having a baby.
	\item Look, I wanted to apologize for today. I think I was a little hard on you.
	\end{itemize}
}
\item verb \\
You use \textbf{want} in questions as a way of making an offer or inviting someone to do something.
 \textit{
	\begin{itemize}
	\item Do you want another cup of coffee?
	\item Do you want to leave your bike here?
	\end{itemize}
}
\item verb \\
If you say to someone that you \textbf{want} something, or ask them if they \textbf{want}  \textbf{to} do it, you are firmly telling them what you want or what you want them to do.
 \textit{
	\begin{itemize}
	\item I want an explanation from you, Jeremy.
	\item If you have a problem with that, I want you to tell me right now.
	\item Do you want to tell me what all this is about?
	\item I want my money back!
	\end{itemize}
}
\item verb \\
If you say that something \textbf{wants} doing, you think that it needs to be done .
 \textit{
	\begin{itemize}
	\item The windows wanted cleaning.
	\item Her hair wants cutting.
	\end{itemize}
}
\item verb \\
If you tell someone that they \textbf{want}  \textbf{to} do a particular thing, you are advising them to do it.
 \textit{
	\begin{itemize}
	\item You want to be very careful not to have a man like Crevecoeur for an enemy.
	\item You want to look where you're going, mate.
	\end{itemize}
}
\item verb \\
If someone \textbf{is wanted} by the police , the police are searching for them because they are thought to have committed a crime .
 \textit{
	\begin{itemize}
	\item They were wanted by the police.
	\item He has killed many in his time, and is wanted in at least three countries.
	\item He was wanted for the murder of a magistrate.
	\end{itemize}
}
\item verb \\
If you \textbf{want} someone, you have a great desire to have sex with them.
 \textit{
	\begin{itemize}
	\item Come on, darling. I want you.
	\end{itemize}
}
\item verb \\
If a child  \textbf{is wanted} , its mother or another person loves it and is willing to look after it.
 \textit{
	\begin{itemize}
	\item Children should be wanted and planned.
	\item I want this baby very much, because it certainly will be the last.
	\end{itemize}
}
\item verb \\
If someone \textbf{wants} you in a particular place or role , they desire you to be in that place or role.
 \textit{
	\begin{itemize}
	\item Albie wants you in his office.
	\item They didn't want her as attorney general.
	\item This is my territory. I want you out of here.
	\end{itemize}
}
\item singular noun \\
A \textbf{want of} something is a lack of it.
 \textit{
	\begin{itemize}
	\item ...a want of manners and charm.
	\item The men were daily becoming weaker from want of rest.
	\end{itemize}
}
\item uncountable noun \\
\textbf{Want} is the state of being extremely  poor .
 \textit{
	\begin{itemize}
	\item He said they were fighting for freedom of speech, freedom of worship, and freedom
from want.
	\end{itemize}
}
\item plural noun \\
Your \textbf{wants} are the things that you want.
 \textit{
	\begin{itemize}
	\item She couldn't lift a spoon without a servant anticipating her wants and getting it
for her.
	\item Supermarkets often claim that they are responding to the wants of consumers.
	\end{itemize}
}
\item  \\
 for want of \textit{
	\begin{itemize}
	\end{itemize}
}
\item  \\
 if you want \textit{
	\begin{itemize}
	\end{itemize}
}
\item  \\
 I don't want to/without wanting to \textit{
	\begin{itemize}
	\end{itemize}
}
\item  \\
 what do you want? \textit{
	\begin{itemize}
	\end{itemize}
}
\end{enumerate}

\section*{wander}
{\large \color{blue}  wanders  wandering  wandered  }
\subsection*{Explain}
\begin{enumerate}
\item verb \\
If you \textbf{wander} in a place, you walk around there in a casual way, often without intending to go in any particular direction.
 \textbf{Wander} is also a noun .
 \textit{
	\begin{itemize}
	\item When he got bored he wandered around the fair.
	\item They wandered off in the direction of the nearest store.
	\item Those who do not have relatives to return to are left to wander the streets and sleep
rough.
	\item A wander around any market will reveal stalls piled high with vegetables.
	\end{itemize}
}
\item verb \\
If a person or animal \textbf{wanders} from a place where they are supposed to stay , they move away from the place without going in a particular direction.
 \textit{
	\begin{itemize}
	\item Because Mother is afraid we'll get lost, we aren't allowed to wander far.
	\item To keep their bees from wandering, beekeepers feed them sugar solutions.
	\end{itemize}
}
\item verb \\
If your mind \textbf{wanders} or your thoughts \textbf{wander} , you stop  concentrating on something and start thinking about other things.
 \textit{
	\begin{itemize}
	\item His mind would wander, and he would lose track of what he was doing.
	\item Jarvis found his attention wandering.
	\item Grace allowed her mind to wander to other things.
	\end{itemize}
}
\item verb \\
If your eyes  \textbf{wander} , you stop looking at one thing and start looking around at other things.
 \textit{
	\begin{itemize}
	\item His eyes wandered restlessly around the room.
	\item His eyes kept wandering to the picture.
	\item Read their body language. Are their eyes wandering?
	\end{itemize}
}
\end{enumerate}

\section*{warn}
{\large \color{blue}  warns  warning  warned  }
\subsection*{Explain}
\begin{enumerate}
\item verb \\
If you \textbf{warn} someone about something such as a possible danger or problem , you tell them about it so that they are aware of it.
 \textit{
	\begin{itemize}
	\item When I had my first baby, friends warned me that children were expensive.
	\item They warned him of the dangers of sailing alone.
	\item Analysts warned that Europe's most powerful economy may be facing trouble.
	\item He also warned of a possible anti-Western backlash.
	\end{itemize}
}
\item verb \\
If you \textbf{warn} someone not \textbf{to} do something, you advise them not to do it so that they can avoid possible danger or punishment .
 \textit{
	\begin{itemize}
	\item Mrs. Blount warned me not to interfere.
	\item Children must be warned to stay away from main roads.
	\item 'Don't do anything yet,' he warned. 'Too risky.'
	\item 'Keep quiet, or they'll all come out,' they warned him.
	\item I wish I'd listened to the people who warned me against having the operation.
	\item Mr Lowe warned against complacency.
	\end{itemize}
}
\item  \\
 be warned \textit{
	\begin{itemize}
	\end{itemize}
}
\end{enumerate}

\section*{wrap}
{\large \color{blue}  wraps  wrapping  wrapped  }
\subsection*{Explain}
\begin{enumerate}
\item verb \\
When you \textbf{wrap} something, you fold paper or cloth tightly round it to cover it completely, for example in order to protect it or so that you can give it to someone as a present.
 \textbf{Wrap up} means the same as wrap .
 \textit{
	\begin{itemize}
	\item Harry had carefully bought and wrapped presents for Mark to give them.
	\item Mexican Indians used to wrap tough meat in leaves from the papaya tree.
	\item Diana is taking the opportunity to wrap up the family presents.
	\end{itemize}
}
\item uncountable noun \\
\textbf{Wrap} is the material that something is wrapped in.
 \textit{
	\begin{itemize}
	\item I tucked some plastic wrap around the sandwiches to keep them from getting stale.
	\item ...gift wrap.
	\end{itemize}
}
\item verb \\
When you \textbf{wrap} something such as a piece of paper or cloth round another thing, you put it around
it.
 \textit{
	\begin{itemize}
	\item She wrapped a handkerchief around her bleeding palm.
	\item Then she stood up, wrapping her coat around her angrily.
	\item Wrap the foil over the fish.
	\end{itemize}
}
\item verb \\
If someone \textbf{wraps} their arms, fingers , or legs around something, they put them firmly around it.
 \textit{
	\begin{itemize}
	\item He wrapped his arms around her.
	\end{itemize}
}
\item countable noun \\
A \textbf{wrap} is a piece of clothing which women wear round their shoulders, either to keep them
 warm when wearing an evening  dress , or for decoration over a coat .
 \textit{
	\begin{itemize}
	\end{itemize}
}
\item  \\
 under wraps \textit{
	\begin{itemize}
	\end{itemize}
}
\end{enumerate}

\section*{acquaint}
{\large \color{blue}  acquaints  acquainting  acquainted  }
\subsection*{Explain}
\begin{enumerate}
\item verb \\
If you \textbf{acquaint} someone \textbf{with} something, you tell them about it so that they know it. If you \textbf{acquaint}  \textbf{yourself with} something, you learn about it.
 \textit{
	\begin{itemize}
	\item ...efforts to acquaint the public with their rights under the new law.
	\item I want to acquaint myself with your strengths and weaknesses.
	\end{itemize}
}
\end{enumerate}

\section*{abide}
{\large \color{blue}  abides  abiding  abided  }
\subsection*{Explain}
\begin{enumerate}
\item  \\
 can't abide \textit{
	\begin{itemize}
	\end{itemize}
}
\end{enumerate}

\section*{advertise}
{\large \color{blue}  advertises  advertising  advertised  }
\subsection*{Explain}
\begin{enumerate}
\item verb \\
If you \textbf{advertise} something such as a product, an event, or a job , you tell people about it online , in newspapers, on television, or on posters in order to encourage them to buy the product, go to the event, or apply for the job.
 \textit{
	\begin{itemize}
	\item The players can advertise baked beans, but not rugby boots.
	\item The property was being advertised for sale in America.
	\item Religious groups are currently not allowed to advertise on television.
	\end{itemize}
}
\item verb \\
If you \textbf{advertise}  \textbf{for} someone to do something for you, for example to work for you or share your accommodation , you announce it online, in a newspaper, on television, or on a notice  board .
 \textit{
	\begin{itemize}
	\item We advertised for staff in a local newspaper.
	\item I shall advertise for someone to go with me.
	\end{itemize}
}
\item verb \\
If someone or something \textbf{advertises} a particular quality, they show it in their appearance or behaviour.
 \textit{
	\begin{itemize}
	\item His hard sinewy body advertised his ruthlessness of purpose.
	\end{itemize}
}
\item verb \\
If you do not \textbf{advertise} the fact that something is the case , you try not to let other people know about it.
 \textit{
	\begin{itemize}
	\item There is no need to advertise the fact that you are a single woman.
	\item I didn't want to advertise the fact that he hadn't driven me to the airport.
	\end{itemize}
}
\end{enumerate}

\section*{adhere}
{\large \color{blue}  adheres  adhering  adhered  }
\subsection*{Explain}
\begin{enumerate}
\item verb \\
If you \textbf{adhere}  \textbf{to} a rule or agreement , you act in the way that it says you should.
 \textit{
	\begin{itemize}
	\item All members of the association adhere to a strict code of practice.
	\item It is only when safety procedures are not strictly adhered to that catastrophes occur.
	\end{itemize}
}
\item verb \\
If you \textbf{adhere}  \textbf{to} an opinion or belief , you support or hold it.
 \textit{
	\begin{itemize}
	\item If you can't adhere to my values, then you have to find another place to live.
	\end{itemize}
}
\item verb \\
If something \textbf{adheres}  \textbf{to} something else, it sticks firmly to it.
 \textit{
	\begin{itemize}
	\item Small particles adhere to the seed.
	\item This sticky compound adheres well on this surface.
	\end{itemize}
}
\end{enumerate}

\section*{allow}
{\large \color{blue}  allows  allowing  allowed  }
\subsection*{Explain}
\begin{enumerate}
\item verb \\
If someone \textbf{is allowed}  \textbf{to} do something, it is all right for them to do it and they will not get into trouble .
 \textit{
	\begin{itemize}
	\item The children are not allowed to watch violent TV programmes.
	\item The Government will allow them to advertise on radio and television.
	\item They will be allowed home.
	\item Smoking will not be allowed.
	\end{itemize}
}
\item verb \\
If you \textbf{are allowed} something, you are given  permission to have it or are given it.
 \textit{
	\begin{itemize}
	\item Gifts like chocolates or flowers are allowed.
	\item He should be allowed the occasional treat.
	\end{itemize}
}
\item verb \\
If you \textbf{allow} something \textbf{to}  happen , you do not prevent it.
 \textit{
	\begin{itemize}
	\item He won't allow himself to fail.
	\item If the soil is allowed to dry out the tree could die.
	\end{itemize}
}
\item verb \\
If one thing \textbf{allows} another thing \textbf{to} happen, the first thing creates the opportunity for the second thing to happen.
 \textit{
	\begin{itemize}
	\item The compromise will allow him to continue his free market reforms.
	\item ...an attempt to allow the Tory majority a greater share of power.
	\item She said this would allow more effective planning.
	\end{itemize}
}
\item verb \\
If you \textbf{allow} a particular length of time or a particular amount of something \textbf{for} a particular purpose , you include it in your planning .
 \textit{
	\begin{itemize}
	\item Please allow 28 days for delivery.
	\item Allow about 75ml (3fl oz) per six servings.
	\end{itemize}
}
\item verb \\
If you \textbf{allow}  \textbf{that} something is true , you admit or agree that it is true.
 \textit{
	\begin{itemize}
	\item Warren allows that capitalist development may result in increased social inequality.
	\end{itemize}
}
\item  \\
 allow me \textit{
	\begin{itemize}
	\end{itemize}
}
\item  \\
 allow me to \textit{
	\begin{itemize}
	\end{itemize}
}
\end{enumerate}

\section*{apply}
{\large \color{blue}  applies  applying  applied  }
\subsection*{Explain}
\begin{enumerate}
\item verb \\
If you \textbf{apply}  \textbf{for} something such as a job or membership of an organization , you write a letter or fill in a form in order to ask formally for it.
 \textit{
	\begin{itemize}
	\item I am continuing to apply for jobs.
	\item They may apply to join the organization.
	\end{itemize}
}
\item verb \\
If you \textbf{apply}  \textbf{yourself to} something or \textbf{apply} your mind  \textbf{to} something, you concentrate  hard on doing it or on thinking about it.
 \textit{
	\begin{itemize}
	\item Faulks has applied himself to this task with considerable energy.
	\item In spare moments he applied his mind to how rockets could be used to make money.
	\end{itemize}
}
\item verb \\
If something such as a rule or a remark  \textbf{applies}  \textbf{to} a person or in a situation , it is relevant to the person or the situation.
 \textit{
	\begin{itemize}
	\item The convention does not apply to us.
	\item The rule applies where a person owns stock in a corporation.
	\end{itemize}
}
\item verb \\
If you \textbf{apply} something such as a rule, system, or skill , you use it in a situation or activity.
 \textit{
	\begin{itemize}
	\item The Government appears to be applying the same principle.
	\item His project is concerned with applying the technology to practical business problems.
	\end{itemize}
}
\item verb \\
A name that \textbf{is applied}  \textbf{to} someone or something is used to refer to them.
 \textit{
	\begin{itemize}
	\item Increasingly the term is applied to people succumbing to stress in a variety of
people-orientated service industries.
	\end{itemize}
}
\item verb \\
If you \textbf{apply} something \textbf{to} a surface, you put it on or rub it into the surface.
 \textit{
	\begin{itemize}
	\item The right thing would be to apply direct pressure to the wound.
	\item Applying the dye can be messy, particularly on long hair.
	\end{itemize}
}
\item verb \\
When the driver of a vehicle  \textbf{applies} the brakes , he or she uses them to slow the vehicle down or to stop it from moving.
 \textit{
	\begin{itemize}
	\item They forgot to apply the handbrake and the car rolled 60ft into the river.
	\end{itemize}
}
\end{enumerate}

\section*{analyse}
{\large \color{blue}  analyses  analysing  analysed  }
\subsection*{Explain}
\begin{enumerate}
\item verb \\
If you \textbf{analyse} something, you consider it carefully or use statistical methods in order to fully  understand it.
 \textit{
	\begin{itemize}
	\item McCarthy was asked to analyse the data from the first phase of trials of the vaccine.
	\item This book teaches you how to analyse what is causing the stress in your life.
	\end{itemize}
}
\item verb \\
If you \textbf{analyse} something, you examine it using scientific methods in order to find out what it consists of.
 \textit{
	\begin{itemize}
	\item We haven't had time to analyse those samples yet.
	\item They had their tablets analysed to find out whether they were getting the real drug
or not.
	\end{itemize}
}
\end{enumerate}

\section*{approach}
{\large \color{blue}  approaches  approaching  approached  }
\subsection*{Explain}
\begin{enumerate}
\item verb \\
When you \textbf{approach} something, you get closer to it.
 \textbf{Approach} is also a noun .
 \textit{
	\begin{itemize}
	\item He didn't approach the front door at once.
	\item When I approached, they grew silent.
	\item We turned to see the approaching car slow down.
	\item At their approach the little boy ran away and hid.
	\item ...the approach of a low-flying helicopter.
	\end{itemize}
}
\item countable noun \\
An \textbf{approach}  \textbf{to} a place is a road , path , or other route that leads to it.
 \textit{
	\begin{itemize}
	\item The path serves as an approach to the boat house.
	\end{itemize}
}
\item verb \\
If you \textbf{approach} someone \textbf{about} something, you speak to them about it for the first time, often making an offer or request .
 \textbf{Approach} is also a noun.
 \textit{
	\begin{itemize}
	\item When Chappel approached me about the job, my first reaction was disbelief.
	\item He approached me to create and design the restaurant.
	\item Anna approached several builders and was fortunate to come across Eddie.
	\item There had already been approaches from buyers interested in the whole of the group.
	\end{itemize}
}
\item verb \\
When you \textbf{approach} a task , problem, or situation in a particular way, you deal with it or think about it in that way.
 \textit{
	\begin{itemize}
	\item The Bank has approached the issue in a practical way.
	\item Employers are interested in how you approach problems.
	\end{itemize}
}
\item countable noun \\
Your \textbf{approach}  \textbf{to} a task, problem, or situation is the way you deal with it or think about it.
 \textit{
	\begin{itemize}
	\item We will be exploring different approaches to gathering information.
	\item ...the adversarial approach of the British legal system.
	\end{itemize}
}
\item verb \\
As a future time or event  \textbf{approaches} , it gradually gets nearer as time passes .
 \textbf{Approach} is also a noun.
 \textit{
	\begin{itemize}
	\item As autumn approached, the plants and colours in the garden changed.
	\item ...the approaching crisis.
	\item ...the festive spirit that permeated the house with the approach of Christmas.
	\end{itemize}
}
\item verb \\
As you \textbf{approach} a future time or event, time passes so that you get gradually nearer to it.
 \textit{
	\begin{itemize}
	\item We approach the end of the year with the economy slowing and little sign of cheer.
	\end{itemize}
}
\item verb \\
If something \textbf{approaches} a particular level or state, it almost  reaches that level or state.
 \textit{
	\begin{itemize}
	\item Oil prices have approached their highest level for almost ten years.
	\end{itemize}
}
\end{enumerate}

\section*{appreciate}
{\large \color{blue}  appreciates  appreciating  appreciated  }
\subsection*{Explain}
\begin{enumerate}
\item verb \\
If you \textbf{appreciate} something, for example a piece of music or good  food , you like it because you recognize its good qualities.
 \textit{
	\begin{itemize}
	\item Anyone can appreciate our music.
	\item In time you'll appreciate the beauty and subtlety of this language.
	\end{itemize}
}
\item verb \\
If you \textbf{appreciate} a situation or problem , you understand it and know what it involves .
 \textit{
	\begin{itemize}
	\item I never really appreciated the size of the club and what it means to the community.
	\item He appreciates that co-operation with the media is part of his professional duties.
	\end{itemize}
}
\item verb \\
If you \textbf{appreciate} something that someone has done for you or is going to do for you, you are grateful for it.
 \textit{
	\begin{itemize}
	\item Peter stood by me when I most needed it. I'll always appreciate that.
	\item Thanks, lads. I appreciate it.
	\item I'd appreciate it if you wouldn't mention it.
	\end{itemize}
}
\item verb \\
If something that you own \textbf{appreciates} over a period of time, its value increases.
 \textit{
	\begin{itemize}
	\item They don't have any confidence that houses will appreciate in value.
	\end{itemize}
}
\end{enumerate}

\section*{aspire}
{\large \color{blue}  aspires  aspiring  aspired  }
\subsection*{Explain}
\begin{enumerate}
\item verb \\
If you \textbf{aspire to} something such as an important  job , you have a strong desire to achieve it.
 \textit{
	\begin{itemize}
	\item ...people who aspire to public office.
	\item They aspired to be gentlemen, though they fell far short of the ideal.
	\end{itemize}
}
\end{enumerate}

\section*{argue}
{\large \color{blue}  argues  arguing  argued  }
\subsection*{Explain}
\begin{enumerate}
\item verb \\
If one person \textbf{argues}  \textbf{with} another, they speak angrily to each other about something that they disagree about. You can  also  say that two people \textbf{argue} .
 \textit{
	\begin{itemize}
	\item The committee is concerned about players' behaviour, especially arguing with referees.
	\item They were still arguing; I could hear them down the road.
	\end{itemize}
}
\item verb \\
If you tell someone not to \textbf{argue}  \textbf{with} you, you want them to do or believe what you say without protest or disagreement .
 \textit{
	\begin{itemize}
	\item Don't argue with me.
	\item The children go to bed at 10.30. No one dares argue.
	\end{itemize}
}
\item verb \\
If you \textbf{argue}  \textbf{with} someone \textbf{about} something, you discuss it with them, with each of you giving your different  opinions .
 \textit{
	\begin{itemize}
	\item He was arguing with the King about the need to maintain the cavalry at full strength.
	\item They are arguing over foreign policy.
	\item The two of them sitting in their office were arguing this point.
	\end{itemize}
}
\item verb \\
If you \textbf{argue}  \textbf{that} something is true , you state it and give the reasons why you think it is true.
 \textit{
	\begin{itemize}
	\item His lawyers are arguing that he is unfit to stand trial.
	\item It could be argued that the exam questions were too difficult.
	\end{itemize}
}
\item verb \\
If you \textbf{argue for} something, you say why you agree with it, in order to persuade people that it is right . If you \textbf{argue against} something, you say why you disagree with it, in order to persuade people that it
is wrong .
 \textit{
	\begin{itemize}
	\item The report argues against tax increases.
	\item I argued the case for an independent central bank.
	\end{itemize}
}
\item verb \\
If you \textbf{argue} , you support your opinions with evidence in an ordered or logical  way .
 \textit{
	\begin{itemize}
	\item I've argued deductively from the text.
	\item He argued persuasively, and was full of confidence.
	\end{itemize}
}
\item verb \\
If you say that no-one can \textbf{argue}  \textbf{with} a particular  fact or opinion, you are emphasizing that it is obviously true and so everyone must  accept it.
 \textit{
	\begin{itemize}
	\item We produced the best soccer of the tournament. Nobody would argue with that.
	\end{itemize}
}
\end{enumerate}

\section*{bark}
{\large \color{blue}  barks  barking  barked  }
\subsection*{Explain}
\begin{enumerate}
\item verb \\
When a dog \textbf{barks} , it makes a short , loud noise , once or several times.
 \textbf{Bark} is also a noun .
 \textit{
	\begin{itemize}
	\item Don't let the dogs bark.
	\item A small dog barked at a seagull he was chasing.
	\item The Doberman let out a string of roaring barks.
	\end{itemize}
}
\item verb \\
If you \textbf{bark}  \textbf{at} someone, you shout at them aggressively in a loud, rough  voice .
 \textit{
	\begin{itemize}
	\item I didn't mean to bark at you.
	\item A policeman held his gun in both hands and barked an order.
	\end{itemize}
}
\item uncountable noun \\
\textbf{Bark} is the tough material that covers the outside of a tree.
 \textit{
	\begin{itemize}
	\end{itemize}
}
\item  \\
 someone's bark is worse than their bite \textit{
	\begin{itemize}
	\end{itemize}
}
\end{enumerate}

\section*{assess}
{\large \color{blue}  assesses  assessing  assessed  }
\subsection*{Explain}
\begin{enumerate}
\item verb \\
When you \textbf{assess} a person, thing, or situation , you consider them in order to make a judgment about them.
 \textit{
	\begin{itemize}
	\item Our correspondent has been assessing the impact of the sanctions.
	\item The test was to assess aptitude rather than academic achievement.
	\item It would be a matter of assessing whether she was well enough to travel.
	\end{itemize}
}
\item verb \\
When you \textbf{assess} the amount of money that something is worth or should be paid, you calculate or estimate it.
 \textit{
	\begin{itemize}
	\item Ask them to send you information on how to assess the value of your belongings.
	\item What's the property's assessed value?
	\end{itemize}
}
\end{enumerate}

\section*{burn}
{\large \color{blue}  burns  burning  burned  burnt  }
\subsection*{Explain}
\begin{enumerate}
\item verb \\
If there is a fire or a flame somewhere , you say that there is a fire or flame \textbf{burning} there.
 \textit{
	\begin{itemize}
	\item Fires were burning out of control in the center of the city.
	\item There was a fire burning in the large fireplace.
	\item The furnace has a design that allows the flame to burn at a lower temperature.
	\end{itemize}
}
\item verb \\
If something \textbf{is burning} , it is on fire.
 \textit{
	\begin{itemize}
	\item When I arrived one of the vehicles was still burning.
	\item That boy was rescued from a burning house.
	\end{itemize}
}
\item verb \\
If you \textbf{burn} something, you destroy or damage it with fire.
 \textit{
	\begin{itemize}
	\item Protesters set cars on fire and burned a building.
	\item Incineration plants should be built to burn household waste.
	\item Coal fell out of the fire, and burned the carpet.
	\end{itemize}
}
\item verb \\
If you \textbf{burn} a fuel or if it \textbf{burns} , it is used to produce heat, light, or energy.
 \textit{
	\begin{itemize}
	\item The power stations burn coal from the Ruhr region.
	\item Manufacturers are working with new fuels to find one that burns more cleanly than
petrol.
	\end{itemize}
}
\item verb \\
If you \textbf{burn} something that you are cooking or if it \textbf{burns} , you spoil it by using too much heat or cooking it for too long.
 \textit{
	\begin{itemize}
	\item I burnt the toast.
	\item Watch them carefully as they finish cooking because they can burn easily.
	\end{itemize}
}
\item verb \\
If you \textbf{burn} part of your body, \textbf{burn}  \textbf{yourself} , or \textbf{are burnt} , you are injured by fire or by something very hot.
 \textbf{Burn} is also a noun .
 \textit{
	\begin{itemize}
	\item Take care not to burn your fingers.
	\item If you are badly burnt, seek medical attention.
	\item She suffered appalling burns to her back.
	\end{itemize}
}
\item verb \\
If someone \textbf{is burnt} or \textbf{burnt} to death, they are killed by fire.
 \textit{
	\begin{itemize}
	\item Women were burned as witches in the middle ages.
	\item At least 80 people were burnt to death when their bus caught fire.
	\end{itemize}
}
\item verb \\
If a light \textbf{is burning} , it is shining .
 \textit{
	\begin{itemize}
	\item The building was darkened except for a single light burning in a third-story window.
	\end{itemize}
}
\item verb \\
If your face \textbf{is burning} , it is red because you are embarrassed or upset .
 \textit{
	\begin{itemize}
	\item Liz's face was burning.
	\end{itemize}
}
\item verb \\
If you \textbf{are burning}  \textbf{with} an emotion or \textbf{are burning}  \textbf{to} do something, you feel that emotion or the desire to do that thing very strongly.
 \textit{
	\begin{itemize}
	\item The young boy was burning with a fierce ambition.
	\item Dan burned to know what the reason could be.
	\end{itemize}
}
\item verb \\
If you \textbf{burn} or get  \textbf{burned} in the sun, the sun makes your skin become red and sore .
 \textit{
	\begin{itemize}
	\item Build up your tan slowly and don't allow your skin to burn.
	\item Summer sun can burn fair skin in minutes.
	\end{itemize}
}
\item verb \\
If a part of your body \textbf{burns} or if something \textbf{burns} it, it has a painful, hot, or stinging feeling.
 \textit{
	\begin{itemize}
	\item My eyes burn from staring at the needle.
	\item His face was burning with cold.
	\item ...delicious Indian recipes which won't burn your throat.
	\end{itemize}
}
\item verb \\
If you are \textbf{burned} or get \textbf{burned} , you lose something as a result of taking a risk , usually in a business deal .
 \textit{
	\begin{itemize}
	\item They always took chances and got burned very badly in past years.
	\end{itemize}
}
\item verb \\
To \textbf{burn} a CD-ROM means to write or copy data onto it.
 \textit{
	\begin{itemize}
	\item You can use this software to burn custom compilations of your favorite tunes.
	\end{itemize}
}
\end{enumerate}

\section*{attract}
{\large \color{blue}  attracts  attracting  attracted  }
\subsection*{Explain}
\begin{enumerate}
\item verb \\
If something \textbf{attracts} people or animals, it has features that cause them to come to it.
 \textit{
	\begin{itemize}
	\item The Cardiff Bay project is attracting many visitors.
	\item Warm weather has attracted the flat fish close to shore.
	\item Summer attracts visitors to the countryside.
	\end{itemize}
}
\item verb \\
If someone or something \textbf{attracts} you, they have particular qualities which cause you to like or admire them. If a particular quality \textbf{attracts} you \textbf{to} a person or thing, it is the reason why you like them.
 \textit{
	\begin{itemize}
	\item He wasn't sure he'd got it right, although the theory attracted him by its logic.
	\item What first attracted me to her was her incredible experience of life.
	\item More people would be attracted to cycling if conditions were right.
	\end{itemize}
}
\item verb \\
If you \textbf{are attracted}  \textbf{to} someone, you are interested in them sexually.
 \textit{
	\begin{itemize}
	\item In spite of her hostility, she was attracted to him.
	\item I was married to a man who had ceased to attract me.
	\end{itemize}
}
\item verb \\
If something \textbf{attracts}  support , publicity , or money , it receives support, publicity, or money.
 \textit{
	\begin{itemize}
	\item President Mwinyi said his country would also like to attract investment from private
companies.
	\item Opinion polls suggest that the two rebels have attracted a lot of sympathy.
	\end{itemize}
}
\item verb \\
If one object \textbf{attracts} another object, it causes the second object to move towards it.
 \textit{
	\begin{itemize}
	\item Anything with strong gravity attracts other things to it.
	\end{itemize}
}
\end{enumerate}

\section*{burst}
{\large \color{blue}  bursts  bursting  }
\subsection*{Explain}
\begin{enumerate}
\item verb \\
If something \textbf{bursts} or if you \textbf{burst} it, it suddenly breaks open or splits open and the air or other substance inside it comes out.
 \textit{
	\begin{itemize}
	\item The driver lost control when a tyre burst.
	\item It is not a good idea to burst a blister.
	\item ...a flood caused by a burst pipe.
	\end{itemize}
}
\item verb \\
If a dam  \textbf{bursts} , or if something \textbf{bursts} it, it breaks apart because the force of the river is too great.
 \textit{
	\begin{itemize}
	\item A dam burst and flooded their villages.
	\end{itemize}
}
\item verb \\
If a river \textbf{bursts} its banks , the water rises and goes on to the land.
 \textit{
	\begin{itemize}
	\item Monsoons caused the river to burst its banks.
	\end{itemize}
}
\item verb \\
When a door or lid  \textbf{bursts} open, it opens very suddenly and violently because someone pushes it or there is great pressure behind it.
 \textit{
	\begin{itemize}
	\item The door burst open and an angry young nurse appeared.
	\end{itemize}
}
\item verb \\
To \textbf{burst}  \textbf{into} or \textbf{out} of a place means to enter or leave it suddenly with a lot of energy or force.
 \textit{
	\begin{itemize}
	\item Gunmen burst into his home and opened fire.
	\item Rachel burst out as the door was flung open again.
	\end{itemize}
}
\item verb \\
If you say that something \textbf{bursts} onto the scene , you mean that it suddenly starts or becomes active , usually after developing quietly for some time.
 \textit{
	\begin{itemize}
	\item He burst onto the fashion scene in the early 1980s.
	\end{itemize}
}
\item verb \\
If you say that someone is about to \textbf{burst}  \textbf{with}  pride , anger , or another emotion , you are emphasizing the intensity of the emotion they are feeling .
 \textit{
	\begin{itemize}
	\item He almost burst with pride when his son John began to excel at football.
	\item He thought his heart would burst with grief.
	\end{itemize}
}
\item verb \\
When a firework or bomb  \textbf{bursts} in the air, it explodes.
 \textit{
	\begin{itemize}
	\item Hundreds of fireworks burst simultaneously in midair.
	\item Every now and then you hear some bombs bursting.
	\end{itemize}
}
\item countable noun \\
A \textbf{burst}  \textbf{of} something is a sudden short period of it.
 \textit{
	\begin{itemize}
	\item ...a burst of machine-gun fire.
	\item It is easier to cope with short bursts of activity than with prolonged exercise.
	\item The current flows in little bursts.
	\end{itemize}
}
\end{enumerate}

\section*{augment}
{\large \color{blue}  augments  augmenting  augmented  }
\subsection*{Explain}
\begin{enumerate}
\item verb \\
To \textbf{augment} something means to make it larger, stronger , or more effective by adding something to it.
 \textit{
	\begin{itemize}
	\item While searching for a way to augment the family income, she began making dolls.
	\end{itemize}
}
\end{enumerate}

\section*{cater}
{\large \color{blue}  caters  catering  catered  }
\subsection*{Explain}
\begin{enumerate}
\item verb \\
In British  English , to \textbf{cater for} a group of people means to provide all the things that they need or want . In American English, you say you \textbf{cater to} a person or group of people.
 \textit{
	\begin{itemize}
	\item Minorca is the sort of place that caters for families.
	\item We cater to an exclusive clientele.
	\end{itemize}
}
\item verb \\
In British English, to \textbf{cater for} something means to take it into account . In American English, you say you \textbf{cater to} something.
 \textit{
	\begin{itemize}
	\item We have to cater for demand.
	\item ...shops that cater for the needs of men.
	\item Exercise classes cater to all levels of fitness.
	\end{itemize}
}
\item verb \\
If a person or company  \textbf{caters for} an occasion such as a wedding or a party , they provide food and drink for all the people there.
 \textit{
	\begin{itemize}
	\item Nunsmere Hall can cater for receptions of up to 300 people.
	\item The chef is pleased to cater for vegetarian diets.
	\item Does he cater parties too?
	\end{itemize}
}
\end{enumerate}

\section*{bother}
{\large \color{blue}  bothers  bothering  bothered  }
\subsection*{Explain}
\begin{enumerate}
\item verb \\
If you do not \textbf{bother}  \textbf{to} do something or if you do not \textbf{bother}  \textbf{with} it, you do not do it, consider it, or use it because you think it is unnecessary or because you are too lazy .
 \textit{
	\begin{itemize}
	\item Lots of people don't bother to go through a marriage ceremony these days.
	\item Most of the papers didn't even bother reporting it.
	\item Nothing I do makes any difference anyway, so why bother?
	\item ...and he does not bother with a helmet either.
	\end{itemize}
}
\item uncountable noun \\
\textbf{Bother}  means trouble or difficulty . You can also use \textbf{bother} to refer to an activity which causes this, especially when you would prefer not to do it or get involved with it.
 \textit{
	\begin{itemize}
	\item I usually buy sliced bread–it's less bother.
	\item The courts take too long and going to the police is a bother.
	\item Most men hate the bother of shaving.
	\end{itemize}
}
\item uncountable noun \\
You use \textbf{bother} to refer to serious trouble, usually when you want to make it sound less serious than it really is.
 \textit{
	\begin{itemize}
	\item Vince is having a spot of bother with the law.
	\end{itemize}
}
\item verb \\
If something \textbf{bothers} you, or if you \textbf{bother} about it, it worries, annoys , or upsets you.
 \textit{
	\begin{itemize}
	\item Is something bothering you?
	\item That kind of jealousy doesn't bother me.
	\item It bothered me that boys weren't interested in me.
	\item Never bother about people's opinions.
	\end{itemize}
}
\item verb \\
If someone \textbf{bothers} you, they talk to you when you want to be left alone or interrupt you when you are busy .
 \textit{
	\begin{itemize}
	\item We are playing a trick on a man who keeps bothering me.
	\item I don't know why he bothers me with this kind of rubbish.
	\end{itemize}
}
\item  \\
 bother/bother it \textit{
	\begin{itemize}
	\end{itemize}
}
\item  \\
 can't be bothered \textit{
	\begin{itemize}
	\end{itemize}
}
\item  \\
 it's no bother \textit{
	\begin{itemize}
	\end{itemize}
}
\end{enumerate}

\section*{clarify}
{\large \color{blue}  clarifies  clarifying  clarified  }
\subsection*{Explain}
\begin{enumerate}
\item verb \\
To \textbf{clarify} something means to make it easier to understand, usually by explaining it in more detail .
 \textit{
	\begin{itemize}
	\item Thank you for writing and allowing me to clarify the present position.
	\item A bank spokesperson was unable to clarify the situation.
	\end{itemize}
}
\end{enumerate}

\section*{break}
{\large \color{blue}  breaks  breaking  broke  broken  }
\subsection*{Explain}
\begin{enumerate}
\item verb \\
When an object \textbf{breaks} or when you \textbf{break} it, it suddenly separates into two or more pieces, often because it has been hit
or dropped.
 \textit{
	\begin{itemize}
	\item He fell through the window, breaking the glass.
	\item The plate broke.
	\item Break the cauliflower into florets.
	\item The plane broke into three pieces.
	\item ...bombed-out buildings, surrounded by broken glass and rubble.
	\item The only sound was the crackle of breaking ice.
	\end{itemize}
}
\item verb \\
If you \textbf{break} a part of your body such as your leg, your arm, or your nose, or if a bone \textbf{breaks} , you are injured because a bone cracks or splits.
 \textbf{Break} is also a noun.
 \textit{
	\begin{itemize}
	\item She broke a leg in a skiing accident.
	\item Old bones break easily.
	\item Several people were treated for broken bones.
	\item It has caused a bad break to Gabriella's leg.
	\end{itemize}
}
\item verb \\
If a surface, cover, or seal \textbf{breaks} or if something \textbf{breaks} it, a hole or tear is made in it, so that a substance can pass through.
 \textit{
	\begin{itemize}
	\item Once you've broken the seal of a bottle there's no way you can put it back together
again.
	\item The bandage must be put on when the blister breaks.
	\item Do not use the cream on broken skin.
	\end{itemize}
}
\item verb \\
When a tool or piece of machinery  \textbf{breaks} or when you \textbf{break} it, it is damaged and no longer works.
 \textit{
	\begin{itemize}
	\item When the clutch broke, the car was locked into second gear.
	\item Tenants do not have to worry about leaking roofs and broken washing machines.
	\item The lead biker broke his bike chain.
	\end{itemize}
}
\item verb \\
If you \textbf{break} a rule, promise, or agreement, you do something that you should not do according
to that rule, promise, or agreement.
 \textit{
	\begin{itemize}
	\item We didn't know we were breaking the law.
	\item The company has consistently denied it had knowingly broken arms embargoes.
	\item ...broken promises.
	\end{itemize}
}
\item verb \\
If you \textbf{break} free or loose, you free yourself from something or escape from it.
 \textit{
	\begin{itemize}
	\item She broke free by thrusting her elbow into his chest.
	\item A young child broke loose from the crowd and ran toward her.
	\end{itemize}
}
\item verb \\
If someone \textbf{breaks} something, especially a difficult or unpleasant situation that has existed for some time, they end it or
change it.
 \textbf{Break} is also a noun.
 \textit{
	\begin{itemize}
	\item New proposals have been put forward to break the deadlock among rival factions.
	\item The country is heading towards elections which may break the party's long hold on
power.
	\item Nothing that might lead to a break in the deadlock has been discussed yet.
	\end{itemize}
}
\item verb \\
If someone or something \textbf{breaks} a silence , they say something or make a noise after a long period of silence.
 \textit{
	\begin{itemize}
	\item Hugh broke the silence. 'Is she always late?' he asked.
	\item The unearthly silence was broken by a shrill screaming.
	\end{itemize}
}
\item countable noun \\
If there is a \textbf{break}  \textbf{in} the cloud or weather, it changes and there is a short period of sunshine or fine weather.
 \textit{
	\begin{itemize}
	\item A sudden break in the cloud allowed rescuers to spot Michael Benson.
	\end{itemize}
}
\item verb \\
If you \textbf{break}  \textbf{with} a group of people or a traditional way of doing things, or you \textbf{break} your connection with them, you stop being involved with that group or stop doing
things in that way.
 \textbf{Break} is also a noun.
 \textit{
	\begin{itemize}
	\item In 1959, Akihito broke with imperial tradition by marrying a commoner.
	\item They were determined to break from precedent.
	\item They have yet to break the link with the trade unions.
	\item Making a completely clean break with the past, the couple got rid of all their old
furniture.
	\end{itemize}
}
\item verb \\
If you \textbf{break} a habit or if someone \textbf{breaks} you \textbf{of} it, you no longer have that habit.
 \textit{
	\begin{itemize}
	\item We don't like breaking habits when it comes to food.
	\item The professor hoped to break the students of the habit of looking for easy answers.
	\end{itemize}
}
\item verb \\
To \textbf{break} someone means to destroy their determination and courage , their success, or their career .
 \textit{
	\begin{itemize}
	\item He never let his jailers break him.
	\item The newspapers and television can make or break you.
	\item Ken's wife, Vicki, said: 'He's a broken man.'
	\end{itemize}
}
\item verb \\
If someone \textbf{breaks}  \textbf{for} a short period of time, they rest or change from what they are doing for a short
period.
 \textit{
	\begin{itemize}
	\item They broke for lunch.
	\end{itemize}
}
\item countable noun \\
A \textbf{break} is a short period of time when you have a rest or a change from what you are doing,
especially if you are working or if you are in a boring or unpleasant situation.
 \textit{
	\begin{itemize}
	\item They may be able to help with childcare so that you can have a break.
	\item I thought a 15 min break from his work would do him good.
	\item She rang Moira during a coffee break.
	\end{itemize}
}
\item countable noun \\
A \textbf{break} is a short holiday.
 \textit{
	\begin{itemize}
	\item They are currently taking a short break in Spain.
	\end{itemize}
}
\item verb \\
If you \textbf{break} your journey somewhere , you stop there for a short time so that you can have a rest.
 \textit{
	\begin{itemize}
	\item Because of the heat we broke our journey at a small country hotel.
	\end{itemize}
}
\item verb \\
To \textbf{break} the force of something such as a blow or fall means to weaken its effect, for example
by getting in the way of it.
 \textit{
	\begin{itemize}
	\item He sustained serious neck injuries after he broke someone's fall.
	\end{itemize}
}
\item verb \\
When a piece of news \textbf{breaks} , people hear about it online , or from the newspapers, television, or radio.
 \textit{
	\begin{itemize}
	\item The news broke that the Prime Minister had resigned.
	\item He resigned from his post as Bishop when the scandal broke.
	\end{itemize}
}
\item verb \\
When you \textbf{break} a piece of bad news to someone, you tell it to them, usually in a kind way.
 \textit{
	\begin{itemize}
	\item Then Louise broke the news that she was leaving me.
	\item I worried for ages and decided that I had better break it to her.
	\end{itemize}
}
\item countable noun \\
A \textbf{break} is a lucky opportunity that someone gets to achieve something.
 \textit{
	\begin{itemize}
	\item He got his first break appearing in a variety show.
	\end{itemize}
}
\item verb \\
If you \textbf{break} a record, you beat the previous record for a particular achievement .
 \textit{
	\begin{itemize}
	\item He has broken the world record in the 100 metres.
	\item The film had broken all box office records.
	\end{itemize}
}
\item verb \\
When day or dawn  \textbf{breaks} , it starts to grow light after the night has ended.
 \textit{
	\begin{itemize}
	\item They continued the search as dawn broke.
	\end{itemize}
}
\item verb \\
When a wave \textbf{breaks} , it passes its highest point and turns downwards , for example when it reaches the shore.
 \textit{
	\begin{itemize}
	\item Danny listened to the waves breaking against the shore.
	\end{itemize}
}
\item verb \\
If you \textbf{break} a secret code, you work out how to understand it.
 \textit{
	\begin{itemize}
	\item It was feared they could break the Allies' codes.
	\end{itemize}
}
\item verb \\
If someone's voice \textbf{breaks} when they are speaking, it changes its sound, for example because they are sad or afraid .
 \textit{
	\begin{itemize}
	\item Godfrey's voice broke, and halted.
	\end{itemize}
}
\item verb \\
When a boy's voice \textbf{breaks} , it becomes deeper and sounds more like a man's voice.
 \textit{
	\begin{itemize}
	\item He sings with the strained discomfort of someone whose voice hasn't quite broken.
	\end{itemize}
}
\item verb \\
If the weather \textbf{breaks} or a storm  \textbf{breaks} , it suddenly becomes rainy or stormy after a period of sunshine.
 \textit{
	\begin{itemize}
	\item I've been waiting for the weather to break.
	\item She hoped she'd be able to reach the hotel before the storm broke.
	\end{itemize}
}
\item verb \\
In tennis , if you \textbf{break} your opponent's serve, you win a game in which your opponent is serving.
 \textbf{Break} is also a noun.
 \textit{
	\begin{itemize}
	\item The world No 5 broke the 25-year-old Cypriot's serve twice.
	\item A single break of serve settled the first two sets.
	\end{itemize}
}
\item  \\
 the break of day/dawn \textit{
	\begin{itemize}
	\end{itemize}
}
\item  \\
 give sb a break \textit{
	\begin{itemize}
	\end{itemize}
}
\item  \\
 to make a break (for it) \textit{
	\begin{itemize}
	\end{itemize}
}
\end{enumerate}

\section*{collide}
{\large \color{blue}  collides  colliding  collided  }
\subsection*{Explain}
\begin{enumerate}
\item verb \\
If two or more moving people or objects \textbf{collide} , they crash into one another. If a moving person or object \textbf{collides}  \textbf{with} a person or object that is not moving, they crash into them.
 \textit{
	\begin{itemize}
	\item Two trains collided head-on in north-eastern Germany early this morning.
	\item Racing up the stairs, he almost collided with Daisy.
	\item He collided with a pine tree near the North Gate.
	\end{itemize}
}
\item verb \\
If the aims , opinions, or interests of one person or group \textbf{collide}  \textbf{with} those of another person or group, they are very different from each other and are
therefore opposed .
 \textit{
	\begin{itemize}
	\item The aims of the negotiators in New York again seem likely to collide with the aims
of the warriors in the field.
	\item What happens when the two interests collide will make a fascinating spectacle.
	\end{itemize}
}
\end{enumerate}

\section*{caress}
{\large \color{blue}  caresses  caressing  caressed  }
\subsection*{Explain}
\begin{enumerate}
\item verb \\
If you \textbf{caress} someone, you stroke them gently and affectionately.
 \textbf{Caress} is also a noun .
 \textit{
	\begin{itemize}
	\item He was gently caressing her golden hair.
	\item Margaret took me to one side, holding my arm in a gentle caress.
	\end{itemize}
}
\end{enumerate}

\section*{condemn}
{\large \color{blue}  condemns  condemning  condemned  }
\subsection*{Explain}
\begin{enumerate}
\item verb \\
If you \textbf{condemn} something, you say that it is very bad and unacceptable .
 \textit{
	\begin{itemize}
	\item Political leaders united yesterday to condemn the latest wave of violence.
	\item Graham was right to condemn his players for lack of ability, attitude and application.
	\item ...a document that condemns sexism as a moral and social evil.
	\end{itemize}
}
\item verb \\
If someone \textbf{is condemned}  \textbf{to} a punishment , they are given this punishment.
 \textit{
	\begin{itemize}
	\item He was condemned to life imprisonment.
	\item ...appeals by prisoners condemned to death.
	\end{itemize}
}
\item verb \\
If circumstances  \textbf{condemn} you \textbf{to} an unpleasant  situation , they make it certain that you will  suffer in that way.
 \textit{
	\begin{itemize}
	\item Their lack of qualifications condemned them to a lifetime of boring, usually poorly-paid
work.
	\item He felt condemned to being alone.
	\item Mark was condemned to do most of the work.
	\end{itemize}
}
\item verb \\
If authorities  \textbf{condemn} a building, they officially  decide that it is not safe and must be pulled down or repaired .
 \textit{
	\begin{itemize}
	\item ...proceedings to condemn buildings in the area.
	\end{itemize}
}
\end{enumerate}

\section*{commence}
{\large \color{blue}  commences  commencing  commenced  }
\subsection*{Explain}
\begin{enumerate}
\item verb \\
When something \textbf{commences} or you \textbf{commence} it, it begins.
 \textit{
	\begin{itemize}
	\item The academic year commences at the beginning of October.
	\item They commenced a systematic search.
	\item The hunter knelt beside the animal carcass and commenced to skin it.
	\end{itemize}
}
\end{enumerate}

\section*{conform}
{\large \color{blue}  conforms  conforming  conformed  }
\subsection*{Explain}
\begin{enumerate}
\item verb \\
If something \textbf{conforms}  \textbf{to} something such as a law or someone's wishes , it is of the required type or quality.
 \textit{
	\begin{itemize}
	\item The lamp has been designed to conform to British safety requirements.
	\item The meat market can continue only if it is radically overhauled to conform with strict
European standards.
	\end{itemize}
}
\item verb \\
If you \textbf{conform} , you behave in the way that you are expected or supposed to behave.
 \textit{
	\begin{itemize}
	\item Many children who can't or don't conform are often bullied.
	\item He did not feel obliged to conform to the rules that applied to ordinary men.
	\item We conformed with social and family expectations.
	\end{itemize}
}
\item verb \\
If someone or something \textbf{conforms to} a pattern or type, they are very similar to it.
 \textit{
	\begin{itemize}
	\item I am well aware that we all conform to one stereotype or another.
	\item Like most 'peacetime wars' it did not conform to preconceived ideas.
	\end{itemize}
}
\end{enumerate}

\section*{compare}
{\large \color{blue}  compares  comparing  compared  }
\subsection*{Explain}
\begin{enumerate}
\item verb \\
When you \textbf{compare} things, you consider them and discover the differences or similarities between them.
 \textit{
	\begin{itemize}
	\item Compare the two illustrations in Fig 60.
	\item Was it fair to compare independent schools with state schools?
	\item Note how smooth the skin of the upper arm is, then compare it to the skin on the
elbow.
	\end{itemize}
}
\item verb \\
If you \textbf{compare} one person or thing \textbf{to} another, you say that they are like the other person or thing.
 \textit{
	\begin{itemize}
	\item Some commentators compared his work to that of James Joyce.
	\item I can only compare the experience to falling in love.
	\end{itemize}
}
\item verb \\
If one thing \textbf{compares} favourably \textbf{with} another, it is better than the other thing. If it \textbf{compares} unfavourably, it is worse than the other thing.
 \textit{
	\begin{itemize}
	\item Our road safety record compares favourably with that of other European countries.
	\item How do the two techniques compare in terms of application?
	\end{itemize}
}
\item verb \\
If you say that something does not \textbf{compare with} something else, you mean that it is much worse.
 \textit{
	\begin{itemize}
	\item The flowers here do not compare with those at home.
	\item The more recent conifer plantations cannot yet compare with the old woodlands.
	\end{itemize}
}
\item  \\
 beyond compare \textit{
	\begin{itemize}
	\end{itemize}
}
\end{enumerate}

\section*{cover}
{\large \color{blue}  covers  covering  covered  }
\subsection*{Explain}
\begin{enumerate}
\item verb \\
If you \textbf{cover} something, you place something else over it in order to protect it, hide it, or close
it.
 \textit{
	\begin{itemize}
	\item Cover the casserole with a tight-fitting lid.
	\item He whimpered and covered his face.
	\item Keep what's left in a covered container in the fridge.
	\end{itemize}
}
\item verb \\
If one thing \textbf{covers} another, it has been placed over it in order to protect it, hide it, or close it.
 \textit{
	\begin{itemize}
	\item His finger went up to touch the black patch which covered his left eye.
	\item His head was covered with a khaki turban.
	\end{itemize}
}
\item verb \\
If one thing \textbf{covers} another, it forms a layer over its surface.
 \textit{
	\begin{itemize}
	\item The clouds had spread and nearly covered the entire sky.
	\item Two oil slicks are covering a total area of seven square miles.
	\item The desk was covered with papers.
	\end{itemize}
}
\item verb \\
To \textbf{cover} something \textbf{with} or \textbf{in} something else means to put a layer of the second thing over its surface.
 \textit{
	\begin{itemize}
	\item The trees in your garden may have covered the ground with apples, pears or plums.
	\item She covered the walls with the signs of the zodiac.
	\end{itemize}
}
\item verb \\
If you \textbf{cover} a particular distance, you travel that distance.
 \textit{
	\begin{itemize}
	\item It would not be easy to cover ten miles on that amount of petrol.
	\item It covered the distance in 28 hours compared with the train's six days.
	\end{itemize}
}
\item verb \\
To \textbf{cover} someone or something means to protect them from attack, for example by pointing a
gun in the direction of people who may attack them, ready to fire the gun if necessary.
 \textit{
	\begin{itemize}
	\item You go first. I'll cover you.
	\end{itemize}
}
\item uncountable noun \\
\textbf{Cover} is protection from enemy attack that is provided for troops or ships carrying out a particular operation, for example by aircraft.
 \textit{
	\begin{itemize}
	\item They said they could not provide adequate air cover for ground operations.
	\end{itemize}
}
\item uncountable noun \\
\textbf{Cover} is trees, rocks, or other places where you shelter from the weather or from an attack,
or hide from someone.
 \textit{
	\begin{itemize}
	\item Charles lit the fuses and they ran for cover.
	\item ...barren wastes of field with no trees and no cover.
	\end{itemize}
}
\item verb \\
An insurance policy that \textbf{covers} a person or thing guarantees that money will be paid by the insurance company in
relation to that person or thing.
 \textit{
	\begin{itemize}
	\item Their insurer paid the £900 bill, even though the policy did not strictly cover it.
	\item These items are not covered by your medical insurance.
	\item You should take out travel insurance covering you and your family against theft.
	\end{itemize}
}
\item uncountable noun \\
Insurance \textbf{cover} is a guarantee from an insurance company that money will be paid by them if it is
needed.
 \textit{
	\begin{itemize}
	\item Make sure that the firm's insurance cover is adequate.
	\end{itemize}
}
\item verb \\
If a law \textbf{covers} a particular set of people, things, or situations, it applies to them.
 \textit{
	\begin{itemize}
	\item The law covers four categories of experiments.
	\item In the US, the matter is covered by the Copyright Act of 1976.
	\end{itemize}
}
\item verb \\
If you \textbf{cover} a particular topic , you discuss it in a lecture , course, or book.
 \textit{
	\begin{itemize}
	\item The Oxford Chemistry Primers aim to cover important topics in organic chemistry.
	\item Other subjects covered included nerves and how to overcome them.
	\end{itemize}
}
\item verb \\
If journalists , newspapers, or television companies \textbf{cover} an event, they report on it.
 \textit{
	\begin{itemize}
	\item Robinson was sent to Italy to cover the World Cup.
	\item The U.S. news media will cover the trial closely.
	\end{itemize}
}
\item verb \\
If a sum of money \textbf{covers} something, it is enough to pay for it.
 \textit{
	\begin{itemize}
	\item Send it to the address given with £1.50 to cover postage and administration.
	\item Those figures might not even cover the cost of breakages.
	\end{itemize}
}
\item countable noun \\
A \textbf{cover} is something which is put over an object, usually in order to protect it.
 \textit{
	\begin{itemize}
	\item ...a family room with washable covers on the furniture.
	\item ...a duvet cover.
	\end{itemize}
}
\item plural noun \\
The \textbf{covers} on your bed are the things such as sheets and blankets that you have on top of you.
 \textit{
	\begin{itemize}
	\item She set her glass down and slid farther under the covers.
	\end{itemize}
}
\item countable noun \\
The \textbf{cover} of a book or a magazine is the outside part of it.
 \textit{
	\begin{itemize}
	\item He was the second jazz musician to be featured on the cover of Time magazine.
	\item ...a small spiral-bound booklet with a green cover.
	\item I used to read every issue from cover to cover.
	\end{itemize}
}
\item countable noun \\
Something that is a \textbf{cover} for secret or illegal activities seems  respectable or normal, and is intended to hide the activities.
 \textit{
	\begin{itemize}
	\item They set up a spurious temple that was a cover for sexual debauchery.
	\item As a cover story he generally tells people he is a freelance photographer.
	\end{itemize}
}
\item verb \\
If you \textbf{cover for} someone who is doing something secret or illegal, you give false information or do
not give all the information you have, in order to protect them.
 \textit{
	\begin{itemize}
	\item Why would she cover for someone who was trying to kill her?
	\end{itemize}
}
\item verb \\
If you \textbf{cover for} someone who is ill or away, you do their work for them while they are not there.
 \textit{
	\begin{itemize}
	\item She did not have enough nurses to cover for those who went ill or took holiday.
	\end{itemize}
}
\item verb \\
To \textbf{cover} a song originally performed by someone else means to record a new version of it.
 \textit{
	\begin{itemize}
	\item He must make a decent living from other artists covering his songs.
	\end{itemize}
}
\item countable noun \\
A \textbf{cover} is the same as a cover version .
 \textit{
	\begin{itemize}
	\item The single is a cover of an old Rolling Stones song.
	\end{itemize}
}
\item  \\
 to blow someone's cover \textit{
	\begin{itemize}
	\end{itemize}
}
\item  \\
 to break cover \textit{
	\begin{itemize}
	\end{itemize}
}
\item  \\
 to take cover \textit{
	\begin{itemize}
	\end{itemize}
}
\item  \\
 under cover \textit{
	\begin{itemize}
	\end{itemize}
}
\item  \\
 under cover of \textit{
	\begin{itemize}
	\end{itemize}
}
\item  \\
 cover your back/cover your rear \textit{
	\begin{itemize}
	\end{itemize}
}
\item  \\
 cover your ass \textit{
	\begin{itemize}
	\end{itemize}
}
\end{enumerate}

\section*{deduce}
{\large \color{blue}  deduces  deducing  deduced  }
\subsection*{Explain}
\begin{enumerate}
\item verb \\
If you \textbf{deduce} something or \textbf{deduce} that something is true , you reach that conclusion because of other things that you know to be true.
 \textit{
	\begin{itemize}
	\item Alison had cleverly deduced that I was the author of the letter.
	\item The date of the document can be deduced from references to the Civil War.
	\item She hoped he hadn't deduced the reason for her visit.
	\end{itemize}
}
\end{enumerate}

\section*{elapse}
{\large \color{blue}  elapses  elapsing  elapsed  }
\subsection*{Explain}
\begin{enumerate}
\item verb \\
When time \textbf{elapses} , it passes.
 \textit{
	\begin{itemize}
	\item Forty-eight hours have elapsed since his arrest.
	\end{itemize}
}
\end{enumerate}

\section*{denounce}
{\large \color{blue}  denounces  denouncing  denounced  }
\subsection*{Explain}
\begin{enumerate}
\item verb \\
If you \textbf{denounce} a person or an action, you criticize them severely and publicly because you feel strongly that they are wrong or evil.
 \textit{
	\begin{itemize}
	\item The letter called for civil rights, but did not openly denounce the regime.
	\item German leaders denounced the attacks and pleaded for tolerance.
	\item Some 25,000 demonstrators denounced him as a traitor.
	\end{itemize}
}
\item verb \\
If you \textbf{denounce} someone who has broken a rule or law, you report them to the authorities.
 \textit{
	\begin{itemize}
	\item ...informers who might denounce you at any moment.
	\end{itemize}
}
\end{enumerate}

\section*{erase}
{\large \color{blue}  erases  erasing  erased  }
\subsection*{Explain}
\begin{enumerate}
\item verb \\
If you \textbf{erase} a thought or feeling , you destroy it completely so that you can no longer remember something or no longer feel a particular emotion .
 \textit{
	\begin{itemize}
	\item They are desperate to erase the memory of that last defeat in Cardiff.
	\item Love was a word he'd erased from his vocabulary since Susan's going.
	\end{itemize}
}
\item verb \\
If you \textbf{erase} sound which has been recorded on a tape or information which has been stored in a computer, you completely remove or destroy it.
 \textit{
	\begin{itemize}
	\item He was in the studio tearfully erasing all the tapes he'd slaved over.
	\item It appears the names were accidentally erased from computer disks.
	\end{itemize}
}
\item verb \\
If you \textbf{erase} something such as writing or a mark , you remove it, usually by rubbing it with a cloth .
 \textit{
	\begin{itemize}
	\item It was unfortunate that she had erased the message.
	\end{itemize}
}
\end{enumerate}

\section*{depart}
{\large \color{blue}  departs  departing  departed  }
\subsection*{Explain}
\begin{enumerate}
\item verb \\
When something or someone \textbf{departs}  \textbf{from} a place, they leave it and start a journey to another place.
 \textit{
	\begin{itemize}
	\item Our tour departs from Heathrow Airport on 31 March and returns 16 April.
	\item In the morning Mr McDonald departed for Sydney.
	\item The coach departs Potsdam in the morning.
	\end{itemize}
}
\item verb \\
If you \textbf{depart}  \textbf{from} a traditional , accepted , or agreed way of doing something, you do it in a different or unexpected way.
 \textit{
	\begin{itemize}
	\item Why is it in this country that we have departed from good educational sense?
	\item It takes a brave cook to depart radically from the traditional Christmas menu.
	\end{itemize}
}
\item verb \\
If someone \textbf{departs} from a job , they resign from it or leave it. In American English, you can say that someone \textbf{departs} a job.
 \textit{
	\begin{itemize}
	\item Lipton is planning to depart from the company he founded.
	\item It is not unusual for staff to depart at this time of year.
	\item He departed baseball in the '60s.
	\end{itemize}
}
\item verb \\
When someone \textbf{departs}  \textbf{this life} , or \textbf{departs}  \textbf{this earth} , they die .
 \textit{
	\begin{itemize}
	\item He departed this world with a sense of having fulfilled his destiny.
	\end{itemize}
}
\end{enumerate}

\section*{giggle}
{\large \color{blue}  giggles  giggling  giggled  }
\subsection*{Explain}
\begin{enumerate}
\item verb \\
If someone \textbf{giggles} , they laugh in a childlike way, because they are amused , nervous , or embarrassed .
 \textbf{Giggle} is also a noun .
 \textit{
	\begin{itemize}
	\item Both girls began to giggle.
	\item 'I beg your pardon?' she giggled.
	\item ...a giggling little girl.
	\item She gave a little giggle.
	\end{itemize}
}
\item plural noun \\
If you say that someone has \textbf{the giggles} , you mean they cannot stop giggling.
 \textit{
	\begin{itemize}
	\item I was so nervous I got the giggles.
	\item She had a fit of the giggles.
	\end{itemize}
}
\item singular noun \\
If you say that something is \textbf{a giggle} , you mean it is fun or is amusing.
 \textit{
	\begin{itemize}
	\item I might buy one for a friend's birthday as a giggle.
	\end{itemize}
}
\end{enumerate}

\section*{detach}
{\large \color{blue}  detaches  detaching  detached  }
\subsection*{Explain}
\begin{enumerate}
\item verb \\
If you \textbf{detach} one thing \textbf{from} another that it is fixed to, you remove it. If one thing \textbf{detaches}  \textbf{from} another, it becomes separated from it.
 \textit{
	\begin{itemize}
	\item Detach the white part of the application form and keep it.
	\item It is easy to detach the currants from the stems.
	\item There was an accident when the towrope detached from the car.
	\end{itemize}
}
\item verb \\
If you \textbf{detach}  \textbf{yourself from} something, you become less involved in it or less concerned about it than you used to be.
 \textit{
	\begin{itemize}
	\item It helps them detach themselves from their problems and become more objective.
	\end{itemize}
}
\item verb \\
If you \textbf{detach}  \textbf{yourself from} a person or place, you leave them.
 \textit{
	\begin{itemize}
	\item Alexis saw his father detach himself from the group and walk away down the hill by
himself.
	\end{itemize}
}
\end{enumerate}

\section*{grind}
{\large \color{blue}  grinds  grinding  ground  }
\subsection*{Explain}
\begin{enumerate}
\item verb \\
If you \textbf{grind} a substance such as corn , you crush it between two hard surfaces or with a machine until it becomes a fine powder .
 \textbf{Grind up} means the same as grind .
 \textit{
	\begin{itemize}
	\item Store the peppercorns in an airtight container and grind the pepper as you need it.
	\item ...the odor of fresh ground coffee.
	\item He makes his own paint, grinding up the pigment with a little oil.
	\end{itemize}
}
\item verb \\
If you \textbf{grind} something \textbf{into} a surface, you press and rub it hard into the surface using small circular or sideways movements.
 \textit{
	\begin{itemize}
	\item 'Well,' I said, grinding my cigarette nervously into the granite step.
	\end{itemize}
}
\item verb \\
If you \textbf{grind} something, you make it smooth or sharp by rubbing it against a hard surface.
 \textit{
	\begin{itemize}
	\item ...a shop where they grind knives.
	\item The tip can be ground to a much sharper edge to cut smoother and faster.
	\end{itemize}
}
\item verb \\
If a vehicle \textbf{grinds}  somewhere , it moves there very slowly and noisily.
 \textit{
	\begin{itemize}
	\item Tanks had crossed the border at five fifteen and were grinding south.
	\end{itemize}
}
\item singular noun \\
The \textbf{grind}  \textbf{of} a machine is the harsh, scraping  noise that it makes, usually because it is old or is working too hard.
 \textit{
	\begin{itemize}
	\item The grind of heavy machines could get on their nerves.
	\end{itemize}
}
\item singular noun \\
If you refer to routine tasks or activities as \textbf{the}  \textbf{grind} , you mean they are boring and take up a lot of time and effort.
 \textit{
	\begin{itemize}
	\item The daily grind of government is done by Her Majesty's Civil Service.
	\item Life continues to be a terrible grind for the ordinary person.
	\end{itemize}
}
\item  \\
 grind to a halt \textit{
	\begin{itemize}
	\end{itemize}
}
\item  \\
 grind to a halt \textit{
	\begin{itemize}
	\end{itemize}
}
\end{enumerate}

\section*{disregard}
{\large \color{blue}  disregards  disregarding  disregarded  }
\subsection*{Explain}
\begin{enumerate}
\item verb \\
If you \textbf{disregard} something, you ignore it or do not take account of it.
 \textbf{Disregard} is also a noun .
 \textit{
	\begin{itemize}
	\item He disregarded the advice of his executives.
	\item Critics say he allowed the police and security forces to disregard human rights.
	\item Whoever planted the bomb showed a total disregard for the safety of the public.
	\end{itemize}
}
\end{enumerate}

\section*{hover}
{\large \color{blue}  hovers  hovering  hovered  }
\subsection*{Explain}
\begin{enumerate}
\item verb \\
To \textbf{hover} means to stay in the same position in the air without moving forwards or backwards . Many birds and insects can hover by moving their wings very quickly.
 \textit{
	\begin{itemize}
	\item Beautiful butterflies hovered above the wild flowers.
	\item A police helicopter hovered overhead.
	\item Mist hovered in all the valleys.
	\end{itemize}
}
\item verb \\
If you \textbf{hover} , you stay in one place and move slightly in a nervous way, for example because you cannot decide what to do.
 \textit{
	\begin{itemize}
	\item Judith was hovering in the doorway.
	\item With no idea of what to do for my next move, my hand hovered over the board.
	\end{itemize}
}
\item verb \\
If you \textbf{hover} , you are in an uncertain  situation or state of mind .
 \textit{
	\begin{itemize}
	\item She hovered on the brink of death for three months as doctors battled to save her.
	\item Just as at the turn of the century, we hover between great hopes and great fears.
	\end{itemize}
}
\item verb \\
If a something such as a price , value, or score  \textbf{hovers} around a particular level , it stays at more or less that level and does not change much.
 \textit{
	\begin{itemize}
	\item Temperatures hovered around freezing.
	\item His golf handicap hovered between 10 and 12.
	\end{itemize}
}
\end{enumerate}

\section*{ebb}
{\large \color{blue}  ebbs  ebbing  ebbed  }
\subsection*{Explain}
\begin{enumerate}
\item verb \\
When the tide or the sea  \textbf{ebbs} , its level gradually falls.
 \textit{
	\begin{itemize}
	\item When the tide ebbs it's a rock pool inhabited by crustaceans.
	\end{itemize}
}
\item countable noun \\
\textbf{The}  \textbf{ebb} or the \textbf{ebb} tide is one of the regular periods, usually two per day , when the sea gradually falls to a lower level as the tide moves away from the land .
 \textit{
	\begin{itemize}
	\item ...the spring ebb tide.
	\item We decided to leave on the ebb at six o'clock next morning.
	\end{itemize}
}
\item verb \\
If someone's life , support , or feeling  \textbf{ebbs} , it becomes weaker and gradually disappears .
 \textbf{Ebb away}  means the same as ebb .
 \textit{
	\begin{itemize}
	\item ...as a man's physical strength ebbs.
	\item Were there occasions when enthusiasm ebbed?
	\item His little girl's life ebbed away.
	\item Their popular support is ebbing away.
	\end{itemize}
}
\item  \\
 at a/o's low(est) ebb \textit{
	\begin{itemize}
	\end{itemize}
}
\item  \\
 ebb and flow \textit{
	\begin{itemize}
	\end{itemize}
}
\end{enumerate}

\section*{insist}
{\large \color{blue}  insists  insisting  insisted  }
\subsection*{Explain}
\begin{enumerate}
\item verb \\
If you \textbf{insist}  \textbf{that} something should be done, you say so very firmly and refuse to give in about it. If you \textbf{insist}  \textbf{on} something, you say firmly that it must be done or provided.
 \textit{
	\begin{itemize}
	\item My family insisted that I should not give in, but stay and fight.
	\item She insisted on being present at all the interviews.
	\item She insists on all her employees coming to the Christmas lunch she gives every year.
	\item I didn't want to join in, but Kenneth insisted.
	\end{itemize}
}
\item verb \\
If you \textbf{insist} that something is the case , you say so very firmly and refuse to say otherwise, even though other people do not believe you.
 \textit{
	\begin{itemize}
	\item The president insisted that he was acting out of compassion, not opportunism.
	\item 'It's not that difficult,' she insists.
	\item Crippen insisted on his innocence.
	\end{itemize}
}
\end{enumerate}

\section*{initiate}
{\large \color{blue}  initiates  initiating  initiated  }
\subsection*{Explain}
\begin{enumerate}
\item verb \\
If you \textbf{initiate} something, you start it or cause it to happen .
 \textit{
	\begin{itemize}
	\item They wanted to initiate a discussion on economics.
	\item The trip was initiated by the manager of the community centre.
	\end{itemize}
}
\item verb \\
If you \textbf{initiate} someone \textbf{into} something, you introduce them to a particular  skill or type of knowledge and teach them about it.
 \textit{
	\begin{itemize}
	\item He initiated her into the study of other cultures.
	\end{itemize}
}
\item verb \\
If someone \textbf{is initiated}  \textbf{into} something such as a religion , secret society , or social group, they become a member of it by taking part in ceremonies at which they learn its special knowledge or customs .
 \textit{
	\begin{itemize}
	\item In many societies, young people are formally initiated into their adult roles.
	\item ...the ceremony that initiated members into the Order.
	\end{itemize}
}
\item countable noun \\
An \textbf{initiate} is a person who has been accepted as a member by a particular group or club and been
taught its secrets and skills.
 \textit{
	\begin{itemize}
	\item Chen was an initiate of a Chinese spiritual discipline.
	\end{itemize}
}
\end{enumerate}

\section*{lash}
{\large \color{blue}  lashes  lashing  lashed  }
\subsection*{Explain}
\begin{enumerate}
\item countable noun \\
Your \textbf{lashes} are the hairs that grow on the edge of your upper and lower  eyelids .
 \textit{
	\begin{itemize}
	\item ...sombre grey eyes, with unusually long lashes.
	\item Joanna studied him through her lashes.
	\end{itemize}
}
\item verb \\
If you \textbf{lash} two or more things together, you tie one of them firmly to the other.
 \textit{
	\begin{itemize}
	\item Secure the anchor by lashing it to the rail.
	\item The shelter is built by lashing poles together to form a small dome.
	\item Cindy lashed her motorboat alongside.
	\item We were worried about the lifeboat which was not lashed down.
	\end{itemize}
}
\item verb \\
If wind, rain, or water \textbf{lashes} someone or something, it hits them violently.
 \textit{
	\begin{itemize}
	\item The worst winter storms of the century lashed the east coast of North America.
	\item Suddenly rain lashed against the windows.
	\item The rain was absolutely lashing down.
	\item ...gales of lashing rain.
	\end{itemize}
}
\item verb \\
If someone \textbf{lashes} you or \textbf{lashes}  \textbf{into} you, they speak very angrily to you, criticizing you or saying you have done something wrong .
 \textbf{Lash} is also a noun .
 \textit{
	\begin{itemize}
	\item She went quiet for a moment while she summoned up the words to lash him.
	\item The report lashes into police commanders for failing to act on intelligence information.
	\item Never before had he felt the full lash of John's temper.
	\end{itemize}
}
\item countable noun \\
A \textbf{lash} is a thin  strip of leather at the end of a whip.
 \textit{
	\begin{itemize}
	\end{itemize}
}
\item countable noun \\
A \textbf{lash} is a blow with a whip, especially a blow on someone's back as a punishment.
 \textit{
	\begin{itemize}
	\item The villagers sentenced one man to five lashes for stealing a ham from his neighbor.
	\end{itemize}
}
\item verb \\
If someone \textbf{lashes} another person, they hit that person with a whip.
 \textit{
	\begin{itemize}
	\item They snatched up whips and lashed the backs of those who had fallen.
	\end{itemize}
}
\item ergative verb \\
If an animal \textbf{lashes} its tail , or if its tail \textbf{lashes} , it moves its tail very fast and violently.
 \textit{
	\begin{itemize}
	\item When in danger, the anteater lashes its tail round a branch.
	\item They tried to get the harpoon into the ray before the sting tail came lashing over
to retaliate.
	\item Don't go near that lashing tail.
	\end{itemize}
}
\end{enumerate}

\section*{install}
{\large \color{blue}  installs  installing  installed  }
\subsection*{Explain}
\begin{enumerate}
\item verb \\
If you \textbf{install} a piece of equipment, you fit it or put it somewhere so that it is ready to be used.
 \textit{
	\begin{itemize}
	\item They had installed a new phone line in the apartment.
	\end{itemize}
}
\item verb \\
If someone \textbf{is installed} in a new job or important position, they are officially given the job or position, often in a special  ceremony .
 \textit{
	\begin{itemize}
	\item Almost a century of upheaval ended when William III of Orange was installed on the
throne.
	\item The opposition candidate of the previous May was installed as president.
	\item The army has promised to install a new government within a week.
	\end{itemize}
}
\item verb \\
If you \textbf{install}  \textbf{yourself} in a particular place, you settle there and make yourself comfortable .
 \textit{
	\begin{itemize}
	\item She had installed herself and her daughter, Cathy, in a villa.
	\end{itemize}
}
\end{enumerate}

\section*{look}
{\large \color{blue}  looks  looking  looked  }
\subsection*{Explain}
\begin{enumerate}
\item verb \\
If you \textbf{look} in a particular direction, you direct your eyes in that direction, especially so that you can see what is there or see what something is like.
 \textbf{Look} is also a noun .
 \textit{
	\begin{itemize}
	\item I looked down the hallway to room number nine.
	\item She turned to look at him.
	\item He looked away, apparently enraged.
	\item If you look, you'll see what was a lake.
	\item Lucille took a last look in the mirror.
	\item Assisi has a couple of churches that are worth a look if you have time.
	\end{itemize}
}
\item verb \\
If you \textbf{look at} a book, newspaper , or magazine , you read it fairly quickly or read part of it.
 \textbf{Look} is also a noun.
 \textit{
	\begin{itemize}
	\item You've just got to look at the last bit of Act Three.
	\item A quick look at Monday's British newspapers shows that there's plenty of interest
in foreign news.
	\end{itemize}
}
\item verb \\
If someone, especially an expert , \textbf{looks} at something, they examine it, and then deal with it or say how it should be dealt with.
 \textbf{Look} is also a noun.
 \textit{
	\begin{itemize}
	\item Can you look at my back? I think something's wrong.
	\item The car has not been running very well and a mechanic had to come over to have a
look at it.
	\end{itemize}
}
\item verb \\
If you \textbf{look at} someone in a particular way, you look at them with your expression showing what you
are feeling or thinking .
 \textbf{Look} is also a noun.
 \textit{
	\begin{itemize}
	\item She looked at him earnestly. 'You don't mind?'
	\item He gave her a blank look, as if he had no idea who she was.
	\item Sally spun round, a feigned look of surprise on her face.
	\end{itemize}
}
\item verb \\
If you \textbf{look}  \textbf{for} something, for example something that you have lost , you try to find it.
 \textbf{Look} is also a noun.
 \textit{
	\begin{itemize}
	\item I'm looking for a child. I believe your husband can help me find her.
	\item I had gone to Maine looking for a place to work.
	\item I looked everywhere for ideas.
	\item Have you looked on the piano?
	\item Go and have another look.
	\end{itemize}
}
\item verb \\
If you \textbf{are looking for} something such as the solution to a problem or a new method , you want it and are trying to obtain it or think of it.
 \textit{
	\begin{itemize}
	\item The working group will be looking for practical solutions to the problems faced by
doctors.
	\item He's looking for a way out from this conflict.
	\end{itemize}
}
\item verb \\
If you \textbf{look at} a subject, problem, or situation , you think about it or study it, so that you know all about it and can perhaps  consider what should be done in relation to it.
 \textbf{Look} is also a noun.
 \textit{
	\begin{itemize}
	\item Next term we'll be looking at the Second World War period.
	\item He visited Florida a few years ago looking at the potential of the area to stage
a big match.
	\item A close look at the statistics reveals a troubling picture.
	\end{itemize}
}
\item verb \\
If you \textbf{look at} a person, situation, or subject from a particular point of view, you judge them or consider them from that point of view.
 \textit{
	\begin{itemize}
	\item Brian had learned to look at her with new respect.
	\item It depends how you look at it.
	\end{itemize}
}
\item convention \\
You say \textbf{look} when you want someone to pay attention to you because you are going to say something important .
 \textit{
	\begin{itemize}
	\item Look, I'm sorry. I didn't mean it.
	\item Now, look, here is how things stand.
	\end{itemize}
}
\item verb \\
You can use \textbf{look} to draw attention to a particular situation, person, or thing, for example because you find
it very surprising , significant , or annoying .
 \textit{
	\begin{itemize}
	\item Hey, look at the time! We'll talk about it tonight. All right?
	\item I mean, look at how many people watch television and how few read books.
	\item Look what a mess you've made of your life.
	\end{itemize}
}
\item verb \\
If something such as a building or window  \textbf{looks}  somewhere , it has a view of a particular place.
 \textbf{Look out} means the same as look1 .
 \textit{
	\begin{itemize}
	\item The castle looks over private parkland.
	\item Each front door looks across a narrow alley to the front door opposite.
	\item Nine windows looked out over the sculpture gardens.
	\item We sit on the terrace, which looks out on the sea.
	\end{itemize}
}
\item verb \\
If you \textbf{are looking} to do something, you are aiming to do it.
 \textit{
	\begin{itemize}
	\item We're not looking to make a fortune.
	\item ...young mums looking to get fit after having kids.
	\end{itemize}
}
\item  \\
 never looked back \textit{
	\begin{itemize}
	\end{itemize}
}
\item  \\
 to look someone in the eye \textit{
	\begin{itemize}
	\end{itemize}
}
\item  \\
 to look the other way \textit{
	\begin{itemize}
	\end{itemize}
}
\item  \\
 look here \textit{
	\begin{itemize}
	\end{itemize}
}
\item  \\
 look out \textit{
	\begin{itemize}
	\end{itemize}
}
\item  \\
 look sb up and down \textit{
	\begin{itemize}
	\end{itemize}
}
\end{enumerate}

\section*{insult}
{\large \color{blue}  insults  insulting  insulted  }
\subsection*{Explain}
\begin{enumerate}
\item verb \\
If someone \textbf{insults} you, they say or do something that is rude or offensive.
 \textit{
	\begin{itemize}
	\item I did not mean to insult you.
	\item Buchanan said he was insulted by the judge's remarks.
	\end{itemize}
}
\item countable noun \\
An \textbf{insult} is a rude remark, or something a person says or does which insults you.
 \textit{
	\begin{itemize}
	\item Their behaviour was an insult to the people they represent.
	\item The prison Governor criticised some of his officers who shouted insults at prisoners
on the roof.
	\end{itemize}
}
\item  \\
 to add insult to injury \textit{
	\begin{itemize}
	\end{itemize}
}
\end{enumerate}

\section*{operate}
{\large \color{blue}  operates  operating  operated  }
\subsection*{Explain}
\begin{enumerate}
\item verb \\
If you \textbf{operate} a business or organization , you work to keep it running properly. If a business or organization \textbf{operates} , it carries out its work.
 \textit{
	\begin{itemize}
	\item Until his death in 1986, Greenwood owned and operated an enormous pear orchard.
	\item ...allowing commercial banks to operate in the country.
	\item Operating costs jumped from £85.3m to £95m.
	\end{itemize}
}
\item verb \\
The way that something \textbf{operates} is the way that it works or has a particular effect.
 \textit{
	\begin{itemize}
	\item Ceiling and wall lights can operate independently.
	\item How do accounting records operate?
	\item The world of work doesn't operate that way.
	\end{itemize}
}
\item verb \\
When you \textbf{operate} a machine or device , or when it \textbf{operates} , you make it work.
 \textit{
	\begin{itemize}
	\item A massive rock fall trapped the men as they operated a tunnelling machine.
	\item The number of these machines operating around the world has now reached ten million.
	\end{itemize}
}
\item verb \\
When surgeons  \textbf{operate}  \textbf{on} a patient in a hospital , they cut  open a patient's body in order to remove , replace , or repair a diseased or damaged part.
 \textit{
	\begin{itemize}
	\item The surgeon who operated on the King released new details of his injuries.
	\item You examine a patient and then you decide whether or not to operate.
	\end{itemize}
}
\item verb \\
If military forces  \textbf{are operating}  \textbf{in} a particular region , they are in that place in order to carry out their orders.
 \textit{
	\begin{itemize}
	\item Up to ten thousand soldiers are operating in the area.
	\item This freed the Austrian army to operate against the French.
	\end{itemize}
}
\end{enumerate}

\section*{introduce}
{\large \color{blue}  introduces  introducing  introduced  }
\subsection*{Explain}
\begin{enumerate}
\item verb \\
To \textbf{introduce} something means to cause it to enter a place or exist in a system for the first time.
 \textit{
	\begin{itemize}
	\item The Government has introduced a number of other money-saving moves.
	\item I kept the birds indoors all winter and introduced them into an aviary the following
June.
	\item The word 'Pagoda' was introduced to Europe by the 17th-century Portuguese.
	\end{itemize}
}
\item verb \\
If you \textbf{introduce} someone \textbf{to} something, you cause them to learn about it or experience it for the first time.
 \textit{
	\begin{itemize}
	\item He introduced us to the delights of natural food.
	\end{itemize}
}
\item verb \\
If you \textbf{introduce} one person \textbf{to} another, or you \textbf{introduce} two people, you tell them each other's names, so that they can get to know each other. If you \textbf{introduce}  \textbf{yourself} to someone, you tell them your name.
 \textit{
	\begin{itemize}
	\item Tim, may I introduce you to my uncle's secretary, Mary Waller?
	\item Someone introduced us and I sat next to him.
	\item We haven't been introduced. My name is Nero Wolfe.
	\item Let me introduce myself.
	\end{itemize}
}
\item verb \\
The person who \textbf{introduces} a television or radio programme speaks at the beginning of it, and often between the different items in it, in order to explain what the programme or the items are about.
 \textit{
	\begin{itemize}
	\item 'Health Matters' is introduced by Dick Oliver on BBC World Service.
	\end{itemize}
}
\end{enumerate}

\section*{perform}
{\large \color{blue}  performs  performing  performed  }
\subsection*{Explain}
\begin{enumerate}
\item verb \\
When you \textbf{perform} a task or action, especially a complicated one, you do it.
 \textit{
	\begin{itemize}
	\item We're looking for people of all ages who have performed outstanding acts of bravery.
	\item His council had to perform miracles on a tiny budget.
	\item Several grafts may be performed at one operation.
	\end{itemize}
}
\item verb \\
If something \textbf{performs} a particular  function , it has that function.
 \textit{
	\begin{itemize}
	\item A complex engine has many separate components, each performing a different function.
	\end{itemize}
}
\item verb \\
If you \textbf{perform} a play, a piece of music , or a dance , you do it in front of an audience.
 \textit{
	\begin{itemize}
	\item Gardiner has pursued relentlessly high standards in performing classical music.
	\item This play was first performed in 411 BC.
	\item He began performing in the early fifties, singing and playing guitar.
	\end{itemize}
}
\item verb \\
If someone or something \textbf{performs well} , they work well or achieve a good  result . If they \textbf{perform badly} , they work badly or achieve a poor result.
 \textit{
	\begin{itemize}
	\item He had not performed well in his exams.
	\item England performed so well in the match at Wembley.
	\item 'State-owned industries will always perform poorly,' John Moore informed readers.
	\item When there's snow and ice, how's this car going to perform?
	\end{itemize}
}
\end{enumerate}

\section*{invade}
{\large \color{blue}  invades  invading  invaded  }
\subsection*{Explain}
\begin{enumerate}
\item verb \\
To \textbf{invade} a country means to enter it by force with an army .
 \textit{
	\begin{itemize}
	\item In autumn 1944 the allies invaded the Italian mainland at Anzio and Salerno.
	\item The Romans and the Normans came to Britain as invading armies.
	\end{itemize}
}
\item verb \\
If you say that people or animals \textbf{invade} a place, you mean that they enter it in large numbers, often in a way that is unpleasant or difficult to deal with.
 \textit{
	\begin{itemize}
	\item People invaded the streets in victory processions almost throughout the day.
	\item Every so often the kitchen would be invaded by ants.
	\end{itemize}
}
\end{enumerate}

\section*{polish}
{\large \color{blue}  }
\subsection*{Explain}
\begin{enumerate}
\item adjective \\
\textbf{Polish}  means belonging or relating to Poland, or to its people, language, or culture .
 \textit{
	\begin{itemize}
	\item The press conference was broadcast live on Polish television.
	\item ...the new Polish government.
	\end{itemize}
}
\item uncountable noun \\
\textbf{Polish} is the language spoken in Poland.
 \textit{
	\begin{itemize}
	\end{itemize}
}
\end{enumerate}

\section*{lie}
{\large \color{blue}  lies  lying  lay  lain  }
\subsection*{Explain}
\begin{enumerate}
\item verb \\
If you \textbf{are lying}  somewhere , you are in a horizontal position and are not standing or sitting .
 \textit{
	\begin{itemize}
	\item There was a child lying on the ground.
	\item The injured man was lying motionless on his back.
	\item He lay awake watching her for a long time.
	\end{itemize}
}
\item verb \\
If an object \textbf{lies} in a particular place, it is in a flat position in that place.
 \textit{
	\begin{itemize}
	\item ...a newspaper lying on a nearby couch.
	\item Broken glass lay scattered on the carpet.
	\item ...a two-page memo lying unread on his desk.
	\end{itemize}
}
\item verb \\
If you say that a place \textbf{lies} in a particular position or direction, you mean that it is situated there.
 \textit{
	\begin{itemize}
	\item The islands lie at the southern end of the Kurile chain.
	\end{itemize}
}
\item link verb \\
You can use \textbf{lie} to say that something is or remains in a particular state or condition. For example,
if something \textbf{lies forgotten} , it has been and remains forgotten.
 \textit{
	\begin{itemize}
	\item She turned back to the Bible lying open in her lap.
	\item The picture lay hidden in the archives for over 40 years.
	\item His country's economy lies in ruins.
	\end{itemize}
}
\item verb \\
You can use \textbf{lie} to say what position a competitor or team is in during a competition .
 \textit{
	\begin{itemize}
	\item I was going well and was lying fourth.
	\item She is lying in second place.
	\end{itemize}
}
\item verb \\
You can talk about where something such as a problem , solution , or fault  \textbf{lies} to say what you think it consists of, involves, or is caused by.
 \textit{
	\begin{itemize}
	\item The problem lay in the large amounts spent on defence.
	\item He realised his future lay elsewhere.
	\item We must be clear about where the responsibility lies.
	\end{itemize}
}
\item verb \\
You use \textbf{lie} in expressions such as \textbf{lie ahead} , \textbf{lie in store} , and \textbf{lie in wait} when you are talking about what someone is going to experience in the future , especially when it is something unpleasant or difficult .
 \textit{
	\begin{itemize}
	\item She'd need all her strength and bravery to cope with what lay in store.
	\item The President's most serious challenges lie ahead.
	\end{itemize}
}
\item verb \\
\textbf{Lie} is used in formal English, especially on gravestones , to say that a dead person is buried in a particular place.
 \textit{
	\begin{itemize}
	\item The inscription reads: Here lies Catin, the son of Magarus.
	\item My father lies in the small cemetery a few miles up this road.
	\end{itemize}
}
\item verb \\
If you say that light, clouds , or fog  \textbf{lie} somewhere, you mean that they exist there or are spread over the area mentioned .
 \textit{
	\begin{itemize}
	\item It had been wet overnight, and a morning mist lay on the field.
	\end{itemize}
}
\item singular noun \\
The \textbf{lie} of an object or area is its position or the way that it is arranged .
 \textit{
	\begin{itemize}
	\item The actual site of a city is determined by the natural lie of the land.
	\end{itemize}
}
\end{enumerate}

\section*{pour}
{\large \color{blue}  pours  pouring  poured  }
\subsection*{Explain}
\begin{enumerate}
\item verb \\
If you \textbf{pour} a liquid or other substance , you make it flow steadily out of a container by holding the container at an angle .
 \textit{
	\begin{itemize}
	\item Pour a pool of sauce on two plates and arrange the meat neatly.
	\item Francis poured milk into a glass.
	\item Heat the oil in a non-stick frying-pan, then pour in the egg mixture.
	\end{itemize}
}
\item verb \\
If you \textbf{pour} someone a drink , you put some of the drink in a cup or glass so that they can drink it.
 \textit{
	\begin{itemize}
	\item He got up and poured himself another drink.
	\item She asked Tillie to pour her a cup of coffee.
	\item Quietly Mark poured and served drinks for all of them.
	\end{itemize}
}
\item verb \\
When a liquid or other substance \textbf{pours}  somewhere , for example through a hole , it flows quickly and in large quantities .
 \textit{
	\begin{itemize}
	\item Blood was pouring from his broken nose.
	\item There was dense smoke pouring from all four engines.
	\item Tears poured down both our faces.
	\item The tide poured in from the south.
	\end{itemize}
}
\item verb \\
When it rains very heavily, you can say that \textbf{it is pouring} .
 \textit{
	\begin{itemize}
	\item It has been pouring almost non stop for the past three days, disrupting normal life.
	\item It has been pouring with rain all week.
	\item The rain was pouring down.
	\item We drove all the way through pouring rain.
	\end{itemize}
}
\item verb \\
If people \textbf{pour} into or out of a place, they go there quickly and in large numbers.
 \textit{
	\begin{itemize}
	\item Any day now, the Northern forces may pour across the new border.
	\item Holidaymakers continued to pour down to the coast in search of surf and sun.
	\item At six p.m. large groups poured from the numerous offices.
	\end{itemize}
}
\item verb \\
If something such as information  \textbf{pours} into a place, a lot of it is obtained or given .
 \textit{
	\begin{itemize}
	\item Martin, 78, died yesterday. Tributes poured in from around the globe.
	\item The commission has invited interested parties to submit comments, and these are now
pouring in.
	\end{itemize}
}
\item  \\
 pour cold water on \textit{
	\begin{itemize}
	\end{itemize}
}
\end{enumerate}

\section*{listen}
{\large \color{blue}  listens  listening  listened  }
\subsection*{Explain}
\begin{enumerate}
\item verb \\
If you \textbf{listen}  \textbf{to} someone who is talking or \textbf{to} a sound, you give your attention to them or it.
 \textit{
	\begin{itemize}
	\item He spent his time listening to the radio.
	\item Sonia was not listening.
	\end{itemize}
}
\item verb \\
If you \textbf{listen}  \textbf{for} a sound, you keep  alert and are ready to hear it if it occurs.
 \textbf{Listen out}  means the same as listen .
 \textit{
	\begin{itemize}
	\item We listen for footsteps approaching.
	\item They're both asleep upstairs, but you don't mind listening just in case of trouble,
do you?
	\item I didn't really listen out for the lyrics.
	\end{itemize}
}
\item verb \\
If you \textbf{listen}  \textbf{to} someone, you do what they advise you to do, or you believe them.
 \textit{
	\begin{itemize}
	\item Anne, you need to listen to me this time.
	\item When I asked him to stop, he would not listen.
	\end{itemize}
}
\item convention \\
You say  \textbf{listen} when you want someone to pay attention to you because you are going to say something important .
 \textit{
	\begin{itemize}
	\item Listen, I finish at one.
	\end{itemize}
}
\item convention \\
You say \textbf{listen up} when you want someone to listen to what you are going to say.
 \textit{
	\begin{itemize}
	\item Okay, listen up, guys. We've got to talk a little about how you look.
	\end{itemize}
}
\item  \\
 listen here \textit{
	\begin{itemize}
	\end{itemize}
}
\end{enumerate}

\section*{praise}
{\large \color{blue}  praises  praising  praised  }
\subsection*{Explain}
\begin{enumerate}
\item verb \\
If you \textbf{praise} someone or something, you express approval for their achievements or qualities.
 \textit{
	\begin{itemize}
	\item The American president praised Turkey for its courage.
	\item Many others praised Sanford for taking a strong stand.
	\item He praised the excellent work of the U.N. weapons inspectors.
	\end{itemize}
}
\item uncountable noun \\
\textbf{Praise} is what you say or write about someone when you are praising them.
 \textit{
	\begin{itemize}
	\item All the guests are full of praise for the staff and service they received.
	\item I have nothing but praise for the police.
	\item That is high praise indeed.
	\end{itemize}
}
\item verb \\
If you \textbf{praise}  God , you express your respect , honour , and thanks to God.
 \textit{
	\begin{itemize}
	\item She asked the church to praise God.
	\end{itemize}
}
\item uncountable noun \\
\textbf{Praise} is the expression of respect, honour, and thanks to God.
 \textit{
	\begin{itemize}
	\item Hindus were singing hymns in praise of the god Rama.
	\end{itemize}
}
\item  \\
 damn with faint praise \textit{
	\begin{itemize}
	\end{itemize}
}
\item  \\
 to sing someone's praises \textit{
	\begin{itemize}
	\end{itemize}
}
\end{enumerate}

\section*{manipulate}
{\large \color{blue}  manipulates  manipulating  manipulated  }
\subsection*{Explain}
\begin{enumerate}
\item verb \\
If you say that someone \textbf{manipulates} people, you disapprove of them because they skilfully force or persuade people to do what they want .
 \textit{
	\begin{itemize}
	\item He is a very difficult character. He manipulates people.
	\item She's always borrowing my clothes and manipulating me to give her money.
	\item They have kids who manipulate them into buying toys.
	\end{itemize}
}
\item verb \\
If you say that someone \textbf{manipulates} an event or situation , you disapprove of them because they use or control it for their own benefit , or cause it to develop in the way they want.
 \textit{
	\begin{itemize}
	\item She was unable, for once, to control and manipulate events.
	\item They felt he had been cowardly in manipulating the system to avoid the draft.
	\end{itemize}
}
\item verb \\
If you \textbf{manipulate} something that requires skill, such as a complicated  piece of equipment or a difficult  idea , you operate it or process it.
 \textit{
	\begin{itemize}
	\item The technology uses a pen to manipulate a computer.
	\item The puppets are expertly manipulated by Liz Walker.
	\item Much of what I do is manipulating data from different sources.
	\end{itemize}
}
\item verb \\
If someone \textbf{manipulates} your bones or muscles , they skilfully move and press them with their hands in order to push the bones into their correct  position or make the muscles less stiff .
 \textit{
	\begin{itemize}
	\item The way he can manipulate my leg has helped my arthritis so much.
	\end{itemize}
}
\end{enumerate}

\section*{rub}
{\large \color{blue}  rubs  rubbing  rubbed  }
\subsection*{Explain}
\begin{enumerate}
\item verb \\
If you \textbf{rub} a part of your body, you move your hand or fingers  backwards and forwards over it while pressing firmly.
 \textit{
	\begin{itemize}
	\item He rubbed his arms and stiff legs.
	\item 'I fell in a ditch', he said, rubbing at a scrape on his hand.
	\end{itemize}
}
\item verb \\
If you \textbf{rub}  \textbf{against} a surface or \textbf{rub} a part of your body \textbf{against} a surface, you move it backwards and forwards while pressing it against the surface.
 \textit{
	\begin{itemize}
	\item A cat was rubbing against my leg.
	\item He kept rubbing his leg against mine.
	\end{itemize}
}
\item verb \\
If you \textbf{rub} an object or a surface, you move a cloth backward and forward over it in order to clean or dry it.
 \textit{
	\begin{itemize}
	\item She took off her glasses and rubbed them hard.
	\item He rubbed and rubbed but couldn't seem to get clean.
	\end{itemize}
}
\item verb \\
If you \textbf{rub} a substance \textbf{into} a surface or \textbf{rub} something such as dirt  \textbf{from} a surface, you spread it over the surface or remove it from the surface using your
hand or something such as a cloth.
 \textit{
	\begin{itemize}
	\item He rubbed oil into my back.
	\item I pretended to rub a fleck of grit from one eye.
	\end{itemize}
}
\item verb \\
If you \textbf{rub} two things \textbf{together} or if they \textbf{rub}  \textbf{together} , they move backwards and forwards, pressing against each other.
 \textit{
	\begin{itemize}
	\item He rubbed his hands together a few times.
	\item ...the 650-mile rift that separates the Pacific and North American geological plates
as they rub together.
	\end{itemize}
}
\item verb \\
If something you are wearing or holding  \textbf{rubs} , it makes you sore because it keeps moving backwards and forwards against your skin.
 \textit{
	\begin{itemize}
	\item Smear cream on to your baby's skin at the edges of the plaster to prevent it from
rubbing.
	\end{itemize}
}
\item singular noun \\
\textbf{Rub} is used in expressions such as \textbf{there's the rub} and \textbf{the rub is} when you are mentioning a difficulty that makes something hard or impossible to achieve .
 \textit{
	\begin{itemize}
	\item 'What do you want to write about?'. And there was the rub, because I didn't yet know.
	\end{itemize}
}
\item countable noun \\
A massage can be referred to as a \textbf{rub} .
 \textit{
	\begin{itemize}
	\item She sometimes asks if I want a back rub.
	\end{itemize}
}
\item countable noun \\
A \textbf{rub} is a substance that you massage into your skin.
 \textit{
	\begin{itemize}
	\item ...a fresh cucumber rub for your whole face.
	\end{itemize}
}
\item  \\
 to rub shoulders with \textit{
	\begin{itemize}
	\end{itemize}
}
\item  \\
 rub sb up the wrong way \textit{
	\begin{itemize}
	\end{itemize}
}
\end{enumerate}

\section*{menace}
{\large \color{blue}  menaces  menacing  menaced  }
\subsection*{Explain}
\begin{enumerate}
\item countable noun \\
If you say that someone or something is a \textbf{menace} to other people or things, you mean that person or thing is likely to cause serious  harm .
 \textit{
	\begin{itemize}
	\item In my view you are a menace to the public.
	\item ...the menace of fascism.
	\end{itemize}
}
\item countable noun \\
You can refer to someone or something as a \textbf{menace} when you want to say that they cause you trouble or annoyance .
 \textit{
	\begin{itemize}
	\item You're a menace to my privacy, Kenworthy.
	\item As I have said earlier in this book, bad shoes are a menace.
	\end{itemize}
}
\item uncountable noun \\
\textbf{Menace} is a quality or atmosphere that gives you the feeling that you are in danger or that someone wants to harm you.
 \textit{
	\begin{itemize}
	\item There is a pervading sense of menace.
	\item ...a voice full of menace.
	\end{itemize}
}
\item verb \\
If you say that one thing \textbf{menaces} another, you mean that the first thing is likely to cause the second thing serious harm.
 \textit{
	\begin{itemize}
	\item The European states retained a latent capability to menace Britain's own security.
	\end{itemize}
}
\item verb \\
If you \textbf{are menaced} by someone, they threaten to harm you.
 \textit{
	\begin{itemize}
	\item She's being menaced by her sister's latest boyfriend.
	\end{itemize}
}
\item  \\
 with menaces \textit{
	\begin{itemize}
	\end{itemize}
}
\end{enumerate}

\section*{scramble}
{\large \color{blue}  scrambles  scrambling  scrambled  }
\subsection*{Explain}
\begin{enumerate}
\item verb \\
If you \textbf{scramble} over rocks or up a hill , you move quickly over them or up it using your hands to help you.
 \textit{
	\begin{itemize}
	\item Tourists were scrambling over the rocks looking for the perfect camera angle.
	\item He scrambled up a steep bank.
	\end{itemize}
}
\item verb \\
If you \textbf{scramble} to a different place or position, you move there in a hurried, awkward way.
 \textit{
	\begin{itemize}
	\item Ann threw back the covers and scrambled out of bed.
	\item He scrambled to his feet.
	\end{itemize}
}
\item verb \\
If a number of people \textbf{scramble}  \textbf{for} something, they compete energetically with each other for it.
 \textbf{Scramble} is also a noun .
 \textit{
	\begin{itemize}
	\item More than three million fans are expected to scramble for tickets.
	\item Business is booming and foreigners are scrambling to invest.
	\item ...the scramble for jobs.
	\item ...a scramble to get a seat on the early-morning flight.
	\end{itemize}
}
\item verb \\
If you \textbf{scramble} eggs, you break them, mix them together and then heat and stir the mixture in a pan.
 \textit{
	\begin{itemize}
	\item Make the toast and scramble the eggs.
	\end{itemize}
}
\item verb \\
If a device \textbf{scrambles} a radio or telephone  message , it interferes with the sound so that the message can only be understood by someone with special  equipment .
 \textit{
	\begin{itemize}
	\item The machine scrambles messages so that the conversations cannot be intercepted.
	\end{itemize}
}
\end{enumerate}

\section*{rouse}
{\large \color{blue}  rouses  rousing  roused  }
\subsection*{Explain}
\begin{enumerate}
\item verb \\
If someone \textbf{rouses} you when you are sleeping or if you \textbf{rouse} , you wake up.
 \textit{
	\begin{itemize}
	\item Hilton roused him at eight-thirty by rapping on the door.
	\item When I put my hand on his, he stirs but doesn't quite rouse.
	\end{itemize}
}
\item verb \\
If you \textbf{rouse}  \textbf{yourself} , you stop being inactive and start doing something.
 \textit{
	\begin{itemize}
	\item She seemed to be unable to rouse herself to do anything.
	\item He roused himself from his lazy contemplation of the scene beneath him.
	\end{itemize}
}
\item verb \\
If something or someone \textbf{rouses} you, they make you very emotional or excited.
 \textit{
	\begin{itemize}
	\item He did more to rouse the crowd there than anybody else.
	\item Ben says his father was good-natured, a man not quickly roused to anger or harsh
opinions.
	\end{itemize}
}
\item verb \\
If something \textbf{rouses} a feeling in you, it causes you to have that feeling.
 \textit{
	\begin{itemize}
	\item It roused a feeling of rebellion in him.
	\item This roused my interest in politics and I went to work for the Democrats.
	\end{itemize}
}
\end{enumerate}

\section*{soar}
{\large \color{blue}  soars  soaring  soared  }
\subsection*{Explain}
\begin{enumerate}
\item verb \\
If the amount, value, level, or volume of something \textbf{soars} , it quickly increases by a great deal .
 \textit{
	\begin{itemize}
	\item Insurance claims are expected to soar.
	\item Shares soared on the stock exchange.
	\item Figures showed customer complaints had soared to record levels and profits were falling.
	\item The temperature in the south will soar into the hundreds.
	\item ...soaring unemployment.
	\end{itemize}
}
\item verb \\
If something such as a bird \textbf{soars} into the air, it goes quickly up into the air.
 \textit{
	\begin{itemize}
	\item If you're lucky, a splendid golden eagle may soar into view.
	\item Buzzards soar overhead at a great height.
	\item The two sheets of flame clashed, soaring hundreds of feet high.
	\end{itemize}
}
\item verb \\
Trees or buildings that \textbf{soar} upwards are very tall .
 \textit{
	\begin{itemize}
	\item The steeple soars skyward.
	\item ...the soaring spires of churches like St Peter's.
	\end{itemize}
}
\item verb \\
If music \textbf{soars} , it rises greatly in volume or pitch .
 \textit{
	\begin{itemize}
	\item The music soared to the rafters, carrying its listeners' hearts.
	\item His soaring voice cuts straight to the heart.
	\end{itemize}
}
\item verb \\
If your spirits  \textbf{soar} , you suddenly  start to feel very happy .
 \textit{
	\begin{itemize}
	\item For the first time in months, my spirits soared.
	\end{itemize}
}
\end{enumerate}

\section*{shrug}
{\large \color{blue}  shrugs  shrugging  shrugged  }
\subsection*{Explain}
\begin{enumerate}
\item verb \\
If you \textbf{shrug} , you raise your shoulders to show that you are not interested in something or that you do not know or care about something.
 \textbf{Shrug} is also a noun .
 \textit{
	\begin{itemize}
	\item I shrugged, as if to say, 'Why not?'
	\item The man shrugged his shoulders.
	\item 'I suppose so,' said Anna with a shrug.
	\end{itemize}
}
\end{enumerate}

\section*{subscribe}
{\large \color{blue}  subscribes  subscribing  subscribed  }
\subsection*{Explain}
\begin{enumerate}
\item verb \\
If you \textbf{subscribe to} an opinion or belief , you are one of a number of people who have this opinion or belief.
 \textit{
	\begin{itemize}
	\item I've personally never subscribed to the view that either sex is superior to the other.
	\end{itemize}
}
\item verb \\
If you \textbf{subscribe to} a service , especially  online , you agree to regularly receive it or receive information from it.
 \textit{
	\begin{itemize}
	\item Viewers must subscribe to a broadband service for £17.99 a month.
	\item Click here to subscribe.
	\end{itemize}
}
\item verb \\
If you \textbf{subscribe to} a magazine or a newspaper , you pay to receive copies of it regularly.
 \textit{
	\begin{itemize}
	\item I subscribe to New Scientist to keep abreast of advances in science.
	\end{itemize}
}
\item verb \\
If you \textbf{subscribe to} a charity or a campaign , you send money to it regularly.
 \textit{
	\begin{itemize}
	\item I subscribe to a few favourite charities.
	\end{itemize}
}
\item verb \\
If you \textbf{subscribe}  \textbf{for}  shares in a company , you apply to buy shares in that company.
 \textit{
	\begin{itemize}
	\item Employees subscribed for far more shares than were available.
	\end{itemize}
}
\end{enumerate}

\section*{stand}
{\large \color{blue}  stands  standing  stood  }
\subsection*{Explain}
\begin{enumerate}
\item verb \\
When you \textbf{are standing} , your body is upright, your legs are straight, and your weight is supported by your
feet.
 \textbf{Stand up} means the same as stand .
 \textit{
	\begin{itemize}
	\item She was standing beside my bed staring down at me.
	\item They told me to stand still and not to turn round.
	\item Overcrowding is so bad that prisoners have to sleep in shifts, while others have
to stand.
	\item We waited, standing up, for an hour.
	\item ...Mrs Fletcher, a shop assistant who has to stand up all day.
	\end{itemize}
}
\item verb \\
When someone who is sitting  \textbf{stands} , they change their position so that they are upright and on their feet.
 \textbf{Stand up} means the same as stand .
 \textit{
	\begin{itemize}
	\item Becker stood and shook hands with Ben.
	\item When I walked in, they all stood up and started clapping.
	\end{itemize}
}
\item verb \\
If you \textbf{stand}  \textbf{aside} or \textbf{stand}  \textbf{back} , you move a short distance sideways or backwards , so that you are standing in a different place.
 \textit{
	\begin{itemize}
	\item I stood aside to let her pass me.
	\item The police officers stood back. Could it be a bomb?
	\end{itemize}
}
\item verb \\
If something such as a building or a piece of furniture \textbf{stands}  somewhere , it is in that position, and is upright.
 \textit{
	\begin{itemize}
	\item The house stands alone on top of a small hill.
	\item I reached for the lamp, which stood in the middle of the table.
	\end{itemize}
}
\item verb \\
You can say that a building \textbf{is standing} when it remains after other buildings around it have fallen down or been destroyed.
 \textit{
	\begin{itemize}
	\item The palace, which was damaged by bombs in World War II, still stood.
	\item There are very few buildings left standing.
	\end{itemize}
}
\item verb \\
If you \textbf{stand} something somewhere, you put it there in an upright position.
 \textit{
	\begin{itemize}
	\item Stand the plant in the open in a sunny, sheltered place.
	\end{itemize}
}
\item verb \\
If you leave food or a mixture of something \textbf{to}  \textbf{stand} , you leave it without disturbing it for some time.
 \textit{
	\begin{itemize}
	\item The salad improves if made in advance and left to stand.
	\end{itemize}
}
\item countable noun \\
If you take or make a \textbf{stand} , you do something or say something in order to make it clear what your attitude to
a particular thing is.
 \textit{
	\begin{itemize}
	\item He felt the need to make a stand against racism.
	\item They must take a stand and cast their votes.
	\item His tough stand won some grudging admiration.
	\end{itemize}
}
\item verb \\
If you ask someone \textbf{where} or \textbf{how} they \textbf{stand}  \textbf{on} a particular issue, you are asking them what their attitude or view is.
 \textit{
	\begin{itemize}
	\item The amendment will force senators to show where they stand on the issue of sexual
harassment.
	\item So far, the bishop hasn't said where he stands.
	\end{itemize}
}
\item verb \\
If you do not know  \textbf{where} you \textbf{stand}  \textbf{with} someone, you do not know exactly what their attitude to you is.
 \textit{
	\begin{itemize}
	\item No-one knows where they stand with him; he is utterly unpredictable.
	\item All children need discipline, to know where they stand.
	\end{itemize}
}
\item link verb \\
You can use \textbf{stand}  instead of 'be' when you are describing the present state or condition of something or someone.
 \textit{
	\begin{itemize}
	\item The alliance stands ready to do what is necessary.
	\item He stands accused of destroying the party in pursuit of his presidential ambitions.
	\item The peace plan as it stands violates basic human rights.
	\end{itemize}
}
\item verb \\
If a decision, law, or offer \textbf{stands} , it still exists and has not been changed or cancelled .
 \textit{
	\begin{itemize}
	\item Although exceptions could be made, the rule still stands.
	\item The Supreme Court says that the convictions can stand.
	\end{itemize}
}
\item verb \\
If something that can be measured \textbf{stands at} a particular level, it is at that level.
 \textit{
	\begin{itemize}
	\item The inflation rate now stands at 3.6 per cent.
	\item Support for the two sides is standing at between 42 and 44 per cent.
	\end{itemize}
}
\item verb \\
You can describe how tall or high someone or something is by saying that they \textbf{stand} a particular height.
 \textit{
	\begin{itemize}
	\item She stood five feet five inches tall and weighed 120 pounds.
	\item The dam will stand 600 feet high.
	\item She stood tall and aloof.
	\end{itemize}
}
\item verb \\
If something can \textbf{stand} a situation or a test, it is good enough or strong enough to experience it without
being damaged, harmed , or shown to be inadequate .
 \textit{
	\begin{itemize}
	\item These are the first machines that can stand the wear and tear of continuously crushing
glass.
	\item I think these books can stand comparison quite happily with works by Dickens.
	\item Ancient wisdom has stood the test of time.
	\end{itemize}
}
\item verb \\
If you cannot \textbf{stand} something, you cannot bear it or tolerate it.
 \textit{
	\begin{itemize}
	\item I can't stand any more. I'm going to run away.
	\item Stoddart can stand any amount of personal criticism.
	\item How does he stand the pain?
	\end{itemize}
}
\item verb \\
If you cannot \textbf{stand} someone or something, you dislike them very strongly.
 \textit{
	\begin{itemize}
	\item I can't stand that man and his arrogance.
	\item He can't stand smoking.
	\end{itemize}
}
\item verb \\
If you \textbf{stand to gain} something, you are likely to gain it. If you \textbf{stand to lose} something, you are likely to lose it.
 \textit{
	\begin{itemize}
	\item The management group would stand to gain millions of dollars if the company were
sold.
	\item As many as 30,000 workers at 22 nuclear weapons sites stand to lose their jobs.
	\end{itemize}
}
\item verb \\
If you \textbf{stand}  \textbf{in} an election , you are a candidate in it.
 \textit{
	\begin{itemize}
	\item He has not yet announced whether he will stand in the election.
	\item Some ardent supporters were urging him to stand.
	\item She is to stand as a Member of the European Parliament.
	\end{itemize}
}
\item verb \\
If you \textbf{stand} someone a meal or a drink, you buy it for them.
 \textit{
	\begin{itemize}
	\item You can stand me a pint.
	\end{itemize}
}
\item countable noun \\
A \textbf{stand} is a small shop or stall, outdoors or in a large public building.
 \textit{
	\begin{itemize}
	\item He ran a newspaper stand outside the American Express office.
	\item She bought a hot dog from a stand on a street corner.
	\end{itemize}
}
\item countable noun \\
A \textbf{stand} at a sports ground is a large structure where people sit or stand to watch what is happening .
 In American English, \textbf{stands} is used with same meaning.
 \textit{
	\begin{itemize}
	\item The people in the stands are cheering with all their might.
	\end{itemize}
}
\item countable noun \\
A \textbf{stand} is an object or piece of furniture that is designed for supporting or holding a particular
kind of thing.
 \textit{
	\begin{itemize}
	\item The teapot came with a stand to catch the drips.
	\end{itemize}
}
\item countable noun \\
A \textbf{stand} is an area where taxis or buses can wait to pick up passengers.
 \textit{
	\begin{itemize}
	\item Luckily there was a taxi stand nearby.
	\end{itemize}
}
\item singular noun \\
In a law court, \textbf{the stand} is the place where a witness stands to answer questions.
 \textit{
	\begin{itemize}
	\item When the father took the stand today, he contradicted his son's testimony.
	\item The government has called nearly 50 witnesses to the stand.
	\end{itemize}
}
\item  \\
 to stand or fall \textit{
	\begin{itemize}
	\end{itemize}
}
\item  \\
 someone's last stand \textit{
	\begin{itemize}
	\end{itemize}
}
\item  \\
 it stands to reason \textit{
	\begin{itemize}
	\end{itemize}
}
\item  \\
 stand in the way of sth/sb \textit{
	\begin{itemize}
	\end{itemize}
}
\end{enumerate}

\section*{suppress}
{\large \color{blue}  suppresses  suppressing  suppressed  }
\subsection*{Explain}
\begin{enumerate}
\item verb \\
If someone in authority  \textbf{suppresses} an activity, they prevent it from continuing , by using force or making it illegal .
 \textit{
	\begin{itemize}
	\item Maritime security patrols protect busy trade routes and suppress illegal activity.
	\item ...nationwide demonstrations for democracy, suppressed after 7 weeks by the army.
	\end{itemize}
}
\item verb \\
If a natural function or reaction of your body \textbf{is suppressed} , it is stopped, for example by drugs or illness .
 \textit{
	\begin{itemize}
	\item The reproduction and growth of cancerous cells can be suppressed by radiation.
	\item ...evidence that ultraviolet light can suppress human immune responses.
	\end{itemize}
}
\item verb \\
If you \textbf{suppress} your feelings or reactions, you do not express them, even though you might  want to.
 \textit{
	\begin{itemize}
	\item Liz thought of Barry and suppressed a smile.
	\item The Professor said that deep sleep allowed suppressed anxieties to surface.
	\end{itemize}
}
\item verb \\
If someone \textbf{suppresses} a piece of information , they prevent other people from learning it.
 \textit{
	\begin{itemize}
	\item At no time did they try to persuade me to suppress the information.
	\item The wrong criminal is in the dock because evidence has been suppressed.
	\end{itemize}
}
\item verb \\
If someone or something \textbf{suppresses} a process or activity, they stop it continuing or developing .
 \textit{
	\begin{itemize}
	\item Technology helps to suppress inflation by boosting efficiency and lowering costs.
	\item Lawyers said today's ruling would suppress innovation of new products.
	\end{itemize}
}
\end{enumerate}

\section*{stay}
{\large \color{blue}  stays  staying  stayed  }
\subsection*{Explain}
\begin{enumerate}
\item verb \\
If you \textbf{stay} where you are, you continue to be there and do not leave.
 \textit{
	\begin{itemize}
	\item 'Stay here,' Trish said. 'I'll bring the car down the drive to take you back.'.
	\item In the old days the woman stayed at home and the man earned the money.
	\end{itemize}
}
\item verb \\
If you \textbf{stay} in a town, or hotel , or at someone's house, you live there for a short time.
 \textbf{Stay} is also a noun .
 \textit{
	\begin{itemize}
	\item Gordon stayed at The Park Hotel, Milan.
	\item He tried to stay a few months every year in Scotland.
	\item An experienced Indian guide is provided during your stay.
	\end{itemize}
}
\item link verb \\
If someone or something \textbf{stays} in a particular state or situation , they continue to be in it.
 \textit{
	\begin{itemize}
	\item The Republican candidate said he would 'work like crazy to stay ahead'.
	\item ...community care networks that offer classes on how to stay healthy.
	\item Nothing stays the same for long.
	\end{itemize}
}
\item verb \\
If you \textbf{stay}  \textbf{away from} a place, you do not go there.
 \textit{
	\begin{itemize}
	\item Government employers and officers also stayed away from work during the strike.
	\item Every single employee turned up at the meeting, even people who usually stayed away.
	\end{itemize}
}
\item verb \\
If you \textbf{stay out of} something, you do not get involved in it.
 \textit{
	\begin{itemize}
	\item In the past, the U.N. has stayed out of the internal affairs of countries unless
invited in.
	\item After months of staying well out of the problem, Washington has expressed a willingness
to help find a solution.
	\end{itemize}
}
\item  \\
 here to stay \textit{
	\begin{itemize}
	\end{itemize}
}
\item  \\
 stay put \textit{
	\begin{itemize}
	\end{itemize}
}
\item  \\
 stay the night \textit{
	\begin{itemize}
	\end{itemize}
}
\end{enumerate}

\section*{terminate}
{\large \color{blue}  terminates  terminating  terminated  }
\subsection*{Explain}
\begin{enumerate}
\item verb \\
When you \textbf{terminate} something or when it \textbf{terminates} , it ends completely.
 \textit{
	\begin{itemize}
	\item Her next remark abruptly terminated the conversation.
	\item His contract terminates at the end of the season.
	\end{itemize}
}
\item verb \\
To \textbf{terminate} a pregnancy  means to end it.
 \textit{
	\begin{itemize}
	\item After a lot of agonizing, she decided to terminate the pregnancy.
	\end{itemize}
}
\item verb \\
When a train or bus  \textbf{terminates}  somewhere , it ends its journey there.
 \textit{
	\begin{itemize}
	\item This train will terminate at Taunton.
	\end{itemize}
}
\end{enumerate}

\section*{sustain}
{\large \color{blue}  sustains  sustaining  sustained  }
\subsection*{Explain}
\begin{enumerate}
\item verb \\
If you \textbf{sustain} something, you continue it or maintain it for a period of time.
 \textit{
	\begin{itemize}
	\item But he has sustained his fierce social conscience from young adulthood through old
age.
	\item The parameters within which life can be sustained on Earth are extraordinarily narrow.
	\item ...a period of sustained economic growth throughout 1995.
	\end{itemize}
}
\item verb \\
If you \textbf{sustain} something such as a defeat , loss, or injury, it happens to you.
 \textit{
	\begin{itemize}
	\item Every aircraft in there has sustained some damage.
	\item A tourist died of injuries sustained in the bomb blast.
	\end{itemize}
}
\item verb \\
If something \textbf{sustains} you, it supports you by giving you help , strength , or encouragement.
 \textit{
	\begin{itemize}
	\item The cash dividends they get from the cash crop would sustain them during the lean
season.
	\item I am sustained by letters of support and what people say to me in ordinary daily
life.
	\item Sustained by this wonderful breakfast, we boarded our plane.
	\end{itemize}
}
\end{enumerate}

\section*{translate}
{\large \color{blue}  translates  translating  translated  }
\subsection*{Explain}
\begin{enumerate}
\item verb \\
If something that someone has said or written \textbf{is translated}  \textbf{from} one language \textbf{into} another, it is said or written again in the second language.
 \textit{
	\begin{itemize}
	\item Only a small number of her books have been translated into English.
	\item Martin Luther translated the Bible into German.
	\item The Celtic word 'geis' is usually translated as 'taboo'.
	\item The girls waited for Mr Esch to translate.
	\item ...Mr Mani by Yehoshua, translated from Hebrew by Hillel Halkin.
	\end{itemize}
}
\item verb \\
If a name, a word, or an expression  \textbf{translates}  \textbf{as} something in a different language, that is what it means in that language.
 \textit{
	\begin{itemize}
	\item His family's Cantonese nickname for him translates as Never Sits Still.
	\end{itemize}
}
\item verb \\
If one thing \textbf{translates} or \textbf{is translated}  \textbf{into} another, the second happens or is done as a result of the first.
 \textit{
	\begin{itemize}
	\item Reforming Warsaw's stagnant economy requires harsh measures that would translate
into job losses.
	\item Your decision must be translated into specific, concrete actions.
	\end{itemize}
}
\item verb \\
If you say that a remark , a gesture, or an action \textbf{translates}  \textbf{as} something, or that you \textbf{translate} it \textbf{as} something, you decide that this is what its significance is.
 \textit{
	\begin{itemize}
	\item 'I love him' often translates as 'He's better than nothing'.
	\item Your body translates this physical sensation as the onset of panic.
	\end{itemize}
}
\end{enumerate}

\section*{throw}
{\large \color{blue}  throws  throwing  threw  thrown  }
\subsection*{Explain}
\begin{enumerate}
\item verb \\
When you \textbf{throw} an object that you are holding, you move your hand or arm quickly and let go of the object, so that it moves through the air.
 \textbf{Throw} is also a noun .
 \textit{
	\begin{itemize}
	\item He spent hours throwing a tennis ball against a wall.
	\item On one occasion, his father threw a radio at his mother.
	\item The crowd began throwing stones.
	\item Sophia jumps up and throws down her knitting.
	\item He threw Brian a rope.
	\item One of the judges thought it was a foul throw.
	\item A throw of the dice allows a player to move himself forward.
	\end{itemize}
}
\item verb \\
If you \textbf{throw} your body or part of your body into a particular position or place, you move it there
suddenly and with a lot of force.
 \textit{
	\begin{itemize}
	\item She threw her arms around his shoulders.
	\item She threatened to throw herself in front of a train.
	\item He set his skinny legs apart and threw back his shoulders.
	\end{itemize}
}
\item verb \\
If you \textbf{throw} something into a particular place or position, you put it there in a quick and careless way.
 \textit{
	\begin{itemize}
	\item He struggled out of his bulky jacket and threw it on to the back seat.
	\item Why not throw it all in the pot and see what happens?
	\end{itemize}
}
\item verb \\
To \textbf{throw} someone into a particular place or position means to force them roughly into that place or position.
 \textit{
	\begin{itemize}
	\item He threw me to the ground and started to kick.
	\item The device exploded, throwing Mr Taylor from his car.
	\end{itemize}
}
\item verb \\
If you say that someone \textbf{is thrown}  \textbf{into}  prison , you mean that they are put there by the authorities, especially if this seems  unfair or cruel .
 \textit{
	\begin{itemize}
	\item Those two should have been thrown in jail.
	\item They will throw us into prison on some pretext.
	\end{itemize}
}
\item verb \\
If a horse \textbf{throws} its rider , it makes him or her fall off, by suddenly jumping or moving violently.
 \textit{
	\begin{itemize}
	\item The horse reared, throwing its rider and knocking down a youth standing beside it.
	\end{itemize}
}
\item verb \\
If a person or thing \textbf{is thrown}  \textbf{into} an unpleasant situation or state, something causes them to be in that situation or state.
 \textit{
	\begin{itemize}
	\item Abidjan was thrown into turmoil because of a protest by taxi drivers.
	\item Economic recession had thrown millions out of work.
	\item The border dispute has thrown next week's meeting into confusion.
	\end{itemize}
}
\item verb \\
If something \textbf{throws} light or a shadow \textbf{on} a surface, it causes that surface to have light or a shadow on it.
 \textit{
	\begin{itemize}
	\item The sunlight is white and blinding, throwing hard-edged shadows on the ground.
	\end{itemize}
}
\item verb \\
If something \textbf{throws}  doubt  \textbf{on} a person or thing, it causes people to doubt or suspect them.
 \textit{
	\begin{itemize}
	\item This new information does throw doubt on their choice.
	\item She did not attempt to throw any suspicion upon you.
	\end{itemize}
}
\item verb \\
If you \textbf{throw} a look or smile at someone or something, you look or smile at them quickly and suddenly.
 \textit{
	\begin{itemize}
	\item Emily turned and threw her a suggestive grin.
	\end{itemize}
}
\item verb \\
If you \textbf{throw} yourself, your energy, or your money \textbf{into} a particular job or activity, you become involved in it very actively or enthusiastically.
 \textit{
	\begin{itemize}
	\item She threw herself into a modelling career.
	\item They threw all their military resources into the battle.
	\end{itemize}
}
\item verb \\
If you \textbf{throw} a fit or a tantrum , you suddenly start to behave in an uncontrolled way.
 \textit{
	\begin{itemize}
	\item I used to get very upset and scream and swear, throwing tantrums all over the place.
	\end{itemize}
}
\item verb \\
If something such as a remark or an experience \textbf{throws} you, it surprises you or confuses you because it is unexpected .
 \textbf{Throw off} means the same as throw .
 \textit{
	\begin{itemize}
	\item The professor rather threw me by asking if I went in for martial arts.
	\item Obviously the puncture threw me a little, but I'm reasonably happy.
	\item I lost my first serve in the first set, it threw me off a bit.
	\end{itemize}
}
\item verb \\
If you \textbf{throw} a punch, you punch someone.
 \textit{
	\begin{itemize}
	\item Everything was fine until someone threw a punch.
	\end{itemize}
}
\item verb \\
When someone \textbf{throws} a party, they organize one, usually in their own home.
 \textit{
	\begin{itemize}
	\item Why not throw a party for your friends?
	\end{itemize}
}
\item verb \\
When someone \textbf{throws} a switch, they turn it on or off.
 \textit{
	\begin{itemize}
	\item The Prince threw the switch to light the illuminations.
	\end{itemize}
}
\item verb \\
In sports, if a player \textbf{throws} a game or contest, they lose it as a result of a deliberate action or intention .
 \textit{
	\begin{itemize}
	\item ...offering him a bribe to throw the game.
	\end{itemize}
}
\item countable noun \\
A \textbf{throw} is a light rug , blanket, or cover for a sofa or bed .
 \textit{
	\begin{itemize}
	\end{itemize}
}
\item  \\
 a throw \textit{
	\begin{itemize}
	\end{itemize}
}
\item  \\
 throw oneself at sb \textit{
	\begin{itemize}
	\end{itemize}
}
\end{enumerate}

\section*{twinkle}
{\large \color{blue}  twinkles  twinkling  twinkled  }
\subsection*{Explain}
\begin{enumerate}
\item verb \\
If a star or a light \textbf{twinkles} , it shines with an unsteady light which rapidly and constantly changes from bright to faint .
 \textit{
	\begin{itemize}
	\item At night, lights twinkle in distant villages across the valleys.
	\item ...a band of twinkling diamonds.
	\end{itemize}
}
\item verb \\
If you say that someone's eyes \textbf{twinkle} , you mean that their face  expresses  good  humour or amusement.
 \textbf{Twinkle} is also a noun .
 \textit{
	\begin{itemize}
	\item She saw her mother's eyes twinkle with amusement.
	\item A kindly twinkle came into her eyes.
	\end{itemize}
}
\end{enumerate}

\section*{tow}
{\large \color{blue}  tows  towing  towed  }
\subsection*{Explain}
\begin{enumerate}
\item verb \\
If one vehicle \textbf{tows} another, it pulls it along behind it.
 \textbf{Tow} is also a noun .
 \textit{
	\begin{itemize}
	\item He had been using the vehicle to tow his work trailer.
	\item They threatened to tow away my car.
	\item A lifeboat towed the 28ft boat to a nearby quay.
	\item I can give you a tow if you want.
	\end{itemize}
}
\item  \\
 in tow \textit{
	\begin{itemize}
	\end{itemize}
}
\end{enumerate}

\section*{understand}
{\large \color{blue}  understands  understanding  understood  }
\subsection*{Explain}
\begin{enumerate}
\item verb \\
If you \textbf{understand} someone or \textbf{understand} what they are saying , you know what they mean .
 \textit{
	\begin{itemize}
	\item I think you heard and also understand me.
	\item Rusty nodded as though she understood the old woman.
	\item I don't understand what you are talking about.
	\item He was speaking poor English, trying to make himself understood.
	\end{itemize}
}
\item verb \\
If you \textbf{understand} a language, you know what someone is saying when they are speaking that language.
 \textit{
	\begin{itemize}
	\item I couldn't read or understand a word of Yiddish, so I asked him to translate.
	\end{itemize}
}
\item verb \\
To \textbf{understand} someone means to know how they feel and why they behave in the way that they do.
 \textit{
	\begin{itemize}
	\item It would be nice to have someone who really understood me, a friend.
	\item Trish had not exactly understood his feelings.
	\item She understands why I get tired and grumpy.
	\end{itemize}
}
\item verb \\
You say that you \textbf{understand} something when you know why or how it happens .
 \textit{
	\begin{itemize}
	\item They are too young to understand what is going on.
	\item She didn't understand why the TV was kept out of reach of the patients.
	\item It is worth making the effort to understand how investment trusts work.
	\end{itemize}
}
\item verb \\
If you \textbf{understand} that something is the case , you think it is true because you have heard or read that it is. You can say that something \textbf{is understood} to be the case to mean that people generally think it is true.
 \textit{
	\begin{itemize}
	\item We understand that she's in the studio recording her second album.
	\item I understand you've heard about David.
	\item The idea, as I understand it, is to make science more relevant.
	\item The management is understood to be very unwilling to agree to this request.
	\item It is understood that the veteran reporter had a heart attack.
	\end{itemize}
}
\item  \\
 give someone to understand \textit{
	\begin{itemize}
	\end{itemize}
}
\item  \\
 do you understand/is that understood \textit{
	\begin{itemize}
	\end{itemize}
}
\end{enumerate}

\section*{transfer}
{\large \color{blue}  transfers  transferring  transferred  }
\subsection*{Explain}
\begin{enumerate}
\item verb \\
If you \textbf{transfer} something or someone \textbf{from} one place \textbf{to} another, or they \textbf{transfer}  \textbf{from} one place \textbf{to} another, they go from the first place to the second.
 \textbf{Transfer} is also a noun .
 \textit{
	\begin{itemize}
	\item Remove the wafers with a spoon and transfer them to a plate.
	\item He was transferred from Weston Hospital to Frenchay.
	\item He wants to transfer some money to the account of his daughter.
	\item The person can transfer from wheelchair to seat with relative ease.
	\item Arrange for the transfer of medical records to your new doctor.
	\item The bank reserves the right to reverse any transfers or payments.
	\end{itemize}
}
\item verb \\
If something \textbf{is transferred} , or \textbf{transfers} , \textbf{from} one person or group of people \textbf{to} another, the second person or group gets it instead of the first.
 \textbf{Transfer} is also a noun.
 \textit{
	\begin{itemize}
	\item I realized she'd transferred all her love from me to you.
	\item The chances of the disease being transferred to humans is extremely remote.
	\item On 1 December the presidency of the Security Council automatically transfers from
the U.S. to Yemen.
	\item ...the transfer of power from the old to the new regimes.
	\end{itemize}
}
\item variable noun \\
Technology  \textbf{transfer} is the process or act by which a country or organization which has developed new
technology enables another country or organization to use the technology.
 \textit{
	\begin{itemize}
	\item These countries need capital and technology transfer.
	\item If the transfer of technology is potentially beneficial to developing countries,
then it is appropriate to consider its cost.
	\end{itemize}
}
\item verb \\
In professional sports, if a player \textbf{transfers} or \textbf{is transferred} from one club to another, they leave one club and begin playing for another.
 \textbf{Transfer} is also a noun.
 \textit{
	\begin{itemize}
	\item He transferred from Spurs to Middlesbrough.
	\item He was transferred from Crystal Palace to Arsenal.
	\item Nobody was expecting his transfer to the Italian club.
	\end{itemize}
}
\item verb \\
If you \textbf{are transferred} , or if you \textbf{transfer} , \textbf{to} a different job or place, you move to a different job or start  working in a different place.
 \textbf{Transfer} is also a noun.
 \textit{
	\begin{itemize}
	\item I was transferred to the book department.
	\item I suspect that she is going to be transferred.
	\item Anton was able to transfer from Lavine's to an American company.
	\item They will be offered transfers to other locations.
	\end{itemize}
}
\item verb \\
When information \textbf{is transferred}  \textbf{onto} a different medium , it is copied from one medium to another.
 \textbf{Transfer} is also a noun.
 \textit{
	\begin{itemize}
	\item Such information is easily transferred onto microfilm.
	\item ...systems to create film-quality computer effects and then transfer them to film.
	\item He has been charged with unauthorised transfer of information from military computers.
	\item ...data transfer.
	\end{itemize}
}
\item verb \\
When property or land \textbf{is transferred} , it stops being owned by one person or institution and becomes owned by another.
 \textbf{Transfer} is also a noun.
 \textit{
	\begin{itemize}
	\item He has already transferred ownership of most of the works to a British foundation.
	\item Certain kinds of property are transferred automatically at death.
	\item ...an outright transfer of property.
	\end{itemize}
}
\item verb \\
If you \textbf{transfer} or \textbf{are transferred} when you are on a journey , you change from one vehicle to another.
 \textit{
	\begin{itemize}
	\item He likes to transfer from the bus to the Blue Line at 103rd Street in Watts.
	\item 1,654 passengers were transferred at sea to a Norwegian cruise ship.
	\end{itemize}
}
\item countable noun \\
\textbf{Transfers} are pieces of paper with a design on one side. The design can be transferred by heat
or pressure onto material, paper, or china for decoration .
 \textit{
	\begin{itemize}
	\item ...gold letter transfers.
	\end{itemize}
}
\item countable noun \\
A \textbf{transfer} is a ticket that you get when you leave a bus or train that allows you to go on a
different bus or train without paying again.
 \textit{
	\begin{itemize}
	\end{itemize}
}
\end{enumerate}

\section*{wake}
{\large \color{blue}  wakes  waking  woke  woken  }
\subsection*{Explain}
\begin{enumerate}
\item verb \\
When you \textbf{wake} or when someone or something \textbf{wakes} you, you become conscious again after being asleep .
 \textbf{Wake up}  means the same as wake .
 \textit{
	\begin{itemize}
	\item It was cold and dark when I woke at 6.30.
	\item Bob woke slowly to sunshine pouring in his window.
	\item She woke to find her dark room lit by flashing lights.
	\item She went upstairs to wake Milton.
	\item One morning I woke up and felt something was wrong.
	\item At dawn I woke him up and said we were leaving.
	\end{itemize}
}
\item countable noun \\
The \textbf{wake} of a boat or other object moving in water is the track of waves that it makes behind it as
it moves through the water.
 \textit{
	\begin{itemize}
	\item The ride was smooth until they got into the merchant ship's wake.
	\item Dolphins sometimes play in the wake of the boats.
	\end{itemize}
}
\item countable noun \\
A \textbf{wake} is a gathering or social  event that is held before or after someone's funeral.
 \textit{
	\begin{itemize}
	\item A funeral wake was in progress.
	\end{itemize}
}
\item phrase \\
If one thing follows  \textbf{in the wake of} another, it happens after the other thing is over, often as a result of it.
 \textit{
	\begin{itemize}
	\item The governor has enjoyed a huge surge in the polls in the wake of last week's convention.
	\item The company is in bankruptcy proceedings in the wake of a strike that began last
spring.
	\end{itemize}
}
\item  \\
 waking hours \textit{
	\begin{itemize}
	\end{itemize}
}
\item  \\
 in sb's wake \textit{
	\begin{itemize}
	\end{itemize}
}
\item  \\
 in sb's wake \textit{
	\begin{itemize}
	\end{itemize}
}
\end{enumerate}

\section*{unfold}
{\large \color{blue}  unfolds  unfolding  unfolded  }
\subsection*{Explain}
\begin{enumerate}
\item verb \\
If a situation  \textbf{unfolds} , it develops and becomes known or understood .
 \textit{
	\begin{itemize}
	\item The outcome depends on conditions as well as how events unfold.
	\item The facts started to unfold before them.
	\end{itemize}
}
\item verb \\
If a story  \textbf{unfolds} or if someone \textbf{unfolds} it, it is told to someone else.
 \textit{
	\begin{itemize}
	\item Don's story unfolded as the cruise got under way.
	\item Mr Wills unfolds his story with evident enjoyment.
	\end{itemize}
}
\item verb \\
If someone \textbf{unfolds} something which has been folded or if it \textbf{unfolds} , it is opened out and becomes flat .
 \textit{
	\begin{itemize}
	\item He quickly unfolded the blankets and spread them on the mattress.
	\item When the bird lifts off into flight, its wings unfold to an impressive six-foot span.
	\end{itemize}
}
\end{enumerate}

\section*{waken}
{\large \color{blue}  wakens  wakening  wakened  }
\subsection*{Explain}
\begin{enumerate}
\item verb \\
When you \textbf{waken} , or when someone or something \textbf{wakens} you, you wake from sleep.
 \textbf{Waken up} means the same as waken .
 \textit{
	\begin{itemize}
	\item The noise of a door slamming wakened her.
	\item Women are much more likely than men to waken because of noise.
	\item 'Drink this coffee–it will waken you up.'
	\item If you do waken up during the night, start the exercises again.
	\end{itemize}
}
\end{enumerate}

\section*{uphold}
{\large \color{blue}  upholds  upholding  upheld  }
\subsection*{Explain}
\begin{enumerate}
\item verb \\
If you \textbf{uphold} something such as a law , a principle , or a decision , you support and maintain it.
 \textit{
	\begin{itemize}
	\item Our policy has been to uphold the law.
	\item It is the responsibility of every government to uphold certain basic principles.
	\item ...upholding the artist's right to creative freedom.
	\end{itemize}
}
\item verb \\
If a court of law \textbf{upholds} a legal decision that has already been made, it decides that it was the correct decision.
 \textit{
	\begin{itemize}
	\item The crown court, however, upheld the magistrate's decision.
	\end{itemize}
}
\end{enumerate}

\section*{zoom}
{\large \color{blue}  zooms  zooming  zoomed  }
\subsection*{Explain}
\begin{enumerate}
\item verb \\
If you \textbf{zoom}  somewhere , you go there very quickly.
 \textit{
	\begin{itemize}
	\item We zoomed through the gallery.
	\item A police car zoomed by very close to them.
	\end{itemize}
}
\item verb \\
If prices or sales  \textbf{zoom} , they increase greatly in a very short time.
 \textit{
	\begin{itemize}
	\item The economy shrank and inflation zoomed.
	\item Profits zoomed from nil five years ago to about 16 million last year.
	\end{itemize}
}
\item countable noun \\
A \textbf{zoom} is the same as a zoom lens .
 \textit{
	\begin{itemize}
	\end{itemize}
}
\end{enumerate}

\section*{agree}
{\large \color{blue}  agrees  agreeing  agreed  }
\subsection*{Explain}
\begin{enumerate}
\item verb \\
If people \textbf{agree}  \textbf{with} each other about something, they have the same opinion about it or say that they have the same opinion.
 \textit{
	\begin{itemize}
	\item If we agreed all the time it would be a bit boring, wouldn't it?
	\item Both have agreed on the need for the money.
	\item So we both agree there's a problem?
	\item I see your point but I'm not sure I agree with you.
	\item I agree with you that the open system is by far the best.
	\item 'It's appalling.'—'It is. I agree.'
	\item I agree that the demise of London zoo would be terrible.
	\item I agree with every word you've just said.
	\item 'Frankly I found it rather frightening.' 'A little startling,' Mark agreed.
	\end{itemize}
}
\item verb \\
If you \textbf{agree}  \textbf{to} do something, you say that you will do it. If you \textbf{agree}  \textbf{to} a proposal , you accept it.
 \textit{
	\begin{itemize}
	\item He agreed to pay me for the drawings.
	\item Donna agreed to both requests.
	\item All 100 senators agree to a postponement.
	\end{itemize}
}
\item verb \\
If people \textbf{agree}  \textbf{on} something, or in British English if they \textbf{agree} something, they all decide to accept or do something.
 \textit{
	\begin{itemize}
	\item The warring sides have agreed on an unconditional ceasefire.
	\item We never agreed a date.
	\item The court had given the unions until September to agree terms with a buyer.
	\end{itemize}
}
\item  \\
 agree to disagree \textit{
	\begin{itemize}
	\end{itemize}
}
\item verb \\
If you \textbf{agree}  \textbf{with} an action or suggestion , you approve of it.
 \textit{
	\begin{itemize}
	\item I don't agree with what they're doing.
	\item In his heart he knew they'd agree with his stand.
	\end{itemize}
}
\item verb \\
If one account of an event or one set of figures  \textbf{agrees}  \textbf{with} another, the two accounts or sets of figures are the same or are consistent with
each other.
 \textit{
	\begin{itemize}
	\item His second statement agrees with facts as stated by the other witnesses.
	\end{itemize}
}
\item verb \\
If some food that you eat  \textbf{does not agree with} you, it makes you feel  ill .
 \textit{
	\begin{itemize}
	\item I don't think the food here agrees with me.
	\end{itemize}
}
\item verb \\
If a place or experience  \textbf{agrees with} you, it makes you feel healthy and happy .
 \textit{
	\begin{itemize}
	\item You look great, Brian. The Bahamas certainly agree with you.
	\end{itemize}
}
\item verb \\
In grammar , if a word \textbf{agrees}  \textbf{with} a noun or pronoun , it has a form that is appropriate to the number or gender of the noun or pronoun. For example , in 'He hates it', the singular  verb agrees with the singular pronoun 'he'.
 \textit{
	\begin{itemize}
	\end{itemize}
}
\end{enumerate}

\section*{abound}
{\large \color{blue}  abounds  abounding  abounded  }
\subsection*{Explain}
\begin{enumerate}
\item verb \\
If things \textbf{abound} , or if a place \textbf{abounds}  \textbf{with} things, there are very large numbers of them.
 \textit{
	\begin{itemize}
	\item Stories abound about when he was in charge.
	\item Venice abounds in famous hotels.
	\item The book abounds with close-up images from space.
	\end{itemize}
}
\end{enumerate}

\section*{appeal}
{\large \color{blue}  appeals  appealing  appealed  }
\subsection*{Explain}
\begin{enumerate}
\item verb \\
If you \textbf{appeal}  \textbf{to} someone \textbf{to} do something, you make a serious and urgent request to them.
 \textit{
	\begin{itemize}
	\item The Prime Minister appealed to young people to use their vote.
	\item He will appeal to the state for an extension of unemployment benefits.
	\item The United Nations has appealed for help from the international community.
	\end{itemize}
}
\item countable noun \\
An \textbf{appeal} is a serious and urgent request.
 \textit{
	\begin{itemize}
	\item His main message was an appeal for unity in the face of the great weather challenge.
	\item Romania's government issued a last-minute appeal to him to call off his trip.
	\end{itemize}
}
\item countable noun \\
An \textbf{appeal} is an attempt to raise money for a charity or for a good cause.
 \textit{
	\begin{itemize}
	\item ...an appeal to save a library containing priceless manuscripts.
	\item This is not another appeal for famine relief.
	\end{itemize}
}
\item verb \\
If you \textbf{appeal}  \textbf{to} someone in authority against a decision, you formally ask them to change it. In British
English, you \textbf{appeal against} something. In American English, you \textbf{appeal} something.
 \textit{
	\begin{itemize}
	\item He said they would appeal against the decision.
	\item We intend to appeal the verdict.
	\item Maguire has appealed to the Supreme Court to stop her extradition.
	\end{itemize}
}
\item variable noun \\
An \textbf{appeal} is a formal request for a decision to be changed.
 \textit{
	\begin{itemize}
	\item Heath's appeal against the sentence was later successful.
	\item The jury agreed with her, but she lost the case on appeal.
	\end{itemize}
}
\item verb \\
If something \textbf{appeals}  \textbf{to} you, you find it attractive or interesting.
 \textit{
	\begin{itemize}
	\item On the other hand, the idea appealed to him.
	\item The range has long appealed to all tastes.
	\end{itemize}
}
\item uncountable noun \\
The \textbf{appeal} of something is a quality that it has which people find attractive or interesting.
 \textit{
	\begin{itemize}
	\item Its new title was meant to give the party greater public appeal.
	\item Johnson's appeal is to people in all walks of life.
	\end{itemize}
}
\end{enumerate}

\section*{appear}
{\large \color{blue}  appears  appearing  appeared  }
\subsection*{Explain}
\begin{enumerate}
\item link verb \\
If you say that something \textbf{appears}  \textbf{to} be the way you describe it, you are reporting what you believe or what you have been told , though you cannot be sure it is true .
 \textit{
	\begin{itemize}
	\item There appears to be increasing support for the leadership to take a more aggressive
stance.
	\item The aircraft appears to have crashed near Katmandu.
	\item It appears that some missiles have been moved.
	\item It appears unlikely that the U.N. would consider making such a move.
	\item The presidency is beginning to appear a political irrelevance.
	\item Nine months later, those talks appear as distant as ever.
	\item He appeared willing to reach an agreement.
	\end{itemize}
}
\item link verb \\
If someone or something \textbf{appears} to have a particular quality or characteristic, they give the impression of having that quality or characteristic.
 \textit{
	\begin{itemize}
	\item She did her best to appear more self-assured than she felt.
	\item He is anxious to appear a gentleman.
	\item Under stress these people will appear to be superficial, over-eager and manipulative.
	\end{itemize}
}
\item verb \\
When someone or something \textbf{appears} , they move into a position where you can see them.
 \textit{
	\begin{itemize}
	\item A woman appeared at the far end of the street.
	\item Last night some of the prisoners appeared on the roof.
	\end{itemize}
}
\item verb \\
When something new \textbf{appears} , it begins to exist or reaches a stage of development where its existence can be noticed .
 \textit{
	\begin{itemize}
	\item ...small white flowers which appear in early summer.
	\item Slogans have appeared on walls around the city.
	\item ...a test which can reveal infection at an early stage, before symptoms appear.
	\end{itemize}
}
\item verb \\
When something such as a book \textbf{appears} , it is published or becomes available for people to buy .
 \textit{
	\begin{itemize}
	\item This is a story based in fact, though few knew about the match before this book appeared.
	\item ...a poem which appeared in his last collection of verse.
	\end{itemize}
}
\item verb \\
When someone \textbf{appears}  \textbf{in} something such as a play, a show , or a television  programme , they take part in it.
 \textit{
	\begin{itemize}
	\item Jill Bennett appeared in several of his plays.
	\item Student leaders appeared on television to ask for calm.
	\end{itemize}
}
\item verb \\
When someone \textbf{appears}  \textbf{before} a court of law or \textbf{before} an official  committee , they go there in order to answer  charges or to give information as a witness .
 \textit{
	\begin{itemize}
	\item Two other executives appeared at Worthing Magistrates' Court charged with tax fraud.
	\item The American will appear before members of the disciplinary committee at Portman
Square.
	\end{itemize}
}
\end{enumerate}

\section*{arrive}
{\large \color{blue}  arrives  arriving  arrived  }
\subsection*{Explain}
\begin{enumerate}
\item verb \\
When a person or vehicle  \textbf{arrives} at a place, they come to it at the end of a journey.
 \textit{
	\begin{itemize}
	\item Fresh groups of guests arrived.
	\item ...a small group of commuters waiting for their train, which arrived on time.
	\item The Princess Royal arrived at Gatwick this morning from Jamaica.
	\end{itemize}
}
\item verb \\
When you \textbf{arrive} at a place, you come to it for the first time in order to stay , live , or work there.
 \textit{
	\begin{itemize}
	\item ...in the old days before the European settlers arrived in the country.
	\item ...a young student newly arrived in England from New Zealand.
	\end{itemize}
}
\item verb \\
When something such as letter or meal  \textbf{arrives} , it is brought or delivered to you.
 \textit{
	\begin{itemize}
	\item Any entry arriving after the closing date will not be considered.
	\item Breakfast arrived while he was in the bathroom.
	\end{itemize}
}
\item verb \\
When something such as a new product or invention  \textbf{arrives} , it becomes available .
 \textit{
	\begin{itemize}
	\item The game was due to arrive in Japanese stores in March.
	\item They'll be ready to embrace the new technology when it arrives.
	\end{itemize}
}
\item verb \\
When a particular  moment or event  \textbf{arrives} , it happens , especially after you have been waiting for it or expecting it.
 \textit{
	\begin{itemize}
	\item The time has arrived when I need to give up smoking.
	\item ...the belief that the army would be much further forward before winter arrived.
	\end{itemize}
}
\item verb \\
When you \textbf{arrive at} something such as a decision , you decide something after thinking about it or discussing it.
 \textit{
	\begin{itemize}
	\item ...if the jury cannot arrive at a unanimous decision.
	\item These figures are arrived at on the basis of dentists' receipts for 1991-2.
	\end{itemize}
}
\item verb \\
When a baby \textbf{arrives} , it is born.
 \textit{
	\begin{itemize}
	\item It's very unlikely that your baby will arrive before you get to hospital.
	\end{itemize}
}
\item verb \\
If you say that someone \textbf{has arrived} , you mean that they have become successful or famous .
 \textit{
	\begin{itemize}
	\item These are cars which show you've arrived and had a good time along the way.
	\end{itemize}
}
\end{enumerate}

\section*{blow}
{\large \color{blue}  blows  blowing  blew  blown  }
\subsection*{Explain}
\begin{enumerate}
\item verb \\
When a wind or breeze  \textbf{blows} , the air moves.
 \textit{
	\begin{itemize}
	\item A chill wind blew at the top of the hill.
	\item We woke to find a gale blowing outside.
	\end{itemize}
}
\item verb \\
If the wind \textbf{blows} something somewhere or if it \textbf{blows} there, the wind moves it there.
 \textit{
	\begin{itemize}
	\item The wind blew her hair back from her forehead.
	\item Strong winds blew away most of the dust.
	\item Her cap fell off in the street and blew away.
	\item Sand blew in our eyes.
	\item The bushes and trees were blowing in the wind.
	\end{itemize}
}
\item verb \\
If you \textbf{blow} , you send out a stream of air from your mouth.
 \textit{
	\begin{itemize}
	\item Danny rubbed his arms and blew on his fingers to warm them.
	\item Take a deep breath and blow.
	\end{itemize}
}
\item verb \\
If you \textbf{blow} something somewhere, you move it by sending out a stream of air from your mouth.
 \textit{
	\begin{itemize}
	\item He picked up his mug and blew off the steam.
	\end{itemize}
}
\item verb \\
If you \textbf{blow}  bubbles or smoke rings, you make them by blowing air out of your mouth through liquid or
smoke.
 \textit{
	\begin{itemize}
	\item He blew a ring of blue smoke.
	\end{itemize}
}
\item verb \\
When a whistle or horn  \textbf{blows} or someone \textbf{blows} it, they make a sound by blowing into it.
 \textit{
	\begin{itemize}
	\item The whistle blew and the train slid forward.
	\item A guard was blowing his whistle.
	\end{itemize}
}
\item verb \\
When you \textbf{blow} your nose, you force air out of it through your nostrils in order to clear it.
 \textit{
	\begin{itemize}
	\item He took out a handkerchief and blew his nose.
	\end{itemize}
}
\item verb \\
To \textbf{blow} something \textbf{out} , \textbf{off} , or \textbf{away} means to remove or destroy it violently with an explosion .
 \textit{
	\begin{itemize}
	\item The can exploded, wrecking the kitchen and bathroom and blowing out windows.
	\item Rival gunmen blew the city to bits.
	\end{itemize}
}
\item verb \\
If you say that something \textbf{blows} an event, situation, or argument into a particular extreme state, especially an uncertain or unpleasant state, you mean that it causes it to be in that state.
 \textit{
	\begin{itemize}
	\item Someone took my comment and tried to blow it into a major controversy.
	\end{itemize}
}
\item verb \\
If you \textbf{blow} a large amount of money, you spend it quickly on luxuries .
 \textit{
	\begin{itemize}
	\item Before you blow it all on a luxury cruise, give a little thought to the future.
	\item My brother lent me some money and I went and blew the lot.
	\end{itemize}
}
\item verb \\
If you \textbf{blow} a chance or attempt to do something, you make a mistake which wastes the chance or causes the attempt to fail.
 \textit{
	\begin{itemize}
	\item He has almost certainly blown his chance of touring India this winter.
	\item ...the high-risk world of real estate, where one careless word could blow a whole
deal.
	\item Oh you fool! You've blown it!
	\end{itemize}
}
\item ergative verb \\
If a fuse \textbf{blows} or if something \textbf{blows} it, the fuse melts because too much electricity has been sent through it, and the electrical current is cut off.
 \textit{
	\begin{itemize}
	\item The fuse blew as he pressed the button.
	\end{itemize}
}
\item ergative verb \\
If you \textbf{blow} a tyre or if it \textbf{blows} , a hole suddenly appears in it and all the air comes out of it.
 \textbf{Blow out} means the same as blow1 .
 \textit{
	\begin{itemize}
	\item A lorry blew a tyre and careered into them.
	\item The car tyre blew.
	\item A tyre blew out when the coach was on its way.
	\end{itemize}
}
\item  \\
 blow your own trumpet \textit{
	\begin{itemize}
	\end{itemize}
}
\end{enumerate}

\section*{ascend}
{\large \color{blue}  ascends  ascending  ascended  }
\subsection*{Explain}
\begin{enumerate}
\item verb \\
If you \textbf{ascend} a hill or staircase , you go up it.
 \textit{
	\begin{itemize}
	\item Mrs Clayton had to hold Lizzie's hand as they ascended the steps.
	\item Then we ascend steeply through forests of rhododendron.
	\end{itemize}
}
\item verb \\
If a staircase or path  \textbf{ascends} , it leads up to a higher position.
 \textit{
	\begin{itemize}
	\item A number of staircases ascend from the cobbled streets onto the ramparts.
	\item ...an ascending spiral path leading to a tower.
	\end{itemize}
}
\item verb \\
If something \textbf{ascends} , it moves up, usually vertically or into the air.
 \textit{
	\begin{itemize}
	\item Keep the drill steady while it ascends and descends.
	\item Nott and Dickinson set a new altitude record when they ascended 55,900 feet in their
balloon.
	\end{itemize}
}
\item verb \\
If someone \textbf{ascends}  \textbf{to} an important position, they achieve it or are appointed to it. When someone \textbf{ascends} a throne , they become king , queen , or pope .
 \textit{
	\begin{itemize}
	\item ...the same year he ascended to power.
	\item Before ascending to the bench, she was a lawyer in a large New York firm.
	\item ...a few years before Sixtus V ascended the papal throne.
	\end{itemize}
}
\item verb \\
If you \textbf{ascend} in your career or in society , you gradually achieve success or a higher status .
 \textit{
	\begin{itemize}
	\item Mobutu ascended through the ranks, eventually becoming commander of the army.
	\item They move freely from one department to another as they ascend the civil service
ladder.
	\end{itemize}
}
\item verb \\
In some religions , when someone's soul  goes to heaven , you can say that they \textbf{ascend}  \textbf{to} heaven.
 \textit{
	\begin{itemize}
	\item ...the belief that the souls of the faithful and virtuous would ascend to heaven.
	\end{itemize}
}
\item verb \\
If something or someone \textbf{ascends}  \textbf{to} a higher level, they reach a state that is better than the one they were in before.
 \textit{
	\begin{itemize}
	\item The story ascends from a gothic tragedy to a miraculous fairy-tale.
	\end{itemize}
}
\end{enumerate}

\section*{coincide}
{\large \color{blue}  coincides  coinciding  coincided  }
\subsection*{Explain}
\begin{enumerate}
\item verb \\
If one event \textbf{coincides}  \textbf{with} another, they happen at the same time.
 \textit{
	\begin{itemize}
	\item The exhibition coincides with the 50th anniversary of his death.
	\item Although his mental illness had coincided with his war service it had not been caused
by it.
	\item The beginning of the solar and lunar years coincided every 13 years.
	\end{itemize}
}
\item verb \\
If the ideas or interests of two or more people \textbf{coincide} , they are the same.
 \textit{
	\begin{itemize}
	\item Our views don't always coincide, but we always voice our opinions.
	\item ...a case in which public and private interests coincide.
	\item Our father was delighted when our opinions coincided with his own.
	\end{itemize}
}
\end{enumerate}

\section*{become}
{\large \color{blue}  becomes  becoming  became  }
\subsection*{Explain}
\begin{enumerate}
\item link verb \\
If someone or something \textbf{becomes} a particular thing, they start to change and develop into that thing, or start to develop the characteristics mentioned .
 \textit{
	\begin{itemize}
	\item I first became interested in Islam while I was doing my nursing training.
	\item The cocoa industry dwindled because it became increasingly difficult to cover costs.
	\item During the 1980s the world's financial systems became more open.
	\item The pilot decided to land, but as we lost altitude the wind became stronger.
	\item She became interested in the idea of starting her own business.
	\item After leaving school, he became a professional footballer.
	\item In 1823 Honduras became a part of the United Provinces of Central America.
	\end{itemize}
}
\item verb \\
If something \textbf{becomes} someone, it makes them look  attractive or it seems right for them.
 \textit{
	\begin{itemize}
	\item Does khaki become you?
	\item Don't be crude tonight, Bernard, it doesn't become you.
	\end{itemize}
}
\item  \\
 what has become of \textit{
	\begin{itemize}
	\end{itemize}
}
\end{enumerate}

\section*{collaborate}
{\large \color{blue}  collaborates  collaborating  collaborated  }
\subsection*{Explain}
\begin{enumerate}
\item verb \\
When one person or group \textbf{collaborates}  \textbf{with} another, they work together, especially on a book or on some research .
 \textit{
	\begin{itemize}
	\item He collaborated with his son Michael on the English translation of the text.
	\item A hospital will collaborate with a retail developer to improve retail and catering
services.
	\item ...a place where professionals and amateurs collaborated in the making of music.
	\item The two men met and agreed to collaborate.
	\end{itemize}
}
\item verb \\
If someone \textbf{collaborates}  \textbf{with} an enemy that is occupying their country during a war, they help them.
 \textit{
	\begin{itemize}
	\item He was accused of having collaborated with the secret police.
	\end{itemize}
}
\end{enumerate}

\section*{browse}
{\large \color{blue}  browses  browsing  browsed  }
\subsection*{Explain}
\begin{enumerate}
\item verb \\
If you \textbf{browse} in a shop, you look at things in a fairly casual way, in the hope that you might  find something you like.
 \textbf{Browse} is also a noun .
 \textit{
	\begin{itemize}
	\item I stopped in several bookstores to browse.
	\item She browsed in an up-market antiques shop.
	\item I'm just browsing around.
	\item ...a browse around the shops.
	\end{itemize}
}
\item verb \\
If you \textbf{browse}  \textbf{through} a book or magazine , you look through it in a fairly casual way.
 \textit{
	\begin{itemize}
	\item ...sitting on the sofa browsing through the TV pages of the paper.
	\item There are plenty of biographies for him to browse over.
	\end{itemize}
}
\item verb \\
If you \textbf{browse} on a computer , you search for information in computer files or on the internet.
 \textit{
	\begin{itemize}
	\item Try browsing around in the network bulletin boards.
	\end{itemize}
}
\item verb \\
When animals \textbf{browse} , they feed on plants.
 \textit{
	\begin{itemize}
	\item ...the three red deer stags browsing 50 yards from my lodge on the fringes of the
forest.
	\end{itemize}
}
\end{enumerate}

\section*{compete}
{\large \color{blue}  competes  competing  competed  }
\subsection*{Explain}
\begin{enumerate}
\item verb \\
When one firm or country  \textbf{competes}  \textbf{with} another, it tries to get people to buy its own goods in preference to those of the other firm or country. You can also  say that two firms or countries \textbf{compete} .
 \textit{
	\begin{itemize}
	\item Its products compete with own-label desserts in most supermarkets.
	\item The stores compete with each other for increased market shares.
	\item Banks and building societies are competing fiercely for business.
	\item The American economy, and its ability to compete abroad, was slowing down according
to the report.
	\end{itemize}
}
\item verb \\
If you \textbf{compete}  \textbf{with} someone \textbf{for} something, you try to get it for yourself and stop the other person getting it. You can also say that two people \textbf{compete}  \textbf{for} something.
 \textit{
	\begin{itemize}
	\item Kangaroos compete with sheep and cattle for sparse supplies of food and water.
	\item Schools should not compete with each other or attempt to poach pupils.
	\item More than 2300 candidates from 93 political parties are competing for 486 seats.
	\end{itemize}
}
\item verb \\
If you \textbf{compete} in a contest or a game , you take part in it.
 \textit{
	\begin{itemize}
	\item He will be competing in the London–Calais–London race.
	\item Dubbed foreign language films will not be allowed to compete for best film.
	\item It is essential for all players who wish to compete that they earn computer ranking
points.
	\end{itemize}
}
\end{enumerate}

\section*{excel}
{\large \color{blue}  excels  excelling  excelled  }
\subsection*{Explain}
\begin{enumerate}
\item verb \\
If someone \textbf{excels}  \textbf{in} something or \textbf{excels}  \textbf{at} it, they are very good at doing it.
 \textit{
	\begin{itemize}
	\item Caine has always been an actor who excels in irony.
	\item Mary was a better rider than either of them and she excelled at outdoor sports.
	\item Academically he began to excel.
	\item I think Krishnan excelled himself in all departments of his game.
	\end{itemize}
}
\end{enumerate}

\section*{die}
{\large \color{blue}  dies  dying  died  }
\subsection*{Explain}
\begin{enumerate}
\item verb \\
When people, animals, and plants \textbf{die} , they stop living.
 \textit{
	\begin{itemize}
	\item A year later my dog died.
	\item Sadly, both he and my mother died of cancer.
	\item I would die a very happy person if I could stay in music my whole life.
	\item ...friends who died young.
	\end{itemize}
}
\item verb \\
If a person, animal, or plant \textbf{is dying} , they are so ill or so badly  injured that they will not live very much longer.
 \textit{
	\begin{itemize}
	\item The elm trees are all dying.
	\item Every working day I treat people who are dying from lung diseases caused by smoking.
	\end{itemize}
}
\item verb \\
If someone \textbf{dies} a violent , unnatural , or painful death, they die in a violent, unnatural, or painful way.
 \textit{
	\begin{itemize}
	\item He watched helplessly as his mother died an agonizing death.
	\item I'm no expert, but I don't think Tracy died a natural death.
	\end{itemize}
}
\item verb \\
If a machine or device \textbf{dies} , it stops completely, especially after a period of working more and more slowly or inefficiently.
 \textit{
	\begin{itemize}
	\item Then suddenly, the engine coughed, spluttered and died.
	\end{itemize}
}
\item verb \\
If a fire or light \textbf{dies} , it stops burning or shining .
 \textit{
	\begin{itemize}
	\item Her cigarette glowed brightly, then died.
	\end{itemize}
}
\item verb \\
If an emotion or facial  expression  \textbf{dies} , it disappears completely, usually after a period of gradually becoming  weaker and less noticeable .
 \textit{
	\begin{itemize}
	\item My love for you will never die.
	\item Kathryn looked down at the floor and the smile died on her lips.
	\end{itemize}
}
\item verb \\
You can say that you \textbf{are dying of}  thirst , hunger , boredom, or curiosity to emphasize that you are very thirsty , hungry , bored , or curious .
 \textit{
	\begin{itemize}
	\item Order me a pot of tea, I'm dying of thirst.
	\end{itemize}
}
\item verb \\
You can say that you \textbf{are dying for} something or \textbf{are dying}  \textbf{to} do something to emphasize that you very much want to have it or do it.
 \textit{
	\begin{itemize}
	\item I'm dying for a breath of fresh air.
	\item She was dying to talk to Frank.
	\end{itemize}
}
\item verb \\
You can use \textbf{die} in expressions such as ' \textbf{I almost died} ' or ' \textbf{I'd die if anything happened} ' where you are emphasizing your feelings about a situation, for example to say that
it is very shocking , upsetting , embarrassing , or amusing .
 \textit{
	\begin{itemize}
	\item I nearly died when I learned where I was ending up.
	\item I nearly died of shame.
	\item I thought I'd die laughing.
	\end{itemize}
}
\item countable noun \\
A \textbf{die} is a specially shaped or patterned block of metal which is used to press or cut other metal into a particular shape.
 \textit{
	\begin{itemize}
	\end{itemize}
}
\item  \\
 the die is cast \textit{
	\begin{itemize}
	\end{itemize}
}
\item  \\
 die hard \textit{
	\begin{itemize}
	\end{itemize}
}
\end{enumerate}

\section*{glare}
{\large \color{blue}  glares  glaring  glared  }
\subsection*{Explain}
\begin{enumerate}
\item verb \\
If you \textbf{glare}  \textbf{at} someone, you look at them with an angry expression on your face .
 \textit{
	\begin{itemize}
	\item The old woman glared at him.
	\item Jacob glared and muttered something.
	\item ...glaring eyes.
	\end{itemize}
}
\item countable noun \\
A \textbf{glare} is an angry, hard , and unfriendly look.
 \textit{
	\begin{itemize}
	\item His glasses magnified his irritable glare.
	\end{itemize}
}
\item verb \\
If the sun or a light \textbf{glares} , it shines with a very bright light which is difficult to look at.
 \textit{
	\begin{itemize}
	\item The sunlight glared.
	\item ...glaring searchlight beams.
	\end{itemize}
}
\item uncountable noun \\
\textbf{Glare} is very bright light that is difficult to look at.
 \textit{
	\begin{itemize}
	\item ...the glare of a car's headlights.
	\item Special-purpose glasses reduce glare.
	\end{itemize}
}
\item singular noun \\
If someone is in \textbf{the glare of}  publicity or public  attention , they are constantly being watched and talked about by a lot of people.
 \textit{
	\begin{itemize}
	\item Norma is said to dislike the glare of publicity.
	\item She attacked police in the full glare of TV cameras.
	\end{itemize}
}
\end{enumerate}

\section*{graze}
{\large \color{blue}  grazes  grazing  grazed  }
\subsection*{Explain}
\begin{enumerate}
\item verb \\
When animals \textbf{graze} or \textbf{are grazed} , they eat the grass or other plants that are growing in a particular place. You can also  say that a field  \textbf{is grazed} by animals.
 \textit{
	\begin{itemize}
	\item Five cows graze serenely around a massive oak.
	\item The hills have been grazed by sheep because they were too steep to be ploughed.
	\item Several horses grazed the meadowland.
	\item ...a large herd of grazing animals.
	\end{itemize}
}
\item verb \\
If you \textbf{graze} a part of your body, you injure your skin by scraping against something.
 \textit{
	\begin{itemize}
	\item I had grazed my knees a little.
	\end{itemize}
}
\item countable noun \\
A \textbf{graze} is a small wound caused by scraping against something.
 \textit{
	\begin{itemize}
	\end{itemize}
}
\item verb \\
If something \textbf{grazes} another thing, it touches that thing lightly as it passes by.
 \textit{
	\begin{itemize}
	\item A bullet had grazed his arm.
	\item Wright managed a shot but it grazed the near post and rolled harmlessly across the
goal.
	\end{itemize}
}
\end{enumerate}

\section*{sit}
{\large \color{blue}  sits  sitting  sat  }
\subsection*{Explain}
\begin{enumerate}
\item verb \\
If you \textbf{are sitting}  somewhere , for example in a chair , your bottom is resting on the chair and the upper part of your body is upright.
 \textit{
	\begin{itemize}
	\item Mother was sitting in her chair in the kitchen.
	\item They sat there in shock and disbelief.
	\item They had been sitting watching television.
	\item He was unable to sit still for longer than a few minutes.
	\end{itemize}
}
\item verb \\
When you \textbf{sit} somewhere, you lower your body until you are sitting on something.
 \textbf{Sit down}  means the same as sit .
 \textit{
	\begin{itemize}
	\item He set the cases against a wall and sat on them.
	\item Eva pulled over a chair and sat beside her husband.
	\item When you stand, they stand; when you sit, they sit.
	\item I sat down, stunned.
	\item Hughes beckoned him to sit down on the sofa.
	\end{itemize}
}
\item verb \\
If you \textbf{sit} someone somewhere, you tell them to sit there or put them in a sitting position.
 To \textbf{sit} someone \textbf{down} somewhere means to sit them there.
 \textit{
	\begin{itemize}
	\item He used to sit me on his lap.
	\item He'll sit you in front of his computer and give you a glimpse of the problem.
	\item She helped him out of the water and sat him down on the rock.
	\item They sat me down and had a serious discussion about sex.
	\end{itemize}
}
\item verb \\
If you \textbf{sit}  \textbf{for} an artist or photographer , you place yourself in a sitting position so you can be painted or photographed .
 \textit{
	\begin{itemize}
	\item A person may well have been sitting for the artist for eight hours at a stretch.
	\end{itemize}
}
\item verb \\
If you \textbf{sit} an examination, you do it.
 \textit{
	\begin{itemize}
	\item June and July are the traditional months for sitting exams.
	\end{itemize}
}
\item verb \\
If you \textbf{sit}  \textbf{on} a committee or other official group, you are a member of it.
 \textit{
	\begin{itemize}
	\item He was asked to sit on numerous committees.
	\item I know of no professional person who has ever sat on a jury.
	\item The party's three MPs will continue to sit in parliament.
	\end{itemize}
}
\item verb \\
When a parliament , legislature , court , or other official body \textbf{sits} , it officially  carries out its work.
 \textit{
	\begin{itemize}
	\item Parliament sits for only 28 weeks out of 52.
	\item The court would sit all night.
	\end{itemize}
}
\item verb \\
If a building or object \textbf{sits} in a particular place, it is in that place.
 \textit{
	\begin{itemize}
	\item Our new house sat next to a stream.
	\item On the table sat a box decorated with little pearl triangles.
	\end{itemize}
}
\item verb \\
To \textbf{sit}  \textbf{for} someone means the same as to babysit for them.
 \textit{
	\begin{itemize}
	\item I've asked Mum to sit for us next Saturday.
	\end{itemize}
}
\item  \\
 to sit tight \textit{
	\begin{itemize}
	\end{itemize}
}
\end{enumerate}

\section*{hurry}
{\large \color{blue}  hurries  hurrying  hurried  }
\subsection*{Explain}
\begin{enumerate}
\item verb \\
If you \textbf{hurry}  somewhere , you go there as quickly as you can.
 \textit{
	\begin{itemize}
	\item Claire hurried along the road.
	\item When she finished work she had to hurry home and look after her son.
	\item Bob hurried to join him, and they rode home together.
	\end{itemize}
}
\item verb \\
If you \textbf{hurry}  \textbf{to} do something, you start doing it as soon as you can, or try to do it quickly.
 \textit{
	\begin{itemize}
	\item Mrs Hardie hurried to make up for her tactlessness by asking her guest about his
holiday.
	\item There was no longer any reason to hurry.
	\end{itemize}
}
\item singular noun \\
If you are \textbf{in a}  \textbf{hurry}  \textbf{to} do something, you need or want to do something quickly. If you do something \textbf{in a}  \textbf{hurry} , you do it quickly or suddenly .
 \textit{
	\begin{itemize}
	\item Kate was in a hurry to grow up, eager for knowledge and experience.
	\item Eric left the barge in a hurry.
	\end{itemize}
}
\item verb \\
To \textbf{hurry} something means the same as to hurry up something.
 \textit{
	\begin{itemize}
	\item ...The President's attempt to hurry the process of independence.
	\end{itemize}
}
\item verb \\
If you \textbf{hurry} someone to a place or into a situation , you try to make them go to that place or get into that situation quickly.
 \textit{
	\begin{itemize}
	\item Rachel hurried him to his bed.
	\item They say they are not going to be hurried into any decision.
	\item I don't want to hurry you.
	\end{itemize}
}
\item  \\
 there's no hurry \textit{
	\begin{itemize}
	\end{itemize}
}
\item  \\
 in no hurry \textit{
	\begin{itemize}
	\end{itemize}
}
\end{enumerate}

\section*{skip}
{\large \color{blue}  skips  skipping  skipped  }
\subsection*{Explain}
\begin{enumerate}
\item verb \\
If you \textbf{skip} along, you move almost as if you are dancing , with a series of little jumps from one foot to the other.
 \textbf{Skip} is also a noun .
 \textit{
	\begin{itemize}
	\item They saw the man with a little girl skipping along behind him.
	\item We went skipping down the street arm in arm.
	\item She was skipping to keep up with him.
	\item The boxer gave a little skip as he came out of his corner.
	\end{itemize}
}
\item verb \\
When someone \textbf{skips} , they jump up and down over a rope which they or two other people are holding at each end and turning  round and round. In American English, you say that someone \textbf{skips rope} .
 \textit{
	\begin{itemize}
	\item Outside, children were skipping and singing a rhyme.
	\item They skip rope and play catch, waiting for the bell.
	\end{itemize}
}
\item verb \\
If you \textbf{skip} something that you usually do or something that most people do, you decide not to do it.
 \textit{
	\begin{itemize}
	\item It is important not to skip meals.
	\item Her daughter started skipping school.
	\end{itemize}
}
\item verb \\
If you \textbf{skip} or \textbf{skip over} a part of something you are reading or a story you are telling , you miss it out or pass over it quickly and move on to something else.
 \textit{
	\begin{itemize}
	\item You might want to skip the exercises in this chapter.
	\item Here it must be noted that Cook skips over the ravages inflicted by the conquistadors.
	\end{itemize}
}
\item verb \\
If you \textbf{skip}  \textbf{from} one subject or activity  \textbf{to} another, you move quickly from one to the other although there is no obvious  connection between them.
 \textit{
	\begin{itemize}
	\item She kept up a continuous chatter, skipping from one subject to the next.
	\end{itemize}
}
\item countable noun \\
A \textbf{skip} is a large, open, metal container which is used to hold and take away large unwanted  items and rubbish .
 \textit{
	\begin{itemize}
	\end{itemize}
}
\end{enumerate}

\section*{intrude}
{\large \color{blue}  intrudes  intruding  intruded  }
\subsection*{Explain}
\begin{enumerate}
\item verb \\
If you say that someone \textbf{is intruding}  \textbf{into} a particular place or situation , you mean that they are not wanted or welcome there.
 \textit{
	\begin{itemize}
	\item The press has been blamed for intruding into people's personal lives in an unacceptable
way.
	\item I don't want to intrude on your meeting.
	\item I hope I'm not intruding.
	\end{itemize}
}
\item verb \\
If something \textbf{intrudes}  \textbf{on} your mood or your life , it disturbs it or has an unwanted  effect on it.
 \textit{
	\begin{itemize}
	\item Do you feel anxious when unforeseen incidents intrude on your day?
	\item There are times when personal feelings cannot be allowed to intrude.
	\end{itemize}
}
\item verb \\
If someone \textbf{intrudes}  \textbf{into} a place, they go there even though they are not allowed to be there.
 \textit{
	\begin{itemize}
	\item The officer on the scene said no one had intruded into the area.
	\item We believe they intruded on to the field of play.
	\item The voyage home began, but not before an intruding aeroplane had repeatedly circled
the ship.
	\end{itemize}
}
\end{enumerate}

\section*{sneak}
{\large \color{blue}  sneaks  sneaking  sneaked  }
\subsection*{Explain}
\begin{enumerate}
\item verb \\
If you \textbf{sneak}  somewhere , you go there very quietly on foot , trying to avoid being seen or heard .
 \textit{
	\begin{itemize}
	\item Sometimes he would sneak out of his house late at night to be with me.
	\item Don't sneak away and hide.
	\end{itemize}
}
\item verb \\
If you \textbf{sneak} something somewhere, you take it there secretly.
 \textit{
	\begin{itemize}
	\item He smuggled papers out each day, photocopied them, and snuck them back.
	\item He reckons he can sneak you some free nachos.
	\end{itemize}
}
\item verb \\
If you \textbf{sneak} a look at someone or something, you secretly have a quick look at them.
 \textit{
	\begin{itemize}
	\item You sneak a look at your watch to see how long you've got to wait.
	\end{itemize}
}
\end{enumerate}

\section*{opt}
{\large \color{blue}  opts  opting  opted  }
\subsection*{Explain}
\begin{enumerate}
\item verb \\
If you \textbf{opt}  \textbf{for} something, or \textbf{opt}  \textbf{to} do something, you choose it or decide to do it in preference to anything else.
 \textit{
	\begin{itemize}
	\item Depending on your circumstances you may wish to opt for one method or the other.
	\item Our students can also opt to stay in residence.
	\end{itemize}
}
\end{enumerate}

\section*{speculate}
{\large \color{blue}  speculates  speculating  speculated  }
\subsection*{Explain}
\begin{enumerate}
\item verb \\
If you \textbf{speculate} about something, you make guesses about its nature or identity , or about what might  happen .
 \textit{
	\begin{itemize}
	\item Critics of the project speculate about how many hospitals could be built instead.
	\item It would be unfair to speculate on the reasons for her resignation.
	\item The doctors speculate that he died of a cerebral haemorrhage caused by a blow on
the head.
	\item The reader can speculate what will happen next.
	\end{itemize}
}
\item verb \\
If someone \textbf{speculates} financially, they buy property, stocks , or shares , in the hope of being able to sell them again at a higher  price and make a profit .
 \textit{
	\begin{itemize}
	\item Big farmers are moving in, in order to speculate with rising land prices.
	\item The banks speculated in property whose value has now dropped.
	\end{itemize}
}
\end{enumerate}

\section*{sneeze}
{\large \color{blue}  sneezes  sneezing  sneezed  }
\subsection*{Explain}
\begin{enumerate}
\item verb \\
When you \textbf{sneeze} , you suddenly  take in your breath and then blow it down your nose noisily without being able to stop yourself, for example because you have a cold .
 \textbf{Sneeze} is also a noun .
 \textit{
	\begin{itemize}
	\item What exactly happens when we sneeze?
	\item See your doctor now to beat summer sneezing.
	\item Coughs and sneezes spread infections.
	\end{itemize}
}
\item  \\
 not to be sneezed at \textit{
	\begin{itemize}
	\end{itemize}
}
\end{enumerate}

\section*{surge}
{\large \color{blue}  surges  surging  surged  }
\subsection*{Explain}
\begin{enumerate}
\item countable noun \\
A \textbf{surge} is a sudden large increase in something that has previously been steady , or has only increased or developed slowly.
 \textit{
	\begin{itemize}
	\item Specialists see various reasons for the recent surge in inflation.
	\item The anniversary is bound to bring a new surge of interest in the poet's work.
	\end{itemize}
}
\item verb \\
If something \textbf{surges} , it increases suddenly and greatly, after being steady or developing only slowly.
 \textit{
	\begin{itemize}
	\item The party's electoral support surged from just under 10 per cent to nearly 17 per
cent.
	\item Surging imports will add to the demand for hard currency.
	\end{itemize}
}
\item verb \\
If a crowd of people \textbf{surge}  forward , they suddenly move forward together .
 \textit{
	\begin{itemize}
	\item The photographers and cameramen surged forward.
	\item ...the crowd surging out from the church.
	\end{itemize}
}
\item countable noun \\
A \textbf{surge} is a sudden powerful  movement of a physical  force such as wind or water.
 \textit{
	\begin{itemize}
	\item The whole car shuddered with an almost frightening surge of power.
	\item In the year 1091, London Bridge was destroyed by a tidal surge during a storm.
	\end{itemize}
}
\item verb \\
If a physical force such as water or electricity  \textbf{surges} through something, it moves through it suddenly and powerfully.
 \textit{
	\begin{itemize}
	\item A paraglider crashed into power lines and survived 11,000 volts surging through his
body.
	\item Fish and seaweed rose, caught motionless in the surging water.
	\end{itemize}
}
\item countable noun \\
If you feel a \textbf{surge}  \textbf{of} a particular  emotion or feeling , you experience it suddenly and powerfully.
 \textit{
	\begin{itemize}
	\item 'It must be very difficult,' said Hunter, feeling a surge of embarrassment for Diane's
predicament.
	\item He was overcome by a sudden surge of jealousy.
	\end{itemize}
}
\item verb \\
If an emotion or sensation  \textbf{surges}  \textbf{in} you or \textbf{through} you, you feel it suddenly and powerfully.
 \textbf{Surge up}  means the same as surge .
 \textit{
	\begin{itemize}
	\item Nausea surged in him and he retched violently.
	\item Panic surged through her.
	\item A slow hatred for Hilton began to surge up in him.
	\item Memories surged up in Don's mind.
	\end{itemize}
}
\end{enumerate}

\section*{sprout}
{\large \color{blue}  sprouts  sprouting  sprouted  }
\subsection*{Explain}
\begin{enumerate}
\item verb \\
When plants, vegetables , or seeds \textbf{sprout} , they produce new shoots or leaves.
 \textit{
	\begin{itemize}
	\item It only takes a few days for beans to sprout.
	\end{itemize}
}
\item verb \\
When leaves, shoots, or plants \textbf{sprout}  somewhere , they grow there.
 \textit{
	\begin{itemize}
	\item Leaf-shoots were beginning to sprout on the hawthorn.
	\item Birch trees sprouted from the rubble and grew into a dense young wood.
	\end{itemize}
}
\item verb \\
If a garden or other area of land  \textbf{sprouts} plants, they start to grow there.
 \textit{
	\begin{itemize}
	\item ...the garden, which had had time to sprout a shocking collection of weeds.
	\end{itemize}
}
\item verb \\
If you \textbf{sprout}  beans or seeds, you make them grow small shoots before eating them. You usually do this by soaking them in water.
 \textit{
	\begin{itemize}
	\item When you sprout seeds their nutritional content increases.
	\item Sprouted beans only need to be cooked for 1-2 minutes.
	\end{itemize}
}
\item verb \\
If something such as hair  \textbf{sprouts} from a person or animal, or if they \textbf{sprout} it, it grows on them.
 \textit{
	\begin{itemize}
	\item She has little wire-rimmed glasses and whiskers sprouting from her chin.
	\item Kevin is sprouting a few grey hairs.
	\end{itemize}
}
\item ergative verb \\
If a large number of things have appeared or developed somewhere, you can  say that they \textbf{have sprouted} there or that the place \textbf{has sprouted} them.
 \textit{
	\begin{itemize}
	\item More than a million satellite dishes have sprouted on homes across the country.
	\item Since its first shop was opened in 1976, it has sprouted outlets in 39 countries.
	\end{itemize}
}
\item countable noun \\
\textbf{Sprouts} are vegetables that look like tiny  cabbages . They are also  called  brussels sprouts .
 \textit{
	\begin{itemize}
	\end{itemize}
}
\item countable noun \\
\textbf{Sprouts} are new shoots on plants.
 \textit{
	\begin{itemize}
	\item After eleven days of growth the number of sprouts was counted.
	\end{itemize}
}
\end{enumerate}

\section*{unite}
{\large \color{blue}  unites  uniting  united  }
\subsection*{Explain}
\begin{enumerate}
\item verb \\
If a group of people or things \textbf{unite} or if something \textbf{unites} them, they join together and act as a group.
 \textit{
	\begin{itemize}
	\item The two parties have been trying to unite since the New Year.
	\item The vast majority of nations have agreed to unite their efforts to bring peace.
	\end{itemize}
}
\end{enumerate}

\section*{stride}
{\large \color{blue}  strides  striding  strode  }
\subsection*{Explain}
\begin{enumerate}
\item verb \\
If you \textbf{stride}  somewhere , you walk there with quick , long steps.
 \textit{
	\begin{itemize}
	\item They were joined by a newcomer who came striding across a field.
	\item He turned abruptly and strode off down the corridor.
	\end{itemize}
}
\item countable noun \\
A \textbf{stride} is a long step which you take when you are walking or running .
 \textit{
	\begin{itemize}
	\item With every stride, runners hit the ground with up to five times their body-weight.
	\item He walked with long strides.
	\end{itemize}
}
\item singular noun \\
Someone's \textbf{stride} is their way of walking with long steps.
 \textit{
	\begin{itemize}
	\item He lengthened his stride to keep up with her.
	\end{itemize}
}
\item countable noun \\
If you \textbf{make}  \textbf{strides} in something that you are doing, you make rapid progress in it.
 \textit{
	\begin{itemize}
	\item The country has made enormous strides politically but not economically.
	\end{itemize}
}
\item  \\
 get into one's stride/ hit one's stride \textit{
	\begin{itemize}
	\end{itemize}
}
\item  \\
 take sth in your stride \textit{
	\begin{itemize}
	\end{itemize}
}
\end{enumerate}

\section*{unload}
{\large \color{blue}  unloads  unloading  unloaded  }
\subsection*{Explain}
\begin{enumerate}
\item verb \\
If you \textbf{unload} goods from a vehicle, or you \textbf{unload} a vehicle, you remove the goods from the vehicle, usually after they have been transported from one place to another.
 \textit{
	\begin{itemize}
	\item Unload everything from the boat and clean it thoroughly.
	\item They were reported to be unloading trucks filled with looted furniture.
	\end{itemize}
}
\item verb \\
If someone \textbf{unloads}  investments , they get rid of them or sell them.
 \textit{
	\begin{itemize}
	\item Since March, he has unloaded 1.3 million shares.
	\end{itemize}
}
\end{enumerate}

\section*{succeed}
{\large \color{blue}  succeeds  succeeding  succeeded  }
\subsection*{Explain}
\begin{enumerate}
\item verb \\
If you \textbf{succeed}  \textbf{in} doing something, you manage to do it.
 \textit{
	\begin{itemize}
	\item We have already succeeded in working out ground rules with the Department of Defense.
	\item Some people will succeed in their efforts to stop smoking.
	\item If they can succeed in America and Europe, then they can succeed here too.
	\end{itemize}
}
\item verb \\
If something \textbf{succeeds} , it works in a satisfactory  way or has the result that is intended .
 \textit{
	\begin{itemize}
	\item Your staff want your business to succeed as much as you do.
	\item ...a move which would make any future talks even more unlikely to succeed.
	\end{itemize}
}
\item verb \\
Someone who \textbf{succeeds}  gains a high position in what they do, for example in business or politics .
 \textit{
	\begin{itemize}
	\item ...the skills and qualities needed to succeed in small and medium-sized businesses.
	\end{itemize}
}
\item verb \\
If you \textbf{succeed} another person, you are the next person to have their job or position.
 \textit{
	\begin{itemize}
	\item David is almost certain to succeed him as chairman on January 1.
	\item George III succeeded to the throne in 1760.
	\end{itemize}
}
\item verb \\
If one thing \textbf{is succeeded}  \textbf{by} another thing, the other thing happens or comes after it.
 \textit{
	\begin{itemize}
	\item All political systems will collapse eventually and be succeeded by others.
	\end{itemize}
}
\end{enumerate}

\section*{vanish}
{\large \color{blue}  vanishes  vanishing  vanished  }
\subsection*{Explain}
\begin{enumerate}
\item verb \\
If someone or something \textbf{vanishes} , they disappear suddenly or in a way that cannot be explained .
 \textit{
	\begin{itemize}
	\item He just vanished and was never seen again.
	\item The aircraft vanished without trace.
	\item Anne vanished from outside her home last Wednesday.
	\item The gunmen paused only to cut the wires to the house, then vanished into the countryside.
	\end{itemize}
}
\item verb \\
If something such as a species of animal or a tradition  \textbf{vanishes} , it stops existing.
 \textit{
	\begin{itemize}
	\item Near the end of Devonian times, thirty percent of all animal life vanished.
	\item He does not think that craftsmanship has vanished from our world.
	\end{itemize}
}
\end{enumerate}

\section*{surrender}
{\large \color{blue}  surrenders  surrendering  surrendered  }
\subsection*{Explain}
\begin{enumerate}
\item verb \\
If you \textbf{surrender} , you stop  fighting or resisting someone and agree that you have been beaten .
 \textbf{Surrender} is also a noun .
 \textit{
	\begin{itemize}
	\item He called on the rebels to surrender.
	\item She surrendered to the police in London last December.
	\item ...the government's apparent surrender to demands made by the religious militants.
	\end{itemize}
}
\item verb \\
If you \textbf{surrender} something you would rather  keep , you give it up or let someone else have it, for example after a struggle .
 \textbf{Surrender} is also a noun.
 \textit{
	\begin{itemize}
	\item Nadja had to fill out forms surrendering all rights to her property.
	\item They had little choice but to surrender their weapons to the government's forces.
	\item ...the sixteen-day deadline for the surrender of weapons and ammunition.
	\end{itemize}
}
\item verb \\
If you \textbf{surrender} something such as a ticket or your passport , you give it to someone in authority when they ask you to.
 \textit{
	\begin{itemize}
	\item They have been ordered to surrender their passports.
	\end{itemize}
}
\item uncountable noun \\
You use \textbf{surrender} to refer to someone's attitude or behaviour when they lose the will to resist their feelings or the demands of other people.
 \textit{
	\begin{itemize}
	\item Depression is a partial surrender to death.
	\item A look of disbelief came into his eyes, but was quickly replaced by one of dismal
surrender.
	\end{itemize}
}
\end{enumerate}

\end{document}