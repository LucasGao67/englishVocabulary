\documentclass[twocolumn]{book}

\title{English book}
\author{lucas gao}
\date{\today}
\usepackage{lipsum}
\usepackage{color}
\usepackage{sectsty}
\sectionfont{\color{cyan} \fontsize{18}{20}\selectfont}

\begin{document}
\section*{barber}
{\large \color{blue}  barbers  }
\subsection*{Explain}
\begin{enumerate}
\item countable noun \\
A \textbf{barber} is a man whose job is cutting men's hair.
 \textit{
	\begin{itemize}
	\item He went to get his hair trimmed by the barber.
	\end{itemize}
}
\item singular noun \\
A \textbf{barber's} is a shop where a barber works.
 \textit{
	\begin{itemize}
	\item My Mum took me to the barber's.
	\end{itemize}
}
\end{enumerate}

\section*{alongside}
{\large \color{blue}  }
\subsection*{Explain}
\begin{enumerate}
\item preposition \\
If one thing is \textbf{alongside} another thing, the first thing is next to the second.
 \textbf{Alongside} is also an adverb .
 \textit{
	\begin{itemize}
	\item He crossed the street and walked alongside Central Park.
	\item Much of the industry was located alongside rivers.
	\item He waited several minutes for a car to pull up alongside.
	\end{itemize}
}
\item preposition \\
If you work \textbf{alongside} other people, you all work together in the same place.
 \textit{
	\begin{itemize}
	\item He had worked alongside Frank and Mark and they had become friends.
	\item Men aged 60 are fighting alongside young boys.
	\end{itemize}
}
\item preposition \\
If one thing exists or develops \textbf{alongside} another, the two things exist or develop together at the same time.
 \textit{
	\begin{itemize}
	\item Her self-confidence will develop alongside her technique.
	\item Alongside work, children, partner, friends and chores, who on earth has a spare second
to be a domestic star?
	\end{itemize}
}
\end{enumerate}

\section*{bat}
{\large \color{blue}  bats  batting  batted  }
\subsection*{Explain}
\begin{enumerate}
\item countable noun \\
A \textbf{bat} is a specially shaped piece of wood that is used for hitting the ball in baseball, softball , cricket, rounders , or table tennis.
 \textit{
	\begin{itemize}
	\item ...a baseball bat.
	\end{itemize}
}
\item verb \\
When you \textbf{bat} , you have a turn at hitting the ball with a bat in baseball, softball, cricket, or
rounders.
 \textit{
	\begin{itemize}
	\item Australia, put in to bat, made a cautious start.
	\end{itemize}
}
\item countable noun \\
A \textbf{bat} is a small flying animal that looks  like a mouse with wings made of skin . Bats are active at night .
 \textit{
	\begin{itemize}
	\end{itemize}
}
\item  \\
 not bat an eyelid \textit{
	\begin{itemize}
	\end{itemize}
}
\item  \\
 go (in) to bat for sb \textit{
	\begin{itemize}
	\end{itemize}
}
\item  \\
 like a bat out of hell \textit{
	\begin{itemize}
	\end{itemize}
}
\item  \\
 off one's own bat \textit{
	\begin{itemize}
	\end{itemize}
}
\item  \\
 right off the bat \textit{
	\begin{itemize}
	\end{itemize}
}
\end{enumerate}

\section*{also}
{\large \color{blue}  }
\subsection*{Explain}
\begin{enumerate}
\item adverb \\
You can use \textbf{also} to give more information about a person or thing, or to add another relevant  fact .
 \textit{
	\begin{itemize}
	\item It is the work of Ivor Roberts-Jones, who also produced the statue of Churchill in
Parliament Square.
	\item He is an asthmatic who was also anaemic.
	\item She has a reputation for brilliance. Also, she is a good communicator.
	\end{itemize}
}
\item adverb \\
You can use \textbf{also} to indicate that something you have just said about one person or thing is true of another person or thing.
 \textit{
	\begin{itemize}
	\item His father, also a top-ranking officer, had perished during the war.
	\item We have been working very hard, and our families have also worked hard.
	\item Not only cancer, but also heart and lung disease are influenced by smoking.
	\end{itemize}
}
\end{enumerate}

\section*{aside}
{\large \color{blue}  asides  }
\subsection*{Explain}
\begin{enumerate}
\item adverb \\
If you move something \textbf{aside} , you move it to one side of you.
 \textit{
	\begin{itemize}
	\item Sarah closed the book and laid it aside.
	\end{itemize}
}
\item adverb \\
If you take or draw someone \textbf{aside} , you take them a little way away from a group of people in order to talk to them in private .
 \textit{
	\begin{itemize}
	\item Billy Ewing grabbed him by the elbow and took him aside.
	\item Will put his arm around her shoulders and drew her aside.
	\end{itemize}
}
\item adverb \\
If you move \textbf{aside} , you get out of someone's way.
 \textit{
	\begin{itemize}
	\item She had been standing in the doorway, but now she stepped aside to let them pass.
	\end{itemize}
}
\item adverb \\
If you set something such as time, money , or space  \textbf{aside} for a particular purpose , you save it and do not use it for anything else.
 \textit{
	\begin{itemize}
	\item She wants to put her pocket-money aside for holidays.
	\item ...the ground set aside for the new cathedral.
	\end{itemize}
}
\item adverb \\
If you brush or sweep  \textbf{aside} a feeling or suggestion , you reject it.
 \textit{
	\begin{itemize}
	\item Talk to a friend who will really listen and not brush aside your feelings.
	\item The Prime Minister swept aside concern about the rising cost of mortgages.
	\end{itemize}
}
\item adverb \\
You use \textbf{aside} to indicate that you have finished talking about something, or that you are leaving it out of your discussion , and that you are about to talk about something else.
 \textit{
	\begin{itemize}
	\item Leaving aside the nutritional argument, these loaves are better value.
	\item Emotional arguments aside, here are the facts.
	\end{itemize}
}
\item countable noun \\
An \textbf{aside} is a comment that a character in a play makes to the audience, which the other characters are supposed not to be able to hear.
 \textit{
	\begin{itemize}
	\item She rolls her eyes and mutters an aside to the camera, 'No wonder I'm stressed!'
	\end{itemize}
}
\item countable noun \\
An \textbf{aside} is something that you say that is not directly  connected with what you are talking about.
 \textit{
	\begin{itemize}
	\item The pace of the book is leisurely, with enjoyable literary and historical asides.
	\end{itemize}
}
\end{enumerate}

\section*{away}
{\large \color{blue}  }
\subsection*{Explain}
\begin{enumerate}
\item adverb \\
If someone or something moves or is moved \textbf{away}  \textbf{from} a place, they move or are moved so that they are no longer there. If you are \textbf{away}  \textbf{from} a place, you are not in the place where people expect you to be.
 \textit{
	\begin{itemize}
	\item An injured police officer was led away by colleagues.
	\item He walked away from his car.
	\item She drove away before either of them could speak again.
	\item Jason was away on a business trip.
	\item Simon had been away a good deal lately.
	\end{itemize}
}
\item adverb \\
If you look or turn  \textbf{away}  \textbf{from} something, you move your head so that you are no longer looking at it.
 \textit{
	\begin{itemize}
	\item She quickly looked away and stared down at her hands.
	\item As he stands up, he turns his face away from her so that she won't see his tears.
	\end{itemize}
}
\item adverb \\
If you put or tidy something \textbf{away} , you put it where it should be. If you hide someone or something \textbf{away} , you put them in a place where nobody can  see them or find them.
 \textit{
	\begin{itemize}
	\item I put my journal away and prepared for bed.
	\item All her letters were carefully filed away in folders.
	\item I have $100m hidden away where no one will ever find it.
	\end{itemize}
}
\item  \\
 away from sb/sth \textit{
	\begin{itemize}
	\end{itemize}
}
\item adverb \\
You use \textbf{away} to talk about future  events . For example , if an event is a week  \textbf{away} , it will  happen after a week.
 \textit{
	\begin{itemize}
	\item ...the Washington summit, now only just over two weeks away.
	\item Peace it seemed might at last be no more than a few months away.
	\end{itemize}
}
\item adverb \\
When a sports  team plays \textbf{away} , it plays on its opponents ' ground.
 \textbf{Away} is also an adjective .
 \textit{
	\begin{itemize}
	\item ...a sensational 4-3 victory for the team playing away.
	\item Charlton are about to play an important away match.
	\end{itemize}
}
\item adverb \\
You can use \textbf{away} to say that something slowly disappears , becomes less significant , or changes so that it is no longer the same.
 \textit{
	\begin{itemize}
	\item So much snow has already melted away.
	\item His voice died away in a whisper.
	\item Once they took office that source of support fell away.
	\end{itemize}
}
\item adverb \\
You use \textbf{away} to show that there has been a change or development from one state or situation to another.
 \textit{
	\begin{itemize}
	\item British courts are increasingly moving away from sending young offenders to prison.
	\item There's been a dramatic shift away from traditional careers towards business and
commerce.
	\end{itemize}
}
\item adverb \\
You can use \textbf{away} to emphasize a continuous or repeated  action .
 \textit{
	\begin{itemize}
	\item He would often be working away on his computer late into the night.
	\item She sighed, her heart banging away against her ribs as she opened the door.
	\end{itemize}
}
\item adverb \\
You use \textbf{away} to show that something is removed .
 \textit{
	\begin{itemize}
	\item If you take my work away I can't be happy anymore.
	\item The waitress whipped the plate away and put down my bill.
	\item Weeks of heavy rain have washed away roads and bridges.
	\end{itemize}
}
\end{enumerate}

\section*{drama}
{\large \color{blue}  dramas  }
\subsection*{Explain}
\begin{enumerate}
\item countable noun \\
A \textbf{drama} is a serious play for the theatre , television, or radio.
 \textit{
	\begin{itemize}
	\item He acted in radio dramas.
	\end{itemize}
}
\item uncountable noun \\
You use \textbf{drama} to refer to plays in general or to work that is connected with plays and the theatre, such as acting or producing.
 \textit{
	\begin{itemize}
	\item He knew nothing of Greek drama.
	\item She met him when she was at drama school.
	\end{itemize}
}
\item variable noun \\
You can refer to a real situation which is exciting or distressing as \textbf{drama} .
 \textit{
	\begin{itemize}
	\item There was none of the drama and relief of a hostage release.
	\item For all its drama, the event was not unexpected.
	\end{itemize}
}
\end{enumerate}

\section*{besides}
{\large \color{blue}  }
\subsection*{Explain}
\begin{enumerate}
\item preposition \\
\textbf{Besides} something or \textbf{beside} something means in addition to it.
 \textbf{Besides} is also an adverb .
 \textit{
	\begin{itemize}
	\item I think she has many good qualities besides being very beautiful.
	\item There was only one person besides Ford who knew Julia Jameson.
	\item You get to sample lots of baked things and take home masses of cookies besides.
	\end{itemize}
}
\item adverb \\
\textbf{Besides} is used to emphasize an additional point that you are making, especially one that you consider to be important.
 \textit{
	\begin{itemize}
	\item The house was too expensive and too big. Besides, I'd grown fond of our little rented
house.
	\end{itemize}
}
\end{enumerate}

\section*{exhibition}
{\large \color{blue}  exhibitions  }
\subsection*{Explain}
\begin{enumerate}
\item countable noun \\
An \textbf{exhibition} is a public event at which pictures , sculptures , or other objects of interest are displayed, for example at a museum or art gallery .
 \textit{
	\begin{itemize}
	\item ...an exhibition of expressionist art.
	\item ...an exhibition on the natural history of the area.
	\end{itemize}
}
\item singular noun \\
An \textbf{exhibition of} a particular skilful activity is a display or example of it that people notice or admire .
 \textit{
	\begin{itemize}
	\item He responded in champion's style by treating the fans to an exhibition of power and
speed.
	\end{itemize}
}
\end{enumerate}

\section*{downstairs}
{\large \color{blue}  }
\subsection*{Explain}
\begin{enumerate}
\item adverb \\
If you go  \textbf{downstairs} in a building, you go down a staircase towards the ground floor.
 \textit{
	\begin{itemize}
	\item Denise went downstairs and made some tea.
	\end{itemize}
}
\item adverb \\
If something or someone is \textbf{downstairs} in a building, they are on the ground floor or on a lower floor than you.
 \textit{
	\begin{itemize}
	\item The telephone was downstairs in the entrance hall.
	\item Everybody was downstairs watching a movie.
	\item ...the woman who lives in the flat downstairs.
	\end{itemize}
}
\item adjective \\
\textbf{Downstairs} means situated on the ground floor of a building or on a lower floor than you are.
 \textit{
	\begin{itemize}
	\item She repainted the downstairs rooms and closed off the second floor.
	\end{itemize}
}
\item singular noun \\
\textbf{The downstairs} of a building is its lower floor or floors.
 \textit{
	\begin{itemize}
	\item The downstairs of the two little houses had been entirely refashioned.
	\end{itemize}
}
\end{enumerate}

\section*{expectation}
{\large \color{blue}  expectations  }
\subsection*{Explain}
\begin{enumerate}
\item plural noun \\
Your \textbf{expectations} are your strong hopes or beliefs that something will happen or that you will get something that you want .
 \textit{
	\begin{itemize}
	\item Students' expectations were as varied as their expertise.
	\item The car has been General Motors' most visible success story, with sales far exceeding
expectations.
	\item The Chancellor's statement lowers expectations of an early election.
	\item Contrary to general expectation, he announced that all four had given their approval.
	\end{itemize}
}
\item countable noun \\
A person's \textbf{expectations} are strong beliefs which they have about the proper way someone should behave or something should happen.
 \textit{
	\begin{itemize}
	\item Stephen Chase had determined to live up to the expectations of the Company.
	\item ...the expectation that the grieving process should have a time limit on it.
	\end{itemize}
}
\end{enumerate}

\section*{downtown}
{\large \color{blue}  }
\subsection*{Explain}
\begin{enumerate}
\item adjective \\
\textbf{Downtown} places are in or towards the centre of a large town or city, where the shops and places of business are.
 \textbf{Downtown} is also an adverb .
 \textbf{Downtown} is also a noun .
 \textit{
	\begin{itemize}
	\item ...an office in downtown Chicago.
	\item By day he worked downtown for American Standard.
	\item You have to be downtown in a hurry.
	\item ...in a large vacant area of the downtown.
	\item ...Ardastra Gardens, a short taxi ride from downtown.
	\end{itemize}
}
\end{enumerate}

\section*{fan}
{\large \color{blue}  fans  fanning  fanned  }
\subsection*{Explain}
\begin{enumerate}
\item countable noun \\
If you are a \textbf{fan} of someone or something, especially a famous person or a sport, you like them very much and are very interested in them.
 \textit{
	\begin{itemize}
	\item They are both ardent Elvis fans.
	\item As a boy he was a Manchester United fan.
	\item I am a great fan of rave music.
	\end{itemize}
}
\item countable noun \\
A \textbf{fan} is a flat object that you hold in your hand and wave in order to move the air and make yourself feel  cooler .
 \textit{
	\begin{itemize}
	\end{itemize}
}
\item verb \\
If you \textbf{fan} yourself or your face when you are hot , you wave a fan or other flat object in order to make yourself feel cooler.
 \textit{
	\begin{itemize}
	\item She would have to wait in the truck, fanning herself with a piece of cardboard.
	\item Mo kept bringing me out refreshments and fanning me as it was that hot.
	\end{itemize}
}
\item countable noun \\
A \textbf{fan} is a piece of electrical or mechanical  equipment with blades that go round and round. It keeps a room or machine cool or gets  rid of unpleasant  smells .
 \textit{
	\begin{itemize}
	\item He cools himself in front of an electric fan.
	\item ...an extractor fan.
	\end{itemize}
}
\item countable noun \\
You can describe anything that has the shape of a wide ' V ' with a curved part above it as a \textbf{fan} .
 \textit{
	\begin{itemize}
	\item ...its fan of tail feathers.
	\item ...a conservatory with an ornate fan-shaped roof.
	\end{itemize}
}
\item verb \\
If you \textbf{fan} a fire, you wave something flat next to it in order to make it burn more strongly. If a wind \textbf{fans} a fire, it blows on it and makes it burn more strongly.
 \textit{
	\begin{itemize}
	\item Kneeling in front of the open hearth, Maria was fanning the smoldering fire.
	\item During the afternoon, hot winds fan the flames.
	\end{itemize}
}
\item verb \\
If someone \textbf{fans} an emotion such as fear , hatred , or passion , they deliberately do things to make people feel the emotion more strongly.
 \textit{
	\begin{itemize}
	\item He said students were fanning social unrest with their violent protests.
	\item ...economic problems which often fan hatred.
	\end{itemize}
}
\end{enumerate}

\section*{elsewhere}
{\large \color{blue}  }
\subsection*{Explain}
\begin{enumerate}
\item adverb \\
\textbf{Elsewhere} means in other places or to another place.
 \textit{
	\begin{itemize}
	\item Almost 80 percent of the state's residents were born elsewhere.
	\item They were living rather well, in comparison with people elsewhere in the world.
	\item But if you are not satisfied then go elsewhere.
	\item Until the doctor arrived from elsewhere on the ward, Amy was in charge.
	\end{itemize}
}
\end{enumerate}

\section*{globe}
{\large \color{blue}  globes  }
\subsection*{Explain}
\begin{enumerate}
\item singular noun \\
You can refer to the world as \textbf{the}  \textbf{globe} when you are emphasizing how big it is or that something happens in many different parts of it.
 \textit{
	\begin{itemize}
	\item ...bottles of beer from every corner of the globe.
	\item 70% of our globe's surface is water.
	\end{itemize}
}
\item countable noun \\
A \textbf{globe} is a ball-shaped object with a map of the world on it. It is usually fixed on a stand .
 \textit{
	\begin{itemize}
	\item ...a globe of the world.
	\item Three large globes stand on the floor.
	\end{itemize}
}
\item countable noun \\
Any ball-shaped object can be referred to as a \textbf{globe} .
 \textit{
	\begin{itemize}
	\item The overhead light was covered now with a white globe.
	\end{itemize}
}
\end{enumerate}

\section*{heir}
{\large \color{blue}  heirs  }
\subsection*{Explain}
\begin{enumerate}
\item countable noun \\
An \textbf{heir} is someone who has the right to inherit a person's money, property, or title when that person dies.
 \textit{
	\begin{itemize}
	\item His heir, Lord Doune, cuts a bit of a dash in the city.
	\item ...the heir to the throne.
	\end{itemize}
}
\end{enumerate}

\section*{forth}
{\large \color{blue}  }
\subsection*{Explain}
\begin{enumerate}
\item adverb \\
When someone goes  \textbf{forth} from a place, they leave it.
 \textit{
	\begin{itemize}
	\item Go forth into the desert.
	\item I came forth to take the air.
	\end{itemize}
}
\item adverb \\
If one thing brings  \textbf{forth} another, the first thing produces the second .
 \textit{
	\begin{itemize}
	\item Nature herself brings forth new forms of life.
	\item My reflections brought forth no conclusion.
	\end{itemize}
}
\item adverb \\
When someone or something is brought \textbf{forth} , they are brought to a place or moved into a position where people can see them.
 \textit{
	\begin{itemize}
	\item Pilate ordered Jesus to be brought forth.
	\item He brought forth a small gold amulet from beneath his robe.
	\end{itemize}
}
\end{enumerate}

\section*{inflation}
{\large \color{blue}  }
\subsection*{Explain}
\begin{enumerate}
\item uncountable noun \\
\textbf{Inflation} is a general increase in the prices of goods and services in a country .
 \textit{
	\begin{itemize}
	\item ...rising unemployment and high inflation.
	\item ...an inflation rate of only 2.2%.
	\end{itemize}
}
\end{enumerate}

\section*{furthermore}
{\large \color{blue}  }
\subsection*{Explain}
\begin{enumerate}
\item adverb \\
\textbf{Furthermore} is used to introduce a piece of information or opinion that adds to or supports the previous one.
 \textit{
	\begin{itemize}
	\item Furthermore, they claim that any such interference is completely ineffective.
	\end{itemize}
}
\end{enumerate}

\section*{information}
{\large \color{blue}  }
\subsection*{Explain}
\begin{enumerate}
\item uncountable noun \\
\textbf{Information} about someone or something consists of facts about them.
 \textit{
	\begin{itemize}
	\item Pat refused to give her any information about Sarah.
	\item Each centre would provide information on technology and training.
	\item For further information contact the number below.
	\item ...an important piece of information.
	\item The information was passed on to another government department.
	\end{itemize}
}
\item uncountable noun \\
\textbf{Information} consists of the facts and figures that are stored and used by a computer program .
 \textit{
	\begin{itemize}
	\item Pictures are scanned into a form of digital information that computers can recognize.
	\end{itemize}
}
\item uncountable noun \\
\textbf{Information} is a service which you can call to find out someone's phone number.
 \textit{
	\begin{itemize}
	\end{itemize}
}
\end{enumerate}

\section*{here}
{\large \color{blue}  }
\subsection*{Explain}
\begin{enumerate}
\item adverb \\
You use \textbf{here} when you are referring to the place where you are.
 \textit{
	\begin{itemize}
	\item I'm here all by myself and I know I'm going to get lost.
	\item Well, I can't stand here chatting all day.
	\item ...the growing number of skiers that come here.
	\item Sheila was in here a minute ago.
	\item My name is Roseanne and I'm in here for shoplifting.
	\item I'm not going to stay here. I'm out of here, back down to San Diego.
	\item When Mommy comes, just tell her I'm up here.
	\end{itemize}
}
\item adverb \\
You use \textbf{here} when you are pointing towards a place that is near you, in order to draw someone else's attention to it.
 \textit{
	\begin{itemize}
	\item ...if you will just sign here.
	\item Come and sit here, Lauren.
	\item 'From there, pulling a line to here,' he said, making invisible drawings in the air.
	\item 'It's on the right-hand side of the shopping centre.'—'Okay. Fine.'—'Oh it's here.'
	\end{itemize}
}
\item adverb \\
You use \textbf{here} in order to indicate that the person or thing that you are talking about is near you or is being held by you.
 \textit{
	\begin{itemize}
	\item My friend here writes for radio.
	\item I have here at my side Mr. Glenn Williams.
	\item I have a little book here by new writer.
	\end{itemize}
}
\item adverb \\
You use \textbf{here} to refer to people in general and their life on Earth .
 \textit{
	\begin{itemize}
	\item ...where we have come from, where we are going to, or what our purpose here is, if
any.
	\item Who are we? What are we doing here?
	\end{itemize}
}
\item adverb \\
If you say that you are \textbf{here}  \textbf{to} do something, that is your role or function .
 \textit{
	\begin{itemize}
	\item I'm here to help you.
	\item I'm not here to listen to your complaints.
	\end{itemize}
}
\item adverb \\
You use \textbf{here} in order to draw attention to something or someone who has just arrived in the place where you are, or to draw attention to the place you have just arrived
at.
 \textit{
	\begin{itemize}
	\item 'Here's the taxi,' she said politely.
	\item 'Mr Cummings is here,' she said, holding the door open.
	\item Here comes your husband.
	\item 'Okay, here we are,' she said, and inserted her key in the lock.
	\item Here's my apartment.
	\end{itemize}
}
\item adverb \\
You use \textbf{here} to refer to a particular point or stage of a situation or subject that you have come to or that you are dealing with.
 \textit{
	\begin{itemize}
	\item Both sides will have to sell the agreement. It's here that the real test will come.
	\item It's here that we come up against the difference of approach.
	\item The book goes into recent work in greater detail than I have attempted here.
	\item Here I think it is appropriate to draw your attention to one specific feature.
	\end{itemize}
}
\item adverb \\
You use \textbf{here} to refer to a period of time, a situation, or an event that is present or happening  now .
 \textit{
	\begin{itemize}
	\item Here comes the summer.
	\item Economic recovery is here.
	\item Here is your opportunity to acquire a luxurious one bedroom home.
	\end{itemize}
}
\item adverb \\
You use \textbf{here} at the beginning of a sentence in order to draw attention to something or to introduce something.
 \textit{
	\begin{itemize}
	\item Here is a summer soup that is almost a meal in itself.
	\item Now here's what I want you to do.
	\item So here's what I think.
	\end{itemize}
}
\item adverb \\
You use \textbf{here} when you are offering or giving something to someone.
 \textit{
	\begin{itemize}
	\item Here's your coffee, just the way you like it.
	\item Here's my card. You know where to find me.
	\item Here's some letters I want you to sign.
	\item Here's your cash.
	\end{itemize}
}
\item  \\
 here sb is \textit{
	\begin{itemize}
	\end{itemize}
}
\item  \\
 here we are \textit{
	\begin{itemize}
	\end{itemize}
}
\item  \\
 here goes \textit{
	\begin{itemize}
	\end{itemize}
}
\item  \\
 here we go again \textit{
	\begin{itemize}
	\end{itemize}
}
\item  \\
 here and now \textit{
	\begin{itemize}
	\end{itemize}
}
\item  \\
 here and there \textit{
	\begin{itemize}
	\end{itemize}
}
\item  \\
 here's to sth \textit{
	\begin{itemize}
	\end{itemize}
}
\end{enumerate}

\section*{jury}
{\large \color{blue}  juries  }
\subsection*{Explain}
\begin{enumerate}
\item countable noun \\
In a court of law, the \textbf{jury} is the group of people who have been chosen from the general public to listen to the facts about a crime and to decide whether the person accused is guilty or not.
 \textit{
	\begin{itemize}
	\item The jury convicted Mr Hampson of all offences.
	\item ...the tradition of trial by jury.
	\end{itemize}
}
\item countable noun \\
A \textbf{jury} is a group of people who choose the winner of a competition.
 \textit{
	\begin{itemize}
	\item I am not surprised that the Booker Prize jury included it on their shortlist.
	\end{itemize}
}
\item  \\
 the jury is out \textit{
	\begin{itemize}
	\end{itemize}
}
\end{enumerate}

\section*{home}
{\large \color{blue}  homes  }
\subsection*{Explain}
\begin{enumerate}
\item countable noun \\
Someone's \textbf{home} is the house or flat where they live .
 \textit{
	\begin{itemize}
	\item Last night they stayed at home and watched TV.
	\item ...his home in Hampstead.
	\item ...the allocation of land for new homes.
	\end{itemize}
}
\item uncountable noun \\
You can use \textbf{home} to refer in a general way to the house, town, or country where someone lives now or where
they were born, often to emphasize that they feel they belong in that place.
 \textit{
	\begin{itemize}
	\item She gives frequent performances of her work, both at home and abroad.
	\item His father worked away from home for much of Jim's first five years.
	\item At seventeen, Daniele was told to leave home by her father.
	\item Ms Highsmith has made Switzerland her home.
	\item Warwick is home to some 550 international students.
	\item Brian decided to leave the U.K. and set up home in Southern Spain.
	\item He has moved back to his home town of Miami.
	\end{itemize}
}
\item adverb \\
\textbf{Home} means to or at the place where you live.
 \textit{
	\begin{itemize}
	\item Hannah wasn't feeling too well and she wanted to go home.
	\item I'll telephone you as soon as I get home.
	\item Hi, Mom, I'm home!
	\item Company officials say striking union members should stay home today.
	\end{itemize}
}
\item adjective \\
\textbf{Home} means made or done in the place where you live.
 \textit{
	\begin{itemize}
	\item ...cheap but healthy home cooking.
	\item ...internet home shopping and grocery delivery.
	\end{itemize}
}
\item adjective \\
\textbf{Home} means relating to your own country as opposed to foreign countries.
 \textit{
	\begin{itemize}
	\item Europe's software companies still have a growing home market.
	\item ...the Guardian's home news pages.
	\end{itemize}
}
\item countable noun \\
A \textbf{home} is a large house or institution where a number of people live and are looked after, instead of living in their own houses or flats. They usually live there because they are
too old or ill to look after themselves or for their families to care for them.
 \textit{
	\begin{itemize}
	\item It's going to be a home for vulnerable adults.
	\item ...an old people's home.
	\end{itemize}
}
\item countable noun \\
You can refer to a family unit as a \textbf{home} .
 \textit{
	\begin{itemize}
	\item She had, at any rate, provided a peaceful and loving home for Harriet.
	\item Single-parent homes are commonplace.
	\end{itemize}
}
\item singular noun \\
If you refer to the \textbf{home}  \textbf{of} something, you mean the place where it began or where it is most typically found.
 \textit{
	\begin{itemize}
	\item This south-west region of France is the home of claret.
	\end{itemize}
}
\item countable noun \\
If you find a \textbf{home}  \textbf{for} something, you find a place where it can be kept.
 \textit{
	\begin{itemize}
	\item The equipment itself is getting smaller, neater and easier to find a home for.
	\end{itemize}
}
\item adverb \\
If you press , drive, or hammer something \textbf{home} , you explain it to people as forcefully as possible .
 \textit{
	\begin{itemize}
	\item It is now up to all of us to debate this issue and press home the argument.
	\end{itemize}
}
\item uncountable noun \\
When a sports team plays \textbf{at}  \textbf{home} , they play a game on their own ground, rather than on the opposing team's ground.
 \textbf{Home} is also an adjective .
 \textit{
	\begin{itemize}
	\item I scored in both games against Barcelona; we drew at home and beat them away.
	\item All three are Chelsea fans, and attend all home games together.
	\end{itemize}
}
\item  \\
 at home \textit{
	\begin{itemize}
	\end{itemize}
}
\item  \\
 bring sth home \textit{
	\begin{itemize}
	\end{itemize}
}
\item  \\
 home and dry \textit{
	\begin{itemize}
	\end{itemize}
}
\item  \\
 to hit home \textit{
	\begin{itemize}
	\end{itemize}
}
\item  \\
 a home from home \textit{
	\begin{itemize}
	\end{itemize}
}
\item  \\
 make yourself at home \textit{
	\begin{itemize}
	\end{itemize}
}
\item  \\
 nothing to write home about \textit{
	\begin{itemize}
	\end{itemize}
}
\item  \\
 to strike home \textit{
	\begin{itemize}
	\end{itemize}
}
\end{enumerate}

\section*{notice}
{\large \color{blue}  notices  noticing  noticed  }
\subsection*{Explain}
\begin{enumerate}
\item verb \\
If you \textbf{notice} something or someone, you become aware of them.
 \textit{
	\begin{itemize}
	\item People should not hesitate to contact the police if they've noticed anyone acting
suspiciously.
	\item I noticed that most academics were writing papers during the summer.
	\item Luckily, I'd noticed where you left the car.
	\item Mrs Shedden noticed a bird sitting on the garage roof.
	\item She needn't worry that he'll think she looks a mess. He won't notice.
	\end{itemize}
}
\item countable noun \\
A \textbf{notice} is a written announcement in a place where everyone can read it.
 \textit{
	\begin{itemize}
	\item There were no new notices on the board.
	\item A few guest houses had 'No Vacancies' notices in their windows.
	\item ...a notice which said 'Beware Flooding'.
	\end{itemize}
}
\item uncountable noun \\
If you give \textbf{notice} about something that is going to happen , you give a warning in advance that it is going to happen.
 \textit{
	\begin{itemize}
	\item Interest is paid monthly. Three months' notice is required for withdrawals.
	\item Unions are required to give seven days' notice of industrial action.
	\item She was transferred without notice.
	\end{itemize}
}
\item countable noun \\
A \textbf{notice} is a formal announcement in a newspaper or magazine about something that has happened or is going to happen.
 \textit{
	\begin{itemize}
	\item I rang The Globe with news of Blake's death, and put notices in the personal column
of The Times.
	\item The request is published in notices in today's national newspapers.
	\end{itemize}
}
\item countable noun \\
A \textbf{notice} is one of a number of letters that are similar or exactly the same which an organization  sends to people in order to give them information or ask them to do something.
 \textit{
	\begin{itemize}
	\item Bonus notices were issued each year from head office to local agents.
	\item Notices will be circulated to all known creditors.
	\end{itemize}
}
\item countable noun \\
A \textbf{notice} is a written article in a newspaper or magazine in which someone gives their opinion of a play, film , or concert .
 \textit{
	\begin{itemize}
	\item Nevertheless, it's good to know you've had good notices, even if you don't read them.
	\end{itemize}
}
\item  \\
 at short notice \textit{
	\begin{itemize}
	\end{itemize}
}
\item  \\
 bring to sb's notice \textit{
	\begin{itemize}
	\end{itemize}
}
\item  \\
 come to sb's notice \textit{
	\begin{itemize}
	\end{itemize}
}
\item  \\
 escape sb's notice \textit{
	\begin{itemize}
	\end{itemize}
}
\item  \\
 until further notice \textit{
	\begin{itemize}
	\end{itemize}
}
\item  \\
 to give notice \textit{
	\begin{itemize}
	\end{itemize}
}
\item  \\
 hand in one's notice/give in one's notice \textit{
	\begin{itemize}
	\end{itemize}
}
\item  \\
 take notice \textit{
	\begin{itemize}
	\end{itemize}
}
\item  \\
 take no notice \textit{
	\begin{itemize}
	\end{itemize}
}
\end{enumerate}

\section*{inward}
{\large \color{blue}  }
\subsection*{Explain}
\begin{enumerate}
\item adjective \\
Your \textbf{inward}  thoughts or feelings are the ones that you do not express or show to other people.
 \textit{
	\begin{itemize}
	\item I sighed with inward relief.
	\item ...a glow of inward satisfaction.
	\end{itemize}
}
\item adjective \\
An \textbf{inward} movement is one towards the inside or centre of something.
 \textit{
	\begin{itemize}
	\item ...a sharp, inward breath like a gasp.
	\item The athlete takes off from one leg from an inward twist.
	\end{itemize}
}
\end{enumerate}

\section*{passage}
{\large \color{blue}  passages  }
\subsection*{Explain}
\begin{enumerate}
\item countable noun \\
A \textbf{passage} is a long narrow space with walls or fences on both sides, which connects one place or room with another.
 \textit{
	\begin{itemize}
	\item Harry stepped into the passage and closed the door behind him.
	\item ...up some stairs and along a narrow passage towards a door.
	\end{itemize}
}
\item countable noun \\
A \textbf{passage} in a book, speech, or piece of music is a section of it that you are considering
separately from the rest .
 \textit{
	\begin{itemize}
	\item He reads a passage from Milton.
	\item ...the passage in which Blake spoke of the world of imagination.
	\item Compare the following passages.
	\end{itemize}
}
\item countable noun \\
A \textbf{passage} is a long narrow hole or tube in your body, which air or liquid can pass along.
 \textit{
	\begin{itemize}
	\item ...cells that line the air passages.
	\item ...blocked nasal passages.
	\end{itemize}
}
\item countable noun \\
A \textbf{passage}  \textbf{through} a crowd of people or things is an empty space that allows you to move through them.
 \textit{
	\begin{itemize}
	\item He cleared a passage for himself through the crammed streets.
	\item Two men suddenly elbowed a passage through the shoppers.
	\end{itemize}
}
\item uncountable noun \\
The \textbf{passage} of someone or something is their movement from one place to another.
 \textit{
	\begin{itemize}
	\item Germany had not requested Franco's consent for the passage of troops through Spain.
	\item The passage of heat through rock is extremely slow.
	\end{itemize}
}
\item uncountable noun \\
The \textbf{passage} of someone or something is their progress from one situation or one stage in their development to another.
 \textit{
	\begin{itemize}
	\item ...to ease their passage to a market economy.
	\item ...the passage from school to college.
	\end{itemize}
}
\item uncountable noun \\
The \textbf{passage}  \textbf{of} a bill or act is the official acceptance of it by a parliament or legislature .
 \textit{
	\begin{itemize}
	\item It's been 200 years since the passage of the Bill of Rights.
	\end{itemize}
}
\item singular noun \\
\textbf{The passage of} a period of time is its passing.
 \textit{
	\begin{itemize}
	\item ...an asset that increases in value with the passage of time.
	\item ...after the passage of eighteen months.
	\end{itemize}
}
\item countable noun \\
A \textbf{passage} is a journey by ship.
 \textit{
	\begin{itemize}
	\item We'd arrived the day before after a 10-hour passage from Swansea.
	\end{itemize}
}
\item uncountable noun \\
If you are granted  \textbf{passage} through a country or area of land, you are given permission to go through it.
 \textit{
	\begin{itemize}
	\item He would be given safe passage to and from the capital.
	\item You may have free passage across our territory whenever you require it.
	\end{itemize}
}
\end{enumerate}

\section*{later}
{\large \color{blue}  }
\subsection*{Explain}
\begin{enumerate}
\item  \\
\textbf{Later} is the comparative of late .
 \textit{
	\begin{itemize}
	\end{itemize}
}
\item adverb \\
You use \textbf{later} to refer to a time or situation that is after the one that you have been talking about or after the present one.
 \textit{
	\begin{itemize}
	\item He resigned ten years later.
	\item I'll join you later.
	\item Burke later admitted he had lied.
	\end{itemize}
}
\item adjective \\
You use \textbf{later} to refer to an event , period of time, or other thing which comes after the one that you have been talking about or after the present one.
 \textit{
	\begin{itemize}
	\item At a later news conference, he said differences should not be dramatized.
	\item The competition should have been re-scheduled for a later date.
	\item A later report said the oil fire on the sea was out.
	\end{itemize}
}
\item adjective \\
You use \textbf{later} to refer to the last part of someone's life or career or the last part of a period of history .
 \textit{
	\begin{itemize}
	\item He found happiness in later life.
	\item In his later years he wrote very little.
	\item ...the later part of the 20th century.
	\end{itemize}
}
\end{enumerate}

\section*{precaution}
{\large \color{blue}  precautions  }
\subsection*{Explain}
\begin{enumerate}
\item countable noun \\
A \textbf{precaution} is an action that is intended to prevent something dangerous or unpleasant from happening .
 \textit{
	\begin{itemize}
	\item Could he not, just as a precaution, move to a place of safety?
	\item I had taken the precaution of doing a little research before I left London.
	\item Extra safety precautions are essential in homes where older people live.
	\end{itemize}
}
\end{enumerate}

\section*{likewise}
{\large \color{blue}  }
\subsection*{Explain}
\begin{enumerate}
\item adverb \\
You use \textbf{likewise} when you are comparing two methods, states , or situations and saying that they are similar.
 \textit{
	\begin{itemize}
	\item All attempts by the Socialists to woo him back were spurned. Similar overtures from
the right have likewise been rejected.
	\item The V2 was not an ordinary weapon: it could only be used against cities. Likewise
the atom bomb.
	\end{itemize}
}
\item adverb \\
If you do something and someone else does \textbf{likewise} , they do the same or a similar thing.
 \textit{
	\begin{itemize}
	\item He lent money, made donations and encouraged others to do likewise.
	\end{itemize}
}
\end{enumerate}

\section*{prey}
{\large \color{blue}  preys  preying  preyed  }
\subsection*{Explain}
\begin{enumerate}
\item uncountable noun \\
A creature's \textbf{prey} are the creatures that it hunts and eats in order to live .
 \textit{
	\begin{itemize}
	\item Electric rays stun their prey with huge electrical discharges.
	\item These animals were the prey of hyenas.
	\end{itemize}
}
\item verb \\
A creature that \textbf{preys on} other creatures lives by catching and eating them.
 \textit{
	\begin{itemize}
	\item The effect was to disrupt the food chain, starving many animals and those that preyed
on them.
	\item The larvae prey upon small aphids.
	\end{itemize}
}
\item uncountable noun \\
You can refer to the people who someone tries to harm or trick as their \textbf{prey} .
 \textit{
	\begin{itemize}
	\item Police officers lie in wait for the gangs who stalk their prey at night.
	\end{itemize}
}
\item verb \\
If someone \textbf{preys on} other people, especially people who are unable to protect themselves, they take advantage of them or harm them in some way.
 \textit{
	\begin{itemize}
	\item The survey claims loan companies prey on weak families already in debt.
	\end{itemize}
}
\item verb \\
If something \textbf{preys on} your mind, you cannot stop  thinking and worrying about it.
 \textit{
	\begin{itemize}
	\item The absence of children at Christmas preyed on Liz's mind.
	\item He had been unwise and it preyed on his conscience.
	\end{itemize}
}
\item uncountable noun \\
If someone is \textbf{prey to} something bad , they have a tendency to let themselves be affected by it.
 \textit{
	\begin{itemize}
	\item He was prey to a growing despair.
	\item You were both a prey to compulsions.
	\end{itemize}
}
\item  \\
 fall prey to something \textit{
	\begin{itemize}
	\end{itemize}
}
\end{enumerate}

\section*{never}
{\large \color{blue}  }
\subsection*{Explain}
\begin{enumerate}
\item adverb \\
\textbf{Never} means at no time in the past or at no time in the future .
 \textit{
	\begin{itemize}
	\item I have never lost the weight I put on in my teens.
	\item Never had he been so free of worry.
	\item That was a mistake. We'll never do it again.
	\item Never say that. Never, do you hear?
	\item He was never really healthy.
	\item This is never to happen again.
	\end{itemize}
}
\item adverb \\
\textbf{Never} means 'not in any circumstances at all'.
 \textit{
	\begin{itemize}
	\item I would never do anything to hurt him.
	\item Even if you are desperate to get married, never let it show.
	\item Divorce is never easy for children.
	\item The golden rule is never to clean a valuable coin.
	\end{itemize}
}
\item  \\
 never ever \textit{
	\begin{itemize}
	\end{itemize}
}
\item adverb \\
\textbf{Never} is used to refer to the past and means 'not'.
 \textit{
	\begin{itemize}
	\item He never achieved anything.
	\item He waited until all the luggage was cleared, but Paula's never appeared.
	\item I never knew the lad.
	\item I'd never have dreamt of doing such a thing.
	\end{itemize}
}
\item exclamation \\
You say ' \textbf{never!} ' to indicate how surprised or shocked you are by something that someone has just said .
 \textit{
	\begin{itemize}
	\end{itemize}
}
\item exclamation \\
You say ' \textbf{Well, I never} ' to indicate that you are very surprised about something that you have just seen or found out.
 \textit{
	\begin{itemize}
	\item 'What were you up to there?'—'I was head of the information department.'—'Well I
never!'
	\end{itemize}
}
\item  \\
 will never do/would never do \textit{
	\begin{itemize}
	\end{itemize}
}
\end{enumerate}

\section*{prophet}
{\large \color{blue}  prophets  }
\subsection*{Explain}
\begin{enumerate}
\item countable noun \\
A \textbf{prophet} is a person who is believed to be chosen by God to say the things that God wants to tell people.
 \textit{
	\begin{itemize}
	\item ...the sacred name of the Holy Prophet of Islam.
	\end{itemize}
}
\item countable noun \\
A \textbf{prophet} is someone who predicts that something will happen in the future.
 \textit{
	\begin{itemize}
	\item I promised myself I'd defy all the prophets of doom and battle back to fitness.
	\end{itemize}
}
\end{enumerate}

\section*{nowhere}
{\large \color{blue}  }
\subsection*{Explain}
\begin{enumerate}
\item adverb \\
You use \textbf{nowhere} to emphasize that a place has more of a particular  quality than any other places, or that it is the only place where something happens or exists .
 \textit{
	\begin{itemize}
	\item Nowhere is language a more serious issue than in Hawaii.
	\item This kind of forest exists nowhere else in the world.
	\item For many birdwatchers, there is nowhere better than Scotland.
	\end{itemize}
}
\item adverb \\
You use \textbf{nowhere} when making negative  statements to say that a suitable place of the specified  kind does not exist.
 \textit{
	\begin{itemize}
	\item There was nowhere to hide and nowhere to run.
	\item I have nowhere else to go, nowhere in the world.
	\item He had nowhere to call home.
	\end{itemize}
}
\item adverb \\
You use \textbf{nowhere} to indicate that something or someone cannot be seen or found .
 \textit{
	\begin{itemize}
	\item Michael glanced anxiously down the corridor, but Wilfred was nowhere to be seen.
	\item The escaped prisoner was nowhere in sight.
	\item The gate was locked and the guards were nowhere.
	\end{itemize}
}
\item adverb \\
You can use \textbf{nowhere} to refer in a general  way to small, unimportant , or uninteresting places.
 \textit{
	\begin{itemize}
	\item ...endless paths that led nowhere in particular.
	\item ...country roads that go from nowhere to nowhere.
	\end{itemize}
}
\item adverb \\
If you say that something or someone appears  \textbf{from nowhere} or \textbf{out of nowhere} , you mean that they appear suddenly and unexpectedly.
 \textit{
	\begin{itemize}
	\item A car came from nowhere, and I had to jump back into the hedge just in time.
	\item Houses had sprung up out of nowhere on the hills.
	\end{itemize}
}
\item adverb \\
You use \textbf{nowhere} to mean not in any part of a text , speech , or argument .
 \textit{
	\begin{itemize}
	\item He nowhere offers concrete historical background to support his arguments.
	\item Point taken, but nowhere did we suggest that this yacht's features were unique.
	\item The most important issue for most ordinary people was nowhere on the proposed agenda.
	\end{itemize}
}
\item  \\
 in the middle of nowhere \textit{
	\begin{itemize}
	\end{itemize}
}
\item  \\
 be getting nowhere \textit{
	\begin{itemize}
	\end{itemize}
}
\item  \\
 nowhere near \textit{
	\begin{itemize}
	\end{itemize}
}
\end{enumerate}

\section*{puzzle}
{\large \color{blue}  puzzles  puzzling  puzzled  }
\subsection*{Explain}
\begin{enumerate}
\item verb \\
If something \textbf{puzzles} you, you do not understand it and feel  confused .
 \textit{
	\begin{itemize}
	\item My sister puzzles me and causes me anxiety.
	\end{itemize}
}
\item verb \\
If you \textbf{puzzle over} something, you try  hard to think of the answer to it or the explanation for it.
 \textit{
	\begin{itemize}
	\item I stayed up nights, puzzling over epic Old English poems in a fog of incomprehension.
	\end{itemize}
}
\item countable noun \\
A \textbf{puzzle} is a question, game, or toy which you have to think about carefully in order to answer
it correctly or put it together properly.
 \textit{
	\begin{itemize}
	\item ...a word puzzle.
	\end{itemize}
}
\item singular noun \\
You can  describe a person or thing that is hard to understand as \textbf{a puzzle} .
 \textit{
	\begin{itemize}
	\item Data from Voyager II has presented astronomers with a puzzle about why our outermost
planet exists.
	\item She was a puzzle.
	\end{itemize}
}
\end{enumerate}

\section*{outside}
{\large \color{blue}  outsides  }
\subsection*{Explain}
\begin{enumerate}
\item countable noun \\
The \textbf{outside} of something is the part which surrounds or encloses the rest of it.
 \textbf{Outside} is also an adjective .
 \textit{
	\begin{itemize}
	\item ...the outside of the building.
	\item Cook over a fairly high heat until the outsides are browned.
	\item ...high up on the outside wall.
	\end{itemize}
}
\item adverb \\
If you are \textbf{outside} , you are not inside a building but are quite close to it.
 \textbf{Outside} is also a preposition .
 \textbf{Outside} is also an adjective.
 \textit{
	\begin{itemize}
	\item 'Was the car inside the garage?'—'No, it was still outside.'.
	\item I stepped outside and pulled up my collar against the cold mist.
	\item Outside, the light was fading rapidly.
	\item The shouting outside grew louder.
	\item The victim was outside a shop when he was attacked.
	\item ...the outside temperature.
	\item ...an outside toilet.
	\end{itemize}
}
\item preposition \\
If you are \textbf{outside} a room , you are not in it but are in the passage or area next to it.
 \textbf{Outside} is also an adverb .
 \textit{
	\begin{itemize}
	\item She'd sent him outside the classroom.
	\item He stood in the narrow hallway just outside the door.
	\item They heard voices coming from outside in the corridor.
	\item She heard the dog on the landing outside.
	\end{itemize}
}
\item adjective \\
When you talk about the \textbf{outside} world, you are referring to things that happen or exist in places other than your own home or community .
 \textbf{Outside} is also an adverb.
 \textit{
	\begin{itemize}
	\item ...a side of Morris's character she hid carefully from the outside world.
	\item It's important to have outside interests.
	\item The scheme was good for the prisoners because it brought them outside into the community.
	\end{itemize}
}
\item preposition \\
People or things \textbf{outside} a country, town, or region are not in it.
 \textbf{Outside} is also a noun .
 \textit{
	\begin{itemize}
	\item ...an old castle outside Budapest.
	\item By broadcasting from 'pirate' radio ships based outside British territorial waters,
they avoided official regulation.
	\item ...theatres both in and outside London.
	\item Peace cannot be imposed from the outside by the United States or anyone else.
	\end{itemize}
}
\item adjective \\
On a road with two separate carriageways, the \textbf{outside} lanes are the ones which are closest to its centre. Compare  inside .
 \textbf{Outside} is also a noun.
 \textit{
	\begin{itemize}
	\item It was travelling in the outside lane at 78mph.
	\item ...coming up on the outside.
	\end{itemize}
}
\item adjective \\
\textbf{Outside} people or organizations are not part of a particular organization or group.
 \textbf{Outside} is also a preposition.
 \textit{
	\begin{itemize}
	\item The company now makes much greater use of outside consultants.
	\item ...church services given on Sundays by outside chaplains.
	\item He is hoping to recruit a chairman from outside the company.
	\end{itemize}
}
\item preposition \\
\textbf{Outside} a particular institution or field of activity means in other fields of activity or in general life.
 \textit{
	\begin{itemize}
	\item The condition is practically unknown outside psychiatry clinics.
	\item ...the largest merger ever to take place outside the oil industry.
	\end{itemize}
}
\item preposition \\
Something that is \textbf{outside} a particular range of things is not included within it.
 \textit{
	\begin{itemize}
	\item She is a beautiful boat, but way, way outside my price range.
	\item When Cathy sings about love, you feel that she's singing about something outside
her experience.
	\end{itemize}
}
\item preposition \\
Something that happens \textbf{outside} a particular period of time happens at a different time from the one mentioned .
 \textit{
	\begin{itemize}
	\item They are open outside normal daily banking hours.
	\item ...nor does it help if your job involves working outside normal office hours.
	\end{itemize}
}
\item prepositional phrase \\
\textbf{Outside of} is used to introduce the only thing or person that prevents your main statement from being completely true .
 \textit{
	\begin{itemize}
	\end{itemize}
}
\item  \\
 at the outside \textit{
	\begin{itemize}
	\end{itemize}
}
\end{enumerate}

\section*{quilt}
{\large \color{blue}  quilts  }
\subsection*{Explain}
\begin{enumerate}
\item countable noun \\
A \textbf{quilt} is a thin cover filled with feathers or some other warm, soft material, which you put over your blankets when you are in bed.
 \textit{
	\begin{itemize}
	\item ...an old patchwork quilt.
	\end{itemize}
}
\item countable noun \\
A \textbf{quilt} is the same as a duvet .
 \textit{
	\begin{itemize}
	\end{itemize}
}
\end{enumerate}

\section*{racket}
{\large \color{blue}  rackets  }
\subsection*{Explain}
\begin{enumerate}
\item singular noun \\
A \textbf{racket} is a loud unpleasant  noise .
 \textit{
	\begin{itemize}
	\item He makes such a racket I'm afraid he disturbs the neighbours.
	\item My dream was interrupted by the most awful racket coming through the walls.
	\item The racket of drills and electric saws went on past midnight.
	\end{itemize}
}
\item countable noun \\
You can refer to an illegal activity used to make money as a \textbf{racket} .
 \textit{
	\begin{itemize}
	\item A smuggling racket is killing thousands of exotic birds each year.
	\item Suspicious fans exposed the racket and police arrested a man last night.
	\end{itemize}
}
\item countable noun \\
A \textbf{racket} is an oval-shaped bat with strings across it. Rackets are used in tennis, squash , and badminton.
 \textit{
	\begin{itemize}
	\item Tennis rackets and balls are provided.
	\end{itemize}
}
\item uncountable noun \\
\textbf{Rackets} is a game which is similar to squash but which is played with a hard ball.
 \textit{
	\begin{itemize}
	\end{itemize}
}
\end{enumerate}

\section*{so}
{\large \color{blue}  }
\subsection*{Explain}
\begin{enumerate}
\item adverb \\
You use \textbf{so} to refer  back to something that has just been mentioned .
 \textit{
	\begin{itemize}
	\item 'Do you think that made much of a difference to the family?'—'I think so.'
	\item If you can't play straight, then say so.
	\item 'Is he the kind of man who can be as flexible as he needs to be?'—' Well, I hope
so.'
	\end{itemize}
}
\item adverb \\
You use \textbf{so} when you are saying that something which has just been said about one person or thing is also true of another one.
 \textit{
	\begin{itemize}
	\item I enjoy Ann's company and so does Martin.
	\item They had a wonderful time and so did I.
	\item The police arrived, and so did reporters and a photographer from the 'Journal'.
	\end{itemize}
}
\item conjunction \\
You use the structures  \textbf{as...so} and \textbf{just as...so} when you want to indicate that two events or situations are similar in some way .
 \textit{
	\begin{itemize}
	\item As computer systems become more sophisticated, so too do the methods of those who
exploit the technology.
	\item Just as John has changed, so has his wife.
	\item Just as the teacher leads in the classroom, so does the headteacher play a leadership
role in the school.
	\end{itemize}
}
\item adverb \\
If you say that a state of affairs  \textbf{is so} , you mean that it is the way it has been described .
 \textit{
	\begin{itemize}
	\item Gold has been a poor investment over the past 20 years, and will continue to be so.
	\item In those days English dances were taught at school, but that seems no longer to be
so.
	\item It is strange to think that he held strong views, but it must have been so.
	\end{itemize}
}
\item adverb \\
You can use \textbf{so} with actions and gestures to show a person how to do something, or to indicate the size , height , or length of something.
 \textit{
	\begin{itemize}
	\item Clasp the chain like so.
	\item ...holding the champagne glass with long red nails positioned just so.
	\end{itemize}
}
\item conjunction \\
You use \textbf{so} and \textbf{so that} to introduce the result of the situation you have just mentioned.
 \textit{
	\begin{itemize}
	\item I am not an emotional type and so cannot bring myself to tell him I love him.
	\item Duvet covers are usually made from cotton, so they can be easily washed.
	\item I was an only child, and so had no experience of large families.
	\item There was snow everywhere, so that the shape of things was difficult to identify.
	\end{itemize}
}
\item conjunction \\
You use \textbf{so} , \textbf{so that} , and \textbf{so as} to introduce the reason for doing the thing that you have just mentioned.
 \textit{
	\begin{itemize}
	\item Come to my suite so I can tell you all about this wonderful play I saw in Boston.
	\item He took her arm and hurried her upstairs so that they wouldn't be overheard.
	\item I was beginning to feel alarm, but kept it to myself so as not to worry our two friends.
	\end{itemize}
}
\item adverb \\
You can use \textbf{so} in stories and accounts to introduce the next event in a series of events or to suggest a connection between two events.
 \textit{
	\begin{itemize}
	\item The woman asked if he could perhaps mend her fences, and so he stayed.
	\item She was free for five whole days, from Christmas Eve. And so she would be going to
Charles, to join her family.
	\item I thought, 'Here's someone who'll understand me.' So I wrote to her.
	\item He said he'd like to meet Sharon. So I said all right.
	\item And so Christmas passed.
	\end{itemize}
}
\item adverb \\
You can use \textbf{so} in conversations to introduce a new  topic , or to introduce a question or comment about something that has been said.
 \textit{
	\begin{itemize}
	\item So how was your day?
	\item So you're a runner, huh?
	\item So as for your question, Miles, the answer still has to be no.
	\item So, as I said to you, natural medicine is also known as holistic medicine.
	\item And so, to answer your question, that's why your mother is disappointed.
	\item 'I didn't find him funny at all.'—'So you won't watch the show again then?'
	\item They cost a fortune so how have these motorbikes become a fashion statement?
	\end{itemize}
}
\item adverb \\
You can use \textbf{so} in conversations to show that you are accepting what someone has just said.
 \textit{
	\begin{itemize}
	\item 'It makes me feel, well, important.'—'And so you are.'
	\item 'You can't possibly use this word.'—'So I won't.'.
	\item 'You know who Diana was, Grandfather.'—'So I do!'
	\item 'Why, this is nothing but common vegetable soup!'—'So it is, madam.'
	\item 'The car, Annie,' said Max rather grimly.—'So okay, the car. What about it?'
	\end{itemize}
}
\item convention \\
You say ' \textbf{So?} ' and ' \textbf{So what?} ' to indicate that you think that something that someone has said is unimportant .
 \textit{
	\begin{itemize}
	\item 'My name's Bruno.'—'So?'
	\item 'You take a chance on the weather if you holiday in the U.K.'—'So what?'
	\item I enjoy someone telling me I'm wonderful, but part of me thinks, 'So what? You won't
say that tomorrow.'
	\end{itemize}
}
\item adverb \\
You can use \textbf{so} in front of adjectives and adverbs to emphasize the quality that they are describing.
 \textit{
	\begin{itemize}
	\item 'I am so afraid,' Francis thought.
	\item He was surprised they had married–they had seemed so different.
	\item What is so compromising about being an employee of the state?
	\end{itemize}
}
\item adverb \\
You can use \textbf{so...that} and \textbf{so...as} to emphasize the degree of something by mentioning the result or consequence of it.
 \textit{
	\begin{itemize}
	\item The tears were streaming so fast she could not see.
	\item The deal seems so attractive it would be ridiculous to say no.
	\item Frescoes are so familiar a feature of Italian churches that it is easy to take them
for granted.
	\item He's not so daft as to listen to rumours.
	\end{itemize}
}
\item  \\
 and so on \textit{
	\begin{itemize}
	\end{itemize}
}
\item  \\
 so much/so many \textit{
	\begin{itemize}
	\end{itemize}
}
\item  \\
 not so much \textit{
	\begin{itemize}
	\end{itemize}
}
\item  \\
 or so \textit{
	\begin{itemize}
	\end{itemize}
}
\end{enumerate}

\section*{reader}
{\large \color{blue}  readers  }
\subsection*{Explain}
\begin{enumerate}
\item countable noun \\
The \textbf{readers} of a newspaper , magazine , or book are the people who read it.
 \textit{
	\begin{itemize}
	\item If you are a regular reader of Homes & Gardens you will know what an invaluable source
of inspiration it is.
	\item These texts give the reader an insight into the Chinese mind.
	\end{itemize}
}
\item countable noun \\
A \textbf{reader} is a person who reads, especially one who reads for pleasure .
 \textit{
	\begin{itemize}
	\item Thanks to that job I became an avid reader.
	\item Their books are loved by young readers the world over.
	\end{itemize}
}
\item countable noun \\
A \textbf{reader} is a person who reads books for a publisher in order to give an opinion on whether they should be published or not.
 \textit{
	\begin{itemize}
	\end{itemize}
}
\item countable noun \\
In British universities, a \textbf{reader} is a senior lecturer, with a rank just below that of a professor.
 \textit{
	\begin{itemize}
	\item John Stevenson is Reader in History at the University of Sheffield.
	\end{itemize}
}
\item countable noun \\
In American universities and colleges , a \textbf{reader} is an assistant to a teacher . Readers' duties include reading and marking examination papers and other work.
 \textit{
	\begin{itemize}
	\end{itemize}
}
\item countable noun \\
A \textbf{reader} is a book to help children to learn to read, or to help people to learn a foreign language. It contains passages of text , and often exercises to give practice in reading and writing .
 \textit{
	\begin{itemize}
	\end{itemize}
}
\end{enumerate}

\section*{then}
{\large \color{blue}  }
\subsection*{Explain}
\begin{enumerate}
\item adverb \\
\textbf{Then} means at a particular time in the past or in the future .
 \textit{
	\begin{itemize}
	\item He wanted an income after his retirement; until then, he wouldn't require additional
money.
	\item The clinic opened for business last October and since then has treated more than
200 people.
	\item I spent years on the dole trying to get bands together and I never worried about
money then.
	\end{itemize}
}
\item adjective \\
\textbf{Then} is used when you refer to something which was true at a particular time in the past but is not true now .
 \textbf{Then} is also an adverb .
 \textit{
	\begin{itemize}
	\item The bill was enacted by the then Labour Government.
	\item He was known by many for his role in the then record-breaking robbery of the mail
train from Glasgow to London in August 1963.
	\item Richard Strauss, then 76 years old, suffered through the war years in silence.
	\item Roberts was then a newly married man.
	\end{itemize}
}
\item adverb \\
You use \textbf{then} to say that one thing happens after another, or is after another on a list .
 \textit{
	\begin{itemize}
	\item Add the oil and then the scallops to the pan, leaving a little space for the garlic.
	\item I felt myself blush. Then I sniffed back a tear.
	\item The couple worked at first individually and then together.
	\end{itemize}
}
\item adverb \\
You use \textbf{then} in conversation to indicate that what you are about to say follows logically in some way from what has just been said or implied .
 \textit{
	\begin{itemize}
	\item 'I wasn't a very good scholar at school.'—'What did you like doing best then?'
	\item You're not gonna tell me, are you? Do I have to guess, then?
	\item 'I got a load of money out of them.'—'So you're okay, then.'
	\end{itemize}
}
\item adverb \\
You use \textbf{then} at the end of a topic or at the end of a conversation.
 \textit{
	\begin{itemize}
	\item 'I can meet you after work. Six o'clock?'—'Fine.'—'Six o'clock, then?'.
	\item He stood up. 'That's settled then.'.
	\item 'I'll talk to you on Friday anyway.'—'Yep. Okay then.'
	\end{itemize}
}
\item adverb \\
You use \textbf{then} with words like 'now', ' well ', and 'okay', to introduce a new topic or a new point of view .
 \textit{
	\begin{itemize}
	\item Now then, you say you walk on the fields out the back?
	\item Well then, I'll put the kettle on and make us some tea.
	\item Okay then let me ask how you do that.
	\end{itemize}
}
\item adverb \\
You use \textbf{then} to introduce a summary of what you have said or the conclusions that you are drawing from it.
 \textit{
	\begin{itemize}
	\item This, then, was the music that dominated in the mid-1960s.
	\item By 1931, then, France alone in Europe was a country of massive immigration.
	\end{itemize}
}
\item adverb \\
You use \textbf{then} to introduce the second part of a sentence which begins with 'if'. The first part of the sentence describes a possible  situation , and \textbf{then} introduces the result of the situation.
 \textit{
	\begin{itemize}
	\item If the answer is 'yes', then we must decide on an appropriate course of action.
	\end{itemize}
}
\item adverb \\
You use \textbf{then} at the beginning of a sentence or after 'and' or 'but' to introduce a comment or an extra piece of information to what you have already said.
 \textit{
	\begin{itemize}
	\item We have to do a lot of reading, and then we have essays to write.
	\item He sounded sincere, but then, he always did.
	\end{itemize}
}
\end{enumerate}

\section*{reading}
{\large \color{blue}  readings  }
\subsection*{Explain}
\begin{enumerate}
\item uncountable noun \\
\textbf{Reading} is the activity of reading books.
 \textit{
	\begin{itemize}
	\item I have always loved reading.
	\item ...young people who find reading and writing difficult.
	\end{itemize}
}
\item countable noun \\
A \textbf{reading} is an event at which poetry or extracts from books are read to an audience .
 \textit{
	\begin{itemize}
	\item This year's event consisted of readings, lectures and workshops.
	\item ...a poetry reading.
	\end{itemize}
}
\item countable noun \\
Your \textbf{reading}  \textbf{of} a word, text, or situation is the way in which you understand or interpret it.
 \textit{
	\begin{itemize}
	\item My reading of her character is that she is a responsible person.
	\item Local public housing authorities disagree with this reading of the law.
	\end{itemize}
}
\item countable noun \\
The \textbf{reading} on a measuring device is the figure or measurement that it shows.
 \textit{
	\begin{itemize}
	\item Once you have recorded the reading, shake the thermometer down to below 36 degrees.
	\item The gauge must be giving a faulty reading.
	\end{itemize}
}
\item countable noun \\
In the British Parliament or the U.S. Congress , a \textbf{reading} is one of the three stages of introducing and discussing a new bill before it can be passed as law.
 \textit{
	\begin{itemize}
	\item The bill is expected to pass its second reading with a comfortable majority.
	\end{itemize}
}
\item  \\
 make/make for interesting/dull/depressing reading \textit{
	\begin{itemize}
	\end{itemize}
}
\end{enumerate}

\section*{there}
{\large \color{blue}  }
\subsection*{Explain}
\begin{enumerate}
\item pronoun \\
\textbf{There} is used as the subject of the verb 'be' to say that something exists or does not exist, or to draw  attention to it.
 \textit{
	\begin{itemize}
	\item There are roadworks between the two towns.
	\item Are there some countries that have been able to tackle these problems successfully?
	\item There were differences of opinion, he added, on very basic issues.
	\item There's nothing in this room; there's no bed, and not a single shelf.
	\item There's no way we can afford to buy a house at the moment.
	\end{itemize}
}
\item pronoun \\
You use \textbf{there} in front of certain verbs when you are saying that something exists, develops , or can be seen . Whether the verb is singular or plural  depends on the noun which follows the verb.
 \textit{
	\begin{itemize}
	\item There remains considerable doubt over when the intended high-speed rail link will
be complete.
	\item There appeared no imminent danger.
	\item There rose before us the great pyramid of Gaza.
	\item There developed a practice that came to a tragic and terrible end.
	\end{itemize}
}
\item convention \\
\textbf{There} is used after 'hello' or 'hi' when you are greeting someone.
 \textit{
	\begin{itemize}
	\item 'Hello there,' said the woman, smiling at them.—'Hi!' they chorused.
	\item Oh, hi there. You must be Sidney.
	\end{itemize}
}
\item adverb \\
If something is \textbf{there} , it exists or is available .
 \textit{
	\begin{itemize}
	\item The group of old buildings on the corner by the main road is still there today.
	\item The book is there for people to read and make up their own mind.
	\item Nothing will be spent until he has made sure the money is there to pay for it.
	\end{itemize}
}
\item adverb \\
You use \textbf{there} to refer to a place which has already been mentioned .
 \textit{
	\begin{itemize}
	\item The next day we drove the 33 miles to Siena (the Villa Arceno is a great place to
stay while you are there) for the Palio.
	\item 'Come on over, if you want.'—'How do I get there?'
	\item It's an amazing train trip, about five days there and back.
	\item We love going to France because we enjoy the culture, lifestyle and food over there.
	\end{itemize}
}
\item adverb \\
You use \textbf{there} to indicate a place that you are pointing to or looking at, in order to draw someone's attention to it.
 \textit{
	\begin{itemize}
	\item There it is, on the corner over there.
	\item There she is on the left up there.
	\item The toilets are over there, dear.
	\item You'll find the details there.
	\end{itemize}
}
\item adverb \\
You use \textbf{there} in expressions such as ' \textbf{there he was} ' or ' \textbf{there we were} ' to sum up part of a story or to slow a story down for dramatic  effect .
 \textit{
	\begin{itemize}
	\item So there we were with Amy and she was driving us crazy.
	\item I looked, and there he was, riding a horse, with a double barreled shotgun on his
shoulder.
	\end{itemize}
}
\item adverb \\
You use \textbf{there} when speaking on the phone to ask if someone is available to speak to you.
 \textit{
	\begin{itemize}
	\item Hello, is Gordon there please?
	\end{itemize}
}
\item adverb \\
You use \textbf{there} to refer to a point that someone has made in a conversation .
 \textit{
	\begin{itemize}
	\item Death is terrible. I agree with you there.
	\item I think you're right there John.
	\item Can I just stop you there sir?
	\item If you'll excuse me, ladies and gentlemen, we'd better leave it there.
	\end{itemize}
}
\item adverb \\
You use \textbf{there} to refer to a stage that has been reached in an activity or process.
 \textit{
	\begin{itemize}
	\item We are making further investigations and will take the matter from there.
	\item And there we end this edition of the programme.
	\item And there we have a question that most women would find uncomfortable to answer.
	\end{itemize}
}
\item adverb \\
You use \textbf{there} to indicate that something has reached a point or level which is completely successful .
 \textit{
	\begin{itemize}
	\item We had hoped to fill the back page with extra news; we're not quite there yet.
	\item Life has not yet returned to normal but we are getting there.
	\end{itemize}
}
\item adverb \\
You can use \textbf{there} in expressions such as \textbf{there you go} or \textbf{there we are} when accepting that an unsatisfactory  situation cannot be changed.
 \textit{
	\begin{itemize}
	\item I'm the oldest and, according to all the books, should be the achiever, but there
you go.
	\item It's the wages that count. Not over-generous, but there you are.
	\item 'They didn't seem to know anything about it.'—'Oh well there we are.'
	\end{itemize}
}
\item adverb \\
You can use \textbf{there} in expressions such as \textbf{there you go} and \textbf{there we are} when emphasizing that something proves that you were right .
 \textit{
	\begin{itemize}
	\item There you go. I knew you'd take it the wrong way.
	\item 'There you are, you see!' she exclaimed. 'I knew you'd say that!'.
	\item Victoria Street, that's the name of the street. There we are, look.
	\end{itemize}
}
\item  \\
 there again \textit{
	\begin{itemize}
	\end{itemize}
}
\item  \\
 there you go again \textit{
	\begin{itemize}
	\end{itemize}
}
\item  \\
 so there \textit{
	\begin{itemize}
	\end{itemize}
}
\item  \\
 there and then \textit{
	\begin{itemize}
	\end{itemize}
}
\item  \\
 there there \textit{
	\begin{itemize}
	\end{itemize}
}
\item  \\
 there you are/go \textit{
	\begin{itemize}
	\end{itemize}
}
\item  \\
 be there for someone \textit{
	\begin{itemize}
	\end{itemize}
}
\end{enumerate}

\section*{recreation}
{\large \color{blue}  recreations  }
\subsection*{Explain}
\begin{enumerate}
\item variable noun \\
\textbf{Recreation} consists of things that you do in your spare time to relax .
 \textit{
	\begin{itemize}
	\item Saturday afternoon is for recreation and outings.
	\item All the family members need to have their own interests and recreations.
	\end{itemize}
}
\item countable noun \\
A \textbf{recreation}  \textbf{of} something is the process of making it exist or seem to exist again in a different time or place.
 \textit{
	\begin{itemize}
	\item This show from Scorsese and Mick Jagger is a stunning recreation of 1970s New York.
	\end{itemize}
}
\end{enumerate}

\section*{thereafter}
{\large \color{blue}  }
\subsection*{Explain}
\begin{enumerate}
\item adverb \\
\textbf{Thereafter} means after the event or date  mentioned .
 \textit{
	\begin{itemize}
	\item Inflation will fall and thereafter so will interest rates.
	\item It was the only time she had ever discouraged him and she regretted it thereafter.
	\end{itemize}
}
\end{enumerate}

\section*{script}
{\large \color{blue}  scripts  scripting  scripted  }
\subsection*{Explain}
\begin{enumerate}
\item countable noun \\
The \textbf{script} of a play, film, or television  programme is the written version of it.
 \textit{
	\begin{itemize}
	\item Jenny's writing a film script.
	\end{itemize}
}
\item verb \\
The person who \textbf{scripts} a film or a radio or television play writes it.
 \textit{
	\begin{itemize}
	\item The film is scripted and directed by Chris McQuarrie.
	\end{itemize}
}
\item variable noun \\
You can refer to a particular system of writing as a particular \textbf{script} .
 \textit{
	\begin{itemize}
	\item ...a text in the Malay language but written in Arabic script.
	\end{itemize}
}
\item singular noun \\
If you say that something which has happened is not in \textbf{the script} , or that someone has not followed  \textbf{the script} , you mean that something has happened which was not expected or intended to happen.
 \textit{
	\begin{itemize}
	\item Losing was not in the script.
	\item The game plan was right. We just didn't follow the script.
	\end{itemize}
}
\end{enumerate}

\section*{thereby}
{\large \color{blue}  }
\subsection*{Explain}
\begin{enumerate}
\item adverb \\
You use \textbf{thereby} to introduce an important result or consequence of the event or action you have just mentioned .
 \textit{
	\begin{itemize}
	\item Our bodies can sweat, thereby losing heat by evaporation.
	\item People are permitted to move into - and thereby spoil - these rural environments.
	\end{itemize}
}
\end{enumerate}

\section*{sheet}
{\large \color{blue}  sheets  }
\subsection*{Explain}
\begin{enumerate}
\item countable noun \\
A \textbf{sheet} is a large rectangular piece of cotton or other cloth that you sleep on or cover yourself with in a bed .
 \textit{
	\begin{itemize}
	\item Once a week, a maid changes the sheets.
	\item ...the luxury of silk sheets.
	\end{itemize}
}
\item countable noun \\
A \textbf{sheet}  \textbf{of} paper is a rectangular piece of paper.
 \textit{
	\begin{itemize}
	\item ...a sheet of newspaper.
	\item I was able to fit it all on one sheet.
	\end{itemize}
}
\item countable noun \\
You can use \textbf{sheet} to refer to a piece of paper which gives information about something.
 \textit{
	\begin{itemize}
	\item ...information sheets on each country in the world.
	\end{itemize}
}
\item countable noun \\
A \textbf{sheet}  \textbf{of} glass, metal, or wood is a large, flat , thin piece of it.
 \textit{
	\begin{itemize}
	\item ...a cracked sheet of glass.
	\item Overhead, cranes were lifting giant sheets of steel.
	\item Vinyl can be laid in sheet or tile form.
	\end{itemize}
}
\item countable noun \\
A \textbf{sheet}  \textbf{of} something is a thin wide  layer of it over the surface of something else.
 \textit{
	\begin{itemize}
	\item ...a sheet of ice.
	\item ...a blue-grey sheet of dust.
	\end{itemize}
}
\item countable noun \\
A \textbf{sheet}  \textbf{of}  fire or water is a fast-moving mass of it that is difficult to see through.
 \textit{
	\begin{itemize}
	\item The streets were now in one fierce sheet of flame.
	\item Sheets of rain slanted across the road.
	\end{itemize}
}
\item countable noun \\
In sailing , a \textbf{sheet} is a line or rope used for controlling the position of a sail on a boat .
 \textit{
	\begin{itemize}
	\end{itemize}
}
\end{enumerate}

\section*{therefore}
{\large \color{blue}  }
\subsection*{Explain}
\begin{enumerate}
\item adverb \\
You use \textbf{therefore} to introduce a logical result or conclusion .
 \textit{
	\begin{itemize}
	\item Muscle cells need lots of fuel and therefore burn lots of calories.
	\item This could bring back competition and therefore better deals for customers.
	\end{itemize}
}
\end{enumerate}

\section*{sleeve}
{\large \color{blue}  sleeves  }
\subsection*{Explain}
\begin{enumerate}
\item countable noun \\
The \textbf{sleeves} of a coat , shirt , or other item of clothing are the parts that cover your arms.
 \textit{
	\begin{itemize}
	\item His sleeves were rolled up to his elbows.
	\item He wore a black band on the left sleeve of his jacket.
	\end{itemize}
}
\item countable noun \\
A record \textbf{sleeve} is the stiff  envelope in which a record is kept .
 \textit{
	\begin{itemize}
	\item There are to be no pictures of him on the sleeve of the new record.
	\item ...an album sleeve.
	\end{itemize}
}
\item  \\
 wear one's heart on one's sleeve \textit{
	\begin{itemize}
	\end{itemize}
}
\item  \\
 have sth up one's sleeve \textit{
	\begin{itemize}
	\end{itemize}
}
\end{enumerate}

\section*{today}
{\large \color{blue}  }
\subsection*{Explain}
\begin{enumerate}
\item adverb \\
You use \textbf{today} to refer to the day on which you are speaking or writing .
 \textbf{Today} is also a noun .
 \textit{
	\begin{itemize}
	\item How are you feeling today?
	\item I wanted him to come with us today, but he couldn't.
	\item Today is Friday, September 14th.
	\item The Prime Minister remains the main story in today's newspapers.
	\end{itemize}
}
\item adverb \\
You can refer to the present period of history as \textbf{today} .
 \textbf{Today} is also a noun.
 \textit{
	\begin{itemize}
	\item Dozens of sayings that remain popular today originally appeared in the King James
Bible.
	\item He thinks pop music today is as exciting as it's ever been.
	\item Living in today's world we are exposed to pollution, traffic, and overcrowding.
	\item ...the Africa of today.
	\end{itemize}
}
\end{enumerate}

\section*{together}
{\large \color{blue}  }
\subsection*{Explain}
\begin{enumerate}
\item adverb \\
If people do something \textbf{together} , they do it with each other.
 \textit{
	\begin{itemize}
	\item We went on long bicycle rides together.
	\item He and I worked together on a book.
	\item They all live together in a three-bedroom house.
	\item Together they swam to the ship.
	\end{itemize}
}
\item adverb \\
If things are joined  \textbf{together} , they are joined with each other so that they touch or form one whole .
 \textit{
	\begin{itemize}
	\item Mix the ingredients together thoroughly.
	\item She clasped her hands together on her lap.
	\item If a window is broken, you can't stick it back together again.
	\end{itemize}
}
\item adverb \\
If things or people are situated  \textbf{together} , they are in the same place and very near to each other.
 \textit{
	\begin{itemize}
	\item The trees grew close together.
	\item Ginette and I gathered our things together.
	\item People stood packed together tightly.
	\end{itemize}
}
\item adverb \\
If a group of people are held or kept \textbf{together} , they are united with each other in some way.
 \textbf{Together} is also an adjective .
 \textit{
	\begin{itemize}
	\item He has done enough to pull the party together.
	\item I want us all to be a happy family together.
	\item His tough brand of social democracy was largely successful in holding the country
together.
	\item We are together in the way we're looking at this situation.
	\end{itemize}
}
\item adjective \\
If two people are \textbf{together} , they are married or having a sexual  relationship with each other.
 \textit{
	\begin{itemize}
	\item We were together for five years.
	\item Towards the end of our time together he was impossible.
	\item Passion kept us together.
	\end{itemize}
}
\item adverb \\
If two things happen or are done \textbf{together} , they happen or are done at the same time.
 \textit{
	\begin{itemize}
	\item Three horses crossed the finish line together.
	\item 'Yes,' they said together.
	\end{itemize}
}
\item adverb \\
You use \textbf{together} when you are adding two or more amounts or things to each other in order to consider a total amount or effect.
 \textit{
	\begin{itemize}
	\item The two main right-wing opposition parties together won 29.8 per cent.
	\item The companies have together spent £600 million.
	\item Together they account for less than five per cent of the population.
	\item The two together are particularly deadly.
	\end{itemize}
}
\item  \\
 go together \textit{
	\begin{itemize}
	\end{itemize}
}
\item adjective \\
If you describe someone as \textbf{together} , you admire them because they are very confident , organized, and know what they want .
 \textit{
	\begin{itemize}
	\item She was very headstrong, and very together.
	\item I know on the surface I appear to be quite a together person.
	\item I had to take a break for a cup of tea before I could really get myself together.
	\end{itemize}
}
\item  \\
 together with \textit{
	\begin{itemize}
	\end{itemize}
}
\end{enumerate}

\section*{talk}
{\large \color{blue}  talks  talking  talked  }
\subsection*{Explain}
\begin{enumerate}
\item verb \\
When you \textbf{talk} , you use spoken language to express your thoughts, ideas, or feelings.
 \textbf{Talk} is also a noun .
 \textit{
	\begin{itemize}
	\item He was too distressed to talk.
	\item A teacher reprimanded a girl for talking in class.
	\item The boys all began to talk at once.
	\item Though she can't talk yet, she understands what is going on.
	\item That's not the kind of talk one usually hears from accountants.
	\end{itemize}
}
\item verb \\
If you \textbf{talk}  \textbf{to} someone, you have a conversation with them. You can also say that two people \textbf{talk} .
 \textbf{Talk} is also a noun.
 \textit{
	\begin{itemize}
	\item We talked and laughed a great deal.
	\item I talked to him yesterday.
	\item A neighbour saw her talking with Craven.
	\item When she came back, they were talking about American food.
	\item Can't you see I'm talking? Don't interrupt.
	\item We had a long talk about her father, Tony, who was a friend of mine.
	\end{itemize}
}
\item verb \\
If you \textbf{talk}  \textbf{to} someone, you tell them about the things that are worrying you. You can also say that two people \textbf{talk} .
 \textbf{Talk} is also a noun.
 \textit{
	\begin{itemize}
	\item Your first step should be to talk to a teacher or school counselor.
	\item There's no one she can talk to, and she's on the verge of collapse.
	\item We need to talk alone.
	\item Do ring if you want to talk about it.
	\item I have to sort some things out. We really needed to talk.
	\item I think it's time we had a talk.
	\end{itemize}
}
\item verb \\
If you \textbf{talk}  \textbf{on} or \textbf{about} something, you make an informal speech telling people what you know or think about it.
 \textbf{Talk} is also a noun.
 \textit{
	\begin{itemize}
	\item She will talk on the issues she cares passionately about including education and
nursery care.
	\item He intends to talk to young people about the dangers of chatrooms.
	\item A guide gives a brief talk on the history of the site.
	\item He then set about campaigning, giving talks and fund-raising.
	\end{itemize}
}
\item plural noun \\
\textbf{Talks} are formal discussions intended to produce an agreement , usually between different countries or between employers and employees .
 \textit{
	\begin{itemize}
	\item ...the next round of peace talks.
	\item Talks between striking workers and the government have broken down.
	\item The Prime Minister flew into Washington for talks on nuclear defence.
	\end{itemize}
}
\item verb \\
If one group of people \textbf{talks}  \textbf{to} another, or if two groups \textbf{talk} , they have formal discussions in order to do a deal or produce an agreement.
 \textit{
	\begin{itemize}
	\item We're talking to some people about opening an office in London.
	\item The company talked with many potential investors.
	\item It triggered broad speculation that the two companies might be talking.
	\end{itemize}
}
\item verb \\
When different countries or different sides in a dispute  \textbf{talk} , or \textbf{talk}  \textbf{to} each other, they discuss their differences in order to try and settle the dispute.
 \textit{
	\begin{itemize}
	\item The Foreign Minister said he was ready to talk to any country that had no hostile
intentions.
	\item The two sides need to sit down and talk.
	\item He has to find a way to make both sides talk to each other.
	\end{itemize}
}
\item verb \\
If people \textbf{are talking}  \textbf{about} another person or \textbf{are talking} , they are discussing that person.
 \textbf{Talk} is also a noun.
 \textit{
	\begin{itemize}
	\item Everyone is talking about him.
	\item People will talk, but you have to get on with your life.
	\item There has been a lot of talk about me getting married.
	\item There was even talk that charges of fraud would be brought.
	\end{itemize}
}
\item verb \\
If someone \textbf{talks} when they are being held by police or soldiers , they reveal important or secret information, usually unwillingly.
 \textit{
	\begin{itemize}
	\item They'll talk, they'll implicate me.
	\end{itemize}
}
\item verb \\
If you \textbf{talk} a particular language or \textbf{talk} with a particular accent , you use that language or have that accent when you speak.
 \textit{
	\begin{itemize}
	\item You don't sound like a foreigner talking English.
	\item They were amazed that I was talking in an Irish accent.
	\end{itemize}
}
\item verb \\
If you \textbf{talk} something such as politics or sport, you discuss it.
 \textit{
	\begin{itemize}
	\item The guests were mostly middle-aged men talking business.
	\end{itemize}
}
\item verb \\
You can use \textbf{talk} to say what you think of the ideas that someone is expressing. For example, if you
say that someone \textbf{is}  \textbf{talking sense} , you mean that you think the opinions they are expressing are sensible .
 \textit{
	\begin{itemize}
	\item You must admit George, you're talking absolute rubbish.
	\end{itemize}
}
\item verb \\
You can say that you \textbf{are talking} a particular thing to draw  attention to your topic or to point out a characteristic of what you are discussing.
 \textit{
	\begin{itemize}
	\item We're not talking murder here; we're talking misdemeanors such as gambling.
	\item We're talking megabucks this time.
	\end{itemize}
}
\item uncountable noun \\
If you say that something such as an idea or threat is just \textbf{talk} , or \textbf{all talk} , you mean that it does not mean or matter much, because people are exaggerating about it or do not really  intend to do anything about it.
 \textit{
	\begin{itemize}
	\item Has much of this actually been tried here? Or is it just talk?
	\item Conditions should be laid down. Otherwise it's all talk.
	\end{itemize}
}
\item  \\
 talk about sth \textit{
	\begin{itemize}
	\end{itemize}
}
\item  \\
 talking of \textit{
	\begin{itemize}
	\end{itemize}
}
\end{enumerate}

\section*{tonight}
{\large \color{blue}  }
\subsection*{Explain}
\begin{enumerate}
\item adverb \\
\textbf{Tonight} is used to refer to the evening of today or the night that follows today.
 \textbf{Tonight} is also a noun .
 \textit{
	\begin{itemize}
	\item I'm at home tonight.
	\item Tonight, I think he proved to everybody what a great player he was.
	\item There they will stay until 11 o'clock tonight.
	\item Tonight is the opening night of the opera.
	\item ...tonight's flight to London.
	\end{itemize}
}
\end{enumerate}

\section*{textbook}
{\large \color{blue}  textbooks  }
\subsection*{Explain}
\begin{enumerate}
\item countable noun \\
A \textbf{textbook} is a book containing facts about a particular subject that is used by people studying that subject.
 \textit{
	\begin{itemize}
	\item She wrote a textbook on international law.
	\item ...a chemistry textbook.
	\end{itemize}
}
\item adjective \\
If you say that something is a \textbf{textbook}  case or example , you are emphasizing that it provides a clear example of a type of situation or event .
 \textit{
	\begin{itemize}
	\item The house is a textbook example of medieval domestic architecture.
	\item The corporation is a textbook model of what can be achieved by a state-owned company.
	\end{itemize}
}
\end{enumerate}

\section*{theory}
{\large \color{blue}  theories  }
\subsection*{Explain}
\begin{enumerate}
\item variable noun \\
A \textbf{theory} is a formal idea or set of ideas that is intended to explain something.
 \textit{
	\begin{itemize}
	\item Marx produced a new theory about historical change based upon conflict.
	\item Einstein formulated the Theory of Relativity in 1905.
	\end{itemize}
}
\item countable noun \\
If you have a \textbf{theory} about something, you have your own opinion about it which you cannot prove but which you think is true .
 \textit{
	\begin{itemize}
	\item There was a theory that he wanted to marry her.
	\item My theory is that you don't need a gym if you have stairs.
	\end{itemize}
}
\item uncountable noun \\
The \textbf{theory} of a practical subject or skill is the set of rules and principles that form the basis of it.
 \textit{
	\begin{itemize}
	\item He taught us music theory.
	\item ...graduates who are well-trained in both the theory and practice of statistics.
	\end{itemize}
}
\item  \\
 in theory \textit{
	\begin{itemize}
	\end{itemize}
}
\end{enumerate}

\section*{upstairs}
{\large \color{blue}  }
\subsection*{Explain}
\begin{enumerate}
\item adverb \\
If you go  \textbf{upstairs} in a building , you go up a staircase towards a higher floor.
 \textit{
	\begin{itemize}
	\item He went upstairs and changed into fresh clothes.
	\item I walked upstairs and unlocked my front door.
	\end{itemize}
}
\item adverb \\
If something or someone is \textbf{upstairs} in a building, they are on a floor that is higher than the ground floor.
 \textit{
	\begin{itemize}
	\item The restaurant is upstairs and consists of a large, open room.
	\item The boys are curled asleep in the small bedroom upstairs.
	\end{itemize}
}
\item adjective \\
An \textbf{upstairs}  room or object is situated on a floor of a building that is higher than the ground floor.
 \textit{
	\begin{itemize}
	\item Marsani moved into the upstairs apartment.
	\item ...an upstairs balcony.
	\end{itemize}
}
\item singular noun \\
\textbf{The upstairs} of a building is the floor or floors that are higher than the ground floor.
 \textit{
	\begin{itemize}
	\item Together we went through the upstairs.
	\item Frances invited them to occupy the upstairs of her home.
	\end{itemize}
}
\end{enumerate}

\section*{trap}
{\large \color{blue}  traps  trapping  trapped  }
\subsection*{Explain}
\begin{enumerate}
\item countable noun \\
A \textbf{trap} is a device which is placed somewhere or a hole which is dug somewhere in order to catch animals or birds.
 \textit{
	\begin{itemize}
	\end{itemize}
}
\item verb \\
If a person \textbf{traps} animals or birds, he or she catches them using traps.
 \textit{
	\begin{itemize}
	\item The locals were encouraged to trap and kill the birds.
	\end{itemize}
}
\item countable noun \\
A \textbf{trap} is a trick that is intended to catch or deceive someone.
 \textit{
	\begin{itemize}
	\item He failed to keep a rendezvous after sensing a police trap.
	\item He was trying to decide whether the question was some sort of a trap.
	\end{itemize}
}
\item verb \\
If you \textbf{trap} someone \textbf{into} doing or saying something, you trick them so that they do or say it, although they did not want to.
 \textit{
	\begin{itemize}
	\item Were you just trying to trap her into making some admission?
	\item She had trapped him so neatly that he wanted to slap her.
	\end{itemize}
}
\item verb \\
To \textbf{trap} someone, especially a criminal , means to capture them.
 \textit{
	\begin{itemize}
	\item The police knew that to trap the killer they had to play him at his own game.
	\item The couple set up a 24-hour security camera to trap the vandal scratching their car.
	\end{itemize}
}
\item countable noun \\
A \textbf{trap} is an unpleasant situation that you cannot easily  escape from.
 \textit{
	\begin{itemize}
	\item The Government has found it's caught in a trap of its own making.
	\end{itemize}
}
\item verb \\
If you \textbf{are trapped} somewhere, something falls onto you or blocks your way and prevents you from moving or escaping.
 \textit{
	\begin{itemize}
	\item The train was trapped underground by a fire.
	\item The light aircraft then cartwheeled, trapping both men.
	\item He saw trapped wagons and animals.
	\end{itemize}
}
\item verb \\
When something \textbf{traps} gas, water, or energy, it prevents it from escaping.
 \textit{
	\begin{itemize}
	\item Wool traps your body heat, keeping the chill at bay.
	\item The volume of gas trapped on these surfaces can be considerable.
	\end{itemize}
}
\item countable noun \\
A \textbf{trap} is a light carriage with two wheels  pulled by horses in which people used to travel .
 \textit{
	\begin{itemize}
	\end{itemize}
}
\item  \\
 to fall into the trap \textit{
	\begin{itemize}
	\end{itemize}
}
\item  \\
 shut one's trap/keep one's trap shut \textit{
	\begin{itemize}
	\end{itemize}
}
\end{enumerate}

\section*{where}
{\large \color{blue}  }
\subsection*{Explain}
\begin{enumerate}
\item adverb \\
You use \textbf{where} to ask questions about the place something is in, or is coming from or going to.
 \textit{
	\begin{itemize}
	\item Where did you meet him?
	\item Where's Anna?
	\item Where are we going?
	\item 'You'll never believe where Julie and I are going.'—'Where?'
	\end{itemize}
}
\item conjunction \\
You use \textbf{where} after certain words, especially  verbs and adjectives , to introduce a clause in which you mention the place in which something is situated or happens .
 \textbf{Where} is also a relative  pronoun .
 \textit{
	\begin{itemize}
	\item People began looking across to see where the noise was coming from.
	\item He knew where Henry Carter had gone.
	\item If he's got something on his mind he knows where to find me.
	\item Ernest Brown lives about a dozen blocks from where the riots began.
	\item ...available at the travel agency where you book your holiday.
	\item Wanchai boasts the Academy of Performing Arts, where everything from Chinese Opera
to Shakespeare is performed.
	\end{itemize}
}
\item adverb \\
You use \textbf{where} to ask questions about a situation , a stage in something, or an aspect of something.
 \textit{
	\begin{itemize}
	\item Where will it all end?
	\item If they get their way, where will it stop?
	\item It's not so simple. They'll have to let the draft board know, and then where will
we be?
	\end{itemize}
}
\item conjunction \\
You use \textbf{where} after certain words, especially verbs and adjectives, to introduce a clause in which
you mention a situation, a stage in something, or an aspect of something.
 \textbf{Where} is also a relative pronoun.
 \textit{
	\begin{itemize}
	\item It's not hard to see where she got her feelings about herself.
	\item She had a feeling she already knew where this conversation was going to lead.
	\item I didn't know where to start.
	\item ...that delicate situation where a friend's confidence can easily be betrayed.
	\item The government is at a stage where it is willing to talk to almost anyone.
	\end{itemize}
}
\item conjunction \\
You use \textbf{where} to introduce a clause that contrasts with the other parts of the sentence .
 \textit{
	\begin{itemize}
	\item Where some would given up, she and her coach were determined to lift their game.
	\item Sometimes a teacher will be listened to, where a parent might not.
	\end{itemize}
}
\end{enumerate}

\section*{tune}
{\large \color{blue}  tunes  tuning  tuned  }
\subsection*{Explain}
\begin{enumerate}
\item countable noun \\
A \textbf{tune} is a series of musical notes that is pleasant and easy to remember .
 \textit{
	\begin{itemize}
	\item She was humming a merry little tune.
	\end{itemize}
}
\item countable noun \\
You can refer to a song or a short piece of music as a \textbf{tune} .
 \textit{
	\begin{itemize}
	\item She'll also be playing your favourite pop tunes.
	\end{itemize}
}
\item verb \\
When someone \textbf{tunes} a musical instrument, they adjust it so that it produces the right notes.
 \textbf{Tune up} means the same as tune .
 \textit{
	\begin{itemize}
	\item 'We do tune our guitars before we go on,' he insisted.
	\item Others were quietly tuning up their instruments.
	\end{itemize}
}
\item verb \\
When an engine or machine \textbf{is tuned} , it is adjusted so that it works well .
 \textbf{Tune up} means the same as tune .
 \textit{
	\begin{itemize}
	\item Drivers are urged to make sure that car engines are properly tuned.
	\item How much do they charge to tune up a Porsche?
	\end{itemize}
}
\item verb \\
If your radio or television \textbf{is tuned}  \textbf{to} a particular broadcasting  station , you are listening to or watching the programmes being broadcast by that station.
 \textit{
	\begin{itemize}
	\item A small television was tuned to an afternoon soap opera.
	\end{itemize}
}
\item  \\
 to call the tune \textit{
	\begin{itemize}
	\end{itemize}
}
\item  \\
 to change your tune \textit{
	\begin{itemize}
	\end{itemize}
}
\item  \\
 to dance to someone's tune \textit{
	\begin{itemize}
	\end{itemize}
}
\item  \\
 in tune/out of tune \textit{
	\begin{itemize}
	\end{itemize}
}
\item  \\
 in tune with/out of tune with \textit{
	\begin{itemize}
	\end{itemize}
}
\item  \\
 to the tune of \textit{
	\begin{itemize}
	\end{itemize}
}
\end{enumerate}

\section*{abroad}
{\large \color{blue}  }
\subsection*{Explain}
\begin{enumerate}
\item adverb \\
If you go  \textbf{abroad} , you go to a foreign country, usually one which is separated from the country where
you live by an ocean or a sea.
 \textit{
	\begin{itemize}
	\item I would love to go abroad this year, perhaps to the South of France.
	\item ...public opposition here and abroad.
	\item He will stand in for Mr Goh when he is abroad.
	\item About 65 per cent of its sales come from abroad.
	\end{itemize}
}
\item adverb \\
If there is a story or feeling  \textbf{abroad} , people generally  know about it or feel it.
 \textit{
	\begin{itemize}
	\item There'll still be a feeling abroad that this change has to be recognised.
	\end{itemize}
}
\end{enumerate}

\section*{acceptance}
{\large \color{blue}  acceptances  }
\subsection*{Explain}
\begin{enumerate}
\item variable noun \\
\textbf{Acceptance}  \textbf{of} an offer or a proposal is the act of saying yes to it or agreeing to it.
 \textit{
	\begin{itemize}
	\item The Party is being degraded by its acceptance of secret donations.
	\item I sent them more than 6,000 cartoons before I had my one and only acceptance by them.
	\item Several shareholders have withdrawn earlier acceptances of the offer.
	\item ...a letter of acceptance.
	\item ...his acceptance speech for the Nobel Peace Prize.
	\end{itemize}
}
\item uncountable noun \\
If there is \textbf{acceptance} of an idea , most people believe or agree that it is true .
 \textit{
	\begin{itemize}
	\item ...a theory that is steadily gaining acceptance.
	\item There was a general acceptance that the defence budget would shrink.
	\end{itemize}
}
\item uncountable noun \\
Your \textbf{acceptance}  \textbf{of} a situation , especially an unpleasant or difficult one, is an attitude or feeling that you cannot change it and that you must  get used to it.
 \textit{
	\begin{itemize}
	\item Their acceptance of the system will probably determine its long-term fate.
	\item ...his calm acceptance of whatever comes his way.
	\end{itemize}
}
\item uncountable noun \\
If there is \textbf{acceptance} of a new product , people start to like it and get used to it.
 \textit{
	\begin{itemize}
	\item Customer acceptance of this technology has been outstanding.
	\item Avant-garde music to this day has not found general public acceptance.
	\end{itemize}
}
\item uncountable noun \\
\textbf{Acceptance} of someone into a group means beginning to think of them as part of the group and to act in a friendly way towards them.
 \textit{
	\begin{itemize}
	\item ...an effort to ensure that people with disabilities achieve real acceptance.
	\end{itemize}
}
\end{enumerate}

\section*{accordingly}
{\large \color{blue}  }
\subsection*{Explain}
\begin{enumerate}
\item adverb \\
You use \textbf{accordingly} to introduce a fact or situation which is a result or consequence of something that you have just referred to.
 \textit{
	\begin{itemize}
	\item We have a different background. Accordingly, we have the right to different futures.
	\item The workforce wants working hours to be reduced. Many companies have accordingly
switched to a five-day week.
	\end{itemize}
}
\item adverb \\
If you consider a situation and then act \textbf{accordingly} , the way you act depends on the nature of the situation.
 \textit{
	\begin{itemize}
	\item It is a difficult job and they should be paid accordingly.
	\item The new government will make a judgment about its interests and act accordingly.
	\end{itemize}
}
\end{enumerate}

\section*{base}
{\large \color{blue}  bases  basing  based  baser  basest  }
\subsection*{Explain}
\begin{enumerate}
\item countable noun \\
The \textbf{base} of something is its lowest edge or part.
 \textit{
	\begin{itemize}
	\item There was a cycle path running along this side of the wall, right at its base.
	\item Line the base and sides of a 20cm deep round cake tin with paper.
	\end{itemize}
}
\item countable noun \\
The \textbf{base} of something is the lowest part of it, where it is attached to something else.
 \textit{
	\begin{itemize}
	\item The surgeon placed catheters through the veins and arteries near the base of the
head.
	\end{itemize}
}
\item countable noun \\
The \textbf{base} of an object such as a box or vase is the lower surface of it that touches the surface it rests on.
 \textit{
	\begin{itemize}
	\item Remove from the heat and plunge the base of the pan into a bowl of very cold water.
	\end{itemize}
}
\item countable noun \\
The \textbf{base} of an object that has several sections and that rests on a surface is the lower section
of it.
 \textit{
	\begin{itemize}
	\item The mattress is best on a solid bed base.
	\item The clock stands on an oval marble base, enclosed by a glass dome.
	\end{itemize}
}
\item countable noun \\
A \textbf{base} is a layer of something which will have another layer added to it.
 \textit{
	\begin{itemize}
	\item Spoon the mixture on to the biscuit base and cook in a pre-heated oven.
	\item On many modern wooden boats, epoxy coatings will have been used as a base for varnishing.
	\end{itemize}
}
\item countable noun \\
A position or thing that is a \textbf{base} for something is one from which that thing can be developed or achieved.
 \textit{
	\begin{itemize}
	\item The post will give him a powerful political base from which to challenge the Kremlin.
	\item The family base was crucial to my development.
	\end{itemize}
}
\item verb \\
If you \textbf{base} one thing \textbf{on} another thing, the first thing develops from the second thing.
 \textit{
	\begin{itemize}
	\item He based his conclusions on the evidence given by the captured prisoners.
	\end{itemize}
}
\item countable noun \\
A company's client  \textbf{base} or customer \textbf{base} is the group of regular clients or customers that the company gets most of its income
from.
 \textit{
	\begin{itemize}
	\item The company has been expanding its customer base using trade magazine advertising.
	\end{itemize}
}
\item countable noun \\
A military \textbf{base} is a place which part of the armed forces works from.
 \textit{
	\begin{itemize}
	\item Gunfire was heard at an army base close to the airport.
	\item ...a massive air base in eastern Saudi Arabia.
	\end{itemize}
}
\item countable noun \\
Your \textbf{base} is the main place where you work, stay , or live.
 \textit{
	\begin{itemize}
	\item For most of the spring and early summer her base was her home in Scotland.
	\end{itemize}
}
\item countable noun \\
If a place is a \textbf{base} for a certain activity, the activity can be carried out at that place or from that
place.
 \textit{
	\begin{itemize}
	\item The two hotels are attractive bases from which to explore southeast Tuscany.
	\item Los Angeles was still my financial base. I was still doing business there.
	\end{itemize}
}
\item countable noun \\
The \textbf{base} of a substance such as paint or food is the main ingredient of it, to which other
substances can be added.
 \textit{
	\begin{itemize}
	\item Drain off any excess marinade and use it as a base for a pouring sauce.
	\item Oils may be mixed with a base oil and massaged into the skin.
	\end{itemize}
}
\item countable noun \\
A \textbf{base} is a system of counting and expressing numbers. The decimal system uses base 10, and the binary system uses base 2.
 \textit{
	\begin{itemize}
	\end{itemize}
}
\item countable noun \\
A \textbf{base} in baseball , softball , or rounders is one of the places at each corner of the square on the pitch.
 \textit{
	\begin{itemize}
	\end{itemize}
}
\item graded adjective \\
\textbf{Base} behaviour is behaviour that is immoral or dishonest .
 \textit{
	\begin{itemize}
	\item Love has the power to overcome the baser emotions.
	\end{itemize}
}
\item  \\
 off base \textit{
	\begin{itemize}
	\end{itemize}
}
\item  \\
 touch base \textit{
	\begin{itemize}
	\end{itemize}
}
\item  \\
 touch/cover all the bases \textit{
	\begin{itemize}
	\end{itemize}
}
\end{enumerate}

\section*{altogether}
{\large \color{blue}  }
\subsection*{Explain}
\begin{enumerate}
\item adverb \\
You use \textbf{altogether} to emphasize that something has stopped , been done, or finished completely.
 \textit{
	\begin{itemize}
	\item When Artie stopped calling altogether, Julie found a new man.
	\item His tour may have to be cancelled altogether.
	\item Clinical test results are certainly encouraging - 10 per cent of wrinkles disappear
altogether.
	\end{itemize}
}
\item adverb \\
You use \textbf{altogether} in front of an adjective or adverb to emphasize a quality that someone or something has.
 \textit{
	\begin{itemize}
	\item The choice of language is altogether different.
	\item Today's celebrations have been altogether more sedate.
	\item Rebuilding the team is an altogether bigger challenge.
	\end{itemize}
}
\item adverb \\
You use \textbf{altogether} to modify a negative  statement and make it less forceful .
 \textit{
	\begin{itemize}
	\item We were not altogether sure that the comet would miss the Earth.
	\item Not altogether surprisingly, the Scottish League took a dim view of this behaviour.

	\item 'I'm not altogether a fool,' she said gruffly.
	\end{itemize}
}
\item adverb \\
You can use \textbf{altogether} to introduce a summary of what you have been saying .
 \textit{
	\begin{itemize}
	\item Altogether, it was a delightful town garden, peaceful and secluded.
	\end{itemize}
}
\item adverb \\
If several amounts add up to a particular amount \textbf{altogether} , that amount is their total .
 \textit{
	\begin{itemize}
	\item Britain has a total of five thousand military personnel in the area altogether.
	\item Altogether seven inmates escaped by scaling a wall and climbing down scaffolding.
	\end{itemize}
}
\end{enumerate}

\section*{basis}
{\large \color{blue}  bases  }
\subsection*{Explain}
\begin{enumerate}
\item singular noun \\
If something is done \textbf{on} a particular \textbf{basis} , it is done according to that method , system, or principle.
 \textit{
	\begin{itemize}
	\item We're going to be meeting there on a regular basis.
	\item They want all groups to be treated on an equal basis.
	\item I've always worked on the basis that any extra money would go into property.
	\end{itemize}
}
\item singular noun \\
If you say that you are acting \textbf{on} the \textbf{basis}  \textbf{of} something, you are giving that as the reason for your action.
 \textit{
	\begin{itemize}
	\item McGregor must remain confined, on the basis of the medical reports we have received.
	\item On the basis that recognising the problem is halfway to a solution, Mulcahy's comments
yesterday should be well received.
	\end{itemize}
}
\item countable noun \\
The \textbf{basis} of something is its starting point or an important part of it from which it can be further developed.
 \textit{
	\begin{itemize}
	\item Both factions have broadly agreed that the U.N. plan is a possible basis for negotiation.
	\item ...the sub-atomic particles that form the basis of nearly all matter on earth.
	\end{itemize}
}
\item countable noun \\
The \textbf{basis} for something is a fact or argument that you can use to prove or justify it.
 \textit{
	\begin{itemize}
	\item ...Japan's attempt to secure the legal basis to send troops overseas.
	\item This is a common fallacy which has no basis in fact.
	\end{itemize}
}
\end{enumerate}

\section*{always}
{\large \color{blue}  }
\subsection*{Explain}
\begin{enumerate}
\item adverb \\
If you \textbf{always} do something, you do it whenever a particular  situation  occurs . If you \textbf{always} did something, you did it whenever a particular situation occurred.
 \textit{
	\begin{itemize}
	\item Whenever I get into a relationship, I always fall madly in love.
	\item She's always late for everything.
	\item We've always done it this way.
	\item Always lock your garage.
	\end{itemize}
}
\item adverb \\
If something is \textbf{always} the case, was \textbf{always} the case, or will  \textbf{always} be the case, it is, was, or will be the case all the time, continuously.
 \textit{
	\begin{itemize}
	\item We will always remember his generous hospitality.
	\item He has always been the family solicitor.
	\item He was always cheerful.
	\end{itemize}
}
\item adverb \\
If you say that something is \textbf{always}  happening , especially something which annoys you, you mean that it happens repeatedly.
 \textit{
	\begin{itemize}
	\item She was always moving things around.
	\end{itemize}
}
\item adverb \\
You use \textbf{always} in expressions such as \textbf{can always} or \textbf{could always} when you are making suggestions or suggesting an alternative  approach or method .
 \textit{
	\begin{itemize}
	\item If you can't find any decent apples, you can always try growing them yourself.
	\item If I failed, I could always go back to being a writer.
	\end{itemize}
}
\item adverb \\
You can say that someone \textbf{always} was, for example , awkward or lucky to indicate that you are not surprised about what they are doing or have just done .
 \textit{
	\begin{itemize}
	\item She's going to be fine. She always was pretty strong.
	\item You always were a good friend.
	\end{itemize}
}
\end{enumerate}

\section*{christ}
{\large \color{blue}  }
\subsection*{Explain}
\begin{enumerate}
\item proper noun \\
\textbf{Christ} is one of the names of Jesus, whom Christians believe to be the son of God and whose teachings are the basis of Christianity .
 \textit{
	\begin{itemize}
	\item ...the teachings of Christ.
	\end{itemize}
}
\item exclamation \\
Some people say ' \textbf{Christ!} ' when they are surprised, shocked , or annoyed , or in order to emphasize what they are saying . This use could cause offence .
 \textit{
	\begin{itemize}
	\item He looked at her watch. 'Christ! We only have three minutes!'
	\end{itemize}
}
\end{enumerate}

\section*{badly}
{\large \color{blue}  worse  worst  }
\subsection*{Explain}
\begin{enumerate}
\item adverb \\
If something is done  \textbf{badly} or goes  \textbf{badly} , it is not very successful or effective .
 \textit{
	\begin{itemize}
	\item I was angry because I played so badly.
	\item The whole project was badly managed.
	\item The coalition did worse than expected, getting just 11.6 per cent of the vote.
	\end{itemize}
}
\item adverb \\
If someone or something is \textbf{badly}  hurt or \textbf{badly}  affected , they are severely hurt or affected.
 \textit{
	\begin{itemize}
	\item The bomb destroyed a police station and badly damaged a church.
	\item One man was killed and another badly injured.
	\item It was a gamble that went badly wrong.
	\end{itemize}
}
\item adverb \\
If you want or need something \textbf{badly} , you want or need it very much.
 \textit{
	\begin{itemize}
	\item Why do you want to go so badly?
	\item Planes landed at Bagram airport today carrying badly needed food and medicine.
	\end{itemize}
}
\item adverb \\
If someone behaves  \textbf{badly} or treats other people \textbf{badly} , they act in an unkind , unpleasant , or unacceptable  way .
 \textit{
	\begin{itemize}
	\item They have both behaved very badly and I am very hurt.
	\item I would like to know why we pensioners are being so badly treated.
	\end{itemize}
}
\item adverb \\
If something reflects  \textbf{badly} on someone or makes others think  \textbf{badly} of them, it harms their reputation .
 \textit{
	\begin{itemize}
	\item Teachers know that low exam results will reflect badly on them.
	\item The male sex comes out of the film very badly.
	\item Despite his illegal act, few people think badly of him.
	\end{itemize}
}
\item adverb \\
If a person or their job is \textbf{badly}  paid , they are not paid very much for what they do.
 \textit{
	\begin{itemize}
	\item You may have to work part-time, in a badly paid job with unsociable hours.
	\item This is the most dangerous professional sport there is, and the worst paid.
	\end{itemize}
}
\end{enumerate}

\section*{christian}
{\large \color{blue}  Christians  }
\subsection*{Explain}
\begin{enumerate}
\item countable noun \\
A \textbf{Christian} is someone who follows the teachings of Jesus Christ.
 \textit{
	\begin{itemize}
	\item He was a devout Christian.
	\item Christians have always been involved in supporting others in times of loss.
	\end{itemize}
}
\item adjective \\
\textbf{Christian}  means relating to Christianity or Christians.
 \textit{
	\begin{itemize}
	\item ...the Christian Church.
	\item ...the Christian faith.
	\item ...Christian areas of Beirut.
	\item Most of my friends are Christian.
	\end{itemize}
}
\end{enumerate}

\section*{barely}
{\large \color{blue}  }
\subsection*{Explain}
\begin{enumerate}
\item adverb \\
You use \textbf{barely} to say that something is only just true or only just the case .
 \textit{
	\begin{itemize}
	\item Anastasia could barely remember the ride to the hospital.
	\item It was 90 degrees and the air conditioning barely cooled the room.
	\item His voice was barely audible.
	\item She was an elfin-like girl who looked barely 10 years old.
	\end{itemize}
}
\item adverb \\
If you say that one thing had \textbf{barely}  happened when something else happened, you mean that the first event was followed  immediately by the second .
 \textit{
	\begin{itemize}
	\item The Boeing 767 had barely taxied to a halt before its doors were flung open.
	\item Barely had she recovered from this trauma when Martin contracted whooping cough.
	\end{itemize}
}
\end{enumerate}

\section*{church}
{\large \color{blue}  churches  }
\subsection*{Explain}
\begin{enumerate}
\item variable noun \\
A \textbf{church} is a building in which Christians worship. You usually refer to this place as \textbf{church} when you are talking about the time that people spend there.
 \textit{
	\begin{itemize}
	\item ...one of Britain's most historic churches.
	\item ...St Helen's Church.
	\item I didn't see you in church on Sunday.
	\end{itemize}
}
\item countable noun \\
A \textbf{Church} is one of the groups of people within the Christian religion, for example  Catholics or Methodists , that have their own beliefs , clergy, and forms of worship.
 \textit{
	\begin{itemize}
	\item ...co-operation with the Church of Scotland.
	\item Church leaders said he was welcome to return.
	\item ...the separation of church and state.
	\end{itemize}
}
\item  \\
 broad church \textit{
	\begin{itemize}
	\end{itemize}
}
\end{enumerate}

\section*{classroom}
{\large \color{blue}  classrooms  }
\subsection*{Explain}
\begin{enumerate}
\item countable noun \\
A \textbf{classroom} is a room in a school where lessons take place.
 \textit{
	\begin{itemize}
	\end{itemize}
}
\end{enumerate}

\section*{conversely}
{\large \color{blue}  }
\subsection*{Explain}
\begin{enumerate}
\item adverb \\
You say  \textbf{conversely} to indicate that the situation you are about to describe is the opposite or reverse of the one you have just described.
 \textit{
	\begin{itemize}
	\item In real life, nobody was all bad, nor, conversely, all good.
	\end{itemize}
}
\end{enumerate}

\section*{clerk}
{\large \color{blue}  clerks  clerking  clerked  }
\subsection*{Explain}
\begin{enumerate}
\item countable noun \\
A \textbf{clerk} is a person who works in an office, bank , or law court and whose job is to look after the records or accounts.
 \textit{
	\begin{itemize}
	\item She was offered a job as an accounts clerk with a travel firm.
	\end{itemize}
}
\item countable noun \\
In a hotel, office, or hospital , a \textbf{clerk} is the person whose job is to answer the telephone and deal with people when they arrive .
 \textit{
	\begin{itemize}
	\item ...a hotel clerk.
	\end{itemize}
}
\item countable noun \\
A \textbf{clerk} is someone who works in a store .
 \textit{
	\begin{itemize}
	\end{itemize}
}
\item verb \\
To \textbf{clerk} means to work as a clerk.
 \textit{
	\begin{itemize}
	\item Gene clerked at the auction.
	\item He clerked for the chief justice of the Supreme Court.
	\end{itemize}
}
\end{enumerate}

\section*{directly}
{\large \color{blue}  }
\subsection*{Explain}
\begin{enumerate}
\item adverb \\
If something is \textbf{directly} above, below, or in front of something, it is in exactly that position.
 \textit{
	\begin{itemize}
	\item The second rainbow will be bigger than the first, and directly above it.
	\item There, directly below me, was a guy holding the ball.
	\item The naked bulb was directly over his head.
	\item They are sleeping in the carpenter's shop directly above.
	\end{itemize}
}
\item adverb \\
If you do one action \textbf{directly}  \textbf{after} another, you do the second action as soon as the first one is finished .
 \textit{
	\begin{itemize}
	\item Directly after the meeting, a senior cabinet minister spoke to the BBC.
	\item Directly after lunch we were packed and ready to go.
	\item Directly following this treatment, he had a hollow, empty feeling in his stomach.
	\end{itemize}
}
\item adverb \\
If something happens  \textbf{directly} , it happens without any delay.
 \textit{
	\begin{itemize}
	\item He will be there directly.
	\end{itemize}
}
\end{enumerate}

\section*{conjunction}
{\large \color{blue}  conjunctions  }
\subsection*{Explain}
\begin{enumerate}
\item countable noun \\
A \textbf{conjunction}  \textbf{of} two or more things is the occurrence of them at the same time or place.
 \textit{
	\begin{itemize}
	\item ...the conjunction of two events.
	\item ...a conjunction of religious and social factors.
	\end{itemize}
}
\item countable noun \\
In grammar , a \textbf{conjunction} is a word or group of words that joins together words, groups, or clauses. In English,
there are co-ordinating conjunctions such as 'and' and 'but', and subordinating conjunctions such as 'although', 'because', and 'when'.
 \textit{
	\begin{itemize}
	\end{itemize}
}
\item  \\
 in conjunction \textit{
	\begin{itemize}
	\end{itemize}
}
\end{enumerate}

\section*{else}
{\large \color{blue}  }
\subsection*{Explain}
\begin{enumerate}
\item adjective \\
You use \textbf{else} after words such as ' anywhere ', 'someone', and 'what', to refer in a vague way to another person, place, or thing.
 \textbf{Else} is also an adverb .
 \textit{
	\begin{itemize}
	\item If I can't make a living at painting, at least I can teach someone else to paint.
	\item We had nothing else to do on those long trips.
	\item What else have you had for your birthday?
	\item There's not much else I can say.
	\item I never wanted to live anywhere else.
	\end{itemize}
}
\item adjective \\
You use \textbf{else} after words such as 'everyone', 'everything', and ' everywhere ' to refer in a vague way to all the other people, things, or places except the one
you are talking about.
 \textbf{Else} is also an adverb.
 \textit{
	\begin{itemize}
	\item As I try to be truthful, I expect everyone else to be truthful.
	\item Batteries are in short supply, like everything else here.
	\item London seems so much dirtier than everywhere else.
	\end{itemize}
}
\item phrase \\
You use \textbf{or else} after stating a logical  conclusion , to indicate that what you are about to say is evidence for that conclusion.
 \textit{
	\begin{itemize}
	\item He must be a good plumber, or else he wouldn't be so busy.
	\item Clearly no lessons have been learnt or else the problem would have been solved.
	\end{itemize}
}
\item phrase \\
You use \textbf{or else} to introduce a statement that indicates the unpleasant results that will occur if someone does or does not do something.
 \textit{
	\begin{itemize}
	\item This time we really need to succeed or else people will start giving us funny looks.
	\item Make sure you are strapped in very well, or else you will fall out.
	\end{itemize}
}
\item phrase \\
You use \textbf{or else} to introduce the second of two possibilities when you do not know which one is true .
 \textit{
	\begin{itemize}
	\item You are either a total genius or else you must be absolutely raving mad.
	\item It's likely someone gave her a lift, or else that she took a taxi.
	\end{itemize}
}
\item  \\
 above all else \textit{
	\begin{itemize}
	\end{itemize}
}
\item  \\
 if nothing else \textit{
	\begin{itemize}
	\end{itemize}
}
\item  \\
 or else \textit{
	\begin{itemize}
	\end{itemize}
}
\end{enumerate}

\section*{console}
{\large \color{blue}  consoles  consoling  consoled  }
\subsection*{Explain}
\begin{enumerate}
\item verb \\
If you \textbf{console} someone who is unhappy about something, you try to make them feel more cheerful .
 \textit{
	\begin{itemize}
	\item 'Never mind, Ned,' he consoled me.
	\item Often they cry, and I have to play the role of a mother, consoling them.
	\item He will have to console himself by reading about the success of his compatriots.
	\item I can console myself with the fact that I'm not alone.
	\item He consoled himself that Emmanuel looked like a nice boy, who could be a good playmate
for his daughter.
	\end{itemize}
}
\item countable noun \\
A \textbf{console} is a panel with a number of switches or knobs that is used to operate a machine.
 \textit{
	\begin{itemize}
	\end{itemize}
}
\end{enumerate}

\section*{especially}
{\large \color{blue}  }
\subsection*{Explain}
\begin{enumerate}
\item adverb \\
You use \textbf{especially} to emphasize that what you are saying applies more to one person, thing, or area than to any others.
 \textit{
	\begin{itemize}
	\item Millions of wild flowers colour the valleys, especially in April and May.
	\item Re-apply sunscreen every two hours, especially if you have been swimming.
	\end{itemize}
}
\item adverb \\
You use \textbf{especially} to emphasize a characteristic or quality.
 \textit{
	\begin{itemize}
	\item Babies are especially vulnerable to the cold in their first month.
	\end{itemize}
}
\end{enumerate}

\section*{council}
{\large \color{blue}  councils  }
\subsection*{Explain}
\begin{enumerate}
\item countable noun \\
A \textbf{council} is a group of people who are elected to govern a local area such as a city or, in Britain , a county .
 \textit{
	\begin{itemize}
	\item ...Cheshire County Council.
	\item The city council has voted almost unanimously in favour.
	\item ...David Ward, one of just two Liberal Democrats on the council.
	\item ...reports of local council meetings.
	\end{itemize}
}
\item adjective \\
\textbf{Council} houses or flats are owned by the local council, and people pay rent to live in them.
 \textit{
	\begin{itemize}
	\item There is a shortage of council housing.
	\item Council tenants around the country are planning a mass lobby of Parliament.
	\end{itemize}
}
\item countable noun \\
\textbf{Council} is used in the names of some organizations .
 \textit{
	\begin{itemize}
	\item ...the National Council for Civil Liberties.
	\item ...the Arts Council.
	\item ...community health councils.
	\end{itemize}
}
\item countable noun \\
In some organizations, the \textbf{council} is the group of people that controls or governs it.
 \textit{
	\begin{itemize}
	\item He was a member of the council of the Royal Northern College of Music.
	\item ...the Bundesbank's central council.
	\end{itemize}
}
\item countable noun \\
A \textbf{council} is a specially organized , formal meeting that is attended by a particular group of people.
 \textit{
	\begin{itemize}
	\item The President said he would call a grand council of all Afghans.
	\item The president meets ministers at inter-ministerial councils.
	\end{itemize}
}
\end{enumerate}

\section*{even}
{\large \color{blue}  }
\subsection*{Explain}
\begin{enumerate}
\item adverb \\
You use \textbf{even} to suggest that what comes just after or just before it in the sentence is rather  surprising .
 \textit{
	\begin{itemize}
	\item He kept calling me for years, even after he got married.
	\item Some of my remarks were so scathing that even Jane was surprised.
	\item I cannot come to a decision about it now or even give any indication of my own views.
	\item He didn't even hear what I said.
	\end{itemize}
}
\item adverb \\
You use \textbf{even} with comparative adjectives and adverbs to emphasize a quality that someone or something has.
 \textit{
	\begin{itemize}
	\item It was on television that he made an even stronger impact as an interviewer.
	\item During his second day Edward looked even more pale and quiet than on his first.
	\item Stan was speaking even more slowly than usual.
	\end{itemize}
}
\item  \\
 even though \textit{
	\begin{itemize}
	\end{itemize}
}
\item phrase \\
If one thing happens  \textbf{even as} something else happens, they both happen at exactly the same time.
 \textit{
	\begin{itemize}
	\item Even as she said this, she knew it was not quite true.
	\item He had been aware, even as he slept, of the noise of the engine.
	\end{itemize}
}
\item  \\
 even so \textit{
	\begin{itemize}
	\end{itemize}
}
\item  \\
 even then \textit{
	\begin{itemize}
	\end{itemize}
}
\end{enumerate}

\section*{cucumber}
{\large \color{blue}  cucumbers  }
\subsection*{Explain}
\begin{enumerate}
\item variable noun \\
A \textbf{cucumber} is a long thin vegetable with a hard green skin and wet  transparent flesh. It is eaten  raw in salads .
 \textit{
	\begin{itemize}
	\end{itemize}
}
\item  \\
 as cool as a cucumber \textit{
	\begin{itemize}
	\end{itemize}
}
\end{enumerate}

\section*{eventually}
{\large \color{blue}  }
\subsection*{Explain}
\begin{enumerate}
\item adverb \\
\textbf{Eventually}  means in the end, especially after a lot of delays, problems , or arguments .
 \textit{
	\begin{itemize}
	\item Eventually, the army caught up with him in Latvia.
	\item The flight eventually got away six hours late.
	\end{itemize}
}
\item adverb \\
\textbf{Eventually} means at the end of a situation or process or as the final  result of it.
 \textit{
	\begin{itemize}
	\item Eventually your child will leave home to lead her own life as a fully independent
adult.
	\item She sees the bar as a starting point and eventually plans to run her own chain of
country inns.
	\end{itemize}
}
\end{enumerate}

\section*{description}
{\large \color{blue}  descriptions  }
\subsection*{Explain}
\begin{enumerate}
\item variable noun \\
A \textbf{description} of someone or something is an account which explains what they are or what they look like.
 \textit{
	\begin{itemize}
	\item Police have issued a description of the man who was aged between fifty and sixty.
	\item They send out detailed descriptions of the job with the application forms.
	\item He has a real gift for vivid description.
	\end{itemize}
}
\item singular noun \\
If something is \textbf{of} a particular \textbf{description} , it belongs to the general  class of items that are mentioned .
 \textit{
	\begin{itemize}
	\item ...the oldest Catholic church of any description in England.
	\item Events of this description occurred daily.
	\end{itemize}
}
\item uncountable noun \\
You can say that something is \textbf{beyond}  \textbf{description} , or that it \textbf{defies}  \textbf{description} , to emphasize that it is very unusual , impressive , terrible , or extreme .
 \textit{
	\begin{itemize}
	\item His face is weary beyond description.
	\item We were in a disaster situation that defies description.
	\end{itemize}
}
\end{enumerate}

\section*{everywhere}
{\large \color{blue}  }
\subsection*{Explain}
\begin{enumerate}
\item adverb \\
You use \textbf{everywhere} to refer to a whole area or to all the places in a particular area.
 \textit{
	\begin{itemize}
	\item Working people everywhere object to paying taxes.
	\item We went everywhere together.
	\item Dust is everywhere.
	\item Tap water is drinkable everywhere in the Algarve.
	\item People come here from everywhere to see these lights.
	\end{itemize}
}
\item adverb \\
You use \textbf{everywhere} to refer to all the places that someone goes to.
 \textit{
	\begin{itemize}
	\item Bradley is still accustomed to travelling everywhere in style.
	\item Everywhere he went he was introduced as the current United States Open Champion.
	\end{itemize}
}
\item adverb \\
You use \textbf{everywhere} to emphasize that you are talking about a large number of places, or all possible places.
 \textit{
	\begin{itemize}
	\item I saw her picture everywhere.
	\item I looked everywhere. I couldn't find him.
	\end{itemize}
}
\item adverb \\
If you say that someone or something is \textbf{everywhere} , you mean that they are present in a place in very large numbers .
 \textit{
	\begin{itemize}
	\item There were books everywhere.
	\item Clothes were everywhere, hanging out of drawers and strewn across the floor.
	\end{itemize}
}
\end{enumerate}

\section*{disposition}
{\large \color{blue}  dispositions  }
\subsection*{Explain}
\begin{enumerate}
\item countable noun \\
Someone's \textbf{disposition} is the way that they tend to behave or feel .
 \textit{
	\begin{itemize}
	\item The rides are unsuitable for people of a nervous disposition.
	\item He was a man of decisive action and an adventurous disposition.
	\item ...his friendly and cheerful disposition.
	\end{itemize}
}
\item singular noun \\
A \textbf{disposition}  \textbf{to} do something is a willingness to do it.
 \textit{
	\begin{itemize}
	\item This has given him a disposition to consider our traditions critically.
	\item They show no disposition to improvise or to take risks.
	\end{itemize}
}
\item singular noun \\
If you refer to \textbf{the disposition of} a number of objects, you mean the pattern in which they are arranged or their positions in relation to each other.
 \textit{
	\begin{itemize}
	\item ...to understand the buildings from the disposition of walls and entrances.
	\end{itemize}
}
\item countable noun \\
The \textbf{disposition of} money or property is the act of giving or distributing it to a number of people.
 \textit{
	\begin{itemize}
	\item Judge Stacks was appointed to oversee the disposition of funds.
	\end{itemize}
}
\end{enumerate}

\section*{exceedingly}
{\large \color{blue}  }
\subsection*{Explain}
\begin{enumerate}
\item adverb \\
\textbf{Exceedingly} means very or very much.
 \textit{
	\begin{itemize}
	\item We had an exceedingly good lunch.
	\item This was an exceedingly difficult decision to take.
	\item I have a case that troubles me exceedingly.
	\end{itemize}
}
\end{enumerate}

\section*{drain}
{\large \color{blue}  drains  draining  drained  }
\subsection*{Explain}
\begin{enumerate}
\item verb \\
If you \textbf{drain} a liquid from a place or object, you remove the liquid by causing it to flow somewhere else. If a liquid \textbf{drains} somewhere, it flows there.
 \textit{
	\begin{itemize}
	\item Miners built the tunnel to drain water out of the mines.
	\item Now the focus is on draining the water.
	\item Springs and rivers that drain into lakes carry dissolved nitrates and phosphates.
	\item The water slowly drained away, down through the porous soil.
	\end{itemize}
}
\item verb \\
If you \textbf{drain} a place or object, you dry it by causing water to flow out of it. If a place or object
 \textbf{drains} , water flows out of it until it is dry.
 \textit{
	\begin{itemize}
	\item Vast numbers of people have been mobilised to drain flooded land.
	\item The soil drains freely and slugs aren't a problem.
	\end{itemize}
}
\item verb \\
If you \textbf{drain} food or if food \textbf{drains} , you remove the liquid that it has been in, especially after it has been cooked or soaked in water.
 \textit{
	\begin{itemize}
	\item Drain the pasta well, arrange on four plates and pour over the sauce.
	\item Wash the leeks thoroughly and allow them to drain.
	\end{itemize}
}
\item countable noun \\
A \textbf{drain} is a pipe that carries water or sewage away from a place, or an opening in a surface that leads to the pipe.
 \textit{
	\begin{itemize}
	\item Tony built his own house and laid his own drains.
	\item ...storm drains.
	\end{itemize}
}
\item verb \\
If someone \textbf{drains} a glass, they empty it by drinking what is in it.
 \textit{
	\begin{itemize}
	\item Pamela drained her glass and refilled it.
	\end{itemize}
}
\item verb \\
If the colour or the blood \textbf{drains} or \textbf{is drained}  \textbf{from} someone's face, they become very pale . You can also  say that someone's face \textbf{drains} or \textbf{is drained}  \textbf{of} colour.
 \textit{
	\begin{itemize}
	\item Harry felt the colour drain from his face.
	\item Thacker's face drained of colour.
	\item Jock's face had been suddenly drained of all colour.
	\item His usually florid complexion seemed drained of colour.
	\end{itemize}
}
\item ergative verb \\
If a feeling  \textbf{drains} or \textbf{is drained} out of you, it gradually becomes less strong until you no longer feel it.
 \textit{
	\begin{itemize}
	\item And then, suddenly, the euphoria began to drain away.
	\item She felt the tension drain out of her.
	\item The happiness and the excitement had been drained completely from her voice.
	\end{itemize}
}
\item verb \\
If something \textbf{drains} you, it leaves you feeling physically and emotionally exhausted.
 \textit{
	\begin{itemize}
	\item My emotional turmoil had drained me.
	\end{itemize}
}
\item ergative verb \\
If energy \textbf{drains} or \textbf{is drained} from you, you lose all energy and become very tired .
 \textit{
	\begin{itemize}
	\item As his energy drained away, his despair and worry grew.
	\item I can help resolve conflicts that drain energy.
	\end{itemize}
}
\item singular noun \\
If you say that something is \textbf{a}  \textbf{drain}  \textbf{on} an organization's finances or resources, you mean that it costs the organization a large amount of money, and you do not consider that it is worth it.
 \textit{
	\begin{itemize}
	\item ...an ultra-modern printing plant, which has been a big drain on resources.
	\item Fraud trials are often complex and have become an expensive drain on the public purse.
	\end{itemize}
}
\item verb \\
If you say that a country's or a company's resources or finances \textbf{are drained} , you mean that they are used or spent completely.
 \textit{
	\begin{itemize}
	\item The state's finances have been drained by war.
	\item The company has steadily drained its cash reserves.
	\end{itemize}
}
\item  \\
 down the drain \textit{
	\begin{itemize}
	\end{itemize}
}
\item  \\
 down the drain \textit{
	\begin{itemize}
	\end{itemize}
}
\end{enumerate}

\section*{farther}
{\large \color{blue}  }
\subsection*{Explain}
\begin{enumerate}
\item  \\
\textbf{Farther} is a comparative form of far .
 \textit{
	\begin{itemize}
	\end{itemize}
}
\end{enumerate}

\section*{education}
{\large \color{blue}  educations  }
\subsection*{Explain}
\begin{enumerate}
\item variable noun \\
\textbf{Education} involves teaching people various subjects, usually at a school or college, or being
 taught .
 \textit{
	\begin{itemize}
	\item They're cutting funds for education.
	\item Paul prolonged his education with six years of advanced study in English.
	\item ...a man with little education.
	\end{itemize}
}
\item uncountable noun \\
\textbf{Education} of a particular kind involves teaching the public about a particular issue.
 \textit{
	\begin{itemize}
	\item ...better health education.
	\end{itemize}
}
\end{enumerate}

\section*{highly}
{\large \color{blue}  }
\subsection*{Explain}
\begin{enumerate}
\item adverb \\
\textbf{Highly} is used before some adjectives to mean 'very'.
 \textit{
	\begin{itemize}
	\item Mr Singh was a highly successful civil engineer.
	\item It seems highly unlikely that she ever existed.
	\item ...the highly controversial nuclear energy programme.
	\end{itemize}
}
\item adverb \\
You use \textbf{highly} to indicate that someone has an important position in an organization or set of people.
 \textit{
	\begin{itemize}
	\item ...a highly placed government advisor.
	\item ...highly ranked soccer teams.
	\end{itemize}
}
\item adverb \\
If someone is \textbf{highly} paid, they receive a large salary .
 \textit{
	\begin{itemize}
	\item He was the most highly paid member of staff.
	\end{itemize}
}
\item adverb \\
If you think  \textbf{highly} of something or someone, you think they are very good indeed.
 \textit{
	\begin{itemize}
	\item Daphne and Michael thought highly of the school.
	\item ...one of the most highly regarded chefs in the French capital.
	\end{itemize}
}
\end{enumerate}

\section*{foundation}
{\large \color{blue}  foundations  }
\subsection*{Explain}
\begin{enumerate}
\item countable noun \\
\textbf{The}  \textbf{foundation}  \textbf{of} something such as a belief or way of life is the things on which it is based.
 \textit{
	\begin{itemize}
	\item Best friends are the foundation of my life.
	\item The issue strikes at the very foundation of our community.
	\item This laid the foundations for later modern economic growth.
	\end{itemize}
}
\item plural noun \\
The \textbf{foundations} of a building or other structure are the layer of bricks or concrete below the ground that it is built on.
 \textit{
	\begin{itemize}
	\end{itemize}
}
\item countable noun \\
A \textbf{foundation} is an organization which provides money for a special  purpose such as research or charity.
 \textit{
	\begin{itemize}
	\item ...the National Foundation for Educational Research.
	\end{itemize}
}
\item uncountable noun \\
If a story , idea , or argument has \textbf{no}  \textbf{foundation} , there are no facts to prove that it is true .
 \textit{
	\begin{itemize}
	\item The allegations were without foundation.
	\item Each complaint is analysed very closely, and if it has no foundation it is rejected.
	\end{itemize}
}
\item variable noun \\
\textbf{Foundation} is a skin-coloured cream that you put on your face before putting on the rest of your make-up.
 \textit{
	\begin{itemize}
	\end{itemize}
}
\end{enumerate}

\section*{hitherto}
{\large \color{blue}  }
\subsection*{Explain}
\begin{enumerate}
\item adverb \\
You use \textbf{hitherto} to indicate that something was true up until the time you are talking about, although it may no longer be the case .
 \textit{
	\begin{itemize}
	\item As a result, workers who had hitherto been reliable now neglected their work.
	\item Hitherto, the main emphasis has been on the need to resist aggression.
	\item The consumer boom had made many hitherto scarce goods more readily available.
	\end{itemize}
}
\end{enumerate}

\section*{incidentally}
{\large \color{blue}  }
\subsection*{Explain}
\begin{enumerate}
\item adverb \\
You use \textbf{incidentally} to introduce a point which is not directly  relevant to what you are saying , often a question or extra information that you have just thought of.
 \textit{
	\begin{itemize}
	\item 'I didn't ask you to come. Incidentally, why have you come?'
	\item The tower, incidentally, dates from the twelfth century.
	\end{itemize}
}
\item adverb \\
If something occurs only \textbf{incidentally} , it is less important than another thing or is not a major part of it.
 \textit{
	\begin{itemize}
	\item The letter mentioned my great-aunt and uncle only incidentally.
	\end{itemize}
}
\end{enumerate}

\section*{impetus}
{\large \color{blue}  }
\subsection*{Explain}
\begin{enumerate}
\item variable noun \\
Something that gives a process \textbf{impetus} or an \textbf{impetus} makes it happen or progress more quickly.
 \textit{
	\begin{itemize}
	\item This decision will give renewed impetus to the economic regeneration of east London.
	\item She was restless and needed a new impetus for her talent.
	\end{itemize}
}
\end{enumerate}

\section*{namely}
{\large \color{blue}  }
\subsection*{Explain}
\begin{enumerate}
\item adverb \\
You use \textbf{namely} to introduce  detailed  information about the subject you are discussing , or a particular  aspect of it.
 \textit{
	\begin{itemize}
	\item One group of people seems to be forgotten, namely pensioners.
	\item They were hardly aware of the challenge facing them, namely, to re-establish prosperity.
	\end{itemize}
}
\end{enumerate}

\section*{inference}
{\large \color{blue}  inferences  }
\subsection*{Explain}
\begin{enumerate}
\item countable noun \\
An \textbf{inference} is a conclusion that you draw about something by using information that you already have about it.
 \textit{
	\begin{itemize}
	\item There were two inferences to be drawn from her letter.
	\end{itemize}
}
\item uncountable noun \\
\textbf{Inference} is the act of drawing conclusions about something on the basis of information that you already have.
 \textit{
	\begin{itemize}
	\item It had an extremely tiny head and, by inference, a tiny brain.
	\end{itemize}
}
\end{enumerate}

\section*{nearly}
{\large \color{blue}  }
\subsection*{Explain}
\begin{enumerate}
\item adverb \\
\textbf{Nearly} is used to indicate that something is not quite the case , or not completely the case.
 \textit{
	\begin{itemize}
	\item Goldsworth stared at me in silence for nearly twenty seconds.
	\item Hunter knew nearly all of this already.
	\item Several times Thorne nearly fell.
	\item I nearly had a heart attack when she told me.
	\item The beach was nearly empty.
	\item They nearly always ate outside.
	\end{itemize}
}
\item adverb \\
\textbf{Nearly} is used to indicate that something will  soon be the case.
 \textit{
	\begin{itemize}
	\item It was already nearly eight o'clock.
	\item I was nearly asleep.
	\item The voyage is nearly over.
	\item You're nearly there.
	\item I've nearly finished the words for your song.
	\end{itemize}
}
\item  \\
 not nearly \textit{
	\begin{itemize}
	\end{itemize}
}
\end{enumerate}

\section*{now}
{\large \color{blue}  }
\subsection*{Explain}
\begin{enumerate}
\item adverb \\
You use \textbf{now} to refer to the present time, often in contrast to a time in the past or the future .
 \textbf{Now} is also a pronoun .
 \textit{
	\begin{itemize}
	\item She's a widow now.
	\item But we are now a much more fragmented society.
	\item Coffee now costs well over 3 dollars a kilo.
	\item She should know that by now.
	\item Now is the time when we must all live as economically as possible.
	\end{itemize}
}
\item adverb \\
If you do something \textbf{now} , you do it immediately.
 \textbf{Now} is also a pronoun.
 \textit{
	\begin{itemize}
	\item I'm sorry, but I must go now.
	\item I fear that if I don't write now I shall never have another opportunity to do so.
	\item Now is your chance to talk to him.
	\end{itemize}
}
\item conjunction \\
You use \textbf{now} or \textbf{now that} to indicate that an event has occurred and as a result something else may or will  happen .
 \textit{
	\begin{itemize}
	\item Now you're settled, why don't you take up some serious study?
	\item Now that she was retired she lived with her sister.
	\end{itemize}
}
\item adverb \\
You use \textbf{now} to indicate that a particular  situation is the result of something that has recently happened.
 \textit{
	\begin{itemize}
	\item She told me not to repeat it, but now I don't suppose it matters.
	\item Diplomats now expect the mission to be much less ambitious.
	\end{itemize}
}
\item adverb \\
In stories and accounts of past events, \textbf{now} is used to refer to the particular time that is being written or spoken about.
 \textit{
	\begin{itemize}
	\item She felt a little better now.
	\item It was too late now for Blake to lock his room door.
	\item By now it was completely dark outside.
	\end{itemize}
}
\item adverb \\
You use \textbf{now} in statements which specify the length of time up to the present that something has lasted .
 \textit{
	\begin{itemize}
	\item They've been married now for 30 years.
	\item They have been missing for a long time now.
	\item It's some days now since I heard anything.
	\end{itemize}
}
\item adverb \\
You say ' \textbf{Now} ' or ' \textbf{Now then} ' to indicate to the person or people you are with that you want their attention , or that you are about to change the subject .
 \textit{
	\begin{itemize}
	\item 'Now then,' Max said, 'to get back to the point.'
	\item She stays at school for drama and doesn't get back till nine. Now, what's everyone
drinking?
	\item Now then, laddie, what's the trouble?
	\item Now, can we move on and discuss the vital business of the day, please.
	\end{itemize}
}
\item adverb \\
Some people say ' \textbf{Now} ' when they are thinking of what to say next .
 \textit{
	\begin{itemize}
	\item Now, er, dogs can live to fifteen.
	\item Now, erm, obviously some of our listeners may have some ideas.
	\end{itemize}
}
\item adverb \\
You use \textbf{now} to give a slight emphasis to a request or command.
 \textit{
	\begin{itemize}
	\item Come on now. You know you must be hungry.
	\item Come and sit down here, now.
	\item Now don't talk so loud and bother him, honey.
	\end{itemize}
}
\item adverb \\
You can say ' \textbf{Now} ' to introduce  information which is relevant to the part of a story or account that you have reached , and which needs to be known before you can continue .
 \textit{
	\begin{itemize}
	\item My son went to Almeria in Southern Spain. Now he is someone who loves a quiet holiday.
	\item Now, I hadn't told him these details, so he must have done some research on his own.
	\end{itemize}
}
\item adverb \\
You say ' \textbf{Now} ' to introduce something which contrasts with what you have just said .
 \textit{
	\begin{itemize}
	\item Now, if it was me, I'd want to do more than just change the locks.
	\end{itemize}
}
\item  \\
 now and again \textit{
	\begin{itemize}
	\end{itemize}
}
\item  \\
 any day/moment/time now \textit{
	\begin{itemize}
	\end{itemize}
}
\item  \\
 now for \textit{
	\begin{itemize}
	\end{itemize}
}
\item  \\
 just now \textit{
	\begin{itemize}
	\end{itemize}
}
\item  \\
 just now \textit{
	\begin{itemize}
	\end{itemize}
}
\item  \\
 it's now or never \textit{
	\begin{itemize}
	\end{itemize}
}
\item  \\
 now, now \textit{
	\begin{itemize}
	\end{itemize}
}
\item  \\
 now, now \textit{
	\begin{itemize}
	\end{itemize}
}
\end{enumerate}

\section*{instruction}
{\large \color{blue}  instructions  }
\subsection*{Explain}
\begin{enumerate}
\item countable noun \\
An \textbf{instruction} is something that someone tells you to do.
 \textit{
	\begin{itemize}
	\item Two lawyers were told not to leave the building but no reason for this instruction
was given.
	\end{itemize}
}
\item uncountable noun \\
If someone gives you \textbf{instruction} in a subject or skill , they teach it to you.
 \textit{
	\begin{itemize}
	\item Each candidate is given instruction in safety.
	\item All schoolchildren must now receive some religious instruction.
	\end{itemize}
}
\item plural noun \\
\textbf{Instructions} are clear and detailed  information on how to do something.
 \textit{
	\begin{itemize}
	\item Always read the instructions before you start taking the medicine.
	\end{itemize}
}
\end{enumerate}

\section*{nowadays}
{\large \color{blue}  }
\subsection*{Explain}
\begin{enumerate}
\item adverb \\
\textbf{Nowadays} means at the present time, in contrast with the past .
 \textit{
	\begin{itemize}
	\item Nowadays it's acceptable for women to be ambitious. But it wasn't then.
	\item I don't see much of Tony nowadays.
	\end{itemize}
}
\end{enumerate}

\section*{interface}
{\large \color{blue}  interfaces  interfacing  interfaced  }
\subsection*{Explain}
\begin{enumerate}
\item countable noun \\
The \textbf{interface} between two subjects or systems is the area in which they affect each other or have links with each other.
 \textit{
	\begin{itemize}
	\item ...a witty exploration of that interface between bureaucracy and the working world.
	\end{itemize}
}
\item countable noun \\
If you refer to the user  \textbf{interface} of a particular piece of computing  software , you are talking about its presentation on screen and how easy it is to operate .
 \textit{
	\begin{itemize}
	\item ...the development of better user interfaces.
	\end{itemize}
}
\item countable noun \\
In computing and electronics , an \textbf{interface} is an electrical circuit which links one machine , especially a computer, with another.
 \textit{
	\begin{itemize}
	\end{itemize}
}
\item verb \\
If one thing \textbf{interfaces}  \textbf{with} another, or if two things \textbf{interface} , they have connections with each other. If you \textbf{interface} one thing \textbf{with} another, you connect the two things.
 \textit{
	\begin{itemize}
	\item ...the way we interface with the environment.
	\item The different components all have to interface smoothly.
	\item He had interfaced all this machinery with a master computer.
	\end{itemize}
}
\end{enumerate}

\section*{joint}
{\large \color{blue}  joints  }
\subsection*{Explain}
\begin{enumerate}
\item adjective \\
\textbf{Joint} means shared by or belonging to two or more people.
 \textit{
	\begin{itemize}
	\item She and Frank had never gotten around to opening a joint account.
	\item Jackie and Ben came to a joint decision as to where they would live.
	\end{itemize}
}
\item countable noun \\
A \textbf{joint} is a part of your body such as your elbow or knee where two bones meet and are able to move together.
 \textit{
	\begin{itemize}
	\item Her joints ache if she exercises.
	\end{itemize}
}
\item countable noun \\
A \textbf{joint} is the place where two things are fastened or fixed together.
 \textit{
	\begin{itemize}
	\end{itemize}
}
\item countable noun \\
A \textbf{joint} is a fairly large piece of meat which is suitable for roasting.
 \textit{
	\begin{itemize}
	\item He carved the joint of lamb.
	\end{itemize}
}
\item countable noun \\
You can refer to a cheap place where people go for some form of entertainment as a \textbf{joint} .
 \textit{
	\begin{itemize}
	\item They had come to the world's most famous pick-up joint.
	\item ...a hamburger joint.
	\end{itemize}
}
\item countable noun \\
A \textbf{joint} is a cigarette which contains cannabis or marijuana .
 \textit{
	\begin{itemize}
	\end{itemize}
}
\item  \\
 to put someone's nose out of joint \textit{
	\begin{itemize}
	\end{itemize}
}
\item  \\
 out of joint \textit{
	\begin{itemize}
	\end{itemize}
}
\end{enumerate}

\section*{partly}
{\large \color{blue}  }
\subsection*{Explain}
\begin{enumerate}
\item adverb \\
You use \textbf{partly} to indicate that something happens or exists to some extent, but not completely.
 \textit{
	\begin{itemize}
	\item It's partly my fault.
	\item He let out a long sigh, mainly of relief, partly of sadness.
	\item I have not worried so much this year, partly because I have had other things to think
about.
	\item I feel partly responsible for the problems we're in.
	\end{itemize}
}
\end{enumerate}

\section*{pile}
{\large \color{blue}  piles  piling  piled  }
\subsection*{Explain}
\begin{enumerate}
\item countable noun \\
A \textbf{pile of} things is a mass of them that is high in the middle and has sloping sides.
 \textit{
	\begin{itemize}
	\item ...a pile of sand.
	\item ...a little pile of crumbs.
	\item The leaves had been swept into huge piles.
	\end{itemize}
}
\item countable noun \\
A \textbf{pile of} things is a quantity of things that have been put neatly somewhere so that each thing is on top of the one below.
 \textit{
	\begin{itemize}
	\item ...a pile of boxes.
	\item We sat in Sam's study, among the piles of books.
	\item The clothes were folded in a neat pile.
	\end{itemize}
}
\item verb \\
If you \textbf{pile} things somewhere, you put them there so that they form a pile.
 \textit{
	\begin{itemize}
	\item He was piling clothes into the suitcase.
	\item A few newspapers and magazines were piled on a table.
	\end{itemize}
}
\item verb \\
If something \textbf{is piled with} things, it is covered or filled with piles of things.
 \textit{
	\begin{itemize}
	\item Tables were piled high with local produce.
	\item ...trucks piled with luggage.
	\end{itemize}
}
\item quantifier \\
If you talk about a \textbf{pile of} something or \textbf{piles of} something, you mean a large amount of it.
 \textit{
	\begin{itemize}
	\item I've got a pile of questions afterwards for you.
	\item ...a whole pile of disasters.
	\end{itemize}
}
\item verb \\
If a group of people \textbf{pile into} or \textbf{out of} a vehicle, they all get into it or out of it in a disorganized way.
 \textit{
	\begin{itemize}
	\item They all piled into Jerrold's car.
	\item A fleet of police cars suddenly arrived. Dozens of officers piled out.
	\end{itemize}
}
\item countable noun \\
You can refer to a large impressive building as a \textbf{pile} , especially when it is the home of a rich important person.
 \textit{
	\begin{itemize}
	\item ...some stately pile in the country.
	\end{itemize}
}
\item countable noun \\
\textbf{Piles} are wooden , concrete, or metal posts which are pushed into the ground and on which buildings or bridges are built. Piles are often used in very wet areas so that the buildings do not flood .
 \textit{
	\begin{itemize}
	\item ...settlements of wooden houses, set on piles along the shore.
	\end{itemize}
}
\item plural noun \\
\textbf{Piles} are painful  swellings that can appear in the veins  inside a person's anus .
 \textit{
	\begin{itemize}
	\end{itemize}
}
\item singular noun \\
The \textbf{pile} of a carpet or of a fabric such as velvet is its soft surface. It consists of a lot of little threads  standing on end.
 \textit{
	\begin{itemize}
	\item ...the carpet's thick pile.
	\end{itemize}
}
\item  \\
 be at the bottom of the pile, be at the top of the pile \textit{
	\begin{itemize}
	\end{itemize}
}
\end{enumerate}

\section*{pretty}
{\large \color{blue}  prettier  prettiest  }
\subsection*{Explain}
\begin{enumerate}
\item adjective \\
If you describe someone, especially a girl , as \textbf{pretty} , you mean that they look  nice and are attractive in a delicate way.
 \textit{
	\begin{itemize}
	\item She's a very charming and very pretty girl.
	\end{itemize}
}
\item adjective \\
A place or a thing that is \textbf{pretty} is attractive and pleasant , in a charming but not particularly  unusual way.
 \textit{
	\begin{itemize}
	\item Whitstable is still a very pretty little town.
	\item ...comfortable sofas covered in a pretty floral print.
	\end{itemize}
}
\item adverb \\
You can use \textbf{pretty} before an adjective or adverb to mean 'quite' or ' rather '.
 \textit{
	\begin{itemize}
	\item I had a pretty good idea what she was going to do.
	\item Pretty soon after my arrival I found lodgings.
	\end{itemize}
}
\item  \\
 pretty much \textit{
	\begin{itemize}
	\end{itemize}
}
\item  \\
 be sitting pretty \textit{
	\begin{itemize}
	\end{itemize}
}
\end{enumerate}

\section*{priest}
{\large \color{blue}  priests  }
\subsection*{Explain}
\begin{enumerate}
\item countable noun \\
A \textbf{priest} is a member of the Christian  clergy in the Catholic , Anglican , or Orthodox church.
 \textit{
	\begin{itemize}
	\item He had trained to be a Catholic priest.
	\end{itemize}
}
\item countable noun \\
In many non-Christian religions a \textbf{priest} is a man who has particular duties and responsibilities in a place where people worship .
 \textit{
	\begin{itemize}
	\end{itemize}
}
\end{enumerate}

\section*{rarely}
{\large \color{blue}  }
\subsection*{Explain}
\begin{enumerate}
\item adverb \\
If something \textbf{rarely}  happens , it does not happen very often.
 \textit{
	\begin{itemize}
	\item June and her daughters laughed a lot and rarely fought.
	\item I very rarely wear a raincoat because I spend most of my time in a car.
	\item Money was plentiful, and rarely did anyone seem very bothered about levels of expenditure.
	\item They were rarely seen together and certainly did not travel together.
	\item Rarely does a grand jury publicly disagree with a prosecutor.
	\end{itemize}
}
\end{enumerate}

\section*{probe}
{\large \color{blue}  probes  probing  probed  }
\subsection*{Explain}
\begin{enumerate}
\item verb \\
If you \textbf{probe}  \textbf{into} something, you ask questions or try to discover  facts about it.
 \textbf{Probe} is also a noun .
 \textit{
	\begin{itemize}
	\item The more they probed into his background, the more inflamed their suspicions would
become.
	\item For three years, I have probed for understanding.
	\item The Office of Fair Trading has been probing banking practices.
	\item The form asks probing questions.
	\item ...a federal grand-jury probe into corruption within the FDA.
	\end{itemize}
}
\item verb \\
If a doctor or dentist  \textbf{probes} , he or she uses a long instrument to examine part of a patient's body.
 \textit{
	\begin{itemize}
	\item The surgeon would pick up his instruments, probe, repair and stitch up again.
	\item Dr Amid probed around the sensitive area.
	\item A doctor probed deep in his shoulder wound for shrapnel.
	\end{itemize}
}
\item countable noun \\
A \textbf{probe} is a long thin instrument that doctors and dentists use to examine parts of the body.
 \textit{
	\begin{itemize}
	\item ...a fibre-optic probe.
	\end{itemize}
}
\item verb \\
If you \textbf{probe} a place, you search it in order to find someone or something that you are looking for.
 \textit{
	\begin{itemize}
	\item A flashlight beam probed the underbrush only yards away from their hiding place.
	\item I probed around for some time in the bushes.
	\end{itemize}
}
\item verb \\
In a conflict such as a war, if one side \textbf{probes} another side's defences , they try to find their weaknesses , for example by attacking them in specific areas using a small number of troops .
 \textbf{Probe} is also a noun.
 \textit{
	\begin{itemize}
	\item He probes the enemy's weak positions, ignoring his strongholds.
	\item Squads of prison officers have been probing the rioters' defences.
	\item Small probes would give the allied armies some combat experience.
	\end{itemize}
}
\item countable noun \\
A \textbf{space probe} is a spacecraft which travels into space with no people in it, usually in order to study the planets and send  information about them back to earth .
 \textit{
	\begin{itemize}
	\item Its rings were discovered by telescope from Earth, but space probes later found that
spectacular rings surround some other planets.
	\item The Pioneer probes have on board ultra-violet instruments which are measuring light
that we can't measure on the earth.
	\end{itemize}
}
\end{enumerate}

\section*{rather}
{\large \color{blue}  }
\subsection*{Explain}
\begin{enumerate}
\item  \\
rather than \textit{
	\begin{itemize}
	\end{itemize}
}
\item adverb \\
You use \textbf{rather} when you are correcting something that you have just said , especially when you are describing a particular situation after saying what it is not.
 \textit{
	\begin{itemize}
	\item Twenty million years ago, Idaho was not the arid place it is now. Rather, it was
warm and damp.
	\item But there must be no talk of final victory; rather, the long, hard slog to a solution.
	\item The process is not a circle but rather a spiral.
	\item He explained what the Crux is, or rather, what it was.
	\end{itemize}
}
\item  \\
 would rather \textit{
	\begin{itemize}
	\end{itemize}
}
\item adverb \\
You use \textbf{rather} to indicate that something is true to a fairly great extent, especially when you are talking about something unpleasant or undesirable .
 \textit{
	\begin{itemize}
	\item I grew up in rather unusual circumstances.
	\item It had made some rather bad mistakes which I thought should be corrected.
	\item He had had an excellent dinner at a rather good local hotel.
	\item The first speaker began to talk, very fast and rather loudly.
	\item We got along rather well.
	\item I'm afraid it's rather a long story.
	\item The reality is rather more complex.
	\item As you can see, he did rather better for himself than I did.
	\item ...a figure rather too good to be true.
	\item The fruit is rather like a sweet chestnut.
	\item Robbie was there with his family, keeping rather in the background.
	\end{itemize}
}
\item adverb \\
You use \textbf{rather} before verbs that introduce your thoughts and feelings , in order to express your opinion politely, especially when a different opinion has been expressed.
 \textit{
	\begin{itemize}
	\item I rather think he was telling the truth.
	\item I rather like the decorative effect.
	\end{itemize}
}
\item convention \\
People sometimes  say  \textbf{rather} to express agreement or acceptance .
 \textit{
	\begin{itemize}
	\item 'Well, he did have a sort of family connection with it, didn't he.'—'Oh yes. Rather.'
	\end{itemize}
}
\end{enumerate}

\section*{profession}
{\large \color{blue}  professions  }
\subsection*{Explain}
\begin{enumerate}
\item countable noun \\
A \textbf{profession} is a type of job that requires advanced  education or training.
 \textit{
	\begin{itemize}
	\item Harper was a teacher by profession.
	\item Only 20 per cent of jobs in the professions are held by women.
	\end{itemize}
}
\item countable noun \\
You can use \textbf{profession} to refer to all the people who have the same profession.
 \textit{
	\begin{itemize}
	\item The attitude of the medical profession is very much more liberal now.
	\end{itemize}
}
\end{enumerate}

\section*{repeatedly}
{\large \color{blue}  }
\subsection*{Explain}
\begin{enumerate}
\item adverb \\
If you do something \textbf{repeatedly} , you do it many times.
 \textit{
	\begin{itemize}
	\item Both men have repeatedly denied the allegations.
	\item The rebel soldiers tried repeatedly to storm the building.
	\end{itemize}
}
\end{enumerate}

\section*{professor}
{\large \color{blue}  professors  }
\subsection*{Explain}
\begin{enumerate}
\item title noun \\
A \textbf{professor} in a British university is the most senior teacher in a department .
 \textit{
	\begin{itemize}
	\item ...Professor Cameron.
	\item In 1979, only 2% of British professors were female.
	\end{itemize}
}
\item countable noun \\
A \textbf{professor} in an American or Canadian university or college is a teacher of the highest  rank .
 \textit{
	\begin{itemize}
	\item Robert Dunn is a professor of economics at George Washington University.
	\end{itemize}
}
\end{enumerate}

\section*{seldom}
{\large \color{blue}  }
\subsection*{Explain}
\begin{enumerate}
\item adverb \\
If something \textbf{seldom}  happens , it happens only occasionally .
 \textit{
	\begin{itemize}
	\item They seldom speak.
	\item I've seldom felt so happy.
	\item We were seldom at home.
	\end{itemize}
}
\end{enumerate}

\section*{quest}
{\large \color{blue}  quests  }
\subsection*{Explain}
\begin{enumerate}
\item countable noun \\
A \textbf{quest} is a long and difficult search for something.
 \textit{
	\begin{itemize}
	\item My quest for a better bank continues.
	\item ...his quest to find true love.
	\end{itemize}
}
\end{enumerate}

\section*{sometime}
{\large \color{blue}  }
\subsection*{Explain}
\begin{enumerate}
\item adverb \\
You use \textbf{sometime} to refer to a time in the future or the past that is unknown or that has not yet been decided .
 \textit{
	\begin{itemize}
	\item The sales figures won't be released until sometime next month.
	\item Why don't you come and see me sometime?
	\item I'm aiming to get to work by nine sometime.
	\item I really want to go to Spain sometime.
	\end{itemize}
}
\item adjective \\
You use \textbf{sometime} to describe a job or role that a person used to have.
 \textit{
	\begin{itemize}
	\item Cecile was in her early thirties, a sometime actress, dancer and singer.
	\end{itemize}
}
\end{enumerate}

\section*{reception}
{\large \color{blue}  receptions  }
\subsection*{Explain}
\begin{enumerate}
\item singular noun \\
The \textbf{reception} in a hotel is the desk or office that books rooms for people and answers their questions .
 \textit{
	\begin{itemize}
	\item Have him bring a car round to the reception.
	\item ...the hotel's reception desk.
	\end{itemize}
}
\item singular noun \\
The \textbf{reception} in an office or hospital is the place where people's appointments and questions are dealt with.
 \textit{
	\begin{itemize}
	\item Wait at reception for me.
	\end{itemize}
}
\item countable noun \\
A \textbf{reception} is a formal party which is given to welcome someone or to celebrate a special event.
 \textit{
	\begin{itemize}
	\item At the reception they served smoked salmon.
	\item ...a glittering wedding reception.
	\end{itemize}
}
\item countable noun \\
If someone or something has a particular kind of \textbf{reception} , that is the way that people react to them.
 \textit{
	\begin{itemize}
	\item The President was given a tumultuous reception.
	\item He received a cool reception to his speech.
	\end{itemize}
}
\item singular noun \\
The \textbf{reception} of guests is the act of formally welcoming them.
 \textit{
	\begin{itemize}
	\item The preparations for the reception of his Royal Highness proceeded.
	\end{itemize}
}
\item uncountable noun \\
If you get good \textbf{reception} from your mobile  phone , radio, or television, the sound or picture is clear because the signal is strong. If the \textbf{reception} is poor, the sound or picture is unclear because the signal is weak .
 \textit{
	\begin{itemize}
	\item Adjust the aerial's position and direction for the best reception.
	\end{itemize}
}
\item singular noun \\
\textbf{Reception} is the same as reception class .
 \textit{
	\begin{itemize}
	\end{itemize}
}
\end{enumerate}

\section*{somewhere}
{\large \color{blue}  }
\subsection*{Explain}
\begin{enumerate}
\item adverb \\
You use \textbf{somewhere} to refer to a place without saying  exactly where you mean.
 \textit{
	\begin{itemize}
	\item I've got a feeling I've seen him before somewhere.
	\item I'm not going home yet. I have to go somewhere else first.
	\item 'Perhaps we can talk somewhere privately,' said Kesler.
	\item Somewhere in Ian's room were some of the letters that she had sent him.
	\item Don't I know you from somewhere?
	\item I needed somewhere to live in London.
	\end{itemize}
}
\item adverb \\
You use \textbf{somewhere} when giving an approximate amount, number, or time.
 \textit{
	\begin{itemize}
	\item It was painted somewhere between 1860 and 1875.
	\item Caray is somewhere between 73 and 80 years of age.
	\item The W.H.O. safety standard for ozone levels is somewhere about a hundred.
	\item Somewhere along the line justice is going to be done.
	\end{itemize}
}
\item  \\
 be getting somewhere \textit{
	\begin{itemize}
	\end{itemize}
}
\end{enumerate}

\section*{rehearsal}
{\large \color{blue}  rehearsals  }
\subsection*{Explain}
\begin{enumerate}
\item variable noun \\
A \textbf{rehearsal} of a play, dance , or piece of music is a practice of it in preparation for a performance.
 \textit{
	\begin{itemize}
	\item The band was scheduled to begin rehearsals for a concert tour.
	\item Mornings were spent in rehearsal, afternoons in performance.
	\end{itemize}
}
\item countable noun \\
You can describe an event or object which is a preparation for a more important event or object as a \textbf{rehearsal for} it.
 \textit{
	\begin{itemize}
	\item The sketch should be a kind of rehearsal for the eventual painting.
	\item Daydreams may seem to be rehearsals for real-life situations.
	\end{itemize}
}
\end{enumerate}

\section*{thus}
{\large \color{blue}  }
\subsection*{Explain}
\begin{enumerate}
\item adverb \\
You use \textbf{thus} to show that what you are about to mention is the result or consequence of something else that you have just mentioned.
 \textit{
	\begin{itemize}
	\item Neither of them thought of turning on the lunch-time news. Thus Caroline didn't hear
of John's death until Peter telephoned.
	\item Some people will be more capable and thus better paid than others.
	\item ...women's access to the basic means of production and thus to political power.
	\end{itemize}
}
\item adverb \\
If you say that something is \textbf{thus} or happens  \textbf{thus} you mean that it is, or happens, as you have just described or as you are just about to describe.
 \textit{
	\begin{itemize}
	\item He stormed four bunkers, completely destroying them. While thus engaged he was seriously
wounded.
	\item Martin helped his father dig the gardens. Thus he discovered his interest in gardening.
	\end{itemize}
}
\end{enumerate}

\section*{shelter}
{\large \color{blue}  shelters  sheltering  sheltered  }
\subsection*{Explain}
\begin{enumerate}
\item countable noun \\
A \textbf{shelter} is a small building or covered place which is made to protect people from bad weather or danger.
 \textit{
	\begin{itemize}
	\item The city's bomb shelters were being prepared for possible air raids.
	\item ...a bus shelter.
	\end{itemize}
}
\item uncountable noun \\
If a place provides \textbf{shelter} , it provides you with a place to stay or live , especially when you need protection from bad weather or danger.
 \textit{
	\begin{itemize}
	\item The number of families seeking shelter rose by 17 percent.
	\item Although horses do not generally mind the cold, shelter from rain and wind is important.
	\item ...the hut where they were given food and shelter.
	\end{itemize}
}
\item countable noun \\
A \textbf{shelter} is a building where homeless people can  sleep and get  food .
 \textit{
	\begin{itemize}
	\item ...a shelter for homeless women.
	\end{itemize}
}
\item verb \\
If you \textbf{shelter} in a place, you stay there and are protected from bad weather or danger.
 \textit{
	\begin{itemize}
	\item ...a man sheltering in a doorway.
	\item Twelve Cubans left the embassy after sheltering there for several days.
	\end{itemize}
}
\item verb \\
If a place or thing \textbf{is sheltered} by something, it is protected by that thing from wind and rain.
 \textit{
	\begin{itemize}
	\item ...a wooden house, sheltered by a low pointed roof.
	\end{itemize}
}
\item verb \\
If you \textbf{shelter} someone, usually someone who is being hunted by police or other people, you provide them with a place to stay or live.
 \textit{
	\begin{itemize}
	\item A neighbor sheltered the boy for seven days.
	\end{itemize}
}
\end{enumerate}

\section*{twice}
{\large \color{blue}  }
\subsection*{Explain}
\begin{enumerate}
\item adverb \\
If something happens  \textbf{twice} , there are two actions or events of the same kind .
 \textit{
	\begin{itemize}
	\item He visited me twice that fall.
	\item The government has twice declined to back the scheme.
	\item Brush teeth and gums twice daily.
	\item Twice before he had been in New York with Gladys on summer vacations.
	\item ...Foster, who is twice the world champion.
	\end{itemize}
}
\item adverb \\
You use \textbf{twice} in expressions such as \textbf{twice a day} and \textbf{twice a week} to indicate that two events or actions of the same kind happen in each day or week.
 \textit{
	\begin{itemize}
	\item I phoned twice a day, leaving messages with his secretary.
	\item This famous horse race has taken place here twice a year since 1310.
	\end{itemize}
}
\item adverb \\
If one thing is, for example , \textbf{twice as}  big or old  \textbf{as} another, the first thing is two times as big or old as the second . People sometimes  say that one thing is \textbf{twice as} good or hard  \textbf{as} another when they want to emphasize that the first thing is much better or harder than the second.
 \textbf{Twice} is also a predeterminer.
 \textit{
	\begin{itemize}
	\item The figure of seventy-million pounds was twice as big as expected.
	\item Remember, brunch is twice as delicious eaten outside.
	\item The barn was twice the size of her father's.
	\item Double cream contains approximately twice the quantity of fat-soluble vitamins as
single cream.
	\end{itemize}
}
\item  \\
 to think twice \textit{
	\begin{itemize}
	\end{itemize}
}
\end{enumerate}

\section*{stack}
{\large \color{blue}  stacks  stacking  stacked  }
\subsection*{Explain}
\begin{enumerate}
\item countable noun \\
A \textbf{stack}  \textbf{of} things is a pile of them.
 \textit{
	\begin{itemize}
	\item There were stacks of books on the bedside table and floor.
	\end{itemize}
}
\item verb \\
If you \textbf{stack} a number of things, you arrange them in neat piles.
 \textbf{Stack up} means the same as stack .
 \textit{
	\begin{itemize}
	\item Mme Cathiard was stacking the clean bottles in crates.
	\item They are stacked neatly in piles of three.
	\item He ordered them to stack up pillows behind his back.
	\item ...plates of delicious food stacked up on the counters.
	\end{itemize}
}
\item plural noun \\
If you say that someone has \textbf{stacks of} something, you mean that they have a lot of it.
 \textit{
	\begin{itemize}
	\item If the job's that good, you'll have stacks of money.
	\end{itemize}
}
\item verb \\
If someone in authority  \textbf{stacks} an organization or body, they fill it with their own supporters so that the decisions it makes will be the ones they want it to make.
 \textit{
	\begin{itemize}
	\item They said they were going to stack the court with anti-abortion judges.
	\item The committee is stacked with members from energy-producing states.
	\end{itemize}
}
\item  \\
 the odds are stacked against sb/things are stacked against sb \textit{
	\begin{itemize}
	\end{itemize}
}
\end{enumerate}

\section*{very}
{\large \color{blue}  }
\subsection*{Explain}
\begin{enumerate}
\item adverb \\
\textbf{Very} is used to give emphasis to an adjective or adverb .
 \textit{
	\begin{itemize}
	\item The problem and the answer are very simple.
	\item It is very, very strong evidence indeed.
	\item I'm very sorry.
	\item They are getting the hang of it very quickly.
	\item Thank you very much.
	\item The men were very much like my father.
	\end{itemize}
}
\item  \\
 not very \textit{
	\begin{itemize}
	\end{itemize}
}
\item adverb \\
You use \textbf{very} to give emphasis to an adjective that is not usually graded, when you want to say that a quality is very obvious .
 \textit{
	\begin{itemize}
	\item Janet looked very pregnant.
	\item His taste strikes the English as very French.
	\item If you think I'm happy with what's left, you're very wrong.
	\end{itemize}
}
\item adverb \\
You use \textbf{very} to give emphasis to a superlative adjective or adverb. For example , if you say that something is \textbf{the very best} , you are emphasizing that it is the best.
 \textit{
	\begin{itemize}
	\item They will be helped by the very latest in navigation aids.
	\item I am feeling in the very best of spirits.
	\item At the very least, the Government must offer some protection to mothers who fear
domestic violence.
	\end{itemize}
}
\item adjective \\
You use \textbf{very} with certain nouns in order to specify an extreme position or extreme point in time.
 \textit{
	\begin{itemize}
	\item At the very back of the yard, several feet from Lenny, was a wooden shack.
	\item I turned to the very end of the book, to read the final words.
	\item ...the Tuileries, in the very heart of Paris.
	\item He was wrong from the very beginning.
	\item We still do not have enough women at the very top.
	\end{itemize}
}
\item adjective \\
You use \textbf{very} with nouns to emphasize that something is exactly the right one or exactly the same one.
 \textit{
	\begin{itemize}
	\item Everybody says he is the very man for the case.
	\item She died in this very house.
	\item In my view, it only perpetuates the very problem that it sets out to cure.
	\item 'Most secret', he called it. Those were his very words.
	\end{itemize}
}
\item adjective \\
You use \textbf{very} with nouns to emphasize the importance or seriousness of what you are saying .
 \textit{
	\begin{itemize}
	\item At one stage, his very life was in danger.
	\item This act undermines the very basis of our democracy.
	\item He stated that such programmes were by their very nature harmful.
	\item History is taking place before your very eyes.
	\end{itemize}
}
\item  \\
 very good \textit{
	\begin{itemize}
	\end{itemize}
}
\item  \\
 very much so \textit{
	\begin{itemize}
	\end{itemize}
}
\item  \\
 very well \textit{
	\begin{itemize}
	\end{itemize}
}
\item  \\
 cannot very well do \textit{
	\begin{itemize}
	\end{itemize}
}
\end{enumerate}

\section*{successor}
{\large \color{blue}  successors  }
\subsection*{Explain}
\begin{enumerate}
\item countable noun \\
Someone's \textbf{successor} is the person who takes their job after they have left.
 \textit{
	\begin{itemize}
	\item He set out several principles that he hopes will guide his successors.
	\item ...several critics hailed him then as the greatest living Scottish poet, the natural
successor to his old friend Hugh MacDiarmid.
	\end{itemize}
}
\end{enumerate}

\section*{well}
{\large \color{blue}  }
\subsection*{Explain}
\begin{enumerate}
\item adverb \\
You say  \textbf{well} to indicate that you are about to say something.
 \textit{
	\begin{itemize}
	\item Sylvia shook hands. 'Well, you go get yourselves some breakfast.'.
	\item Well, I don't like the look of that.
	\end{itemize}
}
\item adverb \\
You say \textbf{well} to indicate that you intend or want to carry on speaking.
 \textit{
	\begin{itemize}
	\item You can, you know, get paranoid? Well, that's something I really try and avoid.
	\item The trouble with City is that they do not have enough quality players. Well, that
can easily be rectified.
	\end{itemize}
}
\item adverb \\
You say \textbf{well} to indicate that you are changing the topic , and are either going back to something that was being discussed earlier or are going on to something new.
 \textit{
	\begin{itemize}
	\item Thank you Lionel, for singing that for us. Well, we'd better tell you what's on the
show between nine and midnight.
	\item Well, let's press on.
	\end{itemize}
}
\item adverb \\
You say \textbf{well} to indicate that you have reached the end of a conversation .
 \textit{
	\begin{itemize}
	\item 'I'm sure you will be an asset,' she added. 'Well, I see it's time for lunch.'
	\item Well, thank you for speaking with us.
	\end{itemize}
}
\item adverb \\
You say \textbf{well} to make a suggestion , criticism , or correction  seem less definite or rude .
 \textit{
	\begin{itemize}
	\item Well, maybe it would be easier to start with a smaller problem.
	\item Well, let's wait and see.
	\item Well, I thought she was a bit unfair about me.
	\end{itemize}
}
\item adverb \\
You say \textbf{well} just before or after you pause , especially to give yourself time to think about what you are going to say.
 \textit{
	\begin{itemize}
	\item Look, I'm really sorry I woke you, and, well, I just wanted to tell you I was all
right.
	\end{itemize}
}
\item adverb \\
You say \textbf{well} when you are correcting something that you have just said .
 \textit{
	\begin{itemize}
	\item The comet is going to come back in 2061 and we are all going to be able to see it.
Well, our offspring are, anyway.
	\item There was a note. Well, not really a note.
	\end{itemize}
}
\item adverb \\
You say \textbf{well} to express your doubt about something that someone has said.
 \textit{
	\begin{itemize}
	\item 'But finance is far more serious.'—'Well, I don't know really.'
	\item 'Go on, Dennis.'—'Well, if you're sure.'
	\end{itemize}
}
\item exclamation \\
You say \textbf{well} to express your surprise or anger at something that someone has just said or done.
 \textit{
	\begin{itemize}
	\item 'Imelda,' said Mrs Kennerly. 'That's my name, Tom.'—'Well,' said Tom. 'Imelda. I
never knew.'.
	\item Well, honestly! They're like an old married couple at times.
	\end{itemize}
}
\item convention \\
You say \textbf{well} to indicate that you are waiting for someone to say something and often to express
your irritation with them.
 \textit{
	\begin{itemize}
	\item 'Well?' asked Barry, 'what does it tell us?'.
	\item 'Well, why don't you ask me?' he said finally.
	\end{itemize}
}
\item convention \\
You use \textbf{well} to indicate that you are amused by something you have heard or seen, and often to introduce a comment on it.
 \textit{
	\begin{itemize}
	\item Well, well, well, look at you.
	\item Bob peered at it. 'Well, well!' he said, 'I haven't seen Spam since the war!' and
laughed.
	\end{itemize}
}
\item  \\
 oh well \textit{
	\begin{itemize}
	\end{itemize}
}
\end{enumerate}

\section*{teacher}
{\large \color{blue}  teachers  }
\subsection*{Explain}
\begin{enumerate}
\item countable noun \\
A \textbf{teacher} is a person who teaches, usually as a job at a school or similar institution .
 \textit{
	\begin{itemize}
	\item I'm a teacher with 21 years' experience.
	\item ...her chemistry teacher.
	\end{itemize}
}
\end{enumerate}

\section*{yes}
{\large \color{blue}  yeses  }
\subsection*{Explain}
\begin{enumerate}
\item convention \\
You use \textbf{yes} to give a positive  response to a question .
 \textit{
	\begin{itemize}
	\item 'Are you a friend of Nick's?'—'Yes.'
	\item 'You actually wrote it down, didn't you?'—'Yes.'
	\item Will she say yes when I ask her out?
	\end{itemize}
}
\item convention \\
You use \textbf{yes} to accept an offer or request , or to give permission .
 \textit{
	\begin{itemize}
	\item 'More coffee?'—'Yes please.'
	\item 'Will you take me there?'—'Yes, I will.'
	\item 'Can I ask you something?'—'Yes, of course.'
	\end{itemize}
}
\item convention \\
You use \textbf{yes} to tell someone that what they have said is correct .
 \textit{
	\begin{itemize}
	\item 'Well I suppose it is based on the old lunar months isn't it.'—'Yes that's right.'
	\item 'That's a type of whitefly, is it?'—'Yes, it is a whitefly.'
	\end{itemize}
}
\item convention \\
You use \textbf{yes} to show that you are ready or willing to speak to the person who wants to speak to you, for example when you are answering a phone or a knock at your door .
 \textit{
	\begin{itemize}
	\item He pushed a button on the intercom. 'Yes?' came a voice.
	\item Yes, can I help you?
	\end{itemize}
}
\item convention \\
You use \textbf{yes} to indicate that you agree with, accept, or understand what the previous  speaker has said.
 \textit{
	\begin{itemize}
	\item 'Not everyone has the gift of a husband like Paul.'—'Oh yes.'
	\item 'It's a fabulous opportunity.'—'Yeah. I know.'
	\end{itemize}
}
\item convention \\
You use \textbf{yes} to encourage someone to continue speaking.
 \textit{
	\begin{itemize}
	\item 'I remembered something funny today.'—'Yeah?'
	\end{itemize}
}
\item convention \\
You use \textbf{yes} , usually followed by 'but', as a polite  way of introducing what you want to say when you disagree with something the previous speaker has just said.
 \textit{
	\begin{itemize}
	\item 'She is entitled to three thousand pounds of income.'—'Yes, but she doesn't earn
any money.'
	\item Ah yes, but think of all the family life they're missing.
	\end{itemize}
}
\item convention \\
You use \textbf{yes} to say that a negative  statement or question that the previous speaker has made is wrong or untrue .
 \textit{
	\begin{itemize}
	\item 'That is not possible,' she said. 'Oh, yes, it is!' Mrs Gruen insisted.
	\item 'I don't know what you're talking about.'—'Yes, you do.'
	\end{itemize}
}
\item convention \\
You can use \textbf{yes} to suggest that you do not believe or agree with what the previous speaker has said, especially when you want to express your annoyance about it.
 \textit{
	\begin{itemize}
	\item 'There was no way to stop it.'—'Oh yes? Well, here's something else you won't be
able to stop.'
	\end{itemize}
}
\item convention \\
You use \textbf{yes} to indicate that you had forgotten something and have just remembered it.
 \textit{
	\begin{itemize}
	\item What was I going to say. Oh yeah, we've finally got our second computer.
	\end{itemize}
}
\item convention \\
You use \textbf{yes} to emphasize and confirm a statement that you are making.
 \textit{
	\begin{itemize}
	\item He collected the £10,000 first prize. Yes, £10,000.
	\end{itemize}
}
\item  \\
 yes and no \textit{
	\begin{itemize}
	\end{itemize}
}
\item countable noun \\
A \textbf{yes} is a person who has answered 'yes' to a question or who has voted in favour of something, or the answer or vote they have made.
 \textit{
	\begin{itemize}
	\item The no-votes are leading the yeses.
	\item The noes have 50 percent, the yeses 35 percent and the rest are undecided.
	\end{itemize}
}
\end{enumerate}

\section*{yet}
{\large \color{blue}  }
\subsection*{Explain}
\begin{enumerate}
\item adverb \\
You use \textbf{yet} in negative  statements to indicate that something has not happened up to the present time, although it probably  will happen. You can also use \textbf{yet} in questions to ask if something has happened up to the present time. In British English the simple  past  tense is not normally used with this meaning of 'yet'.
 \textit{
	\begin{itemize}
	\item They haven't finished yet.
	\item No decision has yet been made.
	\item She hasn't yet set a date for her retirement.
	\item 'Has the murderer been caught?'—'Not yet.'
	\item Have you met my husband yet?
	\item Hammer-throwing for women is not yet a major event.
	\end{itemize}
}
\item adverb \\
You use \textbf{yet} with a negative statement when you are talking about the past, to report something that was not the case then, although it became the case later.
 \textit{
	\begin{itemize}
	\item There was so much that Sam didn't know yet.
	\item He had asked around and learned that Billy was not yet here.
	\end{itemize}
}
\item adverb \\
If you say that something should not or cannot be done \textbf{yet} , you mean that it should not or cannot be done now, although it will have to be done
at a later time.
 \textit{
	\begin{itemize}
	\item Don't get up yet.
	\item The hostages cannot go home just yet.
	\item We should not yet abandon this option for the disposal of highly radioactive waste.
	\end{itemize}
}
\item adverb \\
You use \textbf{yet} after a superlative to indicate, for example , that something is the worst or the best of its kind up to the present time.
 \textit{
	\begin{itemize}
	\item This is the BBC's worst idea yet.
	\item Her latest novel is her best yet.
	\item ...one of the toughest warnings yet delivered.
	\end{itemize}
}
\item adverb \\
You can use \textbf{yet} to say that there is still a possibility that something will happen.
 \textit{
	\begin{itemize}
	\item Like the best stories, this one may yet have a happy end.
	\item A negotiated settlement might yet be possible.
	\end{itemize}
}
\item adverb \\
You can use \textbf{yet} after expressions which refer to a period of time, when you want to say how much longer a situation will continue for.
 \textit{
	\begin{itemize}
	\item Unemployment will go on rising for some time yet.
	\item Nothing will happen for a few years yet.
	\item They'll be ages yet.
	\end{itemize}
}
\item adverb \\
If you say that you have \textbf{yet}  \textbf{to} do something, you mean that you have never done it, especially when this is surprising or bad .
 \textit{
	\begin{itemize}
	\item She has yet to spend a Christmas with her partner.
	\item He has been nominated three times for the Oscar but has yet to win.
	\end{itemize}
}
\item conjunction \\
You can use \textbf{yet} to introduce a fact which is rather surprising after the previous fact you have just mentioned .
 \textit{
	\begin{itemize}
	\item I don't eat much, yet I am a size 16.
	\item They were terrified James would die–yet there were moments when they almost wished
he would.
	\item It is completely waterproof, yet light and comfortable.
	\end{itemize}
}
\item adverb \\
You can use \textbf{yet} to emphasize a word, especially when you are saying that something is surprising because it is more extreme than previous things of its kind, or a further case of them.
 \textit{
	\begin{itemize}
	\item Yet bigger satellites will be sent up into orbit.
	\item I saw yet another doctor.
	\item They would criticize me, or worse yet, pay me no attention.
	\item By then governments may have woken up to a yet more radical option.
	\item We will not have anything to eat yet again.
	\end{itemize}
}
\item  \\
 as yet \textit{
	\begin{itemize}
	\end{itemize}
}
\end{enumerate}

\section*{volleyball}
{\large \color{blue}  }
\subsection*{Explain}
\begin{enumerate}
\item uncountable noun \\
\textbf{Volleyball} is a game in which two teams hit a large ball with their hands backwards and forwards over a high net. If you allow the ball to touch the ground , the other team wins a point.
 \textit{
	\begin{itemize}
	\end{itemize}
}
\end{enumerate}

\section*{adverb}
{\large \color{blue}  adverbs  }
\subsection*{Explain}
\begin{enumerate}
\item countable noun \\
An \textbf{adverb} is a word such as 'slowly', ' now ', 'very', 'politically', or ' fortunately ' which adds  information about the action , event , or situation  mentioned in a clause .
 \textit{
	\begin{itemize}
	\end{itemize}
}
\end{enumerate}

\section*{active}
{\large \color{blue}  }
\subsection*{Explain}
\begin{enumerate}
\item adjective \\
Someone who is \textbf{active} moves around a lot or does a lot of things.
 \textit{
	\begin{itemize}
	\item Having an active youngster about the house can be quite wearing.
	\item ...a long and active life.
	\end{itemize}
}
\item adjective \\
If you have an \textbf{active}  mind or imagination , you are always  thinking of new things.
 \textit{
	\begin{itemize}
	\item ...the tragedy of an active mind trapped by failing physical health.
	\end{itemize}
}
\item adjective \\
If someone is \textbf{active} in an organization, cause, or campaign , they do things for it rather than just giving it their support.
 \textit{
	\begin{itemize}
	\item ...a chance for fathers to play a more active role in childcare.
	\item I am an active member of the Conservative Party.
	\item He is active on Tyler's behalf.
	\end{itemize}
}
\item adjective \\
\textbf{Active} is used to emphasize that someone is taking action in order to achieve something, rather than just hoping for it or achieving it in an indirect way.
 \textit{
	\begin{itemize}
	\item Companies need to take active steps to increase exports.
	\item ...active discouragement from teachers.
	\end{itemize}
}
\item adjective \\
If you say that a person or animal is \textbf{active} in a particular place or at a particular time, you mean that they are performing
their usual activities or performing a particular activity.
 \textit{
	\begin{itemize}
	\item Guerrilla groups are active in the province.
	\item ...animals which are active at night.
	\item ...men who are sexually active.
	\end{itemize}
}
\item adjective \\
An \textbf{active} volcano has erupted recently or is expected to erupt quite  soon .
 \textit{
	\begin{itemize}
	\item ...molten lava from an active volcano.
	\end{itemize}
}
\item adjective \\
An \textbf{active} substance has a chemical or biological effect on things.
 \textit{
	\begin{itemize}
	\item The active ingredient in some of the mouthwashes was simply detergent.
	\end{itemize}
}
\item singular noun \\
In grammar , \textbf{the active} or \textbf{the active voice} means the forms of a verb which are used when the subject refers to a person or thing that does something. For example , in 'I saw her yesterday ', the verb is in the active. Compare  passive .
 \textit{
	\begin{itemize}
	\end{itemize}
}
\end{enumerate}

\section*{ambulance}
{\large \color{blue}  ambulances  }
\subsection*{Explain}
\begin{enumerate}
\item countable noun \\
An \textbf{ambulance} is a vehicle for taking people to and from hospital .
 \textit{
	\begin{itemize}
	\end{itemize}
}
\end{enumerate}

\section*{adequate}
{\large \color{blue}  }
\subsection*{Explain}
\begin{enumerate}
\item adjective \\
If something is \textbf{adequate} , there is enough of it or it is good enough to be used or accepted .
 \textit{
	\begin{itemize}
	\item One in four people worldwide are without adequate homes.
	\item The old methods weren't adequate to meet current needs.
	\item The western diet should be perfectly adequate for most people.
	\end{itemize}
}
\end{enumerate}

\section*{baby}
{\large \color{blue}  babies  }
\subsection*{Explain}
\begin{enumerate}
\item countable noun \\
A \textbf{baby} is a very young child, especially one that cannot yet walk or talk .
 \textit{
	\begin{itemize}
	\item She used to take care of me when I was a baby.
	\item My wife has just had a baby.
	\item Claire had to dress her baby sister.
	\end{itemize}
}
\item countable noun \\
A \textbf{baby} animal is a very young animal.
 \textit{
	\begin{itemize}
	\item ...a baby elephant.
	\item ...baby birds.
	\end{itemize}
}
\item adjective \\
\textbf{Baby}  vegetables are vegetables picked when they are very small.
 \textit{
	\begin{itemize}
	\item Serve with baby new potatoes.
	\end{itemize}
}
\item countable noun \\
Some people use \textbf{baby} as an affectionate way of addressing someone, especially a young woman, or referring to them.
 \textit{
	\begin{itemize}
	\item You have to wake up now, baby.
	\item He was confused, poor baby.
	\end{itemize}
}
\item  \\
 to throw the baby out with the bath water \textit{
	\begin{itemize}
	\end{itemize}
}
\item  \\
 be left holding the baby \textit{
	\begin{itemize}
	\end{itemize}
}
\end{enumerate}

\section*{afraid}
{\large \color{blue}  }
\subsection*{Explain}
\begin{enumerate}
\item adjective \\
If you are \textbf{afraid}  \textbf{of} someone or \textbf{afraid}  \textbf{to} do something, you are frightened because you think that something very unpleasant is going to happen to you.
 \textit{
	\begin{itemize}
	\item She did not seem at all afraid.
	\item I was afraid of the other boys.
	\item I'm still afraid to sleep in my own bedroom.
	\end{itemize}
}
\item adjective \\
If you are \textbf{afraid}  \textbf{for} someone else, you are worried that something horrible is going to happen to them.
 \textit{
	\begin{itemize}
	\item They were afraid for their own safety.
	\end{itemize}
}
\item adjective \\
If you are \textbf{afraid} that something unpleasant will happen, you are worried that it may happen and you want to avoid it.
 \textit{
	\begin{itemize}
	\item I was afraid that nobody would believe me.
	\item The Government is afraid of losing the election.
	\end{itemize}
}
\item  \\
 I'm afraid \textit{
	\begin{itemize}
	\end{itemize}
}
\end{enumerate}

\section*{bank}
{\large \color{blue}  banks  banking  banked  }
\subsection*{Explain}
\begin{enumerate}
\item countable noun \\
A \textbf{bank} is an institution where people or businesses can keep their money.
 \textit{
	\begin{itemize}
	\item Which bank offers you the service that best suits your financial needs?
	\item I had £10,000 in the bank.
	\end{itemize}
}
\item countable noun \\
A \textbf{bank} is a building where a bank offers its services.
 \textit{
	\begin{itemize}
	\end{itemize}
}
\item verb \\
If you \textbf{bank} money, you pay it into a bank.
 \textit{
	\begin{itemize}
	\item Once the agency has banked your cheque, the process begins.
	\end{itemize}
}
\item verb \\
If you \textbf{bank}  \textbf{with} a particular bank, you have an account with that bank.
 \textit{
	\begin{itemize}
	\item I've banked with the Co-op for over 20 years.
	\end{itemize}
}
\item countable noun \\
You use \textbf{bank} to refer to a store of something. For example, a blood \textbf{bank} is a store of blood that is kept ready for use.
 \textit{
	\begin{itemize}
	\item Detectives examined the syringe for DNA traces and deposited the information in a
central data bank.
	\end{itemize}
}
\item  \\
 to break the bank \textit{
	\begin{itemize}
	\end{itemize}
}
\end{enumerate}

\section*{almost}
{\large \color{blue}  }
\subsection*{Explain}
\begin{enumerate}
\item adverb \\
You use \textbf{almost} to indicate that something is not completely the case but is nearly the case.
 \textit{
	\begin{itemize}
	\item The couple had been dating for almost three years.
	\item Storms have been hitting almost all of Britain recently.
	\item The effect is almost impossible to describe.
	\item He was almost as tall as Pete, but skinnier.
	\item The arrested man will almost certainly be kept at this police station.
	\item He contracted Spanish flu, which almost killed him.
	\end{itemize}
}
\end{enumerate}

\section*{barrel}
{\large \color{blue}  barrels  barrelling  barrelled  }
\subsection*{Explain}
\begin{enumerate}
\item countable noun \\
A \textbf{barrel} is a large, round container for liquids or food.
 \textit{
	\begin{itemize}
	\item The wine is aged for almost a year in oak barrels.
	\item ...barrels of pickled fish.
	\end{itemize}
}
\item countable noun \\
In the oil industry, a \textbf{barrel} is a unit of measurement equal to 159 litres .
 \textit{
	\begin{itemize}
	\item Fully operational, the pipe can pump one million barrels of oil a day.
	\item Oil prices were closing at $19.76 a barrel.
	\end{itemize}
}
\item countable noun \\
The \textbf{barrel} of a gun is the tube through which the bullet moves when the gun is fired .
 \textit{
	\begin{itemize}
	\item He pushed the barrel of the gun into the other man's open mouth.
	\end{itemize}
}
\item verb \\
If a vehicle or person \textbf{is barrelling} in a particular direction, they are moving very quickly in that direction.
 \textit{
	\begin{itemize}
	\item The car was barreling down the street at a crazy speed.
	\end{itemize}
}
\item  \\
 lock, stock, and barrel \textit{
	\begin{itemize}
	\end{itemize}
}
\item  \\
 have someone over a barrel \textit{
	\begin{itemize}
	\end{itemize}
}
\item  \\
 to scrape the barrel \textit{
	\begin{itemize}
	\end{itemize}
}
\item  \\
 a barrel of laughs \textit{
	\begin{itemize}
	\end{itemize}
}
\end{enumerate}

\section*{aloud}
{\large \color{blue}  }
\subsection*{Explain}
\begin{enumerate}
\item adverb \\
When you say something, read , or laugh  \textbf{aloud} , you speak or laugh so that other people can hear you.
 \textit{
	\begin{itemize}
	\item When we were children, our father read aloud to us.
	\item 'You fool,' he said aloud.
	\end{itemize}
}
\item  \\
 think aloud \textit{
	\begin{itemize}
	\end{itemize}
}
\end{enumerate}

\section*{beer}
{\large \color{blue}  beers  }
\subsection*{Explain}
\begin{enumerate}
\item variable noun \\
\textbf{Beer} is a bitter alcoholic drink made from grain .
 A glass of beer can be referred to as a \textbf{beer} .
 \textit{
	\begin{itemize}
	\item He sat in the kitchen drinking beer.
	\item We have quite a good range of beers.
	\item Would you like a beer?
	\end{itemize}
}
\end{enumerate}

\section*{beforehand}
{\large \color{blue}  }
\subsection*{Explain}
\begin{enumerate}
\item adverb \\
If you do something \textbf{beforehand} , you do it earlier than a particular event.
 \textit{
	\begin{itemize}
	\item How could she tell beforehand that I was going to go out?
	\item Saunas can be hazardous if misused. Avoid a big meal beforehand.
	\end{itemize}
}
\end{enumerate}

\section*{brass}
{\large \color{blue}  brasses  }
\subsection*{Explain}
\begin{enumerate}
\item uncountable noun \\
\textbf{Brass} is a yellow-coloured metal made from copper and zinc. It is used especially for making ornaments and musical instruments.
 \textit{
	\begin{itemize}
	\item The instrument is beautifully made in brass.
	\end{itemize}
}
\item uncountable noun \\
\textbf{Brass} instruments are musical instruments such as trumpets and horns that you play by blowing into them.
 \textit{
	\begin{itemize}
	\end{itemize}
}
\item singular noun \\
\textbf{The}  \textbf{brass} is the section of an orchestra which consists of brass wind instruments such as trumpets
and horns.
 \textit{
	\begin{itemize}
	\item He once again raised his baton and brought in the brass.
	\end{itemize}
}
\item countable noun \\
\textbf{Brasses} are flat pieces of brass with writing or a picture cut into them, which are often found in churches.
 \textit{
	\begin{itemize}
	\end{itemize}
}
\item collective singular noun \\
In the army or in other organizations, \textbf{the}  \textbf{brass} are the people in the highest positions.
 \textit{
	\begin{itemize}
	\item The brass are reluctant to fraternise with the enlisted men.
	\end{itemize}
}
\item  \\
 to get down to brass tacks \textit{
	\begin{itemize}
	\end{itemize}
}
\end{enumerate}

\section*{bizarre}
{\large \color{blue}  }
\subsection*{Explain}
\begin{enumerate}
\item adjective \\
Something that is \textbf{bizarre} is very odd and strange .
 \textit{
	\begin{itemize}
	\item The game was also notable for the bizarre behaviour of the team's manager.
	\item You know, that book you lent me is really bizarre.
	\end{itemize}
}
\end{enumerate}

\section*{breakdown}
{\large \color{blue}  breakdowns  }
\subsection*{Explain}
\begin{enumerate}
\item countable noun \\
The \textbf{breakdown}  \textbf{of} something such as a relationship , plan , or discussion is its failure or ending .
 \textit{
	\begin{itemize}
	\item ...the breakdown of talks between the U.S. and E.U. officials.
	\item ...the irretrievable breakdown of a marriage.
	\item He argues that the breakdown in the legal system has spawned a black market.
	\end{itemize}
}
\item countable noun \\
If you have a \textbf{breakdown} , you become very depressed , so that you are unable to cope with your life.
 \textit{
	\begin{itemize}
	\item Obviously we were under a lot of stress. And I basically had a breakdown.
	\item They often seem depressed and close to emotional breakdown.
	\end{itemize}
}
\item countable noun \\
If a car or a piece of machinery has a \textbf{breakdown} , it stops working.
 \textit{
	\begin{itemize}
	\item Her old car was unreliable, so the trip was plagued by breakdowns.
	\item If you stop on the hard shoulder, wait for the police or breakdown service.
	\end{itemize}
}
\item countable noun \\
A \textbf{breakdown}  \textbf{of} something is a list of its separate parts.
 \textit{
	\begin{itemize}
	\item The organisers were given a breakdown of the costs.
	\end{itemize}
}
\end{enumerate}

\section*{can}
{\large \color{blue}  }
\subsection*{Explain}
\begin{enumerate}
\item modal verb \\
You use \textbf{can} when you are mentioning a quality or fact about something which people may make use of if they want to.
 \textit{
	\begin{itemize}
	\item Chicken is also the most versatile of meats. It can be roasted whole or in pieces.
	\item Luckily, iron can be reworked and mistakes don't have to be thrown away.
	\item A central reservation number can direct you to accommodations that best suit your
needs.
	\item A selected list of some of those stocking a comprehensive range can be found in Chapter
8.
	\item ...the statue which can still be seen in the British Museum.
	\end{itemize}
}
\item modal verb \\
You use \textbf{can} to indicate that someone has the ability or opportunity to do something.
 \textit{
	\begin{itemize}
	\item Don't worry yourself about me, I can take care of myself.
	\item I can't give you details because I don't actually have any details.
	\item Oh Stephen darling, how can I ever thank you for being so kind?
	\item See if you can find Karlov and tell him we are ready for dinner.
	\item 'You're needed here, Livy'—'But what can I do?'.
	\item The United States will do whatever it can to help Greece.
	\item I cannot describe it, I can't find the words.
	\item Customers can choose from sixty hit titles before buying.
	\item You can't be with your baby all the time.
	\end{itemize}
}
\item modal verb \\
You use \textbf{cannot} to indicate that someone is not able to do something because circumstances make it impossible for them to do it.
 \textit{
	\begin{itemize}
	\item We cannot buy food, clothes and pay for rent and utilities on $20 a week.
	\item She cannot sleep and the pain is often so bad she wants to scream.
	\end{itemize}
}
\item modal verb \\
You use \textbf{can} to indicate that something is true sometimes or is true in some circumstances.
 \textit{
	\begin{itemize}
	\item ...long-term therapy that can last five years or more.
	\item A vacant lot can produce an extraordinary variety of flora and fauna.
	\item I've quite forgotten how closed in London can seem.
	\item Exercising alone can be boring.
	\item The speed at which we talk can also convey a great deal.
	\item Coral can be yellow, blue, or green.
	\end{itemize}
}
\item modal verb \\
You use \textbf{cannot} and \textbf{can't} to state that you are certain that something is not the case or will not happen .
 \textit{
	\begin{itemize}
	\item From her knowledge of Douglas's habits, she feels sure that the attacker can't have
been Douglas.
	\item Things can't be that bad.
	\item You can't be serious, Mrs Lorimer?
	\end{itemize}
}
\item modal verb \\
You use \textbf{can} to indicate that someone is allowed to do something. You use \textbf{cannot} or \textbf{can't} to indicate that someone is not allowed to do something.
 \textit{
	\begin{itemize}
	\item You must buy the credit life insurance before you can buy the disability insurance.
	\item No-one can set up a waste disposal company without proper certification.
	\item Here, can I really have your jeans when you grow out of them?
	\item We can't answer any questions, I'm afraid.
	\item I can't tell you what he said.
	\item You cannot ask for your money back before the agreed date.
	\item I'm on tablets and I can't drive.
	\end{itemize}
}
\item modal verb \\
You use \textbf{cannot} or \textbf{can't} when you think it is very important that something should not happen or that someone should not
do something.
 \textit{
	\begin{itemize}
	\item It is an intolerable situation and it can't be allowed to go on.
	\item The committee can't demand from her more than it demands from its own members.
	\end{itemize}
}
\item modal verb \\
You use \textbf{can} , usually in questions , in order to make suggestions or to offer to do something.
 \textit{
	\begin{itemize}
	\item What can I do around here?
	\item This old lady was struggling out of the train and I said, 'Oh, can I help you?'.
	\item Hello John. What can we do for you?
	\item You can always try the beer you know–it's usually all right in this bar.
	\end{itemize}
}
\item modal verb \\
You use \textbf{can} in questions in order to make polite  requests . You use \textbf{can't} in questions in order to request strongly that someone does something.
 \textit{
	\begin{itemize}
	\item Can I have a look at that?
	\item Can you please help?
	\item Can you just lift the table for a second?
	\item Can you fill in some of the details of your career?
	\item Why can't you leave me alone?
	\end{itemize}
}
\item modal verb \\
You use \textbf{can} as a polite way of interrupting someone or of introducing what you are going to say  next .
 \textit{
	\begin{itemize}
	\item Can I interrupt you just for a minute?
	\item But if I can interrupt, Joe, I don't think anybody here is personally blaming you.
	\item Can I just ask something 'cos I'm really quite interested in this.
	\end{itemize}
}
\item modal verb \\
You use \textbf{can} with verbs such as ' imagine ', 'think', and ' believe ' in order to emphasize how you feel about a particular situation.
 \textit{
	\begin{itemize}
	\item You can imagine he was terribly upset.
	\item You can't think how glad I was to see them all go.
	\item It's been an appallingly busy morning, I can't tell you.
	\end{itemize}
}
\item modal verb \\
You use \textbf{can} in questions with 'how' to indicate that you feel strongly about something.
 \textit{
	\begin{itemize}
	\item How can you complain about higher taxes?
	\item How can millions of dollars go astray?
	\item How can you say such a thing?
	\item How can you expect me to believe your promises?
	\end{itemize}
}
\end{enumerate}

\section*{butter}
{\large \color{blue}  butters  buttering  buttered  }
\subsection*{Explain}
\begin{enumerate}
\item variable noun \\
\textbf{Butter} is a soft  yellow substance made from cream. You spread it on bread or use it in cooking.
 \textit{
	\begin{itemize}
	\item ...bread and butter.
	\item Pour the melted butter into a large mixing bowl.
	\end{itemize}
}
\item verb \\
If you \textbf{butter} something such as bread or toast , you spread butter on it.
 \textit{
	\begin{itemize}
	\item She spread pieces of bread on the counter and began buttering them.
	\item ...buttered scones.
	\end{itemize}
}
\item  \\
 to know what side your bread is buttered on \textit{
	\begin{itemize}
	\end{itemize}
}
\end{enumerate}

\section*{do}
{\large \color{blue}  does  doing  did  done  }
\subsection*{Explain}
\begin{enumerate}
\item auxiliary verb \\
\textbf{Do} is used to form the negative of main verbs, by putting 'not' after 'do' and before the main verb in its infinitive form, that is the form without 'to'.
 \textit{
	\begin{itemize}
	\item They don't want to work.
	\item I did not know Jamie had a knife.
	\item It doesn't matter if you win or lose.
	\end{itemize}
}
\item auxiliary verb \\
\textbf{Do} is used to form questions, by putting the subject after 'do' and before the main
verb in its infinitive form, that is the form without 'to'.
 \textit{
	\begin{itemize}
	\item Do you like music?
	\item What did he say?
	\item Where does she live?
	\end{itemize}
}
\item auxiliary verb \\
\textbf{Do} is used in question tags .
 \textit{
	\begin{itemize}
	\item You know about Andy, don't you?
	\item I'm sure they had some of the same questions last year didn't they?
	\end{itemize}
}
\item auxiliary verb \\
You use \textbf{do} when you are confirming or contradicting a statement containing 'do', or giving a negative or positive answer to a question.
 \textit{
	\begin{itemize}
	\item 'Did he think there was anything suspicious going on?'—'Yes, he did.'
	\item 'Do you have a metal detector?'—'No, I don't.'
	\item They say they don't care, but they do.
	\end{itemize}
}
\item auxiliary verb \\
\textbf{Do} is used with a negative to tell someone not to behave in a certain way.
 \textit{
	\begin{itemize}
	\item Don't be silly.
	\item Don't touch that!
	\end{itemize}
}
\item auxiliary verb \\
\textbf{Do} is used to give emphasis to the main verb when there is no other auxiliary.
 \textit{
	\begin{itemize}
	\item Veronica, I do understand.
	\item You did have your phone with you.
	\end{itemize}
}
\item auxiliary verb \\
\textbf{Do} is used as a polite way of inviting or trying to persuade someone to do something.
 \textit{
	\begin{itemize}
	\item Do sit down.
	\item Do help yourself to another drink.
	\end{itemize}
}
\item verb \\
\textbf{Do} can be used to refer back to another verb group when you are comparing or contrasting two things, or saying that they are the same.
 \textit{
	\begin{itemize}
	\item I make more money than he does.
	\item One day she will walk out, just as her own mother did.
	\item I had fantasies, as do all mothers, about how life would be when my girls were grown.
	\item Girls receive less health care and less education in the developing world than do
boys.
	\end{itemize}
}
\item verb \\
You use \textbf{do} after 'so' and 'nor' to say that the same statement is true for two people or groups.
 \textit{
	\begin{itemize}
	\item You know that's true, and so do I.
	\item We don't forget that. Nor does he.
	\item Her actions and thoughts became distorted. So did her behavior.
	\end{itemize}
}
\end{enumerate}

\section*{camp}
{\large \color{blue}  camps  camping  camped  }
\subsection*{Explain}
\begin{enumerate}
\item countable noun \\
A \textbf{camp} is a collection of huts and other buildings that is provided for a particular group of people, such as refugees , prisoners , or soldiers, as a place to live or stay .
 \textit{
	\begin{itemize}
	\item ...a refugee camp.
	\item 2,500 foreign prisoners-of-war, including Americans, had been held in camps near
Tambov.
	\end{itemize}
}
\item variable noun \\
A \textbf{camp} is an outdoor area with buildings, tents, or caravans where people stay on holiday.
 \textit{
	\begin{itemize}
	\end{itemize}
}
\item variable noun \\
A \textbf{camp} is a collection of tents or caravans where people are living or staying, usually
temporarily while they are travelling .
 \textit{
	\begin{itemize}
	\item ...gypsy camps.
	\item We'll make camp on that hill ahead.
	\end{itemize}
}
\item verb \\
If you \textbf{camp}  somewhere , you stay or live there for a short time in a tent or caravan, or in the open air.
 \textbf{Camp out} means the same as camp .
 \textit{
	\begin{itemize}
	\item We camped near the beach.
	\item ...the men, who are camping on the pavement in sleeping bags.
	\item For six months they camped out in a caravan in a meadow at the back of the house.
	\end{itemize}
}
\item countable noun \\
You can refer to a group of people who all support a particular person, policy , or idea as a particular \textbf{camp} .
 \textit{
	\begin{itemize}
	\item While the 'yes' camp remains in the lead, the jump in numbers of undecideds will
cause alarm.
	\item ...a close colleague, who had sided with the opposite camp in an office dispute.
	\end{itemize}
}
\item adjective \\
If you describe someone's behaviour , performance , or style of dress as \textbf{camp} , you mean that it is exaggerated and amusing , often in a way that is thought to be typical of some male  homosexuals .
 \textbf{Camp} is also a noun .
 \textit{
	\begin{itemize}
	\item James Barron turns in a delightfully camp performance.
	\item The video was seven minutes of high camp and melodrama.
	\end{itemize}
}
\item  \\
 camp it up \textit{
	\begin{itemize}
	\end{itemize}
}
\end{enumerate}

\section*{cigar}
{\large \color{blue}  cigars  }
\subsection*{Explain}
\begin{enumerate}
\item countable noun \\
\textbf{Cigars} are rolls of dried tobacco leaves which people smoke.
 \textit{
	\begin{itemize}
	\item He was sitting alone smoking a big cigar.
	\end{itemize}
}
\end{enumerate}

\section*{everyday}
{\large \color{blue}  }
\subsection*{Explain}
\begin{enumerate}
\item adjective \\
You use \textbf{everyday} to describe something which happens or is used every day, or forms a regular and basic part of your life, so it is not especially interesting or unusual .
 \textit{
	\begin{itemize}
	\item In the course of my everyday life, I had very little contact with teenagers.
	\item ...opportunities for improving fitness in your everyday routine.
	\item ...the everyday problems of living in the city.
	\item A paint finish can transform something everyday and mundane into something more elaborate.
	\end{itemize}
}
\end{enumerate}

\section*{cliff}
{\large \color{blue}  cliffs  }
\subsection*{Explain}
\begin{enumerate}
\item countable noun \\
A \textbf{cliff} is a high area of land with a very steep side, especially one next to the sea.
 \textit{
	\begin{itemize}
	\item The car rolled over the edge of a cliff.
	\end{itemize}
}
\end{enumerate}

\section*{finally}
{\large \color{blue}  }
\subsection*{Explain}
\begin{enumerate}
\item adverb \\
You use \textbf{finally} to suggest that something happens after a long period of time, usually later than you wanted or expected it to happen.
 \textit{
	\begin{itemize}
	\item The word was finally given for us to get on board.
	\item The food finally arrived at the end of last week and distribution began.
	\item Finally, after ten hours of negotiations, the gunman gave himself up.
	\end{itemize}
}
\item adverb \\
You use \textbf{finally} to indicate that something is last in a series of actions or events.
 \textit{
	\begin{itemize}
	\item The action slips from comedy to melodrama and finally to tragedy.
	\end{itemize}
}
\item adverb \\
You use \textbf{finally} in speech or writing to introduce a final point, question , or topic .
 \textit{
	\begin{itemize}
	\item Finally, who needs the theatre?
	\item And finally, a word about the winner and runner-up.
	\end{itemize}
}
\end{enumerate}

\section*{comb}
{\large \color{blue}  combs  combing  combed  }
\subsection*{Explain}
\begin{enumerate}
\item countable noun \\
A \textbf{comb} is a flat piece of plastic or metal with narrow pointed teeth along one side, which you use to tidy your hair.
 \textit{
	\begin{itemize}
	\end{itemize}
}
\item verb \\
When you \textbf{comb} your hair, you tidy it using a comb.
 \textit{
	\begin{itemize}
	\item Salvatore combed his hair carefully.
	\item Her reddish hair was cut short and neatly combed.
	\end{itemize}
}
\item verb \\
If you \textbf{comb} a place, you search everywhere in it in order to find someone or something.
 \textit{
	\begin{itemize}
	\item Officers combed the woods for the murder weapon.
	\item They fanned out and carefully combed the temple grounds.
	\end{itemize}
}
\item verb \\
If you \textbf{comb}  \textbf{through} information, you look at it very carefully in order to find something.
 \textit{
	\begin{itemize}
	\item Eight police officers then spent two years combing through the evidence.
	\end{itemize}
}
\end{enumerate}

\section*{forever}
{\large \color{blue}  }
\subsection*{Explain}
\begin{enumerate}
\item adverb \\
If you say that something will  happen or continue  \textbf{forever} , you mean that it will always happen or continue.
 \textit{
	\begin{itemize}
	\item I think that we will live together forever.
	\item It was great fun but we knew it wouldn't go on for ever.
	\item I will forever be grateful for his considerable input.
	\end{itemize}
}
\item adverb \\
If something has gone or changed  \textbf{forever} , it has gone or changed completely and permanently.
 \textit{
	\begin{itemize}
	\item The old social order was gone forever.
	\item Their lives changed forever.
	\end{itemize}
}
\item adverb \\
If you say that something takes  \textbf{forever} or lasts  \textbf{forever} , you are emphasizing that it takes or lasts a very long time, or that it seems to.
 \textit{
	\begin{itemize}
	\item The drive seemed to take forever.
	\item They didn't cost anything and they lasted forever.
	\end{itemize}
}
\item adverb \\
If you say that someone is \textbf{forever} doing a particular thing, especially something which annoys or amuses you, you are emphasizing that they do it very often.
 \textit{
	\begin{itemize}
	\item He was forever attempting to arrange deals.
	\item I was forever dragging him away from the fireplace.
	\end{itemize}
}
\item adverb \\
You use \textbf{forever} to emphasize that someone always has or shows the quality  mentioned .
 \textit{
	\begin{itemize}
	\item Katherine was forever secretive.
	\item To this end the young child is forever watchful.
	\end{itemize}
}
\end{enumerate}

\section*{copper}
{\large \color{blue}  coppers  }
\subsection*{Explain}
\begin{enumerate}
\item uncountable noun \\
\textbf{Copper} is reddish-brown metal that is used to make things such as coins and electrical wires .
 \textit{
	\begin{itemize}
	\item Chile is the world's largest producer of copper.
	\item ...a copper mine.
	\end{itemize}
}
\item adjective \\
\textbf{Copper} is sometimes used to describe things that are reddish-brown in colour.
 \textit{
	\begin{itemize}
	\item His hair has reverted back to its original copper hue.
	\end{itemize}
}
\item countable noun \\
\textbf{Coppers} are brown metal coins of low value.
 \textit{
	\begin{itemize}
	\end{itemize}
}
\item countable noun \\
A \textbf{copper} is a police  officer .
 \textit{
	\begin{itemize}
	\item ...your friendly neighbourhood copper.
	\end{itemize}
}
\end{enumerate}

\section*{however}
{\large \color{blue}  }
\subsection*{Explain}
\begin{enumerate}
\item adverb \\
You use \textbf{however} when you are adding a comment which is surprising or which contrasts with what has just been said .
 \textit{
	\begin{itemize}
	\item This was not an easy decision. It is, however, a decision that we feel is dictated
by our duty.
	\item Some of the food crops failed. However, the cotton did quite well.
	\item Higher sales have not helped profits, however.
	\end{itemize}
}
\item adverb \\
You use \textbf{however} before an adjective or adverb to emphasize that the degree or extent of something cannot change a situation .
 \textit{
	\begin{itemize}
	\item You should always strive to achieve more, however well you have done before.
	\item However hard she tried, nothing seemed to work.
	\item However famous you are and however rich, losing a baby is the same the world over.
	\item However much it hurt, he could do it.
	\end{itemize}
}
\item conjunction \\
You use \textbf{however} when you want to say that it makes no difference how something is done .
 \textit{
	\begin{itemize}
	\item However we adopt healthcare reform, it isn't going to save major amounts of money.
	\item Wear your hair however you want.
	\end{itemize}
}
\item adverb \\
You use \textbf{however} in expressions such as \textbf{or however long it takes} and \textbf{or however many there were} to indicate that the figure you have just mentioned  may not be accurate .
 \textit{
	\begin{itemize}
	\item The 20,000 or however many who come to watch would love to be out on the pitch.
	\item Wait 30 to 60 minutes or however long it takes.
	\end{itemize}
}
\item adverb \\
You can use \textbf{however} to ask in an emphatic way how something has happened which you are very surprised about. Some speakers of English think that this form is incorrect and prefer to use 'how ever '.
 \textit{
	\begin{itemize}
	\item However did you find this place in such weather?
	\end{itemize}
}
\end{enumerate}

\section*{cousin}
{\large \color{blue}  cousins  }
\subsection*{Explain}
\begin{enumerate}
\item countable noun \\
Your \textbf{cousin} is the child of your uncle or aunt.
 \textit{
	\begin{itemize}
	\item My cousin Mark helped me.
	\item We are cousins.
	\end{itemize}
}
\item countable noun \\
If you refer to two things or groups of people as \textbf{cousins} , you mean that they are equivalents or that there is a connection between them.
 \textit{
	\begin{itemize}
	\item Whereas West Germans drink wine, their Eastern cousins prefer Schnapps.
	\item The average European kitchen is smaller than its American cousin.
	\item ...misanthropy and its cousin racism.
	\end{itemize}
}
\end{enumerate}

\section*{indeed}
{\large \color{blue}  }
\subsection*{Explain}
\begin{enumerate}
\item adverb \\
You use \textbf{indeed} to confirm or agree with something that has just been said .
 \textit{
	\begin{itemize}
	\item Later, he admitted that the payments had indeed been made.
	\item He did indeed keep important documents inside his hat.
	\item 'Did you know him?'—'I did indeed.'.
	\item 'Know what I mean?'—'Indeed I do.'.
	\item 'Isn't it a gorgeous day, Father?'—'Yes, indeed!'.
	\item 'That's a topic which has come to the fore very much recently.'—'Indeed.'
	\end{itemize}
}
\item adverb \\
You use \textbf{indeed} to introduce a further comment or statement which strengthens the point you have already made.
 \textit{
	\begin{itemize}
	\item We have nothing against diversity; indeed, we want more of it.
	\item When we asked to see more we were refused. Indeed we were escorted away by men with
guns.
	\end{itemize}
}
\item adverb \\
You use \textbf{indeed} at the end of a clause to give extra force to the word 'very', or to emphasize a particular word.
 \textit{
	\begin{itemize}
	\item The engine began to sound very loud indeed.
	\item The wine was very good indeed.
	\item Of course, these occasions are rare indeed.
	\end{itemize}
}
\item adverb \\
You can use \textbf{indeed} as a way of repeating a question in order to emphasize it, especially when you do not know the answer .
 \textit{
	\begin{itemize}
	\item 'Now where are the real villains?'—'Where indeed?'.
	\item 'And what do we do here?'—'What, indeed?'
	\end{itemize}
}
\end{enumerate}

\section*{cyberspace}
{\large \color{blue}  }
\subsection*{Explain}
\begin{enumerate}
\item uncountable noun \\
In computer technology , \textbf{cyberspace}  refers to data banks and networks, considered as a place.
 \textit{
	\begin{itemize}
	\end{itemize}
}
\end{enumerate}

\section*{lately}
{\large \color{blue}  }
\subsection*{Explain}
\begin{enumerate}
\item adverb \\
You use \textbf{lately} to describe events in the recent past , or situations that started a short time ago .
 \textit{
	\begin{itemize}
	\item Dad's health hasn't been too good lately.
	\item Lord Tomas had lately been appointed Chairman of the Centre for Policy Studies.
	\item 'Have you talked to her lately?'
	\item Optimism about the U.S. economy has been a rare commodity lately.
	\end{itemize}
}
\item adverb \\
You can use \textbf{lately} to refer to the job a person has been doing until recently.
 \textit{
	\begin{itemize}
	\item ...Timothy Jean Geoffrey Pratt, lately deputy treasury solicitor.
	\end{itemize}
}
\end{enumerate}

\section*{distance}
{\large \color{blue}  distances  distancing  distanced  }
\subsection*{Explain}
\begin{enumerate}
\item variable noun \\
The \textbf{distance}  \textbf{between} two points or places is the amount of space between them.
 \textit{
	\begin{itemize}
	\item ...the distance between the island and the nearby shore.
	\item Everything is within walking distance.
	\item Geographical distance is also a factor.
	\end{itemize}
}
\item uncountable noun \\
When two things are very far apart, you talk about the \textbf{distance} between them.
 \textit{
	\begin{itemize}
	\item The distance wouldn't be a problem.
	\end{itemize}
}
\item adjective \\
\textbf{Distance}  learning or \textbf{distance}  education involves studying at home and sending your work to a college or university , rather than attending the college or university in person.
 \textit{
	\begin{itemize}
	\item I'm doing a theology degree by distance learning.
	\end{itemize}
}
\item uncountable noun \\
When you want to emphasize that two people or things do not have a close relationship or are not the same, you can refer to the \textbf{distance}  \textbf{between} them.
 \textit{
	\begin{itemize}
	\item ...the emotional distance between them.
	\item There was a vast distance between psychological clues and concrete proof.
	\item The ruling party wants too put distance between itself and its predecessor.
	\end{itemize}
}
\item singular noun \\
If you can see something \textbf{in the distance} , you can see it, far away from you.
 \textit{
	\begin{itemize}
	\item We suddenly saw her in the distance.
	\item Mr. Dambar found himself gazing into the distance for a moment or two.
	\end{itemize}
}
\item uncountable noun \\
\textbf{Distance} is coolness or unfriendliness in the way that someone behaves towards you.
 \textit{
	\begin{itemize}
	\item There were periods of sulking, of pronounced distance, of coldness.
	\end{itemize}
}
\item verb \\
If you \textbf{distance}  \textbf{yourself from} a person or thing, or if something \textbf{distances} you \textbf{from} them, you feel less friendly or positive towards them, or become less involved with them.
 \textit{
	\begin{itemize}
	\item The author distanced himself from some of the comments in his book.
	\item Television may actually be distancing the public from the war.
	\end{itemize}
}
\item  \\
 at/from a distance \textit{
	\begin{itemize}
	\end{itemize}
}
\item  \\
 go the distance \textit{
	\begin{itemize}
	\end{itemize}
}
\item  \\
 keep one's distance \textit{
	\begin{itemize}
	\end{itemize}
}
\item  \\
 keep one's distance \textit{
	\begin{itemize}
	\end{itemize}
}
\end{enumerate}

\section*{literally}
{\large \color{blue}  }
\subsection*{Explain}
\begin{enumerate}
\item adverb \\
You can use \textbf{literally} to emphasize a statement . Some careful  speakers of English  think that this use is incorrect .
 \textit{
	\begin{itemize}
	\item We've got to get the economy under control or it will literally eat us up.
	\item The views are literally breath-taking.
	\end{itemize}
}
\item adverb \\
You use \textbf{literally} to emphasize that what you are saying is true , even though it seems  exaggerated or surprising .
 \textit{
	\begin{itemize}
	\item Putting on an opera is a tremendous enterprise involving literally hundreds of people.
	\item I literally crawled to the car.
	\end{itemize}
}
\item adverb \\
If a word or expression is translated  \textbf{literally} , its most simple or basic  meaning is translated.
 \textit{
	\begin{itemize}
	\item The word 'volk' translates literally as 'folk'.
	\item A stanza is, literally, a room.
	\end{itemize}
}
\item  \\
 take sth literally \textit{
	\begin{itemize}
	\end{itemize}
}
\end{enumerate}

\section*{eye}
{\large \color{blue}  eyes  eyeing  eying  eyed  }
\subsection*{Explain}
\begin{enumerate}
\item countable noun \\
Your \textbf{eyes} are the parts of your body with which you see.
 \textit{
	\begin{itemize}
	\item I opened my eyes and looked.
	\item Maria's eyes filled with tears.
	\item ...a tall, thin white-haired lady with piercing dark brown eyes.
	\item He is now blind in one eye.
	\end{itemize}
}
\item verb \\
If you \textbf{eye} someone or something in a particular way, you look at them carefully in that way.
 \textit{
	\begin{itemize}
	\item Sally eyed Claire with interest.
	\item We eyed each other thoughtfully.
	\item Martin eyed the bottle at Marianne's elbow.
	\end{itemize}
}
\item countable noun \\
You use \textbf{eye} when you are talking about a person's ability to judge things or about the way in which they are considering
or dealing with things.
 \textit{
	\begin{itemize}
	\item William was a man of discernment, with an eye for quality.
	\item Their chief negotiator turned his critical eye on the United States.
	\item I cast a practised eye over the sky to determine what the weather would be like.
	\item He first learnt to fish under the watchful eye of his grandmother.
	\end{itemize}
}
\item countable noun \\
An electric  \textbf{eye} or infrared  \textbf{eye} is a device which can recognize the presence of people or objects by detecting the light or heat coming from them.
 \textit{
	\begin{itemize}
	\item An infra-red eye is said to detect the movement of any animal within 10 metres.
	\end{itemize}
}
\item singular noun \\
People sometimes talk about the \textbf{eye} of the camera when they are talking about something being filmed or photographed , or the way something appears in a photograph or film.
 \textit{
	\begin{itemize}
	\item I was again using the cold, unflinching eye of the camera to probe a sick society.
	\end{itemize}
}
\item countable noun \\
An \textbf{eye} on a potato is one of the dark spots from which new stems grow.
 \textit{
	\begin{itemize}
	\end{itemize}
}
\item countable noun \\
An \textbf{eye} is a small metal loop which a hook  fits into, as a fastening on a piece of clothing.
 \textit{
	\begin{itemize}
	\end{itemize}
}
\item countable noun \\
The \textbf{eye} of a needle is the small hole at one end which the thread passes through.
 \textit{
	\begin{itemize}
	\end{itemize}
}
\item singular noun \\
\textbf{The eye of} a storm , tornado, or hurricane is the centre of it.
 \textit{
	\begin{itemize}
	\item The eye of the hurricane hit Florida just south of Miami.
	\end{itemize}
}
\item  \\
 before/in front of/under your eyes \textit{
	\begin{itemize}
	\end{itemize}
}
\item  \\
 cast/run your eye \textit{
	\begin{itemize}
	\end{itemize}
}
\item  \\
 catch someone's eye \textit{
	\begin{itemize}
	\end{itemize}
}
\item  \\
 catch someone's eye \textit{
	\begin{itemize}
	\end{itemize}
}
\item  \\
 to clap eyes on someone \textit{
	\begin{itemize}
	\end{itemize}
}
\item  \\
 to make eye contact \textit{
	\begin{itemize}
	\end{itemize}
}
\item  \\
 to close your eyes to something \textit{
	\begin{itemize}
	\end{itemize}
}
\item  \\
 to cry your eyes out \textit{
	\begin{itemize}
	\end{itemize}
}
\item  \\
 an eye for an eye \textit{
	\begin{itemize}
	\end{itemize}
}
\item  \\
 as far as the eye can see \textit{
	\begin{itemize}
	\end{itemize}
}
\item  \\
 have an eye for something \textit{
	\begin{itemize}
	\end{itemize}
}
\item  \\
 in/to someone's eyes \textit{
	\begin{itemize}
	\end{itemize}
}
\item  \\
 to keep your eyes open \textit{
	\begin{itemize}
	\end{itemize}
}
\item  \\
 to keep your eyes peeled \textit{
	\begin{itemize}
	\end{itemize}
}
\item  \\
 keep your eye on something \textit{
	\begin{itemize}
	\end{itemize}
}
\item  \\
 make eyes at someone \textit{
	\begin{itemize}
	\end{itemize}
}
\item  \\
 there's more to this than meets the eye \textit{
	\begin{itemize}
	\end{itemize}
}
\item  \\
 to meet someone's eyes \textit{
	\begin{itemize}
	\end{itemize}
}
\item  \\
 all eyes are on something \textit{
	\begin{itemize}
	\end{itemize}
}
\item  \\
 have your eye on someone \textit{
	\begin{itemize}
	\end{itemize}
}
\item  \\
 have your eye on something \textit{
	\begin{itemize}
	\end{itemize}
}
\item  \\
 with your eyes open \textit{
	\begin{itemize}
	\end{itemize}
}
\item  \\
 to open your eyes \textit{
	\begin{itemize}
	\end{itemize}
}
\item  \\
 to see eye to eye \textit{
	\begin{itemize}
	\end{itemize}
}
\item  \\
 the eye of the storm \textit{
	\begin{itemize}
	\end{itemize}
}
\item  \\
 take your eyes off something \textit{
	\begin{itemize}
	\end{itemize}
}
\item  \\
 through someone's eyes \textit{
	\begin{itemize}
	\end{itemize}
}
\item  \\
 up to your eyes \textit{
	\begin{itemize}
	\end{itemize}
}
\end{enumerate}

\section*{may}
{\large \color{blue}  }
\subsection*{Explain}
\begin{enumerate}
\item modal verb \\
You use \textbf{may} to indicate that something will possibly happen or be true in the future , but you cannot be certain.
 \textit{
	\begin{itemize}
	\item We may have some rain today.
	\item Rates may rise, but it won't be by much and it won't be for long.
	\item I may be back next year.
	\item I don't know if they'll publish it or not. They may.
	\item Scientists know that cancer may not show up for many years.
	\end{itemize}
}
\item modal verb \\
You use \textbf{may} to indicate that there is a possibility that something is true, but you cannot be
certain.
 \textit{
	\begin{itemize}
	\item Civil rights officials say there may be hundreds of other cases of racial violence.
	\item Throwing good money after bad may not be a good idea, they say.
	\end{itemize}
}
\item modal verb \\
You use \textbf{may} to indicate that something is sometimes true or is true in some circumstances .
 \textit{
	\begin{itemize}
	\item A vegetarian diet may not provide enough calories for a child's normal growth.
	\item Up to five inches of snow may cover the mountains.
	\item ...families that may have both parents working.
	\end{itemize}
}
\item modal verb \\
You use \textbf{may have} with a past  participle when suggesting that it is possible that something happened or was true, or when giving a possible explanation for something.
 \textit{
	\begin{itemize}
	\item He may have been to some of those places.
	\item The chaos may have contributed to the deaths of up to 20 people.
	\item Investigators say that a fuel explosion may have caused the crash.
	\item The events may or may not have been connected.
	\end{itemize}
}
\item modal verb \\
You use \textbf{may} in statements where you are accepting the truth of a situation , but contrasting it with something that is more important .
 \textit{
	\begin{itemize}
	\item I may be almost 50, but there aren't a lot of things I've forgotten.
	\item The elderly man may not be typical, but he speaks for a significant body of opinion.
	\item Walking may be boring at times but on a sunny morning there is nothing finer.
	\end{itemize}
}
\item modal verb \\
You use \textbf{may} when you are mentioning a quality or fact about something that people can make use of if they want to.
 \textit{
	\begin{itemize}
	\item The bag has narrow straps, so it may be worn over the shoulder or carried in the
hand.
	\item Some of the diseases of middle age may be prevented by improving nutrition.
	\end{itemize}
}
\item modal verb \\
You use \textbf{may} to indicate that someone is allowed to do something, usually because of a rule or law . You use \textbf{may not} to indicate that someone is not allowed to do something.
 \textit{
	\begin{itemize}
	\item What is the nearest you may park to a junction?
	\item Adolescents under the age of 18 may not work in jobs that require them to drive.
	\end{itemize}
}
\item modal verb \\
You use \textbf{may} when you are giving permission to someone to do something, or when asking for permission.
 \textit{
	\begin{itemize}
	\item Mr Hobbs? May we come in?
	\item If you wish, you may now have a glass of milk.
	\item 'You may leave.'—'Yes, sir.'
	\end{itemize}
}
\item modal verb \\
You use \textbf{may} when you are making polite requests.
 \textit{
	\begin{itemize}
	\item I'd like the use of your living room, if I may.
	\item May I come with you to Southampton?
	\item Ah, Julia, my dear, here is our guest. May we have some tea?
	\end{itemize}
}
\item modal verb \\
You use \textbf{may} , usually in questions, when you are politely making suggestions or offering to do something.
 \textit{
	\begin{itemize}
	\item May we suggest you try one of our guest houses.
	\item May we recommend a weekend in Stockholm?
	\item Do sit down. And may we offer you something to drink?
	\item May I help you?
	\end{itemize}
}
\item modal verb \\
You use \textbf{may} as a polite way of interrupting someone, asking a question, or introducing what you are going to say  next .
 \textit{
	\begin{itemize}
	\item 'If I may interrupt for a moment,' Kenneth said.
	\item Anyway, may I just ask you one other thing?
	\item If I may return to what we were talking about earlier.
	\end{itemize}
}
\item modal verb \\
You use \textbf{may} when you are mentioning the reaction or attitude that you think someone is likely to have to something you are about to say.
 \textit{
	\begin{itemize}
	\item You know, Brian, whatever you may think, I work hard for a living.
	\item You may consider it useless, but for our customers it's an all-important sign of
good service.
	\end{itemize}
}
\item modal verb \\
You use \textbf{may} in expressions such as \textbf{I may add} and \textbf{I may say} in order to emphasize a statement that you are making.
 \textit{
	\begin{itemize}
	\item They spent their afternoons playing golf–extremely badly, I may add–around Loch Lomond.
	\item Both of them, I may say, are thoroughly reliable men.
	\end{itemize}
}
\item modal verb \\
If you do something so that a particular thing \textbf{may} happen, you do it so that there is an opportunity for that thing to happen.
 \textit{
	\begin{itemize}
	\item ...the need for more surgeons so that patients may be treated more quickly.
	\item The door is shut so that no one may overhear what is said.
	\end{itemize}
}
\item modal verb \\
People sometimes use \textbf{may} to express hopes and wishes.
 \textit{
	\begin{itemize}
	\item Courage seems now to have deserted him. May it quickly reappear.
	\end{itemize}
}
\end{enumerate}

\section*{fiction}
{\large \color{blue}  fictions  }
\subsection*{Explain}
\begin{enumerate}
\item uncountable noun \\
\textbf{Fiction}  refers to books and stories about imaginary people and events , rather than books about real people or events.
 \textit{
	\begin{itemize}
	\item Immigrant tales have always been popular themes in fiction.
	\item Diana is a writer of historical fiction.
	\end{itemize}
}
\item uncountable noun \\
A statement or account that is \textbf{fiction} is not true.
 \textit{
	\begin{itemize}
	\item The truth or fiction of this story has never been truly determined.
	\end{itemize}
}
\item countable noun \\
If something is a \textbf{fiction} , it is not true, although people sometimes  pretend that it is true.
 \textit{
	\begin{itemize}
	\item Is the idea of 'true love' a fiction in itself?
	\end{itemize}
}
\end{enumerate}

\section*{maybe}
{\large \color{blue}  }
\subsection*{Explain}
\begin{enumerate}
\item adverb \\
You use \textbf{maybe} to express  uncertainty , for example when you do not know that something is definitely  true , or when you are mentioning something that may possibly happen in the future in the way you describe .
 \textit{
	\begin{itemize}
	\item Maybe she is in love.
	\item Maybe he sincerely wanted to help his country.
	\item I do think about having children, maybe when I'm 40.
	\item Things are maybe not as good as they should be.
	\item Bill will come on then maybe Ralph, then Bobby and Johnny doing their hits.
	\end{itemize}
}
\item adverb \\
You use \textbf{maybe} when you are making suggestions or giving  advice . \textbf{Maybe} is also used to introduce  polite  requests .
 \textit{
	\begin{itemize}
	\item Maybe we can go to the movies or something.
	\item Maybe you'd better tell me what this is all about.
	\item Maybe you shouldn't eat in that restaurant anymore.
	\item Maybe if you tell me a little about her?
	\item Wait a while, maybe a few days.
	\end{itemize}
}
\item adverb \\
You use \textbf{maybe} to indicate that, although a comment is partly true, there is also another point of view that should be considered .
 \textit{
	\begin{itemize}
	\item Maybe there is jealousy, but I think the envy is more powerful.
	\item OK, maybe I am a failure, but, in my opinion, no more than the rest of this country.
	\end{itemize}
}
\item adverb \\
You can  say  \textbf{maybe} as a response to a question or remark , when you do not want to agree or disagree .
 \textit{
	\begin{itemize}
	\item 'Is she coming back?'—'Maybe. No one hears from her.'
	\item 'People will like you the way you are.'—'Maybe.'
	\end{itemize}
}
\item adverb \\
You use \textbf{maybe} when you are making a rough  guess at a number , quantity , or value , rather than stating it exactly .
 \textit{
	\begin{itemize}
	\item The men were maybe a hundred feet away and coming closer.
	\end{itemize}
}
\item adverb \\
People use \textbf{maybe} to mean ' sometimes ', particularly in a series of general  statements about what someone does, or about something that regularly happens.
 \textit{
	\begin{itemize}
	\item They'll come to the bar for a year, or maybe even two.
	\end{itemize}
}
\end{enumerate}

\section*{infant}
{\large \color{blue}  infants  }
\subsection*{Explain}
\begin{enumerate}
\item countable noun \\
An \textbf{infant} is a baby or very young child.
 \textit{
	\begin{itemize}
	\item ...young mums with infants in prams.
	\item The family were forced to flee with their infant son.
	\item ...the infant mortality rate in Britain.
	\end{itemize}
}
\item countable noun \\
\textbf{Infants} are children between the ages of five and seven, who go to an infant school .
 You use \textbf{the infants} to refer to a school or class for such children.
 \textit{
	\begin{itemize}
	\item You've been my best friend ever since we started in the infants.
	\end{itemize}
}
\item adjective \\
\textbf{Infant} means designed especially for very young children.
 \textit{
	\begin{itemize}
	\item ...an infant carrier in the back of a car.
	\end{itemize}
}
\item adjective \\
An \textbf{infant} organization or system is new and has not developed very much.
 \textit{
	\begin{itemize}
	\item The infant company was based in Germany.
	\item ...the infant health service.
	\end{itemize}
}
\end{enumerate}

\section*{merely}
{\large \color{blue}  }
\subsection*{Explain}
\begin{enumerate}
\item adverb \\
You use \textbf{merely} to emphasize that something is only what you say and not better , more important , or more exciting .
 \textit{
	\begin{itemize}
	\item Michael is now merely a good friend.
	\item Francis Watson was far from being merely a furniture expert.
	\item Merely because you believe a thing is right, it isn't automatically so.
	\item They are offering merely technical assistance.
	\end{itemize}
}
\item adverb \\
You use \textbf{merely} to emphasize that a particular amount or quantity is very small.
 \textit{
	\begin{itemize}
	\item The brain accounts for merely three per cent of body weight.
	\end{itemize}
}
\item  \\
 not merely sth \textit{
	\begin{itemize}
	\end{itemize}
}
\end{enumerate}

\section*{ladder}
{\large \color{blue}  ladders  }
\subsection*{Explain}
\begin{enumerate}
\item countable noun \\
A \textbf{ladder} is a piece of equipment used for climbing up something or down from something. It consists of two long pieces
of wood, metal, or rope with steps fixed between them.
 \textit{
	\begin{itemize}
	\end{itemize}
}
\item singular noun \\
You can use \textbf{the}  \textbf{ladder} to refer to something such as a society , organization , or system which has different levels that people can progress up or drop down.
 \textit{
	\begin{itemize}
	\item If they want to climb the ladder of success they should be given that opportunity.
	\item She admired her mother's sister for moving up the social ladder.
	\end{itemize}
}
\item countable noun \\
A \textbf{ladder} is a hole or torn part in a woman's stocking or tights, where some of the vertical  threads have broken , leaving only the horizontal threads.
 \textit{
	\begin{itemize}
	\item Her hair was a mess and there was a ladder in her tights.
	\end{itemize}
}
\end{enumerate}

\section*{might}
{\large \color{blue}  }
\subsection*{Explain}
\begin{enumerate}
\item modal verb \\
You use \textbf{might} to indicate that something will possibly happen or be true in the future, but you cannot be certain.
 \textit{
	\begin{itemize}
	\item Smoking might be banned totally in most buildings.
	\item The two countries might go to war.
	\item I might well regret it later.
	\item He said he might not be back until tonight.
	\end{itemize}
}
\item modal verb \\
You use \textbf{might} to indicate that there is a possibility that something is true, but you cannot be
certain.
 \textit{
	\begin{itemize}
	\item She and Simon's father had not given up hope that he might be alive.
	\item You might be right.
	\item They haven't seen each other for five years; he might not be interested in her any
more.
	\item ...a suit that looks as though it might contain polyester.
	\end{itemize}
}
\item modal verb \\
You use \textbf{might} to indicate that something could happen or be true in particular circumstances .
 \textit{
	\begin{itemize}
	\item Your child might do better with a different teacher.
	\item ...the type of person who might appear in a fashion magazine.
	\end{itemize}
}
\item modal verb \\
You use \textbf{might have} with a past  participle to indicate that it is possible that something happened or was true, or when giving a possible explanation for something.
 \textit{
	\begin{itemize}
	\item I heard what might have been an explosion.
	\item She thought the shooting might have been an accident.
	\item The equipment needed to clean up the spill might not have arrived yet.
	\item The letters might not have been meant for me at all.
	\end{itemize}
}
\item modal verb \\
You use \textbf{might have} with a past participle to indicate that something was a possibility in the past,
although it did not actually happen.
 \textit{
	\begin{itemize}
	\item If she had had to give up riding she might have taken up sailing competitively.
	\item Had the bomb dropped on a city, there might have been a great deal of damage.
	\item The report might have been better written.
	\item I didn't give my name because if I did I thought you might not have come.
	\end{itemize}
}
\item modal verb \\
You use \textbf{might} in statements where you are accepting the truth of a situation , but contrasting it with something that is more important .
 \textit{
	\begin{itemize}
	\item He might be a bore, but he was as quick-witted as a weasel.
	\item She might not have much energy but she still has a stinging wit.
	\end{itemize}
}
\item modal verb \\
You use \textbf{might} when you are saying  emphatically that someone ought to do the thing mentioned , especially when you are annoyed because they have not done it.
 \textit{
	\begin{itemize}
	\item And while I'm out you might clean up the kitchen.
	\item You might have told me that before!
	\end{itemize}
}
\item modal verb \\
You use \textbf{might} to make a suggestion or to give advice in a very polite way.
 \textit{
	\begin{itemize}
	\item They might be wise to stop advertising on television.
	\item You might try the gas station down the street.
	\item You might want to consider cycling.
	\item I was just wondering if you might like to go feed the cat.
	\item I thought we might go for a drive on Sunday.
	\item It might be a good idea to tell your husband.
	\end{itemize}
}
\item modal verb \\
You use \textbf{might} as a polite way of interrupting someone, asking a question , making a request , or introducing what you are going to say  next .
 \textit{
	\begin{itemize}
	\item Might I make a suggestion?
	\item Might I ask what you're doing here?
	\item Might I trouble you for a drop more tea?
	\item I was wondering if I might talk to you for a moment.
	\item Might I draw your readers' attention to the dangers in the Government's proposal.
	\end{itemize}
}
\item modal verb \\
You use \textbf{might} in expressions such as \textbf{as you might expect} and \textbf{as you might imagine} in order to indicate that the statement you are making is not surprising .
 \textit{
	\begin{itemize}
	\item 'How's Jan?' she asked.—'Bad. As you might expect.'.
	\item The drivers, as you might imagine, didn't care much for that.
	\end{itemize}
}
\item modal verb \\
You use \textbf{might} in expressions such as \textbf{I might add} and \textbf{I might say} in order to emphasize a statement that you are making.
 \textit{
	\begin{itemize}
	\item Relatives ring up constantly, not always for the best motives, I might add.
	\item It didn't come as a great surprise to me, I might say.
	\end{itemize}
}
\item modal verb \\
You use \textbf{might} in expressions such as \textbf{I might have known} and \textbf{I might have guessed} to indicate that you are not surprised at a disappointing event or fact .
 \textit{
	\begin{itemize}
	\item I might have known I'd find you with her.
	\item 'I detest clutter, you know.'—'I didn't know, but I might have guessed.'
	\end{itemize}
}
\end{enumerate}

\section*{logic}
{\large \color{blue}  }
\subsection*{Explain}
\begin{enumerate}
\item uncountable noun \\
\textbf{Logic} is a method of reasoning that involves a series of statements , each of which must be true if the statement before it is true.
 \textit{
	\begin{itemize}
	\item Students learn forensic medicine, philosophy and logic.
	\end{itemize}
}
\item uncountable noun \\
The \textbf{logic} of a conclusion or an argument is its quality of being correct and reasonable .
 \textit{
	\begin{itemize}
	\item I don't follow the logic of your argument.
	\item There would be no logic in upsetting the agreements.
	\end{itemize}
}
\item uncountable noun \\
A particular kind of \textbf{logic} is the way of thinking and reasoning about things that is characteristic of a particular type of person
or particular field of activity.
 \textit{
	\begin{itemize}
	\item The plan was based on sound commercial logic.
	\end{itemize}
}
\end{enumerate}

\section*{loop}
{\large \color{blue}  loops  looping  looped  }
\subsection*{Explain}
\begin{enumerate}
\item countable noun \\
A \textbf{loop} is a curved or circular shape in something long, for example in a piece of string.
 \textit{
	\begin{itemize}
	\item Mrs. Morrell reached for a loop of garden hose.
	\end{itemize}
}
\item verb \\
If you \textbf{loop} something such as a piece of rope around an object, you tie a length of it in a loop around the object, for example in order to fasten it to
the object.
 \textit{
	\begin{itemize}
	\item He looped the rope over the wood.
	\item He wore the watch and chain looped round his neck like a medallion.
	\end{itemize}
}
\item verb \\
If something \textbf{loops}  somewhere , it goes there in a circular direction that makes the shape of a loop.
 \textit{
	\begin{itemize}
	\item The enemy was looping around the south side.
	\item The helicopter took off and headed north. Then it looped west, heading for the hills.
	\end{itemize}
}
\item  \\
 be in the loop \textit{
	\begin{itemize}
	\end{itemize}
}
\end{enumerate}

\section*{need}
{\large \color{blue}  needs  needing  needed  }
\subsection*{Explain}
\begin{enumerate}
\item verb \\
If you \textbf{need} something, or \textbf{need}  \textbf{to} do something, you cannot successfully achieve what you want or live properly without it.
 \textbf{Need} is also a noun.
 \textit{
	\begin{itemize}
	\item He desperately needed money.
	\item These diets provide everything your body needs.
	\item I need to make a phone call.
	\item A baby does not need to wear shoes until he starts to walk.
	\item I need you to do something for me.
	\item I need you here, Wally.
	\item I need you sane and sober.
	\item Charles has never felt the need to compete with anyone.
	\item ...the child who never had his need for attention and importance satisfied.
	\item ...the special nutritional needs of the elderly.
	\end{itemize}
}
\item verb \\
If an object or place \textbf{needs} something doing to it, that action should be done to improve the object or place. If a task  \textbf{needs} doing, it should be done to improve a particular situation.
 \textit{
	\begin{itemize}
	\item The building needs quite a few repairs.
	\item ...a garden that needs tidying.
	\item The taste of vitamins is not too nice so the flavour sometimes needs to be disguised.
	\end{itemize}
}
\item singular noun \\
If there is a \textbf{need}  \textbf{for} something, that thing would improve a situation or something cannot happen without it.
 \textit{
	\begin{itemize}
	\item Mr Forrest believes there is a need for other similar schools throughout Britain.
	\item 'I think we should see a specialist.'—'I don't think there's any need for that.'
	\item There's no need for you to stay.
	\end{itemize}
}
\item modal verb \\
If you say that someone \textbf{needn't} do something, you are telling them not to do it, or advising or suggesting that they should not do it.
 \textbf{Need} is also a verb .
 \textit{
	\begin{itemize}
	\item 'I'll put the key in the window.'—'You needn't bother,' he said gruffly.
	\item Look, you needn't shout.
	\item She need not know I'm here.
	\item Well, for Heaven's sake, you don't need to apologize.
	\item Come along, Mother, we don't need to take up any more of Mr Kemp's time.
	\end{itemize}
}
\item modal verb \\
If you tell someone that they \textbf{needn't} do something, or that something \textbf{needn't} happen, you are telling them that that thing is not necessary, in order to make them
 feel  better .
 \textbf{Need} is also a verb.
 \textit{
	\begin{itemize}
	\item You needn't worry.
	\item This needn't take long, Simon.
	\item Buying budget-priced furniture needn't mean compromising on quality or style.
	\item Loneliness can be horrible, but it need not remain that way.
	\item He replied, with a reassuring smile, 'Oh, you don't need to worry about them.'
	\item You don't need to be a millionaire to consider having a bank account in Switzerland.
	\end{itemize}
}
\item modal verb \\
You use \textbf{needn't} when you are giving someone permission not to do something.
 \textbf{Need} is also a verb.
 \textit{
	\begin{itemize}
	\item You needn't come again, if you don't want to.
	\item Well, you needn't tell me anything if you don't want to.
	\item You don't need to wait for me.
	\item Mommy, you don't need to stay while we talk.
	\end{itemize}
}
\item modal verb \\
If something \textbf{need not} be true , it is not necessarily true or not always true.
 \textit{
	\begin{itemize}
	\item What is right for us need not be right for others.
	\item Freedom need not mean independence.
	\end{itemize}
}
\item modal verb \\
If someone \textbf{needn't}  \textbf{have} done something, it was not necessary or useful for them to do it, although they did it.
 If someone \textbf{didn't need to} do something, they needn't have done it.
 \textit{
	\begin{itemize}
	\item I was a little nervous when I announced my engagement to Grace, but I needn't have
worried.
	\item We spent a hell of a lot of money that we needn't have spent.
	\item You didn't need to give me any more money you know, but thank you.
	\end{itemize}
}
\item modal verb \\
You use \textbf{need} in expressions such as \textbf{I need hardly say} and \textbf{I needn't add} to emphasize that the person you are talking to already  knows what you are going to say.
 \textbf{Need} is also a verb.
 \textit{
	\begin{itemize}
	\item I needn't add that if you fail to do as I ask, you will suffer the consequences.
	\item I hardly need to say that I have never lost contact with him.
	\end{itemize}
}
\item modal verb \\
You can use \textbf{need} in expressions such as ' \textbf{Need I say more} ' and ' \textbf{Need I go on} ' when you want to avoid stating an obvious  consequence of something you have just said .
 \textit{
	\begin{itemize}
	\item Mid-fifties, short black hair, grey moustache, distinctive Russian accent. Need I
go on?
	\end{itemize}
}
\item  \\
 in need \textit{
	\begin{itemize}
	\end{itemize}
}
\item  \\
 in need of \textit{
	\begin{itemize}
	\end{itemize}
}
\item  \\
 if need be/if needs be \textit{
	\begin{itemize}
	\end{itemize}
}
\item  \\
 someone needs to get out more \textit{
	\begin{itemize}
	\end{itemize}
}
\item  \\
 there's no need/no need \textit{
	\begin{itemize}
	\end{itemize}
}
\item  \\
 who needs sth? \textit{
	\begin{itemize}
	\end{itemize}
}
\end{enumerate}

\section*{luxury}
{\large \color{blue}  luxuries  }
\subsection*{Explain}
\begin{enumerate}
\item uncountable noun \\
\textbf{Luxury} is very great  comfort , especially among beautiful and expensive surroundings.
 \textit{
	\begin{itemize}
	\item By all accounts he leads a life of considerable luxury.
	\item She was brought up in an atmosphere of luxury and wealth.
	\end{itemize}
}
\item countable noun \\
A \textbf{luxury} is something expensive which is not necessary but which gives you pleasure .
 \textit{
	\begin{itemize}
	\item A week by the sea is a luxury they can no longer afford.
	\item Tablets are still a luxury here.
	\end{itemize}
}
\item adjective \\
A \textbf{luxury}  item is something expensive which is not necessary but which gives you pleasure.
 \textit{
	\begin{itemize}
	\item He could not afford luxury food on his pay.
	\item He rode on the president's luxury train through his own state.
	\end{itemize}
}
\item singular noun \\
A \textbf{luxury} is a pleasure which you do not often have the opportunity to enjoy .
 \textit{
	\begin{itemize}
	\item Hot baths are my favourite luxury.
	\item We were going to have the luxury of a free weekend, to rest and do whatever we pleased.
	\end{itemize}
}
\end{enumerate}

\section*{often}
{\large \color{blue}  }
\subsection*{Explain}
\begin{enumerate}
\item adverb \\
If something \textbf{often}  happens , it happens many times or much of the time.
 \textit{
	\begin{itemize}
	\item They often spent Christmas at Prescott Hill.
	\item Early American weathervanes were most often cut from flat wooden boards.
	\item They used these words freely, often in front of their parents too.
	\item It was often hard to work and do the course at the same time.
	\item That doesn't happen very often.
	\end{itemize}
}
\item adverb \\
You use \textbf{how often} to ask  questions about frequency . You also use \textbf{often} in reported  clauses and other statements to give information about the frequency of something.
 \textit{
	\begin{itemize}
	\item How often do you brush your teeth?
	\item I don't know how often I heard the same awful jokes.
	\item Unemployed people were victims of personal crime twice as often as employed people.
	\end{itemize}
}
\item  \\
 every so often \textit{
	\begin{itemize}
	\end{itemize}
}
\item  \\
 as often as not \textit{
	\begin{itemize}
	\end{itemize}
}
\end{enumerate}

\section*{melon}
{\large \color{blue}  melons  }
\subsection*{Explain}
\begin{enumerate}
\item variable noun \\
A \textbf{melon} is a large fruit which is sweet and juicy inside and has a hard green or yellow  skin .
 \textit{
	\begin{itemize}
	\end{itemize}
}
\end{enumerate}

\section*{ought}
{\large \color{blue}  }
\subsection*{Explain}
\begin{enumerate}
\item phrase \\
You use \textbf{ought to} to mean that it is morally right to do a particular thing or that it is morally right for a particular situation to exist , especially when giving or asking for advice or opinions .
 \textit{
	\begin{itemize}
	\item If you get something good, you ought to share it.
	\item People who own a bit of money ought to have a voice in saying where it goes.
	\item You ought to be ashamed of yourselves. You've created this problem.
	\end{itemize}
}
\item phrase \\
You use \textbf{ought to} when saying that you think it is a good  idea and important for you or someone else to do a particular thing, especially when giving or asking
for advice or opinions.
 \textit{
	\begin{itemize}
	\item You don't have to be alone with him and I don't think you ought to be.
	\item You ought to ask a lawyer's advice.
	\item She wondered if she ought to take some coffee out to Alfred.
	\item We ought not to be quarrelling now.
	\end{itemize}
}
\item phrase \\
You use \textbf{ought to} to indicate that you expect something to be true or to happen . You use \textbf{ought to have} to indicate that you expect something to have happened already .
 \textit{
	\begin{itemize}
	\item 'This ought to be fun,' he told Alex, eyes gleaming.
	\end{itemize}
}
\item phrase \\
You use \textbf{ought to} to indicate that you think that something should be the case , but might not be.
 \textit{
	\begin{itemize}
	\item By rights the Social Democrats ought to be the favourites in the election. But nothing
looks less certain.
	\item Though this gives them a nice feeling, it really ought to worry them.
	\end{itemize}
}
\item phrase \\
You use \textbf{ought to} to indicate that you think that something has happened because of what you know about the situation, but you are not certain .
 \textit{
	\begin{itemize}
	\item He ought to have reached the house some time ago.
	\end{itemize}
}
\item phrase \\
You use \textbf{ought to have} with a past  participle to indicate that something was expected to happen or be the case, but it did not
happen or was not the case.
 \textit{
	\begin{itemize}
	\item Basically the system ought to have worked.
	\item The money to build the power station ought to have been sufficient.
	\end{itemize}
}
\item phrase \\
You use \textbf{ought to have} with a past participle to indicate that although it was best or correct for someone to do something in the past, they did not actually do it.
 \textit{
	\begin{itemize}
	\item I realize I ought to have told you about it.
	\item Perhaps we ought to have trusted people more.
	\item I ought not to have asked you a thing like that. I'm sorry.
	\item I'm beginning to feel now we oughtn't to have let her go away like that.
	\end{itemize}
}
\item phrase \\
You use \textbf{ought to} when politely telling someone that you must do something, for example that you must leave .
 \textit{
	\begin{itemize}
	\item I really ought to be getting back now.
	\item I think I ought to go.
	\end{itemize}
}
\end{enumerate}

\section*{menu}
{\large \color{blue}  menus  }
\subsection*{Explain}
\begin{enumerate}
\item countable noun \\
In a restaurant or café, or at a formal meal, the \textbf{menu} is a list of the food and drinks that are available .
 \textit{
	\begin{itemize}
	\item A waiter offered him the menu.
	\item Even the most elaborate dishes on the menu were quite low on calories.
	\end{itemize}
}
\item countable noun \\
A \textbf{menu} is the food that you serve at a meal.
 \textit{
	\begin{itemize}
	\item Try out the menu on a few friends.
	\item The menu is all-important. Every component of every meal should create contrasts.
	\end{itemize}
}
\item countable noun \\
On a computer  screen , a \textbf{menu} is a list of choices . Each choice represents something that you can do using the computer.
 \textit{
	\begin{itemize}
	\end{itemize}
}
\end{enumerate}

\section*{presumably}
{\large \color{blue}  }
\subsection*{Explain}
\begin{enumerate}
\item adverb \\
If you say that something is \textbf{presumably} the case , you mean that you think it is very likely to be the case, although you are not certain.
 \textit{
	\begin{itemize}
	\item Presumably the front door was locked when you came down this morning?
	\item The spear is presumably the murder weapon.
	\item He had gone to the reception desk, presumably to check out.
	\end{itemize}
}
\end{enumerate}

\section*{nutrition}
{\large \color{blue}  }
\subsection*{Explain}
\begin{enumerate}
\item uncountable noun \\
\textbf{Nutrition} is the process of taking food into the body and absorbing the nutrients in those foods.
 \textit{
	\begin{itemize}
	\item There are alternative sources of nutrition to animal meat.
	\item As in all experimental sciences, we still do not know everything about nutrition.
	\end{itemize}
}
\end{enumerate}

\section*{readily}
{\large \color{blue}  }
\subsection*{Explain}
\begin{enumerate}
\item adverb \\
If you do something \textbf{readily} , you do it in a way which shows that you are very willing to do it.
 \textit{
	\begin{itemize}
	\item I asked her if she would allow me to interview her, and she readily agreed.
	\item When I was invited to the party, I readily accepted.
	\end{itemize}
}
\item adverb \\
You also use \textbf{readily} to say that something can be done or obtained quickly and easily. For example , if you say that something can be readily understood , you mean that people can understand it quickly and easily.
 \textit{
	\begin{itemize}
	\item The components are readily available in hardware shops.
	\item I don't readily make friends.
	\end{itemize}
}
\end{enumerate}

\section*{pail}
{\large \color{blue}  pails  }
\subsection*{Explain}
\begin{enumerate}
\item countable noun \\
A \textbf{pail} is a bucket, usually made of metal or wood.
 \textit{
	\begin{itemize}
	\end{itemize}
}
\end{enumerate}

\section*{really}
{\large \color{blue}  }
\subsection*{Explain}
\begin{enumerate}
\item adverb \\
You can use \textbf{really} to emphasize a statement .
 \textit{
	\begin{itemize}
	\item I'm very sorry. I really am.
	\item It really is best to manage without any medication if you possibly can.
	\item I really do feel that some people are being unfair.
	\item You know, we really ought to get another car.
	\item I'm fine, really I'm fine.
	\end{itemize}
}
\item adverb \\
You can use \textbf{really} to emphasize an adjective or adverb .
 \textit{
	\begin{itemize}
	\item It was really good.
	\item They were really nice people.
	\item I know her really well.
	\end{itemize}
}
\item adverb \\
You use \textbf{really} when you are discussing the real  facts about something, in contrast to the ones someone wants you to believe .
 \textit{
	\begin{itemize}
	\item My father didn't really love her.
	\item What was really going on?
	\item You make them feel that it was their decision when it wasn't really.
	\end{itemize}
}
\item adverb \\
People use \textbf{really} in questions and negative statements when they want you to answer 'no'.
 \textit{
	\begin{itemize}
	\item Do you really think he would be that stupid?
	\item You can't really expect me to believe you didn't know him.
	\end{itemize}
}
\item adverb \\
If you refer to a time when something \textbf{really}  begins to happen , you are emphasizing that it starts to happen at that time to a much greater  extent and much more seriously than before.
 \textit{
	\begin{itemize}
	\item That's when the pressure really started.
	\item He only really started going out with girls at college.
	\end{itemize}
}
\item adverb \\
People sometimes use \textbf{really} to slightly  reduce the force of a negative statement.
 \textit{
	\begin{itemize}
	\item I'm not really surprised.
	\item 'Did they hurt you?'—'Not really'.
	\item I didn't really notice what I was eating.
	\item I don't think that's very fair really.
	\end{itemize}
}
\item adverb \\
People sometimes add  \textbf{really} to statements in order to make them less definite and more uncertain .
 \textit{
	\begin{itemize}
	\item She is a quiet girl really.
	\item I'm happy most of the time, really.
	\end{itemize}
}
\item adverb \\
People use \textbf{really} to show that they are surprised or that the person they are speaking to may be surprised about something.
 \textit{
	\begin{itemize}
	\item Actually it was quite good really.
	\item I was really rather fond of Arthur.
	\end{itemize}
}
\item convention \\
You can say  \textbf{really} to express surprise or disbelief at what someone has said .
 \textit{
	\begin{itemize}
	\item 'We discovered it was totally the wrong decision.'—'Really?'.
	\item 'We saw a very bright shooting star.'—'Did you really?'
	\end{itemize}
}
\item convention \\
You can say ' \textbf{really} ' in a conversation to show that you are interested in what someone is saying .
 \textit{
	\begin{itemize}
	\item 'We had a very interesting chat.'—'Really? About what?'
	\end{itemize}
}
\item exclamation \\
Some people say \textbf{really} when they are slightly annoyed or offended by something.
 \textit{
	\begin{itemize}
	\item Really, Mr Riss, I expected better of you.
	\end{itemize}
}
\end{enumerate}

\section*{pear}
{\large \color{blue}  pears  }
\subsection*{Explain}
\begin{enumerate}
\item countable noun \\
A \textbf{pear} is a sweet, juicy fruit which is narrow near its stalk , and wider and rounded at the bottom . Pears have white flesh and thin  green or yellow  skin .
 \textit{
	\begin{itemize}
	\end{itemize}
}
\end{enumerate}

\section*{selfish}
{\large \color{blue}  }
\subsection*{Explain}
\begin{enumerate}
\item adjective \\
If you say that someone is \textbf{selfish} , you mean that he or she cares only about himself or herself, and not about other people.
 \textit{
	\begin{itemize}
	\item I think I've been very selfish. I've been mainly concerned with myself.
	\item ...the selfish interests of a few people.
	\end{itemize}
}
\end{enumerate}

\section*{picnic}
{\large \color{blue}  picnics  picnicking  picnicked  }
\subsection*{Explain}
\begin{enumerate}
\item countable noun \\
When people have a \textbf{picnic} , they eat a meal out of doors , usually in a field or a forest , or at the beach .
 \textit{
	\begin{itemize}
	\item We're going on a picnic tomorrow.
	\item We'll take a picnic lunch.
	\end{itemize}
}
\item verb \\
When people \textbf{picnic}  somewhere , they have a picnic.
 \textit{
	\begin{itemize}
	\item Afterwards, we picnicked on the riverbank.
	\item ...such a perfect day for picnicking.
	\end{itemize}
}
\item  \\
 be no picnic \textit{
	\begin{itemize}
	\end{itemize}
}
\end{enumerate}

\section*{shall}
{\large \color{blue}  }
\subsection*{Explain}
\begin{enumerate}
\item modal verb \\
You use \textbf{shall} with 'I' and 'we' in questions in order to make offers or suggestions , or to ask for advice .
 \textit{
	\begin{itemize}
	\item Shall I get the keys?
	\item I bought some lovely raisin buns at the bakery. Shall I bring you one with some tea?
	\item Shall I call her and ask her to come here?
	\item Well, shall we go?
	\item Let's have a nice little stroll, shall we?
	\item What shall I do?
	\end{itemize}
}
\item modal verb \\
You use \textbf{shall} , usually with 'I' and 'we', when you are referring to something that you intend to do, or when you are referring to something that you are sure  will  happen to you in the future.
 \textit{
	\begin{itemize}
	\item We shall be landing in Paris in sixteen minutes, exactly on time.
	\item I shall sail out on the twenty-second.
	\item I shall know more next month, I hope.
	\item I shall miss him terribly.
	\end{itemize}
}
\item modal verb \\
You use \textbf{shall} with 'I' or 'we' during a speech or piece of writing to say what you are going to discuss or explain  later .
 \textit{
	\begin{itemize}
	\item In Chapter 3, I shall describe some of the documentation that I gathered.
	\item We shall refer here to three significant trends that arose in the previous decade.
	\item The building, as we shall see, is very different in its internal planning.
	\end{itemize}
}
\item modal verb \\
You use \textbf{shall} to indicate that something must happen, usually because of a rule or law . You use \textbf{shall not} to indicate that something must not happen.
 \textit{
	\begin{itemize}
	\item The president shall hold office for five years.
	\item Member states shall decide the conditions for granting access to the labour market
for the applicant.
	\item The bank shall be entitled to debit the amount of such liability and all costs incurred
in connection with it to your Account.
	\item You shall not make this speech.
	\item If you want to pry into other people's business you shall not do it here, young man.
	\end{itemize}
}
\item modal verb \\
You use \textbf{shall} , usually with 'you', when you are telling someone that they will be able to do or have something they want .
 \textit{
	\begin{itemize}
	\item Very well, if you want to go, go you shall.
	\item 'I want to hear all the gossip, all the scandal.'—'You shall, dearie, you shall!'
	\item 'What I would like, is a membership list and some information on how the Society
is run.'—'Then that is what you shall have.'
	\end{itemize}
}
\item modal verb \\
You use \textbf{shall} with verbs such as ' look  forward to' and ' hope ' to say politely that you are looking forward to something or hoping to do something.
 \textit{
	\begin{itemize}
	\item Well, we shall look forward to seeing him tomorrow.
	\item I shall hope to see you in my office, young lady, and we'll review your portfolio.
	\end{itemize}
}
\item modal verb \\
You use \textbf{shall} when you are referring to the likely  result or consequence of a particular  action or situation .
 \textit{
	\begin{itemize}
	\item When large finance companies cut down on their entertainments, we shall know that
times really are hard.
	\item This is our last chance and we shall need to take it if we are to compete and survive.
	\end{itemize}
}
\end{enumerate}

\section*{silicon}
{\large \color{blue}  }
\subsection*{Explain}
\begin{enumerate}
\item uncountable noun \\
\textbf{Silicon} is an element that is found in sand and in minerals such as quartz and granite. Silicon
is used to make parts of computers and other electronic  equipment .
 \textit{
	\begin{itemize}
	\item ...a thin layer of silicon oxide.
	\item A chip is a piece of silicon about the size of a postage stamp.
	\end{itemize}
}
\end{enumerate}

\section*{should}
{\large \color{blue}  }
\subsection*{Explain}
\begin{enumerate}
\item modal verb \\
You use \textbf{should} when you are saying what would be the right thing to do or the right state for something to be in.
 \textit{
	\begin{itemize}
	\item I should exercise more.
	\item The diet should be maintained unchanged for about a year.
	\item He's never going to be able to forget it. And I don't think he should.
	\item Sometimes I am not as brave as I should be.
	\item Should our children be taught to swim at school?
	\end{itemize}
}
\item modal verb \\
You use \textbf{should} to give someone an order to do something, or to report an official order.
 \textit{
	\begin{itemize}
	\item All visitors should register with the British Embassy.
	\item The European Commission ruled that the company should pay back tens of millions of
pounds.
	\end{itemize}
}
\item modal verb \\
If you say that something \textbf{should have}  happened , you mean that it did not happen, but that you wish it had. If you say that something \textbf{should not have} happened, you mean that it did happen, but that you wish it had not.
 \textit{
	\begin{itemize}
	\item I should have gone this morning but I was feeling a bit ill.
	\item I should have been in the shade, then I wouldn't have got burned.
	\item You should have done that yesterday you idiot!
	\item You should have written to the area manager again.
	\item I shouldn't have said what I did.
	\end{itemize}
}
\item modal verb \\
You use \textbf{should} when you are saying that something is probably the case or will probably happen in the way you are describing . If you say that something \textbf{should have} happened by a particular time, you mean that it will probably have happened by that time.
 \textit{
	\begin{itemize}
	\item You should have no problem with reading this language.
	\item The voters should by now be in no doubt what the parties stand for.
	\item The doctor said it will take six weeks and I should be fine by then.
	\item We should have finished by a quarter past two and the bus doesn't leave till half
past.
	\end{itemize}
}
\item modal verb \\
You use \textbf{should} in questions when you are asking someone for advice , permission , or information .
 \textit{
	\begin{itemize}
	\item Should I or shouldn't I go to university?
	\item What should I do?
	\item Please could you advise me what I should do?
	\item Should I go back to the motel and wait for you to phone?
	\item Should I fetch your slippers?
	\item Should we tell her about it?
	\end{itemize}
}
\item modal verb \\
You say ' \textbf{I should} ', usually with the expression 'if I were you', when you are giving someone advice by telling them what you would do if you were in their position .
 \textit{
	\begin{itemize}
	\item I should look out if I were you!
	\item James, I should refuse that consultancy with Shapiro, if I were you.
	\item I should go if I were you.
	\end{itemize}
}
\item modal verb \\
You use \textbf{should} in conditional  clauses when you are talking about things that might happen.
 \textit{
	\begin{itemize}
	\item If you should be fired, your health and pension benefits will not be automatically
cut off.
	\item Should you buy a home from them, the company promises to buy it back at the same
price after three years.
	\item Should Havelock become the first Englishman to retain his world title, he will be
the last to do so under the present system.
	\end{itemize}
}
\item modal verb \\
You use \textbf{should} in 'that' clauses after certain verbs, nouns , and adjectives when you are talking about a future  event or situation .
 \textit{
	\begin{itemize}
	\item He raised his glass and indicated that I should do the same.
	\item I insisted that we should have a look at every car.
	\item My father was very keen that I should fulfill my potential.
	\item George was sincerely anxious that his son should find happiness and security.
	\item It seems such a pity that a distinguished name should be commercialized in such a
manner.
	\item There is a wish among competitors that the test should be changed every four years.
	\end{itemize}
}
\item modal verb \\
You use \textbf{should} in expressions such as \textbf{I should think} and \textbf{I should imagine} to indicate that you think something is true but you are not sure .
 \textit{
	\begin{itemize}
	\item I should think it's going to rain soon.
	\item 'I suppose that was the right thing to do.'—'I should imagine so.'.
	\item 'Can we be talking about the same thing?'—'I should hope so.'
	\end{itemize}
}
\item modal verb \\
You use \textbf{should} in expressions such as \textbf{I should like} and \textbf{I should be happy} to show politeness when you are saying what you want to do, or when you are requesting , offering , or accepting something.
 \textit{
	\begin{itemize}
	\item I should be happy if you would bring them this evening.
	\item 'I should like to know anything you can tell me,' said Kendal.
	\item I should like a word with the carpenter.
	\item I should like to ask you to come with us for a quiet supper.
	\item That is very kind of you both. I should like to come.
	\item 'You can go and see her if you like.'—'I should be delighted to do so.'.
	\item She thought, 'I should like her for a friend.'.
	\end{itemize}
}
\item modal verb \\
You use \textbf{should} in expressions such as \textbf{You should have seen us} and \textbf{You should have heard him} to emphasize how funny , shocking , or impressive something that you experienced was.
 \textit{
	\begin{itemize}
	\item You should have heard him last night!
	\item You should have seen him when he first came out–it was so sad.
	\item He started crying and I cried too. You should have seen us.
	\item You should have seen his roses! As good a show as in the Botanic Garden.
	\item You should have seen his face when she tapped him on the shoulder. Talk about surprise!
	\end{itemize}
}
\item modal verb \\
You use \textbf{should} in question structures which begin with words like 'who' and 'what' and are followed by 'but' to emphasize how surprising or shocking a particular event was.
 \textit{
	\begin{itemize}
	\item I'm making these plans and who should I meet but this blonde guy and John.
	\end{itemize}
}
\end{enumerate}

\section*{silver}
{\large \color{blue}  silvers  }
\subsection*{Explain}
\begin{enumerate}
\item uncountable noun \\
\textbf{Silver} is a valuable pale-grey metal that is used for making jewellery and ornaments .
 \textit{
	\begin{itemize}
	\item ...a hand-crafted brooch made from silver.
	\item ...amber earrings set in silver.
	\item ...silver teaspoons.
	\end{itemize}
}
\item uncountable noun \\
\textbf{Silver} consists of coins that are made from silver or that look like silver.
 \textit{
	\begin{itemize}
	\item ...the basement where £150,000 in silver was buried.
	\end{itemize}
}
\item uncountable noun \\
You can use \textbf{silver} to refer to all the things in a house that are made of silver, especially the cutlery and dishes .
 \textit{
	\begin{itemize}
	\item He beat the rugs and polished the silver.
	\end{itemize}
}
\item colour \\
\textbf{Silver} is used to describe things that are shiny and pale  grey in colour.
 \textit{
	\begin{itemize}
	\item He had thick silver hair which needed cutting.
	\item ...a silver sports car.
	\item Using silver tape, they taped all the doors and windows shut.
	\end{itemize}
}
\item variable noun \\
A \textbf{silver} is the same as a silver medal .
 \textit{
	\begin{itemize}
	\item The British sprinter won silver in the women's 100m.
	\end{itemize}
}
\end{enumerate}

\section*{simply}
{\large \color{blue}  }
\subsection*{Explain}
\begin{enumerate}
\item adverb \\
You use \textbf{simply} to emphasize that something consists of only one thing, happens for only one reason , or is done in only one way.
 \textit{
	\begin{itemize}
	\item The table is simply a chipboard circle on a base.
	\item Most of the damage that's occurred was simply because of fallen trees.
	\item Many people switch on the television simply to stave off boredom over the holiday
weekend.
	\item A sitting room can be transformed into a guest bedroom simply by adding a sofabed.
	\end{itemize}
}
\item adverb \\
You use \textbf{simply} to emphasize what you are saying .
 \textit{
	\begin{itemize}
	\item This sort of increase simply cannot be justified.
	\item So many of these questions simply don't have answers.
	\item In a poll of those leaving the theatre and nine out of ten thought it was simply
marvellous.
	\end{itemize}
}
\end{enumerate}

\section*{ski}
{\large \color{blue}  skis  skiing  skied  }
\subsection*{Explain}
\begin{enumerate}
\item countable noun \\
\textbf{Skis} are long, flat , narrow pieces of wood, metal, or plastic that are fastened to boots so that you can move easily on snow or water.
 \textit{
	\begin{itemize}
	\item ...a pair of skis.
	\end{itemize}
}
\item verb \\
When people \textbf{ski} , they move over snow or water on skis.
 \textit{
	\begin{itemize}
	\item They surf, ski and ride.
	\item The whole party then skied off.
	\end{itemize}
}
\item adjective \\
You use \textbf{ski} to refer to things that are concerned with skiing.
 \textit{
	\begin{itemize}
	\item ...the Swiss ski resort of Klosters.
	\item ...a private ski instructor.
	\item ...artificial ski slopes.
	\end{itemize}
}
\end{enumerate}

\section*{spacious}
{\large \color{blue}  }
\subsection*{Explain}
\begin{enumerate}
\item adjective \\
A \textbf{spacious}  room or other place is large in size or area, so that you can move around freely in it.
 \textit{
	\begin{itemize}
	\item The house has a spacious kitchen and dining area.
	\end{itemize}
}
\end{enumerate}

\section*{snake}
{\large \color{blue}  snakes  snaking  snaked  }
\subsection*{Explain}
\begin{enumerate}
\item countable noun \\
A \textbf{snake} is a long, thin reptile without legs .
 \textit{
	\begin{itemize}
	\end{itemize}
}
\item verb \\
Something that \textbf{snakes} in a particular direction goes in that direction in a line with a lot of bends .
 \textit{
	\begin{itemize}
	\item The road snaked through forested mountains.
	\item The three-mile procession snaked its way through the richest streets of the capital.
	\end{itemize}
}
\end{enumerate}

\section*{snow}
{\large \color{blue}  snows  snowing  snowed  }
\subsection*{Explain}
\begin{enumerate}
\item uncountable noun \\
\textbf{Snow} consists of a lot of soft white bits of frozen water that fall from the sky in cold  weather .
 \textit{
	\begin{itemize}
	\item In Mid-Wales six inches of snow blocked roads.
	\item They tramped through the falling snow.
	\end{itemize}
}
\item plural noun \\
You can refer to a great deal of snow in an area as the \textbf{snows} .
 \textit{
	\begin{itemize}
	\item ...the first snows of winter.
	\item As the snows melt, the flood waters rise.
	\end{itemize}
}
\item verb \\
When \textbf{it snows} , snow falls from the sky.
 \textit{
	\begin{itemize}
	\item It had been snowing all night.
	\end{itemize}
}
\item verb \\
If someone \textbf{snows} you, they persuade you to do something or convince you of something by flattering or deceiving you.
 \textit{
	\begin{itemize}
	\item I'd been a fool letting him snow me with his big ideas.
	\end{itemize}
}
\end{enumerate}

\section*{until}
{\large \color{blue}  }
\subsection*{Explain}
\begin{enumerate}
\item preposition \\
If something happens  \textbf{until} a particular time, it happens during the period before that time and stops at that time.
 \textbf{Until} is also a conjunction .
 \textit{
	\begin{itemize}
	\item Until 2016, he was a high-ranking officer in the army.
	\item ...consumers who have waited until after the Christmas holiday to do that holiday
shopping.
	\item I waited until it got dark.
	\item Stir with a metal spoon until the sugar has dissolved.
	\end{itemize}
}
\item preposition \\
You use \textbf{until} with a negative to emphasize the moment in time after which the rest of your statement becomes true , or the condition which would make it true.
 \textbf{Until} is also a conjunction.
 \textit{
	\begin{itemize}
	\item The traffic laws don't take effect until the end of the year.
	\item It was not until 1911 that the first of the vitamins was identified.
	\item The E.U. will not lift its sanctions until that country makes political changes.
	\end{itemize}
}
\end{enumerate}

\section*{spoon}
{\large \color{blue}  spoons  spooning  spooned  }
\subsection*{Explain}
\begin{enumerate}
\item countable noun \\
A \textbf{spoon} is an object used for eating, stirring, and serving food. One end of it is shaped like a shallow bowl and it has a long handle.
 \textit{
	\begin{itemize}
	\item He stirred his coffee with a spoon.
	\end{itemize}
}
\item countable noun \\
You can refer to an amount of food resting on a spoon as a \textbf{spoon}  \textbf{of} food.
 \textit{
	\begin{itemize}
	\item ...tea with two spoons of sugar.
	\end{itemize}
}
\item verb \\
If you \textbf{spoon} food into something, you put it there with a spoon.
 \textit{
	\begin{itemize}
	\item He spooned instant coffee into two of the mugs.
	\item Spoon the sauce over the meat.
	\end{itemize}
}
\item  \\
 to be born with a silver spoon in your mouth \textit{
	\begin{itemize}
	\end{itemize}
}
\end{enumerate}

\section*{usually}
{\large \color{blue}  }
\subsection*{Explain}
\begin{enumerate}
\item adverb \\
If something \textbf{usually}  happens , it is the thing that most often happens in a particular situation.
 \textit{
	\begin{itemize}
	\item The best information about hotels usually comes from friends who have been there.
	\item They ate, as they usually did, in the kitchen.
	\item Usually, the work is boring.
	\item Offering only one loan, usually an installment loan, is part of the plan.
	\end{itemize}
}
\item  \\
 more than usually \textit{
	\begin{itemize}
	\end{itemize}
}
\end{enumerate}

\section*{supper}
{\large \color{blue}  suppers  }
\subsection*{Explain}
\begin{enumerate}
\item variable noun \\
Some people refer to the main meal eaten in the early part of the evening as \textbf{supper} .
 \textit{
	\begin{itemize}
	\item Some guests like to dress for supper.
	\end{itemize}
}
\item variable noun \\
\textbf{Supper} is a simple meal eaten just before you go to bed at night .
 \textit{
	\begin{itemize}
	\item She gives the children their supper, then puts them to bed.
	\end{itemize}
}
\item  \\
 sing for your supper \textit{
	\begin{itemize}
	\end{itemize}
}
\end{enumerate}

\section*{will}
{\large \color{blue}  }
\subsection*{Explain}
\begin{enumerate}
\item modal verb \\
You use \textbf{will} to indicate that you hope , think , or have evidence that something is going to happen or be the case in the future.
 \textit{
	\begin{itemize}
	\item The Prime Minister is now 64 years old and in all probability this will be the last
election that he is likely to contest.
	\item You will find a wide variety of choices available in school cafeterias.
	\item Representatives from across the horse industry will attend the meeting.
	\item 70 per cent of airports will have to be upgraded.
	\item Will you ever feel at home here?
	\item The ship will not be ready for a month.
	\end{itemize}
}
\item modal verb \\
You use \textbf{will} in order to make statements about official arrangements in the future.
 \textit{
	\begin{itemize}
	\item The show will be open to the public at 2pm; admission will be 50p.
	\item When will I be released, sir?
	\end{itemize}
}
\item modal verb \\
You use \textbf{will} in order to make promises and threats about what is going to happen or be the case in the future.
 \textit{
	\begin{itemize}
	\item I'll call you tonight.
	\item Price quotes on selected product categories will be sent on request.
	\item If she refuses to follow rules about car safety, she won't be allowed to use the
car.
	\end{itemize}
}
\item modal verb \\
You use \textbf{will} to indicate someone's intention to do something.
 \textit{
	\begin{itemize}
	\item I will say no more on these matters, important though they are.
	\item We will describe these techniques in Chapters 20 and 21.
	\item 'Dinner's ready.'—'Thanks, Carrie, but we'll have a drink first.'
	\item He will be devoting more time to writing, broadcasting and lecturing.
	\item What will you do next?
	\item Where will you stay when you get to San Francisco?
	\item Will you be remaining in the city?
	\end{itemize}
}
\item modal verb \\
You use \textbf{will} in questions in order to make polite invitations or offers .
 \textit{
	\begin{itemize}
	\item Will you stay for supper?
	\item Will you join me for a drink?
	\item Won't you sit down?
	\end{itemize}
}
\item modal verb \\
You use \textbf{will} in questions in order to ask or tell someone to do something.
 \textit{
	\begin{itemize}
	\item Will you drive me home?
	\item Will you listen again, Andrew?
	\item Wipe the jam off my mouth, will you?
	\end{itemize}
}
\item modal verb \\
You can use \textbf{will} in statements to give an order to someone.
 \textit{
	\begin{itemize}
	\item You will do as I request, if you please.
	\item You will now maintain radio silence.
	\item You will not make jokes about him. He has been very good to me.
	\item You will not discuss this matter with anyone.
	\end{itemize}
}
\item modal verb \\
You use \textbf{will} to say that someone is willing to do something. You use \textbf{will not} or \textbf{won't} to indicate that someone refuses to do something.
 \textit{
	\begin{itemize}
	\item All right, I'll forgive you.
	\item I'll answer the phone.
	\item If you won't let me pay for a taxi, then at least allow me to lend you something.
	\item He has insisted that his organisation will not negotiate with the government.
	\end{itemize}
}
\item modal verb \\
You use \textbf{will} to say that a person or thing is able to do something in the future.
 \textit{
	\begin{itemize}
	\item How the country will defend itself in the future has become increasingly important.
	\item How will I recognize you?
	\end{itemize}
}
\item modal verb \\
You use \textbf{will} to indicate that an action usually happens in the particular way mentioned .
 \textit{
	\begin{itemize}
	\item The thicker the material, the less susceptible the garment will be to wet conditions.
	\item There's no snake that will habitually attack human beings unless threatened.
	\item Art thieves will often hide an important work for years after it has been stolen.
	\end{itemize}
}
\item modal verb \\
You use \textbf{will} in the main clause of some 'if' and 'unless' sentences to indicate something that you consider to be fairly  likely to happen.
 \textit{
	\begin{itemize}
	\item If you overcook the pancakes they will be difficult to roll.
	\item If a nuclear war breaks out, every living thing will be wiped off the face of the
Earth.
	\item He won't stop drinking unless he's told by a doctor that it's killing him.
	\end{itemize}
}
\item modal verb \\
You use \textbf{will} to say that someone insists on behaving or doing something in a particular way and you cannot change them. You emphasize  \textbf{will} when you use it in this way.
 \textit{
	\begin{itemize}
	\item He will leave his socks lying all over the place and it drives me mad.
	\end{itemize}
}
\item modal verb \\
You use \textbf{will have} with a past  participle when you are saying that you are fairly certain that something will be true by a particular time in the future.
 \textit{
	\begin{itemize}
	\item As many as ten million children will have been infected by the end of the decade.
	\item He will have left by January the fifteenth.
	\end{itemize}
}
\item modal verb \\
You use \textbf{will have} with a past participle to indicate that you are fairly sure that something is the case.
 \textit{
	\begin{itemize}
	\item If someone has been in captivity, he will have changed as a result of his experience.
	\item The holiday will have done him the world of good.
	\end{itemize}
}
\end{enumerate}

\section*{ticket}
{\large \color{blue}  tickets  }
\subsection*{Explain}
\begin{enumerate}
\item countable noun \\
A \textbf{ticket} is a small, official piece of paper or card which shows that you have paid to enter a place such as a theatre or a sports ground, or shows that you have paid for a journey .
 \textit{
	\begin{itemize}
	\item I queued for two hours to get a ticket to see the football game.
	\item I love opera and last year I got tickets for Covent Garden.
	\item Entrance is free, but by ticket only.
	\item He became a ticket collector at Waterloo Station.
	\end{itemize}
}
\item countable noun \\
A \textbf{ticket} is an official piece of paper which orders you to pay a fine or to appear in court because you have committed a driving or parking offence.
 \textit{
	\begin{itemize}
	\item I want to know at what point I break the speed limit and get a ticket.
	\end{itemize}
}
\item countable noun \\
A \textbf{ticket} for a game of chance such as a raffle or a lottery is a piece of paper with a number on it. If the number on your ticket matches the number chosen , you win a prize .
 \textit{
	\begin{itemize}
	\item She bought a lottery ticket and won more than $33 million.
	\end{itemize}
}
\item singular noun \\
The particular \textbf{ticket} on which a person fights an election is the party they represent or the policies they support.
 \textit{
	\begin{itemize}
	\item He first ran for president on a far-left ticket.
	\item She would want to fight the election on a ticket of parliamentary democracy.
	\item It's a ticket that was designed to appeal to suburban and small town voters.
	\end{itemize}
}
\item countable noun \\
A \textbf{ticket} is the list of candidates who are representing a particular political party or group in an election.
 \textit{
	\begin{itemize}
	\item He plans to remain on the Republican ticket for the November election.
	\end{itemize}
}
\item  \\
 just the ticket \textit{
	\begin{itemize}
	\end{itemize}
}
\end{enumerate}

\section*{would}
{\large \color{blue}  }
\subsection*{Explain}
\begin{enumerate}
\item modal verb \\
You use \textbf{would} when you are saying what someone believed , hoped , or expected to happen or be the case .
 \textit{
	\begin{itemize}
	\item No one believed he would actually kill himself.
	\item Would he always be like this?
	\item Once inside, I found that the flat would be perfect for my life in Paris.
	\item He expressed the hope that on Monday elementary schools would be reopened.
	\item A report yesterday that said unemployment would continue to rise.
	\item I don't think that he would take such a decision.
	\end{itemize}
}
\item modal verb \\
You use \textbf{would} when saying what someone intended to do.
 \textit{
	\begin{itemize}
	\item The statement added that these views would be discussed by both sides.
	\item George decided it was such a rare car that he would only use it for a few shows.
	\item He did not think he would marry Beth.
	\end{itemize}
}
\item modal verb \\
You use \textbf{would} when you are referring to the result or effect of a possible situation.
 \textit{
	\begin{itemize}
	\item Ordinarily it would be fun to be taken to fabulous restaurants.
	\item It would be wrong to suggest that police officers were not annoyed by acts of indecency.
	\item It would cost very much more for the four of us to go from Italy.
	\item ...identity cards without which fans would not be able to get into stadiums.
	\end{itemize}
}
\item modal verb \\
You use \textbf{would} , or \textbf{would have} with a past participle , to indicate that you are assuming or guessing that something is true , because you have good reasons for thinking it.
 \textit{
	\begin{itemize}
	\item You wouldn't know him.
	\item His fans would already be familiar with Caroline.
	\item That would have been Della's car.
	\item He made a promise to his great-grandfather? That would have been a long time ago.
	\item It was half seven; her mother would be annoyed because he was so late.
	\end{itemize}
}
\item modal verb \\
You use \textbf{would} in the main clause of some 'if' and 'unless' sentences to indicate something you consider to be fairly  unlikely to happen.
 \textit{
	\begin{itemize}
	\item If only I could get some sleep, I would be able to cope.
	\item I think if I went to look at more gardens, I would be better on planning and designing
them.
	\item A policeman would not live one year if he obeyed these regulations.
	\item the targets would not be achieved unless other departments showed equal commitment.
	\end{itemize}
}
\item modal verb \\
You use \textbf{would} to say that someone was willing to do something. You use \textbf{would not} to indicate that they refused to do something.
 \textit{
	\begin{itemize}
	\item They said they would give the police their full cooperation.
	\item She indicated that she would help her boss.
	\item David would not accept this.
	\item He wouldn't say where he had picked up the information.
	\end{itemize}
}
\item modal verb \\
You use \textbf{would not} to indicate that something did not happen, often in spite of a lot of effort .
 \textit{
	\begin{itemize}
	\item He kicked, pushed, and hurled his shoulder at the door. It wouldn't open.
	\item The battery got flatter and flatter, until it wouldn't turn the engine at all.
	\item The paint wouldn't stick to the wallpaper.
	\end{itemize}
}
\item modal verb \\
You use \textbf{would} , especially with 'like', ' love ', and 'wish', when saying that someone wants to do or have a particular thing or wants a particular thing to happen.
 \textit{
	\begin{itemize}
	\item She asked me what I would like to do and mentioned a particular job.
	\item Right now, your mom would like a cup of coffee.
	\item Ideally, she would love to become pregnant again.
	\item He wished it would end.
	\item Anne wouldn't mind going to Italy or France to live.
	\end{itemize}
}
\item modal verb \\
You use \textbf{would} with 'if' clauses in questions when you are asking for permission to do something.
 \textit{
	\begin{itemize}
	\item Do you think it would be all right if I opened a window?
	\item Mr. Cutler, would you mind if I asked a question?
	\end{itemize}
}
\item modal verb \\
You use \textbf{would} , usually in questions with 'like', when you are making a polite offer or invitation .
 \textit{
	\begin{itemize}
	\item Would you like a drink?
	\item Would you like to stay?
	\item Perhaps you would like to pay a visit to London.
	\end{itemize}
}
\item modal verb \\
You use \textbf{would} , usually in questions, when you are politely asking someone to do something.
 \textit{
	\begin{itemize}
	\item Would you do me a favour and get rid of this letter I've just received?
	\item Would you come in here a moment, please?
	\item Would you excuse us for a minute, Cassandra?
	\item Oh dear, there's the doorbell. See who it is, would you, darling.
	\end{itemize}
}
\item modal verb \\
You say that someone \textbf{would} do something when it is typical of them and you are critical of it. You emphasize the word \textbf{would} when you use it in this way.
 \textit{
	\begin{itemize}
	\item Well, you would say that: you're a man.
	\item 'Well, then Francesca turned round and said, "That's a stupid question."'—'She would,
wouldn't she.'
	\end{itemize}
}
\item modal verb \\
You use \textbf{would} , or sometimes \textbf{would have} with a past participle, when you are expressing your opinion about something or seeing if people agree with you, especially when you are uncertain about what you are saying.
 \textit{
	\begin{itemize}
	\item I think you'd agree he's a very respected columnist.
	\item I would have thought it a proper job for the Army to fight rebellion.
	\item 'Was it much different for you when you started at the Foreign Office?'—'Worse, I'd
expect.'.
	\item I would imagine she's quite lonely living on her own.
	\end{itemize}
}
\item modal verb \\
You use \textbf{I would} when you are giving someone advice in an informal way.
 \textit{
	\begin{itemize}
	\item If I were you I would simply ring your friend's bell and ask for your bike back.
	\item I would not, if I were you, be inclined to discuss private business with the landlady.
	\item There could be more unrest, but I wouldn't exaggerate the problems.
	\end{itemize}
}
\item modal verb \\
You use \textbf{you would} in negative sentences with verbs such as 'guess' and ' know ' when you want to say that something is not obvious , especially something surprising .
 \textit{
	\begin{itemize}
	\item Chris is so full of artistic temperament you'd never think she was the daughter of
a banker.
	\item Inside, he admits, his emotions may be churning, but you would never guess it.
	\end{itemize}
}
\item modal verb \\
You use \textbf{would} to talk about something which happened regularly in the past but which no longer happens.
 \textit{
	\begin{itemize}
	\item Sunday mornings my mother would bake. I'd stand by the fridge and help.
	\item 'Beauty is only skin deep,' my mother would say.
	\end{itemize}
}
\item modal verb \\
You use \textbf{would have} with a past participle when you are saying what was likely to have happened by a particular time.
 \textit{
	\begin{itemize}
	\item Within ten weeks, 34 million people would have been reached by our commercials.
	\end{itemize}
}
\item modal verb \\
You use \textbf{would have} with a past participle when you are referring to the result or effect of a possible
event in the past.
 \textit{
	\begin{itemize}
	\item My daughter would have been 17 this week if she had lived.
	\item If I had known how he felt, I would never have let him adopt those children.
	\item If I had not been enjoying the work, I would not have done so much of it.
	\end{itemize}
}
\item modal verb \\
If you say that someone \textbf{would have} liked or preferred something, you mean that they wanted to do it or have it but were unable to.
 \textit{
	\begin{itemize}
	\item I would have liked a life in politics.
	\item She would have liked to ask questions, but he had moved on to another topic.
	\item He dined there regularly, though he would have preferred being at home.
	\end{itemize}
}
\item modal verb \\
You use \textbf{would} , usually in negative sentences, to criticize something that someone has done and to express your disapproval of it.
 \textit{
	\begin{itemize}
	\item I would never have done what they did.
	\end{itemize}
}
\item  \\
 would that \textit{
	\begin{itemize}
	\end{itemize}
}
\end{enumerate}

\section*{vanity}
{\large \color{blue}  vanities  }
\subsection*{Explain}
\begin{enumerate}
\item uncountable noun \\
If you refer to someone's \textbf{vanity} , you are critical of them because they take great pride in their appearance or abilities .
 \textit{
	\begin{itemize}
	\item Men who use steroids are motivated by sheer vanity.
	\item With my usual vanity, I thought he might be falling in love with me.
	\end{itemize}
}
\end{enumerate}

\section*{although}
{\large \color{blue}  }
\subsection*{Explain}
\begin{enumerate}
\item conjunction \\
You use \textbf{although} to introduce a subordinate  clause which contains a statement which contrasts with the statement in the main clause.
 \textit{
	\begin{itemize}
	\item Although he is known to only a few, his reputation among them is very great.
	\item Although the shooting has stopped for now, the destruction left behind is enormous.
	\end{itemize}
}
\item conjunction \\
You use \textbf{although} to introduce a subordinate clause which contains a statement which makes the main
clause of the sentence  seem  surprising or unexpected .
 \textit{
	\begin{itemize}
	\item Although I was only six, I can remember seeing it on TV.
	\item Although he was twice as old as us, he became the life and soul of the company.
	\end{itemize}
}
\item conjunction \\
You use \textbf{although} to introduce a subordinate clause which gives some information that is relevant to the main clause but modifies the strength of that statement.
 \textit{
	\begin{itemize}
	\item He was in love with her, although he did not put that name to it.
	\end{itemize}
}
\item conjunction \\
You use \textbf{although} when admitting a fact about something which you regard as less important than a contrasting fact.
 \textit{
	\begin{itemize}
	\item Although they're expensive, they last forever and never go out of style.
	\item Although not ideal, this attitude is not entirely destructive.
	\end{itemize}
}
\end{enumerate}

\section*{box}
{\large \color{blue}  boxes  boxing  boxed  }
\subsection*{Explain}
\begin{enumerate}
\item countable noun \\
A \textbf{box} is a square or rectangular container with hard or stiff sides. Boxes often have lids.
 A \textbf{box}  \textbf{of} something is an amount of it contained in a box.
 \textit{
	\begin{itemize}
	\item He reached into the cardboard box beside him.
	\item They sat on wooden boxes.
	\item ...the box of tissues on her desk.
	\item She ate two boxes of liqueurs.
	\end{itemize}
}
\item countable noun \\
A \textbf{box} is a square or rectangle that is printed or drawn on a piece of paper, a road, or on some other surface.
 \textit{
	\begin{itemize}
	\end{itemize}
}
\item singular noun \\
In football , \textbf{the box} is the penalty area of the field.
 \textit{
	\begin{itemize}
	\item He scored from the penalty spot after being brought down in the box.
	\end{itemize}
}
\item countable noun \\
A \textbf{box} is a small separate area in a theatre or at a sports ground or stadium , where a small number of people can sit to watch the performance or game.
 \textit{
	\begin{itemize}
	\end{itemize}
}
\item singular noun \\
Television is sometimes referred to as \textbf{the box} .
 \textit{
	\begin{itemize}
	\item Do you watch it live at all or do you watch it on the box?
	\end{itemize}
}
\item countable noun \\
\textbf{Box} is used before a number as a postal address by organizations that receive a lot of mail.
 \textit{
	\begin{itemize}
	\item ...Country Crafts, Box 111, Landisville.
	\end{itemize}
}
\item uncountable noun \\
\textbf{Box} is a small evergreen tree with dark leaves which is often used to form hedges.
 \textit{
	\begin{itemize}
	\item ...box hedges.
	\end{itemize}
}
\item verb \\
To \textbf{box} means to fight someone according to the rules of boxing.
 \textit{
	\begin{itemize}
	\item At school I boxed and played rugby.
	\item The two fighters had previously boxed a 12-round match.
	\end{itemize}
}
\end{enumerate}

\section*{aluminum}
{\large \color{blue}  }
\subsection*{Explain}
\begin{enumerate}
\end{enumerate}

\section*{cargo}
{\large \color{blue}  cargoes  }
\subsection*{Explain}
\begin{enumerate}
\item variable noun \\
The \textbf{cargo} of a ship or plane is the goods that it is carrying.
 \textit{
	\begin{itemize}
	\item The boat calls at the main port to load its regular cargo of bananas.
	\item ...cargo planes.
	\end{itemize}
}
\end{enumerate}

\section*{cassette}
{\large \color{blue}  cassettes  }
\subsection*{Explain}
\begin{enumerate}
\item countable noun \\
A \textbf{cassette} is a small, flat , rectangular plastic case containing magnetic tape which was used in the past for recording and playing back sound or film.
 \textit{
	\begin{itemize}
	\item I started very early, writing my first tune at three. I still have it on cassette.
	\end{itemize}
}
\end{enumerate}

\section*{cat}
{\large \color{blue}  cats  }
\subsection*{Explain}
\begin{enumerate}
\item countable noun \\
A \textbf{cat} is a furry animal that has a long tail and sharp  claws . Cats are often kept as pets.
 \textit{
	\begin{itemize}
	\end{itemize}
}
\item countable noun \\
\textbf{Cats} are lions , tigers, and other wild animals in the same family.
 \textit{
	\begin{itemize}
	\end{itemize}
}
\item  \\
 to let the cat out of the bag \textit{
	\begin{itemize}
	\end{itemize}
}
\item  \\
 curiosity killed the cat \textit{
	\begin{itemize}
	\end{itemize}
}
\item  \\
 look like something the cat dragged in \textit{
	\begin{itemize}
	\end{itemize}
}
\item  \\
 game of cat and mouse \textit{
	\begin{itemize}
	\end{itemize}
}
\item  \\
 to put the cat among the pigeons \textit{
	\begin{itemize}
	\end{itemize}
}
\item  \\
 no room to swing a cat \textit{
	\begin{itemize}
	\end{itemize}
}
\end{enumerate}

\section*{axe}
{\large \color{blue}  axes  axing  axed  }
\subsection*{Explain}
\begin{enumerate}
\item countable noun \\
An \textbf{axe} is a tool used for cutting wood. It consists of a heavy metal blade which is sharp at one edge and attached by its other edge to the end of a long handle .
 \textit{
	\begin{itemize}
	\end{itemize}
}
\item verb \\
If someone's job or something such as a public service or a television programme  \textbf{is axed} , it is ended suddenly and without discussion .
 \textit{
	\begin{itemize}
	\item Community projects are being axed by hard-pressed social services departments.
	\end{itemize}
}
\item singular noun \\
If a person or institution is facing  \textbf{the axe} , that person is likely to lose their job or that institution is likely to be closed, usually in order to save money.
 \textit{
	\begin{itemize}
	\item Hundreds more staff face the axe as managers prepare to present a five-year plan
to investors.
	\end{itemize}
}
\item  \\
 to have an axe to grind \textit{
	\begin{itemize}
	\end{itemize}
}
\end{enumerate}

\section*{decrease}
{\large \color{blue}  decreases  decreasing  decreased  }
\subsection*{Explain}
\begin{enumerate}
\item verb \\
When something \textbf{decreases} or when you \textbf{decrease} it, it becomes less in quantity , size, or intensity .
 \textit{
	\begin{itemize}
	\item Population growth is decreasing by 1.4% each year.
	\item The number of independent firms decreased from 198 to 96.
	\item Raw-steel production by the nation's mills decreased 2.1% last week.
	\item Since 1945 air forces have decreased in size.
	\item Gradually decrease the amount of vitamin C you are taking.
	\item We've got stable labor, decreasing interest rates, low oil prices.
	\end{itemize}
}
\item countable noun \\
A \textbf{decrease}  \textbf{in} the quantity, size, or intensity of something is a reduction in it.
 \textit{
	\begin{itemize}
	\item ...a decrease in the number of young people out of work.
	\item Bank base rates have fallen from 10 per cent to 6 per cent–a decrease of 40 per cent.
	\end{itemize}
}
\end{enumerate}

\section*{because}
{\large \color{blue}  }
\subsection*{Explain}
\begin{enumerate}
\item conjunction \\
You use \textbf{because} when stating the reason for something.
 \textit{
	\begin{itemize}
	\item He is called Mitch, because his name is Mitchell.
	\item Because it is an area of outstanding natural beauty, you can't build on it.
	\item Temple could make nothing of it, partly because he did not know German well enough.

	\item 'Why didn't you tell me, Archie?'—'Because you might have casually mentioned it to
somebody else.'
	\end{itemize}
}
\item conjunction \\
You use \textbf{because} when stating the explanation for a statement you have just made.
 \textit{
	\begin{itemize}
	\item Maybe they didn't want to ask questions, because they rented us a room without even
asking to see our papers.
	\item The President has played a shrewd diplomatic game because from the outset he called
for direct talks with the United States.
	\item I had a sense of déjà vu because I could recognise everything in London.
	\end{itemize}
}
\item  \\
 because of \textit{
	\begin{itemize}
	\end{itemize}
}
\item  \\
 just because \textit{
	\begin{itemize}
	\end{itemize}
}
\end{enumerate}

\section*{dock}
{\large \color{blue}  docks  docking  docked  }
\subsection*{Explain}
\begin{enumerate}
\item countable noun \\
A \textbf{dock} is an enclosed area in a harbour where ships go to be loaded, unloaded, and repaired.
 \textit{
	\begin{itemize}
	\item ...the loading dock.
	\item She headed for the docks, thinking that Ricardo might be hiding in one of the boats.
	\item What other ships are in dock here?
	\end{itemize}
}
\item verb \\
When a ship \textbf{docks} or \textbf{is docked} , it is brought into a dock.
 \textit{
	\begin{itemize}
	\item The vessel docked at Liverpool in April 1811.
	\item Russian commanders docked a huge aircraft carrier in a Russian port.
	\item The aircraft carrier has been docked there since last month.
	\end{itemize}
}
\item verb \\
When one spacecraft \textbf{docks} or \textbf{is docked}  \textbf{with} another, the two crafts join together in space.
 \textit{
	\begin{itemize}
	\item The shuttle should be capable of docking with other spacecraft in orbit.
	\item They have docked a robot module alongside the orbiting space station.
	\item The shuttle was docked at the International Space Station 220 miles above Earth.
	\end{itemize}
}
\item countable noun \\
A \textbf{dock} is a platform for loading vehicles or trains.
 \textit{
	\begin{itemize}
	\item The truck left the loading dock with hoses still attached.
	\end{itemize}
}
\item countable noun \\
A \textbf{dock} is a small structure at the edge of water where boats can tie up, especially one that is privately owned.
 \textit{
	\begin{itemize}
	\item He had a house there and a dock and a little aluminum boat.
	\end{itemize}
}
\item singular noun \\
In a law court, \textbf{the}  \textbf{dock} is where the person accused of a crime stands or sits.
 \textit{
	\begin{itemize}
	\item What about the odd chance that you do put an innocent man in the dock?
	\end{itemize}
}
\item verb \\
If you \textbf{dock} someone's wages or money, you take some of the money away. If you \textbf{dock} someone points in a contest , you take away some of the points that they have.
 \textit{
	\begin{itemize}
	\item He threatens to dock her fee.
	\item To dock points would be wrong.
	\end{itemize}
}
\item variable noun \\
A \textbf{dock} is a plant with large leaves which grows wild in Britain, the United States , and some other northern countries. Dock leaves are supposed to heal  nettle  stings .
 \textit{
	\begin{itemize}
	\end{itemize}
}
\end{enumerate}

\section*{but}
{\large \color{blue}  buts  }
\subsection*{Explain}
\begin{enumerate}
\item conjunction \\
You use \textbf{but} to introduce something which contrasts with what you have just said , or to introduce something which adds to what you have just said.
 \textit{
	\begin{itemize}
	\item 'You said you'd stay till tomorrow.'—'I know, Bel, but I think I would rather go
back.'
	\item Place the saucepan over moderate heat until the cider is very hot but not boiling.
	\item He not only wants to be taken seriously as a musician, but as a poet too.
	\end{itemize}
}
\item conjunction \\
You use \textbf{but} when you are about to add something further in a discussion or to change the subject.
 \textit{
	\begin{itemize}
	\item They need to recruit more people into the prison service. But another point I'd like
to make is that many prisons were built in the nineteenth century.
	\end{itemize}
}
\item conjunction \\
You use \textbf{but} after you have made an excuse or apologized for what you are just about to say .
 \textit{
	\begin{itemize}
	\item Please excuse me, but there is something I must say.
	\item I'm sorry, but it's nothing to do with you.
	\item Forgive my asking, but you're not very happy, are you?
	\end{itemize}
}
\item conjunction \\
You use \textbf{but} to introduce a reply to someone when you want to indicate surprise , disbelief , refusal , or protest .
 \textit{
	\begin{itemize}
	\item 'I don't think I should stay in this house.'—'But why?'
	\item 'Somebody wants you on the telephone'—'But no one knows I'm here!'
	\end{itemize}
}
\item preposition \\
\textbf{But} is used to mean 'except'.
 \textit{
	\begin{itemize}
	\item Europe will be represented in all but two of the seven races.
	\item He didn't speak anything but Greek.
	\item The crew of the ship gave them nothing but bread to eat.
	\end{itemize}
}
\item adverb \\
\textbf{But} is used to mean 'only'.
 \textit{
	\begin{itemize}
	\item Orbit is but one of the sculptor's striking creations.
	\item Lots of interesting different flavours combine - mixed spice and wild berries to
name but two.
	\end{itemize}
}
\item plural noun \\
You use \textbf{buts} in expressions like ' \textbf{no buts} ' and ' \textbf{ifs and buts} ' to refer to reasons someone gives for not doing something, especially when you do not think that they are good reasons.
 \textit{
	\begin{itemize}
	\item 'B-b-b-b-but' I stuttered.—'Never mind the buts,' she ranted.
	\item He committed a crime, no ifs or buts about it.
	\end{itemize}
}
\item  \\
 cannot but \textit{
	\begin{itemize}
	\end{itemize}
}
\item  \\
 but for \textit{
	\begin{itemize}
	\end{itemize}
}
\item  \\
 but then/but then again \textit{
	\begin{itemize}
	\end{itemize}
}
\item  \\
 but then \textit{
	\begin{itemize}
	\end{itemize}
}
\end{enumerate}

\section*{dove}
{\large \color{blue}  doves  }
\subsection*{Explain}
\begin{enumerate}
\item countable noun \\
A \textbf{dove} is a bird that looks  like a pigeon but is smaller and lighter in colour. Doves are often used as a symbol of peace .
 \textit{
	\begin{itemize}
	\end{itemize}
}
\item countable noun \\
In politics , you can refer to people who support the use of peaceful  methods to solve  difficult  situations as \textbf{doves} . Compare  hawk .
 \textit{
	\begin{itemize}
	\item A clear split over tactics appears to be emerging between doves and hawks in the
party.
	\end{itemize}
}
\item  \\
In American English, \textbf{dove} is sometimes used as the past  tense of dive .
 \textit{
	\begin{itemize}
	\end{itemize}
}
\end{enumerate}

\section*{chapter}
{\large \color{blue}  chapters  }
\subsection*{Explain}
\begin{enumerate}
\item countable noun \\
A \textbf{chapter} is one of the parts that a book is divided into. Each chapter has a number, and sometimes a title.
 \textit{
	\begin{itemize}
	\item As we shall see in Chapter 9, there is a totally different explanation.
	\item I took the title of this chapter from one of my favorite books.
	\end{itemize}
}
\item countable noun \\
A \textbf{chapter}  \textbf{in} someone's life or \textbf{in} history is a period of time during which a major event or series of related events takes place.
 \textit{
	\begin{itemize}
	\item This had been a particularly difficult chapter in the country's recent history.
	\item ...one of the most dramatic chapters of recent British politics.
	\end{itemize}
}
\item collective countable noun \\
A \textbf{chapter} is a group of Christian  clergy who work in or who are connected with a cathedral.
 \textit{
	\begin{itemize}
	\item The Archbishop began his address, thanking the Dean and Chapter of Westminster for
inviting him to the Abbey.
	\end{itemize}
}
\item countable noun \\
A \textbf{chapter} is a branch of a society or club.
 \textit{
	\begin{itemize}
	\item To find out more about cancer warning signs call your local chapter of the American
Cancer Society.
	\end{itemize}
}
\item  \\
 chapter and verse \textit{
	\begin{itemize}
	\end{itemize}
}
\end{enumerate}

\section*{face}
{\large \color{blue}  faces  }
\subsection*{Explain}
\begin{enumerate}
\item countable noun \\
Your \textbf{face} is the front part of your head from your chin to the top of your forehead, where your mouth, eyes, nose , and other features are.
 \textit{
	\begin{itemize}
	\item He rolled down his window and stuck his face out.
	\item A strong wind was blowing right in my face.
	\item He was going red in the face and breathing with difficulty.
	\item She had a beautiful face.
	\end{itemize}
}
\item countable noun \\
If your \textbf{face} is happy , sad , or serious , for example, the expression on your face shows that you are happy, sad, or serious.
 \textit{
	\begin{itemize}
	\item He was walking around with a sad face.
	\item The priest frowned into the light, his face puzzled.
	\end{itemize}
}
\item countable noun \\
The \textbf{face} of a cliff , mountain, or building is a vertical surface or side of it.
 \textit{
	\begin{itemize}
	\item ...the north face of the Eiger.
	\item He scrambled 200 feet up the cliff face.
	\end{itemize}
}
\item countable noun \\
The \textbf{face} of a clock or watch is the surface with the numbers or hands on it, which shows the time.
 \textit{
	\begin{itemize}
	\end{itemize}
}
\item singular noun \\
If you say that \textbf{the face of} an area, institution, or field of activity is changing, you mean its appearance or
nature is changing.
 \textit{
	\begin{itemize}
	\item ...the changing face of the British countryside.
	\item This would change the face of Malaysian politics.
	\end{itemize}
}
\item singular noun \\
If you refer to something as \textbf{the} particular \textbf{face of} an activity, belief, or system, you mean that it is one particular aspect of it, in contrast to other aspects.
 \textit{
	\begin{itemize}
	\item We aim to expose the ugly face of Western authoritarianism in the third world.
	\end{itemize}
}
\item  \\
 to lose face \textit{
	\begin{itemize}
	\end{itemize}
}
\item  \\
 blow up in sb's face \textit{
	\begin{itemize}
	\end{itemize}
}
\item  \\
 until sb is blue in the face \textit{
	\begin{itemize}
	\end{itemize}
}
\item  \\
 face down/up \textit{
	\begin{itemize}
	\end{itemize}
}
\item  \\
 the face of the earth \textit{
	\begin{itemize}
	\end{itemize}
}
\item  \\
 off/from the face of the earth \textit{
	\begin{itemize}
	\end{itemize}
}
\item  \\
 face to face \textit{
	\begin{itemize}
	\end{itemize}
}
\item  \\
 face to face \textit{
	\begin{itemize}
	\end{itemize}
}
\item  \\
 to fly in the face of \textit{
	\begin{itemize}
	\end{itemize}
}
\item  \\
 in the face of sth \textit{
	\begin{itemize}
	\end{itemize}
}
\item  \\
 to laugh in someone's face \textit{
	\begin{itemize}
	\end{itemize}
}
\item  \\
 a long face \textit{
	\begin{itemize}
	\end{itemize}
}
\item  \\
 to make a face \textit{
	\begin{itemize}
	\end{itemize}
}
\item  \\
 on the face of it \textit{
	\begin{itemize}
	\end{itemize}
}
\item  \\
 put a brave face on sth/put on a brave face \textit{
	\begin{itemize}
	\end{itemize}
}
\item  \\
 to set your face against sth \textit{
	\begin{itemize}
	\end{itemize}
}
\item  \\
 to show your face \textit{
	\begin{itemize}
	\end{itemize}
}
\item  \\
 a straight face \textit{
	\begin{itemize}
	\end{itemize}
}
\item  \\
 to sb's face \textit{
	\begin{itemize}
	\end{itemize}
}
\item  \\
 to be written all over someone's face \textit{
	\begin{itemize}
	\end{itemize}
}
\end{enumerate}

\section*{child}
{\large \color{blue}  children  }
\subsection*{Explain}
\begin{enumerate}
\item countable noun \\
A \textbf{child} is a human being who is not yet an adult .
 \textit{
	\begin{itemize}
	\item When I was a child I lived in a country village.
	\item He's just a child.
	\item ...a child of six.
	\item It was only suitable for children.
	\end{itemize}
}
\item countable noun \\
Someone's \textbf{children} are their sons and daughters of any age .
 \textit{
	\begin{itemize}
	\item How are the children?
	\item His children have left home.
	\item The young couple decided to have a child.
	\end{itemize}
}
\end{enumerate}

\section*{fake}
{\large \color{blue}  fakes  faking  faked  }
\subsection*{Explain}
\begin{enumerate}
\item adjective \\
A \textbf{fake}  fur or a \textbf{fake}  painting , for example , is a fur or painting that has been made to look valuable or genuine, often in order to deceive people.
 A \textbf{fake} is something that is fake.
 \textit{
	\begin{itemize}
	\item The bank manager is said to have issued fake certificates.
	\item It is filled with famous works of art, and every one of them is a fake.
	\end{itemize}
}
\item verb \\
If someone \textbf{fakes} something, they try to make it look valuable or genuine, although in fact it is not.
 \textit{
	\begin{itemize}
	\item It's safer to fake a tan with make-up than spend a lot of time in the sun.
	\item He faked his own death last year to collect on a $1 million insurance policy.
	\item ...faked evidence.
	\end{itemize}
}
\item countable noun \\
Someone who is a \textbf{fake} is not what they claim to be, for example because they do not have the qualifications that they claim to have.
 \textit{
	\begin{itemize}
	\end{itemize}
}
\item verb \\
If you \textbf{fake} a feeling , emotion, or reaction , you pretend that you are experiencing it when you are not.
 \textit{
	\begin{itemize}
	\item Jon faked nonchalance.
	\item Maturity and emotional sophistication can't be faked.
	\end{itemize}
}
\end{enumerate}

\section*{civilization}
{\large \color{blue}  civilizations  }
\subsection*{Explain}
\begin{enumerate}
\item variable noun \\
A \textbf{civilization} is a human society with its own social organization and culture.
 \textit{
	\begin{itemize}
	\item The ancient civilizations of Central and Latin America were founded upon corn.
	\item It seemed to him that western civilization was in grave economic and cultural danger.
	\end{itemize}
}
\item uncountable noun \\
\textbf{Civilization} is the state of having an advanced level of social organization and a comfortable way of life.
 \textit{
	\begin{itemize}
	\item ...our advanced state of civilisation.
	\end{itemize}
}
\item uncountable noun \\
You can refer to a place where you can enjoy the comforts that you consider to be necessary as \textbf{civilization} .
 \textit{
	\begin{itemize}
	\item ...when I returned to civilization.
	\end{itemize}
}
\end{enumerate}

\section*{flush}
{\large \color{blue}  flushes  flushing  flushed  }
\subsection*{Explain}
\begin{enumerate}
\item verb \\
If you \textbf{flush} , your face goes red because you are hot or ill , or because you are feeling a strong emotion such as embarrassment or anger .
 \textbf{Flush} is also a noun .
 \textit{
	\begin{itemize}
	\item Do you sweat a lot or flush a lot?
	\item He turned away embarrassed, his face flushing red.
	\item There was a slight flush on his cheeks.
	\end{itemize}
}
\item verb \\
When someone \textbf{flushes} a toilet after using it, they fill the toilet bowl with water in order to clean it, usually by pressing a handle or pulling a chain . You can also say that a toilet \textbf{flushes} .
 \textbf{Flush} is also a noun.
 \textit{
	\begin{itemize}
	\item She flushed the toilet and went back in the bedroom.
	\item ...the sound of the toilet flushing.
	\item He heard the flush of a toilet.
	\end{itemize}
}
\item verb \\
If you \textbf{flush} something \textbf{down} the toilet, you get  rid of it by putting it into the toilet bowl and flushing the toilet.
 \textit{
	\begin{itemize}
	\item He was found trying to flush banknotes down the toilet.
	\end{itemize}
}
\item verb \\
If you \textbf{flush} a part of your body, you clean it or make it healthier by using a large amount of liquid to get rid of dirt or harmful substances.
 \textbf{Flush out} means the same as flush .
 \textit{
	\begin{itemize}
	\item Flush the eye with clean cold water for at least 15 minutes.
	\item Water is ideal to flush the kidneys and the urinary tract.
	\item ...an 'alternative' therapy that gently flushes out the colon to remove toxins.
	\end{itemize}
}
\item verb \\
If you \textbf{flush} dirt or a harmful substance \textbf{out} of a place, you get rid of it by using a large amount of liquid.
 \textit{
	\begin{itemize}
	\item That won't flush all the sewage out, but it should unclog some stinking drains.
	\end{itemize}
}
\item verb \\
If you \textbf{flush} people or animals \textbf{out} of a place where they are hiding , you find or capture them by forcing them to come out of that place.
 \textit{
	\begin{itemize}
	\item They flushed them out of their hiding places.
	\item Police conduct raids to flush out illegal traders.
	\end{itemize}
}
\item adjective \\
If one object or surface is \textbf{flush}  \textbf{with} another, they are at the same height or distance from something else, so that they
form a single smooth surface.
 \textit{
	\begin{itemize}
	\item Make sure the tile is flush with the surrounding tiles.
	\end{itemize}
}
\item graded adjective \\
If you are \textbf{flush}  \textbf{with} money, you have a lot of it, usually only for a short time.
 \textit{
	\begin{itemize}
	\item At that time, many developing countries were flush with dollars earned from exports.
	\item If we're feeling flush we'll probably give them champagne.
	\end{itemize}
}
\item singular noun \\
\textbf{The}  \textbf{flush of} something is an intense feeling of excitement or pleasure that you have when you are experiencing it and for a short time afterwards .
 \textit{
	\begin{itemize}
	\item ...the first flush of young love.
	\item ...in the flush of victory.
	\end{itemize}
}
\item singular noun \\
A \textbf{flush of} something is a large quantity of it that comes suddenly or quickly.
 \textit{
	\begin{itemize}
	\item ...the flush of recent victories.
	\item ...a flush of memories.
	\end{itemize}
}
\end{enumerate}

\section*{color}
{\large \color{blue}  }
\subsection*{Explain}
\begin{enumerate}
\end{enumerate}

\section*{footstep}
{\large \color{blue}  footsteps  }
\subsection*{Explain}
\begin{enumerate}
\item countable noun \\
A \textbf{footstep} is the sound or mark that is made by someone walking each time their foot  touches the ground .
 \textit{
	\begin{itemize}
	\item I heard footsteps outside.
	\end{itemize}
}
\item  \\
 to follow in someone's footsteps \textit{
	\begin{itemize}
	\end{itemize}
}
\end{enumerate}

\section*{cry}
{\large \color{blue}  cries  crying  cried  }
\subsection*{Explain}
\begin{enumerate}
\item verb \\
When you \textbf{cry} , tears come from your eyes , usually because you are unhappy or hurt .
 \textbf{Cry} is also a noun .
 \textit{
	\begin{itemize}
	\item I hung up the phone and started to cry.
	\item Please don't cry.
	\item He cried with anger and frustration.
	\item ...a crying baby.
	\item A nurse patted me on the shoulder and said, 'You have a good cry, dear.'
	\end{itemize}
}
\item verb \\
If you \textbf{cry} something, you shout it or say it loudly.
 \textbf{Cry out} means the same as cry .
 \textit{
	\begin{itemize}
	\item 'Nancy Drew,' she cried, 'you're under arrest!'.
	\item I cried: 'It's wonderful news!'
	\item 'You're wrong, quite wrong!' Henry cried out, suddenly excited.
	\item She cried out that no storm was going to stop her.
	\end{itemize}
}
\item countable noun \\
A \textbf{cry} is a loud, high sound that you make when you feel a strong  emotion such as fear, pain, or pleasure .
 \textit{
	\begin{itemize}
	\item A cry of horror broke from me.
	\item Her brother gave a cry of recognition.
	\item With a cry, she rushed forward.
	\end{itemize}
}
\item countable noun \\
A \textbf{cry} is a shouted word or phrase, usually one that is intended to attract someone's attention .
 \textit{
	\begin{itemize}
	\item Thousands of people burst into cries of 'bravo' on the steps of the parliament.
	\item Passers-by heard his cries for help.
	\end{itemize}
}
\item countable noun \\
You can refer to a public protest about something or an appeal for something as a \textbf{cry} of some kind .
 \textit{
	\begin{itemize}
	\item There have been cries of outrage about this expenditure.
	\item Many other countries have turned a deaf ear to their cries for help.
	\end{itemize}
}
\item countable noun \\
A bird's or animal's \textbf{cry} is the loud, high sound that it makes.
 \textit{
	\begin{itemize}
	\item ...the cry of a seagull.
	\end{itemize}
}
\item  \\
 a far cry from \textit{
	\begin{itemize}
	\end{itemize}
}
\item  \\
 in full cry \textit{
	\begin{itemize}
	\end{itemize}
}
\item  \\
 for crying out loud \textit{
	\begin{itemize}
	\end{itemize}
}
\end{enumerate}

\section*{guide}
{\large \color{blue}  guides  guiding  guided  }
\subsection*{Explain}
\begin{enumerate}
\item countable noun \\
A \textbf{guide} is a book that gives you information or instructions to help you do or understand something.
 \textit{
	\begin{itemize}
	\item Our 10-page guide will help you to change your life for the better.
	\item ...the Pocket Guide to Butterflies of Britain and Europe.
	\end{itemize}
}
\item countable noun \\
A \textbf{guide} is a book that gives tourists information about a town , area, or country .
 \textit{
	\begin{itemize}
	\item Follow your guide to Angkor Wat.
	\end{itemize}
}
\item countable noun \\
A \textbf{guide} is someone who shows tourists around places such as museums or cities .
 \textit{
	\begin{itemize}
	\item We've arranged a walking tour of the city with your guide.
	\end{itemize}
}
\item verb \\
If you \textbf{guide} someone around a city, museum, or building , you show it to them and explain points of interest .
 \textit{
	\begin{itemize}
	\item ...a young Egyptologist who guided us through tombs and temples with enthusiasm.
	\item There will be guided walks around the site.
	\end{itemize}
}
\item countable noun \\
A \textbf{guide} is someone who shows people the way to a place in a difficult or dangerous  region .
 \textit{
	\begin{itemize}
	\item The mountain people say that, with guides, the journey can be done in fourteen days.
	\end{itemize}
}
\item countable noun \\
A \textbf{guide} is something that can be used to help you plan your actions or to form an opinion about something.
 \textit{
	\begin{itemize}
	\item As a rough guide, a horse needs 2.5 per cent of his body weight in food every day.
	\item When selecting fresh fish, let your taste buds be your guide.
	\end{itemize}
}
\item verb \\
If you \textbf{guide} someone somewhere , you go there with them in order to show them the way.
 \textit{
	\begin{itemize}
	\item He took the bewildered Elliott by the arm and guided him out.
	\item Khandoo guided me through the dark alleys until the smell told me we had arrived.
	\end{itemize}
}
\item verb \\
If you \textbf{guide} a vehicle somewhere, you control it carefully to make sure that it goes in the right  direction .
 \textit{
	\begin{itemize}
	\item Captain Shelton guided his plane down the runway and took off.
	\end{itemize}
}
\item verb \\
If something \textbf{guides} you somewhere, it gives you the information you need in order to go in the right direction.
 \textit{
	\begin{itemize}
	\item They sailed across the Baltic and North Seas with only a compass to guide them.
	\end{itemize}
}
\item verb \\
If something or someone \textbf{guides} you, they influence your actions or decisions .
 \textit{
	\begin{itemize}
	\item He should have let his instinct guide him.
	\item Development has been guided by a concern for the ecology of the area.
	\item My mother, whose guiding principle in life was doing right, had a far greater influence
on me.
	\end{itemize}
}
\item verb \\
If you \textbf{guide} someone through something that is difficult to understand or to achieve , you help them to understand it or to achieve success in it.
 \textit{
	\begin{itemize}
	\item ...a free helpline to guide you through the application process.
	\item The 42-year-old Scot has guided the team to victory in three of their last five games.
	\end{itemize}
}
\end{enumerate}

\section*{damn}
{\large \color{blue}  damns  damning  damned  }
\subsection*{Explain}
\begin{enumerate}
\item exclamation \\
\textbf{Damn} , \textbf{damn it} , and \textbf{dammit} are used by some people to express  anger or impatience .
 \textit{
	\begin{itemize}
	\item Don't be flippant, damn it! This is serious.
	\end{itemize}
}
\item adjective \\
\textbf{Damn} is used by some people to emphasize what they are saying .
 \textbf{Damn} is also an adverb .
 \textit{
	\begin{itemize}
	\item There's not a damn thing you can do about it now.
	\item As it turned out, I was damn right.
	\item Let's have a damn good party.
	\end{itemize}
}
\item verb \\
If you say that a person or a news  report  \textbf{damns} something such as a policy or action, you mean that they are very critical of it.
 \textit{
	\begin{itemize}
	\item ...a sensational book in which she damns the ultra-right party.
	\item ...a report damning the chocolate advertising people for targeting women in their
campaigns.
	\end{itemize}
}
\item  \\
 not give a damn \textit{
	\begin{itemize}
	\end{itemize}
}
\item  \\
 damn near \textit{
	\begin{itemize}
	\end{itemize}
}
\item  \\
 near as damn it \textit{
	\begin{itemize}
	\end{itemize}
}
\end{enumerate}

\section*{heel}
{\large \color{blue}  heels  heeling  heeled  }
\subsection*{Explain}
\begin{enumerate}
\item countable noun \\
Your \textbf{heel} is the back part of your foot, just below your ankle.
 \textit{
	\begin{itemize}
	\end{itemize}
}
\item countable noun \\
The \textbf{heel} of a shoe is the raised part on the bottom at the back.
 \textit{
	\begin{itemize}
	\item He kicked it shut with the heel of his boot.
	\item ...the shoes with the high heels.
	\end{itemize}
}
\item plural noun \\
\textbf{Heels} are women's shoes that are raised very high at the back.
 \textit{
	\begin{itemize}
	\item ...two well-dressed ladies in high heels.
	\item ...the old adage that you shouldn't wear heels with trousers.
	\end{itemize}
}
\item countable noun \\
The \textbf{heel} of a sock or stocking is the part that covers your heel.
 \textit{
	\begin{itemize}
	\end{itemize}
}
\item countable noun \\
The \textbf{heel of} your hand is the rounded  pad at the bottom of your palm.
 \textit{
	\begin{itemize}
	\end{itemize}
}
\item  \\
 at sb's heels \textit{
	\begin{itemize}
	\end{itemize}
}
\item  \\
 bring sb to heel \textit{
	\begin{itemize}
	\end{itemize}
}
\item  \\
 to click your heels \textit{
	\begin{itemize}
	\end{itemize}
}
\item  \\
 to cool your heels \textit{
	\begin{itemize}
	\end{itemize}
}
\item  \\
 to dig one's heels in \textit{
	\begin{itemize}
	\end{itemize}
}
\item  \\
 hard on the heels of/hot on the heels of \textit{
	\begin{itemize}
	\end{itemize}
}
\item  \\
 hot on sb's heels \textit{
	\begin{itemize}
	\end{itemize}
}
\item  \\
 kick one's heels \textit{
	\begin{itemize}
	\end{itemize}
}
\item  \\
 on one's heel \textit{
	\begin{itemize}
	\end{itemize}
}
\item  \\
 take to one's heels \textit{
	\begin{itemize}
	\end{itemize}
}
\end{enumerate}

\section*{farewell}
{\large \color{blue}  farewells  }
\subsection*{Explain}
\begin{enumerate}
\item convention \\
\textbf{Farewell} means the same as goodbye .
 \textbf{Farewell} is also a noun .
 \textit{
	\begin{itemize}
	\item They said their farewells there at the cafe.
	\end{itemize}
}
\end{enumerate}

\section*{holiday}
{\large \color{blue}  holidays  holidaying  holidayed  }
\subsection*{Explain}
\begin{enumerate}
\item countable noun \\
A \textbf{holiday} is a period of time during which you relax and enjoy yourself away from home . People sometimes  refer to their holiday as their \textbf{holidays} .
 \textit{
	\begin{itemize}
	\item I've just come back from a holiday in the United States.
	\item We rang Duncan to ask where he was going on holiday.
	\item Ischia is a popular seaside holiday resort.
	\item We're going to Scotland for our holidays.
	\end{itemize}
}
\item countable noun \\
A \textbf{holiday} is a day when people do not go to work or school because of a religious or national festival.
 \textit{
	\begin{itemize}
	\item New Year's Day is a public holiday throughout Britain.
	\item He invited her to spend the Fourth of July holiday at his summer home on Fire Island.
	\item Bad weather has caused dozens of flight cancellations over the holiday weekend.
	\end{itemize}
}
\item plural noun \\
\textbf{The}  \textbf{holidays} are the time when children do not have to go to school.
 \textit{
	\begin{itemize}
	\item ...the first day of the school holidays.
	\end{itemize}
}
\item uncountable noun \\
If you have a particular number of days' or weeks ' \textbf{holiday} , you do not have to go to work for that number of days or weeks.
 \textit{
	\begin{itemize}
	\item Every worker will be entitled to four weeks' paid holiday a year.
	\end{itemize}
}
\item verb \\
If you \textbf{are holidaying} in a place away from home, you are on holiday there.
 \textit{
	\begin{itemize}
	\item Sampling the local cuisine is one of the delights of holidaying abroad.
	\item Vacant rooms on the campus were being used by holidaying families.
	\end{itemize}
}
\end{enumerate}

\section*{hound}
{\large \color{blue}  hounds  hounding  hounded  }
\subsection*{Explain}
\begin{enumerate}
\item countable noun \\
A \textbf{hound} is a type of dog that is often used for hunting or racing.
 \textit{
	\begin{itemize}
	\end{itemize}
}
\item verb \\
If someone \textbf{hounds} you, they constantly disturb or speak to you in an annoying or upsetting way.
 \textit{
	\begin{itemize}
	\item Newcomers are constantly hounding them for advice.
	\item From the start of the season, the Arsenal striker has been hounded by the press.
	\end{itemize}
}
\item verb \\
If someone \textbf{is hounded}  \textbf{out of} a job or place, they are forced to leave it, often because other people are constantly
 criticizing them.
 \textit{
	\begin{itemize}
	\item There is a general view around that he has been hounded out of office by the press.
	\end{itemize}
}
\end{enumerate}

\section*{hello}
{\large \color{blue}  hellos  }
\subsection*{Explain}
\begin{enumerate}
\item convention \\
You say ' \textbf{Hello} ' to someone when you meet them.
 \textbf{Hello} is also a noun .
 \textit{
	\begin{itemize}
	\item Hello, Trish.
	\item Do you want to pop your head in and say hallo to my girlfriend?
	\item The salesperson greeted me with a warm hello.
	\end{itemize}
}
\item convention \\
You say ' \textbf{Hello} ' to someone at the beginning of a phone  conversation , either when you answer the phone or before you give your name or say why you are phoning.
 \textit{
	\begin{itemize}
	\item A moment later, Cohen picked up the phone. 'Hello?'
	\item Hallo, may I speak to Frank, please.
	\end{itemize}
}
\item convention \\
You can call ' \textbf{hello} ' to attract someone's attention.
 \textit{
	\begin{itemize}
	\item She could see the open door of a departmental office. 'Hello! Excuse me. This is
the department of French, isn't it?'.
	\item Very softly, she called out: 'Hallo? Who's there?'
	\end{itemize}
}
\end{enumerate}

\section*{hypothesis}
{\large \color{blue}  hypotheses  }
\subsection*{Explain}
\begin{enumerate}
\item variable noun \\
A \textbf{hypothesis} is an idea which is suggested as a possible explanation for a particular situation or condition, but which has not yet been proved to be correct .
 \textit{
	\begin{itemize}
	\item To test this hypothesis, scientists can construct a simplified laboratory experiment.
	\item Different hypotheses have been put forward to explain why these foods are more likely
to cause problems.
	\end{itemize}
}
\end{enumerate}

\section*{hi}
{\large \color{blue}  }
\subsection*{Explain}
\begin{enumerate}
\item convention \\
In informal situations, you say ' \textbf{hi} ' to greet someone.
 \textit{
	\begin{itemize}
	\item 'Hi, Liz,' she said shyly.
	\end{itemize}
}
\end{enumerate}

\section*{leadership}
{\large \color{blue}  leaderships  }
\subsection*{Explain}
\begin{enumerate}
\item countable noun \\
You refer to people who are in control of a group or organization as the \textbf{leadership} .
 \textit{
	\begin{itemize}
	\item He is expected to hold talks with both the Croatian and Slovenian leaderships.
	\item ...the Labour leadership of Haringey council in north London.
	\end{itemize}
}
\item uncountable noun \\
Someone's \textbf{leadership} is their position or state of being in control of a group of people.
 \textit{
	\begin{itemize}
	\item He praised her leadership during the crisis.
	\end{itemize}
}
\item uncountable noun \\
\textbf{Leadership} refers to the qualities that make someone a good leader, or the methods a leader uses to do his or her job .
 \textit{
	\begin{itemize}
	\item What most people want to see is determined, decisive action and firm leadership.
	\end{itemize}
}
\end{enumerate}

\section*{highland}
{\large \color{blue}  }
\subsection*{Explain}
\begin{enumerate}
\item noun \\
1.  2.  \textit{
	\begin{itemize}
	\end{itemize}
}
\end{enumerate}

\section*{owl}
{\large \color{blue}  owls  }
\subsection*{Explain}
\begin{enumerate}
\item countable noun \\
An \textbf{owl} is a bird with a flat face, large eyes, and a small sharp  beak . Most owls obtain their food by hunting small animals at night .
 \textit{
	\begin{itemize}
	\end{itemize}
}
\end{enumerate}

\section*{if}
{\large \color{blue}  }
\subsection*{Explain}
\begin{enumerate}
\item conjunction \\
You use \textbf{if} in conditional  sentences to introduce the circumstances in which an event or situation  might  happen , might be happening , or might have happened.
 \textit{
	\begin{itemize}
	\item She gets very upset if I exclude her.
	\item You'll feel a lot better about yourself if you work on solutions to your upsetting
situations.
	\item You can go if you want.
	\item If you went into town, you'd notice all the pubs have loud jukeboxes.
	\item What I did was right and if I had done anything less it would have been wrong.
	\item Fry remaining peppers, adding a little more dressing if necessary.
	\item Do you have a knack for coming up with ideas? If so, we would love to hear from you.
	\end{itemize}
}
\item conjunction \\
You use \textbf{if} in indirect questions where the answer is either 'yes' or 'no'.
 \textit{
	\begin{itemize}
	\item He asked if I had left with you, and I said no.
	\item I wonder if I might have a word with Mr Abbot?
	\end{itemize}
}
\item  \\
 if not \textit{
	\begin{itemize}
	\end{itemize}
}
\item conjunction \\
You use \textbf{if} , usually with 'can', 'could', ' may ', or 'might', at a point in a conversation when you are politely trying to make a point, change the subject , or interrupt another speaker .
 \textit{
	\begin{itemize}
	\item If I could just make another small point.
	\item So, if we may return strictly to athletics again for a few minutes.
	\item Well, it's the old argument Max, which is a bit ridiculous if you don't mind me saying
so.
	\item Well if you want my opinion, unless you do it soon you're gonna lose the opportunity.
	\end{itemize}
}
\item conjunction \\
You use \textbf{if} at or near the beginning of a clause when politely asking someone to do something.
 \textit{
	\begin{itemize}
	\item I wonder if you'd be kind enough to give us some information, please?
	\item If you will just sign here, please.
	\end{itemize}
}
\item conjunction \\
You use \textbf{if} to introduce a subordinate clause in which you admit a fact which you regard as less important than the statement in the main clause.
 \textit{
	\begin{itemize}
	\item If there was any disappointment it was probably temporary.
	\item So what if sometimes they stayed rather late, it doesn't mean anything.
	\end{itemize}
}
\item phrase \\
You use \textbf{if not} in front of a word or phrase to indicate that your statement does not apply to that word or phrase, but to something closely related to it that you also  mention .
 \textit{
	\begin{itemize}
	\item A number of recent advances hold out if not the hope of a cure, then at least the
possibility of a treatment which could stop the spread of the virus.
	\item She understood his meaning, if not his words.
	\end{itemize}
}
\item phrase \\
You use \textbf{if ever} with past  tenses when you are introducing a description of a person or thing, to emphasize how appropriate it is.
 \textit{
	\begin{itemize}
	\item I became a distraught, worried mother, a useless role if ever there was one.
	\item If ever there was the right person in the right job it was she.
	\item If ever a man needed your love, I need it.
	\end{itemize}
}
\item  \\
 if only \textit{
	\begin{itemize}
	\end{itemize}
}
\item phrase \\
You use \textbf{if only} to express a wish or desire, especially one that cannot be fulfilled.
 \textit{
	\begin{itemize}
	\item If only you had told me that some time ago.
	\item If only it were that simple!
	\item 'Hey, listen to me, all that nonsense is over.'—'If only, Timothy, if only.'
	\end{itemize}
}
\item phrase \\
You use \textbf{as if} when you are making a judgment about something that you see or notice . Your belief or impression might be correct , or it might be wrong .
 \textit{
	\begin{itemize}
	\item The whole room looks as if it has been lovingly put together over the years.
	\item His heart was pounding, as if he were frightened.
	\end{itemize}
}
\item phrase \\
You use \textbf{as if} to describe something or someone by comparing them with another thing or person.
 \textit{
	\begin{itemize}
	\item He points two fingers at his head, as if he were holding a gun.
	\item The two cousins looked as if they'd been carved from blocks of ice.
	\end{itemize}
}
\item phrase \\
You use \textbf{as if} to emphasize that something is not true .
 \textit{
	\begin{itemize}
	\item He suggested that I loved my birds more than him: as if I would.
	\item Getting my work done! My God! As if it mattered.
	\end{itemize}
}
\item  \\
 if anything \textit{
	\begin{itemize}
	\end{itemize}
}
\item  \\
 it's not as if \textit{
	\begin{itemize}
	\end{itemize}
}
\item  \\
 if I were you \textit{
	\begin{itemize}
	\end{itemize}
}
\end{enumerate}

\section*{pie}
{\large \color{blue}  pies  }
\subsection*{Explain}
\begin{enumerate}
\item variable noun \\
A \textbf{pie} consists of meat , vegetables , or fruit baked in pastry.
 \textit{
	\begin{itemize}
	\item ...a chicken pie.
	\item ...apple pie and custard.
	\end{itemize}
}
\item  \\
 pie in the sky \textit{
	\begin{itemize}
	\end{itemize}
}
\end{enumerate}

\section*{lest}
{\large \color{blue}  }
\subsection*{Explain}
\begin{enumerate}
\item conjunction \\
If you do something \textbf{lest} something unpleasant should happen , you do it to try to prevent the unpleasant thing from happening .
 \textit{
	\begin{itemize}
	\item I was afraid to open the door lest he should follow me.
	\item The president has gone along with the hardliners lest they be tempted to oust him.
	\item And, lest we forget, Einstein wrote his most influential papers while working as
a clerk.
	\end{itemize}
}
\end{enumerate}

\section*{pig}
{\large \color{blue}  pigs  pigging  pigged  }
\subsection*{Explain}
\begin{enumerate}
\item countable noun \\
A \textbf{pig} is a pink or black animal with short legs and not much hair on its skin. Pigs are often kept on farms for their meat, which is called pork, ham , bacon , or gammon .
 \textit{
	\begin{itemize}
	\item ...the grunting of the pigs.
	\item ...a pig farmer.
	\end{itemize}
}
\item countable noun \\
If you call someone a \textbf{pig} , you think that they are unpleasant in some way, especially that they are greedy or unkind .
 \textit{
	\begin{itemize}
	\end{itemize}
}
\item singular noun \\
If you say that something is, for example , \textbf{a pig of} a job , you mean it is very difficult.
 \textit{
	\begin{itemize}
	\item It's been a pig of a week.
	\end{itemize}
}
\item verb \\
If you say that people \textbf{are pigging}  \textbf{themselves} , you are criticizing them for eating a very large amount at one meal .
 \textit{
	\begin{itemize}
	\item After pigging herself on ice cream she went upstairs.
	\end{itemize}
}
\item  \\
 to make a pig's ear of \textit{
	\begin{itemize}
	\end{itemize}
}
\item  \\
 pigs might fly \textit{
	\begin{itemize}
	\end{itemize}
}
\item  \\
 make a pig of oneself \textit{
	\begin{itemize}
	\end{itemize}
}
\end{enumerate}

\section*{moreover}
{\large \color{blue}  }
\subsection*{Explain}
\begin{enumerate}
\item adverb \\
You use \textbf{moreover} to introduce a piece of information that adds to or supports the previous statement.
 \textit{
	\begin{itemize}
	\item There was a man immediately behind her. Moreover, he was observing her strangely.
	\item It is a dull place during mid-year vacations and, moreover, a hot one.
	\end{itemize}
}
\end{enumerate}

\section*{pigeon}
{\large \color{blue}  pigeons  }
\subsection*{Explain}
\begin{enumerate}
\item countable noun \\
A \textbf{pigeon} is a bird, usually grey in colour, which has a fat body. Pigeons often live in towns.
 \textit{
	\begin{itemize}
	\end{itemize}
}
\end{enumerate}

\section*{nevertheless}
{\large \color{blue}  }
\subsection*{Explain}
\begin{enumerate}
\item adverb \\
You use \textbf{nevertheless} when saying something that contrasts with what has just been said .
 \textit{
	\begin{itemize}
	\item Many marriages fail. Nevertheless, people continue to get married.
	\item His father, though ill-equipped for the project, had nevertheless tried his best.
	\end{itemize}
}
\end{enumerate}

\section*{pork}
{\large \color{blue}  }
\subsection*{Explain}
\begin{enumerate}
\item uncountable noun \\
\textbf{Pork} is meat from a pig , usually fresh and not smoked or salted .
 \textit{
	\begin{itemize}
	\item ...fried pork chops.
	\item ...a packet of pork sausages.
	\end{itemize}
}
\end{enumerate}

\section*{prejudice}
{\large \color{blue}  prejudices  prejudicing  prejudiced  }
\subsection*{Explain}
\begin{enumerate}
\item variable noun \\
\textbf{Prejudice} is an unreasonable dislike of a particular group of people or things, or a preference for one group of people or things over another.
 \textit{
	\begin{itemize}
	\item Ghaffur alleged that he was repeatedly subjected to racial prejudice.
	\item There is widespread prejudice against workers over 45.
	\item He said he hoped the Swiss authorities would investigate the case thoroughly and
without prejudice.
	\end{itemize}
}
\item verb \\
If you \textbf{prejudice} someone or something, you influence them so that they are unfair in some way.
 \textit{
	\begin{itemize}
	\item I think your South American youth has prejudiced you.
	\item The report was held back for fear of prejudicing his trial.
	\item He claimed his case would be prejudiced if it became known he was refusing to answer
questions.
	\end{itemize}
}
\item verb \\
If someone \textbf{prejudices} another person's situation , they do something which makes it worse than it should be.
 \textit{
	\begin{itemize}
	\item Her study was not in any way intended to prejudice the future development of the
college.
	\item They claim the council has prejudiced their health by failing to deal with asbestos.
	\end{itemize}
}
\item  \\
 without prejudice to \textit{
	\begin{itemize}
	\end{itemize}
}
\end{enumerate}

\section*{reduction}
{\large \color{blue}  reductions  }
\subsection*{Explain}
\begin{enumerate}
\item countable noun \\
When there is a \textbf{reduction}  \textbf{in} something, it is made smaller.
 \textit{
	\begin{itemize}
	\item ...a future reduction in U.K. interest rates.
	\item Many companies have announced dramatic reductions in staff.
	\end{itemize}
}
\item uncountable noun \\
\textbf{Reduction} is the act of making something smaller in size or amount, or less in degree .
 \textit{
	\begin{itemize}
	\item ...a new strategic arms reduction agreement.
	\end{itemize}
}
\end{enumerate}

\section*{provided}
{\large \color{blue}  }
\subsection*{Explain}
\begin{enumerate}
\item conjunction \\
If you say that something will  happen  \textbf{provided} or \textbf{provided that} something else happens, you mean that the first thing will happen only if the second thing also happens.
 \textit{
	\begin{itemize}
	\item It should all work nicely, provided that nobody loses faith in the idea.
	\item Provided they are fit I see no reason why they shouldn't go on playing.
	\end{itemize}
}
\end{enumerate}

\section*{removal}
{\large \color{blue}  removals  }
\subsection*{Explain}
\begin{enumerate}
\item uncountable noun \\
The \textbf{removal} of something is the act of removing it.
 \textit{
	\begin{itemize}
	\item What they expected to be the removal of a small lump turned out to be major surgery.
	\item Parliament had decided that his removal from power was illegal.
	\item ...popular methods of hair removal.
	\end{itemize}
}
\item variable noun \\
\textbf{Removal} is the process of transporting  furniture or equipment from one building to another.
 \textit{
	\begin{itemize}
	\item Home removals are best done in cool weather.
	\item They were in the furniture removal business.
	\item ...a removal van.
	\end{itemize}
}
\end{enumerate}

\section*{remark}
{\large \color{blue}  remarks  remarking  remarked  }
\subsection*{Explain}
\begin{enumerate}
\item verb \\
If you \textbf{remark} that something is the case , you say that it is the case.
 \textit{
	\begin{itemize}
	\item I remarked that I would go shopping that afternoon.
	\item 'Some people have more money than sense,' Winston had remarked.
	\item On several occasions she had remarked on the boy's improvement.
	\end{itemize}
}
\item countable noun \\
If you make a \textbf{remark} about something, you say something about it.
 \textit{
	\begin{itemize}
	\item He made some remarks about the President for which he had to apologise.
	\item ...the FIFA president's controversial remarks that the Portuguese superstar and other
top footballers are modern-day slaves.
	\end{itemize}
}
\end{enumerate}

\section*{sack}
{\large \color{blue}  sacks  sacking  sacked  }
\subsection*{Explain}
\begin{enumerate}
\item countable noun \\
A \textbf{sack} is a large bag made of rough  woven material. Sacks are used to carry or store things such as vegetables or coal .
 \textit{
	\begin{itemize}
	\item ...a sack of potatoes.
	\end{itemize}
}
\item verb \\
If your employers  \textbf{sack} you, they tell you that you can no longer work for them because you have done something that they
did not like or because your work was not good enough.
 \textbf{Sack} is also a noun .
 \textit{
	\begin{itemize}
	\item He had sacked the teacher as soon as he heard of her misconduct.
	\item An airport worker who was sacked for wearing a nose stud has won back her job.
	\item People who make mistakes can be given the sack the same day.
	\end{itemize}
}
\item verb \\
When an army \textbf{sacks} a town or city, they destroy it, taking  away all valuable things.
 \textbf{Sack} is also a noun.
 \textit{
	\begin{itemize}
	\item In 1527 Imperial troops sacked the French ambassador's residence in Rome.
	\item The Odyssey tells what happened to the Greek heroes after the sack of Troy.
	\end{itemize}
}
\item singular noun \\
Some people refer to bed as \textbf{the sack} .
 \textit{
	\begin{itemize}
	\end{itemize}
}
\end{enumerate}

\section*{stroll}
{\large \color{blue}  strolls  strolling  strolled  }
\subsection*{Explain}
\begin{enumerate}
\item verb \\
If you \textbf{stroll}  somewhere , you walk there in a slow , relaxed way.
 \textbf{Stroll} is also a noun .
 \textit{
	\begin{itemize}
	\item A young couple stroll past me hand in hand.
	\item Afterwards, we strolled back, put the kettle on and settled down with the newspapers.
	\item After dinner, I took a stroll round the city.
	\end{itemize}
}
\end{enumerate}

\section*{salesman}
{\large \color{blue}  salesmen  }
\subsection*{Explain}
\begin{enumerate}
\item countable noun \\
A \textbf{salesman} is a man whose job is to sell things, especially  directly to shops or other businesses on behalf of a company.
 \textit{
	\begin{itemize}
	\item ...an insurance salesman.
	\end{itemize}
}
\end{enumerate}

\section*{than}
{\large \color{blue}  }
\subsection*{Explain}
\begin{enumerate}
\item preposition \\
You use \textbf{than} after a comparative  adjective or adverb in order to link two parts of a comparison.
 \textbf{Than} is also a conjunction .
 \textit{
	\begin{itemize}
	\item The radio only weighs a few ounces and is smaller than a mobile phone.
	\item She walks far more slowly than we do.
	\item He wished he could have helped her more than he did.
	\item Sometimes patients are more depressed six months later than when they first hear
the bad news.
	\end{itemize}
}
\item preposition \\
You use \textbf{than} when you are stating a number, quantity, or value approximately by saying that it is above or below another number, quantity, or value.
 \textit{
	\begin{itemize}
	\item They talked on the phone for more than an hour.
	\item The tournament starts in less than two months' time.
	\item Head teachers yesterday demanded a nine per cent rise, more than twice the rate of
inflation.
	\end{itemize}
}
\item conjunction \\
You use \textbf{than} in order to link two parts of a contrast , for example in order to state a preference.
 \textit{
	\begin{itemize}
	\item The arrangement was more a formality than a genuine partnership of two nations.
	\item I would rather stare at a clear, star-filled sky than a TV set.
	\item I would sooner give up sleep than miss my evening class.
	\end{itemize}
}
\end{enumerate}

\section*{score}
{\large \color{blue}  scores  scoring  scored  }
\subsection*{Explain}
\begin{enumerate}
\item verb \\
In a sport or game, if a player \textbf{scores} a goal or a point, they gain a goal or point.
 \textit{
	\begin{itemize}
	\item He scored three goals in the first three minutes.
	\item England scored 282 in their first innings.
	\item Gascoigne almost scored in the opening minute.
	\end{itemize}
}
\item verb \\
If you \textbf{score} a particular number or amount, for example as a mark in a test, you achieve that
number or amount.
 \textit{
	\begin{itemize}
	\item Kelly had scored an average of 147 on three separate IQ tests.
	\item Congress as an institution scores low in public opinion polls.
	\end{itemize}
}
\item countable noun \\
Someone's \textbf{score} in a game or test is a number, for example, a number of points or runs, which shows
what they have achieved or what level they have reached .
 \textit{
	\begin{itemize}
	\item The 26-year-old finished ninth with a score of 100.985.
	\item ...the highest score by an England batsman in this form of cricket.
	\item There was a strong link between parents' numeracy and children's maths scores.
	\end{itemize}
}
\item countable noun \\
The \textbf{score} in a game is the result of it or the current situation, as indicated by the number
of goals, runs, or points obtained by the two teams or players.
 \textit{
	\begin{itemize}
	\item 4-1 was the final score.
	\item They beat the Giants by a score of 7 to 3.
	\item Even in Zurich he kept up with the County cricket scores.
	\end{itemize}
}
\item verb \\
If you \textbf{score} a success, a victory , or a hit , you are successful in what you are doing.
 \textit{
	\begin{itemize}
	\item His abiding passion was ocean racing, at which he scored many successes.
	\item In recent months, the rebels have scored some significant victories.
	\item Soldiers using a multiple rocket launcher scored a direct hit on the steeple of a
church.
	\end{itemize}
}
\item countable noun \\
The \textbf{score} of a film, play, or similar production is the music which is written or used for
it.
 \textit{
	\begin{itemize}
	\item The dance is accompanied by an original score by Henry Torgue.
	\item ...the composer of classic film scores such as West Side Story.
	\end{itemize}
}
\item countable noun \\
The \textbf{score} of a piece of music is the written version of it.
 \textit{
	\begin{itemize}
	\item He recognizes enough notation to be able to follow a score.
	\end{itemize}
}
\item verb \\
If you \textbf{score} a piece of music, you write it or arrange it for specific instruments or voices.
 \textit{
	\begin{itemize}
	\item Strauss spent much of 1941 scoring his last opera, Capriccio.
	\item He wrote and scored a piece for a chamber music ensemble.
	\end{itemize}
}
\item quantifier \\
If you refer to \textbf{scores of} things or people, you are emphasizing that there are very many of them.
 You can also use \textbf{scores} as a pronoun .
 \textit{
	\begin{itemize}
	\item Campaigners lit scores of bonfires in ceremonies to mark the anniversary.
	\item Two people were killed and scores were injured.
	\end{itemize}
}
\item number \\
A \textbf{score} is twenty or approximately twenty.
 \textit{
	\begin{itemize}
	\item A score of countries may be producing or planning to obtain chemical weapons.
	\item The company already has around four score titles commissioned and planned for publication.
	\item The Bible states that the life of man is three score and ten.
	\end{itemize}
}
\item verb \\
If you \textbf{score} a surface with something sharp , you cut a line or number of lines in it.
 \textit{
	\begin{itemize}
	\item Lightly score the surface of the steaks with a knife.
	\end{itemize}
}
\item verb \\
If someone \textbf{scores} drugs, they buy them illegally.
 \textit{
	\begin{itemize}
	\item Me and my mate went to score a kilo of amphetamine.
	\end{itemize}
}
\item  \\
 by the score \textit{
	\begin{itemize}
	\end{itemize}
}
\item  \\
 keep score \textit{
	\begin{itemize}
	\end{itemize}
}
\item  \\
 know the score \textit{
	\begin{itemize}
	\end{itemize}
}
\item  \\
 on that/this score \textit{
	\begin{itemize}
	\end{itemize}
}
\item  \\
 score a point over / score points off \textit{
	\begin{itemize}
	\end{itemize}
}
\item  \\
 to settle a score \textit{
	\begin{itemize}
	\end{itemize}
}
\end{enumerate}

\section*{trench}
{\large \color{blue}  trenches  }
\subsection*{Explain}
\begin{enumerate}
\item countable noun \\
A \textbf{trench} is a long narrow  channel that is cut into the ground , for example in order to lay  pipes or get  rid of water.
 \textit{
	\begin{itemize}
	\end{itemize}
}
\item countable noun \\
A \textbf{trench} is a long narrow channel in the ground used by soldiers in order to protect themselves from the enemy . People often refer to the battle grounds of the First World  War in Northern  France and Belgium as \textbf{the trenches} .
 \textit{
	\begin{itemize}
	\item We fought with them in the trenches.
	\item ...trench warfare.
	\end{itemize}
}
\end{enumerate}

\section*{unless}
{\large \color{blue}  }
\subsection*{Explain}
\begin{enumerate}
\item conjunction \\
You use \textbf{unless} to introduce the only circumstances in which an event you are mentioning will not take place or in which a statement you are making is not true .
 \textit{
	\begin{itemize}
	\item Unless you are trying to lose weight to please yourself, it's hard to stay motivated.
	\item We cannot understand disease unless we understand the person who has the disease.
	\item I'm not happy unless I ride or drive every day.
	\end{itemize}
}
\end{enumerate}

\section*{slope}
{\large \color{blue}  slopes  sloping  sloped  }
\subsection*{Explain}
\begin{enumerate}
\item countable noun \\
A \textbf{slope} is the side of a mountain , hill, or valley .
 \textit{
	\begin{itemize}
	\item Saint-Christo is perched on a mountain slope.
	\item ...the lower slopes of the Himalayas.
	\end{itemize}
}
\item countable noun \\
A \textbf{slope} is a surface that is at an angle, so that one end is higher than the other.
 \textit{
	\begin{itemize}
	\item The street must have been on a slope.
	\end{itemize}
}
\item verb \\
If a surface \textbf{slopes} , it is at an angle, so that one end is higher than the other.
 \textit{
	\begin{itemize}
	\item The bank sloped down sharply to the river.
	\item The garden sloped quite steeply.
	\end{itemize}
}
\item verb \\
If something \textbf{slopes} , it leans to the right or to the left  rather than being upright .
 \textit{
	\begin{itemize}
	\item The writing sloped backwards.
	\item He wonders why the digits on his calculator slope to the right.
	\end{itemize}
}
\item countable noun \\
The \textbf{slope} of something is the angle at which it slopes.
 \textit{
	\begin{itemize}
	\item The slope increases as you go up the curve.
	\item ...a slope of ten degrees.
	\end{itemize}
}
\item verb \\
If someone \textbf{slopes}  \textbf{into} or \textbf{out of} a place, they enter or leave it quickly and quietly , especially because they are trying to avoid or escape something.
 \textit{
	\begin{itemize}
	\item She sloped off quietly on Saturday afternoon.
	\item They sloped into their hotel at 6am.
	\end{itemize}
}
\end{enumerate}

\section*{welcome}
{\large \color{blue}  welcomes  welcoming  welcomed  }
\subsection*{Explain}
\begin{enumerate}
\item verb \\
If you \textbf{welcome} someone, you greet them in a friendly  way when they arrive  somewhere .
 \textbf{Welcome} is also a noun .
 \textit{
	\begin{itemize}
	\item Several people came by to welcome me.
	\item She was there to welcome him home from war.
	\item The delegates received a welcoming speech by the President.
	\item There would be a fantastic welcome awaiting him back here.
	\end{itemize}
}
\item convention \\
You use \textbf{welcome} in expressions such as \textbf{welcome home} , \textbf{welcome to London} , and \textbf{welcome back} when you are greeting someone who has just arrived somewhere.
 \textit{
	\begin{itemize}
	\item Welcome to Washington.
	\item Welcome back, Deborah–It's good to have you here.
	\end{itemize}
}
\item verb \\
If you \textbf{welcome} an action, decision , or situation , you approve of it and are pleased that it has occurred.
 \textbf{Welcome} is also a noun.
 \textit{
	\begin{itemize}
	\item She welcomed this move but said that overall the changes didn't go far enough.
	\item In Germany, the move was welcomed by the Bundesbank president.
	\item Environmental groups have given a guarded welcome to the Prime Minister's proposal.
	\end{itemize}
}
\item adjective \\
If you describe something as \textbf{welcome} , you mean that people wanted it and are happy that it has occurred.
 \textit{
	\begin{itemize}
	\item This was certainly a welcome change of fortune.
	\item The new 25 metre pool for more serious swimmers is a welcome addition.
	\item Any progress in reducing chemical weapons is welcome.
	\end{itemize}
}
\item verb \\
If you say that you \textbf{welcome} certain people or actions, you are inviting and encouraging people to do something, for example to come to a particular place.
 \textit{
	\begin{itemize}
	\item We welcome you to join us on a special tour which explores this unique Australian
attraction.
	\item We would welcome your views about the survey.
	\end{itemize}
}
\item adjective \\
If you say that someone is \textbf{welcome} in a particular place, you are encouraging them to go there by telling them that they will be liked and accepted.
 \textit{
	\begin{itemize}
	\item New members are always welcome.
	\item I told him he wasn't welcome in my home.
	\end{itemize}
}
\item adjective \\
If you tell someone that they are \textbf{welcome}  \textbf{to} do something, you are encouraging them to do it by telling them that they are allowed to do it.
 \textit{
	\begin{itemize}
	\item You are welcome to visit the hospital at any time.
	\item ...a conservatory which guests are welcome to use.
	\end{itemize}
}
\item adjective \\
If you say that someone is \textbf{welcome to} something, you mean that you do not want it yourself because you do not like it and
you are very willing for them to have it.
 \textit{
	\begin{itemize}
	\item If women want to take on the business world, they are welcome to it as far as I'm
concerned.
	\end{itemize}
}
\item  \\
 make sb welcome \textit{
	\begin{itemize}
	\end{itemize}
}
\item  \\
 to outstay your welcome \textit{
	\begin{itemize}
	\end{itemize}
}
\item  \\
 you're welcome \textit{
	\begin{itemize}
	\end{itemize}
}
\end{enumerate}

\section*{symbol}
{\large \color{blue}  symbols  }
\subsection*{Explain}
\begin{enumerate}
\item countable noun \\
Something that is a \textbf{symbol}  \textbf{of} a society or an aspect of life  seems to represent it because it is very typical of it.
 \textit{
	\begin{itemize}
	\item To them, the monarchy is the special symbol of nationhood.
	\item She was put under house arrest but remained a powerful symbol in the election.
	\end{itemize}
}
\item countable noun \\
A \textbf{symbol}  \textbf{of} something such as an idea is a shape or design that is used to represent it.
 \textit{
	\begin{itemize}
	\item Later in this same passage Yeats resumes his argument for the Rose as an Irish symbol.
	\item I frequently use sunflowers as symbols of strength.
	\end{itemize}
}
\item countable noun \\
A \textbf{symbol}  \textbf{for} an item in a calculation or scientific  formula is a number, letter, or shape that represents that item.
 \textit{
	\begin{itemize}
	\item What's the chemical symbol for mercury?
	\end{itemize}
}
\end{enumerate}

\section*{whenever}
{\large \color{blue}  }
\subsection*{Explain}
\begin{enumerate}
\item conjunction \\
You use \textbf{whenever} to refer to any time or every time that something happens or is true.
 \textit{
	\begin{itemize}
	\item She always called at the house whenever she was in the area.
	\item You can have my cottage whenever you like.
	\item I recommend that you avoid processed foods whenever possible.
	\end{itemize}
}
\item conjunction \\
You use \textbf{whenever} to refer to a time that you do not know or are not sure about.
 \textit{
	\begin{itemize}
	\item He married Miss Vancouver in 1963, or whenever it was.
	\end{itemize}
}
\end{enumerate}

\section*{territory}
{\large \color{blue}  territories  }
\subsection*{Explain}
\begin{enumerate}
\item variable noun \\
\textbf{Territory} is land which is controlled by a particular country or ruler .
 \textit{
	\begin{itemize}
	\item The government denies that any of its territory is under rebel control.
	\item ...Russian territory.
	\end{itemize}
}
\item countable noun \\
A \textbf{territory} is a country or region that is controlled by another country.
 \textit{
	\begin{itemize}
	\item They just want to return to their families in the occupied territories.
	\item He toured some of the disputed territories now under U.N. control.
	\end{itemize}
}
\item uncountable noun \\
You can use \textbf{territory} to refer to an area of knowledge or experience .
 \textit{
	\begin{itemize}
	\item Following the futuristic The Handmaid's Tale, her next novel returned to more familiar
territory.
	\item This is difficult territory because it involves uncomfortable issues.
	\end{itemize}
}
\item variable noun \\
An animal's \textbf{territory} is an area which it regards as its own and which it defends when other animals try to enter it.
 \textit{
	\begin{itemize}
	\item The territory of a cat only remains fixed for as long as the cat dominates the area.
	\end{itemize}
}
\item uncountable noun \\
\textbf{Territory} is land with a particular character.
 \textit{
	\begin{itemize}
	\item ...mountainous territory.
	\item ...a vast and uninhabited territory.
	\end{itemize}
}
\item  \\
 come with the territory \textit{
	\begin{itemize}
	\end{itemize}
}
\end{enumerate}

\section*{whereas}
{\large \color{blue}  }
\subsection*{Explain}
\begin{enumerate}
\item conjunction \\
You use \textbf{whereas} to introduce a comment which contrasts with what is said in the main clause .
 \textit{
	\begin{itemize}
	\item Pensions are linked to inflation, whereas they should be linked to the cost of living.
	\item Whereas the population of working age increased by 1 million between 1981 and 1986,
today it is barely growing.
	\end{itemize}
}
\end{enumerate}

\section*{theft}
{\large \color{blue}  thefts  }
\subsection*{Explain}
\begin{enumerate}
\item variable noun \\
\textbf{Theft} is the crime of stealing.
 \textit{
	\begin{itemize}
	\item Art theft is now part of organised crime.
	\item ...the theft of classified documents from a car in London.
	\end{itemize}
}
\end{enumerate}

\section*{wherever}
{\large \color{blue}  }
\subsection*{Explain}
\begin{enumerate}
\item conjunction \\
You use \textbf{wherever} to indicate that something happens or is true in any place or situation .
 \textit{
	\begin{itemize}
	\item Some people enjoy themselves wherever they are.
	\item Jack believed in finding happiness wherever possible.
	\item By simply planning a route, you can explore at will and stop whenever and wherever
you like.
	\end{itemize}
}
\item conjunction \\
You use \textbf{wherever} when you indicate that you do not know where a person or place is.
 \textit{
	\begin{itemize}
	\item I'd like to leave as soon as possible and join my children, wherever they are.
	\item 'Till we meet again, wherever that is,' said the chairman.
	\end{itemize}
}
\item adverb \\
You use \textbf{wherever} in questions as an emphatic form of 'where', usually when you are surprised about something.
 \textit{
	\begin{itemize}
	\item Wherever did you get that idea?
	\item Wherever have you been?
	\end{itemize}
}
\item  \\
 or wherever \textit{
	\begin{itemize}
	\end{itemize}
}
\end{enumerate}

\section*{tie}
{\large \color{blue}  ties  tying  tied  }
\subsection*{Explain}
\begin{enumerate}
\item verb \\
If you \textbf{tie} two things \textbf{together} or \textbf{tie} them, you fasten them together with a knot.
 \textit{
	\begin{itemize}
	\item He tied the ends of the plastic bag together.
	\item Mr Saunders tied her hands and feet.
	\end{itemize}
}
\item verb \\
If you \textbf{tie} something or someone in a particular place or position, you put them there and fasten
them using rope or string.
 \textit{
	\begin{itemize}
	\item He had tied the dog to one of the trees near the canal.
	\item He tied her hands behind her back.
	\end{itemize}
}
\item verb \\
If you \textbf{tie} a piece of string or cloth around something or \textbf{tie} something \textbf{with} a piece of string or cloth, you put the piece of string or cloth around it and fasten
the ends together.
 \textit{
	\begin{itemize}
	\item She tied her scarf over her head.
	\item Roll the meat and tie it with string.
	\item Dad handed me a big box wrapped in gold foil and tied with a red ribbon.
	\end{itemize}
}
\item verb \\
If you \textbf{tie} a knot or bow \textbf{in} something or \textbf{tie} something \textbf{in} a knot or bow, you fasten the ends together.
 \textit{
	\begin{itemize}
	\item He took a short length of rope and swiftly tied a slip knot.
	\item She tied a knot in her scarf.
	\item She grabbed her hair in both hands and swept it back, tying it in a loose knot.
	\item She wore a checked shirt tied in a knot above the navel.
	\end{itemize}
}
\item verb \\
When you \textbf{tie} something or when something \textbf{ties} , you close or fasten it using a bow or knot.
 \textit{
	\begin{itemize}
	\item He pulled on his heavy suede shoes and tied the laces.
	\item ...a long white thing around his neck that tied in front in a floppy bow.
	\end{itemize}
}
\item countable noun \\
A \textbf{tie} is a long narrow piece of cloth that is worn round the neck under a shirt collar and tied in a knot at the front. Ties are worn mainly by men.
 \textit{
	\begin{itemize}
	\item Jason had taken off his jacket and loosened his tie.
	\end{itemize}
}
\item verb \\
If one thing \textbf{is tied}  \textbf{to} another or two things \textbf{are tied} , the two things have a close connection or link.
 \textit{
	\begin{itemize}
	\item Their cancers are not so clearly tied to radiation exposure.
	\item My social life and business life are closely tied.
	\end{itemize}
}
\item verb \\
If you \textbf{are tied}  \textbf{to} a particular place or situation, you are forced to accept it and cannot change it.
 \textit{
	\begin{itemize}
	\item They had children and were consequently tied to the school holidays.
	\item I wouldn't like to be tied to catching the last train home.
	\end{itemize}
}
\item countable noun \\
\textbf{Ties} are the connections you have with people or a place.
 \textit{
	\begin{itemize}
	\item Quebec has always had particularly close ties to France.
	\item I can't find any tie between her and the town.
	\item Louise herself had family ties in Nimes.
	\end{itemize}
}
\item countable noun \\
Railroad  \textbf{ties} are large heavy beams that support the rails of a railway  track .
 \textit{
	\begin{itemize}
	\end{itemize}
}
\item verb \\
If two people \textbf{tie} in a competition or game or if they \textbf{tie}  \textbf{with} each other, they have the same number of points or the same degree of success .
 \textbf{Tie} is also a noun .
 \textit{
	\begin{itemize}
	\item Both teams had tied on points and goal difference.
	\item We tied with Spain in fifth place.
	\item The first game ended in a tie.
	\end{itemize}
}
\item countable noun \\
In sport, a \textbf{tie} is a match that is part of a competition. The losers leave the competition and the winners go on to the next round.
 \textit{
	\begin{itemize}
	\item They'll meet the winners of the first round tie.
	\end{itemize}
}
\end{enumerate}

\section*{whether}
{\large \color{blue}  }
\subsection*{Explain}
\begin{enumerate}
\item conjunction \\
You use \textbf{whether} when you are talking about a choice or doubt between two or more alternatives.
 \textit{
	\begin{itemize}
	\item To this day, it's unclear whether he shot himself or was murdered.
	\item Whether it turns out to be a good idea or a bad idea, we'll find out.
	\item They now have two weeks to decide whether or not to buy.
	\item The council is considering whether to approve of the use of firearms.
	\item I don't know whether they've found anybody yet.
	\end{itemize}
}
\item conjunction \\
You use \textbf{whether} to say that something is true in any of the circumstances that you mention .
 \textit{
	\begin{itemize}
	\item This happens whether the children are in two-parent or one-parent families.
	\item The more muscle you have, the more fat you'll burn, whether you're working out or
fast asleep.
	\item Babies, whether breast-fed or bottle-fed, should receive additional vitamin D.
	\end{itemize}
}
\end{enumerate}

\section*{toe}
{\large \color{blue}  toes  toeing  toed  }
\subsection*{Explain}
\begin{enumerate}
\item countable noun \\
Your \textbf{toes} are the five  movable parts at the end of each foot.
 \textit{
	\begin{itemize}
	\end{itemize}
}
\item countable noun \\
The \textbf{toe} of a shoe or sock is the part that covers the end of your foot.
 \textit{
	\begin{itemize}
	\end{itemize}
}
\item  \\
 to dip your toes \textit{
	\begin{itemize}
	\end{itemize}
}
\item  \\
 keep someone on their toes \textit{
	\begin{itemize}
	\end{itemize}
}
\item  \\
 toe the line \textit{
	\begin{itemize}
	\end{itemize}
}
\item  \\
 to tread on someone's toes \textit{
	\begin{itemize}
	\end{itemize}
}
\end{enumerate}

\section*{while}
{\large \color{blue}  }
\subsection*{Explain}
\begin{enumerate}
\item conjunction \\
If something happens  \textbf{while} something else is happening , the two things are happening at the same time.
 \textit{
	\begin{itemize}
	\item They were grinning and watching while one man laughed and poured his drink over the
head of another.
	\item I sat on the settee to unwrap the package while he stood by.
	\item Racing was halted for an hour while the track was repaired.
	\item Her parents could help with child care while she works.
	\end{itemize}
}
\item conjunction \\
If something happens \textbf{while} something else happens, the first thing happens at some point during the time that
the second thing is happening.
 \textit{
	\begin{itemize}
	\item The two ministers have yet to meet, but may do so while in New York.
	\item Never apply water to a burn while the casualty is still in contact with electric
current.
	\end{itemize}
}
\item conjunction \\
You use \textbf{while} at the beginning of a clause to introduce information which contrasts with information in the main clause.
 \textit{
	\begin{itemize}
	\item Marianne was tempted to turn the large rooms into traditional French-style salons,
while Howard was in favour of a typically English look.
	\item The first two services are free, while the third costs £35.00.
	\end{itemize}
}
\item conjunction \\
You use \textbf{while} , before making a statement , in order to introduce information that partly  conflicts with your statement.
 \textit{
	\begin{itemize}
	\item While the numbers of such developments are relatively small, the potential market
is large.
	\item While the modelling business is not easy to get into, the good model will always
be in demand.
	\item While the news, so far, has been good, there may be days ahead when it is bad.
	\end{itemize}
}
\end{enumerate}

\section*{tray}
{\large \color{blue}  trays  }
\subsection*{Explain}
\begin{enumerate}
\item countable noun \\
A \textbf{tray} is a flat piece of wood , plastic, or metal, which usually has raised edges and which is used for carrying
things, especially  food and drinks .
 \textit{
	\begin{itemize}
	\end{itemize}
}
\end{enumerate}

\section*{yesterday}
{\large \color{blue}  yesterdays  }
\subsection*{Explain}
\begin{enumerate}
\item adverb \\
You use \textbf{yesterday} to refer to the day before today.
 \textbf{Yesterday} is also a noun .
 \textit{
	\begin{itemize}
	\item She left yesterday.
	\item Yesterday she announced that she is quitting her job.
	\item In yesterday's games, we beat our opponents two-one.
	\end{itemize}
}
\item uncountable noun \\
You can refer to the past, especially the recent past, as \textbf{yesterday} .
 \textit{
	\begin{itemize}
	\item The worker of today is different from the worker of yesterday.
	\item ...a world without yesterdays or tomorrows.
	\end{itemize}
}
\end{enumerate}

\section*{violence}
{\large \color{blue}  }
\subsection*{Explain}
\begin{enumerate}
\item uncountable noun \\
\textbf{Violence} is behaviour which is intended to hurt , injure , or kill people.
 \textit{
	\begin{itemize}
	\item Twenty people were killed in the violence.
	\item They threaten them with violence.
	\item ...domestic violence between family members.
	\end{itemize}
}
\item uncountable noun \\
If you do or say something with \textbf{violence} , you use a lot of force and energy in doing or saying it, often because you are angry .
 \textit{
	\begin{itemize}
	\item 'There's no need,' Amy said, with sudden violence.
	\item The violence in her tone gave Alistair a shock.
	\end{itemize}
}
\end{enumerate}

\section*{behavior}
{\large \color{blue}  }
\subsection*{Explain}
\begin{enumerate}
\item noun \\
1.  2.  3.  \textit{
	\begin{itemize}
	\end{itemize}
}
\end{enumerate}

\section*{alarm}
{\large \color{blue}  alarms  alarming  alarmed  }
\subsection*{Explain}
\begin{enumerate}
\item uncountable noun \\
\textbf{Alarm} is a feeling of fear or anxiety that something unpleasant or dangerous  might  happen .
 \textit{
	\begin{itemize}
	\item The news was greeted with alarm by MPs.
	\item She sat up in alarm.
	\item The moves reflect growing alarm over recent events.
	\end{itemize}
}
\item verb \\
If something \textbf{alarms} you, it makes you afraid or anxious that something unpleasant or dangerous might happen.
 \textit{
	\begin{itemize}
	\item We could not see what had alarmed him.
	\end{itemize}
}
\item countable noun \\
An \textbf{alarm} is an automatic device that warns you of danger, for example by ringing a bell.
 \textit{
	\begin{itemize}
	\item He heard the alarm go off.
	\item ...an extremely sophisticated alarm system.
	\item The other man rang the alarm bell.
	\end{itemize}
}
\item countable noun \\
An \textbf{alarm} is the same as an alarm clock .
 \textit{
	\begin{itemize}
	\end{itemize}
}
\item  \\
 alarm bells \textit{
	\begin{itemize}
	\end{itemize}
}
\item  \\
 to raise the alarm \textit{
	\begin{itemize}
	\end{itemize}
}
\end{enumerate}

\section*{bicycle}
{\large \color{blue}  bicycles  bicycling  bicycled  }
\subsection*{Explain}
\begin{enumerate}
\item countable noun \\
A \textbf{bicycle} is a vehicle with two wheels which you ride by sitting on it and pushing two pedals with your feet . You steer it by turning a bar that is connected to the front wheel.
 \textit{
	\begin{itemize}
	\end{itemize}
}
\item verb \\
If you \textbf{bicycle}  somewhere , you cycle there.
 \textit{
	\begin{itemize}
	\item I bicycled on towards the sea.
	\end{itemize}
}
\end{enumerate}

\section*{beast}
{\large \color{blue}  beasts  }
\subsection*{Explain}
\begin{enumerate}
\item countable noun \\
You can refer to an animal as a \textbf{beast} , especially if it is a large, dangerous , or unusual one.
 \textit{
	\begin{itemize}
	\item ...the threats our ancestors faced from wild beasts.
	\item ...a centaur: half man, half beast.
	\end{itemize}
}
\item countable noun \\
If you refer to a man as a \textbf{beast} , you mean that his behaviour , especially his sexual behaviour, is very violent and uncontrolled .
 \textit{
	\begin{itemize}
	\end{itemize}
}
\item countable noun \\
If you call someone a \textbf{beast} , you think that they are behaving in a selfish , unkind , or unpleasant way.
 \textit{
	\begin{itemize}
	\item Bully! Hooligan! Beast! Let me go, let go!
	\end{itemize}
}
\item countable noun \\
You can use \textbf{beast} to refer to something or someone in a light-hearted way, and to mention that they have a particular quality.
 \textit{
	\begin{itemize}
	\item ...that rare beast, a sports movie that isn't boring.
	\end{itemize}
}
\end{enumerate}

\section*{bypass}
{\large \color{blue}  bypasses  bypassing  bypassed  }
\subsection*{Explain}
\begin{enumerate}
\item verb \\
If you \textbf{bypass} someone or something that you would normally have to get involved with, you ignore them, often because you want to achieve something more quickly.
 \textit{
	\begin{itemize}
	\item A growing number of employers are trying to bypass the unions altogether.
	\item Regulators worry that controls could easily be bypassed.
	\end{itemize}
}
\item countable noun \\
A \textbf{bypass} is a surgical  operation performed on or near the heart, in which the flow of blood is redirected so that
it does not flow through a part of the heart which is diseased or blocked .
 \textit{
	\begin{itemize}
	\item ...heart bypass surgery.
	\end{itemize}
}
\item verb \\
If a surgeon  \textbf{bypasses} a diseased artery or other part of the body, he or she performs an operation so that blood or other
 bodily fluids do not flow through it.
 \textit{
	\begin{itemize}
	\item Small veins are removed from the leg and used to bypass the blocked stretch of coronary
arteries.
	\end{itemize}
}
\item countable noun \\
A \textbf{bypass} is a main road which takes traffic around the edge of a town  rather than through its centre .
 \textit{
	\begin{itemize}
	\item A new bypass around the city is being built.
	\item ...the Hereford bypass.
	\end{itemize}
}
\item verb \\
If a road \textbf{bypasses} a place, it goes around it rather than through it.
 \textit{
	\begin{itemize}
	\item ...money for new roads to bypass cities.
	\end{itemize}
}
\item verb \\
If you \textbf{bypass} a place when you are travelling , you avoid going through it.
 \textit{
	\begin{itemize}
	\item The rebel forces simply bypassed Zwedru on their way further south.
	\end{itemize}
}
\end{enumerate}

\section*{blend}
{\large \color{blue}  blends  blending  blended  }
\subsection*{Explain}
\begin{enumerate}
\item verb \\
If you \textbf{blend} substances together or if they \textbf{blend} , you mix them together so that they become one substance.
 \textit{
	\begin{itemize}
	\item Blend the butter with the sugar and beat until light and creamy.
	\item Blend the ingredients until you have a smooth cream.
	\item Put the soap and water in a pan and leave to stand until they have blended.
	\item Most whiskies are blended whiskies.
	\end{itemize}
}
\item countable noun \\
A \textbf{blend}  \textbf{of} things is a mixture or combination of them that is useful or pleasant .
 \textit{
	\begin{itemize}
	\item The public areas offer a subtle blend of traditional charm with modern amenities.
	\item ...a blend of wine and sparkling water.
	\item He makes up his own blends of flour.
	\end{itemize}
}
\item verb \\
When colours, sounds, or styles  \textbf{blend} , they come together or are combined in a pleasing way.
 \textit{
	\begin{itemize}
	\item You could paint the walls and ceilings the same colour so they blend together.
	\item ...the picture, furniture and porcelain collections that blend so well with the house
itself.
	\end{itemize}
}
\item verb \\
If you \textbf{blend}  ideas , policies , or styles, you use them together in order to achieve something.
 \textit{
	\begin{itemize}
	\item The Glasgow-based cartoonist is a master at blending humour with the macabre.
	\item ...a band that blended jazz, folk and classical music.
	\end{itemize}
}
\end{enumerate}

\section*{catalog}
{\large \color{blue}  }
\subsection*{Explain}
\begin{enumerate}
\item noun \\
1.  \textit{
	\begin{itemize}
	\end{itemize}
}
\item verb transitive \\
2.  3.  \textit{
	\begin{itemize}
	\end{itemize}
}
\end{enumerate}

\section*{brand}
{\large \color{blue}  brands  branding  branded  }
\subsection*{Explain}
\begin{enumerate}
\item countable noun \\
A \textbf{brand} of a product is the version of it that is made by one particular manufacturer .
 \textit{
	\begin{itemize}
	\item This is my favourite brand of shampoo.
	\item I bought one of the leading brands.
	\item ...a supermarket's own brand.
	\end{itemize}
}
\item countable noun \\
A \textbf{brand of} something such as a way of thinking or behaving is a particular kind of it.
 \textit{
	\begin{itemize}
	\item The British brand of socialism was more interested in reform than revolution.
	\end{itemize}
}
\item verb \\
If someone \textbf{is branded} as something bad , people think they are that thing.
 \textit{
	\begin{itemize}
	\item I was instantly branded as a rebel.
	\item The company has been branded racist by some of its own staff.
	\item The U.S. administration recently branded him a war criminal.
	\end{itemize}
}
\item verb \\
When you \textbf{brand} an animal, you put a permanent mark on its skin in order to show who it belongs to, usually by burning a mark onto its skin.
 \textbf{Brand} is also a noun .
 \textit{
	\begin{itemize}
	\item The owner couldn't be bothered to brand the cattle.
	\item A brand was a mark of ownership burned into the hide of an animal with a hot iron.
	\end{itemize}
}
\end{enumerate}

\section*{centimetre}
{\large \color{blue}  centimetres  }
\subsection*{Explain}
\begin{enumerate}
\item countable noun \\
A \textbf{centimetre} is a unit of length in the metric system equal to ten  millimetres or one-hundredth of a metre.
 \textit{
	\begin{itemize}
	\item ...a tiny fossil plant, only a few centimetres high.
	\end{itemize}
}
\end{enumerate}

\section*{business}
{\large \color{blue}  businesses  }
\subsection*{Explain}
\begin{enumerate}
\item uncountable noun \\
\textbf{Business} is work relating to the production, buying , and selling of goods or services.
 \textit{
	\begin{itemize}
	\item ...young people seeking a career in business.
	\item Jennifer has an impressive academic and business background.
	\item ...Harvard Business School.
	\end{itemize}
}
\item uncountable noun \\
\textbf{Business} is used when talking about how many products or services a company is able to sell. If \textbf{business} is good, a lot of products or services are being sold and if \textbf{business} is bad , few of them are being sold.
 \textit{
	\begin{itemize}
	\item They worried that German companies would lose business.
	\item Business is booming.
	\end{itemize}
}
\item countable noun \\
A \textbf{business} is an organization which produces and sells goods or which provides a service.
 \textit{
	\begin{itemize}
	\item The company was a family business.
	\item The majority of small businesses go broke within the first twenty-four months.
	\item He was short of cash after the collapse of his business.
	\end{itemize}
}
\item uncountable noun \\
\textbf{Business} is work or some other activity that you do as part of your job and not for pleasure .
 \textit{
	\begin{itemize}
	\item I'm here on business.
	\item You can't mix business with pleasure.
	\item ...business trips.
	\end{itemize}
}
\item singular noun \\
You can use \textbf{business} to refer to a particular area of work or activity in which the aim is to make a profit .
 \textit{
	\begin{itemize}
	\item May I ask you what business you're in?
	\item ...the music business.
	\end{itemize}
}
\item singular noun \\
You can use \textbf{business} to refer to something that you are doing or concerning yourself with.
 \textit{
	\begin{itemize}
	\item ...recording Ben as he goes about his business.
	\item There was nothing left for the teams to do but get on with the business of racing.
	\end{itemize}
}
\item uncountable noun \\
You can use \textbf{business} to refer to important matters that you have to deal with.
 \textit{
	\begin{itemize}
	\item The most important business was left to the last.
	\item I've got some unfinished business to attend to.
	\end{itemize}
}
\item uncountable noun \\
If you say that something is your \textbf{business} , you mean that it concerns you personally and that other people have no right to ask  questions about it or disagree with it.
 \textit{
	\begin{itemize}
	\item My sex life is my business.
	\item If she doesn't want the police involved, that's her business.
	\item It's not our business.
	\end{itemize}
}
\item singular noun \\
You can use \textbf{business} to refer in a general way to an event, situation, or activity. For example, you can
say something is 'a wretched business' or you can refer to 'this assassination business'.
 \textit{
	\begin{itemize}
	\item We have sorted out this wretched business at last.
	\item This whole business is very puzzling.
	\end{itemize}
}
\item singular noun \\
You can use \textbf{business} when describing a task that is unpleasant in some way. For example, if you say that doing something is a costly  \textbf{business} , you mean that it costs a lot.
 \textit{
	\begin{itemize}
	\item Coastal defence is a costly business.
	\item Parenting can be a stressful business.
	\end{itemize}
}
\item  \\
 do business \textit{
	\begin{itemize}
	\end{itemize}
}
\item  \\
 have no business \textit{
	\begin{itemize}
	\end{itemize}
}
\item  \\
 be in business \textit{
	\begin{itemize}
	\end{itemize}
}
\item  \\
 be in business \textit{
	\begin{itemize}
	\end{itemize}
}
\item  \\
 to mean business \textit{
	\begin{itemize}
	\end{itemize}
}
\item  \\
 to mind your own business \textit{
	\begin{itemize}
	\end{itemize}
}
\item  \\
 make it one's business to do sth \textit{
	\begin{itemize}
	\end{itemize}
}
\item  \\
 not be in the business of doing sth \textit{
	\begin{itemize}
	\end{itemize}
}
\item  \\
 out of business \textit{
	\begin{itemize}
	\end{itemize}
}
\item  \\
 the business \textit{
	\begin{itemize}
	\end{itemize}
}
\item  \\
 business as usual \textit{
	\begin{itemize}
	\end{itemize}
}
\end{enumerate}

\section*{cheque}
{\large \color{blue}  cheques  }
\subsection*{Explain}
\begin{enumerate}
\item countable noun \\
A \textbf{cheque} is a printed form on which you write an amount of money and who it is to be paid to. Your bank
then pays the money to that person from your account.
 \textit{
	\begin{itemize}
	\item He wrote them a cheque for £10,000.
	\item I'd like to pay by cheque.
	\end{itemize}
}
\end{enumerate}

\section*{chaos}
{\large \color{blue}  }
\subsection*{Explain}
\begin{enumerate}
\item uncountable noun \\
\textbf{Chaos} is a state of complete disorder and confusion.
 \textit{
	\begin{itemize}
	\item The world's first transatlantic balloon race ended in chaos last night.
	\item It is impossible to establish democracy amid economic chaos.
	\end{itemize}
}
\end{enumerate}

\section*{clarity}
{\large \color{blue}  }
\subsection*{Explain}
\begin{enumerate}
\item uncountable noun \\
The \textbf{clarity} of something such as a book or argument is its quality of being well  explained and easy to understand .
 \textit{
	\begin{itemize}
	\item ...the clarity with which the author explains technical subjects.
	\end{itemize}
}
\item uncountable noun \\
\textbf{Clarity} is the ability to think  clearly .
 \textit{
	\begin{itemize}
	\item In business circles he is noted for his flair and clarity of vision.
	\end{itemize}
}
\item uncountable noun \\
\textbf{Clarity} is the quality of being clear in outline or sound.
 \textit{
	\begin{itemize}
	\item This remarkable technology provides far greater clarity than conventional x-rays.
	\end{itemize}
}
\item uncountable noun \\
The \textbf{clarity} of a liquid , of glass , or of the air is the degree to which it is clear.
 \textit{
	\begin{itemize}
	\item The first thing to strike me was the amazing clarity of the water.
	\end{itemize}
}
\end{enumerate}

\section*{confidence}
{\large \color{blue}  confidences  }
\subsection*{Explain}
\begin{enumerate}
\item uncountable noun \\
If you have \textbf{confidence}  \textbf{in} someone, you feel that you can trust them.
 \textit{
	\begin{itemize}
	\item I have every confidence in you.
	\item This has contributed to the lack of confidence in the police.
	\item His record on ceasefires inspires no confidence.
	\end{itemize}
}
\item uncountable noun \\
If you have \textbf{confidence} , you feel sure about your abilities, qualities, or ideas .
 \textit{
	\begin{itemize}
	\item The band is on excellent form and brimming with confidence.
	\item I always thought the worst of myself and had no confidence whatsoever.
	\end{itemize}
}
\item uncountable noun \\
If you can say something \textbf{with}  \textbf{confidence} , you feel certain it is correct .
 \textit{
	\begin{itemize}
	\item I can say with confidence that such rumours were totally groundless.
	\end{itemize}
}
\item uncountable noun \\
If you tell someone something \textbf{in}  \textbf{confidence} , you tell them a secret.
 \textit{
	\begin{itemize}
	\item We told you all these things in confidence.
	\item Even telling Lois seemed a betrayal of confidence.
	\end{itemize}
}
\item countable noun \\
A \textbf{confidence} is a secret that you tell someone.
 \textit{
	\begin{itemize}
	\item Gregory shared confidences with Carmen.
	\end{itemize}
}
\end{enumerate}

\section*{clay}
{\large \color{blue}  clays  }
\subsection*{Explain}
\begin{enumerate}
\item variable noun \\
\textbf{Clay} is a kind of earth that is soft when it is wet and hard when it is dry . Clay is shaped and baked to make things such as pots and bricks.
 \textit{
	\begin{itemize}
	\item ...the heavy clay soils of Cambridgeshire.
	\item As the wheel turned, the potter shaped and squeezed the lump of clay into a graceful
shape.
	\item ...a little clay pot.
	\end{itemize}
}
\item uncountable noun \\
In tennis , matches played on \textbf{clay} are played on courts whose surface is covered with finely  crushed  stones or brick.
 \textit{
	\begin{itemize}
	\item Most tennis is played on hard courts, but a substantial amount is played on clay.
	\item He was a clay-court specialist who won Wimbledon five times.
	\end{itemize}
}
\item  \\
 feet of clay \textit{
	\begin{itemize}
	\end{itemize}
}
\end{enumerate}

\section*{connection}
{\large \color{blue}  connections  }
\subsection*{Explain}
\begin{enumerate}
\item variable noun \\
A \textbf{connection} is a relationship between two things, people, or groups.
 \textit{
	\begin{itemize}
	\item There was no evidence of a connection between BSE and the brain diseases recently
confirmed in cats.
	\item The police say he had no connection with the security forces.
	\item He has denied any connection to the bombing.
	\end{itemize}
}
\item countable noun \\
A \textbf{connection} is a joint where two wires or pipes are joined together.
 \textit{
	\begin{itemize}
	\item Check all radiators for small leaks, especially round pipework connections.
	\end{itemize}
}
\item countable noun \\
If a place has good road, rail , or air \textbf{connections} , many places can be directly  reached from there by car, train, or plane .
 \textit{
	\begin{itemize}
	\item Fukuoka has excellent air and rail connections to the rest of the country.
	\end{itemize}
}
\item countable noun \\
If you get a \textbf{connection} at a station or airport , you catch a train, bus, or plane, after getting off another train, bus, or plane, in order
to continue your journey .
 \textit{
	\begin{itemize}
	\item My flight was late and I missed the connection.
	\end{itemize}
}
\item plural noun \\
Your \textbf{connections} are the people who you know or are related to, especially when they are in a position to help you.
 \textit{
	\begin{itemize}
	\item She used her connections to full advantage.
	\end{itemize}
}
\item  \\
 in connection with \textit{
	\begin{itemize}
	\end{itemize}
}
\item  \\
 in this connection/in that connection \textit{
	\begin{itemize}
	\end{itemize}
}
\end{enumerate}

\section*{commerce}
{\large \color{blue}  }
\subsection*{Explain}
\begin{enumerate}
\item uncountable noun \\
\textbf{Commerce} is the activities and procedures involved in buying and selling things.
 \textit{
	\begin{itemize}
	\item They have made their fortunes from industry and commerce.
	\end{itemize}
}
\end{enumerate}

\section*{data}
{\large \color{blue}  }
\subsection*{Explain}
\begin{enumerate}
\item uncountable noun \\
You can refer to information as \textbf{data} , especially when it is in the form of facts or statistics that you can analyse . In American English, \textbf{data} is usually a plural  noun . In technical or formal British English, \textbf{data} is sometimes a plural noun, but at other times, it is an uncount noun.
 \textit{
	\begin{itemize}
	\item The study was based on data from 2,100 women.
	\item ...the latest year for which data is available.
	\item To cope with these data, hospitals bought large mainframe computers.
	\end{itemize}
}
\item uncountable noun \\
\textbf{Data} is information that can be stored and used by a computer program.
 \textit{
	\begin{itemize}
	\item No important data is stored on the devices.
	\end{itemize}
}
\end{enumerate}

\section*{compound}
{\large \color{blue}  compounds  compounding  compounded  }
\subsection*{Explain}
\begin{enumerate}
\item countable noun \\
A \textbf{compound} is an enclosed area of land that is used for a particular purpose.
 \textit{
	\begin{itemize}
	\item Police fired on them as they fled into the embassy compound.
	\item ...a military compound.
	\end{itemize}
}
\item countable noun \\
In chemistry , a \textbf{compound} is a substance that consists of two or more elements.
 \textit{
	\begin{itemize}
	\item Organic compounds contain carbon in their molecules.
	\end{itemize}
}
\item countable noun \\
If something is a \textbf{compound}  \textbf{of} different things, it consists of those things.
 \textit{
	\begin{itemize}
	\item Salt in its essential form is a compound of sodium and chlorine.
	\end{itemize}
}
\item adjective \\
\textbf{Compound} is used to indicate that something consists of two or more parts or things.
 \textit{
	\begin{itemize}
	\item ...a tall shrub with shiny compound leaves.
	\item ...the compound microscope.
	\end{itemize}
}
\item adjective \\
In grammar , a \textbf{compound}  noun , adjective , or verb is one that is made up of two or more words, for example ' fire engine', 'bottle-green', and 'force-feed'.
 \textit{
	\begin{itemize}
	\end{itemize}
}
\item adjective \\
In grammar, a \textbf{compound} sentence is one that is made up of two or more main clauses . Compare  complex , , simple .
 \textit{
	\begin{itemize}
	\end{itemize}
}
\item verb \\
To \textbf{compound} a problem , difficulty , or mistake means to make it worse by adding to it.
 \textit{
	\begin{itemize}
	\item Additional bloodshed and loss of life will only compound the tragedy.
	\item The problem is compounded by the medical system here.
	\end{itemize}
}
\end{enumerate}

\section*{database}
{\large \color{blue}  databases  }
\subsection*{Explain}
\begin{enumerate}
\item countable noun \\
A \textbf{database} is a collection of data that is stored in a computer and that can easily be used and added to.
 \textit{
	\begin{itemize}
	\item They maintain a database of hotels that cater for those travelling with pets.
	\end{itemize}
}
\end{enumerate}

\section*{conviction}
{\large \color{blue}  convictions  }
\subsection*{Explain}
\begin{enumerate}
\item countable noun \\
A \textbf{conviction} is a strong belief or opinion.
 \textit{
	\begin{itemize}
	\item It is our firm conviction that a step forward has been taken.
	\item Their religious convictions prevented them from taking up arms.
	\end{itemize}
}
\item uncountable noun \\
If you have \textbf{conviction} , you have great  confidence in your beliefs or opinions.
 \textit{
	\begin{itemize}
	\item 'We shall, sir,' said Thorne, with conviction.
	\end{itemize}
}
\item  \\
 to carry conviction \textit{
	\begin{itemize}
	\end{itemize}
}
\item countable noun \\
If someone has a \textbf{conviction} , they have been found  guilty of a crime in a court of law .
 \textit{
	\begin{itemize}
	\item He will appeal against his conviction.
	\item The man was known to the police because of previous convictions.
	\end{itemize}
}
\end{enumerate}

\section*{defence}
{\large \color{blue}  defences  }
\subsection*{Explain}
\begin{enumerate}
\item uncountable noun \\
\textbf{Defence} is action that is taken to protect someone or something against attack.
 \textit{
	\begin{itemize}
	\item The land was flat, giving no scope for defence.
	\item By wielding a knife in defence you run the risk of having it used against you.
	\end{itemize}
}
\item uncountable noun \\
\textbf{Defence} is the organization of a country's armies and weapons , and their use to protect the country or its interests .
 \textit{
	\begin{itemize}
	\item Twenty eight percent of the federal budget is spent on defense.
	\item ...the French defence minister.
	\item ...a five per cent cut in defence spending.
	\end{itemize}
}
\item plural noun \\
The \textbf{defences} of a country or region are all its armed forces and weapons.
 \textit{
	\begin{itemize}
	\item ...the need to maintain Britain's defences at a sufficiently high level.
	\end{itemize}
}
\item countable noun \\
A \textbf{defence} is something that people or animals can use or do to protect themselves.
 \textit{
	\begin{itemize}
	\item The immune system is our main defence against disease.
	\item The boy could have adopted hardened cynicism as a defense.
	\end{itemize}
}
\item countable noun \\
A \textbf{defence} is something that you say or write which supports ideas or actions that have been criticized or questioned .
 \textit{
	\begin{itemize}
	\item Chomsky's defence of his approach goes further.
	\item The Party Congress has closed with a spirited defence of the government's economic
programme from the Deputy Prime Minister.
	\item I must say in his defence that he is concerned about people.
	\end{itemize}
}
\item countable noun \\
In a court of law, an accused person's \textbf{defence} is the process of presenting  evidence in their favour .
 \textit{
	\begin{itemize}
	\item He has insisted on conducting his own defence.
	\end{itemize}
}
\item singular noun \\
\textbf{The}  \textbf{defence} is the case that is presented by a lawyer in a trial for the person who has been accused of a crime . You can also  refer to this person's lawyers as \textbf{the}  \textbf{defence} .
 \textit{
	\begin{itemize}
	\item The defence was that the records of the interviews were fabricated by the police.
	\item The defence pleaded insanity, but the defendant was found guilty and sentenced.
	\item ...defence lawyers.
	\end{itemize}
}
\item singular noun \\
In games such as football or hockey , the \textbf{defence} is the group of players in a team who try to stop the opposing players scoring a goal or a point.
 \textit{
	\begin{itemize}
	\item Their defence, so strong last season, has now conceded 12 goals in six games.
	\item I still prefer to play in defence.
	\end{itemize}
}
\item  \\
 to sb's defence \textit{
	\begin{itemize}
	\end{itemize}
}
\end{enumerate}

\section*{dealer}
{\large \color{blue}  dealers  }
\subsection*{Explain}
\begin{enumerate}
\item countable noun \\
A \textbf{dealer} is a person whose business involves buying and selling things.
 \textit{
	\begin{itemize}
	\item ...an antique dealer.
	\item ...dealers in commodities and financial securities.
	\end{itemize}
}
\item countable noun \\
A \textbf{dealer} is someone who buys and sells illegal drugs.
 \textit{
	\begin{itemize}
	\item They aim to clear every dealer from the street.
	\end{itemize}
}
\end{enumerate}

\section*{dialog}
{\large \color{blue}  }
\subsection*{Explain}
\begin{enumerate}
\item noun \\
 alt. sp. of \textit{
	\begin{itemize}
	\end{itemize}
}
\end{enumerate}

\section*{deer}
{\large \color{blue}  deer  }
\subsection*{Explain}
\begin{enumerate}
\item countable noun \\
A \textbf{deer} is a large wild animal that eats  grass and leaves. A male deer usually has large, branching  horns .
 \textit{
	\begin{itemize}
	\end{itemize}
}
\end{enumerate}

\section*{disc}
{\large \color{blue}  discs  }
\subsection*{Explain}
\begin{enumerate}
\item countable noun \\
A \textbf{disc} is a flat, circular shape or object.
 \textit{
	\begin{itemize}
	\item Most shredding machines are based on a revolving disc fitted with replaceable blades.
	\end{itemize}
}
\item countable noun \\
A \textbf{disc} is one of the thin, circular pieces of cartilage which separates the bones in your back.
 \textit{
	\begin{itemize}
	\item I had slipped a disc and was frozen in a spasm of pain.
	\end{itemize}
}
\item countable noun \\
A \textbf{disc} is a record that you play on a record player.
 \textit{
	\begin{itemize}
	\item This disc includes the piano sonata in C minor.
	\end{itemize}
}
\end{enumerate}

\section*{depth}
{\large \color{blue}  depths  }
\subsection*{Explain}
\begin{enumerate}
\item variable noun \\
The \textbf{depth} of something such as a river or hole is the distance downwards from its top surface, or between its upper and lower surfaces.
 \textit{
	\begin{itemize}
	\item The smaller lake ranges from five to fourteen feet in depth.
	\item The depth of the shaft is 520 yards.
	\item Pour the vegetable oil into a frying pan to a depth of about 1cm.
	\item They were detected at depths of more than a kilometre in the sea.
	\end{itemize}
}
\item variable noun \\
The \textbf{depth} of something such as a cupboard or drawer is the distance between its front surface and its back.
 \textit{
	\begin{itemize}
	\end{itemize}
}
\item variable noun \\
If an emotion is very strongly or intensely felt , you can talk about its \textbf{depth} .
 \textit{
	\begin{itemize}
	\item I am well aware of the depth of feeling that exists here.
	\item 'Tough, isn't it?' was all she said, but Amy felt the depth of her unspoken sympathy.
	\end{itemize}
}
\item uncountable noun \\
The \textbf{depth} of a situation is its extent and seriousness.
 \textit{
	\begin{itemize}
	\item The country's leadership had underestimated the depth of the crisis.
	\end{itemize}
}
\item uncountable noun \\
The \textbf{depth} of someone's knowledge is the great amount that they know .
 \textit{
	\begin{itemize}
	\item We felt at home with her and were impressed with the depth of her knowledge.
	\item She wants to acquire a greater depth of understanding of the subject.
	\end{itemize}
}
\item uncountable noun \\
The \textbf{depth} of a colour is its intensity and strength .
 \textit{
	\begin{itemize}
	\item White wines tend to gain depth of colour with age.
	\item The blue base gives the red paint more depth.
	\end{itemize}
}
\item uncountable noun \\
In photography and art, you say that a picture has \textbf{depth} or \textbf{depth of field} when you mean that it appears  three-dimensional rather than flat .
 \textit{
	\begin{itemize}
	\item All the paintings are startlingly dramatic as a result of their depth of field and
colour.
	\end{itemize}
}
\item uncountable noun \\
If you say that someone or something has \textbf{depth} , you mean that they have serious and interesting qualities which are not immediately  obvious and which you have to think about carefully before you can fully  understand them.
 \textit{
	\begin{itemize}
	\item His music lacks depth.
	\item There are hidden depths in all of us.
	\end{itemize}
}
\item plural noun \\
\textbf{The depths} are places that are a long way below the surface of the sea or earth.
 \textit{
	\begin{itemize}
	\item The ship vanished into the depths.
	\end{itemize}
}
\item plural noun \\
If you talk about \textbf{the depths of} an area, you mean the parts of it which are very far from the edge .
 \textit{
	\begin{itemize}
	\item ...the depths of the countryside.
	\item Somewhere in the depths of the pine forest an identical sound reverberated.
	\end{itemize}
}
\item plural noun \\
If you are \textbf{in}  \textbf{the depths of} an unpleasant emotion, you feel that emotion very strongly.
 \textit{
	\begin{itemize}
	\item I was in the depths of despair when the baby was sick.
	\end{itemize}
}
\item plural noun \\
If something happens in \textbf{the depths of} a difficult or unpleasant period of time, it happens in the middle and most severe or intense part of it.
 \textit{
	\begin{itemize}
	\item The country is in the depths of a recession.
	\item ...the depths of winter.
	\end{itemize}
}
\item  \\
 in depth \textit{
	\begin{itemize}
	\end{itemize}
}
\item  \\
 out of one's depth \textit{
	\begin{itemize}
	\end{itemize}
}
\item  \\
 out of one's depth \textit{
	\begin{itemize}
	\end{itemize}
}
\end{enumerate}

\section*{dorm}
{\large \color{blue}  dorms  }
\subsection*{Explain}
\begin{enumerate}
\item countable noun \\
A \textbf{dorm} is the same as a dormitory .
 \textit{
	\begin{itemize}
	\end{itemize}
}
\end{enumerate}

\section*{disorder}
{\large \color{blue}  disorders  }
\subsection*{Explain}
\begin{enumerate}
\item variable noun \\
A \textbf{disorder} is a problem or illness which affects someone's mind or body.
 \textit{
	\begin{itemize}
	\item ...a rare nerve disorder that can cause paralysis of the arms.
	\item ...a psychiatrist who specialises in eating disorders.
	\item He appeared to be suffering from a severe mental disorder and had served a term in
prison.
	\end{itemize}
}
\item uncountable noun \\
\textbf{Disorder} is a state of being untidy , badly  prepared , or badly organized .
 \textit{
	\begin{itemize}
	\item The emergency room was in disorder.
	\item Inside all was disorder: drawers fallen out, shoes and boots scattered.
	\end{itemize}
}
\item variable noun \\
\textbf{Disorder} is violence or rioting in public.
 \textit{
	\begin{itemize}
	\item He called on the authorities to stop public disorder.
	\item There are other forms of civil disorder–most notably, football hooliganism.
	\end{itemize}
}
\end{enumerate}

\section*{mail}
{\large \color{blue}  mails  mailing  mailed  }
\subsection*{Explain}
\begin{enumerate}
\item singular noun \\
\textbf{The mail} is the public service or system by which letters and parcels are collected and delivered.
 \textit{
	\begin{itemize}
	\item Your check is in the mail.
	\item People had to renew their motor vehicle registrations through the mail.
	\item The firm has offices in several large cities, but does most of its business by mail.
	\end{itemize}
}
\item uncountable noun \\
You can refer to letters and parcels that are delivered to you as \textbf{mail} .
 \textit{
	\begin{itemize}
	\item There was no mail except the usual junk addressed to the occupier.
	\item Nora looked through the mail.
	\end{itemize}
}
\item verb \\
If you \textbf{mail} a letter or parcel to someone, you send it to them by putting it in a post box or taking it to a post office.
 \textit{
	\begin{itemize}
	\item Last year, he mailed the documents to French journalists.
	\item He mailed me the contract.
	\item The Government has already mailed some 18 million households with details of the
public offer.
	\end{itemize}
}
\item verb \\
To \textbf{mail} a message to someone means to send it to them by means of email or a computer network .
 \textbf{Mail} is also a noun .
 \textit{
	\begin{itemize}
	\item ...if a report must be electronically mailed to an office by 9 am the next day.
	\item If you have any problems then send me some mail.
	\end{itemize}
}
\end{enumerate}

\section*{emotion}
{\large \color{blue}  emotions  }
\subsection*{Explain}
\begin{enumerate}
\item variable noun \\
An \textbf{emotion} is a feeling such as happiness, love , fear, anger , or hatred , which can be caused by the situation that you are in or the people you are with.
 \textit{
	\begin{itemize}
	\item Happiness was an emotion that Reynolds was having to relearn.
	\item Her voice trembled with emotion.
	\end{itemize}
}
\item uncountable noun \\
\textbf{Emotion} is the part of a person's character that consists of their feelings, as opposed to their thoughts .
 \textit{
	\begin{itemize}
	\item ...the split between reason and emotion.
	\end{itemize}
}
\end{enumerate}

\section*{encyclopedia}
{\large \color{blue}  encyclopedias  }
\subsection*{Explain}
\begin{enumerate}
\item countable noun \\
An \textbf{encyclopedia} is a book or set of books in which facts about many different subjects or about one particular subject are arranged for reference , usually in alphabetical order.
 \textit{
	\begin{itemize}
	\end{itemize}
}
\end{enumerate}

\section*{fisherman}
{\large \color{blue}  fishermen  }
\subsection*{Explain}
\begin{enumerate}
\item countable noun \\
A \textbf{fisherman} is a person who catches fish as a job or for sport.
 \textit{
	\begin{itemize}
	\end{itemize}
}
\end{enumerate}

\section*{exam}
{\large \color{blue}  exams  }
\subsection*{Explain}
\begin{enumerate}
\item countable noun \\
An \textbf{exam} is a formal  test that you take to show your knowledge or ability in a particular subject, or to obtain a qualification .
 \textit{
	\begin{itemize}
	\item I don't want to take any more exams.
	\item Kate's exam results were excellent.
	\end{itemize}
}
\item countable noun \\
If you have a medical  \textbf{exam} , a doctor  looks at your body, feels it, or does simple tests in order to check how healthy you are.
 \textit{
	\begin{itemize}
	\end{itemize}
}
\end{enumerate}

\section*{lid}
{\large \color{blue}  lids  }
\subsection*{Explain}
\begin{enumerate}
\item countable noun \\
A \textbf{lid} is the top of a box or other container which can be removed or raised when you want to open the container.
 \textit{
	\begin{itemize}
	\end{itemize}
}
\item countable noun \\
Your \textbf{lids} are the pieces of skin which cover your eyes when you close them.
 \textit{
	\begin{itemize}
	\item A dull pain began to throb behind his lids.
	\end{itemize}
}
\item singular noun \\
If you say that someone is \textbf{keeping the}  \textbf{lid}  \textbf{on} an activity or a piece of information , you mean that they are restricting the activity or are keeping the information secret .
 \textit{
	\begin{itemize}
	\item The soldiers' presence seemed to keep a lid on the violence.
	\item Their finance ministry is still trying to put a lid on the long-simmering securities
scandal.
	\end{itemize}
}
\item singular noun \\
If you say that you want to \textbf{keep a}  \textbf{lid}  \textbf{on} the cost of doing something, you mean that you want to prevent it costing you more than you feel is reasonable .
 \textit{
	\begin{itemize}
	\item If our industry is to remain competitive it must mean keeping a lid on prices.
	\end{itemize}
}
\end{enumerate}

\section*{favor}
{\large \color{blue}  }
\subsection*{Explain}
\begin{enumerate}
\item noun \\
1.  2.  3.  4.  5.  6.  7.  8.  \textit{
	\begin{itemize}
	\item to do someone a favor
	\item your favor of the 15th June
	\end{itemize}
}
\item verb transitive \\
9.  10.  11.  12.  13.  14.  15.  \textit{
	\begin{itemize}
	\item rain favored his escape
	\item to favor one's mother
	\item to favor an injured leg
	\end{itemize}
}
\end{enumerate}

\section*{liquid}
{\large \color{blue}  liquids  }
\subsection*{Explain}
\begin{enumerate}
\item variable noun \\
A \textbf{liquid} is a substance which is not solid but which flows and can be poured , for example water.
 \textit{
	\begin{itemize}
	\item Drink plenty of liquid.
	\item Boil for 20 minutes until the liquid has reduced by half.
	\item Solids turn to liquids at certain temperatures.
	\end{itemize}
}
\item adjective \\
A \textbf{liquid} substance is in the form of a liquid rather than being solid or a gas .
 \textit{
	\begin{itemize}
	\item Wash in warm water with liquid detergent.
	\item ...liquid nitrogen.
	\item Fats are solid at room temperature, and oil is liquid at room temperature.
	\end{itemize}
}
\item adjective \\
\textbf{Liquid} assets are the things that a person or company owns which can be quickly turned into cash if necessary .
 \textit{
	\begin{itemize}
	\item The bank had sufficient liquid assets to continue operations.
	\end{itemize}
}
\end{enumerate}

\section*{favorite}
{\large \color{blue}  }
\subsection*{Explain}
\begin{enumerate}
\item noun \\
1.  2.  \textit{
	\begin{itemize}
	\end{itemize}
}
\item adjective \\
3.  \textit{
	\begin{itemize}
	\end{itemize}
}
\end{enumerate}

\section*{mark}
{\large \color{blue}  marks  marking  marked  }
\subsection*{Explain}
\begin{enumerate}
\item countable noun \\
A \textbf{mark} is a small area of something such as dirt that has accidentally got onto a surface or piece of clothing.
 \textit{
	\begin{itemize}
	\item The dogs are always rubbing against the wall and making dirty marks.
	\item A properly fitting bra should never leave red marks.
	\end{itemize}
}
\item verb \\
If something \textbf{marks} a surface, or if the surface \textbf{marks} , the surface is damaged by marks or a mark.
 \textit{
	\begin{itemize}
	\item Leather overshoes were put on the horses' hooves to stop them marking the turf.
	\item I have to be more careful with the work tops, as wood marks easily.
	\end{itemize}
}
\item countable noun \\
A \textbf{mark} is a written or printed symbol, for example a letter of the alphabet .
 \textit{
	\begin{itemize}
	\item He made marks with a pencil.
	\end{itemize}
}
\item verb \\
If you \textbf{mark} something with a particular word or symbol, you write that word or symbol on it.
 \textit{
	\begin{itemize}
	\item The bank marks the check 'certified'.
	\item Mark the frame with your postcode.
	\item For more details about these products, send a postcard marked HB/FF.
	\end{itemize}
}
\item countable noun \\
A \textbf{mark} is a point that is given for a correct  answer or for doing something well in an exam or competition . A \textbf{mark} can also be a written symbol such as a letter that indicates how good a student's
or competitor's work or performance is.
 \textit{
	\begin{itemize}
	\item ...a simple scoring device of marks out of 10, where '1' equates to 'Very poor performance'.
	\item Candidates who answered 'b' could be awarded half marks.
	\item He did well to get such a good mark.
	\end{itemize}
}
\item plural noun \\
If someone gets good or high \textbf{marks} for doing something, they have done it well. If they get poor or low \textbf{marks} , they have done it badly .
 \textit{
	\begin{itemize}
	\item You have to give her top marks for moral guts.
	\item His administration has earned low marks for its economic policies.
	\end{itemize}
}
\item verb \\
When a teacher  \textbf{marks} a student's work, the teacher decides how good it is and writes a number or letter on it to indicate this opinion.
 \textit{
	\begin{itemize}
	\item He was marking essays in his small study.
	\end{itemize}
}
\item countable noun \\
A particular \textbf{mark} is a particular number, point, or stage which has been reached or might be reached, especially a significant one.
 \textit{
	\begin{itemize}
	\item Unemployment is rapidly approaching the one million mark.
	\end{itemize}
}
\item countable noun \\
The \textbf{mark}  \textbf{of} something is the characteristic feature that enables you to recognize it.
 \textit{
	\begin{itemize}
	\item The mark of a civilized society is that it looks after its weakest members.
	\end{itemize}
}
\item singular noun \\
If you say that a type of behaviour or an event is \textbf{a}  \textbf{mark}  \textbf{of} a particular quality, feeling, or situation, you mean it shows that that quality,
feeling, or situation exists.
 \textit{
	\begin{itemize}
	\item It was a mark of his unfamiliarity with Hollywood that he didn't understand that
an agent was paid out of his client's share.
	\item Shopkeepers closed their shutters as a mark of respect.
	\end{itemize}
}
\item verb \\
If something \textbf{marks} a place or position, it shows where something else is or where it used to be.
 \textit{
	\begin{itemize}
	\item A huge crater marks the spot where the explosion happened.
	\item ...the river which marks the border with Thailand.
	\end{itemize}
}
\item verb \\
An event that \textbf{marks} a particular stage or point is a sign that something different is about to happen .
 \textit{
	\begin{itemize}
	\item The announcement marks the end of an extraordinary period in European history.
	\item That programme received critical acclaim and marked a turning point in Sonita's career.
	\end{itemize}
}
\item verb \\
If you do something to \textbf{mark} an event or occasion , you do it to show that you are aware of the importance of the event or occasion.
 \textit{
	\begin{itemize}
	\item The four new stamps mark the 100th anniversary of the British Astronomical Association.
	\item Hundreds of thousands of people took to the streets to mark the occasion.
	\end{itemize}
}
\item verb \\
If a particular quality or feature \textbf{marks} something, it is a quality or feature which that thing typically has.
 \textit{
	\begin{itemize}
	\item Tragedy has marked Wilmette's life.
	\item The style is marked by simplicity, clarity, and candor.
	\end{itemize}
}
\item verb \\
Something that \textbf{marks} someone \textbf{as} a particular type of person indicates that they are that type of person.
 \textit{
	\begin{itemize}
	\item Her opposition to abortion and feminism mark her as a convinced traditionalist.
	\end{itemize}
}
\item verb \\
In a team game, when a defender \textbf{is marking} an attacker, they are trying to stay close to the attacker and prevent them from getting the ball.
 \textit{
	\begin{itemize}
	\item Every player knows who to mark when we start a game.
	\end{itemize}
}
\item countable noun \\
The \textbf{mark} was the unit of money that was used in Germany. In 2002 it was replaced by the euro .
 \textbf{The mark} was also used to refer to the German currency system.
 \textit{
	\begin{itemize}
	\item The government gave 30 million marks for new school books.
	\item The mark appreciated 12 per cent against the dollar.
	\end{itemize}
}
\item uncountable noun \\
\textbf{Mark} is used before a number to indicate a particular temperature level in a gas oven.
 \textit{
	\begin{itemize}
	\item Set the oven at gas mark 4.
	\end{itemize}
}
\item uncountable noun \\
\textbf{Mark} is used before a number to indicate a particular version or model of a vehicle, machine, or device.
 \textit{
	\begin{itemize}
	\item All eyes will be on the unveiling of the mark III model at the Detroit car show.
	\end{itemize}
}
\item  \\
 to leave your/a mark \textit{
	\begin{itemize}
	\end{itemize}
}
\item  \\
 to make your/a mark \textit{
	\begin{itemize}
	\end{itemize}
}
\item  \\
 quick off the mark \textit{
	\begin{itemize}
	\end{itemize}
}
\item  \\
 on your marks \textit{
	\begin{itemize}
	\end{itemize}
}
\item  \\
 on/off the mark \textit{
	\begin{itemize}
	\end{itemize}
}
\item  \\
 up to the mark \textit{
	\begin{itemize}
	\end{itemize}
}
\item  \\
 wide of the mark \textit{
	\begin{itemize}
	\end{itemize}
}
\item  \\
 mark you \textit{
	\begin{itemize}
	\end{itemize}
}
\end{enumerate}

\section*{fertilizer}
{\large \color{blue}  fertilizers  }
\subsection*{Explain}
\begin{enumerate}
\item variable noun \\
\textbf{Fertilizer} is a substance such as solid animal waste or a chemical mixture that you spread on the ground in order to make plants grow more successfully.
 \textit{
	\begin{itemize}
	\item ...farming without any purchased chemical, fertilizer or pesticide.
	\item Work in a balanced fertiliser before planting.
	\end{itemize}
}
\end{enumerate}

\section*{melody}
{\large \color{blue}  melodies  }
\subsection*{Explain}
\begin{enumerate}
\item countable noun \\
A \textbf{melody} is a tune.
 \textit{
	\begin{itemize}
	\end{itemize}
}
\item uncountable noun \\
\textbf{Melody} is the quality of having a pleasant tune.
 \textit{
	\begin{itemize}
	\item Her voice was full of melody.
	\end{itemize}
}
\end{enumerate}

\section*{fiber}
{\large \color{blue}  }
\subsection*{Explain}
\begin{enumerate}
\item noun \\
 the usual US spelling of  fibre \textit{
	\begin{itemize}
	\end{itemize}
}
\end{enumerate}

\section*{merchandise}
{\large \color{blue}  }
\subsection*{Explain}
\begin{enumerate}
\item uncountable noun \\
\textbf{Merchandise} is goods that are bought , sold , or traded.
 \textit{
	\begin{itemize}
	\end{itemize}
}
\end{enumerate}

\section*{flavor}
{\large \color{blue}  }
\subsection*{Explain}
\begin{enumerate}
\item noun \\
1.  2.  3.  4.  5.  \textit{
	\begin{itemize}
	\item the flavor of the city
	\end{itemize}
}
\item verb transitive \\
6.  \textit{
	\begin{itemize}
	\end{itemize}
}
\end{enumerate}

\section*{merchant}
{\large \color{blue}  merchants  }
\subsection*{Explain}
\begin{enumerate}
\item countable noun \\
A \textbf{merchant} is a person who buys or sells goods in large quantities , especially one who imports and exports them.
 \textit{
	\begin{itemize}
	\item Any knowledgeable wine merchant would be able to advise you.
	\end{itemize}
}
\item countable noun \\
A \textbf{merchant} is a person who owns or runs a shop , store , or other business .
 \textit{
	\begin{itemize}
	\item The family was forced to live on credit from local merchants.
	\end{itemize}
}
\item adjective \\
\textbf{Merchant}  seamen or ships are involved in carrying goods for trade.
 \textit{
	\begin{itemize}
	\item There's been a big reduction in the size of the British merchant fleet in recent
years.
	\end{itemize}
}
\end{enumerate}

\section*{flu}
{\large \color{blue}  }
\subsection*{Explain}
\begin{enumerate}
\item uncountable noun \\
\textbf{Flu} is an illness which is similar to a bad  cold but more serious . It often makes you feel very weak and makes your muscles  hurt .
 \textit{
	\begin{itemize}
	\item I got flu.
	\item He had come down with the flu.
	\end{itemize}
}
\end{enumerate}

\section*{mess}
{\large \color{blue}  messes  messing  messed  }
\subsection*{Explain}
\begin{enumerate}
\item singular noun \\
If you say that something is \textbf{a}  \textbf{mess} or \textbf{in a}  \textbf{mess} , you think that it is in an untidy state.
 \textit{
	\begin{itemize}
	\item The house is a mess.
	\item The wrong shampoo can leave curly hair in a tangled mess.
	\item Linda can't stand mess.
	\end{itemize}
}
\item variable noun \\
If you say that a situation is \textbf{a}  \textbf{mess} , you mean that it is full of trouble or problems . You can also say that something is \textbf{in a}  \textbf{mess} .
 \textit{
	\begin{itemize}
	\item I've made such a mess of my life.
	\item ...the many reasons why the economy is in such a mess.
	\item She'd got herself into a mess, of that he was certain.
	\end{itemize}
}
\item variable noun \\
\textbf{A}  \textbf{mess} is something liquid or sticky that has been accidentally dropped on something.
 \textit{
	\begin{itemize}
	\item Finally, making a dreadful mess, they devour the fruit.
	\item I'll clear up the mess later.
	\end{itemize}
}
\item countable noun \\
The \textbf{mess} at a military  base or military barracks is the building in which members of the armed  forces can eat or relax .
 \textit{
	\begin{itemize}
	\item ...a party at the officers' mess.
	\item He hurried to the Mess to find the control officer.
	\end{itemize}
}
\end{enumerate}

\section*{gasoline}
{\large \color{blue}  }
\subsection*{Explain}
\begin{enumerate}
\item uncountable noun \\
\textbf{Gasoline} is the same as petrol .
 \textit{
	\begin{itemize}
	\end{itemize}
}
\end{enumerate}

\section*{mixture}
{\large \color{blue}  mixtures  }
\subsection*{Explain}
\begin{enumerate}
\item singular noun \\
A \textbf{mixture}  \textbf{of} things consists of several different things together.
 \textit{
	\begin{itemize}
	\item They looked at him with a mixture of horror, envy, and awe.
	\item ...a mixture of spiced, grilled vegetables served cold.
	\end{itemize}
}
\item countable noun \\
A \textbf{mixture} is a substance that consists of other substances which have been stirred or shaken together.
 \textit{
	\begin{itemize}
	\item Prepare the gravy mixture.
	\item ...a mixture of water and sugar and salt.
	\end{itemize}
}
\end{enumerate}

\section*{glamor}
{\large \color{blue}  }
\subsection*{Explain}
\begin{enumerate}
\end{enumerate}

\section*{parasite}
{\large \color{blue}  parasites  }
\subsection*{Explain}
\begin{enumerate}
\item countable noun \\
A \textbf{parasite} is a small animal or plant that lives on or inside a larger animal or plant, and gets its food from it.
 \textit{
	\begin{itemize}
	\end{itemize}
}
\item countable noun \\
If you disapprove of someone because you think that they get money or other things from other people but do not do anything in return , you can call them a \textbf{parasite} .
 \textit{
	\begin{itemize}
	\end{itemize}
}
\end{enumerate}

\section*{gramme}
{\large \color{blue}  }
\subsection*{Explain}
\begin{enumerate}
\end{enumerate}

\section*{particle}
{\large \color{blue}  particles  }
\subsection*{Explain}
\begin{enumerate}
\item countable noun \\
A \textbf{particle}  \textbf{of} something is a very small piece or amount of it.
 \textit{
	\begin{itemize}
	\item ...a particle of hot metal.
	\item There is a particle of truth in his statement.
	\item ...food particles.
	\end{itemize}
}
\item countable noun \\
In physics , a \textbf{particle} is a piece of matter smaller than an atom , for example an electron or a proton .
 \textit{
	\begin{itemize}
	\end{itemize}
}
\item countable noun \\
In grammar , a \textbf{particle} is a preposition such as 'into' or an adverb such as 'out' which can combine with a verb to form a phrasal verb.
 \textit{
	\begin{itemize}
	\end{itemize}
}
\end{enumerate}

\section*{gymnasium}
{\large \color{blue}  gymnasiums  gymnasia  }
\subsection*{Explain}
\begin{enumerate}
\item countable noun \\
A \textbf{gymnasium} is the same as a gym .
 \textit{
	\begin{itemize}
	\end{itemize}
}
\end{enumerate}

\section*{plight}
{\large \color{blue}  plights  }
\subsection*{Explain}
\begin{enumerate}
\item countable noun \\
If you refer to someone's \textbf{plight} , you mean that they are in a difficult or distressing  situation that is full of problems .
 \textit{
	\begin{itemize}
	\item ...the worsening plight of Third World countries plagued by debts.
	\end{itemize}
}
\end{enumerate}

\section*{harbor}
{\large \color{blue}  }
\subsection*{Explain}
\begin{enumerate}
\item noun \\
1.  2.  \textit{
	\begin{itemize}
	\end{itemize}
}
\item verb transitive \\
3.  4.  5.  \textit{
	\begin{itemize}
	\item to harbor a grudge
	\end{itemize}
}
\item verb intransitive \\
6.  7.  \textit{
	\begin{itemize}
	\end{itemize}
}
\end{enumerate}

\section*{recipe}
{\large \color{blue}  recipes  }
\subsection*{Explain}
\begin{enumerate}
\item countable noun \\
A \textbf{recipe} is a list of ingredients and a set of instructions that tell you how to cook something.
 \textit{
	\begin{itemize}
	\item ...a traditional recipe for oatmeal biscuits.
	\item ...a recipe book.
	\end{itemize}
}
\item singular noun \\
If you say that something is \textbf{a recipe for} a particular situation , you mean that it is likely to result in that situation.
 \textit{
	\begin{itemize}
	\item Large-scale inflation is a recipe for disaster.
	\end{itemize}
}
\end{enumerate}

\section*{honor}
{\large \color{blue}  }
\subsection*{Explain}
\begin{enumerate}
\end{enumerate}

\section*{recovery}
{\large \color{blue}  recoveries  }
\subsection*{Explain}
\begin{enumerate}
\item variable noun \\
If a sick person makes a \textbf{recovery} , he or she becomes well again.
 \textit{
	\begin{itemize}
	\item He made a remarkable recovery from a shin injury.
	\item He had been given less than a one in 500 chance of recovery by his doctors.
	\end{itemize}
}
\item variable noun \\
When there is a \textbf{recovery} in a country's economy , it improves .
 \textit{
	\begin{itemize}
	\item Interest-rate cuts have failed to bring about economic recovery.
	\item In many sectors of the economy the recovery has started.
	\end{itemize}
}
\item uncountable noun \\
You talk about the \textbf{recovery}  \textbf{of} something when you get it back after it has been lost or stolen .
 \textit{
	\begin{itemize}
	\item A substantial reward is being offered for the recovery of a painting by Turner.
	\item She has a reasonable prospect of recovery from the insurer.
	\end{itemize}
}
\item uncountable noun \\
You talk about the \textbf{recovery}  \textbf{of} someone's physical or mental state when they return to this state.
 \textit{
	\begin{itemize}
	\item ...the abrupt loss and recovery of consciousness.
	\end{itemize}
}
\item  \\
 in recovery \textit{
	\begin{itemize}
	\end{itemize}
}
\end{enumerate}

\section*{humor}
{\large \color{blue}  }
\subsection*{Explain}
\begin{enumerate}
\end{enumerate}

\section*{scare}
{\large \color{blue}  scares  scaring  scared  }
\subsection*{Explain}
\begin{enumerate}
\item verb \\
If something \textbf{scares} you, it frightens or worries you.
 \textit{
	\begin{itemize}
	\item You're scaring me.
	\item What scares me most is that I'm going to end up not being married.
	\item The prospect of failure scares me rigid.
	\item It scared him to realise how close he had come to losing everything.
	\end{itemize}
}
\item singular noun \\
If a sudden unpleasant  experience gives you a \textbf{scare} , it frightens you.
 \textit{
	\begin{itemize}
	\item Don't you realize what a scare you've given us all?
	\item We got a bit of a scare.
	\end{itemize}
}
\item countable noun \\
A \textbf{scare} is a situation in which many people are afraid or worried because they think something dangerous is happening which will  affect them all.
 \textit{
	\begin{itemize}
	\item He's had a prostate cancer scare.
	\item Despite the scare there are no plans to withdraw the drug.
	\end{itemize}
}
\item countable noun \\
A bomb  \textbf{scare} or a security  \textbf{scare} is a situation in which there is believed to be a bomb in a place.
 \textit{
	\begin{itemize}
	\item Despite many recent bomb scares, no one has yet been hurt.
	\item ...a security scare over a suspect package.
	\end{itemize}
}
\end{enumerate}

\section*{inquiry}
{\large \color{blue}  inquiries  }
\subsection*{Explain}
\begin{enumerate}
\item countable noun \\
An \textbf{inquiry} is a question which you ask in order to get some information.
 \textit{
	\begin{itemize}
	\item He made some inquiries and discovered she had gone to the Continent.
	\item After a brief inquiry about the Christmas holiday, he returned to the subject of
music.
	\end{itemize}
}
\item countable noun \\
An \textbf{inquiry} is an official investigation.
 \textit{
	\begin{itemize}
	\item This is the most difficult and shocking murder inquiry I have had to open in the
last 25 years.
	\item The Democratic Party has called for an independent inquiry into the incident.
	\end{itemize}
}
\item uncountable noun \\
\textbf{Inquiry} is the process of asking about or investigating something in order to find out more about it.
 \textit{
	\begin{itemize}
	\item The investigation has suddenly switched to a new line of inquiry.
	\end{itemize}
}
\item  \\
 help the police with their inquiries \textit{
	\begin{itemize}
	\end{itemize}
}
\end{enumerate}

\section*{scissors}
{\large \color{blue}  }
\subsection*{Explain}
\begin{enumerate}
\item plural noun \\
\textbf{Scissors} are a small cutting tool with two sharp blades that are screwed together. You use scissors for cutting things such as paper and cloth.
 \textit{
	\begin{itemize}
	\item He told me to get some scissors.
	\item She picked up a pair of scissors from the windowsill.
	\end{itemize}
}
\end{enumerate}

\section*{installment}
{\large \color{blue}  }
\subsection*{Explain}
\begin{enumerate}
\item noun \\
1.  2.  \textit{
	\begin{itemize}
	\end{itemize}
}
\end{enumerate}

\section*{section}
{\large \color{blue}  sections  sectioning  sectioned  }
\subsection*{Explain}
\begin{enumerate}
\item countable noun \\
A \textbf{section} of something is one of the parts into which it is divided or from which it is formed.
 \textit{
	\begin{itemize}
	\item He acknowledged that his family belonged to a section of society known as 'the idle
rich'.
	\item They moulded a complete new bow section for the boat.
	\item ...a large orchestra, with a vast percussion section.
	\item ...the Georgetown section of Washington, D.C.
	\item ...a geological section of a rock.
	\end{itemize}
}
\item verb \\
If something \textbf{is sectioned} , it is divided into sections.
 \textit{
	\begin{itemize}
	\item It holds vegetables in place while they are being peeled or sectioned.
	\end{itemize}
}
\item countable noun \\
A \textbf{section} of an official document such as a report , law, or constitution is one of the parts into which it is divided.
 \textit{
	\begin{itemize}
	\item ...section 14 of the Trade Descriptions Act 1968.
	\item ...the all-important section on the powers of the federal government.
	\end{itemize}
}
\item countable noun \\
A \textbf{section} is a diagram of something such as a building or a part of the body. It shows how the object would
 appear to you if it were cut from top to bottom and looked at from the side.
 \textit{
	\begin{itemize}
	\item For some buildings a vertical section is more informative than a plan.
	\end{itemize}
}
\end{enumerate}

\section*{interest}
{\large \color{blue}  interests  interesting  interested  }
\subsection*{Explain}
\begin{enumerate}
\item variable noun \\
If you have an \textbf{interest}  \textbf{in} something, you want to learn or hear more about it.
 \textit{
	\begin{itemize}
	\item There has been a lively interest in the elections in the last two weeks.
	\item His parents tried to discourage his interest in music, but he persisted.
	\item She'd liked him at first, but soon lost interest.
	\item Food was of no interest to her at all.
	\end{itemize}
}
\item countable noun \\
Your \textbf{interests} are the things that you enjoy doing.
 \textit{
	\begin{itemize}
	\item Encourage your child in her interests and hobbies.
	\item He developed a wide range of sporting interests as a pupil at Millfield.
	\end{itemize}
}
\item verb \\
If something \textbf{interests} you, it attracts your attention so that you want to learn or hear more about it or continue doing it.
 \textit{
	\begin{itemize}
	\item Animation had always interested me.
	\item It may interest you to know that the housekeeper witnessed the attack.
	\end{itemize}
}
\item verb \\
If you are trying to persuade someone to buy or do something, you can say that you are trying to \textbf{interest} them \textbf{in} it.
 \textit{
	\begin{itemize}
	\item In the meantime I can't interest you in a new car, I suppose?
	\end{itemize}
}
\item countable noun \\
If something is in the \textbf{interests} of a particular person or group, it will benefit them in some way.
 \textit{
	\begin{itemize}
	\item Did those directors act in the best interests of their club?
	\item The social worker would try to get her to see she was acting against the boy's interests.
	\end{itemize}
}
\item countable noun \\
You can use \textbf{interests} to refer to groups of people who you think use their power or money to benefit themselves.
 \textit{
	\begin{itemize}
	\item The government accused unnamed 'foreign interests' of inciting the trouble.
	\item He resigned as finance minister weeks before the election and stood against big-business
interests.
	\end{itemize}
}
\item countable noun \\
A person or organization that has \textbf{interests} in a company or in a particular type of business owns shares in this company or this type of business.
 \textit{
	\begin{itemize}
	\item Her other business interests include a theme park in Scandinavia and hotels in the
West Country.
	\item ...the Hatch family, who controlled large dairy interests.
	\item Disney will retain a 51 percent controlling interest in the venture.
	\end{itemize}
}
\item countable noun \\
If a person, country, or organization has an \textbf{interest}  \textbf{in} a possible event or situation , they want that event or situation to happen because they are likely to benefit from it.
 \textit{
	\begin{itemize}
	\item The West has an interest in promoting democratic forces in Eastern Europe.
	\item Many people have an interest in not remembering what happened that night.
	\end{itemize}
}
\item uncountable noun \\
\textbf{Interest} is extra money that you receive if you have invested a sum of money. \textbf{Interest} is also the extra money that you pay if you have borrowed money or are buying something on credit.
 \textit{
	\begin{itemize}
	\item Does your current account pay interest?
	\item This is an important step toward lower interest rates.
	\end{itemize}
}
\item  \\
 in the interests of/in the interest of \textit{
	\begin{itemize}
	\end{itemize}
}
\end{enumerate}

\section*{shop}
{\large \color{blue}  shops  shopping  shopped  }
\subsection*{Explain}
\begin{enumerate}
\item countable noun \\
A \textbf{shop} is a building or part of a building where things are sold .
 \textit{
	\begin{itemize}
	\item ...health-food shops.
	\item ...a record shop.
	\item It's not available in the shops.
	\end{itemize}
}
\item verb \\
When you \textbf{shop} , you go to shops and buy things.
 \textit{
	\begin{itemize}
	\item He always shopped at the Co-op.
	\item ...some advice that's worth bearing in mind when shopping for a new carpet.
	\item ...customers who shop once a week.
	\end{itemize}
}
\item countable noun \\
You can refer to a place where a particular service is offered as a particular type of \textbf{shop} .
 \textit{
	\begin{itemize}
	\item ...the barber shop where Rodney sometimes had his hair cut.
	\item ...betting shops.
	\item ...your local pet shop.
	\end{itemize}
}
\item countable noun \\
You can refer to a place where things are made or done as a particular kind of \textbf{shop} .
 \textit{
	\begin{itemize}
	\item ...the blacksmith's shop.
	\item ...a repair shop.
	\end{itemize}
}
\item verb \\
If you \textbf{shop} someone, you report them to the police for doing something illegal .
 \textit{
	\begin{itemize}
	\item His appalled family shopped him to the police.
	\item Fraudsters are often shopped by honest friends and neighbours.
	\end{itemize}
}
\item  \\
 all over the shop \textit{
	\begin{itemize}
	\end{itemize}
}
\item  \\
 set up shop \textit{
	\begin{itemize}
	\end{itemize}
}
\item  \\
 shop till you drop \textit{
	\begin{itemize}
	\end{itemize}
}
\item  \\
 shut up shop \textit{
	\begin{itemize}
	\end{itemize}
}
\item  \\
 to talk shop \textit{
	\begin{itemize}
	\end{itemize}
}
\end{enumerate}

\section*{interference}
{\large \color{blue}  }
\subsection*{Explain}
\begin{enumerate}
\item uncountable noun \\
\textbf{Interference} by a person or group is their unwanted or unnecessary involvement in something.
 \textit{
	\begin{itemize}
	\item The parliament described the decree as interference in the republic's internal affairs.
	\item Airlines will be able to set cheap fares without interference from the government.
	\end{itemize}
}
\item uncountable noun \\
When there is \textbf{interference} , a radio signal is affected by other radio waves or electrical activity so that it cannot be received properly.
 \textit{
	\begin{itemize}
	\item ...electrical interference.
	\item They have been accused of deliberately causing interference to transmissions.
	\end{itemize}
}
\end{enumerate}

\section*{store}
{\large \color{blue}  stores  storing  stored  }
\subsection*{Explain}
\begin{enumerate}
\item countable noun \\
A \textbf{store} is a building or part of a building where things are sold . In British English, \textbf{store} is used mainly to refer to a large shop selling a variety of goods, but in American English a \textbf{store} can be any size of shop.
 \textit{
	\begin{itemize}
	\item ...the decline in independent stores in the town centre.
	\item ...grocery stores.
	\item ...a record store.
	\end{itemize}
}
\item verb \\
When you \textbf{store} things, you put them in a container or other place and leave them there until they are needed .
 \textbf{Store away}  means the same as store .
 \textit{
	\begin{itemize}
	\item Store the cookies in an airtight tin.
	\item Some types of garden furniture must be stored inside in the winter.
	\item I took a photograph of the plaque and stored it away.
	\item He's stored away nearly one ton of potatoes.
	\end{itemize}
}
\item verb \\
When you \textbf{store} information, you keep it in your memory , in a file , or in a computer .
 \textit{
	\begin{itemize}
	\item Where in the brain do we store information about colours?
	\item ...chips for storing data in electronic equipment.
	\end{itemize}
}
\item countable noun \\
A \textbf{store}  \textbf{of} things is a supply of them that you keep somewhere until you need them.
 \textit{
	\begin{itemize}
	\item I handed over my secret store of chocolate biscuits.
	\item Dolly's store of drinking glasses had run out.
	\end{itemize}
}
\item countable noun \\
A \textbf{store} is a place where things are kept while they are not being used.
 \textit{
	\begin{itemize}
	\item ...a decision taken in 1982 to build a store for spent fuel from submarines.
	\item ...a grain store.
	\end{itemize}
}
\item countable noun \\
If you have a \textbf{store}  \textbf{of}  knowledge , jokes , or stories , you have a large amount of them ready to be used.
 \textit{
	\begin{itemize}
	\item He possessed a vast store of knowledge.
	\item Jessica dipped into her store of theatrical anecdotes.
	\end{itemize}
}
\item  \\
 in store \textit{
	\begin{itemize}
	\end{itemize}
}
\item  \\
 to set great store by or on something \textit{
	\begin{itemize}
	\end{itemize}
}
\end{enumerate}

\section*{preference}
{\large \color{blue}  preferences  }
\subsection*{Explain}
\begin{enumerate}
\item variable noun \\
If you have a \textbf{preference}  \textbf{for} something, you would like to have or do that thing rather than something else.
 \textit{
	\begin{itemize}
	\item Parents can express a preference for the school their child attends.
	\item Many of these products were bought in preference to their own.
	\end{itemize}
}
\item uncountable noun \\
If you \textbf{give preference}  \textbf{to} someone with a particular qualification or feature , you choose them rather than someone else.
 \textit{
	\begin{itemize}
	\item Firms often give preference to present employees in promotions, transfers, and other
career-enhancing opportunities.
	\end{itemize}
}
\end{enumerate}

\section*{telescope}
{\large \color{blue}  telescopes  }
\subsection*{Explain}
\begin{enumerate}
\item countable noun \\
A \textbf{telescope} is a long instrument shaped like a tube . It has lenses inside it that make distant things seem larger and nearer when you look through it.
 \textit{
	\begin{itemize}
	\item The telescope will enable scientists to see deeper into the universe than ever before.
	\end{itemize}
}
\end{enumerate}

\section*{proficiency}
{\large \color{blue}  }
\subsection*{Explain}
\begin{enumerate}
\item uncountable noun \\
If you show  \textbf{proficiency}  \textbf{in} something, you show ability or skill at it.
 \textit{
	\begin{itemize}
	\item Evidence of basic proficiency in English is part of the admission requirement.
	\end{itemize}
}
\end{enumerate}

\section*{trace}
{\large \color{blue}  traces  tracing  traced  }
\subsection*{Explain}
\begin{enumerate}
\item verb \\
If you \textbf{trace} the origin or development of something, you find out or describe how it started or developed.
 \textbf{Trace back} means the same as trace .
 \textit{
	\begin{itemize}
	\item The exhibition traces the history of graphic design in America from the 19th century
to the present.
	\item I first went there to trace my roots, visiting my mum's home island of Jamaica.
	\item The psychiatrist successfully traced some of her problems to severe childhood traumas.
	\item Britain's Parliament can trace its history back to the English Parliament of the
13th century.
	\item She has never traced back her lineage, but believes her grandparents were from Aberdeenshire.
	\end{itemize}
}
\item verb \\
If you \textbf{trace} someone or something, you find them after looking for them.
 \textit{
	\begin{itemize}
	\item Police are anxious to trace two men seen leaving the house just before 8am.
	\item We are currently trying to trace the whereabouts of certain sums of money.
	\item They traced the van to a New Jersey car rental agency.
	\end{itemize}
}
\item verb \\
If you \textbf{trace} something such as a pattern or a shape, for example with your finger or toe , you mark its outline on a surface.
 \textit{
	\begin{itemize}
	\item I traced the course of the river on the map.
	\end{itemize}
}
\item verb \\
If you \textbf{trace} a picture , you copy it by covering it with a piece of transparent paper and drawing over the
lines underneath.
 \textit{
	\begin{itemize}
	\item She learned to draw by tracing pictures out of old storybooks.
	\end{itemize}
}
\item countable noun \\
A \textbf{trace}  \textbf{of} something is a very small amount of it.
 \textit{
	\begin{itemize}
	\item Wash them in cold water to remove all traces of sand.
	\item He took great pains to write on his subject without a trace of sensationalism.
	\end{itemize}
}
\item countable noun \\
A \textbf{trace} is a sign which shows you that someone or something has been in a place.
 \textit{
	\begin{itemize}
	\item The local church has traces of fifteenth-century frescoes.
	\item There's been no trace of my aunt and uncle.
	\item Finally, and mysteriously, she disappeared without trace.
	\end{itemize}
}
\item  \\
 to sink without trace \textit{
	\begin{itemize}
	\end{itemize}
}
\end{enumerate}

\section*{requirement}
{\large \color{blue}  requirements  }
\subsection*{Explain}
\begin{enumerate}
\item countable noun \\
A \textbf{requirement} is a quality or qualification that you must have in order to be allowed to do something or to be suitable for something.
 \textit{
	\begin{itemize}
	\item Its products met all legal requirements.
	\item Graduate status is the minimum requirement for entry to the teaching profession.
	\item I knew that concentration was the first requirement for learning.
	\end{itemize}
}
\item countable noun \\
Your \textbf{requirements} are the things that you need.
 \textit{
	\begin{itemize}
	\item Variations of this programme can be arranged to suit your requirements.
	\item ...a packaged food which provides 100 percent of your daily requirement of one vitamin.
	\end{itemize}
}
\end{enumerate}

\section*{trademark}
{\large \color{blue}  trademarks  }
\subsection*{Explain}
\begin{enumerate}
\item countable noun \\
A \textbf{trademark} is a name or symbol that a company uses on its products and that cannot legally be used by another company.
 \textit{
	\begin{itemize}
	\end{itemize}
}
\item countable noun \\
If you say that something is the \textbf{trademark} of a particular person or place, you mean that it is characteristic of them or typically associated with them.
 \textit{
	\begin{itemize}
	\item ...the spiky punk hairdo that became his trademark.
	\end{itemize}
}
\end{enumerate}

\section*{reward}
{\large \color{blue}  rewards  rewarding  rewarded  }
\subsection*{Explain}
\begin{enumerate}
\item countable noun \\
A \textbf{reward} is something that you are given, for example because you have behaved  well , worked hard , or provided a service to the community .
 \textit{
	\begin{itemize}
	\item Never give children sweets, biscuits or cakes as a reward for being good.
	\item He was given the job as a reward for running a successful leadership bid.
	\end{itemize}
}
\item countable noun \\
A \textbf{reward} is a sum of money offered to anyone who can give information about lost or stolen property or about someone who is wanted by the police .
 \textit{
	\begin{itemize}
	\item The firm offered a £10,000 reward for information leading to the arrest of the robbers.
	\end{itemize}
}
\item verb \\
If you do something and \textbf{are rewarded} with a particular benefit , you receive that benefit as a result of doing that thing.
 \textit{
	\begin{itemize}
	\item Make the extra effort to impress the buyer and you will be rewarded with a quicker
sale.
	\end{itemize}
}
\item countable noun \\
The \textbf{rewards} of something are the benefits that you receive as a result of doing or having that
thing.
 \textit{
	\begin{itemize}
	\item The company is only just starting to reap the rewards of long-term investments.
	\item Potentially high financial rewards are attached to senior hospital posts.
	\end{itemize}
}
\item verb \\
If you say that something \textbf{rewards} your attention or effort , you mean that it is worth  spending time or effort on it.
 \textit{
	\begin{itemize}
	\item It is a difficult book to read, but it richly rewards the effort.
	\end{itemize}
}
\end{enumerate}

\section*{wonder}
{\large \color{blue}  wonders  wondering  wondered  }
\subsection*{Explain}
\begin{enumerate}
\item verb \\
If you \textbf{wonder}  \textbf{about} something, you think about it, either because it interests you and you want to know more about it, or because you are worried or suspicious about it.
 \textit{
	\begin{itemize}
	\item I wondered what that noise was.
	\item 'He claims to be her father,' said Max. 'We've been wondering about him.'
	\item It makes you wonder about the effect on men's behaviour.
	\item 'Why does she want to get in there?' Pete wondered.
	\item But there was something else, too. Not hard evidence, but it made me wonder.
	\end{itemize}
}
\item verb \\
If you \textbf{wonder}  \textbf{at} something, you are very surprised about it or think about it in a very surprised
 way .
 \textit{
	\begin{itemize}
	\item He liked to sit and wonder at all that had happened.
	\item Walk down Castle Street, admire our little jewel of a cathedral, then wonder at the
castle.
	\item We all wonder that you're still alive.
	\end{itemize}
}
\item singular noun \\
If you say that it is a \textbf{wonder}  \textbf{that} something happened , you mean that it is very surprising and unexpected .
 \textit{
	\begin{itemize}
	\item It's a wonder that it took almost ten years.
	\item The wonder is that Olivier was not seriously hurt.
	\end{itemize}
}
\item uncountable noun \\
\textbf{Wonder} is a feeling of great surprise and pleasure that you have, for example when you see something that is very beautiful , or when something happens that you thought was impossible .
 \textit{
	\begin{itemize}
	\item 'That's right!' Bobby exclaimed in wonder. 'How did you remember that?'
	\item I was expressing some amazement and wonder at her good fortune.
	\item Cross shook his head in wonder.
	\end{itemize}
}
\item countable noun \\
A \textbf{wonder} is something that causes people to feel great surprise or admiration .
 \textit{
	\begin{itemize}
	\item ...a lecture on the wonders of space and space exploration.
	\item ...the wonder of seeing his name in print.
	\item The East Window is a wonder of medieval glazing.
	\end{itemize}
}
\item adjective \\
If you refer , for example, to a young  man as a \textbf{wonder}  boy , or to a new  product as a \textbf{wonder}  drug , you mean that they are believed by many people to be very good or very effective .
 \textit{
	\begin{itemize}
	\item He was hailed as the wonder boy of American golf.
	\item Dr Williams describes it as a potential wonder drug.
	\end{itemize}
}
\item  \\
 nine/seven/ten-day wonder \textit{
	\begin{itemize}
	\end{itemize}
}
\item  \\
 I wonder \textit{
	\begin{itemize}
	\end{itemize}
}
\item  \\
 no/little wonder \textit{
	\begin{itemize}
	\end{itemize}
}
\item  \\
 no wonder \textit{
	\begin{itemize}
	\end{itemize}
}
\item  \\
 work/do wonders \textit{
	\begin{itemize}
	\end{itemize}
}
\end{enumerate}

\section*{asset}
{\large \color{blue}  assets  }
\subsection*{Explain}
\begin{enumerate}
\item countable noun \\
Something or someone that is an \textbf{asset} is considered useful or helps a person or organization to be successful .
 \textit{
	\begin{itemize}
	\item He considered that the greatest asset of a bank was its reputation for probity.
	\item His Republican credentials made him an asset.
	\end{itemize}
}
\item plural noun \\
The \textbf{assets} of a company or a person are all the things that they own.
 \textit{
	\begin{itemize}
	\item By the end of 1989 the group had assets of 3.5 billion francs.
	\end{itemize}
}
\end{enumerate}

\section*{aid}
{\large \color{blue}  aids  aiding  aided  }
\subsection*{Explain}
\begin{enumerate}
\item uncountable noun \\
\textbf{Aid} is money, equipment , or services that are provided for people, countries, or organizations who need them but cannot provide them for themselves.
 \textit{
	\begin{itemize}
	\item ...convoys delivering humanitarian aid to besieged or isolated communities.
	\item They have already pledged billions of dollars in aid.
	\item ...food aid convoys.
	\end{itemize}
}
\item verb \\
To \textbf{aid} a country, organization, or person means to provide them with money, equipment, or
services that they need.
 \textit{
	\begin{itemize}
	\item The ministry is working through international associations to aid the refugees.
	\item ...a charitable organization that has spent millions aiding pharmaceutical research.
	\end{itemize}
}
\item verb \\
To \textbf{aid} someone means to help or assist them.
 \textbf{Aid} is also a noun .
 \textit{
	\begin{itemize}
	\item ...a software system to aid managers in advanced decision-making.
	\item The hunt for her killer will continue, with police aided by the army and air force.
	\item He was forced to turn for aid to his former enemy.
	\end{itemize}
}
\item uncountable noun \\
If you perform a task  \textbf{with the}  \textbf{aid}  \textbf{of} something, you need or use that thing to perform that task.
 \textit{
	\begin{itemize}
	\item He succeeded with the aid of a completely new method he discovered.
	\item Gently raise your upper body to a sitting position, without the aid of your hands.
	\end{itemize}
}
\item countable noun \\
An \textbf{aid} is an object, device, or technique that makes something easier to do.
 \textit{
	\begin{itemize}
	\item The new law gives authorities a responsibility to provide aids to people with disabilities.
	\item The book is an invaluable aid to teachers of literature.
	\item Colonel Hardy would like to see every tank with a computerized aid.
	\end{itemize}
}
\item verb \\
If something \textbf{aids} a process, it makes it easier or more likely to happen .
 \textit{
	\begin{itemize}
	\item The export sector will aid the economic recovery.
	\item Calcium may aid in the prevention of colon cancer.
	\end{itemize}
}
\item  \\
 in aid of \textit{
	\begin{itemize}
	\end{itemize}
}
\item  \\
 come/go to someone's aid \textit{
	\begin{itemize}
	\end{itemize}
}
\end{enumerate}

\section*{bacterium}
{\large \color{blue}  }
\subsection*{Explain}
\begin{enumerate}
\item  \\
\textbf{Bacterium} is the singular of bacteria .
 \textit{
	\begin{itemize}
	\end{itemize}
}
\end{enumerate}

\section*{arrest}
{\large \color{blue}  arrests  arresting  arrested  }
\subsection*{Explain}
\begin{enumerate}
\item verb \\
If the police  \textbf{arrest} you, they take  charge of you and take you to a police station , because they believe you may have committed a crime .
 \textbf{Arrest} is also a noun .
 \textit{
	\begin{itemize}
	\item Police arrested five young men in connection with one of the attacks.
	\item The police say seven people were arrested for minor offences.
	\item The Police Department wasted no time in making an arrest.
	\item Murder squad detectives approached the man and placed him under arrest.
	\end{itemize}
}
\item verb \\
If something or someone \textbf{arrests} a process, they stop it continuing .
 \textit{
	\begin{itemize}
	\item A quarantine was put in place to arrest the spread of the disease.
	\item The law could arrest the development of good research if applied prematurely.
	\end{itemize}
}
\item verb \\
If something interesting or surprising  \textbf{arrests} your attention, you suddenly  notice it and then continue to look at it or consider it carefully.
 \textit{
	\begin{itemize}
	\item The work of an architect of genius always arrests the attention no matter how little
remains.
	\item As he reached the hall after her, he saw what had arrested her.
	\end{itemize}
}
\end{enumerate}

\section*{classic}
{\large \color{blue}  classics  }
\subsection*{Explain}
\begin{enumerate}
\item adjective \\
A \textbf{classic}  example of a thing or situation has all the features which you expect such a thing or situation to have.
 \textbf{Classic} is also a noun .
 \textit{
	\begin{itemize}
	\item The debate in the mainstream press has been a classic example of British hypocrisy.
	\item His first two goals were classic cases of being in the right place at the right time.
	\item It was a classic of interrogation: first the bully, then the kind one who offers
sympathy.
	\end{itemize}
}
\item adjective \\
A \textbf{classic} film, piece of writing , or piece of music is of very high quality and has become a standard against which
similar things are judged .
 \textbf{Classic} is also a noun.
 \textit{
	\begin{itemize}
	\item ...the classic children's film Huckleberry Finn.
	\item ...a classic study of the American penal system.
	\item The record won a gold award and remains one of the classics of modern popular music.
	\item ...a film classic.
	\end{itemize}
}
\item countable noun \\
A \textbf{classic} is a book which is well-known and considered to be of a high literary standard. You can refer to such books generally as \textbf{the classics} .
 \textit{
	\begin{itemize}
	\item As I grow older, I like to reread the classics regularly.
	\end{itemize}
}
\item adjective \\
\textbf{Classic} style is simple and traditional and is not affected by changes in fashion.
 \textit{
	\begin{itemize}
	\item Wear classic clothes which feel good and look good.
	\item These are classic designs which will fit in well anywhere.
	\end{itemize}
}
\item adjective \\
A \textbf{classic}  car is a model of car that is greatly admired because of its style, and is considered to be one of the best of its kind; used especially to talk about cars which are no longer being produced.
 \textit{
	\begin{itemize}
	\end{itemize}
}
\item uncountable noun \\
\textbf{Classics} is the study of the ancient Greek and Roman civilizations , especially their languages, literature, and philosophy .
 \textit{
	\begin{itemize}
	\item ...a Classics degree.
	\end{itemize}
}
\end{enumerate}

\section*{association}
{\large \color{blue}  associations  }
\subsection*{Explain}
\begin{enumerate}
\item countable noun \\
An \textbf{association} is an official group of people who have the same job , aim , or interest.
 \textit{
	\begin{itemize}
	\item ...the Association of British Travel Agents.
	\item Research associations are often linked to a particular industry.
	\end{itemize}
}
\item countable noun \\
Your \textbf{association}  \textbf{with} a person or a thing such as an organization is the connection that you have with
them.
 \textit{
	\begin{itemize}
	\item ...the company's six-year association with retailer J.C. Penney Co.
	\item Blyth's association with the sea began in 1966.
	\item The association between the two companies stretches back thirty years.
	\end{itemize}
}
\item countable noun \\
If something has particular \textbf{associations} for you, it is connected in your mind with a particular memory , idea, or feeling.
 \textit{
	\begin{itemize}
	\item He has a shelf full of things, each of which has associations for him.
	\item Black was considered inappropriate because of its associations with death.
	\end{itemize}
}
\item  \\
 in association with \textit{
	\begin{itemize}
	\end{itemize}
}
\item  \\
 in association with \textit{
	\begin{itemize}
	\end{itemize}
}
\end{enumerate}

\section*{commodity}
{\large \color{blue}  commodities  }
\subsection*{Explain}
\begin{enumerate}
\item countable noun \\
A \textbf{commodity} is something that is sold for money.
 \textit{
	\begin{itemize}
	\item The government increased prices on several basic commodities like bread and meat.
	\end{itemize}
}
\end{enumerate}

\section*{bandage}
{\large \color{blue}  bandages  bandaging  bandaged  }
\subsection*{Explain}
\begin{enumerate}
\item countable noun \\
A \textbf{bandage} is a long strip of cloth which is wrapped around a wounded part of someone's body to protect or support it.
 \textit{
	\begin{itemize}
	\item We put some ointment and a bandage on his knee.
	\item His chest was swathed in bandages.
	\end{itemize}
}
\item verb \\
If you \textbf{bandage} a wound or part of someone's body, you tie a bandage around it.
 \textbf{Bandage up} means the same as bandage .
 \textit{
	\begin{itemize}
	\item Apply a dressing to the wound and bandage it.
	\item ...a bandaged hand.
	\item I bandaged the leg up and gave her aspirin for the pain.
	\end{itemize}
}
\end{enumerate}

\section*{compliment}
{\large \color{blue}  compliments  complimenting  complimented  }
\subsection*{Explain}
\begin{enumerate}
\item countable noun \\
A \textbf{compliment} is a polite remark that you say to someone to show that you like their appearance , appreciate their qualities , or approve of what they have done .
 \textit{
	\begin{itemize}
	\item He has never paid me a compliment.
	\item I try to graciously accept both compliments and criticism.
	\end{itemize}
}
\item verb \\
If you \textbf{compliment} someone, you pay them a compliment.
 \textit{
	\begin{itemize}
	\item They complimented me on the way I looked each time they saw me.
	\item Firstly I compliment you on most of your excellent Spring issue of 'Triangle'.
	\end{itemize}
}
\item countable noun \\
If you consider something that a person says or does as a \textbf{compliment} , it convinces you of your own good qualities, or that the person appreciates you.
 \textit{
	\begin{itemize}
	\item We consider it a compliment to be called 'conservative'.
	\item It's obvious he's worried about us and I'm taking it as a compliment.
	\end{itemize}
}
\item plural noun \\
You can  refer to your \textbf{compliments} when you want to express thanks , good wishes , or respect to someone in a formal  way .
 \textit{
	\begin{itemize}
	\item My compliments to the chef.
	\item Our compliments to all involved.
	\end{itemize}
}
\item  \\
 return the compliment \textit{
	\begin{itemize}
	\end{itemize}
}
\item  \\
 with sb's compliments \textit{
	\begin{itemize}
	\end{itemize}
}
\end{enumerate}

\section*{butcher}
{\large \color{blue}  butchers  butchering  butchered  }
\subsection*{Explain}
\begin{enumerate}
\item countable noun \\
A \textbf{butcher} is a shopkeeper who cuts up and sells meat. Some butchers also kill animals for meat and make foods such as sausages and meat pies .
 \textit{
	\begin{itemize}
	\end{itemize}
}
\item countable noun \\
A \textbf{butcher} or a \textbf{butcher's} is a shop where meat is sold.
 \textit{
	\begin{itemize}
	\end{itemize}
}
\item verb \\
To \textbf{butcher} an animal means to kill it and cut it up for meat.
 \textit{
	\begin{itemize}
	\item All his meat is butchered on site before being sold in the farm shop.
	\end{itemize}
}
\item countable noun \\
You can  refer to someone as a \textbf{butcher} when they have killed a lot of people in a very cruel  way , and you want to express your horror and disgust .
 \textit{
	\begin{itemize}
	\item ...the Duke of Cumberland, infamous still as the butcher of Culloden.
	\end{itemize}
}
\item verb \\
You can say that someone \textbf{has butchered} people when they have killed a lot of people in a very cruel way, and you want to
express your horror and disgust.
 \textit{
	\begin{itemize}
	\item Guards butchered 1,350 prisoners.
	\item Our people are being butchered in their homes.
	\end{itemize}
}
\end{enumerate}

\section*{confident}
{\large \color{blue}  }
\subsection*{Explain}
\begin{enumerate}
\item adjective \\
If you are \textbf{confident} about something, you are certain that it will  happen in the way you want it to.
 \textit{
	\begin{itemize}
	\item I am confident that everything will come out right in time.
	\item Mr Ryan is confident of success.
	\item Management is confident about the way business is progressing.
	\end{itemize}
}
\item adjective \\
If a person or their manner is \textbf{confident} , they feel sure about their own abilities , qualities, or ideas .
 \textit{
	\begin{itemize}
	\item In time he became more confident and relaxed.
	\item She is a confident woman who is certain of her views.
	\end{itemize}
}
\item adjective \\
If you are \textbf{confident}  \textbf{that} something is true , you are sure that it is true. A \textbf{confident}  statement is one that the speaker is sure is true.
 \textit{
	\begin{itemize}
	\item She is confident that everybody is on her side.
	\item 'Bet you I can', comes the confident reply.
	\end{itemize}
}
\end{enumerate}

\section*{comedy}
{\large \color{blue}  comedies  }
\subsection*{Explain}
\begin{enumerate}
\item uncountable noun \\
\textbf{Comedy} consists of types of entertainment , such as plays and films, or particular scenes in them, that are intended to make people laugh .
 \textit{
	\begin{itemize}
	\item He dropped out of university in Manchester to pursue a career in comedy.
	\item ...a TV comedy series.
	\end{itemize}
}
\item countable noun \\
A \textbf{comedy} is a play, film, or television  programme that is intended to make people laugh.
 \textit{
	\begin{itemize}
	\end{itemize}
}
\item uncountable noun \\
The \textbf{comedy} of a situation  involves those aspects of it that make you laugh.
 \textit{
	\begin{itemize}
	\item Jackie sees the comedy in her millionaire husband's thrifty habits.
	\end{itemize}
}
\end{enumerate}

\section*{deceit}
{\large \color{blue}  deceits  }
\subsection*{Explain}
\begin{enumerate}
\item variable noun \\
\textbf{Deceit} is behaviour that is deliberately intended to make people believe something which is not true .
 \textit{
	\begin{itemize}
	\item They have been involved in a campaign of deceit.
	\end{itemize}
}
\end{enumerate}

\section*{gaol}
{\large \color{blue}  gaols  gaoling  gaoled  }
\subsection*{Explain}
\begin{enumerate}
\end{enumerate}

\section*{density}
{\large \color{blue}  densities  }
\subsection*{Explain}
\begin{enumerate}
\item variable noun \\
\textbf{Density} is the extent to which something is filled or covered with people or things.
 \textit{
	\begin{itemize}
	\item ...a law which restricts the density of housing.
	\item The region has a very high population density.
	\item ...areas with high densities of immigrant populations.
	\end{itemize}
}
\item variable noun \\
In science , the \textbf{density} of a substance or object is the relation of its mass or weight to its volume.
 \textit{
	\begin{itemize}
	\item Jupiter's moon Io, whose density is 3.5 grams per cubic centimetre, is all rock.
	\end{itemize}
}
\end{enumerate}

\section*{goods}
{\large \color{blue}  }
\subsection*{Explain}
\begin{enumerate}
\item plural noun \\
\textbf{Goods} are things that are made to be sold .
 \textit{
	\begin{itemize}
	\item Money can be exchanged for goods or services.
	\item ...a wide range of consumer goods.
	\end{itemize}
}
\item plural noun \\
Your \textbf{goods} are the things that you own and that can be moved.
 \textit{
	\begin{itemize}
	\item All his worldly goods were packed into a neat checked carrier bag.
	\item You can give your unwanted goods to charity.
	\end{itemize}
}
\item  \\
 come up with the goods \textit{
	\begin{itemize}
	\end{itemize}
}
\item  \\
 get/have the goods on \textit{
	\begin{itemize}
	\end{itemize}
}
\end{enumerate}

\section*{doctorate}
{\large \color{blue}  doctorates  }
\subsection*{Explain}
\begin{enumerate}
\item countable noun \\
A \textbf{doctorate} is the highest degree awarded by a university.
 \textit{
	\begin{itemize}
	\item He obtained his doctorate in Social Psychology.
	\end{itemize}
}
\end{enumerate}

\section*{implement}
{\large \color{blue}  implements  implementing  implemented  }
\subsection*{Explain}
\begin{enumerate}
\item verb \\
If you \textbf{implement} something such as a plan , you ensure that what has been planned is done.
 \textit{
	\begin{itemize}
	\item The government promised to implement a new system to control financial loan institutions.
	\item If such measures were implemented, the problems could be overcome in twelve months.
	\end{itemize}
}
\item countable noun \\
An \textbf{implement} is a tool or other piece of equipment.
 \textit{
	\begin{itemize}
	\item ...knives and other useful implements.
	\item ...writing implements.
	\end{itemize}
}
\end{enumerate}

\section*{egg}
{\large \color{blue}  eggs  egging  egged  }
\subsection*{Explain}
\begin{enumerate}
\item countable noun \\
An \textbf{egg} is an oval object that is produced by a female bird and which contains a baby bird. Other animals such as reptiles and fish also lay eggs.
 \textit{
	\begin{itemize}
	\item ...a baby bird hatching from its egg.
	\item ...ant eggs.
	\end{itemize}
}
\item variable noun \\
In Western countries, \textbf{eggs} often means hen's eggs, eaten as food.
 \textit{
	\begin{itemize}
	\item Break the eggs into a shallow bowl and beat them lightly.
	\item ...bacon and eggs.
	\end{itemize}
}
\item countable noun \\
\textbf{Egg} is used to refer to an object in the shape of a hen's egg.
 \textit{
	\begin{itemize}
	\item ...a chocolate egg.
	\end{itemize}
}
\item countable noun \\
An \textbf{egg} is a cell that is produced in the bodies of female animals and humans. If it is fertilized by a sperm , a baby develops from it.
 \textit{
	\begin{itemize}
	\item It only takes one sperm to fertilize an egg.
	\end{itemize}
}
\item  \\
 to put all your eggs in one basket \textit{
	\begin{itemize}
	\end{itemize}
}
\item  \\
 to have egg on your face \textit{
	\begin{itemize}
	\end{itemize}
}
\end{enumerate}

\section*{influence}
{\large \color{blue}  influences  influencing  influenced  }
\subsection*{Explain}
\begin{enumerate}
\item uncountable noun \\
\textbf{Influence} is the power to make other people agree with your opinions or do what you want .
 \textit{
	\begin{itemize}
	\item I have rather a large influence over a good many people.
	\item He denies exerting any political influence over them.
	\item The government should continue to use its influence for the release of all hostages.
	\end{itemize}
}
\item verb \\
If you \textbf{influence} someone, you use your power to make them agree with you or do what you want.
 \textit{
	\begin{itemize}
	\item He is trying to improperly influence a witness.
	\item The angry crowds could influence the government.
	\item My dad influenced me to do electronics.
	\end{itemize}
}
\item countable noun \\
To have an \textbf{influence}  \textbf{on} people or situations  means to affect what they do or what happens .
 \textit{
	\begin{itemize}
	\item Van Gogh had a major influence on the development of modern painting.
	\item The Shropshire landscape was an influence on Owen too.
	\item Many other medications have an influence on cholesterol levels.
	\end{itemize}
}
\item verb \\
If someone or something \textbf{influences} a person or situation, they have an effect on that person's behaviour or that situation.
 \textit{
	\begin{itemize}
	\item We became the best of friends and he influenced me deeply.
	\item What you eat may influence your risk of getting cancer.
	\item They still influence what's played on the radio.
	\end{itemize}
}
\item countable noun \\
Someone or something that is a good or bad  \textbf{influence}  \textbf{on} people has a good or bad effect on them.
 \textit{
	\begin{itemize}
	\item I thought Sue would be a good influence on you.
	\end{itemize}
}
\item  \\
 under the influence of \textit{
	\begin{itemize}
	\end{itemize}
}
\item  \\
 under the influence \textit{
	\begin{itemize}
	\end{itemize}
}
\end{enumerate}

\section*{fraud}
{\large \color{blue}  frauds  }
\subsection*{Explain}
\begin{enumerate}
\item variable noun \\
\textbf{Fraud} is the crime of gaining money or financial  benefits by a trick or by lying .
 \textit{
	\begin{itemize}
	\item He was jailed for two years for fraud and deception.
	\item Tax frauds are dealt with by HMRC.
	\end{itemize}
}
\item countable noun \\
A \textbf{fraud} is something or someone that deceives people in a way that is illegal or dishonest .
 \textit{
	\begin{itemize}
	\item Unfortunately the portraits were frauds.
	\item He believes many 'psychics' are frauds who rely on perception and subtle deception.
	\end{itemize}
}
\item countable noun \\
If you call someone or something a \textbf{fraud} , you are criticizing them because you think that they are not genuine , or are less good than they claim or appear to be.
 \textit{
	\begin{itemize}
	\item I suspect her of being a fashion fraud, pretending to care about it more than she
really does.
	\end{itemize}
}
\end{enumerate}

\section*{jewelry}
{\large \color{blue}  }
\subsection*{Explain}
\begin{enumerate}
\item noun \\
 ornaments such as rings, brooches, bracelets , etc., collectively
 \textit{
	\begin{itemize}
	\end{itemize}
}
\end{enumerate}

\section*{grape}
{\large \color{blue}  grapes  }
\subsection*{Explain}
\begin{enumerate}
\item countable noun \\
\textbf{Grapes} are small green or dark purple fruit which grow in bunches . Grapes can be eaten raw, used for making wine, or dried.
 \textit{
	\begin{itemize}
	\end{itemize}
}
\item  \\
 sour grapes \textit{
	\begin{itemize}
	\end{itemize}
}
\end{enumerate}

\section*{maintenance}
{\large \color{blue}  }
\subsection*{Explain}
\begin{enumerate}
\item uncountable noun \\
The \textbf{maintenance} of a building, vehicle , road , or machine is the process of keeping it in good condition by regularly checking it and repairing it when necessary .
 \textit{
	\begin{itemize}
	\item ...maintenance work on government buildings.
	\item The window had been replaced last week during routine maintenance.
	\item ...car maintenance lessons.
	\end{itemize}
}
\item uncountable noun \\
\textbf{Maintenance} is money that someone gives regularly to another person to pay for the things that the person needs .
 \textit{
	\begin{itemize}
	\item ...the government's plan to make absent fathers pay maintenance for their children.
	\end{itemize}
}
\item uncountable noun \\
If you ensure the \textbf{maintenance}  \textbf{of} a state or process, you make sure that it continues.
 \textit{
	\begin{itemize}
	\item ...the maintenance of peace and stability in Asia.
	\item ...the importance of natural food to the maintenance of health.
	\end{itemize}
}
\end{enumerate}

\section*{metaphor}
{\large \color{blue}  metaphors  }
\subsection*{Explain}
\begin{enumerate}
\item variable noun \\
A \textbf{metaphor} is an imaginative way of describing something by referring to something else which is the same in a particular way. For example, if you want to say that someone is very shy and frightened of things, you might say that they are a mouse .
 \textit{
	\begin{itemize}
	\item ...the avoidance of 'violent expressions and metaphors' like 'kill two birds with
one stone'.
	\item ...the writer's use of metaphor.
	\end{itemize}
}
\item variable noun \\
If one thing is a \textbf{metaphor}  \textbf{for} another, it is intended or regarded as a symbol of it.
 \textit{
	\begin{itemize}
	\item The divided family remains a powerful metaphor for a society tearing itself apart.
	\end{itemize}
}
\item  \\
 to mix your metaphors \textit{
	\begin{itemize}
	\end{itemize}
}
\end{enumerate}

\section*{kilometre}
{\large \color{blue}  kilometres  }
\subsection*{Explain}
\begin{enumerate}
\item countable noun \\
A \textbf{kilometre} is a metric unit of distance or length. One kilometre is a thousand metres and is equal to 0.62 miles.
 \textit{
	\begin{itemize}
	\item ...about twenty kilometres from the border.
	\item The fire destroyed some 40,000 square kilometres of forest.
	\end{itemize}
}
\end{enumerate}

\section*{museum}
{\large \color{blue}  museums  }
\subsection*{Explain}
\begin{enumerate}
\item countable noun \\
A \textbf{museum} is a building where a large number of interesting and valuable objects, such as works of art or historical items , are kept , studied, and displayed to the public .
 \textit{
	\begin{itemize}
	\item For months Malcolm had wanted to visit the Parisian art museums.
	\item ...the American Museum of Natural History.
	\end{itemize}
}
\end{enumerate}

\section*{lab}
{\large \color{blue}  labs  }
\subsection*{Explain}
\begin{enumerate}
\item countable noun \\
A \textbf{lab} is the same as a laboratory .
 \textit{
	\begin{itemize}
	\end{itemize}
}
\item  \\
In Britain, \textbf{Lab} is the written abbreviation for labour .
 \textit{
	\begin{itemize}
	\item ...Diane Abbott (Lab , Hackney North and Stoke Newington).
	\end{itemize}
}
\end{enumerate}

\section*{neck}
{\large \color{blue}  necks  necking  necked  }
\subsection*{Explain}
\begin{enumerate}
\item countable noun \\
Your \textbf{neck} is the part of your body which joins your head to the rest of your body.
 \textit{
	\begin{itemize}
	\item She threw her arms round his neck and hugged him warmly.
	\item He was short and stocky, and had a thick neck.
	\end{itemize}
}
\item countable noun \\
The \textbf{neck} of an article of clothing such as a shirt , dress , or sweater is the part which surrounds your neck.
 \textit{
	\begin{itemize}
	\item ...the low, ruffled neck of her blouse.
	\item He wore a blue shirt open at the neck.
	\end{itemize}
}
\item countable noun \\
The \textbf{neck} of something such as a bottle or a guitar is the long narrow part at one end of it.
 \textit{
	\begin{itemize}
	\item Catherine gripped the broken neck of the bottle.
	\item ...cancer of the neck of the womb.
	\end{itemize}
}
\item verb \\
If two people \textbf{are necking} , they are kissing each other in a sexual way.
 \textit{
	\begin{itemize}
	\item They sat talking and necking in the car for another ten minutes.
	\item I found myself behind a curtain, necking with my best friend.
	\end{itemize}
}
\item singular noun \\
If a horse wins a race \textbf{by a neck} , it wins by a very small distance.
 \textit{
	\begin{itemize}
	\item Four of the seven races were won by a neck or less.
	\end{itemize}
}
\item  \\
 to be breathing down someone's neck \textit{
	\begin{itemize}
	\end{itemize}
}
\item  \\
 neck and neck \textit{
	\begin{itemize}
	\end{itemize}
}
\item  \\
 to risk your neck \textit{
	\begin{itemize}
	\end{itemize}
}
\item  \\
 save one's own neck/save sb's neck \textit{
	\begin{itemize}
	\end{itemize}
}
\item  \\
 to stick your neck out \textit{
	\begin{itemize}
	\end{itemize}
}
\item  \\
 round one's neck/around one's neck \textit{
	\begin{itemize}
	\end{itemize}
}
\item  \\
 up to one's neck \textit{
	\begin{itemize}
	\end{itemize}
}
\item  \\
 your neck of the woods \textit{
	\begin{itemize}
	\end{itemize}
}
\end{enumerate}

\section*{labor}
{\large \color{blue}  }
\subsection*{Explain}
\begin{enumerate}
\end{enumerate}

\section*{nickname}
{\large \color{blue}  nicknames  nicknaming  nicknamed  }
\subsection*{Explain}
\begin{enumerate}
\item countable noun \\
A \textbf{nickname} is an informal name for someone or something.
 \textit{
	\begin{itemize}
	\item Red got his nickname for his red hair.
	\end{itemize}
}
\item verb \\
If you \textbf{nickname} someone or something, you give them an informal name.
 \textit{
	\begin{itemize}
	\item When he got older I nicknamed him Little Alf.
	\item Which newspaper was once nicknamed The Thunderer?
	\end{itemize}
}
\end{enumerate}

\section*{license}
{\large \color{blue}  licenses  licensing  licensed  }
\subsection*{Explain}
\begin{enumerate}
\item verb \\
To \textbf{license} a person or activity means to give official permission for the person to do something or for the activity to take place.
 \textit{
	\begin{itemize}
	\item ...a proposal that would require the state to license guns.
	\item Under the agreement, the council can license a U.S. company to produce the drug.
	\end{itemize}
}
\end{enumerate}

\section*{periodical}
{\large \color{blue}  periodicals  }
\subsection*{Explain}
\begin{enumerate}
\item countable noun \\
\textbf{Periodicals} are magazines , especially  serious or academic ones, that are published at regular intervals.
 \textit{
	\begin{itemize}
	\item The walls would be lined with books and periodicals.
	\item ...a monthly periodical.
	\end{itemize}
}
\item adjective \\
\textbf{Periodical}  events or situations  happen  occasionally , at fairly regular intervals.
 \textit{
	\begin{itemize}
	\item She made periodical visits to her dentist.
	\item ...periodical screening for cancer.
	\end{itemize}
}
\end{enumerate}

\section*{litre}
{\large \color{blue}  litres  }
\subsection*{Explain}
\begin{enumerate}
\item countable noun \\
A \textbf{litre} is a metric unit of volume that is a thousand cubic centimetres . It is equal to 1.76 British pints or 2.11 American pints.
 \textit{
	\begin{itemize}
	\item ...15 litres of water.
	\item This tax would raise petrol prices by about 3.5p per litre.
	\item ...a Ford Escort with a 1.9-litre engine.
	\end{itemize}
}
\end{enumerate}

\section*{proposal}
{\large \color{blue}  proposals  }
\subsection*{Explain}
\begin{enumerate}
\item countable noun \\
A \textbf{proposal} is a plan or an idea , often a formal or written one, which is suggested for people to think about and decide upon.
 \textit{
	\begin{itemize}
	\item The President is to put forward new proposals for resolving the country's constitutional
crisis.
	\item ...the government's proposals to abolish free health care.
	\item The Security Council has rejected the latest peace proposal.
	\end{itemize}
}
\item countable noun \\
A \textbf{proposal} is the act of asking someone to marry you.
 \textit{
	\begin{itemize}
	\item After a three-weekend courtship, Pamela accepted Randolph's proposal of marriage.
	\end{itemize}
}
\end{enumerate}

\section*{lord}
{\large \color{blue}  lords  lording  lorded  }
\subsection*{Explain}
\begin{enumerate}
\item countable noun \\
In Britain, a \textbf{lord} is a man who has a high rank in the nobility, for example an earl, a viscount, or
a marquis .
 \textit{
	\begin{itemize}
	\item She married a lord and lives in this huge house in the Cotswolds.
	\item A few days earlier he had received a telegram from Lord Lloyd.
	\end{itemize}
}
\item countable noun \\
In Britain, judges , bishops, and some male members of the nobility are addressed as ' \textbf{my Lord} '.
 \textit{
	\begin{itemize}
	\item My lord, I am instructed by my client to claim that the evidence has been tampered
with.
	\end{itemize}
}
\item  \\
In Britain, \textbf{Lord} is used in the titles of some officials of very high rank.
 \textit{
	\begin{itemize}
	\item He was Lord Chancellor from 1970 until 1974.
	\item ...the head of the judiciary, the Lord Chief Justice.
	\end{itemize}
}
\item proper noun \\
\textbf{The Lords} is the same as \textbf{the}  House of Lords .
 \textit{
	\begin{itemize}
	\item It's very likely the bill will be defeated in the Lords.
	\end{itemize}
}
\item countable noun \\
In former times, especially in medieval times, a \textbf{lord} was a man who owned land or property and who had power and authority over people.
 \textit{
	\begin{itemize}
	\item It was the home of the powerful lords of Baux.
	\end{itemize}
}
\item proper noun \\
In the Christian church, people refer to God and to Jesus Christ as the \textbf{Lord} .
 \textit{
	\begin{itemize}
	\item I know the Lord will look after him.
	\item She prayed now. 'Lord, help me to find courage.'
	\item ...the birth of the Lord Jesus Christ.
	\end{itemize}
}
\item countable noun \\
If you describe a man as the \textbf{lord}  \textbf{of} a particular area, industry, or thing, you mean that he has total authority and power over it.
 \textit{
	\begin{itemize}
	\item A century ago the aristocracy were truly lords of the earth.
	\item ...the lords of the black market.
	\end{itemize}
}
\item  \\
 good lord \textit{
	\begin{itemize}
	\end{itemize}
}
\item  \\
 lord knows \textit{
	\begin{itemize}
	\end{itemize}
}
\item  \\
 lord knows \textit{
	\begin{itemize}
	\end{itemize}
}
\item  \\
 lord it over sb \textit{
	\begin{itemize}
	\end{itemize}
}
\end{enumerate}

\section*{prose}
{\large \color{blue}  }
\subsection*{Explain}
\begin{enumerate}
\item uncountable noun \\
\textbf{Prose} is ordinary written language, in contrast to poetry.
 \textit{
	\begin{itemize}
	\item Shute's prose is stark and chillingly unsentimental.
	\item What he has to say is expressed in prose of exceptional lucidity and grace.
	\end{itemize}
}
\end{enumerate}

\section*{madame}
{\large \color{blue}  }
\subsection*{Explain}
\begin{enumerate}
\item noun \\
a married Frenchwoman: usually used as a title  equivalent to Mrs, and sometimes  extended to older unmarried women to show  respect and to women of other nationalities \textit{
	\begin{itemize}
	\end{itemize}
}
\end{enumerate}

\section*{protein}
{\large \color{blue}  proteins  }
\subsection*{Explain}
\begin{enumerate}
\item variable noun \\
\textbf{Protein} is a substance found in food and drink such as meat , eggs , and milk . You need protein in order to grow and be healthy .
 \textit{
	\begin{itemize}
	\item Fish was a major source of protein for the working man.
	\item ...a high protein diet.
	\end{itemize}
}
\end{enumerate}

\section*{maneuver}
{\large \color{blue}  }
\subsection*{Explain}
\begin{enumerate}
\end{enumerate}

\section*{reference}
{\large \color{blue}  references  referencing  referenced  }
\subsection*{Explain}
\begin{enumerate}
\item variable noun \\
\textbf{Reference}  \textbf{to} someone or something is the act of talking about them or mentioning them. A \textbf{reference} is a particular example of this.
 \textit{
	\begin{itemize}
	\item He made no reference to any agreement.
	\item ...a reference to a fictitious voyage by the buccaneer John Coxton.
	\item He summed up his philosophy, with reference to Calvin.
	\end{itemize}
}
\item uncountable noun \\
\textbf{Reference} is the act of consulting someone or something in order to get information or advice .
 \textit{
	\begin{itemize}
	\item This might be done without reference to Parliament.
	\item Please keep this sheet in a safe place for reference.
	\end{itemize}
}
\item adjective \\
\textbf{Reference} books are ones that you look at when you need specific information or facts about a subject.
 \textit{
	\begin{itemize}
	\item There are several reference books which have been compiled to help you make your
choice.
	\item ...a useful reference work for teachers.
	\end{itemize}
}
\item countable noun \\
A \textbf{reference} is a word, phrase, or idea which comes from something such as a book, poem , or play and which you use when making a point about something.
 \textit{
	\begin{itemize}
	\item ...a reference from the Quran.
	\item ...historical references.
	\end{itemize}
}
\item countable noun \\
A \textbf{reference} is something such as a number or a name that tells you where you can obtain the information you want .
 \textit{
	\begin{itemize}
	\item ...a map reference.
	\item Make a note of the reference number shown on the form.
	\end{itemize}
}
\item countable noun \\
A \textbf{reference} is a letter that is written by someone who knows you and which describes your character and abilities. When you apply for a job , an employer  might  ask for \textbf{references} .
 \textit{
	\begin{itemize}
	\item The firm offered to give her a reference.
	\end{itemize}
}
\item countable noun \\
A \textbf{reference} is a person who gives you a reference, for example when you are applying for a job.
 \textit{
	\begin{itemize}
	\end{itemize}
}
\item verb \\
If you \textbf{reference} a particular book or writer, you make a precise reference to them in what you are saying or writing.
 \textit{
	\begin{itemize}
	\item His final scene is frequently referenced as one of the most memorable and frightening
in cinema history.
	\end{itemize}
}
\item  \\
 for future reference \textit{
	\begin{itemize}
	\end{itemize}
}
\item  \\
 with/in reference to \textit{
	\begin{itemize}
	\end{itemize}
}
\end{enumerate}

\section*{maths}
{\large \color{blue}  }
\subsection*{Explain}
\begin{enumerate}
\item uncountable noun \\
\textbf{Maths} is the same as mathematics .
 \textit{
	\begin{itemize}
	\item He taught science and maths.
	\end{itemize}
}
\end{enumerate}

\section*{revelation}
{\large \color{blue}  revelations  }
\subsection*{Explain}
\begin{enumerate}
\item countable noun \\
A \textbf{revelation} is a surprising or interesting fact that is made known to people.
 \textit{
	\begin{itemize}
	\item ...the seemingly everlasting revelations about his private life.
	\item ...the revelation that William had survived the initial attack.
	\end{itemize}
}
\item variable noun \\
The \textbf{revelation}  \textbf{of} something is the act of making it known.
 \textit{
	\begin{itemize}
	\item ...following the revelation of his affair with a former secretary.
	\item Further revelations are expected.
	\end{itemize}
}
\item singular noun \\
If you say that something you experienced was \textbf{a}  \textbf{revelation} , you are saying that it was very surprising or very good.
 \textit{
	\begin{itemize}
	\item The noise, the buildings, the people, came as a revelation.
	\item Degas's work had been a revelation to her.
	\end{itemize}
}
\item variable noun \\
A divine \textbf{revelation} is a sign or explanation from God about his nature or purpose.
 \textit{
	\begin{itemize}
	\item The whole system was based on divine revelation in the Scriptures.
	\end{itemize}
}
\end{enumerate}

\section*{meantime}
{\large \color{blue}  }
\subsection*{Explain}
\begin{enumerate}
\item  \\
 (in the) meantime \textit{
	\begin{itemize}
	\end{itemize}
}
\item  \\
 for the meantime \textit{
	\begin{itemize}
	\end{itemize}
}
\end{enumerate}

\section*{riddle}
{\large \color{blue}  riddles  riddling  riddled  }
\subsection*{Explain}
\begin{enumerate}
\item countable noun \\
A \textbf{riddle} is a puzzle or joke in which you ask a question that seems to be nonsense but which has a clever or amusing answer.
 \textit{
	\begin{itemize}
	\end{itemize}
}
\item countable noun \\
You can describe something as a \textbf{riddle} if people have been trying to understand or explain it but have not been able to.
 \textit{
	\begin{itemize}
	\item Hawking's equation is a clue to the riddle of black holes.
	\end{itemize}
}
\item verb \\
If someone \textbf{riddles} something \textbf{with}  bullets or bullet holes, they fire a lot of bullets into it.
 \textit{
	\begin{itemize}
	\item Unknown attackers riddled two homes with gunfire.
	\item The darkness saved me from being riddled with bullets.
	\end{itemize}
}
\end{enumerate}

\section*{memo}
{\large \color{blue}  memos  }
\subsection*{Explain}
\begin{enumerate}
\item countable noun \\
A \textbf{memo} is a short official note that is sent by one person to another within the same company or organization.
 \textit{
	\begin{itemize}
	\end{itemize}
}
\end{enumerate}

\section*{rope}
{\large \color{blue}  ropes  roping  roped  }
\subsection*{Explain}
\begin{enumerate}
\item variable noun \\
A \textbf{rope} is a thick cord or wire that is made by twisting together several thinner cords or wires. Ropes are used for jobs such as pulling  cars , tying up boats , or tying things together.
 \textit{
	\begin{itemize}
	\item He tied the rope around his waist.
	\item ...a climbing rope.
	\item ...a piece of rope.
	\end{itemize}
}
\item verb \\
If you \textbf{rope} one thing \textbf{to} another, you tie the two things together with a rope.
 \textit{
	\begin{itemize}
	\item I roped myself to the chimney.
	\end{itemize}
}
\item plural noun \\
\textbf{The ropes}  refers to the fence made of rope that surrounds a boxing  ring or a wrestling ring.
 \textit{
	\begin{itemize}
	\item He was knocked through the ropes by Tafer.
	\end{itemize}
}
\item  \\
 give sb enough rope to hang \textit{
	\begin{itemize}
	\end{itemize}
}
\item  \\
 to learn the ropes \textit{
	\begin{itemize}
	\end{itemize}
}
\item  \\
 to know the ropes \textit{
	\begin{itemize}
	\end{itemize}
}
\item  \\
 money for old rope \textit{
	\begin{itemize}
	\end{itemize}
}
\item  \\
 on the ropes \textit{
	\begin{itemize}
	\end{itemize}
}
\item  \\
 show sb the ropes \textit{
	\begin{itemize}
	\end{itemize}
}
\end{enumerate}

\section*{metre}
{\large \color{blue}  metres  }
\subsection*{Explain}
\begin{enumerate}
\item countable noun \\
A \textbf{metre} is a metric unit of length equal to 100 centimetres .
 \textit{
	\begin{itemize}
	\item She set a world record in the 100 metre sprint at her national championships.
	\item The tunnel is 10 metres wide and 600 metres long.
	\end{itemize}
}
\item variable noun \\
In the study of poetry , \textbf{metre} is the regular and rhythmic arrangement of syllables according to particular patterns .
 \textit{
	\begin{itemize}
	\item They must each compose a poem in strict alliterative metre.
	\item All of the poems are written in traditional metres and rhyme schemes.
	\end{itemize}
}
\end{enumerate}

\section*{silence}
{\large \color{blue}  silences  silencing  silenced  }
\subsection*{Explain}
\begin{enumerate}
\item variable noun \\
If there is \textbf{silence} , nobody is speaking.
 \textit{
	\begin{itemize}
	\item They stood in silence.
	\item He never lets those long silences develop during dinner.
	\item Then he bellowed 'Silence!'
	\end{itemize}
}
\item uncountable noun \\
\textbf{The}  \textbf{silence}  \textbf{of} a place is the extreme quietness there.
 \textit{
	\begin{itemize}
	\item ...the silence of that rainless, all-concealing fog.
	\item She breathed deeply, savouring the silence.
	\end{itemize}
}
\item uncountable noun \\
Someone's \textbf{silence} about something is their failure or refusal to speak to other people about it.
 \textit{
	\begin{itemize}
	\item The district court ruled that Popper's silence in court today should be entered as
a plea of not guilty.
	\end{itemize}
}
\item verb \\
To \textbf{silence} someone or something means to stop them speaking or making a noise.
 \textit{
	\begin{itemize}
	\item A ringing phone silenced her.
	\item The shock silenced him completely.
	\end{itemize}
}
\item verb \\
If someone \textbf{silences} you, they stop you expressing  opinions that they do not agree with.
 \textit{
	\begin{itemize}
	\item Like other tyrants, he tried to silence anyone who spoke out against him.
	\item ...an unsuccessful attempt by the government to silence the debate.
	\end{itemize}
}
\item verb \\
To \textbf{silence} someone means to kill them in order to stop them revealing something secret .
 \textit{
	\begin{itemize}
	\item A hit man had been sent to silence her over the affair.
	\end{itemize}
}
\end{enumerate}

\section*{millimeter}
{\large \color{blue}  }
\subsection*{Explain}
\begin{enumerate}
\item noun \\
one thousandth of a meter (0.03937 inch )
 \textit{
	\begin{itemize}
	\end{itemize}
}
\end{enumerate}

\section*{slaughter}
{\large \color{blue}  slaughters  slaughtering  slaughtered  }
\subsection*{Explain}
\begin{enumerate}
\item verb \\
If large numbers of people or animals \textbf{are slaughtered} , they are killed in a way that is cruel or unnecessary .
 \textbf{Slaughter} is also a noun .
 \textit{
	\begin{itemize}
	\item Thirty four people were slaughtered while queuing up to cast their votes.
	\item Whales and dolphins are still being slaughtered for commercial gain.
	\item ...a war where the slaughter of civilians was commonplace.
	\item The annual slaughter of wildlife in Italy is horrific.
	\end{itemize}
}
\item verb \\
To \textbf{slaughter} animals such as cows and sheep  means to kill them for their meat .
 \textbf{Slaughter} is also a noun.
 \textit{
	\begin{itemize}
	\item Lack of chicken feed means that chicken farms are having to slaughter their stock.
	\item More than 491,000 sheep were exported for slaughter last year.
	\end{itemize}
}
\end{enumerate}

\section*{supermarket}
{\large \color{blue}  supermarkets  }
\subsection*{Explain}
\begin{enumerate}
\item countable noun \\
A \textbf{supermarket} is a large shop which sells all kinds of food and some household goods .
 \textit{
	\begin{itemize}
	\item Most of us do our food shopping in the supermarket.
	\item How do those prawns find their way from Norway to the supermarket shelf?
	\end{itemize}
}
\end{enumerate}

\section*{neighbor}
{\large \color{blue}  }
\subsection*{Explain}
\begin{enumerate}
\item noun \\
1.  2.  3.  4.  \textit{
	\begin{itemize}
	\item love thy neighbor
	\end{itemize}
}
\item adjective \\
5.  \textit{
	\begin{itemize}
	\end{itemize}
}
\item verb transitive \\
6.  7.  \textit{
	\begin{itemize}
	\end{itemize}
}
\item verb intransitive \\
8.  9.  \textit{
	\begin{itemize}
	\end{itemize}
}
\end{enumerate}

\section*{syndrome}
{\large \color{blue}  syndromes  }
\subsection*{Explain}
\begin{enumerate}
\item countable noun \\
A \textbf{syndrome} is a medical condition that is characterized by a particular group of signs and symptoms.
 \textit{
	\begin{itemize}
	\item Irritable bowel syndrome seems to affect more women than men.
	\item The syndrome is more likely to strike those whose immune systems are already below
par.
	\end{itemize}
}
\item countable noun \\
You can refer to an undesirable condition that is characterized by a particular type of activity or behaviour as a \textbf{syndrome} .
 \textit{
	\begin{itemize}
	\item Avoid sweep-it-under-the-carpet syndrome where you ignore problems.
	\item Scientists call this the 'it won't affect me' syndrome.
	\end{itemize}
}
\end{enumerate}

\section*{tendency}
{\large \color{blue}  tendencies  }
\subsection*{Explain}
\begin{enumerate}
\item countable noun \\
A \textbf{tendency} is a typical or repeated  habit , action or belief .
 \textit{
	\begin{itemize}
	\item She has a tendency to glance around to see if there's someone more important to talk
to.
	\item ...the government's tendency towards secrecy in recent years.
	\item The war strengthened reformist tendencies in British trade unions.
	\end{itemize}
}
\item countable noun \\
A \textbf{tendency} is a part of your character that makes you behave in an unpleasant or worrying way.
 \textit{
	\begin{itemize}
	\item He is spoiled, arrogant and has a tendency towards snobbery.
	\item Helen had been struggling against suicidal tendencies.
	\end{itemize}
}
\end{enumerate}

\section*{odor}
{\large \color{blue}  }
\subsection*{Explain}
\begin{enumerate}
\end{enumerate}

\section*{tower}
{\large \color{blue}  towers  towering  towered  }
\subsection*{Explain}
\begin{enumerate}
\item countable noun \\
A \textbf{tower} is a tall, narrow building, that either stands  alone or forms part of another building such as a church or castle.
 \textit{
	\begin{itemize}
	\item ...an eleventh century castle with 120-foot high towers.
	\item ...the Leaning Tower of Pisa.
	\end{itemize}
}
\item verb \\
Someone or something that \textbf{towers}  \textbf{over}  surrounding people or things is a lot taller than they are.
 \textit{
	\begin{itemize}
	\item He stood up and towered over her.
	\item At school, a girl may tower over most boys her age.
	\item The icebergs towered above them.
	\end{itemize}
}
\item countable noun \\
A \textbf{tower} is a tall structure that is used for sending  radio or television  signals .
 \textit{
	\begin{itemize}
	\item Troops are still in control of the television and radio tower.
	\end{itemize}
}
\item countable noun \\
A \textbf{tower} is the same as a tower block .
 \textit{
	\begin{itemize}
	\item ...his design for a new office tower in Frankfurt.
	\end{itemize}
}
\item countable noun \\
A \textbf{tower} is a tall box that contains the main parts of a computer, such as the hard  disk and the drives .
 \textit{
	\begin{itemize}
	\end{itemize}
}
\item  \\
 tower of strength \textit{
	\begin{itemize}
	\end{itemize}
}
\end{enumerate}

\section*{organization}
{\large \color{blue}  organizations  }
\subsection*{Explain}
\begin{enumerate}
\item countable noun \\
An \textbf{organization} is an official group of people, for example a political party, a business, a charity , or a club .
 \textit{
	\begin{itemize}
	\item Most of these specialized schools are provided by voluntary organizations.
	\item ...a report by the International Labour Organisation.
	\end{itemize}
}
\item uncountable noun \\
The \textbf{organization} of an event or activity involves making all the necessary  arrangements for it.
 \textit{
	\begin{itemize}
	\item ...the exceptional attention to detail that goes into the organisation of this event.
	\item Several projects have been delayed by poor organisation.
	\end{itemize}
}
\item uncountable noun \\
The \textbf{organization}  \textbf{of} something is the way in which its different parts are arranged or relate to each other.
 \textit{
	\begin{itemize}
	\item I am aware that the organization of the book leaves something to be desired.
	\end{itemize}
}
\end{enumerate}

\section*{treason}
{\large \color{blue}  }
\subsection*{Explain}
\begin{enumerate}
\item uncountable noun \\
\textbf{Treason} is the crime of betraying your country, for example by helping its enemies or by trying to remove its government using violence .
 \textit{
	\begin{itemize}
	\end{itemize}
}
\end{enumerate}

\section*{percent}
{\large \color{blue}  }
\subsection*{Explain}
\begin{enumerate}
\item noun \\
percentage or proportion \textit{
	\begin{itemize}
	\end{itemize}
}
\end{enumerate}

\section*{union}
{\large \color{blue}  unions  }
\subsection*{Explain}
\begin{enumerate}
\item countable noun \\
A \textbf{union} is a workers ' organization which represents its members and which aims to improve things such as their working conditions and pay.
 \textit{
	\begin{itemize}
	\item I feel that women in all types of employment can benefit from joining a union.
	\item ...union officials.
	\end{itemize}
}
\item uncountable noun \\
When the \textbf{union} of two or more things occurs, they are joined together and become one thing.
 \textit{
	\begin{itemize}
	\item Long before union with England, Scotland had a vibrant musical tradition.
	\end{itemize}
}
\item singular noun \\
When two or more things, for example countries or organizations, have been joined together to form one thing, you can
 refer to them as a \textbf{union} .
 \textit{
	\begin{itemize}
	\item ...the union of African states.
	\item ...the question of which countries should join the currency union.
	\end{itemize}
}
\item countable noun \\
The marriage of two people is sometimes referred to as a \textbf{union} .
 \textit{
	\begin{itemize}
	\item Even Louis began to think their union was not blessed in the eyes of God.
	\end{itemize}
}
\item countable noun \\
\textbf{Union} is used in the name of some clubs , societies , and organizations.
 \textit{
	\begin{itemize}
	\item The naming of stars is at the discretion of the International Astronomical Union.
	\item ...the Mothers' Union.
	\end{itemize}
}
\end{enumerate}

\section*{vitamin}
{\large \color{blue}  vitamins  }
\subsection*{Explain}
\begin{enumerate}
\item countable noun \\
\textbf{Vitamins} are substances that you need in order to remain  healthy , which are found in food or can be eaten in the form of pills .
 \textit{
	\begin{itemize}
	\item Butter, margarine, and oily fish are all good sources of vitamin D.
	\item Healthy people do not need vitamin supplements.
	\end{itemize}
}
\end{enumerate}

\section*{refreshment}
{\large \color{blue}  refreshments  }
\subsection*{Explain}
\begin{enumerate}
\item plural noun \\
\textbf{Refreshments} are drinks and small amounts of food that are provided, for example , during a meeting or a journey .
 \textit{
	\begin{itemize}
	\end{itemize}
}
\item uncountable noun \\
You can refer to food and drink as \textbf{refreshment} .
 \textit{
	\begin{itemize}
	\item May I offer you some refreshment?
	\end{itemize}
}
\end{enumerate}

\section*{wedding}
{\large \color{blue}  weddings  }
\subsection*{Explain}
\begin{enumerate}
\item countable noun \\
A \textbf{wedding} is a marriage ceremony and the party or special  meal that often takes place after the ceremony.
 \textit{
	\begin{itemize}
	\item Most Britons want a traditional wedding.
	\item ...a wedding present.
	\item ...the couple's 22nd wedding anniversary.
	\end{itemize}
}
\end{enumerate}

\section*{zebra}
{\large \color{blue}  zebras  zebra  }
\subsection*{Explain}
\begin{enumerate}
\item countable noun \\
A \textbf{zebra} is an African  wild horse which has black and white stripes.
 \textit{
	\begin{itemize}
	\end{itemize}
}
\end{enumerate}

\section*{analysis}
{\large \color{blue}  analyses  }
\subsection*{Explain}
\begin{enumerate}
\item variable noun \\
\textbf{Analysis} is the process of considering something carefully or using statistical  methods in order to understand it or explain it.
 \textit{
	\begin{itemize}
	\item Her criteria defy analysis.
	\item We did an analysis of the way that government money has been spent in the past.
	\end{itemize}
}
\item variable noun \\
\textbf{Analysis} is the scientific process of examining something in order to find out what it consists of.
 \textit{
	\begin{itemize}
	\item They collect blood samples for analysis at a national laboratory.
	\item Jacobsen based his conclusion on an analysis of the decay of samarium-147 into neodymium-143.
	\end{itemize}
}
\item countable noun \\
An \textbf{analysis} is an explanation or description that results from considering something carefully.
 \textit{
	\begin{itemize}
	\item He started with an analysis of the situation as it stood in 1947.
	\end{itemize}
}
\item  \\
 in the final analysis/in the last analysis \textit{
	\begin{itemize}
	\end{itemize}
}
\end{enumerate}

\section*{awe}
{\large \color{blue}  awes  awed  }
\subsection*{Explain}
\begin{enumerate}
\item uncountable noun \\
\textbf{Awe} is the feeling of respect and amazement that you have when you are faced with something wonderful and often rather frightening .
 \textit{
	\begin{itemize}
	\item She gazed in awe at the great stones.
	\item His fellow officers regarded him with awe as some sort of genius.
	\item She filled me with a sense of awe.
	\end{itemize}
}
\item verb \\
If you \textbf{are awed}  \textbf{by} someone or something, they make you feel  respectful and amazed , though often rather frightened.
 \textit{
	\begin{itemize}
	\item I am still awed by David's courage.
	\item The crowd listened in awed silence.
	\end{itemize}
}
\item  \\
 be in awe of/stand in awe of \textit{
	\begin{itemize}
	\end{itemize}
}
\end{enumerate}

\section*{crisis}
{\large \color{blue}  crises  }
\subsection*{Explain}
\begin{enumerate}
\item variable noun \\
A \textbf{crisis} is a situation in which something or someone is affected by one or more very serious  problems .
 \textit{
	\begin{itemize}
	\item Natural disasters have obviously contributed to the continent's economic crisis.
	\item He had made arrangements for additional funding before the company was in crisis.
	\item ...children's illnesses or other family crises.
	\item He's having a mid-life crisis.
	\item ...someone to turn to in moments of crisis.
	\end{itemize}
}
\end{enumerate}

\section*{baseball}
{\large \color{blue}  baseballs  }
\subsection*{Explain}
\begin{enumerate}
\item uncountable noun \\
In America , \textbf{baseball} is a game played by two teams of nine players. Each player from one team hits a ball with a bat and then tries to run around three bases and get to the home  base before the other team can get the ball back .
 \textit{
	\begin{itemize}
	\end{itemize}
}
\item countable noun \\
A \textbf{baseball} is a small hard ball which is used in the game of baseball.
 \textit{
	\begin{itemize}
	\end{itemize}
}
\end{enumerate}

\section*{criterion}
{\large \color{blue}  criteria  }
\subsection*{Explain}
\begin{enumerate}
\item countable noun \\
A \textbf{criterion} is a factor on which you judge or decide something.
 \textit{
	\begin{itemize}
	\item The most important criterion for entry is that applicants must design and make their
own work.
	\item British defence policy had to meet three criteria if it was to succeed.
	\end{itemize}
}
\end{enumerate}

\section*{chair}
{\large \color{blue}  chairs  chairing  chaired  }
\subsection*{Explain}
\begin{enumerate}
\item countable noun \\
A \textbf{chair} is a piece of furniture for one person to sit on. Chairs have a back and four legs.
 \textit{
	\begin{itemize}
	\item He rose from his chair and walked to the window.
	\end{itemize}
}
\item countable noun \\
At a university , a \textbf{chair} is the post of professor .
 \textit{
	\begin{itemize}
	\item He has been appointed to the chair of sociology at Southampton University.
	\item He gave London University £600,000 to establish a chair in Islamic art.
	\end{itemize}
}
\item countable noun \\
The person who is the \textbf{chair}  \textbf{of} a committee or meeting is the person in charge of it.
 \textit{
	\begin{itemize}
	\item She shared her concerns with the chair of the church's finance council.
	\end{itemize}
}
\item verb \\
If you \textbf{chair} a meeting or a committee, you are the person in charge of it.
 \textit{
	\begin{itemize}
	\item He was about to chair a meeting in Venice of E.U. foreign ministers.
	\end{itemize}
}
\item singular noun \\
\textbf{The chair} is the same as the electric chair .
 \textit{
	\begin{itemize}
	\end{itemize}
}
\item  \\
 be in the chair/take the chair \textit{
	\begin{itemize}
	\end{itemize}
}
\end{enumerate}

\section*{curriculum}
{\large \color{blue}  curriculums  curricula  }
\subsection*{Explain}
\begin{enumerate}
\item countable noun \\
A \textbf{curriculum} is all the different courses of study that are taught in a school, college, or university .
 \textit{
	\begin{itemize}
	\item There should be a broader curriculum in schools for post-16-year-old pupils.
	\item Russian is the one compulsory foreign language on the school curriculum.
	\end{itemize}
}
\item countable noun \\
A particular  \textbf{curriculum} is one particular course of study that is taught in a school, college, or university.
 \textit{
	\begin{itemize}
	\item ...the history curriculum.
	\end{itemize}
}
\end{enumerate}

\section*{chess}
{\large \color{blue}  }
\subsection*{Explain}
\begin{enumerate}
\item uncountable noun \\
\textbf{Chess} is a game for two people, played on a chessboard. Each player has 16 pieces, including
a king. Your aim is to move your pieces so that your opponent's king cannot escape being taken.
 \textit{
	\begin{itemize}
	\item ...the world chess championships.
	\end{itemize}
}
\end{enumerate}

\section*{fish}
{\large \color{blue}  fish  fishes  fishes  fishing  fished  }
\subsection*{Explain}
\begin{enumerate}
\item countable noun \\
A \textbf{fish} is a creature that lives in water and has a tail and fins. There are many different kinds of fish.
 \textit{
	\begin{itemize}
	\item I was chatting to an islander who had just caught a fish.
	\item The fish were counted and an average weight recorded.
	\end{itemize}
}
\item uncountable noun \\
\textbf{Fish} is the flesh of a fish eaten as food.
 \textit{
	\begin{itemize}
	\item Does dry white wine go best with fish?
	\end{itemize}
}
\item verb \\
If you \textbf{fish} , you try to catch fish, either for food or as a form of sport or recreation .
 \textit{
	\begin{itemize}
	\item Brian remembers learning to fish in the River Cam.
	\end{itemize}
}
\item verb \\
If you \textbf{fish} a particular area of water, you try to catch fish in it.
 \textit{
	\begin{itemize}
	\item On Saturday we fished the River Arno.
	\end{itemize}
}
\item verb \\
If you say that someone is \textbf{fishing}  \textbf{for} information or praise , you disapprove of the fact that they are trying to get it from someone in an indirect way.
 \textit{
	\begin{itemize}
	\item He didn't want to create the impression that he was fishing for information.
	\item 'Lucinda, you don't have to talk to him!' Mike shouted. 'He's just fishing.'
	\end{itemize}
}
\item  \\
 like a fish out of water \textit{
	\begin{itemize}
	\end{itemize}
}
\item  \\
 there are plenty more fish in the sea \textit{
	\begin{itemize}
	\end{itemize}
}
\end{enumerate}

\section*{coin}
{\large \color{blue}  coins  coining  coined  }
\subsection*{Explain}
\begin{enumerate}
\item countable noun \\
A \textbf{coin} is a small piece of metal which is used as money.
 \textit{
	\begin{itemize}
	\item ...50 pence coins.
	\item ...Frederick's gold coin collection.
	\end{itemize}
}
\item verb \\
If you \textbf{coin} a word or a phrase, you are the first person to say it.
 \textit{
	\begin{itemize}
	\item Jaron Lanier coined the term 'virtual reality' and pioneered its early development.
	\end{itemize}
}
\item verb \\
If you say that someone \textbf{is coining}  \textbf{it} or \textbf{is coining} money, you are emphasizing that they are making a lot of money very quickly, often without really  earning it.
 \textbf{Coining in}  means the same as \textbf{coining} .
 \textit{
	\begin{itemize}
	\item Many private colleges are coining it.
	\item One wine shop is coining money selling Wembley-label champagne.
	\end{itemize}
}
\item  \\
 to coin a phrase \textit{
	\begin{itemize}
	\end{itemize}
}
\item  \\
 the other side of the coin \textit{
	\begin{itemize}
	\end{itemize}
}
\item  \\
 two sides of the same coin \textit{
	\begin{itemize}
	\end{itemize}
}
\end{enumerate}

\section*{foot}
{\large \color{blue}  feet  }
\subsection*{Explain}
\begin{enumerate}
\item countable noun \\
Your \textbf{feet} are the parts of your body that are at the ends of your legs, and that you stand on.
 \textit{
	\begin{itemize}
	\item She stamped her foot again.
	\item ...a foot injury.
	\item ...his aching arms and sore feet.
	\end{itemize}
}
\item singular noun \\
\textbf{The}  \textbf{foot}  \textbf{of} something is the part that is farthest from its top.
 \textit{
	\begin{itemize}
	\item David called to the children from the foot of the stairs.
	\item ...the foot of Highgate Hill.
	\item A single word at the foot of a page caught her eye.
	\end{itemize}
}
\item singular noun \\
\textbf{The}  \textbf{foot}  \textbf{of} a bed is the end nearest to the feet of the person lying in it.
 \textit{
	\begin{itemize}
	\item Friends stood at the foot of the bed, looking at her with serious faces.
	\end{itemize}
}
\item countable noun \\
A \textbf{foot} is a unit for measuring length, height, or depth , and is equal to 12 inches or 30.48 centimetres . When you are giving measurements , the form 'foot' is often used as the plural  instead of the plural form 'feet'.
 \textit{
	\begin{itemize}
	\item This beautiful and curiously shaped lake lies at around fifteen thousand feet.
	\item ...a shopping and leisure complex of one million square feet.
	\item He occupies a cell 10 foot long, 6 foot wide and 10 foot high.
	\item I have to give my height in feet and inches.
	\end{itemize}
}
\item adjective \\
A \textbf{foot}  brake or \textbf{foot}  pump is operated by your foot rather than by your hand.
 \textit{
	\begin{itemize}
	\item I tried to reach the foot brakes but I couldn't.
	\end{itemize}
}
\item adjective \\
A \textbf{foot}  patrol or \textbf{foot} soldiers walk rather than travelling in vehicles or on horseback .
 \textit{
	\begin{itemize}
	\item Paratroopers and foot-soldiers entered the building on the government's behalf.
	\end{itemize}
}
\item countable noun \\
In poetry , a \textbf{foot} is one of the basic units of rhythm into which a line is divided.
 \textit{
	\begin{itemize}
	\end{itemize}
}
\item  \\
 to get cold feet \textit{
	\begin{itemize}
	\end{itemize}
}
\item  \\
 to find one's feet \textit{
	\begin{itemize}
	\end{itemize}
}
\item  \\
 keep/have your feet on the ground \textit{
	\begin{itemize}
	\end{itemize}
}
\item  \\
 on foot \textit{
	\begin{itemize}
	\end{itemize}
}
\item  \\
 on your feet \textit{
	\begin{itemize}
	\end{itemize}
}
\item  \\
 on your feet \textit{
	\begin{itemize}
	\end{itemize}
}
\item  \\
 to fall on your feet \textit{
	\begin{itemize}
	\end{itemize}
}
\item  \\
 have one foot in the grave \textit{
	\begin{itemize}
	\end{itemize}
}
\item  \\
 the boot/shoe is on the other foot \textit{
	\begin{itemize}
	\end{itemize}
}
\item  \\
 put your best foot forward \textit{
	\begin{itemize}
	\end{itemize}
}
\item  \\
 put your foot down \textit{
	\begin{itemize}
	\end{itemize}
}
\item  \\
 put your foot down \textit{
	\begin{itemize}
	\end{itemize}
}
\item  \\
 put your foot in it \textit{
	\begin{itemize}
	\end{itemize}
}
\item  \\
 put your feet up \textit{
	\begin{itemize}
	\end{itemize}
}
\item  \\
 not put a foot wrong \textit{
	\begin{itemize}
	\end{itemize}
}
\item  \\
 to set foot somewhere \textit{
	\begin{itemize}
	\end{itemize}
}
\item  \\
 to stand on your own two feet \textit{
	\begin{itemize}
	\end{itemize}
}
\item  \\
 to your feet \textit{
	\begin{itemize}
	\end{itemize}
}
\item  \\
 under your feet \textit{
	\begin{itemize}
	\end{itemize}
}
\item  \\
 to get off on the wrong foot \textit{
	\begin{itemize}
	\end{itemize}
}
\end{enumerate}

\section*{colony}
{\large \color{blue}  colonies  }
\subsection*{Explain}
\begin{enumerate}
\item countable noun \\
A \textbf{colony} is a country which is controlled by a more powerful country.
 \textit{
	\begin{itemize}
	\item He was born in Algeria, a former colony of France.
	\item ...the quantity of gold and silver raised in the Spanish colonies.
	\end{itemize}
}
\item countable noun \\
You can refer to a place where a particular group of people lives as a particular kind of \textbf{colony} .
 \textit{
	\begin{itemize}
	\item The pretty town of Cranbrook became a thriving artists' colony.
	\item ...a penal colony.
	\item ...industrial colonies.
	\end{itemize}
}
\item plural noun \\
\textbf{The colonies} means all the countries that used to be British colonies.
 \textit{
	\begin{itemize}
	\item Many of our troops and officers were scattered around the world in the service of
His Majesty in the colonies.
	\end{itemize}
}
\item plural noun \\
\textbf{The}  \textbf{colonies} means the 13 British colonies in North America which formed the original  United  States .
 \textit{
	\begin{itemize}
	\item On the eve of the Revolution, the colonies produced thirty thousand tons of crude
iron a year.
	\end{itemize}
}
\item countable noun \\
A \textbf{colony}  \textbf{of} birds, insects , or animals is a group of them that live together.
 \textit{
	\begin{itemize}
	\item The Shetlands are famed for their colonies of sea birds.
	\item The caterpillars feed in large colonies.
	\end{itemize}
}
\end{enumerate}

\section*{formula}
{\large \color{blue}  formulae  formulas  }
\subsection*{Explain}
\begin{enumerate}
\item countable noun \\
A \textbf{formula} is a plan that is invented in order to deal with a particular problem .
 \textit{
	\begin{itemize}
	\item It is difficult to imagine how the North and South could ever agree on a formula
to unify the divided peninsula.
	\item ...a peace formula.
	\end{itemize}
}
\item singular noun \\
A \textbf{formula for} a particular situation , usually a good one, is a course of action or a combination of actions that is certain or likely to result in that situation.
 \textit{
	\begin{itemize}
	\item As world's oldest man, he offered a simple formula for a long and happy life.
	\item Clever exploitation of the latest technology would be a sure formula for success.
	\item Socialism does not after all offer a magic formula for prosperity and human dignity.
	\end{itemize}
}
\item countable noun \\
A \textbf{formula} is a group of letters, numbers , or other symbols which represents a scientific or mathematical rule.
 \textit{
	\begin{itemize}
	\item He developed a mathematical formula describing the distances of the planets from
the Sun.
	\end{itemize}
}
\item countable noun \\
In science , the \textbf{formula} for a substance is a list of the amounts of various substances which make up that substance, or an indication of the atoms that it is composed of.
 \textit{
	\begin{itemize}
	\end{itemize}
}
\item uncountable noun \\
\textbf{Formula} is used followed by a number to indicate a particular type of racing car or something relating to
that type of car.
 \textit{
	\begin{itemize}
	\item ...Formula 1 racing cars.
	\item ...Formula 3000 racing.
	\end{itemize}
}
\item uncountable noun \\
\textbf{Formula} is a powder which you mix with water to make artificial  milk for babies .
 \textit{
	\begin{itemize}
	\item ...bottles of formula.
	\end{itemize}
}
\end{enumerate}

\section*{cook}
{\large \color{blue}  cooks  cooking  cooked  }
\subsection*{Explain}
\begin{enumerate}
\item verb \\
When you \textbf{cook} a meal , you prepare food for eating by heating it.
 \textit{
	\begin{itemize}
	\item I have to go and cook the dinner.
	\item Chefs at the St James Court restaurant have cooked for the Queen.
	\item We'll cook them a nice Italian meal.
	\end{itemize}
}
\item verb \\
When you \textbf{cook} food, or when food \textbf{cooks} , it is heated until it is ready to be eaten.
 \textit{
	\begin{itemize}
	\item ...some basic instructions on how to cook a turkey.
	\item Let the vegetables cook gently for about 10 minutes.
	\item Drain the pasta as soon as it is cooked.
	\end{itemize}
}
\item countable noun \\
A \textbf{cook} is a person whose job is to prepare and cook food, especially in someone's home or in an institution .
 \textit{
	\begin{itemize}
	\item They had a butler, a cook, and a maid.
	\end{itemize}
}
\item countable noun \\
If you say that someone is a good \textbf{cook} , you mean they are good at preparing and cooking food.
 \textit{
	\begin{itemize}
	\end{itemize}
}
\item  \\
 to cook the books \textit{
	\begin{itemize}
	\end{itemize}
}
\end{enumerate}

\section*{fridge}
{\large \color{blue}  fridges  }
\subsection*{Explain}
\begin{enumerate}
\item countable noun \\
A \textbf{fridge} is a large metal  container which is kept  cool , usually by electricity , so that food that is put in it stays  fresh .
 \textit{
	\begin{itemize}
	\end{itemize}
}
\end{enumerate}

\section*{cotton}
{\large \color{blue}  cottons  cottoning  cottoned  }
\subsection*{Explain}
\begin{enumerate}
\item variable noun \\
\textbf{Cotton} is a type of cloth made from soft fibres from a particular plant.
 \textit{
	\begin{itemize}
	\item ...a cotton shirt.
	\end{itemize}
}
\item uncountable noun \\
\textbf{Cotton} is a plant which is grown in warm countries and which produces soft fibres used in making cotton cloth.
 \textit{
	\begin{itemize}
	\item ...a large cotton plantation in Tennessee.
	\end{itemize}
}
\item variable noun \\
\textbf{Cotton} is thread that is used for sewing , especially thread that is made from cotton.
 \textit{
	\begin{itemize}
	\item There's a needle and cotton there.
	\end{itemize}
}
\item uncountable noun \\
\textbf{Cotton} or \textbf{absorbent cotton} is a soft mass of cotton, used especially for applying  liquids or creams to your skin .
 \textit{
	\begin{itemize}
	\end{itemize}
}
\end{enumerate}

\section*{goose}
{\large \color{blue}  geese  }
\subsection*{Explain}
\begin{enumerate}
\item countable noun \\
A \textbf{goose} is a large bird that has a long neck and webbed  feet . Geese are often farmed for their meat .
 \textit{
	\begin{itemize}
	\end{itemize}
}
\item uncountable noun \\
\textbf{Goose} is the meat from a goose that has been cooked .
 \textit{
	\begin{itemize}
	\item ...roast goose.
	\end{itemize}
}
\item  \\
 cook someone's goose \textit{
	\begin{itemize}
	\end{itemize}
}
\item  \\
 the goose that lays the golden egg \textit{
	\begin{itemize}
	\end{itemize}
}
\item  \\
 what's sauce for the goose is sauce for the gander \textit{
	\begin{itemize}
	\end{itemize}
}
\end{enumerate}

\section*{crack}
{\large \color{blue}  cracks  cracking  cracked  }
\subsection*{Explain}
\begin{enumerate}
\item verb \\
If something hard \textbf{cracks} , or if you \textbf{crack} it, it becomes slightly damaged, with lines appearing on its surface.
 \textit{
	\begin{itemize}
	\item A gas main had cracked under my neighbour's garage and gas had seeped into our homes.
	\item Crack the salt crust on the fish and you will find the skin just peels off.
	\end{itemize}
}
\item verb \\
If something \textbf{cracks} , or if you \textbf{crack} it, it makes a sharp sound like the sound of a piece of wood breaking.
 \textit{
	\begin{itemize}
	\item Thunder cracked in the sky.
	\item He cracked his fingers nervously.
	\end{itemize}
}
\item verb \\
If you \textbf{crack} a hard part of your body, such as your knee or your head, you hurt it by accidentally hitting it hard against something.
 \textit{
	\begin{itemize}
	\item He cracked his head on the pavement and was knocked cold.
	\end{itemize}
}
\item verb \\
When you \textbf{crack} something that has a shell , such as an egg or a nut , you break the shell in order to reach the inside part.
 \textit{
	\begin{itemize}
	\item Crack the eggs into a bowl.
	\end{itemize}
}
\item verb \\
If you \textbf{crack} a problem or a code, you solve it, especially after a lot of thought.
 \textit{
	\begin{itemize}
	\item He has finally cracked the system after years of painstaking research.
	\end{itemize}
}
\item verb \\
If someone \textbf{cracks} , they lose control of their emotions or actions because they are under a lot of pressure.
 \textit{
	\begin{itemize}
	\item She's calm and strong, and she is just not going to crack.
	\item He was said to have cracked under the pressure and resigned.
	\end{itemize}
}
\item verb \\
If your voice \textbf{cracks} when you are speaking or singing , it changes in pitch because you are feeling a strong emotion.
 \textit{
	\begin{itemize}
	\item Her voice cracked and she began to cry.
	\end{itemize}
}
\item verb \\
If you \textbf{crack} a joke, you tell it.
 \textit{
	\begin{itemize}
	\item Somebody cracked a joke and we all laughed.
	\end{itemize}
}
\item  \\
 not all sth is cracked up to be \textit{
	\begin{itemize}
	\end{itemize}
}
\end{enumerate}

\section*{index}
{\large \color{blue}  indices  indexes  indexing  indexed  }
\subsection*{Explain}
\begin{enumerate}
\item countable noun \\
An \textbf{index} is a system by which changes in the value of something and the rate at which it changes can be recorded, measured, or interpreted .
 \textit{
	\begin{itemize}
	\item ...the U.K. retail price index.
	\item ...economic indices.
	\end{itemize}
}
\item countable noun \\
An \textbf{index} is an alphabetical list that is printed at the back of a book and tells you on which pages important topics are referred to.
 \textit{
	\begin{itemize}
	\item There's even a special subject index.
	\end{itemize}
}
\item verb \\
If you \textbf{index} a book or a collection of information, you make an alphabetical list of the items in it.
 \textit{
	\begin{itemize}
	\item This vast archive has been indexed and made accessible to researchers.
	\item Painters and sculptors are indexed separately.
	\item She's indexed the book by author, by age, and by illustrator.
	\end{itemize}
}
\item verb \\
If a quantity or value \textbf{is indexed}  \textbf{to} another, a system is arranged so that it increases or decreases whenever the other one increases or decreases.
 \textit{
	\begin{itemize}
	\item Minimum pensions and wages are to be indexed to inflation.
	\end{itemize}
}
\item countable noun \\
If one thing is an \textbf{index}  \textbf{of} another, it indicates what the other thing will be like.
 \textit{
	\begin{itemize}
	\item Weeds are an index to the character of the soil.
	\end{itemize}
}
\item countable noun \\
In mathematics , \textbf{indices} are the little numbers that show how many times you must  multiply a number by itself. In the equation 3² = 9, the number 2 is an index.
 \textit{
	\begin{itemize}
	\end{itemize}
}
\end{enumerate}

\section*{cripple}
{\large \color{blue}  cripples  crippling  crippled  }
\subsection*{Explain}
\begin{enumerate}
\item countable noun \\
A person with a physical disability or a serious  permanent injury is sometimes  referred to as a \textbf{cripple} .
 \textit{
	\begin{itemize}
	\item She has gone from being a healthy, fit, and sporty young woman to being a cripple.
	\end{itemize}
}
\item verb \\
If someone \textbf{is crippled} by an injury, it is so serious that they can never move their body properly again.
 \textit{
	\begin{itemize}
	\item Mr Easton was seriously crippled in an accident and had to leave his job.
	\item He had been warned that another bad fall could cripple him for life.
	\item He heaved his crippled leg into an easier position.
	\end{itemize}
}
\item countable noun \\
If you describe someone as an emotional  \textbf{cripple} , you mean that they have a particular  psychological or emotional problem which prevents them from living a normal  life .
 \textit{
	\begin{itemize}
	\end{itemize}
}
\item verb \\
If something \textbf{cripples} a person, it causes them severe psychological or emotional problems.
 \textit{
	\begin{itemize}
	\item Howard wanted to be a popular singer, but stage fright crippled him.
	\item I'm not perfect but I'm also not emotionally crippled or lonely.
	\end{itemize}
}
\item verb \\
To \textbf{cripple} a machine , organization , or system means to damage it severely or prevent it from working properly.
 \textit{
	\begin{itemize}
	\item Let's try to cripple their communications.
	\item A total cut-off of supplies would cripple the country's economy.
	\item The pilot was able to maneuver the crippled aircraft out of the hostile area.
	\end{itemize}
}
\end{enumerate}

\section*{deadline}
{\large \color{blue}  deadlines  }
\subsection*{Explain}
\begin{enumerate}
\item countable noun \\
A \textbf{deadline} is a time or date before which a particular task  must be finished or a particular thing must be done.
 \textit{
	\begin{itemize}
	\item We were not able to meet the deadline because of manufacturing delays.
	\item The deadline for submissions to the competition will be Easter Sunday.
	\item Negotiations will now resume in September, with a final deadline set for November.
	\end{itemize}
}
\end{enumerate}

\section*{medium}
{\large \color{blue}  mediums  media  }
\subsection*{Explain}
\begin{enumerate}
\item adjective \\
If something is of \textbf{medium} size, it is neither large nor small, but approximately  half way between the two.
 \textit{
	\begin{itemize}
	\item A medium dose produces severe nausea within hours.
	\item He was of medium height with blond hair and light blue eyes.
	\end{itemize}
}
\item adjective \\
You use \textbf{medium} to describe something which is average in degree or amount, or approximately half way along a
 scale between two extremes.
 \textbf{Medium} is also an adverb .
 \textit{
	\begin{itemize}
	\item Foods that contain only medium levels of sodium are bread, cakes, milk, butter and
margarine.
	\item ...a sweetish, medium-strength beer.
	\item Cook under a medium-hot grill.
	\end{itemize}
}
\item adjective \\
If something is of a \textbf{medium} colour, it is neither light nor dark, but approximately half way between the two.
 \textit{
	\begin{itemize}
	\item Andrea has medium brown hair, grey eyes and very pale skin.
	\item When violet is added to the medium blue a particularly striking, warm coloration
is created.
	\end{itemize}
}
\item countable noun \\
A \textbf{medium} is a way or means of expressing your ideas or of communicating with people.
 \textit{
	\begin{itemize}
	\item In Sierra Leone, English is used as the medium of instruction for all primary education.
	\item But Artaud was increasingly dissatisfied with film as a medium.
	\end{itemize}
}
\item countable noun \\
A \textbf{medium} is a substance or material which is used for a particular purpose or in order to
produce a particular effect.
 \textit{
	\begin{itemize}
	\item Blood is the medium in which oxygen is carried to all parts of the body.
	\item Hyatt has found a way of creating these qualities using the more permanent medium
of oil paint.
	\end{itemize}
}
\item countable noun \\
A \textbf{medium} is a person who claims to be able to contact and speak to people who are dead, and to pass messages between them and people who are still  alive .
 \textit{
	\begin{itemize}
	\end{itemize}
}
\item  \\
 happy medium \textit{
	\begin{itemize}
	\end{itemize}
}
\end{enumerate}

\section*{flaw}
{\large \color{blue}  flaws  }
\subsection*{Explain}
\begin{enumerate}
\item countable noun \\
A \textbf{flaw}  \textbf{in} something such as a theory or argument is a mistake in it, which causes it to be less effective or valid .
 \textit{
	\begin{itemize}
	\item There were, however, a number of crucial flaws in his monetary theory.
	\item Almost all of these studies have serious flaws.
	\end{itemize}
}
\item countable noun \\
A \textbf{flaw}  \textbf{in} someone's character is an undesirable quality that they have.
 \textit{
	\begin{itemize}
	\item The only flaw in his character seems to be a short temper.
	\end{itemize}
}
\item countable noun \\
A \textbf{flaw}  \textbf{in} something such as a pattern or material is a fault in it that should not be there.
 \textit{
	\begin{itemize}
	\end{itemize}
}
\end{enumerate}

\section*{mouse}
{\large \color{blue}  mice  }
\subsection*{Explain}
\begin{enumerate}
\item countable noun \\
A \textbf{mouse} is a small furry animal with a long tail .
 \textit{
	\begin{itemize}
	\item ...a mouse running in a wheel in its cage.
	\item ...the problem of rats and mice.
	\end{itemize}
}
\item countable noun \\
A \textbf{mouse} is a device that is connected to a computer. By moving it over a flat surface and pressing its buttons , you can move the cursor around the screen and do things without using the keyboard .
 \textit{
	\begin{itemize}
	\end{itemize}
}
\end{enumerate}

\section*{forest}
{\large \color{blue}  forests  }
\subsection*{Explain}
\begin{enumerate}
\item variable noun \\
A \textbf{forest} is a large area where trees grow close together.
 \textit{
	\begin{itemize}
	\item Parts of the forest are still dense and inaccessible.
	\item ...25 million hectares of forest.
	\end{itemize}
}
\item countable noun \\
A \textbf{forest} of tall or narrow objects is a group of them standing or sticking  upright .
 \textit{
	\begin{itemize}
	\item They descended from the plane into a forest of microphones and cameras.
	\end{itemize}
}
\end{enumerate}

\section*{nucleus}
{\large \color{blue}  nuclei  }
\subsection*{Explain}
\begin{enumerate}
\item countable noun \\
The \textbf{nucleus} of an atom or cell is the central part of it.
 \textit{
	\begin{itemize}
	\item Neutrons and protons are bound together in the nucleus of an atom.
	\end{itemize}
}
\item countable noun \\
\textbf{The}  \textbf{nucleus}  \textbf{of} a group of people or things is the small number of members which form the most important part of the group.
 \textit{
	\begin{itemize}
	\item A small group of shareholders formed the nucleus of a new management team.
	\end{itemize}
}
\end{enumerate}

\section*{fracture}
{\large \color{blue}  fractures  fracturing  fractured  }
\subsection*{Explain}
\begin{enumerate}
\item countable noun \\
A \textbf{fracture} is a slight crack or break in something, especially a bone.
 \textit{
	\begin{itemize}
	\item At least one-third of all women over ninety have sustained a hip fracture.
	\end{itemize}
}
\item verb \\
If something such as a bone \textbf{is fractured} or \textbf{fractures} , it gets a slight crack or break in it.
 \textit{
	\begin{itemize}
	\item You've fractured a rib, maybe more than one.
	\item One strut had fractured and been crudely repaired in several places.
	\item He suffered a fractured skull.
	\end{itemize}
}
\item verb \\
If something such as an organization or society  \textbf{is fractured} or \textbf{fractures} , it splits into several parts or stops  existing .
 \textit{
	\begin{itemize}
	\item His policy risks fracturing the coalition.
	\item It might be a society that could fracture along class lines.
	\end{itemize}
}
\end{enumerate}

\section*{grief}
{\large \color{blue}  griefs  }
\subsection*{Explain}
\begin{enumerate}
\item variable noun \\
\textbf{Grief} is a feeling of extreme sadness.
 \textit{
	\begin{itemize}
	\item ...a huge outpouring of national grief for the victims of the shootings.
	\item Their grief soon gave way to anger.
	\end{itemize}
}
\item  \\
 come to grief \textit{
	\begin{itemize}
	\end{itemize}
}
\item  \\
 good grief \textit{
	\begin{itemize}
	\end{itemize}
}
\end{enumerate}

\section*{heritage}
{\large \color{blue}  heritages  }
\subsection*{Explain}
\begin{enumerate}
\item variable noun \\
A country's \textbf{heritage} is all the qualities, traditions, or features of life there that have continued over many years and have been passed on from one generation to another.
 \textit{
	\begin{itemize}
	\item The historic building is as much part of our heritage as the paintings.
	\item ...the rich heritage of Russian folk music.
	\end{itemize}
}
\end{enumerate}

\section*{plough}
{\large \color{blue}  ploughs  ploughing  ploughed  }
\subsection*{Explain}
\begin{enumerate}
\item countable noun \\
A \textbf{plough} is a large farming  tool with sharp blades which is pulled across the soil to turn it over, usually before seeds are planted.
 \textit{
	\begin{itemize}
	\end{itemize}
}
\item verb \\
When someone \textbf{ploughs} an area of land, they turn over the soil using a plough.
 \textit{
	\begin{itemize}
	\item They ploughed nearly 100,000 acres of virgin moorland.
	\item ...a carefully ploughed field.
	\end{itemize}
}
\item  \\
 under the plough \textit{
	\begin{itemize}
	\end{itemize}
}
\end{enumerate}

\section*{jet}
{\large \color{blue}  jets  jetting  jetted  }
\subsection*{Explain}
\begin{enumerate}
\item countable noun \\
A \textbf{jet} is an aircraft that is powered by jet engines .
 \textit{
	\begin{itemize}
	\item Her private jet landed in the republic on the way to Japan.
	\item He had arrived from Jersey by jet.
	\item ...America's first jet aircraft.
	\end{itemize}
}
\item verb \\
If you \textbf{jet}  somewhere , you travel there in a fast  plane .
 \textit{
	\begin{itemize}
	\item They will be jetting off on a two-week holiday in America.
	\item They spend a great deal of time jetting around the world.
	\end{itemize}
}
\item countable noun \\
A \textbf{jet} of liquid or gas is a strong , fast, thin stream of it.
 \textit{
	\begin{itemize}
	\item A jet of water poured through the windows.
	\end{itemize}
}
\item uncountable noun \\
\textbf{Jet} is a hard black stone that is used in jewellery.
 \textit{
	\begin{itemize}
	\end{itemize}
}
\end{enumerate}

\section*{programme}
{\large \color{blue}  programmes  programming  programmed  }
\subsection*{Explain}
\begin{enumerate}
\item countable noun \\
A \textbf{programme} of actions or events is a series of actions or events that are planned to be done.
 \textit{
	\begin{itemize}
	\item The general argued that the nuclear programme should still continue.
	\item The programme of sell-offs has been implemented by the new chief executive.
	\end{itemize}
}
\item countable noun \\
A television or radio \textbf{programme} is something that is broadcast on television or radio.
 \textit{
	\begin{itemize}
	\item ...a series of TV programmes on global environment.
	\item ...local news programmes.
	\end{itemize}
}
\item countable noun \\
A theatre or concert  \textbf{programme} is a small book or sheet of paper which gives information about the play or concert you are attending .
 \textit{
	\begin{itemize}
	\end{itemize}
}
\item verb \\
When you \textbf{programme} a machine or system, you set its controls so that it will work in a particular way.
 \textit{
	\begin{itemize}
	\item Parents can programme the machine not to turn on at certain times.
	\end{itemize}
}
\item verb \\
If a living  creature  \textbf{is programmed}  \textbf{to}  behave in a particular way, they are likely to behave in that way because of social or biological  factors that they cannot control.
 \textit{
	\begin{itemize}
	\item We are all genetically programmed to develop certain illnesses.
	\end{itemize}
}
\end{enumerate}

\section*{kitchen}
{\large \color{blue}  kitchens  }
\subsection*{Explain}
\begin{enumerate}
\item countable noun \\
A \textbf{kitchen} is a room that is used for cooking and for household  jobs such as washing  dishes .
 \textit{
	\begin{itemize}
	\end{itemize}
}
\end{enumerate}

\section*{railroad}
{\large \color{blue}  railroads  railroading  railroaded  }
\subsection*{Explain}
\begin{enumerate}
\item countable noun \\
A \textbf{railroad} is a route between two places along which trains travel on steel  rails .
 \textit{
	\begin{itemize}
	\item ...railroad tracks that led to nowhere.
	\item The railroad finally reached Santa Barbara in 1877.
	\end{itemize}
}
\item countable noun \\
A \textbf{railroad} is a company or organization that operates  railway routes.
 \textit{
	\begin{itemize}
	\item ...The Chicago and Northwestern Railroad.
	\end{itemize}
}
\item verb \\
If you \textbf{railroad} someone \textbf{into} doing something, you make them do it although they do not really  want to, by hurrying them and putting pressure on them.
 \textit{
	\begin{itemize}
	\item She is a very fine actor who has refused to be railroaded into rom-coms.
	\item He railroaded the reforms through.
	\end{itemize}
}
\end{enumerate}

\section*{legacy}
{\large \color{blue}  legacies  }
\subsection*{Explain}
\begin{enumerate}
\item countable noun \\
A \textbf{legacy} is money or property which someone leaves to you when they die .
 \textit{
	\begin{itemize}
	\item You could make a real difference to someone's life by leaving them a generous legacy.
	\end{itemize}
}
\item countable noun \\
A \textbf{legacy}  \textbf{of} an event or period of history is something which is a direct result of it and which continues to exist after it is over.
 \textit{
	\begin{itemize}
	\item ...the legacy of inequality and injustice created by Apartheid.
	\item The old system has left a mixed legacy.
	\end{itemize}
}
\end{enumerate}

\section*{reflection}
{\large \color{blue}  reflections  }
\subsection*{Explain}
\begin{enumerate}
\item countable noun \\
A \textbf{reflection} is an image that you can see in a mirror or in glass or water.
 \textit{
	\begin{itemize}
	\item Meg stared at her reflection in the bedroom mirror.
	\end{itemize}
}
\item uncountable noun \\
\textbf{Reflection} is the process by which light and heat are sent back from a surface and do not pass through it.
 \textit{
	\begin{itemize}
	\item ...the reflection of a beam of light off a mirror.
	\end{itemize}
}
\item countable noun \\
If you say that something is a \textbf{reflection}  \textbf{of} a particular person's attitude or \textbf{of} a situation , you mean that it is caused by that attitude or situation and therefore reveals something about it.
 \textit{
	\begin{itemize}
	\item Inhibition in adulthood is a reflection of a person's experiences as a child.
	\end{itemize}
}
\item singular noun \\
If something is a \textbf{reflection} or a \textbf{sad reflection}  \textbf{on} a person or thing, it gives a bad  impression of them.
 \textit{
	\begin{itemize}
	\item Infection with head lice is no reflection on personal hygiene.
	\item The break-up of the library would be a sad reflection on the value we place on our
heritage.
	\end{itemize}
}
\item variable noun \\
\textbf{Reflection} is careful thought about a particular subject . Your \textbf{reflections} are your thoughts about a particular subject.
 \textit{
	\begin{itemize}
	\item After days of reflection she decided to write back.
	\item He paused, absorbed by his reflections.
	\end{itemize}
}
\item countable noun \\
\textbf{Reflections}  \textbf{on} something are comments or writings that express someone's ideas about it.
 \textit{
	\begin{itemize}
	\item In his latest collection of poems readers are confronted with a series of reflections
on death.
	\end{itemize}
}
\end{enumerate}

\section*{majesty}
{\large \color{blue}  majesties  }
\subsection*{Explain}
\begin{enumerate}
\item countable noun \\
You use majesty in expressions such as \textbf{Your Majesty} or \textbf{Her Majesty} when you are addressing or referring to a King or Queen .
 \textit{
	\begin{itemize}
	\item I quite agree, Your Majesty.
	\item His Majesty requests your presence in the royal chambers.
	\item Their Majesties celebrated our arrival by giving us each a little silver spoon.
	\end{itemize}
}
\item uncountable noun \\
\textbf{Majesty} is the quality of being beautiful , dignified , and impressive .
 \textit{
	\begin{itemize}
	\item ...the majesty of the mainland mountains.
	\end{itemize}
}
\end{enumerate}

\section*{remains}
{\large \color{blue}  }
\subsection*{Explain}
\begin{enumerate}
\item plural noun \\
1.  2.  3.  \textit{
	\begin{itemize}
	\item the remains of a meal
	\item archaeological remains
	\end{itemize}
}
\end{enumerate}

\section*{maximum}
{\large \color{blue}  }
\subsection*{Explain}
\begin{enumerate}
\item adjective \\
You use \textbf{maximum} to describe an amount which is the largest that is possible, allowed , or required .
 \textbf{Maximum} is also a noun .
 \textit{
	\begin{itemize}
	\item Under planning law the maximum height for a fence or hedge is 2 metres.
	\item China headed the table with maximum points.
	\item The law provides for a maximum of two years in prison.
	\item Twelve hours is the minimum, sixty hours the maximum.
	\end{itemize}
}
\item adjective \\
You use \textbf{maximum} to indicate how great an amount is.
 \textit{
	\begin{itemize}
	\item ...the maximum amount of information.
	\item It was achieved with minimum fuss and maximum efficiency.
	\item ...a maximum security prison.
	\end{itemize}
}
\item adverb \\
If you say that something is a particular amount \textbf{maximum} , you mean that this is the greatest amount it should be or could possibly be, although a smaller
amount is acceptable or very possible.
 \textit{
	\begin{itemize}
	\item We need an extra 6g a day maximum.
	\end{itemize}
}
\item  \\
 to the maximum \textit{
	\begin{itemize}
	\end{itemize}
}
\end{enumerate}

\section*{resource}
{\large \color{blue}  resources  }
\subsection*{Explain}
\begin{enumerate}
\item countable noun \\
The \textbf{resources} of an organization or person are the materials, money, and other things that they have and can use in
order to function properly.
 \textit{
	\begin{itemize}
	\item Some families don't have the resources to feed themselves properly.
	\item There's a great shortage of resource materials in many schools.
	\end{itemize}
}
\item countable noun \\
A country's \textbf{resources} are the things that it has and can use to increase its wealth, such as coal , oil , or land.
 \textit{
	\begin{itemize}
	\item ...resources like coal, tungsten, oil and copper.
	\item Today we are overpopulated, straining the earth's resources.
	\end{itemize}
}
\end{enumerate}

\section*{minimum}
{\large \color{blue}  }
\subsection*{Explain}
\begin{enumerate}
\item adjective \\
You use \textbf{minimum} to describe an amount which is the smallest that is possible, allowed, or required .
 \textbf{Minimum} is also a noun .
 \textit{
	\begin{itemize}
	\item He was only five feet nine, the minimum height for a policeman.
	\item ...a rise in the minimum wage.
	\item This will take a minimum of one hour.
	\item Four foot should be seen as an absolute minimum.
	\end{itemize}
}
\item adjective \\
You use \textbf{minimum} to state how small an amount is.
 \textbf{Minimum} is also a noun.
 \textit{
	\begin{itemize}
	\item The basic needs of life are available with minimum effort.
	\item Neil and Chris try to spend the minimum amount of time on the garden.
	\item With a minimum of fuss, she produced the grandson he had so desperately wished for.
	\end{itemize}
}
\item adverb \\
If you say that something is a particular amount \textbf{minimum} , you mean that this is the smallest amount it should be or could possibly be, although a larger
amount is acceptable or very possible.
 \textit{
	\begin{itemize}
	\item You're talking over a thousand pounds minimum for one course.
	\end{itemize}
}
\item  \\
 at the most \textit{
	\begin{itemize}
	\end{itemize}
}
\item  \\
 to a/the minimum \textit{
	\begin{itemize}
	\end{itemize}
}
\end{enumerate}

\section*{rumor}
{\large \color{blue}  }
\subsection*{Explain}
\begin{enumerate}
\end{enumerate}

\section*{mutter}
{\large \color{blue}  mutters  muttering  muttered  }
\subsection*{Explain}
\begin{enumerate}
\item verb \\
If you \textbf{mutter} , you speak very quietly so that you cannot easily be heard , often because you are complaining about something.
 \textbf{Mutter} is also a noun .
 \textit{
	\begin{itemize}
	\item 'God knows what's happening in that madman's mind,' she muttered.
	\item She can hear the old woman muttering about consideration.
	\item He sat there shaking his head, muttering to himself.
	\item She was staring into the fire muttering.
	\item They make no more than a mutter of protest.
	\end{itemize}
}
\end{enumerate}

\section*{sock}
{\large \color{blue}  socks  socking  socked  }
\subsection*{Explain}
\begin{enumerate}
\item countable noun \\
\textbf{Socks} are pieces of clothing which cover your foot and ankle and are worn inside shoes.
 \textit{
	\begin{itemize}
	\item ...a pair of knee-length socks.
	\end{itemize}
}
\item  \\
 to sock it to \textit{
	\begin{itemize}
	\end{itemize}
}
\item  \\
 to pull your socks up \textit{
	\begin{itemize}
	\end{itemize}
}
\end{enumerate}

\section*{outlook}
{\large \color{blue}  outlooks  }
\subsection*{Explain}
\begin{enumerate}
\item variable noun \\
Your \textbf{outlook} is your general attitude towards life.
 \textit{
	\begin{itemize}
	\item The illness had a profound effect on his outlook.
	\item I adopted a positive outlook on life.
	\item We were quite different in outlook, Philip and I.
	\end{itemize}
}
\item singular noun \\
The \textbf{outlook} for something is what people think  will  happen in relation to it.
 \textit{
	\begin{itemize}
	\item The economic outlook is one of rising unemployment.
	\item ...the uncertain outlook for the motor industry.
	\end{itemize}
}
\end{enumerate}

\section*{speciality}
{\large \color{blue}  specialities  }
\subsection*{Explain}
\begin{enumerate}
\item countable noun \\
Someone's \textbf{speciality} is a particular type of work that they do most or do best , or a subject that they know a lot about.
 \textit{
	\begin{itemize}
	\item My father was a historian of repute. His speciality was the history of Germany.
	\item Handpainted tiled murals for kitchens are her speciality.
	\end{itemize}
}
\item countable noun \\
A \textbf{speciality} of a particular place is a special food or product that is always very good there.
 \textit{
	\begin{itemize}
	\item Rhineland dishes are a speciality of the restaurant.
	\item I started with the Viennese speciality frittatensuppe, or pancake soup.
	\end{itemize}
}
\end{enumerate}

\section*{plant}
{\large \color{blue}  plants  planting  planted  }
\subsection*{Explain}
\begin{enumerate}
\item countable noun \\
A \textbf{plant} is a living thing that grows in the earth and has a stem, leaves, and roots.
 \textit{
	\begin{itemize}
	\item Water each plant as often as required.
	\item ...exotic plants.
	\end{itemize}
}
\item verb \\
When you \textbf{plant} a seed, plant, or young tree, you put it into the ground so that it will grow there.
 \textit{
	\begin{itemize}
	\item He says he plans to plant fruit trees and vegetables.
	\end{itemize}
}
\item verb \\
When someone \textbf{plants} land \textbf{with} a particular type of plant or crop, they put plants, seeds, or young trees into the
land to grow them there.
 \textit{
	\begin{itemize}
	\item They plan to plant the area with grass and trees.
	\item Recently much of their energy has gone into planting a large vegetable garden.
	\item ...newly planted fields.
	\end{itemize}
}
\item countable noun \\
A \textbf{plant} is a factory or a place where power is produced.
 \textit{
	\begin{itemize}
	\item ...Ford's British car assembly plants.
	\item The plant provides forty per cent of the country's electricity.
	\end{itemize}
}
\item uncountable noun \\
\textbf{Plant} is large machinery that is used in industrial processes.
 \textit{
	\begin{itemize}
	\item ...investment in plant and equipment.
	\end{itemize}
}
\item verb \\
If you \textbf{plant} something somewhere , you put it there firmly.
 \textit{
	\begin{itemize}
	\item She planted her feet wide and bent her knees slightly.
	\item ...with his enormous feet planted heavily apart.
	\end{itemize}
}
\item verb \\
To \textbf{plant} something such as a bomb means to hide it somewhere so that it explodes or works there.
 \textit{
	\begin{itemize}
	\item So far no one has admitted planting the bomb.
	\end{itemize}
}
\item verb \\
If something such as a weapon or drugs \textbf{is planted} on someone, it is put among their possessions or in their house so that they will be wrongly accused of a crime .
 \textit{
	\begin{itemize}
	\item He claimed that the evidence had been planted to incriminate him.
	\end{itemize}
}
\item verb \\
If an organization \textbf{plants} someone somewhere, they send that person there so that they can get information or watch someone secretly.
 \textit{
	\begin{itemize}
	\item Journalists informed police, who planted an undercover detective to trap their target.
	\end{itemize}
}
\item verb \\
If you \textbf{plant a kiss}  \textbf{on} someone, you give them a kiss.
 \textit{
	\begin{itemize}
	\item She planted a kiss on each of his leathery cheeks.
	\end{itemize}
}
\item verb \\
If you \textbf{plant an idea} in someone's mind, they begin to accept the idea without realizing that it has originally come from you and not from them.
 \textit{
	\begin{itemize}
	\item He hoped that he could plant the idea in such a way that the manager would believe
it was his own.
	\end{itemize}
}
\end{enumerate}

\section*{spectacle}
{\large \color{blue}  spectacles  }
\subsection*{Explain}
\begin{enumerate}
\item plural noun \\
Glasses are sometimes  referred to as \textbf{spectacles} .
 \textit{
	\begin{itemize}
	\item He looked at me over the tops of his spectacles.
	\item ...thick spectacle frames.
	\end{itemize}
}
\item countable noun \\
A \textbf{spectacle} is a strange or interesting sight .
 \textit{
	\begin{itemize}
	\item It was a spectacle not to be missed.
	\end{itemize}
}
\item variable noun \\
A \textbf{spectacle} is a grand and impressive event or performance.
 \textit{
	\begin{itemize}
	\item 94,000 people turned up for the spectacle.
	\item ...a director passionate about music and spectacle.
	\end{itemize}
}
\end{enumerate}

\section*{prospect}
{\large \color{blue}  prospects  prospecting  prospected  }
\subsection*{Explain}
\begin{enumerate}
\item variable noun \\
If there is some \textbf{prospect}  \textbf{of} something happening , there is a possibility that it will happen .
 \textit{
	\begin{itemize}
	\item Unfortunately, there is little prospect of seeing these big questions answered.
	\item The prospects for peace in the country's eight-year civil war are becoming brighter.
	\item There is a real prospect that the bill will be defeated in parliament.
	\end{itemize}
}
\item singular noun \\
A particular \textbf{prospect} is something that you expect or know is going to happen.
 \textit{
	\begin{itemize}
	\item They now face the prospect of having to wear a cycling helmet by law.
	\item After supper he'd put his feet up and read. It was a pleasant prospect.
	\end{itemize}
}
\item plural noun \\
Someone's \textbf{prospects} are their chances of being successful , especially in their career .
 \textit{
	\begin{itemize}
	\item I chose to work abroad to improve my career prospects.
	\item ...a detailed review of the company's prospects.
	\end{itemize}
}
\item verb \\
When people \textbf{prospect for} oil, gold, or some other valuable substance, they look for it in the ground or under
the sea.
 \textit{
	\begin{itemize}
	\item He had prospected for minerals everywhere from the Gobi Desert to the Transvaal.
	\item In fact, the oil companies are already prospecting not far from here.
	\end{itemize}
}
\end{enumerate}

\section*{stair}
{\large \color{blue}  stairs  }
\subsection*{Explain}
\begin{enumerate}
\item plural noun \\
\textbf{Stairs} are a set of steps inside a building which go from one floor to another.
 \textit{
	\begin{itemize}
	\item Nancy began to climb the stairs.
	\item We walked up a flight of stairs.
	\item He learned to walk safely up and down stairs.
	\item He stopped at the top of the stairs.
	\item ...a stair carpet.
	\end{itemize}
}
\item singular noun \\
A \textbf{stair} is a flight of stairs.
 \textit{
	\begin{itemize}
	\item I followed her down the stair.
	\end{itemize}
}
\item countable noun \\
A \textbf{stair} is one of the steps in a flight of stairs.
 \textit{
	\begin{itemize}
	\item Terry was sitting on the bottom stair.
	\end{itemize}
}
\end{enumerate}

\section*{quantity}
{\large \color{blue}  quantities  }
\subsection*{Explain}
\begin{enumerate}
\item variable noun \\
A \textbf{quantity} is an amount that you can measure or count.
 \textit{
	\begin{itemize}
	\item ...a small quantity of water.
	\item ...vast quantities of food.
	\item It takes a long time to make a large quantity.
	\item Cheap goods are available, but not in sufficient quantities to satisfy demand.
	\item Uranium is available in considerable quantity from various areas of the world.
	\end{itemize}
}
\item uncountable noun \\
Things that are produced or available in \textbf{quantity} are produced or available in large amounts.
 \textit{
	\begin{itemize}
	\item After some initial problems, acetone was successfully produced in quantity.
	\item But even with those databases, the sheer quantity of data can still cause problems.
	\end{itemize}
}
\item uncountable noun \\
You can use \textbf{quantity} to refer to the amount of something that there is, especially when you want to contrast it with its quality.
 \textit{
	\begin{itemize}
	\item ...the less discerning drinker who prefers quantity to quality.
	\item In terms of quantity, production grew faster than ever before.
	\end{itemize}
}
\item  \\
 unknown quantity \textit{
	\begin{itemize}
	\end{itemize}
}
\end{enumerate}

\section*{staircase}
{\large \color{blue}  staircases  }
\subsection*{Explain}
\begin{enumerate}
\item countable noun \\
A \textbf{staircase} is a set of stairs inside a building.
 \textit{
	\begin{itemize}
	\item They walked down the staircase together.
	\end{itemize}
}
\end{enumerate}

\section*{scene}
{\large \color{blue}  scenes  }
\subsection*{Explain}
\begin{enumerate}
\item countable noun \\
A \textbf{scene} in a play, film, or book is part of it in which a series of events happen in the same place.
 \textit{
	\begin{itemize}
	\item I found the proposal scene tremendously poignant.
	\item ...the opening scene of 'A Christmas Carol'.
	\item ...love scenes.
	\item ...Act I, scene 1.
	\end{itemize}
}
\item countable noun \\
You refer to a place as a \textbf{scene} when you are describing its appearance and indicating what impression it makes on you.
 \textit{
	\begin{itemize}
	\item It's a scene of complete devastation.
	\item Thick black smoke billowed over the scene.
	\item You can just picture the scene, can't you?
	\end{itemize}
}
\item countable noun \\
You can describe an event that you see , or that is broadcast or shown in a picture , as a \textbf{scene} of a particular  kind .
 \textit{
	\begin{itemize}
	\item There were emotional scenes as the refugees enjoyed their first breath of freedom.
	\item Television broadcasters were warned to exercise caution over depicting scenes of
violence.
	\item It was a bizarre scene.
	\end{itemize}
}
\item countable noun \\
The \textbf{scene}  \textbf{of} an event is the place where it happened.
 \textit{
	\begin{itemize}
	\item The area has been the scene of fierce fighting for three months.
	\item ...traces left at the scene of a crime.
	\item Fire and police crews rushed to the scene, but the couple were already dead.
	\item Riot vans were on the scene in minutes.
	\end{itemize}
}
\item singular noun \\
You can refer to an area of activity as a particular type of \textbf{scene} .
 \textit{
	\begin{itemize}
	\item Sandman is a cult figure on the local music scene.
	\item ...when he first burst onto the national political scene at age 28.
	\item ...a youth guide to London's club scene.
	\end{itemize}
}
\item countable noun \\
Paintings and drawings of places are sometimes  called  \textbf{scenes} .
 \textit{
	\begin{itemize}
	\item ...James Lynch's country scenes.
	\end{itemize}
}
\item countable noun \\
If you make a \textbf{scene} , you embarrass people by publicly  showing your anger about something.
 \textit{
	\begin{itemize}
	\item I'm sorry I made such a scene.
	\end{itemize}
}
\item  \\
 behind the scenes \textit{
	\begin{itemize}
	\end{itemize}
}
\item  \\
 behind the scenes \textit{
	\begin{itemize}
	\end{itemize}
}
\item  \\
 a change of scene \textit{
	\begin{itemize}
	\end{itemize}
}
\item  \\
 set the scene \textit{
	\begin{itemize}
	\end{itemize}
}
\item  \\
 set the scene for sth \textit{
	\begin{itemize}
	\end{itemize}
}
\item  \\
 on the scene/from the scene \textit{
	\begin{itemize}
	\end{itemize}
}
\item  \\
 not be one's scene \textit{
	\begin{itemize}
	\end{itemize}
}
\end{enumerate}

\section*{sorrow}
{\large \color{blue}  }
\subsection*{Explain}
\begin{enumerate}
\item uncountable noun \\
\textbf{Sorrow} is a feeling of deep sadness or regret.
 \textit{
	\begin{itemize}
	\item It was a time of great sorrow.
	\item Words cannot express my sorrow.
	\end{itemize}
}
\end{enumerate}

\section*{sulfur}
{\large \color{blue}  }
\subsection*{Explain}
\begin{enumerate}
\end{enumerate}

\section*{spray}
{\large \color{blue}  sprays  spraying  sprayed  }
\subsection*{Explain}
\begin{enumerate}
\item variable noun \\
\textbf{Spray} is a lot of small drops of water which are being thrown into the air.
 \textit{
	\begin{itemize}
	\item The moon was casting a rainbow through the spray from the waterfall.
	\item The rope whipped clear of the water, throwing up a spray of droplets.
	\end{itemize}
}
\item variable noun \\
A \textbf{spray} is a liquid kept under pressure in a can or other container , which you can force out in very small drops.
 \textit{
	\begin{itemize}
	\item ...hair spray.
	\item ...a can of insect spray.
	\end{itemize}
}
\item verb \\
If you \textbf{spray} a liquid somewhere or if it \textbf{sprays} somewhere, drops of the liquid cover a place or shower someone.
 \textit{
	\begin{itemize}
	\item A sprayer hooked to a tractor can spray five gallons onto ten acres.
	\item Two inmates hurled slates at prison officers spraying them with a hose.
	\item Drops of blood sprayed across the room.
	\end{itemize}
}
\item verb \\
If a lot of small things \textbf{spray} somewhere or if something \textbf{sprays} them, they are scattered somewhere with a lot of force.
 \textit{
	\begin{itemize}
	\item A shower of mustard seeds sprayed into the air and fell into the grass.
	\item The intensity of the blaze shattered windows, spraying glass on the streets below.
	\item The bullet slammed into the ceiling, spraying them with bits of plaster.
	\end{itemize}
}
\item verb \\
If someone \textbf{sprays}  bullets somewhere, they fire a lot of bullets at a group of people or things.
 \textit{
	\begin{itemize}
	\item He ran to the top of the building spraying bullets into shoppers below.
	\item The army lorries were sprayed with machine gun fire from guerrillas in the woods.
	\end{itemize}
}
\item verb \\
If something \textbf{is sprayed} , it is painted using paint kept under pressure in a container.
 \textit{
	\begin{itemize}
	\item The bare metal was sprayed with several coats of primer.
	\end{itemize}
}
\item verb \\
When someone \textbf{sprays} against insects , they cover plants or crops with a chemical which prevents insects feeding on them.
 \textit{
	\begin{itemize}
	\item He doesn't spray against pests or diseases.
	\item Confine the use of insecticides to the evening and do not spray plants that are in
flower.
	\item Because of the immunity of the immature insects, it's important to spray regularly.
	\end{itemize}
}
\item countable noun \\
A \textbf{spray} is a piece of equipment for spraying water or another liquid, especially over growing plants.
 \textit{
	\begin{itemize}
	\end{itemize}
}
\item countable noun \\
A \textbf{spray of} flowers or leaves is a number of flowers or leaves on one stem or branch.
 \textit{
	\begin{itemize}
	\item ...a small spray of freesias.
	\end{itemize}
}
\end{enumerate}

\section*{synthesis}
{\large \color{blue}  syntheses  }
\subsection*{Explain}
\begin{enumerate}
\item countable noun \\
A \textbf{synthesis}  \textbf{of} different ideas or styles is a mixture or combination of these ideas or styles.
 \textit{
	\begin{itemize}
	\item His novels are a rich synthesis of Balkan history and mythology.
	\item Her synthesis of feminism and socialism ran counter to all other historical currents.
	\end{itemize}
}
\item variable noun \\
The \textbf{synthesis} of a substance is the production of it by means of chemical or biological reactions.
 \textit{
	\begin{itemize}
	\item ...the genes that regulate the synthesis of these compounds.
	\item This kind of lighting encourages vitamin D synthesis in the skin.
	\end{itemize}
}
\end{enumerate}

\section*{stick}
{\large \color{blue}  sticks  }
\subsection*{Explain}
\begin{enumerate}
\item countable noun \\
A \textbf{stick} is a thin branch which has fallen off a tree.
 \textit{
	\begin{itemize}
	\item ...people carrying bundles of dried sticks to sell for firewood.
	\end{itemize}
}
\item countable noun \\
A \textbf{stick} is a long thin piece of wood which is used for supporting someone's weight or for
hitting people or animals.
 \textit{
	\begin{itemize}
	\item He looks old and walks with a stick.
	\item Crowds armed with sticks and stones took to the streets.
	\end{itemize}
}
\item countable noun \\
A \textbf{stick} is a long thin piece of wood which is used for a particular purpose.
 \textit{
	\begin{itemize}
	\item ...kebab sticks.
	\item ...lolly sticks.
	\item ...drum sticks.
	\end{itemize}
}
\item countable noun \\
Some long thin objects that are used in sports are called \textbf{sticks} .
 \textit{
	\begin{itemize}
	\item ...lacrosse sticks.
	\item ...hockey sticks.
	\item ...ski-sticks.
	\end{itemize}
}
\item countable noun \\
A \textbf{stick}  \textbf{of} something is a long thin piece of it.
 \textit{
	\begin{itemize}
	\item ...a stick of celery.
	\item ...cinnamon sticks.
	\end{itemize}
}
\item uncountable noun \\
If you give someone some \textbf{stick} , you criticize them or tease them roughly .
 \textit{
	\begin{itemize}
	\item It's not motorists who give you the most stick, it's the general public.
	\item I get some stick from the lads because of my faith but I don't mind.
	\end{itemize}
}
\item plural noun \\
If you say that someone lives in \textbf{the sticks} , you mean that they live a long way from any large cities.
 \textit{
	\begin{itemize}
	\item He lived out in the sticks somewhere.
	\end{itemize}
}
\item  \\
 a stick to beat someone with \textit{
	\begin{itemize}
	\end{itemize}
}
\item  \\
 to get the wrong end of the stick \textit{
	\begin{itemize}
	\end{itemize}
}
\item  \\
 more...than you could shake a stick at \textit{
	\begin{itemize}
	\end{itemize}
}
\end{enumerate}

\section*{stumble}
{\large \color{blue}  stumbles  stumbling  stumbled  }
\subsection*{Explain}
\begin{enumerate}
\item verb \\
If you \textbf{stumble} , you put your foot down awkwardly while you are walking or running and nearly fall over.
 \textbf{Stumble} is also a noun .
 \textit{
	\begin{itemize}
	\item He stumbled and almost fell.
	\item I stumbled into the phone box and dialled 999.
	\item I make it into the darkness with only one stumble.
	\end{itemize}
}
\item verb \\
If you \textbf{stumble} while you are reading  aloud or speaking, you make a mistake, and have to pause before saying the words properly.
 \textit{
	\begin{itemize}
	\item ...his voice wavered and he stumbled over the words at one point.
	\end{itemize}
}
\end{enumerate}

\section*{tear}
{\large \color{blue}  tears  }
\subsection*{Explain}
\begin{enumerate}
\item countable noun \\
\textbf{Tears} are the drops of salty liquid that come out of your eyes when you are crying .
 \textit{
	\begin{itemize}
	\item Her eyes filled with tears.
	\item I just broke down and wept with tears of joy.
	\item I didn't shed a single tear.
	\end{itemize}
}
\item plural noun \\
You can use \textbf{tears} in expressions such as \textbf{in tears} , \textbf{burst into tears} , and \textbf{close to tears} to indicate that someone is crying or is almost crying.
 \textit{
	\begin{itemize}
	\item He was in floods of tears on the phone.
	\item She burst into tears.
	\item She was conscious of being very near to tears.
	\end{itemize}
}
\end{enumerate}

\section*{track}
{\large \color{blue}  tracks  tracking  tracked  }
\subsection*{Explain}
\begin{enumerate}
\item countable noun \\
A \textbf{track} is a narrow road or path.
 \textit{
	\begin{itemize}
	\item We set off once more, over a rough mountain track.
	\end{itemize}
}
\item countable noun \\
A \textbf{track} is a piece of ground, often oval-shaped , that is used for races involving athletes , cars, bicycles , horses, or dogs called greyhounds .
 \textit{
	\begin{itemize}
	\item The two men turned to watch the horses going round the track.
	\item ...the athletics track.
	\end{itemize}
}
\item countable noun \\
Railway \textbf{tracks} are the rails that a train travels along.
 \textit{
	\begin{itemize}
	\item A woman fell on to the tracks.
	\end{itemize}
}
\item countable noun \\
A \textbf{track} is one of the songs or pieces of music on a recording.
 \textit{
	\begin{itemize}
	\item All 10 tracks on the album are love songs.
	\end{itemize}
}
\item plural noun \\
\textbf{Tracks} are marks left in the ground by the feet of animals or people.
 \textit{
	\begin{itemize}
	\item The only evidence of pandas was their tracks in the snow.
	\item McKee suddenly noticed tire tracks on the bank ahead.
	\end{itemize}
}
\item verb \\
If you \textbf{track} animals or people, you try to follow them by looking for the signs that they have left behind, for example the
marks left by their feet.
 \textit{
	\begin{itemize}
	\item He thought he had better track this wolf and kill it.
	\item I followed him, tracking him in the snow until finally he got tired.
	\end{itemize}
}
\item verb \\
To \textbf{track} someone or something means to follow their movements by means of a special device,
such as a satellite or radar.
 \textit{
	\begin{itemize}
	\item Our radar began tracking the jets.
	\item Forecasters are also tracking the hurricane.
	\end{itemize}
}
\item verb \\
If you \textbf{track} someone or something, you investigate them, because you are interested in finding out more about them.
 \textit{
	\begin{itemize}
	\item If it's possible, track the rumour back to its origin.
	\item The player is being tracked by Juventus.
	\end{itemize}
}
\item countable noun \\
In a school, a \textbf{track} is a group of children of the same age and ability who are taught together.
 \textit{
	\begin{itemize}
	\end{itemize}
}
\item verb \\
To \textbf{track} students means to divide them into groups according to their ability.
 \textit{
	\begin{itemize}
	\item Students are already being tracked.
	\end{itemize}
}
\item  \\
 cover your tracks \textit{
	\begin{itemize}
	\end{itemize}
}
\item  \\
 have the inside track \textit{
	\begin{itemize}
	\end{itemize}
}
\item  \\
 to keep track \textit{
	\begin{itemize}
	\end{itemize}
}
\item  \\
 to lose track of \textit{
	\begin{itemize}
	\end{itemize}
}
\item  \\
 make tracks \textit{
	\begin{itemize}
	\end{itemize}
}
\item  \\
 on track \textit{
	\begin{itemize}
	\end{itemize}
}
\item  \\
 on the track of \textit{
	\begin{itemize}
	\end{itemize}
}
\item  \\
 on the right/wrong track \textit{
	\begin{itemize}
	\end{itemize}
}
\item  \\
 to stop someone in their tracks \textit{
	\begin{itemize}
	\end{itemize}
}
\item  \\
 stop sth (dead) in its tracks \textit{
	\begin{itemize}
	\end{itemize}
}
\end{enumerate}

\section*{theater}
{\large \color{blue}  }
\subsection*{Explain}
\begin{enumerate}
\item noun \\
1.  2.  3.  4.  5.  \textit{
	\begin{itemize}
	\item a commander in the Southern Pacific theater
	\item a play that is good theater
	\end{itemize}
}
\end{enumerate}

\section*{tragedy}
{\large \color{blue}  tragedies  }
\subsection*{Explain}
\begin{enumerate}
\item variable noun \\
A \textbf{tragedy} is an extremely sad event or situation .
 \textit{
	\begin{itemize}
	\item They have suffered an enormous personal tragedy.
	\item Maskell's life had not been without tragedy.
	\end{itemize}
}
\item variable noun \\
\textbf{Tragedy} is a type of literature, especially drama, that is serious and sad, and often ends with the death of the main  character .
 \textit{
	\begin{itemize}
	\item The story has elements of tragedy and farce.
	\item ...a classic Greek tragedy.
	\end{itemize}
}
\end{enumerate}

\section*{thesis}
{\large \color{blue}  theses  }
\subsection*{Explain}
\begin{enumerate}
\item countable noun \\
A \textbf{thesis} is an idea or theory that is expressed as a statement and is discussed in a logical way.
 \textit{
	\begin{itemize}
	\item This thesis does not stand up to close inspection.
	\item ...the thesis that computers can be programmed to do anything a human mind does.
	\end{itemize}
}
\item countable noun \\
A \textbf{thesis} is a long piece of writing based on your own ideas and research that you do as part of a university degree,
 especially a higher degree such as a PhD.
 \textit{
	\begin{itemize}
	\item He was awarded his PhD for a thesis on industrial robots.
	\end{itemize}
}
\end{enumerate}

\section*{transistor}
{\large \color{blue}  transistors  }
\subsection*{Explain}
\begin{enumerate}
\item countable noun \\
A \textbf{transistor} is a small electronic part in something such as a television or radio, which controls the flow of electricity .
 \textit{
	\begin{itemize}
	\end{itemize}
}
\item countable noun \\
A \textbf{transistor} or a \textbf{transistor radio} is a small portable radio.
 \textit{
	\begin{itemize}
	\end{itemize}
}
\end{enumerate}

\section*{thief}
{\large \color{blue}  thieves  }
\subsection*{Explain}
\begin{enumerate}
\item countable noun \\
A \textbf{thief} is a person who steals something from another person.
 \textit{
	\begin{itemize}
	\item The thieves snatched the camera.
	\item ...car thieves.
	\end{itemize}
}
\end{enumerate}

\section*{trumpet}
{\large \color{blue}  trumpets  trumpeting  trumpeted  }
\subsection*{Explain}
\begin{enumerate}
\item variable noun \\
A \textbf{trumpet} is a musical instrument of the brass family which plays quite high notes. You play the trumpet by blowing into it.
 \textit{
	\begin{itemize}
	\end{itemize}
}
\item verb \\
If someone \textbf{trumpets} something that they are proud of or that they think is important , they speak about it publicly in a very forceful way.
 \textit{
	\begin{itemize}
	\item The government has been trumpeting tourism as a growth industry.
	\item ...Mark Morris, who is trumpeted as the dance talent of his generation.
	\item Nobody should be trumpeting about chemical weapons.
	\item It was trumpeted that the nation's health was improving.
	\item ...the much trumpeted 'tax cuts' in the 1980s.
	\end{itemize}
}
\item verb \\
When an elephant  \textbf{trumpets} , it makes a loud sound.
 \textit{
	\begin{itemize}
	\item The elephants trumpeted and stamped their feet at their approach.
	\end{itemize}
}
\end{enumerate}

\section*{tooth}
{\large \color{blue}  teeth  }
\subsection*{Explain}
\begin{enumerate}
\item countable noun \\
Your \textbf{teeth} are the hard white objects in your mouth, which you use for biting and chewing.
 \textit{
	\begin{itemize}
	\item She had very pretty straight teeth.
	\item If a tooth feels very loose, your dentist may recommend that it's taken out.
	\end{itemize}
}
\item plural noun \\
The \textbf{teeth} of something such as a comb , saw , cog , or zip are the parts that stick out in a row on its edge .
 \textit{
	\begin{itemize}
	\item The front cog has 44 teeth.
	\end{itemize}
}
\item plural noun \\
If you say that something such as an official group or a law has \textbf{teeth} , you mean that it has power and is able to be effective .
 \textit{
	\begin{itemize}
	\item The opposition argues that the new council will be unconstitutional and without teeth.
	\item The law must have teeth, and it must be enforced.
	\end{itemize}
}
\item  \\
 armed to the teeth \textit{
	\begin{itemize}
	\end{itemize}
}
\item  \\
 to cut your teeth on something \textit{
	\begin{itemize}
	\end{itemize}
}
\item  \\
 to set your teeth on edge \textit{
	\begin{itemize}
	\end{itemize}
}
\item  \\
 to fight tooth and nail \textit{
	\begin{itemize}
	\end{itemize}
}
\item  \\
 get one's teeth into sth \textit{
	\begin{itemize}
	\end{itemize}
}
\item  \\
 in the teeth of \textit{
	\begin{itemize}
	\end{itemize}
}
\item  \\
 lie through one's teeth \textit{
	\begin{itemize}
	\end{itemize}
}
\item  \\
 long in the tooth \textit{
	\begin{itemize}
	\end{itemize}
}
\item  \\
 a sweet tooth \textit{
	\begin{itemize}
	\end{itemize}
}
\end{enumerate}

\section*{vegetation}
{\large \color{blue}  }
\subsection*{Explain}
\begin{enumerate}
\item uncountable noun \\
Plants, trees, and flowers can be referred to as \textbf{vegetation} .
 \textit{
	\begin{itemize}
	\item The inn has a garden of semi-tropical vegetation.
	\item ...a smell of gently-rotting vegetation.
	\end{itemize}
}
\end{enumerate}

\section*{troop}
{\large \color{blue}  troops  trooping  trooped  }
\subsection*{Explain}
\begin{enumerate}
\item plural noun \\
\textbf{Troops} are soldiers, especially when they are in a large organized group doing a particular task .
 \textit{
	\begin{itemize}
	\item The operation will involve more than 35,000 troops from a dozen countries.
	\item There were reports of troop movements.
	\end{itemize}
}
\item countable noun \\
A \textbf{troop} is a group of soldiers within a cavalry or armoured  regiment .
 \textit{
	\begin{itemize}
	\item ...a troop of enemy cavalry trotting towards the Dutch right flank.
	\end{itemize}
}
\item countable noun \\
A \textbf{troop} of boy scouts is a local group of them that meets regularly.
 \textit{
	\begin{itemize}
	\item ...a Scout troop.
	\end{itemize}
}
\item countable noun \\
A \textbf{troop of} people or animals is a group of them.
 \textit{
	\begin{itemize}
	\item Amy was aware of the little troop of travellers watching the two of them.
	\item Out of beams and cracks came troops of beetles, ants and spiders.
	\end{itemize}
}
\item verb \\
If people \textbf{troop}  somewhere , they walk there in a group, often in a sad or tired way.
 \textit{
	\begin{itemize}
	\item They all trooped back to the house for a rest.
	\item The men trooped into work with resignation.
	\end{itemize}
}
\end{enumerate}

\section*{violet}
{\large \color{blue}  violets  }
\subsection*{Explain}
\begin{enumerate}
\item countable noun \\
A \textbf{violet} is a small plant that has purple or white flowers in the spring .
 \textit{
	\begin{itemize}
	\end{itemize}
}
\item colour \\
Something that is \textbf{violet} is a bluish-purple colour.
 \textit{
	\begin{itemize}
	\item The light was beginning to drain from a violet sky.
	\end{itemize}
}
\item  \\
 no shrinking violet \textit{
	\begin{itemize}
	\end{itemize}
}
\end{enumerate}

\section*{tv}
{\large \color{blue}  TVs  }
\subsection*{Explain}
\begin{enumerate}
\item variable noun \\
\textbf{TV} means the same as television .
 \textit{
	\begin{itemize}
	\item The TV was on.
	\item I prefer going to the cinema to watching TV.
	\item ...a TV commercial.
	\end{itemize}
}
\end{enumerate}

\section*{wine}
{\large \color{blue}  wines  wining  wined  }
\subsection*{Explain}
\begin{enumerate}
\item variable noun \\
\textbf{Wine} is an alcoholic drink which is made from grapes. You can also  refer to alcoholic drinks made from other fruits or vegetables as \textbf{wine} .
 A glass of wine can be referred to as a \textbf{wine} .
 \textit{
	\begin{itemize}
	\item ...a bottle of white wine.
	\item This is a nice wine.
	\item ...homemade parsnip wine.
	\end{itemize}
}
\item colour \\
\textbf{Wine} is used to describe things that are very dark red in colour.
 \textit{
	\begin{itemize}
	\item She wore her wine-coloured gaberdine raincoat.
	\item ...an olive and wine wool sweater.
	\end{itemize}
}
\item  \\
 to wine and dine \textit{
	\begin{itemize}
	\end{itemize}
}
\end{enumerate}

\section*{airline}
{\large \color{blue}  airlines  }
\subsection*{Explain}
\begin{enumerate}
\item countable noun \\
An \textbf{airline} is a company which provides regular services carrying people or goods in aeroplanes .
 \textit{
	\begin{itemize}
	\item ...the Dutch national airline KLM.
	\end{itemize}
}
\end{enumerate}

\section*{assembly}
{\large \color{blue}  assemblies  }
\subsection*{Explain}
\begin{enumerate}
\item countable noun \\
An \textbf{assembly} is a large group of people who meet regularly to make decisions or laws for a particular region or country.
 \textit{
	\begin{itemize}
	\item ...the campaign for the first free election to the National Assembly.
	\item ...an assembly of party members from the Russian republic.
	\end{itemize}
}
\item countable noun \\
An \textbf{assembly} is a group of people gathered together for a particular purpose.
 \textit{
	\begin{itemize}
	\item He waited until complete quiet settled on the assembly.
	\end{itemize}
}
\item uncountable noun \\
When you refer to rights of \textbf{assembly} or restrictions on \textbf{assembly} , you are referring to the legal right that people have to gather together.
 \textit{
	\begin{itemize}
	\item The U.S. Constitution guarantees free speech, freedom of assembly and equal protection.
	\item They were accused of unlawful assembly.
	\end{itemize}
}
\item variable noun \\
In a school, \textbf{assembly} is a gathering of all the teachers and pupils at the beginning of every school day.
 \textit{
	\begin{itemize}
	\item By 9, the juniors are in the hall for assembly.
	\item ...a long room with a stage at one end for assemblies.
	\end{itemize}
}
\item uncountable noun \\
The \textbf{assembly} of a machine, device, or object is the process of fitting its different parts together.
 \textit{
	\begin{itemize}
	\item For the rest of the day, he worked on the assembly of an explosive device.
	\item ...workers at Sao Paolo's car assembly plants.
	\end{itemize}
}
\end{enumerate}

\section*{anguish}
{\large \color{blue}  }
\subsection*{Explain}
\begin{enumerate}
\item uncountable noun \\
\textbf{Anguish} is great mental suffering or physical pain.
 \textit{
	\begin{itemize}
	\item A cry of anguish burst from her lips.
	\item Mark looked at him in anguish.
	\end{itemize}
}
\end{enumerate}

\section*{basket}
{\large \color{blue}  baskets  }
\subsection*{Explain}
\begin{enumerate}
\item countable noun \\
A \textbf{basket} is a stiff container that is used for carrying or storing  objects . Baskets are made from thin strips of materials such as straw, plastic, or wire  woven  together .
 A \textbf{basket}  \textbf{of} things is a number of things contained in a basket.
 \textit{
	\begin{itemize}
	\item ...big wicker picnic baskets filled with sandwiches.
	\item ...a laundry basket.
	\item ...a small basket of fruit and snacks.
	\end{itemize}
}
\item countable noun \\
In economics , a \textbf{basket}  \textbf{of}  currencies or goods is the average or total  value of a number of different currencies or goods.
 \textit{
	\begin{itemize}
	\item The pound's value against a basket of currencies hit a new low of 76.9.
	\item ...an inflation measure that gauges the price of a fixed basket of goods and services.
	\end{itemize}
}
\item countable noun \\
In basketball , the \textbf{basket} is a net  hanging from a ring through which players try to throw the ball in order to score points. A \textbf{basket} is also the point scored when the ball is thrown through the ring.
 \textit{
	\begin{itemize}
	\end{itemize}
}
\end{enumerate}

\section*{antenna}
{\large \color{blue}  antennae  antennas  }
\subsection*{Explain}
\begin{enumerate}
\item countable noun \\
The \textbf{antennae} of something such as an insect or crustacean are the two long, thin parts attached
to its head that it uses to feel things with.
 \textit{
	\begin{itemize}
	\end{itemize}
}
\item countable noun \\
An \textbf{antenna} is a device that sends and receives television or radio  signals .
 \textit{
	\begin{itemize}
	\end{itemize}
}
\end{enumerate}

\section*{blackboard}
{\large \color{blue}  blackboards  }
\subsection*{Explain}
\begin{enumerate}
\item countable noun \\
A \textbf{blackboard} is a dark-coloured board that you can write on with chalk.
 \textit{
	\begin{itemize}
	\end{itemize}
}
\end{enumerate}

\section*{ballot}
{\large \color{blue}  ballots  balloting  balloted  }
\subsection*{Explain}
\begin{enumerate}
\item countable noun \\
A \textbf{ballot} is a secret vote in which people select a candidate in an election, or express their opinion about something.
 \textit{
	\begin{itemize}
	\item The result of the ballot will not be known for two weeks.
	\item Fifty of its members will be elected by direct ballot.
	\end{itemize}
}
\item countable noun \\
A \textbf{ballot} is a piece of paper on which you indicate your choice or opinion in a secret vote.
 \textit{
	\begin{itemize}
	\item Election boards will count the ballots by hand.
	\item ...the first senator to be re-elected without her name appearing on the ballot.
	\end{itemize}
}
\item verb \\
If you \textbf{ballot} a group of people, you find out what they think about a subject by organizing a secret vote.
 \textit{
	\begin{itemize}
	\item The union said they will ballot members on whether to strike.
	\end{itemize}
}
\end{enumerate}

\section*{blanket}
{\large \color{blue}  blankets  blanketing  blanketed  }
\subsection*{Explain}
\begin{enumerate}
\item countable noun \\
A \textbf{blanket} is a large square or rectangular piece of thick cloth, especially one which you put on a bed to keep you warm .
 \textit{
	\begin{itemize}
	\end{itemize}
}
\item countable noun \\
A \textbf{blanket}  \textbf{of} something such as snow is a continuous layer of it which hides what is below or beyond it.
 \textit{
	\begin{itemize}
	\item The mud disappeared under a blanket of snow.
	\item Cold damp air brought in the new year under a blanket of fog.
	\end{itemize}
}
\item singular noun \\
You can refer to something such as an unpleasant  emotion or an undesirable quality that seems to affect every aspect of a particular situation as a \textbf{blanket of} that emotion or quality.
 \textit{
	\begin{itemize}
	\item ...the blanket of depression.
	\item A blanket of silence descended.
	\end{itemize}
}
\item verb \\
If something such as snow \textbf{blankets} an area, it covers it.
 \textit{
	\begin{itemize}
	\item More than a foot of snow blanketed parts of Michigan.
	\item With a thick mist now blanketing the trees, I got thoroughly lost.
	\end{itemize}
}
\item adjective \\
You use \textbf{blanket} to describe something when you want to emphasize that it affects or refers to every person or thing in a group, without any exceptions .
 \textit{
	\begin{itemize}
	\item It is tempting to support a blanket ban on junk food advertising.
	\item ...the blanket coverage of the Olympics.
	\end{itemize}
}
\end{enumerate}

\section*{beverage}
{\large \color{blue}  beverages  }
\subsection*{Explain}
\begin{enumerate}
\item countable noun \\
\textbf{Beverages} are drinks.
 \textit{
	\begin{itemize}
	\item Alcoholic beverages are served in the hotel lounge.
	\item ...artificially sweetened beverages.
	\item ...foods and beverages.
	\end{itemize}
}
\end{enumerate}

\section*{chain}
{\large \color{blue}  chains  chaining  chained  }
\subsection*{Explain}
\begin{enumerate}
\item countable noun \\
A \textbf{chain} consists of metal rings connected together in a line.
 \textit{
	\begin{itemize}
	\item His open shirt revealed a fat gold chain.
	\item The dogs were leaping and growling at the full stretch of their chains.
	\end{itemize}
}
\item plural noun \\
If prisoners are \textbf{in chains} , they have thick rings of metal round their wrists or ankles to prevent them from escaping .
 \textit{
	\begin{itemize}
	\item He'd spent four and a half years in windowless cells, much of the time in chains.
	\end{itemize}
}
\item plural noun \\
You can refer to feelings and duties which prevent you from doing what you want to do as \textbf{chains} .
 \textit{
	\begin{itemize}
	\item He had to break right now the chains of habit that bound him to the present.
	\end{itemize}
}
\item verb \\
If a person or thing \textbf{is chained}  \textbf{to} something, they are fastened to it with a chain.
 \textbf{Chain up} means the same as chain .
 \textit{
	\begin{itemize}
	\item The dog was chained to the leg of the one solid garden seat.
	\item She chained her bike to the railings.
	\item Some demonstrators chained themselves to railings inside the court building.
	\item We were sitting together in our cell, chained to the wall.
	\item I'll lock the doors and chain you up.
	\item They kept me chained up every night and released me each day.
	\item All the rowing boats were chained up.
	\end{itemize}
}
\item countable noun \\
A \textbf{chain of} things is a group of them existing or arranged in a line.
 \textit{
	\begin{itemize}
	\item ...a chain of islands known as the Windward Islands.
	\item Students tried to form a human chain around the parliament.
	\end{itemize}
}
\item countable noun \\
A \textbf{chain}  \textbf{of} shops, hotels, or other businesses is a number of them owned by the same person or
company.
 \textit{
	\begin{itemize}
	\item ...a large supermarket chain.
	\item ...Italy's leading chain of cinemas.
	\end{itemize}
}
\item singular noun \\
A \textbf{chain of} events is a series of them happening one after another.
 \textit{
	\begin{itemize}
	\item ...the bizarre chain of events that led to his departure in January 1938.
	\end{itemize}
}
\end{enumerate}

\section*{bracket}
{\large \color{blue}  brackets  bracketing  bracketed  }
\subsection*{Explain}
\begin{enumerate}
\item countable noun \\
If you say that someone or something is in a particular \textbf{bracket} , you mean that they come within a particular range , for example a range of incomes , ages , or prices .
 \textit{
	\begin{itemize}
	\item ...a 33% top tax rate on everyone in these high-income brackets.
	\item Do you fall outside that age bracket?
	\end{itemize}
}
\item countable noun \\
\textbf{Brackets} are pieces of metal, wood , or plastic that are fastened to a wall in order to support something such as a shelf.
 \textit{
	\begin{itemize}
	\item Fix the beam with the brackets and screws.
	\item ...adjustable wall brackets.
	\end{itemize}
}
\item verb \\
If two or more people or things \textbf{are bracketed}  \textbf{together} , they are considered to be similar or related in some way.
 \textit{
	\begin{itemize}
	\item Small businesses are being bracketed together as high risk, regardless of their business
plans and previous histories.
	\item Austrian wine styles are often bracketed with those of northern Germany.
	\end{itemize}
}
\item countable noun \\
\textbf{Brackets} are a pair of written marks that you place round a word, expression , or sentence in order to indicate that you are giving extra  information . In British English, curved marks like these are also  called  \textbf{brackets} , but in American English, they are called parenthesis .
 \textit{
	\begin{itemize}
	\item The prices in brackets are special rates for the under 18s.
	\item My annotations appear in square brackets.
	\end{itemize}
}
\item countable noun \\
\textbf{Brackets} are pair of marks that are placed around a series of symbols in a mathematical expression to indicate that those symbols function as one item within the expression.
 \textit{
	\begin{itemize}
	\end{itemize}
}
\end{enumerate}

\section*{coke}
{\large \color{blue}  }
\subsection*{Explain}
\begin{enumerate}
\item uncountable noun \\
\textbf{Coke} is a solid black substance that is produced from coal and is burned as a fuel.
 \textit{
	\begin{itemize}
	\item ...a coke-burning stove.
	\end{itemize}
}
\item uncountable noun \\
\textbf{Coke} is the same as cocaine .
 \textit{
	\begin{itemize}
	\end{itemize}
}
\end{enumerate}

\section*{brain}
{\large \color{blue}  brains  braining  brained  }
\subsection*{Explain}
\begin{enumerate}
\item countable noun \\
Your \textbf{brain} is the organ inside your head that controls your body's activities and enables you to think and to feel things such as heat and pain .
 \textit{
	\begin{itemize}
	\item Her father died of a brain tumour.
	\end{itemize}
}
\item countable noun \\
Your \textbf{brain} is your mind and the way that you think.
 \textit{
	\begin{itemize}
	\item Once you stop using your brain you soon go stale.
	\item Stretch your brain with this puzzle.
	\end{itemize}
}
\item countable noun \\
If someone has \textbf{brains} or a good \textbf{brain} , they have the ability to learn and understand things quickly, to solve  problems , and to make good decisions .
 \textit{
	\begin{itemize}
	\item They were not the only ones to have brains and ambition.
	\item I had a good brain and the teachers liked me.
	\end{itemize}
}
\item countable noun \\
If someone is \textbf{the}  \textbf{brains} behind an idea or an organization, he or she had that idea or makes the important
decisions about how that organization is managed .
 \textit{
	\begin{itemize}
	\item Mr White was the brains behind the scheme.
	\item Some investigators regarded her as the brains of the gang.
	\end{itemize}
}
\item verb \\
To \textbf{brain} someone means to hit them forcefully on the head.
 \textit{
	\begin{itemize}
	\item He had threatened to brain him then and there.
	\end{itemize}
}
\item  \\
 to beat someone's brains out \textit{
	\begin{itemize}
	\end{itemize}
}
\item  \\
 blow someone's brains out \textit{
	\begin{itemize}
	\end{itemize}
}
\item  \\
 have on the brain \textit{
	\begin{itemize}
	\end{itemize}
}
\item  \\
 to pick someone's brains \textit{
	\begin{itemize}
	\end{itemize}
}
\end{enumerate}

\section*{collective}
{\large \color{blue}  collectives  }
\subsection*{Explain}
\begin{enumerate}
\item adjective \\
\textbf{Collective} actions, situations , or feelings involve or are shared by every member of a group of people.
 \textit{
	\begin{itemize}
	\item It was a collective decision.
	\item The country's politicians are already heaving a collective sigh of relief.
	\end{itemize}
}
\item adjective \\
A \textbf{collective}  amount of something is the total  obtained by adding  together the amounts that each person or thing in a group has.
 \textit{
	\begin{itemize}
	\item Their collective volume wasn't very large.
	\end{itemize}
}
\item adjective \\
The \textbf{collective}  term for two or more types of thing is a general word or expression which refers to all of them.
 \textit{
	\begin{itemize}
	\item Social science is a collective name, covering a series of individual sciences.
	\end{itemize}
}
\item countable noun \\
A \textbf{collective} is a business or farm which is run , and often owned, by a group of people who take an equal share of any profits .
 \textit{
	\begin{itemize}
	\item He will see that he is participating in all the decisions of the collective.
	\end{itemize}
}
\end{enumerate}

\section*{bunch}
{\large \color{blue}  bunches  bunching  bunched  }
\subsection*{Explain}
\begin{enumerate}
\item countable noun \\
A \textbf{bunch}  \textbf{of} people is a group of people who share one or more characteristics or who are doing something together.
 \textit{
	\begin{itemize}
	\item My neighbours are a bunch of busybodies.
	\item We were a pretty inexperienced bunch of people really.
	\item The players were a great bunch.
	\end{itemize}
}
\item countable noun \\
A \textbf{bunch}  \textbf{of} flowers is a number of flowers with their stalks  held or tied together.
 \textit{
	\begin{itemize}
	\item He had left a huge bunch of flowers in her hotel room.
	\end{itemize}
}
\item countable noun \\
A \textbf{bunch}  \textbf{of}  bananas or grapes is a group of them growing on the same stem .
 \textit{
	\begin{itemize}
	\item Lili had fallen asleep clutching a fat bunch of grapes.
	\end{itemize}
}
\item countable noun \\
A \textbf{bunch}  \textbf{of}  keys is a set of keys kept together on a metal  ring .
 \textit{
	\begin{itemize}
	\item George took out a bunch of keys and went to work on the complicated lock.
	\end{itemize}
}
\item quantifier \\
A \textbf{bunch}  \textbf{of} things is a number of things, especially a large number.
 \textbf{Bunch} is also a pronoun .
 \textit{
	\begin{itemize}
	\item We did a bunch of songs together.
	\item I'd like to adopt a multi-racial child. In fact, I'd love a whole bunch.
	\end{itemize}
}
\item plural noun \\
If a girl has her hair  \textbf{in}  \textbf{bunches} , it is parted down the middle and tied on each side of her head .
 \textit{
	\begin{itemize}
	\end{itemize}
}
\item verb \\
If clothing  \textbf{bunches}  \textbf{around} a part of your body, it forms a set of creases around it.
 \textit{
	\begin{itemize}
	\item She clutches the sides of her skirt until it bunches around her waist.
	\end{itemize}
}
\item  \\
 the best of the bunch \textit{
	\begin{itemize}
	\end{itemize}
}
\end{enumerate}

\section*{comfort}
{\large \color{blue}  comforts  comforting  comforted  }
\subsection*{Explain}
\begin{enumerate}
\item uncountable noun \\
If you are doing something \textbf{in}  \textbf{comfort} , you are physically relaxed and contented , and are not feeling any pain or other unpleasant  sensations .
 \textit{
	\begin{itemize}
	\item This will enable the audience to sit in comfort while watching the shows.
	\item The shoe has padding around the collar, heel and tongue for added comfort.
	\end{itemize}
}
\item uncountable noun \\
\textbf{Comfort} is a style of life in which you have enough money to have everything you need .
 \textit{
	\begin{itemize}
	\item Thanks to the success of her books, she lives in comfort.
	\end{itemize}
}
\item uncountable noun \\
\textbf{Comfort} is what you feel when worries or unhappiness stop .
 \textit{
	\begin{itemize}
	\item He welcomed the truce, but pointed out it was of little comfort to families spending
Christmas without a loved one.
	\item He will be able to take some comfort from inflation figures due on Friday.
	\item He found comfort in Eva's blind faith in him.
	\end{itemize}
}
\item countable noun \\
If you refer to a person, thing, or idea as a \textbf{comfort} , you mean that it helps you to stop worrying or makes you feel less unhappy .
 \textit{
	\begin{itemize}
	\item It's a comfort talking to you.
	\item Being able to afford a drink would be a comfort in these tough times.
	\end{itemize}
}
\item verb \\
If you \textbf{comfort} someone, you make them feel less worried, unhappy, or upset , for example by saying  kind things to them.
 \textit{
	\begin{itemize}
	\item Ned put his arm around her, trying to comfort her.
	\end{itemize}
}
\item countable noun \\
\textbf{Comforts} are things which make your life easier and more pleasant , such as electrical devices you have in your home .
 \textit{
	\begin{itemize}
	\item She enjoys the material comforts this jet-set lifestyle has to offer.
	\item Electricity provides us with warmth and light and all our modern home comforts.
	\item I do like my comforts.
	\end{itemize}
}
\item  \\
 too close etc for comfort \textit{
	\begin{itemize}
	\end{itemize}
}
\end{enumerate}

\section*{burglar}
{\large \color{blue}  burglars  }
\subsection*{Explain}
\begin{enumerate}
\item countable noun \\
A \textbf{burglar} is a thief who enters a house or other building by force.
 \textit{
	\begin{itemize}
	\item Burglars broke into their home.
	\end{itemize}
}
\end{enumerate}

\section*{degree}
{\large \color{blue}  degrees  }
\subsection*{Explain}
\begin{enumerate}
\item countable noun \\
You use \textbf{degree} to indicate the extent to which something happens or is the case , or the amount which something is felt .
 \textit{
	\begin{itemize}
	\item These man-made barriers will ensure a very high degree of protection.
	\item Politicians have used television with varying degrees of success.
	\end{itemize}
}
\item uncountable noun \\
You use \textbf{degree} in expressions such as \textbf{a matter of degree} and \textbf{different in degree} to indicate that you are talking about the comparative quantity, scale, or extent of something, rather than other
 factors .
 \textit{
	\begin{itemize}
	\item The first change is a matter of degree, the second is a fundamental shift.
	\item The jobs are different in degree of difficulty.
	\end{itemize}
}
\item countable noun \\
A \textbf{degree} is a unit of measurement that is used to measure temperatures. It is often written
as °, for example 23°.
 \textit{
	\begin{itemize}
	\item It's over 80 degrees outside.
	\item Pure water sometimes does not freeze until it reaches minus 40 degrees Celsius.
	\end{itemize}
}
\item countable noun \\
A \textbf{degree} is a unit of measurement that is used to measure angles, and also longitude and latitude. It is often written as °, for example 23°.
 \textit{
	\begin{itemize}
	\item It was pointing outward at an angle of 45 degrees.
	\item ...McMurdo Station in Antarctica, which is at 78 degrees South.
	\end{itemize}
}
\item countable noun \\
A \textbf{degree} at a university or college is a course of study that you take there, or the qualification that you get when you have passed the course.
 \textit{
	\begin{itemize}
	\item He took a master's degree in economics at Yale.
	\item ...an engineering degree.
	\item ...the first year of a degree course.
	\end{itemize}
}
\item  \\
 by degrees \textit{
	\begin{itemize}
	\end{itemize}
}
\item  \\
 to some/a certain degree (etc) \textit{
	\begin{itemize}
	\end{itemize}
}
\item  \\
 to what/that degree/the degree that(etc) \textit{
	\begin{itemize}
	\end{itemize}
}
\end{enumerate}

\section*{campus}
{\large \color{blue}  campuses  }
\subsection*{Explain}
\begin{enumerate}
\item countable noun \\
A \textbf{campus} is an area of land that contains the main buildings of a university or college.
 \textit{
	\begin{itemize}
	\item ...during a rally at the campus.
	\item Private automobiles are not allowed on campus.
	\end{itemize}
}
\end{enumerate}

\section*{evening}
{\large \color{blue}  evenings  }
\subsection*{Explain}
\begin{enumerate}
\item variable noun \\
The \textbf{evening} is the part of each day between the end of the afternoon and the time when you go to bed .
 \textit{
	\begin{itemize}
	\item All he did that evening was sit around the flat.
	\item Supper is from 5.00 to 6.00 in the evening.
	\item Towards evening the carnival entered its final stage.
	\end{itemize}
}
\end{enumerate}

\section*{canteen}
{\large \color{blue}  canteens  }
\subsection*{Explain}
\begin{enumerate}
\item countable noun \\
A \textbf{canteen} is a place in a factory, shop, or college where meals are served to the people who work or study there.
 \textit{
	\begin{itemize}
	\item Rennie had eaten his tea in the canteen.
	\item ...a school canteen.
	\item ...canteen food.
	\end{itemize}
}
\item countable noun \\
A \textbf{canteen} is a small plastic  bottle for carrying water and other drinks. Canteens are used by soldiers.
 \textit{
	\begin{itemize}
	\item ...a full canteen of water.
	\end{itemize}
}
\item countable noun \\
A \textbf{canteen}  \textbf{of} cutlery is a set of knives , forks , and spoons in a specially designed box.
 \textit{
	\begin{itemize}
	\end{itemize}
}
\end{enumerate}

\section*{excursion}
{\large \color{blue}  excursions  }
\subsection*{Explain}
\begin{enumerate}
\item countable noun \\
You can refer to a short journey as an \textbf{excursion} , especially if it is made for pleasure or enjoyment .
 \textit{
	\begin{itemize}
	\item In Bermuda, Sam's father took him on an excursion to a coral barrier.
	\end{itemize}
}
\item countable noun \\
An \textbf{excursion} is a trip or visit to an interesting place, especially one that is arranged or recommended by a holiday company or tourist organization.
 \textit{
	\begin{itemize}
	\item We also recommend a full day optional excursion to the Upper Douro.
	\item Another pleasant excursion is Malaga, 18 miles away.
	\end{itemize}
}
\item countable noun \\
If you describe an activity as an \textbf{excursion}  \textbf{into} something, you mean that it is an attempt to develop or understand something new that you have not experienced before.
 \textit{
	\begin{itemize}
	\item ...Radio 3's latest excursion into ethnic music, dance and literature.
	\item The few excursions into stylistic experiment do not entirely come off.
	\end{itemize}
}
\end{enumerate}

\section*{cardinal}
{\large \color{blue}  cardinals  }
\subsection*{Explain}
\begin{enumerate}
\item countable noun \\
A \textbf{cardinal} is a high-ranking  priest in the Catholic Church.
 \textit{
	\begin{itemize}
	\item In 1448, Nicholas was appointed a cardinal.
	\item They were encouraged by a promise from Cardinal Winning.
	\end{itemize}
}
\item adjective \\
A \textbf{cardinal}  rule or quality is the one that is considered to be the most important.
 \textit{
	\begin{itemize}
	\item As a salesperson, your cardinal rule is to do everything you can to satisfy a customer.
	\item Harmony, balance and order are cardinal virtues to the French.
	\end{itemize}
}
\item countable noun \\
A \textbf{cardinal} is a common North American bird. The male has bright red feathers .
 \textit{
	\begin{itemize}
	\end{itemize}
}
\end{enumerate}

\section*{file}
{\large \color{blue}  files  filing  filed  }
\subsection*{Explain}
\begin{enumerate}
\item countable noun \\
A \textbf{file} is a box or a folded piece of heavy paper or plastic in which letters or documents are kept.
 \textit{
	\begin{itemize}
	\item He sat behind a table on which were half a dozen files.
	\item ...a file of insurance papers.
	\end{itemize}
}
\item countable noun \\
A \textbf{file} is a collection of information about a particular person or thing.
 \textit{
	\begin{itemize}
	\item There was stuff in that file that was private between me and Dr Denny.
	\item We already have files on people's tax details, mortgages and poll tax.
	\item You must record and keep a file of all expenses.
	\end{itemize}
}
\item verb \\
If you \textbf{file} a document, you put it in the correct file.
 \textit{
	\begin{itemize}
	\item They are all filed alphabetically under author.
	\end{itemize}
}
\item countable noun \\
In computing , a \textbf{file} is a set of related data that has its own name.
 \textit{
	\begin{itemize}
	\end{itemize}
}
\item verb \\
If you \textbf{file} a formal or legal accusation , complaint , or request , you make it officially .
 \textit{
	\begin{itemize}
	\item A number of them have filed formal complaints against the police.
	\item I filed for divorce on the grounds of adultery a few months later.
	\end{itemize}
}
\item verb \\
When someone \textbf{files} a report or a news story , they send or give it to their employer .
 \textit{
	\begin{itemize}
	\item Catherine Bond filed that report for the BBC from Nairobi.
	\item He had to rush back to the office and file a housing story before the secretaries
went home.
	\end{itemize}
}
\item verb \\
When a group of people \textbf{files}  somewhere , they walk one behind the other in a line.
 \textit{
	\begin{itemize}
	\item She paused as the group of children filed out of the house.
	\item Slowly, people filed into the room and sat down.
	\end{itemize}
}
\item countable noun \\
A \textbf{file} is a hand tool which is used for rubbing hard objects to make them smooth, shape them, or cut through them.
 \textit{
	\begin{itemize}
	\end{itemize}
}
\item verb \\
If you \textbf{file} an object, you smooth it, shape it, or cut it with a file.
 \textit{
	\begin{itemize}
	\item Manicurists are skilled at shaping and filing nails.
	\end{itemize}
}
\item  \\
 on file/ on sb's files \textit{
	\begin{itemize}
	\end{itemize}
}
\item  \\
 in single file \textit{
	\begin{itemize}
	\end{itemize}
}
\end{enumerate}

\section*{career}
{\large \color{blue}  careers  careering  careered  }
\subsection*{Explain}
\begin{enumerate}
\item countable noun \\
A \textbf{career} is the job or profession that someone does for a long period of their life.
 \textit{
	\begin{itemize}
	\item She is now concentrating on a career as a fashion designer.
	\item Dennis had recently begun a successful career conducting opera.
	\item ...a career in journalism.
	\item ...a political career.
	\end{itemize}
}
\item countable noun \\
Your \textbf{career} is the part of your life that you spend  working .
 \textit{
	\begin{itemize}
	\item During his career, he wrote more than fifty plays.
	\item She began her career as a teacher.
	\end{itemize}
}
\item adjective \\
\textbf{Careers}  advice or guidance in British English, or \textbf{career} advice or guidance in American English, consists of information about different  jobs and help with deciding what kind of job you want to do.
 \textit{
	\begin{itemize}
	\item She received very little careers guidance when young.
	\item Get hold of the company list from your careers advisory service.
	\end{itemize}
}
\item verb \\
If a person or vehicle  \textbf{careers}  somewhere , they move fast and in an uncontrolled way.
 \textit{
	\begin{itemize}
	\item His car careered into a river.
	\item He went careering off down the track.
	\end{itemize}
}
\end{enumerate}

\section*{focus}
{\large \color{blue}  foci  focuses  focusing  focused  }
\subsection*{Explain}
\begin{enumerate}
\item verb \\
If you \textbf{focus}  \textbf{on} a particular topic or if your attention \textbf{is focused}  \textbf{on} it, you concentrate on it and think about it, discuss it, or deal with it, rather than dealing with other topics.
 \textit{
	\begin{itemize}
	\item The research effort has focused on tracing the effects of growing levels of five
compounds.
	\item He is currently focusing on assessment and development.
	\item Today he was able to focus his message exclusively on the economy.
	\item Many of the papers focus their attention on the controversy surrounding the Foreign
Secretary.
	\end{itemize}
}
\item countable noun \\
\textbf{The}  \textbf{focus} of something is the main topic or main thing that it is concerned with.
 \textit{
	\begin{itemize}
	\item The U.N.'s role in promoting peace is increasingly the focus of international attention.
	\item The new system is the focus of controversy.
	\item Her children are the main focus of her life.
	\end{itemize}
}
\item countable noun \\
Your \textbf{focus} on something is the special attention that you pay it.
 \textit{
	\begin{itemize}
	\item He said his sudden focus on foreign policy was not motivated by presidential politics.
	\item The report's focus is on how technology affects human life rather than business.
	\item IBM has also shifted its focus from mainframes to personal computers.
	\end{itemize}
}
\item uncountable noun \\
If you say that something has a \textbf{focus} , you mean that you can see a purpose in it.
 \textit{
	\begin{itemize}
	\item Somehow, though, their latest album has a focus that the others have lacked.
	\item Suddenly all of the seemingly isolated examples took on a meaningful focus.
	\end{itemize}
}
\item verb \\
If you \textbf{focus} your eyes or if your eyes \textbf{focus} , your eyes adjust so that you can clearly see the thing that you want to look at. If you \textbf{focus} a camera , telescope , or other instrument, you adjust it so that you can see clearly through it.
 \textit{
	\begin{itemize}
	\item Kelly couldn't focus his eyes well enough to tell if the figure was male or female.
	\item His eyes slowly began to focus on what looked like a small dark ball.
	\item He found the binoculars and focused them on the boat.
	\item Had she kept the camera focused on the river bank she might have captured a vital
scene.
	\end{itemize}
}
\item uncountable noun \\
You use \textbf{focus} to refer to the fact of adjusting your eyes or a camera, telescope, or other instrument, and to the degree to which you can see clearly.
 \textit{
	\begin{itemize}
	\item His focus switched to the little white ball.
	\item Together these factors determine the depth of focus.
	\item It has no manual focus facility.
	\end{itemize}
}
\item verb \\
If you \textbf{focus}  rays of light on a particular point, you pass them through a lens or reflect them from a mirror so that they meet at that point.
 \textit{
	\begin{itemize}
	\item Magnetic coils focus the electron beams into fine spots.
	\end{itemize}
}
\item countable noun \\
The \textbf{focus} of a number of rays or lines is the point at which they meet.
 \textit{
	\begin{itemize}
	\end{itemize}
}
\item  \\
 in focus \textit{
	\begin{itemize}
	\end{itemize}
}
\item  \\
 in focus \textit{
	\begin{itemize}
	\end{itemize}
}
\item  \\
 out of focus \textit{
	\begin{itemize}
	\end{itemize}
}
\item  \\
 out of focus \textit{
	\begin{itemize}
	\end{itemize}
}
\end{enumerate}

\section*{cart}
{\large \color{blue}  carts  carting  carted  }
\subsection*{Explain}
\begin{enumerate}
\item countable noun \\
A \textbf{cart} is an old-fashioned  wooden vehicle that is used for transporting goods or people. Some carts are pulled by animals.
 \textit{
	\begin{itemize}
	\item ...a country where horse-drawn carts far outnumber cars.
	\end{itemize}
}
\item verb \\
If you \textbf{cart} things or people somewhere , you carry them or transport them there, often with difficulty .
 \textit{
	\begin{itemize}
	\item One of their father's relatives carted off the entire contents of the house.
	\item One of them protests loudly, and the Americans cart him away in plastic handcuffs.
	\item I've been trying to cut down on the stuff that I cart around with me.
	\end{itemize}
}
\item countable noun \\
A \textbf{cart} is a small vehicle with a motor .
 \textit{
	\begin{itemize}
	\item Cars are prohibited, so transportation is by electric cart or by horse and buggy.
	\item He drove up in a golf cart to watch them.
	\end{itemize}
}
\item countable noun \\
A \textbf{cart} or a \textbf{shopping cart} is a large metal  basket on wheels which is provided by shops such as supermarkets for customers to use while they are in the shop.
 \textit{
	\begin{itemize}
	\end{itemize}
}
\item  \\
 to put the cart before the horse \textit{
	\begin{itemize}
	\end{itemize}
}
\end{enumerate}

\section*{frog}
{\large \color{blue}  frogs  }
\subsection*{Explain}
\begin{enumerate}
\item countable noun \\
A \textbf{frog} is a small creature with smooth skin, big  eyes , and long back legs which it uses for jumping . Frogs usually live near water.
 \textit{
	\begin{itemize}
	\end{itemize}
}
\item countable noun \\
\textbf{Frogs} is sometimes used to refer to French people. 
 \textit{
	\begin{itemize}
	\end{itemize}
}
\end{enumerate}

\section*{cloth}
{\large \color{blue}  cloths  }
\subsection*{Explain}
\begin{enumerate}
\item variable noun \\
\textbf{Cloth} is fabric which is made by weaving or knitting a substance such as cotton, wool, silk , or nylon . Cloth is used especially for making clothes.
 \textit{
	\begin{itemize}
	\item She began cleaning the wound with a piece of cloth.
	\end{itemize}
}
\item countable noun \\
A \textbf{cloth} is a piece of cloth which you use for a particular purpose, such as cleaning something or covering something.
 \textit{
	\begin{itemize}
	\item Clean the surface with a damp cloth.
	\item ...a tray covered with a cloth.
	\end{itemize}
}
\item singular noun \\
\textbf{The cloth} is sometimes used to refer to Christian  priests and ministers .
 \textit{
	\begin{itemize}
	\item I've got as much respect for the cloth as the next man.
	\item ...a man of the cloth.
	\end{itemize}
}
\end{enumerate}

\section*{gasp}
{\large \color{blue}  gasps  gasping  gasped  }
\subsection*{Explain}
\begin{enumerate}
\item countable noun \\
A \textbf{gasp} is a short quick breath of air that you take in through your mouth , especially when you are surprised , shocked , or in pain .
 \textit{
	\begin{itemize}
	\item An audible gasp went round the court as the jury announced the verdict.
	\item She gave a small gasp of pain.
	\end{itemize}
}
\item verb \\
When you \textbf{gasp} , you take a short quick breath through your mouth, especially when you are surprised,
shocked, or in pain.
 \textit{
	\begin{itemize}
	\item She gasped for air and drew in a lungful of water.
	\item I heard myself gasp and cry out.
	\end{itemize}
}
\item  \\
 last gasp \textit{
	\begin{itemize}
	\end{itemize}
}
\end{enumerate}

\section*{conference}
{\large \color{blue}  conferences  }
\subsection*{Explain}
\begin{enumerate}
\item countable noun \\
A \textbf{conference} is a meeting, often lasting a few days, which is organized on a particular subject or to bring together people who have a common interest .
 \textit{
	\begin{itemize}
	\item The President summoned all the state governors to a conference on education.
	\item ...the Conservative Party conference.
	\item Last weekend the Roman Catholic Church in Scotland held a conference, attended by
450 delegates.
	\end{itemize}
}
\item countable noun \\
A \textbf{conference} is a meeting at which formal discussions take place.
 \textit{
	\begin{itemize}
	\item They sat down at the dinner table, as they always did, before the meal, for a conference.
	\item Her employer was in conference with two lawyers and did not want to be interrupted.
	\end{itemize}
}
\end{enumerate}

\section*{context}
{\large \color{blue}  contexts  }
\subsection*{Explain}
\begin{enumerate}
\item variable noun \\
The \textbf{context}  \textbf{of} an idea or event is the general  situation that relates to it, and which helps it to be understood .
 \textit{
	\begin{itemize}
	\item We are doing this work in the context of reforms in the economic, social and cultural
spheres.
	\item ...the historical context in which Chaucer wrote.
	\item This is the context in which the President must decide his policy.
	\end{itemize}
}
\item variable noun \\
The \textbf{context} of a word, sentence , or text consists of the words, sentences, or text before and after it which help to make
its meaning clear .
 \textit{
	\begin{itemize}
	\item Without a context, I would have assumed it was written by a man.
	\end{itemize}
}
\item  \\
 in context \textit{
	\begin{itemize}
	\end{itemize}
}
\item  \\
 out of context \textit{
	\begin{itemize}
	\end{itemize}
}
\end{enumerate}

\section*{hat}
{\large \color{blue}  hats  }
\subsection*{Explain}
\begin{enumerate}
\item countable noun \\
A \textbf{hat} is a head covering, often with a brim round it, which is usually worn out of doors to give  protection from the weather .
 \textit{
	\begin{itemize}
	\end{itemize}
}
\item countable noun \\
If you say that someone is wearing a particular  \textbf{hat} , you mean that they are performing a particular role at that time. If you say that they wear several \textbf{hats} , you mean that they have several roles or jobs .
 \textit{
	\begin{itemize}
	\item ...putting on my nationalistic hat.
	\item ...various problems, including too many people wearing too many hats.
	\end{itemize}
}
\item  \\
 at the drop of a hat \textit{
	\begin{itemize}
	\end{itemize}
}
\item  \\
 keep sth under your hat \textit{
	\begin{itemize}
	\end{itemize}
}
\item  \\
 old hat \textit{
	\begin{itemize}
	\end{itemize}
}
\item  \\
 pass the hat (around) \textit{
	\begin{itemize}
	\end{itemize}
}
\item  \\
 to take your hat off to someone \textit{
	\begin{itemize}
	\end{itemize}
}
\item  \\
 hats off to sb \textit{
	\begin{itemize}
	\end{itemize}
}
\item  \\
 pull sth out of a hat \textit{
	\begin{itemize}
	\end{itemize}
}
\item  \\
 draw/pick/pull sth out of a hat \textit{
	\begin{itemize}
	\end{itemize}
}
\end{enumerate}

\section*{couple}
{\large \color{blue}  couples  coupling  coupled  }
\subsection*{Explain}
\begin{enumerate}
\item quantifier \\
If you refer to \textbf{a couple of} people or things, you mean two or approximately two of them, although the exact number is not important or you are not sure of it.
 \textbf{Couple} is also a determiner in spoken American English, and before 'more' and 'less'.
 \textbf{Couple} is also a pronoun .
 \textit{
	\begin{itemize}
	\item Across the street from me there are a couple of police officers standing guard.
	\item I think the trouble will clear up in a couple of days.
	\item ...a small town a couple of hundred miles from New York City.
	\item ...a couple weeks before the election.
	\item I think I can play maybe for a couple more years.
	\item I've got a couple that don't look too bad.
	\end{itemize}
}
\item countable noun \\
A \textbf{couple} is two people who are married , living together, or having a sexual relationship .
 \textit{
	\begin{itemize}
	\item The couple have no children.
	\item ...after burglars ransacked an elderly couple's home.
	\item ...an isolated spot popular with courting couples.
	\end{itemize}
}
\item countable noun \\
A \textbf{couple} is two people that you see together on a particular occasion or that have some association .
 \textit{
	\begin{itemize}
	\item ...as the four couples began the opening dance.
	\item They were an odd couple.
	\end{itemize}
}
\item verb \\
If you say that one thing produces a particular effect when it \textbf{is coupled with} another, you mean that the two things combine to produce that effect.
 \textit{
	\begin{itemize}
	\item ...a problem that is coupled with lower demand for the machines themselves.
	\item Over-use of those drugs, coupled with poor diet, leads to physical degeneration.
	\item ...memories or past failures, coupled with a feeling of guilt.
	\end{itemize}
}
\item verb \\
If one piece of equipment  \textbf{is coupled}  \textbf{to} another, it is joined to it so that the two pieces of equipment work together.
 \textit{
	\begin{itemize}
	\item Its engine is coupled to a semiautomatic gearbox.
	\item The various systems are coupled together in complex arrays.
	\end{itemize}
}
\end{enumerate}

\section*{hierarchy}
{\large \color{blue}  hierarchies  }
\subsection*{Explain}
\begin{enumerate}
\item variable noun \\
A \textbf{hierarchy} is a system of organizing people into different ranks or levels of importance , for example in society or in a company.
 \textit{
	\begin{itemize}
	\item Like most other American companies with a rigid hierarchy, workers and managers had
strictly defined duties.
	\item She rose up the Tory hierarchy by the local government route.
	\item Even in the desert there was a kind of social hierarchy.
	\end{itemize}
}
\item countable noun \\
The \textbf{hierarchy} of an organization such as the Church is the group of people who manage and control it.
 \textit{
	\begin{itemize}
	\end{itemize}
}
\item countable noun \\
A \textbf{hierarchy}  \textbf{of}  ideas and beliefs involves organizing them into a system or structure.
 \textit{
	\begin{itemize}
	\item The notion of 'cultural imperialism' implies a hierarchy of cultures, some of which
are stronger than others.
	\end{itemize}
}
\end{enumerate}

\section*{dean}
{\large \color{blue}  deans  }
\subsection*{Explain}
\begin{enumerate}
\item countable noun \\
A \textbf{dean} is an important official at a university or college.
 \textit{
	\begin{itemize}
	\item She was Dean of the Science faculty at Sophia University.
	\end{itemize}
}
\item countable noun \\
A \textbf{dean} is a priest who is the main administrator of a large church.
 \textit{
	\begin{itemize}
	\item ...Alan Webster, former Dean of St Paul's.
	\end{itemize}
}
\item countable noun \\
The \textbf{dean} of a group is the most important member of that group.
 \textit{
	\begin{itemize}
	\end{itemize}
}
\end{enumerate}

\section*{intelligence}
{\large \color{blue}  }
\subsection*{Explain}
\begin{enumerate}
\item uncountable noun \\
\textbf{Intelligence} is the quality of being intelligent or clever .
 \textit{
	\begin{itemize}
	\item She's a woman of exceptional intelligence.
	\end{itemize}
}
\item uncountable noun \\
\textbf{Intelligence} is the ability to think , reason , and understand  instead of doing things automatically or by instinct .
 \textit{
	\begin{itemize}
	\item Nerve cells, after all, do not have intelligence of their own.
	\end{itemize}
}
\item uncountable noun \\
\textbf{Intelligence} is information that is gathered by the government or the army about their country's enemies and their activities.
 \textit{
	\begin{itemize}
	\item She first moved into the intelligence services 22 years ago.
	\item The purpose of intelligence is to provide information on how the enemy can be beaten.
	\item Why was military intelligence so lacking?
	\end{itemize}
}
\end{enumerate}

\section*{flock}
{\large \color{blue}  flocks  flocking  flocked  }
\subsection*{Explain}
\begin{enumerate}
\item countable noun \\
A \textbf{flock}  \textbf{of} birds, sheep, or goats is a group of them.
 \textit{
	\begin{itemize}
	\item They kept a small flock of sheep.
	\item They are gregarious birds and feed in flocks.
	\end{itemize}
}
\item countable noun \\
You can refer to a group of people or things as a \textbf{flock of} them to emphasize that there are a lot of them.
 \textit{
	\begin{itemize}
	\item These cases all attracted flocks of famous writers.
	\item ...his flock of advisers.
	\end{itemize}
}
\item verb \\
If people \textbf{flock}  \textbf{to} a particular place or event , a very large number of them go there, usually because it is pleasant or interesting .
 \textit{
	\begin{itemize}
	\item The public have flocked to the show.
	\item The criticisms will not stop people flocking to see the film.
	\item His greatest wish must be that huge crowds flock into the beautiful park.
	\end{itemize}
}
\item countable noun \\
A clergyman's \textbf{flock} is the group of Christians who come to his church or live in the area that he has responsibility for.
 \textit{
	\begin{itemize}
	\end{itemize}
}
\end{enumerate}

\section*{jacket}
{\large \color{blue}  jackets  }
\subsection*{Explain}
\begin{enumerate}
\item countable noun \\
A \textbf{jacket} is a short coat with long sleeves.
 \textit{
	\begin{itemize}
	\item ...a black leather jacket.
	\end{itemize}
}
\item countable noun \\
Potatoes baked in their \textbf{jackets} are baked with their skin on.
 \textit{
	\begin{itemize}
	\end{itemize}
}
\item countable noun \\
The \textbf{jacket} of a book is the paper cover that protects the book.
 \textit{
	\begin{itemize}
	\end{itemize}
}
\item countable noun \\
A record \textbf{jacket} is the cover in which a record is kept .
 \textit{
	\begin{itemize}
	\end{itemize}
}
\end{enumerate}

\section*{garment}
{\large \color{blue}  garments  }
\subsection*{Explain}
\begin{enumerate}
\item countable noun \\
A \textbf{garment} is a piece of clothing; used especially in contexts where you are talking about the manufacture or sale of clothes.
 \textit{
	\begin{itemize}
	\item Many of the garments have the customers' name tags sewn into the linings.
	\end{itemize}
}
\end{enumerate}

\section*{jaw}
{\large \color{blue}  jaws  }
\subsection*{Explain}
\begin{enumerate}
\item countable noun \\
Your \textbf{jaw} is the lower part of your face below your mouth. The movement of your jaw is sometimes  considered to express a particular emotion . For example , if your \textbf{jaw drops} , you are very surprised .
 \textit{
	\begin{itemize}
	\item He thought for a moment, stroking his well-defined jaw.
	\item Meg's jaw dropped in amazement.
	\item His jaw was set, but his voice sounded thin and unsure.
	\end{itemize}
}
\item countable noun \\
A person's or animal's \textbf{jaws} are the two bones in their head which their teeth are attached to.
 \textit{
	\begin{itemize}
	\item ...a forest rodent with powerful jaws.
	\end{itemize}
}
\item plural noun \\
If you talk about the \textbf{jaws of} something unpleasant such as death or hell , you are referring to a dangerous or unpleasant situation .
 \textit{
	\begin{itemize}
	\item A family dog rescued a newborn boy from the jaws of death.
	\item ...caught in the jaws of world recession.
	\end{itemize}
}
\end{enumerate}

\section*{heap}
{\large \color{blue}  heaps  heaping  heaped  }
\subsection*{Explain}
\begin{enumerate}
\item countable noun \\
A \textbf{heap}  \textbf{of} things is a pile of them, especially a pile arranged in a rather untidy way.
 \textit{
	\begin{itemize}
	\item ...a heap of bricks.
	\item ...a compost heap.
	\item He has dug up the tiles that cover the floor and left them in a heap.
	\end{itemize}
}
\item verb \\
If you \textbf{heap} things somewhere , you arrange them in a large pile.
 \textbf{Heap up}  means the same as heap .
 \textit{
	\begin{itemize}
	\item Mrs. Madrigal heaped more carrots onto Michael's plate.
	\item Off to one side, the militia was heaping up wood for a bonfire.
	\end{itemize}
}
\item verb \\
If you \textbf{heap}  praise or criticism  \textbf{on} someone or something, you give them a lot of praise or criticism.
 \textit{
	\begin{itemize}
	\item The head of the navy heaped scorn on both the methods and motives of the conspirators.
	\end{itemize}
}
\item quantifier \\
\textbf{Heaps of} something or a \textbf{heap of} something is a large quantity of it.
 \textit{
	\begin{itemize}
	\item You have heaps of time.
	\item ...a job that might suit someone with heaps of experience.
	\item I got in a heap of trouble.
	\end{itemize}
}
\item  \\
 at the bottom of the heap \textit{
	\begin{itemize}
	\end{itemize}
}
\item  \\
 in a heap \textit{
	\begin{itemize}
	\end{itemize}
}
\end{enumerate}

\section*{lock}
{\large \color{blue}  locks  locking  locked  }
\subsection*{Explain}
\begin{enumerate}
\item verb \\
When you \textbf{lock} something such as a door, drawer, or case, you fasten it, usually with a key , so that other people cannot open it.
 \textit{
	\begin{itemize}
	\item Are you sure you locked the front door?
	\item Wolfgang moved along the corridor towards the locked door at the end.
	\end{itemize}
}
\item countable noun \\
The \textbf{lock} on something such as a door or a drawer is the device which is used to keep it shut and prevent other people from opening it. Locks are opened with a key.
 \textit{
	\begin{itemize}
	\item At that moment he heard Gill's key turning in the lock of the door.
	\item An intruder forced open a lock on French windows at the house.
	\end{itemize}
}
\item verb \\
If you \textbf{lock} something or someone in a place, room, or container, you put them there and fasten
the lock.
 \textit{
	\begin{itemize}
	\item Her maid locked the case in the safe.
	\item They beat them up and locked them in a cell.
	\end{itemize}
}
\item verb \\
If you \textbf{lock} something in a particular position or if it \textbf{lock} there, it is held or fitted firmly in that position.
 \textit{
	\begin{itemize}
	\item He leaned back in the swivel chair and locked his fingers behind his head.
	\item There was a whine of hydraulics as the undercarriage locked into position.
	\end{itemize}
}
\item countable noun \\
On a canal or river, a \textbf{lock} is a place where walls have been built with gates at each end so that boats can move
to a higher or lower section of the canal or river, by gradually changing the water level inside the gates.
 \textit{
	\begin{itemize}
	\end{itemize}
}
\item countable noun \\
A \textbf{lock}  \textbf{of} hair is a small bunch of hairs on your head that grow together and curl or curve in the same direction.
 \textit{
	\begin{itemize}
	\item She brushed a lock of hair off his forehead.
	\end{itemize}
}
\item plural noun \\
Your \textbf{locks} are your hair.
 \textit{
	\begin{itemize}
	\item ...women with long, wavy locks.
	\end{itemize}
}
\item  \\
 under lock and key \textit{
	\begin{itemize}
	\end{itemize}
}
\end{enumerate}

\section*{horn}
{\large \color{blue}  horns  }
\subsection*{Explain}
\begin{enumerate}
\item countable noun \\
On a vehicle such as a car , the \textbf{horn} is the device that makes a loud noise as a signal or warning.
 \textit{
	\begin{itemize}
	\item He sounded the car horn.
	\end{itemize}
}
\item countable noun \\
The \textbf{horns} of an animal such as a cow or deer are the hard pointed things that grow from its head.
 \textit{
	\begin{itemize}
	\item A mature cow has horns.
	\end{itemize}
}
\item uncountable noun \\
\textbf{Horn} is the hard substance that the horns of animals are made of. Horn is sometimes used to make objects such as spoons , buttons , or ornaments .
 \textit{
	\begin{itemize}
	\end{itemize}
}
\item countable noun \\
A \textbf{horn} is a musical instrument of the brass family. It is a long circular metal tube, wide at one end, which you play by blowing .
 \textit{
	\begin{itemize}
	\end{itemize}
}
\item countable noun \\
A \textbf{horn} is a simple musical instrument consisting of a metal tube that is wide at one end and narrow at the other. You play it by blowing into it.
 \textit{
	\begin{itemize}
	\item ...a hunting horn.
	\end{itemize}
}
\item countable noun \\
A \textbf{horn} is a hollow curved object that is narrow at one end and wide at the other.
 \textit{
	\begin{itemize}
	\item ...a wind-up gramophone with a big horn.
	\end{itemize}
}
\item  \\
 blow one's own horn \textit{
	\begin{itemize}
	\end{itemize}
}
\item  \\
 lock horns \textit{
	\begin{itemize}
	\end{itemize}
}
\item  \\
 on the horns of a dilemma \textit{
	\begin{itemize}
	\end{itemize}
}
\item  \\
 pull in one's horns/draw in one's horns \textit{
	\begin{itemize}
	\end{itemize}
}
\end{enumerate}

\section*{market}
{\large \color{blue}  markets  marketing  marketed  }
\subsection*{Explain}
\begin{enumerate}
\item countable noun \\
A \textbf{market} is a place where goods are bought and sold, usually outdoors .
 \textit{
	\begin{itemize}
	\item He sold boots on a market stall.
	\end{itemize}
}
\item countable noun \\
The \textbf{market} for a particular type of thing is the number of people who want to buy it, or the area of the world in which it is sold.
 \textit{
	\begin{itemize}
	\item The foreign market was increasingly crucial.
	\item ...the Russian market for personal computers.
	\item But there is no youth market in cars.
	\end{itemize}
}
\item singular noun \\
The \textbf{market}  refers to the total  amount of a product that is sold each year , especially when you are talking about the competition between the companies who sell that product.
 \textit{
	\begin{itemize}
	\item The two big companies control 72% of the market.
	\end{itemize}
}
\item adjective \\
If you talk about a \textbf{market}  economy , or the \textbf{market}  price of something, you are referring to an economic system in which the prices of things depend on how many are available and how many people want to buy them, rather than prices being fixed by governments .
 \textit{
	\begin{itemize}
	\item Their ultimate aim was a market economy for Hungary.
	\item He must sell the house for the current market value.
	\item ...the market price of cocoa.
	\end{itemize}
}
\item verb \\
To \textbf{market} a product means to organize its sale, by deciding on its price, where it should be sold, and how it should be advertised .
 \textit{
	\begin{itemize}
	\item ...if you marketed our music the way you market pop music.
	\item They have been marketed largely to buyers in America.
	\item ...if a soap is marketed as an anti-acne product.
	\end{itemize}
}
\item singular noun \\
\textbf{The job market} or \textbf{the labour market} refers to the people who are looking for work and the jobs available for them to do.
 \textit{
	\begin{itemize}
	\item Every year, 250,000 people enter the job market.
	\item ...the changes in the labour market during the 1980s.
	\end{itemize}
}
\item singular noun \\
The stock market is sometimes referred to as \textbf{the market} .
 \textit{
	\begin{itemize}
	\item The market collapsed last October.
	\end{itemize}
}
\item  \\
 a buyer's/seller's market \textit{
	\begin{itemize}
	\end{itemize}
}
\item  \\
 in the market for something \textit{
	\begin{itemize}
	\end{itemize}
}
\item  \\
 on the market \textit{
	\begin{itemize}
	\end{itemize}
}
\item  \\
 to price yourself out of the market \textit{
	\begin{itemize}
	\end{itemize}
}
\end{enumerate}

\section*{introduction}
{\large \color{blue}  introductions  }
\subsection*{Explain}
\begin{enumerate}
\item countable noun \\
The \textbf{introduction}  \textbf{to} a book or talk is the part that comes at the beginning and tells you what the rest of the book or talk is about.
 \textit{
	\begin{itemize}
	\item Ellen Malos, in her introduction to 'The Politics of Housework', provides a summary
of the debates.
	\end{itemize}
}
\item countable noun \\
If you refer to a book as an \textbf{introduction}  \textbf{to} a particular subject, you mean that it explains the basic facts about that subject.
 \textit{
	\begin{itemize}
	\item On balance, the book is a friendly, down-to-earth introduction to physics.
	\item ...'Psychology and Language: An Introduction to Psycholinguistics'.
	\end{itemize}
}
\item countable noun \\
You can refer to a new product as an \textbf{introduction} when it becomes available in a place for the first time.
 \textit{
	\begin{itemize}
	\item There are two among their recent introductions that have greatly impressed me.
	\end{itemize}
}
\item  \\
 needs no introduction \textit{
	\begin{itemize}
	\end{itemize}
}
\end{enumerate}

\section*{negro}
{\large \color{blue}  Negroes  }
\subsection*{Explain}
\begin{enumerate}
\item countable noun \\
A \textbf{Negro} is someone with dark  skin who comes from Africa or whose ancestors  came from Africa.
 \textit{
	\begin{itemize}
	\end{itemize}
}
\end{enumerate}

\section*{juice}
{\large \color{blue}  juices  }
\subsection*{Explain}
\begin{enumerate}
\item variable noun \\
\textbf{Juice} is the liquid that can be obtained from a fruit.
 \textit{
	\begin{itemize}
	\item ...fresh orange juice.
	\item Soak the couscous overnight in the juice of about six lemons.
	\end{itemize}
}
\item plural noun \\
The \textbf{juices} of a piece of meat are the liquid that comes out of it when you cook it.
 \textit{
	\begin{itemize}
	\item When cooked, drain off the juices and put the meat in a processor or mincer.
	\end{itemize}
}
\item plural noun \\
The \textbf{juices} in your stomach are the fluids that help you to digest  food .
 \textit{
	\begin{itemize}
	\end{itemize}
}
\end{enumerate}

\section*{noise}
{\large \color{blue}  noises  }
\subsection*{Explain}
\begin{enumerate}
\item uncountable noun \\
\textbf{Noise} is a loud or unpleasant sound.
 \textit{
	\begin{itemize}
	\item There was too much noise in the room and he needed peace.
	\item The noise of bombs and guns was incessant.
	\item The baby was filled with alarm at the darkness and the noise.
	\end{itemize}
}
\item countable noun \\
A \textbf{noise} is a sound that someone or something makes.
 \textit{
	\begin{itemize}
	\item Sir Gerald made a small noise in his throat.
	\item ...birdsong and other animal noises.
	\item She'd been working in her room till a noise had disturbed her.
	\end{itemize}
}
\item plural noun \\
If someone \textbf{makes noises} of a particular kind about something, they say things that indicate their attitude to it in a rather indirect or vague way.
 \textit{
	\begin{itemize}
	\item The President took care to make encouraging noises about the future.
	\item His mother had also started making noises about it being time for him to leave home.
	\end{itemize}
}
\item  \\
 make the right noises/make all the right noises \textit{
	\begin{itemize}
	\end{itemize}
}
\end{enumerate}

\section*{novel}
{\large \color{blue}  novels  }
\subsection*{Explain}
\begin{enumerate}
\item countable noun \\
A \textbf{novel} is a long written story about imaginary people and events.
 \textit{
	\begin{itemize}
	\item ...a novel by Herman Hesse.
	\item ...historical novels set in the time of the Pharaohs.
	\end{itemize}
}
\item adjective \\
\textbf{Novel} things are new and different from anything that has been done, experienced , or made before.
 \textit{
	\begin{itemize}
	\item Protesters found a novel way of demonstrating against steeply rising oil prices.
	\item The very idea of a sixth form college was novel in 1962.
	\end{itemize}
}
\end{enumerate}

\section*{pant}
{\large \color{blue}  pants  panting  panted  }
\subsection*{Explain}
\begin{enumerate}
\item verb \\
If you \textbf{pant} , you breathe quickly and loudly with your mouth open, because you have been doing something energetic .
 \textit{
	\begin{itemize}
	\item She climbed rapidly until she was panting with the effort.
	\end{itemize}
}
\end{enumerate}

\section*{objection}
{\large \color{blue}  objections  }
\subsection*{Explain}
\begin{enumerate}
\item variable noun \\
If you make or raise an \textbf{objection}  \textbf{to} something, you say that you do not like it or agree with it.
 \textit{
	\begin{itemize}
	\item Some managers have recently raised objection to the PFA handling these negotiations.
	\item Despite objections by the public, the government voted today to cut off aid.
	\end{itemize}
}
\item uncountable noun \\
If you say that you have \textbf{no}  \textbf{objection}  \textbf{to} something, you mean that you are not annoyed or bothered by it.
 \textit{
	\begin{itemize}
	\item I have no objection to banks making money.
	\item I no longer have any objection to your going to see her.
	\end{itemize}
}
\end{enumerate}

\section*{phrase}
{\large \color{blue}  phrases  phrasing  phrased  }
\subsection*{Explain}
\begin{enumerate}
\item countable noun \\
A \textbf{phrase} is a short group of words that people often use as a way of saying something. The meaning of a phrase is often not obvious from the meaning of the individual words in it.
 \textit{
	\begin{itemize}
	\item He used a phrase I hate: 'You have to be cruel to be kind.'
	\item ...the American phrase 'laying an egg' meaning to fail at something.
	\end{itemize}
}
\item countable noun \\
A \textbf{phrase} is a small group of words which forms a unit, either on its own or within a sentence .
 \textit{
	\begin{itemize}
	\item It is impossible to hypnotise someone simply by saying a particular word or phrase.
	\end{itemize}
}
\item verb \\
If you \textbf{phrase} something in a particular way, you express it in words in that way.
 \textit{
	\begin{itemize}
	\item I would have phrased it quite differently.
	\item The speech was carefully phrased.
	\item They phrased it as a question.
	\end{itemize}
}
\item  \\
 turn of phrase \textit{
	\begin{itemize}
	\end{itemize}
}
\end{enumerate}

\section*{peasant}
{\large \color{blue}  peasants  }
\subsection*{Explain}
\begin{enumerate}
\item countable noun \\
A \textbf{peasant} is a poor person of low social status who works on the land; used of people who live in countries where farming is still a common way of life.
 \textit{
	\begin{itemize}
	\item ...the peasants in the Peruvian highlands.
	\item Chinese peasants farm their own plots.
	\end{itemize}
}
\end{enumerate}

\section*{procedure}
{\large \color{blue}  procedures  }
\subsection*{Explain}
\begin{enumerate}
\item variable noun \\
A \textbf{procedure} is a way of doing something, especially the usual or correct way.
 \textit{
	\begin{itemize}
	\item A biopsy is usually a minor surgical procedure.
	\item Police insist that he did not follow the correct procedure in applying for a visa.
	\item The White House said there would be no change in procedure.
	\end{itemize}
}
\end{enumerate}

\section*{remainder}
{\large \color{blue}  remainders  remaindering  remaindered  }
\subsection*{Explain}
\begin{enumerate}
\item quantifier \\
\textbf{The remainder}  \textbf{of} a group are the things or people that still  remain after the other things or people have gone or have been dealt with.
 \textbf{Remainder} is also a pronoun .
 \textit{
	\begin{itemize}
	\item He gulped down the remainder of his coffee.
	\item I spent the remainder of the day feeling terrible.
	\item Only 5.9 per cent of the area is covered in trees. Most of the remainder is farmland.
	\item A quarter of finalists hoped to go travelling. The remainder were undecided about
their plans.
	\end{itemize}
}
\item singular noun \\
In arithmetic , \textbf{the remainder} is the amount that remains when one amount cannot be exactly divided by another.
For example , if you divide 22 by 7, the answer is 3 and the remainder is 1.
 \textit{
	\begin{itemize}
	\end{itemize}
}
\item verb \\
If a book \textbf{is remaindered} , it is sold at a reduced price because it has not been selling very well and will not be published again.
 \textit{
	\begin{itemize}
	\item It failed to sell and was soon remaindered.
	\end{itemize}
}
\item countable noun \\
A \textbf{remainder} is a book that has been remaindered.
 \textit{
	\begin{itemize}
	\end{itemize}
}
\end{enumerate}

\section*{privacy}
{\large \color{blue}  }
\subsection*{Explain}
\begin{enumerate}
\item uncountable noun \\
If you have \textbf{privacy} , you are in a place or situation which allows you to do things without other people seeing you or disturbing you.
 \textit{
	\begin{itemize}
	\item He saw the publication of this book as an embarrassing invasion of his privacy.
	\item Thatched pavilions provide shady retreats for relaxing and reading in privacy.
	\item ...a collection of over 60 designs to try on in the privacy of your own home.
	\end{itemize}
}
\item  \\
 to invade someone's privacy \textit{
	\begin{itemize}
	\end{itemize}
}
\end{enumerate}

\section*{remnant}
{\large \color{blue}  remnants  }
\subsection*{Explain}
\begin{enumerate}
\item countable noun \\
The \textbf{remnants}  \textbf{of} something are small parts of it that are left over when the main part has disappeared or been destroyed .
 \textit{
	\begin{itemize}
	\item After twenty-four hours of fighting, the remnants of the force were fleeing.
	\item Beneath the present church were remnants of Roman flooring.
	\end{itemize}
}
\item countable noun \\
A \textbf{remnant} is a small piece of cloth that is left over when most of the cloth has been sold. Shops usually sell remnants cheaply.
 \textit{
	\begin{itemize}
	\end{itemize}
}
\end{enumerate}

\section*{resistance}
{\large \color{blue}  resistances  }
\subsection*{Explain}
\begin{enumerate}
\item uncountable noun \\
\textbf{Resistance} to something such as a change or a new idea is a refusal to accept it.
 \textit{
	\begin{itemize}
	\item The U.S. wants big cuts in European agricultural export subsidies, but this is meeting
resistance.
	\end{itemize}
}
\item uncountable noun \\
\textbf{Resistance} to an attack consists of fighting back against the people who have attacked you.
 \textit{
	\begin{itemize}
	\item The troops are encountering stiff resistance.
	\item Police in riot gear cleared the noisy demonstrators, who offered no resistance.
	\end{itemize}
}
\item uncountable noun \\
The \textbf{resistance} of your body \textbf{to}  germs or diseases is its power to remain  unharmed or unaffected by them.
 \textit{
	\begin{itemize}
	\item This disease is surprisingly difficult to catch as most people have a natural resistance
to it.
	\end{itemize}
}
\item uncountable noun \\
Wind or air \textbf{resistance} is a force which slows down a moving object or vehicle.
 \textit{
	\begin{itemize}
	\item The design of the bicycle has managed to reduce the effects of wind resistance and
drag.
	\end{itemize}
}
\item variable noun \\
In electrical  engineering or physics , \textbf{resistance} is the ability of a substance or an electrical circuit to stop the flow of an electrical current through it.
 \textit{
	\begin{itemize}
	\item ...materials that lose all their electrical resistance.
	\end{itemize}
}
\item singular noun \\
In a country which is occupied by the army of another country, or which has a very harsh and strict government, \textbf{the resistance} is an organized group of people who are involved in illegal activities against the people in power.
 \textit{
	\begin{itemize}
	\item They managed to escape after being arrested by the resistance.
	\end{itemize}
}
\item  \\
 the line of least resistance \textit{
	\begin{itemize}
	\end{itemize}
}
\end{enumerate}

\section*{reserve}
{\large \color{blue}  reserves  reserving  reserved  }
\subsection*{Explain}
\begin{enumerate}
\item verb \\
If something \textbf{is reserved}  \textbf{for} a particular person or purpose, it is kept specially for that person or purpose.
 \textit{
	\begin{itemize}
	\item A double room with a balcony overlooking the sea had been reserved for him.
	\item In the United States lanes are reserved for cars with more than one occupant.
	\end{itemize}
}
\item verb \\
If you \textbf{reserve} something such as a table, ticket , or magazine , you arrange for it to be kept specially for you, rather than sold or given to someone
else.
 \textit{
	\begin{itemize}
	\item I'll reserve a table for five.
	\item Demand will be huge, so ask your newsagent to reserve your copy today.
	\end{itemize}
}
\item countable noun \\
A \textbf{reserve} is a supply of something that is available for use when it is needed.
 \textit{
	\begin{itemize}
	\item The Gulf has 65 per cent of the world's oil reserves.
	\item A friend can be a reserve of help in times of trouble.
	\end{itemize}
}
\item countable noun \\
In sports, a \textbf{reserve} is someone who is available to play as part of a team if one of the members is ill or cannot play.
 \textit{
	\begin{itemize}
	\item He ended up as a reserve, but still qualified for a team gold medal.
	\end{itemize}
}
\item countable noun \\
A nature \textbf{reserve} is an area of land where the animals, birds, and plants are officially protected.
 \textit{
	\begin{itemize}
	\item Marine biologists are calling for Cardigan Bay to be created a marine nature reserve
to protect the dolphins.
	\end{itemize}
}
\item uncountable noun \\
If someone shows \textbf{reserve} , they keep their feelings hidden .
 \textit{
	\begin{itemize}
	\item The subject is one which must be discussed with reserve.
	\item His natural reserve made him appear self-conscious.
	\end{itemize}
}
\item  \\
 in reserve \textit{
	\begin{itemize}
	\end{itemize}
}
\item countable noun \\
A military \textbf{reserve} is a group of soldiers who are ready to join a military operation if they are needed.
 \textit{
	\begin{itemize}
	\end{itemize}
}
\end{enumerate}

\section*{row}
{\large \color{blue}  rows  }
\subsection*{Explain}
\begin{enumerate}
\item countable noun \\
A \textbf{row}  \textbf{of} things or people is a number of them arranged in a line.
 \textit{
	\begin{itemize}
	\item ...a row of pretty little cottages.
	\item Several men are pushing school desks and chairs into neat rows.
	\end{itemize}
}
\item countable noun \\
In a theatre or cinema, or on a plane , each line of seats is called a \textbf{row} .
 \textit{
	\begin{itemize}
	\item She was sitting in the front row.
	\end{itemize}
}
\item countable noun \\
\textbf{Row} is sometimes used in the names of streets.
 \textit{
	\begin{itemize}
	\item ...the house at 236 Larch Row.
	\end{itemize}
}
\item  \\
 in a row \textit{
	\begin{itemize}
	\end{itemize}
}
\end{enumerate}

\section*{shortcoming}
{\large \color{blue}  shortcomings  }
\subsection*{Explain}
\begin{enumerate}
\item countable noun \\
Someone's or something's \textbf{shortcomings} are the faults or weaknesses which they have.
 \textit{
	\begin{itemize}
	\item Marriages usually break down as a result of the shortcomings of both partners.
	\item His book has its shortcomings.
	\end{itemize}
}
\end{enumerate}

\section*{rug}
{\large \color{blue}  rugs  }
\subsection*{Explain}
\begin{enumerate}
\item countable noun \\
A \textbf{rug} is a piece of thick material that you put on a floor. It is like a carpet but covers a smaller area.
 \textit{
	\begin{itemize}
	\item A Persian rug covered the hardwood floors.
	\end{itemize}
}
\item countable noun \\
A \textbf{rug} is a small blanket which you use to cover your shoulders or your knees to keep them warm .
 \textit{
	\begin{itemize}
	\item The old lady was seated in her chair at the window, a rug over her knees.
	\item ...a travel rug.
	\end{itemize}
}
\item  \\
 pull the rug from under someone/pull the rug from under someone's feet \textit{
	\begin{itemize}
	\end{itemize}
}
\end{enumerate}

\section*{street}
{\large \color{blue}  streets  }
\subsection*{Explain}
\begin{enumerate}
\item countable noun \\
A \textbf{street} is a road in a city, town, or village , usually with houses along it.
 \textit{
	\begin{itemize}
	\item He lived at 66 Bingfield Street.
	\item Boppard is a small, quaint town with narrow streets.
	\end{itemize}
}
\item countable noun \\
You can use \textbf{street} or \textbf{streets} when talking about activities that happen out of doors in a town rather than inside a building.
 \textit{
	\begin{itemize}
	\item Changing money on the street is illegal-always use a bank.
	\item Their aim is to raise a million pounds to get the homeless off the streets.
	\item ...a New York street gang.
	\end{itemize}
}
\item  \\
 be streets ahead of sb \textit{
	\begin{itemize}
	\end{itemize}
}
\item  \\
 the man in the street \textit{
	\begin{itemize}
	\end{itemize}
}
\item  \\
 up sb's street \textit{
	\begin{itemize}
	\end{itemize}
}
\end{enumerate}

\section*{seminar}
{\large \color{blue}  seminars  }
\subsection*{Explain}
\begin{enumerate}
\item countable noun \\
A \textbf{seminar} is a meeting where a group of people discuss a problem or topic .
 \textit{
	\begin{itemize}
	\item ...courses and seminars on nutrition and natural health.
	\end{itemize}
}
\item countable noun \\
A \textbf{seminar} is a class at a college or university in which the teacher and a small group of students discuss a topic.
 \textit{
	\begin{itemize}
	\item Students are asked to prepare material in advance of each weekly seminar.
	\end{itemize}
}
\end{enumerate}

\section*{tax}
{\large \color{blue}  taxes  taxing  taxed  }
\subsection*{Explain}
\begin{enumerate}
\item variable noun \\
\textbf{Tax} is an amount of money that you have to pay to the government so that it can pay for
public services.
 \textit{
	\begin{itemize}
	\item No-one enjoys paying tax.
	\item They are calling for large spending cuts and tax increases.
	\item ...a cut in tax on new cars.
	\item ...a pledge not to raise taxes on people below a certain income.
	\item His decision to return to a form of property tax is the right one.
	\end{itemize}
}
\item verb \\
When a person or company \textbf{is taxed} , they have to pay a part of their income or profits to the government. When goods \textbf{are taxed} , a percentage of their price has to be paid to the government.
 \textit{
	\begin{itemize}
	\item Husband and wife are now taxed separately on their incomes.
	\item ...the government's commitment to simplifying the way companies are taxed.
	\item The Bonn government taxes profits of corporations at a rate that is among the highest
in Europe.
	\end{itemize}
}
\item verb \\
If something \textbf{taxes} your strength , your patience , or your resources , it uses nearly all of them, so that you have great difficulty in carrying out what you are trying to do.
 \textit{
	\begin{itemize}
	\item Overcrowding has taxed the city's ability to deal with waste.
	\item These dilemmas would tax the best of statesmen.
	\end{itemize}
}
\end{enumerate}

\section*{session}
{\large \color{blue}  sessions  }
\subsection*{Explain}
\begin{enumerate}
\item countable noun \\
A \textbf{session} is a meeting of a court, parliament , or other official group.
 \textit{
	\begin{itemize}
	\item ...an emergency session of parliament.
	\item After two late night sessions, the Security Council has failed to reach agreement.
	\item The court was in session.
	\end{itemize}
}
\item countable noun \\
A \textbf{session} is a period during which the meetings of a court, parliament, or other official group
are regularly held.
 \textit{
	\begin{itemize}
	\item The parliamentary session ends on October 4th.
	\item From September until December, Congress remained in session.
	\end{itemize}
}
\item countable noun \\
A \textbf{session} of a particular activity is a period of that activity.
 \textit{
	\begin{itemize}
	\item The two leaders emerged for a photo session.
	\item ...group therapy sessions.
	\end{itemize}
}
\item adjective \\
\textbf{Session} musicians are employed to play  backing  music in recording studios.
 \textit{
	\begin{itemize}
	\item He established himself as a session musician.
	\item ...a session drummer.
	\end{itemize}
}
\end{enumerate}

\section*{temper}
{\large \color{blue}  tempers  tempering  tempered  }
\subsection*{Explain}
\begin{enumerate}
\item variable noun \\
If you refer to someone's \textbf{temper} or say that they have a \textbf{temper} , you mean that they become angry very easily .
 \textit{
	\begin{itemize}
	\item He had a temper and could be nasty.
	\item His short temper had become notorious.
	\item I hope he can control his temper.
	\end{itemize}
}
\item variable noun \\
Your \textbf{temper} is the way you are feeling at a particular time. If you are \textbf{in} a good \textbf{temper} , you feel  cheerful . If you are \textbf{in} a bad  \textbf{temper} , you feel angry and impatient .
 \textit{
	\begin{itemize}
	\item I was in a bad temper last night.
	\item He was in a very good temper.
	\item In a fit of bad temper, Dougie threw the deep fat fryer overboard.
	\end{itemize}
}
\item verb \\
To \textbf{temper} something means to make it less extreme .
 \textit{
	\begin{itemize}
	\item For others, especially the young and foolish, the state will temper justice with
mercy.
	\item He had to learn to temper his enthusiasm.
	\end{itemize}
}
\item  \\
 in/into a temper \textit{
	\begin{itemize}
	\end{itemize}
}
\item  \\
 to lose your temper \textit{
	\begin{itemize}
	\end{itemize}
}
\end{enumerate}

\section*{student}
{\large \color{blue}  students  }
\subsection*{Explain}
\begin{enumerate}
\item countable noun \\
A \textbf{student} is a person who is studying at a university or college.
 \textit{
	\begin{itemize}
	\item Warren's eldest son is an art student, at St Martin's.
	\item ...a 23-year-old medical student.
	\end{itemize}
}
\item countable noun \\
A \textbf{student} is a child who is studying at a secondary school.
 \textit{
	\begin{itemize}
	\end{itemize}
}
\item countable noun \\
Someone who is a \textbf{student of} a particular subject is interested in the subject and spends time learning about it.
 \textit{
	\begin{itemize}
	\item ...a passionate student of nineteenth century history.
	\end{itemize}
}
\end{enumerate}

\section*{throne}
{\large \color{blue}  thrones  }
\subsection*{Explain}
\begin{enumerate}
\item countable noun \\
A \textbf{throne} is a decorative  chair used by a king , queen , or emperor on important  official occasions.
 \textit{
	\begin{itemize}
	\end{itemize}
}
\item singular noun \\
You can talk about \textbf{the throne} as a way of referring to the position of being king, queen, or emperor.
 \textit{
	\begin{itemize}
	\item When Queen Victoria was on the throne, the horse was the main form of transport.
	\item ...the heir to the throne.
	\end{itemize}
}
\end{enumerate}

\section*{tumour}
{\large \color{blue}  tumours  }
\subsection*{Explain}
\begin{enumerate}
\item countable noun \\
A \textbf{tumour} is a mass of diseased or abnormal cells that has grown in a person's or animal's body.
 \textit{
	\begin{itemize}
	\end{itemize}
}
\end{enumerate}

\section*{wisdom}
{\large \color{blue}  wisdoms  }
\subsection*{Explain}
\begin{enumerate}
\item uncountable noun \\
\textbf{Wisdom} is the ability to use your experience and knowledge in order to make sensible  decisions or judgments .
 \textit{
	\begin{itemize}
	\item ...the patience and wisdom that comes from old age.
	\item ...a great man, who spoke words of great wisdom.
	\end{itemize}
}
\item variable noun \\
\textbf{Wisdom} is the store of knowledge that a society or culture has collected over a long period of time.
 \textit{
	\begin{itemize}
	\item ...a folksy piece of wisdom.
	\item ...this church's original Semitic wisdom, religion and faith.
	\item ...a simpler and more humane approach, based on ancient wisdoms and 'natural' mechanisms.
	\end{itemize}
}
\item singular noun \\
If you talk about \textbf{the wisdom of} a particular decision or action, you are talking about how sensible it is.
 \textit{
	\begin{itemize}
	\item Many Lithuanians have expressed doubts about the wisdom of the decision.
	\end{itemize}
}
\item variable noun \\
You can use \textbf{wisdom} to refer to ideas that are accepted by a large number of people.
 \textit{
	\begin{itemize}
	\item Health education wisdom in the U.K. differs from that of the United States.
	\item Unchallenged wisdoms flow swiftly among the middle classes.
	\end{itemize}
}
\end{enumerate}

\section*{valley}
{\large \color{blue}  valleys  }
\subsection*{Explain}
\begin{enumerate}
\item countable noun \\
A \textbf{valley} is a low stretch of land between hills , especially one that has a river flowing through it.
 \textit{
	\begin{itemize}
	\item ...a wooded valley set against the backdrop of Monte Rosa.
	\item ...the Loire valley.
	\end{itemize}
}
\end{enumerate}

\section*{wit}
{\large \color{blue}  wits  }
\subsection*{Explain}
\begin{enumerate}
\item uncountable noun \\
\textbf{Wit} is the ability to use words or ideas in an amusing , clever, and imaginative way.
 \textit{
	\begin{itemize}
	\item Boulding was known for his biting wit.
	\item They love her practical attitude to life, her zest and wit.
	\end{itemize}
}
\item countable noun \\
If you describe someone as a \textbf{wit} , you mean that they have the ability to use words or ideas in an amusing, clever,
and imaginative way.
 \textit{
	\begin{itemize}
	\item Holmes was gregarious, a great wit, a man of wide interests.
	\end{itemize}
}
\item singular noun \\
If you say that someone has \textbf{the wit}  \textbf{to} do something, you mean that they have the intelligence and understanding to make the right decision or take the right action in a particular situation .
 \textit{
	\begin{itemize}
	\item The information is there and waiting to be accessed by anyone with the wit to use
it.
	\end{itemize}
}
\item plural noun \\
You can refer to your ability to think quickly and cleverly in a difficult situation as your \textbf{wits} .
 \textit{
	\begin{itemize}
	\item She has used her wits to progress to the position she holds today.
	\end{itemize}
}
\item plural noun \\
You can use \textbf{wits} in expressions such as \textbf{frighten} someone \textbf{out of their wits} and \textbf{scare the wits out of} someone to emphasize that a person or thing worries or frightens someone very much.
 \textit{
	\begin{itemize}
	\item You scared us out of our wits. We heard you had an accident.
	\item ...a huge bass drum which frightened the wits out of the organist each time it was
banged.
	\end{itemize}
}
\item  \\
 have one's wits about one/keep one's wits about one \textit{
	\begin{itemize}
	\end{itemize}
}
\item  \\
 to be at your wits' end \textit{
	\begin{itemize}
	\end{itemize}
}
\item  \\
 pit one's wits against sb \textit{
	\begin{itemize}
	\end{itemize}
}
\item  \\
 to wit \textit{
	\begin{itemize}
	\end{itemize}
}
\end{enumerate}

\section*{vapour}
{\large \color{blue}  vapours  }
\subsection*{Explain}
\begin{enumerate}
\item variable noun \\
\textbf{Vapour} consists of tiny  drops of water or other liquids in the air, which appear as mist .
 \textit{
	\begin{itemize}
	\item ...water vapour.
	\end{itemize}
}
\end{enumerate}

\section*{zinc}
{\large \color{blue}  }
\subsection*{Explain}
\begin{enumerate}
\item uncountable noun \\
\textbf{Zinc} is a bluish-white metal which is used to make other metals such as brass, or to cover
other metals such as iron to stop a brown substance called  rust from forming.
 \textit{
	\begin{itemize}
	\end{itemize}
}
\end{enumerate}

\section*{advent}
{\large \color{blue}  }
\subsection*{Explain}
\begin{enumerate}
\item uncountable noun \\
\textbf{The}  \textbf{advent}  \textbf{of} an important event, invention , or situation is the fact of it starting or coming into existence .
 \textit{
	\begin{itemize}
	\item ...the leap forward in communication made possible by the advent of the mobile phone.
	\item The advent of war led to a greater austerity.
	\end{itemize}
}
\item uncountable noun \\
The \textbf{advent}  \textbf{of} a person at a place is their arrival there.
 \textit{
	\begin{itemize}
	\item Deptford had come alive with the advent of the new priest at St Paul's.
	\end{itemize}
}
\end{enumerate}

\section*{anger}
{\large \color{blue}  angers  angering  angered  }
\subsection*{Explain}
\begin{enumerate}
\item uncountable noun \\
\textbf{Anger} is the strong  emotion that you feel when you think that someone has behaved in an unfair , cruel , or unacceptable way.
 \textit{
	\begin{itemize}
	\item He cried with anger and frustration.
	\item Ellen felt both despair and anger at her mother.
	\end{itemize}
}
\item verb \\
If something \textbf{angers} you, it makes you feel angry.
 \textit{
	\begin{itemize}
	\item The decision to allow more offshore oil drilling angered some Californians.
	\end{itemize}
}
\end{enumerate}

\section*{album}
{\large \color{blue}  albums  }
\subsection*{Explain}
\begin{enumerate}
\item countable noun \\
An \textbf{album} is a collection of songs that is available for download , or as a CD or record. You can also  refer to the CD or record as an \textbf{album} .
 \textit{
	\begin{itemize}
	\item Chris likes music and has a large collection of albums.
	\end{itemize}
}
\item countable noun \\
An \textbf{album} is a book in which you keep things such as photographs or stamps that you have collected .
 \textit{
	\begin{itemize}
	\item Theresa showed me her photo album.
	\end{itemize}
}
\end{enumerate}

\section*{apology}
{\large \color{blue}  apologies  }
\subsection*{Explain}
\begin{enumerate}
\item variable noun \\
An \textbf{apology} is something that you say or write in order to tell someone that you are sorry that you have hurt them or caused trouble for them.
 \textit{
	\begin{itemize}
	\item I didn't get an apology.
	\item We received a letter of apology.
	\item He made a public apology for the team's performance.
	\end{itemize}
}
\item plural noun \\
If you offer or make your \textbf{apologies} , you apologize .
 \textit{
	\begin{itemize}
	\item His mother offered her apologies to the Jones family.
	\item When Mary finally appeared, she made her apologies to Mrs Madrigal.
	\end{itemize}
}
\item  \\
 make no apologies \textit{
	\begin{itemize}
	\end{itemize}
}
\end{enumerate}

\section*{attorney}
{\large \color{blue}  attorneys  }
\subsection*{Explain}
\begin{enumerate}
\item countable noun \\
In the United  States , an \textbf{attorney} or \textbf{attorney at law} is a lawyer.
 \textit{
	\begin{itemize}
	\item ...a prosecuting attorney.
	\end{itemize}
}
\end{enumerate}

\section*{chill}
{\large \color{blue}  chills  chilling  chilled  }
\subsection*{Explain}
\begin{enumerate}
\item verb \\
When you \textbf{chill} something or when it \textbf{chills} , you lower its temperature so that it becomes colder but does not freeze.
 \textit{
	\begin{itemize}
	\item Chill the fruit salad until serving time.
	\item These doughs can be rolled out while you wait for the pastry to chill.
	\item ...a glass of chilled champagne.
	\end{itemize}
}
\item verb \\
When cold weather or something cold \textbf{chills} a person or a place, it makes that person or that place feel very cold.
 \textit{
	\begin{itemize}
	\item The marble floor was beginning to chill me.
	\item An exposed garden may be chilled by cold winds.
	\item Wade placed his chilled hands on the radiator.
	\item The boulder sheltered them from the chilling wind.
	\end{itemize}
}
\item verb \\
If you say that something you see , hear , or feel \textbf{chills} you, you mean that it frightens you.
 \textit{
	\begin{itemize}
	\item There was a coldness in her that chilled him.
	\item Some films chill you to the marrow of your bones.
	\end{itemize}
}
\item countable noun \\
If something sends a \textbf{chill} through you, it gives you a sudden feeling of fear or anxiety .
 \textit{
	\begin{itemize}
	\item The violence used against the students sent a chill through Indonesia.
	\item He smiled, an odd, dreamy smile that sent chills up my back.
	\end{itemize}
}
\item countable noun \\
A \textbf{chill} is a mild  illness which can give you a slight  fever and headache .
 \textit{
	\begin{itemize}
	\item He caught a chill while performing at a rain-soaked open-air venue.
	\end{itemize}
}
\item adjective \\
\textbf{Chill} weather is cold and unpleasant .
 \textbf{Chill} is also a noun .
 \textit{
	\begin{itemize}
	\item ...chill winds, rain and choppy seas.
	\item September is here, bringing with it a chill in the mornings.
	\item ...the cold chill of the night.
	\end{itemize}
}
\end{enumerate}

\section*{banquet}
{\large \color{blue}  banquets  }
\subsection*{Explain}
\begin{enumerate}
\item countable noun \\
A \textbf{banquet} is a grand  formal  dinner .
 \textit{
	\begin{itemize}
	\item Last night he attended a state banquet at Buckingham Palace.
	\end{itemize}
}
\end{enumerate}

\section*{courtesy}
{\large \color{blue}  courtesies  }
\subsection*{Explain}
\begin{enumerate}
\item uncountable noun \\
\textbf{Courtesy} is politeness, respect , and consideration for others.
 \textit{
	\begin{itemize}
	\item ...a gentleman who behaves with the utmost courtesy towards everyone he meets.
	\item He did not even have the courtesy to reply to my email.
	\end{itemize}
}
\item singular noun \\
If you refer to \textbf{the}  \textbf{courtesy}  \textbf{of} doing something, you are referring to a polite action.
 \textit{
	\begin{itemize}
	\item By extending the courtesy of a phone call to my clients, I was building a personal
relationship with them.
	\item At least if they're arguing, they're doing you the courtesy of being interested.
	\end{itemize}
}
\item countable noun \\
\textbf{Courtesies} are polite, conventional things that people say in formal  situations .
 \textit{
	\begin{itemize}
	\item Hugh and John were exchanging faintly barbed courtesies.
	\end{itemize}
}
\item adjective \\
\textbf{Courtesy} is used to describe  services that are provided free of charge by an organization to its customers , or to the general  public .
 \textit{
	\begin{itemize}
	\item A courtesy shuttle bus operates between the hotel and the town.
	\item ...a courtesy phone.
	\end{itemize}
}
\item adjective \\
A \textbf{courtesy}  call or a \textbf{courtesy}  visit is a formal visit that you pay someone as a way of showing them politeness or respect.
 \textit{
	\begin{itemize}
	\item The President paid a courtesy call on Emperor Akihito.
	\end{itemize}
}
\item uncountable noun \\
A \textbf{courtesy}  title is a title that someone is allowed to use, although it has no legal or official  status .
 \textit{
	\begin{itemize}
	\item Both were accorded the courtesy title of Lady.
	\item My title, by courtesy only, is the Honourable Amalia Lovell.
	\end{itemize}
}
\item  \\
 (by) courtesy of \textit{
	\begin{itemize}
	\end{itemize}
}
\item  \\
 (by) courtesy of \textit{
	\begin{itemize}
	\end{itemize}
}
\end{enumerate}

\section*{belt}
{\large \color{blue}  belts  belting  belted  }
\subsection*{Explain}
\begin{enumerate}
\item countable noun \\
A \textbf{belt} is a strip of leather or cloth that you fasten round your waist.
 \textit{
	\begin{itemize}
	\item He wore a belt with a large brass buckle.
	\end{itemize}
}
\item countable noun \\
A \textbf{belt} in a machine is a circular strip of rubber that is used to drive moving parts or to move objects along.
 \textit{
	\begin{itemize}
	\item The turning disc is connected by a drive belt to an electric motor.
	\end{itemize}
}
\item countable noun \\
A \textbf{belt} of land or sea is a long, narrow area of it that has some special  feature .
 \textit{
	\begin{itemize}
	\item Miners in Zambia's northern copper belt have gone on strike.
	\item Behind him was a belt of trees, and behind the trees hills and fields.
	\end{itemize}
}
\item verb \\
If someone \textbf{belts} you, they hit you very hard.
 \textbf{Belt} is also a noun .
 \textit{
	\begin{itemize}
	\item 'Is it right she belted old George in the gut?' she asked.
	\item Father would give you a belt over the head with the scrubbing brush.
	\end{itemize}
}
\item verb \\
If you \textbf{belt}  somewhere , you move or travel there very fast.
 \textit{
	\begin{itemize}
	\item We belted down Iveagh Parade to where the motor was.
	\end{itemize}
}
\item countable noun \\
If someone is or has a \textbf{belt} of a particular colour in judo or karate, they have reached the standard which that colour represents.
 \textit{
	\begin{itemize}
	\item He is a black belt in karate.
	\end{itemize}
}
\item  \\
 below the belt \textit{
	\begin{itemize}
	\end{itemize}
}
\item  \\
 to tighten your belt \textit{
	\begin{itemize}
	\end{itemize}
}
\item  \\
 under your belt \textit{
	\begin{itemize}
	\end{itemize}
}
\end{enumerate}

\section*{curtain}
{\large \color{blue}  curtains  }
\subsection*{Explain}
\begin{enumerate}
\item countable noun \\
\textbf{Curtains} are large pieces of material which you hang from the top of a window.
 \textit{
	\begin{itemize}
	\item Her bedroom curtains were drawn.
	\end{itemize}
}
\item countable noun \\
\textbf{Curtains} are pieces of very thin material which you hang in front of windows in order to prevent people from seeing in.
 \textit{
	\begin{itemize}
	\end{itemize}
}
\item singular noun \\
In a theatre, \textbf{the curtain} is the large piece of material that hangs in front of the stage until a performance
 begins .
 \textit{
	\begin{itemize}
	\item The curtain rises toward the end of the Prelude.
	\end{itemize}
}
\item singular noun \\
You can refer to something as a \textbf{curtain} when it is thick and difficult to see through or get  past .
 \textit{
	\begin{itemize}
	\item ...a curtain of cigarette smoke.
	\item Something dark disappeared behind the curtain of leaves.
	\end{itemize}
}
\item  \\
 bring down the curtain \textit{
	\begin{itemize}
	\end{itemize}
}
\end{enumerate}

\section*{bibliography}
{\large \color{blue}  bibliographies  }
\subsection*{Explain}
\begin{enumerate}
\item countable noun \\
A \textbf{bibliography} is a list of books on a particular subject.
 \textit{
	\begin{itemize}
	\item At the end of this chapter there is a select bibliography of useful books.
	\end{itemize}
}
\item countable noun \\
A \textbf{bibliography} is a list of the books and articles that are referred to in a particular book.
 \textit{
	\begin{itemize}
	\end{itemize}
}
\end{enumerate}

\section*{device}
{\large \color{blue}  devices  }
\subsection*{Explain}
\begin{enumerate}
\item countable noun \\
A \textbf{device} is an object that has been invented for a particular purpose , for example for recording or measuring something.
 \textit{
	\begin{itemize}
	\item ...an electronic device that protects your vehicle 24 hours a day.
	\item ...a device that could measure minute quantities of matter.
	\item We believe that an explosive device had been left inside a container.
	\end{itemize}
}
\item countable noun \\
A \textbf{device} is a method of achieving something.
 \textit{
	\begin{itemize}
	\item They claim that military spending is used as a device for managing the economy.
	\item ...the literary device of the metaphor.
	\end{itemize}
}
\item  \\
 to leave someone to their own devices \textit{
	\begin{itemize}
	\end{itemize}
}
\end{enumerate}

\section*{emphasis}
{\large \color{blue}  emphases  }
\subsection*{Explain}
\begin{enumerate}
\item variable noun \\
\textbf{Emphasis} is special or extra importance that is given to an activity or to a part or aspect of something.
 \textit{
	\begin{itemize}
	\item Too much emphasis is placed on research.
	\item Grant puts a special emphasis on weather in his paintings.
	\end{itemize}
}
\item variable noun \\
\textbf{Emphasis} is extra force that you put on a syllable, word, or phrase when you are speaking in order to make it seem more important .
 \textit{
	\begin{itemize}
	\item 'I might have known it!' Miss Burnett said with emphasis.
	\item The emphasis is on the first syllable of the last word.
	\end{itemize}
}
\end{enumerate}

\section*{employee}
{\large \color{blue}  employees  }
\subsection*{Explain}
\begin{enumerate}
\item countable noun \\
An \textbf{employee} is a person who is paid to work for an organization or for another person.
 \textit{
	\begin{itemize}
	\item He is an employee of Fuji Bank.
	\item Many of its employees are women.
	\item ...a government employee.
	\end{itemize}
}
\end{enumerate}

\section*{carriage}
{\large \color{blue}  carriages  }
\subsection*{Explain}
\begin{enumerate}
\item countable noun \\
A \textbf{carriage} is an old-fashioned vehicle, usually for a small number of passengers, which is pulled by horses .
 \textit{
	\begin{itemize}
	\item The President-elect followed in an open carriage drawn by six beautiful gray horses.
	\end{itemize}
}
\item countable noun \\
A \textbf{carriage} is one of the separate , long sections of a train that carries passengers.
 \textit{
	\begin{itemize}
	\end{itemize}
}
\item countable noun \\
A \textbf{carriage} is the same as a baby carriage .
 \textit{
	\begin{itemize}
	\end{itemize}
}
\item uncountable noun \\
\textbf{Carriage} is the cost or action of transporting or delivering  goods .
 \textit{
	\begin{itemize}
	\item It costs £10.86 for one litre including carriage.
	\item If the Government introduces a carbon tax on road haulage, then carriage by water
will become more attractive.
	\end{itemize}
}
\item uncountable noun \\
Your \textbf{carriage} is the way you hold your body and head when you are walking , standing , or sitting .
 \textit{
	\begin{itemize}
	\item Her legs were long and fine, her hips slender, her carriage erect.
	\end{itemize}
}
\end{enumerate}

\section*{employer}
{\large \color{blue}  employers  }
\subsection*{Explain}
\begin{enumerate}
\item countable noun \\
Your \textbf{employer} is the person or organization that you work for.
 \textit{
	\begin{itemize}
	\item He had been sent to Rome by his employer.
	\item The telephone company is the country's largest employer.
	\end{itemize}
}
\end{enumerate}

\section*{chancellor}
{\large \color{blue}  Chancellors  }
\subsection*{Explain}
\begin{enumerate}
\item title noun \\
\textbf{Chancellor} is the title of the head of government in Germany and Austria .
 \textit{
	\begin{itemize}
	\item ...Chancellor Angela Merkel of Germany.
	\item ...as the Chancellor arrived.
	\end{itemize}
}
\item countable noun \\
In Britain , the \textbf{Chancellor} is the Chancellor of the Exchequer .
 \textit{
	\begin{itemize}
	\end{itemize}
}
\item countable noun \\
The \textbf{Chancellor} of a British university is the official head of the university. The Chancellor does not take part in running the university.
 \textit{
	\begin{itemize}
	\end{itemize}
}
\item countable noun \\
The head of some American universities is called  \textbf{the}  \textbf{Chancellor} .
 \textit{
	\begin{itemize}
	\end{itemize}
}
\end{enumerate}

\section*{employment}
{\large \color{blue}  }
\subsection*{Explain}
\begin{enumerate}
\item uncountable noun \\
\textbf{Employment} is the fact of having a paid job .
 \textit{
	\begin{itemize}
	\item She was unable to find employment.
	\item He regularly drove from his home to his place of employment.
	\end{itemize}
}
\item uncountable noun \\
\textbf{Employment} is the fact of employing someone.
 \textit{
	\begin{itemize}
	\item ...the employment of children under nine.
	\end{itemize}
}
\item uncountable noun \\
\textbf{Employment} is the work that is available in a country or area.
 \textit{
	\begin{itemize}
	\item ...economic policies designed to secure full employment.
	\end{itemize}
}
\end{enumerate}

\section*{clause}
{\large \color{blue}  clauses  }
\subsection*{Explain}
\begin{enumerate}
\item countable noun \\
A \textbf{clause} is a section of a legal document.
 \textit{
	\begin{itemize}
	\item He has a clause in his contract which entitles him to a percentage of the profits.
	\item ...a compromise document sprinkled with escape clauses.
	\item ...a complaint alleging a breach of clause 4 of the code.
	\end{itemize}
}
\item countable noun \\
In grammar , a \textbf{clause} is a group of words containing a verb. Sentences contain one or more clauses. There
are finite clauses and non-finite clauses.
 \textit{
	\begin{itemize}
	\end{itemize}
}
\end{enumerate}

\section*{esteem}
{\large \color{blue}  esteems  esteeming  esteemed  }
\subsection*{Explain}
\begin{enumerate}
\item uncountable noun \\
\textbf{Esteem} is the admiration and respect that you feel towards another person.
 \textit{
	\begin{itemize}
	\item He is held in high esteem by colleagues in the construction industry.
	\item Their public esteem has never been lower.
	\item He said he retained immense regard and esteem for the prime minister.
	\end{itemize}
}
\item verb \\
If you \textbf{esteem} someone or something, you respect or admire them.
 \textit{
	\begin{itemize}
	\item I greatly esteem your message in the midst of our hard struggle.
	\end{itemize}
}
\end{enumerate}

\section*{colonel}
{\large \color{blue}  colonels  }
\subsection*{Explain}
\begin{enumerate}
\item countable noun \\
A \textbf{colonel} is a senior officer in an army , air force, or the marines .
 \textit{
	\begin{itemize}
	\item This particular place was run by an ex-Army colonel.
	\item ...Colonel Edward Staley.
	\end{itemize}
}
\end{enumerate}

\section*{fable}
{\large \color{blue}  fables  }
\subsection*{Explain}
\begin{enumerate}
\item variable noun \\
A \textbf{fable} is a story which teaches a moral lesson . Fables sometimes have animals as the main characters.
 \textit{
	\begin{itemize}
	\item ...the fable of the tortoise and the hare.
	\item Each tale has the timeless quality of fable.
	\end{itemize}
}
\item variable noun \\
You can describe a statement or explanation that is untrue but that many people believe as \textbf{fable} .
 \textit{
	\begin{itemize}
	\item Is reincarnation fact or fable?
	\item ...little-known horticultural facts and fables.
	\end{itemize}
}
\end{enumerate}

\section*{congress}
{\large \color{blue}  congresses  }
\subsection*{Explain}
\begin{enumerate}
\item countable noun \\
A \textbf{congress} is a large meeting that is held to discuss  ideas and policies .
 \textit{
	\begin{itemize}
	\item A lot has changed after the party congress.
	\item ...a congress of coal miners.
	\end{itemize}
}
\end{enumerate}

\section*{game}
{\large \color{blue}  games  }
\subsection*{Explain}
\begin{enumerate}
\item countable noun \\
A \textbf{game} is an activity or sport usually involving skill, knowledge , or chance, in which you follow fixed rules and try to win against an opponent or to solve a puzzle .
 \textit{
	\begin{itemize}
	\item ...the wonderful game of football.
	\item ...a playful game of hide-and-seek.
	\item ...a video game.
	\end{itemize}
}
\item countable noun \\
A \textbf{game} is one particular occasion on which a game is played.
 \textit{
	\begin{itemize}
	\item It was the first game of the season.
	\item He regularly watched our games from the stands.
	\item We won three games against Australia.
	\end{itemize}
}
\item countable noun \\
A \textbf{game} is a part of a match, for example in tennis or bridge , consisting of a fixed number of points.
 \textit{
	\begin{itemize}
	\item She won six games to love in the second set.
	\item ...the last three points of the second game.
	\end{itemize}
}
\item plural noun \\
\textbf{Games} are an organized event in which competitions in several sports take place.
 \textit{
	\begin{itemize}
	\item ...the Commonwealth Games.
	\end{itemize}
}
\item plural noun \\
\textbf{Games} are organized sports activities that children do at school.
 \textit{
	\begin{itemize}
	\item At his grammar school he is remembered for being bad at games but good in debates.
	\end{itemize}
}
\item singular noun \\
Someone's \textbf{game} is the degree of skill or the style that they use when playing a particular game.
 \textit{
	\begin{itemize}
	\item Once I was through the first set my game picked up.
	\end{itemize}
}
\item countable noun \\
You can describe a situation that you do not treat seriously as a \textbf{game} .
 \textit{
	\begin{itemize}
	\item Many people regard life as a game: you win some, you lose some.
	\item It's a cat-and-mouse game to him, and I'm the mouse.
	\end{itemize}
}
\item countable noun \\
You can use \textbf{game} to describe a way of behaving in which a person uses a particular plan, usually in order to gain an advantage for himself or herself.
 \textit{
	\begin{itemize}
	\item When the uncertainties become greater than the certainties, we end up in a game of
bluff.
	\item Until now, the Americans have been playing a very delicate political game.
	\end{itemize}
}
\item uncountable noun \\
Wild animals or birds that are hunted for sport and sometimes cooked and eaten are
 referred to as \textbf{game} .
 \textit{
	\begin{itemize}
	\item ...men who shot game for food.
	\end{itemize}
}
\item adjective \\
If you are \textbf{game} for something, you are willing to do something new, unusual , or risky .
 \textit{
	\begin{itemize}
	\item After all this time he still had new ideas and was game to try them.
	\item He said he's game for a similar challenge next year.
	\end{itemize}
}
\item  \\
 game on \textit{
	\begin{itemize}
	\end{itemize}
}
\item  \\
 to give the game away \textit{
	\begin{itemize}
	\end{itemize}
}
\item  \\
 new to a game \textit{
	\begin{itemize}
	\end{itemize}
}
\item  \\
 on the game \textit{
	\begin{itemize}
	\end{itemize}
}
\item  \\
 at their own game \textit{
	\begin{itemize}
	\end{itemize}
}
\item  \\
 all part of the game \textit{
	\begin{itemize}
	\end{itemize}
}
\item  \\
 playing games \textit{
	\begin{itemize}
	\end{itemize}
}
\item  \\
 someone has raised their game \textit{
	\begin{itemize}
	\end{itemize}
}
\item  \\
 the game is up \textit{
	\begin{itemize}
	\end{itemize}
}
\end{enumerate}

\section*{contract}
{\large \color{blue}  contracts  contracting  contracted  }
\subsection*{Explain}
\begin{enumerate}
\item countable noun \\
A \textbf{contract} is a legal agreement, usually between two companies or between an employer and employee , which involves doing work for a stated sum of money.
 \textit{
	\begin{itemize}
	\item The company won a prestigious contract for work on Europe's tallest building.
	\item He was given a seven-year contract with an annual salary of $150,000.
	\end{itemize}
}
\item verb \\
If you \textbf{contract}  \textbf{with} someone \textbf{to} do something, you legally agree to do it for them or for them to do it for you.
 \textit{
	\begin{itemize}
	\item You can contract with us to deliver your cargo.
	\item The Boston Museum of Fine Arts has already contracted to lease part of its collection
to a museum in Japan.
	\end{itemize}
}
\item verb \\
When something \textbf{contracts} or when something \textbf{contracts} it, it becomes smaller or shorter.
 \textit{
	\begin{itemize}
	\item Blood is only expelled from the heart when it contracts.
	\item New research shows that an excess of meat and salt can contract muscles.
	\end{itemize}
}
\item verb \\
When something such as an economy or market  \textbf{contracts} , it becomes smaller.
 \textit{
	\begin{itemize}
	\item The manufacturing economy contracted in October for the sixth consecutive month.
	\end{itemize}
}
\item verb \\
If you \textbf{contract} a serious  illness , you become ill with it.
 \textit{
	\begin{itemize}
	\item He contracted the disease from a blood transfusion.
	\item Ovarian cancer is the sixth most common cancer contracted by women.
	\end{itemize}
}
\item verb \\
If you \textbf{contract} a marriage, alliance , or other relationship with someone, you arrange to have that relationship with them.
 \textit{
	\begin{itemize}
	\item She contracted a formal marriage to a British ex-serviceman.
	\end{itemize}
}
\item countable noun \\
If there is a \textbf{contract}  \textbf{on} a person or on their life , someone has made an arrangement to have them killed.
 \textit{
	\begin{itemize}
	\item The convictions resulted in the local crime bosses putting a contract on him.
	\item The police advised her to get out of town because there was a contract on her life.
	\end{itemize}
}
\item  \\
 under contract \textit{
	\begin{itemize}
	\end{itemize}
}
\end{enumerate}

\section*{goodness}
{\large \color{blue}  }
\subsection*{Explain}
\begin{enumerate}
\item exclamation \\
People sometimes  say ' \textbf{goodness} ' or ' \textbf{my goodness} ' to express  surprise .
 \textit{
	\begin{itemize}
	\item Goodness, I wonder if he knows.
	\item My goodness, he's earned millions in his career.
	\end{itemize}
}
\item uncountable noun \\
\textbf{Goodness} is the quality of being kind , helpful , and honest .
 \textit{
	\begin{itemize}
	\item He retains a faith in human goodness.
	\end{itemize}
}
\end{enumerate}

\section*{county}
{\large \color{blue}  counties  }
\subsection*{Explain}
\begin{enumerate}
\item countable noun \\
A \textbf{county} is a region of Britain , Ireland , or the USA which has its own local government.
 \textit{
	\begin{itemize}
	\item He is living now in his mother's home county of Oxfordshire.
	\item Over 50 events are planned throughout the county.
	\end{itemize}
}
\end{enumerate}

\section*{greenhouse}
{\large \color{blue}  greenhouses  }
\subsection*{Explain}
\begin{enumerate}
\item countable noun \\
A \textbf{greenhouse} is a glass building in which you grow plants that need to be protected from bad  weather .
 \textit{
	\begin{itemize}
	\end{itemize}
}
\item adjective \\
\textbf{Greenhouse}  means relating to or causing the greenhouse effect .
 \textit{
	\begin{itemize}
	\end{itemize}
}
\end{enumerate}

\section*{dilemma}
{\large \color{blue}  dilemmas  }
\subsection*{Explain}
\begin{enumerate}
\item countable noun \\
A \textbf{dilemma} is a difficult situation in which you have to choose between two or more alternatives.
 \textit{
	\begin{itemize}
	\item He was faced with the dilemma of whether or not to return to his country.
	\item The issue raises a moral dilemma.
	\end{itemize}
}
\end{enumerate}

\section*{humidity}
{\large \color{blue}  }
\subsection*{Explain}
\begin{enumerate}
\item uncountable noun \\
You say there is \textbf{humidity} when the air feels very heavy and damp .
 \textit{
	\begin{itemize}
	\item The heat and humidity were insufferable.
	\end{itemize}
}
\item uncountable noun \\
\textbf{Humidity} is the amount of water in the air.
 \textit{
	\begin{itemize}
	\item The humidity is relatively low.
	\end{itemize}
}
\end{enumerate}

\section*{elephant}
{\large \color{blue}  elephants  }
\subsection*{Explain}
\begin{enumerate}
\item countable noun \\
An \textbf{elephant} is a very large animal with a long, flexible  nose  called a trunk , which it uses to pick up things. Elephants live in India and Africa.
 \textit{
	\begin{itemize}
	\end{itemize}
}
\end{enumerate}

\section*{hurricane}
{\large \color{blue}  hurricanes  }
\subsection*{Explain}
\begin{enumerate}
\item countable noun \\
A \textbf{hurricane} is an extremely  violent wind or storm.
 \textit{
	\begin{itemize}
	\end{itemize}
}
\end{enumerate}

\section*{eve}
{\large \color{blue}  eves  }
\subsection*{Explain}
\begin{enumerate}
\item countable noun \\
\textbf{The}  \textbf{eve}  \textbf{of} a particular event or occasion is the day before it, or the period of time just before it.
 \textit{
	\begin{itemize}
	\item ...on the eve of his 27th birthday.
	\end{itemize}
}
\end{enumerate}

\section*{indignation}
{\large \color{blue}  }
\subsection*{Explain}
\begin{enumerate}
\item uncountable noun \\
\textbf{Indignation} is the feeling of shock and anger which you have when you think that something is unjust or unfair.
 \textit{
	\begin{itemize}
	\item She was filled with indignation at the conditions under which miners were forced
to work.
	\item No wonder he could hardly contain his indignation.
	\end{itemize}
}
\end{enumerate}

\section*{format}
{\large \color{blue}  formats  formatting  formatted  }
\subsection*{Explain}
\begin{enumerate}
\item countable noun \\
The \textbf{format} of something is the way or order in which it is arranged and presented .
 \textit{
	\begin{itemize}
	\item I had met with him to explain the format of the programme and what we had in mind.
	\item ...a large-format book.
	\end{itemize}
}
\item countable noun \\
The \textbf{format} of a piece of computer  software or a musical recording is the type of equipment on which it is designed to be used or played. For example, possible formats for a musical recording are CD and cassette .
 \textit{
	\begin{itemize}
	\item His latest album is available on all formats.
	\end{itemize}
}
\item verb \\
To \textbf{format} a computer disk means to run a program so that the disk can be written on.
 \textit{
	\begin{itemize}
	\end{itemize}
}
\item verb \\
To \textbf{format} a piece of computer text or graphics means to arrange the way in which it appears when it is printed or is displayed on a screen.
 \textit{
	\begin{itemize}
	\item The saved text was often badly formatted with many short lines.
	\end{itemize}
}
\end{enumerate}

\section*{intensity}
{\large \color{blue}  }
\subsection*{Explain}
\begin{enumerate}
\item noun \\
1.  2.  3.  4.  \textit{
	\begin{itemize}
	\end{itemize}
}
\end{enumerate}

\section*{gender}
{\large \color{blue}  genders  }
\subsection*{Explain}
\begin{enumerate}
\item uncountable noun \\
\textbf{Gender} is the state of being male or female in relation to the social and cultural  roles that are considered  appropriate for men and women.
 \textit{
	\begin{itemize}
	\item It is illegal to discriminate on the grounds of race, gender or sexual orientation.
	\item Gender stereotyping can be as damaging for men as it can for women.
	\item Some people experience a mismatch between their gender identity and their biological
sex.
	\end{itemize}
}
\item countable noun \\
You can use \textbf{gender} to refer to one of a range of identities that includes female, male, a combination of both, and neither.
 \textit{
	\begin{itemize}
	\item Membership is open to people of all genders.
	\item The new law would allow people to change gender by filling in a form.
	\item Each of them identifies with a different gender from the one they were born with.
	\end{itemize}
}
\item variable noun \\
Some people refer to the fact that a person is male or female as his or her \textbf{gender} .
 \textit{
	\begin{itemize}
	\item Women are sometimes denied opportunities solely because of their gender.
	\end{itemize}
}
\item countable noun \\
Some people refer to all male people or all female people as a particular \textbf{gender} .
 \textit{
	\begin{itemize}
	\item ...the different abilities and skills of the two genders.
	\end{itemize}
}
\item variable noun \\
In grammar , the \textbf{gender} of a noun, pronoun , or adjective is whether it is masculine, feminine, or neuter. A word's gender can affect its form and behaviour . In English, only personal pronouns such as 'she', reflexive pronouns such as 'itself', and possessive  determiners such as 'his' have gender.
 \textit{
	\begin{itemize}
	\item In both Welsh and Irish the word for 'moon' is of feminine gender.
	\end{itemize}
}
\end{enumerate}

\section*{lake}
{\large \color{blue}  lakes  }
\subsection*{Explain}
\begin{enumerate}
\item countable noun \\
A \textbf{lake} is a large area of fresh water, surrounded by land.
 \textit{
	\begin{itemize}
	\item They can go fishing in the lake.
	\item The Nile flows from Lake Victoria in East Africa north to the Mediterranean Sea.
	\end{itemize}
}
\end{enumerate}

\section*{habitat}
{\large \color{blue}  habitats  }
\subsection*{Explain}
\begin{enumerate}
\item variable noun \\
The \textbf{habitat} of an animal or plant is the natural environment in which it normally lives or grows.
 \textit{
	\begin{itemize}
	\item In its natural habitat, the hibiscus will grow up to 25ft.
	\item Few countries have as rich a diversity of habitat as South Africa.
	\end{itemize}
}
\end{enumerate}

\section*{misery}
{\large \color{blue}  miseries  }
\subsection*{Explain}
\begin{enumerate}
\item variable noun \\
\textbf{Misery} is great unhappiness.
 \textit{
	\begin{itemize}
	\item All that money brought nothing but sadness and misery and tragedy.
	\item ...the miseries of his youth.
	\end{itemize}
}
\item uncountable noun \\
\textbf{Misery} is the way of life and unpleasant living conditions of people who are very poor .
 \textit{
	\begin{itemize}
	\item A tiny, educated elite profited from the misery of their two million fellow countrymen.
	\end{itemize}
}
\item countable noun \\
If you say that someone is a \textbf{misery} , you are critical of them because they are always  complaining .
 \textit{
	\begin{itemize}
	\end{itemize}
}
\item  \\
 make sb's life a misery \textit{
	\begin{itemize}
	\end{itemize}
}
\item  \\
 put sb out of their misery \textit{
	\begin{itemize}
	\end{itemize}
}
\item  \\
 put sth out of its misery \textit{
	\begin{itemize}
	\end{itemize}
}
\end{enumerate}

\section*{homework}
{\large \color{blue}  }
\subsection*{Explain}
\begin{enumerate}
\item uncountable noun \\
\textbf{Homework} is school work that teachers  give to pupils to do at home in the evening or at the weekend .
 \textit{
	\begin{itemize}
	\item Have you done your homework, Gemma?
	\end{itemize}
}
\item uncountable noun \\
If you \textbf{do} your \textbf{homework} , you find out what you need to know in preparation for something.
 \textit{
	\begin{itemize}
	\item Before you go near a stockbroker, do your homework.
	\end{itemize}
}
\end{enumerate}

\section*{monkey}
{\large \color{blue}  monkeys  }
\subsection*{Explain}
\begin{enumerate}
\item countable noun \\
A \textbf{monkey} is an animal with a long tail which lives in hot countries and climbs trees.
 \textit{
	\begin{itemize}
	\end{itemize}
}
\item countable noun \\
If you refer to a child as a \textbf{monkey} , you are saying in an affectionate way that he or she is very lively and naughty.
 \textit{
	\begin{itemize}
	\item She's such a little monkey.
	\end{itemize}
}
\end{enumerate}

\section*{jug}
{\large \color{blue}  jugs  }
\subsection*{Explain}
\begin{enumerate}
\item countable noun \\
A \textbf{jug} is a cylindrical container with a handle and is used for holding and pouring liquids.
 A \textbf{jug} of liquid is the amount that the jug contains.
 \textit{
	\begin{itemize}
	\item ...a jug of water.
	\end{itemize}
}
\end{enumerate}

\section*{morality}
{\large \color{blue}  moralities  }
\subsection*{Explain}
\begin{enumerate}
\item uncountable noun \\
\textbf{Morality} is the belief that some behaviour is right and acceptable and that other behaviour is wrong .
 \textit{
	\begin{itemize}
	\item ...standards of morality and justice in society.
	\item ...an effort to preserve traditional morality.
	\end{itemize}
}
\item countable noun \\
A \textbf{morality} is a system of principles and values  concerning people's behaviour, which is generally  accepted by a society or by a particular group of people.
 \textit{
	\begin{itemize}
	\item ...a morality that is sexist.
	\item ...communities and their shared moralities.
	\end{itemize}
}
\item uncountable noun \\
\textbf{The}  \textbf{morality}  \textbf{of} something is how right or acceptable it is.
 \textit{
	\begin{itemize}
	\item ...the arguments about the morality of blood sports.
	\end{itemize}
}
\end{enumerate}

\section*{lesson}
{\large \color{blue}  lessons  }
\subsection*{Explain}
\begin{enumerate}
\item countable noun \\
A \textbf{lesson} is a fixed period of time when people are taught about a particular subject or taught how to do something.
 \textit{
	\begin{itemize}
	\item It would be his last French lesson for months.
	\item Johanna took piano lessons.
	\end{itemize}
}
\item countable noun \\
You use \textbf{lesson} to refer to an experience which acts as a warning to you or an example from which you should learn.
 \textit{
	\begin{itemize}
	\item I had learned a very important lesson: adults must take responsibility for their
own fate.
	\end{itemize}
}
\item countable noun \\
In a church service, the \textbf{lesson} is a short piece of text which is read aloud from the bible .
 \textit{
	\begin{itemize}
	\item The Rev. Nicola Judd read the lesson.
	\end{itemize}
}
\end{enumerate}

\section*{nest}
{\large \color{blue}  nests  nesting  nested  }
\subsection*{Explain}
\begin{enumerate}
\item countable noun \\
A bird's \textbf{nest} is the home that it makes to lay its eggs in.
 \textit{
	\begin{itemize}
	\item I can see an eagle's nest on the rocks.
	\end{itemize}
}
\item verb \\
When a bird \textbf{nests}  somewhere , it builds a nest and settles there to lay its eggs.
 \textit{
	\begin{itemize}
	\item Some species may nest in close proximity to each other.
	\item ...nesting sites.
	\end{itemize}
}
\item countable noun \\
A \textbf{nest} is a home that a group of insects or other creatures make in order to live in and give birth to their young in.
 \textit{
	\begin{itemize}
	\item Some solitary bees make their nests in burrows in the soil.
	\item ...a rat's nest.
	\end{itemize}
}
\item countable noun \\
You can refer to a place as your \textbf{nest} when it is your home or where you feel  comfortable and relaxed .
 \textit{
	\begin{itemize}
	\item Your house is your nest, your sanctuary.
	\item The baby had been asleep in her nest of pink and white blankets.
	\end{itemize}
}
\item countable noun \\
You can use \textbf{nest} to refer to a place where something bad is being done or to the people there who are doing it.
 \textit{
	\begin{itemize}
	\item Are you telling me that you've got your own little nest of informers in the Police
Department?
	\end{itemize}
}
\item  \\
 to feather one's nest \textit{
	\begin{itemize}
	\end{itemize}
}
\item  \\
 fly the nest \textit{
	\begin{itemize}
	\end{itemize}
}
\end{enumerate}

\section*{make}
{\large \color{blue}  makes  making  made  }
\subsection*{Explain}
\begin{enumerate}
\item verb \\
You can use \textbf{make} with a wide range of nouns to indicate that someone performs an action or says something. For example, if you \textbf{make} a suggestion , you suggest something.
 \textit{
	\begin{itemize}
	\item I'd just like to make a comment.
	\item I made a few phone calls.
	\item I think you're making a serious mistake.
	\item Science and technology have made major changes to the way we live.
	\item She had made us an offer too good to refuse.
	\end{itemize}
}
\item verb \\
You can use \textbf{make} with certain nouns to indicate that someone does something well or badly . For example, if you \textbf{make} a success of something, you do it successfully, and if you \textbf{make} a mess of something, you do it very badly.
 \textit{
	\begin{itemize}
	\item Apparently he made a mess of his audition.
	\item Are you really going to make a better job of it this time?
	\end{itemize}
}
\item verb \\
If you \textbf{make}  \textbf{as if to} do something or \textbf{make}  \textbf{to} do something, you behave in a way that makes it seem that you are just about to do it.
 \textit{
	\begin{itemize}
	\item Mary made as if to protest, then hesitated.
	\item He made to chase Davey, who ran back laughing.
	\end{itemize}
}
\item verb \\
In cricket , if a player \textbf{makes} a particular number of runs, they score that number of runs. In baseball or American football , if a player \textbf{makes} a particular score, they achieve that score.
 \textit{
	\begin{itemize}
	\item He made 1,972 runs for the county.
	\end{itemize}
}
\item  \\
 to make do \textit{
	\begin{itemize}
	\end{itemize}
}
\item  \\
 to make like sth/sb \textit{
	\begin{itemize}
	\end{itemize}
}
\end{enumerate}

\section*{pain}
{\large \color{blue}  pains  pained  }
\subsection*{Explain}
\begin{enumerate}
\item variable noun \\
\textbf{Pain} is the feeling of great discomfort you have, for example when you have been hurt or when you are ill .
 \textit{
	\begin{itemize}
	\item ...back pain.
	\item ...a bone disease that caused excruciating pain.
	\item To help ease the pain, heat can be applied to the area with a hot water bottle.
	\item I felt a sharp pain in my lower back.
	\item The illness began with a nagging pain.
	\item ...chest pains.
	\end{itemize}
}
\item uncountable noun \\
\textbf{Pain} is the feeling of unhappiness that you have when something unpleasant or upsetting  happens .
 \textit{
	\begin{itemize}
	\item ...grey eyes that seemed filled with pain.
	\end{itemize}
}
\item verb \\
If a fact or idea  \textbf{pains} you, it makes you feel upset and disappointed .
 \textit{
	\begin{itemize}
	\item This public acknowledgment of Ted's disability pained my mother.
	\item It pains me to think of you struggling all alone.
	\end{itemize}
}
\item  \\
 a pain in the arse \textit{
	\begin{itemize}
	\end{itemize}
}
\item  \\
 be at pains to do sth \textit{
	\begin{itemize}
	\end{itemize}
}
\item  \\
 for their pains \textit{
	\begin{itemize}
	\end{itemize}
}
\item  \\
 on pain of sth/under pain of sth \textit{
	\begin{itemize}
	\end{itemize}
}
\item  \\
 take pains to do sth, go to great pains to do sth \textit{
	\begin{itemize}
	\end{itemize}
}
\end{enumerate}

\section*{ministry}
{\large \color{blue}  ministries  }
\subsection*{Explain}
\begin{enumerate}
\item countable noun \\
In Britain and some other countries, a \textbf{ministry} is a government department which deals with a particular thing or area of activity, for example  trade , defence , or transport .
 \textit{
	\begin{itemize}
	\item ...the Ministry of Justice.
	\item ...a spokesperson for the Agriculture Ministry.
	\end{itemize}
}
\item countable noun \\
The \textbf{ministry} of a religious person is the work that they do that is based on or inspired by their religious beliefs .
 \textit{
	\begin{itemize}
	\item His ministry is among the poor.
	\end{itemize}
}
\item collective singular noun \\
Members of the clergy belonging to some branches of the Christian  church are referred to as \textbf{the ministry} .
 \textit{
	\begin{itemize}
	\item So what prompted him to enter the ministry?
	\end{itemize}
}
\end{enumerate}

\section*{pants}
{\large \color{blue}  }
\subsection*{Explain}
\begin{enumerate}
\item plural noun \\
\textbf{Pants} are a piece of underwear which have two holes to put your legs through and elastic around the top to hold them up round your waist or hips .
 \textit{
	\begin{itemize}
	\item I put on my bra and pants.
	\end{itemize}
}
\item plural noun \\
\textbf{Pants} are a piece of clothing that covers the lower part of your body and each leg.
 \textit{
	\begin{itemize}
	\item He wore brown corduroy pants and a white cotton shirt.
	\end{itemize}
}
\item uncountable noun \\
If you say that something is \textbf{pants} , you mean that it is very poor in quality.
 \textit{
	\begin{itemize}
	\item The place is pants, yet so popular.
	\end{itemize}
}
\item  \\
 the pants off \textit{
	\begin{itemize}
	\end{itemize}
}
\item  \\
 by the seat of your pants \textit{
	\begin{itemize}
	\end{itemize}
}
\end{enumerate}

\section*{moon}
{\large \color{blue}  moons  mooning  mooned  }
\subsection*{Explain}
\begin{enumerate}
\item singular noun \\
\textbf{The}  \textbf{moon} is the object that you can often see in the sky at night . It goes round the Earth once every four weeks , and as it does so its appearance changes from a circle to part of a circle.
 \textit{
	\begin{itemize}
	\item ...the first man on the moon.
	\item There will be no moon.
	\item ...the light of a full moon.
	\end{itemize}
}
\item countable noun \\
A \textbf{moon} is an object similar to a small planet that travels around a planet.
 \textit{
	\begin{itemize}
	\item ...Neptune's large moon.
	\end{itemize}
}
\item verb \\
If you \textbf{are mooning}  \textbf{around} , you are spending time doing nothing in particular, for example because you feel  unhappy or lazy , or are worried about something.
 \textit{
	\begin{itemize}
	\item Lettie was mooning around all morning, doing nothing.
	\item My working days were spent mooning round his department, trying to sneak a chance
encounter.
	\end{itemize}
}
\item verb \\
If you \textbf{moon} at someone, you turn your back to them and show them your bare  bottom .
 \textit{
	\begin{itemize}
	\end{itemize}
}
\item  \\
 blue moon \textit{
	\begin{itemize}
	\end{itemize}
}
\item  \\
 over the moon \textit{
	\begin{itemize}
	\end{itemize}
}
\end{enumerate}

\section*{pleasure}
{\large \color{blue}  pleasures  }
\subsection*{Explain}
\begin{enumerate}
\item uncountable noun \\
If something gives you \textbf{pleasure} , you get a feeling of happiness, satisfaction , or enjoyment from it.
 \textit{
	\begin{itemize}
	\item Watching sport gave him great pleasure.
	\item Everybody takes pleasure in eating.
	\item He gets huge pleasure from ballet and contemporary dance.
	\end{itemize}
}
\item uncountable noun \\
\textbf{Pleasure} is the activity of enjoying yourself, especially  rather than working or doing what you have a duty to do.
 \textit{
	\begin{itemize}
	\item He mixed business and pleasure in a perfect and dynamic way.
	\item I read for pleasure.
	\end{itemize}
}
\item countable noun \\
A \textbf{pleasure} is an activity, experience or aspect of something that you find very enjoyable or satisfying .
 \textit{
	\begin{itemize}
	\item Watching TV is our only pleasure.
	\item ...the pleasure of seeing a smiling face.
	\item ...the conveniences and pleasures of modern life.
	\end{itemize}
}
\item  \\
 a pleasure/the pleasure \textit{
	\begin{itemize}
	\end{itemize}
}
\item  \\
 It's a pleasure/my pleasure \textit{
	\begin{itemize}
	\end{itemize}
}
\item  \\
 with pleasure \textit{
	\begin{itemize}
	\end{itemize}
}
\end{enumerate}

\section*{motel}
{\large \color{blue}  motels  }
\subsection*{Explain}
\begin{enumerate}
\item countable noun \\
A \textbf{motel} is a hotel intended for people who are travelling by car.
 \textit{
	\begin{itemize}
	\end{itemize}
}
\end{enumerate}

\section*{port}
{\large \color{blue}  ports  }
\subsection*{Explain}
\begin{enumerate}
\item countable noun \\
A \textbf{port} is a town by the sea or on a river, which has a harbour .
 \textit{
	\begin{itemize}
	\item Port-Louis is an attractive little fishing port.
	\item ...the Mediterranean port of Marseilles.
	\end{itemize}
}
\item countable noun \\
A \textbf{port} is a harbour area where ships load and unload goods or passengers .
 \textit{
	\begin{itemize}
	\item ...the bridges which link the port area to the city centre.
	\item ...the city's port authority.
	\end{itemize}
}
\item countable noun \\
A \textbf{port} on a computer is a place where you can attach another piece of equipment, for example
a printer .
 \textit{
	\begin{itemize}
	\end{itemize}
}
\item adjective \\
In sailing , the \textbf{port} side of a ship is the left side when you are on it and facing towards the front.
 \textbf{Port} is also a noun .
 \textit{
	\begin{itemize}
	\item Her official number is carved on the port side of the forecabin.
	\item USS Ogden turned to port.
	\end{itemize}
}
\item uncountable noun \\
\textbf{Port} is a type of strong, sweet red wine.
 \textit{
	\begin{itemize}
	\item He asked for a glass of port after dinner.
	\end{itemize}
}
\end{enumerate}

\section*{mug}
{\large \color{blue}  mugs  mugging  mugged  }
\subsection*{Explain}
\begin{enumerate}
\item countable noun \\
A \textbf{mug} is a large deep  cup with straight sides and a handle, used for hot drinks.
 A \textbf{mug} of something is the amount of it contained in a mug.
 \textit{
	\begin{itemize}
	\item He spooned instant coffee into two of the mugs.
	\item He had been drinking mugs of coffee to keep himself awake.
	\end{itemize}
}
\item verb \\
If someone \textbf{mugs} you, they attack you in order to steal your money .
 \textit{
	\begin{itemize}
	\item I was walking out to my car when this guy tried to mug me.
	\item He has been mugged more than once.
	\end{itemize}
}
\item countable noun \\
If you say that someone is a \textbf{mug} , you mean that they are stupid and easily deceived by other people.
 \textit{
	\begin{itemize}
	\item He's a mug as far as women are concerned.
	\item I feel such a mug for signing the agreement.
	\end{itemize}
}
\item  \\
 mug's game \textit{
	\begin{itemize}
	\end{itemize}
}
\item countable noun \\
Someone's \textbf{mug} is their face.
 \textit{
	\begin{itemize}
	\item He managed to get his ugly mug on TV.
	\end{itemize}
}
\end{enumerate}

\section*{rage}
{\large \color{blue}  rages  raging  raged  }
\subsection*{Explain}
\begin{enumerate}
\item variable noun \\
\textbf{Rage} is strong anger that is difficult to control.
 \textit{
	\begin{itemize}
	\item He was red-cheeked with rage.
	\item I flew into a rage.
	\item He admitted shooting the man in a fit of rage.
	\end{itemize}
}
\item verb \\
You say that something powerful or unpleasant  \textbf{rages} when it continues with great force or violence.
 \textit{
	\begin{itemize}
	\item Train services were halted as the fire raged for more than four hours.
	\item ...the fierce arguments raging over the future of the Holy City.
	\item The war rages on and the time has come to take sides.
	\end{itemize}
}
\item verb \\
If you \textbf{rage} about something, you speak or think very angrily about it.
 \textit{
	\begin{itemize}
	\item Monroe was on the phone, raging about her mistreatment by the brothers.
	\item Inside, Frannie was raging.
	\item 'I can't see it's any of your business,' he raged.
	\end{itemize}
}
\item uncountable noun \\
You can refer to the strong anger that someone feels in a particular situation as a particular \textbf{rage} , especially when this results in violent or aggressive behaviour.
 \textit{
	\begin{itemize}
	\item Cabin crews are reporting up to nine cases of air rage a week.
	\end{itemize}
}
\item singular noun \\
When something is popular and fashionable , you can say that it is \textbf{the rage} or \textbf{all the rage} .
 \textit{
	\begin{itemize}
	\item The 1950s look is all the rage at the moment.
	\end{itemize}
}
\end{enumerate}

\section*{outbreak}
{\large \color{blue}  outbreaks  }
\subsection*{Explain}
\begin{enumerate}
\item countable noun \\
If there is an \textbf{outbreak}  \textbf{of} something unpleasant , such as violence or a disease, it suddenly  starts to happen .
 \textit{
	\begin{itemize}
	\item The festival ended a day early after an outbreak of violence involving hundreds of
youths.
	\item At the outbreak of war, he enlisted as a private.
	\item The cholera outbreak continued to spread.
	\end{itemize}
}
\end{enumerate}

\section*{shipment}
{\large \color{blue}  shipments  }
\subsection*{Explain}
\begin{enumerate}
\item countable noun \\
A \textbf{shipment} is an amount of a particular  kind of cargo that is sent to another country on a ship, train , aeroplane , or other vehicle .
 \textit{
	\begin{itemize}
	\item Food shipments could begin in a matter of weeks.
	\item ...a shipment of weapons.
	\end{itemize}
}
\item uncountable noun \\
The \textbf{shipment} of a cargo somewhere is the sending of it there by ship, train, aeroplane, or some other vehicle.
 \textit{
	\begin{itemize}
	\item Bananas are packed before being transported to the docks for shipment overseas.
	\end{itemize}
}
\end{enumerate}

\section*{pilgrim}
{\large \color{blue}  pilgrims  }
\subsection*{Explain}
\begin{enumerate}
\item countable noun \\
\textbf{Pilgrims} are people who make a journey to a holy place for a religious reason .
 \textit{
	\begin{itemize}
	\end{itemize}
}
\end{enumerate}

\section*{skirt}
{\large \color{blue}  skirts  skirting  skirted  }
\subsection*{Explain}
\begin{enumerate}
\item countable noun \\
A \textbf{skirt} is a piece of clothing that fastens at the waist and hangs down around the legs.
 \textit{
	\begin{itemize}
	\end{itemize}
}
\item verb \\
Something that \textbf{skirts} an area is situated around the edge of it.
 \textit{
	\begin{itemize}
	\item We raced across a large field that skirted the slope of a hill.
	\end{itemize}
}
\item verb \\
If you \textbf{skirt} something, you go around the edge of it.
 \textit{
	\begin{itemize}
	\item We shall be skirting the island on our way.
	\item She skirted round the edge of the room to the door.
	\end{itemize}
}
\item verb \\
If you \textbf{skirt} a problem or question , you avoid dealing with it.
 \textit{
	\begin{itemize}
	\item He skirted the hardest issues, concentrating on areas of possible agreement.
	\item He skirted round his main differences with her.
	\end{itemize}
}
\end{enumerate}

\section*{plea}
{\large \color{blue}  pleas  }
\subsection*{Explain}
\begin{enumerate}
\item countable noun \\
A \textbf{plea} is an appeal or request for something, made in an intense or emotional  way .
 \textit{
	\begin{itemize}
	\item Mr Nicholas made his emotional plea for help in solving the killing.
	\item ...an impassioned plea to mankind to act to save the planet.
	\end{itemize}
}
\item countable noun \\
In a court of law, a person's \textbf{plea} is the answer that they give when they have been charged with a crime , saying whether or not they are guilty of that crime.
 \textit{
	\begin{itemize}
	\item The judge questioned him about his guilty plea.
	\item We will enter a plea of not guilty.
	\item Her plea of guilty to manslaughter through provocation was rejected.
	\end{itemize}
}
\item countable noun \\
A \textbf{plea} is a reason which is given , to a court of law or to other people, as an excuse for doing something or for not
doing something.
 \textit{
	\begin{itemize}
	\item The jury rejected his plea of insanity.
	\item Mr Dunn's pleas of poverty are only partly justified.
	\end{itemize}
}
\end{enumerate}

\section*{tackle}
{\large \color{blue}  tackles  tackling  tackled  }
\subsection*{Explain}
\begin{enumerate}
\item verb \\
If you \textbf{tackle} a difficult problem or task, you deal with it in a very determined or efficient way.
 \textit{
	\begin{itemize}
	\item The first reason to tackle these problems is to save children's lives.
	\end{itemize}
}
\item verb \\
If you \textbf{tackle} someone in a game such as hockey or football, you try to take the ball away from them. If you \textbf{tackle} someone in rugby or American football, you knock them to the ground .
 \textbf{Tackle} is also a noun .
 \textit{
	\begin{itemize}
	\item Foley tackled the quarterback.
	\item ...a tackle by full-back Brian Burrows.
	\end{itemize}
}
\item verb \\
If you \textbf{tackle} someone about a particular matter , you speak to them honestly about it, usually in order to get it changed or done.
 \textit{
	\begin{itemize}
	\item I tackled him about how anyone could live amidst so much poverty.
	\end{itemize}
}
\item verb \\
If you \textbf{tackle} someone, you attack them and fight them.
 \textit{
	\begin{itemize}
	\item He claims his attacker overtook and tackled him, pushing him into the dirt.
	\end{itemize}
}
\item uncountable noun \\
\textbf{Tackle} is the equipment that you need for a sport or activity, especially fishing.
 \textit{
	\begin{itemize}
	\item ...fishing tackle.
	\end{itemize}
}
\item uncountable noun \\
\textbf{Tackle} is the equipment, usually consisting of ropes and pulleys, needed for lifting or
pulling something.
 \textit{
	\begin{itemize}
	\item I finally hoisted him up with a block and tackle.
	\end{itemize}
}
\end{enumerate}

\section*{pub}
{\large \color{blue}  pubs  }
\subsection*{Explain}
\begin{enumerate}
\item countable noun \\
A \textbf{pub} is a building where people can have drinks, especially alcoholic drinks, and talk to their friends . Many pubs also serve food.
 \textit{
	\begin{itemize}
	\item He was in the pub until closing time.
	\item Richard used to run a pub.
	\end{itemize}
}
\end{enumerate}

\section*{temperature}
{\large \color{blue}  temperatures  }
\subsection*{Explain}
\begin{enumerate}
\item variable noun \\
The \textbf{temperature} of something is a measure of how hot or cold it is.
 \textit{
	\begin{itemize}
	\item The temperature soared to above 100 degrees in the shade.
	\item The temperature of the water was about 40 degrees.
	\item Coping with severe drops in temperature can be very difficult.
	\end{itemize}
}
\item uncountable noun \\
Your \textbf{temperature} is the temperature of your body. A normal temperature is about 37° centigrade.
 \textit{
	\begin{itemize}
	\item His temperature continued to rise alarmingly.
	\end{itemize}
}
\item countable noun \\
You can use \textbf{temperature} to talk about the feelings and emotions that people have in particular situations .
 \textit{
	\begin{itemize}
	\item There's also been a noticeable rise in the political temperature.
	\end{itemize}
}
\item  \\
 room temperature \textit{
	\begin{itemize}
	\end{itemize}
}
\item  \\
 run a temperature \textit{
	\begin{itemize}
	\end{itemize}
}
\item  \\
 take someone's temperature \textit{
	\begin{itemize}
	\end{itemize}
}
\end{enumerate}

\section*{quota}
{\large \color{blue}  quotas  }
\subsection*{Explain}
\begin{enumerate}
\item countable noun \\
A \textbf{quota} is the limited number or quantity of something which is officially  allowed .
 \textit{
	\begin{itemize}
	\item The quota of four tickets per person had been reduced to two.
	\end{itemize}
}
\item countable noun \\
A \textbf{quota} is a fixed  maximum or minimum  proportion of people from a particular group who are allowed to do something, such as come and live in a country or work for the government.
 \textit{
	\begin{itemize}
	\item The bill would force employers to adopt a quota system when recruiting workers.
	\end{itemize}
}
\item countable noun \\
Someone's \textbf{quota}  \textbf{of} something is their expected or deserved share of it.
 \textit{
	\begin{itemize}
	\item They have the usual quota of human weaknesses, no doubt.
	\end{itemize}
}
\end{enumerate}

\section*{refugee}
{\large \color{blue}  refugees  }
\subsection*{Explain}
\begin{enumerate}
\item countable noun \\
\textbf{Refugees} are people who have been forced to leave their homes or their country, either because there is a war there or because of their political
or religious  beliefs .
 \textit{
	\begin{itemize}
	\end{itemize}
}
\end{enumerate}

\section*{thirst}
{\large \color{blue}  thirsts  thirsting  thirsted  }
\subsection*{Explain}
\begin{enumerate}
\item variable noun \\
\textbf{Thirst} is the feeling that you need to drink something.
 \textit{
	\begin{itemize}
	\item Coca is well-known for reducing hunger, thirst and fatigue.
	\item Instead of tea or coffee, drink water to quench your thirst.
	\item I had such a thirst.
	\end{itemize}
}
\item uncountable noun \\
\textbf{Thirst} is the condition of not having enough to drink.
 \textit{
	\begin{itemize}
	\item They died of thirst on the voyage.
	\end{itemize}
}
\item singular noun \\
A \textbf{thirst}  \textbf{for} something is a very strong  desire for that thing.
 \textit{
	\begin{itemize}
	\item Children show a real thirst for learning.
	\item ...their ever-growing thirst for cash.
	\end{itemize}
}
\item verb \\
If you say that someone \textbf{thirsts}  \textbf{for} something, you mean that they have a strong desire for it.
 \textit{
	\begin{itemize}
	\item We all thirst for the same things.
	\end{itemize}
}
\end{enumerate}

\section*{species}
{\large \color{blue}  species  }
\subsection*{Explain}
\begin{enumerate}
\item countable noun \\
A \textbf{species} is a class of plants or animals whose members have the same main characteristics and are able to breed with each other.
 \textit{
	\begin{itemize}
	\item Pandas are an endangered species.
	\item There are several thousand species of trees here.
	\end{itemize}
}
\end{enumerate}

\section*{trousers}
{\large \color{blue}  }
\subsection*{Explain}
\begin{enumerate}
\item plural noun \\
\textbf{Trousers} are a piece of clothing that you wear over your body from the waist downwards , and that cover each leg separately.
 \textit{
	\begin{itemize}
	\item He was smartly dressed in a shirt, dark trousers and boots.
	\item Alexander rolled up his trouser legs.
	\end{itemize}
}
\end{enumerate}

\section*{sport}
{\large \color{blue}  sports  sporting  sported  }
\subsection*{Explain}
\begin{enumerate}
\item variable noun \\
\textbf{Sports} are games such as football and basketball and other competitive leisure activities which need physical effort and skill .
 \textit{
	\begin{itemize}
	\item I'd say football is my favourite sport.
	\item She excels at sport.
	\item Mark was mainly interested in sport at school, playing rugby as well as soccer.
	\item Billy turned on a radio to get the sports news.
	\end{itemize}
}
\item countable noun \\
If you say that someone is a \textbf{sport} or a good \textbf{sport} , you mean that they cope with a difficult situation or teasing in a cheerful way.
 \textit{
	\begin{itemize}
	\item He was accused of having no sense of humor, of not being a good sport.
	\end{itemize}
}
\item verb \\
If you say that someone \textbf{sports} something such as a distinctive  item of clothing, you mean that they wear it without any shyness.
 \textit{
	\begin{itemize}
	\item He sported a collarless jacket with pleated black panels.
	\item He was heavily-built and sported a red moustache.
	\end{itemize}
}
\end{enumerate}

\section*{texture}
{\large \color{blue}  textures  }
\subsection*{Explain}
\begin{enumerate}
\item variable noun \\
The \textbf{texture} of something is the way that it feels when you touch it, for example how smooth or rough it is.
 \textit{
	\begin{itemize}
	\item Aloe Vera is used in moisturisers to give them a wonderfully silky texture.
	\item Her skin is pale, the texture of fine wax.
	\end{itemize}
}
\item variable noun \\
The \textbf{texture} of something, especially food or soil , is its structure, for example whether it is light with lots of holes , or very heavy and solid .
 \textit{
	\begin{itemize}
	\item This cheese has an open, crumbly texture with a strong flavour.
	\item Earthworms consume large amounts of soil, and produce a rich humus, perfect in texture.
	\end{itemize}
}
\item variable noun \\
The \textbf{texture} of a piece of music or a work of literature is the impression that it makes on you as a result of the way that its different elements are combined .
 \textit{
	\begin{itemize}
	\item The dense but still subtle orchestral textures often overpowered the voices.
	\end{itemize}
}
\end{enumerate}

\section*{window}
{\large \color{blue}  windows  }
\subsection*{Explain}
\begin{enumerate}
\item countable noun \\
A \textbf{window} is a space in the wall of a building or in the side of a vehicle, which has glass
in it so that light can come in and you can see out.
 \textit{
	\begin{itemize}
	\item He stood at the window, moodily staring out.
	\item The room felt very hot and she wondered why someone did not open a window.
	\item ...my car window.
	\end{itemize}
}
\item countable noun \\
A \textbf{window} is a large piece of glass along the front of a shop, behind which some of the goods
that the shop sells are displayed.
 \textit{
	\begin{itemize}
	\item I stood for a few moments in front of the nearest shop window.
	\end{itemize}
}
\item countable noun \\
A \textbf{window} is a glass-covered opening above a counter , for example in a bank , post office, railway  station , or museum , which the person serving you sits behind.
 \textit{
	\begin{itemize}
	\item The woman at the ticket window told me that the admission fee was $17.50.
	\end{itemize}
}
\item countable noun \\
On a computer screen , a \textbf{window} is one of the work areas that the screen can be divided into.
 \textit{
	\begin{itemize}
	\end{itemize}
}
\item countable noun \\
If you have a \textbf{window} in your diary for something, or if you can make a \textbf{window} for it, you are free at a particular time and can do it then.
 \textit{
	\begin{itemize}
	\item Tell her I've got a window in my diary later on this week.
	\end{itemize}
}
\item  \\
 go/fly out of the window \textit{
	\begin{itemize}
	\end{itemize}
}
\item  \\
 window of opportunity \textit{
	\begin{itemize}
	\end{itemize}
}
\end{enumerate}

\section*{visa}
{\large \color{blue}  visas  }
\subsection*{Explain}
\begin{enumerate}
\item countable noun \\
A \textbf{visa} is an official document, or a stamp  put in your passport, which allows you to enter or leave a particular country.
 \textit{
	\begin{itemize}
	\item His visitor's visa expired.
	\item ...an exit visa.
	\item ...a tightening of U.S. visa requirements.
	\end{itemize}
}
\end{enumerate}

\section*{wrist}
{\large \color{blue}  wrists  }
\subsection*{Explain}
\begin{enumerate}
\item countable noun \\
Your \textbf{wrist} is the part of your body between your hand and your arm which bends when you move your hand.
 \textit{
	\begin{itemize}
	\end{itemize}
}
\end{enumerate}

\section*{academy}
{\large \color{blue}  academies  }
\subsection*{Explain}
\begin{enumerate}
\item countable noun \\
\textbf{Academy} is sometimes used in the names of secondary schools and colleges , or private high schools in the United  States .
 \textit{
	\begin{itemize}
	\item ...the Royal Academy of Music.
	\item ...her experience as a police academy instructor.
	\end{itemize}
}
\item countable noun \\
\textbf{Academy}  appears in the names of some societies formed to improve or maintain  standards in a particular field .
 \textit{
	\begin{itemize}
	\item ...the American Academy of Psychotherapists.
	\item The British Academy of Film and Television Arts.
	\end{itemize}
}
\end{enumerate}

\section*{applause}
{\large \color{blue}  }
\subsection*{Explain}
\begin{enumerate}
\item uncountable noun \\
\textbf{Applause} is the noise made by a group of people clapping their hands to show approval .
 \textit{
	\begin{itemize}
	\item They greeted him with thunderous applause.
	\item ...a round of applause.
	\end{itemize}
}
\end{enumerate}

\section*{agony}
{\large \color{blue}  agonies  }
\subsection*{Explain}
\begin{enumerate}
\item uncountable noun \\
\textbf{Agony} is great physical or mental pain.
 \textit{
	\begin{itemize}
	\item She called out in agony.
	\item As a young man he suffered agonies of religious doubt.
	\end{itemize}
}
\end{enumerate}

\section*{blueprint}
{\large \color{blue}  blueprints  }
\subsection*{Explain}
\begin{enumerate}
\item countable noun \\
A \textbf{blueprint}  \textbf{for} something is a plan or set of proposals that shows how it is expected to work.
 \textit{
	\begin{itemize}
	\item The country's president will offer delegates his blueprint for the country's future.
	\item ...the blueprint of a new plan of economic reform.
	\end{itemize}
}
\item countable noun \\
A \textbf{blueprint} of an architect's building plans or a designer's pattern is a photographic print consisting of white lines on a blue background. Blueprints
contain all of the information that is needed to build or make something.
 \textit{
	\begin{itemize}
	\item ...a blueprint of the whole place, complete with heating ducts and wiring.
	\item The documents contain a blueprint for a nuclear device.
	\end{itemize}
}
\item countable noun \\
A genetic  \textbf{blueprint} is a pattern which is contained within all living  cells . This pattern decides how the organism  develops and what it looks like.
 \textit{
	\begin{itemize}
	\item The offspring contain a mixture of the genetic blueprint of each parent.
	\item DNA is the genetic material in every cell that carries the blueprints for everything
from hair color to the risk of cancer.
	\end{itemize}
}
\end{enumerate}

\section*{aisle}
{\large \color{blue}  aisles  }
\subsection*{Explain}
\begin{enumerate}
\item countable noun \\
An \textbf{aisle} is a long narrow  gap that people can walk along between rows of seats in a public building such as a church or between rows of shelves in a supermarket .
 \textit{
	\begin{itemize}
	\item He started down the centre aisle.
	\item ...the frozen food aisle.
	\end{itemize}
}
\item singular noun \\
\textbf{The aisle} is used in expressions such as \textbf{walking down the aisle} to refer to the activity of getting  married .
 \textit{
	\begin{itemize}
	\item He was in no hurry to walk down the aisle.
	\end{itemize}
}
\end{enumerate}

\section*{boot}
{\large \color{blue}  boots  booting  booted  }
\subsection*{Explain}
\begin{enumerate}
\item countable noun \\
\textbf{Boots} are shoes that cover your whole foot and the lower part of your leg.
 \textit{
	\begin{itemize}
	\item He sat in a kitchen chair, reached down and pulled off his boots.
	\item He was wearing riding pants, high boots, and spurs.
	\end{itemize}
}
\item countable noun \\
\textbf{Boots} are strong, heavy shoes which cover your ankle and which have thick  soles . You wear them to protect your feet, for example when you are walking or taking part in sport.
 \textit{
	\begin{itemize}
	\item The soldiers' boots resounded in the street.
	\item Equip yourself with stout walking boots and sticks.
	\end{itemize}
}
\item verb \\
If you \textbf{boot} something such as a ball, you kick it hard.
 \textit{
	\begin{itemize}
	\item He booted the ball 40 yards back up field.
	\item One guy booted the door down.
	\end{itemize}
}
\item countable noun \\
The \textbf{boot} of a car is a covered space at the back or front, in which you carry things such
as luggage and shopping .
 \textit{
	\begin{itemize}
	\item He opened the boot to put my bags in.
	\item Harris got a rope from the car boot.
	\end{itemize}
}
\item verb \\
To \textbf{boot} a car means to fit a Denver boot to one of its wheels so that it cannot be driven away.
 \textit{
	\begin{itemize}
	\item 'If we're gettin' booted, we sure as hell ain't leavin' it for the locals.
	\end{itemize}
}
\item  \\
 get/be given the boot \textit{
	\begin{itemize}
	\end{itemize}
}
\item  \\
 put the boot in \textit{
	\begin{itemize}
	\end{itemize}
}
\item  \\
 to boot \textit{
	\begin{itemize}
	\end{itemize}
}
\end{enumerate}

\section*{beam}
{\large \color{blue}  beams  beaming  beamed  }
\subsection*{Explain}
\begin{enumerate}
\item verb \\
If you say that someone \textbf{is beaming} , you mean that they have a big smile on their face because they are happy , pleased , or proud about something.
 \textit{
	\begin{itemize}
	\item Frances beamed at her friend with undisguised admiration.
	\item 'Welcome back,' she beamed.
	\item ...the beaming face of a 41-year-old man on the brink of achieving his dreams.
	\end{itemize}
}
\item countable noun \\
A \textbf{beam} is a line of energy, radiation, or particles sent in a particular direction.
 \textit{
	\begin{itemize}
	\item ...high-energy laser beams.
	\item ...a beam of neutrons.
	\end{itemize}
}
\item verb \\
If something \textbf{beams} radio signals or television pictures or they \textbf{are beamed}  somewhere , they are sent there by means of electronic equipment.
 \textit{
	\begin{itemize}
	\item The interview was beamed live across the state.
	\item The live satellite broadcast was beamed into homes across America.
	\item ...a ship which is due to begin beaming radio broadcasts to South East Asia.
	\end{itemize}
}
\item countable noun \\
A \textbf{beam}  \textbf{of} light is a line of light that shines from an object such as a lamp .
 \textit{
	\begin{itemize}
	\end{itemize}
}
\item verb \\
If something such as the sun or a lamp \textbf{beams} down, it sends light to a place and shines on it.
 \textit{
	\begin{itemize}
	\item A sharp white spot-light beamed down on a small stage.
	\item All you see of the outside world is the sunlight beaming through the cracks in the
roof.
	\end{itemize}
}
\item countable noun \\
A \textbf{beam} is a long thick bar of wood, metal, or concrete, especially one used to support the roof of a building.
 \textit{
	\begin{itemize}
	\item The ceilings are supported by oak beams.
	\end{itemize}
}
\item singular noun \\
In gymnastics , \textbf{the}  \textbf{beam} is a piece of equipment that consists of a narrow wooden bar on which gymnasts balance and perform movements.
 \textit{
	\begin{itemize}
	\end{itemize}
}
\end{enumerate}

\section*{bowl}
{\large \color{blue}  bowls  bowling  bowled  }
\subsection*{Explain}
\begin{enumerate}
\item countable noun \\
A \textbf{bowl} is a round container with a wide  uncovered top. Some kinds of bowl are used, for example , for serving or eating food from, or in cooking , while other larger kinds are used for washing or cleaning .
 \textit{
	\begin{itemize}
	\item Put all the ingredients into a large bowl.
	\item Your dog should have his own bowls for food and water.
	\end{itemize}
}
\item countable noun \\
The contents of a bowl can be referred to as a \textbf{bowl}  \textbf{of} something.
 \textit{
	\begin{itemize}
	\item ...a bowl of soup.
	\end{itemize}
}
\item countable noun \\
You can refer to the hollow rounded part of an object as its \textbf{bowl} .
 \textit{
	\begin{itemize}
	\item He smacked the bowl of his pipe into his hand.
	\item ...the toilet bowl.
	\end{itemize}
}
\item uncountable noun \\
\textbf{Bowls} is a game in which players try to roll large wooden balls as near as possible to a small wooden ball. Bowls is usually played outdoors on grass .
 \textit{
	\begin{itemize}
	\end{itemize}
}
\item countable noun \\
A set of \textbf{bowls} is a set of round wooden balls that you play bowls with.
 \textit{
	\begin{itemize}
	\end{itemize}
}
\item verb \\
If you \textbf{bowl} , you play the game of bowls or the game of bowling.
 \textit{
	\begin{itemize}
	\item Everyone wanted to bowl, hence everyone wanted to open a bowling alley.
	\end{itemize}
}
\item verb \\
In a sport such as cricket , when a bowler  \textbf{bowls} a ball, he or she sends it down the pitch towards a batsman.
 \textit{
	\begin{itemize}
	\item I can't see the point of bowling a ball like that.
	\item He bowled so well that we won two matches.
	\end{itemize}
}
\item verb \\
In a sport such as cricket, when a batsman \textbf{is bowled} , he has to leave the pitch because the bowler has hit the wicket with the ball.
 To \textbf{bowl} someone \textbf{out} means the same as to bowl them.
 \textit{
	\begin{itemize}
	\item He was bowled out first ball.
	\end{itemize}
}
\item verb \\
If you \textbf{bowl}  \textbf{along} in a car or on a boat , you move along very quickly, especially when you are enjoying yourself.
 \textit{
	\begin{itemize}
	\item Veronica looked at him, smiling, as they bowled along.
	\item It felt just like old times, to bowl down Knightsbridge.
	\end{itemize}
}
\item countable noun \\
A large stadium where sports or concerts take place is sometimes  called a \textbf{Bowl} .
 \textit{
	\begin{itemize}
	\item ...the Crystal Palace Bowl.
	\item ...the Rose Bowl.
	\end{itemize}
}
\item countable noun \\
A \textbf{bowl} or \textbf{bowl game} is a competition in which the best  college  teams play, after the main season has ended.
 \textit{
	\begin{itemize}
	\item ...the Fiesta college football bowl.
	\end{itemize}
}
\end{enumerate}

\section*{boost}
{\large \color{blue}  boosts  boosting  boosted  }
\subsection*{Explain}
\begin{enumerate}
\item verb \\
If one thing \textbf{boosts} another, it causes it to increase, improve, or be more successful .
 \textbf{Boost} is also a noun .
 \textit{
	\begin{itemize}
	\item It wants the government to take action to boost the economy.
	\item The move is designed to boost sales during the peak booking months of January and
February.
	\item It would get the economy going and give us the boost that we need.
	\item The proposal received a boost on Sunday when The New York Times endorsed it in a
leading article.
	\end{itemize}
}
\item verb \\
If something \textbf{boosts} your confidence or morale , it improves it.
 \textbf{Boost} is also a noun.
 \textit{
	\begin{itemize}
	\item We need a big win to boost our confidence.
	\item Do what you can to give her confidence and boost her morale.
	\item It did give me a boost to win such a big event.
	\end{itemize}
}
\end{enumerate}

\section*{carrier}
{\large \color{blue}  carriers  }
\subsection*{Explain}
\begin{enumerate}
\item countable noun \\
A \textbf{carrier} is a vehicle that is used for carrying people, especially  soldiers , or things.
 \textit{
	\begin{itemize}
	\item There were armoured personnel carriers and tanks on the streets.
	\item Deliveries are made by common carrier or van line.
	\end{itemize}
}
\item countable noun \\
A \textbf{carrier} is a passenger airline .
 \textit{
	\begin{itemize}
	\item Aer Lingus, Ireland's national carrier, will report its full-year results tomorrow.
	\end{itemize}
}
\item countable noun \\
A \textbf{carrier} is a person or an animal that is infected with a disease and so can make other people or animals ill .
 \textit{
	\begin{itemize}
	\item ...screening of Ebola carriers.
	\item ...carriers of disease such as mosquitoes and worms.
	\end{itemize}
}
\end{enumerate}

\section*{chef}
{\large \color{blue}  chefs  }
\subsection*{Explain}
\begin{enumerate}
\item countable noun \\
A \textbf{chef} is a cook in a restaurant or hotel .
 \textit{
	\begin{itemize}
	\end{itemize}
}
\end{enumerate}

\section*{collision}
{\large \color{blue}  collisions  }
\subsection*{Explain}
\begin{enumerate}
\item variable noun \\
A \textbf{collision} occurs when a moving object crashes into something.
 \textit{
	\begin{itemize}
	\item They were on their way to the Shropshire Union Canal when their van was involved
in a collision with a car.
	\item I saw a head-on collision between two aeroplanes.
	\end{itemize}
}
\item countable noun \\
A \textbf{collision}  \textbf{of}  cultures or ideas occurs when two very different cultures or people meet and conflict.
 \textit{
	\begin{itemize}
	\item It's the collision of disparate ideas that alters one's perspective.
	\item The play represents the collision of three generations.
	\end{itemize}
}
\end{enumerate}

\section*{closet}
{\large \color{blue}  closets  }
\subsection*{Explain}
\begin{enumerate}
\item countable noun \\
A \textbf{closet} is a piece of furniture with doors at the front and shelves  inside , which is used for storing things.
 \textit{
	\begin{itemize}
	\end{itemize}
}
\item countable noun \\
A \textbf{closet} is a very small room for storing things, especially one without windows .
 \textit{
	\begin{itemize}
	\end{itemize}
}
\item adjective \\
\textbf{Closet} is used to describe a person who has beliefs , habits , or feelings which they keep secret, often because they are embarrassed about them. \textbf{Closet} is also used of their beliefs, habits, or feelings.
 \textit{
	\begin{itemize}
	\item He is a closet Fascist.
	\item ...closet misogyny.
	\end{itemize}
}
\item  \\
 come out of the closet \textit{
	\begin{itemize}
	\end{itemize}
}
\item  \\
 come out of the closet \textit{
	\begin{itemize}
	\end{itemize}
}
\end{enumerate}

\section*{concept}
{\large \color{blue}  concepts  }
\subsection*{Explain}
\begin{enumerate}
\item countable noun \\
A \textbf{concept} is an idea or abstract principle.
 \textit{
	\begin{itemize}
	\item She added that the concept of arranged marriages is misunderstood in the west.
	\item ...basic legal concepts.
	\end{itemize}
}
\end{enumerate}

\section*{consensus}
{\large \color{blue}  }
\subsection*{Explain}
\begin{enumerate}
\item singular noun \\
A \textbf{consensus} is general agreement among a group of people.
 \textit{
	\begin{itemize}
	\item The consensus amongst scientists is that the world will warm up over the next few
decades.
	\item The question of when the troops should leave would be decided by consensus.
	\end{itemize}
}
\end{enumerate}

\section*{cradle}
{\large \color{blue}  cradles  cradling  cradled  }
\subsection*{Explain}
\begin{enumerate}
\item countable noun \\
A \textbf{cradle} is a baby's bed with high sides. Cradles often have curved  bases so that they rock from side to side.
 \textit{
	\begin{itemize}
	\end{itemize}
}
\item countable noun \\
The \textbf{cradle} is the part of a telephone on which the receiver rests while it is not being used.
 \textit{
	\begin{itemize}
	\item I dropped the receiver back in the cradle.
	\end{itemize}
}
\item countable noun \\
A \textbf{cradle} is a frame which supports or protects something.
 \textit{
	\begin{itemize}
	\item He fixed the towing cradle round the hull.
	\end{itemize}
}
\item countable noun \\
A place that is referred to as \textbf{the cradle of} something is the place where it began .
 \textit{
	\begin{itemize}
	\item Mali is the cradle of some of Africa's richest civilizations.
	\item ...New York, the cradle of capitalism.
	\end{itemize}
}
\item verb \\
If you \textbf{cradle} someone or something \textbf{in} your arms or hands , you hold them carefully and gently.
 \textit{
	\begin{itemize}
	\item I cradled her in my arms.
	\item He was sitting at the big table cradling a large bowl of milky coffee.
	\end{itemize}
}
\item  \\
 from the cradle to the grave \textit{
	\begin{itemize}
	\end{itemize}
}
\end{enumerate}

\section*{costume}
{\large \color{blue}  costumes  }
\subsection*{Explain}
\begin{enumerate}
\item variable noun \\
An actor's or performer's \textbf{costume} is the set of clothes they wear while they are performing .
 \textit{
	\begin{itemize}
	\item Even from a distance the effect of his fox costume was stunning.
	\item The performers, in costume and make-up, were walking up and down backstage.
	\item In all, she has eight costume changes.
	\end{itemize}
}
\item uncountable noun \\
The clothes worn by people at a particular time in history , or in a particular country, are referred to as a particular type of \textbf{costume} .
 \textit{
	\begin{itemize}
	\item ...men and women in eighteenth-century costume.
	\item We were greeted by dancers in traditional costume and a rousing version of Midnight
in Moscow.
	\end{itemize}
}
\item adjective \\
A \textbf{costume} play or drama is one which is set in the past and in which the actors wear the type of clothes that were worn in that period.
 \textit{
	\begin{itemize}
	\item ...a lavish costume drama set in Ireland and the U.S. in the 1890s.
	\end{itemize}
}
\end{enumerate}

\section*{cupboard}
{\large \color{blue}  cupboards  }
\subsection*{Explain}
\begin{enumerate}
\item countable noun \\
A \textbf{cupboard} is a piece of furniture that has one or two doors, usually contains shelves , and is used to store things. In British English, \textbf{cupboard}  refers to all kinds of furniture like this. In American English, closet is usually used instead to refer to larger pieces of furniture.
 \textit{
	\begin{itemize}
	\item The kitchen cupboard was stocked with tins of soup and food.
	\end{itemize}
}
\item countable noun \\
A \textbf{cupboard} is a very small room that is used to store things, especially one without windows .
 \textit{
	\begin{itemize}
	\end{itemize}
}
\end{enumerate}

\section*{dignity}
{\large \color{blue}  }
\subsection*{Explain}
\begin{enumerate}
\item uncountable noun \\
If someone behaves or moves with \textbf{dignity} , they are calm , controlled , and admirable .
 \textit{
	\begin{itemize}
	\item ...her extraordinary dignity and composure.
	\end{itemize}
}
\item uncountable noun \\
If you talk about the \textbf{dignity} of people or their lives or activities , you mean that they are valuable and worthy of respect .
 \textit{
	\begin{itemize}
	\item ...the sense of human dignity.
	\item ...the integrity and the dignity of our lives and feelings.
	\end{itemize}
}
\item uncountable noun \\
Your \textbf{dignity} is the sense that you have of your own importance and value , and other people's respect for you.
 \textit{
	\begin{itemize}
	\item If you were wrong, admit it. You won't lose dignity, but will gain respect.
	\item She still has her dignity.
	\end{itemize}
}
\end{enumerate}

\section*{drum}
{\large \color{blue}  drums  drumming  drummed  }
\subsection*{Explain}
\begin{enumerate}
\item countable noun \\
A \textbf{drum} is a musical instrument consisting of a skin stretched tightly over a round frame . You play a drum by beating it with sticks or with your hands .
 \textit{
	\begin{itemize}
	\end{itemize}
}
\item countable noun \\
A \textbf{drum} is a large cylindrical container which is used to store fuel or other substances.
 \textit{
	\begin{itemize}
	\item ...an oil drum.
	\item ...a drum of chemical waste.
	\end{itemize}
}
\item countable noun \\
A \textbf{drum} is a hollow cylindrical structure which is part of a machine, for example a washing machine.
 \textit{
	\begin{itemize}
	\end{itemize}
}
\item countable noun \\
A \textbf{drum} is a circular object on which wire or rope is wound and kept.
 \textit{
	\begin{itemize}
	\item He had found a drum of electric cable.
	\end{itemize}
}
\item verb \\
If something \textbf{drums}  \textbf{on} a surface, or if you \textbf{drum} something \textbf{on} a surface, it hits it regularly, making a continuous beating sound.
 \textit{
	\begin{itemize}
	\item He drummed his fingers on the leather top of his desk.
	\item Rain drummed on the roof of the car.
	\end{itemize}
}
\item  \\
 to beat the drum for someone or something \textit{
	\begin{itemize}
	\end{itemize}
}
\end{enumerate}

\section*{dollar}
{\large \color{blue}  dollars  }
\subsection*{Explain}
\begin{enumerate}
\item countable noun \\
The \textbf{dollar} is the unit of money used in the USA, Canada, Australia, and some other countries.
It is represented by the symbol $. A dollar is divided into one hundred smaller units called cents.
 \textbf{The dollar} is also used to refer to the American currency system.
 \textit{
	\begin{itemize}
	\item She gets paid seven dollars an hour.
	\item The government is spending billions of dollars on new urban rail projects.
	\item In early trading in Tokyo, the dollar fell sharply against the yen.
	\end{itemize}
}
\item  \\
 top dollar \textit{
	\begin{itemize}
	\end{itemize}
}
\end{enumerate}

\section*{editor}
{\large \color{blue}  editors  }
\subsection*{Explain}
\begin{enumerate}
\item countable noun \\
An \textbf{editor} is the person who is in charge of a newspaper or magazine and who decides what will be published in each edition of it.
 \textit{
	\begin{itemize}
	\end{itemize}
}
\item countable noun \\
An \textbf{editor} is a journalist who is responsible for a particular section of a newspaper or magazine.
 \textit{
	\begin{itemize}
	\item Cookery Editor Moyra Fraser takes you behind the scenes.
	\end{itemize}
}
\item countable noun \\
An \textbf{editor} is a person who checks and corrects  texts before they are published.
 \textit{
	\begin{itemize}
	\item Your role as editor is important, for you can look at a piece of writing objectively.
	\end{itemize}
}
\item countable noun \\
An \textbf{editor} is a radio or television journalist who reports on a particular type of news.
 \textit{
	\begin{itemize}
	\item ...our economics editor, Dominic Harrod.
	\end{itemize}
}
\item countable noun \\
An \textbf{editor} is a person who prepares a film, or a radio or television programme, by selecting some of what has been filmed or recorded and putting it in a particular order.
 \textit{
	\begin{itemize}
	\item She worked at 20th Century Fox as a film editor.
	\end{itemize}
}
\item countable noun \\
An \textbf{editor} is a person who collects  pieces of writing by different  authors and prepares them for publication in a book or a series of books.
 \textit{
	\begin{itemize}
	\item Michael Rosen is the editor of the anthology.
	\end{itemize}
}
\item countable noun \\
An \textbf{editor} is a computer program that enables you to change and correct stored data.
 \textit{
	\begin{itemize}
	\end{itemize}
}
\end{enumerate}

\section*{domain}
{\large \color{blue}  domains  }
\subsection*{Explain}
\begin{enumerate}
\item countable noun \\
A \textbf{domain} is a particular field of thought , activity, or interest , especially one over which someone has control, influence , or rights.
 \textit{
	\begin{itemize}
	\item ...the great experimenters in the domain of art.
	\item This information should be in the public domain.
	\end{itemize}
}
\item countable noun \\
Someone's \textbf{domain} is the area they own or have control over.
 \textit{
	\begin{itemize}
	\item ...the mighty king's domain.
	\end{itemize}
}
\item countable noun \\
On the internet, a \textbf{domain} is a set of addresses that shows , for example , the category or geographical area that an internet address belongs to.
 \textit{
	\begin{itemize}
	\end{itemize}
}
\end{enumerate}

\section*{feeling}
{\large \color{blue}  feelings  }
\subsection*{Explain}
\begin{enumerate}
\item countable noun \\
A \textbf{feeling} is an emotion, such as anger or happiness.
 \textit{
	\begin{itemize}
	\item It gave me a feeling of satisfaction.
	\item Strong feelings of pride welled up in me.
	\item I think our main feeling would be of an immense gratitude.
	\item He was unable to contain his own destructive feelings.
	\end{itemize}
}
\item plural noun \\
Your \textbf{feelings} about something are the things that you think and feel about it, or your attitude towards it.
 \textit{
	\begin{itemize}
	\item Everyone knows what my feelings are on that.
	\item I have also begun to reassess my own feelings about being a woman.
	\item I think that sums up the feelings of most discerning and intelligent Indians.
	\item He made no real secret of his feelings to his friends.
	\end{itemize}
}
\item plural noun \\
When you refer to someone's \textbf{feelings} , you are talking about the things that might  embarrass , offend , or upset them. For example , if you hurt someone's \textbf{feelings} , you upset them by something that you say or do.
 \textit{
	\begin{itemize}
	\item He was afraid of hurting my feelings.
	\item He has no respect, no regard for anyone's feelings.
	\item What about my feelings?
	\end{itemize}
}
\item uncountable noun \\
\textbf{Feeling} is a way of thinking and reacting to things which is emotional and not planned  rather than logical and practical .
 \textit{
	\begin{itemize}
	\item He was prompted to a rare outburst of feeling.
	\item ...a voice that trembles with feeling.
	\end{itemize}
}
\item uncountable noun \\
\textbf{Feeling} for someone is love , affection , sympathy, or concern for them.
 \textit{
	\begin{itemize}
	\item Thomas never lost his feeling for Harriet.
	\item It's incredible that Peter can behave with such stupid lack of feeling.
	\end{itemize}
}
\item countable noun \\
If you have a \textbf{feeling} of hunger , tiredness, or other physical sensation, you experience it.
 \textit{
	\begin{itemize}
	\item I also had a strange feeling in my neck.
	\item Focus on the feeling of relaxation.
	\item He experienced feelings of claustrophobia from being in a small place.
	\end{itemize}
}
\item uncountable noun \\
\textbf{Feeling} in part of your body is the ability to experience the sense of touch in this part
of the body.
 \textit{
	\begin{itemize}
	\item After the accident he had no feeling in his legs.
	\end{itemize}
}
\item countable noun \\
If you have \textbf{a}  \textbf{feeling}  \textbf{that} something is the case or \textbf{that} something is going to happen , you think that is probably the case or that it is probably going to happen.
 \textit{
	\begin{itemize}
	\item I have a feeling that everything will come right for us one day.
	\item You have a feeling about people, and I just felt she was going to be good.
	\end{itemize}
}
\item uncountable noun \\
\textbf{Feeling} is used to refer to a general  opinion that a group of people has about something.
 \textit{
	\begin{itemize}
	\item There is still some feeling in the art world that the market for such works may be
declining.
	\item It seemed that anti-Fascist feeling was not being encouraged.
	\end{itemize}
}
\item singular noun \\
If you have a \textbf{feeling of} being in a particular situation , you feel that you are in that situation.
 \textit{
	\begin{itemize}
	\item I had the terrible feeling of being left behind to bring up the baby while he had
fun.
	\end{itemize}
}
\item singular noun \\
If you have \textbf{a feeling for} something, you have an understanding of it or a natural ability to do it.
 \textit{
	\begin{itemize}
	\item Try to get a feeling for the people who live here.
	\item You seem to have a feeling for drawing.
	\end{itemize}
}
\item singular noun \\
If something such as a place or book creates a particular kind of \textbf{feeling} , it creates a particular kind of atmosphere.
 \textit{
	\begin{itemize}
	\item That's what we tried to portray in the book, this feeling of opulence and grandeur.
	\end{itemize}
}
\item  \\
 bad feeling/ill feeling \textit{
	\begin{itemize}
	\end{itemize}
}
\item  \\
 hard feelings \textit{
	\begin{itemize}
	\end{itemize}
}
\item  \\
 I know the feeling \textit{
	\begin{itemize}
	\end{itemize}
}
\item  \\
 have mixed feelings about sth \textit{
	\begin{itemize}
	\end{itemize}
}
\end{enumerate}

\section*{equation}
{\large \color{blue}  equations  }
\subsection*{Explain}
\begin{enumerate}
\item countable noun \\
An \textbf{equation} is a mathematical statement saying that two amounts or values are the same, for example 6 x 4=12x2.
 \textit{
	\begin{itemize}
	\end{itemize}
}
\item countable noun \\
An \textbf{equation} is a situation in which two or more parts have to be considered together so that the whole situation can be understood or explained .
 \textit{
	\begin{itemize}
	\item The equation is simple: research breeds new products.
	\item The party fears the equation between higher spending and higher taxes.
	\item New plans have taken chance out of the equation.
	\end{itemize}
}
\end{enumerate}

\section*{ferry}
{\large \color{blue}  ferries  ferrying  ferried  }
\subsection*{Explain}
\begin{enumerate}
\item countable noun \\
A \textbf{ferry} is a boat that transports passengers and sometimes  also vehicles, usually across rivers or short stretches of sea.
 \textit{
	\begin{itemize}
	\item They had recrossed the River Gambia by ferry.
	\end{itemize}
}
\item verb \\
If a vehicle \textbf{ferries} people or goods, it transports them, usually by means of regular journeys between the same two places.
 \textit{
	\begin{itemize}
	\item Every day, a plane arrives to ferry guests to and from Bird Island Lodge.
	\item It was still dark when five coaches started to ferry the miners the 140 miles from
the Silverhill colliery.
	\item A helicopter ferried in more soldiers to help in the search.
	\end{itemize}
}
\end{enumerate}

\section*{flash}
{\large \color{blue}  flashes  flashing  flashed  }
\subsection*{Explain}
\begin{enumerate}
\item countable noun \\
A \textbf{flash} is a sudden burst of light or of something shiny or bright.
 \textit{
	\begin{itemize}
	\item A sudden flash of lightning lit everything up for a second.
	\item The wire snapped at the wall plug with a blue flash and the light fused.
	\item A jay emerged from the juniper bush in a flash of blue feathers.
	\end{itemize}
}
\item verb \\
If a light \textbf{flashes} or if you \textbf{flash} a light, it shines with a sudden bright light, especially as quick , regular flashes of light.
 \textit{
	\begin{itemize}
	\item Lightning flashed among the distant dark clouds.
	\item He lost his temper after a driver flashed her headlights as he overtook.
	\item He flashed his light into the boat and saw the fishing-line.
	\item He saw the flashing lights of the highway patrol car in his driving mirror.
	\end{itemize}
}
\item countable noun \\
You talk about \textbf{a}  \textbf{flash}  \textbf{of} something when you are saying that it happens very suddenly and unexpectedly.
 \textit{
	\begin{itemize}
	\item 'What did Moira tell you?' Liz demanded with a flash of anger.
	\item When pursued, he made his escape with a flash of speed.
	\item The essays could do with a flash of wit or humor.
	\end{itemize}
}
\item verb \\
If something \textbf{flashes}  past or by, it moves past you so fast that you cannot see it properly.
 \textit{
	\begin{itemize}
	\item It was a busy road, cars flashed by every few minutes.
	\item ...the ball flashed across the face of the goal.
	\end{itemize}
}
\item verb \\
If something \textbf{flashes through} or \textbf{into} your mind, you suddenly think about it.
 \textit{
	\begin{itemize}
	\item A ludicrous thought flashed through Harry's mind.
	\item Those lines of Milton flashed into my mind.
	\end{itemize}
}
\item verb \\
If you \textbf{flash} something such as an identity card, you show it to people quickly and then put it away again.
 \textit{
	\begin{itemize}
	\item Halim flashed his official card, and managed to get hold of a soldier to guard the
Land Rover.
	\end{itemize}
}
\item verb \\
If a picture or message  \textbf{flashes}  \textbf{up on} a screen, or if you \textbf{flash} it \textbf{onto} a screen, it is displayed there briefly or suddenly, and often repeatedly.
 \textit{
	\begin{itemize}
	\item The figures flash up on the scoreboard.
	\item The words 'Good Luck' were flashing on the screen.
	\item Researchers flash two groups of different letters onto a computer screen.
	\item The screen flashes a message: Try again.
	\item A list of items is repeatedly flashed up on the screen.
	\end{itemize}
}
\item verb \\
If you \textbf{flash} news or information to a place, you send it there quickly by computer, satellite , or other system.
 \textit{
	\begin{itemize}
	\item They had told their offices to flash the news as soon as it broke.
	\item This is, of course, international news and soon it was being flashed around the world.
	\end{itemize}
}
\item verb \\
If you \textbf{flash} a look or a smile at someone, you suddenly look at them or smile at them.
 \textit{
	\begin{itemize}
	\item I flashed a look at Sue.
	\item Meg flashed Cissie a grateful smile.
	\end{itemize}
}
\item verb \\
If someone's eyes \textbf{flash} , they suddenly show a strong emotion, especially anger .
 \textit{
	\begin{itemize}
	\item Her dark eyes flashed and she spoke rapidly.
	\end{itemize}
}
\item uncountable noun \\
\textbf{Flash} is the use of special bulbs to give more light when taking a photograph.
 \textit{
	\begin{itemize}
	\item He was one of the first people to use high speed flash in bird photography.
	\end{itemize}
}
\item countable noun \\
A \textbf{flash} is the same as a flashlight .
 \textit{
	\begin{itemize}
	\item Stopping to rest, Pete shut off the flash.
	\end{itemize}
}
\item adjective \\
If you describe something as \textbf{flash} , you mean that it looks expensive , fashionable , and new.
 \textit{
	\begin{itemize}
	\item ...a flash uptown restaurant.
	\item You can go for a 'rostrum' system, which sounds flash, but can be assembled quite
cheaply.
	\end{itemize}
}
\item  \\
 a flash in the pan \textit{
	\begin{itemize}
	\end{itemize}
}
\item  \\
 in a flash \textit{
	\begin{itemize}
	\end{itemize}
}
\item  \\
 quick as a flash \textit{
	\begin{itemize}
	\end{itemize}
}
\end{enumerate}

\section*{fog}
{\large \color{blue}  fogs  fogging  fogged  }
\subsection*{Explain}
\begin{enumerate}
\item variable noun \\
When there is \textbf{fog} , there are tiny  drops of water in the air which form a thick cloud and make it difficult to see things.
 \textit{
	\begin{itemize}
	\item The crash happened in thick fog.
	\item These ocean fogs can last for days.
	\end{itemize}
}
\item singular noun \\
A \textbf{fog} is an unpleasant cloud of something such as smoke  inside a building or room.
 \textit{
	\begin{itemize}
	\item ...a fog of stale cigarette smoke.
	\end{itemize}
}
\item singular noun \\
You can use \textbf{fog} to refer to a situation which stops people from being able to notice things, understand things, or think  clearly .
 \textit{
	\begin{itemize}
	\item The most basic facts about him are lost in a fog of mythology.
	\item Synchronizing these attacks may be difficult in the fog of war.
	\item His mind was in a fog when he finally got up.
	\end{itemize}
}
\item ergative verb \\
If a window , mirror , or other glass surface \textbf{fogs} or \textbf{is fogged} , it becomes covered with very small drops of water so that you cannot see things
clearly through it or in it.
 \textbf{Fog up} means the same as fog .
 \textit{
	\begin{itemize}
	\item The windows fogged immediately.
	\item Water had fogged his diving mask and he couldn't remember how to clear it.
	\item The car windows were fogged with vapor.
	\item The car windows fogged up.
	\item It'd fog up their telescopes.
	\item His hair was all wet and his glasses were fogged up.
	\end{itemize}
}
\end{enumerate}

\section*{fume}
{\large \color{blue}  fumes  fuming  fumed  }
\subsection*{Explain}
\begin{enumerate}
\item plural noun \\
\textbf{Fumes} are the unpleasant and often unhealthy  smoke and gases that are produced by fires or by things such as chemicals, fuel , or cooking .
 \textit{
	\begin{itemize}
	\item ...car exhaust fumes.
	\item They have been protesting about fumes from a chlorine factory.
	\end{itemize}
}
\item verb \\
If you \textbf{fume} over something, you express  annoyance and anger about it.
 \textit{
	\begin{itemize}
	\item He was still fuming over the remark.
	\item 'It's monstrous!' Jackie fumed.
	\end{itemize}
}
\end{enumerate}

\section*{fraction}
{\large \color{blue}  fractions  }
\subsection*{Explain}
\begin{enumerate}
\item countable noun \\
A \textbf{fraction}  \textbf{of} something is a tiny amount or proportion of it.
 \textit{
	\begin{itemize}
	\item She hesitated for a fraction of a second before responding.
	\item Here's how to eat like the stars, at a fraction of the cost.
	\item I opened my eyes just a fraction.
	\end{itemize}
}
\item countable noun \\
A \textbf{fraction} is a number that can be expressed as a proportion of two whole numbers. For example , ½ and ¾ are both fractions.
 \textit{
	\begin{itemize}
	\item The students had a grasp of decimals, percentages and fractions.
	\end{itemize}
}
\end{enumerate}

\section*{garden}
{\large \color{blue}  gardens  gardening  gardened  }
\subsection*{Explain}
\begin{enumerate}
\item countable noun \\
In British English, a \textbf{garden} is a piece of land next to a house, with flowers, vegetables, other plants, and often grass. In American
English, the usual word is yard , and a \textbf{garden}  refers only to land which is used for growing flowers and vegetables.
 \textit{
	\begin{itemize}
	\item ...the most beautiful garden on Earth.
	\end{itemize}
}
\item verb \\
If you \textbf{garden} , you do work in your garden such as weeding or planting .
 \textit{
	\begin{itemize}
	\item Jim gardened at the homes of friends on weekends.
	\end{itemize}
}
\item plural noun \\
\textbf{Gardens} are places like a park that have areas of plants, trees, and grass, and that people
can visit and walk around.
 \textit{
	\begin{itemize}
	\item The Gardens are open from 10.30am until 5pm.
	\item ...Kensington Gardens.
	\end{itemize}
}
\item countable noun \\
\textbf{Gardens} is sometimes used as part of the name of a street .
 \textit{
	\begin{itemize}
	\item He lives at 9, Acacia Gardens.
	\end{itemize}
}
\end{enumerate}

\section*{fragment}
{\large \color{blue}  fragments  fragmenting  fragmented  }
\subsection*{Explain}
\begin{enumerate}
\item countable noun \\
A \textbf{fragment}  \textbf{of} something is a small piece or part of it.
 \textit{
	\begin{itemize}
	\item ...fragments of metal in my shoulder.
	\item She read everything, digesting every fragment of news.
	\item ...glass fragments.
	\end{itemize}
}
\item verb \\
If something \textbf{fragments} or \textbf{is fragmented} , it breaks or separates into small pieces or parts.
 \textit{
	\begin{itemize}
	\item The clouds fragmented and out came the sun.
	\item Fierce rivalries have traditionally fragmented the region.
	\item By the first century BC, Buddhism was in danger of fragmenting into small sects.
	\end{itemize}
}
\end{enumerate}

\section*{grey}
{\large \color{blue}  greyer  greyest  }
\subsection*{Explain}
\begin{enumerate}
\item colour \\
\textbf{Grey} is the colour of ashes or of clouds on a rainy day.
 \textit{
	\begin{itemize}
	\item ...a grey suit.
	\end{itemize}
}
\item adjective \\
You use \textbf{grey} to describe the colour of people's hair when it changes from its original colour, usually as
they get old.
 \textit{
	\begin{itemize}
	\item ...my grey hair.
	\item Eddie was going grey.
	\end{itemize}
}
\item adjective \\
If the weather is \textbf{grey} , there are many clouds in the sky and the light is dull.
 \textit{
	\begin{itemize}
	\item It was a grey, wet April Sunday.
	\end{itemize}
}
\item adjective \\
If you describe a situation as \textbf{grey} , you mean that it is dull, unpleasant , or difficult .
 \textit{
	\begin{itemize}
	\item Brazilians look gloomily forward to a New Year that even the president admits will
be grey and cheerless.
	\end{itemize}
}
\item adjective \\
If you describe someone or something as \textbf{grey} , you think that they are boring and unattractive , and very similar to other things or other people.
 \textit{
	\begin{itemize}
	\item ...little grey men in suits.
	\end{itemize}
}
\item adjective \\
Journalists sometimes use \textbf{grey} to describe things concerning old people.
 \textit{
	\begin{itemize}
	\item There was further evidence of grey consumer power last week, when Ford revealed a
car designed with elderly people in mind.
	\end{itemize}
}
\end{enumerate}

\section*{frequency}
{\large \color{blue}  frequencies  }
\subsection*{Explain}
\begin{enumerate}
\item uncountable noun \\
The \textbf{frequency} of an event is the number of times it happens during a particular period.
 \textit{
	\begin{itemize}
	\item The frequency of Kara's phone calls increased rapidly.
	\item The tanks broke down with increasing frequency.
	\end{itemize}
}
\item variable noun \\
In physics , the \textbf{frequency} of a sound  wave or a radio wave is the number of times it vibrates within a specified period of time.
 \textit{
	\begin{itemize}
	\item You can't hear waves of such a high frequency.
	\item ...a frequency of 24 kilohertz.
	\item ...low frequency waves.
	\end{itemize}
}
\end{enumerate}

\section*{ingredient}
{\large \color{blue}  ingredients  }
\subsection*{Explain}
\begin{enumerate}
\item countable noun \\
\textbf{Ingredients} are the things that are used to make something, especially all the different foods you use when you are cooking a particular dish .
 \textit{
	\begin{itemize}
	\item Mix in the remaining ingredients.
	\end{itemize}
}
\item countable noun \\
An \textbf{ingredient} of a situation is one of the essential parts of it.
 \textit{
	\begin{itemize}
	\item The meeting had all the ingredients of high political drama.
	\item I think that is one of the major ingredients in his success.
	\end{itemize}
}
\end{enumerate}

\section*{garlic}
{\large \color{blue}  }
\subsection*{Explain}
\begin{enumerate}
\item uncountable noun \\
\textbf{Garlic} is the small, white , round bulb of a plant that is related to the onion plant. Garlic has a very strong smell and taste and is used in cooking.
 \textit{
	\begin{itemize}
	\item ...a clove of garlic.
	\end{itemize}
}
\end{enumerate}

\section*{knot}
{\large \color{blue}  knots  knotting  knotted  }
\subsection*{Explain}
\begin{enumerate}
\item countable noun \\
If you tie a \textbf{knot} in a piece of string, rope, cloth , or other material, you pass one end or part of it through a loop and pull it tight .
 \textit{
	\begin{itemize}
	\item One lace had broken and been tied in a knot.
	\end{itemize}
}
\item verb \\
If you \textbf{knot} a piece of string, rope, cloth, or other material, you pass one end or part of it
through a loop and pull it tight.
 \textit{
	\begin{itemize}
	\item He knotted the laces securely together.
	\item He knotted the bandanna around his neck.
	\item ...a knotted rope.
	\end{itemize}
}
\item countable noun \\
A \textbf{knot}  \textbf{of} people is a group of people who are standing very close together.
 \textit{
	\begin{itemize}
	\item A little knot of men stood clapping.
	\end{itemize}
}
\item countable noun \\
If you feel a \textbf{knot} in your stomach , you get an uncomfortable tight feeling in your stomach, usually because you are afraid or excited .
 \textit{
	\begin{itemize}
	\item There was a knot of tension in his stomach.
	\end{itemize}
}
\item verb \\
If your stomach \textbf{knots} or if something \textbf{knots} it, it feels tight because you are afraid or excited.
 \textit{
	\begin{itemize}
	\item I felt my stomach knot with apprehension.
	\item The old dread knotted her stomach.
	\end{itemize}
}
\item verb \\
If part of your face or your muscles \textbf{knot} , they become tense , usually because you are worried or angry .
 \textit{
	\begin{itemize}
	\item His forehead knotted in a frown.
	\item ...his knotted muscles.
	\end{itemize}
}
\item countable noun \\
A \textbf{knot} in a piece of wood is a small hard area where a branch grew.
 \textit{
	\begin{itemize}
	\end{itemize}
}
\item countable noun \\
A \textbf{knot} is a unit of speed. The speed of ships, aircraft, and winds is measured in knots.
 \textit{
	\begin{itemize}
	\item They travel at speeds of up to 30 knots.
	\item ...thirteen knot winds.
	\end{itemize}
}
\item  \\
 to tie yourself in knots \textit{
	\begin{itemize}
	\end{itemize}
}
\item  \\
 to tie the knot \textit{
	\begin{itemize}
	\end{itemize}
}
\end{enumerate}

\section*{gratitude}
{\large \color{blue}  }
\subsection*{Explain}
\begin{enumerate}
\item uncountable noun \\
\textbf{Gratitude} is the state of feeling grateful .
 \textit{
	\begin{itemize}
	\item ...a sense of gratitude.
	\item I wish to express my gratitude to Kathy Davis for her immense practical help.
	\end{itemize}
}
\end{enumerate}

\section*{machinery}
{\large \color{blue}  }
\subsection*{Explain}
\begin{enumerate}
\item uncountable noun \\
You can use \textbf{machinery} to refer to machines in general , or machines that are used in a factory or on a farm .
 \textit{
	\begin{itemize}
	\item ...quality tools and machinery.
	\item ...your local garden machinery specialist.
	\item Farmers import most of their machinery and materials.
	\end{itemize}
}
\item singular noun \\
\textbf{The}  \textbf{machinery} of a government or organization is the system and all the procedures that it uses to deal with things.
 \textit{
	\begin{itemize}
	\item The machinery of democracy could be created quickly.
	\item ...the government machinery and administrative procedures concerned with social provision.
	\item The full state and police machinery ground into action.
	\end{itemize}
}
\end{enumerate}

\section*{harassment}
{\large \color{blue}  }
\subsection*{Explain}
\begin{enumerate}
\item uncountable noun \\
\textbf{Harassment} is behaviour which is intended to trouble or annoy someone, for example  repeated attacks on them or attempts to cause them problems .
 \textit{
	\begin{itemize}
	\item The survey found that 51 per cent of women had experienced some form of sexual harassment.
	\item ...racial harassment.
	\item The party has accused the police of harassment.
	\end{itemize}
}
\end{enumerate}

\section*{meat}
{\large \color{blue}  meats  }
\subsection*{Explain}
\begin{enumerate}
\item variable noun \\
\textbf{Meat} is flesh taken from a dead animal that people cook and eat .
 \textit{
	\begin{itemize}
	\item Meat and fish are relatively expensive.
	\item ...imported meat products.
	\item ...a buffet of cold meats and salads.
	\end{itemize}
}
\item  \\
 be meat and drink to sb \textit{
	\begin{itemize}
	\end{itemize}
}
\item  \\
 meat and potatoes \textit{
	\begin{itemize}
	\end{itemize}
}
\end{enumerate}

\section*{idea}
{\large \color{blue}  ideas  }
\subsection*{Explain}
\begin{enumerate}
\item countable noun \\
An \textbf{idea} is a plan, suggestion , or possible  course of action.
 \textit{
	\begin{itemize}
	\item It's a good idea to plan ahead.
	\item I really like the idea of helping people.
	\item She told me she'd had a brilliant idea.
	\end{itemize}
}
\item countable noun \\
An \textbf{idea} is an opinion or belief about what something is like or should be like.
 \textit{
	\begin{itemize}
	\item Some of his ideas about democracy are entirely his own.
	\item ...the idea that reading too many books ruins your eyes.
	\item My idea of physical perfection is to be very slender.
	\end{itemize}
}
\item singular noun \\
If someone gives you an \textbf{idea of} something, they give you information about it without being very exact or giving a lot of detail .
 \textit{
	\begin{itemize}
	\item This table will give you some idea of how levels of ability can be measured.
	\item Could you give us an idea of the range of complaints you've been receiving?
	\item If you cannot remember the exact date give a rough idea of when it was.
	\end{itemize}
}
\item singular noun \\
If you have an \textbf{idea} of something, you know about it to some extent .
 \textit{
	\begin{itemize}
	\item By the end of the week you will have a clear idea of what your eating habits are.
	\item No one has any real idea how much the company will make next year.
	\end{itemize}
}
\item singular noun \\
If you have an \textbf{idea}  \textbf{that} something is the case, you think that it may be the case, although you are not certain.
 \textit{
	\begin{itemize}
	\item I had an idea that he joined the army later, but I may be wrong.
	\end{itemize}
}
\item singular noun \\
\textbf{The idea} of an action or activity is its aim or purpose.
 \textit{
	\begin{itemize}
	\item The idea is to encourage people to get to know their neighbours.
	\end{itemize}
}
\item countable noun \\
If you have the \textbf{idea of} doing something, you intend to do it.
 \textit{
	\begin{itemize}
	\item He sent for a number of books he admired with the idea of re-reading them.
	\item I had to postpone ideas of a career and stay at home.
	\end{itemize}
}
\item singular noun \\
You can use \textbf{idea} in expressions such as \textbf{I've no idea} or \textbf{I haven't the faintest idea} to emphasize that you do not know something.
 \textit{
	\begin{itemize}
	\item 'Is she coming by coach?'—'Well I've no idea.'
	\item We haven't the faintest idea where he is.
	\end{itemize}
}
\item  \\
 get the idea \textit{
	\begin{itemize}
	\end{itemize}
}
\item  \\
 you have no idea/you've no idea \textit{
	\begin{itemize}
	\end{itemize}
}
\end{enumerate}

\section*{middle}
{\large \color{blue}  middles  }
\subsection*{Explain}
\begin{enumerate}
\item countable noun \\
\textbf{The}  \textbf{middle}  \textbf{of} something is the part of it that is furthest from its edges , ends, or outside surface.
 \textit{
	\begin{itemize}
	\item Howard stood in the middle of the room, sipping a cup of coffee.
	\item Hyde accelerated away from the kerb, swerving out into the middle of the street.
	\item I was in the middle of the back row.
	\item Make sure the roast potatoes aren't raw in the middle.
	\end{itemize}
}
\item adjective \\
The \textbf{middle} object in a row of objects is the one that has an equal number of objects on each
side.
 \textit{
	\begin{itemize}
	\item The middle button of his uniform jacket was strained over his belly.
	\item ...the middle finger of her left hand.
	\end{itemize}
}
\item countable noun \\
Your \textbf{middle} is the part of your body around your stomach .
 \textit{
	\begin{itemize}
	\item At age fifty-three, he now has a few extra pounds around his middle.
	\item The cook's apron covered her middle.
	\end{itemize}
}
\item singular noun \\
\textbf{The}  \textbf{middle}  \textbf{of} an event or period of time is the part that comes after the first part and before the last part.
 \textbf{Middle} is also an adjective .
 \textit{
	\begin{itemize}
	\item I woke up in the middle of the night and could hear a tapping on the window.
	\item It was now the middle of November, cold and often foggy.
	\item She was born in the middle of a rain storm.
	\item The month began and ended dry, but the middle fortnight saw nearly 100mm of rain.
	\end{itemize}
}
\item adjective \\
If someone is in their \textbf{middle}  thirties , for example , they are aged  somewhere  approximately between thirty-four and thirty-six.
 \textit{
	\begin{itemize}
	\item She knew he was in his middle fifties, although he looked much younger.
	\item I went on competing till I was in my middle forties.
	\end{itemize}
}
\item adjective \\
The \textbf{middle} child in a family has equal numbers of younger and older  brothers and sisters .
 \textit{
	\begin{itemize}
	\item His middle son died in a drowning accident five years back.
	\end{itemize}
}
\item adjective \\
The \textbf{middle}  course or way is a moderate course of action that lies between two opposite and extreme courses.
 \textit{
	\begin{itemize}
	\item He favoured a middle course between free enterprise and state intervention.
	\end{itemize}
}
\item  \\
 down the middle \textit{
	\begin{itemize}
	\end{itemize}
}
\item  \\
 in the middle of \textit{
	\begin{itemize}
	\end{itemize}
}
\end{enumerate}

\section*{imagination}
{\large \color{blue}  imaginations  }
\subsection*{Explain}
\begin{enumerate}
\item variable noun \\
Your \textbf{imagination} is the ability that you have to form pictures or ideas in your mind of things that are new and exciting , or things that you have not experienced.
 \textit{
	\begin{itemize}
	\item Antonia is a woman with a vivid imagination.
	\item Alistair had a logical mind, and little imagination.
	\item The Government approach displays a lack of imagination.
	\end{itemize}
}
\item countable noun \\
Your \textbf{imagination} is the part of your mind which allows you to form pictures or ideas of things that
do not necessarily exist in real life.
 \textit{
	\begin{itemize}
	\item Long before I ever went there, Africa was alive in my imagination.
	\end{itemize}
}
\item  \\
 capture/catch sb's imagination \textit{
	\begin{itemize}
	\end{itemize}
}
\item  \\
 stretch one's imagination \textit{
	\begin{itemize}
	\end{itemize}
}
\end{enumerate}

\section*{noodle}
{\large \color{blue}  noodles  }
\subsection*{Explain}
\begin{enumerate}
\item countable noun \\
\textbf{Noodles} are long, thin , curly strips of pasta. They are used especially in Chinese and Italian  cooking .
 \textit{
	\begin{itemize}
	\end{itemize}
}
\end{enumerate}

\section*{notion}
{\large \color{blue}  notions  }
\subsection*{Explain}
\begin{enumerate}
\item countable noun \\
A \textbf{notion} is an idea or belief about something.
 \textit{
	\begin{itemize}
	\item We each have a notion of just what kind of person we'd like to be.
	\item I reject absolutely the notion that privatisation of our industry is now inevitable.
	\item I'd had a few notions about being a journalist.
	\end{itemize}
}
\item plural noun \\
\textbf{Notions} are small articles for sewing , such as buttons , zips , and thread .
 \textit{
	\begin{itemize}
	\end{itemize}
}
\end{enumerate}

\section*{pendulum}
{\large \color{blue}  pendulums  }
\subsection*{Explain}
\begin{enumerate}
\item countable noun \\
The \textbf{pendulum} of a clock is a rod with a weight at the end which swings from side to side in order to make the clock work.
 \textit{
	\begin{itemize}
	\end{itemize}
}
\item singular noun \\
You can use the idea of a \textbf{pendulum} and the way it swings regularly as a way of talking about regular changes in a situation or in people's opinions .
 \textit{
	\begin{itemize}
	\item The pendulum has swung back and the American car companies have made dramatic advances
in safety.
	\item The political pendulum has swung in favour of the liberals.
	\end{itemize}
}
\end{enumerate}

\section*{plumber}
{\large \color{blue}  plumbers  }
\subsection*{Explain}
\begin{enumerate}
\item countable noun \\
A \textbf{plumber} is a person whose job is to connect and repair things such as water and drainage pipes, baths , and toilets .
 \textit{
	\begin{itemize}
	\end{itemize}
}
\end{enumerate}

\section*{radar}
{\large \color{blue}  radars  }
\subsection*{Explain}
\begin{enumerate}
\item variable noun \\
\textbf{Radar} is a way of discovering the position or speed of objects such as aircraft or ships when they cannot be seen , by using radio signals.
 \textit{
	\begin{itemize}
	\item Pilots complained that the radars in the Mirages malfunctioned during conditions
of high humidity.
	\item The aircraft was on a flight from Milan when it disappeared from radar screens.
	\end{itemize}
}
\end{enumerate}

\section*{relief}
{\large \color{blue}  reliefs  }
\subsection*{Explain}
\begin{enumerate}
\item variable noun \\
If you feel a sense of \textbf{relief} , you feel happy because something unpleasant has not happened or is no longer happening .
 \textit{
	\begin{itemize}
	\item I breathed a sigh of relief.
	\item The news will come as a great relief to the French authorities.
	\item To his relief a loud knock on the door spared him from giving an explanation.
	\item It's a relief to get out of the office once in a while.
	\end{itemize}
}
\item uncountable noun \\
If something provides \textbf{relief}  \textbf{from} pain or distress, it stops the pain or distress.
 \textit{
	\begin{itemize}
	\item This brought considerable relief from the pain.
	\item ...a self-help programme which can give lasting relief from the torment of hay fever.
	\end{itemize}
}
\item uncountable noun \\
\textbf{Relief} is money, food, or clothing that is provided for people who are very poor, or who
have been affected by war or a natural disaster .
 \textit{
	\begin{itemize}
	\item Relief agencies are stepping up efforts to provide food, shelter and agricultural
equipment.
	\item ...famine relief.
	\end{itemize}
}
\item countable noun \\
A \textbf{relief} worker is someone who does your work when you go home, or who is employed to do it
 instead of you when you are sick .
 \textit{
	\begin{itemize}
	\item No relief drivers were available.
	\end{itemize}
}
\item countable noun \\
A \textbf{relief} is a sculpture that is carved out of a flat vertical surface.
 \textit{
	\begin{itemize}
	\end{itemize}
}
\end{enumerate}

\section*{retail}
{\large \color{blue}  retails  retailing  retailed  }
\subsection*{Explain}
\begin{enumerate}
\item uncountable noun \\
\textbf{Retail} is the activity of selling goods direct to the public, usually in small quantities. Compare  wholesale .
 \textit{
	\begin{itemize}
	\item ...retail stores.
	\item Retail sales grew just 3.8 percent last year.
	\end{itemize}
}
\item adverb \\
If something is sold \textbf{retail} , it is sold in ordinary  shops direct to the public.
 \textit{
	\begin{itemize}
	\end{itemize}
}
\item verb \\
If an item in a shop \textbf{retails}  \textbf{at} or \textbf{for} a particular price, it is on sale at that price.
 \textit{
	\begin{itemize}
	\item It originally retailed at £23.50.
	\end{itemize}
}
\item verb \\
If someone \textbf{retails} a story or event, they tell it to someone else, often in detail and in an exciting way.
 \textit{
	\begin{itemize}
	\item Mr Hastings gleefully retailed the story to Mr Anderson over lunch.
	\end{itemize}
}
\end{enumerate}

\section*{rim}
{\large \color{blue}  rims  }
\subsection*{Explain}
\begin{enumerate}
\item countable noun \\
The \textbf{rim} of a container such as a cup or glass is the edge that goes all the way round the top .
 \textit{
	\begin{itemize}
	\item She looked at him over the rim of her glass.
	\end{itemize}
}
\item countable noun \\
The \textbf{rim} of a circular object is its outside edge.
 \textit{
	\begin{itemize}
	\item ...a round mirror with white metal rim.
	\end{itemize}
}
\item countable noun \\
If there is a \textbf{rim} of dirt around a surface there is a dirty  mark around it.
 \textit{
	\begin{itemize}
	\item There was already a rim of dark hairs and soap round the basin.
	\end{itemize}
}
\item countable noun \\
In basketball , the \textbf{rim} is the metal ring that holds the net through which players have to try to throw the ball to score .
 \textit{
	\begin{itemize}
	\end{itemize}
}
\end{enumerate}

\section*{ribbon}
{\large \color{blue}  ribbons  }
\subsection*{Explain}
\begin{enumerate}
\item variable noun \\
A \textbf{ribbon} is a long, narrow piece of cloth that you use for tying things together or as a decoration.
 \textit{
	\begin{itemize}
	\item She had tied back her hair with a peach satin ribbon.
	\item ...a piece of ribbon.
	\end{itemize}
}
\item countable noun \\
A typewriter or printer  \textbf{ribbon} is a long, narrow piece of cloth containing ink and is used in a typewriter or printer.
 \textit{
	\begin{itemize}
	\end{itemize}
}
\item countable noun \\
A \textbf{ribbon} is a small decorative strip of cloth which is given to someone to wear on their clothes as an award or to show that they are linked with a particular organization .
 \textit{
	\begin{itemize}
	\end{itemize}
}
\end{enumerate}

\section*{sake}
{\large \color{blue}  sakes  }
\subsection*{Explain}
\begin{enumerate}
\item  \\
 for the sake of sthg \textit{
	\begin{itemize}
	\end{itemize}
}
\item  \\
 for it's/their own sake \textit{
	\begin{itemize}
	\end{itemize}
}
\item  \\
 for sb's sake \textit{
	\begin{itemize}
	\end{itemize}
}
\item  \\
 for God's sake \textit{
	\begin{itemize}
	\end{itemize}
}
\end{enumerate}

\section*{sheep}
{\large \color{blue}  sheep  }
\subsection*{Explain}
\begin{enumerate}
\item countable noun \\
A \textbf{sheep} is a farm animal which is covered with thick  curly hair called wool. Sheep are kept for their wool or for their meat.
 \textit{
	\begin{itemize}
	\item ...grassland on which a flock of sheep were grazing.
	\end{itemize}
}
\item plural noun \\
If you say that a group of people are like \textbf{sheep} , you disapprove of them because if one person does something, all the others copy that person.
 \textit{
	\begin{itemize}
	\end{itemize}
}
\end{enumerate}

\section*{scrap}
{\large \color{blue}  scraps  scrapping  scrapped  }
\subsection*{Explain}
\begin{enumerate}
\item countable noun \\
A \textbf{scrap}  \textbf{of} something is a very small piece or amount of it.
 \textit{
	\begin{itemize}
	\item A crumpled scrap of paper was found in her handbag.
	\item ...a fire fuelled by scraps of wood.
	\item They need every scrap of information they can get.
	\end{itemize}
}
\item plural noun \\
\textbf{Scraps} are pieces of unwanted food which are thrown  away or given to animals.
 \textit{
	\begin{itemize}
	\item ...the scraps from the Sunday dinner table.
	\end{itemize}
}
\item verb \\
If you \textbf{scrap} something, you get  rid of it or cancel it.
 \textit{
	\begin{itemize}
	\item President Hussein called on all countries in the Middle East to scrap nuclear or
chemical weapons.
	\item It had been thought that passport controls would be scrapped.
	\end{itemize}
}
\item adjective \\
\textbf{Scrap} metal or paper is no longer wanted for its original  purpose , but may have some other use.
 \textit{
	\begin{itemize}
	\item There's always tons of scrap paper in Dad's office.
	\end{itemize}
}
\item uncountable noun \\
\textbf{Scrap} is metal from old or damaged  machinery or cars .
 \textit{
	\begin{itemize}
	\item Thousands of tanks, artillery pieces and armored vehicles will be cut up for scrap.
	\end{itemize}
}
\item countable noun \\
You can  refer to a fight or a quarrel as a \textbf{scrap} , especially if it is not very serious .
 \textit{
	\begin{itemize}
	\item Billy Bonds has never been one to avoid a scrap.
	\end{itemize}
}
\end{enumerate}

\section*{southeast}
{\large \color{blue}  }
\subsection*{Explain}
\begin{enumerate}
\item noun \\
1.  2.  \textit{
	\begin{itemize}
	\end{itemize}
}
\item adjective \\
3.  4.  5.  \textit{
	\begin{itemize}
	\end{itemize}
}
\item adverb \\
6.  \textit{
	\begin{itemize}
	\end{itemize}
}
\end{enumerate}

\section*{sensation}
{\large \color{blue}  sensations  }
\subsection*{Explain}
\begin{enumerate}
\item countable noun \\
A \textbf{sensation} is a physical feeling.
 \textit{
	\begin{itemize}
	\item Floating can be a very pleasant sensation.
	\item A sensation of burning or tingling may be experienced in the hands.
	\end{itemize}
}
\item uncountable noun \\
\textbf{Sensation} is your ability to feel things physically, especially through your sense of touch .
 \textit{
	\begin{itemize}
	\item The pain was so bad that she lost all sensation.
	\item ...nerve damage which can lead to loss of sensation in the limbs.
	\end{itemize}
}
\item countable noun \\
You can use \textbf{sensation} to refer to the general feeling or impression caused by a particular experience.
 \textit{
	\begin{itemize}
	\item It's a funny sensation to know someone's talking about you in a language you don't
understand.
	\end{itemize}
}
\item countable noun \\
If a person, event , or situation is a \textbf{sensation} , it causes great excitement or interest .
 \textit{
	\begin{itemize}
	\item ...the film that turned her into an overnight sensation.
	\end{itemize}
}
\item singular noun \\
If a person, event, or situation causes \textbf{a sensation} , they cause great interest or excitement.
 \textit{
	\begin{itemize}
	\item She was just 14 when she caused a sensation in Montreal.
	\end{itemize}
}
\end{enumerate}

\section*{symptom}
{\large \color{blue}  symptoms  }
\subsection*{Explain}
\begin{enumerate}
\item countable noun \\
A \textbf{symptom} of an illness is something wrong with your body or mind that is a sign of the illness.
 \textit{
	\begin{itemize}
	\item One of the most common symptoms of schizophrenia is hearing imaginary voices.
	\item ...patients with flu symptoms.
	\item If the symptoms persist, it is important to go to your doctor.
	\end{itemize}
}
\item countable noun \\
A \textbf{symptom}  \textbf{of} a bad  situation is something that happens which is considered to be a sign of this situation.
 \textit{
	\begin{itemize}
	\item With some people lateness is a symptom of general unreliability.
	\item The contradictory statements are symptoms of disarray in the administration.
	\end{itemize}
}
\end{enumerate}

\section*{sense}
{\large \color{blue}  senses  sensing  sensed  }
\subsection*{Explain}
\begin{enumerate}
\item countable noun \\
Your \textbf{senses} are the physical abilities of sight, smell, hearing, touch, and taste.
 \textit{
	\begin{itemize}
	\item She stared at him again, unable to believe the evidence of her senses.
	\item ...a keen sense of smell.
	\end{itemize}
}
\item verb \\
If you \textbf{sense} something, you become aware of it or you realize it, although it is not very obvious .
 \textit{
	\begin{itemize}
	\item She probably sensed that I wasn't telling her the whole story.
	\item He looks about him, sensing danger.
	\item Prost had sensed what might happen.
	\end{itemize}
}
\item singular noun \\
If you have a \textbf{sense}  \textbf{that} something is the case, you think that it is the case, although you may not have firm , clear evidence for this belief.
 \textit{
	\begin{itemize}
	\item Suddenly you got this sense that people were drawing themselves away from each other.
	\item There is no sense of urgency on either side.
	\end{itemize}
}
\item singular noun \\
If you have a \textbf{sense of}  guilt or relief , for example , you feel  guilty or relieved .
 \textit{
	\begin{itemize}
	\item When your child is struggling for life, you feel this overwhelming sense of guilt.
	\item Lulled into a false sense of security, we eagerly awaited their return.
	\end{itemize}
}
\item singular noun \\
If you have a \textbf{sense of} something such as duty or justice , you are aware of it and believe it is important.
 \textit{
	\begin{itemize}
	\item My sense of justice was offended.
	\item We must keep a sense of proportion about all this.
	\item She needs to regain a sense of her own worth.
	\end{itemize}
}
\item singular noun \\
Someone who has a \textbf{sense}  \textbf{of}  timing or style has a natural ability with regard to timing or style. You can also say that someone has a bad  \textbf{sense}  \textbf{of} timing or style.
 \textit{
	\begin{itemize}
	\item He has an impeccable sense of timing.
	\item Her dress sense is appalling.
	\item ...his astute business sense.
	\end{itemize}
}
\item uncountable noun \\
\textbf{Sense} is the ability to make good judgments and to behave sensibly.
 \textit{
	\begin{itemize}
	\item ...when he was younger and had a bit more sense.
	\item When that doesn't work they sometimes have the sense to seek help.
	\item And I'll buzz over to talk some sense into old Ocker.
	\end{itemize}
}
\item singular noun \\
If you say that there is no \textbf{sense} or little \textbf{sense}  \textbf{in} doing something, you mean that it is not a sensible thing to do because nothing useful would be gained by doing it.
 \textit{
	\begin{itemize}
	\item There's no sense in pretending this doesn't happen.
	\item There's little sense in trying to outspend a competitor with a much larger service
factory.
	\end{itemize}
}
\item countable noun \\
A \textbf{sense} of a word or expression is one of its possible meanings.
 \textit{
	\begin{itemize}
	\item ...a noun which has two senses.
	\item Then she remembered that they had no mind in any real sense of that word.
	\end{itemize}
}
\item  \\
 in a sense \textit{
	\begin{itemize}
	\end{itemize}
}
\item  \\
 make sense \textit{
	\begin{itemize}
	\end{itemize}
}
\item  \\
 make sense of sth \textit{
	\begin{itemize}
	\end{itemize}
}
\item  \\
 make sense \textit{
	\begin{itemize}
	\end{itemize}
}
\item  \\
 come to one's senses/bring sb to their senses \textit{
	\begin{itemize}
	\end{itemize}
}
\item  \\
 have taken leave of one's senses \textit{
	\begin{itemize}
	\end{itemize}
}
\item  \\
 talk sense \textit{
	\begin{itemize}
	\end{itemize}
}
\item  \\
 have a sense that/get a sense that \textit{
	\begin{itemize}
	\end{itemize}
}
\end{enumerate}

\section*{tan}
{\large \color{blue}  tans  tanning  tanned  }
\subsection*{Explain}
\begin{enumerate}
\item singular noun \\
If you have a \textbf{tan} , your skin has become darker than usual because you have been in the sun.
 \textit{
	\begin{itemize}
	\item She is tall and blonde, with a permanent tan.
	\end{itemize}
}
\item verb \\
If a part of your body \textbf{tans} or if you \textbf{tan} it, your skin becomes darker than usual because you spend a lot of time in the sun.
 \textit{
	\begin{itemize}
	\item I have very pale skin that never tans.
	\item I don't tan.
	\item Leigh rolled over on her stomach to tan her back.
	\end{itemize}
}
\item colour \\
Something that is \textbf{tan} is a light brown colour.
 \textit{
	\begin{itemize}
	\item ...a tan leather jacket.
	\end{itemize}
}
\item verb \\
To \textbf{tan} animal skins means to make them into leather by treating them with tannin or other chemicals .
 \textit{
	\begin{itemize}
	\item ...the process of tanning animal hides.
	\end{itemize}
}
\end{enumerate}

\section*{sentiment}
{\large \color{blue}  sentiments  }
\subsection*{Explain}
\begin{enumerate}
\item variable noun \\
A \textbf{sentiment} that people have is an attitude which is based on their thoughts and feelings.
 \textit{
	\begin{itemize}
	\item Public sentiment rapidly turned anti-American.
	\item He's found growing sentiment for military action.
	\item ...nationalist sentiments that threaten to split the country.
	\end{itemize}
}
\item countable noun \\
A \textbf{sentiment} is an idea or feeling that someone expresses in words.
 \textit{
	\begin{itemize}
	\item I must agree with the sentiments expressed by the previous speaker.
	\item The Foreign Secretary echoed this sentiment.
	\end{itemize}
}
\item uncountable noun \\
\textbf{Sentiment} is feelings such as pity or love , especially for things in the past , and may be considered exaggerated and foolish .
 \textit{
	\begin{itemize}
	\item Laura kept that letter out of sentiment.
	\item The coronation was an occasion for extravagant myth and sentiment.
	\end{itemize}
}
\end{enumerate}

\section*{tension}
{\large \color{blue}  tensions  }
\subsection*{Explain}
\begin{enumerate}
\item uncountable noun \\
\textbf{Tension} is the feeling that is produced in a situation when people are anxious and do not trust each other, and when there is a possibility of sudden  violence or conflict .
 \textit{
	\begin{itemize}
	\item The tension between the two countries is likely to remain.
	\item ...continued tension over the killing of demonstrators.
	\item The years of his government are remembered for political tension and conflict.
	\end{itemize}
}
\item uncountable noun \\
\textbf{Tension} is a feeling of worry and anxiety which makes it difficult for you to relax .
 \textit{
	\begin{itemize}
	\item She has done her best to keep calm but finds herself trembling with tension and indecision.
	\item Smiling and laughing has actually been shown to relieve tension and stress.
	\end{itemize}
}
\item variable noun \\
If there is a \textbf{tension} between forces, arguments , or influences , there are differences between them that cause difficulties .
 \textit{
	\begin{itemize}
	\item The film explored the tension between public duty and personal affections.
	\end{itemize}
}
\item uncountable noun \\
The \textbf{tension} in something such as a rope or wire is the extent to which it is stretched tight .
 \textit{
	\begin{itemize}
	\end{itemize}
}
\end{enumerate}

\section*{shatter}
{\large \color{blue}  shatters  shattering  shattered  }
\subsection*{Explain}
\begin{enumerate}
\item verb \\
If something \textbf{shatters} or \textbf{is shattered} , it breaks into a lot of small pieces.
 \textit{
	\begin{itemize}
	\item ...safety glass that won't shatter if it's broken.
	\item The car shattered into a thousand burning pieces in a 200mph crash.
	\item One bullet shattered his skull.
	\end{itemize}
}
\item verb \\
If something \textbf{shatters} your dreams , hopes , or beliefs , it completely destroys them.
 \textit{
	\begin{itemize}
	\item A failure would shatter the hopes of many people.
	\item Something like that really shatters your confidence.
	\item ...broken hearts and shattered dreams.
	\end{itemize}
}
\item verb \\
If someone \textbf{is shattered} by an event , it shocks and upsets them very much.
 \textit{
	\begin{itemize}
	\item He had been shattered by his son's death.
	\item ...the tragedy which had shattered his life.
	\end{itemize}
}
\end{enumerate}

\section*{toll}
{\large \color{blue}  tolls  tolling  tolled  }
\subsection*{Explain}
\begin{enumerate}
\item verb \\
When a bell  \textbf{tolls} or when someone \textbf{tolls} it, it rings slowly and repeatedly, often as a sign that someone has died .
 \textit{
	\begin{itemize}
	\item Church bells tolled and black flags fluttered.
	\item The pilgrims tolled the bell.
	\end{itemize}
}
\item countable noun \\
A \textbf{toll} is a small sum of money that you have to pay in order to use a particular bridge or road.
 \textit{
	\begin{itemize}
	\end{itemize}
}
\item countable noun \\
A \textbf{toll} road or \textbf{toll} bridge is a road or bridge where you have to pay in order to use it.
 \textit{
	\begin{itemize}
	\end{itemize}
}
\item countable noun \\
A \textbf{toll} is a total number of deaths , accidents, or disasters that occur in a particular period of time.
 \textit{
	\begin{itemize}
	\item There are fears that the casualty toll may be higher.
	\item ...the second highest annual murder toll in that city's history.
	\end{itemize}
}
\item  \\
 take its toll \textit{
	\begin{itemize}
	\end{itemize}
}
\end{enumerate}

\section*{spur}
{\large \color{blue}  spurs  spurring  spurred  }
\subsection*{Explain}
\begin{enumerate}
\item verb \\
If one thing \textbf{spurs} you \textbf{to} do another, it encourages you to do it.
 \textbf{Spur on} means the same as spur .
 \textit{
	\begin{itemize}
	\item It's the money that spurs these fishermen to risk a long ocean journey in their flimsy
boats.
	\item His friend's plight had spurred him into taking part.
	\item Their attitude, rather than reining him back, only seemed to spur Philip on.
	\item We may not like criticism, but it can spur us on to greater things.
	\end{itemize}
}
\item verb \\
If something \textbf{spurs} a change or event, it makes it happen faster or sooner .
 \textit{
	\begin{itemize}
	\item The administration may put more emphasis on spurring economic growth.
	\item The trade pacts will spur an exodus of U.S. businesses to Mexico.
	\end{itemize}
}
\item countable noun \\
Something that acts as a \textbf{spur}  \textbf{to} something else encourages a person or organization to do that thing or makes it happen
more quickly.
 \textit{
	\begin{itemize}
	\item ...a belief in competition as a spur to efficiency.
	\item Redundancy is the spur for many to embark on new careers.
	\end{itemize}
}
\item countable noun \\
\textbf{Spurs} are small metal wheels with sharp points that are attached to the heels of a rider's
 boots . The rider uses them to make their horse go faster.
 \textit{
	\begin{itemize}
	\end{itemize}
}
\item countable noun \\
The \textbf{spur} of a hill or mountain is a piece of ground which sticks out from its side.
 \textit{
	\begin{itemize}
	\end{itemize}
}
\item  \\
 on the spur of the moment \textit{
	\begin{itemize}
	\end{itemize}
}
\item  \\
 win one's spurs/earn one's spurs \textit{
	\begin{itemize}
	\end{itemize}
}
\end{enumerate}

\section*{university}
{\large \color{blue}  universities  }
\subsection*{Explain}
\begin{enumerate}
\item variable noun \\
A \textbf{university} is an institution where students study for degrees and where academic research is done.
 \textit{
	\begin{itemize}
	\item Patrick is now at London University.
	\item They want their daughter to go to university, but they are also keen that she get
a summer job.
	\item The university refused to let the controversial politician speak on campus.
	\end{itemize}
}
\end{enumerate}

\section*{verdict}
{\large \color{blue}  verdicts  }
\subsection*{Explain}
\begin{enumerate}
\item countable noun \\
In a court of law , the \textbf{verdict} is the decision that is given by the jury or judge at the end of a trial.
 \textit{
	\begin{itemize}
	\item The jury returned a unanimous guilty verdict.
	\item Three judges will deliver their verdict in October.
	\end{itemize}
}
\item countable noun \\
Someone's \textbf{verdict} on something is their opinion of it, after thinking about it or investigating it.
 \textit{
	\begin{itemize}
	\item The doctor's verdict was that he was entirely healthy.
	\item The critics were too quick to give their verdict on us.
	\end{itemize}
}
\end{enumerate}

\section*{thunder}
{\large \color{blue}  thunders  thundering  thundered  }
\subsection*{Explain}
\begin{enumerate}
\item uncountable noun \\
\textbf{Thunder} is the loud noise that you hear from the sky after a flash of lightning, especially during a storm .
 \textit{
	\begin{itemize}
	\item There was frequent thunder and lightning, and torrential rain.
	\item ...a distant clap of thunder.
	\end{itemize}
}
\item verb \\
When \textbf{it}  \textbf{thunders} , a loud noise comes from the sky after a flash of lightning.
 \textit{
	\begin{itemize}
	\item The day was heavy and still. It would probably thunder later.
	\end{itemize}
}
\item uncountable noun \\
The \textbf{thunder of} something that is moving or making a sound is the loud deep noise it makes.
 \textit{
	\begin{itemize}
	\item The thunder of the sea on the rocks seemed to blank out other thoughts.
	\item Khalil heard the thunder of an avalanche.
	\end{itemize}
}
\item verb \\
If something or someone \textbf{thunders}  somewhere , they move there quickly and with a lot of noise.
 \textit{
	\begin{itemize}
	\item The horses thundered across the valley floor.
	\item Niccolini was thundering up the stairs, taking them two at a time.
	\item A lorry thundered by.
	\end{itemize}
}
\item verb \\
If something \textbf{thunders} , it makes a very loud noise, usually continuously.
 \textit{
	\begin{itemize}
	\item She heard the sound of the guns thundering in the fog.
	\item ...thundering applause.
	\end{itemize}
}
\item verb \\
If you \textbf{thunder} something, you say it loudly and forcefully, especially because you are angry .
 \textit{
	\begin{itemize}
	\item 'It's your money. Ask for it!' she thundered.
	\item The Prosecutor looked toward Napoleon, waiting for him to thunder an objection.
	\end{itemize}
}
\item  \\
 to steal someone's thunder \textit{
	\begin{itemize}
	\end{itemize}
}
\end{enumerate}

\section*{wage}
{\large \color{blue}  wages  waging  waged  }
\subsection*{Explain}
\begin{enumerate}
\item countable noun \\
Someone's \textbf{wages} are the amount of money that is regularly paid to them for the work that they do.
 \textit{
	\begin{itemize}
	\item His wages have gone up.
	\item This may end efforts to set a minimum wage well above the poverty line.
	\end{itemize}
}
\item verb \\
If a person, group, or country \textbf{wages} a campaign or a war, they start it and continue it over a period of time.
 \textit{
	\begin{itemize}
	\item ...the three factions that had been waging a civil war.
	\item They waged a price war.
	\end{itemize}
}
\end{enumerate}

\section*{uproar}
{\large \color{blue}  }
\subsection*{Explain}
\begin{enumerate}
\item uncountable noun \\
If there is \textbf{uproar} , there is a lot of shouting and noise because people are very angry or upset about something.
 \textit{
	\begin{itemize}
	\item The announcement caused uproar in the crowd.
	\item The courtroom was in an uproar.
	\end{itemize}
}
\item uncountable noun \\
You can also use \textbf{uproar} to refer to a lot of public criticism and debate about something that has made people angry.
 \textit{
	\begin{itemize}
	\item The town is in uproar over the dispute.
	\item The surprise announcement could cause an uproar in the United States.
	\end{itemize}
}
\end{enumerate}

\section*{west}
{\large \color{blue}  }
\subsection*{Explain}
\begin{enumerate}
\item uncountable noun \\
The \textbf{west} is the direction which you look towards in the evening in order to see the sun set.
 \textit{
	\begin{itemize}
	\item I pushed on towards Flagstaff, a hundred miles to the west.
	\item The sun crosses the sky from east to west.
	\end{itemize}
}
\item singular noun \\
\textbf{The}  \textbf{west}  \textbf{of} a place, country, or region is the part of it which is in the west.
 \textit{
	\begin{itemize}
	\item ...physicists working at Bristol University in the west of England.
	\end{itemize}
}
\item adverb \\
If you go \textbf{west} , you travel towards the west.
 \textit{
	\begin{itemize}
	\item We are going west to California.
	\end{itemize}
}
\item adverb \\
Something that is \textbf{west}  \textbf{of} a place is positioned to the west of it.
 \textit{
	\begin{itemize}
	\item ...their home town of Paisley, several miles west of Glasgow.
	\end{itemize}
}
\item adjective \\
The \textbf{west} part of a place, country, or region is the part which is towards the west.
 \textit{
	\begin{itemize}
	\item ...a small island off the west coast of South Korea.
	\end{itemize}
}
\item adjective \\
\textbf{West} is used in the names of some countries, states, and regions in the west of a larger
area.
 \textit{
	\begin{itemize}
	\item Mark has been working in West Africa for about six months.
	\item ...his West London home.
	\item ...Charleston, West Virginia.
	\end{itemize}
}
\item adjective \\
A \textbf{west} wind is a wind that blows from the west.
 \textit{
	\begin{itemize}
	\end{itemize}
}
\item singular noun \\
\textbf{The West} is used to refer to the United States, Canada , and the countries of Western, Northern , and Southern Europe.
 \textit{
	\begin{itemize}
	\item ...relations between Iran and the West.
	\end{itemize}
}
\end{enumerate}

\section*{wedge}
{\large \color{blue}  wedges  wedging  wedged  }
\subsection*{Explain}
\begin{enumerate}
\item verb \\
If you \textbf{wedge} something, you force it to remain in a particular position by holding it there tightly or by fixing something next to it to prevent it from moving.
 \textit{
	\begin{itemize}
	\item I shut the shed door and wedged it with a log of wood.
	\item We slammed the gate after them, wedging it shut with planks.
	\end{itemize}
}
\item verb \\
If you \textbf{wedge} something somewhere , you fit it there tightly.
 \textit{
	\begin{itemize}
	\item Wedge the plug into the hole.
	\item The hotel's wedged right between the two airports.
	\end{itemize}
}
\item countable noun \\
A \textbf{wedge} is an object with one pointed edge and one thick edge, which you put under a door to keep it firmly in position.
 \textit{
	\begin{itemize}
	\end{itemize}
}
\item countable noun \\
A \textbf{wedge} is a piece of metal with a pointed edge which is used for splitting a material such as stone or wood, by being hammered into a crack in the material.
 \textit{
	\begin{itemize}
	\end{itemize}
}
\item countable noun \\
A \textbf{wedge}  \textbf{of} something such as fruit or cheese is a piece of it that has a thick triangular shape.
 \textit{
	\begin{itemize}
	\end{itemize}
}
\item  \\
 drive a wedge \textit{
	\begin{itemize}
	\end{itemize}
}
\item  \\
 the thin end of the wedge \textit{
	\begin{itemize}
	\end{itemize}
}
\end{enumerate}

\section*{zigzag}
{\large \color{blue}  zigzags  zigzagging  zigzagged  }
\subsection*{Explain}
\begin{enumerate}
\item countable noun \\
A \textbf{zigzag} is a line which has a series of angles in it like a continuous series of 'W's.
 \textit{
	\begin{itemize}
	\item They staggered in a zigzag across the tarmac.
	\item ...a zigzag pattern.
	\end{itemize}
}
\item verb \\
If you \textbf{zigzag} , you move forward by going at an angle first to one side then to the other.
 \textit{
	\begin{itemize}
	\item I zigzagged down a labyrinth of alleys.
	\item Expertly he zigzagged his way across the field.
	\end{itemize}
}
\end{enumerate}

\section*{zero}
{\large \color{blue}  zeros  zeroes  zeroing  zeroed  }
\subsection*{Explain}
\begin{enumerate}
\item number \\
\textbf{Zero} is the number 0.
 \textit{
	\begin{itemize}
	\item Visibility at the city's airport came down to zero, bringing air traffic to a standstill.
	\item ...a scale ranging from zero to seven.
	\end{itemize}
}
\item uncountable noun \\
\textbf{Zero} is a temperature of 0°. It is freezing point on the Centigrade and Celsius  scales , and 32° below freezing point on the Fahrenheit scale.
 \textit{
	\begin{itemize}
	\item It's a sunny late winter day, just a few degrees above zero.
	\item That night the mercury fell to thirty degrees below zero.
	\end{itemize}
}
\item adjective \\
You can use \textbf{zero} to say that there is none at all of the thing mentioned .
 \textit{
	\begin{itemize}
	\item This new ministry was being created with zero assets and zero liabilities.
	\item ...zero inflation.
	\item His chances are zero.
	\end{itemize}
}
\end{enumerate}

\section*{abdomen}
{\large \color{blue}  abdomens  }
\subsection*{Explain}
\begin{enumerate}
\item countable noun \\
Your \textbf{abdomen} is the part of your body below your chest where your stomach and intestines are.
 \textit{
	\begin{itemize}
	\item He was suffering from pains in his abdomen.
	\end{itemize}
}
\end{enumerate}

\section*{bias}
{\large \color{blue}  biases  biasing  biased  }
\subsection*{Explain}
\begin{enumerate}
\item variable noun \\
\textbf{Bias} is a tendency to prefer one person or thing to another, and to favour that person or thing.
 \textit{
	\begin{itemize}
	\item Bias against women permeates every level of the judicial system.
	\item There were fierce attacks on the BBC for alleged political bias.
	\end{itemize}
}
\item variable noun \\
\textbf{Bias} is a concern with or interest in one thing more than others.
 \textit{
	\begin{itemize}
	\item The Department has a strong bias towards neuroscience.
	\end{itemize}
}
\item verb \\
To \textbf{bias} someone means to influence them in favour of a particular choice .
 \textit{
	\begin{itemize}
	\item We mustn't allow it to bias our teaching.
	\end{itemize}
}
\item  \\
 on the bias \textit{
	\begin{itemize}
	\end{itemize}
}
\end{enumerate}

\section*{bargain}
{\large \color{blue}  bargains  bargaining  bargained  }
\subsection*{Explain}
\begin{enumerate}
\item countable noun \\
Something that is a \textbf{bargain} is good  value for money , usually because it has been sold at a lower price than normal .
 \textit{
	\begin{itemize}
	\item At this price the wine is a bargain.
	\item Fresh salmon is a bargain at the supermarket this week.
	\end{itemize}
}
\item countable noun \\
A \textbf{bargain} is an agreement, especially a formal  business agreement, in which two people or groups agree what each of them will do, pay , or receive.
 \textit{
	\begin{itemize}
	\item I'll make a bargain with you. I'll play hostess if you'll include Matthew in your
guest-list.
	\item The treaty was based on a bargain between the French and German governments.
	\end{itemize}
}
\item verb \\
When people \textbf{bargain}  \textbf{with} each other, they discuss what each of them will do, pay, or receive.
 \textit{
	\begin{itemize}
	\item They prefer to bargain with individual clients, for cash.
	\item Shop in small local markets and don't be afraid to bargain.
	\end{itemize}
}
\item  \\
 to drive a hard bargain \textit{
	\begin{itemize}
	\end{itemize}
}
\item  \\
 into the bargain \textit{
	\begin{itemize}
	\end{itemize}
}
\item  \\
 keep one's side of the bargain \textit{
	\begin{itemize}
	\end{itemize}
}
\end{enumerate}

\section*{boycott}
{\large \color{blue}  boycotts  boycotting  boycotted  }
\subsection*{Explain}
\begin{enumerate}
\item verb \\
If a country, group, or person \textbf{boycotts} a country, organization, or activity, they refuse to be involved with it in any way
because they disapprove of it.
 \textbf{Boycott} is also a noun .
 \textit{
	\begin{itemize}
	\item The main opposition parties are boycotting the elections.
	\item Opposition leaders had called for a boycott of the vote.
	\item ...a successful national boycott against the company's products.
	\end{itemize}
}
\end{enumerate}

\section*{bee}
{\large \color{blue}  bees  }
\subsection*{Explain}
\begin{enumerate}
\item countable noun \\
A \textbf{bee} is an insect with a yellow-and-black striped body that makes a buzzing noise as it flies . Bees make honey , and can sting .
 \textit{
	\begin{itemize}
	\end{itemize}
}
\item  \\
 to have a bee in your bonnet \textit{
	\begin{itemize}
	\end{itemize}
}
\item countable noun \\
A \textbf{bee} is a social event where people get together for a competition or to do something such as sew .
 \textit{
	\begin{itemize}
	\item That year I won first prize in the spelling bee.
	\item ...a group of friends at a quilting bee.
	\end{itemize}
}
\end{enumerate}

\section*{brown}
{\large \color{blue}  browner  brownest  browns  browning  browned  }
\subsection*{Explain}
\begin{enumerate}
\item colour \\
Something that is \textbf{brown} is the colour of earth or of wood.
 \textit{
	\begin{itemize}
	\item ...her deep brown eyes.
	\item The stairs are decorated in golds and earthy browns.
	\end{itemize}
}
\item adjective \\
You can describe a white-skinned person as \textbf{brown} when they have been sitting in the sun until their skin has become darker than usual .
 \textit{
	\begin{itemize}
	\item I don't want to be really really brown, just have a nice light golden colour.
	\end{itemize}
}
\item verb \\
If someone \textbf{browns} in the sun they become brown in colour.
 \textit{
	\begin{itemize}
	\item Her skin was of the fortunate kind that could brown in the sun without burning.
	\end{itemize}
}
\item adjective \\
\textbf{Brown} is used to describe grains that have not had their outer layers removed, and foods made from these grains.
 \textit{
	\begin{itemize}
	\item ...brown bread.
	\item ...spicy tomato sauce served over a bed of brown rice.
	\end{itemize}
}
\item verb \\
When food \textbf{browns} or when you \textbf{brown} food, you cook it, usually for a short time on a high flame .
 \textit{
	\begin{itemize}
	\item Cook for ten minutes until the sugar browns.
	\item He browned the chicken in a frying pan.
	\end{itemize}
}
\end{enumerate}

\section*{breeze}
{\large \color{blue}  breezes  breezing  breezed  }
\subsection*{Explain}
\begin{enumerate}
\item countable noun \\
A \textbf{breeze} is a gentle wind.
 \textit{
	\begin{itemize}
	\item ...a cool summer breeze.
	\end{itemize}
}
\item verb \\
If you \textbf{breeze}  \textbf{into} a place or a position, you enter it in a very casual or relaxed manner.
 \textit{
	\begin{itemize}
	\item Lopez breezed into the quarter-finals of the tournament.
	\item 'Are you all right?' Francine asked as she breezed in with the mail.
	\end{itemize}
}
\item verb \\
If you \textbf{breeze through} something such as a game or test , you cope with it easily .
 \textit{
	\begin{itemize}
	\item John seems to breeze effortlessly through his many commitments at work.
	\end{itemize}
}
\item singular noun \\
If you say that something is \textbf{a breeze} , you mean that it is very easy to do or to achieve .
 \textit{
	\begin{itemize}
	\item And after being manager for 20 people, handling my own tiny staff of three is a breeze!
	\item Making the pastry is a breeze if you have a food processor.
	\end{itemize}
}
\end{enumerate}

\section*{centigrade}
{\large \color{blue}  }
\subsection*{Explain}
\begin{enumerate}
\item adjective \\
\textbf{Centigrade} is a scale for measuring temperature , in which water freezes at 0 degrees and boils at 100 degrees. It is represented by the symbol °C.
 \textbf{Centigrade} is also a noun .
 \textit{
	\begin{itemize}
	\item ...daytime temperatures of up to forty degrees centigrade.
	\item The number at the bottom is the recommended water temperature in Centigrade.
	\end{itemize}
}
\end{enumerate}

\section*{bride}
{\large \color{blue}  brides  }
\subsection*{Explain}
\begin{enumerate}
\item countable noun \\
A \textbf{bride} is a woman who is getting married or who has just got married.
 \textit{
	\begin{itemize}
	\end{itemize}
}
\end{enumerate}

\section*{chap}
{\large \color{blue}  chaps  }
\subsection*{Explain}
\begin{enumerate}
\item countable noun \\
A \textbf{chap} is a man or boy.
 \textit{
	\begin{itemize}
	\item She thought he was a very nice chap.
	\end{itemize}
}
\end{enumerate}

\section*{bureau}
{\large \color{blue}  }
\subsection*{Explain}
\begin{enumerate}
\item countable noun \\
A \textbf{bureau} is an office, organization , or government department that collects and distributes  information .
 \textit{
	\begin{itemize}
	\item ...the Federal Bureau of Investigation.
	\item ...the Citizens' Advice Bureau.
	\end{itemize}
}
\item countable noun \\
A \textbf{bureau} is an office of a company or organization which has its main office in another town or country .
 \textit{
	\begin{itemize}
	\item ...the Wall Street Journal's Washington bureau.
	\end{itemize}
}
\item countable noun \\
A \textbf{bureau} is a writing desk with shelves and drawers and a lid that opens to form the writing surface.
 \textit{
	\begin{itemize}
	\end{itemize}
}
\item countable noun \\
A \textbf{bureau} is a chest of drawers.
 \textit{
	\begin{itemize}
	\end{itemize}
}
\end{enumerate}

\section*{command}
{\large \color{blue}  commands  commanding  commanded  }
\subsection*{Explain}
\begin{enumerate}
\item verb \\
If someone in authority \textbf{commands} you to do something, they tell you that you must do it.
 \textbf{Command} is also a noun .
 \textit{
	\begin{itemize}
	\item He commanded his troops to attack.
	\item 'Get in your car and follow me,' he commanded.
	\item He commanded that roads be built to link castles across the land.
	\item 'Don't panic,' I commanded myself.
	\item The tanker failed to respond to a command to stop.
	\item I closed my eyes at his command.
	\item ...the note of command in his voice.
	\end{itemize}
}
\item verb \\
If you \textbf{command} something such as respect or obedience , you obtain it because you are popular , famous , or important .
 \textit{
	\begin{itemize}
	\item ...an excellent physician who commanded the respect of all his colleagues.
	\item There is no limit to what can be achieved here because of the fantastic support we
command.
	\end{itemize}
}
\item verb \\
If an army or country \textbf{commands} a place, they have total control over it.
 \textbf{Command} is also a noun.
 \textit{
	\begin{itemize}
	\item The Royal Navy would command the seas.
	\item Yemen commands the strait at the southern end of the Red Sea.
	\item ...the struggle for command of the air.
	\end{itemize}
}
\item verb \\
An officer who \textbf{commands} part of an army, navy , or air force is responsible for controlling and organizing it.
 \textbf{Command} is also a noun.
 \textit{
	\begin{itemize}
	\item ...the French general who commands the U.N. troops in the region.
	\item He didn't just command. He personally fought in several heavy battles.
	\item ...a small garrison under the command of Major James Craig.
	\item He took command of 108 Squadron.
	\end{itemize}
}
\item countable noun \\
In the armed forces, a \textbf{command} is a group of officers who are responsible for organizing and controlling part of
an army, navy, or air force.
 \textit{
	\begin{itemize}
	\item He had authorisation from the military command to retaliate.
	\item The army's supreme command has said the army will withdraw.
	\end{itemize}
}
\item collective countable noun \\
In the armed forces, a \textbf{command} is a group of soldiers that a particular officer is in charge of.
 \textit{
	\begin{itemize}
	\item There would continue to be a joint command of U.S. and Saudi forces operating within
Saudi borders.
	\item ...the Strategic Air Command.
	\end{itemize}
}
\item countable noun \\
In computing , a \textbf{command} is an instruction that you give to a computer.
 \textit{
	\begin{itemize}
	\end{itemize}
}
\item uncountable noun \\
If someone has \textbf{command} of a situation, they have control of it because they have, or seem to have, power or authority.
 \textit{
	\begin{itemize}
	\item Whoever was waiting for them there had command of the situation.
	\item Mr Baker would take command of the campaign.
	\item It was his senior partner who was in command.
	\end{itemize}
}
\item uncountable noun \\
Your \textbf{command of} something, such as a foreign language, is your knowledge of it and your ability to use this knowledge.
 \textit{
	\begin{itemize}
	\item His command of English was excellent.
	\item ...a singer with a natural command of melody.
	\end{itemize}
}
\item verb \\
If a place \textbf{commands} a view, especially an impressive one, you can see the view clearly from that place. If a person \textbf{commands} a view of something, they can see it clearly from where they are.
 \textit{
	\begin{itemize}
	\item The house commanded some splendid views of Delaware Bay.
	\item ...a point of rock, from which we could command a view of the loch.
	\end{itemize}
}
\item  \\
 have sth at one's command \textit{
	\begin{itemize}
	\end{itemize}
}
\item  \\
 be in command/be in command of yourself \textit{
	\begin{itemize}
	\end{itemize}
}
\end{enumerate}

\section*{camera}
{\large \color{blue}  cameras  }
\subsection*{Explain}
\begin{enumerate}
\item countable noun \\
A \textbf{camera} is a piece of equipment that is used for taking  photographs , making films, or producing television  pictures . Many cameras are now included as part of other digital devices such as phones and tablets .
 \textit{
	\begin{itemize}
	\item Her gran lent her a camera for a school trip to Venice and Egypt.
	\item ...a video camera.
	\item They were caught speeding by hidden cameras.
	\end{itemize}
}
\item  \\
 on camera \textit{
	\begin{itemize}
	\end{itemize}
}
\item  \\
 off camera \textit{
	\begin{itemize}
	\end{itemize}
}
\item  \\
 in camera \textit{
	\begin{itemize}
	\end{itemize}
}
\end{enumerate}

\section*{copyright}
{\large \color{blue}  copyrights  }
\subsection*{Explain}
\begin{enumerate}
\item variable noun \\
If someone has \textbf{copyright} on a piece of writing or music, it is illegal to reproduce or perform it without their permission .
 \textit{
	\begin{itemize}
	\item To order a book one first had to get permission from the monastery that held the
copyright.
	\item She threatened legal action for breach of copyright.
	\end{itemize}
}
\end{enumerate}

\section*{crime}
{\large \color{blue}  crimes  }
\subsection*{Explain}
\begin{enumerate}
\item variable noun \\
A \textbf{crime} is an illegal action or activity for which a person can be punished by law.
 \textit{
	\begin{itemize}
	\item He and Lieutenant Cassidy were checking the scene of the crime.
	\item Mr Steele has committed no crime and poses no danger to the public.
	\item Endangering their lives will be regarded as a crime against humanity.
	\item ...the growing problem of organised crime.
	\item We need a positive programme of crime prevention.
	\end{itemize}
}
\item countable noun \\
If you say that doing something is a \textbf{crime} , you think it is very wrong or a serious  mistake .
 \textit{
	\begin{itemize}
	\item A language is a finely tuned instrument which it is a crime to damage.
	\item It would be a crime to travel all the way to Australia and not stop in Sydney.
	\end{itemize}
}
\end{enumerate}

\section*{crush}
{\large \color{blue}  crushes  crushing  crushed  }
\subsection*{Explain}
\begin{enumerate}
\item verb \\
To \textbf{crush} something means to press it very hard so that its shape is destroyed or so that it breaks into pieces.
 \textit{
	\begin{itemize}
	\item Andrew crushed his empty can.
	\item Their vehicle was crushed by an army tank.
	\item Peel and crush the garlic.
	\item ...crushed ice.
	\end{itemize}
}
\item verb \\
To \textbf{crush} a protest or movement, or a group of opponents , means to defeat it completely, usually by force.
 \textit{
	\begin{itemize}
	\item The military operation was the first step in a plan to crush the uprising.
	\item ...in his bid to crush the rebels.
	\end{itemize}
}
\item verb \\
If you \textbf{are crushed} by something, it upsets you a great  deal .
 \textit{
	\begin{itemize}
	\item Listen to criticism but don't be crushed by it.
	\end{itemize}
}
\item verb \\
If you \textbf{are crushed} against someone or something, you are pushed or pressed against them.
 \textit{
	\begin{itemize}
	\item We were at the front, crushed against the stage.
	\end{itemize}
}
\item countable noun \\
A \textbf{crush} is a crowd of people close together, in which it is difficult to move.
 \textit{
	\begin{itemize}
	\item Franklin and his thirteen-year-old son somehow got separated in the crush.
	\item Everywhere he went he was mobbed by a crush of fans.
	\end{itemize}
}
\item countable noun \\
If you have a \textbf{crush}  \textbf{on} someone, you are in love with them but do not have a relationship with them.
 \textit{
	\begin{itemize}
	\item She had a crush on you, you know.
	\item I'd got over my schoolgirl crush.
	\end{itemize}
}
\end{enumerate}

\section*{criminal}
{\large \color{blue}  criminals  }
\subsection*{Explain}
\begin{enumerate}
\item countable noun \\
A \textbf{criminal} is a person who regularly commits crimes.
 \textit{
	\begin{itemize}
	\item A group of gunmen attacked a prison and set free nine criminals in Moroto.
	\end{itemize}
}
\item adjective \\
\textbf{Criminal} means connected with crime.
 \textit{
	\begin{itemize}
	\item He faces various criminal charges.
	\item At 17, he had a criminal record for petty theft.
	\item Doug was found guilty of criminal assault and sentenced to six months in jail.
	\end{itemize}
}
\item adjective \\
If you describe an action as \textbf{criminal} , you think it is very wrong or a serious  mistake .
 \textit{
	\begin{itemize}
	\item He said a full-scale dispute involving strikes would be criminal.
	\end{itemize}
}
\end{enumerate}

\section*{cut}
{\large \color{blue}  cuts  cutting  }
\subsection*{Explain}
\begin{enumerate}
\item verb \\
If you \textbf{cut} something, you use a knife or a similar tool to divide it into pieces, or to mark
it or damage it. If you \textbf{cut} a shape or a hole in something, you make the shape or hole by using a knife or similar
tool.
 \textbf{Cut} is also a noun.
 \textit{
	\begin{itemize}
	\item Mrs. Haines stood nearby, holding scissors to cut a ribbon.
	\item Cut the tomatoes in half vertically.
	\item The thieves cut a hole in the fence.
	\item Mr. Long was now cutting himself a piece of the pink cake.
	\item You can hear the saw as it cuts through the bones.
	\item ...thinly-cut cucumber sandwiches.
	\item The operation involves making several cuts in the cornea.
	\end{itemize}
}
\item verb \\
If you \textbf{cut}  \textbf{yourself} or \textbf{cut} a part of your body, you accidentally injure yourself on a sharp object so that you
 bleed .
 \textbf{Cut} is also a noun.
 \textit{
	\begin{itemize}
	\item Johnson cut himself shaving.
	\item I started to cry because I cut my finger.
	\item Zoe was badly cut as she scrambled down rocks to reach him.
	\item Blood from his cut lip trickled over his chin.
	\item He had sustained a cut on his left eyebrow.
	\item ...cuts and bruises.
	\end{itemize}
}
\item verb \\
If you \textbf{cut} something such as grass, your hair, or your fingernails , you shorten them using scissors or another tool.
 \textbf{Cut} is also a noun.
 \textit{
	\begin{itemize}
	\item The most recent tenants hadn't even cut the grass.
	\item You have to learn not to cut your toenails in the living room.
	\item You've had your hair cut, it looks great.
	\item She had dark red hair, cut short.
	\item Prices vary from salon to salon, starting at £17 for a cut and blow-dry.
	\end{itemize}
}
\item verb \\
The way that clothes \textbf{are cut} is the way they are designed and made.
 \textit{
	\begin{itemize}
	\item ...badly-cut blue suits.
	\end{itemize}
}
\item verb \\
To \textbf{cut through} something means to move or pass through it easily.
 \textit{
	\begin{itemize}
	\item I could see long canoes cutting through the waves.
	\end{itemize}
}
\item verb \\
If you \textbf{cut across} or \textbf{through} a place, you go through it because it is the shortest route to another place.
 \textit{
	\begin{itemize}
	\item He decided to cut across the Heath, through Greenwich Park.
	\end{itemize}
}
\item verb \\
If you \textbf{cut} something, you reduce it.
 \textbf{Cut} is also a noun.
 \textit{
	\begin{itemize}
	\item The first priority is to cut costs.
	\item The U.N. force is to be cut by 90%.
	\item ...a deal to cut 50 billion dollars from the federal deficit.
	\item The economy needs an immediate 2 per cent cut in interest rates.
	\item ...the government's plans for tax cuts.
	\end{itemize}
}
\item verb \\
If you \textbf{cut} a text, broadcast, or performance, you shorten it. If you \textbf{cut} a part of a text, broadcast, or performance, you do not publish, broadcast, or perform
that part.
 \textbf{Cut} is also a noun.
 \textit{
	\begin{itemize}
	\item The audience wants more music and less drama, so we've cut some scenes.
	\item It has been found necessary to make some cuts in the text.
	\end{itemize}
}
\item verb \\
To \textbf{cut} a supply of something means to stop providing it or stop it being provided.
 \textbf{Cut} is also a noun.
 \textit{
	\begin{itemize}
	\item They used pressure tactics to force them to return, including cutting food and water
supplies.
	\item The strike had already led to cuts in electricity and water supplies in many areas.
	\end{itemize}
}
\item verb \\
If you \textbf{cut} a pack of playing cards, you divide it into two.
 \textit{
	\begin{itemize}
	\item Place the cards face down on the table and cut them.
	\end{itemize}
}
\item convention \\
When the director of a film says ' \textbf{cut} ', they want the actors and the camera crew to stop filming .
 \textit{
	\begin{itemize}
	\end{itemize}
}
\item verb \\
When a singer or band \textbf{cuts} a CD, they make a recording of their music.
 \textit{
	\begin{itemize}
	\item She eventually cut her own album.
	\end{itemize}
}
\item verb \\
When a child \textbf{cuts} a tooth, a new tooth starts to grow through the gum.
 \textit{
	\begin{itemize}
	\item Many infants do not cut their first tooth until they are a year old.
	\end{itemize}
}
\item verb \\
If a child \textbf{cuts}  classes or \textbf{cuts} school, they do not go to classes or to school when they are supposed to.
 \textit{
	\begin{itemize}
	\item Cutting school more than once in three months is a sign of trouble.
	\end{itemize}
}
\item verb \\
If you tell someone to \textbf{cut} something, you are telling them in an irritated way to stop it.
 \textit{
	\begin{itemize}
	\item 'Cut the euphemisms, Daniel,' Brenda snapped.
	\item Why don't you just cut the crap and open the door.
	\end{itemize}
}
\item countable noun \\
A \textbf{cut} of meat is a piece or type of meat which is cut in a particular way from the animal,
or from a particular part of it.
 \textit{
	\begin{itemize}
	\item Use a cheap cut such as spare rib chops.
	\end{itemize}
}
\item singular noun \\
Someone's \textbf{cut} of the profits or winnings from something, especially ones that have been obtained dishonestly, is their share.
 \textit{
	\begin{itemize}
	\item The lawyers, of course, take their cut of the little guy's winnings.
	\end{itemize}
}
\item countable noun \\
A \textbf{cut} is a narrow valley which has been cut through a hill so that a road or railroad track can pass through.
 \textit{
	\begin{itemize}
	\end{itemize}
}
\item  \\
 a cut above \textit{
	\begin{itemize}
	\end{itemize}
}
\item  \\
 cut sb dead \textit{
	\begin{itemize}
	\end{itemize}
}
\item  \\
 cut and dried \textit{
	\begin{itemize}
	\end{itemize}
}
\item  \\
 to cut loose \textit{
	\begin{itemize}
	\end{itemize}
}
\item  \\
 to cut and run \textit{
	\begin{itemize}
	\end{itemize}
}
\item  \\
 cut it \textit{
	\begin{itemize}
	\end{itemize}
}
\item  \\
 cut and thrust \textit{
	\begin{itemize}
	\end{itemize}
}
\item  \\
 to cut both ways \textit{
	\begin{itemize}
	\end{itemize}
}
\end{enumerate}

\section*{curb}
{\large \color{blue}  curbs  curbing  curbed  }
\subsection*{Explain}
\begin{enumerate}
\item verb \\
If you \textbf{curb} something, you control it and keep it within limits .
 \textbf{Curb} is also a noun .
 \textit{
	\begin{itemize}
	\item ...advertisements aimed at curbing the spread of the disease.
	\item He called for energy consumption to be curbed.
	\item He called for much stricter curbs on immigration.
	\end{itemize}
}
\item verb \\
If you \textbf{curb} an emotion or your behaviour , you keep it under control.
 \textit{
	\begin{itemize}
	\item He curbed his temper.
	\item You must curb your extravagant tastes.
	\end{itemize}
}
\end{enumerate}

\section*{despatch}
{\large \color{blue}  }
\subsection*{Explain}
\begin{enumerate}
\end{enumerate}

\section*{destruction}
{\large \color{blue}  }
\subsection*{Explain}
\begin{enumerate}
\item uncountable noun \\
\textbf{Destruction} is the act of destroying something, or the state of being destroyed.
 \textit{
	\begin{itemize}
	\item ...an international agreement aimed at halting the destruction of the ozone layer.
	\item ...weapons of mass destruction.
	\end{itemize}
}
\end{enumerate}

\section*{dislike}
{\large \color{blue}  dislikes  disliking  disliked  }
\subsection*{Explain}
\begin{enumerate}
\item verb \\
If you \textbf{dislike} someone or something, you consider them to be unpleasant and do not like them.
 \textit{
	\begin{itemize}
	\item We don't serve liver often because so many people dislike it.
	\item David began to dislike anyone who smoked.
	\end{itemize}
}
\item uncountable noun \\
\textbf{Dislike} is the feeling that you do not like someone or something.
 \textit{
	\begin{itemize}
	\item He made no attempt to conceal his dislike of me.
	\item Years of dislike boiled over and blows were exchanged.
	\end{itemize}
}
\item countable noun \\
Your \textbf{dislikes} are the things that you do not like.
 \textit{
	\begin{itemize}
	\item Consider what your likes and dislikes are about your job.
	\item Strong irrational dislikes of other people can easily be picked up from others.
	\end{itemize}
}
\item  \\
 take a dislike \textit{
	\begin{itemize}
	\end{itemize}
}
\end{enumerate}

\section*{dwarf}
{\large \color{blue}  dwarfs  dwarves  dwarfs  dwarfing  dwarfed  }
\subsection*{Explain}
\begin{enumerate}
\item verb \\
If one person or thing \textbf{is dwarfed} by another, the second is so much bigger than the first that it makes them look very small.
 \textit{
	\begin{itemize}
	\item His figure is dwarfed by the huge red McDonald's sign.
	\item The U.S. air travel market dwarfs that of Britain.
	\end{itemize}
}
\item countable noun \\
\textbf{Dwarf} is used to describe a particular  kind of star which is quite small and not very bright .
 \textit{
	\begin{itemize}
	\item ...a white dwarf star.
	\item ...a red dwarf.
	\end{itemize}
}
\item adjective \\
\textbf{Dwarf} is used to describe varieties or species of plants and animals which are much smaller than the usual size for their kind.
 \textit{
	\begin{itemize}
	\item ...dwarf shrubs.
	\end{itemize}
}
\item countable noun \\
In children's stories , a \textbf{dwarf} is an imaginary creature that is like a small man . Dwarfs often have magical powers.
 \textit{
	\begin{itemize}
	\end{itemize}
}
\item countable noun \\
In former times, people who were much smaller than normal were called  \textbf{dwarfs} .
 \textit{
	\begin{itemize}
	\end{itemize}
}
\end{enumerate}

\section*{dismay}
{\large \color{blue}  dismays  dismaying  dismayed  }
\subsection*{Explain}
\begin{enumerate}
\item uncountable noun \\
\textbf{Dismay} is a strong feeling of fear , worry , or sadness that is caused by something unpleasant and unexpected .
 \textit{
	\begin{itemize}
	\item Local councillors have reacted with dismay and indignation.
	\item Lucy discovered to her dismay that she was pregnant.
	\item The ministers expressed dismay at the continued practice of ethnic cleansing.
	\item Meg looked up at her in dismay.
	\end{itemize}
}
\item verb \\
If you \textbf{are dismayed} by something, it makes you feel  afraid , worried, or sad .
 \textit{
	\begin{itemize}
	\item The committee was dismayed by what it had been told.
	\item The thought that she was crying dismayed him.
	\end{itemize}
}
\end{enumerate}

\section*{epoch}
{\large \color{blue}  epochs  }
\subsection*{Explain}
\begin{enumerate}
\item countable noun \\
If you refer to a long period of time as an \textbf{epoch} , you mean that important  events or great changes took place during it.
 \textit{
	\begin{itemize}
	\item The birth of Christ was the beginning of a major epoch of world history.
	\end{itemize}
}
\item countable noun \\
An \textbf{epoch} is a very long period of time in the earth's development , marked by particular physical or biological characteristics.
 \textit{
	\begin{itemize}
	\item Two main glacial epochs affected both areas during the last 100 million years of
Precambrian times.
	\end{itemize}
}
\end{enumerate}

\section*{doubt}
{\large \color{blue}  doubts  doubting  doubted  }
\subsection*{Explain}
\begin{enumerate}
\item variable noun \\
If you have \textbf{doubt} or \textbf{doubts} about something, you feel uncertain about it and do not know whether it is true or possible . If you say you have \textbf{no}  \textbf{doubt}  \textbf{about} it, you mean that you are certain it is true.
 \textit{
	\begin{itemize}
	\item This raises doubts about the point of advertising.
	\item I had my doubts when she started, but she's getting really good.
	\item They were troubled and full of doubt.
	\item There can be little doubt that he will offend again.
	\item Local inhabitants haven't the slightest doubt as to who is the rightful owner.
	\end{itemize}
}
\item verb \\
If you \textbf{doubt} whether something is true or possible, you believe that it is probably not true or possible.
 \textit{
	\begin{itemize}
	\item Others doubted whether that would happen.
	\item He doubted if he would learn anything new from Marie.
	\item She doubted that the accident could have been avoided.
	\end{itemize}
}
\item verb \\
If you \textbf{doubt} something, you believe that it might not be true or genuine .
 \textit{
	\begin{itemize}
	\item No one doubted his ability.
	\item Nobody that I spoke to doubted his sincerity as a politician.
	\end{itemize}
}
\item verb \\
If you \textbf{doubt} someone or \textbf{doubt} their word, you think that they may not be telling the truth.
 \textit{
	\begin{itemize}
	\item No one directly involved with the case doubted him.
	\item I still have no reason to doubt his word.
	\end{itemize}
}
\item  \\
 beyond doubt \textit{
	\begin{itemize}
	\end{itemize}
}
\item  \\
 in doubt \textit{
	\begin{itemize}
	\end{itemize}
}
\item  \\
 I doubt it \textit{
	\begin{itemize}
	\end{itemize}
}
\item  \\
 in doubt \textit{
	\begin{itemize}
	\end{itemize}
}
\item  \\
 no doubt \textit{
	\begin{itemize}
	\end{itemize}
}
\item  \\
 no doubt \textit{
	\begin{itemize}
	\end{itemize}
}
\item  \\
 without (a) doubt \textit{
	\begin{itemize}
	\end{itemize}
}
\end{enumerate}

\section*{error}
{\large \color{blue}  errors  }
\subsection*{Explain}
\begin{enumerate}
\item variable noun \\
An \textbf{error} is something you have done which is considered to be incorrect or wrong, or which should not have been done.
 \textit{
	\begin{itemize}
	\item NASA discovered a mathematical error in its calculations.
	\item MPs attacked lax management and errors of judgment.
	\end{itemize}
}
\item  \\
 in error \textit{
	\begin{itemize}
	\end{itemize}
}
\item  \\
 to see the error of your ways \textit{
	\begin{itemize}
	\end{itemize}
}
\end{enumerate}

\section*{fatigue}
{\large \color{blue}  fatigues  }
\subsection*{Explain}
\begin{enumerate}
\item uncountable noun \\
\textbf{Fatigue} is a feeling of extreme physical or mental tiredness.
 \textit{
	\begin{itemize}
	\item She continued to have severe stomach cramps, aches, fatigue, and depression.
	\item His team lasted another 15 days before fatigue began to take its toll.
	\end{itemize}
}
\item uncountable noun \\
You can say that people are suffering from a particular kind of \textbf{fatigue} when they have been doing something for a long time and feel they can no longer continue to do it.
 \textit{
	\begin{itemize}
	\item ...compassion fatigue caused by endless TV and celebrity appeals.
	\item ...the result of four months of battle fatigue.
	\end{itemize}
}
\item plural noun \\
\textbf{Fatigues} are clothes that soldiers wear when they are fighting or when they are doing routine  jobs .
 \textit{
	\begin{itemize}
	\item He never expected to return home wearing U.S. combat fatigues.
	\end{itemize}
}
\item uncountable noun \\
\textbf{Fatigue} in metal or wood is a weakness in it that is caused by repeated stress. Fatigue can cause the metal or wood to break.
 \textit{
	\begin{itemize}
	\item The problem turned out to be metal fatigue in the fuselage.
	\end{itemize}
}
\end{enumerate}

\section*{explanation}
{\large \color{blue}  explanations  }
\subsection*{Explain}
\begin{enumerate}
\item countable noun \\
If you give an \textbf{explanation} of something that has happened , you give people reasons for it, especially in an attempt to justify it.
 \textit{
	\begin{itemize}
	\item She told the court she would give a full explanation of the prosecution's decision
on Monday.
	\item There was a hint of schoolboy shyness in his explanation.
	\item 'It's my ulcer,' he added by way of explanation.
	\end{itemize}
}
\item countable noun \\
If you say there is an \textbf{explanation}  \textbf{for} something, you mean that there is a reason for it.
 \textit{
	\begin{itemize}
	\item The deputy airport manager said there was no apparent explanation for the crash.
	\item Scientific explanations for natural phenomena are widely accepted.
	\item It's the only explanation for these results.
	\end{itemize}
}
\item countable noun \\
If you give an \textbf{explanation}  \textbf{of} something, you give details about it or describe it so that it can be understood .
 \textit{
	\begin{itemize}
	\item Haig was immediately impressed by Charteris's expertise and by his lucid explanation
of the work.
	\end{itemize}
}
\end{enumerate}

\section*{fax}
{\large \color{blue}  faxes  faxing  faxed  }
\subsection*{Explain}
\begin{enumerate}
\item countable noun \\
A \textbf{fax} or a \textbf{fax machine} is a piece of equipment that was used in the past to copy documents by sending information electronically along a telephone  line , and to receive copies that were sent in this way .
 \textit{
	\begin{itemize}
	\item ...a modern reception desk with telephone and fax.
	\item Back then, cartoonists sent in their work by fax.
	\end{itemize}
}
\item verb \\
In the past, if you \textbf{faxed} a document to someone, you sent it from one fax machine to another.
 \textit{
	\begin{itemize}
	\item I faxed a copy of the agreement to each of the investors.
	\item Did you fax him a reply?
	\item Pop it in the post, or get your secretary to fax it.
	\item I faxed 10 hotels in the area to check room size.
	\end{itemize}
}
\item countable noun \\
You can  refer to a copy of a document that was transmitted by a fax machine as a \textbf{fax} .
 \textit{
	\begin{itemize}
	\item I sent him a long fax, saying I didn't need any help.
	\item ...a 2,000 word fax message.
	\end{itemize}
}
\end{enumerate}

\section*{fool}
{\large \color{blue}  fools  fooling  fooled  }
\subsection*{Explain}
\begin{enumerate}
\item countable noun \\
If you call someone a \textbf{fool} , you are indicating that you think they are not at all sensible and show a lack of good judgment.
 \textit{
	\begin{itemize}
	\item 'You fool!' she shouted.
	\item He'd been a fool to get involved with her!
	\end{itemize}
}
\item adjective \\
\textbf{Fool} is used to describe an action or person that is not at all sensible and shows a lack of good judgment.
 \textit{
	\begin{itemize}
	\item What a damn fool thing to do!
	\item What can that fool guard be thinking of?
	\end{itemize}
}
\item verb \\
If someone \textbf{fools} you, they deceive or trick you.
 \textit{
	\begin{itemize}
	\item Art dealers fool a lot of people.
	\item Don't be fooled by his appearance.
	\item They tried to fool you into coming after us.
	\end{itemize}
}
\item verb \\
If you say that a person \textbf{is fooling with} something or someone, you mean that the way they are behaving is likely to cause problems .
 \textit{
	\begin{itemize}
	\item What are you doing fooling with such a staggering sum of money?
	\item He kept telling her that here you did not fool with officials.
	\end{itemize}
}
\item countable noun \\
In the courts of kings and queens in medieval  Europe , \textbf{the}  \textbf{fool} was the person whose job was to do silly things in order to make people laugh .
 \textit{
	\begin{itemize}
	\end{itemize}
}
\item variable noun \\
\textbf{Fool} is a dessert made by mixing  soft  cooked fruit with whipped cream or with custard.
 \textit{
	\begin{itemize}
	\item ...gooseberry fool.
	\end{itemize}
}
\item  \\
 make a fool of someone \textit{
	\begin{itemize}
	\end{itemize}
}
\item  \\
 make a fool of yourself \textit{
	\begin{itemize}
	\end{itemize}
}
\item  \\
 more fool (you) \textit{
	\begin{itemize}
	\end{itemize}
}
\item  \\
 to play the fool \textit{
	\begin{itemize}
	\end{itemize}
}
\end{enumerate}

\section*{fuss}
{\large \color{blue}  fusses  fussing  fussed  }
\subsection*{Explain}
\begin{enumerate}
\item singular noun \\
\textbf{Fuss} is anxious or excited  behaviour which serves no useful  purpose .
 \textit{
	\begin{itemize}
	\item I don't know what all the fuss is about.
	\item He just gets down to work without any fuss.
	\end{itemize}
}
\item verb \\
If you \textbf{fuss} , you worry or behave in a nervous, anxious way about unimportant  matters or rush around doing unnecessary things.
 \textit{
	\begin{itemize}
	\item Carol fussed about getting me a drink.
	\item Jack was fussing over the food and clothing we were going to take.
	\item She fussed with a wisp of hair over her ear.
	\item A team of waiters began fussing around the table.
	\item 'Stop fussing,' he snapped.
	\end{itemize}
}
\item verb \\
If you \textbf{fuss over} someone, you pay them a lot of attention and do things to make them happy or comfortable .
 \textit{
	\begin{itemize}
	\item Auntie Hilda and Uncle Jack couldn't fuss over them enough.
	\end{itemize}
}
\item  \\
 make a fuss \textit{
	\begin{itemize}
	\end{itemize}
}
\item  \\
 make a fuss of \textit{
	\begin{itemize}
	\end{itemize}
}
\end{enumerate}

\section*{germ}
{\large \color{blue}  germs  }
\subsection*{Explain}
\begin{enumerate}
\item countable noun \\
A \textbf{germ} is a very small organism that causes disease.
 \textit{
	\begin{itemize}
	\item Chlorine is widely used to kill germs.
	\item ...a germ that destroyed hundreds of millions of lives.
	\end{itemize}
}
\item singular noun \\
The \textbf{germ of} something such as an idea is something which developed or might develop into that thing.
 \textit{
	\begin{itemize}
	\item The germ of an idea took root in Rosemary's mind.
	\item This was the germ of a book.
	\end{itemize}
}
\end{enumerate}

\section*{gamble}
{\large \color{blue}  gambles  gambling  gambled  }
\subsection*{Explain}
\begin{enumerate}
\item countable noun \\
A \textbf{gamble} is a risky action or decision that you take in the hope of gaining money, success , or an advantage over other people.
 \textit{
	\begin{itemize}
	\item Yesterday, he named his cabinet and took a big gamble in the process.
	\item ...the president's risky gamble in calling a referendum.
	\end{itemize}
}
\item verb \\
If you \textbf{gamble}  \textbf{on} something, you take a risky action or decision in the hope of gaining money, success,
or an advantage over other people.
 \textit{
	\begin{itemize}
	\item Few firms will be willing to gamble on new products.
	\item They are not prepared to gamble their careers on this matter.
	\item Who wants to gamble with the life of a friend?
	\end{itemize}
}
\item verb \\
If you \textbf{gamble} an amount of money, you bet it in a game such as cards or on the result of a race or competition . People who \textbf{gamble} usually do it frequently.
 \textit{
	\begin{itemize}
	\item Most people visit Las Vegas to gamble their hard-earned money.
	\item John gambled heavily on the horses.
	\item He was gambling heavily, to the alarm of his family.
	\item He gambled away his family estate on a single throw of the dice.
	\end{itemize}
}
\end{enumerate}

\section*{guilt}
{\large \color{blue}  }
\subsection*{Explain}
\begin{enumerate}
\item uncountable noun \\
\textbf{Guilt} is an unhappy feeling that you have because you have done something wrong or think that you have done something wrong.
 \textit{
	\begin{itemize}
	\item Her emotions had ranged from anger to guilt in the space of a few seconds.
	\item Some cancer patients experience strong feelings of guilt.
	\end{itemize}
}
\item uncountable noun \\
\textbf{Guilt} is the fact that you have done something wrong or illegal .
 \textit{
	\begin{itemize}
	\item The trial is concerned only with the determination of guilt according to criminal
law.
	\item You weren't convinced of Mr Matthews' guilt.
	\end{itemize}
}
\end{enumerate}

\section*{glimpse}
{\large \color{blue}  glimpses  glimpsing  glimpsed  }
\subsection*{Explain}
\begin{enumerate}
\item countable noun \\
If you get a \textbf{glimpse}  \textbf{of} someone or something, you see them very briefly and not very well .
 \textit{
	\begin{itemize}
	\item Some of the fans had waited 24 hours outside the hotel to catch a glimpse of the
singer.
	\end{itemize}
}
\item verb \\
If you \textbf{glimpse} someone or something, you see them very briefly and not very well.
 \textit{
	\begin{itemize}
	\item She glimpsed a group of people standing on the bank of a river.
	\end{itemize}
}
\item countable noun \\
A \textbf{glimpse}  \textbf{of} something is a brief experience of it or an idea about it that helps you understand or appreciate it better .
 \textit{
	\begin{itemize}
	\item The programme offers a glimpse of their working methods.
	\item ...a glimpse into the future.
	\end{itemize}
}
\end{enumerate}

\section*{hint}
{\large \color{blue}  hints  hinting  hinted  }
\subsection*{Explain}
\begin{enumerate}
\item countable noun \\
A \textbf{hint} is a suggestion about something that is made in an indirect way.
 \textit{
	\begin{itemize}
	\item The Minister gave a strong hint that the government were thinking of introducing
tax concessions for mothers.
	\item I'd dropped a hint about having an exhibition of his work up here.
	\item The statement gave no hint as to what the measures would be.
	\end{itemize}
}
\item verb \\
If you \textbf{hint}  \textbf{at} something, you suggest it in an indirect way.
 \textit{
	\begin{itemize}
	\item She suggested a trip to the shops and hinted at the possibility of a treat of some
sort.
	\item Criticism is hinted at, but never made explicit.
	\item The papers also hinted that the problem was really his private life.
	\item The President hinted he might make some changes in the government.
	\end{itemize}
}
\item countable noun \\
A \textbf{hint} is a helpful piece of advice, usually about how to do something.
 \textit{
	\begin{itemize}
	\item Here are some helpful hints to make your journey easier.
	\item I'm hoping to get some fashion hints.
	\end{itemize}
}
\item singular noun \\
A \textbf{hint}  \textbf{of} something is a very small amount of it.
 \textit{
	\begin{itemize}
	\item She added only a hint of vermouth to the gin.
	\item I glanced at her and saw no hint of irony on her face.
	\end{itemize}
}
\end{enumerate}

\section*{gossip}
{\large \color{blue}  gossips  gossiping  gossiped  }
\subsection*{Explain}
\begin{enumerate}
\item variable noun \\
\textbf{Gossip} is informal conversation, often about other people's private  affairs .
 \textit{
	\begin{itemize}
	\item He spent the first hour talking gossip.
	\item There has been much gossip about the possible reasons for his absence.
	\item Don't you like a good gossip?
	\end{itemize}
}
\item verb \\
If you \textbf{gossip with} someone, you talk informally, especially about other people or local  events . You can also  say that two people \textbf{gossip} .
 \textit{
	\begin{itemize}
	\item We spoke, debated, gossiped into the night.
	\item Eva gossiped with Sarah.
	\item Mrs Lilywhite never gossiped.
	\end{itemize}
}
\item countable noun \\
If you describe someone as a \textbf{gossip} , you mean that they enjoy talking informally to people about the private affairs of others.
 \textit{
	\begin{itemize}
	\item He was a vicious gossip.
	\end{itemize}
}
\end{enumerate}

\section*{keyboard}
{\large \color{blue}  keyboards  }
\subsection*{Explain}
\begin{enumerate}
\item countable noun \\
The \textbf{keyboard} a computer, tablet , or on some phones is the set of keys that you press in order to operate it.
 \textit{
	\begin{itemize}
	\end{itemize}
}
\item countable noun \\
The \textbf{keyboard} of a piano or organ is the set of black and white keys that you press in order to play it.
 \textit{
	\begin{itemize}
	\item Tanya's hands rippled over the keyboard.
	\end{itemize}
}
\item countable noun \\
People sometimes  refer to musical instruments that have a keyboard as \textbf{keyboards} .
 \textit{
	\begin{itemize}
	\item ...Sean O'Hagan on keyboards.
	\item ...the keyboard player.
	\end{itemize}
}
\end{enumerate}

\section*{harm}
{\large \color{blue}  harms  harming  harmed  }
\subsection*{Explain}
\begin{enumerate}
\item verb \\
To \textbf{harm} a person or animal means to cause them physical injury, usually on purpose .
 \textit{
	\begin{itemize}
	\item The hijackers seemed anxious not to harm anyone.
	\end{itemize}
}
\item uncountable noun \\
\textbf{Harm} is physical injury to a person or an animal which is usually caused on purpose.
 \textit{
	\begin{itemize}
	\item All dogs are capable of doing harm to human beings.
	\end{itemize}
}
\item verb \\
To \textbf{harm} a thing, or sometimes a person, means to damage them or make them less effective or successful than they were.
 \textit{
	\begin{itemize}
	\item ...a warning that the product may harm the environment.
	\item Low-priced imports will harm the industry.
	\end{itemize}
}
\item uncountable noun \\
\textbf{Harm} is the damage to something which is caused by a particular course of action.
 \textit{
	\begin{itemize}
	\item The abuse of your powers does harm to all other officers who do their job properly.
	\item To cut taxes would probably do the economy more harm than good.
	\end{itemize}
}
\item  \\
 sb will come to no harm/no harm will come to sb \textit{
	\begin{itemize}
	\end{itemize}
}
\item  \\
 do no/little harm; no harm in doing \textit{
	\begin{itemize}
	\end{itemize}
}
\item  \\
 do no harm/do sb no harm \textit{
	\begin{itemize}
	\end{itemize}
}
\item  \\
 no harm done \textit{
	\begin{itemize}
	\end{itemize}
}
\item  \\
 in harm's way \textit{
	\begin{itemize}
	\end{itemize}
}
\item  \\
 out of harm's way \textit{
	\begin{itemize}
	\end{itemize}
}
\end{enumerate}

\section*{leg}
{\large \color{blue}  legs  legging  legged  }
\subsection*{Explain}
\begin{enumerate}
\item countable noun \\
A person or animal's \textbf{legs} are the long parts of their body that they use to stand on.
 \textit{
	\begin{itemize}
	\item He was tapping his walking stick against his leg.
	\end{itemize}
}
\item countable noun \\
The \textbf{legs} of a pair of trousers are the parts that cover your legs.
 \textit{
	\begin{itemize}
	\item He moved on through wet grass that soaked his trouser legs.
	\end{itemize}
}
\item countable noun \\
A \textbf{leg} of lamb , pork , chicken , or other meat is a piece of meat that consists of the animal's or bird's leg, especially the thigh.
 \textit{
	\begin{itemize}
	\item ...a chicken leg.
	\item ...a leg of mutton.
	\end{itemize}
}
\item countable noun \\
The \textbf{legs} of a table , chair, or other piece of furniture are the parts that rest on the floor and support the furniture's weight .
 \textit{
	\begin{itemize}
	\item His ankles were tied to the legs of the chair.
	\item The teak table has fluted legs.
	\end{itemize}
}
\item countable noun \\
A \textbf{leg} of a long journey is one part of it, usually between two points where you stop .
 \textit{
	\begin{itemize}
	\item The first leg of the journey was by boat to Lake Naivasha in Kenya.
	\end{itemize}
}
\item countable noun \\
A \textbf{leg} of a sports  competition is one of a series of games that are played to find an overall  winner .
 \textit{
	\begin{itemize}
	\item The first round of the cup was decided over two legs.
	\item The Arsenal centre back scored in both legs of the Champions League qualifier .
	\end{itemize}
}
\item  \\
 leg it \textit{
	\begin{itemize}
	\end{itemize}
}
\item  \\
 on one's last legs \textit{
	\begin{itemize}
	\end{itemize}
}
\item  \\
 to pull someone's leg \textit{
	\begin{itemize}
	\end{itemize}
}
\item  \\
 have a leg to stand on \textit{
	\begin{itemize}
	\end{itemize}
}
\item  \\
 have legs \textit{
	\begin{itemize}
	\end{itemize}
}
\end{enumerate}

\section*{huddle}
{\large \color{blue}  huddles  huddling  huddled  }
\subsection*{Explain}
\begin{enumerate}
\item verb \\
If you \textbf{huddle}  somewhere , you sit , stand , or lie there holding your arms and legs  close to your body, usually because you are cold or frightened .
 \textit{
	\begin{itemize}
	\item She huddled inside the porch as she rang the bell.
	\item Myrtle sat huddled on the side of the bed, weeping.
	\end{itemize}
}
\item verb \\
If people \textbf{huddle}  \textbf{together} or \textbf{huddle}  \textbf{round} something, they stand, sit, or lie close to each other, usually because they all
 feel cold or frightened.
 \textit{
	\begin{itemize}
	\item Tired and lost, we huddled together.
	\item ...strangers huddling together for warmth.
	\item Hundreds of people huddled around a single radio listening to the announcement.
	\item The survivors spent the night huddled around bonfires.
	\end{itemize}
}
\item verb \\
If people \textbf{huddle} in a group, they gather together to discuss something quietly or secretly.
 \textit{
	\begin{itemize}
	\item The only people in the store were three young employees, huddled in the corner chatting.
	\item The president has been huddling with his most senior aides.
	\item He was huddled with his advisers in Ottawa yesterday to review strategy.
	\end{itemize}
}
\item countable noun \\
A \textbf{huddle} is a small group of people or things that are standing very close together or lying on top of each other, usually in a disorganized way.
 \textit{
	\begin{itemize}
	\item We lay there: a huddle of bodies, gasping for air.
	\item Les kept seeing Eric and Tam in a huddle and he knew they were talking about him.
	\item ...the huddle of dark houses on the other side of the reservoir.
	\end{itemize}
}
\end{enumerate}

\section*{mistake}
{\large \color{blue}  mistakes  mistaking  mistook  mistaken  }
\subsection*{Explain}
\begin{enumerate}
\item countable noun \\
If you make a \textbf{mistake} , you do something which you did not intend to do, or which produces a result that you do not want .
 \textit{
	\begin{itemize}
	\item She made the mistake of going against her doctor's advice.
	\item I think it's a serious mistake to confuse books with life.
	\item Jonathan says it was his mistake.
	\item There must be some mistake.
	\item He has been arrested by mistake.
	\end{itemize}
}
\item countable noun \\
A \textbf{mistake} is something or part of something which is incorrect or not right.
 \textit{
	\begin{itemize}
	\item Her mother sighed and rubbed out another mistake in the crossword puzzle.
	\item ...spelling mistakes.
	\end{itemize}
}
\item verb \\
If you \textbf{mistake} one person or thing \textbf{for} another, you wrongly think that they are the other person or thing.
 \textit{
	\begin{itemize}
	\item I mistook you for Carlos.
	\item When hay fever first occurs it is often mistaken for a summer cold.
	\end{itemize}
}
\item verb \\
If you \textbf{mistake} something, you fail to recognize or understand it.
 \textit{
	\begin{itemize}
	\item The government completely mistook the feeling of the country.
	\item No one should mistake how serious the issue is.
	\end{itemize}
}
\item  \\
 there's no mistaking \textit{
	\begin{itemize}
	\end{itemize}
}
\end{enumerate}

\section*{hunger}
{\large \color{blue}  hungers  hungering  hungered  }
\subsection*{Explain}
\begin{enumerate}
\item uncountable noun \\
\textbf{Hunger} is the feeling of weakness or discomfort that you get when you need something to eat .
 \textit{
	\begin{itemize}
	\item Hunger is the body's signal that levels of blood sugar are too low.
	\item Protein helps to fill you up and curb hunger pangs.
	\end{itemize}
}
\item uncountable noun \\
\textbf{Hunger} is a severe lack of food which causes suffering or death .
 \textit{
	\begin{itemize}
	\item Three hundred people in this town are dying of hunger every day.
	\end{itemize}
}
\item singular noun \\
If you have a \textbf{hunger}  \textbf{for} something, you want or need it very much.
 \textit{
	\begin{itemize}
	\item Geffen has a hunger for success that seems bottomless.
	\item ...his hunger to equal Vardon's record of six wins.
	\end{itemize}
}
\item verb \\
If you say that someone \textbf{hungers}  \textbf{for} something or \textbf{hungers}  \textbf{after} it, you are emphasizing that they want it very much.
 \textit{
	\begin{itemize}
	\item But Jules was not eager for classroom learning, he hungered for adventure.
	\end{itemize}
}
\end{enumerate}

\section*{news}
{\large \color{blue}  }
\subsection*{Explain}
\begin{enumerate}
\item uncountable noun \\
\textbf{News} is information about a recently changed  situation or a recent event.
 \textit{
	\begin{itemize}
	\item We waited and waited for news of him.
	\item They still haven't had any news about when they'll be able to go home.
	\item I wish I had better news for you.
	\item He's thrilled to bits at the news.
	\end{itemize}
}
\item uncountable noun \\
\textbf{News} is information that is published in newspapers and broadcast on radio and television about recent events in the country or world or in a particular area of activity .
 \textit{
	\begin{itemize}
	\item Foreign News is on Page 16.
	\item We'll also have the latest sports news.
	\item The announcement was made at a news conference.
	\item Those are some of the top stories in the news.
	\end{itemize}
}
\item singular noun \\
\textbf{The news} is a television or radio broadcast which consists of information about recent events
in the country or the world.
 \textit{
	\begin{itemize}
	\item I heard all about the bombs on the news.
	\item ...the six o'clock news.
	\end{itemize}
}
\item noun, in names \\
\textbf{News} is sometimes used in the names of newspapers.
 \textit{
	\begin{itemize}
	\item ...the New York Daily News.
	\end{itemize}
}
\item uncountable noun \\
If you say that someone or something is \textbf{news} , you mean that they are considered to be interesting and important at the moment , and that people want to hear about them on the radio and television and in newspapers.
 \textit{
	\begin{itemize}
	\item A murder was big news.
	\item If you are a celebrity, you are headline news.
	\end{itemize}
}
\item  \\
 bad news/good news \textit{
	\begin{itemize}
	\end{itemize}
}
\item  \\
 be news to sb \textit{
	\begin{itemize}
	\end{itemize}
}
\end{enumerate}

\section*{kiss}
{\large \color{blue}  kisses  kissing  kissed  }
\subsection*{Explain}
\begin{enumerate}
\item verb \\
If you \textbf{kiss} someone, you touch them with your lips to show  affection or sexual desire, or to greet them or say goodbye.
 \textbf{Kiss} is also a noun .
 \textit{
	\begin{itemize}
	\item She leaned up and kissed him on the cheek.
	\item Her parents kissed her goodbye as she set off from their home.
	\item They kissed for almost half a minute.
	\item We kissed goodbye.
	\item I put my arms around her and gave her a kiss.
	\item We gave each other hugs and kisses every morning.
	\end{itemize}
}
\item verb \\
If you \textbf{kiss} something, you touch it lightly with your lips, usually as a sign of respect.
 \textit{
	\begin{itemize}
	\item The men stepped forward to kiss the hand of their mentor.
	\item She bowed her head and kissed the Archbishop's ring.
	\end{itemize}
}
\item verb \\
If you say that something \textbf{kisses} another thing, you mean that it touches that thing very gently.
 \textit{
	\begin{itemize}
	\item The wheels of the aircraft kissed the runway.
	\end{itemize}
}
\item  \\
 to blow a kiss \textit{
	\begin{itemize}
	\end{itemize}
}
\item  \\
 to kiss something goodbye \textit{
	\begin{itemize}
	\end{itemize}
}
\end{enumerate}

\section*{novelty}
{\large \color{blue}  novelties  }
\subsection*{Explain}
\begin{enumerate}
\item uncountable noun \\
\textbf{Novelty} is the quality of being different , new, and unusual.
 \textit{
	\begin{itemize}
	\item After a few hours, however, the novelty wore off.
	\end{itemize}
}
\item countable noun \\
A \textbf{novelty} is something that is new and therefore interesting.
 \textit{
	\begin{itemize}
	\item Seeing people queuing for food was a novelty.
	\item It came from the days when a motor car was a novelty.
	\end{itemize}
}
\item countable noun \\
\textbf{Novelties} are cheap toys, ornaments, or other objects that are sold as presents or souvenirs .
 \textit{
	\begin{itemize}
	\item At Easter, we give them plastic eggs filled with small toys, novelties and coins.
	\end{itemize}
}
\end{enumerate}

\section*{male}
{\large \color{blue}  males  }
\subsection*{Explain}
\begin{enumerate}
\item adjective \\
Someone who is \textbf{male} is a man or a boy.
 \textit{
	\begin{itemize}
	\item It was the first new prison for adult male prisoners to be opened in Scotland for
60 years.
	\item The London City Ballet has engaged two male dancers from the Bolshoi.
	\item Most of the demonstrators were white and male.
	\end{itemize}
}
\item countable noun \\
Men and boys are sometimes  referred to as \textbf{males} when they are being considered as a type.
 \textit{
	\begin{itemize}
	\item ...the remains of a Caucasian male, aged 65-70.
	\item A high proportion of crime is perpetrated by young males in their teens and twenties.
	\end{itemize}
}
\item adjective \\
\textbf{Male}  means relating, belonging, or affecting men rather than women.
 \textit{
	\begin{itemize}
	\item Women entered the workforce in greater numbers and male unemployment rose.
	\item ...a deep male voice.
	\end{itemize}
}
\item countable noun \\
You can refer to any creature that belongs to the sex that cannot lay eggs or have babies as a \textbf{male} .
 \textbf{Male} is also an adjective .
 \textit{
	\begin{itemize}
	\item Males and females take turns brooding the eggs.
	\item After mating the male wasps tunnel through the sides of their nursery.
	\end{itemize}
}
\item adjective \\
A \textbf{male} flower or plant fertilizes the part that will become the fruit .
 \textit{
	\begin{itemize}
	\end{itemize}
}
\end{enumerate}

\section*{rat}
{\large \color{blue}  rats  ratting  ratted  }
\subsection*{Explain}
\begin{enumerate}
\item countable noun \\
A \textbf{rat} is an animal which has a long tail and looks like a large mouse.
 \textit{
	\begin{itemize}
	\item This was demonstrated in a laboratory experiment with rats.
	\item ...a rat-infested derelict building.
	\end{itemize}
}
\item countable noun \\
If you call someone a \textbf{rat} , you mean that you are angry with them or dislike them, often because they have cheated you or betrayed you.
 \textit{
	\begin{itemize}
	\item What did you do with the gun you took from that little rat Turner?
	\end{itemize}
}
\item verb \\
If someone \textbf{rats on} you, they tell someone in authority about things that you have done, especially  bad things.
 \textit{
	\begin{itemize}
	\item They were accused of encouraging children to rat on their parents.
	\end{itemize}
}
\item verb \\
If someone \textbf{rats on} an agreement , they do not do what they said they would do.
 \textit{
	\begin{itemize}
	\item She claims he ratted on their deal.
	\end{itemize}
}
\item  \\
 to smell a rat \textit{
	\begin{itemize}
	\end{itemize}
}
\end{enumerate}

\section*{mortgage}
{\large \color{blue}  mortgages  mortgaging  mortgaged  }
\subsection*{Explain}
\begin{enumerate}
\item countable noun \\
A \textbf{mortgage} is a loan of money which you get from a bank or building  society in order to buy a house.
 \textit{
	\begin{itemize}
	\item ...an increase in mortgage rates.
	\end{itemize}
}
\item verb \\
If you \textbf{mortgage} your house or land , you use it as a guarantee to a company in order to borrow money from them.
 \textit{
	\begin{itemize}
	\item They had to mortgage their home to pay the bills.
	\item ...mortgaged homes.
	\end{itemize}
}
\end{enumerate}

\section*{road}
{\large \color{blue}  roads  }
\subsection*{Explain}
\begin{enumerate}
\item countable noun \\
A \textbf{road} is a long piece of hard ground which is built between two places so that people can drive or ride  easily from one place to the other.
 \textit{
	\begin{itemize}
	\item There was very little traffic on the roads.
	\item We just go straight up the Bristol Road.
	\item He was coming down the road the same time as the girl was turning into the lane.
	\item Buses carry 30 per cent of those travelling by road.
	\item ...road accidents.
	\end{itemize}
}
\item countable noun \\
The \textbf{road}  \textbf{to} a particular result is the means of achieving it or the process of achieving it.
 \textit{
	\begin{itemize}
	\item We are bound to see some ups and downs along the road to recovery.
	\end{itemize}
}
\item  \\
 to hit the road \textit{
	\begin{itemize}
	\end{itemize}
}
\item  \\
 on the road \textit{
	\begin{itemize}
	\end{itemize}
}
\item  \\
 on the road \textit{
	\begin{itemize}
	\end{itemize}
}
\end{enumerate}

\section*{panic}
{\large \color{blue}  panics  panicking  panicked  }
\subsection*{Explain}
\begin{enumerate}
\item variable noun \\
\textbf{Panic} is a very strong feeling of anxiety or fear , which makes you act without thinking carefully.
 \textit{
	\begin{itemize}
	\item An earthquake hit the capital, causing panic among the population.
	\item I phoned the doctor in a panic, worried about the pain in my chest.
	\end{itemize}
}
\item uncountable noun \\
\textbf{Panic} or \textbf{a}  \textbf{panic} is a situation in which people are affected by a strong feeling of anxiety.
 \textit{
	\begin{itemize}
	\item There was a moment of panic in Britain as it became clear just how vulnerable the
nation was.
	\item I'm in a panic about getting everything done in time.
	\item The policy announcement caused panic buying of petrol.
	\end{itemize}
}
\item verb \\
If you \textbf{panic} or if someone \textbf{panics} you, you suddenly feel anxious or afraid , and act quickly and without thinking carefully.
 \textit{
	\begin{itemize}
	\item Guests panicked and screamed when the bomb exploded.
	\item The unexpected and sudden memory briefly panicked her.
	\item He will not be panicked into a hasty decision.
	\end{itemize}
}
\end{enumerate}

\section*{route}
{\large \color{blue}  routes  routing  routed  }
\subsection*{Explain}
\begin{enumerate}
\item countable noun \\
A \textbf{route} is a way from one place to another.
 \textit{
	\begin{itemize}
	\item ...the most direct route to the town centre.
	\item All escape routes were blocked by armed police.
	\item Tens of thousands lined the route from Dublin airport.
	\end{itemize}
}
\item countable noun \\
A bus , air, or shipping \textbf{route} is the way between two places along which buses, planes , or ships travel regularly.
 \textit{
	\begin{itemize}
	\item ...the main shipping routes to Japan.
	\end{itemize}
}
\item countable noun \\
In the United  States , \textbf{Route} is used in front of a number in the names of main roads between major cities.
 \textit{
	\begin{itemize}
	\item ...the Broadway-Webster exit on Route 580.
	\end{itemize}
}
\item countable noun \\
Your \textbf{route} is the series of visits you make to different people or places, as part of your job .
 \textit{
	\begin{itemize}
	\item He began cracking open big blue tins of butter cookies and feeding the dogs on his
route.
	\item They would go out on his route and check him.
	\end{itemize}
}
\item countable noun \\
You can refer to a way of achieving something as a \textbf{route} .
 \textit{
	\begin{itemize}
	\item Researchers are trying to get at the same information through an indirect route.
	\item Buying the best is as sure a route to success in investment as in any other field.
	\end{itemize}
}
\item verb \\
If vehicles, goods, or passengers  \textbf{are routed} in a particular direction, they are made to travel in that direction.
 \textit{
	\begin{itemize}
	\item Double-stack trains are taking a lot of freight that used to be routed via trucks.
	\item Approaching cars will be routed into two lanes.
	\end{itemize}
}
\item verb \\
If telephone  calls or other electronic  signals  \textbf{are routed} in a particular way, the signals are sent through a particular series of connections .
 \textit{
	\begin{itemize}
	\item ...plans to route every emergency call in the country through just three telephone
exchanges.
	\end{itemize}
}
\item  \\
 en route \textit{
	\begin{itemize}
	\end{itemize}
}
\item  \\
 en route \textit{
	\begin{itemize}
	\end{itemize}
}
\item  \\
 go the route \textit{
	\begin{itemize}
	\end{itemize}
}
\end{enumerate}

\section*{prick}
{\large \color{blue}  pricks  pricking  pricked  }
\subsection*{Explain}
\begin{enumerate}
\item verb \\
If you \textbf{prick} something or \textbf{prick} holes in it, you make small holes in it with a sharp object such as a pin .
 \textit{
	\begin{itemize}
	\item Prick the potatoes and rub the skins with salt.
	\item He pricks holes in the foil with a pin.
	\end{itemize}
}
\item verb \\
If something sharp \textbf{pricks} you or if you \textbf{prick}  \textbf{yourself with} something sharp, it sticks into you or presses your skin and causes you pain.
 \textit{
	\begin{itemize}
	\item She had just pricked her finger with the needle.
	\end{itemize}
}
\item verb \\
If tears  \textbf{prick} your eyes , you feel as if you are about to cry .
 \textit{
	\begin{itemize}
	\item Davydd felt tears prick his eyes.
	\end{itemize}
}
\item verb \\
If something \textbf{pricks} your \textbf{conscience} , you suddenly feel guilty about it. If you \textbf{are pricked}  \textbf{by} an emotion , you suddenly experience that emotion.
 \textit{
	\begin{itemize}
	\item Most were sympathetic once we pricked their consciences.
	\item I was pricked by the needle of curiosity.
	\end{itemize}
}
\item countable noun \\
A \textbf{prick} is a small, sharp pain that you get when something pricks you.
 \textit{
	\begin{itemize}
	\item At the same time she felt a prick on her neck.
	\end{itemize}
}
\item countable noun \\
If someone calls a man a \textbf{prick} , they are indicating that they do not like him and that they think he is stupid .
 \textit{
	\begin{itemize}
	\end{itemize}
}
\item countable noun \\
A man's \textbf{prick} is his penis .
 \textit{
	\begin{itemize}
	\end{itemize}
}
\end{enumerate}

\section*{sauce}
{\large \color{blue}  sauces  }
\subsection*{Explain}
\begin{enumerate}
\item variable noun \\
A \textbf{sauce} is a thick liquid which is served with other food.
 \textit{
	\begin{itemize}
	\item ...pasta cooked in a sauce of garlic, tomatoes, and cheese.
	\item ...vanilla ice cream with chocolate sauce.
	\end{itemize}
}
\end{enumerate}

\section*{report}
{\large \color{blue}  reports  reporting  reported  }
\subsection*{Explain}
\begin{enumerate}
\item verb \\
If you \textbf{report} something that has happened , you tell people about it.
 \textit{
	\begin{itemize}
	\item They had been called in to clear drains after local people reported a foul smell.
	\item I reported the theft to the police.
	\item The RSPCA reported that 10,610 cats were abandoned last year.
	\item 'He seems to be all right now,' reported a relieved Taylor.
	\item The foreign secretary is reported as saying that force will have to be used if diplomacy
fails.
	\item She reported him missing the next day.
	\item Between forty and fifty people are reported to have died in the fighting.
	\end{itemize}
}
\item verb \\
If you \textbf{report}  \textbf{on} an event or subject, you tell people about it, because it is your job or duty to do so.
 \textit{
	\begin{itemize}
	\item Many journalists enter the country to report on political affairs.
	\item I'll report to you later.
	\end{itemize}
}
\item countable noun \\
A \textbf{report} is a news article or broadcast which gives information about something that has just happened.
 \textit{
	\begin{itemize}
	\item ...a report in London's Independent newspaper.
	\item With a report on these developments, here's Jim Fish in Belgrade.
	\item Press reports said that 65mm of water fell in twenty four hours.
	\end{itemize}
}
\item countable noun \\
A \textbf{report} is an official document which a group of people issue after investigating a situation or event.
 \textit{
	\begin{itemize}
	\item After an inspection, the inspectors must publish a report.
	\item A report has found that only 22 per cent of lecturers in our universities are women.
	\end{itemize}
}
\item countable noun \\
If you give someone a \textbf{report} on something, you tell them what has been happening .
 \textit{
	\begin{itemize}
	\item She came back to give us a progress report on how the project is going.
	\item It seemed obvious that you were trying to focus suspicion on Mr Hirsch.
	\end{itemize}
}
\item countable noun \\
If you say that there are \textbf{reports} that something has happened, you mean that some people say it has happened but you
have no direct evidence of it.
 \textit{
	\begin{itemize}
	\item There were unconfirmed reports of several arrests.
	\item There were no reports of casualties.
	\end{itemize}
}
\item verb \\
If someone \textbf{reports} you \textbf{to} a person in authority, they tell that person about something wrong that you have done.
 \textit{
	\begin{itemize}
	\item His daughter reported him to police a few days later.
	\item She was reported for speeding twice on the same road within a week.
	\end{itemize}
}
\item verb \\
If you \textbf{report}  \textbf{to} a person or place, you go to that person or place and say that you are ready to start work or say that you are present.
 \textit{
	\begin{itemize}
	\item According to protocol, he first reported to the Director of the hospital.
	\item He has to surrender his passport and report to the police every five days.
	\item None of the men had reported for duty.
	\end{itemize}
}
\item verb \\
If you say that one employee  \textbf{reports}  \textbf{to} another, you mean that the first employee is told what to do by the second one and
is responsible to them.
 \textit{
	\begin{itemize}
	\item He reported to a section chief, who reported to a division chief, and so on up the
line.
	\end{itemize}
}
\item countable noun \\
A school \textbf{report} is an official written account of how well or how badly a pupil has done during the term or year that has just finished .
 \textit{
	\begin{itemize}
	\item And now she was getting bad school reports.
	\end{itemize}
}
\item countable noun \\
A \textbf{report} is a sudden loud noise, for example the sound of a gun being fired or an explosion .
 \textit{
	\begin{itemize}
	\item Soon afterwards there was a loud report as the fuel tanks exploded.
	\end{itemize}
}
\end{enumerate}

\section*{saw}
{\large \color{blue}  saws  sawing  sawed  sawn  }
\subsection*{Explain}
\begin{enumerate}
\item  \\
\textbf{Saw} is the past  tense of see .
 \textit{
	\begin{itemize}
	\end{itemize}
}
\item countable noun \\
A \textbf{saw} is a tool for cutting wood, which has a blade with sharp teeth along one edge. Some saws are pushed  backwards and forwards by hand, and others are powered by electricity .
 \textit{
	\begin{itemize}
	\end{itemize}
}
\item verb \\
If you \textbf{saw} something, you cut it with a saw.
 \textit{
	\begin{itemize}
	\item He escaped by sawing through the bars of his cell.
	\item Your father is sawing wood.
	\end{itemize}
}
\end{enumerate}

\section*{retort}
{\large \color{blue}  retorts  retorting  retorted  }
\subsection*{Explain}
\begin{enumerate}
\item verb \\
To \textbf{retort} means to reply angrily to someone.
 \textbf{Retort} is also a noun .
 \textit{
	\begin{itemize}
	\item Was he afraid, he was asked. 'Afraid of what?' he retorted.
	\item Others retort that strong central power is a dangerous thing in Russia.
	\item His sharp retort clearly made an impact.
	\end{itemize}
}
\end{enumerate}

\section*{settlement}
{\large \color{blue}  settlements  }
\subsection*{Explain}
\begin{enumerate}
\item countable noun \\
A \textbf{settlement} is an official agreement between two sides who were involved in a conflict or argument .
 \textit{
	\begin{itemize}
	\item Our objective must be to secure a peace settlement.
	\item They are not optimistic about a settlement of the eleven-year conflict.
	\end{itemize}
}
\item countable noun \\
A \textbf{settlement} is an agreement to end a disagreement or dispute without going to a court of law , for example by offering someone money.
 \textit{
	\begin{itemize}
	\item She accepted an out-of-court settlement of £4,000.
	\item ...a libel settlement.
	\end{itemize}
}
\item uncountable noun \\
The \textbf{settlement}  \textbf{of} a debt is the act of paying  back money that you owe .
 \textit{
	\begin{itemize}
	\item ...ways to delay the settlement of debts.
	\end{itemize}
}
\item countable noun \\
A \textbf{settlement} is a place where people have come to live and have built  homes .
 \textit{
	\begin{itemize}
	\item The village is a settlement of just fifty houses.
	\item ...a Muslim settlement.
	\end{itemize}
}
\item uncountable noun \\
The \textbf{settlement of} a group of people is the process in which they settle in a place where people from
their country have never lived before.
 \textit{
	\begin{itemize}
	\end{itemize}
}
\end{enumerate}

\section*{roar}
{\large \color{blue}  roars  roaring  roared  }
\subsection*{Explain}
\begin{enumerate}
\item verb \\
If something, usually a vehicle , \textbf{roars}  somewhere , it goes there very fast , making a loud noise.
 \textit{
	\begin{itemize}
	\item A police car roared past.
	\item The plane roared down the runway for takeoff.
	\item Flames roared hundreds of feet into the air.
	\end{itemize}
}
\item verb \\
If something \textbf{roars} , it makes a very loud noise.
 \textbf{Roar} is also a noun .
 \textit{
	\begin{itemize}
	\item The engine roared, and the vehicle leapt forward.
	\item Her heart was pounding and the blood roared in her ears.
	\item ...the roaring waters of Niagara Falls.
	\item ...the roar of traffic.
	\item Local residents saw it plunge towards Earth with a deafening roar.
	\end{itemize}
}
\item verb \\
If someone \textbf{roars}  \textbf{with} laughter, they laugh in a very noisy way.
 \textbf{Roar} is also a noun.
 \textit{
	\begin{itemize}
	\item Max threw back his head and roared with laughter.
	\item There were roars of laughter as he stood up.
	\end{itemize}
}
\item verb \\
If someone \textbf{roars} , they shout something in a very loud voice .
 \textbf{Roar} is also a noun.
 \textit{
	\begin{itemize}
	\item 'I'll kill you for that,' he roared.
	\item During the playing of the national anthem the crowd roared and whistled.
	\item The audience roared its approval.
	\item There was a roar of approval.
	\end{itemize}
}
\item verb \\
When a lion  \textbf{roars} , it makes the loud sound that lions typically make.
 \textbf{Roar} is also a noun.
 \textit{
	\begin{itemize}
	\item The lion roared once, and sprang.
	\item ...the roar of lions in the distance.
	\end{itemize}
}
\end{enumerate}

\section*{signature}
{\large \color{blue}  signatures  }
\subsection*{Explain}
\begin{enumerate}
\item countable noun \\
Your \textbf{signature} is your name, written in your own characteristic way, often at the end of a document to indicate that you wrote the document or that you agree with what it says .
 \textit{
	\begin{itemize}
	\item I was writing my signature at the bottom of the page.
	\item ...a petition containing 170 signatures.
	\end{itemize}
}
\item adjective \\
A \textbf{signature}  item is typical of or associated with a particular person.
 \textit{
	\begin{itemize}
	\item Rabbit stew is one of chef Giancarlo Moeri's signature dishes.
	\item The dress reflects our signature style of understated elegance with individuality.
	\end{itemize}
}
\item  \\
 put one's signature to sth \textit{
	\begin{itemize}
	\end{itemize}
}
\end{enumerate}

\section*{sanction}
{\large \color{blue}  sanctions  sanctioning  sanctioned  }
\subsection*{Explain}
\begin{enumerate}
\item verb \\
If someone in authority \textbf{sanctions} an action or practice , they officially  approve of it and allow it to be done.
 \textbf{Sanction} is also a noun .
 \textit{
	\begin{itemize}
	\item He may now be ready to sanction the use of force.
	\item He seemed to be preparing to sanction an increase in public borrowing.
	\item The king could not enact laws without the sanction of Parliament.
	\end{itemize}
}
\item plural noun \\
\textbf{Sanctions} are measures taken by countries to restrict  trade and official  contact with a country that has broken international law.
 \textit{
	\begin{itemize}
	\item The continued abuse of human rights has now led the United States to impose sanctions
against the regime.
	\item He expressed his opposition to the lifting of sanctions.
	\end{itemize}
}
\item countable noun \\
A \textbf{sanction} is a severe  course of action which is intended to make people obey  instructions , customs , or laws.
 \textit{
	\begin{itemize}
	\item As an ultimate sanction, they can sell their shares.
	\end{itemize}
}
\item verb \\
If a country or an authority \textbf{sanctions} another country or a person for doing something, it declares that the country or person is guilty of doing it and imposes sanctions on them.
 \textit{
	\begin{itemize}
	\item ...their failure to sanction the country for butchering whales in violation of international
conservation treaties.
	\end{itemize}
}
\end{enumerate}

\section*{simplicity}
{\large \color{blue}  }
\subsection*{Explain}
\begin{enumerate}
\item uncountable noun \\
The \textbf{simplicity} of something is the fact that it is not complicated and can be understood or done easily .
 \textit{
	\begin{itemize}
	\item The apparent simplicity of his plot is deceptive.
	\item Because of its simplicity, this test could be carried out easily by a family doctor.
	\end{itemize}
}
\item uncountable noun \\
When you talk about something's \textbf{simplicity} , you approve of it because it has no unnecessary parts or complicated details .
 \textit{
	\begin{itemize}
	\item ...fussy details that ruin the simplicity of the design.
	\item A pair of jewelled earrings will liven up this dress without detracting from its
simplicity.
	\end{itemize}
}
\item  \\
 simplicity itself \textit{
	\begin{itemize}
	\end{itemize}
}
\end{enumerate}

\section*{sham}
{\large \color{blue}  shams  }
\subsection*{Explain}
\begin{enumerate}
\item countable noun \\
Something that is a \textbf{sham} is not real or is not really what it seems to be.
 \textit{
	\begin{itemize}
	\item The government's promises were exposed as a hollow sham.
	\item Many of the world's leaders have already denounced this election as a sham.
	\item ...sham marriages.
	\end{itemize}
}
\end{enumerate}

\section*{sin}
{\large \color{blue}  sins  sinning  sinned  }
\subsection*{Explain}
\begin{enumerate}
\item variable noun \\
\textbf{Sin} or a \textbf{sin} is an action or type of behaviour which is believed to break the laws of God.
 \textit{
	\begin{itemize}
	\item The Vatican's teaching on abortion is clear: it is a sin.
	\item Was it the sin of pride to have believed too much in themselves?
	\end{itemize}
}
\item verb \\
If you \textbf{sin} , you do something that is believed to break the laws of God.
 \textit{
	\begin{itemize}
	\item The Spanish Inquisition charged him with sinning against God and man.
	\item You have sinned and must repent your ways.
	\end{itemize}
}
\item countable noun \\
A \textbf{sin} is any action or behaviour that people disapprove of or consider morally wrong .
 \textit{
	\begin{itemize}
	\item ...the sin of arrogant hard-heartedness.
	\item The ultimate sin was not infidelity, but public mention which led to scandal.
	\end{itemize}
}
\item  \\
 to live in sin \textit{
	\begin{itemize}
	\end{itemize}
}
\end{enumerate}

\section*{slap}
{\large \color{blue}  slaps  slapping  slapped  }
\subsection*{Explain}
\begin{enumerate}
\item verb \\
If you \textbf{slap} someone, you hit them with the palm of your hand.
 \textbf{Slap} is also a noun .
 \textit{
	\begin{itemize}
	\item He would push or slap her once in a while.
	\item I slapped him hard across the face.
	\item He reached forward and gave him a slap.
	\end{itemize}
}
\item verb \\
If you \textbf{slap} someone \textbf{on} the back , you hit them in a friendly manner on their back.
 \textit{
	\begin{itemize}
	\item A large middle-aged lady slapped me on the back and said 'Nice to see you again.'
	\end{itemize}
}
\item verb \\
If you \textbf{slap} something \textbf{onto} a surface, you put it there quickly, roughly , or carelessly.
 \textit{
	\begin{itemize}
	\item The barman slapped the cup on to the waiting saucer.
	\end{itemize}
}
\item verb \\
If journalists  say that the authorities  \textbf{slap} something such as a tax or a ban  \textbf{on} something, they think it is unreasonable or put on without careful  thought .
 \textit{
	\begin{itemize}
	\item The government slapped a ban on the export of unprocessed logs.
	\item Thankfully the Government still hasn't discovered a way of slapping a tax on love,
sunshine or air.
	\end{itemize}
}
\item  \\
 a slap in the face \textit{
	\begin{itemize}
	\end{itemize}
}
\item  \\
 a slap on the wrist \textit{
	\begin{itemize}
	\end{itemize}
}
\end{enumerate}

\section*{smile}
{\large \color{blue}  smiles  smiling  smiled  }
\subsection*{Explain}
\begin{enumerate}
\item verb \\
When you \textbf{smile} , the corners of your mouth curve up and you sometimes show your teeth . People smile when they are pleased or amused, or when they are being friendly .
 \textit{
	\begin{itemize}
	\item When he saw me, he smiled and waved.
	\item He rubbed the back of his neck and smiled ruefully at me.
	\item His smiling face appears on T-shirts, billboards, and posters.
	\end{itemize}
}
\item countable noun \\
A \textbf{smile} is the expression that you have on your face when you smile.
 \textit{
	\begin{itemize}
	\item She gave a wry smile.
	\item 'There are some sandwiches if you're hungry,' she said with a smile.
	\item She had a big smile on her face.
	\end{itemize}
}
\item verb \\
If you \textbf{smile} something, you say it with a smile or express it by a smile.
 \textit{
	\begin{itemize}
	\item 'Aren't we daft?' she smiled.
	\item She smiled her thanks and arranged the guitar under her arm.
	\end{itemize}
}
\item verb \\
If you say that something such as fortune  \textbf{smiles}  \textbf{on} someone, you mean that they are lucky or successful .
 \textit{
	\begin{itemize}
	\item When fortune smiled on him, he made the most of it.
	\item God is not smiling on our cause.
	\end{itemize}
}
\item  \\
 all smiles \textit{
	\begin{itemize}
	\end{itemize}
}
\end{enumerate}

\section*{southwest}
{\large \color{blue}  }
\subsection*{Explain}
\begin{enumerate}
\item noun \\
1.  2.  \textit{
	\begin{itemize}
	\end{itemize}
}
\item adjective \\
3.  4.  5.  \textit{
	\begin{itemize}
	\item southwest Italy
	\end{itemize}
}
\item adverb \\
6.  \textit{
	\begin{itemize}
	\end{itemize}
}
\end{enumerate}

\section*{solution}
{\large \color{blue}  solutions  }
\subsection*{Explain}
\begin{enumerate}
\item countable noun \\
A \textbf{solution}  \textbf{to} a problem or difficult  situation is a way of dealing with it so that the difficulty is removed .
 \textit{
	\begin{itemize}
	\item Although he has sought to find a peaceful solution, he is facing pressure to use
military force.
	\item ...the ability to sort out simple, effective solutions to practical problems.
	\end{itemize}
}
\item countable noun \\
The \textbf{solution}  \textbf{to} a puzzle is the answer to it.
 \textit{
	\begin{itemize}
	\item ...the solution to crossword No. 19721.
	\end{itemize}
}
\item countable noun \\
A \textbf{solution} is a liquid in which a solid substance has been dissolved.
 \textit{
	\begin{itemize}
	\item ...a warm solution of liquid detergent.
	\item Vitamins in solution are more affected than those in solid foods.
	\end{itemize}
}
\end{enumerate}

\section*{struggle}
{\large \color{blue}  struggles  struggling  struggled  }
\subsection*{Explain}
\begin{enumerate}
\item verb \\
If you \textbf{struggle}  \textbf{to} do something, you try  hard to do it, even though other people or things may be making it difficult for you to succeed .
 \textit{
	\begin{itemize}
	\item They had to struggle against all kinds of adversity.
	\item Those who have lost their jobs struggle to pay their supermarket bills.
	\end{itemize}
}
\item variable noun \\
A \textbf{struggle} is a long and difficult attempt to achieve something such as freedom or political  rights .
 \textit{
	\begin{itemize}
	\item Life became a struggle for survival.
	\item ...a young lad's struggle to support his poverty-stricken family.
	\item He is currently locked in a power struggle with his Prime Minister.
	\end{itemize}
}
\item verb \\
If you \textbf{struggle} when you are being held , you twist , kick , and move violently in order to get  free .
 \textit{
	\begin{itemize}
	\item I struggled, but he was a tall man, well-built.
	\end{itemize}
}
\item verb \\
If two people \textbf{struggle}  \textbf{with} each other, they fight.
 \textbf{Struggle} is also a noun .
 \textit{
	\begin{itemize}
	\item She screamed at him to 'stop it' as they struggled on the ground.
	\item We were struggling for the gun when it went off!
	\item There were signs that she struggled with her attacker.
	\item He died in a struggle with prison officers.
	\end{itemize}
}
\item verb \\
If you \textbf{struggle}  \textbf{to} move yourself or \textbf{to} move a heavy object, you try to do it, but it is difficult.
 \textit{
	\begin{itemize}
	\item I could see the young boy struggling to free himself.
	\item I struggled with my bags, desperately looking for a porter.
	\end{itemize}
}
\item verb \\
If you \textbf{struggle}  somewhere , you succeed in moving there, but only with great difficulty.
 \textit{
	\begin{itemize}
	\item The pilot struggled out of the wreck almost uninjured.
	\item Catherine struggled to her feet.
	\item I struggled into a bathrobe and staggered down the stairs.
	\end{itemize}
}
\item verb \\
If a person or organization  \textbf{is struggling} , they are likely to fail in what they are doing, even though they might be trying very hard.
 \textit{
	\begin{itemize}
	\item The company is struggling to find buyers for its new product.
	\item One in five young adults was struggling with everyday mathematics.
	\item By the 1960s, many shipyards were struggling.
	\end{itemize}
}
\item singular noun \\
An action or activity that is \textbf{a struggle} is very difficult to do.
 \textit{
	\begin{itemize}
	\item Losing weight was a terrible struggle.
	\end{itemize}
}
\end{enumerate}

\section*{span}
{\large \color{blue}  spans  spanning  spanned  }
\subsection*{Explain}
\begin{enumerate}
\item countable noun \\
A \textbf{span} is the period of time between two dates or events during which something exists , functions , or happens .
 \textit{
	\begin{itemize}
	\item The batteries had a life span of six hours.
	\item Gradually the time span between sessions will increase.
	\end{itemize}
}
\item countable noun \\
Your concentration  \textbf{span} or your attention  \textbf{span} is the length of time you are able to concentrate on something or be interested in it.
 \textit{
	\begin{itemize}
	\item His ability to absorb information was astonishing, but his concentration span was
short.
	\item Young children have a limited attention span.
	\end{itemize}
}
\item verb \\
If something \textbf{spans} a long period of time, it lasts throughout that period of time or relates to that whole period of time.
 \textit{
	\begin{itemize}
	\item His professional career spanned 16 years.
	\item The film spans almost a quarter-century.
	\item Lining a corridor is a wall of photographs spanning his rugby days.
	\end{itemize}
}
\item verb \\
If something \textbf{spans} a range of things, all those things are included in it.
 \textit{
	\begin{itemize}
	\item Bernstein's compositions spanned all aspects of music, from symphonies to musicals.
	\item ...a remarkable man whose interests spanned almost every aspect of nature.
	\end{itemize}
}
\item countable noun \\
The \textbf{span} of something that extends or is spread out sideways is the total width of it from one end to the other.
 \textit{
	\begin{itemize}
	\item It is a very pretty butterfly, with a 2 inch wing span.
	\item The hip joint is a hand span below the waist.
	\end{itemize}
}
\item verb \\
A bridge or other structure that \textbf{spans} something such as a river or a valley stretches right across it.
 \textit{
	\begin{itemize}
	\item Travellers cross a footbridge that spans a little stream.
	\item ...the humped iron bridge spanning the railway.
	\item Floors can span 100 metres without any visible means of support.
	\end{itemize}
}
\end{enumerate}

\section*{swell}
{\large \color{blue}  swells  swelling  swelled  swollen  }
\subsection*{Explain}
\begin{enumerate}
\item verb \\
If the amount or size of something \textbf{swells} or if something \textbf{swells} it, it becomes larger than it was before.
 \textit{
	\begin{itemize}
	\item The human population swelled, at least temporarily, as migrants moved south.
	\item By the end of this month the size of the mission is expected to swell to 280 people.
	\item His bank balance has swelled by £222,000 in the last three weeks.
	\item Offers from other countries should swell the force to 35,000.
	\item ...the ever-swelling numbers of the homeless.
	\item Its population is swollen by 360,000 refugees.
	\end{itemize}
}
\item verb \\
If something such as a part of your body \textbf{swells} , it becomes larger and rounder than normal .
 \textbf{Swell up} means the same as swell .
 \textit{
	\begin{itemize}
	\item Do your ankles swell at night?
	\item The limbs swell to an enormous size.
	\item When you develop a throat infection or catch a cold the glands in the neck swell
up.
	\end{itemize}
}
\item verb \\
If you \textbf{swell}  \textbf{with} a feeling , you are suddenly full of that feeling.
 \textit{
	\begin{itemize}
	\item She could see her two sons swell with pride.
	\end{itemize}
}
\item verb \\
If sounds \textbf{swell} , they get  louder .
 \textit{
	\begin{itemize}
	\item Heavenly music swelled from nowhere.
	\end{itemize}
}
\item countable noun \\
A \textbf{swell} is the regular movement of waves up and down in the open sea.
 \textit{
	\begin{itemize}
	\item We bobbed gently up and down on the swell of the incoming tide.
	\end{itemize}
}
\item adjective \\
You can describe something as \textbf{swell} if you think it is really  nice .
 \textit{
	\begin{itemize}
	\item I've had a swell time.
	\end{itemize}
}
\end{enumerate}

\section*{swarm}
{\large \color{blue}  swarms  swarming  swarmed  }
\subsection*{Explain}
\begin{enumerate}
\item countable noun \\
A \textbf{swarm}  \textbf{of} bees or other insects is a large group of them flying together.
 \textit{
	\begin{itemize}
	\end{itemize}
}
\item verb \\
When bees or other insects \textbf{swarm} , they move or fly in a large group.
 \textit{
	\begin{itemize}
	\item A dark cloud of bees comes swarming out of the hive.
	\end{itemize}
}
\item verb \\
When people \textbf{swarm}  somewhere , they move there quickly in a large group.
 \textit{
	\begin{itemize}
	\item People swarmed to the shops, buying up everything in sight.
	\end{itemize}
}
\item countable noun \\
A \textbf{swarm}  \textbf{of} people is a large group of them moving about quickly.
 \textit{
	\begin{itemize}
	\item A swarm of people encircled the hotel.
	\item Today at the crossing there were swarms of tourists taking photographs.
	\end{itemize}
}
\item verb \\
If a place \textbf{is swarming}  \textbf{with} people, it is full of people moving about in a busy way.
 \textit{
	\begin{itemize}
	\item Within minutes the area was swarming with officers who began searching a nearby wood.
	\end{itemize}
}
\end{enumerate}

\section*{veto}
{\large \color{blue}  vetoes  vetoing  vetoed  }
\subsection*{Explain}
\begin{enumerate}
\item verb \\
If someone in authority  \textbf{vetoes} something, they forbid it, or stop it being put into action.
 \textbf{Veto} is also a noun .
 \textit{
	\begin{itemize}
	\item They vetoed a draft resolution condemning the violence.
	\item The Treasury vetoed any economic aid.
	\item The veto was a calculated political risk.
	\end{itemize}
}
\item uncountable noun \\
\textbf{Veto} is the right that someone in authority has to forbid something.
 \textit{
	\begin{itemize}
	\item ...the President's power of veto.
	\end{itemize}
}
\end{enumerate}

\section*{waist}
{\large \color{blue}  waists  }
\subsection*{Explain}
\begin{enumerate}
\item countable noun \\
Your \textbf{waist} is the middle part of your body where it narrows  slightly above your hips.
 \textit{
	\begin{itemize}
	\item Ricky kept his arm round her waist.
	\item He was stripped to the waist.
	\end{itemize}
}
\item countable noun \\
The \textbf{waist} of a garment such as a dress , coat , or pair of trousers is the part of it which covers the middle part of your body.
 \textit{
	\begin{itemize}
	\end{itemize}
}
\end{enumerate}

\section*{volunteer}
{\large \color{blue}  volunteers  volunteering  volunteered  }
\subsection*{Explain}
\begin{enumerate}
\item countable noun \\
A \textbf{volunteer} is someone who does work without being paid for it, because they want to do it.
 \textit{
	\begin{itemize}
	\item She now helps in a local school as a volunteer three days a week.
	\item Mike was a member of the local volunteer fire brigade.
	\end{itemize}
}
\item countable noun \\
A \textbf{volunteer} is someone who offers to do a particular task or job without being forced to do it.
 \textit{
	\begin{itemize}
	\item Right. What I want now is two volunteers to come down to the front.
	\item Any volunteers?
	\end{itemize}
}
\item verb \\
If you \textbf{volunteer}  \textbf{to} do something, you offer to do it without being forced to do it.
 \textit{
	\begin{itemize}
	\item Aunt Mary volunteered to clean up the kitchen.
	\item He volunteered for the army in 1939.
	\item She volunteered as a nurse in a soldiers' rest-home.
	\item He's volunteered his services as a chauffeur.
	\end{itemize}
}
\item verb \\
If you \textbf{volunteer}  information , you tell someone something without being asked .
 \textit{
	\begin{itemize}
	\item The room was quiet; no one volunteered any further information.
	\item 'They were both great supporters of Franco,' Ryle volunteered.
	\item The next week, Phillida volunteered that they were getting on better.
	\end{itemize}
}
\item countable noun \\
A \textbf{volunteer} is someone who chooses to join the armed forces, especially during a war, as opposed to someone who is forced to join by law .
 \textit{
	\begin{itemize}
	\item They fought as volunteers with the rebels.
	\item Victory in the civil war had been achieved by a mainly volunteer army.
	\end{itemize}
}
\end{enumerate}

\section*{warmth}
{\large \color{blue}  }
\subsection*{Explain}
\begin{enumerate}
\item uncountable noun \\
The \textbf{warmth} of something is the heat that it has or produces.
 \textit{
	\begin{itemize}
	\item She went further into the room, drawn by the warmth of the fire.
	\item June had brought with it the first of the summer warmth.
	\end{itemize}
}
\item uncountable noun \\
The \textbf{warmth} of something such as a garment or blanket is the protection that it gives you against the cold .
 \textit{
	\begin{itemize}
	\item The blanket will provide additional warmth and comfort in bed.
	\end{itemize}
}
\item uncountable noun \\
Someone who has \textbf{warmth} is friendly and enthusiastic in their behaviour towards other people.
 \textit{
	\begin{itemize}
	\item He greeted us both with warmth and affection.
	\end{itemize}
}
\end{enumerate}

\section*{affair}
{\large \color{blue}  affairs  }
\subsection*{Explain}
\begin{enumerate}
\item singular noun \\
If an event or a series of events has been mentioned and you want to talk about it again, you can refer to it as \textbf{the affair} .
 \textit{
	\begin{itemize}
	\item The government has mishandled the whole affair.
	\item The affair began when customs officials inspected a convoy of 60 tankers.
	\item The industry minister described the affair as 'an absolute scandal'.
	\end{itemize}
}
\item singular noun \\
You can refer to an important or interesting event or situation as ' \textbf{the} ... \textbf{affair} '.
 \textit{
	\begin{itemize}
	\item ...a reduction of defence expenditures in the wake of the Suez affair.
	\end{itemize}
}
\item singular noun \\
You can describe the main quality of an event by saying that it is a particular kind of \textbf{affair} .
 \textit{
	\begin{itemize}
	\item Michael said that his planned 10-day visit would be a purely private affair.
	\item Breakfast will be a cheerless affair for the Prime Minister this morning.
	\end{itemize}
}
\item singular noun \\
You can describe an object as a particular kind of \textbf{affair} when you want to draw  attention to a particular feature , or indicate that it is unusual .
 \textit{
	\begin{itemize}
	\item All their beds were distinctive; Mac's was an iron affair with brass knobs.
	\item He divided it into two bundles, tied them to his walking stick, and slung the whole
affair across his back.
	\end{itemize}
}
\item countable noun \\
If two people who are not married to each other have an \textbf{affair} , they have a sexual relationship.
 \textit{
	\begin{itemize}
	\item She was having an affair with someone at work.
	\end{itemize}
}
\item plural noun \\
You can use \textbf{affairs} to refer to all the important facts or activities that are connected with a particular subject .
 \textit{
	\begin{itemize}
	\item He does not want to interfere in the internal affairs of another country.
	\item With more details, here's our foreign affairs correspondent.
	\end{itemize}
}
\item plural noun \\
Your \textbf{affairs} are all the matters connected with your life which you consider to be private and normally  deal with yourself.
 \textit{
	\begin{itemize}
	\item He was rational and consistent in the conduct of his affairs.
	\item The unexpectedness of my father's death meant that his affairs were not entirely
in order.
	\end{itemize}
}
\item singular noun \\
If you say that a decision or situation is someone's \textbf{affair} , you mean that it is their responsibility , and other people should not interfere .
 \textit{
	\begin{itemize}
	\item If you wish to make a fool of yourself, that is your affair.
	\item If they want to stay and fight, then I guess that's their affair.
	\end{itemize}
}
\end{enumerate}

\section*{accident}
{\large \color{blue}  accidents  }
\subsection*{Explain}
\begin{enumerate}
\item countable noun \\
An \textbf{accident}  happens when a vehicle  hits a person, an object , or another vehicle, causing injury or damage .
 \textit{
	\begin{itemize}
	\item She was involved in a serious car accident last week.
	\item Six passengers were killed in the accident.
	\end{itemize}
}
\item countable noun \\
If someone has an \textbf{accident} , something unpleasant happens to them that was not intended , sometimes causing injury or death.
 \textit{
	\begin{itemize}
	\item 5,000 people die every year because of accidents in the home.
	\item The police say the killing of the young man was an accident.
	\end{itemize}
}
\item variable noun \\
If something happens \textbf{by}  \textbf{accident} , it happens completely by chance.
 \textit{
	\begin{itemize}
	\item She discovered the problem by accident.
	\item Almost like an accident of nature, this family has produced more talent than seems
possible.
	\end{itemize}
}
\item  \\
 an accident waiting to happen \textit{
	\begin{itemize}
	\end{itemize}
}
\item  \\
 it's no accident \textit{
	\begin{itemize}
	\end{itemize}
}
\end{enumerate}

\section*{anchor}
{\large \color{blue}  anchors  anchoring  anchored  }
\subsection*{Explain}
\begin{enumerate}
\item countable noun \\
An \textbf{anchor} is a heavy  hooked object that is dropped from a boat into the water at the end of a chain in order to make the boat stay in one place.
 \textit{
	\begin{itemize}
	\end{itemize}
}
\item verb \\
When a boat \textbf{anchors} or when you \textbf{anchor} it, its anchor is dropped into the water in order to make it stay in one place.
 \textit{
	\begin{itemize}
	\item We could anchor off the pier.
	\item They anchored the boat.
	\end{itemize}
}
\item verb \\
If you \textbf{anchor} an object somewhere , you fix it to something to prevent it moving from that place.
 \textit{
	\begin{itemize}
	\item The roots anchor the plant in the earth.
	\item The child seat belt was not properly anchored to the car.
	\end{itemize}
}
\item countable noun \\
If one thing is the \textbf{anchor}  \textbf{for} something else, it makes that thing stable and secure .
 \textit{
	\begin{itemize}
	\item He provided an emotional anchor for her.
	\item He remains the anchor of the country's fragile political balance.
	\end{itemize}
}
\item verb \\
If something \textbf{is anchored}  \textbf{in} something or \textbf{to} something, it has strong  links with it.
 \textit{
	\begin{itemize}
	\item Bilbao is firmly anchored in Basque culture.
	\item His basic outlook remains anchored in the liberal tradition.
	\end{itemize}
}
\item verb \\
The person who \textbf{anchors} a television or radio  programme , especially a news programme, is the person who presents it and acts as a link between interviews and reports which come from other places or studios .
 \textit{
	\begin{itemize}
	\item Viewers saw him anchoring a five-minute summary of regional news.
	\item ...a series of reports on the Vietnam War, anchored by Mr. Cronkite.
	\end{itemize}
}
\item countable noun \\
The \textbf{anchor} on a television or radio programme, especially a news programme, is the person who
presents it.
 \textit{
	\begin{itemize}
	\item He was the anchor of the 15-minute evening newscast.
	\end{itemize}
}
\item  \\
 at anchor \textit{
	\begin{itemize}
	\end{itemize}
}
\item  \\
 drop anchor \textit{
	\begin{itemize}
	\end{itemize}
}
\item  \\
 weigh anchor/up anchor \textit{
	\begin{itemize}
	\end{itemize}
}
\end{enumerate}

\section*{acid}
{\large \color{blue}  acids  }
\subsection*{Explain}
\begin{enumerate}
\item variable noun \\
An \textbf{acid} is a chemical substance, usually a liquid, which contains hydrogen and can react with other substances to form salts . Some acids burn or dissolve other substances that they come into contact with.
 \textit{
	\begin{itemize}
	\item ...citric acid.
	\item Acids in the stomach destroy the virus.
	\end{itemize}
}
\item adjective \\
An \textbf{acid} substance contains acid.
 \textit{
	\begin{itemize}
	\item These shrubs must have an acid, lime-free soil.
	\end{itemize}
}
\item graded adjective \\
An \textbf{acid} fruit or drink has a sour or sharp taste.
 \textit{
	\begin{itemize}
	\item A tomatillo is a small green Mexican fruit with a delicate and slightly acid taste.
	\end{itemize}
}
\item graded adjective \\
An \textbf{acid}  remark , or \textbf{acid}  humour , is very unkind or critical .
 \textit{
	\begin{itemize}
	\item This comedy of contemporary manners is told with compassion and acid humour.
	\end{itemize}
}
\item uncountable noun \\
The drug  LSD is sometimes  referred to as \textbf{acid} .
 \textit{
	\begin{itemize}
	\end{itemize}
}
\end{enumerate}

\section*{auction}
{\large \color{blue}  auctions  auctioning  auctioned  }
\subsection*{Explain}
\begin{enumerate}
\item variable noun \\
An \textbf{auction} is a public sale where goods are sold to the person who offers the highest price.
 \textit{
	\begin{itemize}
	\item He bought the picture at auction in London some years ago.
	\end{itemize}
}
\item verb \\
If something \textbf{is auctioned} , it is sold in an auction.
 \textit{
	\begin{itemize}
	\item The airline plans to auction its international routes to former competitors.
	\item We'll auction them for charity.
	\end{itemize}
}
\end{enumerate}

\section*{blur}
{\large \color{blue}  blurs  blurring  blurred  }
\subsection*{Explain}
\begin{enumerate}
\item countable noun \\
A \textbf{blur} is a shape or area which you cannot see  clearly because it has no distinct outline or because it is moving very fast .
 \textit{
	\begin{itemize}
	\item Out of the corner of my eye I saw a blur of movement on the other side of the glass.
	\item Her face is a blur.
	\end{itemize}
}
\item verb \\
When a thing \textbf{blurs} or when something \textbf{blurs} it, you cannot see it clearly because its edges are no longer distinct.
 \textit{
	\begin{itemize}
	\item This creates a spectrum of colours at the edges of objects which blurs the image.
	\item If you move your eyes and your head, the picture will blur.
	\end{itemize}
}
\item verb \\
If something \textbf{blurs} an idea or a distinction between things, that idea or distinction no longer seems clear.
 \textit{
	\begin{itemize}
	\item His latest work blurs the distinction between fact and fiction.
	\item The evidence is blurred by central banks' reluctance to reveal their blunders.
	\end{itemize}
}
\item verb \\
If your vision  \textbf{blurs} , or if something \textbf{blurs} it, you cannot see things clearly.
 \textit{
	\begin{itemize}
	\item Her eyes, behind her glasses, began to blur.
	\item Sweat ran from his forehead into his eyes, blurring his vision.
	\end{itemize}
}
\end{enumerate}

\section*{avail}
{\large \color{blue}  avails  availing  availed  }
\subsection*{Explain}
\begin{enumerate}
\item  \\
 to/of no avail, to/of little avail \textit{
	\begin{itemize}
	\end{itemize}
}
\item verb \\
If you \textbf{avail}  \textbf{yourself of} an offer or an opportunity , you accept the offer or make use of the opportunity.
 \textit{
	\begin{itemize}
	\item Guests should feel at liberty to avail themselves of your facilities.
	\end{itemize}
}
\end{enumerate}

\section*{bounce}
{\large \color{blue}  bounces  bouncing  bounced  }
\subsection*{Explain}
\begin{enumerate}
\item verb \\
When an object such as a ball \textbf{bounces} or when you \textbf{bounce} it, it moves upwards from a surface or away from it immediately after hitting it.
 \textbf{Bounce} is also a noun .
 \textit{
	\begin{itemize}
	\item I bounced a ball against the house.
	\item My father would burst into the kitchen bouncing a football.
	\item ...a falling pebble, bouncing down the eroded cliff.
	\item They watched the dodgem cars bang and bounce.
	\item The wheelchair tennis player is allowed two bounces of the ball.
	\end{itemize}
}
\item uncountable noun \\
The \textbf{bounce} of a sports field is the condition of it, which determines how high a ball will bounce on it.
 \textit{
	\begin{itemize}
	\end{itemize}
}
\item verb \\
If sound or light \textbf{bounces off} a surface or \textbf{is bounced off} it, it reaches the surface and is reflected back.
 \textit{
	\begin{itemize}
	\item Your arms and legs need protection from light bouncing off glass.
	\item They work by bouncing microwaves off solid objects.
	\end{itemize}
}
\item verb \\
If something \textbf{bounces} or if something \textbf{bounces} it, it swings or moves up and down.
 \textit{
	\begin{itemize}
	\item Her long black hair bounced as she walked.
	\item Then I noticed the car was bouncing up and down as if someone were jumping on it.
	\item The wind was bouncing the branches of the big oak trees.
	\end{itemize}
}
\item verb \\
If you \textbf{bounce} on a soft surface, you jump up and down on it repeatedly.
 \textit{
	\begin{itemize}
	\item She lets us do anything, even bounce on our beds.
	\end{itemize}
}
\item verb \\
If you \textbf{bounce} a child on your knee , you lift him or her up and down quickly and repeatedly for fun .
 \textit{
	\begin{itemize}
	\item Patsy had picked up the baby and was bouncing him on her knee.
	\end{itemize}
}
\item verb \\
If someone \textbf{bounces}  somewhere , they move there in an energetic way, because they are feeling  happy .
 \textit{
	\begin{itemize}
	\item Moira bounced into the office.
	\end{itemize}
}
\item verb \\
If you \textbf{bounce} your ideas  \textbf{off} someone, you tell them to that person, in order to find out what they think about them.
 \textit{
	\begin{itemize}
	\item It was good to bounce ideas off another mind.
	\item Let's bounce a few ideas around.
	\end{itemize}
}
\item verb \\
If someone \textbf{bounces} you \textbf{into} doing something you do not really  want to do, they make you do it, usually by starting a process which cannot easily be stopped .
 \textit{
	\begin{itemize}
	\item Attempts have been made to bounce member states into decisions.
	\end{itemize}
}
\item verb \\
If a cheque \textbf{bounces} or if a bank \textbf{bounces} it, the bank refuses to accept it and pay out the money, because the person who wrote it does not have enough money
in their account.
 \textit{
	\begin{itemize}
	\item Our only complaint would be if the cheque bounced.
	\item His bank wrongly bounced cheques worth £75,000.
	\end{itemize}
}
\item verb \\
If an email or other electronic message \textbf{bounces} , it is returned to the person who sent it because the address was wrong or because of a problem with one of the computers involved in sending it.
 \textit{
	\begin{itemize}
	\end{itemize}
}
\end{enumerate}

\section*{carbon}
{\large \color{blue}  carbons  }
\subsection*{Explain}
\begin{enumerate}
\item uncountable noun \\
\textbf{Carbon} is a chemical element that diamonds and coal are made up of.
 \textit{
	\begin{itemize}
	\end{itemize}
}
\item countable noun \\
A \textbf{carbon} is a sheet of carbon paper.
 \textit{
	\begin{itemize}
	\item He inserted the paper and two carbons.
	\end{itemize}
}
\end{enumerate}

\section*{demand}
{\large \color{blue}  demands  demanding  demanded  }
\subsection*{Explain}
\begin{enumerate}
\item verb \\
If you \textbf{demand} something such as information or action, you ask for it in a very forceful way.
 \textit{
	\begin{itemize}
	\item Mr Byers last night demanded an immediate explanation from the Education Secretary.
	\item Russia demanded that Unita send a delegation to the peace talks.
	\item The hijackers are demanding to speak to representatives of both governments.
	\item 'What did you expect me to do about it?' she demanded.
	\end{itemize}
}
\item verb \\
If one thing \textbf{demands} another, the first needs the second in order to happen or be dealt with successfully.
 \textit{
	\begin{itemize}
	\item He said the task of reconstruction would demand much patience, hard work and sacrifice.
	\item There would be fewer international crises demanding his attention.
	\item But he could also turn on the style when the occasion demanded.
	\end{itemize}
}
\item countable noun \\
A \textbf{demand} is a firm request for something.
 \textit{
	\begin{itemize}
	\item There have been demands for services from tenants up there.
	\item They consistently rejected the demand to remove U.S. troops.
	\item He grew ever more fierce in his demands.
	\end{itemize}
}
\item uncountable noun \\
If you refer to \textbf{demand} , or to the \textbf{demand}  \textbf{for} something, you are referring to how many people want to have it, do it, or buy it.
 \textit{
	\begin{itemize}
	\item Another flight would be arranged on Saturday if sufficient demand arose.
	\item Demand for coal is down and so are prices.
	\item The demand to see her work is much greater than expected.
	\item Because of the slump in domestic demand, production has stopped.
	\end{itemize}
}
\item plural noun \\
\textbf{The}  \textbf{demands}  \textbf{of} something or its \textbf{demands}  \textbf{on} you are the things which it needs or the things which you have to do for it.
 \textit{
	\begin{itemize}
	\item Researchers wrongly assumed that people were quite clear about the demands of the
task.
	\item ...the demands and challenges of a new job.
	\item There were too many other demands on his loyalty now.
	\end{itemize}
}
\item  \\
 in (great) demand \textit{
	\begin{itemize}
	\end{itemize}
}
\item  \\
 make demands \textit{
	\begin{itemize}
	\end{itemize}
}
\item  \\
 on demand: usu PHR after v \textit{
	\begin{itemize}
	\end{itemize}
}
\end{enumerate}

\section*{caution}
{\large \color{blue}  cautions  cautioning  cautioned  }
\subsection*{Explain}
\begin{enumerate}
\item uncountable noun \\
\textbf{Caution} is great care which you take in order to avoid  possible danger.
 \textit{
	\begin{itemize}
	\item Extreme caution should be exercised when buying part-worn tyres.
	\item The Chancellor is a man of caution.
	\end{itemize}
}
\item verb \\
If someone \textbf{cautions} you, they warn you about problems or danger.
 \textbf{Caution} is also a noun .
 \textit{
	\begin{itemize}
	\item Tony cautioned against misrepresenting the situation.
	\item The statement clearly was intended to caution Seoul against attempting to block the
council's action again.
	\item But experts caution that instant gratification comes at a price.
	\item There was a note of caution for the Treasury in the figures.
	\end{itemize}
}
\item verb \\
If someone who has broken the law  \textbf{is cautioned} by the police , they are warned that if they break the law again official action will be taken against them.
 \textbf{Caution} is also a noun.
 \textit{
	\begin{itemize}
	\item The two men were cautioned but police say they will not be charged.
	\item Liam was eventually let off with a caution.
	\end{itemize}
}
\item verb \\
If someone who has been arrested  \textbf{is cautioned} , the police warn them that anything they say may be used as evidence in a trial .
 \textit{
	\begin{itemize}
	\item Nobody was cautioned after arrest.
	\end{itemize}
}
\item  \\
 to throw caution to the wind \textit{
	\begin{itemize}
	\end{itemize}
}
\end{enumerate}

\section*{disguise}
{\large \color{blue}  disguises  disguising  disguised  }
\subsection*{Explain}
\begin{enumerate}
\item variable noun \\
If you are \textbf{in}  \textbf{disguise} , you are not wearing your usual  clothes or you have altered your appearance in other ways, so that people will not recognize you.
 \textit{
	\begin{itemize}
	\item You'll have to travel in disguise.
	\item He was wearing that ridiculous disguise.
	\item She's adopted so many disguises her own mother wouldn't recognize her.
	\end{itemize}
}
\item verb \\
If you \textbf{disguise}  \textbf{yourself} , you put on clothes which make you look like someone else or alter your appearance in other ways, so that people will not
recognize you.
 \textit{
	\begin{itemize}
	\item She disguised herself as a man so she could fight on the battlefield.
	\end{itemize}
}
\item verb \\
To \textbf{disguise} something means to hide it or make it appear different so that people will not know about it or will not recognize it.
 \textit{
	\begin{itemize}
	\item He made no attempt to disguise his agitation.
	\item Their healthy image disguises the fact that they are highly processed foods.
	\item I played along, and disguised my voice.
	\end{itemize}
}
\end{enumerate}

\section*{charity}
{\large \color{blue}  charities  }
\subsection*{Explain}
\begin{enumerate}
\item countable noun \\
A \textbf{charity} is an organization which raises money in order to help people who are sick or very poor , or who have a disability .
 \textit{
	\begin{itemize}
	\item The National Trust is a registered charity.
	\item She was working as a volunteer at a homeless charity in Cambridge.
	\end{itemize}
}
\item uncountable noun \\
If you give money \textbf{to}  \textbf{charity} , you give it to one or more charitable organizations. If you do something \textbf{for}  \textbf{charity} , you do it in order to raise money for one or more charitable organizations.
 \textit{
	\begin{itemize}
	\item He made substantial donations to charity.
	\item Gooch will be raising money for charity.
	\item ...a charity event.
	\end{itemize}
}
\item uncountable noun \\
People who live on \textbf{charity} live on money or goods which other people give them because they are poor.
 \textit{
	\begin{itemize}
	\item My mum was very proud. She wouldn't accept charity.
	\item Her husband is unemployed and the family depends on charity.
	\end{itemize}
}
\item uncountable noun \\
\textbf{Charity} is kindness and understanding towards other people.
 \textit{
	\begin{itemize}
	\end{itemize}
}
\item  \\
 charity begins at home \textit{
	\begin{itemize}
	\end{itemize}
}
\end{enumerate}

\section*{dream}
{\large \color{blue}  dreams  dreaming  dreamed  dreamt  }
\subsection*{Explain}
\begin{enumerate}
\item countable noun \\
A \textbf{dream} is an imaginary series of events that you experience in your mind while you are asleep .
 \textit{
	\begin{itemize}
	\item He had a dream about Claire.
	\item I had a dream that I was in an old study, surrounded by leather books.
	\end{itemize}
}
\item verb \\
When you \textbf{dream} , you experience imaginary events in your mind while you are asleep.
 \textit{
	\begin{itemize}
	\item Ivor dreamed that he was on a bus.
	\item She dreamed about her baby.
	\end{itemize}
}
\item countable noun \\
You can refer to a situation or event as a \textbf{dream} if you often think about it because you would like it to happen .
 \textit{
	\begin{itemize}
	\item He had finally accomplished his dream of becoming a pilot.
	\item My dream is to have a house in the country.
	\item You can make that dream come true.
	\end{itemize}
}
\item verb \\
If you often think about something that you would very much like to happen or have,
you can say that you \textbf{dream}  \textbf{of} it.
 \textit{
	\begin{itemize}
	\item As a schoolgirl, she had dreamed of becoming an actress.
	\item For most of us, a brand new designer kitchen is something we can only dream about.
	\item I dream that my son will attend college and find a good job.
	\end{itemize}
}
\item adjective \\
You can use \textbf{dream} to describe something that you think is ideal or perfect , especially if it is something that you thought you would never be able to have or experience.
 \textit{
	\begin{itemize}
	\item He had his dream house built on the banks of the river Bure.
	\item ...a dream holiday to Jamaica.
	\end{itemize}
}
\item singular noun \\
If you describe something as \textbf{a} particular person's \textbf{dream} , you think that it would be ideal for that person and that he or she would like it
very much.
 \textit{
	\begin{itemize}
	\item Greece is said to be a botanist's dream.
	\end{itemize}
}
\item singular noun \\
If you say that something is \textbf{a dream} , you mean that it is wonderful .
 \textit{
	\begin{itemize}
	\end{itemize}
}
\item countable noun \\
You can refer to a situation or event that does not seem  real as a \textbf{dream} , especially if it is very strange or unpleasant .
 \textit{
	\begin{itemize}
	\item When the right woman comes along, this bad dream will be over.
	\end{itemize}
}
\item verb \\
If you say that you \textbf{would not}  \textbf{dream of} doing something, you are emphasizing that you would never do it because you think it is wrong or is not possible or suitable for you.
 \textit{
	\begin{itemize}
	\item I wouldn't dream of making fun of you.
	\item My sons would never dream of expecting their clothes to be ironed.
	\end{itemize}
}
\item verb \\
If you say that you \textbf{never}  \textbf{dreamed}  \textbf{that} something would happen, you are emphasizing that you did not think that it would
happen because it seemed very unlikely .
 \textit{
	\begin{itemize}
	\item I never dreamed that I would be able to afford a home here.
	\item Who could ever dream of a disaster like this?
	\item I find life more charming and more astonishing than I'd ever dreamed.
	\end{itemize}
}
\item  \\
 dream on \textit{
	\begin{itemize}
	\end{itemize}
}
\item  \\
 in a dream \textit{
	\begin{itemize}
	\end{itemize}
}
\item  \\
 in your dreams! \textit{
	\begin{itemize}
	\end{itemize}
}
\item  \\
 like a dream \textit{
	\begin{itemize}
	\end{itemize}
}
\item  \\
 of your dreams \textit{
	\begin{itemize}
	\end{itemize}
}
\item  \\
 in your wildest dreams \textit{
	\begin{itemize}
	\end{itemize}
}
\item  \\
 beyond your wildest dreams \textit{
	\begin{itemize}
	\end{itemize}
}
\end{enumerate}

\section*{coal}
{\large \color{blue}  coals  }
\subsection*{Explain}
\begin{enumerate}
\item uncountable noun \\
\textbf{Coal} is a hard black substance that is extracted from the ground and burned as fuel.
 \textit{
	\begin{itemize}
	\item Gas-fired electricity is cheaper than coal.
	\item Today, oil and natural gas have replaced coal and wood in most areas.
	\end{itemize}
}
\item plural noun \\
\textbf{Coals} are burning pieces of coal.
 \textit{
	\begin{itemize}
	\item The iron teakettle was hissing splendidly over live coals.
	\item It is important to get the coals white-hot before you start cooking.
	\end{itemize}
}
\item  \\
 haul/drag sb over the coals \textit{
	\begin{itemize}
	\end{itemize}
}
\item  \\
 coals to Newcastle \textit{
	\begin{itemize}
	\end{itemize}
}
\end{enumerate}

\section*{employ}
{\large \color{blue}  employs  employing  employed  }
\subsection*{Explain}
\begin{enumerate}
\item verb \\
If a person or company  \textbf{employs} you, they pay you to work for them.
 \textit{
	\begin{itemize}
	\item The company employs 18 staff.
	\item More than 3,000 local workers are employed in the tourism industry.
	\item Her first husband had been employed in a chemicals company.
	\end{itemize}
}
\item verb \\
If you \textbf{employ} certain methods , materials, or expressions , you use them.
 \textit{
	\begin{itemize}
	\item The tactics the police are now to employ are definitely uncompromising.
	\item ...the vocabulary that she employs.
	\item ...the approaches and methods employed in the study.
	\end{itemize}
}
\item verb \\
If your time \textbf{is employed}  \textbf{in} doing something, you are using the time you have to do that thing.
 \textit{
	\begin{itemize}
	\item Your time could be usefully employed in attending to professional matters.
	\item The journalists would be better employed in explaining how the costs can be justified.
	\end{itemize}
}
\item  \\
 in the employ of sb/sth \textit{
	\begin{itemize}
	\end{itemize}
}
\end{enumerate}

\section*{encounter}
{\large \color{blue}  encounters  encountering  encountered  }
\subsection*{Explain}
\begin{enumerate}
\item verb \\
If you \textbf{encounter}  problems or difficulties , you experience them.
 \textit{
	\begin{itemize}
	\item Every day of our lives we encounter stresses of one kind or another.
	\item Thousands of customers have encountered problems.
	\end{itemize}
}
\item verb \\
If you \textbf{encounter} someone, you meet them, usually unexpectedly.
 \textit{
	\begin{itemize}
	\item Did you encounter anyone in the building?
	\item Renata wrote him that she had encountered her long-estranged father.
	\end{itemize}
}
\item countable noun \\
An \textbf{encounter}  \textbf{with} someone is a meeting with them, particularly one that is unexpected or significant .
 \textit{
	\begin{itemize}
	\item ...offering supporters the chance to win an encounter with the President in return
for any size of donation.
	\end{itemize}
}
\item countable noun \\
An \textbf{encounter} is a particular type of experience.
 \textit{
	\begin{itemize}
	\item ...a sexual encounter.
	\item She was not the only person to feel daunted by her encounter with the court.
	\end{itemize}
}
\end{enumerate}

\section*{dish}
{\large \color{blue}  dishes  dishing  dished  }
\subsection*{Explain}
\begin{enumerate}
\item countable noun \\
A \textbf{dish} is a shallow container with a wide  uncovered  top . You eat and serve food from dishes and cook food in them.
 \textit{
	\begin{itemize}
	\item ...plastic bowls and dishes.
	\item Pile potatoes into a warm serving dish.
	\end{itemize}
}
\item countable noun \\
The contents of a dish can be referred to as a \textbf{dish} of something.
 \textit{
	\begin{itemize}
	\item Nicholas ate a dish of spaghetti.
	\end{itemize}
}
\item countable noun \\
Food that is prepared in a particular style or combination can be referred to as a \textbf{dish} .
 \textit{
	\begin{itemize}
	\item This dish is best served cold.
	\item There are plenty of vegetarian dishes to choose from.
	\item ...a delicious fish dish.
	\end{itemize}
}
\item plural noun \\
All the objects that have been used to cook, serve, and eat a meal can be referred to as \textbf{the}  \textbf{dishes} .
 \textit{
	\begin{itemize}
	\item There were dirty dishes in the sink.
	\item He'd cooked dinner and washed the dishes.
	\end{itemize}
}
\item countable noun \\
You can use \textbf{dish} to refer to anything that is round and hollow in shape with a wide uncovered top.
 \textit{
	\begin{itemize}
	\item ...a dish used to receive satellite broadcasts.
	\end{itemize}
}
\item  \\
 do the dishes \textit{
	\begin{itemize}
	\end{itemize}
}
\end{enumerate}

\section*{excerpt}
{\large \color{blue}  excerpts  excerpted  }
\subsection*{Explain}
\begin{enumerate}
\item countable noun \\
An \textbf{excerpt} is a short  piece of writing or music which is taken from a larger piece.
 \textit{
	\begin{itemize}
	\item ...an excerpt from Tchaikovsky's Nutcracker.
	\end{itemize}
}
\item passive verb \\
If a long piece of writing or music \textbf{is excerpted} , short pieces from it are printed or played on their own.
 \textit{
	\begin{itemize}
	\item The following is excerpted from his journal.
	\end{itemize}
}
\end{enumerate}

\section*{filter}
{\large \color{blue}  filters  filtering  filtered  }
\subsection*{Explain}
\begin{enumerate}
\item verb \\
To \textbf{filter} a substance means to pass it through a device which is designed to remove certain
particles contained in it.
 \textit{
	\begin{itemize}
	\item The best prevention for cholera is to boil or filter water, and eat only well-cooked
food.
	\end{itemize}
}
\item countable noun \\
A \textbf{filter} is a device through which a substance is passed when it is being filtered.
 \textit{
	\begin{itemize}
	\item ...a paper coffee filter.
	\end{itemize}
}
\item countable noun \\
A \textbf{filter} is a device through which sound or light is passed and which blocks or reduces particular
sound or light frequencies.
 \textit{
	\begin{itemize}
	\item You might use a yellow filter to improve the clarity of a hazy horizon.
	\end{itemize}
}
\item verb \\
If light or sound \textbf{filters}  \textbf{into} a place, it comes in weakly or slowly, either through a partly covered opening , or from a long distance  away .
 \textit{
	\begin{itemize}
	\item Light filtered into my kitchen through the soft, green shade of the cherry tree.
	\end{itemize}
}
\item verb \\
When news or information \textbf{filters} through to people, it gradually reaches them.
 \textit{
	\begin{itemize}
	\item It took months before the findings began to filter through to the politicians.
	\item News of the attack quickly filtered through the college.
	\item ...as indications filter in from polling stations.
	\item ...the horror stories which were beginning to filter out of Germany.
	\end{itemize}
}
\item countable noun \\
A traffic \textbf{filter} is a traffic signal or lane which controls the movement of traffic wanting to turn left or right.
 \textit{
	\begin{itemize}
	\end{itemize}
}
\end{enumerate}

\section*{experience}
{\large \color{blue}  experiences  experiencing  experienced  }
\subsection*{Explain}
\begin{enumerate}
\item uncountable noun \\
\textbf{Experience} is knowledge or skill in a particular job or activity, which you have gained because you have done that job or activity for a long time.
 \textit{
	\begin{itemize}
	\item He has also had managerial experience on every level.
	\item His mother's had plenty of experience taking care of the twins for him.
	\end{itemize}
}
\item uncountable noun \\
\textbf{Experience} is used to refer to the past events, knowledge, and feelings that make up someone's life or character .
 \textit{
	\begin{itemize}
	\item I should not be in any danger here, but experience has taught me caution.
	\item She had learned from experience to take little rests in between her daily routine.
	\item 'If you act afraid, they won't let go,' he says, speaking from experience.
	\end{itemize}
}
\item countable noun \\
An \textbf{experience} is something that you do or that happens to you, especially something important that affects you.
 \textit{
	\begin{itemize}
	\item Moving had become a common experience for me.
	\item His only experience of gardening so far proved immensely satisfying.
	\item Many of his clients are unbelievably nervous, usually because of a bad experience
in the past.
	\end{itemize}
}
\item verb \\
If you \textbf{experience} a particular situation , you are in that situation or it happens to you.
 \textit{
	\begin{itemize}
	\item We had never experienced this kind of holiday before and had no idea what to expect.
	\item British business is now experiencing a severe recession.
	\end{itemize}
}
\item verb \\
If you \textbf{experience} a feeling, you feel it or are affected by it.
 \textbf{Experience} is also a noun .
 \textit{
	\begin{itemize}
	\item If you experience dizziness or drowsiness, do not drive or operate machinery.
	\item ...the experience of pain.
	\end{itemize}
}
\end{enumerate}

\section*{group}
{\large \color{blue}  groups  grouping  grouped  }
\subsection*{Explain}
\begin{enumerate}
\item countable noun \\
A \textbf{group}  \textbf{of} people or things is a number of people or things which are together in one place
at one time.
 \textit{
	\begin{itemize}
	\item The trouble involved a small group of football supporters.
	\item The students work in groups on complex problems.
	\end{itemize}
}
\item countable noun \\
A \textbf{group} is a set of people who have the same interests or aims , and who organize themselves to work or act together.
 \textit{
	\begin{itemize}
	\item ...the Minority Rights Group.
	\item Members of an environmental group are staging a protest inside a chemical plant.
	\end{itemize}
}
\item countable noun \\
A \textbf{group} is a set of people, organizations, or things which are considered together because
they have something in common.
 \textit{
	\begin{itemize}
	\item She is among the most promising players in her age group.
	\item As a group, today's old people are still relatively deprived.
	\end{itemize}
}
\item countable noun \\
A \textbf{group} is a number of separate commercial or industrial  firms which all have the same owner .
 \textit{
	\begin{itemize}
	\item The group made a pre-tax profit of £1.05 million.
	\item ...a French-based insurance group.
	\end{itemize}
}
\item countable noun \\
A \textbf{group} is a number of musicians who perform together, especially ones who play popular music.
 \textit{
	\begin{itemize}
	\item At school he played bass in a pop group called The Urge.
	\item ...Robbie Williams' backing group.
	\end{itemize}
}
\item verb \\
If a number of things or people \textbf{are grouped together} or \textbf{group together} , they are together in one place or within one organization or system.
 \textit{
	\begin{itemize}
	\item The fact sheets are grouped into seven sections.
	\item The G-7 organization groups together the world's seven leading industrialized nations.
	\item We want to encourage them to group together to act as a big purchaser.
	\end{itemize}
}
\end{enumerate}

\section*{flourish}
{\large \color{blue}  flourishes  flourishing  flourished  }
\subsection*{Explain}
\begin{enumerate}
\item verb \\
If something \textbf{flourishes} , it is successful , active , or common , and developing quickly and strongly.
 \textit{
	\begin{itemize}
	\item Business flourished and within six months they were earning 18,000 roubles a day.
	\end{itemize}
}
\item verb \\
If a plant or animal \textbf{flourishes} , it grows  well or is healthy because the conditions are right for it.
 \textit{
	\begin{itemize}
	\item The plant flourishes particularly well in slightly harsher climes.
	\end{itemize}
}
\item verb \\
If you \textbf{flourish} an object, you wave it about in a way that makes people notice it.
 \textbf{Flourish} is also a noun .
 \textit{
	\begin{itemize}
	\item He flourished the glass to emphasize the point.
	\item He took his peaked cap from under his arm with a flourish and pulled it low over
his eyes.
	\end{itemize}
}
\item countable noun \\
If you do something \textbf{with a}  \textbf{flourish} , you do in a showy way so that people notice it.
 \textit{
	\begin{itemize}
	\item She tended to finish dancing with a flourish.
	\end{itemize}
}
\item countable noun \\
A \textbf{flourish} is a curly line or piece of decoration.
 \textit{
	\begin{itemize}
	\item He scrawled his name across the bill, underlining it with a showy flourish.
	\end{itemize}
}
\end{enumerate}

\section*{herd}
{\large \color{blue}  herds  herding  herded  }
\subsection*{Explain}
\begin{enumerate}
\item countable noun \\
A \textbf{herd} is a large group of animals of one kind that live together.
 \textit{
	\begin{itemize}
	\item ...large herds of elephant and buffalo.
	\item ...dairy herds.
	\end{itemize}
}
\item singular noun \\
If you say that someone has joined  \textbf{the herd} or follows  \textbf{the herd} , you are criticizing them because you think that they behave just like everyone else and do not think for themselves.
 \textit{
	\begin{itemize}
	\item They are individuals; they will not follow the herd.
	\end{itemize}
}
\item verb \\
If you \textbf{herd} people somewhere , you make them move there in a group.
 \textit{
	\begin{itemize}
	\item He began to herd the prisoners out.
	\item The group was herded into a bus.
	\end{itemize}
}
\item verb \\
If you \textbf{herd} animals, you make them move along as a group.
 \textit{
	\begin{itemize}
	\item Stefano used a motor cycle to herd the sheep.
	\item A boy herded half a dozen camels down towards the water trough.
	\end{itemize}
}
\end{enumerate}

\section*{glide}
{\large \color{blue}  glides  gliding  glided  }
\subsection*{Explain}
\begin{enumerate}
\item verb \\
If you \textbf{glide}  somewhere , you move silently and in a smooth and effortless way.
 \textit{
	\begin{itemize}
	\item Waiters glide between tightly packed tables bearing trays of pasta.
	\end{itemize}
}
\item verb \\
When birds or aeroplanes  \textbf{glide} , they float on air currents .
 \textit{
	\begin{itemize}
	\item ... the albatross, which glides effortlessly and gracefully behind the yacht.
	\end{itemize}
}
\end{enumerate}

\section*{ideology}
{\large \color{blue}  ideologies  }
\subsection*{Explain}
\begin{enumerate}
\item variable noun \\
An \textbf{ideology} is a set of beliefs, especially the political beliefs on which people, parties , or countries base their actions.
 \textit{
	\begin{itemize}
	\item ...capitalist ideology.
	\end{itemize}
}
\end{enumerate}

\section*{grin}
{\large \color{blue}  grins  grinning  grinned  }
\subsection*{Explain}
\begin{enumerate}
\item verb \\
When you \textbf{grin} , you smile broadly.
 \textit{
	\begin{itemize}
	\item He grins, delighted at the memory.
	\item Sarah tried several times to catch Philip's eye, but he just grinned at her.
	\item I'll never forget his evil grinning face staring at me.
	\end{itemize}
}
\item countable noun \\
A \textbf{grin} is a broad smile.
 \textit{
	\begin{itemize}
	\item She came out of his office with a big grin on her face.
	\item Bobby looked at her with a sheepish grin.
	\end{itemize}
}
\item  \\
 to grin and bear it \textit{
	\begin{itemize}
	\end{itemize}
}
\end{enumerate}

\section*{imitation}
{\large \color{blue}  imitations  }
\subsection*{Explain}
\begin{enumerate}
\item countable noun \\
An \textbf{imitation} of something is a copy of it.
 \textit{
	\begin{itemize}
	\item ...the most accurate imitation of Chinese architecture in Europe.
	\end{itemize}
}
\item uncountable noun \\
\textbf{Imitation} means copying someone else's actions.
 \textit{
	\begin{itemize}
	\item They discussed important issues in imitation of their elders.
	\item Molly learned her golf by imitation.
	\end{itemize}
}
\item adjective \\
\textbf{Imitation} things are not genuine but are made to look as if they are.
 \textit{
	\begin{itemize}
	\item ...a complete set of Dickens bound in imitation leather.
	\end{itemize}
}
\item countable noun \\
If someone does an \textbf{imitation}  \textbf{of} another person, they copy the way they speak or behave , sometimes in order to be funny .
 \textit{
	\begin{itemize}
	\item He gave his imitation of Queen Elizabeth's royal wave.
	\item I could do a pretty good imitation of him.
	\end{itemize}
}
\end{enumerate}

\section*{grope}
{\large \color{blue}  gropes  groping  groped  }
\subsection*{Explain}
\begin{enumerate}
\item verb \\
If you \textbf{grope}  \textbf{for} something that you cannot see , you try to find it by moving your hands around in order to feel it.
 \textit{
	\begin{itemize}
	\item With his left hand he groped for the knob, turned it, and pulled the door open.
	\item Bunbury groped in his breast pocket for his wallet.
	\end{itemize}
}
\item verb \\
If you \textbf{grope} your \textbf{way} to a place, you move there, holding your hands in front of you and feeling the way because you cannot see anything.
 \textit{
	\begin{itemize}
	\item I didn't turn on the light, but groped my way across the room.
	\end{itemize}
}
\item verb \\
If you \textbf{grope}  \textbf{for} something, for example the solution to a problem , you try to think of it, when you have no real  idea what it could be.
 \textit{
	\begin{itemize}
	\item He groped for solutions to the problems facing the country.
	\item She groped for a simple word to express a simple idea.
	\end{itemize}
}
\item verb \\
If one person \textbf{gropes} another, they touch or take  hold of them in a rough , sexual way.
 \textbf{Grope} is also a noun .
 \textit{
	\begin{itemize}
	\item He would try to grope her breasts and put his hand up her skirt.
	\item She even boasted of having a grope in a cupboard with a 13-year-old.
	\end{itemize}
}
\end{enumerate}

\section*{intention}
{\large \color{blue}  intentions  }
\subsection*{Explain}
\begin{enumerate}
\item variable noun \\
An \textbf{intention} is an idea or plan of what you are going to do.
 \textit{
	\begin{itemize}
	\item Beveridge announced his intention of standing for parliament.
	\item It is my intention to remain in my position until a successor is elected.
	\item Unfortunately, his good intentions never seemed to last long.
	\end{itemize}
}
\item  \\
 have no intention/have every intention \textit{
	\begin{itemize}
	\end{itemize}
}
\end{enumerate}

\section*{gut}
{\large \color{blue}  guts  gutting  gutted  }
\subsection*{Explain}
\begin{enumerate}
\item plural noun \\
A person's or animal's \textbf{guts} are all the organs inside them.
 \textit{
	\begin{itemize}
	\item By the time they finish, the crewmen are standing ankle-deep in fish guts.
	\end{itemize}
}
\item verb \\
When someone \textbf{guts} a dead animal or fish, they prepare it for cooking by removing all the organs from inside it.
 \textit{
	\begin{itemize}
	\item It is not always necessary to gut the fish prior to freezing.
	\end{itemize}
}
\item singular noun \\
\textbf{The gut} is the tube inside the body of a person or animal through which food passes while it is being
 digested .
 \textit{
	\begin{itemize}
	\end{itemize}
}
\item uncountable noun \\
\textbf{Guts} is the will and courage to do something which is difficult or unpleasant , or which might have unpleasant results.
 \textit{
	\begin{itemize}
	\item The new Chancellor has the guts to push through unpopular tax increases.
	\item It takes more guts than I've usually got to go and see him.
	\end{itemize}
}
\item singular noun \\
A \textbf{gut} feeling is based on instinct or emotion rather than reason .
 \textit{
	\begin{itemize}
	\item Let's have your gut reaction to the facts as we know them.
	\end{itemize}
}
\item countable noun \\
You can refer to someone's stomach as their \textbf{gut} , especially when it is very large and sticks out.
 \textit{
	\begin{itemize}
	\item His gut sagged out over his belt.
	\end{itemize}
}
\item plural noun \\
The \textbf{guts}  \textbf{of} something, for example a subject or a machine, are the key elements of it, which make it work.
 \textit{
	\begin{itemize}
	\item She has a reputation for getting at the guts of a subject and never pulling her punches.
	\item The guts of the reactor have to be hauled out of the pressure vessel.
	\end{itemize}
}
\item verb \\
To \textbf{gut} a building means to destroy the inside of it so that only its outside walls remain .
 \textit{
	\begin{itemize}
	\item Over the weekend, a firebomb gutted a building where 60 people lived.
	\item A factory stands gutted and deserted.
	\end{itemize}
}
\item uncountable noun \\
\textbf{Gut} is string made from part of the stomach of an animal. Traditionally, it is used to make the
strings of sports rackets or musical instruments such as violins .
 \textit{
	\begin{itemize}
	\end{itemize}
}
\item  \\
 to bust a gut \textit{
	\begin{itemize}
	\end{itemize}
}
\item  \\
 to hate someone's guts \textit{
	\begin{itemize}
	\end{itemize}
}
\item  \\
 spill your guts \textit{
	\begin{itemize}
	\end{itemize}
}
\item  \\
 work your guts out \textit{
	\begin{itemize}
	\end{itemize}
}
\end{enumerate}

\section*{magnet}
{\large \color{blue}  magnets  }
\subsection*{Explain}
\begin{enumerate}
\item countable noun \\
If you say that something is a \textbf{magnet} or is like a \textbf{magnet} , you mean that people are very attracted by it and want to go to it or look at it.
 \textit{
	\begin{itemize}
	\item Prospect Park, with its vast lake, is a magnet for all health freaks.
	\item Lower interest rates are acting like a magnet, dragging consumers back to the shops.
	\end{itemize}
}
\item countable noun \\
A \textbf{magnet} is a piece of iron or other material which attracts iron towards it.
 \textit{
	\begin{itemize}
	\item ...a fridge magnet.
	\end{itemize}
}
\end{enumerate}

\section*{hire}
{\large \color{blue}  hires  hiring  hired  }
\subsection*{Explain}
\begin{enumerate}
\item verb \\
If you \textbf{hire} someone, you employ them or pay them to do a particular job for you.
 \textit{
	\begin{itemize}
	\item Sixteen of the contestants have hired lawyers and are suing the organisers.
	\item The rest of the staff have been hired on short-term contracts.
	\item He will be in charge of all hiring and firing at PHA.
	\item ...the mystery assassin (who turned out to be a hired killer).
	\end{itemize}
}
\item verb \\
If you \textbf{hire} something, you pay money to the owner so that you can use it for a period of time.
 \textit{
	\begin{itemize}
	\item To hire a car you must produce a passport and a current driving licence.
	\item Her hired car was found abandoned at Beachy Head.
	\end{itemize}
}
\item uncountable noun \\
You use \textbf{hire} to refer to the activity or business of hiring something.
 \textit{
	\begin{itemize}
	\item They booked our hotel, and organised car hire.
	\item Hire of skis, boots and clothing, are all available.
	\item ...a day's outing by hire car to the southern coast of Crete.
	\end{itemize}
}
\item  \\
 for hire \textit{
	\begin{itemize}
	\end{itemize}
}
\end{enumerate}

\section*{meaning}
{\large \color{blue}  meanings  }
\subsection*{Explain}
\begin{enumerate}
\item variable noun \\
The \textbf{meaning} of a word, expression, or gesture is the thing or idea that it refers to or represents and which can be explained using other words.
 \textit{
	\begin{itemize}
	\item I hadn't a clue to the meaning of 'activism'.
	\item I became more aware of the symbols and their meanings.
	\end{itemize}
}
\item variable noun \\
The \textbf{meaning} of what someone says or of something such as a book or film is the thoughts or ideas that are intended to be expressed by it.
 \textit{
	\begin{itemize}
	\item Unsure of the meaning of this remark, Ryle chose to remain silent.
	\item Her book is not without autobiographical meaning.
	\end{itemize}
}
\item uncountable noun \\
If an activity or action has \textbf{meaning} , it has a purpose and is worthwhile .
 \textit{
	\begin{itemize}
	\item Art has real meaning when it helps people to understand themselves.
	\item ...a challenge that gives meaning to life.
	\end{itemize}
}
\item  \\
 not to know the meaning of the word \textit{
	\begin{itemize}
	\end{itemize}
}
\end{enumerate}

\section*{input}
{\large \color{blue}  inputs  inputting  }
\subsection*{Explain}
\begin{enumerate}
\item variable noun \\
\textbf{Input} consists of information or resources that a group or project  receives .
 \textit{
	\begin{itemize}
	\item We listen to our employees and value their input.
	\item Even large firms often fail to provide these necessary inputs for themselves.
	\end{itemize}
}
\item uncountable noun \\
\textbf{Input} is information that is put into a computer.
 \textit{
	\begin{itemize}
	\end{itemize}
}
\item verb \\
If you \textbf{input} information into a computer, you feed it in, for example by typing it on a keyboard .
 \textit{
	\begin{itemize}
	\item All this information had to be input onto the computer.
	\end{itemize}
}
\end{enumerate}

\section*{number}
{\large \color{blue}  numbers  numbering  numbered  }
\subsection*{Explain}
\begin{enumerate}
\item countable noun \\
A \textbf{number} is a word such as 'two', ' nine ', or ' twelve ', or a symbol such as 1, 3, or 47. You use numbers to say how many things you are referring to or where something comes in a series.
 \textit{
	\begin{itemize}
	\item No, I don't know the room number.
	\item Stan Laurel was born at number 3, Argyll Street.
	\item The number 47 bus leaves in 10 minutes.
	\end{itemize}
}
\item countable noun \\
You use \textbf{number} with words such as 'large' or 'small' to say approximately how many things or people there are.
 \textit{
	\begin{itemize}
	\item Quite a considerable number of interviews are going on.
	\item I have had an enormous number of letters from single parents.
	\item Growing numbers of people in the rural areas are too frightened to vote.
	\end{itemize}
}
\item singular noun \\
If there are \textbf{a}  \textbf{number}  \textbf{of} things or people, there are several of them. If there are \textbf{any}  \textbf{number}  \textbf{of} things or people, there is a large quantity of them.
 \textit{
	\begin{itemize}
	\item I seem to remember that Sam told a number of lies.
	\item There must be any number of people in my position.
	\end{itemize}
}
\item uncountable noun \\
You can refer to someone's or something's position in a list of the most successful or most popular of a particular type of thing as, for example , \textbf{number} one or \textbf{number} two.
 \textit{
	\begin{itemize}
	\item ...the 19-year-old Norwegian who is already ranked world number one.
	\item Before you knew it, the single was at Number 90 in the U.S. singles charts.
	\item His book is number two in the best-seller lists.
	\end{itemize}
}
\item verb \\
If a group of people or things \textbf{numbers} a particular total, that is how many there are.
 \textit{
	\begin{itemize}
	\item They told me that their village numbered 100.
	\item This time the dead were numbered in hundreds, not dozens.
	\end{itemize}
}
\item countable noun \\
A \textbf{number} is the series of numbers that you dial when you are making a phone call.
 \textit{
	\begin{itemize}
	\item Sarah sat down and dialled a number.
	\item ...a list of names and phone numbers.
	\item My number is 414-3925.
	\item 'You must have a wrong number,' she said. 'There's no one of that name here'.
	\end{itemize}
}
\item countable noun \\
You can refer to a short piece of music, a song , or a dance as a \textbf{number} .
 \textit{
	\begin{itemize}
	\item ...'Unforgettable', a number that was written and performed in 1951.
	\item Responsibility for the dance numbers was split between Robert Alton and the young
George Balanchine.
	\end{itemize}
}
\item verb \\
If someone or something \textbf{is numbered}  \textbf{among} a particular group, they are believed to belong in that group.
 \textit{
	\begin{itemize}
	\item The Leicester Swannington Railway is numbered among Britain's railway pioneers.
	\item He numbered several Americans among his friends.
	\end{itemize}
}
\item verb \\
If you \textbf{number} something, you mark it with a number, usually starting at 1.
 \textit{
	\begin{itemize}
	\item He cut his paper up into tiny squares, and he numbered each one.
	\item Each factor has been numbered.
	\end{itemize}
}
\item  \\
 someone's days are numbered \textit{
	\begin{itemize}
	\end{itemize}
}
\item  \\
 one of sb's number \textit{
	\begin{itemize}
	\end{itemize}
}
\item  \\
 the numbers game/racket \textit{
	\begin{itemize}
	\end{itemize}
}
\end{enumerate}

\section*{internet}
{\large \color{blue}  }
\subsection*{Explain}
\begin{enumerate}
\item proper noun \\
\textbf{The internet} is the computer network which allows computer users to connect with computers all over the world, and which carries email.
 \textit{
	\begin{itemize}
	\end{itemize}
}
\end{enumerate}

\section*{oath}
{\large \color{blue}  oaths  }
\subsection*{Explain}
\begin{enumerate}
\item countable noun \\
An \textbf{oath} is a formal  promise , especially a promise to be loyal to a person or country.
 \textit{
	\begin{itemize}
	\item He took an oath of loyalty to the government.
	\item He swore an oath promising to uphold and protect the country's laws and constitution.
	\end{itemize}
}
\item singular noun \\
In a court of law , when someone takes \textbf{the}  \textbf{oath} , they make a formal promise to tell the truth. You can say that someone is \textbf{on oath} or \textbf{under oath} when they have made this promise.
 \textit{
	\begin{itemize}
	\item His girlfriend had gone into the witness box and taken the oath.
	\item Under oath, he finally admitted that he had lied.
	\item Three officers gave evidence on oath against him.
	\end{itemize}
}
\item countable noun \\
An \textbf{oath} is an offensive or emphatic word or expression which you use when you are angry or shocked .
 \textit{
	\begin{itemize}
	\item Wellor let out a foul oath and hurled himself upon him.
	\end{itemize}
}
\end{enumerate}

\section*{kick}
{\large \color{blue}  kicks  kicking  kicked  }
\subsection*{Explain}
\begin{enumerate}
\item verb \\
If you \textbf{kick} someone or something, you hit them forcefully with your foot.
 \textbf{Kick} is also a noun .
 \textit{
	\begin{itemize}
	\item He kicked the door hard.
	\item He threw me to the ground and started to kick.
	\item He escaped by kicking open the window.
	\item The fiery actress kicked him in the shins.
	\item An ostrich can kick a man to death.
	\item He suffered a kick to the knee.
	\end{itemize}
}
\item verb \\
When you \textbf{kick} a ball or other object, you hit it with your foot so that it moves through the air.
 \textbf{Kick} is also a noun.
 \textit{
	\begin{itemize}
	\item I went to kick the ball and I completely missed it.
	\item He kicked the ball away.
	\item A furious player kicked his racket into the grandstand.
	\item After just one kick from England, the referee blew his whistle.
	\end{itemize}
}
\item verb \\
If you \textbf{kick} or if you \textbf{kick} your legs, you move your legs with very quick , small, and forceful movements, once or repeatedly.
 \textbf{Kick out} means the same as kick .
 \textit{
	\begin{itemize}
	\item They were dragged away struggling and kicking.
	\item First he kicked the left leg, then he kicked the right.
	\item He kicked his feet away from the window.
	\item As its rider tried to free it, the horse kicked out.
	\end{itemize}
}
\item verb \\
If you \textbf{kick} your legs, you lift your legs up very high one after the other, for example when you are dancing.
 \textit{
	\begin{itemize}
	\item He was kicking his legs like a Can Can dancer.
	\item She begins dancing, kicking her legs high in the air.
	\end{itemize}
}
\item verb \\
If you \textbf{kick} a habit , you stop doing something that is bad for you and that you find  difficult to stop doing.
 \textit{
	\begin{itemize}
	\item She's kicked her drug habit and learned that her life has value.
	\item I've kicked cigarettes and booze.
	\end{itemize}
}
\item singular noun \\
If something gives you \textbf{a kick} , it makes you feel very excited or very happy for a short period of time.
 \textit{
	\begin{itemize}
	\item I got a kick out of seeing my name in print.
	\end{itemize}
}
\item  \\
 kick you when you are down \textit{
	\begin{itemize}
	\end{itemize}
}
\item  \\
 for kicks \textit{
	\begin{itemize}
	\end{itemize}
}
\item  \\
 kicking and screaming \textit{
	\begin{itemize}
	\end{itemize}
}
\item  \\
 a kick in the teeth \textit{
	\begin{itemize}
	\end{itemize}
}
\item  \\
 kick oneself \textit{
	\begin{itemize}
	\end{itemize}
}
\end{enumerate}

\section*{obstacle}
{\large \color{blue}  obstacles  }
\subsection*{Explain}
\begin{enumerate}
\item countable noun \\
An \textbf{obstacle} is an object that makes it difficult for you to go where you want to go, because it is in your way.
 \textit{
	\begin{itemize}
	\item Most competition cars will only roll over if they hit an obstacle.
	\item He left her to navigate her own way round the trolleys and other obstacles.
	\end{itemize}
}
\item countable noun \\
You can refer to anything that makes it difficult for you to do something as an \textbf{obstacle} .
 \textit{
	\begin{itemize}
	\item Overcrowding remains a large obstacle to improving conditions.
	\item To succeed, you must learn to overcome obstacles.
	\end{itemize}
}
\end{enumerate}

\section*{opera}
{\large \color{blue}  operas  }
\subsection*{Explain}
\begin{enumerate}
\item variable noun \\
An \textbf{opera} is a play with music in which all the words are sung .
 \textit{
	\begin{itemize}
	\item ...a one-act opera about contemporary women in America.
	\item ...Donizetti's opera 'Lucia di Lammermoor'.
	\item ...an opera singer.
	\item He was also learned in classical music with a great love of opera.
	\end{itemize}
}
\end{enumerate}

\section*{lecture}
{\large \color{blue}  lectures  lecturing  lectured  }
\subsection*{Explain}
\begin{enumerate}
\item countable noun \\
A \textbf{lecture} is a talk someone gives in order to teach people about a particular subject, usually at a university or college .
 \textit{
	\begin{itemize}
	\item ...a series of lectures by Professor Eric Robinson.
	\item In his lecture Riemann covered an enormous variety of topics.
	\end{itemize}
}
\item verb \\
If you \textbf{lecture}  \textbf{on} a particular subject, you give a lecture or a series of lectures about it.
 \textit{
	\begin{itemize}
	\item She then invited him to Atlanta to lecture on the history of art.
	\item She has danced, choreographed, lectured and taught all over the world.
	\item Wendy Rigby was recently invited to lecture a group of doctors on the benefits of
aromatherapy.
	\end{itemize}
}
\item verb \\
If someone \textbf{lectures} you about something, they criticize you or tell you how they think you should behave .
 \textbf{Lecture} is also a noun .
 \textit{
	\begin{itemize}
	\item He used to lecture me about getting too much sun.
	\item Chuck would lecture me, telling me to get a haircut.
	\item She was no longer interrogating but lecturing.
	\item Our captain gave us a stern lecture on safety.
	\end{itemize}
}
\end{enumerate}

\section*{opinion}
{\large \color{blue}  opinions  }
\subsection*{Explain}
\begin{enumerate}
\item countable noun \\
Your \textbf{opinion} about something is what you think or believe about it.
 \textit{
	\begin{itemize}
	\item I wasn't asking for your opinion, Dick.
	\item He held the opinion that a government should think before introducing a tax.
	\item Most who expressed an opinion spoke favorably of Thomas.
	\end{itemize}
}
\item singular noun \\
Your \textbf{opinion}  \textbf{of} someone is your judgment of their character or ability .
 \textit{
	\begin{itemize}
	\item That improved Mrs Goole's already favourable opinion of him.
	\end{itemize}
}
\item uncountable noun \\
You can refer to the beliefs or views that people have as \textbf{opinion} .
 \textit{
	\begin{itemize}
	\item Some, I suppose, might even be in positions to influence opinion.
	\item There is a broad consensus of opinion about the policies which should be pursued.
	\end{itemize}
}
\item countable noun \\
An \textbf{opinion} from an expert is the advice or judgment that they give you in the subject that they know a lot about.
 \textit{
	\begin{itemize}
	\item Severe, prolonged pain, especially at rest, needs a medical opinion.
	\end{itemize}
}
\item  \\
 in sb's opinion \textit{
	\begin{itemize}
	\end{itemize}
}
\item  \\
 of the opinion \textit{
	\begin{itemize}
	\end{itemize}
}
\end{enumerate}

\section*{nurture}
{\large \color{blue}  nurtures  nurturing  nurtured  }
\subsection*{Explain}
\begin{enumerate}
\item verb \\
If you \textbf{nurture} something such as a young child or a young plant, you care for it while it is growing and developing .
 \textit{
	\begin{itemize}
	\item Parents want to know the best way to nurture and raise their child to adulthood.
	\item The modern conservatory is not an environment for nurturing plants.
	\end{itemize}
}
\item verb \\
If you \textbf{nurture}  plans , ideas , or people, you encourage them or help them to develop.
 \textit{
	\begin{itemize}
	\item She had always nurtured great ambitions for her son.
	\item ...parents whose political views were nurtured in the sixties.
	\item The University of Notre Dame has nurtured a number of top scholars who reject these
values.
	\end{itemize}
}
\item uncountable noun \\
\textbf{Nurture} is care that is given to someone while they are growing and developing.
 \textit{
	\begin{itemize}
	\item The human organism learns partly by nature, partly by nurture.
	\end{itemize}
}
\end{enumerate}

\section*{party}
{\large \color{blue}  parties  partying  partied  }
\subsection*{Explain}
\begin{enumerate}
\item countable noun \\
A \textbf{party} is a political organization whose members have similar aims and beliefs . Usually the organization tries to get its members elected to the government of a country.
 \textit{
	\begin{itemize}
	\item ...a member of the Labour party.
	\item ...India's ruling party.
	\item ...opposition parties.
	\item ...her resignation as party leader.
	\end{itemize}
}
\item countable noun \\
A \textbf{party} is a social event, often in someone's home , at which people enjoy themselves doing things such as eating , drinking, dancing, talking , or playing games.
 \textit{
	\begin{itemize}
	\item The couple met at a party.
	\item We threw a huge birthday party.
	\item Most teenagers like to go to parties.
	\end{itemize}
}
\item verb \\
If you \textbf{party} , you enjoy yourself doing things such as going out to parties, drinking, dancing, and talking to people.
 \textit{
	\begin{itemize}
	\item They come to eat and drink, to swim, to party. Sometimes they never go to bed.
	\item After a long evening of partying he looked tired.
	\end{itemize}
}
\item countable noun \\
A \textbf{party}  \textbf{of} people is a group of people who are doing something together, for example  travelling together.
 \textit{
	\begin{itemize}
	\item They became separated from their party.
	\item ...a party of sightseers.
	\item ...a research party of scientists.
	\end{itemize}
}
\item countable noun \\
One of the people involved in a legal agreement or dispute can be referred to as a particular \textbf{party} .
 \textit{
	\begin{itemize}
	\item It has to be proved that they are the guilty party.
	\item ...he was the injured party.
	\item ...a court, the decision of which may not satisfy either party.
	\end{itemize}
}
\item  \\
 be a party to sth/be party to sth \textit{
	\begin{itemize}
	\end{itemize}
}
\end{enumerate}

\section*{oxide}
{\large \color{blue}  oxides  }
\subsection*{Explain}
\begin{enumerate}
\item variable noun \\
An \textbf{oxide} is a compound of oxygen and another chemical element.
 \textit{
	\begin{itemize}
	\item ...molecules of iron oxide, or rust.
	\end{itemize}
}
\end{enumerate}

\section*{pattern}
{\large \color{blue}  patterns  }
\subsection*{Explain}
\begin{enumerate}
\item countable noun \\
A \textbf{pattern} is the repeated or regular way in which something happens or is done.
 \textit{
	\begin{itemize}
	\item All three attacks followed the same pattern.
	\item A change in the pattern of his breathing became apparent.
	\end{itemize}
}
\item countable noun \\
A \textbf{pattern} is an arrangement of lines or shapes, especially a design in which the same shape is repeated at regular intervals over a surface.
 \textit{
	\begin{itemize}
	\item ...a golden robe embroidered with red and purple thread stitched into a pattern of
flames.
	\end{itemize}
}
\item countable noun \\
A \textbf{pattern} is a diagram or shape that you can use as a guide when you are making something such
as a model or a piece of clothing.
 \textit{
	\begin{itemize}
	\item ...cutting out a pattern for trousers.
	\item Send for our free patterns to knit yourself.
	\item ...sewing patterns.
	\end{itemize}
}
\end{enumerate}

\section*{parade}
{\large \color{blue}  parades  parading  paraded  }
\subsection*{Explain}
\begin{enumerate}
\item countable noun \\
A \textbf{parade} is a procession of people or vehicles moving through a public place in order to celebrate an important  day or event .
 \textit{
	\begin{itemize}
	\item A military parade marched slowly and solemnly down Pennsylvania Avenue.
	\end{itemize}
}
\item verb \\
When people \textbf{parade}  somewhere , they walk together in a formal group or a line, usually with other people watching them.
 \textit{
	\begin{itemize}
	\item More than four thousand soldiers, sailors and airmen paraded down the Champs Elysees.
	\item Everybody was beginning to parade back to the village.
	\end{itemize}
}
\item variable noun \\
\textbf{Parade} is a formal occasion when soldiers  stand in lines to be seen by an officer or important person, or march in a group.
 \textit{
	\begin{itemize}
	\item He had them on parade at six o'clock in the morning.
	\item Morning parade was in progress on the parade ground.
	\end{itemize}
}
\item verb \\
If flags or statues  \textbf{are paraded} , they are carried in a procession.
 \textit{
	\begin{itemize}
	\item Banners were paraded from church to church on feast days.
	\end{itemize}
}
\item verb \\
If prisoners  \textbf{are paraded} through the streets of a town or on television , they are shown to the public, usually in order to make the people who are holding them seem more powerful or important.
 \textit{
	\begin{itemize}
	\item Five leading fighter pilots have been captured and paraded before the media.
	\end{itemize}
}
\item verb \\
If you say that someone \textbf{parades} a person, you mean that they show that person to others only in order to gain some advantage for themselves.
 \textit{
	\begin{itemize}
	\item Children are paraded on television alongside the party leaders to win votes.
	\end{itemize}
}
\item verb \\
If people \textbf{parade} something, they show it in public so that it can be admired .
 \textit{
	\begin{itemize}
	\item Valentino is keen to see celebrities parading his clothes at big occasions.
	\end{itemize}
}
\item verb \\
If someone \textbf{parades} , they walk about somewhere in order to be seen and admired.
 \textit{
	\begin{itemize}
	\item I love to put on a bathing suit and parade on the beach.
	\item They danced and paraded around.
	\end{itemize}
}
\item verb \\
If someone \textbf{parades} a real or pretended  feeling or quality, they draw  attention to themselves by displaying it.
 \textit{
	\begin{itemize}
	\item They parade their virtuous beliefs and hide their vices.
	\end{itemize}
}
\item verb \\
If you say that something \textbf{parades as} or \textbf{is paraded as} a good or important thing, you mean that some people say that it is good or important
but you think it probably is not.
 \textit{
	\begin{itemize}
	\item He paraded his cut in interest rates as a small victory.
	\item ...all the fashions that parade as modern movements in art.
	\end{itemize}
}
\item countable noun \\
If you talk about a \textbf{parade of} people or things, you mean that there is a series of them that seems never to end.
 \textit{
	\begin{itemize}
	\item When I ask Nick about his childhood, he remembers a parade of baby-sitters.
	\item ...an endless parade of advertisements.
	\end{itemize}
}
\item countable noun \\
A \textbf{parade} is a short row of shops, usually set back from the main street.
 \textit{
	\begin{itemize}
	\end{itemize}
}
\item countable noun \\
\textbf{Parade} is used as part of the name of a street.
 \textit{
	\begin{itemize}
	\item ...Queens Hotel, Clarence Parade, Southsea.
	\end{itemize}
}
\end{enumerate}

\section*{pledge}
{\large \color{blue}  pledges  pledging  pledged  }
\subsection*{Explain}
\begin{enumerate}
\item countable noun \\
When someone makes a \textbf{pledge} , they make a serious promise that they will do something.
 \textit{
	\begin{itemize}
	\item The meeting ended with a pledge to step up cooperation between the six states of
the region.
	\item ...a £1.1m pledge of support from the Spanish ministry of culture.
	\end{itemize}
}
\item verb \\
When someone \textbf{pledges}  \textbf{to} do something, they promise in a serious way to do it. When they \textbf{pledge} something, they promise to give it.
 \textit{
	\begin{itemize}
	\item Mr Dudley has pledged to give any award to charity.
	\item Philip pledges support and offers to help in any way that he can.
	\item I pledge that by next year we will have the problem solved.
	\end{itemize}
}
\item verb \\
If you \textbf{pledge} a sum of money to an organization or activity, you promise to pay that amount of money
to it at a particular time or over a particular period.
 \textbf{Pledge} is also a noun .
 \textit{
	\begin{itemize}
	\item The French President is pledging $150 million in French aid next year.
	\item The government has now pledged £170m over the next six years for improving primary
care.
	\item ...a pledge of forty-two million dollars a month.
	\end{itemize}
}
\item verb \\
If you \textbf{pledge}  \textbf{yourself to} something, you commit yourself to following a particular course of action or to supporting a particular person, group, or idea.
 \textit{
	\begin{itemize}
	\item He has pledged himself to assist Mr. Williams with money.
	\item He has pledged himself to Everton Football Club for another three years.
	\item The treaties renounce the use of force and pledge the two countries to co-operation.
	\end{itemize}
}
\item verb \\
If you \textbf{pledge} something such as a valuable  possession or a sum of money, you leave it with someone as a guarantee that you will repay money that you have borrowed .
 \textit{
	\begin{itemize}
	\item He asked her to pledge the house as security for a loan.
	\end{itemize}
}
\end{enumerate}

\section*{pirate}
{\large \color{blue}  pirates  pirating  pirated  }
\subsection*{Explain}
\begin{enumerate}
\item countable noun \\
\textbf{Pirates} are sailors who attack other ships and steal  property from them.
 \textit{
	\begin{itemize}
	\item In the nineteenth century, pirates roamed the seas.
	\end{itemize}
}
\item countable noun \\
You can  refer to someone who behaves in an immoral or illegal  way as a \textbf{pirate} .
 \textit{
	\begin{itemize}
	\item Of course I knew Max was a rogue, a bit of a pirate.
	\end{itemize}
}
\item verb \\
Someone who \textbf{pirates}  films , books , files or computer  programs  copies and sells them when they have no right to do so.
 \textit{
	\begin{itemize}
	\item He was accused of pirating music, movies and other web material.
	\item ...American manufacturers who've seen their designs pirated in other countries.
	\end{itemize}
}
\item adjective \\
A \textbf{pirate}  version of something is an illegal copy of it.
 \textit{
	\begin{itemize}
	\item Pirate copies can be downloaded easily.
	\end{itemize}
}
\end{enumerate}

\section*{plug}
{\large \color{blue}  plugs  plugging  plugged  }
\subsection*{Explain}
\begin{enumerate}
\item countable noun \\
A \textbf{plug} on a piece of electrical equipment is a small plastic object with two or three metal pins which fit into the holes of an electric socket and connects the equipment to the electricity supply.
 \textit{
	\begin{itemize}
	\end{itemize}
}
\item countable noun \\
A \textbf{plug} is an electric socket.
 \textit{
	\begin{itemize}
	\end{itemize}
}
\item countable noun \\
A \textbf{plug} is a thick, circular piece of rubber or plastic that you use to block the hole in a bath or sink when it is filled with water.
 \textit{
	\begin{itemize}
	\item She put the plug in the sink and filled it with cold water.
	\end{itemize}
}
\item countable noun \\
A \textbf{plug} is a small, round piece of wood, plastic, or wax which is used to block holes.
 \textit{
	\begin{itemize}
	\item A plug had been inserted in the drill hole.
	\end{itemize}
}
\item verb \\
If you \textbf{plug} a hole, you block it with something.
 \textit{
	\begin{itemize}
	\item Crews are working to plug a major oil leak.
	\end{itemize}
}
\item verb \\
If someone \textbf{plugs} a commercial product, especially a book or a film, they praise it in order to encourage people to buy it or see it because they have an interest in it doing well.
 \textbf{Plug} is also a noun .
 \textit{
	\begin{itemize}
	\item We did not want people on the show who are purely interested in plugging a book or
film.
	\item Let's do this show tonight and it'll be a great plug, a great promotion.
	\end{itemize}
}
\item  \\
 pull the plug \textit{
	\begin{itemize}
	\end{itemize}
}
\end{enumerate}

\section*{rape}
{\large \color{blue}  rapes  raping  raped  }
\subsection*{Explain}
\begin{enumerate}
\item verb \\
If someone \textbf{is raped} , they are forced to have sex , usually by violence or threats of violence.
 \textit{
	\begin{itemize}
	\item A young woman was brutally raped in her own home.
	\item They'd held him down and raped him.
	\end{itemize}
}
\item variable noun \\
\textbf{Rape} is the crime of forcing someone to have sex.
 \textit{
	\begin{itemize}
	\item Almost ninety per cent of all rapes and violent assaults went unreported.
	\end{itemize}
}
\item singular noun \\
\textbf{The rape of} an area or of a country is the destruction or spoiling of it.
 \textit{
	\begin{itemize}
	\item As a result of the rape of the forests, parts of the country are now short of water.
	\end{itemize}
}
\item  \\
\textbf{Rape} is a plant with yellow flowers which is grown as a crop . Its seeds are crushed to make cooking oil.
 \textit{
	\begin{itemize}
	\end{itemize}
}
\end{enumerate}

\section*{satisfaction}
{\large \color{blue}  }
\subsection*{Explain}
\begin{enumerate}
\item uncountable noun \\
\textbf{Satisfaction} is the pleasure that you feel when you do something or get something that you wanted or needed to do or get.
 \textit{
	\begin{itemize}
	\item She felt a small glow of satisfaction.
	\item Both sides expressed satisfaction with the progress so far.
	\item I doubt I'll ever get rich, but I get job satisfaction.
	\end{itemize}
}
\item uncountable noun \\
If you get \textbf{satisfaction} from someone, you get money or an apology from them because you have been treated badly .
 \textit{
	\begin{itemize}
	\item If you can't get any satisfaction, complain to the park owner.
	\end{itemize}
}
\item  \\
 to someone's satisfaction \textit{
	\begin{itemize}
	\end{itemize}
}
\end{enumerate}

\section*{register}
{\large \color{blue}  registers  registering  registered  }
\subsection*{Explain}
\begin{enumerate}
\item countable noun \\
A \textbf{register} is an official list or record of people or things.
 \textit{
	\begin{itemize}
	\item ...registers of births, deaths and marriages.
	\item He signed the register at the hotel.
	\item She calls the register for her class of thirty 12 year olds.
	\end{itemize}
}
\item verb \\
If you \textbf{register} to do something, you put your name on an official list, in order to be able to do
that thing or to receive a service.
 \textit{
	\begin{itemize}
	\item Have you come to register at the school?
	\item Thousands lined up to register to vote.
	\item Many students register for these courses to widen skills for use in their current
job.
	\item About 26 million people are not registered with a dentist.
	\item ...registered voters.
	\end{itemize}
}
\item verb \\
If you \textbf{register} something, such as the name of a person who has just died or information about something you own, you have these facts recorded on an official list.
 \textit{
	\begin{itemize}
	\item The council said the car was not registered.
	\item We registered his birth.
	\item The house is registered in her name.
	\item ...a registered charity.
	\end{itemize}
}
\item verb \\
When something \textbf{registers}  \textbf{on} a scale or measuring instrument, it shows on the scale or instrument. You can also  say that something \textbf{registers} a certain amount or level \textbf{on} a scale or measuring instrument.
 \textit{
	\begin{itemize}
	\item It will only register on sophisticated X-ray equipment.
	\item The earthquake registered 5.3 points on the Richter scale.
	\item The scales registered a gain of 1.3 kilograms.
	\end{itemize}
}
\item verb \\
If you \textbf{register} your feelings or opinions about something, you do something that makes them clear to other people.
 \textit{
	\begin{itemize}
	\item Voters wish to register their dissatisfaction with the ruling party.
	\item Workers stopped work to register their protest.
	\end{itemize}
}
\item verb \\
If a feeling \textbf{registers}  \textbf{on} someone's face, their expression shows clearly that they have that feeling.
 \textit{
	\begin{itemize}
	\item Surprise again registered on Rodney's face.
	\end{itemize}
}
\item verb \\
If a piece of information does not \textbf{register} or if you do not \textbf{register} it, you do not really pay attention to it, and so you do not remember it or react to it.
 \textit{
	\begin{itemize}
	\item What I said sometimes didn't register in her brain.
	\item The sound was so familiar that she didn't register it.
	\end{itemize}
}
\item countable noun \\
If you sing or play something in a high or low \textbf{register} , you sing, or play it using high or low notes. If you say something in a high or
low \textbf{register} , you say it in a high or low voice.
 \textit{
	\begin{itemize}
	\end{itemize}
}
\item variable noun \\
In linguistics , the \textbf{register} of a piece of speech or writing is its level and style of language, which is usually appropriate to the situation or circumstances in which it is used.
 \textit{
	\begin{itemize}
	\end{itemize}
}
\end{enumerate}

\section*{seam}
{\large \color{blue}  seams  }
\subsection*{Explain}
\begin{enumerate}
\item countable noun \\
A \textbf{seam} is a line of stitches which joins two pieces of cloth together.
 \textit{
	\begin{itemize}
	\end{itemize}
}
\item countable noun \\
A \textbf{seam} of coal is a long, narrow  layer of it underneath the ground.
 \textit{
	\begin{itemize}
	\item The average U.K. coal seam is one metre thick.
	\end{itemize}
}
\item  \\
 come apart at the seams/fall apart at the seams \textit{
	\begin{itemize}
	\end{itemize}
}
\item  \\
 bursting at the seams \textit{
	\begin{itemize}
	\end{itemize}
}
\end{enumerate}

\section*{reign}
{\large \color{blue}  reigns  reigning  reigned  }
\subsection*{Explain}
\begin{enumerate}
\item verb \\
If you say , for example , that silence  \textbf{reigns} in a place or confusion  \textbf{reigns} in a situation , you mean that the place is silent or the situation is confused .
 \textit{
	\begin{itemize}
	\item Confusion reigned about how the debate would end.
	\item A relative calm reigned over the city.
	\end{itemize}
}
\item verb \\
When a king or queen  \textbf{reigns} , he or she rules a country.
 \textbf{Reign} is also a noun .
 \textit{
	\begin{itemize}
	\item ...Henry II, who reigned from 1154 to 1189.
	\item ...George III, Britain's longest reigning monarch.
	\item ...Queen Victoria's reign.
	\end{itemize}
}
\item verb \\
If you say that a person \textbf{reigns} in a situation or area, you mean that they are very powerful or successful .
 \textbf{Reign} is also a noun.
 \textit{
	\begin{itemize}
	\item He reigned as the male sex symbol of the 1950s.
	\item ...the girls that reigned over 1960s pop.
	\item ...Klitschko's 11-year reign as a world champion.
	\end{itemize}
}
\item  \\
 sb/sth reigns supreme \textit{
	\begin{itemize}
	\end{itemize}
}
\item  \\
 reign of terror \textit{
	\begin{itemize}
	\end{itemize}
}
\end{enumerate}

\section*{significance}
{\large \color{blue}  }
\subsection*{Explain}
\begin{enumerate}
\item uncountable noun \\
The \textbf{significance} of something is the importance that it has, usually because it will have an effect on a situation or shows something about a situation.
 \textit{
	\begin{itemize}
	\item Ideas about the social significance of religion have changed over time.
	\item A sacred site might be a mountain that is of some significance to a tribe.
	\end{itemize}
}
\end{enumerate}

\section*{renaissance}
{\large \color{blue}  }
\subsection*{Explain}
\begin{enumerate}
\item proper noun \\
\textbf{The Renaissance} was the period in Europe , especially  Italy , in the 14th, 15th, and 16th centuries , when there was a new interest in art, literature , science, and learning.
 \textit{
	\begin{itemize}
	\item ...the Renaissance masterpieces in London's galleries.
	\item Science took a new and different turn in the Renaissance.
	\end{itemize}
}
\item singular noun \\
If something experiences a \textbf{renaissance} , it becomes popular or successful again after a time when people were not interested in it.
 \textit{
	\begin{itemize}
	\item Popular art is experiencing a renaissance.
	\item They gathered to protest against the renaissance of the extreme right.
	\end{itemize}
}
\end{enumerate}

\section*{song}
{\large \color{blue}  songs  }
\subsection*{Explain}
\begin{enumerate}
\item countable noun \\
A \textbf{song} is words sung to a tune .
 \textit{
	\begin{itemize}
	\item ...a voice singing a Spanish song.
	\item ...a love song.
	\end{itemize}
}
\item uncountable noun \\
\textbf{Song} is the art of singing.
 \textit{
	\begin{itemize}
	\item ...dance, music, mime and song.
	\item ...the history of American popular song.
	\end{itemize}
}
\item countable noun \\
A bird's \textbf{song} is the pleasant , musical sounds that it makes.
 \textit{
	\begin{itemize}
	\item It's been a long time since I heard a blackbird's song in the evening.
	\end{itemize}
}
\item  \\
 break into song/burst into song \textit{
	\begin{itemize}
	\end{itemize}
}
\item  \\
 for a song \textit{
	\begin{itemize}
	\end{itemize}
}
\item  \\
 on song \textit{
	\begin{itemize}
	\end{itemize}
}
\end{enumerate}

\section*{respect}
{\large \color{blue}  respects  respecting  respected  }
\subsection*{Explain}
\begin{enumerate}
\item verb \\
If you \textbf{respect} someone, you have a good opinion of their character or ideas .
 \textit{
	\begin{itemize}
	\item I want him to respect me as a career woman.
	\item He needs the advice of people he respects, and he respects you.
	\end{itemize}
}
\item uncountable noun \\
If you have \textbf{respect}  \textbf{for} someone, you have a good opinion of them.
 \textit{
	\begin{itemize}
	\item I have tremendous respect for Dean.
	\item His voice was warm with friendship and respect.
	\end{itemize}
}
\item verb \\
If you \textbf{respect} someone's wishes , rights , or customs , you avoid doing things that they would dislike or regard as wrong .
 \textit{
	\begin{itemize}
	\item Finally, trying to respect her wishes, I said I'd leave.
	\end{itemize}
}
\item uncountable noun \\
If you show \textbf{respect}  \textbf{for} someone's wishes, rights, or customs, you avoid doing anything they would dislike
or regard as wrong.
 \textit{
	\begin{itemize}
	\item They will campaign for the return of traditional lands and respect for aboriginal
rights and customs.
	\end{itemize}
}
\item verb \\
If you \textbf{respect} a law or moral  principle , you agree not to break it.
 \textbf{Respect} is also a noun .
 \textit{
	\begin{itemize}
	\item It is about time tour operators respected the law and their own code of conduct.
	\item ...respect for the law and the rejection of the use of violence.
	\end{itemize}
}
\item phrase \\
You can say  \textbf{with respect} when you are politely disagreeing with someone or criticizing them.
 \textit{
	\begin{itemize}
	\item With respect, Minister, you still haven't answered my question.
	\item With respect, I hardly think that's the point.
	\end{itemize}
}
\item  \\
 pay one's respects \textit{
	\begin{itemize}
	\end{itemize}
}
\item  \\
 pay one's last respects \textit{
	\begin{itemize}
	\end{itemize}
}
\item  \\
 in this respect/in many respects \textit{
	\begin{itemize}
	\end{itemize}
}
\item  \\
 with respect to/in respect of \textit{
	\begin{itemize}
	\end{itemize}
}
\end{enumerate}

\section*{source}
{\large \color{blue}  sources  sourcing  sourced  }
\subsection*{Explain}
\begin{enumerate}
\item countable noun \\
The \textbf{source}  \textbf{of} something is the person, place, or thing which you get it from.
 \textit{
	\begin{itemize}
	\item ...over 40 per cent of British adults use television as their major source of information
about the arts.
	\item Renewable sources of energy must be used where practical.
	\item Tourism, which is a major source of income for the city, may be seriously affected.
	\end{itemize}
}
\item verb \\
In business, if a person or firm  \textbf{sources} a product or a raw material, they find someone who will  supply it.
 \textit{
	\begin{itemize}
	\item Together they travel the world, sourcing clothes for the small, privately owned company.
	\item About 60 per cent of an average car is sourced from outside of the manufacturer.
	\item ...furniture sourced from all over the world.
	\end{itemize}
}
\item countable noun \\
A \textbf{source} is a person or book that provides information for a news story or for a piece of
 research .
 \textit{
	\begin{itemize}
	\item Military sources say the boat was heading south at high speed.
	\item She quotes secondary and primary sources without distinction.
	\end{itemize}
}
\item countable noun \\
The \textbf{source of} a difficulty is its cause.
 \textit{
	\begin{itemize}
	\item This gave me a clue as to the source of the problem.
	\end{itemize}
}
\item countable noun \\
The \textbf{source} of a river or stream is the place where it begins .
 \textit{
	\begin{itemize}
	\item ...the source of the Tiber.
	\end{itemize}
}
\end{enumerate}

\section*{shake}
{\large \color{blue}  shakes  shaking  shook  shaken  }
\subsection*{Explain}
\begin{enumerate}
\item verb \\
If you \textbf{shake} something, you hold it and move it quickly backwards and forwards or up and down. You can also  \textbf{shake} a person, for example , because you are angry with them or because you want them to wake up.
 \textbf{Shake} is also a noun .
 \textit{
	\begin{itemize}
	\item The nurse shook the thermometer and put it under my armpit.
	\item Shake the rugs well and hang them for a few hours before replacing on the floor.
	\item She picked up the bag of salad and gave it a shake.
	\end{itemize}
}
\item verb \\
If you \textbf{shake}  \textbf{yourself} or your body, you make a lot of quick, small, repeated movements without moving from the place where you are.
 \textbf{Shake} is also a noun.
 \textit{
	\begin{itemize}
	\item As soon as he got inside, the dog shook himself.
	\item He shook his hands to warm them up.
	\item Take some slow, deep breaths and give your body a bit of a shake.
	\end{itemize}
}
\item verb \\
If you \textbf{shake} your \textbf{head} , you turn it from side to side in order to say 'no' or to show  disbelief or sadness.
 \textbf{Shake} is also a noun.
 \textit{
	\begin{itemize}
	\item 'Anything else?' Colum asked. Kathryn shook her head wearily.
	\item We were amazed, shocked, dumbfounded, shaking our heads in disbelief.
	\item Palmer gave a sad shake of his head.
	\end{itemize}
}
\item verb \\
If you \textbf{are shaking} , or a part of your body \textbf{is shaking} , you are making quick, small movements that you cannot control, for example because
you are cold or afraid .
 \textit{
	\begin{itemize}
	\item He roared with laughter, shaking in his chair.
	\item My hand shook so much that I could hardly hold the microphone.
	\item I stood there, crying and shaking with fear.
	\end{itemize}
}
\item plural noun \\
If you have \textbf{the shakes} , your body is shaking a lot because you are afraid or ill , or because you have drunk too much alcohol .
 \textit{
	\begin{itemize}
	\item I felt dizzy and had the shakes.
	\end{itemize}
}
\item verb \\
If you \textbf{shake} your fist or an object such as a stick  \textbf{at} someone, you wave it in the air in front of them because you are angry with them.
 \textit{
	\begin{itemize}
	\item The colonel rushed up to Earle, shaking his gun at him.
	\item The protesters burst through police lines into the cathedral square, shaking clenched
fists.
	\end{itemize}
}
\item verb \\
If a force \textbf{shakes} something, or if something \textbf{shakes} , it moves from side to side or up and down with quick, small, but sometimes  violent movements.
 \textit{
	\begin{itemize}
	\item ...an explosion that shook buildings several kilometers away.
	\item The hiccups may shake your baby's body from head to foot.
	\item The breeze grew in strength, the flags shook, plastic bunting creaked.
	\end{itemize}
}
\item verb \\
To \textbf{shake} something into a certain place or state means to bring it into that place or state
by moving it quickly up and down or from side to side.
 \textit{
	\begin{itemize}
	\item Small insects can be collected by shaking them into a jar.
	\item She frees her mass of hair from a rubber band and shakes it off her shoulders.
	\item Shake off any excess flour before putting the liver in the pan.
	\item The prop shaft vibrated like mad and shook the exhaust mounting loose.
	\end{itemize}
}
\item verb \\
If your voice  \textbf{is shaking} , you cannot control it properly and it sounds very unsteady , for example because you are nervous or angry.
 \textit{
	\begin{itemize}
	\item His voice shaking with rage, he asked why the report was kept from the public.
	\end{itemize}
}
\item verb \\
If an event or a piece of news  \textbf{shakes} you, or \textbf{shakes} your confidence , it makes you feel upset and unable to think calmly.
 \textit{
	\begin{itemize}
	\item The news of Tandy's escape had shaken them all.
	\item She was close to both of her parents and was undeniably shaken by their divorce.
	\item Your optimism has been badly shaken over the past months.
	\end{itemize}
}
\item verb \\
If an event \textbf{shakes} a group of people or their beliefs , it causes great uncertainty and makes them question their beliefs.
 \textit{
	\begin{itemize}
	\item It won't shake the football world if we beat Torquay.
	\item When events happen that shake these beliefs, our fear takes control.
	\item The reforms aim to win back confidence in a system shaken by a major scandal.
	\end{itemize}
}
\item verb \\
If you \textbf{shake} someone \textbf{out of} an attitude or belief that you disapprove of, you cause them to change their attitude or belief to one that is more responsible or sensible .
 \textit{
	\begin{itemize}
	\item No amount of reasoning could shake him out of his conviction.
	\item Many businessmen still find it hard to shake themselves out of the old state-dependent
habit.
	\end{itemize}
}
\item countable noun \\
A \textbf{shake} is the same as a milkshake .
 \textit{
	\begin{itemize}
	\item He sent his driver to fetch him a strawberry shake.
	\end{itemize}
}
\item  \\
 a fair shake \textit{
	\begin{itemize}
	\end{itemize}
}
\item  \\
 no great shakes \textit{
	\begin{itemize}
	\end{itemize}
}
\item  \\
 to shake someone's hand \textit{
	\begin{itemize}
	\end{itemize}
}
\item  \\
 to shake hands \textit{
	\begin{itemize}
	\end{itemize}
}
\end{enumerate}

\section*{statue}
{\large \color{blue}  statues  }
\subsection*{Explain}
\begin{enumerate}
\item countable noun \\
A \textbf{statue} is a large sculpture of a person or an animal, made of stone or metal.
 \textit{
	\begin{itemize}
	\end{itemize}
}
\end{enumerate}

\section*{snatch}
{\large \color{blue}  snatches  snatching  snatched  }
\subsection*{Explain}
\begin{enumerate}
\item verb \\
If you \textbf{snatch} something or \textbf{snatch}  \textbf{at} something, you take it or pull it away quickly.
 \textit{
	\begin{itemize}
	\item Mick snatched the cards from Archie's hand.
	\item He snatched up the phone.
	\item The thin wind snatched at her skirt.
	\end{itemize}
}
\item verb \\
If something \textbf{is snatched} from you, it is stolen , usually using force . If a person \textbf{is snatched} , they are taken away by force.
 \textit{
	\begin{itemize}
	\item If your bag is snatched, let it go.
	\item Mr Hillman was snatched by kidnappers last Thursday.
	\end{itemize}
}
\item verb \\
If you \textbf{snatch} an opportunity , you take it quickly. If you \textbf{snatch} something to eat or a rest , you have it quickly in between doing other things.
 \textit{
	\begin{itemize}
	\item I snatched a glance at the mirror.
	\item You can even snatch a few hours off.
	\item He was going out for a run, then snatching a piece of toast and a cup of coffee.
	\end{itemize}
}
\item verb \\
If you \textbf{snatch}  victory in a competition , you defeat your opponent by a small amount or just before the end of the contest .
 \textit{
	\begin{itemize}
	\item The American came from behind to snatch victory by a mere eight seconds.
	\item Chesterfield snatched a third goal.
	\end{itemize}
}
\item countable noun \\
A \textbf{snatch}  \textbf{of} a conversation or a song is a very small piece of it.
 \textit{
	\begin{itemize}
	\item I heard snatches of the conversation.
	\end{itemize}
}
\item  \\
 to snatch defeat from the jaws of victory \textit{
	\begin{itemize}
	\end{itemize}
}
\end{enumerate}

\section*{summary}
{\large \color{blue}  summaries  }
\subsection*{Explain}
\begin{enumerate}
\item countable noun \\
A \textbf{summary}  \textbf{of} something is a short account of it, which gives the main points but not the details .
 \textit{
	\begin{itemize}
	\item What follows is a brief summary of the process.
	\item Here's a summary of the day's news.
	\item Milligan gives a fair summary of his subject within a relatively short space.
	\end{itemize}
}
\item adjective \\
\textbf{Summary} actions are done without delay, often when something else should have been done first
or done instead .
 \textit{
	\begin{itemize}
	\item It says torture and summary execution are common.
	\item There is no doubt that some considered that a beating was no more than summary justice.
	\item The four men were killed after a summary trial.
	\end{itemize}
}
\end{enumerate}

\section*{strike}
{\large \color{blue}  strikes  striking  struck  stricken  }
\subsection*{Explain}
\begin{enumerate}
\item countable noun \\
When there is a \textbf{strike} , workers stop doing their work for a period of time, usually in order to try to get
 better pay or conditions for themselves.
 \textit{
	\begin{itemize}
	\item French air traffic controllers have begun a three-day strike in a dispute over pay.
	\item Staff at the hospital went on strike in protest at the incidents.
	\item ...a call for strike action.
	\end{itemize}
}
\item verb \\
When workers \textbf{strike} , they go on strike.
 \textit{
	\begin{itemize}
	\item ...their recognition of the workers' right to strike.
	\item They shouldn't be striking for more money.
	\item The government agreed not to sack any of the striking workers.
	\end{itemize}
}
\item verb \\
If you \textbf{strike} someone or something, you deliberately hit them.
 \textit{
	\begin{itemize}
	\item She took two quick steps forward and struck him across the mouth.
	\item He struck the ball straight into the hospitality tents.
	\item I struck it away and got a bite on my forearm.
	\item It is impossible to say who struck the fatal blow.
	\end{itemize}
}
\item verb \\
If something that is falling or moving \textbf{strikes} something, it hits it.
 \textit{
	\begin{itemize}
	\item His head struck the bottom when he dived into the 6ft end of the pool.
	\item One 16-inch shell struck the control tower.
	\item ...the fire which began when the installation was struck by lightning.
	\end{itemize}
}
\item verb \\
If you \textbf{strike} one thing against another, or if one thing \textbf{strikes} against another, the first thing hits the second thing.
 \textit{
	\begin{itemize}
	\item Wilde fell and struck his head on the stone floor.
	\item My right toe struck against a submerged rock.
	\end{itemize}
}
\item verb \\
If something such as an illness or disaster  \textbf{strikes} , it suddenly happens .
 \textit{
	\begin{itemize}
	\item Bank of England officials continued to insist that the pound would soon return to
stability but disaster struck.
	\item Both of them were afflicted with a rare genetic disease, which struck in their thirties.
	\item A powerful earthquake struck the island early this morning.
	\item He was suddenly struck with such a sense of grief, of loss, that his eyes filled
with tears.
	\end{itemize}
}
\item verb \\
To \textbf{strike} means to attack someone or something quickly and violently.
 \textit{
	\begin{itemize}
	\item The attacker struck as she was walking near the town centre.
	\item The killer says he will strike again.
	\item Then the scorpion struck.
	\end{itemize}
}
\item countable noun \\
A military \textbf{strike} is a military attack, especially an air attack.
 \textit{
	\begin{itemize}
	\item ...a punitive air strike.
	\item ...a nuclear strike.
	\item ...strategic strikes against enemy forces.
	\end{itemize}
}
\item verb \\
If something \textbf{strikes at} the heart or root of something, it attacks or conflicts with the basic elements or
principles of that thing.
 \textit{
	\begin{itemize}
	\item ...a rejection of her core beliefs and values, which strikes at the very heart of
her being.
	\item The issue strikes at the very foundation of our community.
	\end{itemize}
}
\item verb \\
If an idea or thought \textbf{strikes} you, it suddenly comes into your mind.
 \textit{
	\begin{itemize}
	\item A thought struck her. Was she jealous of her mother, then?
	\item At this point, it suddenly struck me that I was wasting my time.
	\end{itemize}
}
\item verb \\
If something \textbf{strikes} you \textbf{as} being a particular thing, it gives you the impression of being that thing.
 \textit{
	\begin{itemize}
	\item He struck me as a very serious but friendly person.
	\item What struck me as interesting is how much we judge other people by the clothes they
wear.
	\item You've always struck me as being an angry man.
	\end{itemize}
}
\item verb \\
If you \textbf{are struck} by something, you think it is very impressive , noticeable , or interesting.
 \textit{
	\begin{itemize}
	\item She was struck by his simple, spellbinding eloquence.
	\item Theresa was struck by her own lack of forethought.
	\item What struck me about the firm is how genuinely friendly and informal it is.
	\end{itemize}
}
\item verb \\
If you \textbf{strike} a deal or a bargain with someone, you come to an agreement with them.
 \textit{
	\begin{itemize}
	\item They struck a deal with their paper supplier, getting two years of newsprint on credit.
	\item The two struck a deal in which Rendell took half of what a manager would.
	\item He insists he has struck no bargains for their release.
	\end{itemize}
}
\item verb \\
If you \textbf{strike} a balance, you do something that is halfway between two extremes.
 \textit{
	\begin{itemize}
	\item At times like that you have to strike a balance between sleep and homework.
	\end{itemize}
}
\item verb \\
If you \textbf{strike} a pose or attitude, you put yourself in a particular position, for example when someone
is taking your photograph.
 \textit{
	\begin{itemize}
	\item She struck a pose, one hand on her hip.
	\end{itemize}
}
\item verb \\
If something \textbf{strikes} fear \textbf{into} people, it makes them very frightened or anxious .
 \textit{
	\begin{itemize}
	\item His name strikes fear into the hearts of his opponents .
	\end{itemize}
}
\item verb \\
If you \textbf{are struck}  dumb or blind , you suddenly become unable to speak or to see.
 \textit{
	\begin{itemize}
	\item I was struck dumb by this and had to think it over for a moment.
	\item For this revelation he was struck blind by the goddess Hera.
	\end{itemize}
}
\item verb \\
When a clock \textbf{strikes} , its bells make a sound to indicate what the time is.
 \textit{
	\begin{itemize}
	\item The clock struck nine.
	\item Finally, the clock strikes.
	\end{itemize}
}
\item verb \\
If you \textbf{strike} words \textbf{from} a document or an official record, you remove them.
 \textbf{Strike out} means the same as strike .
 \textit{
	\begin{itemize}
	\item Strike that from the minutes.
	\item Her achievements were struck from the record book.
	\item The censor struck out the next two lines.
	\end{itemize}
}
\item verb \\
When you \textbf{strike} a match, you make it produce a flame by moving it quickly against something rough.
 \textit{
	\begin{itemize}
	\item Robina struck a match and held it to the crumpled newspaper in the grate.
	\end{itemize}
}
\item verb \\
If someone \textbf{strikes} oil or gold, they discover it in the ground as a result of mining or drilling .
 \textit{
	\begin{itemize}
	\item Hamilton Oil announced that it had struck oil in the Liverpool Bay area of the Irish
Sea.
	\end{itemize}
}
\item verb \\
When a coin or medal \textbf{is struck} , it is made.
 \textit{
	\begin{itemize}
	\item Another medal was specially struck for him.
	\end{itemize}
}
\item countable noun \\
If someone has \textbf{two strikes against} them, things cause them to be in a bad situation or at a disadvantage .
 \textit{
	\begin{itemize}
	\item The Hotel has two strikes against it. One, it's an immense ugly concrete building.
Second, it lies in a rather awkward position.
	\item When I got out I couldn't find any work, and for being an ex-con, that was a strike
against me.
	\end{itemize}
}
\item  \\
 within striking distance \textit{
	\begin{itemize}
	\end{itemize}
}
\item  \\
 to strike gold \textit{
	\begin{itemize}
	\end{itemize}
}
\item  \\
 strike it rich \textit{
	\begin{itemize}
	\end{itemize}
}
\end{enumerate}

\section*{tap}
{\large \color{blue}  taps  tapping  tapped  }
\subsection*{Explain}
\begin{enumerate}
\item countable noun \\
A \textbf{tap} is a device that controls the flow of a liquid or gas from a pipe or container , for example on a sink .
 \textit{
	\begin{itemize}
	\item She turned on the taps.
	\item ...a cold-water tap.
	\item The honey runs out of a tap at the bottom of the drum.
	\end{itemize}
}
\item verb \\
If you \textbf{tap} something, you hit it with a quick light blow or a series of quick light blows.
 \textbf{Tap} is also a noun .
 \textit{
	\begin{itemize}
	\item He tapped the table to still the shouts of protest.
	\item Tap the eggs gently with a teaspoon to crack the shells.
	\item Grace tapped on the bedroom door and went in.
	\item There was a comfortable-looking clerk on duty, tapping away on a manual typewriter.
	\item To hold the carpet in place, it's a good idea to tap in a few nails temporarily.
	\item A tap on the door interrupted him and Sally Pierce came in.
	\end{itemize}
}
\item verb \\
If you \textbf{tap} your fingers or feet, you make a regular  pattern of sound by hitting a surface lightly and repeatedly with them, especially while you are listening to music.
 \textit{
	\begin{itemize}
	\item The song's so catchy it makes you bounce round the living room or tap your feet.
	\end{itemize}
}
\item verb \\
If you \textbf{tap} a resource or situation , you make use of it by getting from it something that you need or want .
 \textit{
	\begin{itemize}
	\item He owes his election to having tapped deep public disillusion with professional politicians.
	\item The company is tapping shareholders for £15.8 million.
	\item The utility group has launched the company in an attempt to tap into the market for
green energy.
	\end{itemize}
}
\item verb \\
If someone \textbf{taps} your phone , they attach a special device to the line so that they can secretly listen to your conversations .
 \textbf{Tap} is also a noun.
 \textit{
	\begin{itemize}
	\item The government passed laws allowing the police to tap phones.
	\item We suspected the telephone line was tapped.
	\item He assured MPs that they were not subjected to phone taps.
	\end{itemize}
}
\item  \\
 See also  phone-tapping , wiretap \textit{
	\begin{itemize}
	\end{itemize}
}
\item  \\
 on tap \textit{
	\begin{itemize}
	\end{itemize}
}
\item  \\
 on tap \textit{
	\begin{itemize}
	\end{itemize}
}
\end{enumerate}

\section*{support}
{\large \color{blue}  supports  supporting  supported  }
\subsection*{Explain}
\begin{enumerate}
\item verb \\
If you \textbf{support} someone or their ideas or aims , you agree with them, and perhaps  help them because you want them to succeed .
 \textbf{Support} is also a noun .
 \textit{
	\begin{itemize}
	\item The vice president insisted that he supported the hard-working people of New York.
	\item They pressed the party to support a total ban on pesticides.
	\item The Prime Minister gave his full support to the government's reforms.
	\item They are prepared to resort to violence in support of their beliefs.
	\end{itemize}
}
\item uncountable noun \\
If you give \textbf{support} to someone during a difficult or unhappy time, you are kind to them and help them.
 \textit{
	\begin{itemize}
	\item We campaign for the rights of sufferers and provide support for the patient and family.
	\item We hope to continue to have her close support and friendship.
	\end{itemize}
}
\item uncountable noun \\
Financial  \textbf{support} is money provided to enable an organization to continue . This money is usually provided by the government.
 \textit{
	\begin{itemize}
	\item ...the government's proposal to cut agricultural support by only about 15%.
	\end{itemize}
}
\item verb \\
If you \textbf{support} someone, you provide them with money or the things that they need .
 \textit{
	\begin{itemize}
	\item I have children to support, money to be earned, and a home to be maintained.
	\item She sold everything she'd ever bought in order to support herself through art school.
	\end{itemize}
}
\item verb \\
If a fact \textbf{supports} a statement or a theory, it helps to show that it is true or correct .
 \textbf{Support} is also a noun.
 \textit{
	\begin{itemize}
	\item The Freudian theory about daughters falling in love with their father has little
evidence to support it.
	\item He offers no factual support for these assertions.
	\end{itemize}
}
\item verb \\
If something \textbf{supports} an object, it is underneath the object and holding it up.
 \textit{
	\begin{itemize}
	\item ...the thick wooden posts that supported the ceiling.
	\item Let your baby sit on the floor propped up with plenty of cushions to support him.
	\end{itemize}
}
\item countable noun \\
A \textbf{support} is a bar or other object that supports something.
 \textit{
	\begin{itemize}
	\end{itemize}
}
\item verb \\
If you \textbf{support}  \textbf{yourself} , you prevent yourself from falling by holding onto something or by leaning on something.
 \textbf{Support} is also a noun.
 \textit{
	\begin{itemize}
	\item He supported himself by means of a nearby post.
	\item Alice, very pale, was leaning against him as if for support.
	\end{itemize}
}
\item verb \\
If you \textbf{support} a sports team , you always want them to win and perhaps go regularly to their games.
 \textit{
	\begin{itemize}
	\item Tim, 17, supports Manchester United.
	\end{itemize}
}
\item uncountable noun \\
At a concert or show, \textbf{the}  \textbf{support} or the \textbf{support} act is a less well-known person or band who performs before the main person or band.
 \textit{
	\begin{itemize}
	\end{itemize}
}
\end{enumerate}

\section*{verse}
{\large \color{blue}  verses  }
\subsection*{Explain}
\begin{enumerate}
\item uncountable noun \\
\textbf{Verse} is writing arranged in lines which have rhythm and which often rhyme at the end .
 \textit{
	\begin{itemize}
	\item ...a slim volume of verse.
	\item I have been moved to write a few lines of verse.
	\end{itemize}
}
\item countable noun \\
A \textbf{verse} is one of the parts into which a poem, a song, or a chapter of the Bible or the Koran is divided.
 \textit{
	\begin{itemize}
	\item This verse describes three signs of spring.
	\item The choir has sung only two verses of the last hymn.
	\end{itemize}
}
\end{enumerate}

\section*{test}
{\large \color{blue}  tests  testing  tested  }
\subsection*{Explain}
\begin{enumerate}
\item verb \\
When you \textbf{test} something, you try it, for example by touching it or using it for a short time, in order to find out what it is, what condition it is in, or how well it works.
 \textit{
	\begin{itemize}
	\item Either measure the temperature with a bath thermometer or test the water with your
wrist.
	\item Here the army has its ranges where it tests missiles and other weaponry.
	\item The drug must first be tested in clinical trials to see if it works on other cancers.
	\end{itemize}
}
\item countable noun \\
A \textbf{test} is a deliberate action or experiment to find out how well something works.
 \textit{
	\begin{itemize}
	\item ...the banning of nuclear tests.
	\end{itemize}
}
\item verb \\
If you \textbf{test} someone, you ask them questions or tell them to perform certain actions in order to find out how much they know about a subject or how well they are able to do something.
 \textit{
	\begin{itemize}
	\item They are not really testing pupils; they are testing the teachers.
	\item She decided to test herself with a training run in London.
	\end{itemize}
}
\item countable noun \\
A \textbf{test} is a series of questions that you must  answer or actions that you must perform in order to show how much you know about a subject
or how well you are able to do something.
 \textit{
	\begin{itemize}
	\item Out of a total of 2,602 pupils, only 922 passed the test.
	\item She had sold her bike, taken a driving test and bought a car.
	\end{itemize}
}
\item verb \\
If you \textbf{test} someone, you deliberately make things difficult for them in order to see how they react .
 \textit{
	\begin{itemize}
	\item She may be testing her mother to see how much she can take before she throws her
out.
	\end{itemize}
}
\item countable noun \\
If an event or situation is a \textbf{test}  \textbf{of} a person or thing, it reveals their qualities or effectiveness.
 \textit{
	\begin{itemize}
	\item It is a commonplace fact that holidays are a major test of any relationship.
	\item The test of any civilised society is how it treats its minorities.
	\end{itemize}
}
\item verb \\
If you \textbf{are tested}  \textbf{for} a particular disease or medical condition, you are examined or go through various procedures in order to find out whether you have that disease or
condition.
 \textit{
	\begin{itemize}
	\item My doctor wants me to be tested for diabetes.
	\item Girls in an affected family can also be tested to see if they carry the defective
gene.
	\end{itemize}
}
\item countable noun \\
A medical \textbf{test} is an examination of a part of your body in order to check that you are healthy or to find out what is wrong with you.
 \textit{
	\begin{itemize}
	\item If necessary, X-rays and blood tests will also be used to aid diagnosis.
	\item The doctor ordered numerous tests, which revealed no physical problem.
	\end{itemize}
}
\item countable noun \\
A \textbf{test} is a sports match between two international  teams , usually in cricket , rugby  union , or rugby league .
 \textit{
	\begin{itemize}
	\end{itemize}
}
\item  \\
 put sth to the test \textit{
	\begin{itemize}
	\end{itemize}
}
\item  \\
 put sth to the test \textit{
	\begin{itemize}
	\end{itemize}
}
\item  \\
 stand the test of time \textit{
	\begin{itemize}
	\end{itemize}
}
\end{enumerate}

\section*{wall}
{\large \color{blue}  walls  walling  walled  }
\subsection*{Explain}
\begin{enumerate}
\item countable noun \\
A \textbf{wall} is one of the vertical sides of a building or room .
 \textit{
	\begin{itemize}
	\item Kathryn leaned against the wall of the church.
	\item The bedroom walls would be painted light blue.
	\item She checked the wall clock.
	\end{itemize}
}
\item countable noun \\
A \textbf{wall} is a long narrow vertical structure made of stone or brick that surrounds or divides an area of land.
 \textit{
	\begin{itemize}
	\item He sat on the wall in the sun.
	\item The well is surrounded by a wall only 12 inches high.
	\end{itemize}
}
\item countable noun \\
The \textbf{wall} of something that is hollow is its side.
 \textit{
	\begin{itemize}
	\item He ran his fingers along the inside walls of the box.
	\end{itemize}
}
\item countable noun \\
A \textbf{wall}  \textbf{of} something is a large amount of it forming a high vertical barrier .
 \textit{
	\begin{itemize}
	\item She gazed at the wall of books.
	\item I was just hit by a wall of water.
	\end{itemize}
}
\item countable noun \\
You can describe something as a \textbf{wall}  \textbf{of} a particular kind when it acts as a barrier and prevents people from understanding something.
 \textit{
	\begin{itemize}
	\item The police say they met the usual wall of silence.
	\end{itemize}
}
\item  \\
 to bang your head against a brick wall \textit{
	\begin{itemize}
	\end{itemize}
}
\item  \\
 to have your back to the wall \textit{
	\begin{itemize}
	\end{itemize}
}
\item  \\
 climb the wall \textit{
	\begin{itemize}
	\end{itemize}
}
\item  \\
 drive someone up the wall \textit{
	\begin{itemize}
	\end{itemize}
}
\item  \\
 go to the wall \textit{
	\begin{itemize}
	\end{itemize}
}
\end{enumerate}

\section*{tramp}
{\large \color{blue}  tramps  tramping  tramped  }
\subsection*{Explain}
\begin{enumerate}
\item countable noun \\
A \textbf{tramp} is a person who has no home or job , and very little money . Tramps go from place to place, and get  food or money by asking people or by doing casual work.
 \textit{
	\begin{itemize}
	\end{itemize}
}
\item verb \\
If you \textbf{tramp}  somewhere , you walk there slowly and with regular, heavy steps, for a long time.
 \textit{
	\begin{itemize}
	\item They put on their coats and tramped through the falling snow.
	\item She spent all day yesterday tramping the streets, gathering evidence.
	\end{itemize}
}
\item uncountable noun \\
The \textbf{tramp} of people is the sound of their heavy, regular walking.
 \textit{
	\begin{itemize}
	\item He heard the slow, heavy tramp of feet on the stairs.
	\item ...the tramp of heavy boots.
	\end{itemize}
}
\item countable noun \\
If someone refers to a woman as a \textbf{tramp} , they are insulting her, because they think that she is immoral in her sexual  behaviour .
 \textit{
	\begin{itemize}
	\end{itemize}
}
\end{enumerate}

\section*{wax}
{\large \color{blue}  waxes  waxing  waxed  }
\subsection*{Explain}
\begin{enumerate}
\item variable noun \\
\textbf{Wax} is a solid, slightly  shiny substance made of fat or oil which is used to make candles and polish. It melts when it is heated.
 \textit{
	\begin{itemize}
	\item There were coloured candles which had spread pools of wax on the furniture.
	\item She loved the scent in the house of wax polish.
	\end{itemize}
}
\item verb \\
If you \textbf{wax} a surface, you put a thin layer of wax onto it, especially in order to polish it.
 \textit{
	\begin{itemize}
	\item We'd have long talks while she helped me wax the floor.
	\item ...all those Sundays spent washing and waxing the car.
	\end{itemize}
}
\item verb \\
If you have your legs \textbf{waxed} , you have the hair removed from your legs by having wax put on them and then pulled off quickly.
 \textit{
	\begin{itemize}
	\item She has just had her legs waxed at the local beauty parlour.
	\item She would wax her legs, ready for the party.
	\end{itemize}
}
\item uncountable noun \\
\textbf{Wax} is the sticky yellow substance found in your ears .
 \textit{
	\begin{itemize}
	\end{itemize}
}
\item  \\
 to wax lyrical \textit{
	\begin{itemize}
	\end{itemize}
}
\item  \\
 wax and wane \textit{
	\begin{itemize}
	\end{itemize}
}
\end{enumerate}

\section*{transplant}
{\large \color{blue}  transplants  transplanting  transplanted  }
\subsection*{Explain}
\begin{enumerate}
\item variable noun \\
A \textbf{transplant} is a medical operation in which a part of a person's body is replaced because it is diseased .
 \textit{
	\begin{itemize}
	\item He was recovering from a heart transplant operation.
	\item ...the controversy over the sale of human organs for transplant.
	\end{itemize}
}
\item verb \\
If doctors  \textbf{transplant} an organ such as a heart or a kidney , they use it to replace a patient's diseased organ.
 \textit{
	\begin{itemize}
	\item The operation to transplant a kidney is now fairly routine.
	\item ...transplanted organs such as hearts and kidneys.
	\end{itemize}
}
\item verb \\
To \textbf{transplant} someone or something means to move them to a different place.
 \textit{
	\begin{itemize}
	\item 15 years later I also transplanted myself to Scotland from England.
	\item In the 19th century, the Santa Claus tradition seems to have been transplanted back
to Europe.
	\item Farmers will be able to seed it directly, rather than having to transplant seedlings.
	\end{itemize}
}
\end{enumerate}

\section*{welfare}
{\large \color{blue}  }
\subsection*{Explain}
\begin{enumerate}
\item uncountable noun \\
The \textbf{welfare} of a person or group is their health, comfort , and happiness.
 \textit{
	\begin{itemize}
	\item I do not think he is considering Emma's welfare.
	\item He was the head of a charity for the welfare of children.
	\end{itemize}
}
\item adjective \\
\textbf{Welfare} services are provided to help with people's living conditions and financial problems .
 \textit{
	\begin{itemize}
	\item Child welfare services are well established and comprehensive.
	\item He has urged complete reform of the welfare system.
	\end{itemize}
}
\item uncountable noun \\
In the United  States , \textbf{welfare} is money that is paid by the government to people who are unemployed , poor , or sick .
 \textit{
	\begin{itemize}
	\item States are making deep cuts in welfare.
	\end{itemize}
}
\end{enumerate}

\section*{accordance}
{\large \color{blue}  }
\subsection*{Explain}
\begin{enumerate}
\item  \\
 in accordance with \textit{
	\begin{itemize}
	\end{itemize}
}
\end{enumerate}

\section*{arithmetic}
{\large \color{blue}  }
\subsection*{Explain}
\begin{enumerate}
\item uncountable noun \\
\textbf{Arithmetic} is the part of mathematics that is concerned with the addition, subtraction, multiplication,
and division of numbers .
 \textit{
	\begin{itemize}
	\item ...teaching the basics of reading, writing and arithmetic.
	\item ...an arithmetic test.
	\end{itemize}
}
\item uncountable noun \\
You can use \textbf{arithmetic} to refer to the process of doing a particular  sum or calculation.
 \textit{
	\begin{itemize}
	\item 4,000 women put in ten rupees each, which if my arithmetic is right adds up to 40,000
rupees.
	\end{itemize}
}
\item uncountable noun \\
If you refer to \textbf{the}  \textbf{arithmetic} of a situation , you are concerned with those aspects of it that can be expressed in numbers, and how they affect the situation.
 \textit{
	\begin{itemize}
	\item The budgetary arithmetic suggests that government borrowing is set to surge.
	\item The arithmetic is finely balanced: the socialists and their allies do not have a
majority.
	\end{itemize}
}
\item adjective \\
\textbf{Arithmetic}  means relating to or consisting of calculations involving numbers.
 \textit{
	\begin{itemize}
	\item ...simple arithmetic operations such as adding or multiplying numbers.
	\end{itemize}
}
\end{enumerate}

\section*{alternative}
{\large \color{blue}  alternatives  }
\subsection*{Explain}
\begin{enumerate}
\item countable noun \\
If one thing is an \textbf{alternative}  \textbf{to} another, the first can be found, used, or done instead of the second .
 \textit{
	\begin{itemize}
	\item New ways to treat arthritis may provide an alternative to painkillers.
	\end{itemize}
}
\item adjective \\
An \textbf{alternative}  plan or offer is different from the one that you already have, and can be done or used instead.
 \textit{
	\begin{itemize}
	\item There were alternative methods of travel available.
	\item They had a right to seek alternative employment.
	\end{itemize}
}
\item adjective \\
\textbf{Alternative} is used to describe something that is different from the usual things of its kind , or the usual ways of doing something, in modern  Western society. For example , an \textbf{alternative} lifestyle does not follow conventional ways of living and working .
 \textit{
	\begin{itemize}
	\item ...unconventional parents who embraced the alternative lifestyle of the Sixties.
	\item If you like alternative comedy you'll love this book.
	\end{itemize}
}
\item adjective \\
\textbf{Alternative}  medicine uses traditional ways of curing people, such as medicines made from plants, massage , and acupuncture .
 \textit{
	\begin{itemize}
	\item ...alternative health care.
	\end{itemize}
}
\item adjective \\
\textbf{Alternative}  energy uses natural  sources of energy such as the sun , wind , or water for power and fuel , rather than oil , coal , or nuclear power.
 \textit{
	\begin{itemize}
	\end{itemize}
}
\end{enumerate}

\section*{banner}
{\large \color{blue}  banners  }
\subsection*{Explain}
\begin{enumerate}
\item countable noun \\
A \textbf{banner} is a long strip of cloth with something written on it. Banners are usually attached to two poles and carried during a protest or rally .
 \textit{
	\begin{itemize}
	\item ...a large crowd of students carrying banners denouncing the government.
	\end{itemize}
}
\item  \\
 under the banner of \textit{
	\begin{itemize}
	\end{itemize}
}
\end{enumerate}

\section*{apartment}
{\large \color{blue}  apartments  }
\subsection*{Explain}
\begin{enumerate}
\item countable noun \\
An \textbf{apartment} is a set of rooms for living in, usually on one floor of a large building.
 \textit{
	\begin{itemize}
	\item Christina has her own apartment, with her own car.
	\item ...bleak cities of concrete apartment blocks.
	\end{itemize}
}
\item plural noun \\
The \textbf{apartments} of an important person such as a king , queen , or president are a set of large rooms which are used by them.
 \textit{
	\begin{itemize}
	\item ...the private apartments of the Prince of Wales at St James's Palace.
	\end{itemize}
}
\end{enumerate}

\section*{bear}
{\large \color{blue}  bears  bearing  bore  borne  }
\subsection*{Explain}
\begin{enumerate}
\item verb \\
If you \textbf{bear} something somewhere , you carry it there or take it there.
 \textit{
	\begin{itemize}
	\item They bore the oblong hardwood box into the kitchen and put it on the table.
	\end{itemize}
}
\item verb \\
If you \textbf{bear} something such as a weapon , you hold it or carry it with you.
 \textit{
	\begin{itemize}
	\item ...the constitutional right to bear arms.
	\end{itemize}
}
\item verb \\
If one thing \textbf{bears} the weight of something else, it supports the weight of that thing.
 \textit{
	\begin{itemize}
	\item The ice was not thick enough to bear the weight of marching men.
	\end{itemize}
}
\item verb \\
If something \textbf{bears} a particular mark or characteristic, it has that mark or characteristic.
 \textit{
	\begin{itemize}
	\item The houses bear the marks of bullet holes.
	\item ...notepaper bearing the Presidential seal.
	\item ...a corporation he owned that bore his name.
	\item The room bore all the signs of a violent struggle.
	\end{itemize}
}
\item verb \\
If you \textbf{bear} an unpleasant experience, you accept it because you are unable to do anything about it.
 \textit{
	\begin{itemize}
	\item They will have to bear the misery of living in constant fear of war.
	\end{itemize}
}
\item verb \\
If you can't \textbf{bear} someone or something, you dislike them very much.
 \textit{
	\begin{itemize}
	\item I can't bear people who make judgements and label me.
	\item I can't bear having to think what I'm going to say.
	\item He can't bear to talk about it, even to me.
	\end{itemize}
}
\item verb \\
If someone \textbf{bears} the cost of something, they pay for it.
 \textit{
	\begin{itemize}
	\item Patients should not have to bear the costs of their own treatment.
	\end{itemize}
}
\item verb \\
If you \textbf{bear} the responsibility for something, you accept responsibility for it.
 \textit{
	\begin{itemize}
	\item If a woman makes a decision to have a child alone, she should bear that responsibility
alone.
	\end{itemize}
}
\item verb \\
If one thing \textbf{bears} no resemblance or no relationship to another thing, they are not at all similar.
 \textit{
	\begin{itemize}
	\item Their daily menus bore no resemblance whatsoever to what they were actually fed.
	\item For many software packages, the price bears little relation to cost.
	\end{itemize}
}
\item verb \\
When a plant or tree \textbf{bears} flowers, fruit, or leaves, it produces them.
 \textit{
	\begin{itemize}
	\item As the plants grow and start to bear fruit they will need a lot of water.
	\end{itemize}
}
\item verb \\
If something such as a bank account or an investment  \textbf{bears} interest, interest is paid on it.
 \textit{
	\begin{itemize}
	\item The eight-year bond will bear annual interest of 10.5%.
	\end{itemize}
}
\item verb \\
When a woman \textbf{bears} a child, she gives birth to him or her.
 \textit{
	\begin{itemize}
	\item Emma bore a son called Karl.
	\item She bore him a daughter, Suzanna.
	\end{itemize}
}
\item verb \\
If you \textbf{bear} someone a feeling such as love or hate , you feel that emotion towards them.
 \textit{
	\begin{itemize}
	\item She bore no ill will. If people didn't like her, too bad.
	\item I have lived with him on the best terms and bear him friendship.
	\end{itemize}
}
\item verb \\
If you \textbf{bear}  \textbf{yourself} in a particular way, you move or behave in that way.
 \textit{
	\begin{itemize}
	\item There was elegance and simple dignity in the way he bore himself.
	\end{itemize}
}
\item verb \\
If you \textbf{bear} left or \textbf{bear} right when you are driving or walking along, you turn and continue in that direction.
 \textit{
	\begin{itemize}
	\item Go left onto the A107 and bear left into Seven Sisters Road.
	\end{itemize}
}
\item  \\
 to bring something to bear \textit{
	\begin{itemize}
	\end{itemize}
}
\item  \\
 bring pressure to bear on \textit{
	\begin{itemize}
	\end{itemize}
}
\end{enumerate}

\section*{august}
{\large \color{blue}  }
\subsection*{Explain}
\begin{enumerate}
\item adjective \\
Someone or something that is \textbf{august} is dignified and impressive .
 \textit{
	\begin{itemize}
	\item ...the august surroundings of the Liberal Club.
	\end{itemize}
}
\end{enumerate}

\section*{beat}
{\large \color{blue}  beats  beating  beaten  }
\subsection*{Explain}
\begin{enumerate}
\item verb \\
If you \textbf{beat} someone or something, you hit them very hard.
 \textit{
	\begin{itemize}
	\item My sister tried to stop them and they beat her.
	\item They were beaten to death with baseball bats.
	\end{itemize}
}
\item verb \\
To \textbf{beat}  \textbf{on} , \textbf{beat}  \textbf{at} , or \textbf{beat}  \textbf{against} something means to hit it hard, usually several times or continuously for a period
of time.
 \textbf{Beat} is also a noun.
 \textit{
	\begin{itemize}
	\item There was dead silence but for a fly beating against the glass.
	\item Nina managed to free herself and began beating at the flames with a pillow.
	\item The rain was beating on the windowpanes.
	\item ...the rhythmic beat of the surf.
	\end{itemize}
}
\item verb \\
When your heart or pulse  \textbf{beats} , it continually makes regular rhythmic movements.
 \textbf{Beat} is also a noun.
 \textit{
	\begin{itemize}
	\item I felt my heart beating faster.
	\item He could hear the beat of his heart.
	\item Most people's pulse rate is more than 70 beats per minute.
	\end{itemize}
}
\item verb \\
If you \textbf{beat} a drum or similar instrument, you hit it in order to make a sound. You can also say that a drum \textbf{beats} .
 \textbf{Beat} is also a noun.
 \textit{
	\begin{itemize}
	\item When you beat the drum, you feel good.
	\item ...drums beating and pipes playing.
	\item ...the rhythmical beat of the drum.
	\end{itemize}
}
\item countable noun \\
The \textbf{beat} of a piece of music is the main rhythm that it has.
 \textit{
	\begin{itemize}
	\item ...the thumping beat of rock music.
	\item ...the dance beats of the last two decades.
	\end{itemize}
}
\item countable noun \\
In music, a \textbf{beat} is a unit of measurement . The number of beats in a bar of a piece of music is indicated by two numbers at
the beginning of the piece.
 \textit{
	\begin{itemize}
	\item It's got four beats to a bar.
	\end{itemize}
}
\item verb \\
If you \textbf{beat} eggs, cream , or butter , you mix them thoroughly using a fork or beater.
 \textit{
	\begin{itemize}
	\item Beat the eggs and sugar until they start to thicken.
	\end{itemize}
}
\item verb \\
When a bird or insect \textbf{beats} its wings or when its wings \textbf{beat} , its wings move up and down.
 \textit{
	\begin{itemize}
	\item Beating their wings they flew off.
	\item Its wings beat slowly.
	\end{itemize}
}
\item verb \\
If you \textbf{beat} someone in a competition or election , you defeat them.
 \textit{
	\begin{itemize}
	\item In yesterday's games, Switzerland beat the United States two-one.
	\item There are men who simply don't like being beaten by a woman.
	\item She was easily beaten into third place.
	\end{itemize}
}
\item verb \\
If someone \textbf{beats} a record or achievement , they do better than it.
 \textit{
	\begin{itemize}
	\item He was as eager as his Captain to beat the record.
	\end{itemize}
}
\item verb \\
If you \textbf{beat} something that you are fighting against, for example an organization, a problem,
or a disease, you defeat it.
 \textit{
	\begin{itemize}
	\item It became clear that the Union was not going to beat the government.
	\item The doctor gave him the news that he'd beaten cancer.
	\item They recognise that tough action offers the only hope of beating inflation.
	\item Both he and his wife have recently beaten cancer and now are taking on some new challenges.
	\end{itemize}
}
\item verb \\
If an attack or an attempt \textbf{is beaten}  \textbf{off} or \textbf{is beaten}  \textbf{back} , it is stopped, often temporarily.
 \textit{
	\begin{itemize}
	\item The rescuers were beaten back by strong winds and currents.
	\item ...the day after government troops beat off a fierce rebel attack on its capital.

	\end{itemize}
}
\item verb \\
If you say that one thing \textbf{beats} another, you mean that it is better than it.
 \textit{
	\begin{itemize}
	\item Being boss of a software firm beats selling insurance.
	\item Nothing quite beats the luxury of soaking in a long, hot bath at the end of a tiring
day.
	\item For an evening stroll the beach at Dieppe is hard to beat.
	\end{itemize}
}
\item verb \\
If you say you can't \textbf{beat} a particular thing you mean that it is the best thing of its kind.
 \textit{
	\begin{itemize}
	\item You can't beat soap and water for cleansing.
	\end{itemize}
}
\item verb \\
To \textbf{beat} a time limit or an event means to achieve something before that time or event.
 \textit{
	\begin{itemize}
	\item They were trying to beat the midnight deadline.
	\item Those who shop on Sunday to beat the rush are wasting their time.
	\end{itemize}
}
\item countable noun \\
A police officer's or journalist's \textbf{beat} is the area for which he or she is responsible.
 \textit{
	\begin{itemize}
	\item Crime on his beat has halved.
	\end{itemize}
}
\item verb \\
You use \textbf{beat} in expressions such as 'It beats me' or 'What beats me is' to indicate that you cannot
 understand or explain something.
 \textit{
	\begin{itemize}
	\item 'What am I doing wrong, anyway?'—'Beats me, Lewis.'
	\item How you can be so insensitive absolutely beats me.
	\end{itemize}
}
\item phrase \\
If you tell someone to \textbf{beat it} , you are telling them to go away.
 \textit{
	\begin{itemize}
	\item Beat it before it's too late.
	\end{itemize}
}
\item convention \\
You can say \textbf{Can you beat it?} or \textbf{Can you beat that?} to show that you are surprised and perhaps  annoyed about something.
 \textit{
	\begin{itemize}
	\item Can you beat it; there was Graham Greene in Freetown and there was I on the other
side of Africa.
	\end{itemize}
}
\item  \\
 beat sb to it \textit{
	\begin{itemize}
	\end{itemize}
}
\item  \\
 If you can't beat them, join them. \textit{
	\begin{itemize}
	\end{itemize}
}
\item  \\
 miss a beat \textit{
	\begin{itemize}
	\end{itemize}
}
\item  \\
 miss a beat \textit{
	\begin{itemize}
	\end{itemize}
}
\item  \\
 on the beat \textit{
	\begin{itemize}
	\end{itemize}
}
\item  \\
 beat time \textit{
	\begin{itemize}
	\end{itemize}
}
\end{enumerate}

\section*{batch}
{\large \color{blue}  batches  }
\subsection*{Explain}
\begin{enumerate}
\item countable noun \\
A \textbf{batch}  \textbf{of} things or people is a group of things or people of the same kind , especially a group that is dealt with at the same time or is sent to a particular place at the same time.
 \textit{
	\begin{itemize}
	\item ...the current batch of trainee priests.
	\item She brought a large batch of newspaper cuttings.
	\item We're still waiting for the first batch to arrive.
	\end{itemize}
}
\end{enumerate}

\section*{candle}
{\large \color{blue}  candles  }
\subsection*{Explain}
\begin{enumerate}
\item countable noun \\
A \textbf{candle} is a stick of hard wax with a piece of string  called a wick through the middle . You light the wick in order to give a steady  flame that provides light.
 \textit{
	\begin{itemize}
	\item The bedroom was lit by a single candle.
	\end{itemize}
}
\item  \\
 to burn the candle at both ends \textit{
	\begin{itemize}
	\end{itemize}
}
\item  \\
 can't hold a candle to \textit{
	\begin{itemize}
	\end{itemize}
}
\item  \\
 the game is not worth the candle \textit{
	\begin{itemize}
	\end{itemize}
}
\end{enumerate}

\section*{cartoon}
{\large \color{blue}  cartoons  }
\subsection*{Explain}
\begin{enumerate}
\item countable noun \\
A \textbf{cartoon} is a humorous drawing or series of drawings in a newspaper or magazine.
 \textit{
	\begin{itemize}
	\item ...a cartoon strip in the Daily Mirror.
	\end{itemize}
}
\item countable noun \\
A \textbf{cartoon} is a film in which all the characters and scenes are drawn  rather than being real people or objects.
 \textit{
	\begin{itemize}
	\item ...the Saturday morning cartoons.
	\end{itemize}
}
\end{enumerate}

\section*{clergy}
{\large \color{blue}  }
\subsection*{Explain}
\begin{enumerate}
\item plural noun \\
The \textbf{clergy} are the official  leaders of the religious activities of a particular group of believers.
 \textit{
	\begin{itemize}
	\item These proposals met opposition from the clergy.
	\end{itemize}
}
\end{enumerate}

\section*{click}
{\large \color{blue}  clicks  clicking  clicked  }
\subsection*{Explain}
\begin{enumerate}
\item verb \\
If something \textbf{clicks} or if you \textbf{click} it, it makes a short, sharp sound.
 \textbf{Click} is also a noun .
 \textit{
	\begin{itemize}
	\item The applause rose to a crescendo and cameras clicked.
	\item He clicked off the radio.
	\item Blake clicked his fingers at a passing waiter, who hurried across to them.
	\item The telephone rang three times before I heard a click and then her recorded voice.
	\end{itemize}
}
\item verb \\
If you \textbf{click}  \textbf{on} an area of a computer  screen , you point the cursor at that area and press one of the buttons on the mouse in order to make something happen .
 \textbf{Click} is also a noun.
 \textit{
	\begin{itemize}
	\item I clicked on a link and recent reviews of the production came up.
	\item You can check your email with a click of your mouse.
	\end{itemize}
}
\item verb \\
When you suddenly understand something, you can say that it \textbf{clicks} .
 \textit{
	\begin{itemize}
	\item When I saw the television report, it all clicked.
	\item It suddenly clicked that this was fantastic fun.
	\end{itemize}
}
\item reciprocal verb \\
If you \textbf{click}  \textbf{with} someone, you like each other and become friendly as soon as you meet . You can also say that two people \textbf{click} .
 \textit{
	\begin{itemize}
	\item They clicked immediately. They loved the same things.
	\end{itemize}
}
\end{enumerate}

\section*{creature}
{\large \color{blue}  creatures  }
\subsection*{Explain}
\begin{enumerate}
\item countable noun \\
You can refer to any living thing that is not a plant as a \textbf{creature} , especially when it is of an unknown or unfamiliar  kind . People also refer to imaginary animals and beings as \textbf{creatures} .
 \textit{
	\begin{itemize}
	\item Alaskan Eskimos believe that every living creature possesses a spirit.
	\item The garden is surrounded by a hedge in which many small creatures can live.
	\item They have been visited by creatures from outer space.
	\end{itemize}
}
\item countable noun \\
If you say that someone is a particular type of \textbf{creature} , you are focusing on a particular quality they have.
 \textit{
	\begin{itemize}
	\item She's charming, a sweet creature.
	\item I am not a vain creature.
	\item She was a creature of the emotions, rather than reason.
	\end{itemize}
}
\item countable noun \\
If you describe someone as someone else's \textbf{creature} , you mean that they are controlled by or depend on that person.
 \textit{
	\begin{itemize}
	\item We are not creatures of the Conservative government.
	\end{itemize}
}
\end{enumerate}

\section*{conductor}
{\large \color{blue}  conductors  }
\subsection*{Explain}
\begin{enumerate}
\item countable noun \\
A \textbf{conductor} is a person who stands in front of an orchestra or choir and directs its performance .
 \textit{
	\begin{itemize}
	\end{itemize}
}
\item countable noun \\
On a bus, the \textbf{conductor} is the person whose job is to help  passengers and check tickets. 
 \textit{
	\begin{itemize}
	\end{itemize}
}
\item countable noun \\
On a train, a \textbf{conductor} is a person whose job is to travel on the train in order to help passengers and check tickets.
 \textit{
	\begin{itemize}
	\end{itemize}
}
\item countable noun \\
A \textbf{conductor} is a substance that heat or electricity can pass through or along.
 \textit{
	\begin{itemize}
	\end{itemize}
}
\end{enumerate}

\section*{dance}
{\large \color{blue}  dances  dancing  danced  }
\subsection*{Explain}
\begin{enumerate}
\item verb \\
When you \textbf{dance} , you move your body and feet in a way which follows a rhythm, usually in time to music.
 \textit{
	\begin{itemize}
	\item Polly had never learned to dance.
	\item I like to dance to the music on the radio.
	\end{itemize}
}
\item countable noun \\
A \textbf{dance} is a particular series of graceful movements of your body and feet, which you usually do in time to music.
 \textit{
	\begin{itemize}
	\item Sometimes the people doing this dance hold brightly colored scarves.
	\item She describes the tango as a very sexy dance.
	\end{itemize}
}
\item verb \\
When you \textbf{dance}  \textbf{with} someone, the two of you take part in a dance together , as partners . You can also  say that two people \textbf{dance} .
 \textbf{Dance} is also a noun .
 \textit{
	\begin{itemize}
	\item It's a terrible thing when nobody wants to dance with you.
	\item Shall we dance?
	\item He asked her to dance.
	\item Come and have a dance with me.
	\end{itemize}
}
\item countable noun \\
A \textbf{dance} is a social event where people dance with each other.
 \textit{
	\begin{itemize}
	\item She often went to parties and dances at Littlecote.
	\item ...the school dance.
	\end{itemize}
}
\item uncountable noun \\
\textbf{Dance} is the activity of performing dances, as a public  entertainment or an art form.
 \textit{
	\begin{itemize}
	\item She loves dance, drama and music.
	\item The story is told through dance.
	\item ...dance classes.
	\end{itemize}
}
\item verb \\
If you \textbf{dance} a particular kind of dance, you do it or perform it.
 \textit{
	\begin{itemize}
	\item Then we put the music on, and we all danced the Charleston.
	\item They will dance two performances of Ashton's 'Romeo and Juliet'.
	\end{itemize}
}
\item verb \\
If you \textbf{dance}  somewhere , you move there lightly and quickly, usually because you are happy or excited .
 \textit{
	\begin{itemize}
	\item He danced off down the road.
	\item Amy went and kissed him, and then danced out of his reach.
	\end{itemize}
}
\item verb \\
If you say that something \textbf{dances} , you mean that it moves about, or seems to move about, lightly and quickly.
 \textit{
	\begin{itemize}
	\item Light danced on the surface of the water.
	\item She read it slowly, but the words danced before her eyes.
	\end{itemize}
}
\item  \\
 to lead someone a merry dance \textit{
	\begin{itemize}
	\end{itemize}
}
\end{enumerate}

\section*{decade}
{\large \color{blue}  decades  }
\subsection*{Explain}
\begin{enumerate}
\item countable noun \\
A \textbf{decade} is a period of ten years, especially one that begins with a year ending in 0, for example 1980 to 1989.
 \textit{
	\begin{itemize}
	\item ...the last decade of the nineteenth century.
	\end{itemize}
}
\end{enumerate}

\section*{drop}
{\large \color{blue}  drops  dropping  dropped  }
\subsection*{Explain}
\begin{enumerate}
\item verb \\
If a level or amount \textbf{drops} or if someone or something \textbf{drops} it, it quickly becomes less.
 \textbf{Drop} is also a noun .
 \textit{
	\begin{itemize}
	\item Temperatures can drop to freezing at night.
	\item Once the rate rises it never drops back to its previous level.
	\item His blood pressure had dropped severely.
	\item He had dropped the price of his London home by £1.25m.
	\item He was prepared to take a drop in wages.
	\item The poll indicates a drop in support for the Conservatives.
	\end{itemize}
}
\item verb \\
If you \textbf{drop} something, you accidentally let it fall.
 \textit{
	\begin{itemize}
	\item I dropped my glasses and broke them.
	\end{itemize}
}
\item verb \\
If something \textbf{drops}  \textbf{onto} something else, it falls onto that thing. If something \textbf{drops}  \textbf{from}  somewhere , it falls from that place.
 \textit{
	\begin{itemize}
	\item He felt hot tears dropping onto his fingers.
	\item Burning embers started dropping from the ceiling.
	\item His toupee dropped off, revealing his bald head.
	\end{itemize}
}
\item verb \\
If you \textbf{drop} something somewhere or if it \textbf{drops} there, you deliberately let it fall there.
 \textit{
	\begin{itemize}
	\item Drop the noodles into the water.
	\item He dropped his plate into the sink.
	\item ...shaped pots that simply drop into their own container.
	\item Bombs drop round us and the floor shudders.
	\end{itemize}
}
\item verb \\
If a person or a part of their body \textbf{drops} to a lower position, or if they \textbf{drop} a part of their body to a lower position, they move to that position, often in a
 tired and lifeless way.
 \textit{
	\begin{itemize}
	\item Nancy dropped into a nearby chair.
	\item She let her head drop.
	\item He dropped his hands on to his knees.
	\end{itemize}
}
\item verb \\
To \textbf{drop} is used in expressions such as \textbf{to be about to drop} and \textbf{to dance until you drop} to emphasize that you are exhausted and can no longer continue doing something.
 \textit{
	\begin{itemize}
	\item She looked about to drop.
	\item You have to run until you drop.
	\end{itemize}
}
\item verb \\
If a man \textbf{drops} his trousers , he pulls them down, usually as a joke or to be rude .
 \textit{
	\begin{itemize}
	\item A couple of boozy revellers dropped their trousers.
	\end{itemize}
}
\item verb \\
If your voice  \textbf{drops} or if you \textbf{drop} your voice, you speak more quietly .
 \textit{
	\begin{itemize}
	\item Her voice will drop to a dismissive whisper.
	\item He dropped his voice and glanced round at the door.
	\end{itemize}
}
\item verb \\
If you \textbf{drop} someone or something somewhere, you take them somewhere and leave them there, usually
in a car or other vehicle.
 \textbf{Drop off} means the same as drop .
 \textit{
	\begin{itemize}
	\item He dropped me outside the hotel.
	\item Many children had been dropped at the stadium by their parents.
	\item Tim had dropped the letter in earlier.
	\item Just drop me off at the airport.
	\item He was dropping off a late birthday present.
	\end{itemize}
}
\item verb \\
If you \textbf{drop} an idea, course of action, or habit , you do not continue with it.
 \textit{
	\begin{itemize}
	\item He was told to drop the idea.
	\item The prosecution was forced to drop the case.
	\item The charges were dropped.
	\end{itemize}
}
\item verb \\
If someone \textbf{is dropped} by a sports team or organization, they are no longer included in that team or employed
by that organization.
 \textit{
	\begin{itemize}
	\item The country's captain was dropped from the tour party to England.
	\end{itemize}
}
\item verb \\
If you \textbf{drop} a game or part of a game in a sports competition , you lose it.
 \textit{
	\begin{itemize}
	\item Oremans has yet to drop a set.
	\end{itemize}
}
\item verb \\
If you \textbf{drop} to a lower position in a sports competition, you move to that position.
 \textit{
	\begin{itemize}
	\item Britain has dropped from second to third place in the league.
	\end{itemize}
}
\item countable noun \\
A \textbf{drop}  \textbf{of} a liquid is a very small amount of it shaped like a little ball. In informal English, you can also use \textbf{drop} when you are referring to a very small amount of something such as a drink.
 \textit{
	\begin{itemize}
	\item ...a drop of blue ink.
	\item Add the cream a few drops at a time.
	\item I'll have another drop of that Italian milk.
	\end{itemize}
}
\item plural noun \\
\textbf{Drops} are a kind of medicine which you put drop by drop into your ears , eyes, or nose .
 \textit{
	\begin{itemize}
	\item ...eye drops.
	\end{itemize}
}
\item countable noun \\
Fruit or chocolate  \textbf{drops} are small round sweets with a fruit or chocolate flavour .
 \textit{
	\begin{itemize}
	\end{itemize}
}
\item countable noun \\
You use \textbf{drop} to talk about vertical distances. For example, a thirty-foot \textbf{drop} is a distance of thirty feet between the top of a cliff or wall and the bottom of it.
 \textit{
	\begin{itemize}
	\item There was a sheer drop just outside my window.
	\item It's only a four-foot drop.
	\end{itemize}
}
\item  \\
 drop a hint \textit{
	\begin{itemize}
	\end{itemize}
}
\item  \\
 drop the subject/drop it/let it drop \textit{
	\begin{itemize}
	\end{itemize}
}
\end{enumerate}

\section*{december}
{\large \color{blue}  Decembers  }
\subsection*{Explain}
\begin{enumerate}
\item variable noun \\
\textbf{December} is the twelfth and last month of the year in the Western  calendar .
 \textit{
	\begin{itemize}
	\item ...a bright morning in mid-December.
	\item Her baby was born on 4 December.
	\item The talks are due to be concluded this December.
	\end{itemize}
}
\end{enumerate}

\section*{erosion}
{\large \color{blue}  }
\subsection*{Explain}
\begin{enumerate}
\item uncountable noun \\
\textbf{Erosion} is the gradual  destruction and removal of rock or soil in a particular area by rivers , the sea , or the weather .
 \textit{
	\begin{itemize}
	\item As their roots are strong and penetrating, they prevent erosion.
	\item ...erosion of the river valleys.
	\item ...soil erosion.
	\end{itemize}
}
\item uncountable noun \\
The \textbf{erosion}  \textbf{of} a person's authority , rights , or confidence is the gradual destruction or removal of them.
 \textit{
	\begin{itemize}
	\item ...the erosion of confidence in world financial markets.
	\item ...an erosion of presidential power.
	\end{itemize}
}
\item uncountable noun \\
The \textbf{erosion}  \textbf{of}  support , values , or money is a gradual decrease in its level or standard .
 \textit{
	\begin{itemize}
	\item ...the erosion of moral standards.
	\item ...a dramatic erosion of support for the program.
	\end{itemize}
}
\end{enumerate}

\section*{dozen}
{\large \color{blue}  dozens  }
\subsection*{Explain}
\begin{enumerate}
\item number \\
If you have \textbf{a}  \textbf{dozen} things, you have twelve of them.
 \textit{
	\begin{itemize}
	\item ...a dozen eggs.
	\item You will be able to take ten dozen bottles free of duty through customs.
	\item He ordered a dozen of their best red roses.
	\item His chicken eggs sell for $22 a dozen.
	\end{itemize}
}
\item number \\
You can refer to a group of approximately twelve things or people as \textbf{a}  \textbf{dozen} . You can refer to a group of approximately six things or people as \textbf{half a dozen} .
 \textit{
	\begin{itemize}
	\item I was sitting only a dozen feet away.
	\item In half a dozen words, he had explained the bond that linked them.
	\item The riot left four people dead and several dozen injured.
	\end{itemize}
}
\item quantifier \\
If you refer to \textbf{dozens of} things or people, you are emphasizing that there are very many of them.
 You can also use \textbf{dozens} as a pronoun.
 \textit{
	\begin{itemize}
	\item ...a storm which destroyed dozens of homes and buildings.
	\item Just as revealing are Mr Johnson's portraits, of which there are dozens.
	\end{itemize}
}
\end{enumerate}

\section*{flag}
{\large \color{blue}  flags  flagging  flagged  }
\subsection*{Explain}
\begin{enumerate}
\item countable noun \\
A \textbf{flag} is a piece of cloth which can be attached to a pole and which is used as a sign , signal, or symbol of something, especially of a particular country.
 \textit{
	\begin{itemize}
	\item The Marines raised the American flag.
	\item They had raised the white flag in surrender.
	\end{itemize}
}
\item countable noun \\
A \textbf{flag} is a small piece of paper or cloth attached to a stick or pin which is sold on a flag day or used to mark a particular spot .
 \textit{
	\begin{itemize}
	\end{itemize}
}
\item countable noun \\
Journalists sometimes refer to the \textbf{flag} of a particular country or organization as a way of referring to the country or organization
itself and its values or power.
 \textit{
	\begin{itemize}
	\item The two brothers are riding under the British flag.
	\item The airport was opened by Canadian troops operating under the flag of the United
Nations.
	\end{itemize}
}
\item verb \\
If you \textbf{flag} or if your spirits  \textbf{flag} , you begin to lose  enthusiasm or energy.
 \textit{
	\begin{itemize}
	\item His enthusiasm was in no way flagging.
	\item By 4,000m he was beginning to flag.
	\end{itemize}
}
\item countable noun \\
A \textbf{flag} is the same as a flagstone .
 \textit{
	\begin{itemize}
	\end{itemize}
}
\item  \\
 to fly the flag \textit{
	\begin{itemize}
	\end{itemize}
}
\end{enumerate}

\section*{elite}
{\large \color{blue}  elites  }
\subsection*{Explain}
\begin{enumerate}
\item countable noun \\
You can refer to the most powerful, rich, or talented people within a particular group, place,
or society as the \textbf{elite} .
 \textit{
	\begin{itemize}
	\item ...a government comprised mainly of the elite.
	\item We have a political elite in this country.
	\end{itemize}
}
\item adjective \\
\textbf{Elite} people or organizations are considered to be the best of their kind .
 \textit{
	\begin{itemize}
	\item ...the elite troops of the President's bodyguard.
	\end{itemize}
}
\end{enumerate}

\section*{honey}
{\large \color{blue}  honeys  }
\subsection*{Explain}
\begin{enumerate}
\item variable noun \\
\textbf{Honey} is a sweet, sticky , yellowish substance that is made by bees.
 \textit{
	\begin{itemize}
	\end{itemize}
}
\item countable noun \\
You call someone \textbf{honey} as a sign of affection .
 \textit{
	\begin{itemize}
	\item Honey, I don't really think that's a good idea.
	\end{itemize}
}
\end{enumerate}

\section*{entrepreneur}
{\large \color{blue}  entrepreneurs  }
\subsection*{Explain}
\begin{enumerate}
\item countable noun \\
An \textbf{entrepreneur} is a person who sets up businesses and business deals .
 \textit{
	\begin{itemize}
	\end{itemize}
}
\end{enumerate}

\section*{mind}
{\large \color{blue}  minds  }
\subsection*{Explain}
\begin{enumerate}
\item countable noun \\
You refer to someone's \textbf{mind} when talking about their thoughts. For example , if you say that something is \textbf{in your mind} , you mean that you are thinking about it, and if you say that something is \textbf{at the back of your mind} , you mean that you are aware of it, although you are not thinking about it very much.
 \textit{
	\begin{itemize}
	\item I'm trying to clear my mind of all this.
	\item There was no doubt in his mind that the man was serious.
	\item I put what happened during that game to the back of my mind.
	\item He spent the next hour going over the trial in his mind.
	\item She found herself thinking thoughts that would never have entered her mind until
now.
	\end{itemize}
}
\item countable noun \\
Your \textbf{mind} is your ability to think and reason .
 \textit{
	\begin{itemize}
	\item You have a good mind.
	\item Studying stretched my mind and got me thinking about things.
	\item ...an excellent training for the young mind.
	\end{itemize}
}
\item countable noun \\
If you have a particular type of \textbf{mind} , you have a particular way of thinking which is part of your character, or a result
of your education or professional  training .
 \textit{
	\begin{itemize}
	\item Andrew, you have a very suspicious mind.
	\item The key to his success is his logical mind.
	\item ...an American writer who has researched the criminal mind.
	\end{itemize}
}
\item countable noun \\
You can refer to someone as a particular kind of \textbf{mind} as a way of saying that they are clever , intelligent , or imaginative .
 \textit{
	\begin{itemize}
	\item She moved to London, meeting some of the best minds of her time.
	\end{itemize}
}
\item  \\
 to bear in mind \textit{
	\begin{itemize}
	\end{itemize}
}
\item  \\
 to call something to mind \textit{
	\begin{itemize}
	\end{itemize}
}
\item  \\
 to cast your mind back \textit{
	\begin{itemize}
	\end{itemize}
}
\item  \\
 to close your mind \textit{
	\begin{itemize}
	\end{itemize}
}
\item  \\
 to change your mind \textit{
	\begin{itemize}
	\end{itemize}
}
\item  \\
 to come to mind \textit{
	\begin{itemize}
	\end{itemize}
}
\item  \\
 to cross your mind \textit{
	\begin{itemize}
	\end{itemize}
}
\item  \\
 in your mind's eye \textit{
	\begin{itemize}
	\end{itemize}
}
\item  \\
 to have a mind to \textit{
	\begin{itemize}
	\end{itemize}
}
\item  \\
 to have a good mind/half a mind to do \textit{
	\begin{itemize}
	\end{itemize}
}
\item  \\
 to have sth in mind \textit{
	\begin{itemize}
	\end{itemize}
}
\item  \\
 to have it in mind to do \textit{
	\begin{itemize}
	\end{itemize}
}
\item  \\
 with sth in mind \textit{
	\begin{itemize}
	\end{itemize}
}
\item  \\
 in the mind \textit{
	\begin{itemize}
	\end{itemize}
}
\item  \\
 to know your own mind \textit{
	\begin{itemize}
	\end{itemize}
}
\item  \\
 to lose your mind \textit{
	\begin{itemize}
	\end{itemize}
}
\item  \\
 to make up your mind \textit{
	\begin{itemize}
	\end{itemize}
}
\item  \\
 mind over matter \textit{
	\begin{itemize}
	\end{itemize}
}
\item  \\
 of one mind \textit{
	\begin{itemize}
	\end{itemize}
}
\item  \\
 a load off your mind \textit{
	\begin{itemize}
	\end{itemize}
}
\item  \\
 on your mind \textit{
	\begin{itemize}
	\end{itemize}
}
\item  \\
 your mind is on sth/to have your mind on sth \textit{
	\begin{itemize}
	\end{itemize}
}
\item  \\
 an open mind \textit{
	\begin{itemize}
	\end{itemize}
}
\item  \\
 to open your mind \textit{
	\begin{itemize}
	\end{itemize}
}
\item  \\
 out of your mind \textit{
	\begin{itemize}
	\end{itemize}
}
\item  \\
 be/go out of your mind with sth \textit{
	\begin{itemize}
	\end{itemize}
}
\item  \\
 bored/stoned out of your mind \textit{
	\begin{itemize}
	\end{itemize}
}
\item  \\
 to put your mind to sth \textit{
	\begin{itemize}
	\end{itemize}
}
\item  \\
 to put you in mind of \textit{
	\begin{itemize}
	\end{itemize}
}
\item  \\
 to read sb's mind \textit{
	\begin{itemize}
	\end{itemize}
}
\item  \\
 to set/put sb's mind at rest \textit{
	\begin{itemize}
	\end{itemize}
}
\item  \\
 nobody in their right mind \textit{
	\begin{itemize}
	\end{itemize}
}
\item  \\
 to set your mind/have your mind set on sth \textit{
	\begin{itemize}
	\end{itemize}
}
\item  \\
 to slip your mind \textit{
	\begin{itemize}
	\end{itemize}
}
\item  \\
 to speak your mind \textit{
	\begin{itemize}
	\end{itemize}
}
\item  \\
 to stick in your mind \textit{
	\begin{itemize}
	\end{itemize}
}
\item  \\
 to take your mind off \textit{
	\begin{itemize}
	\end{itemize}
}
\item  \\
 to my mind \textit{
	\begin{itemize}
	\end{itemize}
}
\item  \\
 in two minds \textit{
	\begin{itemize}
	\end{itemize}
}
\end{enumerate}

\section*{episode}
{\large \color{blue}  episodes  }
\subsection*{Explain}
\begin{enumerate}
\item countable noun \\
You can refer to an event or a short period of time as an \textbf{episode} if you want to suggest that it is important or unusual , or has some particular quality.
 \textit{
	\begin{itemize}
	\item This episode is bound to be a deep embarrassment for Washington.
	\item Unfortunately it was a rather sordid episode of my life.
	\end{itemize}
}
\item countable noun \\
An \textbf{episode} of something such as a series on radio or television or a story in a magazine is one of the separate parts in which it is broadcast or published .
 \textit{
	\begin{itemize}
	\item The final episode will be shown next Sunday.
	\end{itemize}
}
\item countable noun \\
An \textbf{episode}  \textbf{of} an illness is short period in which a person who suffers from it is affected by it particularly  badly .
 \textit{
	\begin{itemize}
	\item ...people who'd suffered a third episode of depression in two years.
	\item The new drug lessens the severity of pneumonia episodes.
	\end{itemize}
}
\end{enumerate}

\section*{miniature}
{\large \color{blue}  miniatures  }
\subsection*{Explain}
\begin{enumerate}
\item adjective \\
\textbf{Miniature} is used to describe something which is very small, especially a smaller version of something which is normally much bigger .
 \textit{
	\begin{itemize}
	\item ...miniature roses.
	\item He looked like a miniature version of his handsome and elegant big brother.
	\end{itemize}
}
\item  \\
 in miniature \textit{
	\begin{itemize}
	\end{itemize}
}
\item countable noun \\
A \textbf{miniature} is a very small detailed painting, often of a person.
 \textit{
	\begin{itemize}
	\end{itemize}
}
\item countable noun \\
A \textbf{miniature} is a very small bottle of strong alcohol such as whisky or brandy , and usually contains enough for one or two drinks.
 \textit{
	\begin{itemize}
	\end{itemize}
}
\end{enumerate}

\section*{february}
{\large \color{blue}  Februaries  }
\subsection*{Explain}
\begin{enumerate}
\item variable noun \\
\textbf{February} is the second month of the year in the Western  calendar .
 \textit{
	\begin{itemize}
	\item He joined the Army in February 1943.
	\item His exhibition opens on 5 February.
	\item Last February the tribunal agreed he had been the victim of racial discrimination.
	\end{itemize}
}
\end{enumerate}

\section*{nose}
{\large \color{blue}  noses  nosing  nosed  }
\subsection*{Explain}
\begin{enumerate}
\item countable noun \\
Your \textbf{nose} is the part of your face which sticks out above your mouth . You use it for smelling and breathing .
 \textit{
	\begin{itemize}
	\item She wiped her nose with a tissue.
	\item She's got funny eyes and a big nose.
	\end{itemize}
}
\item countable noun \\
The \textbf{nose} of a vehicle such as a car or aeroplane is the front part of it.
 \textit{
	\begin{itemize}
	\item Sue parked off the main street, with the van's nose pointing away from the street.
	\end{itemize}
}
\item countable noun \\
You can refer to your sense of smell as your \textbf{nose} .
 \textit{
	\begin{itemize}
	\item The river that runs through Middlesbrough became ugly on the eye and hard on the
nose.
	\end{itemize}
}
\item singular noun \\
If a horse wins a race  \textbf{by a}  \textbf{nose} , it wins by a very small distance .
 \textit{
	\begin{itemize}
	\item Chirkpar rattled past him on the right to snatch the prize by a nose.
	\end{itemize}
}
\item verb \\
If a vehicle \textbf{noses} in a certain direction or if you \textbf{nose} it there, you move it slowly and carefully in that direction.
 \textit{
	\begin{itemize}
	\item He could not see the driver as the car nosed forward.
	\item A motorboat nosed out of the mist and nudged into the branches of a tree.
	\item Ben drove past them, nosing his car into the garage.
	\end{itemize}
}
\item  \\
 to keep your nose clean \textit{
	\begin{itemize}
	\end{itemize}
}
\item  \\
 to follow your nose \textit{
	\begin{itemize}
	\end{itemize}
}
\item  \\
 follow one's nose \textit{
	\begin{itemize}
	\end{itemize}
}
\item  \\
 have a nose for sth \textit{
	\begin{itemize}
	\end{itemize}
}
\item  \\
 get up sb's nose \textit{
	\begin{itemize}
	\end{itemize}
}
\item  \\
 to look down your nose at someone \textit{
	\begin{itemize}
	\end{itemize}
}
\item  \\
 to pay through the nose \textit{
	\begin{itemize}
	\end{itemize}
}
\item  \\
 to poke your nose into something \textit{
	\begin{itemize}
	\end{itemize}
}
\item  \\
 rub sb's nose in sth \textit{
	\begin{itemize}
	\end{itemize}
}
\item  \\
 to cut off your nose to spite your face \textit{
	\begin{itemize}
	\end{itemize}
}
\item  \\
 nose to tail \textit{
	\begin{itemize}
	\end{itemize}
}
\item  \\
 to thumb your nose at someone \textit{
	\begin{itemize}
	\end{itemize}
}
\item  \\
 turn up one's nose at sth \textit{
	\begin{itemize}
	\end{itemize}
}
\item  \\
 under sb's nose \textit{
	\begin{itemize}
	\end{itemize}
}
\end{enumerate}

\section*{fellow}
{\large \color{blue}  fellows  }
\subsection*{Explain}
\begin{enumerate}
\item adjective \\
You use \textbf{fellow} to describe people who are in the same situation as you, or people you feel you have something in common with.
 \textit{
	\begin{itemize}
	\item She discovered to her pleasure, a talent for making her fellow guests laugh.
	\item Even in jail, my fellow inmates treated me with kindness.
	\end{itemize}
}
\item countable noun \\
A \textbf{fellow} is a man or boy.
 \textit{
	\begin{itemize}
	\item By all accounts, Rodger would appear to be a fine fellow.
	\end{itemize}
}
\item plural noun \\
Your \textbf{fellows} are the people who you work with, do things with, or who are like you in some way.
 \textit{
	\begin{itemize}
	\item He stood out in terms of competence from all his fellows.
	\item People looked out for one another and were concerned about the welfare of their fellows.
	\end{itemize}
}
\item countable noun \\
A \textbf{fellow}  \textbf{of} an academic or professional  association is someone who is a specially elected member of it, usually because of their work or achievements or as a mark of honour .
 \textit{
	\begin{itemize}
	\item ...the fellows of the Zoological Society of London.
	\end{itemize}
}
\end{enumerate}

\section*{orchestra}
{\large \color{blue}  orchestras  }
\subsection*{Explain}
\begin{enumerate}
\item countable noun \\
An \textbf{orchestra} is a large group of musicians who play a variety of different instruments together . Orchestras usually play classical  music .
 \textit{
	\begin{itemize}
	\item ...the Royal Liverpool Philharmonic Orchestra.
	\end{itemize}
}
\item singular noun \\
\textbf{The}  \textbf{orchestra} or \textbf{the}  \textbf{orchestra seats} in a theatre or concert  hall are the seats on the ground  floor  directly in front of the stage.
 \textit{
	\begin{itemize}
	\end{itemize}
}
\end{enumerate}

\section*{gang}
{\large \color{blue}  gangs  ganging  ganged  }
\subsection*{Explain}
\begin{enumerate}
\item countable noun \\
A \textbf{gang} is a group of people, especially young people, who go around together and often deliberately cause trouble .
 \textit{
	\begin{itemize}
	\item During the fight with a rival gang he lashed out with his flick knife.
	\item Gang members were behind a lot of the violence.
	\item He was attacked by a gang of youths.
	\end{itemize}
}
\item countable noun \\
A \textbf{gang} is a group of criminals who work together to commit  crimes .
 \textit{
	\begin{itemize}
	\item Police were hunting for a gang who had allegedly stolen fifty-five cars.
	\item ...an underworld gang.
	\item ...a gang of masked robbers.
	\end{itemize}
}
\item singular noun \\
\textbf{The}  \textbf{gang} is a group of friends who frequently meet .
 \textit{
	\begin{itemize}
	\item Come on over, we've got lots of the old gang here.
	\end{itemize}
}
\item countable noun \\
A \textbf{gang} is a group of workers who do physical work together.
 \textit{
	\begin{itemize}
	\item ...a gang of labourers.
	\end{itemize}
}
\end{enumerate}

\section*{painter}
{\large \color{blue}  painters  }
\subsection*{Explain}
\begin{enumerate}
\item countable noun \\
A \textbf{painter} is an artist who paints pictures.
 \textit{
	\begin{itemize}
	\end{itemize}
}
\item countable noun \\
A \textbf{painter} is someone who paints walls , doors , and some other parts of buildings as their job .
 \textit{
	\begin{itemize}
	\end{itemize}
}
\end{enumerate}

\section*{handful}
{\large \color{blue}  handfuls  }
\subsection*{Explain}
\begin{enumerate}
\item singular noun \\
A \textbf{handful}  \textbf{of} people or things is a small number of them.
 \textit{
	\begin{itemize}
	\item He surveyed the handful of customers at the bar.
	\item One spring morning a handful of potential investors assembled in Quincy.
	\end{itemize}
}
\item countable noun \\
A \textbf{handful}  \textbf{of} something is the amount of it that you can hold in your hand.
 \textit{
	\begin{itemize}
	\item She scooped up a handful of sand and let it trickle through her fingers.
	\end{itemize}
}
\item singular noun \\
If you say that someone, especially a child, is a \textbf{handful} , you mean that they are difficult to control.
 \textit{
	\begin{itemize}
	\item Zara can be a handful sometimes.
	\end{itemize}
}
\end{enumerate}

\section*{panda}
{\large \color{blue}  pandas  }
\subsection*{Explain}
\begin{enumerate}
\item countable noun \\
A \textbf{panda} or a \textbf{giant panda} is a large animal rather like a bear, which has black and white fur and lives in the bamboo forests of China.
 \textit{
	\begin{itemize}
	\end{itemize}
}
\end{enumerate}

\section*{horsepower}
{\large \color{blue}  }
\subsection*{Explain}
\begin{enumerate}
\item uncountable noun \\
\textbf{Horsepower} is a unit of power used for measuring how powerful an engine is.
 \textit{
	\begin{itemize}
	\item The engine has more than 4,000 horsepower.
	\item ...a 300-horsepower engine.
	\end{itemize}
}
\end{enumerate}

\section*{pipe}
{\large \color{blue}  pipes  piping  piped  }
\subsection*{Explain}
\begin{enumerate}
\item countable noun \\
A \textbf{pipe} is a long, round, hollow object, usually made of metal or plastic, through which
a liquid or gas can flow.
 \textit{
	\begin{itemize}
	\item They had accidentally damaged a gas pipe while drilling.
	\item The plant makes plastic covered steel pipes for the oil and gas industries.
	\end{itemize}
}
\item countable noun \\
A \textbf{pipe} is an object which is used for smoking tobacco. You put the tobacco into the cup-shaped
part at the end of the pipe, light it, and breathe in the smoke through a narrow tube.
 \textit{
	\begin{itemize}
	\end{itemize}
}
\item countable noun \\
A \textbf{pipe} is a simple musical instrument in the shape of a tube with holes in it. You play
a pipe by blowing into it while covering and uncovering the holes with your fingers .
 \textit{
	\begin{itemize}
	\end{itemize}
}
\item plural noun \\
\textbf{Pipes} are the same as bagpipes .
 \textit{
	\begin{itemize}
	\end{itemize}
}
\item countable noun \\
An \textbf{organ pipe} is one of the long hollow tubes in which air vibrates and produces a musical note.
 \textit{
	\begin{itemize}
	\end{itemize}
}
\item verb \\
If liquid or gas \textbf{is piped}  somewhere , it is transferred from one place to another through a pipe.
 \textit{
	\begin{itemize}
	\item The heated gas is piped through a coil surrounded by water.
	\item The villagers piped in drinking water from the reservoir.
	\item Most of the houses in the capital don't have piped water.
	\end{itemize}
}
\item verb \\
If someone, especially a child, \textbf{pipes} something, they say it in a high-pitched voice.
 \textit{
	\begin{itemize}
	\item 'But I want to help,' Bessie piped.
	\end{itemize}
}
\item countable noun \\
In some computer systems, a \textbf{pipe} is a method of passing information from one program to another.
 \textit{
	\begin{itemize}
	\end{itemize}
}
\end{enumerate}

\section*{january}
{\large \color{blue}  Januaries  }
\subsection*{Explain}
\begin{enumerate}
\item variable noun \\
\textbf{January} is the first month of the year in the Western  calendar .
 \textit{
	\begin{itemize}
	\item We always have snow in January.
	\item She was born on January 6, 1946.
	\item I haven't seen my own daughter since last January.
	\end{itemize}
}
\end{enumerate}

\section*{plague}
{\large \color{blue}  plagues  plaguing  plagued  }
\subsection*{Explain}
\begin{enumerate}
\item countable noun \\
A \textbf{plague} is a very infectious disease that spreads quickly and kills large numbers of people.
 \textit{
	\begin{itemize}
	\item A cholera plague had been killing many prisoners of war at the time.
	\end{itemize}
}
\item uncountable noun \\
\textbf{Plague} or \textbf{the plague} is a very infectious disease which usually results in death . The patient has a severe  fever and swellings on his or her body.
 \textit{
	\begin{itemize}
	\item ...a fresh outbreak of plague.
	\item ...illnesses such as smallpox, typhus and the plague.
	\end{itemize}
}
\item countable noun \\
A \textbf{plague of}  unpleasant things is a large number of them that arrive or happen at the same time.
 \textit{
	\begin{itemize}
	\item The city is under threat from a plague of rats.
	\item Last year there was a plague of robbery and housebreaking.
	\end{itemize}
}
\item countable noun \\
If you describe something as a \textbf{plague} , you mean that it causes a great deal of trouble or harm .
 \textit{
	\begin{itemize}
	\item Inflation will remain a recurrent plague.
	\item Tim seems to have escaped the cynicism which is the absolute plague of our generation.
	\end{itemize}
}
\item verb \\
If you \textbf{are}  \textbf{plagued}  \textbf{by} unpleasant things, they continually cause you a lot of trouble or suffering .
 \textit{
	\begin{itemize}
	\item She was plagued by weakness, fatigue, and dizziness.
	\item Fears about job security plague nearly half the workforce.
	\end{itemize}
}
\item verb \\
If someone \textbf{plagues} you, they keep bothering you or asking you for something.
 \textit{
	\begin{itemize}
	\item I'm not going to plague you with a lot more questions.
	\item Tommy Cook had been plaguing Pinner for months.
	\end{itemize}
}
\item  \\
 to avoid someone or something like the plague \textit{
	\begin{itemize}
	\end{itemize}
}
\item  \\
 a plague on sb/sth \textit{
	\begin{itemize}
	\end{itemize}
}
\end{enumerate}

\section*{july}
{\large \color{blue}  Julys  }
\subsection*{Explain}
\begin{enumerate}
\item variable noun \\
\textbf{July} is the seventh month of the year in the Western  calendar .
 \textit{
	\begin{itemize}
	\item In late July 1914, he and Violet spent a few days with friends near Berwick-upon-Tweed.
	\item I expect you to report for work on 28 July.
	\item She met him for the first time last July.
	\end{itemize}
}
\end{enumerate}

\section*{precision}
{\large \color{blue}  }
\subsection*{Explain}
\begin{enumerate}
\item uncountable noun \\
If you do something \textbf{with}  \textbf{precision} , you do it exactly as it should be done .
 \textit{
	\begin{itemize}
	\item The choir sang with precision.
	\item The interior is planned with a precision the military would be proud of.
	\end{itemize}
}
\end{enumerate}

\section*{lifetime}
{\large \color{blue}  lifetimes  }
\subsection*{Explain}
\begin{enumerate}
\item countable noun \\
A \textbf{lifetime} is the length of time that someone is alive.
 \textit{
	\begin{itemize}
	\item During my lifetime I haven't got around to much travelling.
	\item ...a trust fund to be administered throughout his wife's lifetime.
	\item ...an extraordinary lifetime of achievement.
	\end{itemize}
}
\item singular noun \\
The \textbf{lifetime} of a particular thing is the period of time that it lasts .
 \textit{
	\begin{itemize}
	\item ...the lifetime of a parliament.
	\item ...a satellite's lifetime.
	\end{itemize}
}
\item  \\
 of a lifetime \textit{
	\begin{itemize}
	\end{itemize}
}
\end{enumerate}

\section*{psychiatry}
{\large \color{blue}  }
\subsection*{Explain}
\begin{enumerate}
\item uncountable noun \\
\textbf{Psychiatry} is the branch of medicine concerned with the treatment of mental illness.
 \textit{
	\begin{itemize}
	\end{itemize}
}
\end{enumerate}

\section*{loaf}
{\large \color{blue}  loaves  loafs  loafing  loafed  }
\subsection*{Explain}
\begin{enumerate}
\item countable noun \\
A \textbf{loaf} of bread is bread which has been shaped and baked in one piece. It is usually large
enough for more than one person and can be cut into slices .
 \textit{
	\begin{itemize}
	\item ...a loaf of crusty bread.
	\item ...freshly baked loaves.
	\end{itemize}
}
\item verb \\
If you \textbf{loaf} , you stand or wait in a place, not doing anything interesting or useful .
 \textbf{Loaf around}  means the same as loaf .
 \textit{
	\begin{itemize}
	\item Soldiers loafed at street corners.
	\item We had been at Cambridge together, she studying medicine and me loafing around.
	\end{itemize}
}
\end{enumerate}

\section*{race}
{\large \color{blue}  races  racing  raced  }
\subsection*{Explain}
\begin{enumerate}
\item countable noun \\
A \textbf{race} is a competition to see who is the fastest, for example in running, swimming, or
driving.
 \textit{
	\begin{itemize}
	\item The first Le Mans 24-hour motor race was won by André Lagache and René Léonard.
	\end{itemize}
}
\item verb \\
If you \textbf{race} , you take part in a race.
 \textit{
	\begin{itemize}
	\item In the 10 years I raced in Europe, 30 drivers were killed.
	\item They may even have raced each other–but not regularly.
	\end{itemize}
}
\item plural noun \\
\textbf{The races} are a series of horse races that are held in a particular place on a particular day.
People go to watch and to bet on which horse will win.
 \textit{
	\begin{itemize}
	\item The high point of this trip was a day at the races.
	\end{itemize}
}
\item countable noun \\
A \textbf{race} is a situation in which people or organizations compete with each other for power or control.
 \textit{
	\begin{itemize}
	\item The race for the White House begins in earnest today.
	\item The race is on to build up membership fast.
	\end{itemize}
}
\item variable noun \\
A \textbf{race} is one of the major groups which human beings can be divided into according to their
physical features, such as the colour of their skin.
 \textit{
	\begin{itemize}
	\item The College welcomes students of all races, faiths, and nationalities.
	\item Discrimination by employers on the grounds of race and nationality was illegal.
	\end{itemize}
}
\item verb \\
If you \textbf{race}  somewhere , you go there as quickly as possible.
 \textit{
	\begin{itemize}
	\item He raced across town to the State House building.
	\item The hares raced away out of sight.
	\end{itemize}
}
\item verb \\
If something \textbf{races} towards a particular state or position, it moves very fast towards that state or
position.
 \textit{
	\begin{itemize}
	\item Do they realize we are racing towards complete economic collapse?
	\item American economic growth raced ahead.
	\end{itemize}
}
\item verb \\
If you \textbf{race} a vehicle or animal, you prepare it for races and make it take part in races.
 \textit{
	\begin{itemize}
	\item He still raced sports cars as often as he could.
	\end{itemize}
}
\item verb \\
If your mind \textbf{races} , or if thoughts \textbf{race} through your mind, you think very fast about something, especially when you are in a difficult or dangerous situation.
 \textit{
	\begin{itemize}
	\item I made sure I sounded calm but my mind was racing.
	\item Already her mind was racing ahead to the hundred and one things she had to do.
	\item Bits and pieces of the past raced through her mind.
	\end{itemize}
}
\item verb \\
If your heart  \textbf{races} , it beats very quickly because you are excited or afraid .
 \textit{
	\begin{itemize}
	\item Her heart raced uncontrollably.
	\end{itemize}
}
\item  \\
 a race against time \textit{
	\begin{itemize}
	\end{itemize}
}
\end{enumerate}

\section*{nitrogen}
{\large \color{blue}  }
\subsection*{Explain}
\begin{enumerate}
\item uncountable noun \\
\textbf{Nitrogen} is a colourless element that has no smell and is usually found as a gas. It forms about 78% of the earth's atmosphere , and is found in all living things.
 \textit{
	\begin{itemize}
	\end{itemize}
}
\end{enumerate}

\section*{saddle}
{\large \color{blue}  saddles  saddling  saddled  }
\subsection*{Explain}
\begin{enumerate}
\item countable noun \\
A \textbf{saddle} is a leather seat that you put on the back of an animal so that you can ride the animal.
 \textit{
	\begin{itemize}
	\end{itemize}
}
\item verb \\
If you \textbf{saddle} a horse, you put a saddle on it so that you can ride it.
 \textbf{Saddle up}  means the same as saddle .
 \textit{
	\begin{itemize}
	\item Why don't we saddle a couple of horses and go for a ride?
	\item I want to be gone from here as soon as we can saddle up.
	\item She saddled up a horse.
	\end{itemize}
}
\item countable noun \\
A \textbf{saddle} is a seat on a bicycle or motorcycle .
 \textit{
	\begin{itemize}
	\end{itemize}
}
\item countable noun \\
A \textbf{saddle}  \textbf{of}  lamb , rabbit , or venison is a piece of meat taken from the middle of the animal's back.
 \textit{
	\begin{itemize}
	\end{itemize}
}
\item verb \\
If you \textbf{saddle} someone \textbf{with} a problem or \textbf{with} a responsibility , you put them in a position where they have to deal with it.
 \textit{
	\begin{itemize}
	\item The war devastated the economy and saddled the country with a huge foreign debt.
	\end{itemize}
}
\item  \\
 in the saddle \textit{
	\begin{itemize}
	\end{itemize}
}
\item  \\
 in the saddle \textit{
	\begin{itemize}
	\end{itemize}
}
\end{enumerate}

\section*{november}
{\large \color{blue}  Novembers  }
\subsection*{Explain}
\begin{enumerate}
\item variable noun \\
\textbf{November} is the eleventh month of the year in the Western  calendar .
 \textit{
	\begin{itemize}
	\item He arrived in London in November 1939.
	\item He died on 24 November 2001, aged 80.
	\item There's no telling what the voters will do next November.
	\end{itemize}
}
\end{enumerate}

\section*{shoe}
{\large \color{blue}  shoes  shoeing  shoed  shod  }
\subsection*{Explain}
\begin{enumerate}
\item countable noun \\
\textbf{Shoes} are objects which you wear on your feet. They cover most of your foot and you wear
them over socks or stockings .
 \textit{
	\begin{itemize}
	\item ...a pair of shoes.
	\item Low-heeled comfortable shoes are best.
	\item You don't mind if I take my shoes off, do you?
	\end{itemize}
}
\item countable noun \\
A \textbf{shoe} is the same as a horseshoe .
 \textit{
	\begin{itemize}
	\end{itemize}
}
\item verb \\
When a blacksmith  \textbf{shoes} a horse, they fix horseshoes onto its feet.
 \textit{
	\begin{itemize}
	\item Blacksmiths spent most of their time repairing tools and shoeing horses.
	\item He helped his father by holding the horses steady while they were being shod.
	\end{itemize}
}
\item  \\
 fill sb's shoes/step into sb's shoes \textit{
	\begin{itemize}
	\end{itemize}
}
\item  \\
 be in sb's shoes \textit{
	\begin{itemize}
	\end{itemize}
}
\end{enumerate}

\section*{october}
{\large \color{blue}  Octobers  }
\subsection*{Explain}
\begin{enumerate}
\item variable noun \\
\textbf{October} is the tenth month of the year in the Western  calendar .
 \textit{
	\begin{itemize}
	\item Most seasonal hiring is done in early October.
	\item The first plane is due to leave on 2 October.
	\item My grandson has been away since last October.
	\end{itemize}
}
\end{enumerate}

\section*{spectrum}
{\large \color{blue}  spectra  spectrums  }
\subsection*{Explain}
\begin{enumerate}
\item singular noun \\
\textbf{The spectrum} is the range of different colours which is produced when light passes through a glass prism or through a drop of water. A rainbow  shows the colours in the spectrum.
 \textit{
	\begin{itemize}
	\end{itemize}
}
\item countable noun \\
A \textbf{spectrum} is a range of a particular type of thing.
 \textit{
	\begin{itemize}
	\item She'd seen his moods range across the emotional spectrum.
	\item Politicians across the political spectrum have denounced the act.
	\item The term 'special needs' covers a wide spectrum of problems.
	\end{itemize}
}
\item countable noun \\
A \textbf{spectrum} is a range of light waves or radio waves within particular frequencies.
 \textit{
	\begin{itemize}
	\item ...from X-rays right through the spectrum down to radio waves.
	\item ...the individual colours within the light spectrum.
	\item ...the ultraviolet spectra of hot stars.
	\end{itemize}
}
\end{enumerate}

\section*{pair}
{\large \color{blue}  pairs  pairing  paired  }
\subsection*{Explain}
\begin{enumerate}
\item countable noun \\
A \textbf{pair}  \textbf{of} things are two things of the same size and shape that are used together or are both part of something, for example  shoes , earrings , or parts of the body.
 \textit{
	\begin{itemize}
	\item ...a pair of socks.
	\item ...trainers that cost up to 90 pounds a pair.
	\item 72,000 pairs of hands clapped in unison to the song.
	\end{itemize}
}
\item countable noun \\
Some objects that have two main parts of the same size and shape are referred to as a \textbf{pair} , for example \textbf{a pair of trousers} or \textbf{a pair of scissors} .
 \textit{
	\begin{itemize}
	\item ...a pair of faded jeans.
	\item ...a pair of binoculars.
	\end{itemize}
}
\item singular noun \\
You can refer to two people as a \textbf{pair} when they are standing or walking together or when they have some kind of relationship with each other.
 \textit{
	\begin{itemize}
	\item John and Jeremy are a pair of friends who work together at a law firm.
	\item The pair announced their separation last month.
	\item He and Paula made an unlikely pair.
	\end{itemize}
}
\item verb \\
If one thing \textbf{is paired with} another, it is put with it or considered with it.
 \textit{
	\begin{itemize}
	\item The trainees will then be paired with experienced managers.
	\end{itemize}
}
\item  \\
 a safe pair of hands \textit{
	\begin{itemize}
	\end{itemize}
}
\end{enumerate}

\section*{speech}
{\large \color{blue}  speeches  }
\subsection*{Explain}
\begin{enumerate}
\item uncountable noun \\
\textbf{Speech} is the ability to speak or the act of speaking.
 \textit{
	\begin{itemize}
	\item ...the development of speech in children.
	\item ...a speech therapist specialising in stammering.
	\end{itemize}
}
\item singular noun \\
Your \textbf{speech} is the way in which you speak.
 \textit{
	\begin{itemize}
	\item His speech became increasingly thick and nasal.
	\item I'd make fun of her dress and imitate her speech.
	\end{itemize}
}
\item uncountable noun \\
\textbf{Speech} is spoken language.
 \textit{
	\begin{itemize}
	\item He could imitate in speech or writing most of those he admired.
	\item ...the way common letter clusters are usually pronounced in speech.
	\end{itemize}
}
\item countable noun \\
A \textbf{speech} is a formal talk which someone gives to an audience.
 \textit{
	\begin{itemize}
	\item She is due to make a speech on the economy next week.
	\item He delivered his speech in French.
	\item ...a dramatic resignation speech.
	\end{itemize}
}
\item countable noun \\
A \textbf{speech} is a group of lines spoken by a character in a play.
 \textit{
	\begin{itemize}
	\item ...a great actor delivering a key speech from Hamlet.
	\end{itemize}
}
\end{enumerate}

\section*{spider}
{\large \color{blue}  spiders  }
\subsection*{Explain}
\begin{enumerate}
\item countable noun \\
A \textbf{spider} is a small creature with eight legs. Most types of spider make structures called  webs in which they catch  insects for food.
 \textit{
	\begin{itemize}
	\end{itemize}
}
\end{enumerate}

\section*{population}
{\large \color{blue}  populations  }
\subsection*{Explain}
\begin{enumerate}
\item countable noun \\
The \textbf{population} of a country or area is all the people who live in it.
 \textit{
	\begin{itemize}
	\item Bangladesh now has a population of about 110 million.
	\item ...the annual rate of population growth.
	\item ...the local population.
	\end{itemize}
}
\item countable noun \\
If you refer to a particular type of \textbf{population} in a country or area, you are referring to all the people or animals of that type
there.
 \textit{
	\begin{itemize}
	\item ...75.6 per cent of the male population over sixteen.
	\item ...areas with a large black population.
	\item ...the elephant populations of Tanzania and Kenya.
	\end{itemize}
}
\end{enumerate}

\section*{spirit}
{\large \color{blue}  spirits  spiriting  spirited  }
\subsection*{Explain}
\begin{enumerate}
\item singular noun \\
Your \textbf{spirit} is the part of you that is not physical and that consists of your character and feelings.
 \textit{
	\begin{itemize}
	\item The human spirit is virtually indestructible.
	\item Marian retains a restless, youthful spirit, in search of new horizons.
	\end{itemize}
}
\item countable noun \\
A person's \textbf{spirit} is the non-physical part of them that is believed to remain  alive after their death .
 \textit{
	\begin{itemize}
	\item His spirit has left him and all that remains is the shell of his body.
	\end{itemize}
}
\item countable noun \\
A \textbf{spirit} is a ghost or supernatural being.
 \textit{
	\begin{itemize}
	\item ...protection against evil spirits.
	\end{itemize}
}
\item uncountable noun \\
\textbf{Spirit} is the courage and determination that helps people to survive in difficult times and to keep their way of life and their beliefs .
 \textit{
	\begin{itemize}
	\item She was a very brave girl and everyone who knew her admired her spirit.
	\end{itemize}
}
\item uncountable noun \\
\textbf{Spirit} is the liveliness and energy that someone shows in what they do.
 \textit{
	\begin{itemize}
	\item They played with spirit.
	\end{itemize}
}
\item singular noun \\
The \textbf{spirit} in which you do something is the attitude you have when you are doing it.
 \textit{
	\begin{itemize}
	\item Their problem can only be solved in a spirit of compromise.
	\item They approached the talks in a conciliatory spirit.
	\end{itemize}
}
\item uncountable noun \\
A particular kind of \textbf{spirit} is the feeling of loyalty to a group that is shared by the people who belong to the group.
 \textit{
	\begin{itemize}
	\item There is a great sense of team spirit among the British squad.
	\item The president has appealed to the people for patriotism and community spirit.
	\end{itemize}
}
\item singular noun \\
A particular kind of \textbf{spirit} is the set of ideas , beliefs, and aims that are held by a group of people.
 \textit{
	\begin{itemize}
	\item ...the real spirit of the Labour movement.
	\end{itemize}
}
\item singular noun \\
\textbf{The spirit of} something such as a law or an agreement is the way that it was intended to be interpreted or applied.
 \textit{
	\begin{itemize}
	\item The requirement for work permits violates the spirit of the 1950 treaty.
	\end{itemize}
}
\item countable noun \\
You can refer to a person as a particular kind of \textbf{spirit} if they show a certain characteristic or if they show a lot of enthusiasm in what they are doing.
 \textit{
	\begin{itemize}
	\item I like to think of myself as a free spirit.
	\item He was the founder and guiding spirit of New York's Shakespeare Festival.
	\end{itemize}
}
\item plural noun \\
Your \textbf{spirits} are your feelings at a particular time, especially feelings of happiness or unhappiness.
 \textit{
	\begin{itemize}
	\item At supper, everyone was in high spirits.
	\item A bit of exercise will help lift his spirits.
	\end{itemize}
}
\item verb \\
If someone or something \textbf{is spirited}  \textbf{away} , or if they \textbf{are spirited}  \textbf{out of}  somewhere , they are taken from a place quickly and secretly without anyone noticing .
 \textit{
	\begin{itemize}
	\item He was spirited away and probably murdered.
	\item His parents had spirited him away to the country.
	\item It is possible that he has been spirited out of the country.
	\end{itemize}
}
\item plural noun \\
\textbf{Spirits} are strong alcoholic drinks such as whisky and gin.
 \textit{
	\begin{itemize}
	\end{itemize}
}
\item uncountable noun \\
\textbf{Spirit} or \textbf{spirits} is an alcoholic liquid that is used as a fuel , for cleaning things, or for other purposes. There are many kinds of spirit.
 \textit{
	\begin{itemize}
	\end{itemize}
}
\item  \\
 enter into the spirit \textit{
	\begin{itemize}
	\end{itemize}
}
\item  \\
 in spirit \textit{
	\begin{itemize}
	\end{itemize}
}
\item  \\
 in spirit \textit{
	\begin{itemize}
	\end{itemize}
}
\item  \\
 the spirit of the age/the spirit of the times \textit{
	\begin{itemize}
	\end{itemize}
}
\end{enumerate}

\section*{portion}
{\large \color{blue}  portions  }
\subsection*{Explain}
\begin{enumerate}
\item countable noun \\
A \textbf{portion of} something is a part of it.
 \textit{
	\begin{itemize}
	\item Damage was confined to a small portion of the castle.
	\item I have spent a fairly considerable portion of my life here.
	\item I had learnt a portion of the Koran.
	\item Insurance can represent a significant portion of the total price of a holiday.
	\end{itemize}
}
\item countable noun \\
A \textbf{portion} is the amount of food that is given to one person at a meal .
 \textit{
	\begin{itemize}
	\item Desserts can be substituted by a portion of fresh fruit.
	\item The portions were generous.
	\item ...fish and chips at about £2.70 a portion.
	\end{itemize}
}
\end{enumerate}

\section*{stability}
{\large \color{blue}  }
\subsection*{Explain}
\begin{enumerate}
\end{enumerate}

\section*{puff}
{\large \color{blue}  puffs  puffing  puffed  }
\subsection*{Explain}
\begin{enumerate}
\item verb \\
If someone \textbf{puffs}  \textbf{at} a cigarette, cigar, or pipe, they smoke it.
 \textbf{Puff} is also a noun .
 \textit{
	\begin{itemize}
	\item He lit a cigar and puffed at it twice.
	\item He nodded and puffed on a stubby pipe as he listened.
	\item She was taking quick puffs at her cigarette.
	\end{itemize}
}
\item verb \\
If you \textbf{puff} smoke or moisture from your mouth or if it \textbf{puffs} from your mouth, you breathe it out.
 \textbf{Puff out}  means the same as puff .
 \textit{
	\begin{itemize}
	\item Richard puffed smoke towards the ceiling.
	\item The weather was dry and cold; wisps of steam puffed from their lips.
	\item He puffed out a cloud of smoke.
	\end{itemize}
}
\item verb \\
If an engine , chimney , or boiler  \textbf{puffs} smoke or steam , clouds of smoke or steam come out of it.
 \textit{
	\begin{itemize}
	\item As I completed my 26th lap the Porsche puffed blue smoke.
	\end{itemize}
}
\item countable noun \\
A \textbf{puff of} something such as air or smoke is a small amount of it that is blown out from somewhere .
 \textit{
	\begin{itemize}
	\item Wind caught the sudden puff of dust and blew it inland.
	\end{itemize}
}
\item verb \\
If you \textbf{are puffing} , you are breathing loudly and quickly with your mouth open because you are out of breath after a lot of physical  effort .
 \textit{
	\begin{itemize}
	\item I know nothing about boxing, but I could see he was unfit, because he was puffing.
	\end{itemize}
}
\item countable noun \\
A \textbf{puff}  \textbf{for} something such as a book, film , product, or organization is something that is done or said in order to attract people's attention and tell them how good it is.
 \textbf{Puff} is also a verb .
 \textit{
	\begin{itemize}
	\item Sometimes there is a gigantic puff for a commercial show.
	\item He puffed the new system by showing how badly his existing system performed by comparison.
	\end{itemize}
}
\item countable noun \\
A \textbf{puff} is the same as a poof .
 \textit{
	\begin{itemize}
	\end{itemize}
}
\end{enumerate}

\section*{stage}
{\large \color{blue}  stages  staging  staged  }
\subsection*{Explain}
\begin{enumerate}
\item countable noun \\
A \textbf{stage}  \textbf{of} an activity, process, or period is one part of it.
 \textit{
	\begin{itemize}
	\item The way children express their feelings depends on their stage of development.
	\item Mr Cook has arrived in Greece on the final stage of a tour which also included Egypt
and Israel.
	\end{itemize}
}
\item countable noun \\
In a theatre, the \textbf{stage} is an area where actors or other entertainers perform.
 \textit{
	\begin{itemize}
	\item The road crew needed more than 24 hours to move and rebuild the stage after a concert.
	\item I went on stage and did my show.
	\end{itemize}
}
\item singular noun \\
You can refer to acting and the production of plays in a theatre as \textbf{the stage} .
 \textit{
	\begin{itemize}
	\item Madge did not want to put her daughter on the stage.
	\item He was the first comedian I ever saw on the stage.
	\end{itemize}
}
\item verb \\
If someone \textbf{stages} a play or other show , they organize and present a performance of it.
 \textit{
	\begin{itemize}
	\item Maya Angelou first staged the play 'And I Still Rise' in the late 1970s.
	\end{itemize}
}
\item verb \\
If you \textbf{stage} an event or ceremony , you organize it and usually take part in it.
 \textit{
	\begin{itemize}
	\item Workers have staged a number of strikes in protest.
	\item At the middle of this year the government staged a huge military parade.
	\end{itemize}
}
\item singular noun \\
You can refer to a particular area of activity as a particular \textbf{stage} , especially when you are talking about politics .
 \textit{
	\begin{itemize}
	\item He was finally forced off the political stage by the deterioration of his health.
	\item The E.U. thought it could boost its credibility as a strong actor on the international
stage.
	\end{itemize}
}
\end{enumerate}

\section*{rectangle}
{\large \color{blue}  rectangles  }
\subsection*{Explain}
\begin{enumerate}
\item countable noun \\
A \textbf{rectangle} is a four-sided shape whose corners are all ninety  degree angles. Each side of a rectangle is the same length as the one opposite to it.
 \textit{
	\begin{itemize}
	\end{itemize}
}
\end{enumerate}

\section*{stool}
{\large \color{blue}  stools  }
\subsection*{Explain}
\begin{enumerate}
\item countable noun \\
A \textbf{stool} is a seat with legs but no support for your arms or back.
 \textit{
	\begin{itemize}
	\item O'Brien sat on a bar stool and leaned his elbows on the counter.
	\end{itemize}
}
\item  \\
 fall between two stools \textit{
	\begin{itemize}
	\end{itemize}
}
\item countable noun \\
\textbf{Stools} are the pieces of solid waste matter that are passed out of a person's body through their bowels.
 \textit{
	\begin{itemize}
	\end{itemize}
}
\end{enumerate}

\section*{series}
{\large \color{blue}  series  }
\subsection*{Explain}
\begin{enumerate}
\item countable noun \\
A \textbf{series}  \textbf{of} things or events is a number of them that come one after the other.
 \textit{
	\begin{itemize}
	\item ...a series of meetings with students and political leaders.
	\item ...a series of explosions.
	\end{itemize}
}
\item countable noun \\
A radio or television \textbf{series} is a set of programmes of a particular kind which have the same title .
 \textit{
	\begin{itemize}
	\item Lisa Kudrow became famous for her role as Phoebe in the world's most popular TV series,
Friends.
	\item ...the world's longest-running radio series, Britain's 'The Archers'.
	\end{itemize}
}
\end{enumerate}

\section*{tick}
{\large \color{blue}  ticks  ticking  ticked  }
\subsection*{Explain}
\begin{enumerate}
\item countable noun \\
A \textbf{tick} is a written mark like a V : ✓. It is used to show that something is correct or has been selected or dealt with.
 \textit{
	\begin{itemize}
	\item His exercise books were full of well deserved red ticks.
	\item Place a tick in the appropriate box.
	\end{itemize}
}
\item verb \\
If you \textbf{tick} something that is written on a piece of paper , you put a tick next to it.
 \textit{
	\begin{itemize}
	\item Please tick this box if you do not wish to receive such mailings.
	\item As each boy said yes, he ticked their name.
	\end{itemize}
}
\item verb \\
When a clock or watch \textbf{ticks} , it makes a regular series of short sounds as it works.
 \textbf{Tick away} means the same as tick .
 \textit{
	\begin{itemize}
	\item A wind-up clock ticked busily from the kitchen counter.
	\item A grandfather clock ticked away in a corner.
	\end{itemize}
}
\item countable noun \\
The \textbf{tick} of a clock or watch is the series of short sounds it makes when it is working , or one of those sounds.
 \textit{
	\begin{itemize}
	\item He sat listening to the tick of the grandfather clock.
	\end{itemize}
}
\item countable noun \\
You can use \textbf{tick} to refer to a very short period of time.
 \textit{
	\begin{itemize}
	\item Just hang on a tick, we may be able to help.
	\item I'll be back in a tick.
	\item I shall be with you in two ticks.
	\end{itemize}
}
\item verb \\
If you talk about what makes someone \textbf{tick} , you are talking about the beliefs , wishes , and feelings that make them behave in the way that they do.
 \textit{
	\begin{itemize}
	\item He wanted to find out what made them tick.
	\item I'm interested in how people tick.
	\end{itemize}
}
\item countable noun \\
A \textbf{tick} is a small creature which lives on the bodies of people or animals and uses their blood as food.
 \textit{
	\begin{itemize}
	\item ...chemicals that destroy ticks and mites.
	\item Tick bites can cause Lyme disease.
	\end{itemize}
}
\end{enumerate}

\section*{set}
{\large \color{blue}  sets  }
\subsection*{Explain}
\begin{enumerate}
\item countable noun \\
A \textbf{set}  \textbf{of} things is a number of things that belong together or that are thought of as a group.
 \textit{
	\begin{itemize}
	\item There must be one set of laws for the whole of the country.
	\item I might need a spare set of clothes.
	\item The computer repeats a set of calculations.
	\item Only she and Mr Cohen had complete sets of keys to the shop.
	\item The mattress and base are normally bought as a set.
	\item ...a chess set.
	\end{itemize}
}
\item countable noun \\
In tennis, a \textbf{set} is one of the groups of six or more games that form part of a match.
 \textit{
	\begin{itemize}
	\item Graf was leading 5-1 in the first set.
	\end{itemize}
}
\item countable noun \\
In mathematics , a \textbf{set} is a group of mathematical quantities that have some characteristic in common.
 \textit{
	\begin{itemize}
	\end{itemize}
}
\item countable noun \\
A band's or musician's \textbf{set} is the group of songs or tunes that they perform at a concert.
 \textit{
	\begin{itemize}
	\item The band continued with their set after a short break.
	\item He plays a solo acoustic set.
	\end{itemize}
}
\item singular noun \\
You can refer to a group of people as a \textbf{set} if they meet together socially or have the same interests and lifestyle .
 \textit{
	\begin{itemize}
	\item He belonged to what the press called 'The Chelsea Set'.
	\end{itemize}
}
\item countable noun \\
The \textbf{set} for a play, film, or television show is the furniture and scenery that is on the
stage when the play is being performed or in the studio where filming takes place.
 \textit{
	\begin{itemize}
	\item From the first moment he got on the set, he wanted to be a director too.
	\item ...his stage sets for the Folies Bergeres.
	\item ...a movie set.
	\item ...stars who behave badly on set.
	\end{itemize}
}
\item singular noun \\
\textbf{The}  \textbf{set}  \textbf{of} someone's face or part of their body is the way that it is fixed in a particular
expression or position, especially one that shows determination .
 \textit{
	\begin{itemize}
	\item Isabelle opened her mouth but stopped when she saw the set of his shoulders and the
look in his eyes.
	\item The set of her face can seem severe, even dour.
	\end{itemize}
}
\item countable noun \\
A \textbf{set} is an appliance . For example, a television set is a television.
 \textit{
	\begin{itemize}
	\item Children spend so much time in front of the television set.
	\item We got our first set–black and white–in 1963.
	\end{itemize}
}
\end{enumerate}

\section*{transmission}
{\large \color{blue}  transmissions  }
\subsection*{Explain}
\begin{enumerate}
\item uncountable noun \\
The \textbf{transmission} of something is the passing or sending of it to a different person or place.
 \textit{
	\begin{itemize}
	\item ...the possible risk for blood-borne disease transmission.
	\item The company is responsible for satellite data transmission .
	\item ...the transmission of knowledge and skills.
	\end{itemize}
}
\item uncountable noun \\
The \textbf{transmission} of television or radio programmes is the broadcasting of them.
 \textit{
	\begin{itemize}
	\end{itemize}
}
\item countable noun \\
A \textbf{transmission} is a broadcast.
 \textit{
	\begin{itemize}
	\end{itemize}
}
\item variable noun \\
The \textbf{transmission} on a car or other vehicle is the system of gears and shafts by which the power from the engine
 reaches and turns the wheels.
 \textit{
	\begin{itemize}
	\item The car was fitted with automatic transmission.
	\item ...a four-speed manual transmission.
	\end{itemize}
}
\end{enumerate}

\section*{stitch}
{\large \color{blue}  stitches  stitching  stitched  }
\subsection*{Explain}
\begin{enumerate}
\item verb \\
If you \textbf{stitch}  cloth , you use a needle and thread to join two pieces together or to make a decoration .
 \textit{
	\begin{itemize}
	\item Fold the fabric and stitch the two layers together.
	\item We stitched incessantly.
	\item ...those patient ladies who stitched the magnificent medieval tapestries.
	\end{itemize}
}
\item countable noun \\
\textbf{Stitches} are the short pieces of thread that have been sewn in a piece of cloth.
 \textit{
	\begin{itemize}
	\item ...a row of straight stitches.
	\item You can use embroidery stitches for further decoration.
	\end{itemize}
}
\item countable noun \\
In knitting and crochet, a \textbf{stitch} is a loop made by one turn of wool around a knitting needle or crochet hook .
 \textit{
	\begin{itemize}
	\item Her mother counted the stitches on her knitting needles.
	\item She kept dropping stitches.
	\end{itemize}
}
\item uncountable noun \\
If you sew or knit something in a particular \textbf{stitch} , you sew or knit in a way that produces a particular pattern .
 \textit{
	\begin{itemize}
	\item The design can be worked in cross stitch.
	\item ...a woolly vest knitted in garter stitch.
	\end{itemize}
}
\item verb \\
When doctors  \textbf{stitch} a wound , they use a special needle and thread to sew the skin together.
 \textbf{Stitch up} means the same as stitch .
 \textit{
	\begin{itemize}
	\item Jill washed and stitched the wound.
	\item Dr Armonson stitched up her wrist wounds.
	\item They've taken him off to hospital to stitch him up.
	\end{itemize}
}
\item countable noun \\
A \textbf{stitch} is a piece of thread that has been used to sew the skin of a wound together.
 \textit{
	\begin{itemize}
	\item He had six stitches in a head wound.
	\end{itemize}
}
\item singular noun \\
A \textbf{stitch} is a sharp pain in your side, usually caused by running or laughing a lot .
 \textit{
	\begin{itemize}
	\end{itemize}
}
\item  \\
 in stitches \textit{
	\begin{itemize}
	\end{itemize}
}
\end{enumerate}

\section*{tube}
{\large \color{blue}  tubes  }
\subsection*{Explain}
\begin{enumerate}
\item countable noun \\
A \textbf{tube} is a long hollow object that is usually round , like a pipe .
 \textit{
	\begin{itemize}
	\item He is fed by a tube that enters his nose.
	\item ...a cardboard tube.
	\end{itemize}
}
\item countable noun \\
A \textbf{tube}  \textbf{of} something such as paste is a long, thin container which you squeeze in order to force the paste out.
 \textit{
	\begin{itemize}
	\item ...a tube of toothpaste.
	\item ...a small tube of moisturizer.
	\end{itemize}
}
\item countable noun \\
Some long, thin, hollow parts in your body are referred to as \textbf{tubes} .
 \textit{
	\begin{itemize}
	\item The lungs are in fact constructed of thousands of tiny tubes.
	\end{itemize}
}
\item singular noun \\
\textbf{The tube} is the underground  railway system in London .
 \textit{
	\begin{itemize}
	\item I took the tube then the train and came straight here.
	\item He travelled by tube.
	\end{itemize}
}
\item countable noun \\
You can refer to the television as \textbf{the tube} .
 \textit{
	\begin{itemize}
	\item The only baseball he saw was on the tube.
	\end{itemize}
}
\item  \\
 go down the tube(s) \textit{
	\begin{itemize}
	\end{itemize}
}
\end{enumerate}

\section*{teenager}
{\large \color{blue}  teenagers  }
\subsection*{Explain}
\begin{enumerate}
\item countable noun \\
A \textbf{teenager} is someone who is between thirteen and nineteen  years  old .
 \textit{
	\begin{itemize}
	\item As a teenager he attended Tulse Hill Senior High School.
	\end{itemize}
}
\end{enumerate}

\section*{tunnel}
{\large \color{blue}  tunnels  tunnelling  tunnelled  }
\subsection*{Explain}
\begin{enumerate}
\item countable noun \\
A \textbf{tunnel} is a long passage which has been made under the ground , usually through a hill or under the sea.
 \textit{
	\begin{itemize}
	\item ...two new railway tunnels through the Alps.
	\item ...the motorway tunnels under the Hudson river.
	\end{itemize}
}
\item verb \\
To \textbf{tunnel}  somewhere  means to make a tunnel there.
 \textit{
	\begin{itemize}
	\item The rebels tunnelled out of a maximum security jail.
	\item The caterpillars tunnel into the fruit to grow and mature.
	\end{itemize}
}
\end{enumerate}

\section*{thrill}
{\large \color{blue}  thrills  thrilling  thrilled  }
\subsection*{Explain}
\begin{enumerate}
\item countable noun \\
If something gives you a \textbf{thrill} , it gives you a sudden feeling of great excitement, pleasure, or fear.
 \textit{
	\begin{itemize}
	\item I can remember the thrill of not knowing what I would get on Christmas morning.
	\item It's a great thrill for a cricket-lover like me to play at the home of cricket.
	\item ...the realization that new adventures, new thrills, and new worlds lie ahead.
	\end{itemize}
}
\item verb \\
If something \textbf{thrills} you, or if you \textbf{thrill}  \textbf{at} it, it gives you a feeling of great pleasure and excitement.
 \textit{
	\begin{itemize}
	\item The electric atmosphere both terrified and thrilled him.
	\item The children will thrill at all their favourite characters.
	\end{itemize}
}
\item  \\
 thrills and spills \textit{
	\begin{itemize}
	\end{itemize}
}
\end{enumerate}

\section*{vegetable}
{\large \color{blue}  vegetables  }
\subsection*{Explain}
\begin{enumerate}
\item countable noun \\
\textbf{Vegetables} are plants such as cabbages, potatoes, and onions which you can cook and eat .
 \textit{
	\begin{itemize}
	\item A good general diet should include plenty of fresh vegetables.
	\item ...traditional Caribbean fruit and vegetables.
	\item ...vegetable soup.
	\end{itemize}
}
\item adjective \\
\textbf{Vegetable}  matter  comes from plants.
 \textit{
	\begin{itemize}
	\item ...compounds, of animal, vegetable or mineral origin.
	\item ...decayed vegetable matter.
	\end{itemize}
}
\item countable noun \\
If someone refers to a brain-damaged person as a \textbf{vegetable} , they mean that the person cannot move, think , or speak .
 \textit{
	\begin{itemize}
	\end{itemize}
}
\end{enumerate}

\section*{ray}
{\large \color{blue}  rays  }
\subsection*{Explain}
\begin{enumerate}
\item countable noun \\
\textbf{Rays} of light are narrow beams of light.
 \textit{
	\begin{itemize}
	\item ...the first rays of light spread over the horizon.
	\item It can be seen clearly in a ray of sunlight or under a lamp.
	\item The sun's rays can penetrate water up to 10 feet.
	\end{itemize}
}
\item countable noun \\
A \textbf{ray of} hope, comfort , or other positive quality is a small amount of it that you welcome because it makes a bad situation seem less bad.
 \textit{
	\begin{itemize}
	\item They could provide a ray of hope amid the general business and economic gloom.
	\item The one ray of sunlight in this depressing history is her meeting and falling in
love with Martin.
	\end{itemize}
}
\item countable noun \\
A \textbf{ray} is a fairly large sea fish which has a flat body, eyes on the top of its body, and a long tail.
 \textit{
	\begin{itemize}
	\end{itemize}
}
\end{enumerate}

\section*{widow}
{\large \color{blue}  widows  }
\subsection*{Explain}
\begin{enumerate}
\item countable noun \\
A \textbf{widow} is a woman whose spouse has died and who has not married again.
 \textit{
	\begin{itemize}
	\end{itemize}
}
\end{enumerate}

\section*{blade}
{\large \color{blue}  blades  }
\subsection*{Explain}
\begin{enumerate}
\item countable noun \\
The \textbf{blade} of a knife , axe , or saw is the edge, which is used for cutting.
 \textit{
	\begin{itemize}
	\item Many of these tools have sharp blades, so be careful.
	\end{itemize}
}
\item countable noun \\
The \textbf{blades} of a propeller are the long, flat parts that turn  round .
 \textit{
	\begin{itemize}
	\end{itemize}
}
\item countable noun \\
The \textbf{blade} of an oar is the thin flat part that you put into the water.
 \textit{
	\begin{itemize}
	\end{itemize}
}
\item countable noun \\
A \textbf{blade} of grass is a single  piece of grass.
 \textit{
	\begin{itemize}
	\end{itemize}
}
\end{enumerate}

\section*{arrow}
{\large \color{blue}  arrows  }
\subsection*{Explain}
\begin{enumerate}
\item countable noun \\
An \textbf{arrow} is a long thin weapon which is sharp and pointed at one end and which often has feathers at the other end. An arrow is
shot from a bow.
 \textit{
	\begin{itemize}
	\item Warriors armed with bows and arrows and spears have invaded their villages.
	\end{itemize}
}
\item countable noun \\
An \textbf{arrow} is a written or printed sign that consists of a straight  line with another line bent at a sharp angle at one end. This is a printed arrow: →. The arrow points in a particular direction to indicate where something is.
 \textit{
	\begin{itemize}
	\item A series of arrows points the way to his grave.
	\end{itemize}
}
\end{enumerate}

\section*{census}
{\large \color{blue}  censuses  }
\subsection*{Explain}
\begin{enumerate}
\item countable noun \\
A \textbf{census} is an official survey of the population of a country that is carried out in order to find out how many people live there and to obtain  details of such things as people's ages and jobs .
 \textit{
	\begin{itemize}
	\end{itemize}
}
\end{enumerate}

\section*{butterfly}
{\large \color{blue}  butterflies  }
\subsection*{Explain}
\begin{enumerate}
\item countable noun \\
A \textbf{butterfly} is an insect with large colourful wings and a thin body.
 \textit{
	\begin{itemize}
	\end{itemize}
}
\item uncountable noun \\
\textbf{Butterfly} is a swimming stroke which you do on your front , kicking your legs and bringing your arms over your head together.
 \textit{
	\begin{itemize}
	\end{itemize}
}
\item  \\
 butterflies in your stomach \textit{
	\begin{itemize}
	\end{itemize}
}
\end{enumerate}

\section*{cheese}
{\large \color{blue}  cheeses  }
\subsection*{Explain}
\begin{enumerate}
\item variable noun \\
\textbf{Cheese} is a solid food made from milk. It is usually white or yellow .
 \textit{
	\begin{itemize}
	\item ...bread and cheese.
	\item ...cheese sauce.
	\item He cut the mould off a piece of cheese.
	\item ...delicious French cheeses.
	\end{itemize}
}
\item  \\
 big cheese \textit{
	\begin{itemize}
	\end{itemize}
}
\item  \\
 say cheese \textit{
	\begin{itemize}
	\end{itemize}
}
\end{enumerate}

\section*{case}
{\large \color{blue}  cases  }
\subsection*{Explain}
\begin{enumerate}
\item countable noun \\
A particular \textbf{case} is a particular situation or incident , especially one that you are using as an individual example or instance of something.
 \textit{
	\begin{itemize}
	\item Surgical training takes at least nine years, or 11 in the case of obstetrics.
	\item One of the effects of dyslexia, in my case at least, is that you pay tremendous attention
to detail.
	\item In extreme cases, insurance companies can prosecute for fraud.
	\item The Honduran press published reports of eighteen cases of alleged baby snatching.
	\end{itemize}
}
\item countable noun \\
A \textbf{case} is a person or their particular problem that a doctor, social worker, or other professional is dealing with.
 \textit{
	\begin{itemize}
	\item ...the case of a 57-year-old man who had suffered a stroke.
	\item Some cases of arthritis respond to a gluten-free diet.
	\item Child protection workers were meeting to discuss her case.
	\end{itemize}
}
\item countable noun \\
If you say that someone is a sad  \textbf{case} or a hopeless  \textbf{case} , you mean that they are in a sad situation or a hopeless situation.
 \textit{
	\begin{itemize}
	\item I knew I was going to make it–that I wasn't a hopeless case.
	\end{itemize}
}
\item countable noun \\
A \textbf{case} is a crime or mystery that the police are investigating .
 \textit{
	\begin{itemize}
	\item The couple were made official suspects in the case.
	\item Mr. Hitchens said you have solved some very unusual cases.
	\end{itemize}
}
\item countable noun \\
The \textbf{case}  \textbf{for} or \textbf{against} a plan or idea consists of the facts and reasons used to support it or oppose it.
 \textit{
	\begin{itemize}
	\item He sat there while I made the case for his dismissal.
	\item Both these facts strengthen the case against hanging.
	\item She argued her case.
	\end{itemize}
}
\item countable noun \\
In law, a \textbf{case} is a trial or other legal inquiry .
 \textit{
	\begin{itemize}
	\item It can be difficult for public figures to win a libel case.
	\item The case was brought by his family, who say their reputation has been damaged.
	\end{itemize}
}
\item  \\
 in any case \textit{
	\begin{itemize}
	\end{itemize}
}
\item  \\
 in any case \textit{
	\begin{itemize}
	\end{itemize}
}
\item  \\
 in case/just in case \textit{
	\begin{itemize}
	\end{itemize}
}
\item  \\
 in case of sth \textit{
	\begin{itemize}
	\end{itemize}
}
\item  \\
 in case \textit{
	\begin{itemize}
	\end{itemize}
}
\item  \\
 in that/which case \textit{
	\begin{itemize}
	\end{itemize}
}
\item  \\
 just in case \textit{
	\begin{itemize}
	\end{itemize}
}
\item  \\
 as/whatever the case may be \textit{
	\begin{itemize}
	\end{itemize}
}
\item  \\
 a case of \textit{
	\begin{itemize}
	\end{itemize}
}
\item  \\
 a case in point \textit{
	\begin{itemize}
	\end{itemize}
}
\item  \\
 to be the case \textit{
	\begin{itemize}
	\end{itemize}
}
\item  \\
 on the case \textit{
	\begin{itemize}
	\end{itemize}
}
\end{enumerate}

\section*{cherry}
{\large \color{blue}  cherries  }
\subsection*{Explain}
\begin{enumerate}
\item countable noun \\
\textbf{Cherries} are small, round fruit with red skins .
 \textit{
	\begin{itemize}
	\end{itemize}
}
\item variable noun \\
A \textbf{cherry} or a \textbf{cherry tree} is a tree that cherries grow on.
 \textit{
	\begin{itemize}
	\end{itemize}
}
\end{enumerate}

\section*{crowd}
{\large \color{blue}  crowds  crowding  crowded  }
\subsection*{Explain}
\begin{enumerate}
\item countable noun \\
A \textbf{crowd} is a large group of people who have gathered together, for example to watch or listen to something interesting , or to protest about something.
 \textit{
	\begin{itemize}
	\item A huge crowd gathered in a square outside the Kremlin walls.
	\item It took some two hours before the crowd was fully dispersed.
	\item The crowd were enormously enthusiastic.
	\item The explosions took place in shopping centres as crowds of people were shopping for
Mother's Day.
	\end{itemize}
}
\item countable noun \\
A particular \textbf{crowd} is a group of friends , or a set of people who share the same interests or job .
 \textit{
	\begin{itemize}
	\item All the old crowd have come out for this occasion.
	\end{itemize}
}
\item verb \\
When people \textbf{crowd}  \textbf{around} someone or something, they gather closely together around them.
 \textit{
	\begin{itemize}
	\item The hungry refugees crowded around the tractors.
	\item Police blocked off the road as hotel staff and guests crowded around.
	\end{itemize}
}
\item verb \\
If people \textbf{crowd}  \textbf{into} a place or \textbf{are crowded}  \textbf{into} a place, large numbers of them enter it so that it becomes very full .
 \textit{
	\begin{itemize}
	\item Hundreds of thousands of people have crowded into the centre of the Lithuanian capital,
Vilnius.
	\item One group of journalists were crowded into a minibus.
	\item 'Bravo, bravo,' chanted party workers crowded in the main hall.
	\end{itemize}
}
\item verb \\
If a group of people \textbf{crowd} a place, there are so many of them there that it is full.
 \textit{
	\begin{itemize}
	\item Thousands of demonstrators crowded the streets shouting slogans.
	\end{itemize}
}
\item verb \\
If people \textbf{crowd} you, they stand very closely around you trying to see or speak to you, so that you feel  uncomfortable .
 \textit{
	\begin{itemize}
	\item It had been a tense day with people crowding her all the time.
	\end{itemize}
}
\end{enumerate}

\section*{damp}
{\large \color{blue}  damper  dampest  damps  damping  damped  }
\subsection*{Explain}
\begin{enumerate}
\item adjective \\
Something that is \textbf{damp} is slightly wet.
 \textit{
	\begin{itemize}
	\item Her hair was still damp.
	\item ...the damp, cold air.
	\item She wiped the table with a damp cloth.
	\end{itemize}
}
\item uncountable noun \\
\textbf{Damp} is moisture that is found on the inside  walls of a house or in the air.
 \textit{
	\begin{itemize}
	\item There was damp everywhere and the entire building was in need of rewiring.
	\end{itemize}
}
\item verb \\
If you \textbf{damp} something, you make it slightly wet.
 \textit{
	\begin{itemize}
	\item Hillsden damped a hand towel and laid it across her forehead.
	\end{itemize}
}
\end{enumerate}

\section*{directory}
{\large \color{blue}  directories  }
\subsection*{Explain}
\begin{enumerate}
\item countable noun \\
A \textbf{directory} is a book which gives lists of facts , for example people's names, addresses, and telephone numbers, or the names and addresses of business  companies , usually arranged in alphabetical order.
 \textit{
	\begin{itemize}
	\item ...a telephone directory.
	\end{itemize}
}
\item countable noun \\
A \textbf{directory} is an area of a computer disk which contains one or more files or other directories.
 \textit{
	\begin{itemize}
	\item This option lets you create new files or directories.
	\end{itemize}
}
\item countable noun \\
On the internet , a \textbf{directory} is a list of the subjects that you can  find  information on.
 \textit{
	\begin{itemize}
	\item Yahoo is the oldest internet directory service.
	\end{itemize}
}
\end{enumerate}

\section*{dawn}
{\large \color{blue}  dawns  dawning  dawned  }
\subsection*{Explain}
\begin{enumerate}
\item variable noun \\
\textbf{Dawn} is the time of day when light first appears in the sky, just before the sun  rises .
 \textit{
	\begin{itemize}
	\item Nancy woke at dawn.
	\end{itemize}
}
\item singular noun \\
\textbf{The}  \textbf{dawn}  \textbf{of} a period of time or a situation is the beginning of it.
 \textit{
	\begin{itemize}
	\item ...the dawn of powered flight.
	\item ...the dawn of the radio age.
	\end{itemize}
}
\item verb \\
If something \textbf{is dawning} , it is beginning to develop or come into existence .
 \textit{
	\begin{itemize}
	\item The age of the computerized toilet has dawned.
	\item A new era seemed to be about to dawn for the coach and his young team.
	\item Now there is a dawning realisation that drastic action is necessary.
	\end{itemize}
}
\item verb \\
When you say that a particular day \textbf{dawned} , you mean it arrived or began, usually when it became light.
 \textit{
	\begin{itemize}
	\item When the great day dawned, the first concern was the weather.
	\item The next day dawned sombre and gloomy.
	\end{itemize}
}
\end{enumerate}

\section*{engineer}
{\large \color{blue}  engineers  engineering  engineered  }
\subsection*{Explain}
\begin{enumerate}
\item countable noun \\
An \textbf{engineer} is a person who uses scientific  knowledge to design, construct, and maintain engines and machines or structures such as roads , railways, and bridges .
 \textit{
	\begin{itemize}
	\end{itemize}
}
\item countable noun \\
An \textbf{engineer} is a person who repairs mechanical or electrical devices.
 \textit{
	\begin{itemize}
	\item They send a service engineer to fix the disk drive.
	\end{itemize}
}
\item countable noun \\
An \textbf{engineer} is a person who is responsible for maintaining the engine of a ship while it is at sea.
 \textit{
	\begin{itemize}
	\end{itemize}
}
\item verb \\
When a vehicle, bridge , or building \textbf{is engineered} , it is planned and constructed using scientific methods .
 \textit{
	\begin{itemize}
	\item Many of Kuwait's spacious freeways were engineered by W S Atkins.
	\item ...the car's better designed and better engineered rivals.
	\end{itemize}
}
\item verb \\
If you \textbf{engineer} an event or situation, you arrange for it to happen , in a clever or indirect way.
 \textit{
	\begin{itemize}
	\item He could stand no more and engineered an escape.
	\item LeBlanc's rise was not entirely a consequence of talent but was engineered by her
maternal grandfather.
	\end{itemize}
}
\end{enumerate}

\section*{exposure}
{\large \color{blue}  exposures  }
\subsection*{Explain}
\begin{enumerate}
\item uncountable noun \\
\textbf{Exposure}  \textbf{to} something dangerous means being in a situation where it might  affect you.
 \textit{
	\begin{itemize}
	\item Exposure to lead is known to damage the brains of young children.
	\item ...the potential exposure of people to nuclear waste.
	\end{itemize}
}
\item uncountable noun \\
\textbf{Exposure} is the harmful effect on your body caused by very cold weather.
 \textit{
	\begin{itemize}
	\item He was suffering from exposure and shock but his condition was said to be stable.
	\item At least two people died of exposure in Chicago overnight.
	\end{itemize}
}
\item uncountable noun \\
The \textbf{exposure} of a well-known person is the revealing of the fact that they are bad or immoral in some way.
 \textit{
	\begin{itemize}
	\item ...the exposure of stars in so-called morally compromising situations.
	\item Their sporting reputation has suffered enormously from Johnson's exposure.
	\end{itemize}
}
\item uncountable noun \\
\textbf{Exposure} is publicity that a person, company, or product receives .
 \textit{
	\begin{itemize}
	\item The candidates are getting an enormous amount of exposure in the media.
	\end{itemize}
}
\item countable noun \\
In photography , an \textbf{exposure} is a single photograph.
 \textit{
	\begin{itemize}
	\item Larger drawings tend to require two or three exposures to cover them.
	\end{itemize}
}
\item variable noun \\
In photography, the \textbf{exposure} is the amount of light that is allowed to enter a camera when taking a photograph.
 \textit{
	\begin{itemize}
	\item A tripod also lets you shoot long exposures at night.
	\item ...an exposure of 1/18sec at f/11.
	\item Against a deep blue sky or dark storm-clouds, you may need to reduce the exposure.
	\end{itemize}
}
\end{enumerate}

\section*{engineering}
{\large \color{blue}  }
\subsection*{Explain}
\begin{enumerate}
\item uncountable noun \\
\textbf{Engineering} is the work involved in designing and constructing engines and machinery, or structures such as roads and bridges. \textbf{Engineering} is also the subject studied by people who want to do this work.
 \textit{
	\begin{itemize}
	\item ...the design and engineering of aircraft and space vehicles.
	\item ...graduates with degrees in engineering.
	\end{itemize}
}
\end{enumerate}

\section*{float}
{\large \color{blue}  floats  floating  floated  }
\subsection*{Explain}
\begin{enumerate}
\item verb \\
If something or someone \textbf{is floating} in a liquid, they are in the liquid, on or just below the surface, and are being
supported by it. You can also \textbf{float} something on a liquid.
 \textit{
	\begin{itemize}
	\item They noticed fifty and twenty dollar bills floating in the water.
	\item ...barges floating quietly by the grassy river banks.
	\item They'll spend some time floating boats in the creek.
	\end{itemize}
}
\item verb \\
Something that \textbf{floats} lies on or just below the surface of a liquid when it is put in it and does not sink.
 \textit{
	\begin{itemize}
	\item Empty things float.
	\end{itemize}
}
\item countable noun \\
A \textbf{float} is a light object that is used to help someone or something float.
 \textit{
	\begin{itemize}
	\end{itemize}
}
\item countable noun \\
A \textbf{float} is a small object attached to a fishing line which floats on the water and moves
when a fish has been caught.
 \textit{
	\begin{itemize}
	\end{itemize}
}
\item verb \\
Something that \textbf{floats} in or through the air hangs in it or moves slowly and gently through it.
 \textit{
	\begin{itemize}
	\item The white cloud of smoke floated away.
	\item ...the sun's rays lighting up the dust floating in the air.
	\end{itemize}
}
\item verb \\
If a sound or smell  \textbf{floats} to a place quite  far away, it can be heard or smelled there.
 \textit{
	\begin{itemize}
	\item Sublime music floats on a scented summer breeze to the spot where you lie on the
lush grass.
	\item The smells of delicious foods floated all around him.
	\end{itemize}
}
\item verb \\
If you \textbf{float}  somewhere , you walk there very lightly and gracefully.
 \textit{
	\begin{itemize}
	\item Caroline floated up the aisle on her father's arm.
	\end{itemize}
}
\item verb \\
If you \textbf{float} a project, plan, or idea, you suggest it for others to think about.
 \textit{
	\begin{itemize}
	\item The French had floated the idea of placing the diplomatic work in the hands of the
U.N.
	\end{itemize}
}
\item verb \\
If a company director  \textbf{floats} their company, they start to sell shares in it to the public.
 \textit{
	\begin{itemize}
	\item He floated his firm on the stock market.
	\item The advisers decided to float 60 per cent of the shares.
	\end{itemize}
}
\item verb \\
If a government \textbf{floats} its country's currency or allows it to \textbf{float} , it allows the currency's value to change freely in relation to other currencies.
 \textit{
	\begin{itemize}
	\item ...the decision to float their currency.
	\item 59 per cent of people believed the pound should be allowed to float freely.
	\end{itemize}
}
\item countable noun \\
A \textbf{float} is a truck on which displays and people in special costumes are carried in a festival  procession .
 \textit{
	\begin{itemize}
	\end{itemize}
}
\item singular noun \\
A \textbf{float} is a small amount of coins and notes of low value that someone has before they start
selling things so that they are able to give customers change if necessary .
 \textit{
	\begin{itemize}
	\end{itemize}
}
\end{enumerate}

\section*{entrance}
{\large \color{blue}  entrances  }
\subsection*{Explain}
\begin{enumerate}
\item countable noun \\
The \textbf{entrance}  \textbf{to} a place is the way into it, for example a door or gate.
 \textit{
	\begin{itemize}
	\item Beside the entrance to the church, turn right.
	\item He was driven out of a side entrance with his hand covering his face.
	\item A marble entrance hall leads to a sitting room.
	\end{itemize}
}
\item countable noun \\
You can  refer to someone's arrival in a place as their \textbf{entrance} , especially when you think that they are trying to be noticed and admired .
 \textit{
	\begin{itemize}
	\item If she had noticed her father's entrance, she gave no indication.
	\end{itemize}
}
\item countable noun \\
When a performer makes his or her \textbf{entrance}  \textbf{on to} the stage, he or she comes on to the stage.
 \textit{
	\begin{itemize}
	\item He made his entrance into the parade ring.
	\end{itemize}
}
\item uncountable noun \\
If you gain  \textbf{entrance}  \textbf{to} a particular place, you manage to get in there.
 \textit{
	\begin{itemize}
	\item Hewitt had gained entrance to the Hall by pretending to be a heating engineer.
	\end{itemize}
}
\item uncountable noun \\
If you gain \textbf{entrance}  \textbf{to} a particular profession , society , or institution , you are accepted as a member of it.
 \textit{
	\begin{itemize}
	\item Entrance to universities and senior secondary schools was restricted.
	\item ...entrance exams for the French civil service.
	\end{itemize}
}
\item singular noun \\
If you make an \textbf{entrance}  \textbf{into} a particular activity or system, you succeed in becoming  involved in it.
 \textit{
	\begin{itemize}
	\item Charlie found his entrance into higher education completely miserable.
	\item ...his entrance into politics in 1993.
	\end{itemize}
}
\end{enumerate}

\section*{hatred}
{\large \color{blue}  }
\subsection*{Explain}
\begin{enumerate}
\item uncountable noun \\
\textbf{Hatred} is an extremely  strong feeling of dislike for someone or something.
 \textit{
	\begin{itemize}
	\item Her hatred of them would never lead her to murder.
	\item My hatred for her is so intense it seems to be destroying me.
	\item He has been accused of inciting racial hatred.
	\end{itemize}
}
\end{enumerate}

\section*{factory}
{\large \color{blue}  factories  }
\subsection*{Explain}
\begin{enumerate}
\item countable noun \\
A \textbf{factory} is a large building where machines are used to make large quantities of goods.
 \textit{
	\begin{itemize}
	\item He owned furniture factories in New York State.
	\end{itemize}
}
\end{enumerate}

\section*{incidence}
{\large \color{blue}  incidences  }
\subsection*{Explain}
\begin{enumerate}
\item variable noun \\
\textbf{The}  \textbf{incidence}  \textbf{of} something bad , such as a disease, is the frequency with which it occurs, or the occasions when it occurs.
 \textit{
	\begin{itemize}
	\item The incidence of breast cancer increases with age.
	\item ...the high incidence of child mortality.
	\item ...a couple of isolated incidences.
	\end{itemize}
}
\end{enumerate}

\section*{faculty}
{\large \color{blue}  faculties  }
\subsection*{Explain}
\begin{enumerate}
\item countable noun \\
Your \textbf{faculties} are your physical and mental abilities.
 \textit{
	\begin{itemize}
	\item He was drunk and not in control of his faculties.
	\item It is also a myth that the faculty of hearing is greatly increased in blind people.
	\end{itemize}
}
\item variable noun \\
A \textbf{faculty} is a group of related departments in some universities, or the people who work in
them.
 \textit{
	\begin{itemize}
	\item ...the Faculty of Social and Political Sciences.
	\end{itemize}
}
\item variable noun \\
A \textbf{faculty} is all the teaching staff of a university or college, or of one department.
 \textit{
	\begin{itemize}
	\item The faculty agreed on a change in the requirements.
	\item How can faculty improve their teaching so as to encourage creativity?
	\item ...eminent Stanford faculty members.
	\end{itemize}
}
\end{enumerate}

\section*{ink}
{\large \color{blue}  inks  inking  inked  }
\subsection*{Explain}
\begin{enumerate}
\item variable noun \\
\textbf{Ink} is the coloured  liquid used for writing or printing.
 \textit{
	\begin{itemize}
	\item The letter was handwritten in black ink.
	\end{itemize}
}
\item verb \\
If you \textbf{ink} something, you put ink on it.
 \textit{
	\begin{itemize}
	\item Ritter took his left hand and inked the fingertips.
	\end{itemize}
}
\end{enumerate}

\section*{folk}
{\large \color{blue}  folks  }
\subsection*{Explain}
\begin{enumerate}
\item plural noun \\
You can refer to people as \textbf{folk} or \textbf{folks} .
 \textit{
	\begin{itemize}
	\item Country folk can tell you that there are certain places which animals avoid.
	\item These are the folks from the local TV station.
	\item ...old folks.
	\end{itemize}
}
\item plural noun \\
You can refer to your close family, especially your mother and father , as your \textbf{folks} .
 \textit{
	\begin{itemize}
	\item I've been avoiding my folks lately.
	\end{itemize}
}
\item plural noun \\
You can use \textbf{folks} as a term of address when you are talking to several people.
 \textit{
	\begin{itemize}
	\item 'It's a question of money, folks,' I announced.
	\item This is it, folks: the best record guide in the business.
	\end{itemize}
}
\item adjective \\
\textbf{Folk} art and customs are traditional or typical of a particular community or nation .
 \textit{
	\begin{itemize}
	\item ...South American folk art.
	\item ...traditional Chinese folk medicine.
	\end{itemize}
}
\item adjective \\
\textbf{Folk} music is music which is traditional or typical of a particular community or nation.
 \textbf{Folk} is also a noun .
 \textit{
	\begin{itemize}
	\item ...Irish folk music.
	\item ...a variety of music including classical, jazz, and folk.
	\end{itemize}
}
\item adjective \\
\textbf{Folk} can be used to describe something that relates to the beliefs and opinions of ordinary people.
 \textit{
	\begin{itemize}
	\item Jack was a folk hero in the Greenwich Village bars.
	\item Folk psychology comes closer to the obvious truth than the most sophisticated theories.
	\end{itemize}
}
\end{enumerate}

\section*{knee}
{\large \color{blue}  knees  kneeing  kneed  }
\subsection*{Explain}
\begin{enumerate}
\item countable noun \\
Your \textbf{knee} is the place where your leg bends.
 \textit{
	\begin{itemize}
	\item He will receive physiotherapy on his damaged left knee.
	\item ...a knee injury.
	\end{itemize}
}
\item countable noun \\
If something or someone is \textbf{on} your \textbf{knee} or \textbf{on} your \textbf{knees} , they are resting or sitting on the upper part of your legs when you are sitting down.
 \textit{
	\begin{itemize}
	\item He sat with the package on his knees.
	\item I sat in the back of the taxi with my son on my knee.
	\end{itemize}
}
\item countable noun \\
The \textbf{knee} on a piece of clothing is the part that covers your knee.
 \textit{
	\begin{itemize}
	\item ...jeans with holes at both knees.
	\end{itemize}
}
\item plural noun \\
If you are \textbf{on} your \textbf{knees} , your legs are bent and your knees are on the ground .
 \textit{
	\begin{itemize}
	\item She fell to the ground on her knees and prayed.
	\item She was on her knees in the kitchen.
	\end{itemize}
}
\item verb \\
If you \textbf{knee} someone, you hit them using your knee.
 \textit{
	\begin{itemize}
	\item Ian kneed him in the groin.
	\end{itemize}
}
\item  \\
 bring to its knees \textit{
	\begin{itemize}
	\end{itemize}
}
\end{enumerate}

\section*{force}
{\large \color{blue}  forces  forcing  forced  }
\subsection*{Explain}
\begin{enumerate}
\item verb \\
If someone \textbf{forces} you \textbf{to} do something, they make you do it even though you do not want to, for example by threatening you.
 \textit{
	\begin{itemize}
	\item The gunman forced a woman to drive him across the city, stopping to shoot at bystanders.
	\item The royal family were forced to flee with their infant son.
	\item I cannot force you in this. You must decide.
	\item They were grabbed by three men who appeared to force them into a car.
	\end{itemize}
}
\item verb \\
If a situation or event \textbf{forces} you to do something, it makes it necessary for you to do something that you would not otherwise have done.
 \textit{
	\begin{itemize}
	\item A back injury forced her to withdraw from the tournament.
	\item He turned right, down a dirt road that forced him into four-wheel drive.
	\item She finally was forced to the conclusion that she wouldn't get another paid job in
her field.
	\end{itemize}
}
\item verb \\
If someone \textbf{forces} something \textbf{on} or \textbf{upon} you, they make you accept or use it when you would prefer not to.
 \textit{
	\begin{itemize}
	\item To force this agreement on the nation is wrong.
	\end{itemize}
}
\item verb \\
If you \textbf{force} something into a particular position, you use a lot of strength to make it move there.
 \textit{
	\begin{itemize}
	\item They were forcing her head under the icy waters, drowning her.
	\end{itemize}
}
\item verb \\
If someone \textbf{forces} a lock, a door, or a window , they break the lock or fastening in order to get into a building without using a key .
 \textit{
	\begin{itemize}
	\item That evening police forced the door of the flat and arrested Mr Roberts.
	\item He tried to force the window open but it was jammed shut.
	\end{itemize}
}
\item uncountable noun \\
If someone uses \textbf{force} to do something, or if it is done by \textbf{force} , strong and violent physical action is taken in order to achieve it.
 \textit{
	\begin{itemize}
	\item The government decided against using force to break up the demonstrations.
	\item ...the guerrillas' efforts to seize power by force.
	\end{itemize}
}
\item uncountable noun \\
\textbf{Force} is the power or strength which something has.
 \textit{
	\begin{itemize}
	\item The force of the explosion shattered the windows of several buildings.
	\item It looked as though the storm had an awful lot of force.
	\end{itemize}
}
\item countable noun \\
If you refer to someone or something as a \textbf{force} in a particular type of activity, you mean that they have a strong influence on it.
 \textit{
	\begin{itemize}
	\item For years the army was the most powerful political force in the country.
	\item The band are still as innovative a force in British music as they were when they
started.
	\item One of the driving forces behind this recent expansion is the growth of services.
	\end{itemize}
}
\item uncountable noun \\
The \textbf{force}  \textbf{of} something is the powerful effect or quality that it has.
 \textit{
	\begin{itemize}
	\item He changed our world through the force of his ideas.
	\item Perhaps your force of argument might have made some difference.
	\end{itemize}
}
\item countable noun \\
You can use \textbf{forces} to refer to processes and events that do not appear to be caused by human beings,
and are therefore difficult to understand or control.
 \textit{
	\begin{itemize}
	\item ...the forces of nature: epidemics, predators, floods, hurricanes.
	\item The principle of market forces was applied to some of the country's most revered
institutions.
	\item Is it really the Holy Spirit moving me, or is it some evil force?
	\end{itemize}
}
\item variable noun \\
In physics , a \textbf{force} is the pulling or pushing effect that something has on something else.
 \textit{
	\begin{itemize}
	\item ...the earth's gravitational force.
	\item ...protons and electrons trapped by magnetic forces in the Van Allen belts.
	\end{itemize}
}
\item uncountable noun \\
\textbf{Force} is used before a number to indicate a wind of a particular speed or strength, especially a very strong wind.
 \textit{
	\begin{itemize}
	\item The airlift was conducted in force ten winds.
	\item Northerly winds will increase to force six by midday.
	\end{itemize}
}
\item verb \\
If you \textbf{force} a smile or a laugh , you manage to smile or laugh, but with an effort because you are unhappy .
 \textit{
	\begin{itemize}
	\item Joe forced a smile, but underneath he was a little disturbed.
	\item 'Why don't you offer me a drink?' he asked, with a forced smile.
	\end{itemize}
}
\item countable noun \\
\textbf{Forces} are groups of soldiers or military vehicles that are organized for a particular purpose.
 \textit{
	\begin{itemize}
	\item ...the deployment of American forces in the region.
	\end{itemize}
}
\item plural noun \\
\textbf{The}  \textbf{forces} means the army, the navy , or the air force, or all three.
 \textit{
	\begin{itemize}
	\item The more senior you become in the forces, the more likely you are to end up in a
desk job.
	\end{itemize}
}
\item singular noun \\
\textbf{The}  \textbf{force} is sometimes used to mean the police force.
 \textit{
	\begin{itemize}
	\item It was hard for a police officer to make friends outside the force.
	\end{itemize}
}
\item  \\
 by force of \textit{
	\begin{itemize}
	\end{itemize}
}
\item  \\
 force of habit \textit{
	\begin{itemize}
	\end{itemize}
}
\item  \\
 in force \textit{
	\begin{itemize}
	\end{itemize}
}
\item  \\
 in force \textit{
	\begin{itemize}
	\end{itemize}
}
\item  \\
 to join forces \textit{
	\begin{itemize}
	\end{itemize}
}
\item  \\
 force one's way somewhere \textit{
	\begin{itemize}
	\end{itemize}
}
\end{enumerate}

\section*{laptop}
{\large \color{blue}  laptops  }
\subsection*{Explain}
\begin{enumerate}
\item countable noun \\
A \textbf{laptop} or a \textbf{laptop computer} is a small portable computer.
 \textit{
	\begin{itemize}
	\item She used to work at her laptop until four in the morning.
	\item ...a laptop computer he called Magic Slate.
	\end{itemize}
}
\end{enumerate}

\section*{geometry}
{\large \color{blue}  }
\subsection*{Explain}
\begin{enumerate}
\item uncountable noun \\
\textbf{Geometry} is the branch of mathematics concerned with the properties and relationships of lines,
 angles , curves, and shapes.
 \textit{
	\begin{itemize}
	\item ...the very ordered way in which mathematics and geometry describe nature.
	\end{itemize}
}
\item uncountable noun \\
The \textbf{geometry} of an object is its shape or the relationship of its parts to each other.
 \textit{
	\begin{itemize}
	\item They have tinkered with the geometry of the car's nose.
	\end{itemize}
}
\end{enumerate}

\section*{meditation}
{\large \color{blue}  meditations  }
\subsection*{Explain}
\begin{enumerate}
\item uncountable noun \\
\textbf{Meditation} is the act of remaining in a silent and calm state for a period of time, as part of a religious training , or so that you are more able to deal with the problems of everyday life.
 A \textbf{meditation} is a particular exercise that is used in meditation.
 \textit{
	\begin{itemize}
	\item Many busy executives have begun to practice yoga and meditation.
	\item It is important that you use that meditation for at least a fortnight.
	\end{itemize}
}
\item  \\
 See also  transcendental meditation \textit{
	\begin{itemize}
	\end{itemize}
}
\item uncountable noun \\
\textbf{Meditation} is the act of thinking about something very carefully and deeply for a long time.
 \textit{
	\begin{itemize}
	\item ...the man, lost in meditation, walking with slow steps along the shore.
	\item In his lonely meditations, Antony had been forced to the conclusion that there had
been rumours.
	\end{itemize}
}
\item countable noun \\
A \textbf{meditation}  \textbf{on} a particular subject is something such as a piece of writing or a speech which expresses  deep  thoughts about that subject.
 \textit{
	\begin{itemize}
	\item In fact, the entire novel is a long meditation on child-bearing and mortality.
	\item The title track is a pointed meditation on a continent gone wrong.
	\end{itemize}
}
\end{enumerate}

\section*{hay}
{\large \color{blue}  }
\subsection*{Explain}
\begin{enumerate}
\item uncountable noun \\
\textbf{Hay} is grass which has been cut and dried so that it can be used to feed animals.
 \textit{
	\begin{itemize}
	\item ...bales of hay.
	\item ...traditional hay making methods.
	\end{itemize}
}
\item  \\
 make hay/make hay while the sun shines \textit{
	\begin{itemize}
	\end{itemize}
}
\end{enumerate}

\section*{moisture}
{\large \color{blue}  }
\subsection*{Explain}
\begin{enumerate}
\item uncountable noun \\
\textbf{Moisture} is tiny  drops of water in the air, on a surface, or in the ground .
 \textit{
	\begin{itemize}
	\item When the soil is dry, more moisture is lost from the plant.
	\item Rainfall affects the moisture content of the atmosphere.
	\end{itemize}
}
\end{enumerate}

\section*{hostage}
{\large \color{blue}  hostages  }
\subsection*{Explain}
\begin{enumerate}
\item countable noun \\
A \textbf{hostage} is someone who has been captured by a person or organization and who may be killed or injured if people do not do what that person or organization demands .
 \textit{
	\begin{itemize}
	\item It is hopeful that two hostages will be freed in the next few days.
	\end{itemize}
}
\item  \\
 take sb hostage/hold sb hostage \textit{
	\begin{itemize}
	\end{itemize}
}
\item variable noun \\
If you say you are \textbf{hostage to} something, you mean that your freedom to take action is restricted by things that you cannot control .
 \textit{
	\begin{itemize}
	\item The bank is shrewdly ensuring that in future it will not be a hostage to strict targets.
	\item Wine growers say they've been held hostage to the interests of the cereal and soybean
farmers.
	\end{itemize}
}
\end{enumerate}

\section*{nickel}
{\large \color{blue}  nickels  }
\subsection*{Explain}
\begin{enumerate}
\item uncountable noun \\
\textbf{Nickel} is a silver-coloured metal that is used in making steel.
 \textit{
	\begin{itemize}
	\end{itemize}
}
\item countable noun \\
In the United  States and Canada , a \textbf{nickel} is a coin worth five cents.
 \textit{
	\begin{itemize}
	\end{itemize}
}
\end{enumerate}

\section*{humanity}
{\large \color{blue}  humanities  }
\subsection*{Explain}
\begin{enumerate}
\item uncountable noun \\
All the people in the world can be referred to as \textbf{humanity} .
 \textit{
	\begin{itemize}
	\item They face charges of committing crimes against humanity.
	\item ...a young lawyer full of illusions and love of humanity.
	\end{itemize}
}
\item uncountable noun \\
A person's \textbf{humanity} is their state of being a human being, rather than an animal or an object.
 \textit{
	\begin{itemize}
	\item ...a man who's almost lost his humanity in his bitter hatred of his rivals.
	\end{itemize}
}
\item uncountable noun \\
\textbf{Humanity} is the quality of being kind , thoughtful , and sympathetic towards others.
 \textit{
	\begin{itemize}
	\item Her speech showed great maturity and humanity.
	\end{itemize}
}
\item plural noun \\
\textbf{The}  \textbf{humanities} are the subjects such as history , philosophy , and literature which are concerned with human ideas and behaviour .
 \textit{
	\begin{itemize}
	\item ...students majoring in the humanities.
	\end{itemize}
}
\end{enumerate}

\section*{oak}
{\large \color{blue}  oaks  }
\subsection*{Explain}
\begin{enumerate}
\item variable noun \\
An \textbf{oak} or an \textbf{oak tree} is a large tree that often grows in woods and forests and has strong , hard wood.
 \textbf{Oak} is the wood of this tree.
 \textit{
	\begin{itemize}
	\item Many large oaks were felled during the war.
	\item ...forests of beech, chestnut, and oak.
	\item The cabinet was made of oak.
	\end{itemize}
}
\end{enumerate}

\section*{industry}
{\large \color{blue}  industries  }
\subsection*{Explain}
\begin{enumerate}
\item uncountable noun \\
\textbf{Industry} is the work and processes involved in collecting raw materials, and making them into products in factories .
 \textit{
	\begin{itemize}
	\item British industry suffers through insufficient investment in research.
	\item ...in countries where industry is developing rapidly.
	\end{itemize}
}
\item countable noun \\
A particular \textbf{industry} consists of all the people and activities involved in making a particular product
or providing a particular service.
 \textit{
	\begin{itemize}
	\item ...the motor vehicle and textile industries.
	\item ...the Scottish tourist industry.
	\end{itemize}
}
\item countable noun \\
If you refer to a social or political activity as an \textbf{industry} , you are criticizing it because you think it involves a lot of people in unnecessary or useless work.
 \textit{
	\begin{itemize}
	\item Some Afro-Caribbeans are rejecting the whole race relations industry.
	\item The multibillion-dollar fitness industry rakes in fat profits from our hunger to
look good.
	\end{itemize}
}
\item uncountable noun \\
\textbf{Industry} is the fact of working very hard.
 \textit{
	\begin{itemize}
	\item No one doubted his ability, his industry or his integrity.
	\end{itemize}
}
\end{enumerate}

\section*{outrage}
{\large \color{blue}  outrages  outraging  outraged  }
\subsection*{Explain}
\begin{enumerate}
\item verb \\
If you \textbf{are outraged} by something, it makes you extremely shocked and angry .
 \textit{
	\begin{itemize}
	\item Many people have been outraged by some of the things that have been said.
	\item Reports of torture and mass executions in Serbia's detention camps have outraged
the world's religious leaders.
	\end{itemize}
}
\item uncountable noun \\
\textbf{Outrage} is an intense feeling of anger and shock.
 \textit{
	\begin{itemize}
	\item The decision provoked outrage from women and human rights groups.
	\item The Treaty has failed to arouse genuine public outrage.
	\end{itemize}
}
\item countable noun \\
You can refer to an act or event which you find very shocking as an \textbf{outrage} .
 \textit{
	\begin{itemize}
	\item The latest outrage was to have been a co-ordinated gun and bomb attack on the station.
	\item Tom, this is an outrage!
	\end{itemize}
}
\end{enumerate}

\section*{instrument}
{\large \color{blue}  instruments  }
\subsection*{Explain}
\begin{enumerate}
\item countable noun \\
An \textbf{instrument} is a tool or device that is used to do a particular task , especially a scientific task.
 \textit{
	\begin{itemize}
	\item ...a thin tube-like optical instrument.
	\item ...instruments for cleaning and polishing teeth.
	\item The environment itself will at the same time be measured by about 60 scientific instruments.
	\end{itemize}
}
\item countable noun \\
A musical \textbf{instrument} is an object such as a piano , guitar , or flute , which you play in order to produce music.
 \textit{
	\begin{itemize}
	\item Learning a musical instrument introduces a child to an understanding of music.
	\end{itemize}
}
\item countable noun \\
An \textbf{instrument} is a device that is used for making measurements of something such as speed , height , or sound, for example on a ship or plane or in a car .
 \textit{
	\begin{itemize}
	\item ...crucial instruments on the control panel.
	\item ...navigation instruments.
	\end{itemize}
}
\item countable noun \\
Something that is an \textbf{instrument} for achieving a particular aim is used by people to achieve that aim.
 \textit{
	\begin{itemize}
	\item The veto has been a traditional instrument of diplomacy for centuries.
	\end{itemize}
}
\end{enumerate}

\section*{paste}
{\large \color{blue}  pastes  pasting  pasted  }
\subsection*{Explain}
\begin{enumerate}
\item variable noun \\
\textbf{Paste} is a soft, wet , sticky mixture of a substance and a liquid, which can be spread easily . Some types of paste are used to stick things together.
 \textit{
	\begin{itemize}
	\item He then sticks it back together with flour paste.
	\item Blend a little milk with the custard powder to form a paste.
	\item ...wallpaper paste.
	\end{itemize}
}
\item variable noun \\
\textbf{Paste} is a soft smooth mixture made of crushed meat, fruit, or vegetables . You can, for example , spread it onto bread or use it in cooking .
 \textit{
	\begin{itemize}
	\item ...tomato paste.
	\item ...fish-paste sandwiches.
	\end{itemize}
}
\item verb \\
If you \textbf{paste} something on a surface, you put glue or paste on it and stick it on the surface.
 \textit{
	\begin{itemize}
	\item ...pasting labels on bottles.
	\item Activists pasted up posters criticizing the leftist leaders.
	\end{itemize}
}
\item uncountable noun \\
\textbf{Paste} is a hard shiny glass that is used for making imitation jewellery .
 \textit{
	\begin{itemize}
	\item ...paste emeralds.
	\end{itemize}
}
\end{enumerate}

\section*{invasion}
{\large \color{blue}  invasions  }
\subsection*{Explain}
\begin{enumerate}
\item variable noun \\
If there is an \textbf{invasion} of a country, a foreign  army  enters it by force.
 \textit{
	\begin{itemize}
	\item ...seven years after the Roman invasion of Britain.
	\item He was commander in chief during the invasion of Panama.
	\end{itemize}
}
\item variable noun \\
If you refer to the arrival of a large number of people or things as an \textbf{invasion} , you are emphasizing that they are unpleasant or difficult to deal with.
 \textit{
	\begin{itemize}
	\item ...this year's annual invasion of flies, wasps and ants.
	\item Seaside resorts such as Blackpool and Brighton are preparing for a tourist invasion.
	\end{itemize}
}
\item variable noun \\
If you describe an action as an \textbf{invasion} , you disapprove of it because it affects someone or something in a way that is not wanted .
 \textit{
	\begin{itemize}
	\item Is reading a child's diary always a gross invasion of privacy?
	\end{itemize}
}
\end{enumerate}

\section*{pea}
{\large \color{blue}  peas  }
\subsection*{Explain}
\begin{enumerate}
\item countable noun \\
\textbf{Peas} are round green seeds which grow in long thin  cases and are eaten as a vegetable.
 \textit{
	\begin{itemize}
	\end{itemize}
}
\end{enumerate}

\section*{knife}
{\large \color{blue}  knives  knifes  knifing  knifed  }
\subsection*{Explain}
\begin{enumerate}
\item countable noun \\
A \textbf{knife} is a tool for cutting or a weapon and consists of a flat  piece of metal with a sharp  edge on the end of a handle.
 \textit{
	\begin{itemize}
	\item ...a knife and fork.
	\item Two robbers broke into her home, held a knife to her throat and stole her savings.
	\end{itemize}
}
\item verb \\
To \textbf{knife} someone means to attack and injure them with a knife.
 \textit{
	\begin{itemize}
	\item Dawson takes revenge on the man by knifing him to death.
	\end{itemize}
}
\item countable noun \\
A surgeon's \textbf{knife} is a piece of equipment used to cut flesh and organs during operations . It is made of metal and has a very thin sharp edge.
 \textit{
	\begin{itemize}
	\end{itemize}
}
\item  \\
 like a knife through butter/like a hot knife through butter \textit{
	\begin{itemize}
	\end{itemize}
}
\item  \\
 you can cut sth with a knife \textit{
	\begin{itemize}
	\end{itemize}
}
\item  \\
 the knives are out for sb \textit{
	\begin{itemize}
	\end{itemize}
}
\item  \\
 to twist the knife \textit{
	\begin{itemize}
	\end{itemize}
}
\end{enumerate}

\section*{pedal}
{\large \color{blue}  pedals  pedalling  pedalled  }
\subsection*{Explain}
\begin{enumerate}
\item countable noun \\
The \textbf{pedals} on a bicycle are the two parts that you push with your feet in order to make the bicycle move.
 \textit{
	\begin{itemize}
	\end{itemize}
}
\item verb \\
When you \textbf{pedal} a bicycle, you push the pedals around with your feet to make it move.
 \textit{
	\begin{itemize}
	\item She climbed on her bike with a feeling of pride and pedalled the five miles home.
	\item She was too tired to pedal back.
	\end{itemize}
}
\item countable noun \\
A \textbf{pedal} in a car or on a machine is a lever that you press with your foot in order to control the car or machine.
 \textit{
	\begin{itemize}
	\item ...the brake or accelerator pedals.
	\end{itemize}
}
\end{enumerate}

\section*{performance}
{\large \color{blue}  performances  }
\subsection*{Explain}
\begin{enumerate}
\item countable noun \\
A \textbf{performance} involves entertaining an audience by doing something such as singing , dancing, or acting.
 \textit{
	\begin{itemize}
	\item Inside the theatre, they were giving a performance of Bizet's Carmen.
	\item ...her performance as the betrayed Medea.
	\item The Festival Of Arts & Music will include two days of live performances.
	\end{itemize}
}
\item variable noun \\
Someone's or something's \textbf{performance} is how successful they are or how well they do something.
 \textit{
	\begin{itemize}
	\item That study looked at the performance of 18 surgeons.
	\item The poor performance has been blamed on the recession and cheaper sports car imports.
	\item The job of the new director-general was to ensure that performance targets were met.
	\end{itemize}
}
\item uncountable noun \\
A car's \textbf{performance} is its ability to go  fast and to increase its speed quickly.
 \textit{
	\begin{itemize}
	\end{itemize}
}
\item adjective \\
A \textbf{performance}  car is one that can go very fast and can increase its speed very quickly.
 \textit{
	\begin{itemize}
	\end{itemize}
}
\item singular noun \\
\textbf{The}  \textbf{performance}  \textbf{of} a task is the fact or action of doing it.
 \textit{
	\begin{itemize}
	\item He devoted in excess of seventy hours a week to the performance of his duties.
	\item The people believe that the performance of this ritual is the will of the Great Spirit.
	\end{itemize}
}
\item singular noun \\
You can describe something that is or looks  complicated or difficult to do as \textbf{a performance} .
 \textit{
	\begin{itemize}
	\item The whole process is quite a performance.
	\item She made a big performance of sprinkling all the spices on.
	\end{itemize}
}
\end{enumerate}

\section*{mankind}
{\large \color{blue}  }
\subsection*{Explain}
\begin{enumerate}
\item uncountable noun \\
You can refer to all human beings as \textbf{mankind} when considering them as a group. Some people dislike this use.
 \textit{
	\begin{itemize}
	\item ...the evolution of mankind.
	\end{itemize}
}
\end{enumerate}

\section*{pound}
{\large \color{blue}  pounds  pounding  pounded  }
\subsection*{Explain}
\begin{enumerate}
\item countable noun \\
The \textbf{pound} is the unit of money which is used in the UK. It is represented by the symbol £.
One UK pound is divided into a hundred pence. Some other countries, for example Egypt, also have a unit of money called
a \textbf{pound} .
 \textbf{The pound} is also used to refer to the UK currency system.
 \textit{
	\begin{itemize}
	\item Coffee is two pounds a cup.
	\item A thousand pounds worth of jewellery and silver has been stolen.
	\item ...multi-million pound profits.
	\item ...a pound coin.
	\item The pound is expected to continue to increase against most other currencies.
	\end{itemize}
}
\item singular noun \\
\textbf{The pound} is used to refer to the British currency system, and sometimes to the currency systems
of other countries which use pounds.
 \textit{
	\begin{itemize}
	\item The pound is expected to continue to increase against most other currencies.
	\end{itemize}
}
\item countable noun \\
A \textbf{pound} is a unit of weight used mainly in Britain, America, and other countries where English
is spoken. One pound is equal to 0.454 kilograms. A \textbf{pound of} something is a quantity of it that weighs one pound.
 \textit{
	\begin{itemize}
	\item Her weight was under ninety pounds.
	\item ...a pound of cheese.
	\end{itemize}
}
\item countable noun \\
A \textbf{pound} is a place where dogs and cats found wandering in the street are taken and kept until they are claimed by their
 owners .
 \textit{
	\begin{itemize}
	\end{itemize}
}
\item countable noun \\
A \textbf{pound} is a place where cars that have been parked illegally are taken by the police and kept until they have been claimed by their
owners.
 \textit{
	\begin{itemize}
	\end{itemize}
}
\item verb \\
If you \textbf{pound} something or \textbf{pound on} it, you hit it with great force, usually loudly and repeatedly.
 \textit{
	\begin{itemize}
	\item He pounded the table with his fist.
	\item Somebody began pounding on the front door.
	\item She came at him, pounding her fists against his chest.
	\item ...the pounding waves.
	\end{itemize}
}
\item verb \\
If you \textbf{pound} something, you crush it into a paste or a powder or into very small pieces.
 \textit{
	\begin{itemize}
	\item She paused as she pounded the maize grains.
	\end{itemize}
}
\item verb \\
If your heart \textbf{is pounding} , it is beating with an unusually strong and fast  rhythm , usually because you are afraid .
 \textit{
	\begin{itemize}
	\item I'm sweating, my heart is pounding. I can't breathe.
	\end{itemize}
}
\item  \\
 pound of flesh \textit{
	\begin{itemize}
	\end{itemize}
}
\end{enumerate}

\section*{march}
{\large \color{blue}  marches  marching  marched  }
\subsection*{Explain}
\begin{enumerate}
\item verb \\
When soldiers  \textbf{march}  somewhere , or when a commanding  officer  \textbf{marches} them somewhere, they walk there with very regular steps, as a group.
 \textbf{March} is also a noun .
 \textit{
	\begin{itemize}
	\item A Scottish battalion was marching down the street.
	\item Captain Ramirez called them to attention and marched them off to the main camp.
	\item We marched fifteen miles to Yadkin River.
	\item The ice was not thick enough to bear the weight of marching men.
	\item After a short march, the column entered the village.
	\end{itemize}
}
\item verb \\
When a large group of people \textbf{march} for a cause, they walk somewhere together in order to express their ideas or to protest about something.
 \textbf{March} is also a noun.
 \textit{
	\begin{itemize}
	\item The demonstrators then marched through the capital chanting slogans and demanding
free elections.
	\item Hundreds of activists marked the holy day by marching for peace and disarmament.
	\item Organisers expect up to 300,000 protesters to join the march.
	\end{itemize}
}
\item verb \\
If you say that someone \textbf{marches} somewhere, you mean that they walk there quickly and in a determined way, for example because they are angry .
 \textit{
	\begin{itemize}
	\item He marched into the kitchen without knocking.
	\end{itemize}
}
\item verb \\
If you \textbf{march} someone somewhere, you force them to walk there with you, for example by holding their arm tightly.
 \textit{
	\begin{itemize}
	\item Nearly 700 prisoners were marched away.
	\item I marched him across the room, down the hall and out onto the doorstep.
	\end{itemize}
}
\item singular noun \\
\textbf{The}  \textbf{march}  \textbf{of} something is its steady  development or progress .
 \textit{
	\begin{itemize}
	\item It is easy to feel trampled by the relentless march of technology.
	\item Society's march toward ever-increasing materialism was continuing.
	\end{itemize}
}
\item countable noun \\
A \textbf{march} is a piece of music with a regular rhythm that you can march to.
 \textit{
	\begin{itemize}
	\item A military band played Russian marches and folk tunes.
	\end{itemize}
}
\item  \\
 your marching orders \textit{
	\begin{itemize}
	\end{itemize}
}
\item  \\
 to steal a march on someone \textit{
	\begin{itemize}
	\end{itemize}
}
\end{enumerate}

\section*{premium}
{\large \color{blue}  premiums  }
\subsection*{Explain}
\begin{enumerate}
\item countable noun \\
A \textbf{premium} is a sum of money that you pay regularly to an insurance company for an insurance policy.
 \textit{
	\begin{itemize}
	\item It is too early to say whether insurance premiums will be affected.
	\end{itemize}
}
\item countable noun \\
A \textbf{premium} is a sum of money that you have to pay for something in addition to the normal  cost .
 \textit{
	\begin{itemize}
	\item Even if customers want 'solutions', most are not willing to pay a premium for them.
	\item Callers are charged a premium rate of 48p a minute.
	\end{itemize}
}
\item adjective \\
\textbf{Premium} goods are of a higher than usual quality and are often expensive .
 \textit{
	\begin{itemize}
	\item At the premium end of the market, business is booming.
	\item ...the most popular premium ice cream in this country.
	\end{itemize}
}
\item  \\
 at a premium \textit{
	\begin{itemize}
	\end{itemize}
}
\item  \\
 at a premium \textit{
	\begin{itemize}
	\end{itemize}
}
\item  \\
 place a high premium on sth/put a high premium on sth \textit{
	\begin{itemize}
	\end{itemize}
}
\end{enumerate}

\section*{pavement}
{\large \color{blue}  pavements  }
\subsection*{Explain}
\begin{enumerate}
\item countable noun \\
A \textbf{pavement} is a path with a hard surface, usually by the side of a road.
 \textit{
	\begin{itemize}
	\item He was hurrying along the pavement.
	\end{itemize}
}
\item countable noun \\
The \textbf{pavement} is the hard surface of a road.
 \textit{
	\begin{itemize}
	\end{itemize}
}
\end{enumerate}

\section*{rice}
{\large \color{blue}  rices  }
\subsection*{Explain}
\begin{enumerate}
\item variable noun \\
\textbf{Rice} consists of white or brown grains taken from a cereal plant. You cook rice and usually eat it with meat or vegetables.
 \textit{
	\begin{itemize}
	\item ...a meal consisting of chicken, rice and vegetables.
	\item Thailand exports its fine rices around the world.
	\end{itemize}
}
\end{enumerate}

\section*{riot}
{\large \color{blue}  riots  rioting  rioted  }
\subsection*{Explain}
\begin{enumerate}
\item countable noun \\
When there is a \textbf{riot} , a crowd of people behave violently in a public place, for example they fight , throw  stones , or damage buildings and vehicles .
 \textit{
	\begin{itemize}
	\item Twelve inmates have been killed during a riot at the prison.
	\end{itemize}
}
\item verb \\
If people \textbf{riot} , they behave violently in a public place.
 \textit{
	\begin{itemize}
	\item Last year 600 inmates rioted, starting fires and building barricades.
	\item They rioted in protest against the government.
	\end{itemize}
}
\item singular noun \\
If you say that there is \textbf{a riot of} something pleasant such as colour, you mean that there is a large amount of various types of it.
 \textit{
	\begin{itemize}
	\item All the cacti were in flower, so that the desert was a riot of colour.
	\item ...a riot of tastes and spices.
	\end{itemize}
}
\item  \\
 read someone the riot act \textit{
	\begin{itemize}
	\end{itemize}
}
\item  \\
 to run riot \textit{
	\begin{itemize}
	\end{itemize}
}
\item  \\
 to run riot \textit{
	\begin{itemize}
	\end{itemize}
}
\end{enumerate}

\section*{personality}
{\large \color{blue}  personalities  }
\subsection*{Explain}
\begin{enumerate}
\item variable noun \\
Your \textbf{personality} is your whole character and nature .
 \textit{
	\begin{itemize}
	\item She has such a kind, friendly personality.
	\item Through sheer force of personality Hugh Trenchard had got his way.
	\item These personality traits get passed on from generation to generation.
	\item The contest was as much about personalities as it was about politics.
	\end{itemize}
}
\item variable noun \\
If someone has \textbf{personality} or is a \textbf{personality} , they have a strong and lively character.
 \textit{
	\begin{itemize}
	\item ...a woman of great personality.
	\item He is such a personality–he is so funny.
	\end{itemize}
}
\item countable noun \\
You can refer to a famous person, especially in entertainment, broadcasting , or sport, as a \textbf{personality} .
 \textit{
	\begin{itemize}
	\item She is one of the country's most highly paid television personalities.
	\end{itemize}
}
\end{enumerate}

\section*{rubber}
{\large \color{blue}  rubbers  }
\subsection*{Explain}
\begin{enumerate}
\item uncountable noun \\
\textbf{Rubber} is a strong, waterproof, elastic substance made from the juice of a tropical tree or produced chemically. It is used for making tyres , boots , and other products.
 \textit{
	\begin{itemize}
	\item ...the smell of burning rubber.
	\end{itemize}
}
\item adjective \\
\textbf{Rubber} things are made of rubber.
 \textit{
	\begin{itemize}
	\item ...rubber gloves.
	\item ...a rubber ball.
	\end{itemize}
}
\item countable noun \\
A \textbf{rubber} is a small piece of rubber or other material that is used to remove mistakes that you have made while writing or drawing .
 \textit{
	\begin{itemize}
	\end{itemize}
}
\item countable noun \\
A \textbf{rubber} is a condom.
 \textit{
	\begin{itemize}
	\end{itemize}
}
\item countable noun \\
In some card games, for example  bridge or whist , a \textbf{rubber} is a match of three games.
 \textit{
	\begin{itemize}
	\item Let's have a few rubbers of bridge.
	\end{itemize}
}
\end{enumerate}

\section*{power}
{\large \color{blue}  powers  powering  powered  }
\subsection*{Explain}
\begin{enumerate}
\item uncountable noun \\
If someone has \textbf{power} , they have a lot of control over people and activities.
 \textit{
	\begin{itemize}
	\item She interviewed six women who have reached positions of great power and influence.
	\item In a democracy, power must be divided.
	\item ...a political power struggle between the Liberals and National Party.
	\end{itemize}
}
\item uncountable noun \\
Your \textbf{power}  \textbf{to} do something is your ability to do it.
 \textit{
	\begin{itemize}
	\item Human societies have the power to solve the problems confronting them.
	\item He was so drunk that he had lost the power of speech.
	\end{itemize}
}
\item uncountable noun \\
If it is \textbf{in} or \textbf{within} your \textbf{power}  \textbf{to} do something, you are able to do it or you have the resources to deal with it.
 \textit{
	\begin{itemize}
	\item Your debt situation is only temporary, and it is within your power to resolve it.
	\item Although it is not in his power to do so, he said he would rebuild the Air Base.
	\item We must do everything in our power to ensure the success of the conference.
	\end{itemize}
}
\item uncountable noun \\
If someone in authority has the \textbf{power} to do something, they have the legal right to do it.
 \textit{
	\begin{itemize}
	\item The Prime Minister has the power to dismiss and appoint senior ministers.
	\item The police have the power of arrest.
	\item The legal powers of British Customs officers are laid out in the Customs and Excise
Management Act of 1969.
	\end{itemize}
}
\item uncountable noun \\
If people take \textbf{power} or come to \textbf{power} , they take charge of a country's affairs . If a group of people are \textbf{in power} , they are in charge of a country's affairs.
 \textit{
	\begin{itemize}
	\item In 1964 Labour came into power.
	\item He first assumed power in 1970.
	\item The party has been in power since independence in 1964.
	\end{itemize}
}
\item countable noun \\
You can use \textbf{power} to refer to a country that is very rich or important, or has strong military forces.
 \textit{
	\begin{itemize}
	\item The country is a major power in an area of great strategic importance.
	\item ...the emergence of a new major economic power.
	\end{itemize}
}
\item uncountable noun \\
The \textbf{power} of something is the ability that it has to move or affect things.
 \textit{
	\begin{itemize}
	\item The Roadrunner had better power, better tyres, and better brakes.
	\item ...massive computing power.
	\end{itemize}
}
\item uncountable noun \\
\textbf{Power} is energy, especially  electricity , that is obtained in large quantities from a fuel source and used to operate lights, heating, and machinery .
 \textit{
	\begin{itemize}
	\item Nuclear power is cleaner than coal.
	\item Power has been restored to most parts that were hit last night by high winds.
	\item There is enough power to run up to four lights.
	\end{itemize}
}
\item verb \\
The device or fuel that \textbf{powers} a machine provides the energy that the machine needs in order to work.
 \textit{
	\begin{itemize}
	\item The battery could power an electric car for 600 miles on a single charge.
	\item The planes are powered by Rolls Royce engines.
	\end{itemize}
}
\item adjective \\
\textbf{Power} tools are operated by electricity.
 \textit{
	\begin{itemize}
	\item ...large power tools, such as chainsaws.
	\item ...a power drill.
	\end{itemize}
}
\item singular noun \\
In mathematics , \textbf{power} is used in expressions such as \textbf{2 to the power of 4} or \textbf{2 to the 4th power} to indicate that 2 must be multiplied by itself 4 times. This is written in numbers as 2⁴, or 2 x 2 x 2 x 2, which equals
16.
 \textit{
	\begin{itemize}
	\item Any number to the power of nought is equal to one.
	\end{itemize}
}
\item  \\
 the powers that be \textit{
	\begin{itemize}
	\end{itemize}
}
\end{enumerate}

\section*{shark}
{\large \color{blue}  sharks  }
\subsection*{Explain}
\begin{enumerate}
\item variable noun \\
A \textbf{shark} is a very large fish. Some sharks have very sharp teeth and may attack people.
 \textit{
	\begin{itemize}
	\end{itemize}
}
\item countable noun \\
If you refer to a person as a \textbf{shark} , you disapprove of them because they trick people out of their money by giving bad  advice about buying , selling, or investments .
 \textit{
	\begin{itemize}
	\item Beware the sharks when you are making up your mind how to invest.
	\end{itemize}
}
\end{enumerate}

\section*{sandwich}
{\large \color{blue}  sandwiches  sandwiching  sandwiched  }
\subsection*{Explain}
\begin{enumerate}
\item countable noun \\
A \textbf{sandwich} usually consists of two slices of bread with a layer of food such as cheese or meat between them.
 \textit{
	\begin{itemize}
	\item ...a ham sandwich.
	\end{itemize}
}
\item verb \\
If you \textbf{sandwich} two things \textbf{together} with something else, you put that other thing between them. If you \textbf{sandwich} one thing between two other things, you put it between them.
 \textit{
	\begin{itemize}
	\item Sandwich the two halves of the sponge together with cream.
	\item I had to sandwich my writing between supervising work on the farm and entertaining
guests.
	\end{itemize}
}
\end{enumerate}

\section*{shock}
{\large \color{blue}  shocks  shocking  shocked  }
\subsection*{Explain}
\begin{enumerate}
\item countable noun \\
If you have a \textbf{shock} , something suddenly  happens which is unpleasant , upsetting , or very surprising.
 \textit{
	\begin{itemize}
	\item The extent of the violence came as a shock.
	\item He has never recovered from the shock of your brother's death.
	\item It was quite a shock to see my face on that screen!
	\end{itemize}
}
\item uncountable noun \\
\textbf{Shock} is a person's emotional and physical condition when something very frightening or upsetting has happened to them.
 \textit{
	\begin{itemize}
	\item The little boy was speechless with shock.
	\item She's still in a state of shock.
	\end{itemize}
}
\item uncountable noun \\
If someone is \textbf{in}  \textbf{shock} , they are suffering from a serious physical condition in which their blood is not flowing round their body properly,
for example because they have had a bad  injury .
 \textit{
	\begin{itemize}
	\item He was found beaten and in shock.
	\item They escaped the blaze but were rushed to hospital suffering from shock.
	\end{itemize}
}
\item verb \\
If something \textbf{shocks} you, it makes you feel very upset, because it involves death or suffering and because you had not expected it.
 \textit{
	\begin{itemize}
	\item After forty years in the police force nothing much shocks me.
	\item Relief workers were shocked by what they saw.
	\end{itemize}
}
\item verb \\
If someone or something \textbf{shocks} you, it upsets or offends you because you think it is rude or morally wrong .
 \textit{
	\begin{itemize}
	\item You can't shock me.
	\item They were easily shocked in those days.
	\item We were always trying to be creative and to shock.
	\end{itemize}
}
\item adjective \\
A \textbf{shock}  announcement or event is one which shocks people because it is unexpected .
 \textit{
	\begin{itemize}
	\item ...the shock announcement that she is to resign.
	\item ...a shock defeat.
	\end{itemize}
}
\item countable noun \\
A \textbf{shock} is something sudden and unexpected that threatens the economy , traditions , or way of life of a group of people.
 \textit{
	\begin{itemize}
	\item This is the latest in a series of shocks to the banking system.
	\item ...the economic pain of two oil shocks.
	\end{itemize}
}
\item variable noun \\
A \textbf{shock} is the force of something suddenly hitting or pulling something else.
 \textit{
	\begin{itemize}
	\item Steel barriers can bend and absorb the shock.
	\end{itemize}
}
\item countable noun \\
A \textbf{shock} is the same as an electric shock .
 \textit{
	\begin{itemize}
	\end{itemize}
}
\item countable noun \\
A \textbf{shock} is a shock absorber .
 \textit{
	\begin{itemize}
	\item Do you think I need new shocks?
	\end{itemize}
}
\item countable noun \\
A \textbf{shock of} hair is a very thick mass of hair on a person's head.
 \textit{
	\begin{itemize}
	\item ...a very old priest with a shock of white hair.
	\end{itemize}
}
\item  \\
 short, sharp shock \textit{
	\begin{itemize}
	\end{itemize}
}
\end{enumerate}

\section*{september}
{\large \color{blue}  Septembers  }
\subsection*{Explain}
\begin{enumerate}
\item variable noun \\
\textbf{September} is the ninth month of the year in the Western  calendar .
 \textit{
	\begin{itemize}
	\item Her son, Jerome, was born in September.
	\item They returned to Moscow on 22 September 1930.
	\item They spent a couple of nights here last September.
	\end{itemize}
}
\end{enumerate}

\section*{skyscraper}
{\large \color{blue}  skyscrapers  }
\subsection*{Explain}
\begin{enumerate}
\item countable noun \\
A \textbf{skyscraper} is a very tall building in a city .
 \textit{
	\begin{itemize}
	\end{itemize}
}
\end{enumerate}

\section*{soldier}
{\large \color{blue}  soldiers  soldiering  soldiered  }
\subsection*{Explain}
\begin{enumerate}
\item countable noun \\
A \textbf{soldier} is a person who works in an army, especially a person who is not an officer.
 \textit{
	\begin{itemize}
	\end{itemize}
}
\end{enumerate}

\section*{snowstorm}
{\large \color{blue}  snowstorms  }
\subsection*{Explain}
\begin{enumerate}
\item countable noun \\
A \textbf{snowstorm} is a very heavy fall of snow, usually when there is also a strong wind  blowing at the same time.
 \textit{
	\begin{itemize}
	\end{itemize}
}
\end{enumerate}

\section*{son}
{\large \color{blue}  sons  }
\subsection*{Explain}
\begin{enumerate}
\item countable noun \\
Someone's \textbf{son} is their male child .
 \textit{
	\begin{itemize}
	\item He shared a pizza with his son Laurence.
	\item Sam is the seven-year-old son of Eric Davies.
	\item They have a son.
	\end{itemize}
}
\item countable noun \\
A man, especially a famous man, can be described as a \textbf{son} of the place he comes from.
 \textit{
	\begin{itemize}
	\item ...New Orleans's most famous son, Louis Armstrong.
	\item ...sons of Africa.
	\end{itemize}
}
\item countable noun \\
Some people use \textbf{son} as a form of address when they are showing  kindness or affection to a boy or a man who is younger than them.
 \textit{
	\begin{itemize}
	\item Don't be frightened by failure, son.
	\end{itemize}
}
\end{enumerate}

\section*{storm}
{\large \color{blue}  storms  storming  stormed  }
\subsection*{Explain}
\begin{enumerate}
\item countable noun \\
A \textbf{storm} is very bad weather, with heavy rain, strong winds, and often thunder and lightning.
 \textit{
	\begin{itemize}
	\item ...the violent storms which whipped America's East Coast.
	\end{itemize}
}
\item countable noun \\
If something causes a \textbf{storm} , it causes an angry or excited reaction from a large number of people.
 \textit{
	\begin{itemize}
	\item The photos caused a storm when they were first published.
	\item The announcement provoked an immediate storm of protest.
	\item ...the storm of publicity that Richard's book had generated.
	\end{itemize}
}
\item countable noun \\
A \textbf{storm}  \textbf{of}  applause or other noise is a sudden  loud amount of it made by an audience or other group of people in reaction to something.
 \textit{
	\begin{itemize}
	\item His speech was greeted with a storm of applause.
	\item The medals ceremony caused a storm of booing.
	\end{itemize}
}
\item verb \\
If you \textbf{storm}  \textbf{into} or \textbf{out of} a place, you enter or leave it quickly and noisily, because you are angry.
 \textit{
	\begin{itemize}
	\item After a bit of an argument, he stormed out.
	\item He stormed into an office, demanding to know where the head of department was.
	\end{itemize}
}
\item verb \\
If you \textbf{storm} , you say something in a very loud voice , because you are extremely angry.
 \textit{
	\begin{itemize}
	\item 'It's a fiasco,' he stormed.
	\end{itemize}
}
\item verb \\
If a place that is being defended  \textbf{is stormed} , a group of people attack it, usually in order to get  inside it.
 \textit{
	\begin{itemize}
	\item Government buildings have been stormed and looted.
	\item The refugees decided to storm the embassy.
	\end{itemize}
}
\item  \\
 take sth by storm \textit{
	\begin{itemize}
	\end{itemize}
}
\item  \\
 to weather the storm \textit{
	\begin{itemize}
	\end{itemize}
}
\end{enumerate}

\section*{talent}
{\large \color{blue}  talents  }
\subsection*{Explain}
\begin{enumerate}
\item variable noun \\
\textbf{Talent} is the natural ability to do something well .
 \textit{
	\begin{itemize}
	\item She is proud that both her children have a talent for music.
	\item The player was given hardly any opportunities to show off his talents.
	\item He's got lots of talent.
	\end{itemize}
}
\end{enumerate}

\section*{straw}
{\large \color{blue}  straws  }
\subsection*{Explain}
\begin{enumerate}
\item uncountable noun \\
\textbf{Straw} consists of the dried, yellowish stalks from crops such as wheat or barley.
 \textit{
	\begin{itemize}
	\item The barn was full of bales of straw.
	\item I stumbled through mud to a yard strewn with straw.
	\item ...a wide-brimmed straw hat.
	\end{itemize}
}
\item countable noun \\
A \textbf{straw} is a thin tube of paper or plastic, which you use to suck a drink into your mouth.
 \textit{
	\begin{itemize}
	\item ...a bottle of lemonade with a straw in it.
	\end{itemize}
}
\item  \\
 to clutch at straws \textit{
	\begin{itemize}
	\end{itemize}
}
\item  \\
 the last straw \textit{
	\begin{itemize}
	\end{itemize}
}
\item  \\
 to draw the short straw \textit{
	\begin{itemize}
	\end{itemize}
}
\item  \\
 straw in the wind \textit{
	\begin{itemize}
	\end{itemize}
}
\end{enumerate}

\section*{submarine}
{\large \color{blue}  submarines  }
\subsection*{Explain}
\begin{enumerate}
\item countable noun \\
A \textbf{submarine} is a type of ship that can travel both above and below the surface of the sea. The abbreviation  sub is also used.
 \textit{
	\begin{itemize}
	\item ...a nuclear submarine.
	\end{itemize}
}
\item adjective \\
\textbf{Submarine} means existing below the surface of the sea.
 \textit{
	\begin{itemize}
	\item ...submarine caves.
	\item ...submarine plants.
	\end{itemize}
}
\item countable noun \\
A \textbf{submarine}  sandwich is a long soft  bread  roll  filled with a combination of things such as meat , cheese , eggs , and salad . The abbreviation sub is also used.
 \textit{
	\begin{itemize}
	\end{itemize}
}
\end{enumerate}

\section*{tool}
{\large \color{blue}  tools  }
\subsection*{Explain}
\begin{enumerate}
\item countable noun \\
A \textbf{tool} is any instrument or simple piece of equipment that you hold in your hands and use to do a particular kind of work. For example , spades, hammers, and knives are all tools.
 \textit{
	\begin{itemize}
	\item I find the best tool for the purpose is a pair of shears.
	\end{itemize}
}
\item countable noun \\
You can refer to anything that you use for a particular purpose as a particular type of \textbf{tool} .
 \textit{
	\begin{itemize}
	\item Writing is a good tool for discharging overwhelming feelings.
	\item The computer has become an invaluable teaching tool.
	\item The threat of bankruptcy is a legitimate tool to extract money from them.
	\end{itemize}
}
\item countable noun \\
If you describe someone as a \textbf{tool} of a particular person, group, or system, you mean that they are controlled and used
by that person, group, or system, especially to do unpleasant or dishonest things.
 \textit{
	\begin{itemize}
	\item He became the tool of the security services.
	\end{itemize}
}
\item  \\
 to down tools \textit{
	\begin{itemize}
	\end{itemize}
}
\item  \\
 tools of one's/the trade \textit{
	\begin{itemize}
	\end{itemize}
}
\end{enumerate}

\section*{theme}
{\large \color{blue}  themes  }
\subsection*{Explain}
\begin{enumerate}
\item countable noun \\
A \textbf{theme} in a piece of writing , a talk , or a discussion is an important idea or subject that runs through it.
 \textit{
	\begin{itemize}
	\item The theme of the summit was women as 'agents of change'.
	\item One of Chomsky's main themes is that the appearance of open debate is illusion.
	\item The need to strengthen the family has been a recurrent theme for the Prime Minister.
	\end{itemize}
}
\item countable noun \\
A \textbf{theme} in an artist's work or in a work of literature is an idea in it that the artist or writer develops or repeats.
 \textit{
	\begin{itemize}
	\item The novel's central theme is the perennial conflict between men and women.
	\item This painting points to another recurring theme in Munch's work.
	\end{itemize}
}
\item countable noun \\
A \textbf{theme} is a short simple  tune on which a piece of music is based.
 \textit{
	\begin{itemize}
	\item ...variations on themes from Mozart's The Magic Flute.
	\end{itemize}
}
\item countable noun \\
\textbf{Theme} music or a \textbf{theme}  song is a piece of music that is played at the beginning and end of a film or of a television or radio  programme .
 \textit{
	\begin{itemize}
	\item ...the theme from Dr Zhivago.
	\item The programme has an appallingly catchy theme tune.
	\end{itemize}
}
\end{enumerate}

\section*{triangle}
{\large \color{blue}  triangles  }
\subsection*{Explain}
\begin{enumerate}
\item countable noun \\
A \textbf{triangle} is an object, arrangement , or flat shape with three straight sides and three angles.
 \textit{
	\begin{itemize}
	\item This design is in pastel colours with three rectangles and three triangles.
	\item Its outline roughly forms an equilateral triangle.
	\item ...triangles of fried bread.
	\item ...the great triangle of the Lancashire textile towns, stretching from Manchester
up as far as Burnley and across to Accrington.
	\end{itemize}
}
\item countable noun \\
The \textbf{triangle} is a musical instrument that consists of a piece of metal shaped like a triangle. You play it by hitting it with a short metal bar.
 \textit{
	\begin{itemize}
	\end{itemize}
}
\item countable noun \\
If you describe a group of three people as a \textbf{triangle} , you mean that they are all connected with each other in a particular situation, but often have different  interests .
 \textit{
	\begin{itemize}
	\item She plays a French woman in a love triangle with her two best friends.
	\item ...the classic triangle of husband, wife and mistress.
	\end{itemize}
}
\end{enumerate}

\section*{tide}
{\large \color{blue}  tides  tiding  tided  }
\subsection*{Explain}
\begin{enumerate}
\item countable noun \\
\textbf{The}  \textbf{tide} is the regular change in the level of the sea on the shore .
 \textit{
	\begin{itemize}
	\item The tide was at its highest.
	\item The tide was going out, and the sand was smooth and glittering.
	\item State police say that high tides and severe flooding have damaged beaches.
	\end{itemize}
}
\item countable noun \\
A \textbf{tide} is a current in the sea that is caused by the regular and continuous movement of large areas of water towards and away from the shore.
 \textit{
	\begin{itemize}
	\item Roman vessels used to sail with the tide from Boulogne to Richborough.
	\end{itemize}
}
\item singular noun \\
The \textbf{tide of}  opinion , for example , is what the majority of people think at a particular time.
 \textit{
	\begin{itemize}
	\item The tide of opinion seems overwhelmingly in his favour.
	\end{itemize}
}
\item singular noun \\
People sometimes  refer to events or forces that are difficult or impossible to control as \textbf{the tide of}  history , for example.
 \textit{
	\begin{itemize}
	\item They talked of reversing the tide of history.
	\item The tide of war swept back across their country.
	\end{itemize}
}
\item singular noun \\
You can talk about a \textbf{tide of} something, especially something which is unpleasant , when there is a large and increasing amount of it.
 \textit{
	\begin{itemize}
	\item ...an ever increasing tide of crime.
	\item The tide of nationalism is still running high in a number of republics.
	\end{itemize}
}
\end{enumerate}

\section*{triple}
{\large \color{blue}  triples  tripling  tripled  }
\subsection*{Explain}
\begin{enumerate}
\item adjective \\
\textbf{Triple}  means consisting of three things or parts.
 \textit{
	\begin{itemize}
	\item ...a triple somersault.
	\item In 1882 Germany, Austria, and Italy formed the Triple Alliance.
	\end{itemize}
}
\item verb \\
If something \textbf{triples} or if you \textbf{triple} it, it becomes three times as large in size or number.
 \textit{
	\begin{itemize}
	\item I got a fantastic new job and my salary tripled.
	\item The Exhibition has tripled in size from last year.
	\item The merger puts the firm in a position to triple its earnings.
	\end{itemize}
}
\item predeterminer \\
If something is \textbf{triple the} amount or size of another thing, it is three times as large.
 \textit{
	\begin{itemize}
	\item The mine reportedly had an accident rate triple the national average.
	\item The kitchen is triple the size it once was.
	\end{itemize}
}
\end{enumerate}

\section*{trail}
{\large \color{blue}  trails  trailing  trailed  }
\subsection*{Explain}
\begin{enumerate}
\item countable noun \\
A \textbf{trail} is a rough path across open country or through forests .
 \textit{
	\begin{itemize}
	\item He was following a broad trail through the trees.
	\end{itemize}
}
\item countable noun \\
A \textbf{trail} is a route along a series of paths or roads, often one that has been planned and marked out for a particular purpose.
 \textit{
	\begin{itemize}
	\item ...a large area of woodland with hiking and walking trails.
	\end{itemize}
}
\item countable noun \\
A \textbf{trail} is a series of marks or other signs of movement or other activities left by someone or something.
 \textit{
	\begin{itemize}
	\item Everywhere in the house was a sticky trail of orange juice.
	\item He left a trail of clues at the scenes of his crimes.
	\item The typhoon has left a trail of death and destruction across much of central Japan.
	\end{itemize}
}
\item verb \\
If you \textbf{trail} someone or something, you follow them secretly, often by finding the marks or signs that they have left.
 \textit{
	\begin{itemize}
	\item Two detectives were trailing him.
	\item I trailed her to a shop in Kensington.
	\end{itemize}
}
\item countable noun \\
You can refer to all the places that a politician  visits in the period before an election as their campaign  \textbf{trail} .
 \textit{
	\begin{itemize}
	\item During a recent speech on the campaign trail, he was interrupted by hecklers.
	\item ...at the end of a hard day on the election trail.
	\end{itemize}
}
\item verb \\
If you \textbf{trail} something or it \textbf{trails} , it hangs down loosely behind you as you move along.
 \textit{
	\begin{itemize}
	\item She came down the stairs slowly, trailing the coat behind her.
	\item He let his fingers trail in the water.
	\end{itemize}
}
\item verb \\
If someone \textbf{trails}  somewhere , they move there slowly, without any energy or enthusiasm , often following someone else.
 \textit{
	\begin{itemize}
	\item He trailed through the wet Manhattan streets.
	\item I spent a long afternoon trailing behind him.
	\end{itemize}
}
\item verb \\
If a person or team in a sports match or other contest  \textbf{is trailing} , they have a lower score than their opponents .
 \textit{
	\begin{itemize}
	\item He scored again, leaving Dartford trailing 3-0 at the break.
	\item She took over as chief executive of the company when it was trailing behind its competitors.
	\end{itemize}
}
\item  \\
 on the trail of \textit{
	\begin{itemize}
	\end{itemize}
}
\end{enumerate}

\section*{work}
{\large \color{blue}  works  working  worked  }
\subsection*{Explain}
\begin{enumerate}
\item verb \\
People who \textbf{work} have a job, usually one which they are paid to do.
 \textit{
	\begin{itemize}
	\item He works for the U.S. Department of Transport.
	\item I started working in a recording studio.
	\item Where do you work?
	\item He worked as a bricklayer's mate.
	\item I want to work, I don't want to be on welfare.
	\end{itemize}
}
\item uncountable noun \\
People who have \textbf{work} or who are \textbf{in work} have a job, usually one which they are paid to do.
 \textit{
	\begin{itemize}
	\item Fewer and fewer people are in work.
	\item I was out of work at the time.
	\item She'd have enough money to provide for her children until she could find work.
	\item What kind of work do you do?
	\end{itemize}
}
\item verb \\
When you \textbf{work} , you do the things that you are paid or required to do in your job.
 \textit{
	\begin{itemize}
	\item I can't talk to you right now–I'm working.
	\item He was working at his desk.
	\item Some firms expect the guards to work twelve hours a day.
	\end{itemize}
}
\item uncountable noun \\
Your \textbf{work} consists of the things you are paid or required to do in your job.
 \textit{
	\begin{itemize}
	\item We're supposed to be running a business here. I've got work to do.
	\item I used to take work home, but I don't do it any more.
	\item There have been days when I have finished work at 2pm.
	\item The film highlights the stressful and difficult aspects of the teacher's work.
	\end{itemize}
}
\item verb \\
When you \textbf{work} , you spend time and effort doing a task that needs to be done or trying to achieve
something.
 \textbf{Work} is also a noun.
 \textit{
	\begin{itemize}
	\item Linda spends all her time working on the garden.
	\item While I was working on my letter the telephone rang.
	\item Leonard was working at his German. His mistakes made her laugh.
	\item The most important reason for coming to university is to work for a degree.
	\item The government expressed hope that all the sides will work towards a political solution.
	\item There was a lot of work to do on their house.
	\item We hadn't appreciated how much work was involved in organizing a wedding.
	\item He said that the peace plan would be rejected because it needed more work.
	\end{itemize}
}
\item uncountable noun \\
\textbf{Work} is the place where you do your job.
 \textit{
	\begin{itemize}
	\item Many people travel to work by car.
	\item She told her friends at work that she was trying to lose weight.
	\end{itemize}
}
\item uncountable noun \\
\textbf{Work} is something which you produce as a result of an activity or as a result of doing
your job.
 \textit{
	\begin{itemize}
	\item It can help to have an impartial third party look over your work.
	\item Tidiness in the workshop is really essential for producing good work.
	\item That's a beautiful piece of work. You should be proud of it.
	\end{itemize}
}
\item countable noun \\
A \textbf{work} is something such as a painting, book, or piece of music produced by an artist , writer, or composer.
 \textit{
	\begin{itemize}
	\item In my opinion, this is Rembrandt's greatest work.
	\item Under his arm, there was a book which looked like the complete works of Shakespeare.
	\item The church has several valuable works of art.
	\end{itemize}
}
\item verb \\
If someone \textbf{is working on} a particular subject or question, they are studying or researching it.
 \textbf{Work} is also a noun.
 \textit{
	\begin{itemize}
	\item Professor Bonnet has been working for many years on molecules of this type.
	\item Our work shows that 10 per cent of families were behind on their rent or mortgage.
	\end{itemize}
}
\item verb \\
If you \textbf{work}  \textbf{with} a person or a group of people, you spend time and effort trying to help them in some
way.
 \textbf{Work} is also a noun.
 \textit{
	\begin{itemize}
	\item She spent a period of time working with people dying of cancer.
	\item He knew then that he wanted to work among the poor.
	\item ...a highly respected doctor who is noted for his work with the poor.
	\item She became involved in social and relief work among the refugees.
	\end{itemize}
}
\item verb \\
If a machine or piece of equipment \textbf{works} , it operates and performs a particular function.
 \textit{
	\begin{itemize}
	\item The pump doesn't work and we have no running water.
	\item Is the telephone working today?
	\item Ned turned on the lanterns, which worked with batteries.
	\end{itemize}
}
\item verb \\
If an idea, system, or way of doing something \textbf{works} , it is successful, effective , or satisfactory .
 \textit{
	\begin{itemize}
	\item 95 per cent of these diets do not work.
	\item If lust is all there is to hold you together, the relationship will never work.
	\item I shouldn't have come, I knew it wouldn't work.
	\item A methodical approach works best.
	\end{itemize}
}
\item verb \\
If a drug or medicine \textbf{works} , it produces a particular physical effect.
 \textit{
	\begin{itemize}
	\item I wake at 6am as the sleeping pill doesn't work for more than nine hours.
	\item The drug works by increasing levels of serotonin in the brain.
	\end{itemize}
}
\item verb \\
If something \textbf{works} in your favour, it helps you in some way. If something \textbf{works} to your disadvantage , it causes problems for you in some way.
 \textit{
	\begin{itemize}
	\item One factor thought to have worked in his favour is his working class image.
	\item This obviously works against the interests of the child.
	\end{itemize}
}
\item verb \\
If something or someone \textbf{works} their magic or \textbf{works} their charms  \textbf{on} a person, they have a powerful positive effect on them.
 \textit{
	\begin{itemize}
	\item As Foreign Secretary, he had to work his charm on leaders from Stalin to Truman.
	\item Our spirits rallied as the hot tea worked its magic.
	\end{itemize}
}
\item verb \\
If your mind or brain \textbf{is working} , you are thinking about something or trying to solve a problem.
 \textit{
	\begin{itemize}
	\item My mind was working frantically, running over the events of the evening.
	\end{itemize}
}
\item verb \\
If you \textbf{work on} an assumption or idea, you act as if it were true or base other ideas on it, until you have more
information.
 \textit{
	\begin{itemize}
	\item We are working on the assumption that it was a gas explosion.
	\end{itemize}
}
\item verb \\
If you \textbf{work} a particular area or type of place, you travel around that area or work in those
places as part of your job, for example trying to sell something there.
 \textit{
	\begin{itemize}
	\item Brand has been working the clubs and the pubs since 1986, developing her comedy act.
	\item This is the seventh year that he has worked the streets of Manhattan.
	\end{itemize}
}
\item verb \\
If you \textbf{work} someone, you make them spend time and effort doing a particular activity or job.
 \textit{
	\begin{itemize}
	\item They're working me too hard. I'm too old for this.
	\item They didn't take my father away, but kept him in the village and worked him to death.
	\end{itemize}
}
\item verb \\
If someone, often a politician or entertainer , \textbf{works} a crowd , they create a good relationship with the people in the crowd and get their support
or interest.
 \textit{
	\begin{itemize}
	\item The Prime Minister has an ability to work a crowd–some might even suggest it is a
kind of charm.
	\item He worked the room like a politician, gripping hands, and slapping backs.
	\end{itemize}
}
\item verb \\
When people \textbf{work} the land, they do all the tasks involved in growing crops.
 \textit{
	\begin{itemize}
	\item Farmers worked the fertile valleys.
	\end{itemize}
}
\item verb \\
When a mine \textbf{is worked} , minerals such as coal or gold are removed from it.
 \textit{
	\begin{itemize}
	\item The mines had first been worked in 1849, when gold was discovered in California.
	\item Only an agreed number of men was allowed to work any given seam at any given time.
	\end{itemize}
}
\item verb \\
If you \textbf{work} a machine or piece of equipment, you use or control it.
 \textit{
	\begin{itemize}
	\item Many adults still depend on their children to work the computer.
	\end{itemize}
}
\item verb \\
If something \textbf{works} into a particular state or condition, it gradually moves so that it is in that state
or condition.
 \textit{
	\begin{itemize}
	\item A screw had worked loose from my glasses.
	\end{itemize}
}
\item verb \\
If you \textbf{work} a substance such as dough or clay, you keep pressing it to make it have a particular texture .
 \textit{
	\begin{itemize}
	\item Work the dough with the palm of your hand until it is very smooth.
	\item Remove rind from the cheese and work it to a firm paste, with a fork.
	\end{itemize}
}
\item verb \\
If you \textbf{work} a material such as metal, leather, or stone, you cut, sew , or shape it in order to make something or to create a design.
 \textit{
	\begin{itemize}
	\item ...the machines needed to extract and work the raw stone.
	\item ...a long, cool tunnel of worked stone.
	\end{itemize}
}
\item verb \\
If you \textbf{work with} a particular substance or material, you use it in order to make something or to create
a design.
 \textit{
	\begin{itemize}
	\item He studied sculpture because he enjoyed working with clay.
	\end{itemize}
}
\item verb \\
If you \textbf{work} a part of your body, or if it \textbf{works} , you move it.
 \textit{
	\begin{itemize}
	\item Each position will work the muscles in a different way.
	\item Her mouth was working in her sleep.
	\end{itemize}
}
\item countable noun \\
A \textbf{works} is a place where something is manufactured or where an industrial process is carried
out. \textbf{Works} is used to refer to one or to more than one of these places.
 \textit{
	\begin{itemize}
	\item The steel works could be seen for miles.
	\item ...a recycling works.
	\item ...the works canteen.
	\end{itemize}
}
\item plural noun \\
\textbf{Works} are activities such as digging the ground or building on a large scale.
 \textit{
	\begin{itemize}
	\item ...six years of disruptive building works, road construction and urban development.
	\end{itemize}
}
\item singular noun \\
You can say  \textbf{the works} after listing things such as someone's possessions or requirements , to emphasize that they possess or require everything you can think of in a particular category .
 \textit{
	\begin{itemize}
	\item Amazing place he's got there–squash courts, swimming pool, jacuzzi, the works.
	\end{itemize}
}
\item  \\
 at work \textit{
	\begin{itemize}
	\end{itemize}
}
\item  \\
 at work \textit{
	\begin{itemize}
	\end{itemize}
}
\item  \\
 to have your work cut out \textit{
	\begin{itemize}
	\end{itemize}
}
\item  \\
 in the works \textit{
	\begin{itemize}
	\end{itemize}
}
\item  \\
 to make short/heavy/easy/quick work of sth \textit{
	\begin{itemize}
	\end{itemize}
}
\item  \\
 a nasty piece of work \textit{
	\begin{itemize}
	\end{itemize}
}
\item  \\
 to put/set sb to work \textit{
	\begin{itemize}
	\end{itemize}
}
\item  \\
 get/go to work \textit{
	\begin{itemize}
	\end{itemize}
}
\item  \\
 to work your way swh \textit{
	\begin{itemize}
	\end{itemize}
}
\item  \\
 nice/good work \textit{
	\begin{itemize}
	\end{itemize}
}
\end{enumerate}

\section*{vinegar}
{\large \color{blue}  vinegars  }
\subsection*{Explain}
\begin{enumerate}
\item variable noun \\
\textbf{Vinegar} is a sharp-tasting liquid, usually made from sour wine or malt , which is used to make things such as salad  dressing .
 \textit{
	\begin{itemize}
	\end{itemize}
}
\end{enumerate}

\section*{virtue}
{\large \color{blue}  virtues  }
\subsection*{Explain}
\begin{enumerate}
\item uncountable noun \\
\textbf{Virtue} is thinking and doing what is right and avoiding what is wrong .
 \textit{
	\begin{itemize}
	\item Virtue is not confined to the Christian world.
	\item She could have established her own innocence and virtue easily enough.
	\end{itemize}
}
\item countable noun \\
A \textbf{virtue} is a good quality or way of behaving .
 \textit{
	\begin{itemize}
	\item His virtue is patience.
	\item Her flaws were as large as her virtues.
	\item Humility is considered a virtue.
	\end{itemize}
}
\item countable noun \\
The \textbf{virtue} of something is an advantage or benefit that it has, especially in comparison with something else.
 \textit{
	\begin{itemize}
	\item There was no virtue in returning to Calvi the way I had come.
	\item It's other great virtue, of course, is its hard-wearing quality.
	\end{itemize}
}
\item phrase \\
You use \textbf{by virtue of} to explain why something happens or is true .
 \textit{
	\begin{itemize}
	\item The article stuck in my mind by virtue of one detail.
	\item Mr Olaechea has British residency by virtue of his marriage.
	\end{itemize}
}
\item  \\
 make a virtue of \textit{
	\begin{itemize}
	\end{itemize}
}
\end{enumerate}

\section*{afternoon}
{\large \color{blue}  afternoons  }
\subsection*{Explain}
\begin{enumerate}
\item variable noun \\
The \textbf{afternoon} is the part of each day which begins at lunchtime and ends at about six o'clock.
 \textit{
	\begin{itemize}
	\item He's arriving in the afternoon.
	\item He had stayed in his room all afternoon.
	\item ...an afternoon news conference.
	\end{itemize}
}
\end{enumerate}

\section*{advocate}
{\large \color{blue}  advocates  advocating  advocated  }
\subsection*{Explain}
\begin{enumerate}
\item verb \\
If you \textbf{advocate} a particular  action or plan , you recommend it publicly.
 \textit{
	\begin{itemize}
	\item Mr Williams is a conservative who advocates fewer government controls on business.
	\item ...the tax policy advocated by the Opposition.
	\end{itemize}
}
\item countable noun \\
An \textbf{advocate}  \textbf{of} a particular action or plan is someone who recommends it publicly.
 \textit{
	\begin{itemize}
	\item He was a strong advocate of free market policies and a multi-party system.
	\end{itemize}
}
\item countable noun \\
An \textbf{advocate} is a lawyer who speaks in favour of someone or defends them in a court of law.
 \textit{
	\begin{itemize}
	\end{itemize}
}
\item countable noun \\
An \textbf{advocate} for a particular group is a person who works for the interests of that group.
 \textit{
	\begin{itemize}
	\item ...advocates for the charity.
	\end{itemize}
}
\end{enumerate}

\section*{ambassador}
{\large \color{blue}  ambassadors  }
\subsection*{Explain}
\begin{enumerate}
\item countable noun \\
An \textbf{ambassador} is an important  official who lives in a foreign country and represents his or her own country's interests there.
 \textit{
	\begin{itemize}
	\item ...the German ambassador to Poland.
	\end{itemize}
}
\end{enumerate}

\section*{apparatus}
{\large \color{blue}  apparatuses  }
\subsection*{Explain}
\begin{enumerate}
\item variable noun \\
The \textbf{apparatus} of an organization or system is its structure and method of operation .
 \textit{
	\begin{itemize}
	\item For many years, the country had been buried under the apparatus of the regime.
	\item ...a massive bureaucratic apparatus.
	\end{itemize}
}
\item variable noun \\
\textbf{Apparatus} is the equipment, such as tools and machines, which is used to do a particular job or activity.
 \textit{
	\begin{itemize}
	\item One of the boys had to be rescued by firefighters wearing breathing apparatus.
	\end{itemize}
}
\end{enumerate}

\section*{array}
{\large \color{blue}  arrays  }
\subsection*{Explain}
\begin{enumerate}
\item countable noun \\
An \textbf{array}  \textbf{of} different things or people is a large number or wide  range of them.
 \textit{
	\begin{itemize}
	\item As the deadline approached she experienced a bewildering array of emotions.
	\item A dazzling array of celebrities are expected at the Mayfair gallery to see the pictures.
	\end{itemize}
}
\item countable noun \\
An \textbf{array}  \textbf{of} objects is a collection of them that is displayed or arranged in a particular way.
 \textit{
	\begin{itemize}
	\item There was an impressive array of pill bottles stacked on top of the fridge.
	\item We visited the local markets and saw wonderful arrays of fruit and vegetables.
	\end{itemize}
}
\item countable noun \\
An \textbf{array}  \textbf{of} instruments such as telescopes or solar panels is a number of them that are connected together to form a single unit.
 \textit{
	\begin{itemize}
	\end{itemize}
}
\item countable noun \\
In science and mathematics , an \textbf{array}  \textbf{of} things such as atoms or numbers is a regular pattern or structure that is formed by them.
 \textit{
	\begin{itemize}
	\item ...methods which can be used to create an ordered array of molecules within materials.
	\item The image is then stored on the computer hard disk as a vast array of black or white
dots.
	\end{itemize}
}
\end{enumerate}

\section*{badge}
{\large \color{blue}  badges  }
\subsection*{Explain}
\begin{enumerate}
\item countable noun \\
A \textbf{badge} is a piece of metal or cloth which you wear to show that you belong to an organization or support a cause. American English usually uses button to refer to a small round metal badge.
 \textit{
	\begin{itemize}
	\end{itemize}
}
\item singular noun \\
Any feature which is regarded as a sign of a particular quality can be referred to as a \textbf{badge} .
 \textit{
	\begin{itemize}
	\item Foreign companies used to consider a New York listing a badge of honour.
	\end{itemize}
}
\end{enumerate}

\section*{atmosphere}
{\large \color{blue}  atmospheres  }
\subsection*{Explain}
\begin{enumerate}
\item countable noun \\
A planet's \textbf{atmosphere} is the layer of air or other gases around it.
 \textit{
	\begin{itemize}
	\item ...dangerous levels of pollution in the Earth's atmosphere.
	\item Even worse, the levels of methane in the atmosphere are rising at more than 1 per
cent a year.
	\end{itemize}
}
\item countable noun \\
The \textbf{atmosphere} of a place is the air that you breathe there.
 \textit{
	\begin{itemize}
	\item These gases pollute the atmosphere of towns and cities.
	\end{itemize}
}
\item singular noun \\
The \textbf{atmosphere} of a place is the general impression that you get of it.
 \textit{
	\begin{itemize}
	\item Pale wooden floors and plenty of natural light add to the relaxed atmosphere.
	\item There's still an atmosphere of great hostility and tension in the city.
	\end{itemize}
}
\item uncountable noun \\
If a place or an event has \textbf{atmosphere} , it is interesting.
 \textit{
	\begin{itemize}
	\item The old harbour is still full of atmosphere and well worth visiting.
	\end{itemize}
}
\end{enumerate}

\section*{basketball}
{\large \color{blue}  basketballs  }
\subsection*{Explain}
\begin{enumerate}
\item uncountable noun \\
\textbf{Basketball} is a game in which two teams of five players each try to score goals by throwing a large ball through a circular  net  fixed to a metal ring at each end of the court.
 \textit{
	\begin{itemize}
	\end{itemize}
}
\item countable noun \\
A \textbf{basketball} is a large ball which is used in the game of basketball.
 \textit{
	\begin{itemize}
	\end{itemize}
}
\end{enumerate}

\section*{block}
{\large \color{blue}  blocks  blocking  blocked  }
\subsection*{Explain}
\begin{enumerate}
\item countable noun \\
A \textbf{block} of flats or offices is a large building containing them.
 \textit{
	\begin{itemize}
	\item ...blocks of council flats.
	\item ...a white-painted apartment block.
	\end{itemize}
}
\item countable noun \\
A \textbf{block} in a town is an area of land with streets on all its sides.
 \textit{
	\begin{itemize}
	\item She walked four blocks down High Street.
	\item He walked around the block three times.
	\end{itemize}
}
\item countable noun \\
A \textbf{block}  \textbf{of} a substance is a large rectangular piece of it.
 \textit{
	\begin{itemize}
	\item ...a block of ice.
	\end{itemize}
}
\item verb \\
To \textbf{block} a road, channel, or pipe means to put an object across it or in it so that nothing
can pass through it or along it.
 \textit{
	\begin{itemize}
	\item Some students today blocked a highway that cuts through the center of the city.
	\item When the shrimp farm is built it will block the stream.
	\item He can clear blocked drains.
	\end{itemize}
}
\item verb \\
If something \textbf{blocks} your view, it prevents you from seeing something because it is between you and that
thing.
 \textit{
	\begin{itemize}
	\item ...a row of spruce trees that blocked his view of the long north slope of the mountain.
	\end{itemize}
}
\item verb \\
If you \textbf{block} someone's way, you prevent them from going  somewhere or entering a place by standing in front of them.
 \textit{
	\begin{itemize}
	\item I started to move round him, but he blocked my way.
	\item Mr Calder tried to leave the shop but the police officer blocked his path.
	\end{itemize}
}
\item verb \\
If you \textbf{block} something that is being arranged, you prevent it from being done.
 \textit{
	\begin{itemize}
	\item For years the country has tried to block imports of various cheap foreign products.
	\item His persistent attempts to complain to his superiors were blocked and ignored.
	\end{itemize}
}
\item verb \\
In some sports, if a player \textbf{blocks} a shot or kick , they stop the ball reaching its target . If one player \textbf{blocks} another, the first stops the second from reaching or moving with the ball.
 \textit{
	\begin{itemize}
	\item The City goalkeeper had superbly blocked a shot.
	\end{itemize}
}
\item countable noun \\
A \textbf{block}  \textbf{of} something such as tickets or shares is a large quantity of them, especially when they are all sold at the same time and are in a particular sequence or order.
 \textit{
	\begin{itemize}
	\item Those booking a block of seats get them at reduced rates.
	\end{itemize}
}
\item countable noun \\
If you have a \textbf{mental block} or a \textbf{block} , you are temporarily unable to do something that you can normally do which involves using, thinking about, or remembering something.
 \textit{
	\begin{itemize}
	\end{itemize}
}
\item  \\
 lay one's head is on the block/put one's head on the block \textit{
	\begin{itemize}
	\end{itemize}
}
\end{enumerate}

\section*{bolt}
{\large \color{blue}  bolts  bolting  bolted  }
\subsection*{Explain}
\begin{enumerate}
\item countable noun \\
A \textbf{bolt} is a long metal object which screws into a nut and is used to fasten things together.
 \textit{
	\begin{itemize}
	\end{itemize}
}
\item verb \\
When you \textbf{bolt} one thing to another, you fasten them firmly together, using a bolt.
 \textit{
	\begin{itemize}
	\item The safety belt is easy to fit as there's no need to bolt it to seat belt anchorage
points.
	\item Bolt the components together.
	\item The doors were bolted on.
	\item ...a wooden bench which was bolted to the floor.
	\end{itemize}
}
\item countable noun \\
A \textbf{bolt} on a door or window is a metal bar that you can slide across in order to fasten the door or window.
 \textit{
	\begin{itemize}
	\item I heard the sound of a bolt being slowly and reluctantly slid open.
	\end{itemize}
}
\item verb \\
When you \textbf{bolt} a door or window, you slide the bolt across to fasten it.
 \textit{
	\begin{itemize}
	\item He reminded her that he would have to lock and bolt the kitchen door after her.
	\item ...the heavy bolted doors .
	\end{itemize}
}
\item verb \\
If a person or animal \textbf{bolts} , they suddenly start to run very fast , often because something has frightened them.
 \textit{
	\begin{itemize}
	\item The horse bolted when a gun went off.
	\item I made some excuse and bolted for the exit.
	\end{itemize}
}
\item verb \\
If you \textbf{bolt} your food, you eat it so quickly that you hardly  chew it or taste it.
 \textbf{Bolt down} means the same as bolt .
 \textit{
	\begin{itemize}
	\item Being under stress can cause you to miss meals, eat on the move, or bolt your food.
	\item Back then I could bolt down three or four burgers and a pile of French fries.
	\end{itemize}
}
\item countable noun \\
A \textbf{bolt of} lightning is a flash of lightning that is seen as a white line in the sky .
 \textit{
	\begin{itemize}
	\item Suddenly a bolt of lightning crackled through the sky.
	\end{itemize}
}
\item countable noun \\
A \textbf{bolt}  \textbf{of} cloth is a long wide piece of it that is wound into a roll round a piece of cardboard .
 \textit{
	\begin{itemize}
	\item ...bolts of black silk.
	\end{itemize}
}
\item verb \\
When vegetables such as lettuces or onions  \textbf{bolt} , they grow too quickly and produce flowers and seeds, and therefore become less good
to eat.
 \textit{
	\begin{itemize}
	\item If the soil dries out the plants may bolt.
	\end{itemize}
}
\item  \\
 make a bolt for swh/make a bolt for it \textit{
	\begin{itemize}
	\end{itemize}
}
\item  \\
 bolt from the blue \textit{
	\begin{itemize}
	\end{itemize}
}
\item  \\
 bolt upright \textit{
	\begin{itemize}
	\end{itemize}
}
\end{enumerate}

\section*{brochure}
{\large \color{blue}  brochures  }
\subsection*{Explain}
\begin{enumerate}
\item countable noun \\
A \textbf{brochure} is a magazine or thin book with pictures that gives you information about a product or service.
 \textit{
	\begin{itemize}
	\item ...travel brochures.
	\end{itemize}
}
\end{enumerate}

\section*{cancer}
{\large \color{blue}  cancers  }
\subsection*{Explain}
\begin{enumerate}
\item variable noun \\
\textbf{Cancer} is a serious  disease in which cells in a person's body increase rapidly in an uncontrolled way, producing abnormal growths.
 \textit{
	\begin{itemize}
	\item Her mother died of breast cancer.
	\item Jane was just 25 when she learned she had cancer.
	\item Ninety per cent of lung cancers are caused by smoking.
	\end{itemize}
}
\end{enumerate}

\section*{cannon}
{\large \color{blue}  cannons  cannoning  cannoned  }
\subsection*{Explain}
\begin{enumerate}
\item countable noun \\
A \textbf{cannon} is a large gun, usually on wheels , which used to be used in battles .
 \textit{
	\begin{itemize}
	\end{itemize}
}
\item countable noun \\
A \textbf{cannon} is a heavy automatic gun, especially one that is fired from an aircraft.
 \textit{
	\begin{itemize}
	\end{itemize}
}
\item verb \\
If one person or thing \textbf{cannons} into another, they bump into them with great force.
 \textit{
	\begin{itemize}
	\item One of the reporters cannoned into Arnold.
	\item The ball cannoned off the back of a Spartak defender and into the net.
	\end{itemize}
}
\item  \\
 a loose cannon \textit{
	\begin{itemize}
	\end{itemize}
}
\end{enumerate}

\section*{candy}
{\large \color{blue}  candies  }
\subsection*{Explain}
\begin{enumerate}
\item variable noun \\
\textbf{Candy} is sweet foods such as toffees or chocolate.
 \textit{
	\begin{itemize}
	\item ...a piece of candy.
	\item ...a large box of candies.
	\end{itemize}
}
\end{enumerate}

\section*{care}
{\large \color{blue}  cares  caring  cared  }
\subsection*{Explain}
\begin{enumerate}
\item verb \\
If you \textbf{care}  \textbf{about} something, you feel that it is important and are concerned about it.
 \textit{
	\begin{itemize}
	\item ...a company that cares about the environment.
	\item ...young men who did not care whether they lived or died.
	\item Does anybody know we're here, does anybody care?
	\end{itemize}
}
\item verb \\
If you \textbf{care}  \textbf{for} someone, you feel a lot of affection for them.
 \textit{
	\begin{itemize}
	\item He wanted me to know that he still cared for me.
	\item ...people who are your friends, who care about you.
	\end{itemize}
}
\item verb \\
If you \textbf{care}  \textbf{for} someone or something, you look after them and keep them in a good state or condition.
 \textbf{Care} is also a noun .
 \textit{
	\begin{itemize}
	\item They hired a nurse to care for her.
	\item ...these distinctive cars, lovingly cared for by private owners.
	\item ...well-cared-for homes.
	\item Most of the staff specialise in the care of children.
	\item ...sensitive teeth which need special care.
	\item She denied the murder of four children who were in her care.
	\end{itemize}
}
\item uncountable noun \\
Children who are \textbf{in}  \textbf{care} are looked after by the state because their parents are dead or unable to look after them properly.
 \textit{
	\begin{itemize}
	\item ...a home for children in care.
	\item She was taken into care as a baby.
	\end{itemize}
}
\item verb \\
If you say that you do not \textbf{care}  \textbf{for} something or someone, you mean that you do not like them.
 \textit{
	\begin{itemize}
	\item She had met both sons and did not care for either.
	\end{itemize}
}
\item verb \\
If you say that someone does something when they \textbf{care}  \textbf{to} do it, you mean that they do it, although they should do it more willingly or more
often.
 \textit{
	\begin{itemize}
	\item The woman tells anyone who cares to listen that she's going through hell.
	\item Experts reveal only as much as they care to.
	\end{itemize}
}
\item verb \\
You can ask someone if they would \textbf{care}  \textbf{for} something or if they would \textbf{care}  \textbf{to} do something as a polite way of asking if they would like to have or do something.
 \textit{
	\begin{itemize}
	\item Would you care for some orange juice?
	\item He said he was off to the beach and would we care to join him.
	\end{itemize}
}
\item uncountable noun \\
If you do something \textbf{with}  \textbf{care} , you give careful attention to it because you do not want to make any mistakes or cause any damage .
 \textit{
	\begin{itemize}
	\item Condoms are an effective method of birth control if used with care.
	\item We'd taken enormous care in choosing the location.
	\end{itemize}
}
\item countable noun \\
Your \textbf{cares} are your worries, anxieties, or fears .
 \textit{
	\begin{itemize}
	\item Lean back in a hot bath and forget all the cares of the day.
	\item Johnson seemed without a care in the world.
	\end{itemize}
}
\item  \\
 for all sb cares \textit{
	\begin{itemize}
	\end{itemize}
}
\item  \\
 couldn't care less \textit{
	\begin{itemize}
	\end{itemize}
}
\item  \\
 care of sb, in care of sb \textit{
	\begin{itemize}
	\end{itemize}
}
\item  \\
 take care of sth/sb \textit{
	\begin{itemize}
	\end{itemize}
}
\item  \\
 take care \textit{
	\begin{itemize}
	\end{itemize}
}
\item  \\
 take care to do sth \textit{
	\begin{itemize}
	\end{itemize}
}
\item  \\
 take care of sth \textit{
	\begin{itemize}
	\end{itemize}
}
\item  \\
 who cares \textit{
	\begin{itemize}
	\end{itemize}
}
\end{enumerate}

\section*{crab}
{\large \color{blue}  crabs  }
\subsection*{Explain}
\begin{enumerate}
\item countable noun \\
A \textbf{crab} is a sea creature with a flat round body covered by a shell , and five pairs of legs with large claws on the front pair. Crabs usually move sideways.
 \textbf{Crab} is the flesh of this creature eaten as food.
 \textit{
	\begin{itemize}
	\end{itemize}
}
\end{enumerate}

\section*{catastrophe}
{\large \color{blue}  catastrophes  }
\subsection*{Explain}
\begin{enumerate}
\item countable noun \\
A \textbf{catastrophe} is an unexpected event that causes great suffering or damage .
 \textit{
	\begin{itemize}
	\item From all points of view, war would be a catastrophe.
	\item ...the economic and environmental catastrophe that the oil leak has caused.
	\end{itemize}
}
\end{enumerate}

\section*{dusk}
{\large \color{blue}  }
\subsection*{Explain}
\begin{enumerate}
\item uncountable noun \\
\textbf{Dusk} is the time just before night when the daylight has almost  gone but when it is not completely dark.
 \textit{
	\begin{itemize}
	\item We arrived home at dusk.
	\end{itemize}
}
\item uncountable noun \\
\textbf{The}  \textbf{dusk} is the dull , shadowy light there is at dusk.
 \textit{
	\begin{itemize}
	\item She turned and disappeared into the dusk.
	\end{itemize}
}
\end{enumerate}

\section*{cathedral}
{\large \color{blue}  cathedrals  }
\subsection*{Explain}
\begin{enumerate}
\item countable noun \\
A \textbf{cathedral} is a very large and important church which has a bishop in charge of it.
 \textit{
	\begin{itemize}
	\item ...St. Paul's Cathedral.
	\item ...the cathedral city of Canterbury.
	\end{itemize}
}
\end{enumerate}

\section*{fence}
{\large \color{blue}  fences  fencing  fenced  }
\subsection*{Explain}
\begin{enumerate}
\item countable noun \\
A \textbf{fence} is a barrier between two areas of land, made of wood or wire supported by posts.
 \textit{
	\begin{itemize}
	\item Villagers say the fence would restrict public access to the hills.
	\end{itemize}
}
\item verb \\
If you \textbf{fence} an area of land, you surround it with a fence.
 \textit{
	\begin{itemize}
	\item The first task was to fence the wood to exclude sheep.
	\item Thomas was playing in a little fenced area full of sand.
	\end{itemize}
}
\item countable noun \\
A \textbf{fence} in show jumping or horse racing is an obstacle or barrier that horses have to jump over.
 \textit{
	\begin{itemize}
	\end{itemize}
}
\item countable noun \\
A \textbf{fence} is a person who receives stolen property and then sells it.
 \textit{
	\begin{itemize}
	\item He originally acted as a fence for another gang before turning to burglary himself.
	\end{itemize}
}
\item  \\
 to mend fences \textit{
	\begin{itemize}
	\end{itemize}
}
\item  \\
 to sit on the fence \textit{
	\begin{itemize}
	\end{itemize}
}
\end{enumerate}

\section*{chin}
{\large \color{blue}  chins  }
\subsection*{Explain}
\begin{enumerate}
\item countable noun \\
Your \textbf{chin} is the part of your face that is below your mouth and above your neck .
 \textit{
	\begin{itemize}
	\item ...a double chin.
	\item He rubbed the gray stubble on his chin.
	\end{itemize}
}
\item  \\
 take sthing on the chin \textit{
	\begin{itemize}
	\end{itemize}
}
\end{enumerate}

\section*{fireplace}
{\large \color{blue}  fireplaces  }
\subsection*{Explain}
\begin{enumerate}
\item countable noun \\
In a room, the \textbf{fireplace} is the place where a fire can be lit and the area on the wall and floor  surrounding this place.
 \textit{
	\begin{itemize}
	\end{itemize}
}
\end{enumerate}

\section*{coat}
{\large \color{blue}  coats  coating  coated  }
\subsection*{Explain}
\begin{enumerate}
\item countable noun \\
A \textbf{coat} is a piece of clothing with long sleeves which you wear over your other clothes when you go outside.
 \textit{
	\begin{itemize}
	\item He turned off the television, put on his coat and walked out.
	\end{itemize}
}
\item countable noun \\
An animal's \textbf{coat} is the fur or hair on its body.
 \textit{
	\begin{itemize}
	\item Vitamin B6 is great for improving the condition of dogs' and horses' coats.
	\end{itemize}
}
\item verb \\
If you \textbf{coat} something \textbf{with} a substance or \textbf{in} a substance, you cover it with a thin layer of the substance.
 \textit{
	\begin{itemize}
	\item Coat the fish with seasoned flour.
	\end{itemize}
}
\item countable noun \\
A \textbf{coat}  \textbf{of}  paint or varnish is a thin layer of it on a surface.
 \textit{
	\begin{itemize}
	\item The front door needs a new coat of paint.
	\item You will need to apply three coats of varnish.
	\end{itemize}
}
\end{enumerate}

\section*{friction}
{\large \color{blue}  frictions  }
\subsection*{Explain}
\begin{enumerate}
\item uncountable noun \\
If there is \textbf{friction} between people, there is disagreement and argument between them.
 \textit{
	\begin{itemize}
	\item Sara sensed that there had been friction between her children.
	\item The plan is likely only to aggravate ethnic frictions.
	\end{itemize}
}
\item uncountable noun \\
\textbf{Friction} is the force that makes it difficult for things to move freely when they are touching each other.
 \textit{
	\begin{itemize}
	\item The pistons are graphite-coated to reduce friction.
	\end{itemize}
}
\item uncountable noun \\
\textbf{Friction} is the rubbing of one object against another.
 \textit{
	\begin{itemize}
	\item ...the friction of his leg against hers.
	\end{itemize}
}
\end{enumerate}

\section*{continent}
{\large \color{blue}  continents  }
\subsection*{Explain}
\begin{enumerate}
\item countable noun \\
A \textbf{continent} is a very large area of land, such as Africa or Asia, that consists of several countries.
 \textit{
	\begin{itemize}
	\item She loved the African continent.
	\item Dinosaurs evolved when most continents were joined in a single land mass.
	\end{itemize}
}
\item proper noun \\
People sometimes use \textbf{the Continent} to refer to the continent of Europe except for Britain .
 \textit{
	\begin{itemize}
	\item Its shops are among the most stylish on the Continent.
	\end{itemize}
}
\end{enumerate}

\section*{frost}
{\large \color{blue}  frosts  }
\subsection*{Explain}
\begin{enumerate}
\item variable noun \\
When there is \textbf{frost} or a \textbf{frost} , the temperature outside falls below freezing point and the ground becomes covered
in ice crystals .
 \textit{
	\begin{itemize}
	\item There is frost on the ground and snow is forecast.
	\item The wind had veered to north, bringing clear skies and a keen frost.
	\end{itemize}
}
\item  \\
 degrees of frost \textit{
	\begin{itemize}
	\end{itemize}
}
\end{enumerate}

\section*{convention}
{\large \color{blue}  conventions  }
\subsection*{Explain}
\begin{enumerate}
\item variable noun \\
A \textbf{convention} is a way of behaving that is considered to be correct or polite by most people in a society .
 \textit{
	\begin{itemize}
	\item It's just a social convention that men don't wear skirts.
	\item Despite her wish to defy convention, she had become pregnant and married at 21.
	\end{itemize}
}
\item countable noun \\
In art , literature , or the theatre , a \textbf{convention} is a traditional  method or style .
 \textit{
	\begin{itemize}
	\item We go offstage and come back for the convention of the encore.
	\item ...the stylistic conventions of Egyptian art.
	\end{itemize}
}
\item countable noun \\
A \textbf{convention} is an official agreement between countries or groups of people.
 \textit{
	\begin{itemize}
	\item ...the U.N. convention on climate change.
	\item ...the Geneva convention.
	\end{itemize}
}
\item countable noun \\
A \textbf{convention} is a large meeting of an organization or political group.
 \textit{
	\begin{itemize}
	\item ...the annual convention of the Society of Professional Journalists.
	\item ...the Republican convention.
	\end{itemize}
}
\end{enumerate}

\section*{fuel}
{\large \color{blue}  fuels  fuelling  fuelled  }
\subsection*{Explain}
\begin{enumerate}
\item variable noun \\
\textbf{Fuel} is a substance such as coal, oil , or petrol that is burned to provide heat or power.
 \textit{
	\begin{itemize}
	\item They ran out of fuel.
	\item ...industrial research into cleaner fuels.
	\end{itemize}
}
\item verb \\
To \textbf{fuel} a situation  means to make it become worse or more intense .
 \textit{
	\begin{itemize}
	\item The result will inevitably fuel speculation about the Prime Minister's future.
	\item The economic boom was fueled by easy credit.
	\end{itemize}
}
\item  \\
 add fuel to something \textit{
	\begin{itemize}
	\end{itemize}
}
\end{enumerate}

\section*{descent}
{\large \color{blue}  descents  }
\subsection*{Explain}
\begin{enumerate}
\item variable noun \\
A \textbf{descent} is a movement from a higher to a lower level or position.
 \textit{
	\begin{itemize}
	\item During their descent from the summit, a storm swept in.
	\item ...the crash of an Airbus A300 on its descent into Kathmandu airport.
	\end{itemize}
}
\item countable noun \\
A \textbf{descent} is a surface that slopes downwards, for example the side of a steep  hill .
 \textit{
	\begin{itemize}
	\item On the descents, cyclists spin past cars, freewheeling downhill at tremendous speed.
	\end{itemize}
}
\item singular noun \\
When you want to emphasize that a situation becomes very bad , you can talk about someone's or something's \textbf{descent} into that situation.
 \textit{
	\begin{itemize}
	\item Without a political settlement, the descent into chaos will be guaranteed.
	\item ...his swift descent from respected academic to homeless alcoholic.
	\end{itemize}
}
\item uncountable noun \\
You use \textbf{descent} to talk about a person's family background , for example their nationality or social  status .
 \textit{
	\begin{itemize}
	\item All the contributors were of African descent.
	\end{itemize}
}
\end{enumerate}

\section*{hedge}
{\large \color{blue}  hedges  hedging  hedged  }
\subsection*{Explain}
\begin{enumerate}
\item countable noun \\
A \textbf{hedge} is a row of bushes or small trees, usually along the edge of a garden, field, or road.
 \textit{
	\begin{itemize}
	\end{itemize}
}
\item verb \\
If you \textbf{hedge}  \textbf{against} something unpleasant or unwanted that might affect you, especially  losing money, you do something which will protect you from it.
 \textit{
	\begin{itemize}
	\item You can hedge against redundancy or illness with insurance.
	\item Today's clever financial instruments make it possible for firms to hedge their risks.
	\end{itemize}
}
\item countable noun \\
Something that is a \textbf{hedge against} something unpleasant will protect you from its effects.
 \textit{
	\begin{itemize}
	\item Gold is traditionally a hedge against inflation.
	\end{itemize}
}
\item verb \\
If you \textbf{hedge} , you avoid  answering a question or committing yourself to a particular action or decision.
 \textit{
	\begin{itemize}
	\item They hedged in answering various questions about the operation.
	\item 'I can't give you an answer now,' he hedged.
	\end{itemize}
}
\item  \\
 hedge one's bets \textit{
	\begin{itemize}
	\end{itemize}
}
\end{enumerate}

\section*{embassy}
{\large \color{blue}  embassies  }
\subsection*{Explain}
\begin{enumerate}
\item countable noun \\
An \textbf{embassy} is a group of government officials, headed by an ambassador, who represent their government in a foreign country. The building in which they work is also  called an \textbf{embassy} .
 \textit{
	\begin{itemize}
	\item The American Embassy has already complained.
	\item Mr Cohen held discussions at the embassy with one of the rebel leaders.
	\end{itemize}
}
\end{enumerate}

\section*{invitation}
{\large \color{blue}  invitations  }
\subsection*{Explain}
\begin{enumerate}
\item countable noun \\
An \textbf{invitation} is a written or spoken  request to come to an event such as a party , a meal , or a meeting .
 \textit{
	\begin{itemize}
	\item ...an invitation to lunch.
	\item The Syrians have not yet accepted an invitation to attend.
	\item He's understood to be there at the personal invitation of the President.
	\end{itemize}
}
\item countable noun \\
An \textbf{invitation} is the card or paper on which an invitation is written or printed .
 \textit{
	\begin{itemize}
	\item Hundreds of invitations are being sent out this week.
	\item ...gold embossed invitation cards.
	\end{itemize}
}
\item singular noun \\
If you believe that someone's action is likely to have a particular result, especially a bad one, you can refer to the action as an \textbf{invitation to} that result.
 \textit{
	\begin{itemize}
	\item ...a war that most liberal Democrats regarded as an invitation to disaster.
	\item Don't leave your shopping on the back seat of your car–it's an open invitation to
a thief.
	\end{itemize}
}
\end{enumerate}

\section*{jazz}
{\large \color{blue}  jazzes  jazzing  jazzed  }
\subsection*{Explain}
\begin{enumerate}
\item uncountable noun \\
\textbf{Jazz} is a style of music that was invented by African American musicians in the early part of the twentieth century. Jazz music has very strong rhythms and often involves improvisation.
 \textit{
	\begin{itemize}
	\item The pub has live jazz on Sundays.
	\end{itemize}
}
\end{enumerate}

\section*{god}
{\large \color{blue}  gods  }
\subsection*{Explain}
\begin{enumerate}
\item proper noun \\
The name \textbf{God} is given to the spirit or being who is worshipped as the creator and ruler of the world, especially by Jews , Christians , and Muslims .
 \textit{
	\begin{itemize}
	\item He believes in God.
	\item God bless you.
	\end{itemize}
}
\item convention \\
People sometimes use \textbf{God} in exclamations to emphasize something that they are saying , or to express surprise, fear , or excitement . This use could cause offence .
 \textit{
	\begin{itemize}
	\item God, how I hated him!
	\item Oh my God he's shot somebody.
	\item Good God, it's Mr Harper!
	\item God Almighty, Hart, you scared me silly.
	\end{itemize}
}
\item countable noun \\
In many religions, a \textbf{god} is one of the spirits or beings that are believed to have power over a particular part of the world or nature .
 \textit{
	\begin{itemize}
	\item ...Pan, the God of nature.
	\item ...Zeus, king of the gods.
	\end{itemize}
}
\item countable noun \\
Someone who is admired very much by a person or group of people, and who influences them a lot , can be referred to as a \textbf{god} .
 \textit{
	\begin{itemize}
	\item To his followers he was a god.
	\end{itemize}
}
\item  \\
 God forbid \textit{
	\begin{itemize}
	\end{itemize}
}
\item  \\
 God's gift \textit{
	\begin{itemize}
	\end{itemize}
}
\item  \\
 God help \textit{
	\begin{itemize}
	\end{itemize}
}
\item  \\
 God help \textit{
	\begin{itemize}
	\end{itemize}
}
\item  \\
 God help us \textit{
	\begin{itemize}
	\end{itemize}
}
\item  \\
 God knows/God only knows/God alone knows \textit{
	\begin{itemize}
	\end{itemize}
}
\item  \\
 God knows \textit{
	\begin{itemize}
	\end{itemize}
}
\item  \\
 a man of God \textit{
	\begin{itemize}
	\end{itemize}
}
\item  \\
 in God's name \textit{
	\begin{itemize}
	\end{itemize}
}
\item  \\
 play God \textit{
	\begin{itemize}
	\end{itemize}
}
\item  \\
 please God \textit{
	\begin{itemize}
	\end{itemize}
}
\item  \\
 to God \textit{
	\begin{itemize}
	\end{itemize}
}
\item  \\
 God willing \textit{
	\begin{itemize}
	\end{itemize}
}
\end{enumerate}

\section*{laser}
{\large \color{blue}  lasers  lasering  lasered  }
\subsection*{Explain}
\begin{enumerate}
\item countable noun \\
A \textbf{laser} is a narrow beam of concentrated light produced by a special  machine . It is used for cutting very hard materials, and in many technical  fields such as surgery and telecommunications .
 \textit{
	\begin{itemize}
	\item ...new laser technology.
	\item Researchers realised that a tunable laser beam might be useful in surgery.
	\end{itemize}
}
\item countable noun \\
A \textbf{laser} is a machine that produces a laser beam.
 \textit{
	\begin{itemize}
	\item ...the first-ever laser, built in 1960.
	\end{itemize}
}
\item verb \\
If you have part of your body \textbf{lasered} , you have it treated using a laser.
 \textit{
	\begin{itemize}
	\item I had my eyes lasered. I was shortsighted and now I'm not.
	\item British men are shaving, waxing and lasering their body hair like never before.
	\end{itemize}
}
\end{enumerate}

\section*{graduate}
{\large \color{blue}  graduates  graduating  graduated  }
\subsection*{Explain}
\begin{enumerate}
\item countable noun \\
In Britain , a \textbf{graduate} is a person who has successfully completed a degree at a university or college and
has received a certificate that shows this.
 \textit{
	\begin{itemize}
	\item They are looking for graduates with humanities or business degrees.
	\item ...graduates in engineering.
	\end{itemize}
}
\item countable noun \\
In the United  States , a \textbf{graduate} is a student who has successfully completed a course at a high school, college, or
university.
 \textit{
	\begin{itemize}
	\item The top one-third of all high school graduates are entitled to an education at the
California State University.
	\end{itemize}
}
\item verb \\
In Britain, when a student \textbf{graduates} from university, they have successfully completed a degree course.
 \textit{
	\begin{itemize}
	\item She graduated in English and Drama from Manchester University.
	\end{itemize}
}
\item verb \\
In the United States, when a student \textbf{graduates} , they complete their studies successfully and leave their school or university. You can  also  say that a school or university \textbf{graduates} a student or students.
 \textit{
	\begin{itemize}
	\item When the boys graduated from high school, Ann moved to a small town in Vermont.
	\item In 1986, American universities graduated a record number of students with degrees
in computer science.
	\end{itemize}
}
\item verb \\
If you \textbf{graduate}  \textbf{from} one thing \textbf{to} another, you go from a less important  job or position to a more important one.
 \textit{
	\begin{itemize}
	\item Bruce graduated to chef at the Bear Hotel.
	\item From commercials she quickly graduated to television shows.
	\end{itemize}
}
\end{enumerate}

\section*{headline}
{\large \color{blue}  headlines  headlining  headlined  }
\subsection*{Explain}
\begin{enumerate}
\item countable noun \\
A \textbf{headline} is the title of a newspaper story, printed in large letters at the top of the story, especially on the front page.
 \textit{
	\begin{itemize}
	\item The Daily Mail has the headline 'The Voice of Conscience'.
	\end{itemize}
}
\item plural noun \\
\textbf{The}  \textbf{headlines} are the main points of the news which are read on radio or television.
 \textit{
	\begin{itemize}
	\item I'm Claudia Polley with the news headlines.
	\end{itemize}
}
\item verb \\
If a newspaper or magazine article \textbf{is headlined} a particular thing, that is the headline that introduces it.
 \textit{
	\begin{itemize}
	\item The article was headlined 'Tell us the truth'.
	\end{itemize}
}
\item verb \\
If someone \textbf{headlines} a show , they are the main performer in it.
 \textit{
	\begin{itemize}
	\item The band are headlining the festival's Saturday programme.
	\end{itemize}
}
\item  \\
 to hit the headlines \textit{
	\begin{itemize}
	\end{itemize}
}
\end{enumerate}

\section*{hill}
{\large \color{blue}  hills  }
\subsection*{Explain}
\begin{enumerate}
\item countable noun \\
A \textbf{hill} is an area of land that is higher than the land that surrounds it.
 \textit{
	\begin{itemize}
	\item We trudged up the hill to the stadium.
	\item ...Maple Hill.
	\item ...the Black Hills of Dakota.
	\end{itemize}
}
\item  \\
 over the hill \textit{
	\begin{itemize}
	\end{itemize}
}
\end{enumerate}

\section*{mill}
{\large \color{blue}  mills  milling  milled  }
\subsection*{Explain}
\begin{enumerate}
\item countable noun \\
A \textbf{mill} is a building in which grain is crushed to make flour.
 \textit{
	\begin{itemize}
	\end{itemize}
}
\item countable noun \\
A \textbf{mill} is a small device used for grinding something such as coffee beans or pepper into powder .
 \textit{
	\begin{itemize}
	\item ...a pepper mill.
	\end{itemize}
}
\item countable noun \\
A \textbf{mill} is a factory used for making and processing materials such as steel , wool , or cotton .
 \textit{
	\begin{itemize}
	\item ...a steel mill.
	\item ...a textile mill.
	\end{itemize}
}
\item verb \\
To \textbf{mill} something such as wheat or pepper means to grind it in a mill.
 \textit{
	\begin{itemize}
	\item They mill 1000 tonnes of flour a day in every Australian state.
	\item ...freshly milled black pepper.
	\end{itemize}
}
\end{enumerate}

\section*{hour}
{\large \color{blue}  hours  }
\subsection*{Explain}
\begin{enumerate}
\item countable noun \\
An \textbf{hour} is a period of sixty minutes.
 \textit{
	\begin{itemize}
	\item They waited for about two hours.
	\item I only slept about half an hour that night.
	\item ...a twenty-four hour strike.
	\item London was an hour away and by the time I arrived the operation had already been
performed.
	\end{itemize}
}
\item plural noun \\
People say that something takes or lasts  \textbf{hours} to emphasize that it takes or lasts a very long time, or what seems like a very long time.
 \textit{
	\begin{itemize}
	\item Getting there would take hours.
	\end{itemize}
}
\item singular noun \\
A clock that strikes  \textbf{the hour} strikes when it is exactly one o'clock, two o'clock, and so on.
 \textit{
	\begin{itemize}
	\end{itemize}
}
\item singular noun \\
You can refer to a particular time or moment as a particular \textbf{hour} .
 \textit{
	\begin{itemize}
	\item ...the hour of his execution.
	\item The gathering storm had made the day even darker than was usual at this hour.
	\end{itemize}
}
\item countable noun \\
If you refer, for example , to someone's \textbf{hour}  \textbf{of}  need or \textbf{hour}  \textbf{of} happiness, you are referring to the time in their life when they are or were experiencing that condition or feeling .
 \textit{
	\begin{itemize}
	\item He was one of the first people to stand by me in my hour of need.
	\item ...the darkest hour of my professional life.
	\end{itemize}
}
\item plural noun \\
You can refer to the period of time during which something happens or operates each day as the \textbf{hours} during which it happens or operates.
 \textit{
	\begin{itemize}
	\item ...the hours of darkness.
	\item Phone us on this number during office hours.
	\item ...outside prison visiting hours.
	\end{itemize}
}
\item plural noun \\
If you refer to the \textbf{hours} involved in a job , you are talking about how long you spend each week doing it and when you do it.
 \textit{
	\begin{itemize}
	\item I worked quite irregular hours.
	\item The job was easy; the hours were good.
	\end{itemize}
}
\item  \\
 after hours \textit{
	\begin{itemize}
	\end{itemize}
}
\item  \\
 at all hours \textit{
	\begin{itemize}
	\end{itemize}
}
\item  \\
 the small hours \textit{
	\begin{itemize}
	\end{itemize}
}
\item  \\
 hour after hour \textit{
	\begin{itemize}
	\end{itemize}
}
\item  \\
 on the hour \textit{
	\begin{itemize}
	\end{itemize}
}
\item  \\
 past the hour/before the hour \textit{
	\begin{itemize}
	\end{itemize}
}
\item  \\
 out of hours \textit{
	\begin{itemize}
	\end{itemize}
}
\end{enumerate}

\section*{mirror}
{\large \color{blue}  mirrors  mirroring  mirrored  }
\subsection*{Explain}
\begin{enumerate}
\item countable noun \\
A \textbf{mirror} is a flat piece of glass which reflects light, so that when you look at it you can see yourself reflected in it.
 \textit{
	\begin{itemize}
	\item He absent-mindedly looked at himself in the mirror.
	\item He checked his mirror and saw that a dark coloured van was immediately behind him.
	\end{itemize}
}
\item verb \\
If something \textbf{mirrors} something else, it has similar features to it, and therefore seems  like a copy or representation of it.
 \textit{
	\begin{itemize}
	\item The book inevitably mirrors my own interests and experiences.
	\item His own shock was mirrored on her face.
	\end{itemize}
}
\item verb \\
If you see something reflected in water, you can say that the water \textbf{mirrors} it.
 \textit{
	\begin{itemize}
	\item ...the sudden glitter where a newly-flooded field mirrors the sky.
	\item The ship would lie there mirrored in a perfectly unmoving glossy sea.
	\end{itemize}
}
\end{enumerate}

\section*{husband}
{\large \color{blue}  husbands  husbanding  husbanded  }
\subsection*{Explain}
\begin{enumerate}
\item countable noun \\
Someone's \textbf{husband} is the man they are married to.
 \textit{
	\begin{itemize}
	\item Eva married her husband Jack in 1957.
	\item Are they husband and wife?
	\end{itemize}
}
\item verb \\
If you \textbf{husband} something valuable , you use it carefully and do not waste it.
 \textit{
	\begin{itemize}
	\item Husbanding precious resources was part of rural life.
	\end{itemize}
}
\end{enumerate}

\section*{mist}
{\large \color{blue}  mists  misting  misted  }
\subsection*{Explain}
\begin{enumerate}
\item variable noun \\
\textbf{Mist} consists of a large number of tiny  drops of water in the air, which make it difficult to see very far .
 \textit{
	\begin{itemize}
	\item Thick mist made flying impossible.
	\item A bluish mist hung in the air.
	\item Mists and fog swirled about the road.
	\end{itemize}
}
\item verb \\
If a piece of glass  \textbf{mists} or \textbf{is misted} , it becomes covered with tiny drops of moisture , so that you cannot see through it easily .
 \textbf{Mist over} and \textbf{mist up}  mean the same as mist .
 \textit{
	\begin{itemize}
	\item The windows misted, blurring the stark streetlight.
	\item The temperature in the car was misting the window.
	\item The front windshield was misting over.
	\item She stood in front of the misted-up mirror.
	\end{itemize}
}
\item verb \\
If someone's eyes  \textbf{mist} , they cannot see easily because there are tears in their eyes.
 \textbf{Mist over} means the same as mist .
 \textit{
	\begin{itemize}
	\item Her eyes misted with tears.
	\item His eyes misted over and he started to shake.
	\end{itemize}
}
\end{enumerate}

\section*{lap}
{\large \color{blue}  laps  lapping  lapped  }
\subsection*{Explain}
\begin{enumerate}
\item countable noun \\
If you have something on your \textbf{lap} when you are sitting down, it is on top of your legs and near to your body.
 \textit{
	\begin{itemize}
	\item She waited quietly with her hands in her lap.
	\item Hugh glanced at the child on her mother's lap.
	\end{itemize}
}
\item countable noun \\
In a race, a competitor completes a \textbf{lap} when they have gone round a course once.
 \textit{
	\begin{itemize}
	\item ...that last lap of the race.
	\item On lap two, Baker edged forward.
	\end{itemize}
}
\item verb \\
In a race, if you \textbf{lap} another competitor, you go past them while they are still on the previous lap.
 \textit{
	\begin{itemize}
	\item He was caught out while lapping a slower rider.
	\end{itemize}
}
\item countable noun \\
A \textbf{lap} of a long journey is one part of it, between two points where you stop.
 \textit{
	\begin{itemize}
	\item I had thought we might travel as far as Oak Valley, but we only managed the first
lap of the journey.
	\end{itemize}
}
\item verb \\
When water \textbf{laps} against something such as the shore or the side of a boat, it touches it gently and makes a soft sound.
 \textit{
	\begin{itemize}
	\item ...the water that lapped against the pillars of the boathouse.
	\item With a rising tide the water was lapping at his chin before rescuers arrived.
	\item The building was right on the river and the water lapped the walls.
	\end{itemize}
}
\item verb \\
When an animal \textbf{laps} a drink, it uses short quick movements of its tongue to take liquid up into its mouth.
 \textbf{Lap up} means the same as lap .
 \textit{
	\begin{itemize}
	\item The cat lapped milk from a dish.
	\item She poured some water into a plastic bowl. Faust, her Great Dane, lapped it up with
relish.
	\end{itemize}
}
\item  \\
 in the lap of the gods \textit{
	\begin{itemize}
	\end{itemize}
}
\item  \\
 in the lap of luxury \textit{
	\begin{itemize}
	\end{itemize}
}
\end{enumerate}

\section*{napkin}
{\large \color{blue}  napkins  }
\subsection*{Explain}
\begin{enumerate}
\item countable noun \\
A \textbf{napkin} is a square of cloth or paper that you use when you are eating to protect your clothes,
or to wipe your mouth or hands .
 \textit{
	\begin{itemize}
	\item She was taking tiny bites of a hot dog and daintily wiping her lips with a napkin.
	\end{itemize}
}
\end{enumerate}

\section*{leaflet}
{\large \color{blue}  leaflets  leafleting  leafleted  }
\subsection*{Explain}
\begin{enumerate}
\item countable noun \\
A \textbf{leaflet} is a little book or a piece of paper containing information about a particular subject .
 \textit{
	\begin{itemize}
	\item Campaigners handed out leaflets on passive smoking.
	\item ...a leaflet called 'Sexual Harassment at Work'.
	\end{itemize}
}
\item verb \\
If you \textbf{leaflet} a place, you distribute leaflets there, for example by handing them to people, or by putting them through letter  boxes .
 \textit{
	\begin{itemize}
	\item We've leafleted the university today to try to drum up some support.
	\item The only reason we leafleted on the Jewish New Year was because more people than
usual go to the synagogue on that day.
	\end{itemize}
}
\end{enumerate}

\section*{operator}
{\large \color{blue}  operators  }
\subsection*{Explain}
\begin{enumerate}
\item countable noun \\
An \textbf{operator} is a person who connects  phone  calls at a telephone exchange or in a place such as an office or hotel .
 \textit{
	\begin{itemize}
	\item He dialled the operator and put in a call for Rome.
	\end{itemize}
}
\item countable noun \\
An \textbf{operator} is a person who is employed to operate or control a machine.
 \textit{
	\begin{itemize}
	\item ...computer operators.
	\end{itemize}
}
\item countable noun \\
An \textbf{operator} is a person or a company that runs a business .
 \textit{
	\begin{itemize}
	\item ...'Tele-Communications', the nation's largest cable TV operator.
	\end{itemize}
}
\item countable noun \\
If you call someone a good \textbf{operator} , you mean that they are skilful at achieving what they want , often in a slightly  dishonest way.
 \textit{
	\begin{itemize}
	\item He was a smart operator. Don't underestimate him.
	\item ...one of the shrewdest political operators in the Arab World.
	\end{itemize}
}
\end{enumerate}

\section*{magnitude}
{\large \color{blue}  magnitudes  }
\subsection*{Explain}
\begin{enumerate}
\item uncountable noun \\
If you talk about the \textbf{magnitude} of something, you are talking about its great size, scale, or importance.
 \textit{
	\begin{itemize}
	\item An operation of this magnitude is going to be difficult.
	\item These are issues of great magnitude.
	\item No one seems to realise the magnitude of this problem.
	\end{itemize}
}
\item variable noun \\
\textbf{Magnitude} is used in stating the size or extent of something such as a star, earthquake, or
 explosion .
 \textit{
	\begin{itemize}
	\item ...the 1.2 magnitude star Fomalhaut.
	\item The San Francisco earthquake of 1906 had a magnitude of 8.3.
	\end{itemize}
}
\item  \\
 order of magnitude \textit{
	\begin{itemize}
	\end{itemize}
}
\item  \\
 order of magnitude \textit{
	\begin{itemize}
	\end{itemize}
}
\end{enumerate}

\section*{organ}
{\large \color{blue}  organs  }
\subsection*{Explain}
\begin{enumerate}
\item countable noun \\
An \textbf{organ} is a part of your body that has a particular purpose or function , for example your heart or lungs .
 \textit{
	\begin{itemize}
	\item ...damage to the muscles and internal organs.
	\item ...the reproductive organs.
	\item ...organ transplants.
	\end{itemize}
}
\item countable noun \\
An \textbf{organ} is a large musical instrument with pipes of different  lengths through which air is forced. It has keys and pedals rather like a piano .
 \textit{
	\begin{itemize}
	\end{itemize}
}
\item countable noun \\
You refer to a newspaper or organization as \textbf{the}  \textbf{organ}  \textbf{of} the government or another group when it is used by them as a means of giving information or getting things done.
 \textit{
	\begin{itemize}
	\item The Security Service is an important organ of the State.
	\end{itemize}
}
\end{enumerate}

\section*{mainland}
{\large \color{blue}  }
\subsection*{Explain}
\begin{enumerate}
\item singular noun \\
You can refer to the largest part of a country or continent as \textbf{the mainland} when contrasting it with the islands around it.
 \textit{
	\begin{itemize}
	\item She was going to Nanaimo to catch the ferry to the mainland.
	\item ...the islands that lie off the coast of mainland Britain.
	\end{itemize}
}
\end{enumerate}

\section*{patron}
{\large \color{blue}  patrons  }
\subsection*{Explain}
\begin{enumerate}
\item countable noun \\
A \textbf{patron} is a person who supports and gives money to artists, writers , or musicians .
 \textit{
	\begin{itemize}
	\item Catherine the Great was a patron of the arts and sciences.
	\end{itemize}
}
\item countable noun \\
The \textbf{patron} of a charity, group, or campaign is an important person who allows his or her name to be used for publicity .
 \textit{
	\begin{itemize}
	\item Fiona and Alastair have become patrons of the National Missing Person's Helpline.
	\end{itemize}
}
\item countable noun \\
The \textbf{patrons} of a place such as a pub , bar, or hotel are its customers.
 \textit{
	\begin{itemize}
	\end{itemize}
}
\end{enumerate}

\section*{marble}
{\large \color{blue}  marbles  }
\subsection*{Explain}
\begin{enumerate}
\item uncountable noun \\
\textbf{Marble} is a type of very hard rock which feels cold when you touch it and which shines when it is cut and polished. Statues and parts of buildings are sometimes made of marble.
 \textit{
	\begin{itemize}
	\item The house has a superb staircase made from oak and marble.
	\item The entrance-hall was paved with black and white marble tiles.
	\end{itemize}
}
\item countable noun \\
\textbf{Marbles} are sculptures made of marble.
 \textit{
	\begin{itemize}
	\item ...marbles and bronzes from the Golden Age of Athens.
	\end{itemize}
}
\item uncountable noun \\
\textbf{Marbles} is a children's game played with small balls, usually made of coloured glass. You roll a ball along the ground and try to hit an opponent's ball with it.
 \textit{
	\begin{itemize}
	\item On the far side of the street, two boys were playing marbles.
	\end{itemize}
}
\item countable noun \\
A \textbf{marble} is one of the small balls used in the game of marbles.
 \textit{
	\begin{itemize}
	\end{itemize}
}
\item  \\
 lose one's marbles \textit{
	\begin{itemize}
	\end{itemize}
}
\end{enumerate}

\section*{prosperity}
{\large \color{blue}  }
\subsection*{Explain}
\begin{enumerate}
\item uncountable noun \\
\textbf{Prosperity} is a condition in which a person or community is doing well financially.
 \textit{
	\begin{itemize}
	\item ...a new era of peace and prosperity.
	\item ...Japan's economic prosperity.
	\end{itemize}
}
\end{enumerate}

\section*{mass}
{\large \color{blue}  masses  massing  massed  }
\subsection*{Explain}
\begin{enumerate}
\item singular noun \\
A \textbf{mass}  \textbf{of} things is a large number of them grouped together.
 \textit{
	\begin{itemize}
	\item On his desk is a mass of books and papers.
	\end{itemize}
}
\item singular noun \\
A \textbf{mass}  \textbf{of} something is a large amount of it.
 \textit{
	\begin{itemize}
	\item She had a mass of auburn hair.
	\end{itemize}
}
\item quantifier \\
\textbf{Masses}  \textbf{of} something means a great deal of it.
 \textit{
	\begin{itemize}
	\item There's masses of work for her to do.
	\item It has masses of flowers each year.
	\end{itemize}
}
\item adjective \\
\textbf{Mass} is used to describe something which involves or affects a very large number of people.
 \textit{
	\begin{itemize}
	\item ...ideas on combating mass unemployment.
	\item All the lights went off, and mass hysteria broke out.
	\item ...weapons of mass destruction.
	\item ...the harm caused by mass tourism.
	\end{itemize}
}
\item countable noun \\
A \textbf{mass}  \textbf{of} a solid substance, a liquid, or a gas is an amount of it, especially a large amount which has no definite shape.
 \textit{
	\begin{itemize}
	\item ...before it cools and sets into a solid mass.
	\item The fourteenth century cathedral was reduced to a mass of rubble.
	\item ...the strong temperature difference between the two masses of air.
	\end{itemize}
}
\item plural noun \\
If you talk about \textbf{the masses} , you mean the ordinary people in society, in contrast to the leaders or the highly  educated people.
 \textit{
	\begin{itemize}
	\item His music is commercial. It is aimed at the masses.
	\item This issue has aroused much resentment among the masses.
	\end{itemize}
}
\item singular noun \\
\textbf{The}  \textbf{mass}  \textbf{of} people are most of the people in a country, society, or group.
 \textit{
	\begin{itemize}
	\item The 1939-45 world war involved the mass of the population.
	\item Schools allowed the mass of children to leave school at 16 with poor qualifications.
	\end{itemize}
}
\item countable noun \\
A \textbf{mass}  \textbf{of} people is a large crowd of them.
 \textit{
	\begin{itemize}
	\item ...masses of excited people clogged the streets.
	\item ...a mass of grinning teenage faces.
	\end{itemize}
}
\item verb \\
When people or things \textbf{mass} , or when you \textbf{mass} them, they gather together into a large crowd or group.
 \textit{
	\begin{itemize}
	\item Shortly after the workers went on strike, police began to mass at the shipyard.
	\item The clouds massed, whipped up by the wind.
	\item The General was massing his troops for a counterattack.
	\end{itemize}
}
\item singular noun \\
If you say that something is \textbf{a}  \textbf{mass of} things, you mean that it is covered with them or full of them.
 \textit{
	\begin{itemize}
	\item His body was a mass of sores.
	\item In the spring, the meadow is a mass of daffodils.
	\end{itemize}
}
\item variable noun \\
In physics , the \textbf{mass} of an object is the amount of physical matter that it has.
 \textit{
	\begin{itemize}
	\item Astronomers know that Pluto and Triton have nearly the same size, mass, and density.
	\end{itemize}
}
\item variable noun \\
\textbf{Mass} is a Christian church ceremony , especially in a Roman Catholic or Orthodox church, during which people eat bread and drink wine in order to remember the last  meal of Jesus  Christ .
 \textit{
	\begin{itemize}
	\item She attended a convent school and went to Mass each day.
	\end{itemize}
}
\item countable noun \\
A \textbf{Mass} is a piece of music which uses the prayers from the Christian ceremony of Mass as the words that are sung.
 \textit{
	\begin{itemize}
	\end{itemize}
}
\end{enumerate}

\section*{refuge}
{\large \color{blue}  refuges  }
\subsection*{Explain}
\begin{enumerate}
\item uncountable noun \\
If you take \textbf{refuge}  somewhere , you try to protect yourself from physical  harm by going there.
 \textit{
	\begin{itemize}
	\item They took refuge in a bomb shelter.
	\item His home became a place of refuge for the believers.
	\end{itemize}
}
\item countable noun \\
A \textbf{refuge} is a place where you go for safety and protection, for example from violence or from bad weather.
 \textit{
	\begin{itemize}
	\item ...a refuge for women who had experienced domestic abuse.
	\item We climbed up a winding track towards a mountain refuge.
	\end{itemize}
}
\item uncountable noun \\
If you take \textbf{refuge}  \textbf{in} a particular way of behaving or thinking , you try to protect yourself from unhappiness or unpleasantness by behaving or thinking
in that way.
 \textit{
	\begin{itemize}
	\item With these restrictions on childhood it's no wonder kids seek refuge in consumerism.
	\item Father Rowan took refuge in silence.
	\end{itemize}
}
\end{enumerate}

\section*{packet}
{\large \color{blue}  packets  }
\subsection*{Explain}
\begin{enumerate}
\item countable noun \\
A \textbf{packet} is a small container in which a quantity of something is sold . Packets are either small boxes made of thin cardboard, or bags or envelopes made of paper or plastic .
 A \textbf{packet}  \textbf{of} something is an amount of it contained in a packet.
 \textit{
	\begin{itemize}
	\item Cook the rice according to instructions on the packet.
	\item ...a crisp packet.
	\item Use half a packet of noodles per person.
	\item Elinor bought her a packet of biscuits.
	\end{itemize}
}
\item countable noun \\
A \textbf{packet} is a small flat parcel.
 \textit{
	\begin{itemize}
	\item ...to send letters and packets abroad.
	\item ...a packet of photographs.
	\end{itemize}
}
\item singular noun \\
You can refer to a lot of money as \textbf{a packet} .
 \textit{
	\begin{itemize}
	\item It'll cost you a packet.
	\item You could save yourself a packet.
	\item Someone's making a packet out of it.
	\end{itemize}
}
\end{enumerate}

\section*{rein}
{\large \color{blue}  reins  reining  reined  }
\subsection*{Explain}
\begin{enumerate}
\item plural noun \\
\textbf{Reins} are the thin leather straps attached  round a horse's neck which are used to control the horse.
 \textit{
	\begin{itemize}
	\end{itemize}
}
\item plural noun \\
Journalists  sometimes use the expression  \textbf{the reins} or \textbf{the reins of power} to refer to the control of a country or organization .
 \textit{
	\begin{itemize}
	\item He was determined to see the party keep a hold on the reins of power.
	\end{itemize}
}
\item  \\
 give free rein to/give full rein to \textit{
	\begin{itemize}
	\end{itemize}
}
\item  \\
 to keep a tight rein on \textit{
	\begin{itemize}
	\end{itemize}
}
\end{enumerate}

\section*{painting}
{\large \color{blue}  paintings  }
\subsection*{Explain}
\begin{enumerate}
\item countable noun \\
A \textbf{painting} is a picture which someone has painted.
 \textit{
	\begin{itemize}
	\item ...a large oil-painting of Queen Victoria.
	\end{itemize}
}
\item uncountable noun \\
\textbf{Painting} is the activity of painting pictures.
 \textit{
	\begin{itemize}
	\item ...two hobbies she really enjoyed, painting and gardening.
	\end{itemize}
}
\item uncountable noun \\
\textbf{Painting} is the activity of painting doors , walls , and some other parts of buildings.
 \textit{
	\begin{itemize}
	\item ...painting and decorating.
	\end{itemize}
}
\end{enumerate}

\section*{restaurant}
{\large \color{blue}  restaurants  }
\subsection*{Explain}
\begin{enumerate}
\item countable noun \\
A \textbf{restaurant} is a place where you can eat a meal and pay for it. In restaurants your food is usually served to you at your table by a waiter or waitress .
 \textit{
	\begin{itemize}
	\item They ate in an Italian restaurant in Forth Street.
	\item We had dinner in the hotel's restaurant.
	\end{itemize}
}
\end{enumerate}

\section*{pamphlet}
{\large \color{blue}  pamphlets  }
\subsection*{Explain}
\begin{enumerate}
\item countable noun \\
A \textbf{pamphlet} is a very thin  book , with a paper cover, which gives information about something.
 \textit{
	\begin{itemize}
	\end{itemize}
}
\end{enumerate}

\section*{salary}
{\large \color{blue}  salaries  }
\subsection*{Explain}
\begin{enumerate}
\item variable noun \\
A \textbf{salary} is the money that someone is paid each month by their employer, especially when they are in a profession such as teaching , law, or medicine .
 \textit{
	\begin{itemize}
	\item ...the lawyer was paid a huge salary.
	\item The government has decided to increase salaries for all civil servants.
	\end{itemize}
}
\end{enumerate}

\section*{screw}
{\large \color{blue}  screws  screwing  screwed  }
\subsection*{Explain}
\begin{enumerate}
\item countable noun \\
A \textbf{screw} is a metal object similar to a nail , with a raised spiral line around it. You turn a screw using a screwdriver so that it goes through two things, for example two pieces of wood , and fastens them together.
 \textit{
	\begin{itemize}
	\item Each bracket is fixed to the wall with just three screws.
	\end{itemize}
}
\item verb \\
If you \textbf{screw} something somewhere or if it \textbf{screws} somewhere, you fix it in place by means of a screw or screws.
 \textit{
	\begin{itemize}
	\item I had screwed the shelf on the wall myself.
	\item Screw down any loose floorboards.
	\item I particularly like the type of shelving that screws to the wall.
	\end{itemize}
}
\item adjective \\
A \textbf{screw}  lid or fitting is one that has a raised spiral line on the inside or outside of it, so that it can be fixed in place by twisting.
 \textit{
	\begin{itemize}
	\item ...an ordinary jam jar with a screw lid.
	\end{itemize}
}
\item verb \\
If you \textbf{screw} something somewhere or if it \textbf{screws} somewhere, you fix it in place by twisting it round and round.
 \textit{
	\begin{itemize}
	\item Kelly screwed the silencer onto the pistol.
	\item Screw down the lid fairly tightly.
	\item This device screws into the shutter release button.
	\item ...several aluminium poles that screw together to give a maximum length of 10 yards.
	\end{itemize}
}
\item verb \\
If you \textbf{screw} something such as a piece of paper \textbf{into} a ball, you squeeze it or twist it tightly so that it is in the shape of a ball.
 \textit{
	\begin{itemize}
	\item He screwed the paper into a ball and tossed it into the fire.
	\end{itemize}
}
\item verb \\
If you \textbf{screw} your face or your eyes  \textbf{into} a particular  expression , you tighten the muscles of your face to form that expression, for example because you are in pain or because the light is too bright .
 \textit{
	\begin{itemize}
	\item He screwed his face into an expression of mock pain.
	\end{itemize}
}
\item verb \\
If someone \textbf{screws} someone else or if two people \textbf{screw} , they have sex together.
 \textit{
	\begin{itemize}
	\end{itemize}
}
\item verb \\
Some people use \textbf{screw} in expressions such as \textbf{screw you} or \textbf{screw that} to show that they are not concerned about someone or something or that they feel  contempt for them.
 \textit{
	\begin{itemize}
	\end{itemize}
}
\item verb \\
If someone says that they \textbf{have been screwed} , they mean that someone else has cheated them, especially by getting money from them dishonestly.
 \textit{
	\begin{itemize}
	\item They haven't given us accurate information. We've been screwed.
	\item The consumer is getting screwed by cover charges as well.
	\end{itemize}
}
\item verb \\
If someone \textbf{screws} something, especially money, \textbf{out of} you, they get it from you by putting  pressure on you.
 \textit{
	\begin{itemize}
	\item For decades rich nations have been screwing money out of poor nations.
	\end{itemize}
}
\item countable noun \\
Prisoners often refer to prison officers as \textbf{screws} .
 \textit{
	\begin{itemize}
	\end{itemize}
}
\item countable noun \\
A \textbf{screw} is a propeller on a ship or an aircraft .
 \textit{
	\begin{itemize}
	\end{itemize}
}
\item  \\
 turn the screw(s) on someone \textit{
	\begin{itemize}
	\end{itemize}
}
\item  \\
 turn of the screw \textit{
	\begin{itemize}
	\end{itemize}
}
\end{enumerate}

\section*{team}
{\large \color{blue}  teams  teaming  teamed  }
\subsection*{Explain}
\begin{enumerate}
\item countable noun \\
A \textbf{team} is a group of people who play a particular sport or game together against other similar groups of people.
 \textit{
	\begin{itemize}
	\item The team failed to qualify for the African Nations Cup finals.
	\item He had lost his place in the England team.
	\end{itemize}
}
\item countable noun \\
You can refer to any group of people who work together as a \textbf{team} .
 \textit{
	\begin{itemize}
	\item Each specialist consultant has a team of doctors under her.
	\end{itemize}
}
\end{enumerate}

\section*{sculpture}
{\large \color{blue}  sculptures  }
\subsection*{Explain}
\begin{enumerate}
\item variable noun \\
A \textbf{sculpture} is a work of art that is produced by carving or shaping stone, wood, clay , or other materials.
 \textit{
	\begin{itemize}
	\item ...stone sculptures of figures and animals.
	\item ...a collection of 20th-century art and sculpture.
	\end{itemize}
}
\item uncountable noun \\
\textbf{Sculpture} is the art of creating sculptures.
 \textit{
	\begin{itemize}
	\item Both studied sculpture.
	\end{itemize}
}
\end{enumerate}

\section*{thigh}
{\large \color{blue}  thighs  }
\subsection*{Explain}
\begin{enumerate}
\item countable noun \\
Your \textbf{thighs} are the top parts of your legs, between your knees and your hips.
 \textit{
	\begin{itemize}
	\end{itemize}
}
\end{enumerate}

\section*{slice}
{\large \color{blue}  slices  slicing  sliced  }
\subsection*{Explain}
\begin{enumerate}
\item countable noun \\
A \textbf{slice}  \textbf{of}  bread , meat , fruit , or other food is a thin piece that has been cut from a larger piece.
 \textit{
	\begin{itemize}
	\item Try to eat at least four slices of bread a day.
	\item ...water flavoured with a slice of lemon.
	\end{itemize}
}
\item verb \\
If you \textbf{slice} bread, meat, fruit, or other food, you cut it into thin pieces.
 \textbf{Slice up}  means the same as slice .
 \textit{
	\begin{itemize}
	\item Helen sliced the cake.
	\item Slice the steak into long thin slices.
	\item I sliced up an onion.
	\item He began slicing the pie up.
	\end{itemize}
}
\item countable noun \\
You can use \textbf{slice} to refer to a part of a situation or activity .
 \textit{
	\begin{itemize}
	\item Fiction takes up a large slice of the publishing market.
	\item ...a car that represents a slice of motoring history.
	\end{itemize}
}
\item verb \\
In tennis , golf , and other sports , if you \textbf{slice} a ball, you hit its edge  rather than its centre, so that it travels at an angle .
 \textbf{Slice} is also a noun .
 \textit{
	\begin{itemize}
	\item The captain swung his left foot, but sliced the ball wide.
	\item ...a ball that would reduce hooks and slices.
	\end{itemize}
}
\item verb \\
If something \textbf{slices}  \textbf{through} a substance , it moves through it quickly, like a knife.
 \textit{
	\begin{itemize}
	\item The ship sliced through the water.
	\end{itemize}
}
\end{enumerate}

\section*{trunk}
{\large \color{blue}  trunks  }
\subsection*{Explain}
\begin{enumerate}
\item countable noun \\
The \textbf{trunk} of a tree is the large main stem from which the branches grow.
 \textit{
	\begin{itemize}
	\item ...the gnarled trunk of a birch tree.
	\item ...toadstools growing on fallen tree trunks.
	\end{itemize}
}
\item countable noun \\
A \textbf{trunk} is a large, strong case or box used for storing things or for taking on a journey .
 \textit{
	\begin{itemize}
	\end{itemize}
}
\item countable noun \\
An elephant's \textbf{trunk} is its very long nose that it uses to lift food and water to its mouth .
 \textit{
	\begin{itemize}
	\end{itemize}
}
\item countable noun \\
The \textbf{trunk} of a car is a covered space at the back or front in which you put luggage or other things.
 \textit{
	\begin{itemize}
	\end{itemize}
}
\item plural noun \\
\textbf{Trunks} are shorts that a man wears when he goes  swimming .
 \textit{
	\begin{itemize}
	\item I wear these trunks because they have a streamline effect in the water.
	\end{itemize}
}
\item countable noun \\
Your \textbf{trunk} is the central part of your body, from your neck to your waist .
 \textit{
	\begin{itemize}
	\end{itemize}
}
\end{enumerate}

\section*{sugar}
{\large \color{blue}  sugars  sugaring  sugared  }
\subsection*{Explain}
\begin{enumerate}
\item uncountable noun \\
\textbf{Sugar} is a sweet substance that is used to make food and drinks sweet. It is usually in the form of small white
or brown  crystals .
 \textit{
	\begin{itemize}
	\item ...bags of sugar.
	\item Ice cream is high in fat and sugar.
	\end{itemize}
}
\item countable noun \\
If someone has one \textbf{sugar} in their tea or coffee , they have one small spoon of sugar or one sugar lump in it.
 \textit{
	\begin{itemize}
	\item How many sugars do you take?
	\item ...a mug of tea with two sugars.
	\end{itemize}
}
\item verb \\
If you \textbf{sugar} food or drink, you add sugar to it.
 \textit{
	\begin{itemize}
	\item He sat down and sugared and stirred his coffee.
	\end{itemize}
}
\item countable noun \\
\textbf{Sugars} are substances that occur naturally in food. When you eat them, the body converts them into energy .
 \textit{
	\begin{itemize}
	\item Plants produce sugars and starch to provide themselves with energy.
	\item ...the natural sugars found in grape juice.
	\end{itemize}
}
\item vocative noun \\
If one person knows another person very well and likes them a lot , they sometimes  call them \textbf{sugar} .
 \textit{
	\begin{itemize}
	\item I know how to make you feel better, sugar. I'll tell you a story.
	\end{itemize}
}
\end{enumerate}

\section*{torrent}
{\large \color{blue}  torrents  }
\subsection*{Explain}
\begin{enumerate}
\item countable noun \\
A \textbf{torrent} is a lot of water falling or flowing rapidly or violently.
 \textit{
	\begin{itemize}
	\item Torrents of water gushed into the reservoir.
	\item The rain came down in torrents, and we could see nothing.
	\item The trip involved crossing a raging torrent.
	\end{itemize}
}
\item countable noun \\
A \textbf{torrent}  \textbf{of}  abuse or questions is a lot of abuse or questions directed continuously at someone.
 \textit{
	\begin{itemize}
	\item He turned round and directed a torrent of abuse at me.
	\item ...a £45,000 offer which prompted a torrent of criticism in the media.
	\end{itemize}
}
\end{enumerate}

\section*{youngster}
{\large \color{blue}  youngsters  }
\subsection*{Explain}
\begin{enumerate}
\item countable noun \\
Young people, especially children, are sometimes referred to as \textbf{youngsters} .
 \textit{
	\begin{itemize}
	\item Other youngsters are not so lucky.
	\item I was only a youngster in 1935.
	\end{itemize}
}
\end{enumerate}

\section*{whale}
{\large \color{blue}  whales  }
\subsection*{Explain}
\begin{enumerate}
\item countable noun \\
\textbf{Whales} are very large mammals that live in the sea.
 \textit{
	\begin{itemize}
	\end{itemize}
}
\item  \\
 have a whale of a time \textit{
	\begin{itemize}
	\end{itemize}
}
\end{enumerate}

\section*{accent}
{\large \color{blue}  accents  }
\subsection*{Explain}
\begin{enumerate}
\item countable noun \\
Someone who speaks with a particular \textbf{accent} pronounces the words of a language in a distinctive way that shows which country, region, or social class they come from.
 \textit{
	\begin{itemize}
	\item He had developed a slight American accent.
	\end{itemize}
}
\item countable noun \\
An \textbf{accent} is a short line or other mark which is written above certain letters in some languages and which indicates the way those letters are pronounced.
 \textit{
	\begin{itemize}
	\end{itemize}
}
\item singular noun \\
If you put the \textbf{accent}  \textbf{on} a particular feature of something, you emphasize it or give it special  importance .
 \textit{
	\begin{itemize}
	\item He is putting the accent on military readiness.
	\item There is often a strong accent on material success.
	\end{itemize}
}
\end{enumerate}

\section*{billion}
{\large \color{blue}  billions  }
\subsection*{Explain}
\begin{enumerate}
\item number \\
A \textbf{billion} is a thousand million.
 \textit{
	\begin{itemize}
	\item ...3 billion dollars.
	\item This year, almost a billion birds will be processed in the region.
	\end{itemize}
}
\item quantifier \\
If you talk about \textbf{billions of} people or things, you mean that there is a very large number of them but you do not know or do not want to say  exactly how many.
 You can also use \textbf{billions} as a pronoun.
 \textit{
	\begin{itemize}
	\item Biological systems have been doing this for billions of years.
	\item He urged U.S. executives to invest billions of dollars in his country.
	\item He thought that it must be worth billions.
	\end{itemize}
}
\end{enumerate}

\section*{boat}
{\large \color{blue}  boats  }
\subsection*{Explain}
\begin{enumerate}
\item countable noun \\
A \textbf{boat} is something in which people can travel across water.
 \textit{
	\begin{itemize}
	\item One of the best ways to see the area is in a small boat.
	\item The island may be reached by boat from the mainland.
	\end{itemize}
}
\item countable noun \\
You can refer to a passenger ship as a \textbf{boat} .
 \textit{
	\begin{itemize}
	\item When the boat reached Cape Town, we said a temporary goodbye.
	\end{itemize}
}
\item  \\
 to miss the boat \textit{
	\begin{itemize}
	\end{itemize}
}
\item  \\
 to push the boat out \textit{
	\begin{itemize}
	\end{itemize}
}
\item  \\
 to rock the boat \textit{
	\begin{itemize}
	\end{itemize}
}
\item  \\
 in the same boat \textit{
	\begin{itemize}
	\end{itemize}
}
\end{enumerate}

\section*{bush}
{\large \color{blue}  bushes  }
\subsection*{Explain}
\begin{enumerate}
\item countable noun \\
A \textbf{bush} is a large plant which is smaller than a tree and has a lot of branches.
 \textit{
	\begin{itemize}
	\item Trees and bushes grew down to the water's edge.
	\end{itemize}
}
\item singular noun \\
The wild , uncultivated parts of some hot countries are referred to as \textbf{the}  \textbf{bush} .
 \textit{
	\begin{itemize}
	\item They walked through the dense Mozambican bush for thirty six hours.
	\item The jeep was found lying in thick bush.
	\end{itemize}
}
\item  \\
 to beat about the bush \textit{
	\begin{itemize}
	\end{itemize}
}
\end{enumerate}

\section*{brook}
{\large \color{blue}  brooks  brooking  brooked  }
\subsection*{Explain}
\begin{enumerate}
\item countable noun \\
A \textbf{brook} is a small stream.
 \textit{
	\begin{itemize}
	\end{itemize}
}
\item verb \\
If someone in a position of authority  will  \textbf{brook no}  interference or opposition , they will not accept any interference or opposition from others.
 \textit{
	\begin{itemize}
	\item She'd had a plan of action, one that would brook no interference.
	\item The army will brook no weakening of its power.
	\end{itemize}
}
\end{enumerate}

\section*{cop}
{\large \color{blue}  cops  copping  copped  }
\subsection*{Explain}
\begin{enumerate}
\item countable noun \\
A \textbf{cop} is a police officer.
 \textit{
	\begin{itemize}
	\item Frank didn't like having the cops know where to find him.
	\end{itemize}
}
\item  \\
 cop it \textit{
	\begin{itemize}
	\end{itemize}
}
\item  \\
 not much cop \textit{
	\begin{itemize}
	\end{itemize}
}
\end{enumerate}

\section*{bullet}
{\large \color{blue}  bullets  }
\subsection*{Explain}
\begin{enumerate}
\item countable noun \\
A \textbf{bullet} is a small piece of metal with a pointed or rounded end, which is fired out of a gun.
 \textit{
	\begin{itemize}
	\end{itemize}
}
\item  \\
 to bite the bullet \textit{
	\begin{itemize}
	\end{itemize}
}
\end{enumerate}

\section*{coward}
{\large \color{blue}  cowards  }
\subsection*{Explain}
\begin{enumerate}
\item countable noun \\
If you call someone a \textbf{coward} , you disapprove of them because they are easily  frightened and avoid dangerous or difficult  situations .
 \textit{
	\begin{itemize}
	\item She accused her husband of being a coward.
	\end{itemize}
}
\end{enumerate}

\section*{daughter}
{\large \color{blue}  daughters  }
\subsection*{Explain}
\begin{enumerate}
\item countable noun \\
Someone's \textbf{daughter} is their female child .
 \textit{
	\begin{itemize}
	\item ...Flora and her daughter Catherine.
	\item ...the daughter of a university professor.
	\item I have two daughters.
	\end{itemize}
}
\end{enumerate}

\section*{devil}
{\large \color{blue}  devils  }
\subsection*{Explain}
\begin{enumerate}
\item proper noun \\
In Judaism , Christianity , and Islam , \textbf{the Devil} is the most powerful evil spirit.
 \textit{
	\begin{itemize}
	\end{itemize}
}
\item countable noun \\
A \textbf{devil} is an evil spirit.
 \textit{
	\begin{itemize}
	\item ...the idea of angels with wings and devils with horns and hoofs.
	\end{itemize}
}
\item countable noun \\
You can use \textbf{devil} to emphasize the way you feel about someone. For example, if you call someone a poor  \textbf{devil} , you are saying that you feel sorry for them. You can call someone you are fond of but who sometimes annoys or irritates you an old \textbf{devil} or a little \textbf{devil} .
 \textit{
	\begin{itemize}
	\item I felt sorry for Blake, poor devil.
	\item Manfred, you're a suspicious old devil.
	\item Susie, you're a determined little devil.
	\end{itemize}
}
\item countable noun \\
If you refer to someone as a \textbf{devil} , you mean that they do not behave very well but you like them and are not angry with them.
 \textit{
	\begin{itemize}
	\item 'I think he was a bit of a devil,' Constance said.
	\end{itemize}
}
\item  \\
 a devil of a \textit{
	\begin{itemize}
	\end{itemize}
}
\item  \\
 better the devil you know \textit{
	\begin{itemize}
	\end{itemize}
}
\item  \\
 have the devil's own job/a devil of a job \textit{
	\begin{itemize}
	\end{itemize}
}
\item  \\
 like the devil \textit{
	\begin{itemize}
	\end{itemize}
}
\item  \\
 the devil take the hindmost \textit{
	\begin{itemize}
	\end{itemize}
}
\item  \\
 between the devil and the deep blue sea \textit{
	\begin{itemize}
	\end{itemize}
}
\item  \\
 sell your soul to the devil \textit{
	\begin{itemize}
	\end{itemize}
}
\item  \\
 talk/speak of the devil \textit{
	\begin{itemize}
	\end{itemize}
}
\item  \\
 what/how/why the devil \textit{
	\begin{itemize}
	\end{itemize}
}
\end{enumerate}

\section*{descendant}
{\large \color{blue}  descendants  }
\subsection*{Explain}
\begin{enumerate}
\item countable noun \\
Someone's \textbf{descendants} are the people in later  generations who are related to them.
 \textit{
	\begin{itemize}
	\item They are descendants of the original English and Scottish settlers.
	\item ...Lord Cochrane and his descendants.
	\end{itemize}
}
\item countable noun \\
Something modern which developed from an older thing can be called a \textbf{descendant}  \textbf{of} it.
 \textit{
	\begin{itemize}
	\item His design was a descendant of a 1956 device.
	\item They are the descendants of plants imported by the early settlers.
	\end{itemize}
}
\end{enumerate}

\section*{dew}
{\large \color{blue}  }
\subsection*{Explain}
\begin{enumerate}
\item uncountable noun \\
\textbf{Dew} is small drops of water that form on the ground and other surfaces outdoors during the night.
 \textit{
	\begin{itemize}
	\item The dew gathered on the leaves.
	\end{itemize}
}
\end{enumerate}

\section*{door}
{\large \color{blue}  doors  }
\subsection*{Explain}
\begin{enumerate}
\item countable noun \\
A \textbf{door} is a piece of wood , glass , or metal , which is moved to open and close the entrance to a building, room, cupboard, or vehicle .
 \textit{
	\begin{itemize}
	\item I knocked at the front door, but there was no answer.
	\item The police officer opened the door and looked in.
	\item I heard a door slamming.
	\end{itemize}
}
\item countable noun \\
A \textbf{door} is the space in a wall when a door is open.
 \textit{
	\begin{itemize}
	\item She looked through the door of the kitchen. Her daughter was at the stove.
	\end{itemize}
}
\item countable noun \\
The \textbf{door} is the entrance to a large building such as a shop , hotel , or theatre .
 \textit{
	\begin{itemize}
	\item The queues at the door wound around the building.
	\end{itemize}
}
\item plural noun \\
\textbf{Doors} is used in expressions such as \textbf{a few doors down} or \textbf{three doors up} to refer to a place that is a particular  number of buildings away from where you are.
 \textit{
	\begin{itemize}
	\item Mrs Cade's house was only a few doors down from her daughter's apartment.
	\end{itemize}
}
\item  \\
 answer the door \textit{
	\begin{itemize}
	\end{itemize}
}
\item  \\
 by/through the back door \textit{
	\begin{itemize}
	\end{itemize}
}
\item  \\
 to close the door on something \textit{
	\begin{itemize}
	\end{itemize}
}
\item  \\
 behind closed doors \textit{
	\begin{itemize}
	\end{itemize}
}
\item  \\
 from door to door/door to door \textit{
	\begin{itemize}
	\end{itemize}
}
\item  \\
 from door to door/door to door \textit{
	\begin{itemize}
	\end{itemize}
}
\item  \\
 foot in the door \textit{
	\begin{itemize}
	\end{itemize}
}
\item  \\
 to shut the door in someone's face \textit{
	\begin{itemize}
	\end{itemize}
}
\item  \\
 to lay something at someone's door \textit{
	\begin{itemize}
	\end{itemize}
}
\item  \\
 open the door to something \textit{
	\begin{itemize}
	\end{itemize}
}
\item  \\
 out of doors \textit{
	\begin{itemize}
	\end{itemize}
}
\item  \\
 see someone to the door \textit{
	\begin{itemize}
	\end{itemize}
}
\item  \\
 to show someone the door \textit{
	\begin{itemize}
	\end{itemize}
}
\end{enumerate}

\section*{eagle}
{\large \color{blue}  eagles  }
\subsection*{Explain}
\begin{enumerate}
\item countable noun \\
An \textbf{eagle} is a large bird that lives by eating small animals.
 \textit{
	\begin{itemize}
	\end{itemize}
}
\item  \\
 eagle eye \textit{
	\begin{itemize}
	\end{itemize}
}
\end{enumerate}

\section*{doorway}
{\large \color{blue}  doorways  }
\subsection*{Explain}
\begin{enumerate}
\item countable noun \\
A \textbf{doorway} is a space in a wall where a door opens and closes .
 \textit{
	\begin{itemize}
	\item Hannah looked up to see David and another man standing in the doorway.
	\item We were escorted through a low doorway.
	\end{itemize}
}
\item countable noun \\
A \textbf{doorway} is a covered space just outside the door of a building.
 \textit{
	\begin{itemize}
	\item ...homeless people sleeping in shop doorways.
	\end{itemize}
}
\end{enumerate}

\section*{eighteen}
{\large \color{blue}  eighteens  }
\subsection*{Explain}
\begin{enumerate}
\item number \\
\textbf{Eighteen} is the number 18.
 \textit{
	\begin{itemize}
	\item He was employed by them for eighteen years.
	\end{itemize}
}
\end{enumerate}

\section*{duty}
{\large \color{blue}  duties  }
\subsection*{Explain}
\begin{enumerate}
\item uncountable noun \\
\textbf{Duty} is work that you have to do for your job.
 \textit{
	\begin{itemize}
	\item Staff must report for duty at their normal place of work.
	\item My duty is to look after the animals.
	\end{itemize}
}
\item plural noun \\
Your \textbf{duties} are tasks which you have to do because they are part of your job.
 \textit{
	\begin{itemize}
	\item I carried out my duties conscientiously.
	\item He was relieved of his duties as presidential adviser.
	\end{itemize}
}
\item singular noun \\
If you say that something is your \textbf{duty} , you believe that you ought to do it because it is your responsibility .
 \textit{
	\begin{itemize}
	\item I consider it my duty to write to you and thank you.
	\end{itemize}
}
\item variable noun \\
\textbf{Duties} are taxes which you pay to the government on goods that you buy .
 \textit{
	\begin{itemize}
	\item Import duties still average 30%.
	\item ...customs duties.
	\item They are pressing the Chancellor to reduce excise duty on beer.
	\end{itemize}
}
\item  \\
 off duty, on duty \textit{
	\begin{itemize}
	\end{itemize}
}
\end{enumerate}

\section*{eighty}
{\large \color{blue}  eighties  }
\subsection*{Explain}
\begin{enumerate}
\item number \\
\textbf{Eighty} is the number 80.
 \textit{
	\begin{itemize}
	\item Eighty horses trotted up.
	\end{itemize}
}
\item plural noun \\
When you talk about the \textbf{eighties} , you are referring to numbers between 80 and 89. For example , if you are \textbf{in} your \textbf{eighties} , you are aged between 80 and 89. If the temperature is \textbf{in the eighties} , the temperature is between 80 and 89 degrees .
 \textit{
	\begin{itemize}
	\item He was in his late eighties and had become the country's most respected elder statesman.
	\end{itemize}
}
\item plural noun \\
\textbf{The eighties} is the decade between 1980 and 1989.
 \textit{
	\begin{itemize}
	\item He ran a property development business in the eighties.
	\end{itemize}
}
\end{enumerate}

\section*{extent}
{\large \color{blue}  }
\subsection*{Explain}
\begin{enumerate}
\item singular noun \\
If you are talking about how great, important , or serious a difficulty or situation is, you can refer to \textbf{the}  \textbf{extent}  \textbf{of} it.
 \textit{
	\begin{itemize}
	\item The government itself has little information on the extent of industrial pollution.
	\item Growing up with him soon made me realise the extent of his determination.
	\item The full extent of the losses was disclosed yesterday.
	\end{itemize}
}
\item singular noun \\
\textbf{The}  \textbf{extent}  \textbf{of} something is its length, area, or size .
 \textit{
	\begin{itemize}
	\item Their commitment was only to maintain the extent of forests, not their biodiversity.
	\end{itemize}
}
\item  \\
 to a large extent \textit{
	\begin{itemize}
	\end{itemize}
}
\item  \\
 to what/that extent/the extent that(etc) \textit{
	\begin{itemize}
	\end{itemize}
}
\item  \\
 to the extent of/that/to such an extent that(etc) \textit{
	\begin{itemize}
	\end{itemize}
}
\end{enumerate}

\section*{eleven}
{\large \color{blue}  elevens  }
\subsection*{Explain}
\begin{enumerate}
\item number \\
\textbf{Eleven} is the number 11.
 \textit{
	\begin{itemize}
	\item ...the Princess and her eleven friends.
	\end{itemize}
}
\end{enumerate}

\section*{goat}
{\large \color{blue}  goats  }
\subsection*{Explain}
\begin{enumerate}
\item countable noun \\
A \textbf{goat} is a farm animal or a wild animal that is about the size of a sheep. Goats have horns , and hairs on their chin which resemble a beard.
 \textit{
	\begin{itemize}
	\end{itemize}
}
\end{enumerate}

\section*{explosion}
{\large \color{blue}  explosions  }
\subsection*{Explain}
\begin{enumerate}
\item countable noun \\
An \textbf{explosion} is a sudden, violent burst of energy, for example one caused by a bomb .
 \textit{
	\begin{itemize}
	\item A second explosion came minutes later and we were enveloped in a dust cloud.
	\item Three people have been killed in a bomb explosion in northwest Spain.
	\end{itemize}
}
\item variable noun \\
\textbf{Explosion} is the act of deliberately causing a bomb or similar  device to explode.
 \textit{
	\begin{itemize}
	\item Bomb disposal experts blew up the bag in a controlled explosion.
	\item France has carried out an underground nuclear explosion on Mururoa Atoll in the South
Pacific.
	\end{itemize}
}
\item countable noun \\
An \textbf{explosion} is a large rapid increase in the number or amount of something.
 \textit{
	\begin{itemize}
	\item The study also forecast an explosion in the diet soft-drink market.
	\item He explains that there was an explosion of courses through the 1960s.
	\item The spread of the suburbs has triggered a population explosion among America's deer.
	\end{itemize}
}
\item countable noun \\
An \textbf{explosion} is a sudden violent expression of someone's feelings , especially  anger .
 \textit{
	\begin{itemize}
	\item Every time they met, Myra anticipated an explosion.
	\item It was an explosion of anger against the practices of the occupying forces.
	\end{itemize}
}
\item countable noun \\
An \textbf{explosion} is a sudden and serious  political  protest or violence .
 \textit{
	\begin{itemize}
	\item They warned him that a referendum might cause an explosion in the country.
	\item ...the explosion of protest and violence sparked off by the killing of seven workers.
	\end{itemize}
}
\item countable noun \\
An \textbf{explosion} is a sudden very loud noise.
 \textit{
	\begin{itemize}
	\item There was an explosion of music.
	\end{itemize}
}
\end{enumerate}

\section*{hostess}
{\large \color{blue}  hostesses  }
\subsection*{Explain}
\begin{enumerate}
\item countable noun \\
The \textbf{hostess} at a party is the woman who has invited the guests and provides the food, drink, or entertainment .
 \textit{
	\begin{itemize}
	\item The hostess introduced them.
	\item She was a superb hostess to us all.
	\end{itemize}
}
\item countable noun \\
A \textbf{hostess} at a night club or dance  hall is a woman who is paid by a man to be with him for the evening .
 \textit{
	\begin{itemize}
	\end{itemize}
}
\end{enumerate}

\section*{fifteen}
{\large \color{blue}  fifteens  }
\subsection*{Explain}
\begin{enumerate}
\item number \\
\textbf{Fifteen} is the number 15.
 \textit{
	\begin{itemize}
	\item In India, there are fifteen official languages.
	\end{itemize}
}
\item collective countable noun \\
A rugby-union team can be referred to as a \textbf{fifteen} .
 \textit{
	\begin{itemize}
	\end{itemize}
}
\end{enumerate}

\section*{hut}
{\large \color{blue}  huts  }
\subsection*{Explain}
\begin{enumerate}
\item countable noun \\
A \textbf{hut} is a small house with only one or two rooms , especially one which is made of wood, mud , grass , or stones .
 \textit{
	\begin{itemize}
	\end{itemize}
}
\item countable noun \\
A \textbf{hut} is a small wooden building in someone's garden , or a temporary building used by builders or repair  workers .
 \textit{
	\begin{itemize}
	\end{itemize}
}
\end{enumerate}

\section*{forty}
{\large \color{blue}  forties  }
\subsection*{Explain}
\begin{enumerate}
\item number \\
\textbf{Forty} is the number 40.
 \textit{
	\begin{itemize}
	\item She will be forty next birthday.
	\end{itemize}
}
\item plural noun \\
When you talk about the \textbf{forties} , you are referring to numbers between 40 and 49. For example , if you are \textbf{in} your \textbf{forties} , you are aged between 40 and 49. If the temperature is \textbf{in the forties} , the temperature is between 40 and 49 degrees .
 \textit{
	\begin{itemize}
	\item He was a big man in his forties, smartly dressed in a suit and tie.
	\end{itemize}
}
\item plural noun \\
\textbf{The forties} is the decade between 1940 and 1949.
 \textit{
	\begin{itemize}
	\item Steel cans were introduced sometime during the forties.
	\end{itemize}
}
\end{enumerate}

\section*{inn}
{\large \color{blue}  inns  }
\subsection*{Explain}
\begin{enumerate}
\item countable noun \\
An \textbf{inn} is a small hotel or pub, usually an old one.
 \textit{
	\begin{itemize}
	\item ...the Waterside Inn.
	\end{itemize}
}
\end{enumerate}

\section*{fourteen}
{\large \color{blue}  fourteens  }
\subsection*{Explain}
\begin{enumerate}
\item number \\
\textbf{Fourteen} is the number 14.
 \textit{
	\begin{itemize}
	\item I'm fourteen years old.
	\end{itemize}
}
\end{enumerate}

\section*{isle}
{\large \color{blue}  isles  }
\subsection*{Explain}
\begin{enumerate}
\item countable noun \\
An \textbf{isle} is an island; often used as part of an island's name, or in literary English.
 \textit{
	\begin{itemize}
	\item ...the paradise isle of Bali.
	\item ...the Isle of Man.
	\end{itemize}
}
\end{enumerate}

\section*{hawk}
{\large \color{blue}  hawks  hawking  hawked  }
\subsection*{Explain}
\begin{enumerate}
\item countable noun \\
A \textbf{hawk} is a large bird with a short, hooked  beak , sharp  claws , and very good eyesight . Hawks catch and eat small birds and animals.
 \textit{
	\begin{itemize}
	\end{itemize}
}
\item countable noun \\
In politics , if you refer to someone as a \textbf{hawk} , you mean that they believe in using force and violence to achieve something, rather than using more peaceful or diplomatic  methods . Compare  dove .
 \textit{
	\begin{itemize}
	\item Both hawks and doves have expanded their conditions for ending the war.
	\end{itemize}
}
\item verb \\
If someone \textbf{hawks} goods, they sell them by walking through the streets or knocking at people's houses , and asking people to buy them.
 \textit{
	\begin{itemize}
	\item ...vendors hawking trinkets.
	\end{itemize}
}
\item verb \\
You can say that someone \textbf{is hawking} something if you do not like the forceful way in which they are asking people to buy it.
 \textbf{Hawk around} means the same as hawk .
 \textit{
	\begin{itemize}
	\item Developers will be hawking cut-price flats and houses.
	\item He is hawking around a 15-minute, £5,000 promotional video.
	\item Most of the women were hawking food around the various prisons.
	\end{itemize}
}
\item verb \\
If someone \textbf{hawks} , they noisily clear mucus from their throat and spit it out.
 \textit{
	\begin{itemize}
	\item He hawked and spat.
	\end{itemize}
}
\item  \\
 watch sb like a hawk \textit{
	\begin{itemize}
	\end{itemize}
}
\end{enumerate}

\section*{kid}
{\large \color{blue}  kids  kidding  kidded  }
\subsection*{Explain}
\begin{enumerate}
\item countable noun \\
You can  refer to a child as a \textbf{kid} .
 \textit{
	\begin{itemize}
	\item They've got three kids.
	\item All the kids in my class could read.
	\end{itemize}
}
\item countable noun \\
Young people who are no longer children are sometimes referred to as \textbf{kids} .
 \textit{
	\begin{itemize}
	\item There were gangs of kids on motorbikes roaming around.
	\item That's a lot for a kid of 22 to cope with.
	\end{itemize}
}
\item adjective \\
You can refer to your younger brother as your \textbf{kid} brother and your younger sister as your \textbf{kid} sister.
 \textit{
	\begin{itemize}
	\item My kid sister woke up and started crying.
	\end{itemize}
}
\item countable noun \\
A \textbf{kid} is a young goat.
 \textit{
	\begin{itemize}
	\end{itemize}
}
\item verb \\
If you \textbf{are kidding} , you are saying something that is not really  true , as a joke .
 \textit{
	\begin{itemize}
	\item I'm not kidding, Frank. There's a cow out there, just standing around.
	\item I'm just kidding.
	\item Are you sure you're not kidding me?
	\end{itemize}
}
\item verb \\
If you \textbf{kid} someone, you tease them.
 \textit{
	\begin{itemize}
	\item He liked to kid Ingrid a lot.
	\item He used to kid me about being chubby.
	\end{itemize}
}
\item verb \\
If people \textbf{kid}  \textbf{themselves} , they allow themselves to believe something that is not true because they wish that it was true.
 \textit{
	\begin{itemize}
	\item We're kidding ourselves, Bill. We're not winning, we're not even doing well.
	\item I could kid myself that you did this for me, but it would be a lie.
	\end{itemize}
}
\item  \\
 no kidding \textit{
	\begin{itemize}
	\end{itemize}
}
\item  \\
 no kidding? \textit{
	\begin{itemize}
	\end{itemize}
}
\item  \\
 sb has got to be kidding/sb must be kidding \textit{
	\begin{itemize}
	\end{itemize}
}
\item  \\
 who is sb kidding/who is sb trying to kid? \textit{
	\begin{itemize}
	\end{itemize}
}
\end{enumerate}

\section*{hip}
{\large \color{blue}  hips  hipper  hippest  }
\subsection*{Explain}
\begin{enumerate}
\item countable noun \\
Your \textbf{hips} are the two areas at the sides of your body between the tops of your legs and your waist.
 \textit{
	\begin{itemize}
	\item Tracey put her hands on her hips and sighed.
	\end{itemize}
}
\item countable noun \\
You refer to the bones between the tops of your legs and your waist as your \textbf{hips} .
 \textit{
	\begin{itemize}
	\end{itemize}
}
\item adjective \\
If you say that someone is \textbf{hip} , you mean that they are very modern and follow all the latest fashions, for example in clothes and ideas.
 \textit{
	\begin{itemize}
	\item ...a hip young character with tight-cropped blond hair and stylish glasses.
	\end{itemize}
}
\item countable noun \\
A \textbf{hip} is a \textbf{rosehip} .
 \textit{
	\begin{itemize}
	\end{itemize}
}
\item  \\
 hip hip hooray \textit{
	\begin{itemize}
	\end{itemize}
}
\item  \\
 to shoot from the hip \textit{
	\begin{itemize}
	\end{itemize}
}
\end{enumerate}

\section*{knob}
{\large \color{blue}  knobs  }
\subsection*{Explain}
\begin{enumerate}
\item countable noun \\
A \textbf{knob} is a round handle on a door or drawer which you use in order to open or close it.
 \textit{
	\begin{itemize}
	\item He turned the knob and pushed against the door.
	\end{itemize}
}
\item countable noun \\
A \textbf{knob} is a rounded lump or ball on top of a post or stick .
 \textit{
	\begin{itemize}
	\item A loose brass knob on the bedstead rattled.
	\end{itemize}
}
\item countable noun \\
A \textbf{knob} is a round switch on a piece of machinery or equipment .
 \textit{
	\begin{itemize}
	\item ...the volume knob.
	\end{itemize}
}
\item countable noun \\
A \textbf{knob of}  butter is a small amount of it.
 \textit{
	\begin{itemize}
	\item Top the steaming hot potatoes with a knob of butter.
	\end{itemize}
}
\end{enumerate}

\section*{jar}
{\large \color{blue}  jars  jarring  jarred  }
\subsection*{Explain}
\begin{enumerate}
\item countable noun \\
A \textbf{jar} is a glass container with a lid that is used for storing  food .
 \textit{
	\begin{itemize}
	\item ...yellow cucumbers in great glass jars.
	\end{itemize}
}
\item countable noun \\
You can use \textbf{jar} to refer to a jar and its contents, or to the contents only.
 \textit{
	\begin{itemize}
	\item She opened up a glass jar of plums.
	\item ...two jars of filter coffee.
	\end{itemize}
}
\item countable noun \\
If you have a \textbf{jar} , you have a drink with friends in a pub .
 \textit{
	\begin{itemize}
	\item They had a few jars together.
	\end{itemize}
}
\item verb \\
If something \textbf{jars}  \textbf{on} you, you find it unpleasant , disturbing, or shocking.
 \textit{
	\begin{itemize}
	\item Sometimes a light remark jarred on her father.
	\item ...televised congressional hearings that jarred the nation's faith in the presidency.
	\item You shouldn't have too many colours in a small space as the effect can jar.
	\end{itemize}
}
\item verb \\
If an object  \textbf{jars} , or if something \textbf{jars} it, the object moves with a fairly  hard  shaking  movement .
 \textit{
	\begin{itemize}
	\item The ship jarred a little.
	\item The impact jarred his arm.
	\end{itemize}
}
\end{enumerate}

\section*{lady}
{\large \color{blue}  ladies  }
\subsection*{Explain}
\begin{enumerate}
\item countable noun \\
You can use \textbf{lady} when you are referring to a woman, especially when you are showing politeness or respect.
 \textit{
	\begin{itemize}
	\item She's a very sweet old lady.
	\item Shall we rejoin the ladies?
	\item ...a lady doctor.
	\item ...a cream-coloured lady's shoe.
	\end{itemize}
}
\item countable noun \\
You can say ' \textbf{ladies} ' when you are addressing a group of women in a formal and respectful  way .
 \textit{
	\begin{itemize}
	\item Your table is ready, ladies, if you'd care to come through.
	\item Good afternoon, ladies and gentlemen.
	\end{itemize}
}
\item countable noun \\
A \textbf{lady} is a woman from the upper classes, especially in former times.
 \textit{
	\begin{itemize}
	\item ...the Empress and ladies of the Imperial Palace.
	\item Our governess was told to make sure we knew how to talk like English ladies.
	\end{itemize}
}
\item title noun \\
In Britain, \textbf{Lady} is a title used in front of the names of some female members of the nobility , or the wives of knights .
 \textit{
	\begin{itemize}
	\item Cockburn's arrival coincided with that of Sir Iain and Lady Noble.
	\item My dear Lady Mary, how very good to see you.
	\end{itemize}
}
\item countable noun \\
If you say that a woman is a \textbf{lady} , you mean that she behaves in a polite, dignified , and graceful way.
 \textit{
	\begin{itemize}
	\item His wife was great as well, beautiful-looking and a real lady.
	\item A lady always sits quietly with her hands in her lap.
	\end{itemize}
}
\item singular noun \\
People sometimes refer to a public  toilet for women as \textbf{the ladies} .
 \textit{
	\begin{itemize}
	\item At Temple station, Charlotte rushed into the Ladies.
	\end{itemize}
}
\item countable noun \\
' \textbf{Lady} ' is sometimes used by men as a form of address when they are talking to a woman that they do not know , especially in shops and in the street .
 \textit{
	\begin{itemize}
	\item What seems to be the trouble, lady?
	\item As she left the litter-strewn lot, an angry voice called out to her. 'Hey, lady!'
	\end{itemize}
}
\end{enumerate}

\section*{magic}
{\large \color{blue}  }
\subsection*{Explain}
\begin{enumerate}
\item uncountable noun \\
\textbf{Magic} is the power to use supernatural forces to make impossible things happen , such as making people disappear or controlling events in nature .
 \textit{
	\begin{itemize}
	\item They believe in magic.
	\item ...the use of magic to combat any adverse powers or influences.
	\item Older legends say that Merlin raised the stones by magic.
	\end{itemize}
}
\item uncountable noun \\
You can use \textbf{magic} when you are referring to an event that is so wonderful, strange , or unexpected that it seems as if supernatural powers have caused it. You can also  say that something happens \textbf{as if by magic} or \textbf{like magic} .
 \textit{
	\begin{itemize}
	\item All this was supposed to work magic.
	\item The picture will now appear, as if by magic!
	\item The fog disappeared like magic.
	\end{itemize}
}
\item adjective \\
You use \textbf{magic} to describe something that does things, or appears to do things, by magic.
 \textit{
	\begin{itemize}
	\item So it's a magic potion?
	\item ...the magic ingredient that helps to keep skin looking smooth.
	\end{itemize}
}
\item uncountable noun \\
\textbf{Magic} is the art and skill of performing mysterious tricks to entertain people, for example by making things appear and disappear.
 \textit{
	\begin{itemize}
	\item His secret hobby: performing magic tricks.
	\end{itemize}
}
\item uncountable noun \\
If you refer to \textbf{the}  \textbf{magic}  \textbf{of} something, you mean that it has a special mysterious quality which makes it seem wonderful and exciting to you and which makes
you feel  happy .
 \textbf{Magic} is also an adjective .
 \textit{
	\begin{itemize}
	\item It infected them with some of the magic of a lost age.
	\item There can be a magic about love that defies all explanation.
	\item There were also moments of pure magic.
	\item Then came those magic moments in the rose-garden.
	\end{itemize}
}
\item uncountable noun \\
If you refer to a person's \textbf{magic} , you mean a special talent or ability that they have, which you admire or consider very impressive .
 \textit{
	\begin{itemize}
	\item The fighter believes he can still regain some of his old magic.
	\end{itemize}
}
\item adjective \\
You can use expressions such as \textbf{the magic number} and \textbf{the magic word} to indicate that a number or word is the one which is significant or desirable in a particular situation .
 \textit{
	\begin{itemize}
	\item ...their quest to gain the magic number of 270 electoral votes on Election Day.
	\item ...the magic word that opened doors onto private worlds.
	\end{itemize}
}
\item adjective \\
\textbf{Magic} is used in expressions such as \textbf{there is no magic formula} and \textbf{there is no magic solution} to say that someone will have to make an effort to solve a problem , because it will not solve itself.
 \textit{
	\begin{itemize}
	\item There is no magic formula for producing winning products.
	\item There is no magic cure.
	\end{itemize}
}
\item adjective \\
If you say that something is \textbf{magic} , you think it is very good or enjoyable.
 \textit{
	\begin{itemize}
	\item It was magic–one of the best days of my life.
	\end{itemize}
}
\end{enumerate}

\section*{lane}
{\large \color{blue}  lanes  }
\subsection*{Explain}
\begin{enumerate}
\item countable noun \\
A \textbf{lane} is a narrow road, especially in the country.
 \textit{
	\begin{itemize}
	\item ...a quiet country lane.
	\item Follow the lane to the river.
	\end{itemize}
}
\item countable noun \\
\textbf{Lane} is also used in the names of roads, either in cities or in the country.
 \textit{
	\begin{itemize}
	\item ...The Dorchester Hotel, Park Lane.
	\end{itemize}
}
\item countable noun \\
A \textbf{lane} is a part of a main road which is marked by the edge of the road and a painted line, or by two painted lines.
 \textit{
	\begin{itemize}
	\item The lorry was travelling at 20mph in the slow lane.
	\item I pulled out into the eastbound lane of Route 2.
	\end{itemize}
}
\item countable noun \\
At a swimming pool or athletics track, a \textbf{lane} is a long narrow section which is marked by lines or ropes .
 \textit{
	\begin{itemize}
	\item The pool is divided into three sections with a crawler lane for beginners.
	\item Drawn in lane three, she had all her rivals in sight.
	\end{itemize}
}
\item countable noun \\
A \textbf{lane} is a route that is frequently used by aircraft or ships.
 \textit{
	\begin{itemize}
	\item The collision took place in the busiest shipping lanes in the world.
	\end{itemize}
}
\item  \\
 the fast lane \textit{
	\begin{itemize}
	\end{itemize}
}
\end{enumerate}

\section*{million}
{\large \color{blue}  millions  }
\subsection*{Explain}
\begin{enumerate}
\item number \\
A \textbf{million} or one \textbf{million} is the number 1,000,000.
 \textit{
	\begin{itemize}
	\item Up to five million people a year visit the county.
	\item Profits for 1999 topped £100 million.
	\end{itemize}
}
\item quantifier \\
If you talk about \textbf{millions of} people or things, you mean that there is a very large number of them but you do not know or do not want to say  exactly how many.
 You can also use \textbf{millions} as a pronoun.
 \textit{
	\begin{itemize}
	\item The programme was viewed on television in millions of homes.
	\item This wretched war has brought misery to millions.
	\end{itemize}
}
\end{enumerate}

\section*{lobby}
{\large \color{blue}  lobbies  lobbying  lobbied  }
\subsection*{Explain}
\begin{enumerate}
\item verb \\
If you \textbf{lobby} someone such as a member of a government or council , you try to persuade them that a particular law should be changed or that a particular thing should be done.
 \textit{
	\begin{itemize}
	\item Carers from all over the U.K. lobbied Parliament last week to demand a better financial
deal.
	\item Gun control advocates are lobbying hard for new laws.
	\item The union has attacked the plan and threatened to lobby against it.
	\item It must be terribly frustrating to lobby and get absolutely nowhere.
	\end{itemize}
}
\item countable noun \\
A \textbf{lobby} is a group of people who represent a particular organization or campaign , and try to persuade a government or council to help or support them.
 \textit{
	\begin{itemize}
	\item Agricultural interests are some of the most powerful lobbies in Washington.
	\item He set up this lobby of independent producers.
	\item ...the Lawyers' Committee for Civil Rights, a housing lobby group.
	\end{itemize}
}
\item countable noun \\
In a hotel or other large building, the \textbf{lobby} is the area near the entrance that usually has corridors and staircases  leading off it.
 \textit{
	\begin{itemize}
	\item I met her in the lobby of the museum.
	\end{itemize}
}
\end{enumerate}

\section*{mushroom}
{\large \color{blue}  mushrooms  mushrooming  mushroomed  }
\subsection*{Explain}
\begin{enumerate}
\item variable noun \\
\textbf{Mushrooms} are fungi that you can eat .
 \textit{
	\begin{itemize}
	\item There are many types of wild mushrooms.
	\item ...eggs, bacon, sausage, and mushrooms.
	\item ...mushroom omelette.
	\end{itemize}
}
\item verb \\
If something such as an industry or a place \textbf{mushrooms} , it grows or comes into existence very quickly.
 \textit{
	\begin{itemize}
	\item The media training industry has mushroomed over the past decade.
	\item A town of a few hundred thousand people mushroomed to a crowded city of 2 million.
	\end{itemize}
}
\end{enumerate}

\section*{miss}
{\large \color{blue}  misses  missing  missed  }
\subsection*{Explain}
\begin{enumerate}
\item verb \\
If you \textbf{miss} something, you fail to hit it, for example when you have thrown something at it or you have shot a bullet at it.
 \textbf{Miss} is also a noun .
 \textit{
	\begin{itemize}
	\item She hurled the ashtray across the room, narrowly missing my head.
	\item When I'd missed a few times, he suggested I rest the rifle on a rock to steady it.
	\item After more misses, they finally put two arrows into the lion's chest.
	\end{itemize}
}
\item verb \\
In sport , if you \textbf{miss} a shot, you fail to get the ball in the goal, net , or hole .
 \textbf{Miss} is also a noun.
 \textit{
	\begin{itemize}
	\item He scored four of the goals but missed a penalty.
	\item The striker was guilty of two glaring misses.
	\end{itemize}
}
\item verb \\
If you \textbf{miss} something, you fail to notice it.
 \textit{
	\begin{itemize}
	\item From this vantage point he watched, his searching eye never missing a detail.
	\item It's the first thing you see as you come round the corner. You can't miss it.
	\item Sergeant Cobbins was an experienced officer and didn't miss much.
	\end{itemize}
}
\item verb \\
If you \textbf{miss} the meaning or importance of something, you fail to understand or appreciate it.
 \textit{
	\begin{itemize}
	\item Tambov had slightly missed the point.
	\item She seems to have missed the joke.
	\end{itemize}
}
\item verb \\
If you \textbf{miss} a chance or opportunity , you fail to take advantage of it.
 \textit{
	\begin{itemize}
	\item Williams knew that she had missed her chance of victory.
	\item It was too good an opportunity to miss.
	\end{itemize}
}
\item verb \\
If you \textbf{miss} someone who is no longer with you or who has died , you feel  sad and wish that they were still with you.
 \textit{
	\begin{itemize}
	\item Your mama and I are gonna miss you at Christmas.
	\item He was a gentle, sensitive, lovable man who will be missed by a host of friends.
	\end{itemize}
}
\item verb \\
If you \textbf{miss} something, you feel sad because you no longer have it or are no longer doing or experiencing it.
 \textit{
	\begin{itemize}
	\item I could happily move back into a flat if it wasn't for the fact that I'd miss my
garden.
	\item He missed having good friends.
	\end{itemize}
}
\item verb \\
If you \textbf{miss} something such as a plane or train , you arrive too late to catch it.
 \textit{
	\begin{itemize}
	\item I had already missed my flight, and the next one wasn't until the following morning.
	\item He missed the last bus home.
	\end{itemize}
}
\item verb \\
If you \textbf{miss} something such as a meeting or an activity, you do not go to it or take part in it.
 \textit{
	\begin{itemize}
	\item It's a pity Makku and I had to miss our lesson last week.
	\item You won't be missing much on TV tonight apart from the usual repeats.
	\item 'Are you coming to the show?'—'I wouldn't miss it for the world.'
	\end{itemize}
}
\item  \\
 give sth a miss \textit{
	\begin{itemize}
	\end{itemize}
}
\end{enumerate}

\section*{nineteen}
{\large \color{blue}  nineteens  }
\subsection*{Explain}
\begin{enumerate}
\item number \\
\textbf{Nineteen} is the number 19.
 \textit{
	\begin{itemize}
	\item They have nineteen days to make up their minds.
	\end{itemize}
}
\end{enumerate}

\section*{mistress}
{\large \color{blue}  mistresses  }
\subsection*{Explain}
\begin{enumerate}
\item countable noun \\
A married man's \textbf{mistress} is a woman who is not his wife and with whom he is having a sexual relationship.
 \textit{
	\begin{itemize}
	\item She was his mistress for three years.
	\item He has a wife and a mistress.
	\end{itemize}
}
\item countable noun \\
A \textbf{mistress} is a female teacher .
 \textit{
	\begin{itemize}
	\item My history mistress was extremely helpful.
	\end{itemize}
}
\item countable noun \\
A servant's \textbf{mistress} is the woman that he or she works for.
 \textit{
	\begin{itemize}
	\item It must be really bad, for her to ignore a summons from her mistress!
	\end{itemize}
}
\item countable noun \\
A dog's \textbf{mistress} is the woman or girl who owns it.
 \textit{
	\begin{itemize}
	\item The huge wolfhound danced in circles around his mistress.
	\end{itemize}
}
\item uncountable noun \\
If a woman is \textbf{mistress} of a situation , she has complete control over it.
 \textit{
	\begin{itemize}
	\item She had always been mistress of her own destiny.
	\end{itemize}
}
\item countable noun \\
If you say that a woman is a \textbf{mistress of} a particular activity , you mean that she is very skilled at it.
 \textit{
	\begin{itemize}
	\item She is a mistress of disguise.
	\item ...another winner from the mistress of historical romance.
	\end{itemize}
}
\end{enumerate}

\section*{mountain}
{\large \color{blue}  mountains  }
\subsection*{Explain}
\begin{enumerate}
\item countable noun \\
A \textbf{mountain} is a very high area of land with steep sides.
 \textit{
	\begin{itemize}
	\item Ben Nevis, in Scotland, is Britain's highest mountain.
	\item ...a lovely little mountain village.
	\end{itemize}
}
\item quantifier \\
If you talk about a \textbf{mountain}  \textbf{of} something, or \textbf{mountains}  \textbf{of} something, you are emphasizing that there is a large amount of it.
 \textit{
	\begin{itemize}
	\item They are faced with a mountain of bureaucracy.
	\item They have mountains of coffee to sell.
	\end{itemize}
}
\item  \\
 a mountain to climb \textit{
	\begin{itemize}
	\end{itemize}
}
\end{enumerate}

\section*{police}
{\large \color{blue}  polices  policing  policed  }
\subsection*{Explain}
\begin{enumerate}
\item singular noun \\
The \textbf{police} are the official organization that is responsible for making sure that people obey the law.
 \textit{
	\begin{itemize}
	\item The police are also looking for a second car.
	\item Police say they have arrested twenty people following the disturbances.
	\item I noticed a police car shadowing us.
	\end{itemize}
}
\item  \\
 See also  secret police \textit{
	\begin{itemize}
	\end{itemize}
}
\item plural noun \\
\textbf{Police} are men and women who are members of the official organization that is responsible
for making sure that people obey the law.
 \textit{
	\begin{itemize}
	\item More than one hundred police have ringed the area.
	\end{itemize}
}
\item verb \\
If the police or military forces \textbf{police} an area or event, they make sure that law and order is preserved in that area or at that event.
 \textit{
	\begin{itemize}
	\item ...the tiny U.N. observer force whose job it is to police the border.
	\item The march was heavily policed.
	\end{itemize}
}
\item verb \\
If a person or group in authority  \textbf{polices} a law or an area of public life, they make sure that what is done is fair and legal .
 \textit{
	\begin{itemize}
	\item ...Imro, the self-regulatory body that polices the investment management business.
	\end{itemize}
}
\end{enumerate}

\section*{mouth}
{\large \color{blue}  mouths  mouthing  mouthed  }
\subsection*{Explain}
\begin{enumerate}
\item countable noun \\
Your \textbf{mouth} is the area of your face where your lips are or the space behind your lips where your teeth and tongue are.
 \textit{
	\begin{itemize}
	\item She clamped her hand against her mouth.
	\item His mouth was full of peas.
	\item ...an inflammation of the mouth.
	\end{itemize}
}
\item countable noun \\
You can say that someone has a particular kind of \textbf{mouth} to indicate that they speak in a particular kind of way or that they say particular kinds of
things.
 \textit{
	\begin{itemize}
	\item I've always had a loud mouth, I refuse to be silenced.
	\item You've got such a crude mouth!
	\end{itemize}
}
\item countable noun \\
The \textbf{mouth} of a cave, hole , or bottle is its entrance or opening.
 \textit{
	\begin{itemize}
	\item By the mouth of the tunnel, he bent to retie his lace.
	\end{itemize}
}
\item countable noun \\
The \textbf{mouth} of a river is the place where it flows into the sea.
 \textit{
	\begin{itemize}
	\item ...the town at the mouth of the River Dart.
	\end{itemize}
}
\item verb \\
If you \textbf{mouth} something, you form words with your lips without making any sound.
 \textit{
	\begin{itemize}
	\item I mouthed a goodbye and hurried in behind Momma.
	\item She winked broadly at him and silently mouthed something.
	\item 'It's for you,' he mouthed.
	\end{itemize}
}
\item verb \\
If you \textbf{mouth} something, you say it, especially without believing it or without understanding it.
 \textit{
	\begin{itemize}
	\item I mouthed some sympathetic platitudes.
	\item They mouthed the values of family and charity, but demonstrated the opposite.
	\end{itemize}
}
\item  \\
 mouths to feed \textit{
	\begin{itemize}
	\end{itemize}
}
\item  \\
 open your mouth \textit{
	\begin{itemize}
	\end{itemize}
}
\end{enumerate}

\section*{policeman}
{\large \color{blue}  policemen  }
\subsection*{Explain}
\begin{enumerate}
\item countable noun \\
A \textbf{policeman} is a man who is a member of the police force.
 \textit{
	\begin{itemize}
	\end{itemize}
}
\end{enumerate}

\section*{pot}
{\large \color{blue}  pots  potting  potted  }
\subsection*{Explain}
\begin{enumerate}
\item countable noun \\
A \textbf{pot} is a deep round container used for cooking stews , soups , and other food.
 A \textbf{pot}  \textbf{of} stew, soup, or other food is an amount of it contained in a pot.
 \textit{
	\begin{itemize}
	\item ...metal cooking pots.
	\item He was stirring a pot of soup.
	\end{itemize}
}
\item countable noun \\
You can use \textbf{pot} to refer to a teapot or coffee pot.
 A \textbf{pot}  \textbf{of}  tea or coffee is an amount of it contained in a pot.
 \textit{
	\begin{itemize}
	\item There's tea in the pot.
	\item He spilt a pot of coffee.
	\end{itemize}
}
\item countable noun \\
A \textbf{pot} is a cylindrical container for jam , paint , or some other thick liquid.
 A \textbf{pot}  \textbf{of} jam, paint, or some other thick liquid is an amount of it contained in a pot.
 \textit{
	\begin{itemize}
	\item Hundreds of jam pots lined her scrubbed shelves.
	\item ...a pot of red paint.
	\end{itemize}
}
\item countable noun \\
A \textbf{pot} is the same as a flowerpot .
 \textit{
	\begin{itemize}
	\end{itemize}
}
\item verb \\
If you \textbf{pot} a young plant, or part of a plant, you put it into a container filled with soil , so it can grow there.
 \textit{
	\begin{itemize}
	\item Pot the cuttings individually.
	\item ...potted plants.
	\end{itemize}
}
\item uncountable noun \\
\textbf{Pot} is sometimes used to refer to the drugs cannabis and marijuana.
 \textit{
	\begin{itemize}
	\end{itemize}
}
\item quantifier \\
If you have \textbf{pots of} money, you have a lot of it.
 \textit{
	\begin{itemize}
	\item He must have pots of money.
	\end{itemize}
}
\item singular noun \\
In a card game, \textbf{the pot} is the money from all the players which the winner of the game will take as a prize.
 \textit{
	\begin{itemize}
	\end{itemize}
}
\item singular noun \\
You can refer to a fund consisting of money from several people as \textbf{the pot} .
 \textit{
	\begin{itemize}
	\item I've taken some money from the pot for wrapping paper.
	\end{itemize}
}
\item countable noun \\
Someone who has a \textbf{pot} has a round, fat  stomach which sticks out, either because they eat or drink too much, or because they have had very little
to eat for some time.
 \textit{
	\begin{itemize}
	\end{itemize}
}
\item countable noun \\
A \textbf{pot} is a deep bowl which a small child uses instead of a toilet .
 \textit{
	\begin{itemize}
	\end{itemize}
}
\item verb \\
In the games of snooker and billiards , if you \textbf{pot} a ball, you succeed in hitting it into one of the pockets.
 \textit{
	\begin{itemize}
	\item He did not pot a ball for the next two frames.
	\end{itemize}
}
\item  \\
 go to pot \textit{
	\begin{itemize}
	\end{itemize}
}
\item  \\
 pot luck \textit{
	\begin{itemize}
	\end{itemize}
}
\end{enumerate}

\section*{obligation}
{\large \color{blue}  obligations  }
\subsection*{Explain}
\begin{enumerate}
\item variable noun \\
If you have an \textbf{obligation}  \textbf{to} do something, it is your duty to do that thing.
 \textit{
	\begin{itemize}
	\item When teachers assign homework, students usually feel an obligation to do it.
	\item Ministers are under no obligation to follow the committee's recommendations.
	\end{itemize}
}
\item variable noun \\
If you have an \textbf{obligation}  \textbf{to} a person, it is your duty to look after them or protect their interests .
 \textit{
	\begin{itemize}
	\item The United States will do that which is necessary to meet its obligations to its
own citizens.
	\item I have an ethical and a moral obligation to my client.
	\end{itemize}
}
\item  \\
 without obligation \textit{
	\begin{itemize}
	\end{itemize}
}
\end{enumerate}

\section*{raid}
{\large \color{blue}  raids  raiding  raided  }
\subsection*{Explain}
\begin{enumerate}
\item verb \\
When soldiers  \textbf{raid} a place, they make a sudden armed attack against it, with the aim of causing damage rather than occupying any of the enemy's land.
 \textbf{Raid} is also a noun .
 \textit{
	\begin{itemize}
	\item The guerrillas raided banks and destroyed a police barracks and an electricity substation.
	\item The rebels attempted a surprise raid on a military camp.
	\item Its planes are carrying out heavy bombing raids against the guerrillas.
	\end{itemize}
}
\item verb \\
If the police \textbf{raid} a building, they enter it suddenly and by force in order to look for dangerous criminals or for evidence of something illegal , such as drugs or weapons .
 \textbf{Raid} is also a noun.
 \textit{
	\begin{itemize}
	\item Fraud squad officers raided the firm's offices.
	\item They were arrested early this morning after a raid on a house by thirty armed police.
	\end{itemize}
}
\item verb \\
If someone \textbf{raids} a building or place, they enter it by force in order to steal something.
 \textbf{Raid} is also a noun.
 \textit{
	\begin{itemize}
	\item A 19-year-old man has been found guilty of raiding a bank.
	\item ...an armed raid on a small Post Office.
	\item He carried out a series of bank raids.
	\end{itemize}
}
\item verb \\
If you \textbf{raid} the fridge or the larder , you take food from it to eat  instead of a meal or in between meals.
 \textit{
	\begin{itemize}
	\item She made her way to the kitchen to raid the fridge.
	\end{itemize}
}
\end{enumerate}

\section*{offspring}
{\large \color{blue}  offspring  }
\subsection*{Explain}
\begin{enumerate}
\item countable noun \\
You can refer to a person's children or to an animal's young as their \textbf{offspring} .
 \textit{
	\begin{itemize}
	\item Eleanor was now less anxious about her offspring than she had once been.
	\end{itemize}
}
\end{enumerate}

\section*{siren}
{\large \color{blue}  sirens  }
\subsection*{Explain}
\begin{enumerate}
\item countable noun \\
A \textbf{siren} is a warning device which makes a long, loud noise . Most fire  engines , ambulances , and police  cars have sirens.
 \textit{
	\begin{itemize}
	\item It sounds like an air raid siren.
	\end{itemize}
}
\item countable noun \\
Some people refer to a woman as a \textbf{siren} when they think that she is attractive to men but dangerous in some way.
 \textit{
	\begin{itemize}
	\item He depicts her as a siren who has drawn him to his ruin.
	\item ...the voluptuous siren with a husky voice.
	\end{itemize}
}
\item  \\
 siren call/siren song \textit{
	\begin{itemize}
	\end{itemize}
}
\end{enumerate}

\section*{tank}
{\large \color{blue}  tanks  }
\subsection*{Explain}
\begin{enumerate}
\item countable noun \\
A \textbf{tank} is a large container for holding liquid or gas.
 A \textbf{tank} of a liquid or gas is an amount of it contained in a tank.
 \textit{
	\begin{itemize}
	\item ...an empty fuel tank.
	\item Two water tanks provide a total capacity of 400 litres.
	\item ...a tank full of goldfish.
	\item A lorry can drive up to 600 miles on a single tank of fuel.
	\end{itemize}
}
\item countable noun \\
A \textbf{tank} is a large military vehicle that is equipped with weapons and moves along on metal tracks that are fitted over the wheels .
 \textit{
	\begin{itemize}
	\end{itemize}
}
\end{enumerate}

\section*{peak}
{\large \color{blue}  peaks  peaking  peaked  }
\subsection*{Explain}
\begin{enumerate}
\item countable noun \\
The \textbf{peak} of a process or an activity is the point at which it is at its strongest, most successful , or most fully developed.
 \textit{
	\begin{itemize}
	\item The party's membership has fallen from a peak of fifty-thousand after the Second
World War.
	\item The bomb went off in a concrete dustbin at the peak of the morning rush hour.
	\item ...a flourishing career that was at its peak at the time of his death.
	\item Economies have peaks and troughs.
	\end{itemize}
}
\item verb \\
When something \textbf{peaks} , it reaches its highest value or its highest level.
 \textit{
	\begin{itemize}
	\item Temperatures have peaked at over thirty degrees Celsius.
	\item The crisis peaked in July 1974.
	\item His career peaked during the 1970s.
	\end{itemize}
}
\item adjective \\
The \textbf{peak} level or value of something is its highest level or value.
 \textit{
	\begin{itemize}
	\item Calls cost 36p (cheap rate) and 48p (peak rate) per minute.
	\item We bought it at the wrong time and paid the peak price.
	\end{itemize}
}
\item adjective \\
\textbf{Peak} times are the times when there is most demand for something or most use of something.
 \textit{
	\begin{itemize}
	\item It's always crowded at peak times.
	\item During peak periods, reservations are difficult to make at some of the hotels.
	\end{itemize}
}
\item countable noun \\
A \textbf{peak} is a mountain or the top of a mountain.
 \textit{
	\begin{itemize}
	\item ...the snow-covered peaks.
	\end{itemize}
}
\item countable noun \\
The \textbf{peak} of a cap is the part at the front that sticks out above your eyes.
 \textit{
	\begin{itemize}
	\item The man touched the peak of his cap.
	\end{itemize}
}
\end{enumerate}

\section*{porch}
{\large \color{blue}  porches  }
\subsection*{Explain}
\begin{enumerate}
\item countable noun \\
A \textbf{porch} is a sheltered area at the entrance to a building . It has a roof and sometimes has walls .
 \textit{
	\begin{itemize}
	\item Is there a light in the porch or garden?
	\end{itemize}
}
\item countable noun \\
A \textbf{porch} is a raised  platform  built along the outside wall of a house and often covered with a roof.
 \textit{
	\begin{itemize}
	\item We'd eat during the hot summer evenings on the front porch.
	\end{itemize}
}
\end{enumerate}

\section*{thirteen}
{\large \color{blue}  thirteens  }
\subsection*{Explain}
\begin{enumerate}
\item number \\
\textbf{Thirteen} is the number 13.
 \textit{
	\begin{itemize}
	\item Thirteen people died in the accident.
	\end{itemize}
}
\end{enumerate}

\section*{queen}
{\large \color{blue}  queens  }
\subsection*{Explain}
\begin{enumerate}
\item title noun \\
A \textbf{queen} is a woman who rules a country as its monarch .
 \textit{
	\begin{itemize}
	\item ...Queen Victoria.
	\item She met the Queen last week.
	\end{itemize}
}
\item title noun \\
A \textbf{queen} is a woman who is married to a king.
 \textit{
	\begin{itemize}
	\item The king and queen had fled.
	\end{itemize}
}
\item countable noun \\
If you refer to a woman as \textbf{the}  \textbf{queen}  \textbf{of} a particular  activity , you mean that she is well-known for being very good at it.
 \textit{
	\begin{itemize}
	\item ...the queen of crime writing.
	\end{itemize}
}
\item countable noun \\
A \textbf{queen} is a male homosexual who dresses and speaks  rather  like a woman.
 \textit{
	\begin{itemize}
	\end{itemize}
}
\item countable noun \\
In chess, the \textbf{queen} is the most powerful piece. It can be moved in any direction.
 \textit{
	\begin{itemize}
	\end{itemize}
}
\item countable noun \\
A \textbf{queen} is a playing card with a picture of a queen on it.
 \textit{
	\begin{itemize}
	\item ...the queen of spades.
	\end{itemize}
}
\item countable noun \\
A \textbf{queen} or a \textbf{queen bee} is a large female bee which can lay eggs.
 \textit{
	\begin{itemize}
	\end{itemize}
}
\end{enumerate}

\section*{thousand}
{\large \color{blue}  thousands  }
\subsection*{Explain}
\begin{enumerate}
\item number \\
\textbf{A}  \textbf{thousand} or \textbf{one}  \textbf{thousand} is the number 1,000.
 \textit{
	\begin{itemize}
	\item ...five thousand acres.
	\item Visitors can expect to pay about a thousand pounds a day.
	\end{itemize}
}
\item quantifier \\
If you refer to \textbf{thousands of} things or people, you are emphasizing that there are very many of them.
 You can also use \textbf{thousands} as a pronoun.
 \textit{
	\begin{itemize}
	\item Thousands of refugees are packed into over-crowded towns and villages.
	\item I must have driven past that place thousands of times.
	\item Hundreds have been killed in the fighting and thousands made homeless.
	\end{itemize}
}
\end{enumerate}

\section*{quiz}
{\large \color{blue}  quizzes  quizzing  quizzed  }
\subsection*{Explain}
\begin{enumerate}
\item countable noun \\
A \textbf{quiz} is a game or competition in which someone tests your knowledge by asking you questions.
 \textit{
	\begin{itemize}
	\item We'll have a quiz at the end of the show.
	\end{itemize}
}
\item verb \\
If you \textbf{are quizzed} by someone about something, they ask you questions about it.
 \textit{
	\begin{itemize}
	\item He was quizzed about his income, debts and eligibility for state benefits.
	\item Sybil quizzed her about life as a working girl.
	\end{itemize}
}
\end{enumerate}

\section*{tin}
{\large \color{blue}  tins  }
\subsection*{Explain}
\begin{enumerate}
\item uncountable noun \\
\textbf{Tin} is a soft silvery-white metal.
 \textit{
	\begin{itemize}
	\item ...a factory that turns scrap metal into tin cans.
	\item ...a tin-roofed hut.
	\end{itemize}
}
\item countable noun \\
A \textbf{tin} is a metal container which is filled with food and sealed in order to preserve the food for long periods of time.
 A \textbf{tin}  \textbf{of} food is the amount of food contained in a tin.
 \textit{
	\begin{itemize}
	\item She popped out to buy a tin of soup.
	\item He had survived by eating a small tin of fruit every day.
	\end{itemize}
}
\item countable noun \\
A \textbf{tin} is a metal container with a lid in which things such as biscuits , cakes , or tobacco can be kept .
 A \textbf{tin}  \textbf{of} something is the amount contained in a tin.
 \textit{
	\begin{itemize}
	\item Store the cookies in an airtight tin.
	\item He reached for a tin of tobacco on the shelf behind him.
	\item They emptied out the remains of the tin of paint and smeared it on the inside of
the van.
	\end{itemize}
}
\item countable noun \\
A baking  \textbf{tin} is a metal container used for baking things such as cakes and bread in an oven .
 \textit{
	\begin{itemize}
	\item Pour the mixture into the cake tin and bake for 45 minutes.
	\item ...a 2 lb loaf tin.
	\end{itemize}
}
\end{enumerate}

\section*{resume}
{\large \color{blue}  résumés  }
\subsection*{Explain}
\begin{enumerate}
\item countable noun \\
A \textbf{résumé} is a short account , either spoken or written , of something that has happened or that someone has said or written.
 \textit{
	\begin{itemize}
	\item I will leave with you a resumé of his most recent speech.
	\item We began each planning meeting with a résumé of how we were doing as a division.
	\end{itemize}
}
\item countable noun \\
Your \textbf{résumé} is a brief account of your personal  details , your education , and the jobs you have had. You are often asked to send a résumé when you are applying for a job .
 \textit{
	\begin{itemize}
	\end{itemize}
}
\end{enumerate}

\section*{translation}
{\large \color{blue}  translations  }
\subsection*{Explain}
\begin{enumerate}
\item countable noun \\
A \textbf{translation} is a piece of writing or speech that has been translated from a different language.
 \textit{
	\begin{itemize}
	\item ...MacNiece's excellent English translation of 'Faust'.
	\item I've only read Solzhenitsyn in translation.
	\end{itemize}
}
\item  \\
 lose sth/be lost in (the) translation \textit{
	\begin{itemize}
	\end{itemize}
}
\end{enumerate}

\section*{satellite}
{\large \color{blue}  satellites  }
\subsection*{Explain}
\begin{enumerate}
\item countable noun \\
A \textbf{satellite} is an object which has been sent into space in order to collect information or to be part of a communications system. Satellites move continually round the Earth or around another planet.
 \textit{
	\begin{itemize}
	\item The rocket launched two communications satellites.
	\item The signals are sent by satellite link.
	\item Worldwide coverage beamed by satellite generates huge audiences.
	\end{itemize}
}
\item adjective \\
\textbf{Satellite} television is broadcast using a satellite.
 \textit{
	\begin{itemize}
	\item They have four satellite channels.
	\end{itemize}
}
\item countable noun \\
A \textbf{satellite} is a natural object in space that moves round a planet or star.
 \textit{
	\begin{itemize}
	\item ...the satellites of Jupiter.
	\end{itemize}
}
\item countable noun \\
You can  refer to a country, area, or organization as a \textbf{satellite} when it is controlled by or depends on a larger and more powerful one.
 \textit{
	\begin{itemize}
	\item Italy became a satellite state of Germany by the end of the 1930s.
	\item ...Russia and its former satellites.
	\end{itemize}
}
\end{enumerate}

\section*{turnover}
{\large \color{blue}  turnovers  }
\subsection*{Explain}
\begin{enumerate}
\item variable noun \\
The \textbf{turnover} of a company is the value of the goods or services sold during a particular period of time.
 \textit{
	\begin{itemize}
	\item Her annual turnover is around £45,000.
	\item The company had a turnover of £3.8 million.
	\end{itemize}
}
\item variable noun \\
The \textbf{turnover} of people in an organization or place is the rate at which people leave and are replaced .
 \textit{
	\begin{itemize}
	\item Short-term contracts increase staff turnover.
	\item The industry has a high turnover of young people.
	\end{itemize}
}
\end{enumerate}

\section*{stream}
{\large \color{blue}  streams  streaming  streamed  }
\subsection*{Explain}
\begin{enumerate}
\item countable noun \\
A \textbf{stream} is a small narrow river.
 \textit{
	\begin{itemize}
	\item There was a small stream at the end of the garden.
	\item ...a mountain stream.
	\end{itemize}
}
\item countable noun \\
A \textbf{stream} of smoke , air, or liquid is a narrow moving mass of it.
 \textit{
	\begin{itemize}
	\item The window was open, letting in streams of cold sea air.
	\item Add the oil in a slow, steady stream.
	\end{itemize}
}
\item countable noun \\
A \textbf{stream} of vehicles or people is a long moving line of them.
 \textit{
	\begin{itemize}
	\item There was a stream of traffic behind him.
	\end{itemize}
}
\item countable noun \\
A \textbf{stream}  \textbf{of} things is a large number of them occurring one after another.
 \textit{
	\begin{itemize}
	\item The discovery triggered a stream of readers' letters.
	\item ...a never-ending stream of jokes.
	\item We had a constant stream of visitors.
	\end{itemize}
}
\item verb \\
If a liquid \textbf{streams}  somewhere , it flows or comes out in large amounts.
 \textit{
	\begin{itemize}
	\item Tears streamed down their faces.
	\item She came in, rain streaming from her clothes and hair.
	\end{itemize}
}
\item verb \\
If your eyes  \textbf{are streaming} , liquid is coming from them, for example because you have a cold . You can also  say that your nose  \textbf{is streaming} .
 \textit{
	\begin{itemize}
	\item Her eyes were streaming now from the wind.
	\item A cold usually starts with a streaming nose and dry throat.
	\end{itemize}
}
\item verb \\
If people or vehicles \textbf{stream} somewhere, they move there quickly and in large numbers .
 \textit{
	\begin{itemize}
	\item Hundreds of people were streaming into the banquet room.
	\item The traffic streamed past him.
	\item The clock in the church struck twelve, and soon after people began to stream out.
	\end{itemize}
}
\item verb \\
When light \textbf{streams} into or out of a place, it shines strongly into or out of it.
 \textit{
	\begin{itemize}
	\item Sunlight was streaming into the courtyard.
	\end{itemize}
}
\item verb \\
If something such as a flag or someone's hair \textbf{streams} in the wind , it is blown so that it is almost  horizontal .
 \textit{
	\begin{itemize}
	\item She was wearing a flimsy pink dress that streamed out behind her.
	\item He had been greeted by the sight of his mother, her red hair wildly streaming.
	\end{itemize}
}
\item verb \\
If you \textbf{stream} music, films, or television  programmes , you play them directly from the internet .
 \textit{
	\begin{itemize}
	\item A smart TV gives you the ability to stream TV shows and movies on demand.
	\item You can stream music to your stereo system from your mobile phone.
	\end{itemize}
}
\item countable noun \\
In a school, a \textbf{stream} is a group of children of the same age and ability who are taught together.
 \textit{
	\begin{itemize}
	\item Examinations may be used to choose which pupils are to move into the top streams.
	\end{itemize}
}
\item verb \\
To \textbf{stream}  pupils means to divide them into groups according to their ability.
 \textit{
	\begin{itemize}
	\item He advocates streaming children, and educating them according to their needs.
	\end{itemize}
}
\item  \\
 on stream \textit{
	\begin{itemize}
	\end{itemize}
}
\end{enumerate}

\section*{twelve}
{\large \color{blue}  twelves  }
\subsection*{Explain}
\begin{enumerate}
\item number \\
\textbf{Twelve} is the number 12.
 \textit{
	\begin{itemize}
	\item Twelve days later Duffy lost his job.
	\end{itemize}
}
\end{enumerate}

\section*{threshold}
{\large \color{blue}  thresholds  }
\subsection*{Explain}
\begin{enumerate}
\item countable noun \\
The \textbf{threshold} of a building or room is the floor in the doorway, or the doorway itself.
 \textit{
	\begin{itemize}
	\item He stopped at the threshold of the bedroom.
	\item The bride was carried over the threshold.
	\end{itemize}
}
\item countable noun \\
A \textbf{threshold} is an amount, level, or limit on a scale . When the \textbf{threshold} is reached , something else happens or changes.
 \textit{
	\begin{itemize}
	\item She has a low threshold of boredom and needs the constant stimulation of physical
activity.
	\item The consensus has clearly shifted in favour of raising the nuclear threshold.
	\item Fewer than forty per cent voted–the threshold for results to be valid.
	\end{itemize}
}
\item  \\
 on the threshold of \textit{
	\begin{itemize}
	\end{itemize}
}
\end{enumerate}

\section*{twenty}
{\large \color{blue}  twenties  }
\subsection*{Explain}
\begin{enumerate}
\item number \\
\textbf{Twenty} is the number 20.
 \textit{
	\begin{itemize}
	\item He spent twenty years in India.
	\end{itemize}
}
\item plural noun \\
When you talk about the \textbf{twenties} , you are referring to numbers between 20 and 29. For example , if you are \textbf{in} your \textbf{twenties} , you are aged between 20 and 29. If the temperature is \textbf{in the twenties} , the temperature is between 20 and 29 degrees .
 \textit{
	\begin{itemize}
	\item They're both in their twenties and both married with children of their own.
	\end{itemize}
}
\item plural noun \\
\textbf{The twenties} is the decade between 1920 and 1929.
 \textit{
	\begin{itemize}
	\item It was written in the Twenties, but it still really stands out.
	\end{itemize}
}
\end{enumerate}

\section*{violin}
{\large \color{blue}  violins  }
\subsection*{Explain}
\begin{enumerate}
\item variable noun \\
A \textbf{violin} is a musical instrument. Violins are made of wood and have four  strings . You play the violin by holding it under your chin and moving a bow across the strings.
 \textit{
	\begin{itemize}
	\item Lizzie used to play the violin.
	\item ...the Brahms violin concerto in D.
	\end{itemize}
}
\end{enumerate}

\section*{two}
{\large \color{blue}  twos  }
\subsection*{Explain}
\begin{enumerate}
\item number \\
\textbf{Two} is the number 2.
 \textit{
	\begin{itemize}
	\item He is now married with two children.
	\end{itemize}
}
\item  \\
 it takes two to tango \textit{
	\begin{itemize}
	\end{itemize}
}
\item  \\
 put two and two together \textit{
	\begin{itemize}
	\end{itemize}
}
\end{enumerate}

\section*{whistle}
{\large \color{blue}  whistles  whistling  whistled  }
\subsection*{Explain}
\begin{enumerate}
\item verb \\
When you \textbf{whistle} or when you \textbf{whistle} a tune , you make a series of musical notes by forcing your breath out between your lips, or your teeth .
 \textit{
	\begin{itemize}
	\item He whistled and sang snatches of songs.
	\item He was whistling softly to himself.
	\item As he washed, he whistled a tune.
	\end{itemize}
}
\item verb \\
When someone \textbf{whistles} , they make a sound by forcing their breath out between their lips or their teeth.
People sometimes whistle when they are surprised or shocked , or to call a dog , or to show that they are impressed .
 \textbf{Whistle} is also a noun .
 \textit{
	\begin{itemize}
	\item He whistled, surprised but not shocked.
	\item Jenkins whistled through his teeth, impressed at last.
	\item Women don't enjoy being whistled at.
	\item Jackson gave a low whistle.
	\end{itemize}
}
\item verb \\
If something such as a train or a kettle \textbf{whistles} , it makes a loud , high sound.
 \textit{
	\begin{itemize}
	\item Somewhere a train whistled.
	\item ...the whistling car radio.
	\end{itemize}
}
\item verb \\
If something such as the wind or a bullet \textbf{whistles}  somewhere , it moves there, making a loud, high sound.
 \textit{
	\begin{itemize}
	\item The wind was whistling through the building.
	\item As I stood up a bullet whistled past my back.
	\end{itemize}
}
\item countable noun \\
A \textbf{whistle} is a loud sound produced by air or steam being forced through a small opening , or by something moving quickly through the air.
 \textit{
	\begin{itemize}
	\item Hugh listened to the whistle of a train.
	\item ...the whistle of the wind.
	\item ...a shrill whistle from the boiling kettle.
	\end{itemize}
}
\item countable noun \\
A \textbf{whistle} is a small metal  tube which you blow in order to produce a loud sound and attract someone's attention .
 \textit{
	\begin{itemize}
	\item On the platform, the guard blew his whistle.
	\item The referee blew his whistle for a penalty.
	\end{itemize}
}
\item countable noun \\
Some factories and other places where people work have a \textbf{whistle} which signals the beginning and the end of the working  day .
 \textit{
	\begin{itemize}
	\item Every night you could hear the whistles of the steel mill.
	\end{itemize}
}
\item countable noun \\
A \textbf{whistle} is a simple musical instrument in the shape of a metal pipe with holes . You play the whistle by blowing into it.
 \textit{
	\begin{itemize}
	\end{itemize}
}
\item  \\
 to blow the whistle \textit{
	\begin{itemize}
	\end{itemize}
}
\item  \\
 sb can whistle for sth \textit{
	\begin{itemize}
	\end{itemize}
}
\item  \\
 clean as a whistle \textit{
	\begin{itemize}
	\end{itemize}
}
\item  \\
 clean as a whistle \textit{
	\begin{itemize}
	\end{itemize}
}
\item  \\
 whistling in the dark \textit{
	\begin{itemize}
	\end{itemize}
}
\item  \\
 to wet your whistle \textit{
	\begin{itemize}
	\end{itemize}
}
\item  \\
 whistle in the wind \textit{
	\begin{itemize}
	\end{itemize}
}
\end{enumerate}

\section*{waterfall}
{\large \color{blue}  waterfalls  }
\subsection*{Explain}
\begin{enumerate}
\item countable noun \\
A \textbf{waterfall} is a place where water flows over the edge of a steep , high cliff in hills or mountains , and falls into a pool below.
 \textit{
	\begin{itemize}
	\item ...Angel Falls, the world's highest waterfall.
	\end{itemize}
}
\end{enumerate}

\section*{whip}
{\large \color{blue}  whips  whipping  whipped  }
\subsection*{Explain}
\begin{enumerate}
\item countable noun \\
A \textbf{whip} is a long thin piece of material such as leather or rope, fastened to a stiff handle. It is used for hitting people or animals.
 \textit{
	\begin{itemize}
	\end{itemize}
}
\item verb \\
If someone \textbf{whips} a person or animal, they beat them or hit them with a whip or something like a whip.
 \textit{
	\begin{itemize}
	\item Eye-witnesses claimed Mr Melton whipped the horse up to 16 times.
	\end{itemize}
}
\item verb \\
If something, for example the wind, \textbf{whips} something, it strikes it sharply.
 \textit{
	\begin{itemize}
	\item A terrible wind whipped our faces.
	\item A branch whipped her across the cheek.
	\end{itemize}
}
\item verb \\
If something flexible \textbf{whips} in a particular way, it moves sharply when it is affected by a force, for example
by the wind.
 \textit{
	\begin{itemize}
	\item Blond strands of hair whipped in the wind.
	\end{itemize}
}
\item verb \\
If someone \textbf{whips} something out or \textbf{whips} it off, they take it out or take it off very quickly and suddenly .
 \textit{
	\begin{itemize}
	\item Bob whipped out his notebook.
	\item Players were whipping their shirts off.
	\item My waitress whipped the plate away and put down my bill.
	\end{itemize}
}
\item verb \\
If something or someone \textbf{whips}  somewhere , they move there or go there very quickly.
 \textit{
	\begin{itemize}
	\item The wind out here is whipping along at about 30 miles an hour.
	\item I whipped into a parking space.
	\end{itemize}
}
\item verb \\
When you \textbf{whip} something liquid such as cream or an egg, you stir it very fast until it is thick or stiff.
 \textit{
	\begin{itemize}
	\item Whip the cream until thick.
	\item Whip the eggs, oils and honey together.
	\item ...strawberries and whipped cream.
	\end{itemize}
}
\item verb \\
If you \textbf{whip} people \textbf{into} an emotional state, you deliberately cause and encourage them to be in that state.
 \textit{
	\begin{itemize}
	\item He could whip a crowd into hysteria.
	\item Politicians and businessmen have whipped themselves into a panic.
	\end{itemize}
}
\item countable noun \\
A \textbf{whip} is a member of a political party in a parliament or legislature who is responsible for making sure that party members are present to vote on important issues and that they vote in
the appropriate way.
 \textit{
	\begin{itemize}
	\item The whips have the job of making sure MPs toe the line.
	\end{itemize}
}
\item countable noun \\
A \textbf{whip} is a notice which tells the members of a political party in parliament that it is important for them to vote
in a particular way on an important issue.
 \textit{
	\begin{itemize}
	\end{itemize}
}
\item  \\
 the whip hand \textit{
	\begin{itemize}
	\end{itemize}
}
\end{enumerate}

\section*{abundance}
{\large \color{blue}  }
\subsection*{Explain}
\begin{enumerate}
\item singular noun \\
An \textbf{abundance}  \textbf{of} something is a large quantity of it.
 \textit{
	\begin{itemize}
	\item The area has an abundance of wildlife.
	\item Food was in abundance.
	\end{itemize}
}
\end{enumerate}

\section*{actress}
{\large \color{blue}  actresses  }
\subsection*{Explain}
\begin{enumerate}
\item countable noun \\
An \textbf{actress} is a woman whose job is acting in plays or films.
 \textit{
	\begin{itemize}
	\item She's a very great dramatic actress.
	\end{itemize}
}
\end{enumerate}

\section*{against}
{\large \color{blue}  }
\subsection*{Explain}
\begin{enumerate}
\item preposition \\
If one thing is leaning or pressing  \textbf{against} another, it is touching it.
 \textit{
	\begin{itemize}
	\item She leaned against him.
	\item On a table pushed against a wall there were bottles of water.
	\item ...the rain beating against the window panes.
	\end{itemize}
}
\item preposition \\
If you are \textbf{against} something such as a plan, policy, or system, you think it is wrong , bad , or stupid .
 \textbf{Against} is also an adverb .
 \textit{
	\begin{itemize}
	\item Taxes are unpopular–it is understandable that voters are against them.
	\item Joan was very much against commencing drug treatment.
	\item ...a march to protest against job losses.
	\item The vote for the suspension of the party was 283 in favour with 29 against.
	\end{itemize}
}
\item preposition \\
If you compete  \textbf{against} someone in a game, you try to beat them.
 \textit{
	\begin{itemize}
	\item The tour will include games against the Australian Barbarians.
	\item He had rescheduled his fight against the British champion.
	\end{itemize}
}
\item preposition \\
If you take action \textbf{against} someone or something, you try to harm them.
 \textit{
	\begin{itemize}
	\item Security forces are still using violence against opponents of the government.
	\item ...demonstrations against the war.
	\end{itemize}
}
\item preposition \\
If you take action \textbf{against} a possible  future event, you try to prevent it.
 \textit{
	\begin{itemize}
	\item ...the fight against crime.
	\item They are arguing against hospital closures.
	\item I must warn you against raising your hopes.
	\end{itemize}
}
\item preposition \\
If you do something \textbf{against} someone's wishes , advice , or orders, you do not do what they want you to do or tell you to do.
 \textit{
	\begin{itemize}
	\item He didn't want to go against the wishes of the government.
	\item He discharged himself from hospital against the advice of doctors.
	\end{itemize}
}
\item preposition \\
If you do something in order to protect yourself \textbf{against} something unpleasant or harmful , you do something which will make its effects on you less serious if it happens .
 \textit{
	\begin{itemize}
	\item A business needs insurance against risks such as fire and flood.
	\item It has been claimed that fish oils protect against heart disease.
	\end{itemize}
}
\item  \\
 have sth against sb \textit{
	\begin{itemize}
	\end{itemize}
}
\item preposition \\
If something is \textbf{against} the law or \textbf{against} the rules, there is a law or a rule which says that you must not do it.
 \textit{
	\begin{itemize}
	\item It is against the law to detain you against your will for any length of time.
	\item We thought cheating was against the rules.
	\end{itemize}
}
\item preposition \\
If you are moving \textbf{against} a current, tide , or wind, you are moving in the opposite direction to it.
 \textit{
	\begin{itemize}
	\item ...swimming upstream against the current.
	\item They were going to sail around the little island, against the tide.
	\end{itemize}
}
\item preposition \\
If something happens or is considered \textbf{against} a particular background of events, it is considered in relation to those events, because those events are relevant to it.
 \textit{
	\begin{itemize}
	\item The profits rise was achieved against a backdrop of falling metal prices.
	\end{itemize}
}
\item preposition \\
If something is measured or valued \textbf{against} something else, it is measured or valued by comparing it with the other thing.
 \textit{
	\begin{itemize}
	\item Our policy has to be judged against a clear test: will it improve the standard of
education?
	\item Check the operator's productivity against agreed targets.
	\item The U.S. dollar is down against most foreign currencies today.
	\end{itemize}
}
\item  \\
 as against \textit{
	\begin{itemize}
	\end{itemize}
}
\item preposition \\
The odds  \textbf{against} something happening are the chances or odds that it will not happen.
 \textbf{Against} is also an adverb.
 \textit{
	\begin{itemize}
	\item The odds against him surviving are incredible.
	\item What were the odds against?
	\end{itemize}
}
\end{enumerate}

\section*{aircraft}
{\large \color{blue}  aircraft  }
\subsection*{Explain}
\begin{enumerate}
\item countable noun \\
An \textbf{aircraft} is a vehicle which can fly, for example an aeroplane or a helicopter.
 \textit{
	\begin{itemize}
	\item The return flight of the aircraft was delayed.
	\item At least three military aircraft were destroyed.
	\end{itemize}
}
\end{enumerate}

\section*{among}
{\large \color{blue}  }
\subsection*{Explain}
\begin{enumerate}
\item preposition \\
Someone or something that is situated or moving  \textbf{among} a group of things or people is surrounded by them.
 \textit{
	\begin{itemize}
	\item ...youths in their late teens sitting among adults.
	\item They walked among the crowds in Red Square.
	\item ...a little house among the trees.
	\end{itemize}
}
\item preposition \\
If you are \textbf{among} people of a particular  kind , you are with them and having contact with them.
 \textit{
	\begin{itemize}
	\item Things weren't so bad, after all. I was among friends again.
	\item I was brought up among people who read and wrote a lot.
	\end{itemize}
}
\item preposition \\
If someone or something is \textbf{among} a group, they are a member of that group and share its characteristics .
 \textit{
	\begin{itemize}
	\item A fifteen year old girl was among the injured.
	\item Also among the speakers was the new American ambassador to Moscow.
	\end{itemize}
}
\item preposition \\
If you want to focus on something that is happening within a particular group of people, you can  say that it is happening \textbf{among} that group.
 \textit{
	\begin{itemize}
	\item Unemployment is quite high, especially among young people.
	\end{itemize}
}
\item preposition \\
If something happens  \textbf{among} a group of people, it happens within the whole of that group or between the members of that group.
 \textit{
	\begin{itemize}
	\item Intense debate continues among the leadership.
	\item I am sick of all the quarrelling among politicians who should be concentrating on
vital issues.
	\end{itemize}
}
\item preposition \\
If something such as a feeling , opinion , or situation  exists  \textbf{among} a group of people, most of them have it or experience it.
 \textit{
	\begin{itemize}
	\item The biggest fear among parents thinking of using the Internet is that their children
will be exposed to pornography.
	\item The resort is popular among ski enthusiasts.
	\end{itemize}
}
\item preposition \\
You use \textbf{among} before a noun to mention a group when talking about a smaller group within it.
 \textit{
	\begin{itemize}
	\item Among those 18 and over, 510,000 benefit claimants were not unemployed.
	\item Among the varieties available, my preference stays with the old and lovely pink-flowered
variety, 'Apple Blossom'.
	\end{itemize}
}
\item preposition \\
If something applies to a particular person or thing \textbf{among}  \textbf{others} , it also applies to other people or things.
 \textit{
	\begin{itemize}
	\item ...a news conference attended among others by our foreign affairs correspondent.
	\item If you are unhappy with the present situation you can contact, among others, your
local MP.
	\end{itemize}
}
\item preposition \\
If something is shared \textbf{among} a number of people, some of it is given to all of them.
 \textit{
	\begin{itemize}
	\item Most of the furniture was left to the neighbours or distributed among friends.
	\item She tried to ensure her affection was equally shared among all three children.
	\end{itemize}
}
\item preposition \\
If people talk, fight , or agree  \textbf{among}  \textbf{themselves} , they do it together , without involving anyone else.
 \textit{
	\begin{itemize}
	\item European farm ministers disagree among themselves.
	\item The directors have been arguing among themselves.
	\end{itemize}
}
\end{enumerate}

\section*{angel}
{\large \color{blue}  angels  }
\subsection*{Explain}
\begin{enumerate}
\item countable noun \\
\textbf{Angels} are spiritual beings that some people believe are God's servants in heaven .
 \textit{
	\begin{itemize}
	\end{itemize}
}
\item countable noun \\
You can call someone you like very much an \textbf{angel} in order to show  affection , especially when they have been kind to you or done you a favour .
 \textit{
	\begin{itemize}
	\item Thank you a thousand times, you're an angel.
	\end{itemize}
}
\item countable noun \\
If you describe someone as an \textbf{angel} , you mean that they seem to be very kind and good.
 \textit{
	\begin{itemize}
	\item Poppa thought her an angel.
	\item He was such an angel to put up with it.
	\end{itemize}
}
\item  \\
 on the side of the angels \textit{
	\begin{itemize}
	\end{itemize}
}
\item  \\
 rush in where angels fear to tread \textit{
	\begin{itemize}
	\end{itemize}
}
\end{enumerate}

\section*{astronomy}
{\large \color{blue}  }
\subsection*{Explain}
\begin{enumerate}
\item uncountable noun \\
\textbf{Astronomy} is the scientific study of the stars , planets , and other natural objects in space .
 \textit{
	\begin{itemize}
	\end{itemize}
}
\end{enumerate}

\section*{before}
{\large \color{blue}  }
\subsection*{Explain}
\begin{enumerate}
\item preposition \\
If something happens  \textbf{before} a particular date , time, or event , it happens earlier than that date, time, or event.
 \textbf{Before} is also a conjunction .
 \textit{
	\begin{itemize}
	\item Annie was born a few weeks before Christmas.
	\item Before World War II, women were not recruited as intelligence officers.
	\item Dan rarely comes to bed before 2 or 3am.
	\item Stock prices climbed close to the peak they'd registered before the stock market
crashed.
	\end{itemize}
}
\item preposition \\
If you do one thing \textbf{before} doing something else, you do it earlier than the other thing.
 \textbf{Before} is also a conjunction.
 \textit{
	\begin{itemize}
	\item He spent his early life in Sri Lanka before moving to England.
	\item Before leaving, he went into his office to fill in the daily time sheet.
	\item He took a cold shower and then towelled off before he put on fresh clothes.
	\end{itemize}
}
\item adverb \\
You use \textbf{before} when you are talking about time. For example , if something happened the day  \textbf{before} a particular date or event, it happened during the previous day.
 \textbf{Before} is also a preposition .
 \textbf{Before} is also a conjunction.
 \textit{
	\begin{itemize}
	\item The war had ended only a month or so before.
	\item It's interesting that he sent me the book twenty days before the deadline for my
book.
	\item Kelman had a book published in the U.S. more than a decade before a British publisher
would touch him.
	\end{itemize}
}
\item conjunction \\
If you do something \textbf{before} someone else can do something, you do it when they have not yet done it.
 \textit{
	\begin{itemize}
	\item Before he could take another one, she laid her fingertips on his mouth.
	\end{itemize}
}
\item adverb \\
If someone has done something \textbf{before} , they have done it on a previous occasion . If someone has not done something \textbf{before} , they have never done it.
 \textit{
	\begin{itemize}
	\item I've been here before.
	\item I had met Professor Lown before.
	\item She had never been to Italy before.
	\end{itemize}
}
\item conjunction \\
If there is a period of time or if several things are done \textbf{before} something happens, it takes that amount of time or effort for this thing to happen.
 \textit{
	\begin{itemize}
	\item It was some time before the door opened in response to his ring.
	\end{itemize}
}
\item conjunction \\
If a particular situation has to happen \textbf{before} something else happens, this situation must happen or exist in order for the other thing to happen.
 \textit{
	\begin{itemize}
	\item There was additional work to be done before all the troops would be ready.
	\end{itemize}
}
\item preposition \\
If someone is \textbf{before} something, they are in front of it.
 \textit{
	\begin{itemize}
	\item They drove through a tall iron gate and stopped before a large white villa.
	\end{itemize}
}
\item preposition \\
If you tell someone that one place is a certain distance  \textbf{before} another, you mean that they will  come to the first place first.
 \textit{
	\begin{itemize}
	\item The turn is about two kilometres before the roundabout.
	\end{itemize}
}
\item preposition \\
If you appear or come \textbf{before} an official person or group, you go there and answer  questions .
 \textit{
	\begin{itemize}
	\item The Governor will appear before the committee next Tuesday.
	\end{itemize}
}
\item preposition \\
If something happens \textbf{before} a particular person or group, it is seen by or happens while this person or this group is present .
 \textit{
	\begin{itemize}
	\item The game followed a colourful opening ceremony before a crowd of seventy-four thousand.
	\end{itemize}
}
\item preposition \\
If you have something such as a journey , a task , or a stage of your life \textbf{before} you, you must do it or live through it in the future .
 \textit{
	\begin{itemize}
	\item Everyone in the room knew it was the single hardest task before them.
	\item I saw before me an idyllic life.
	\end{itemize}
}
\item preposition \\
When you want to say that one person or thing is more important than another, you can say that they come \textbf{before} the other person or thing.
 \textit{
	\begin{itemize}
	\item Life is still a juggling act, but my children come before anything else.
	\end{itemize}
}
\end{enumerate}

\section*{beginning}
{\large \color{blue}  beginnings  }
\subsection*{Explain}
\begin{enumerate}
\item countable noun \\
The \textbf{beginning}  \textbf{of} an event or process is the first part of it.
 \textit{
	\begin{itemize}
	\item This was also the beginning of her recording career.
	\item Think of this as a new beginning.
	\end{itemize}
}
\item plural noun \\
\textbf{The}  \textbf{beginnings}  \textbf{of} something are the signs or events which form the first part of it.
 \textit{
	\begin{itemize}
	\item I had the beginnings of a headache.
	\item The discussions were the beginnings of a dialogue with Moscow.
	\end{itemize}
}
\item singular noun \\
\textbf{The}  \textbf{beginning}  \textbf{of} a period of time is the time at which it starts.
 \textit{
	\begin{itemize}
	\item The wedding will be at the beginning of March.
	\end{itemize}
}
\item countable noun \\
The \textbf{beginning}  \textbf{of} a piece of written material is the first words or sentences of it.
 \textit{
	\begin{itemize}
	\item ...the question which was raised at the beginning of this chapter.
	\end{itemize}
}
\item plural noun \\
If you talk about the \textbf{beginnings} of a person, company , or group, you are referring to their backgrounds or origins.
 \textit{
	\begin{itemize}
	\item His views come from his own humble beginnings.
	\end{itemize}
}
\item adjective \\
You use \textbf{beginning} to describe someone who is in the early stages of learning to do something.
 \textit{
	\begin{itemize}
	\item The people that she had in her classroom were beginning learners.
	\end{itemize}
}
\end{enumerate}

\section*{behind}
{\large \color{blue}  }
\subsection*{Explain}
\begin{enumerate}
\item preposition \\
If something is \textbf{behind} a thing or person, it is on the other side of them from you, or nearer their back
rather than their front .
 \textbf{Behind} is also an adverb .
 \textit{
	\begin{itemize}
	\item I put one of the cushions behind his head.
	\item They were parked behind the truck.
	\item The moon disappeared behind a cloud.
	\item Rising into the hills behind are 800 acres of parkland.
	\item She was attacked from behind.
	\end{itemize}
}
\item preposition \\
If you are walking or travelling  \textbf{behind} someone or something, you are following them.
 \textbf{Behind} is also an adverb.
 \textit{
	\begin{itemize}
	\item Keith wandered along behind him.
	\item Myra and Sam and the children were driving behind them.
	\item The troopers followed behind, every muscle tensed for the sudden gunfire.
	\end{itemize}
}
\item preposition \\
If someone is \textbf{behind} a desk , counter , or bar , they are on the other side of it from where you are.
 \textit{
	\begin{itemize}
	\item The colonel was sitting behind a cheap wooden desk.
	\item He could just about see the little man behind the counter.
	\end{itemize}
}
\item preposition \\
When you shut a door or gate  \textbf{behind} you, you shut it after you have gone through it.
 \textit{
	\begin{itemize}
	\item I walked out and closed the door behind me.
	\item He slammed the gate shut behind him.
	\end{itemize}
}
\item preposition \\
The people, reason , or events \textbf{behind} a situation are the causes of it or are responsible for it.
 \textit{
	\begin{itemize}
	\item It is still not clear who was behind the killing.
	\item He is embarrassed about the motives behind his decision.
	\end{itemize}
}
\item preposition \\
If something or someone is \textbf{behind} you, they support you and help you.
 \textit{
	\begin{itemize}
	\item He had the state's judicial power behind him.
	\end{itemize}
}
\item preposition \\
If you refer to what is \textbf{behind} someone's outside appearance , you are referring to a characteristic which you cannot immediately  see or is not obvious , but which you think is there.
 \textit{
	\begin{itemize}
	\item What lay behind his anger was really the hurt he felt at Grace's refusal.
	\item Behind the smiling eyes lurks the evil that led her to murder the two babies in her
care.
	\end{itemize}
}
\item preposition \\
If you are \textbf{behind} someone, you are less successful than them, or have done less or advanced less.
 \textbf{Behind} is also an adverb.
 \textit{
	\begin{itemize}
	\item She finished second behind the American in the 800 metres.
	\item Food production has already fallen behind the population growth.
	\item The rapid development of technology means that she is now far behind, and will need
retraining.
	\item The accounts are more than three months behind.
	\end{itemize}
}
\item preposition \\
If an experience is \textbf{behind} you, it happened in your past and will not happen again, or no longer affects you.
 \textit{
	\begin{itemize}
	\item Maureen put the nightmare behind her.
	\item He will attempt to put behind him the misery of failing to win a medal in his individual
event.
	\end{itemize}
}
\item preposition \\
If you have a particular achievement  \textbf{behind} you, you have managed to reach this achievement, and other people consider it to be important or valuable .
 \textit{
	\begin{itemize}
	\item He has 20 years of loyal service to Barclays Bank behind him.
	\item Birgit was a popular actress with half a decade of filmmaking behind her.
	\end{itemize}
}
\item preposition \\
If something is \textbf{behind}  schedule , it is not as far advanced as people had planned . If someone is \textbf{behind} schedule, they are not progressing as quickly at something as they had planned.
 \textit{
	\begin{itemize}
	\item The work is 22 weeks behind schedule.
	\item We were two months behind schedule, and already over budget.
	\end{itemize}
}
\item adverb \\
If you stay  \textbf{behind} , you remain in a place after other people have gone.
 \textit{
	\begin{itemize}
	\item If they've disrupted a lesson or been rude to a member of staff, they have to stay
behind until six o'clock.
	\item About 1,200 personnel will remain behind to take care of the air base.
	\end{itemize}
}
\item adverb \\
If you leave something or someone \textbf{behind} , you do not take them with you when you go.
 \textit{
	\begin{itemize}
	\item The rebels fled into the mountains, leaving behind their weapons and supplies.
	\item They moved to Vancouver, leaving her father behind to run his business.
	\end{itemize}
}
\end{enumerate}

\section*{blouse}
{\large \color{blue}  blouses  }
\subsection*{Explain}
\begin{enumerate}
\item countable noun \\
A \textbf{blouse} is a kind of shirt worn by a girl or woman.
 \textit{
	\begin{itemize}
	\end{itemize}
}
\end{enumerate}

\section*{below}
{\large \color{blue}  }
\subsection*{Explain}
\begin{enumerate}
\item preposition \\
If something is \textbf{below} something else, it is in a lower position.
 \textbf{Below} is also an adverb .
 \textit{
	\begin{itemize}
	\item He appeared from the apartment directly below Leonard's.
	\item The path runs below a long brick wall.
	\item The sun had already sunk below the horizon.
	\item The boat dipped below the surface of the water.
	\item We climbed rather perilously down a rope-ladder to the boat below.
	\item ...a view to the street below.
	\item Spread out below was a great crowd.
	\end{itemize}
}
\item  \\
 below ground \textit{
	\begin{itemize}
	\end{itemize}
}
\item adverb \\
You use \textbf{below} in a piece of writing to refer to something that is mentioned later.
 \textit{
	\begin{itemize}
	\item Please contact me on the number below.
	\item For a step-by-step guide see below.
	\end{itemize}
}
\item preposition \\
If something is \textbf{below} a particular amount, rate , or level , it is less than that amount, rate, or level.
 \textbf{Below} is also an adverb.
 \textit{
	\begin{itemize}
	\item Night temperatures can drop below 15 degrees Celsius.
	\item The company agreed to keep overall price increases to 7.5 per cent below inflation.
	\item Rainfall has been below average.
	\item ...temperatures at zero or below.
	\end{itemize}
}
\item preposition \\
If someone is \textbf{below} you in an organization, they are lower in rank .
 \textit{
	\begin{itemize}
	\item Such people often experience less stress than those in the ranks immediately below
them.
	\item ...pay rises awarded to all white-collar staff below chief officer level.
	\end{itemize}
}
\end{enumerate}

\section*{ceiling}
{\large \color{blue}  ceilings  }
\subsection*{Explain}
\begin{enumerate}
\item countable noun \\
A \textbf{ceiling} is the horizontal surface that forms the top part or roof  inside a room.
 \textit{
	\begin{itemize}
	\item The rooms were spacious, with tall windows and high ceilings.
	\item The study was lined from floor to ceiling on every wall with bookcases.
	\end{itemize}
}
\item countable noun \\
A \textbf{ceiling}  \textbf{on} something such as prices or wages is an official upper limit that cannot be broken .
 \textit{
	\begin{itemize}
	\item ...an informal agreement to put a ceiling on salaries.
	\item The agreement sets the ceiling of twenty-two-point-five million barrels a day on
OPEC production.
	\end{itemize}
}
\item countable noun \\
A \textbf{ceiling} is the greatest  height at which a particular aircraft can fly safely.
 \textit{
	\begin{itemize}
	\end{itemize}
}
\end{enumerate}

\section*{beneath}
{\large \color{blue}  }
\subsection*{Explain}
\begin{enumerate}
\item preposition \\
Something that is \textbf{beneath} another thing is under the other thing.
 \textbf{Beneath} is also an adverb .
 \textit{
	\begin{itemize}
	\item She could see the muscles of his shoulders beneath his T-shirt.
	\item She found pleasure in sitting beneath the trees.
	\item Four storeys of parking beneath the theatre was not enough.
	\item ...the frozen grass crunching beneath his feet.
	\item On a shelf beneath he spotted a photo album.
	\item ...aeroplanes roaring above, subways rattling beneath.
	\end{itemize}
}
\item preposition \\
If you talk about what is \textbf{beneath} the surface of something, you are talking about the aspects of it which are hidden or not obvious .
 \textit{
	\begin{itemize}
	\item ...emotional strains beneath the surface.
	\item Somewhere deep beneath the surface lay a caring character.
	\item Beneath the festive mood there is an underlying apprehension.
	\end{itemize}
}
\item preposition \\
If you say that someone or something is \textbf{beneath} you, you feel that they are not good enough for you or not suitable for you.
 \textit{
	\begin{itemize}
	\item They decided she was marrying beneath her.
	\item Many find themselves having to take jobs far beneath them.
	\end{itemize}
}
\end{enumerate}

\section*{circus}
{\large \color{blue}  circuses  }
\subsection*{Explain}
\begin{enumerate}
\item countable noun \\
A \textbf{circus} is a group that consists of clowns, acrobats, and animals which travels around to different places and performs  shows .
 \textbf{The circus} is the show performed by these people.
 \textit{
	\begin{itemize}
	\item My real ambition was to work in a circus.
	\item ...circus performers.
	\item My dad took me to the circus.
	\end{itemize}
}
\item singular noun \\
If you describe a group of people or an event as a \textbf{circus} , you disapprove of them because they attract a lot of attention but do not achieve anything useful .
 \textit{
	\begin{itemize}
	\item It could well turn into some kind of a media circus.
	\item ...the travelling circus of political journalists.
	\end{itemize}
}
\item noun, in names \\
\textbf{Circus} is sometimes used as part of the name of a street which goes in a circle .
 \textit{
	\begin{itemize}
	\item ...Piccadilly Circus.
	\end{itemize}
}
\end{enumerate}

\section*{beside}
{\large \color{blue}  }
\subsection*{Explain}
\begin{enumerate}
\item preposition \\
Something that is \textbf{beside} something else is at the side of it or next to it.
 \textit{
	\begin{itemize}
	\item On the table beside an empty plate was a pile of books.
	\item I moved from behind my desk to sit beside her.
	\end{itemize}
}
\item  \\
 beside oneself \textit{
	\begin{itemize}
	\end{itemize}
}
\end{enumerate}

\section*{countryside}
{\large \color{blue}  }
\subsection*{Explain}
\begin{enumerate}
\item uncountable noun \\
\textbf{The}  \textbf{countryside} is land which is away from towns and cities.
 \textit{
	\begin{itemize}
	\item I've always loved the English countryside.
	\item We are surrounded by lots of beautiful countryside.
	\end{itemize}
}
\end{enumerate}

\section*{between}
{\large \color{blue}  }
\subsection*{Explain}
\begin{enumerate}
\item preposition \\
If something is \textbf{between} two things or is \textbf{in between} them, it has one of the things on one side of it and the other thing on the other
side.
 \textit{
	\begin{itemize}
	\item She left the table to stand between the two men.
	\item Charlie crossed between the traffic to the far side of the street.
	\end{itemize}
}
\item preposition \\
If people or things travel  \textbf{between} two places, they travel regularly from one place to the other and back again.
 \textit{
	\begin{itemize}
	\item I spent a lot of time travelling between London and Bradford.
	\end{itemize}
}
\item preposition \\
A relationship , discussion , or difference  \textbf{between} two people, groups, or things is one that involves them both or relates to them both.
 \textit{
	\begin{itemize}
	\item I think the relationship between patients and doctors has got a lot less personal.
	\item There have been intensive discussions between the two governments in recent days.
	\item There has always been a difference between community radio and commercial radio.
	\end{itemize}
}
\item preposition \\
If something stands  \textbf{between} you and what you want , it prevents you from having it.
 \textit{
	\begin{itemize}
	\item His sense of duty often stood between him and the enjoyment of life.
	\end{itemize}
}
\item preposition \\
If something is \textbf{between} two amounts or ages , it is greater or older than the first one and smaller or younger than the second one.
 \textit{
	\begin{itemize}
	\item Increase the amount of time you spend exercising by walking between 15 and 20 minutes.
	\item Amsterdam is fun–a third of its population is aged between 18 and 30.
	\end{itemize}
}
\item preposition \\
If something happens  \textbf{between} or \textbf{in between} two times or events, it happens after the first time or event and before the second
one.
 \textbf{Between} is also an adverb .
 \textit{
	\begin{itemize}
	\item The canal was built between 1793 and 1797.
	\item Berlin was well known for its good living in between the two world wars.
	\item They come in peaks lasting two or three minutes, with periods of calm in between.
	\end{itemize}
}
\item preposition \\
If you must  choose  \textbf{between} two or more things, you must choose just one of them.
 \textit{
	\begin{itemize}
	\item Students will be able to choose between English, French and Russian as their first
foreign language.
	\end{itemize}
}
\item preposition \\
If people or places have a particular amount of something \textbf{between} them, this is the total amount that they have.
 \textit{
	\begin{itemize}
	\item The three sites employ 12,500 people between them.
	\item Between them, they train over fifty horses in Lambourn.
	\end{itemize}
}
\item preposition \\
When something is divided or shared  \textbf{between} people, they each have a share of it.
 \textit{
	\begin{itemize}
	\item His company was bought out for £3.5m, divided between five partners.
	\item There is only one bathroom shared between eight bedrooms.
	\end{itemize}
}
\item  \\
 between you and me/between ourselves \textit{
	\begin{itemize}
	\end{itemize}
}
\end{enumerate}

\section*{crown}
{\large \color{blue}  crowns  crowning  crowned  }
\subsection*{Explain}
\begin{enumerate}
\item countable noun \\
A \textbf{crown} is a circular  ornament , usually made of gold and jewels , which a king or queen wears on their head at official ceremonies . You can also use \textbf{crown} to refer to anything circular that is worn on someone's head.
 \textit{
	\begin{itemize}
	\item ...a crown of flowers.
	\end{itemize}
}
\item proper noun \\
The government of a country that has a king or queen is sometimes referred to as \textbf{the Crown} . In British criminal cases the prosecutor is \textbf{the Crown} .
 \textit{
	\begin{itemize}
	\item She says the sovereignty of the Crown must be preserved.
	\item ...a Minister of the Crown.
	\item ...chief witness for the Crown.
	\end{itemize}
}
\item verb \\
When a king or queen \textbf{is crowned} , a crown is placed on their head as part of a ceremony in which they are officially made king or queen.
 \textit{
	\begin{itemize}
	\item Elizabeth was crowned in Westminster Abbey on 2 June 1953.
	\item Two days later, Juan Carlos was crowned king.
	\item ...the newly-crowned King.
	\end{itemize}
}
\item verb \\
If one thing \textbf{crowns} another, it is on top of it.
 \textit{
	\begin{itemize}
	\item Here another rugged castle crowns the cliffs and crags.
	\item ...a very striking face, crowned by an abundance of hair.
	\end{itemize}
}
\item countable noun \\
Your \textbf{crown} is the top part of your head, at the back.
 \textit{
	\begin{itemize}
	\item He laid his hand gently on the crown of her head.
	\end{itemize}
}
\item countable noun \\
The \textbf{crown} of a hat is the part which covers the top of your head.
 \textit{
	\begin{itemize}
	\end{itemize}
}
\item countable noun \\
A \textbf{crown} was a British coin worth five shillings.
 \textit{
	\begin{itemize}
	\end{itemize}
}
\item countable noun \\
A \textbf{crown} is an artificial top piece fixed over a broken or decayed tooth.
 \textit{
	\begin{itemize}
	\end{itemize}
}
\item countable noun \\
In sport, winning an important competition is sometimes referred to as a \textbf{crown} .
 \textit{
	\begin{itemize}
	\item ...his dream of a fourth Wimbledon crown.
	\end{itemize}
}
\item verb \\
An achievement or event that \textbf{crowns} something makes it perfect , successful , or complete.
 \textit{
	\begin{itemize}
	\item It is an important moment, crowning the efforts of the tournament organisers.
	\item The summit was crowned by the signing of the historic START treaty.
	\item ...the crowning achievement of his career.
	\end{itemize}
}
\item verb \\
If you \textbf{crown} your career  \textbf{with} a success or achievement, you have a final success or achievement which is greater than all the others you have had.
 \textit{
	\begin{itemize}
	\item He went on to crown a distinguished career in radio and television with his book
'The Price of Victory'.
	\end{itemize}
}
\end{enumerate}

\section*{beyond}
{\large \color{blue}  }
\subsection*{Explain}
\begin{enumerate}
\item preposition \\
If something is \textbf{beyond} a place or barrier , it is on the other side of it.
 \textbf{Beyond} is also an adverb .
 \textit{
	\begin{itemize}
	\item They heard footsteps in the main room, beyond a door.
	\item On his right was a thriving vegetable garden and beyond it a small orchard of apple
trees.
	\item The house had a fabulous view out to the Strait of Georgia and the Rockies beyond.
	\item ...the need to defend itself against its enemies inside its borders and beyond.
	\end{itemize}
}
\item preposition \\
If something happens  \textbf{beyond} a particular time or date , it continues after that time or date has passed.
 \textbf{Beyond} is also an adverb.
 \textit{
	\begin{itemize}
	\item Few jockeys continue race-riding beyond the age of 40.
	\item You may be entitled to Child Benefit if a child continues getting full-time education
beyond the date already notified by you.
	\item She is confident about the company's prospects for the current financial year and
beyond.
	\end{itemize}
}
\item preposition \\
If something extends  \textbf{beyond} a particular thing, it affects or includes other things.
 \textit{
	\begin{itemize}
	\item His interests extended beyond the fine arts to international politics and philosophy.
	\end{itemize}
}
\item preposition \\
You use \textbf{beyond} to introduce an exception to what you are saying .
 \textit{
	\begin{itemize}
	\item He appears to have almost no personal staff, beyond a secretary who can't make coffee.
	\item I knew nothing beyond a few random facts.
	\end{itemize}
}
\item preposition \\
If something goes  \textbf{beyond} a particular point or stage , it progresses or increases so that it passes that point or stage.
 \textit{
	\begin{itemize}
	\item Their five-year relationship was strained beyond breaking point.
	\item It seems to me he's beyond caring about what anybody does.
	\end{itemize}
}
\item preposition \\
If something is, for example , \textbf{beyond}  understanding or \textbf{beyond}  belief , it is so extreme in some way that it cannot be understood or believed .
 \textit{
	\begin{itemize}
	\item What Jock had done was beyond my comprehension.
	\item Sweden is lovely in summer–cold beyond belief in winter.
	\item Business computing has changed beyond recognition.
	\end{itemize}
}
\item preposition \\
If you say that something is \textbf{beyond} someone, you mean that they cannot deal with it.
 \textit{
	\begin{itemize}
	\item Although he could give her sympathy, any practical help would almost certainly be
beyond him.
	\item The situation was beyond her control.
	\end{itemize}
}
\end{enumerate}

\section*{custom}
{\large \color{blue}  customs  }
\subsection*{Explain}
\begin{enumerate}
\item variable noun \\
A \textbf{custom} is an activity, a way of behaving , or an event which is usual or traditional in a particular society or in particular circumstances .
 \textit{
	\begin{itemize}
	\item The custom of lighting the famous flame goes back centuries.
	\item Chung has tried to adapt to local customs.
	\end{itemize}
}
\item singular noun \\
If it is your \textbf{custom}  \textbf{to} do something, you usually do it in particular circumstances.
 \textit{
	\begin{itemize}
	\item It was his custom to approach every problem cautiously.
	\item As is the custom, police forensic experts carried out a painstaking search of the
debris.
	\end{itemize}
}
\item uncountable noun \\
If a shop has your \textbf{custom} , you regularly buy things there.
 \textit{
	\begin{itemize}
	\item You have the right to withhold your custom if you so wish.
	\item Providing discounts is not the only way to win custom.
	\end{itemize}
}
\item adjective \\
If you use \textbf{custom} to describe something such as a vehicle or a piece of clothing , you mean that it has been designed for one particular customer.
 \textit{
	\begin{itemize}
	\item Her one-of -a-kind custom garments are priced from one hundred dollars.
	\end{itemize}
}
\end{enumerate}

\section*{despite}
{\large \color{blue}  }
\subsection*{Explain}
\begin{enumerate}
\item preposition \\
You use \textbf{despite} to introduce a fact which makes the other part of the sentence  surprising .
 \textit{
	\begin{itemize}
	\item It is possible to lead happy and productive lives despite their loss.
	\item Despite being warned to be on time they both arrived late.
	\end{itemize}
}
\item preposition \\
If you do something \textbf{despite}  \textbf{yourself} you do it although you did not really  intend or expect to.
 \textit{
	\begin{itemize}
	\item Despite myself, Harry's remarks had caused me to stop and reflect.
	\end{itemize}
}
\end{enumerate}

\section*{eight}
{\large \color{blue}  eights  }
\subsection*{Explain}
\begin{enumerate}
\item number \\
\textbf{Eight} is the number 8.
 \textit{
	\begin{itemize}
	\item So far eight workers have been killed.
	\end{itemize}
}
\end{enumerate}

\section*{dynasty}
{\large \color{blue}  dynasties  }
\subsection*{Explain}
\begin{enumerate}
\item countable noun \\
A \textbf{dynasty} is a series of rulers of a country who all belong to the same family.
 \textit{
	\begin{itemize}
	\item The Seljuk dynasty of Syria was founded in 1094.
	\end{itemize}
}
\item countable noun \\
A \textbf{dynasty} is a period of time during which a country is ruled by members of the same family.
 \textit{
	\begin{itemize}
	\item ...carvings dating back to the Ming dynasty.
	\end{itemize}
}
\item countable noun \\
A \textbf{dynasty} is a family which has members from two or more generations who are important in a particular field of activity, for example in business or politics .
 \textit{
	\begin{itemize}
	\item ...the Kennedy dynasty.
	\end{itemize}
}
\end{enumerate}

\section*{fifty}
{\large \color{blue}  fifties  }
\subsection*{Explain}
\begin{enumerate}
\item number \\
\textbf{Fifty} is the number 50.
 \textit{
	\begin{itemize}
	\item Fifty years is a long time in journalism.
	\end{itemize}
}
\item plural noun \\
When you talk about the \textbf{fifties} , you are referring to numbers between 50 and 59. For example , if you are \textbf{in} your \textbf{fifties} , you are aged between 50 and 59. If the temperature is \textbf{in the fifties} , the temperature is between 50 and 59 degrees .
 \textit{
	\begin{itemize}
	\item I probably look as if I'm in my fifties rather than my seventies.
	\end{itemize}
}
\item plural noun \\
\textbf{The fifties} is the decade between 1950 and 1959.
 \textit{
	\begin{itemize}
	\item He began performing in the early fifties.
	\end{itemize}
}
\end{enumerate}

\section*{female}
{\large \color{blue}  females  }
\subsection*{Explain}
\begin{enumerate}
\item adjective \\
Someone who is \textbf{female} is a woman or a girl.
 \textit{
	\begin{itemize}
	\item ...a sixteen-piece dance band with a female singer.
	\item Their aim is to have equal numbers of male and female MPs.
	\item Only 13 per cent of consultants are female.
	\end{itemize}
}
\item countable noun \\
Women and girls are sometimes  referred to as \textbf{females} when they are being considered as a type.
 \textit{
	\begin{itemize}
	\item Hay fever affects males more than females.
	\end{itemize}
}
\item adjective \\
\textbf{Female}  matters and things relate to, belong to, or affect women rather than men .
 \textit{
	\begin{itemize}
	\item ...female infertility.
	\item ...a purveyor of female undergarments.
	\item I realize there's no consensus on what are male or female values.
	\end{itemize}
}
\item countable noun \\
You can refer to any creature that can lay  eggs or produce babies from its body as a \textbf{female} .
 \textbf{Female} is also an adjective .
 \textit{
	\begin{itemize}
	\item Each female will lay just one egg in April or May.
	\item ...the scent given off by the female aphid to attract the male.
	\end{itemize}
}
\item adjective \\
A \textbf{female} flower or plant contains the part that will  become the fruit when it is fertilized.
 \textit{
	\begin{itemize}
	\item Figs have male and female flowers.
	\end{itemize}
}
\end{enumerate}

\section*{five}
{\large \color{blue}  fives  }
\subsection*{Explain}
\begin{enumerate}
\item number \\
\textbf{Five} is the number 5.
 \textit{
	\begin{itemize}
	\item Eric Edward Bullus was born in Peterborough, the second of five children.
	\end{itemize}
}
\item uncountable noun \\
\textbf{Fives} is a British ball game in which you hit a small hard ball with a glove or bat against three walls of a court .
 \textit{
	\begin{itemize}
	\end{itemize}
}
\end{enumerate}

\section*{flight}
{\large \color{blue}  flights  }
\subsection*{Explain}
\begin{enumerate}
\item countable noun \\
A \textbf{flight} is a journey made by flying, usually in an aeroplane .
 \textit{
	\begin{itemize}
	\item The flight will take four hours.
	\end{itemize}
}
\item countable noun \\
You can refer to an aeroplane carrying passengers on a particular journey as a particular \textbf{flight} .
 \textit{
	\begin{itemize}
	\item I'll try to get on the flight down to Karachi tonight.
	\item BA flight 286 was two hours late.
	\end{itemize}
}
\item uncountable noun \\
\textbf{Flight} is the action of flying, or the ability to fly.
 \textit{
	\begin{itemize}
	\item These hawks are magnificent in flight, soaring and circling for long periods.
	\item Supersonic flight could become a routine form of travel in the 21st century.
	\end{itemize}
}
\item countable noun \\
A \textbf{flight of} birds is a group of them flying together.
 \textit{
	\begin{itemize}
	\item A flight of green parrots shot out of the cedar forest.
	\end{itemize}
}
\item uncountable noun \\
\textbf{Flight} is the act of running away from a dangerous or unpleasant  situation or place.
 \textit{
	\begin{itemize}
	\item Frank was in full flight when he reached them.
	\item The family was often in flight, hiding out in friends' houses.
	\item ...her hurried flight from the palace in a cart.
	\end{itemize}
}
\item countable noun \\
A \textbf{flight}  \textbf{of} steps or stairs is a set of steps or stairs that lead from one level to another without
changing direction.
 \textit{
	\begin{itemize}
	\item We walked in silence up a flight of stairs and down a long corridor.
	\end{itemize}
}
\item  \\
 flight of fancy \textit{
	\begin{itemize}
	\end{itemize}
}
\item  \\
 take flight \textit{
	\begin{itemize}
	\end{itemize}
}
\end{enumerate}

\section*{fly}
{\large \color{blue}  flies  flying  flew  flown  }
\subsection*{Explain}
\begin{enumerate}
\item countable noun \\
A \textbf{fly} is a small insect with two wings. There are many kinds of flies, and the most common
are black in colour.
 \textit{
	\begin{itemize}
	\end{itemize}
}
\item verb \\
When something such as a bird, insect, or aircraft \textbf{flies} , it moves through the air.
 \textit{
	\begin{itemize}
	\item The planes flew through the clouds.
	\item The bird flew away.
	\end{itemize}
}
\item verb \\
If you \textbf{fly}  somewhere , you travel there in an aircraft.
 \textit{
	\begin{itemize}
	\item He flew to Los Angeles.
	\item He flew back to London.
	\item Mr Baker flew in from Moscow.
	\end{itemize}
}
\item verb \\
When someone \textbf{flies} an aircraft, they control its movement in the air.
 \textit{
	\begin{itemize}
	\item Parker had successfully flown both aircraft.
	\item He flew a small plane to Cuba.
	\item His inspiration to fly came even before he joined the Army.
	\end{itemize}
}
\item verb \\
To \textbf{fly} someone or something somewhere means to take or send them there in an aircraft.
 \textit{
	\begin{itemize}
	\item It may be possible to fly the women and children out on Thursday.
	\item The relief supplies are being flown from a warehouse in Pisa.
	\end{itemize}
}
\item verb \\
If something such as your hair \textbf{is flying} about, it is moving about freely and loosely in the air.
 \textit{
	\begin{itemize}
	\item His long, uncovered hair flew back in the wind.
	\item She was running down the stairs, her hair flying.
	\end{itemize}
}
\item verb \\
If you \textbf{fly} a flag or if it \textbf{is flying} , you display it at the top of a pole .
 \textit{
	\begin{itemize}
	\item They flew the flag of the African National Congress.
	\item A flag was flying on the new military HQ.
	\end{itemize}
}
\item verb \\
If you say that someone or something \textbf{flies} in a particular direction, you are emphasizing that they move there with a lot of speed or force.
 \textit{
	\begin{itemize}
	\item She flew to their bedsides when they were ill.
	\item I flew downstairs.
	\item There are bullets flying around your head.
	\end{itemize}
}
\item verb \\
If you tell someone that you must  \textbf{fly} , you are indicating that you have to leave in a great hurry .
 \textit{
	\begin{itemize}
	\item I must fly or I'll miss my plane.
	\item I'll have to fly.
	\end{itemize}
}
\item verb \\
If stories or rumours  \textbf{are flying} around a place, they are being discussed a great deal and by a lot of people within a short period of time.
 \textit{
	\begin{itemize}
	\item Rumours had been flying around the workrooms all morning.
	\item Rumours were flying about possible deals.
	\end{itemize}
}
\item countable noun \\
The front opening on a pair of trousers is referred to as the \textbf{fly} , or in British English the \textbf{flies} . It usually consists of a zip or row of buttons behind a band of cloth.
 \textit{
	\begin{itemize}
	\end{itemize}
}
\item countable noun \\
In fishing, a \textbf{fly} is a model of a small winged insect that is used as a bait .
 \textit{
	\begin{itemize}
	\end{itemize}
}
\item  \\
 wouldn't harm a fly \textit{
	\begin{itemize}
	\end{itemize}
}
\item  \\
 to let fly \textit{
	\begin{itemize}
	\end{itemize}
}
\item  \\
 on the fly \textit{
	\begin{itemize}
	\end{itemize}
}
\item  \\
 send someone/something flying \textit{
	\begin{itemize}
	\end{itemize}
}
\item  \\
 a fly on the wall \textit{
	\begin{itemize}
	\end{itemize}
}
\end{enumerate}

\section*{four}
{\large \color{blue}  fours  }
\subsection*{Explain}
\begin{enumerate}
\item number \\
\textbf{Four} is the number 4.
 \textit{
	\begin{itemize}
	\item Judith is married with four children.
	\end{itemize}
}
\item countable noun \\
In cricket , if a player hits a \textbf{four} , they score four runs by hitting the ball along the ground so that it crosses the boundary at the edge of the playing area.
 \textit{
	\begin{itemize}
	\item Taylor hit 13 fours and batted for 140 minutes.
	\end{itemize}
}
\item countable noun \\
A \textbf{four} is a narrow racing boat that is rowed by a team of four people.
 \textit{
	\begin{itemize}
	\end{itemize}
}
\item  \\
 on all fours \textit{
	\begin{itemize}
	\end{itemize}
}
\end{enumerate}

\section*{fork}
{\large \color{blue}  forks  forking  forked  }
\subsection*{Explain}
\begin{enumerate}
\item countable noun \\
A \textbf{fork} is a tool used for eating food which has a row of three or four long metal points at the end.
 \textit{
	\begin{itemize}
	\item ...knives and forks.
	\end{itemize}
}
\item verb \\
If you \textbf{fork} food \textbf{into} your mouth or \textbf{onto} a plate , you put it there using a fork.
 \textit{
	\begin{itemize}
	\item Ann forked some fish into her mouth.
	\item He forked an egg onto a piece of bread and folded it into a sandwich.
	\end{itemize}
}
\item countable noun \\
A garden  \textbf{fork} is a tool used for breaking up soil which has a row of three or four long metal points at the end.
 \textit{
	\begin{itemize}
	\end{itemize}
}
\item verb \\
If you \textbf{fork} something such as manure or hay , you move it from one place to another using a large garden fork.
 \textit{
	\begin{itemize}
	\item They started me off in the gardens as a handyman. Digging, forking manure, that kind
of thing.
	\item Farmers cut the hay, fork it on to a cart and then store it in barns.
	\end{itemize}
}
\item countable noun \\
A \textbf{fork} in a road, path , or river is a point at which it divides into two parts and forms a ' Y ' shape.
 \textit{
	\begin{itemize}
	\item We arrived at a fork in the road.
	\item The road divides; you should take the right fork.
	\item ...the fork of the Delaware and Lehigh rivers.
	\end{itemize}
}
\item verb \\
If a road, path, or river \textbf{forks} , it forms a fork.
 \textit{
	\begin{itemize}
	\item Beyond the village the road forked.
	\item The path dipped down to a sort of cove, and then it forked in two directions.
	\end{itemize}
}
\item verb \\
If you \textbf{fork} in a particular  direction when you are travelling along a road or path, you choose one of the forks in it and travel down it.
 \textit{
	\begin{itemize}
	\item Just before the town boundary fork left onto a minor road.
	\end{itemize}
}
\end{enumerate}

\section*{from}
{\large \color{blue}  }
\subsection*{Explain}
\begin{enumerate}
\item preposition \\
If something comes  \textbf{from} a particular person or thing, or if you get something \textbf{from} them, they give it to you or they are the source of it.
 \textit{
	\begin{itemize}
	\item He appealed for information from anyone who saw the attackers.
	\item ...an anniversary present from his wife.
	\item The results were taken from six surveys.
	\item The dirt from the fields drifted like snow.
	\end{itemize}
}
\item preposition \\
Someone who comes \textbf{from} a particular place lives in that place or originally  lived there. Something that comes \textbf{from} a particular place was made in that place.
 \textit{
	\begin{itemize}
	\item ...an art dealer from Zurich.
	\item Katy Jones is nineteen and comes from Birmingham.
	\item ...wines from Coteaux d'Aix-en-Provence.
	\end{itemize}
}
\item preposition \\
A person \textbf{from} a particular organization works for that organization.
 \textit{
	\begin{itemize}
	\item ...a representative from the Israeli embassy.
	\end{itemize}
}
\item preposition \\
If someone or something moves or is moved \textbf{from} a place, they leave it or are removed, so that they are no longer there.
 \textit{
	\begin{itemize}
	\item The guests watched as she fled from the room.
	\end{itemize}
}
\item preposition \\
If you take one thing or person \textbf{from} another, you move that thing or person so that they are no longer with the other
or attached to the other.
 \textit{
	\begin{itemize}
	\item In many bone transplants, bone can be taken from other parts of the patient's body.
	\item Remove the bowl from the ice and stir in the cream.
	\end{itemize}
}
\item preposition \\
If you take something \textbf{from} an amount, you reduce the amount by that much.
 \textit{
	\begin{itemize}
	\item The £103 is deducted from Mrs Adams' salary every month.
	\item Three from six leaves three.
	\end{itemize}
}
\item preposition \\
\textbf{From} is used in expressions such as \textbf{away from} or \textbf{absent from} to say that someone or something is not present in a place where they are usually found.
 \textit{
	\begin{itemize}
	\item Her husband worked away from home a lot.
	\item Jo was absent from the house all the next day.
	\end{itemize}
}
\item preposition \\
If you return  \textbf{from} a place or an activity, you return after being in that place or doing that activity.
 \textit{
	\begin{itemize}
	\item My son Colin has just returned from Amsterdam.
	\item ...a group of men travelling home from a darts match.
	\end{itemize}
}
\item preposition \\
If you are back \textbf{from} a place or activity, you have left it and have returned to your former place.
 \textit{
	\begin{itemize}
	\item Our economics correspondent, James Morgan, is just back from Germany.
	\item One afternoon when I was home from school, he asked me to come to see a movie with
him.
	\end{itemize}
}
\item preposition \\
If you see or hear something \textbf{from} a particular place, you are in that place when you see it or hear it.
 \textit{
	\begin{itemize}
	\item Visitors see the painting from behind a plate glass window.
	\item Viewed from above, the valleys form the shape of a man.
	\end{itemize}
}
\item preposition \\
If something hangs or sticks out \textbf{from} an object, it is attached to it or held by it.
 \textit{
	\begin{itemize}
	\item Hanging from his right wrist is a heavy gold bracelet.
	\item ...large fans hanging from ceilings.
	\item He saw the corner of a magazine sticking out from under the blanket.
	\end{itemize}
}
\item preposition \\
You can use \textbf{from} when giving distances. For example , if a place is fifty miles \textbf{from} another place, the distance between the two places is fifty miles.
 \textit{
	\begin{itemize}
	\item The centre of the town is 4 kilometres from the station.
	\item ...a small park only a few hundred yards from Zurich's main shopping centre.
	\item How far is it from here?
	\end{itemize}
}
\item preposition \\
If a road or railway line goes  \textbf{from} one place to another, you can travel along it between the two places.
 \textit{
	\begin{itemize}
	\item ...the road from St Petersburg to Tallinn.
	\end{itemize}
}
\item preposition \\
\textbf{From} is used, especially in the expression \textbf{made from} , to say what substance has been used to make something.
 \textit{
	\begin{itemize}
	\item ...bread made from white flour.
	\item ...a luxurious resort built from the island's native coral stone.
	\end{itemize}
}
\item preposition \\
You can use \textbf{from} when you are talking about the beginning of a period of time.
 \textit{
	\begin{itemize}
	\item She studied painting from 1926 and also worked as a commercial artist.
	\item Breakfast is available to fishermen from 6 a.m.
	\item From 1922 till 1925 she lived in Prague.
	\end{itemize}
}
\item preposition \\
You say \textbf{from} one thing \textbf{to} another when you are stating the range of things that are possible , or when saying that the range of things includes everything in a certain category .
 \textit{
	\begin{itemize}
	\item There are 94 countries represented at the Games, from Algeria to Zimbabwe.
	\item Over 150 companies will be there, covering everything from finance to fixtures and
fittings.
	\end{itemize}
}
\item preposition \\
If something changes \textbf{from} one thing \textbf{to} another, it stops being the first thing and becomes the second thing.
 \textit{
	\begin{itemize}
	\item The expression on his face changed from sympathy to surprise.
	\item Unemployment has fallen from 7.5 to 7.2%.
	\item I made a switch from butter to olive oil for much of my cooking.
	\end{itemize}
}
\item preposition \\
You use \textbf{from} after some verbs and nouns when mentioning the cause of something.
 \textit{
	\begin{itemize}
	\item The problem simply resulted from a difference of opinion.
	\item He is suffering from eye ulcers, brought on by the intense light in Australia.
	\item They really do get pleasure from spending money on other people.
	\item Most of the wreckage from the 1985 quake has been cleared.
	\end{itemize}
}
\item preposition \\
You use \textbf{from} when you are giving the reason for an opinion .
 \textit{
	\begin{itemize}
	\item She knew from experience that Dave was about to tell her the truth.
	\item He sensed from the expression on her face that she had something to say.
	\end{itemize}
}
\item preposition \\
\textbf{From} is used after verbs with meanings such as ' protect ', ' free ', 'keep', and ' prevent ' to introduce the action that does not happen , or that someone does not want to happen.
 \textit{
	\begin{itemize}
	\item Such laws could protect the consumer from harmful or dangerous remedies.
	\item 300 tons of Peruvian mangoes were kept from entering France.
	\end{itemize}
}
\end{enumerate}

\section*{genius}
{\large \color{blue}  geniuses  }
\subsection*{Explain}
\begin{enumerate}
\item uncountable noun \\
\textbf{Genius} is very great ability or skill in a particular subject or activity.
 \textit{
	\begin{itemize}
	\item This is the mark of her real genius as a designer.
	\item The man had genius and had made his mark in the aviation world.
	\item Its very title is a stroke of genius.
	\end{itemize}
}
\item countable noun \\
A \textbf{genius} is a highly talented, creative, or intelligent person.
 \textit{
	\begin{itemize}
	\item Chaplin was not just a genius, he was among the most influential figures in film
history.
	\end{itemize}
}
\end{enumerate}

\section*{hundred}
{\large \color{blue}  hundreds  }
\subsection*{Explain}
\begin{enumerate}
\item number \\
\textbf{A}  \textbf{hundred} or \textbf{one}  \textbf{hundred} is the number 100.
 \textit{
	\begin{itemize}
	\item According to one official more than a hundred people have been arrested.
	\end{itemize}
}
\item quantifier \\
If you refer to \textbf{hundreds of} things or people, you are emphasizing that there are very many of them.
 You can also use \textbf{hundreds} as a pronoun.
 \textit{
	\begin{itemize}
	\item Hundreds of tree species face extinction.
	\item Today you can buy hundreds of flavours of ice-cream.
	\item Hundreds have been killed in the fighting and thousands made homeless.
	\end{itemize}
}
\item  \\
 a hundred per cent/one hundred percent \textit{
	\begin{itemize}
	\end{itemize}
}
\end{enumerate}

\section*{habit}
{\large \color{blue}  habits  }
\subsection*{Explain}
\begin{enumerate}
\item variable noun \\
A \textbf{habit} is something that you do often or regularly.
 \textit{
	\begin{itemize}
	\item He has an endearing habit of licking his lips when he's nervous.
	\item Many people add salt to their food out of habit, without even tasting it first.
	\item ...a survey on eating habits in the U.K.
	\end{itemize}
}
\item countable noun \\
A \textbf{habit} is an action which is considered bad that someone does repeatedly and finds it difficult to stop doing.
 \textit{
	\begin{itemize}
	\item Break the habit of eating too quickly by putting your knife and fork down after each
mouthful.
	\item After twenty years as a chain smoker Mr Nathe has given up the habit.
	\end{itemize}
}
\item countable noun \\
A drug \textbf{habit} is an addiction to a drug such as heroin or cocaine .
 \textit{
	\begin{itemize}
	\item She became a prostitute in order to pay for her cocaine habit.
	\end{itemize}
}
\item countable noun \\
A \textbf{habit} is a piece of clothing shaped like a long loose dress, which a nun or monk wears.
 \textit{
	\begin{itemize}
	\end{itemize}
}
\item  \\
 a creature of habit \textit{
	\begin{itemize}
	\end{itemize}
}
\item  \\
 in the habit of/into the habit of \textit{
	\begin{itemize}
	\end{itemize}
}
\item  \\
 make a habit of \textit{
	\begin{itemize}
	\end{itemize}
}
\item  \\
 habit of mind \textit{
	\begin{itemize}
	\end{itemize}
}
\end{enumerate}

\section*{nine}
{\large \color{blue}  nines  }
\subsection*{Explain}
\begin{enumerate}
\item number \\
\textbf{Nine} is the number 9.
 \textit{
	\begin{itemize}
	\item We still sighted nine yachts.
	\item ...nine hundred pounds.
	\end{itemize}
}
\item  \\
 dressed up to the nines \textit{
	\begin{itemize}
	\end{itemize}
}
\end{enumerate}

\section*{heaven}
{\large \color{blue}  heavens  }
\subsection*{Explain}
\begin{enumerate}
\item proper noun \\
In some religions , \textbf{heaven} is said to be the place where God lives, where good people go when they die, and where everyone is always  happy . It is usually imagined as being high up in the sky.
 \textit{
	\begin{itemize}
	\item I believed that when I died I would go to heaven and see God.
	\end{itemize}
}
\item uncountable noun \\
You can use \textbf{heaven} to refer to a place or situation that you like very much.
 \textit{
	\begin{itemize}
	\item We went touring in Wales and Ireland. It was heaven.
	\item I was in cinematic heaven.
	\end{itemize}
}
\item plural noun \\
\textbf{The heavens} are the sky.
 \textit{
	\begin{itemize}
	\item He walked out into the middle of the road, looking up at the heavens.
	\item ...a detailed map of the heavens.
	\end{itemize}
}
\item  \\
 heaven forbid \textit{
	\begin{itemize}
	\end{itemize}
}
\item  \\
 good heavens \textit{
	\begin{itemize}
	\end{itemize}
}
\item  \\
 heaven help sb \textit{
	\begin{itemize}
	\end{itemize}
}
\item  \\
 heaven knows \textit{
	\begin{itemize}
	\end{itemize}
}
\item  \\
 heaven knows \textit{
	\begin{itemize}
	\end{itemize}
}
\item  \\
 to move heaven and earth \textit{
	\begin{itemize}
	\end{itemize}
}
\item  \\
 in heaven's name \textit{
	\begin{itemize}
	\end{itemize}
}
\item  \\
 the heavens open \textit{
	\begin{itemize}
	\end{itemize}
}
\end{enumerate}

\section*{ninety}
{\large \color{blue}  nineties  }
\subsection*{Explain}
\begin{enumerate}
\item number \\
\textbf{Ninety} is the number 90.
 \textit{
	\begin{itemize}
	\item It was decided she had to stay another ninety days.
	\end{itemize}
}
\item plural noun \\
When you talk about the \textbf{nineties} , you are referring to numbers between 90 and 99. For example , if you are \textbf{in} your \textbf{nineties} , you are aged between 90 and 99. If the temperature is \textbf{in the nineties} , the temperature is between 90 and 99 degrees.
 \textit{
	\begin{itemize}
	\item By this time she was in her nineties and needed help more and more frequently.
	\end{itemize}
}
\item plural noun \\
\textbf{The nineties} is the decade between 1990 and 1999.
 \textit{
	\begin{itemize}
	\item These trends only got worse as we moved into the nineties.
	\end{itemize}
}
\end{enumerate}

\section*{horse}
{\large \color{blue}  horses  horsing  horsed  }
\subsection*{Explain}
\begin{enumerate}
\item countable noun \\
A \textbf{horse} is a large animal which people can ride. Some horses are used for pulling  ploughs and carts .
 \textit{
	\begin{itemize}
	\item A small man on a grey horse had appeared.
	\end{itemize}
}
\item plural noun \\
When you talk about \textbf{the horses} , you mean horse races in which people bet money on the horse which they think  will  win .
 \textit{
	\begin{itemize}
	\item He still likes to bet on the horses.
	\end{itemize}
}
\item countable noun \\
A vaulting \textbf{horse} is a tall piece of gymnastics  equipment for jumping over.
 \textit{
	\begin{itemize}
	\end{itemize}
}
\item  \\
 from the horse's mouth \textit{
	\begin{itemize}
	\end{itemize}
}
\end{enumerate}

\section*{idiom}
{\large \color{blue}  idioms  }
\subsection*{Explain}
\begin{enumerate}
\item countable noun \\
A particular \textbf{idiom} is a particular style of something such as music , dance , or architecture .
 \textit{
	\begin{itemize}
	\item McCartney was also keen to write in a classical idiom, rather than a pop one.
	\item It was an old building in the local idiom.
	\end{itemize}
}
\item countable noun \\
An \textbf{idiom} is a group of words which have a different meaning when used together from the one they would have if you took the meaning of each word separately.
 \textit{
	\begin{itemize}
	\item Proverbs and idioms may become worn with over-use.
	\item She is, in fact, a perfect illustration of the French idiom 'to be comfortable in
one's own skin.'
	\end{itemize}
}
\item uncountable noun \\
\textbf{Idiom} of a particular kind is the language that people use at a particular time or in a particular place.
 \textit{
	\begin{itemize}
	\item And nothing was so irritating as the confident way he used archaic idiom.
	\item ...her command of the Chinese idiom.
	\end{itemize}
}
\end{enumerate}

\section*{kingdom}
{\large \color{blue}  kingdoms  }
\subsection*{Explain}
\begin{enumerate}
\item countable noun \\
A \textbf{kingdom} is a country or region that is ruled by a king or queen.
 \textit{
	\begin{itemize}
	\item The kingdom's power declined.
	\item ...the United Kingdom.
	\item ...the Kingdom of Denmark.
	\end{itemize}
}
\item countable noun \\
A \textbf{kingdom} is a place or area that is thought to be under the control of a person or organization .
 \textit{
	\begin{itemize}
	\item It was infamous as a kingdom of brigands, scoundrels, and slave-traders.
	\end{itemize}
}
\item singular noun \\
All the animals, birds , and insects in the world can be referred to together as the animal \textbf{kingdom} . All the plants can be referred to as the plant \textbf{kingdom} .
 \textit{
	\begin{itemize}
	\end{itemize}
}
\end{enumerate}

\section*{plus}
{\large \color{blue}  pluses  plusses  }
\subsection*{Explain}
\begin{enumerate}
\item conjunction \\
You say  \textbf{plus} to show that one number or quantity is being added to another.
 \textit{
	\begin{itemize}
	\item Send a cheque for £18.99 plus £2 for postage and packing.
	\item They will pay about $673 million plus interest.
	\end{itemize}
}
\item adjective \\
\textbf{Plus} before a number or quantity means that the number or quantity is greater than zero .
 \textit{
	\begin{itemize}
	\item The aircraft was subjected to temperatures of minus 65 degrees and plus 120 degrees.
	\end{itemize}
}
\item conjunction \\
You can use \textbf{plus} when mentioning an additional  item or fact .
 \textit{
	\begin{itemize}
	\item There's easily enough room for two adults and three children, plus a dog in the boot.
	\item I haven't exercised at all for years, plus I haven't got age on my side!
	\end{itemize}
}
\item adjective \\
You use \textbf{plus} after a number or quantity to indicate that the actual number or quantity is greater than the one mentioned.
 \textit{
	\begin{itemize}
	\item There are only 35 staff to serve 30,000-plus customers.
	\item Among the guests were 16 high-flying executives, all on salaries of £50,000 a year
plus.
	\end{itemize}
}
\item  \\
Teachers use \textbf{plus} in grading work in schools and colleges . ' B plus' is a better grade than 'B', but it is not as good as 'A'.
 \textit{
	\begin{itemize}
	\end{itemize}
}
\item countable noun \\
A \textbf{plus} is an advantage or benefit .
 \textit{
	\begin{itemize}
	\item Experience of any career in sales is a big plus.
	\item There are plenty of plus points about being an older first-time mum.
	\end{itemize}
}
\end{enumerate}

\section*{landlady}
{\large \color{blue}  landladies  }
\subsection*{Explain}
\begin{enumerate}
\item countable noun \\
Someone's \textbf{landlady} is the woman who allows them to live or work in a building which she owns, in return for rent .
 \textit{
	\begin{itemize}
	\item We had been made homeless by our landlady.
	\end{itemize}
}
\item countable noun \\
The \textbf{landlady} of a pub is the woman who owns or runs it, or the wife of the person who owns or
runs it.
 \textit{
	\begin{itemize}
	\item She became the landlady of a rural pub.
	\end{itemize}
}
\item countable noun \\
A \textbf{landlady} is the woman who owns or runs a boarding house or inn .
 \textit{
	\begin{itemize}
	\end{itemize}
}
\end{enumerate}

\section*{poster}
{\large \color{blue}  posters  }
\subsection*{Explain}
\begin{enumerate}
\item countable noun \\
A \textbf{poster} is a large notice or picture that you stick on a wall or board , often in order to advertise something.
 \textit{
	\begin{itemize}
	\end{itemize}
}
\end{enumerate}

\section*{mosaic}
{\large \color{blue}  mosaics  }
\subsection*{Explain}
\begin{enumerate}
\item variable noun \\
A \textbf{mosaic} is a design which consists of small pieces of coloured glass, pottery , or stone set in concrete or plaster .
 \textit{
	\begin{itemize}
	\item ...a Roman villa which once housed a fine collection of mosaics.
	\item He has used a mixture of mosaic, collage and felt-tip pen.
	\end{itemize}
}
\end{enumerate}

\section*{seven}
{\large \color{blue}  sevens  }
\subsection*{Explain}
\begin{enumerate}
\item number \\
\textbf{Seven} is the number 7.
 \textit{
	\begin{itemize}
	\item Sarah and Ella have been friends for seven years.
	\end{itemize}
}
\end{enumerate}

\section*{opening}
{\large \color{blue}  openings  }
\subsection*{Explain}
\begin{enumerate}
\item adjective \\
The \textbf{opening} event, item , day , or week in a series is the first one.
 \textit{
	\begin{itemize}
	\item They returned to take part in the season's opening game.
	\item ...the opening day of the fifth General Synod.
	\end{itemize}
}
\item countable noun \\
\textbf{The}  \textbf{opening}  \textbf{of} something such as a book, play, or concert is the first part of it.
 \textit{
	\begin{itemize}
	\item The opening of the scene depicts Akhnaten and his family in a moment of intimacy.
	\end{itemize}
}
\item countable noun \\
An \textbf{opening} is a hole or empty space through which things or people can pass .
 \textit{
	\begin{itemize}
	\item He squeezed through a narrow opening in the fence.
	\end{itemize}
}
\item countable noun \\
An \textbf{opening} in a forest is a small area where there are no trees or bushes .
 \textit{
	\begin{itemize}
	\item I glanced down at the beach as we passed an opening in the trees.
	\end{itemize}
}
\item countable noun \\
An \textbf{opening} is a good opportunity to do something, for example to show people how good you are.
 \textit{
	\begin{itemize}
	\item Her capabilities were always there; all she needed was an opening to show them.
	\end{itemize}
}
\item countable noun \\
An \textbf{opening} is a job that is available .
 \textit{
	\begin{itemize}
	\item We don't have any openings now, but we'll call you if something comes up.
	\end{itemize}
}
\end{enumerate}

\section*{six}
{\large \color{blue}  sixes  }
\subsection*{Explain}
\begin{enumerate}
\item number \\
\textbf{Six} is the number 6.
 \textit{
	\begin{itemize}
	\item ...a glorious career spanning more than six decades.
	\end{itemize}
}
\item countable noun \\
In cricket , if a player  hits a \textbf{six} , they score six runs by hitting the ball so that it crosses the boundary at the edge of the playing area before it touches the ground.
 \textit{
	\begin{itemize}
	\end{itemize}
}
\item  \\
 to hit someone for six \textit{
	\begin{itemize}
	\end{itemize}
}
\item  \\
 at sixes and sevens \textit{
	\begin{itemize}
	\end{itemize}
}
\end{enumerate}

\section*{outset}
{\large \color{blue}  }
\subsection*{Explain}
\begin{enumerate}
\item  \\
 at the outset/from the outset \textit{
	\begin{itemize}
	\end{itemize}
}
\end{enumerate}

\section*{third}
{\large \color{blue}  thirds  }
\subsection*{Explain}
\begin{enumerate}
\item ordinal number \\
The \textbf{third}  item in a series is the one that you count as number three.
 \textit{
	\begin{itemize}
	\item I sleep on the third floor.
	\item It was the third time one of his cars had gone up in flames.
	\item He came third in the poll with 149 votes.
	\item The attack was the third so far this year.
	\end{itemize}
}
\item fraction \\
A \textbf{third} is one of three equal parts of something.
 \textit{
	\begin{itemize}
	\item A third of the cost went into technology and services.
	\item He divided their kingdom into thirds.
	\end{itemize}
}
\item adverb \\
You say  \textbf{third} when you want to make a third point or give a third reason for something.
 \textit{
	\begin{itemize}
	\item First, interest rates may take longer to fall. Second, lending may fall. Third, bad
loans could wipe out any improvement.
	\end{itemize}
}
\item countable noun \\
A \textbf{third} is the lowest honours degree that can be obtained from a British university.
 \textit{
	\begin{itemize}
	\end{itemize}
}
\end{enumerate}

\section*{pilot}
{\large \color{blue}  pilots  piloting  piloted  }
\subsection*{Explain}
\begin{enumerate}
\item countable noun \\
A \textbf{pilot} is a person who is trained to fly an aircraft.
 \textit{
	\begin{itemize}
	\item He spent seventeen years as an airline pilot.
	\item ...fighter pilots of the British Royal Air Force.
	\end{itemize}
}
\item countable noun \\
A \textbf{pilot} is a person who steers a ship through a difficult  stretch of water, for example the entrance to a harbour .
 \textit{
	\begin{itemize}
	\end{itemize}
}
\item verb \\
If someone \textbf{pilots} an aircraft or ship, they act as its pilot.
 \textit{
	\begin{itemize}
	\item He piloted his own plane part of the way to Washington.
	\end{itemize}
}
\item countable noun \\
A \textbf{pilot}  scheme or a \textbf{pilot} project is one which is used to test an idea before deciding whether to introduce it on a larger scale .
 \textit{
	\begin{itemize}
	\item The service is being expanded following the success of a pilot scheme.
	\item ...a ten-year pilot project backed by the trade and industry department.
	\end{itemize}
}
\item verb \\
If a government or organization \textbf{pilots} a programme or a scheme, they test it, before deciding whether to introduce it on
a larger scale.
 \textit{
	\begin{itemize}
	\item The trust is looking for 50 schools to pilot a programme aimed at teenage pupils.
	\end{itemize}
}
\item verb \\
If a government minister \textbf{pilots} a new law or bill through parliament , he or she makes sure that it is introduced successfully.
 \textit{
	\begin{itemize}
	\item His achievement in piloting the Bill through a querulous House of Commons was an
outstanding parliamentary feat.
	\end{itemize}
}
\item countable noun \\
A \textbf{pilot} or a \textbf{pilot episode} is a single television programme that is shown in order to find out whether a particular series of programmes is likely to be popular .
 \textit{
	\begin{itemize}
	\item A pilot episode has been shot and a full series has been ordered.
	\end{itemize}
}
\item countable noun \\
A \textbf{pilot} is the pilot light on a gas cooker or stove , boiler , or fire.
 \textit{
	\begin{itemize}
	\end{itemize}
}
\end{enumerate}

\section*{thirty}
{\large \color{blue}  thirties  }
\subsection*{Explain}
\begin{enumerate}
\item number \\
\textbf{Thirty} is the number 30.
 \textit{
	\begin{itemize}
	\item The building was built about thirty years ago.
	\end{itemize}
}
\item plural noun \\
When you talk about the \textbf{thirties} , you are referring to numbers between 30 and 39. For example , if you are \textbf{in} your \textbf{thirties} , you are aged between 30 and 39. If the temperature is \textbf{in the thirties} , the temperature is between 30 and 39 degrees .
 \textit{
	\begin{itemize}
	\item Mozart clearly enjoyed good health throughout his twenties and early thirties.
	\end{itemize}
}
\item plural noun \\
\textbf{The thirties} is the decade between 1930 and 1939.
 \textit{
	\begin{itemize}
	\item She became quite a notable director in the thirties and forties.
	\end{itemize}
}
\end{enumerate}

\section*{plane}
{\large \color{blue}  planes  planing  planed  }
\subsection*{Explain}
\begin{enumerate}
\item countable noun \\
A \textbf{plane} is a vehicle with wings and one or more engines, which can fly through the air.
 \textit{
	\begin{itemize}
	\item He had plenty of time to catch his plane.
	\item Her mother was killed in a plane crash.
	\item ...fighter planes.
	\end{itemize}
}
\item countable noun \\
A \textbf{plane} is a flat, level surface which may be sloping at a particular angle .
 \textit{
	\begin{itemize}
	\item ...a building with angled planes.
	\end{itemize}
}
\item singular noun \\
If a number of points are in the same \textbf{plane} , one line or one flat surface could pass through them all.
 \textit{
	\begin{itemize}
	\item All the planets orbit the Sun in roughly the same plane, round its equator.
	\end{itemize}
}
\item countable noun \\
If you say that something is \textbf{on a higher plane} , you mean that it is more spiritual or less concerned with ordinary things.
 \textit{
	\begin{itemize}
	\item ...life on a higher plane of existence.
	\item We felt we were living life on several different planes.
	\end{itemize}
}
\item countable noun \\
A \textbf{plane} is a tool that has a flat bottom with a sharp blade in it. You move the plane over a piece of wood in order to remove thin pieces of its surface.
 \textit{
	\begin{itemize}
	\end{itemize}
}
\item verb \\
If you \textbf{plane} a piece of wood, you make it smaller or smoother by using a plane.
 \textbf{Plane down} means the same as plane .
 \textit{
	\begin{itemize}
	\item She watches him plane the surface of a walnut board.
	\item Again I planed the surface flush.
	\item The piece was reduced in size by planing down the four corners.
	\end{itemize}
}
\item verb \\
If something such as a boat \textbf{planes}  \textbf{across} water, it moves quickly across the water, just touching the surface.
 \textit{
	\begin{itemize}
	\item All four of the boats planed across the Solent with the greatest of ease.
	\end{itemize}
}
\item countable noun \\
A \textbf{plane} or a \textbf{plane tree} is a large tree with broad leaves which often grows in towns.
 \textit{
	\begin{itemize}
	\end{itemize}
}
\end{enumerate}

\section*{three}
{\large \color{blue}  threes  }
\subsection*{Explain}
\begin{enumerate}
\item number \\
\textbf{Three} is the number 3.
 \textit{
	\begin{itemize}
	\item We waited three months before going back to see the specialist.
	\end{itemize}
}
\end{enumerate}

\section*{plenty}
{\large \color{blue}  }
\subsection*{Explain}
\begin{enumerate}
\item quantifier \\
If there is \textbf{plenty of} something, there is a large amount of it. If there are \textbf{plenty of} things, there are many of them. \textbf{Plenty} is used especially to indicate that there is enough of something, or more than you need .
 \textbf{Plenty} is also a pronoun.
 \textit{
	\begin{itemize}
	\item There was still plenty of time to take Jill out for pizza.
	\item Most businesses face plenty of competition.
	\item Taking plenty of exercise can be of great benefit.
	\item Are there plenty of fresh fruits and vegetables in your diet?
	\item I don't believe in long interviews. Fifteen minutes is plenty.
	\item She's got plenty to do these days.
	\end{itemize}
}
\item uncountable noun \\
\textbf{Plenty} is a situation in which people have a lot to eat or a lot of money to live on.
 \textit{
	\begin{itemize}
	\item You are all fortunate to be growing up in a time of peace and plenty.
	\item ...an area that has become a symbol of despair in the midst of America's plenty.
	\end{itemize}
}
\item adverb \\
You use \textbf{plenty} in front of adjectives or adverbs to emphasize the degree of the quality they are describing .
 \textit{
	\begin{itemize}
	\item The water looked plenty deep.
	\item The compartment is plenty big enough.
	\item Keep in mind you're going to be fighting lots of men who hit plenty hard.
	\end{itemize}
}
\item  \\
 in plenty \textit{
	\begin{itemize}
	\end{itemize}
}
\end{enumerate}

\section*{through}
{\large \color{blue}  }
\subsection*{Explain}
\begin{enumerate}
\item preposition \\
To move \textbf{through} something such as a hole , opening , or pipe means to move directly from one side or end of it to the other.
 \textbf{Through} is also an adverb .
 \textit{
	\begin{itemize}
	\item The theatre was evacuated when rain poured through the roof at the Liverpool Playhouse.
	\item Go straight through that door under the EXIT sign.
	\item Visitors enter through a side entrance.
	\item The main path continues through a tunnel of trees.
	\item He went straight through to the kitchen and took a can of cola from the fridge.
	\item She opened the door and stood back to allow the man to pass through.
	\end{itemize}
}
\item preposition \\
To cut \textbf{through} something means to cut it in two pieces or to make a hole in it.
 \textbf{Through} is also an adverb.
 \textit{
	\begin{itemize}
	\item A fish knife is designed to cut through the flesh but not the bones.
	\item Some rabbits have even taken to gnawing through the metal.
	\item Score deeper each time until the board is cut through.
	\end{itemize}
}
\item preposition \\
To go \textbf{through} a town, area, or country means to travel across it or in it.
 \textbf{Through} is also an adverb.
 \textit{
	\begin{itemize}
	\item Go up to Ramsgate, cross into France, go through Andorra and into Spain.
	\item ...travelling through pathless woods.
	\item The couple set off in August from Morocco, drove through the Sahara, visited Nigeria
and were heading for Zimbabwe.
	\item ...a memorable road trip through the California vineyards.
	\item Few know that the tribe was just passing through.
	\end{itemize}
}
\item preposition \\
If you move \textbf{through} a group of things or a mass of something, it is on either side of you or all around
you.
 \textbf{Through} is also an adverb.
 \textit{
	\begin{itemize}
	\item We made our way through the crowd to the river.
	\item Sybil's fingers ran through the water.
	\item Nancy kept running, plunging through the sand.
	\item He hurried through the rain, to the patrol car.
	\item He pushed his way through to the edge of the crowd where he waited.
	\end{itemize}
}
\item preposition \\
To get  \textbf{through} a barrier or obstacle means to get from one side of it to the other.
 \textbf{Through} is also an adverb.
 \textit{
	\begin{itemize}
	\item Allow twenty-five minutes to get through Passport Control and Customs.
	\item He was one of the last of the crowd to pass through the barrier.
	\item Traders generally travel safely through the border.
	\item ...a maze of barriers, designed to prevent vehicles driving straight through.
	\end{itemize}
}
\item preposition \\
If a driver  goes  \textbf{through} a red light, they keep driving even though they should stop .
 \textit{
	\begin{itemize}
	\item He was killed at a road junction by a van driver who went through a red light.
	\item We drove through red traffic lights, the horn blaring.
	\end{itemize}
}
\item preposition \\
If something goes into an object and comes out of the other side, you can say that it passes \textbf{through} the object.
 \textbf{Through} is also an adverb.
 \textit{
	\begin{itemize}
	\item The ends of the net pass through a wooden bar at each end.
	\item Zita was herself unconventional, keeping a safety-pin stuck through her ear lobe.
	\item I bored a hole so that the fixing bolt would pass through.
	\end{itemize}
}
\item preposition \\
To go \textbf{through} a system means to move around it or to pass from one end of it to the other.
 \textbf{Through} is also an adverb.
 \textit{
	\begin{itemize}
	\item ...electric currents travelling through copper wires.
	\item What a lot of cards you've got through the post!
	\item ...a child's successful passage through the education system.
	\item ...a resolution which would allow food aid to go through with fewer restrictions.
	\end{itemize}
}
\item preposition \\
If you see , hear , or feel something \textbf{through} a particular thing, that thing is between you and the thing you can see, hear, or
feel.
 \textit{
	\begin{itemize}
	\item Alice gazed pensively through the wet glass.
	\item They could hear music pulsing through the walls of the house.
	\item I am sure I can feel a vibration through the soles of my feet.
	\end{itemize}
}
\item preposition \\
If something such as a feeling, attitude , or quality, happens  \textbf{through} an area, organization, or a person's body, it happens everywhere in it or affects all of it.
 \textit{
	\begin{itemize}
	\item An atmosphere of anticipation vibrated through the crowd.
	\item The melody that ran through his brain was composed of bad notes.
	\item What was going through his mind when he spoke those amazing words?
	\item A mood of optimism swept through the company and its customers.
	\end{itemize}
}
\item preposition \\
If something happens or exists \textbf{through} a period of time, it happens or exists from the beginning until the end.
 \textbf{Through} is also an adverb.
 \textit{
	\begin{itemize}
	\item We're playing in New Zealand, Australia and Japan through November.
	\item Saga features trips for older people at home and abroad all through the year.
	\item She kept quiet all through breakfast.
	\item We've got a tough programme, hard work right through to the summer.
	\item He worked right through.
	\end{itemize}
}
\item preposition \\
If something happens from a particular period of time \textbf{through} another, it starts at the first period and continues until the end of the second period.
 \textit{
	\begin{itemize}
	\item ...open Monday through Sunday from 7:00 am to 10:00 pm.
	\item During her busy season (March through June), she often completes as many as fifty
paintings a week.
	\end{itemize}
}
\item preposition \\
If you go \textbf{through} a particular experience or event, you experience it, and if you behave in a particular way \textbf{through} it, you behave in that way while it is happening .
 \textit{
	\begin{itemize}
	\item Men go through a change of life emotionally just like women.
	\item ...a humorous woman who had lived through two world wars in Paris.
	\item Why was I putting myself through all this misery?
	\item Through it all, Mark was outwardly calm.
	\end{itemize}
}
\item adjective \\
If you are \textbf{through}  \textbf{with} something or if it is \textbf{through} , you have finished doing it and will never do it again. If you are \textbf{through}  \textbf{with} someone, you do not want to have anything to do with them again.
 \textit{
	\begin{itemize}
	\item I'm through with the explaining.
	\item Training as a counsellor would guarantee her employment once her schooling was through.
	\item They were through. They wanted out. Forever.
	\item I'm through with women.
	\end{itemize}
}
\item preposition \\
You use \textbf{through} in expressions such as \textbf{half-way through} and \textbf{all the way through} to indicate to what extent an action or task is completed.
 \textbf{Through} is also an adverb.
 \textit{
	\begin{itemize}
	\item A competitor collapsed half-way through the marathon.
	\item Stir the meat about until it turns white all the way through.
	\end{itemize}
}
\item preposition \\
If something happens because of something else, you can say that it happens \textbf{through} it.
 \textit{
	\begin{itemize}
	\item They are understood to have retired through age or ill health.
	\item The thought of someone suffering through a mistake of mine makes me shiver.
	\end{itemize}
}
\item preposition \\
You use \textbf{through} when stating the means by which a particular thing is achieved .
 \textit{
	\begin{itemize}
	\item Those who seek to grab power through violence deserve punishment.
	\item You simply can't get a ticket through official channels.
	\end{itemize}
}
\item preposition \\
If you do something \textbf{through} someone else, they take the necessary action for you.
 \textit{
	\begin{itemize}
	\item Do I need to go through my doctor or can I make an appointment direct?
	\item Speaking through an interpreter, he called for some new thinking from the West.
	\end{itemize}
}
\item adverb \\
If something such as a proposal or idea goes \textbf{through} , it is accepted by people in authority and is made legal or official.
 \textbf{Through} is also a preposition .
 \textit{
	\begin{itemize}
	\item It is possible that the present Governor General will be made interim President,
if the proposals go through.
	\item The secretary of state during the Nixon-Ford transition did not wish to push the
proposals through.
	\item They want to get the plan through Congress as quickly as possible.
	\end{itemize}
}
\item preposition \\
If someone gets \textbf{through} an examination or a round of a competition , they succeed or win .
 \textbf{Through} is also an adverb.
 \textit{
	\begin{itemize}
	\item She was bright, learned languages quickly, and sailed through her exams.
	\item All the seeded players got through the first round.
	\item Nigeria also go through from that group.
	\end{itemize}
}
\item adverb \\
When you get \textbf{through} while making a phone call, the call is connected and you can speak to the person you are phoning.
 \textit{
	\begin{itemize}
	\item He may find the line cut on the phone so that he can't get through.
	\item Smith tried to get through to Frank at Warm Springs the next morning.
	\end{itemize}
}
\item preposition \\
If you look or go \textbf{through} a lot of things, you look at them or deal with them one after the other.
 \textit{
	\begin{itemize}
	\item Let's go through the numbers together and see if a workable deal is possible.
	\item When you have finished your list of personal preferences, go through it again.
	\item David ran through the agreement with Guy, point by point.
	\item He, too, had a lot of paperwork to get through.
	\end{itemize}
}
\item preposition \\
If you read  \textbf{through} something, you read it from beginning to end.
 \textbf{Through} is also an adverb.
 \textit{
	\begin{itemize}
	\item She read through pages and pages of the music I had brought her.
	\item I only had time to skim through the script before I flew over here.
	\item He read the article straight through.
	\end{itemize}
}
\item adjective \\
A \textbf{through} train goes directly to a particular place, so that the people who want to go there
do not have to change trains.
 \textit{
	\begin{itemize}
	\item ...Britain's longest through train journey, 685 miles.
	\end{itemize}
}
\item adverb \\
If you say that someone or something is wet  \textbf{through} , you are emphasizing how wet they are.
 \textit{
	\begin{itemize}
	\item I returned to the inn cold and wet, soaked through by the drizzling rain.
	\item She went on crying, and cried and cried until the pillow was wet through.
	\end{itemize}
}
\item  \\
 through and through \textit{
	\begin{itemize}
	\end{itemize}
}
\end{enumerate}

\section*{potato}
{\large \color{blue}  potatoes  }
\subsection*{Explain}
\begin{enumerate}
\item variable noun \\
\textbf{Potatoes} are quite  round vegetables with brown or red skins and white  insides . They grow under the ground .
 \textit{
	\begin{itemize}
	\end{itemize}
}
\item  \\
 hot potato \textit{
	\begin{itemize}
	\end{itemize}
}
\end{enumerate}

\section*{prince}
{\large \color{blue}  princes  }
\subsection*{Explain}
\begin{enumerate}
\item title noun \\
A \textbf{prince} is a male member of a royal family, especially the son of the king or queen of a country.
 \textit{
	\begin{itemize}
	\item ...Prince Edward and other royal guests.
	\item The Prince won warm applause for his ideas.
	\end{itemize}
}
\item title noun \\
A \textbf{prince} is the male royal ruler of a small country or state.
 \textit{
	\begin{itemize}
	\item He was speaking without the prince's authority.
	\end{itemize}
}
\item countable noun \\
If someone describes a man as the \textbf{prince}  \textbf{of} a particular type of work, they mean that he is the best man doing that type of work.
 \textit{
	\begin{itemize}
	\item Britain's prince of pop is back.
	\end{itemize}
}
\end{enumerate}

\section*{toward}
{\large \color{blue}  }
\subsection*{Explain}
\begin{enumerate}
\item adjective \\
1.  2.  3.  \textit{
	\begin{itemize}
	\end{itemize}
}
\item preposition \\
4.  \textit{
	\begin{itemize}
	\end{itemize}
}
\end{enumerate}

\section*{realm}
{\large \color{blue}  realms  }
\subsection*{Explain}
\begin{enumerate}
\item countable noun \\
You can use \textbf{realm} to refer to any area of activity, interest, or thought .
 \textit{
	\begin{itemize}
	\item ...the realm of politics.
	\item Students' interests are mostly limited to the academic realm.
	\end{itemize}
}
\item countable noun \\
A \textbf{realm} is a country that has a king or queen .
 \textit{
	\begin{itemize}
	\item Defence of the realm is crucial.
	\end{itemize}
}
\item  \\
 the realm of possibility \textit{
	\begin{itemize}
	\end{itemize}
}
\end{enumerate}

\section*{shot}
{\large \color{blue}  shots  }
\subsection*{Explain}
\begin{enumerate}
\item  \\
\textbf{Shot} is the past  tense and past participle of shoot .
 \textit{
	\begin{itemize}
	\end{itemize}
}
\item countable noun \\
A \textbf{shot} is an act of firing a gun .
 \textit{
	\begin{itemize}
	\item He had murdered Perceval at point blank range with a single shot.
	\item They fired a volley of shots at the target.
	\end{itemize}
}
\item countable noun \\
Someone who is a good \textbf{shot} can shoot well . Someone who is a bad  \textbf{shot} cannot shoot well.
 \textit{
	\begin{itemize}
	\item He was not a particularly good shot because of his eyesight.
	\end{itemize}
}
\item countable noun \\
In sports such as football , golf , or tennis , a \textbf{shot} is an act of kicking , hitting, or throwing the ball, especially in an attempt to score a point.
 \textit{
	\begin{itemize}
	\item He had only one shot at goal.
	\end{itemize}
}
\item countable noun \\
A \textbf{shot} is a photograph or a particular sequence of pictures in a film.
 \textit{
	\begin{itemize}
	\item ...a shot of a fox peering from the bushes.
	\item The gleaming monochrome shots of cobbled streets are drained of colour.
	\end{itemize}
}
\item countable noun \\
If you have a \textbf{shot at} something, you attempt to do it.
 \textit{
	\begin{itemize}
	\item The heavyweight champion will be given a shot at Holyfield's world title.
	\end{itemize}
}
\item countable noun \\
A \textbf{shot}  \textbf{of} a drug is an injection of it.
 \textit{
	\begin{itemize}
	\item He administered a shot of Nembutal.
	\end{itemize}
}
\item countable noun \\
A \textbf{shot}  \textbf{of} a strong alcoholic drink is a small glass of it.
 \textit{
	\begin{itemize}
	\item ...a shot of vodka.
	\item ...spirits and liqueurs, served in a shot glass.
	\end{itemize}
}
\item  \\
 give something your best shot \textit{
	\begin{itemize}
	\end{itemize}
}
\item  \\
 a shot across the bow \textit{
	\begin{itemize}
	\end{itemize}
}
\item  \\
 call the shots \textit{
	\begin{itemize}
	\end{itemize}
}
\item  \\
 like a shot \textit{
	\begin{itemize}
	\end{itemize}
}
\item  \\
 a long shot \textit{
	\begin{itemize}
	\end{itemize}
}
\item  \\
 by a long shot \textit{
	\begin{itemize}
	\end{itemize}
}
\item  \\
 be shot through with \textit{
	\begin{itemize}
	\end{itemize}
}
\end{enumerate}

\section*{underneath}
{\large \color{blue}  }
\subsection*{Explain}
\begin{enumerate}
\item preposition \\
If one thing is \textbf{underneath} another, it is directly under it, and may be covered or hidden by it.
 \textbf{Underneath} is also an adverb .
 \textit{
	\begin{itemize}
	\item The device exploded underneath a van.
	\item ...using dogs to locate people trapped underneath collapsed buildings.
	\item ...a table for two underneath the olive trees.
	\item Her apartment was underneath a bar, called 'The Lift'.
	\item He has on a denim shirt with a T-shirt underneath.
	\item ...if we could maybe pull back a bit of this carpet to see what's underneath.
	\item The shooting-range is lit from underneath by rows of ruby-red light fittings.
	\end{itemize}
}
\item adverb \\
The part of something which is \textbf{underneath} is the part which normally  touches the ground or faces towards the ground.
 \textbf{Underneath} is also an adjective .
 \textbf{Underneath} is also a noun .
 \textit{
	\begin{itemize}
	\item Check the actual construction of the chair by looking underneath.
	\item The sand martin is a brown bird with white underneath.
	\item His bare feet were smooth on top and rough-skinned underneath.
	\item Some objects had got entangled with the underneath mechanism of the engine.
	\item Now I know what the underneath of a car looks like.
	\end{itemize}
}
\item adverb \\
You use \textbf{underneath} when talking about feelings and emotions that people do not show in their behaviour.
 \textbf{Underneath} is also a preposition .
 \textit{
	\begin{itemize}
	\item He was as violent as Nick underneath.
	\item Underneath, Sofia was deeply committed to her partner.
	\item Underneath his outgoing behaviour Luke was shy.
	\end{itemize}
}
\end{enumerate}

\section*{switch}
{\large \color{blue}  switches  switching  switched  }
\subsection*{Explain}
\begin{enumerate}
\item countable noun \\
A \textbf{switch} is a small control for an electrical device which you use to turn the device on or
off.
 \textit{
	\begin{itemize}
	\item Leona put some detergent into the dishwasher, shut the door and pressed the switch.
	\item ...a light switch.
	\end{itemize}
}
\item verb \\
If you \textbf{switch}  \textbf{to} something different, for example to a different system, task , or subject of conversation , you change to it from what you were doing or saying before.
 \textbf{Switch} is also a noun .
 \textbf{Switch over} means the same as switch .
 \textit{
	\begin{itemize}
	\item Estonia is switching to a market economy.
	\item The law would encourage companies to switch from coal to cleaner fuels.
	\item The encouragement of a friend spurred Chris into switching jobs.
	\item New technology made a switch to oil possible.
	\item The spokesman implicitly condemned the government's policy switch.
	\item ...a professional man who started out in law but switched over to medicine.
	\end{itemize}
}
\item verb \\
If you \textbf{switch} your attention from one thing \textbf{to} another or if your attention \textbf{switches} , you stop paying attention to the first thing and start paying attention to the second.
 \textit{
	\begin{itemize}
	\item My mother's interest had switched to my health.
	\item As the era wore on, she switched her attention to films.
	\end{itemize}
}
\item verb \\
If you \textbf{switch} two things, you replace one with the other.
 \textit{
	\begin{itemize}
	\item In half an hour, they'd switched the tags on every cable.
	\item The ballot boxes have been switched.
	\end{itemize}
}
\end{enumerate}

\section*{upon}
{\large \color{blue}  }
\subsection*{Explain}
\begin{enumerate}
\item preposition \\
If one thing is \textbf{upon} another, it is on it.
 \textit{
	\begin{itemize}
	\item He set the tray upon the table.
	\item He bent forward and laid a kiss softly upon her forehead.
	\item I imagined the eyes of the others in the room upon me.
	\end{itemize}
}
\item preposition \\
You use \textbf{upon} when mentioning an event that is followed immediately by another event.
 \textit{
	\begin{itemize}
	\item The door on the left, upon entering the church, leads to the Crypt of St Issac.
	\item Upon conclusion of these studies, the patient was told that she had a severe problem.
	\end{itemize}
}
\item preposition \\
You use \textbf{upon} between two occurrences of the same noun in order to say that there are large numbers of the thing mentioned.
 \textit{
	\begin{itemize}
	\item Row upon row of women surged forwards.
	\item I looked across the mountains, ridge upon ridge.
	\end{itemize}
}
\item preposition \\
If an event is \textbf{upon} you, it is just about to happen .
 \textit{
	\begin{itemize}
	\item The long-threatened storm was upon us.
	\item The wedding season is upon us.
	\item They had to conserve the candles now with winter upon them.
	\end{itemize}
}
\end{enumerate}

\section*{waitress}
{\large \color{blue}  waitresses  waitressing  waitressed  }
\subsection*{Explain}
\begin{enumerate}
\item countable noun \\
A \textbf{waitress} is a woman who works in a restaurant, serving people with food and drink .
 \textit{
	\begin{itemize}
	\end{itemize}
}
\item verb \\
A woman who \textbf{waitresses} works in a restaurant serving food and drink.
 \textit{
	\begin{itemize}
	\item She had been working in a pub, cooking and waitressing.
	\end{itemize}
}
\end{enumerate}

\section*{versus}
{\large \color{blue}  }
\subsection*{Explain}
\begin{enumerate}
\item preposition \\
You use \textbf{versus} to indicate that two figures , ideas , or choices are opposed.
 \textit{
	\begin{itemize}
	\item Only 18.8% of the class of 1982 had some kind of diploma four years after high school,
versus 45% of the class of 1972.
	\item ...bottle-feeding versus breastfeeding.
	\end{itemize}
}
\item preposition \\
\textbf{Versus} is used to indicate that two teams or people are competing against each other in a sporting event.
 \textit{
	\begin{itemize}
	\item Italy versus Japan is turning out to be a surprisingly well-matched competition.
	\item ...two seats for the Red Sox versus New York Yankees match.
	\end{itemize}
}
\end{enumerate}

\section*{weather}
{\large \color{blue}  weathers  weathering  weathered  }
\subsection*{Explain}
\begin{enumerate}
\item uncountable noun \\
The \textbf{weather} is the condition of the atmosphere in one area at a particular time, for example if it is raining , hot , or windy .
 \textit{
	\begin{itemize}
	\item The weather was bad.
	\item I like cold weather.
	\item Fishing is possible in virtually any weather.
	\item ...the weather conditions.
	\end{itemize}
}
\item verb \\
If something such as wood or rock \textbf{weathers} or \textbf{is weathered} , it changes colour or shape as a result of the wind, sun , rain, or cold .
 \textit{
	\begin{itemize}
	\item Unpainted wooden furniture weathers to a grey colour.
	\item This rock has been weathered and eroded.
	\end{itemize}
}
\item verb \\
If you \textbf{weather} a difficult time or a difficult situation, you survive it and are able to continue  normally after it has passed or ended.
 \textit{
	\begin{itemize}
	\item The company has weathered the recession.
	\item The government has weathered its worst political crisis.
	\end{itemize}
}
\item  \\
 keep a weather eye on sb/sth \textit{
	\begin{itemize}
	\end{itemize}
}
\item  \\
 to make heavy weather of something \textit{
	\begin{itemize}
	\end{itemize}
}
\item  \\
 under the weather \textit{
	\begin{itemize}
	\end{itemize}
}
\end{enumerate}

\section*{without}
{\large \color{blue}  }
\subsection*{Explain}
\begin{enumerate}
\item preposition \\
You use \textbf{without} to indicate that someone or something does not have or use the thing mentioned .
 \textit{
	\begin{itemize}
	\item I don't like myself without a beard.
	\item She wore a brown shirt pressed without a wrinkle.
	\item ...a meal without barbecue sauce.
	\end{itemize}
}
\item preposition \\
If one thing happens  \textbf{without} another thing, or if you do something \textbf{without} doing something else, the second thing does not happen or occur.
 \textit{
	\begin{itemize}
	\item He was offered a generous pension provided he left without a fuss.
	\item They worked without a break until about eight in the evening.
	\item Alex had done this without consulting her.
	\end{itemize}
}
\item preposition \\
If you do something \textbf{without} a particular feeling, you do not have that feeling when you do it.
 \textit{
	\begin{itemize}
	\item Janet Magnusson watched his approach without enthusiasm.
	\item 'Hello, Swanson,' he said without surprise.
	\end{itemize}
}
\item preposition \\
If you do something \textbf{without} someone else, they are not in the same place as you are or are not involved in the
same action as you.
 \textit{
	\begin{itemize}
	\item I told Franklin he would have to start dinner without me.
	\item How can I rebuild my life without you?
	\item We would never go anywhere without you.
	\end{itemize}
}
\end{enumerate}

\section*{artist}
{\large \color{blue}  artists  }
\subsection*{Explain}
\begin{enumerate}
\item countable noun \\
An \textbf{artist} is someone who draws or paints  pictures or creates sculptures as a job or a hobby .
 \textit{
	\begin{itemize}
	\item ...the studio of a great artist.
	\item Each poster is signed by the artist.
	\item I'm not a good artist.
	\end{itemize}
}
\item countable noun \\
An \textbf{artist} is a person who creates novels , poems , films, or other things which can be considered as works of art.
 \textit{
	\begin{itemize}
	\item His books are enormously easy to read, yet he is a serious artist.
	\end{itemize}
}
\item countable noun \\
An \textbf{artist} is a performer such as a musician , actor , or dancer.
 \textit{
	\begin{itemize}
	\item ...a popular artist who has sold millions of records.
	\end{itemize}
}
\item countable noun \\
If you say that someone is an \textbf{artist} at a particular activity, you mean they are very skilled at it.
 \textit{
	\begin{itemize}
	\item He is an exceptional footballer–an artist.
	\end{itemize}
}
\end{enumerate}

\section*{across}
{\large \color{blue}  }
\subsection*{Explain}
\begin{enumerate}
\item preposition \\
If someone or something goes  \textbf{across} a place or a boundary, they go from one side of it to the other.
 \textbf{Across} is also an adverb .
 \textit{
	\begin{itemize}
	\item She walked across the floor and lay down on the bed.
	\item He watched Karl run across the street to Tommy.
	\item ...an expedition across Africa.
	\item Richard stood up and walked across to the window.
	\end{itemize}
}
\item preposition \\
If something is situated or stretched  \textbf{across} something else, it is situated or stretched from one side of it to the other.
 \textbf{Across} is also an adverb.
 \textit{
	\begin{itemize}
	\item ...the floating bridge across Lake Washington in Seattle.
	\item He scrawled his name across the bill.
	\item Lucy had strung a banner across the wall saying 'Welcome Home'.
	\item Trim toenails straight across using nail clippers.
	\end{itemize}
}
\item preposition \\
If something is lying  \textbf{across} an object or place, it is resting on it and partly  covering it.
 \textit{
	\begin{itemize}
	\item She found her clothes lying across the chair.
	\item The wind pushed his hair across his face.
	\end{itemize}
}
\item preposition \\
Something that is \textbf{across} something such as a street , river, or area is on the other side of it.
 \textbf{Across} is also an adverb.
 \textit{
	\begin{itemize}
	\item Anyone from the houses across the road could see him.
	\item When I saw you across the room I knew I'd met you before.
	\item They parked across from the Castro Theatre.
	\item He pulled up a chair and sat down across from Michael.
	\end{itemize}
}
\item adverb \\
If you look  \textbf{across} at a place, person, or thing, you look towards them.
 \textit{
	\begin{itemize}
	\item He glanced across at his sleeping wife.
	\item She rose from the chair and gazed across at him.
	\item ...breathtaking views across to the hills.
	\end{itemize}
}
\item preposition \\
You use \textbf{across} to say that a particular expression is shown on someone's face .
 \textit{
	\begin{itemize}
	\item An enormous grin spread across his face.
	\item For a moment a shadow seemed to pass across Roy's face.
	\end{itemize}
}
\item preposition \\
If someone hits you \textbf{across} the face or head , they hit you on that part.
 \textit{
	\begin{itemize}
	\item Graham hit him across the face with the gun.
	\end{itemize}
}
\item preposition \\
When something happens  \textbf{across} a place or organization , it happens equally everywhere within it.
 \textit{
	\begin{itemize}
	\item The film opens across America in December.
	\item Thousands of farmers from across Europe have held a huge demonstration in the centre
of Brussels.
	\item 2,000 workers across all state agencies are to be fired by March 31st.
	\end{itemize}
}
\item preposition \\
When something happens \textbf{across} a political , religious , or social barrier, it involves people in different groups.
 \textit{
	\begin{itemize}
	\item ...parties competing across the political spectrum.
	\end{itemize}
}
\item adverb \\
\textbf{Across} is used in measurements to show the width of something.
 \textit{
	\begin{itemize}
	\item This hand-decorated plate measures 30cm across.
	\item The snails are no larger than one centimetre across.
	\end{itemize}
}
\end{enumerate}

\section*{astronaut}
{\large \color{blue}  astronauts  }
\subsection*{Explain}
\begin{enumerate}
\item countable noun \\
An \textbf{astronaut} is a person who is trained for travelling in a spacecraft .
 \textit{
	\begin{itemize}
	\end{itemize}
}
\end{enumerate}

\section*{amid}
{\large \color{blue}  }
\subsection*{Explain}
\begin{enumerate}
\item preposition \\
If something happens  \textbf{amid}  noises or events of some kind , it happens while the other things are happening .
 \textit{
	\begin{itemize}
	\item A senior leader cancelled a trip to Britain yesterday amid growing signs of a possible
political crisis.
	\item Children were changing classrooms amid laughter and shouts.
	\end{itemize}
}
\item preposition \\
If something is \textbf{amid} other things, it is surrounded by them.
 \textit{
	\begin{itemize}
	\item ...a tiny bungalow amid clusters of trees.
	\end{itemize}
}
\end{enumerate}

\section*{carpenter}
{\large \color{blue}  carpenters  }
\subsection*{Explain}
\begin{enumerate}
\item countable noun \\
A \textbf{carpenter} is a person whose job is making and repairing  wooden things.
 \textit{
	\begin{itemize}
	\end{itemize}
}
\end{enumerate}

\section*{anybody}
{\large \color{blue}  }
\subsection*{Explain}
\begin{enumerate}
\item pronoun \\
\textbf{Anybody} means the same as anyone .
 \textit{
	\begin{itemize}
	\end{itemize}
}
\end{enumerate}

\section*{clone}
{\large \color{blue}  clones  cloning  cloned  }
\subsection*{Explain}
\begin{enumerate}
\item countable noun \\
If someone or something is a \textbf{clone} of another person or thing, they are so similar to this person or thing that they
 seem to be exactly the same as them.
 \textit{
	\begin{itemize}
	\item Tom was in some ways a younger clone of his handsome father.
	\item Designers are mistaken if they believe we all want to be supermodel clones.
	\end{itemize}
}
\item countable noun \\
A \textbf{clone} is an animal or plant that has been produced artificially, for example in a laboratory, from the cells of another animal or plant. A clone is exactly the
same as the original animal or plant.
 \textit{
	\begin{itemize}
	\end{itemize}
}
\item verb \\
To \textbf{clone} an animal or plant means to produce it as a clone.
 \textit{
	\begin{itemize}
	\item The idea of cloning extinct life forms still belongs to science fiction.
	\end{itemize}
}
\end{enumerate}

\section*{anyone}
{\large \color{blue}  }
\subsection*{Explain}
\begin{enumerate}
\item pronoun \\
You use \textbf{anyone} or \textbf{anybody} in statements with negative  meaning to indicate in a general way that nobody is present or involved in an action.
 \textit{
	\begin{itemize}
	\item I won't tell anyone I saw you here.
	\item You needn't talk to anyone if you don't want to.
	\item He was far too scared to tell anybody.
	\item Presidents are not any different from anybody else; they're human beings.
	\end{itemize}
}
\item pronoun \\
You use \textbf{anyone} or \textbf{anybody} in questions and conditional  clauses to ask or talk about whether someone is present or doing something.
 \textit{
	\begin{itemize}
	\item Why would anyone want that job?
	\item How can anyone look sad at an occasion like this?
	\item If anyone deserves to be happy, you do.
	\end{itemize}
}
\item pronoun \\
You use \textbf{anyone} or \textbf{anybody} before words which indicate the kind of person you are talking about.
 \textit{
	\begin{itemize}
	\item I always had been the person who achieved things before anyone else at my age.
	\item It's not a job for anyone who is slow with numbers.
	\item Anybody interested in pop culture at all should buy this movie.
	\end{itemize}
}
\item pronoun \\
You use \textbf{anyone} or \textbf{anybody} to refer to a person when you are emphasizing that it could be any person out of a very large number of people.
 \textit{
	\begin{itemize}
	\item Anyone could be doing what I'm doing.
	\item Al Smith could make anybody laugh.
	\end{itemize}
}
\item  \\
 anyone who is anyone/anybody who is anybody \textit{
	\begin{itemize}
	\end{itemize}
}
\end{enumerate}

\section*{cloud}
{\large \color{blue}  clouds  clouding  clouded  }
\subsection*{Explain}
\begin{enumerate}
\item variable noun \\
A \textbf{cloud} is a mass of water vapour that floats in the sky. Clouds are usually white or grey in colour.
 \textit{
	\begin{itemize}
	\item ...the varied shapes of the clouds.
	\item The sky was almost entirely obscured by cloud.
	\item ...the risks involved in flying through cloud.
	\end{itemize}
}
\item countable noun \\
A \textbf{cloud}  \textbf{of} something such as smoke or dust is a mass of it floating in the air.
 \textit{
	\begin{itemize}
	\item The hens darted away on all sides, raising a cloud of dust.
	\end{itemize}
}
\item verb \\
If you say that something \textbf{clouds} your view of a situation, you mean that it makes you unable to understand the situation or judge it properly.
 \textit{
	\begin{itemize}
	\item Perhaps anger had clouded his vision, perhaps his judgment had been faulty.
	\item In his latter years religious mania clouded his mind.
	\end{itemize}
}
\item verb \\
If you say that something \textbf{clouds} a situation, you mean that it makes it unpleasant .
 \textit{
	\begin{itemize}
	\item His last years were clouded by financial difficulties.
	\end{itemize}
}
\item ergative verb \\
If your eyes or face \textbf{cloud} or if sadness or anger  \textbf{clouds} them, your eyes or your face suddenly show sadness or anger.
 \textbf{Cloud over} means the same as cloud .
 \textit{
	\begin{itemize}
	\item Trish's face clouded with disappointment.
	\item As he looked at Katherine, great sorrow clouded his eyes.
	\item I saw Sean's face cloud over at this blatant lie.
	\end{itemize}
}
\item verb \\
If glass \textbf{clouds} or if moisture  \textbf{clouds} it, tiny  drops of water cover the glass, making it difficult to see through.
 \textit{
	\begin{itemize}
	\item The mirror clouded beside her cheek.
	\item I run the water very hot, clouding the mirror.
	\end{itemize}
}
\item  \\
 cloud-based \textit{
	\begin{itemize}
	\end{itemize}
}
\item  \\
 have one's head in the clouds \textit{
	\begin{itemize}
	\end{itemize}
}
\item  \\
 on cloud nine \textit{
	\begin{itemize}
	\end{itemize}
}
\item  \\
 under a cloud \textit{
	\begin{itemize}
	\end{itemize}
}
\end{enumerate}

\section*{both}
{\large \color{blue}  }
\subsection*{Explain}
\begin{enumerate}
\item determiner \\
You use \textbf{both} when you are referring to two people or things and saying that something is true about each of them.
 \textbf{Both} is also a quantifier .
 \textbf{Both} is also a pronoun.
 \textbf{Both} is also an emphasizing pronoun.
 \textbf{Both} is also a predeterminer.
 \textit{
	\begin{itemize}
	\item She cried out in fear and flung both arms up to protect her face.
	\item Put both vegetables into a bowl and crush with a potato masher.
	\item Both of these women have strong memories of the Vietnam War.
	\item We're going to Andreas's Boutique to pick out something original for both of us.
	\item Miss Brown and her friend, both from Stoke, were arrested on the 8th of June.
	\item Will there be public-works programmes, or community service, or both?
	\item He visited the Institute of Neurology in Havana where they both worked.
	\item 'Well, I'll leave you both, then,' said Gregory.
	\item Both the band's writers are fascinating lyricists.
	\item Both the horses were out, tacked up and ready to ride.
	\end{itemize}
}
\item conjunction \\
You use the structure \textbf{both...and} when you are giving two facts or alternatives and emphasizing that each of them is true or possible .
 \textit{
	\begin{itemize}
	\item Now women work both before and after having their children.
	\item Any such action would have to be approved by both American and Saudi leaders.
	\end{itemize}
}
\end{enumerate}

\section*{distinction}
{\large \color{blue}  distinctions  }
\subsection*{Explain}
\begin{enumerate}
\item countable noun \\
A \textbf{distinction}  \textbf{between}  similar things is a difference .
 \textit{
	\begin{itemize}
	\item There are obvious distinctions between the two wine-making areas.
	\item The distinction between craft and fine art is more controversial.
	\end{itemize}
}
\item uncountable noun \\
\textbf{Distinction} is the quality of being very good or better than other things of the same type.
 \textit{
	\begin{itemize}
	\item Lewis emerges as a composer of distinction and sensitivity.
	\item ...pieces of furniture of distinction.
	\end{itemize}
}
\item countable noun \\
A \textbf{distinction} is a special award or honour that is given to someone because of their very high  level of achievement .
 \textit{
	\begin{itemize}
	\item The order was created in 1902 as a special distinction for eminent men and women.
	\item I did an M.A. at Liverpool University in Latin American Studies and got a distinction.
	\end{itemize}
}
\item singular noun \\
If you say that someone or something has \textbf{the}  \textbf{distinction}  \textbf{of} being something, you are drawing  attention to the fact that they have the special quality of being that thing. \textbf{Distinction} is normally used to refer to good qualities, but can  sometimes  also be used to refer to bad qualities.
 \textit{
	\begin{itemize}
	\item He has the distinction of being regarded as the country's greatest living writer.
	\end{itemize}
}
\end{enumerate}

\section*{concerning}
{\large \color{blue}  }
\subsection*{Explain}
\begin{enumerate}
\item preposition \\
You use \textbf{concerning} to indicate what a question or piece of information is about.
 \textit{
	\begin{itemize}
	\item For more information concerning the club contact I. Coldwell.
	\item ...various questions concerning pollution and the environment.
	\end{itemize}
}
\item adjective \\
If something is \textbf{concerning} , it causes you to feel concerned about it.
 \textit{
	\begin{itemize}
	\item It is particularly concerning that he is working for non-British companies while
advising on foreign policy.
	\end{itemize}
}
\end{enumerate}

\section*{doom}
{\large \color{blue}  dooms  dooming  doomed  }
\subsection*{Explain}
\begin{enumerate}
\item uncountable noun \\
\textbf{Doom} is a terrible future state or event which you cannot prevent .
 \textit{
	\begin{itemize}
	\item ...his warnings of impending doom.
	\item ...a wicked mermaid who lured sailors to their doom.
	\end{itemize}
}
\item uncountable noun \\
If you have a sense or feeling of \textbf{doom} , you feel that things are going very badly and are likely to get  even worse.
 \textit{
	\begin{itemize}
	\item Why are people so full of gloom and doom?
	\item Attendance figures have dropped, creating a mood of doom among theatre directors.
	\end{itemize}
}
\item verb \\
If a fact or event \textbf{dooms} someone or something \textbf{to} a particular fate, it makes certain that they are going to suffer in some way.
 \textit{
	\begin{itemize}
	\item That argument doomed their marriage to failure.
	\end{itemize}
}
\end{enumerate}

\section*{during}
{\large \color{blue}  }
\subsection*{Explain}
\begin{enumerate}
\item preposition \\
If something happens  \textbf{during} a period of time or an event, it happens continuously, or happens several times between
the beginning and end of that period or event.
 \textit{
	\begin{itemize}
	\item Sandstorms are common during the Saudi Arabian winter.
	\item Plants need to be looked after and protected during bad weather.
	\end{itemize}
}
\item preposition \\
If something develops \textbf{during} a period of time, it develops gradually from the beginning to the end of that period.
 \textit{
	\begin{itemize}
	\item Wages have fallen by more than twenty percent during the past two months.
	\item Industrial production has expanded during the last three decades.
	\end{itemize}
}
\item preposition \\
An event that happens \textbf{during} a period of time happens at some point or moment in that period.
 \textit{
	\begin{itemize}
	\item During his visit, the Pope will also bless the new hospital.
	\end{itemize}
}
\end{enumerate}

\section*{element}
{\large \color{blue}  elements  }
\subsection*{Explain}
\begin{enumerate}
\item countable noun \\
The different \textbf{elements} of something are the different parts it contains.
 \textit{
	\begin{itemize}
	\item The exchange of prisoners of war was one of the key elements of the U.N.'s peace
plan.
	\end{itemize}
}
\item countable noun \\
A particular \textbf{element} of a situation, activity, or process is an important quality or feature that it has or needs .
 \textit{
	\begin{itemize}
	\item Fitness has now become an important element in our lives.
	\end{itemize}
}
\item countable noun \\
When you talk about \textbf{elements} within a society or organization, you are referring to groups of people who have similar aims , beliefs , or habits .
 \textit{
	\begin{itemize}
	\item ...criminal elements within the security forces.
	\item ...the hooligan element.
	\end{itemize}
}
\item countable noun \\
If something has an \textbf{element}  \textbf{of} a particular quality or emotion , it has a certain amount of this quality or emotion.
 \textit{
	\begin{itemize}
	\item These reports clearly contain elements of propaganda.
	\end{itemize}
}
\item countable noun \\
An \textbf{element} is a substance such as gold , oxygen , or carbon that consists of only one type of atom.
 \textit{
	\begin{itemize}
	\end{itemize}
}
\item countable noun \\
The \textbf{element} in an electric fire or water heater is the metal part which changes the electric current into heat .
 \textit{
	\begin{itemize}
	\item With its unique heating element it makes perfect coffee.
	\end{itemize}
}
\item plural noun \\
You can refer to the weather , especially wind and rain, as \textbf{the elements} .
 \textit{
	\begin{itemize}
	\item The area where most refugees are waiting is exposed to the elements.
	\end{itemize}
}
\item  \\
 in one's element \textit{
	\begin{itemize}
	\end{itemize}
}
\end{enumerate}

\section*{everybody}
{\large \color{blue}  }
\subsection*{Explain}
\begin{enumerate}
\item  \\
\textbf{Everybody} means the same as everyone .
 \textit{
	\begin{itemize}
	\end{itemize}
}
\end{enumerate}

\section*{expert}
{\large \color{blue}  experts  }
\subsection*{Explain}
\begin{enumerate}
\item countable noun \\
An \textbf{expert} is a person who is very skilled at doing something or who knows a lot about a particular subject .
 \textit{
	\begin{itemize}
	\item Our team of experts will be on hand to offer help and advice between 12 noon and
7pm daily.
	\item ...a yoga expert.
	\item ...an expert on trade in that area.
	\end{itemize}
}
\item adjective \\
Someone who is \textbf{expert}  \textbf{at} doing something is very skilled at it.
 \textit{
	\begin{itemize}
	\item The Japanese are expert at lowering manufacturing costs.
	\item There is a great deal to learn from Hal's expert approach.
	\end{itemize}
}
\item adjective \\
If you say that someone has \textbf{expert}  hands or an \textbf{expert}  eye , you mean that they are very skilful or experienced in using their hands or eyes for a particular purpose .
 \textit{
	\begin{itemize}
	\item When the horse suffered a back injury Harvey cured it with his own expert hands.
	\item The symptoms are very mild and it takes an expert eye to see them.
	\end{itemize}
}
\item adjective \\
\textbf{Expert}  advice or help is given by someone who has studied a subject thoroughly or who is very skilled at a particular job .
 \textit{
	\begin{itemize}
	\item You'll also get expert advice on keeping your hair in good condition.
	\item We'll need an expert opinion.
	\item The good news is that expert help is now available.
	\end{itemize}
}
\end{enumerate}

\section*{everyone}
{\large \color{blue}  }
\subsection*{Explain}
\begin{enumerate}
\item pronoun \\
You use \textbf{everyone} or \textbf{everybody} to refer to all the people in a particular group.
 \textit{
	\begin{itemize}
	\item Everyone in the street was shocked when they heard the news.
	\item When everyone else goes home around 5 p.m. Lynn is still hard at work.
	\item Not everyone thinks that the government is being particularly generous.
	\end{itemize}
}
\item pronoun \\
You use \textbf{everyone} or \textbf{everybody} to refer to all people.
 \textit{
	\begin{itemize}
	\item Everyone feels like a failure at times.
	\item Everyone needs some free time for rest and relaxation.
	\item You can't keep everybody happy.
	\end{itemize}
}
\end{enumerate}

\section*{expertise}
{\large \color{blue}  }
\subsection*{Explain}
\begin{enumerate}
\item uncountable noun \\
\textbf{Expertise} is special skill or knowledge that is acquired by training, study, or practice.
 \textit{
	\begin{itemize}
	\item The problem is that most local authorities lack the expertise to deal sensibly in
this market.
	\end{itemize}
}
\end{enumerate}

\section*{except}
{\large \color{blue}  }
\subsection*{Explain}
\begin{enumerate}
\item preposition \\
You use \textbf{except} to introduce the only thing or person that a statement does not apply to, or a fact that prevents
a statement from being completely true.
 \textbf{Except} is also a conjunction .
 \textit{
	\begin{itemize}
	\item I wouldn't have accepted anything except a job in Europe.
	\item I don't take any drugs whatsoever, except aspirin for colds.
	\item Children who take exams early will be allowed to drop a subject except in the case
of maths, English and science.
	\item Freddie would tell me nothing about what he was writing, except that it was to be
a Christmas play.
	\item The log cabin stayed empty, except when we came.
	\item Ida would not speak to him except to answer questions.
	\item Nothing more to do now except wait.
	\end{itemize}
}
\item  \\
 except for \textit{
	\begin{itemize}
	\end{itemize}
}
\end{enumerate}

\section*{fare}
{\large \color{blue}  fares  faring  fared  }
\subsection*{Explain}
\begin{enumerate}
\item countable noun \\
A \textbf{fare} is the money that you pay for a journey that you make, for example , in a bus, train, or taxi.
 \textit{
	\begin{itemize}
	\item He could barely afford the railway fare.
	\item ...taxi fares.
	\end{itemize}
}
\item uncountable noun \\
The \textbf{fare} at a restaurant or café is the type of food that is served there.
 \textit{
	\begin{itemize}
	\item The fare has much improved since Hugh has taken charge of the kitchen.
	\item ...traditional Portuguese fare in a traditional setting.
	\end{itemize}
}
\item verb \\
If you say that someone or something \textbf{fares}  well or badly , you are referring to the degree of success they achieve in a particular situation or activity .
 \textit{
	\begin{itemize}
	\item It is unlikely that the marine industry will fare any better in September.
	\item Some later expeditions fared better, though they were no better equipped.
	\end{itemize}
}
\end{enumerate}

\section*{her}
{\large \color{blue}  }
\subsection*{Explain}
\begin{enumerate}
\item pronoun \\
You use \textbf{her} to refer to a woman, girl , or female animal.
 \textbf{Her} is also a possessive  determiner .
 \textit{
	\begin{itemize}
	\item I went in the room and told her I had something to say to her.
	\item Catherine could not give her the advice she most needed.
	\item I really thought I'd lost her. Everybody kept asking me, 'Have you found your cat?'
	\item Liz travelled round the world for a year with her boyfriend James.
	\item We admire her courage, compassion and dedication.
	\item ...a black dog, her hair erect along the centre of her back.
	\end{itemize}
}
\item pronoun \\
In written English, \textbf{her} is sometimes used to refer to a person without saying whether that person is a man or a woman. Some people dislike this use and prefer to use 'him or her' or 'them'.
 \textbf{Her} is also a possessive determiner.
 \textit{
	\begin{itemize}
	\item Talk to your baby, play games, and show her how much you enjoy her company.
	\item The non-drinking, non smoking model should do nothing to risk her reputation.
	\end{itemize}
}
\item pronoun \\
\textbf{Her} is sometimes used to refer to a country or nation.
 \textbf{Her} is also a possessive determiner.
 \textit{
	\begin{itemize}
	\item Our reporter looks at reactions to Britain's apparently deep-rooted distrust of her
E.U. partner.
	\end{itemize}
}
\item singular pronoun \\
People sometimes use \textbf{her} to refer to a car , machine , or ship.
 \textbf{Her} is also a possessive determiner.
 \textit{
	\begin{itemize}
	\item Kemp got out of his car. 'Just fill her up, thanks.'
	\item This dramatic photograph was taken from Carpathia's deck by one of her passengers.
	\end{itemize}
}
\end{enumerate}

\section*{friendship}
{\large \color{blue}  friendships  }
\subsection*{Explain}
\begin{enumerate}
\item variable noun \\
A \textbf{friendship} is a relationship between two or more friends .
 \textit{
	\begin{itemize}
	\item Giving advice when it's not called for is the quickest way to end a good friendship.
	\item She struck up a close friendship with Desiree during the week of rehearsals.
	\item After seven years of friendship, she still couldn't tell when he was kidding.
	\end{itemize}
}
\item uncountable noun \\
You use \textbf{friendship} to refer in a general way to the state of being friends, or the feelings that friends have
for each other.
 \textit{
	\begin{itemize}
	\item ...a hobby which led to a whole new world of friendship and adventure.
	\end{itemize}
}
\item singular noun \\
If you have someone's \textbf{friendship} , they are your friend.
 \textit{
	\begin{itemize}
	\item He enjoyed the friendship of George Washington.
	\item Steve really values your friendship more than anything else.
	\end{itemize}
}
\item variable noun \\
\textbf{Friendship} is a relationship between two countries in which they help and support each other.
 \textit{
	\begin{itemize}
	\item ...as president of a foundation that promotes friendship between Italy and America.
	\item They were reaching out the hand of friendship to their former adversaries.
	\item Restoring ties with Israel would not affect Bulgaria's traditional friendships with
other countries.
	\end{itemize}
}
\end{enumerate}

\section*{hers}
{\large \color{blue}  }
\subsection*{Explain}
\begin{enumerate}
\item pronoun \\
You use \textbf{hers} to indicate that something belongs or relates to a woman, girl , or female animal.
 \textit{
	\begin{itemize}
	\item His hand as it shook hers was warm and firm.
	\item He'd never seen eyes as green as hers.
	\item Professor Camm was a great friend of hers.
	\item Hers was the suggestion I acted upon.
	\end{itemize}
}
\item pronoun \\
In written English, \textbf{hers} is sometimes used to refer to a person without saying whether that person is a man or a woman. Some people dislike this use and prefer to use 'his or hers' or 'theirs'.
 \textit{
	\begin{itemize}
	\item The author can report other people's results which more or less agree with hers.
	\end{itemize}
}
\item pronoun \\
\textbf{Hers} is sometimes used to refer to a country or nation .
 \textit{
	\begin{itemize}
	\end{itemize}
}
\item possessive pronoun \\
People sometimes use \textbf{hers} to refer to a car , machine , or ship.
 \textit{
	\begin{itemize}
	\end{itemize}
}
\end{enumerate}

\section*{garage}
{\large \color{blue}  garages  }
\subsection*{Explain}
\begin{enumerate}
\item countable noun \\
A \textbf{garage} is a building in which you keep a car . A garage is often built  next to or as part of a house.
 \textit{
	\begin{itemize}
	\end{itemize}
}
\item countable noun \\
A \textbf{garage} is a place where you can  get your car repaired. In Britain , you can also buy fuel for your car, or buy cars.
 \textit{
	\begin{itemize}
	\item Nancy took her car to a local garage for a check-up.
	\item Nelson Garage has the used car you're after.
	\end{itemize}
}
\end{enumerate}

\section*{his}
{\large \color{blue}  }
\subsection*{Explain}
\begin{enumerate}
\item determiner \\
You use \textbf{his} to indicate that something belongs or relates to a man, boy , or male animal.
 \textbf{His} is also a possessive pronoun.
 \textit{
	\begin{itemize}
	\item Brian splashed water on his face, then brushed his teeth.
	\item He spent a large part of his career in Hollywood.
	\item The past 10 years have been the happiest and most fulfilling of his life.
	\item The dog let his head thump on the floor again.
	\item He had taken advice, but the decision was his.
	\item Anna reached out her hand to him and clasped his.
	\end{itemize}
}
\item determiner \\
In written English, \textbf{his} is sometimes used to refer to a person without saying whether that person is a man or a woman. Some people dislike this use and prefer to use 'his or her' or 'their'.
 \textbf{His} is also a possessive pronoun.
 \textit{
	\begin{itemize}
	\item Formerly, the relations between a teacher and his pupils were dominated by fear on
the part of the pupils.
	\item Everyone should receive a fair price for the product of his labour.
	\item The student going to art or drama school will be very enthusiastic about further
education. His is not a narrow mind, but one eager to grasp every facet of anything
he studies.
	\end{itemize}
}
\item possessive determiner \\
In some religions , \textbf{His} is used to refer to God.
 \textbf{His} is also a possessive pronoun.
 \textit{
	\begin{itemize}
	\item ...humble faith in God, and trust in His Church.
	\item In what way do you feel called to serve God as a clergyman? Is it your way, or His?
	\end{itemize}
}
\end{enumerate}

\section*{giant}
{\large \color{blue}  giants  }
\subsection*{Explain}
\begin{enumerate}
\item adjective \\
Something that is described as \textbf{giant} is much larger or more important than most others of its kind .
 \textit{
	\begin{itemize}
	\item ...Italy's giant car maker, Fiat.
	\item ...a giant oak table.
	\item ...a giant step towards unification with the introduction of monetary union.
	\end{itemize}
}
\item countable noun \\
\textbf{Giant} is often used to refer to any large, successful  business organization or country.
 \textit{
	\begin{itemize}
	\item ...Japanese electronics giant Sony.
	\item ...one of Germany's industrial giants, Daimler-Benz.
	\end{itemize}
}
\item countable noun \\
A \textbf{giant} is an imaginary person who is very big and strong , especially one mentioned in old stories .
 \textit{
	\begin{itemize}
	\item ...a Nordic saga of giants.
	\end{itemize}
}
\item countable noun \\
You can refer to someone, especially a man, as a \textbf{giant} , if they seem important or powerful or if they are big and strong.
 \textit{
	\begin{itemize}
	\item He has enormous charisma. He is a giant of a man.
	\item The biggest man in the patrol, a giant of a man, lifted Mattie on to his shoulders.
	\end{itemize}
}
\item countable noun \\
You can refer to someone such as a famous  musician or writer as a \textbf{giant} , if they are regarded as one of the most important or successful people in their field .
 \textit{
	\begin{itemize}
	\item ...Pete Seeger, the giant of American folk music
	\item He was without question one of the giants of Japanese literature.
	\end{itemize}
}
\end{enumerate}

\section*{hardware}
{\large \color{blue}  }
\subsection*{Explain}
\begin{enumerate}
\item uncountable noun \\
In computer systems, \textbf{hardware}  refers to the machines themselves as opposed to the programs which tell the machines what to do. Compare  software .
 \textit{
	\begin{itemize}
	\end{itemize}
}
\item uncountable noun \\
Military \textbf{hardware} is the machinery and equipment that is used by the armed forces, such as tanks, aircraft , and missiles.
 \textit{
	\begin{itemize}
	\end{itemize}
}
\item uncountable noun \\
\textbf{Hardware} refers to tools and equipment that are used in the home and garden , for example  saucepans , screwdrivers, and lawnmowers.
 \textit{
	\begin{itemize}
	\end{itemize}
}
\end{enumerate}

\section*{historian}
{\large \color{blue}  historians  }
\subsection*{Explain}
\begin{enumerate}
\item countable noun \\
A \textbf{historian} is a person who specializes in the study of history, and who writes books and articles about it.
 \textit{
	\begin{itemize}
	\end{itemize}
}
\end{enumerate}

\section*{into}
{\large \color{blue}  }
\subsection*{Explain}
\begin{enumerate}
\item preposition \\
If you put one thing \textbf{into} another, you put the first thing inside the second .
 \textit{
	\begin{itemize}
	\item Combine the remaining ingredients and put them into a dish.
	\item Until the 1980s almost all olives were packed into jars by hand.
	\end{itemize}
}
\item preposition \\
If you go  \textbf{into} a place or vehicle , you move from being outside it to being inside it.
 \textit{
	\begin{itemize}
	\item I have no idea how he got into Iraq.
	\item She got up and went into an inner office.
	\item He got into bed and started to read.
	\end{itemize}
}
\item preposition \\
If one thing goes  \textbf{into} another, the first thing moves from the outside to the inside of the second thing,
by breaking or damaging the surface of it.
 \textit{
	\begin{itemize}
	\item Flavell had accidentally discharged a pistol, firing it into the ceiling.
	\item The rider came off and the handlebar went into his neck.
	\end{itemize}
}
\item preposition \\
If one thing gets  \textbf{into} another, the first thing enters the second and becomes part of it.
 \textit{
	\begin{itemize}
	\item Poisonous smoke had got into the water supply.
	\item The money went into a common fund.
	\end{itemize}
}
\item preposition \\
If you are walking or driving a vehicle and you bump  \textbf{into} something or crash  \textbf{into} something, you hit it accidentally.
 \textit{
	\begin{itemize}
	\item A train plowed into the barrier at the end of the platform.
	\item Joanna heard him bump into the table and curse again.
	\end{itemize}
}
\item preposition \\
When you get \textbf{into} a piece of clothing , you put it on.
 \textit{
	\begin{itemize}
	\item She could change into a different outfit in two minutes.
	\item He put on his underwear and got into his suit.
	\end{itemize}
}
\item preposition \\
If someone or something gets \textbf{into} a particular state, they start being in that state.
 \textit{
	\begin{itemize}
	\item He had too much time on his hands and that caused him to get into trouble.
	\item I slid into a depression.
	\item ...the group's plunge into financial crisis earlier in the year.
	\end{itemize}
}
\item preposition \\
If you talk someone \textbf{into} doing something, you persuade them to do it.
 \textit{
	\begin{itemize}
	\item Gerome tried to talk her into taking an apartment in Paris.
	\end{itemize}
}
\item preposition \\
If something changes \textbf{into} something else, it then has a new form, shape, or nature .
 \textit{
	\begin{itemize}
	\item ...his attempt to turn a nasty episode into a joke.
	\item ...learning what she needs to know to grow into a competent adult.
	\item ...Irish fairytales that had been translated into English.
	\end{itemize}
}
\item preposition \\
If something is cut or split  \textbf{into} a number of pieces or sections , it is divided so that it becomes several smaller pieces or sections.
 \textit{
	\begin{itemize}
	\item Sixteen teams are taking part, divided into four groups.
	\item Roll out the pastry and cut into narrow strips.
	\item Now if a great lake like Victoria were to dry up partially, it would be split into
a number of separate, smaller lakes.
	\end{itemize}
}
\item preposition \\
An investigation  \textbf{into} a subject or event is concerned with that subject or event.
 \textit{
	\begin{itemize}
	\item The concert will raise funds for research into lung cancer.
	\item We are beginning to have some insight into drug therapy.
	\end{itemize}
}
\item preposition \\
If you move or go \textbf{into} a particular career or business , you start working in it.
 \textit{
	\begin{itemize}
	\item In the early 1990s, it was easy to get into the rental business.
	\item He closed down the business and went into politics.
	\end{itemize}
}
\item preposition \\
If something continues  \textbf{into} a period of time, it continues until after that period of time has begun .
 \textit{
	\begin{itemize}
	\item He had three children, and lived on into his sixties.
	\item The Open Golf Championship will be getting into its second day in a few hours.
	\end{itemize}
}
\item preposition \\
If you are very interested in something and like it very much, you can say that you are \textbf{into} it.
 \textit{
	\begin{itemize}
	\item I'm into electronics myself.
	\end{itemize}
}
\end{enumerate}

\section*{history}
{\large \color{blue}  histories  }
\subsection*{Explain}
\begin{enumerate}
\item uncountable noun \\
You can refer to the events of the past as \textbf{history} . You can also refer to the past events which concern a particular topic or place as its history.
 \textit{
	\begin{itemize}
	\item History is full of seemingly minor events leading to international showdowns.
	\item ...the most evil mass killer in history.
	\item ...the history of Birmingham.
	\item ...religious history.
	\end{itemize}
}
\item uncountable noun \\
\textbf{History} is a subject studied in schools, colleges , and universities that deals with events that have happened in the past.
 \textit{
	\begin{itemize}
	\end{itemize}
}
\item countable noun \\
A \textbf{history} is an account of events that have happened in the past.
 \textit{
	\begin{itemize}
	\item ...his magnificent history of broadcasting in Canada.
	\item ...oral histories taken from elderly people in Rochester.
	\end{itemize}
}
\item countable noun \\
If a person or a place has \textbf{a}  \textbf{history}  \textbf{of} something, it has been very common or has happened frequently in their past.
 \textit{
	\begin{itemize}
	\item ...a history of drink problems.
	\item The boy's mother had a history of abusing her children.
	\end{itemize}
}
\item countable noun \\
Someone's \textbf{history} is the set of facts that are known about their past.
 \textit{
	\begin{itemize}
	\item He couldn't get a new job because of his medical history.
	\item ...an exhibition documenting the personal history of Anne Frank.
	\end{itemize}
}
\item uncountable noun \\
If you say that an event, thing, or person is \textbf{history} , you mean that they are no longer important .
 \textit{
	\begin{itemize}
	\item The Charlottetown agreement is history.
	\end{itemize}
}
\item  \\
 the rest is history \textit{
	\begin{itemize}
	\end{itemize}
}
\end{enumerate}

\section*{its}
{\large \color{blue}  }
\subsection*{Explain}
\begin{enumerate}
\item determiner \\
You use \textbf{its} to indicate that something belongs or relates to a thing, place, or animal that has just been
 mentioned or whose identity is known . You can use \textbf{its} to indicate that something belongs or relates to a child or baby .
 \textit{
	\begin{itemize}
	\item The British Labor Party concludes its annual conference today in Brighton.
	\item ...Japan, with its extreme housing shortage.
	\item The dog lifted its head, listening.
	\end{itemize}
}
\end{enumerate}

\section*{ignorance}
{\large \color{blue}  }
\subsection*{Explain}
\begin{enumerate}
\item uncountable noun \\
\textbf{Ignorance}  \textbf{of} something is lack of knowledge about it.
 \textit{
	\begin{itemize}
	\item I am embarrassed by my complete ignorance of history.
	\item There is so much ignorance about mental illness.
	\item In my ignorance I had never heard country & western music.
	\end{itemize}
}
\end{enumerate}

\section*{itself}
{\large \color{blue}  }
\subsection*{Explain}
\begin{enumerate}
\item pronoun \\
\textbf{Itself} is used as the object of a verb or preposition when it refers to something that is the same thing as the subject of the verb.
 \textit{
	\begin{itemize}
	\item ...remarkable new evidence showing how the body rebuilds itself while we sleep.
	\item Unemployment does not correct itself.
	\item ...the threat of Europe building trade business around itself.
	\end{itemize}
}
\item pronoun \\
You use \textbf{itself} to emphasize the thing you are referring to.
 \textit{
	\begin{itemize}
	\item I think life itself is a learning process.
	\item The involvement of the foreign ministers was itself a sign of progress.
	\item He cheered up on Christmas Day itself.
	\end{itemize}
}
\item pronoun \\
If you say that someone is, for example , politeness \textbf{itself} or kindness  \textbf{itself} , you are emphasizing they are extremely  polite or extremely kind .
 \textit{
	\begin{itemize}
	\item I was never really happy there, although the people were kindness itself.
	\item He is rarely satisfied with anything less than perfection itself.
	\end{itemize}
}
\end{enumerate}

\section*{inertia}
{\large \color{blue}  }
\subsection*{Explain}
\begin{enumerate}
\item uncountable noun \\
If you have a feeling of \textbf{inertia} , you feel very lazy and unwilling to move or be active .
 \textit{
	\begin{itemize}
	\item ...her inertia, her lack of energy.
	\item This might help you overcome inertia.
	\end{itemize}
}
\item uncountable noun \\
\textbf{Inertia} is the tendency of a physical object to remain  still or to continue moving, unless a force is applied to it.
 \textit{
	\begin{itemize}
	\end{itemize}
}
\end{enumerate}

\section*{me}
{\large \color{blue}  }
\subsection*{Explain}
\begin{enumerate}
\item pronoun \\
A speaker or writer uses \textbf{me} to refer to himself or herself. \textbf{Me} is a first person singular  pronoun . \textbf{Me} is used as the object of a verb or a preposition .
 \textit{
	\begin{itemize}
	\item I had to make important decisions that would affect me for the rest of my life.
	\item He asked me to go to Cambridge with him.
	\item Give me a few hours to think about it.
	\item She looked up at me, smiling.
	\end{itemize}
}
\end{enumerate}

\section*{lumber}
{\large \color{blue}  lumbers  lumbering  lumbered  }
\subsection*{Explain}
\begin{enumerate}
\item uncountable noun \\
\textbf{Lumber} consists of trees and large pieces of wood that have been roughly cut up.
 \textit{
	\begin{itemize}
	\item It was made of soft lumber, spruce by the look of it.
	\item He was going to have to purchase all his lumber at full retail price.
	\end{itemize}
}
\item verb \\
If someone or something \textbf{lumbers} from one place to another, they move there very slowly and clumsily.
 \textit{
	\begin{itemize}
	\item He turned and lumbered back to his chair.
	\item The truck lumbered across the parking lot toward the road.
	\item He looked straight ahead and overtook a lumbering lorry.
	\end{itemize}
}
\end{enumerate}

\section*{mine}
{\large \color{blue}  }
\subsection*{Explain}
\begin{enumerate}
\item pronoun \\
\textbf{Mine} is the first person singular  possessive  pronoun . A speaker or writer uses \textbf{mine} to refer to something that belongs or relates to himself or herself.
 \textit{
	\begin{itemize}
	\item Her right hand is inches from mine.
	\item That wasn't his fault, it was mine.
	\item I'm looking for a friend of mine who lives here.
	\end{itemize}
}
\end{enumerate}

\section*{may}
{\large \color{blue}  }
\subsection*{Explain}
\begin{enumerate}
\item modal verb \\
You use \textbf{may} to indicate that something will possibly happen or be true in the future , but you cannot be certain.
 \textit{
	\begin{itemize}
	\item We may have some rain today.
	\item Rates may rise, but it won't be by much and it won't be for long.
	\item I may be back next year.
	\item I don't know if they'll publish it or not. They may.
	\item Scientists know that cancer may not show up for many years.
	\end{itemize}
}
\item modal verb \\
You use \textbf{may} to indicate that there is a possibility that something is true, but you cannot be
certain.
 \textit{
	\begin{itemize}
	\item Civil rights officials say there may be hundreds of other cases of racial violence.
	\item Throwing good money after bad may not be a good idea, they say.
	\end{itemize}
}
\item modal verb \\
You use \textbf{may} to indicate that something is sometimes true or is true in some circumstances .
 \textit{
	\begin{itemize}
	\item A vegetarian diet may not provide enough calories for a child's normal growth.
	\item Up to five inches of snow may cover the mountains.
	\item ...families that may have both parents working.
	\end{itemize}
}
\item modal verb \\
You use \textbf{may have} with a past  participle when suggesting that it is possible that something happened or was true, or when giving a possible explanation for something.
 \textit{
	\begin{itemize}
	\item He may have been to some of those places.
	\item The chaos may have contributed to the deaths of up to 20 people.
	\item Investigators say that a fuel explosion may have caused the crash.
	\item The events may or may not have been connected.
	\end{itemize}
}
\item modal verb \\
You use \textbf{may} in statements where you are accepting the truth of a situation , but contrasting it with something that is more important .
 \textit{
	\begin{itemize}
	\item I may be almost 50, but there aren't a lot of things I've forgotten.
	\item The elderly man may not be typical, but he speaks for a significant body of opinion.
	\item Walking may be boring at times but on a sunny morning there is nothing finer.
	\end{itemize}
}
\item modal verb \\
You use \textbf{may} when you are mentioning a quality or fact about something that people can make use of if they want to.
 \textit{
	\begin{itemize}
	\item The bag has narrow straps, so it may be worn over the shoulder or carried in the
hand.
	\item Some of the diseases of middle age may be prevented by improving nutrition.
	\end{itemize}
}
\item modal verb \\
You use \textbf{may} to indicate that someone is allowed to do something, usually because of a rule or law . You use \textbf{may not} to indicate that someone is not allowed to do something.
 \textit{
	\begin{itemize}
	\item What is the nearest you may park to a junction?
	\item Adolescents under the age of 18 may not work in jobs that require them to drive.
	\end{itemize}
}
\item modal verb \\
You use \textbf{may} when you are giving permission to someone to do something, or when asking for permission.
 \textit{
	\begin{itemize}
	\item Mr Hobbs? May we come in?
	\item If you wish, you may now have a glass of milk.
	\item 'You may leave.'—'Yes, sir.'
	\end{itemize}
}
\item modal verb \\
You use \textbf{may} when you are making polite requests.
 \textit{
	\begin{itemize}
	\item I'd like the use of your living room, if I may.
	\item May I come with you to Southampton?
	\item Ah, Julia, my dear, here is our guest. May we have some tea?
	\end{itemize}
}
\item modal verb \\
You use \textbf{may} , usually in questions, when you are politely making suggestions or offering to do something.
 \textit{
	\begin{itemize}
	\item May we suggest you try one of our guest houses.
	\item May we recommend a weekend in Stockholm?
	\item Do sit down. And may we offer you something to drink?
	\item May I help you?
	\end{itemize}
}
\item modal verb \\
You use \textbf{may} as a polite way of interrupting someone, asking a question, or introducing what you are going to say  next .
 \textit{
	\begin{itemize}
	\item 'If I may interrupt for a moment,' Kenneth said.
	\item Anyway, may I just ask you one other thing?
	\item If I may return to what we were talking about earlier.
	\end{itemize}
}
\item modal verb \\
You use \textbf{may} when you are mentioning the reaction or attitude that you think someone is likely to have to something you are about to say.
 \textit{
	\begin{itemize}
	\item You know, Brian, whatever you may think, I work hard for a living.
	\item You may consider it useless, but for our customers it's an all-important sign of
good service.
	\end{itemize}
}
\item modal verb \\
You use \textbf{may} in expressions such as \textbf{I may add} and \textbf{I may say} in order to emphasize a statement that you are making.
 \textit{
	\begin{itemize}
	\item They spent their afternoons playing golf–extremely badly, I may add–around Loch Lomond.
	\item Both of them, I may say, are thoroughly reliable men.
	\end{itemize}
}
\item modal verb \\
If you do something so that a particular thing \textbf{may} happen, you do it so that there is an opportunity for that thing to happen.
 \textit{
	\begin{itemize}
	\item ...the need for more surgeons so that patients may be treated more quickly.
	\item The door is shut so that no one may overhear what is said.
	\end{itemize}
}
\item modal verb \\
People sometimes use \textbf{may} to express hopes and wishes.
 \textit{
	\begin{itemize}
	\item Courage seems now to have deserted him. May it quickly reappear.
	\end{itemize}
}
\end{enumerate}

\section*{misfortune}
{\large \color{blue}  misfortunes  }
\subsection*{Explain}
\begin{enumerate}
\item variable noun \\
A \textbf{misfortune} is something unpleasant or unlucky that happens to someone.
 \textit{
	\begin{itemize}
	\item She seemed to enjoy the misfortunes of others.
	\item He had his full share of misfortune.
	\end{itemize}
}
\end{enumerate}

\section*{ours}
{\large \color{blue}  }
\subsection*{Explain}
\begin{enumerate}
\item pronoun \\
You use \textbf{ours} to refer to something that belongs or relates both to yourself and to one or more other people.
 \textit{
	\begin{itemize}
	\item There are few strangers in a town like ours.
	\item Half the houses had been fitted with alarms and ours hadn't.
	\end{itemize}
}
\end{enumerate}

\section*{odds}
{\large \color{blue}  }
\subsection*{Explain}
\begin{enumerate}
\item plural noun \\
You refer to how likely something is to happen as the \textbf{odds} that it will happen.
 \textit{
	\begin{itemize}
	\item What are the odds of finding a parking space right outside the door?
	\item The odds are that you are going to fail.
	\end{itemize}
}
\item plural noun \\
In betting , \textbf{odds} are expressions with numbers such as '10 to 1' and '7 to 2' that show how likely something is thought to be, for example how likely a particular horse is to lose or win a race .
 \textit{
	\begin{itemize}
	\item Gavin Jones, who put £25 on Eugene, at odds of 50 to 1, has won £1,250.
	\end{itemize}
}
\item  \\
 at odds \textit{
	\begin{itemize}
	\end{itemize}
}
\item  \\
 odds are against \textit{
	\begin{itemize}
	\end{itemize}
}
\item  \\
 against all odds \textit{
	\begin{itemize}
	\end{itemize}
}
\item  \\
 the odds are in sb's favour \textit{
	\begin{itemize}
	\end{itemize}
}
\item  \\
 to shorten the odds \textit{
	\begin{itemize}
	\end{itemize}
}
\end{enumerate}

\section*{payment}
{\large \color{blue}  payments  }
\subsection*{Explain}
\begin{enumerate}
\item countable noun \\
A \textbf{payment} is an amount of money that is paid to someone, or the act of paying this money.
 \textit{
	\begin{itemize}
	\item Thousands of its customers are in arrears with loans and mortgage payments.
	\item The fund will make payments of just over £1 billion next year.
	\end{itemize}
}
\item uncountable noun \\
\textbf{Payment} is the act of paying money to someone or of being paid.
 \textit{
	\begin{itemize}
	\item He had sought to obtain payment of a sum which he had claimed was owed to him.
	\end{itemize}
}
\end{enumerate}

\section*{regarding}
{\large \color{blue}  }
\subsection*{Explain}
\begin{enumerate}
\item preposition \\
You can use \textbf{regarding} to indicate the subject that is being talked or written about.
 \textit{
	\begin{itemize}
	\item He refused to divulge any information regarding the man's whereabouts.
	\end{itemize}
}
\end{enumerate}

\section*{puppet}
{\large \color{blue}  puppets  }
\subsection*{Explain}
\begin{enumerate}
\item countable noun \\
A \textbf{puppet} is a doll that you can move, either by pulling strings which are attached to it or by putting your hand inside its body and moving your fingers .
 \textit{
	\begin{itemize}
	\end{itemize}
}
\item countable noun \\
You can refer to a person or country as a \textbf{puppet} when you mean that their actions are controlled by a more powerful person or government , even though they may appear to be independent.
 \textit{
	\begin{itemize}
	\item When the invasion occurred he ruled as a puppet of the occupiers.
	\item There are fears he will be a puppet President.
	\end{itemize}
}
\end{enumerate}

\section*{shortage}
{\large \color{blue}  shortages  }
\subsection*{Explain}
\begin{enumerate}
\item variable noun \\
If there is a \textbf{shortage}  \textbf{of} something, there is not enough of it.
 \textit{
	\begin{itemize}
	\item A shortage of funds is preventing the U.N. from monitoring relief.
	\item The country is suffering from a food shortage.
	\item There's no shortage of ideas when it comes to improving the education of children.
	\end{itemize}
}
\end{enumerate}

\section*{since}
{\large \color{blue}  }
\subsection*{Explain}
\begin{enumerate}
\item preposition \\
You use \textbf{since} when you are mentioning a time or event in the past and indicating that a situation has continued from then until now .
 \textbf{Since} is also an adverb .
 \textbf{Since} is also a conjunction .
 \textit{
	\begin{itemize}
	\item Jacques Arnold has been a Member of Parliament since 1987.
	\item She had a breakthrough in her research some years ago, and since then she has been
very successful.
	\item I've been here since the end of June.
	\item When we first met, we had a row, and we have rowed frequently ever since.
	\item They went to Dartmouth College together in the 1960s and have frequently done business
together since.
	\item I returned home to Sussex and have since worked as a solicitor.
	\item I've earned my own living since I was seven, doing all kinds of jobs.
	\item Ever since he was a boy, de Forest had dreamed of making his fortune as an inventor.
	\end{itemize}
}
\item preposition \\
You use \textbf{since} to mention a time or event in the past when you are describing an event or situation that has happened after that time.
 \textbf{Since} is also a conjunction.
 \textit{
	\begin{itemize}
	\item The percentage increase in reported crime in England and Wales this year is the highest
since the war.
	\item They were the first band since the Beatles to reach No 1 with each of their first
four albums.
	\item So much has changed in the sport since I was a teenager.
	\item Since I have become a mother, the sound of children's voices has lost its charm.
	\item ...a slight accent she had acquired since he last saw her.
	\end{itemize}
}
\item adverb \\
When you are talking about an event or situation in the past, you use \textbf{since} to indicate that another event happened at some point later in time.
 \textit{
	\begin{itemize}
	\item About six thousand people were arrested, several hundred of whom have since been
released.
	\item There is increasing criticism among his supporters, many of whom have since left
Central Office.
	\end{itemize}
}
\item  \\
 long since \textit{
	\begin{itemize}
	\end{itemize}
}
\item conjunction \\
You use \textbf{since} to introduce  reasons or explanations .
 \textit{
	\begin{itemize}
	\item I'm forever losing things since I'm quite forgetful.
	\item Since she did not make enough money to live in her own house, she went back to live
with her mother.
	\end{itemize}
}
\end{enumerate}

\section*{sky}
{\large \color{blue}  skies  }
\subsection*{Explain}
\begin{enumerate}
\item variable noun \\
\textbf{The}  \textbf{sky} is the space around the Earth which you can see when you stand outside and look upwards.
 \textit{
	\begin{itemize}
	\item The sun is already high in the sky.
	\item ...warm sunshine and clear blue skies.
	\item The night sky was lit up by flashes of light.
	\end{itemize}
}
\end{enumerate}

\section*{somewhat}
{\large \color{blue}  }
\subsection*{Explain}
\begin{enumerate}
\item adverb \\
You use \textbf{somewhat} to indicate that something is the case to a limited  extent or degree .
 \textit{
	\begin{itemize}
	\item He concluded that Oswald was somewhat abnormal.
	\item He explained somewhat unconvincingly that the company was paying for everything.
	\item Although his relationship with his mother had improved somewhat, he was still depressed.
	\item 'I believe you know him'—'Somewhat.'
	\end{itemize}
}
\end{enumerate}

\section*{specialist}
{\large \color{blue}  specialists  }
\subsection*{Explain}
\begin{enumerate}
\item countable noun \\
A \textbf{specialist} is a person who has a particular skill or knows a lot about a particular subject.
 \textit{
	\begin{itemize}
	\item If you are housebound, you can arrange for a home visit from a specialist adviser.
	\item Peckham, himself a cancer specialist, is well aware of the wide variations in medical
practice.
	\item ...a specialist in diseases of the nervous system.
	\end{itemize}
}
\end{enumerate}

\section*{their}
{\large \color{blue}  }
\subsection*{Explain}
\begin{enumerate}
\item determiner \\
You use \textbf{their} to indicate that something belongs or relates to the group of people, animals, or
things that you are talking about.
 \textit{
	\begin{itemize}
	\item Janis and Kurt have announced their engagement.
	\item Horses were poking their heads over their stall doors.
	\item ...as the trees shed their leaves and the year begins to die.
	\end{itemize}
}
\item determiner \\
You use \textbf{their}  instead of 'his or her' to indicate that something belongs or relates to a person without
 saying whether that person is a man or a woman. Some people think this use is incorrect .
 \textit{
	\begin{itemize}
	\item Every member will receive their own 'Welcome to Labour' brochure.
	\item But anyone looking for income from their investments is in a much worse state.
	\end{itemize}
}
\end{enumerate}

\section*{station}
{\large \color{blue}  stations  stationing  stationed  }
\subsection*{Explain}
\begin{enumerate}
\item countable noun \\
A \textbf{station} is a building by a railway line where trains stop so that people can get on or off.
 \textit{
	\begin{itemize}
	\item Ingrid went with him to the railway station to see him off.
	\end{itemize}
}
\item countable noun \\
A bus \textbf{station} is a building, usually in a town or city, where buses stop, usually for a while, so that people can get on or off.
 \textit{
	\begin{itemize}
	\end{itemize}
}
\item countable noun \\
If you talk about a particular radio or television \textbf{station} , you are referring to the programmes  broadcast by a particular radio or television company .
 \textit{
	\begin{itemize}
	\item ...an independent local radio station.
	\item It claims to be the most popular television station in the U.K.
	\end{itemize}
}
\item passive verb \\
If soldiers or officials \textbf{are stationed} in a place, they are sent there to do a job or to work for a period of time.
 \textit{
	\begin{itemize}
	\item Reports from the capital, Lome, say troops are stationed on the streets.
	\item I was stationed there just after the war.
	\item ...United States military personnel stationed in the Philippines.
	\end{itemize}
}
\item verb \\
If you \textbf{station}  \textbf{yourself}  somewhere , you go there and wait , usually for a particular purpose.
 \textit{
	\begin{itemize}
	\item The musicians stationed themselves quickly on either side of the stairs.
	\item He stationed himself at the door.
	\end{itemize}
}
\end{enumerate}

\section*{theirs}
{\large \color{blue}  }
\subsection*{Explain}
\begin{enumerate}
\item pronoun \\
You use \textbf{theirs} to indicate that something belongs or relates to the group of people, animals, or
things that you are talking about.
 \textit{
	\begin{itemize}
	\item There was a big group of a dozen people at the table next to theirs.
	\item It would cost about £3000 to install a new heating system in a flat such as theirs.
	\item Theirs had been a happy and satisfactory marriage.
	\end{itemize}
}
\item pronoun \\
You use \textbf{theirs}  instead of 'his or hers' to indicate that something belongs or relates to a person without
 saying whether that person is a man or a woman. Some people think this use is incorrect .
 \textit{
	\begin{itemize}
	\item If someone wanted it, it would be theirs for the taking.
	\end{itemize}
}
\end{enumerate}

\section*{sun}
{\large \color{blue}  suns  sunning  sunned  }
\subsection*{Explain}
\begin{enumerate}
\item singular noun \\
\textbf{The}  \textbf{sun} is the ball of fire in the sky that the Earth goes round, and that gives us heat and light.
 \textit{
	\begin{itemize}
	\item The sun was now high in the southern sky.
	\item The sun came out, briefly.
	\item ...the sun's rays.
	\item The sun was shining.
	\end{itemize}
}
\item uncountable noun \\
You refer to the light and heat that reach us from the sun as \textbf{the}  \textbf{sun} .
 \textit{
	\begin{itemize}
	\item Dena took them into the courtyard to sit in the sun.
	\item They were trying to soak up some sun.
	\end{itemize}
}
\item verb \\
If you \textbf{are sunning yourself} , you are sitting or lying in a place where the sun is shining on you.
 \textit{
	\begin{itemize}
	\item She was last seen sunning herself in a riverside park.
	\end{itemize}
}
\item countable noun \\
A \textbf{sun} is any star which has planets  going around it.
 \textit{
	\begin{itemize}
	\end{itemize}
}
\item  \\
 everythng/anything under the sun \textit{
	\begin{itemize}
	\end{itemize}
}
\end{enumerate}

\section*{these}
{\large \color{blue}  }
\subsection*{Explain}
\begin{enumerate}
\item determiner \\
You use \textbf{these} at the beginning of noun groups to refer to someone or something that you have already  mentioned or identified .
 \textbf{These} is also a pronoun.
 \textit{
	\begin{itemize}
	\item Switch to an interest-paying current account. Most banks now offer these accounts.
	\item A steering committee has been formed. These people can make decisions much more quickly.
	\item AIDS kills mostly young people. These are the people who contribute most to a country's
economic development.
	\end{itemize}
}
\item determiner \\
You use \textbf{these} to introduce people or things that you are going to talk about.
 \textbf{These} is also a pronoun.
 \textit{
	\begin{itemize}
	\item Look for an account with these features: simple fees; online or phone access to account
information.
	\item If you're converting your loft, these addresses will be useful.
	\item These are some of the things you can do for yourself.
	\end{itemize}
}
\item determiner \\
In spoken English, people use \textbf{these} to introduce people or things into a story .
 \textit{
	\begin{itemize}
	\item I was on my own and these fellows came along towards me.
	\item She used to make these chocolate puddle puddings, you know, with the sauce underneath.
	\end{itemize}
}
\item pronoun \\
You use \textbf{these} when you are identifying someone or asking about their identity .
 \textit{
	\begin{itemize}
	\item These are my children.
	\end{itemize}
}
\item determiner \\
You use \textbf{these} to refer to people or things that are near you, especially when you touch them or point to them.
 \textbf{These} is also a pronoun.
 \textit{
	\begin{itemize}
	\item I put these pictures up here to show how children are solving the problem.
	\item These scissors are awfully heavy.
	\item These are the people who are doing our loft conversion for us.
	\item These are my favourite biscuits.
	\end{itemize}
}
\item determiner \\
You use \textbf{these} when you refer to something which you expect the person you are talking to to know about, or when you are checking that you are both thinking of the same person or thing.
 \textit{
	\begin{itemize}
	\item You know these last few months when we've been expecting it to warm up a little bit?
	\item You know these new houses that are controlled by touchscreen and all of that?
	\end{itemize}
}
\item determiner \\
You use \textbf{these} in the expression  \textbf{these days} to mean 'at the present time'.
 \textit{
	\begin{itemize}
	\item Living in Bootham these days can be depressing.
	\item Trying to make it as a single parent is the main concern to me these days.
	\end{itemize}
}
\end{enumerate}

\section*{swan}
{\large \color{blue}  swans  swanning  swanned  }
\subsection*{Explain}
\begin{enumerate}
\item countable noun \\
A \textbf{swan} is a large bird with a very long neck. Swans live on rivers and lakes and are usually white.
 \textit{
	\begin{itemize}
	\end{itemize}
}
\item verb \\
If you describe someone as \textbf{swanning around} or \textbf{swanning off} , you mean that they go and have fun , rather than working or taking  care of their responsibilities .
 \textit{
	\begin{itemize}
	\item She spends her time swanning around the world.
	\item The mother was widowed and had swanned off.
	\end{itemize}
}
\end{enumerate}

\section*{those}
{\large \color{blue}  }
\subsection*{Explain}
\begin{enumerate}
\item determiner \\
You use \textbf{those} to refer to people or things which have already been mentioned .
 \textbf{Those} is also a pronoun .
 \textit{
	\begin{itemize}
	\item Theoretically he had control over more than $400 million in U.S. accounts. But, in
fact, it was the U.S. Treasury and State Department who controlled those accounts.
	\item I was helped by people who cared and I shall never be able to thank those people
enough.
	\item I understand there are several projects going on. Could you tell us a bit about those?
	\item Waterfalls never fail to attract and the Falls of Clyde are no exception.
	\end{itemize}
}
\item determiner \\
You use \textbf{those} when you are referring to people or things that are a distance  away from you in position or time, especially when you indicate or point to them.
 \textbf{Those} is also a pronoun.
 \textit{
	\begin{itemize}
	\item What are those buildings?
	\item Oh, those books! I meant to put them away before this afternoon.
	\item Those are nice shoes. Where'd you get them?
	\item Excuse me. What are those for?
	\item I think those are my earrings.
	\end{itemize}
}
\item determiner \\
You use \textbf{those} to refer to someone or something when you are going to give details or information about them.
 \textit{
	\begin{itemize}
	\item Those people who took up weapons to defend themselves are political prisoners.
	\item The point of home bread-making is to avoid those additives used in much commercial
baking.
	\end{itemize}
}
\item pronoun \\
You use \textbf{those} to introduce more information about something already mentioned, instead of repeating the noun which refers to it.
 \textit{
	\begin{itemize}
	\item The interests he enjoys most are those which enable him to show off his talents.
	\item The cells of the body, especially those of the brain, need circulating blood.
	\end{itemize}
}
\item pronoun \\
You use \textbf{those} to mean 'people'.
 \textit{
	\begin{itemize}
	\item A little selfish behaviour is unlikely to cause real damage to those around us.
	\item A number of leading opposition figures were said to be among those arrested.
	\end{itemize}
}
\item determiner \\
You use \textbf{those} when you refer to things that you expect the person you are talking to to know about or when you are checking that you are both thinking of the same people or things.
 \textit{
	\begin{itemize}
	\item He did buy me those daffodils a week or so ago.
	\item I have been putting pressure on the council to fill in those potholes.
	\item I believe they've doubled their turnover since those advertisements appeared.
	\item ...those embarrassing moments we all have.
	\end{itemize}
}
\end{enumerate}

\section*{timber}
{\large \color{blue}  timbers  }
\subsection*{Explain}
\begin{enumerate}
\item uncountable noun \\
\textbf{Timber} is wood that is used for building houses and making furniture . You can also  refer to trees that are grown for this purpose as \textbf{timber} .
 \textit{
	\begin{itemize}
	\item These Severn Valley woods have been exploited for timber since Saxon times.
	\item ...a single-story timber building.
	\end{itemize}
}
\item countable noun \\
The \textbf{timbers} of a ship or house are the large pieces of wood that have been used to build it.
 \textit{
	\begin{itemize}
	\item ...a bird nestling in the timbers of the roof.
	\end{itemize}
}
\end{enumerate}

\section*{throughout}
{\large \color{blue}  }
\subsection*{Explain}
\begin{enumerate}
\item preposition \\
If you say that something happens  \textbf{throughout} a particular period of time, you mean that it happens during the whole of that period.
 \textbf{Throughout} is also an adverb .
 \textit{
	\begin{itemize}
	\item The national tragedy of rival groups killing each other continued throughout 1990.
	\item Movie music can be made memorable because its themes are repeated throughout the
film.
	\item ...a single-minded devotion to racing which Gaye has shown throughout her career.
	\item The first song, 'Blue Moon', didn't go too badly except that everyone talked throughout.
	\end{itemize}
}
\item preposition \\
If you say that something happens or exists \textbf{throughout} a place, you mean that it happens or exists in all parts of that place.
 \textbf{Throughout} is also an adverb.
 \textit{
	\begin{itemize}
	\item 'Sight Savers', founded in 1950, now runs projects throughout Africa, the Caribbean
and South East Asia.
	\item These terms will be used throughout the book.
	\item The route is well sign-posted throughout.
	\item Throughout, the walls are white.
	\end{itemize}
}
\end{enumerate}

\section*{tub}
{\large \color{blue}  tubs  }
\subsection*{Explain}
\begin{enumerate}
\item countable noun \\
A \textbf{tub} is a deep container of any size .
 A \textbf{tub}  \textbf{of} something is the amount of it contained in a tub.
 \textit{
	\begin{itemize}
	\item He peeled the paper top off a little white tub and poured the cream into his coffee.
	\item Shrubs can be grown in tubs or large containers.
	\item She would eat four tubs of ice cream in one sitting.
	\end{itemize}
}
\item countable noun \\
A \textbf{tub} is the same as a bathtub .
 \textit{
	\begin{itemize}
	\item She lay back in the tub.
	\end{itemize}
}
\end{enumerate}

\section*{till}
{\large \color{blue}  tills  tilling  tilled  }
\subsection*{Explain}
\begin{enumerate}
\item preposition \\
In spoken English and informal  written English, \textbf{till} is often used instead of until .
 \textbf{Till} is also a conjunction .
 \textit{
	\begin{itemize}
	\item They had to wait till Monday to ring the bank manager.
	\item I've survived till now, and will go on doing so without help from you.
	\item I hadn't left home till I was nineteen.
	\item They slept till the alarm bleeper woke them at four.
	\end{itemize}
}
\item countable noun \\
In a shop or other place of business , a \textbf{till} is a counter or cash register where money is kept , and where customers pay for what they have bought .
 \textit{
	\begin{itemize}
	\item ...long queues at tills that make customers angry.
	\end{itemize}
}
\item countable noun \\
A \textbf{till} is the drawer of a cash register, in which the money is kept.
 \textit{
	\begin{itemize}
	\item He checked the register. There was money in the till.
	\end{itemize}
}
\item verb \\
When people \textbf{till} land, they prepare the earth and work on it in order to grow crops.
 \textit{
	\begin{itemize}
	\item Workers were singing as they tilled the rice paddy fields.
	\item ...freshly tilled fields.
	\end{itemize}
}
\end{enumerate}

\section*{vehicle}
{\large \color{blue}  vehicles  }
\subsection*{Explain}
\begin{enumerate}
\item countable noun \\
A \textbf{vehicle} is a machine such as a car , bus , or truck which has an engine and is used to carry people from place to place.
 \textit{
	\begin{itemize}
	\item The vehicle would not have the capacity to make the journey on one tank of fuel.
	\item ...a vehicle which was somewhere between a tractor and a truck.
	\end{itemize}
}
\item countable noun \\
You can use \textbf{vehicle} to refer to something that you use in order to achieve a particular purpose .
 \textit{
	\begin{itemize}
	\item Her art became a vehicle for her political beliefs.
	\item The vehicle that permitted both communication and acceptability was social revolution.
	\end{itemize}
}
\end{enumerate}

\section*{via}
{\large \color{blue}  }
\subsection*{Explain}
\begin{enumerate}
\item preposition \\
If you go  somewhere  \textbf{via} a particular place, you go through that place on the way to your destination .
 \textit{
	\begin{itemize}
	\item We drove via Lovech to the old Danube town of Ruse.
	\item Mr Baker will return home via Britain and France.
	\end{itemize}
}
\item preposition \\
If you do something \textbf{via} a particular means or person, you do it by making use of that means or person.
 \textit{
	\begin{itemize}
	\item Technology allows relief workers to contact the outside world via satellite.
	\item Translators can now work from home, via electronic mail systems.
	\item The executive's meeting had finished and Sir Marcus had reported its conclusions
to the prime minister via Richard Ryder.
	\end{itemize}
}
\end{enumerate}

\section*{vowel}
{\large \color{blue}  vowels  }
\subsection*{Explain}
\begin{enumerate}
\item countable noun \\
A \textbf{vowel} is a sound such as the ones represented in writing by the letters 'a', 'e' 'i', ' o ' and 'u', which you pronounce with your mouth open, allowing the air to flow through it. Compare  consonant .
 \textit{
	\begin{itemize}
	\item The vowel in words like 'my' and 'thigh' is not very difficult.
	\item ...English vowel sounds.
	\end{itemize}
}
\end{enumerate}

\section*{whichever}
{\large \color{blue}  }
\subsection*{Explain}
\begin{enumerate}
\item determiner \\
You use \textbf{whichever} in order to indicate that it does not matter which of the possible  alternatives  happens or is chosen .
 \textbf{Whichever} is also a conjunction .
 \textit{
	\begin{itemize}
	\item Whichever way you look at it, nuclear power is the energy of the future.
	\item Israel offers automatic citizenship to all Jews who want it, whatever colour they
are and whichever language they speak.
	\item We will gladly exchange your goods, or refund your money, whichever you prefer.
	\end{itemize}
}
\item determiner \\
You use \textbf{whichever} to specify which of a number of possibilities is the right one or the one you mean.
 \textbf{Whichever} is also a conjunction.
 \textit{
	\begin{itemize}
	\item ...learning to relax by whichever method suits you best.
	\item Fishing is from 6 am to dusk or 10.30pm, whichever is sooner.
	\item Management has offered 7 per cent or the rate of inflation, whichever is higher.
	\item Whichever of the fitness classes you opt for, trained instructors are there to help
you.
	\end{itemize}
}
\end{enumerate}

\section*{wood}
{\large \color{blue}  woods  }
\subsection*{Explain}
\begin{enumerate}
\item variable noun \\
\textbf{Wood} is the material which forms the trunks and branches of trees.
 \textit{
	\begin{itemize}
	\item Their dishes were made of wood.
	\item There was a smell of damp wood and machine oil.
	\item ...a short piece of wood.
	\end{itemize}
}
\item countable noun \\
A \textbf{wood} is a fairly large area of trees growing near each other. You can refer to one or several of these areas as \textbf{woods} , and this is the usual form in American English.
 \textit{
	\begin{itemize}
	\item After dinner Alice slipped away for a walk in the woods with Artie.
	\item About a mile to the west of town he came upon a large wood.
	\end{itemize}
}
\item  \\
 not out of the woods \textit{
	\begin{itemize}
	\end{itemize}
}
\item  \\
 touch wood \textit{
	\begin{itemize}
	\end{itemize}
}
\end{enumerate}

\section*{workshop}
{\large \color{blue}  workshops  }
\subsection*{Explain}
\begin{enumerate}
\item countable noun \\
A \textbf{workshop} is a period of discussion or practical work on a particular subject in which a group of people share their knowledge or experience .
 \textit{
	\begin{itemize}
	\item Trumpeter Marcus Belgrave ran a jazz workshop for young artists.
	\item The Jamaica Festival is planning a series of workshops and business seminars.
	\end{itemize}
}
\item countable noun \\
A \textbf{workshop} is a building which contains tools or machinery for making or repairing things, especially using wood or metal.
 \textit{
	\begin{itemize}
	\item ...a modestly equipped workshop.
	\item ...the railway workshops.
	\end{itemize}
}
\end{enumerate}

\section*{within}
{\large \color{blue}  }
\subsection*{Explain}
\begin{enumerate}
\item preposition \\
If something is \textbf{within} a place, area, or object, it is inside it or surrounded by it.
 \textbf{Within} is also an adverb .
 \textit{
	\begin{itemize}
	\item Clients are entertained within private dining rooms.
	\item An olive-coloured tent stood within a thicket of trees.
	\item Land-use can vary enormously even within a small country.
	\item A small voice called from within. 'Yes, just coming.'
	\end{itemize}
}
\item preposition \\
Something that happens or exists  \textbf{within} a society , organization, or system, happens or exists inside it.
 \textbf{Within} is also an adverb.
 \textit{
	\begin{itemize}
	\item The motives that attract people to work within a social service are variable.
	\item ...the spirit of self-sacrifice within an army.
	\item Within criminal law almost anything could be defined as 'crime'.
	\item The Church of England, with threats of split from within, has still to make up its
mind.
	\end{itemize}
}
\item preposition \\
If you experience a particular feeling , you can say that it is \textbf{within} you.
 \textbf{Within} is also an adverb.
 \textit{
	\begin{itemize}
	\item He's coping much better within himself.
	\item You've got to identify these inadequacies within yourself.
	\item 'God!' cried Dennis from within. 'Oh, my God!'
	\end{itemize}
}
\item preposition \\
If something is \textbf{within} a particular limit or set of rules , it does not go beyond it or is not more than what is allowed .
 \textit{
	\begin{itemize}
	\item Troops have agreed to stay within specific boundaries to avoid confrontations.
	\item Exercise within your comfortable limit.
	\item The film will be finished within its budget.
	\end{itemize}
}
\item preposition \\
If you are \textbf{within} a particular distance of a place, you are less than that distance from it.
 \textit{
	\begin{itemize}
	\item The man was within a few feet of him.
	\item It was within easy walking distance of the hotel.
	\item The rebels have advanced to within 150 kms of the capital.
	\end{itemize}
}
\item preposition \\
\textbf{Within} a particular length of time means before that length of time has passed .
 \textit{
	\begin{itemize}
	\item About 40% of all students entering as freshmen graduate within 4 years.
	\item Within 24 hours the deal was completed.
	\end{itemize}
}
\item preposition \\
If something is \textbf{within sight} , \textbf{within earshot} , or \textbf{within reach} , you can see it, hear it, or reach it.
 \textit{
	\begin{itemize}
	\item His boat was moored within sight of West Church.
	\item The people at every table within earshot fell silent instantly.
	\item Amy looked to see if there was anything within reach that she could give him.
	\end{itemize}
}
\end{enumerate}

\section*{bucket}
{\large \color{blue}  buckets  bucketing  bucketed  }
\subsection*{Explain}
\begin{enumerate}
\item countable noun \\
A \textbf{bucket} is a round metal or plastic container with a handle  attached to its sides. Buckets are often used for holding and carrying water.
 A \textbf{bucket}  \textbf{of} water is the amount of water contained in a bucket.
 \textit{
	\begin{itemize}
	\item We drew water in a bucket from the well outside the door.
	\item The girls happily played in the sand and sea with buckets and spades.
	\item She threw a bucket of water over them.
	\end{itemize}
}
\item quantifier \\
\textbf{Buckets} or \textbf{bucket-loads}  \textbf{of} something means a large amount of it.
 \textit{
	\begin{itemize}
	\item They obviously have buckets of confidence.
	\item They didn't exactly sell bucket-loads of records the first time around.
	\end{itemize}
}
\item plural noun \\
If someone cries  \textbf{buckets} , they cry a lot because they are very upset . If it rains \textbf{buckets} , it rains a lot.
 \textit{
	\begin{itemize}
	\item He was weeping buckets.
	\item The rain was still coming down in buckets when we went back out.
	\end{itemize}
}
\item  \\
 to kick the bucket \textit{
	\begin{itemize}
	\end{itemize}
}
\end{enumerate}

\section*{anything}
{\large \color{blue}  }
\subsection*{Explain}
\begin{enumerate}
\item pronoun \\
You use \textbf{anything} in statements with negative  meaning to indicate in a general way that nothing is present or that an action or event does not or cannot happen .
 \textit{
	\begin{itemize}
	\item We can't do anything.
	\item Dad sat, not saying anything.
	\item She couldn't see or hear anything at all.
	\item By the time I get home, I'm too tired to do anything active.
	\item I couldn't manage anything without you.
	\end{itemize}
}
\item pronoun \\
You use \textbf{anything} in questions and conditional  clauses to ask or talk about whether something is present or happening .
 \textit{
	\begin{itemize}
	\item What happened, is anything wrong?
	\item Did you find anything?
	\item Is there anything you can do to help?
	\item If there's anything I could do for him, I would.
	\end{itemize}
}
\item pronoun \\
You can use \textbf{anything} before words which indicate the kind of thing you are talking about.
 \textit{
	\begin{itemize}
	\item More than anything else, he wanted to become a teacher.
	\item Anything that's cheap this year will be even cheaper next year.
	\item She collects anything that has charm.
	\end{itemize}
}
\item pronoun \\
You use \textbf{anything} to emphasize a possible thing, event, or situation , when you are saying that it could be any one of a very large number of things.
 \textit{
	\begin{itemize}
	\item He is young, fresh, and ready for anything.
	\item At that point, anything could happen.
	\item He is convinced he just has to say 'please' and he can have anything.
	\end{itemize}
}
\item pronoun \\
You use \textbf{anything} in expressions such as \textbf{anything near} , \textbf{anything close to} and \textbf{anything like} to emphasize a statement that you are making.
 \textit{
	\begin{itemize}
	\item The only way he can live anything near a normal life is to have an operation.
	\item Only Cowans played anything close to his true form.
	\item Plainer examples of the early period do not fetch anything like these sums.
	\end{itemize}
}
\item pronoun \\
When you do not want to be exact , you use \textbf{anything} to talk about a particular range of things or quantities .
 \textit{
	\begin{itemize}
	\item The cows produce anything from 25 to 40 litres of milk per day.
	\end{itemize}
}
\item  \\
 as anything \textit{
	\begin{itemize}
	\end{itemize}
}
\item  \\
 anything but \textit{
	\begin{itemize}
	\end{itemize}
}
\item  \\
 would not do sth for anything/would not be sth for anything \textit{
	\begin{itemize}
	\end{itemize}
}
\item  \\
 if anything \textit{
	\begin{itemize}
	\end{itemize}
}
\item  \\
 or anything \textit{
	\begin{itemize}
	\end{itemize}
}
\end{enumerate}

\section*{cabinet}
{\large \color{blue}  cabinets  }
\subsection*{Explain}
\begin{enumerate}
\item countable noun \\
A \textbf{cabinet} is a cupboard used for storing things such as medicine or alcoholic drinks or for displaying decorative things in.
 \textit{
	\begin{itemize}
	\item The star of my medicine cabinet is the humble aspirin.
	\item He looked at the display cabinet with its gleaming sets of glasses.
	\end{itemize}
}
\item countable noun \\
The \textbf{Cabinet} is a group of the most senior ministers in a government, who meet regularly to discuss  policies .
 \textit{
	\begin{itemize}
	\item The announcement came after a three-hour Cabinet meeting.
	\item ...a former Cabinet Minister.
	\end{itemize}
}
\end{enumerate}

\section*{bump}
{\large \color{blue}  bumps  bumping  bumped  }
\subsection*{Explain}
\begin{enumerate}
\item verb \\
If you \textbf{bump} into something or someone, you accidentally hit them while you are moving.
 \textbf{Bump} is also a noun .
 \textit{
	\begin{itemize}
	\item They stopped walking and he almost bumped into them.
	\item There was a jerk as the boat bumped against something.
	\item He bumped his head on the low beams of the house.
	\item Small children often cry after a minor bump.
	\end{itemize}
}
\item countable noun \\
A \textbf{bump} is the action or the dull sound of two heavy objects hitting each other.
 \textit{
	\begin{itemize}
	\item I felt a little bump and I knew instantly what had happened.
	\item The child took five steps, and then sat down with a bump.
	\end{itemize}
}
\item countable noun \\
A \textbf{bump} is a minor  injury or swelling that you get if you bump into something or if something hits you.
 \textit{
	\begin{itemize}
	\item She fell against our coffee table and got a large bump on her forehead.
	\end{itemize}
}
\item countable noun \\
If you have a \textbf{bump} while you are driving a car , you have a minor accident in which you hit something.
 \textit{
	\begin{itemize}
	\end{itemize}
}
\item countable noun \\
A \textbf{bump} on a road is a raised, uneven part.
 \textit{
	\begin{itemize}
	\item The truck hit a bump and bounced.
	\end{itemize}
}
\item verb \\
If a vehicle \textbf{bumps}  \textbf{over} a surface, it travels in a rough , bouncing way because the surface is very uneven.
 \textit{
	\begin{itemize}
	\item We left the road, and again bumped over the mountainside.
	\item The aircraft bumped along erratically without gathering anything like sufficient
speed.
	\end{itemize}
}
\item  \\
 with a bump \textit{
	\begin{itemize}
	\end{itemize}
}
\end{enumerate}

\section*{calendar}
{\large \color{blue}  calendars  }
\subsection*{Explain}
\begin{enumerate}
\item countable noun \\
A \textbf{calendar} is a chart or device which displays the date and the day of the week , and often the whole of a particular year divided up into months , weeks, and days.
 \textit{
	\begin{itemize}
	\item There was a calendar on the wall above, with large squares around the dates.
	\end{itemize}
}
\item countable noun \\
A \textbf{calendar} is a particular system for dividing time into periods such as years, months, and weeks, often starting from a particular point in history .
 \textit{
	\begin{itemize}
	\item The Christian calendar was originally based on the Julian calendar of the Romans.
	\end{itemize}
}
\item countable noun \\
You can use \textbf{calendar} to refer to a series or list of events and activities which take place on particular dates, and which are important for a particular organization , community , or person.
 \textit{
	\begin{itemize}
	\item It is one of the British sporting calendar's most prestigious events.
	\item Franklin joined her and the children whenever his crowded calendar allowed.
	\item They tried to make a calendar of Spain's festivals.
	\end{itemize}
}
\end{enumerate}

\section*{civilize}
{\large \color{blue}  civilizes  civilizing  civilized  }
\subsection*{Explain}
\begin{enumerate}
\item verb \\
To \textbf{civilize} a person or society  means to educate them and improve their way of life.
 \textit{
	\begin{itemize}
	\item ...efforts to civilise the children.
	\item It exerts a civilizing influence on mankind.
	\end{itemize}
}
\end{enumerate}

\section*{cement}
{\large \color{blue}  cements  cementing  cemented  }
\subsection*{Explain}
\begin{enumerate}
\item uncountable noun \\
\textbf{Cement} is a grey powder which is mixed with sand and water in order to make concrete.
 \textit{
	\begin{itemize}
	\item ...a mixture of wet sand and cement.
	\end{itemize}
}
\item uncountable noun \\
\textbf{Cement} is the same as concrete .
 \textit{
	\begin{itemize}
	\item ...the hard cold cement floor.
	\end{itemize}
}
\item uncountable noun \\
Glue that is made for sticking particular substances together is sometimes  called  \textbf{cement} .
 \textit{
	\begin{itemize}
	\item Stick the pieces on with tile cement.
	\end{itemize}
}
\item verb \\
Something that \textbf{cements} a relationship or agreement makes it stronger .
 \textit{
	\begin{itemize}
	\item Nothing cements a friendship between countries so much as trade.
	\end{itemize}
}
\item uncountable noun \\
The \textbf{cement} of a relationship or agreement is something that makes it stronger and more long-lasting .
 \textit{
	\begin{itemize}
	\item Good communication, not structure, is the cement that holds any organization together.
	\item In the old days, television was the cement of society.
	\end{itemize}
}
\item verb \\
If things \textbf{are cemented} together, they are stuck or fastened together.
 \textit{
	\begin{itemize}
	\item Most artificial joints are cemented into place.
	\end{itemize}
}
\end{enumerate}

\section*{complain}
{\large \color{blue}  complains  complaining  complained  }
\subsection*{Explain}
\begin{enumerate}
\item verb \\
If you \textbf{complain}  \textbf{about} a situation, you say that you are not satisfied with it.
 \textit{
	\begin{itemize}
	\item Miners have complained bitterly that the government did not fulfill their promises.
	\item The American couple complained about the high cost of visiting Europe.
	\item For my own part, I have nothing to complain of.
	\item They are liable to face more mistreatment if they complain to the police.
	\item People should complain when they consider an advert offensive.
	\item 'I do everything you ask of me,' he complained.
	\end{itemize}
}
\item verb \\
If you \textbf{complain of} pain or illness, you say that you are feeling pain or feeling ill .
 \textit{
	\begin{itemize}
	\item He complained of a headache.
	\end{itemize}
}
\end{enumerate}

\section*{centre}
{\large \color{blue}  centres  centring  centred  }
\subsection*{Explain}
\begin{enumerate}
\item countable noun \\
A \textbf{centre} is a building where people have meetings , take part in a particular activity, or get  help of some kind .
 \textit{
	\begin{itemize}
	\item We went to a party at the leisure centre.
	\item She now also does pottery classes at a community centre.
	\item ...the National Exhibition Centre.
	\end{itemize}
}
\item countable noun \\
If an area or town is a \textbf{centre} for an industry or activity, that industry or activity is very important there.
 \textit{
	\begin{itemize}
	\item London is also the major international insurance centre.
	\end{itemize}
}
\item countable noun \\
The \textbf{centre} of something is the middle of it.
 \textit{
	\begin{itemize}
	\item A large wooden table dominates the centre of the room.
	\item Bake until light golden and crisp around the edges and slightly soft in the centre.
	\end{itemize}
}
\item countable noun \\
The \textbf{centre} of a town or city is the part where there are the most shops and businesses and where a lot of people come from other areas to work or shop.
 \textit{
	\begin{itemize}
	\item ...the city centre.
	\end{itemize}
}
\item countable noun \\
If something or someone is at the \textbf{centre}  \textbf{of} a situation , they are the most important thing or person involved.
 \textit{
	\begin{itemize}
	\item ...the man at the centre of the controversy.
	\item At the centre of the inquiry has been concern for the pensioners involved.
	\end{itemize}
}
\item countable noun \\
If someone or something is the \textbf{centre}  \textbf{of}  attention or interest, people are giving them a lot of attention.
 \textit{
	\begin{itemize}
	\item The rest of the cast was used to her being the centre of attention.
	\item The centre of attraction was Pierre Auguste Renoir's oil painting.
	\end{itemize}
}
\item singular noun \\
In politics , \textbf{the centre}  refers to groups and their beliefs , when they are considered to be neither left-wing nor right-wing .
 \textit{
	\begin{itemize}
	\item The Democrats have become a party of the centre.
	\item ...the centre parties.
	\end{itemize}
}
\item verb \\
If you \textbf{centre} something, you move it so that it is at the centre of something else.
 \textit{
	\begin{itemize}
	\item Centre the design on the cloth before you start.
	\end{itemize}
}
\item verb \\
If something \textbf{centres} or \textbf{is centred}  \textbf{on} a particular thing or person, that thing or person is the main subject of attention.
 \textit{
	\begin{itemize}
	\item ...a plan which centred on academic achievement and personal motivation.
	\item All his concerns were centred around himself rather than Rachel.
	\item When working with patients, my efforts are centred on helping them to overcome illness.
	\end{itemize}
}
\item verb \\
If an industry or event \textbf{is centred} in a place, or if it \textbf{centres} there, it takes place to the greatest  extent there.
 \textit{
	\begin{itemize}
	\item Chinese restaurants have traditionally been centred around Chinatown.
	\item The disturbances have centred round the two main university areas.
	\item Between 100 and 150 travellers' vehicles were scattered around the county, with the
largest gathering centred on Ampfield.
	\end{itemize}
}
\end{enumerate}

\section*{everything}
{\large \color{blue}  }
\subsection*{Explain}
\begin{enumerate}
\item pronoun \\
You use \textbf{everything} to refer to all the objects , actions , activities , or facts in a particular  situation .
 \textit{
	\begin{itemize}
	\item He'd gone to Seattle long after everything else in his life had changed.
	\item Early in the morning, hikers pack everything that they will need for the day's hike.
	\item Everything in the building had gone silent.
	\end{itemize}
}
\item pronoun \\
You use \textbf{everything} to refer to all possible or likely actions, activities, or situations.
 \textit{
	\begin{itemize}
	\item 'This should have been decided long before now.'—'We can't think of everything.'.
	\item Cathy thought that she had the answer to everything.
	\item Noel and I do everything together.
	\item Are you doing everything possible to reduce your budget?
	\end{itemize}
}
\item pronoun \\
You use \textbf{everything} to refer to a whole situation or to life in general .
 \textit{
	\begin{itemize}
	\item She says everything is going smoothly.
	\item Is everything all right?
	\item Everything's going to be just fine.
	\end{itemize}
}
\item pronoun \\
If you say that someone or something is \textbf{everything} , you mean you consider them to be the most important thing in your life, or the most important thing that
there is.
 \textit{
	\begin{itemize}
	\item I love him. He is everything to me.
	\item Crime cases were something that agents solved, and to him the case was everything.
	\item Money isn't everything.
	\end{itemize}
}
\item pronoun \\
If you say that someone or something has \textbf{everything} , you mean they have all the things or qualities that most people consider to be desirable .
 \textit{
	\begin{itemize}
	\item She has everything: beauty, talent, children.
	\item It was a garden that had everything. It was rich and wild and beautiful, and exciting.
	\end{itemize}
}
\item  \\
 and everything \textit{
	\begin{itemize}
	\end{itemize}
}
\end{enumerate}

\section*{chip}
{\large \color{blue}  chips  chipping  chipped  }
\subsection*{Explain}
\begin{enumerate}
\item countable noun \\
\textbf{Chips} are long, thin pieces of potato fried in oil or fat and eaten hot , usually with a meal .
 \textit{
	\begin{itemize}
	\item I had fish and chips in a cafe.
	\item Frank Browne shook more sauce over his chips.
	\end{itemize}
}
\item countable noun \\
\textbf{Chips} or \textbf{potato chips} are very thin slices of fried potato that are eaten cold as a snack.
 \textit{
	\begin{itemize}
	\item ...a package of onion-flavored potato chips.
	\end{itemize}
}
\item countable noun \\
A silicon \textbf{chip} is a very small piece of silicon with electronic circuits on it which is part of a computer or other piece of machinery .
 \textit{
	\begin{itemize}
	\end{itemize}
}
\item countable noun \\
A \textbf{chip} is a small piece of something or a small piece which has been broken off something.
 \textit{
	\begin{itemize}
	\item It contains real chocolate chips.
	\item He was burning wood chips to make charcoal.
	\item Teichler's eyes gleamed like chips of blue glass.
	\end{itemize}
}
\item countable noun \\
A \textbf{chip} in something such as a piece of china or furniture is where a small piece has been broken off it.
 \textit{
	\begin{itemize}
	\item The washbasin had a small chip.
	\end{itemize}
}
\item verb \\
If you \textbf{chip} something or if it \textbf{chips} , a small piece is broken off it.
 \textit{
	\begin{itemize}
	\item The blow chipped the woman's tooth.
	\item Steel baths are lighter but chip easily.
	\end{itemize}
}
\item countable noun \\
\textbf{Chips} are plastic counters used in gambling to represent money.
 \textit{
	\begin{itemize}
	\item He put the pile of chips in the centre of the table and drew a card.
	\end{itemize}
}
\item countable noun \\
In discussions between people or governments, a \textbf{chip} or a \textbf{bargaining chip} is something of value which one side holds, which can be exchanged for something they want from the other side.
 \textit{
	\begin{itemize}
	\item The information could be used as a bargaining chip to extract some parallel information
from Britain.
	\item He was not expected to be released because he was considered a valuable chip in this
game.
	\end{itemize}
}
\item  \\
 a chip off the old block \textit{
	\begin{itemize}
	\end{itemize}
}
\item  \\
 when the chips are down \textit{
	\begin{itemize}
	\end{itemize}
}
\item  \\
 a chip on one's shoulder \textit{
	\begin{itemize}
	\end{itemize}
}
\end{enumerate}

\section*{formulate}
{\large \color{blue}  formulates  formulating  formulated  }
\subsection*{Explain}
\begin{enumerate}
\item verb \\
If you \textbf{formulate} something such as a plan or proposal , you invent it, thinking about the details carefully.
 \textit{
	\begin{itemize}
	\item Little by little, he formulated his plan for escape.
	\end{itemize}
}
\item verb \\
If you \textbf{formulate} a thought , opinion , or idea , you express it or describe it using particular words.
 \textit{
	\begin{itemize}
	\item I was impressed by the way he could formulate his ideas.
	\end{itemize}
}
\end{enumerate}

\section*{comparison}
{\large \color{blue}  comparisons  }
\subsection*{Explain}
\begin{enumerate}
\item variable noun \\
When you make a \textbf{comparison} , you consider two or more things and discover the differences between them.
 \textit{
	\begin{itemize}
	\item ...a comparison of the British and German economies.
	\item Its recommendations are based on detailed comparisons between the public and private
sectors.
	\item There are no previous statistics for comparison.
	\end{itemize}
}
\item countable noun \\
When you make a \textbf{comparison} , you say that one thing is like another in some way.
 \textit{
	\begin{itemize}
	\item It is demonstrably an unfair comparison.
	\item He finds the comparison of insect wings with a sailing boat useful up to a point.
	\item The comparison of her life to a sea voyage simplifies her experience.
	\end{itemize}
}
\item  \\
 in comparison/by comparison \textit{
	\begin{itemize}
	\end{itemize}
}
\item  \\
 there is no comparison \textit{
	\begin{itemize}
	\end{itemize}
}
\item  \\
 to stand comparison with sth \textit{
	\begin{itemize}
	\end{itemize}
}
\end{enumerate}

\section*{dam}
{\large \color{blue}  dams  damming  dammed  }
\subsection*{Explain}
\begin{enumerate}
\item countable noun \\
A \textbf{dam} is a wall that is built across a river in order to stop the water flowing and to make a lake .
 \textit{
	\begin{itemize}
	\item ...plans to build a dam on the Danube River.
	\item ...the Aswan Dam.
	\end{itemize}
}
\item verb \\
To \textbf{dam} a river means to build a dam across it.
 \textit{
	\begin{itemize}
	\item ...plans to dam the nearby Delaware River.
	\item This reservoir was formed by damming the River Blith.
	\end{itemize}
}
\item countable noun \\
In the breeding of farm animals, an animal's \textbf{dam} is its mother .
 \textit{
	\begin{itemize}
	\end{itemize}
}
\end{enumerate}

\section*{herself}
{\large \color{blue}  }
\subsection*{Explain}
\begin{enumerate}
\item pronoun \\
You use \textbf{herself} to refer to a woman, girl , or female animal.
 \textit{
	\begin{itemize}
	\item She let herself out of the room.
	\item Jennifer believes she will move out on her own when she is financially able to support
herself.
	\item Robin didn't feel good about herself.
	\end{itemize}
}
\item pronoun \\
In written English, \textbf{herself} is sometimes used to refer to a person without saying whether that person is a man or a woman. Some people dislike this use and prefer to use 'himself or herself' or 'themselves'.
 \textit{
	\begin{itemize}
	\item How can anyone blame her for actions for which she feels herself to be in no way
responsible?
	\end{itemize}
}
\item pronoun \\
\textbf{Herself} is sometimes used to refer to a country or nation .
 \textit{
	\begin{itemize}
	\item Britain's dream of herself began to fade.
	\end{itemize}
}
\item reflexive pronoun \\
People sometimes use \textbf{herself} to refer to a car, machine, or ship.
 \textit{
	\begin{itemize}
	\item The ship adjusted herself to the roll and rhythm of the sea.
	\end{itemize}
}
\item pronoun \\
You use \textbf{herself} to emphasize the person or thing that you are referring to. \textbf{Herself} is sometimes used instead of 'her' as the object of a verb or preposition .
 \textit{
	\begin{itemize}
	\item She's so beautiful herself.
	\item Has anyone thought of consulting Bethan herself?
	\item She herself was not a keen gardener.
	\end{itemize}
}
\end{enumerate}

\section*{date}
{\large \color{blue}  dates  dating  dated  }
\subsection*{Explain}
\begin{enumerate}
\item countable noun \\
A \textbf{date} is a specific time that can be named , for example a particular day or a particular year.
 \textit{
	\begin{itemize}
	\item What's the date today?
	\item You will need to give the dates you wish to stay and the number of rooms you require.
	\end{itemize}
}
\item verb \\
If you \textbf{date} something, you give or discover the date when it was made or when it began .
 \textit{
	\begin{itemize}
	\item You cannot date the carving and it is difficult to date the stone itself.
	\item I think we can date the decline of Western Civilization quite precisely.
	\item Archaeologists have dated the fort to the reign of Emperor Antoninus Pius.
	\end{itemize}
}
\item verb \\
When you \textbf{date} something such as a letter or a cheque , you write that day's date on it.
 \textit{
	\begin{itemize}
	\item Once the decision is reached, he can date and sign the sheet.
	\item The letter is dated 2 July 1993.
	\end{itemize}
}
\item singular noun \\
If you want to refer to an event without saying  exactly when it will  happen or when it happened, you can say that it will happen or happened \textbf{at} some \textbf{date} in the future or past .
 \textit{
	\begin{itemize}
	\item Retain copies of all correspondence, since you may need them at a later date.
	\item He did leave open the possibility of direct American aid at some unspecified date
in the future.
	\item He was content for her wedding to be at some date between July and September.
	\end{itemize}
}
\item  \\
 to date \textit{
	\begin{itemize}
	\end{itemize}
}
\item verb \\
If something \textbf{dates} , it goes out of fashion and becomes unacceptable to modern  tastes .
 \textit{
	\begin{itemize}
	\item A black coat always looks smart and will never date.
	\item This album has hardly dated at all.
	\end{itemize}
}
\item verb \\
If your ideas , what you say, or the things that you like or can remember  \textbf{date} you, they show that you are quite  old or older than the people you are with.
 \textit{
	\begin{itemize}
	\item It's going to date me now. I attended that school from 1969 to 1972.
	\end{itemize}
}
\item countable noun \\
A \textbf{date} is an appointment to meet someone or go out with them, especially someone with whom you are having, or may  soon have, a romantic  relationship .
 \textit{
	\begin{itemize}
	\item I have a date with Bob.
	\item I think we should make a date to go and see Gwendolen soon.
	\end{itemize}
}
\item countable noun \\
If you have a date with someone with whom you are having, or may soon have, a romantic
relationship, you can refer to that person as your \textbf{date} .
 \textit{
	\begin{itemize}
	\item He lied to Essie, saying his date was one of the girls in the show.
	\end{itemize}
}
\item verb \\
If you \textbf{are dating} someone, you go out with them regularly because you are having, or may soon have,
a romantic relationship with them. You can also say that two people \textbf{are dating} .
 \textit{
	\begin{itemize}
	\item For a year I dated a woman who was a research assistant.
	\item They've been dating for three months.
	\item In high school, he did not date very much.
	\end{itemize}
}
\item countable noun \\
A \textbf{date} is a small, dark-brown, sticky fruit with a stone  inside . Dates grow on palm trees in hot  countries .
 \textit{
	\begin{itemize}
	\end{itemize}
}
\end{enumerate}

\section*{him}
{\large \color{blue}  }
\subsection*{Explain}
\begin{enumerate}
\item pronoun \\
You use \textbf{him} to refer to a man, boy , or male animal.
 \textit{
	\begin{itemize}
	\item John's aunt died suddenly and left him a surprisingly large sum.
	\item Is Sam there? Let me talk to him.
	\item On his arrival, Elaine met him at the bus station.
	\item My brother had a lovely dog. I looked after him for about a week.
	\end{itemize}
}
\item pronoun \\
In written English, \textbf{him} is sometimes used to refer to a person without saying whether that person is a man or a woman. Some people dislike this use and prefer to use 'him or her' or 'them'.
 \textit{
	\begin{itemize}
	\item If the child sees the word 'hear', we should show him that this is the base word
in 'hearing'.
	\end{itemize}
}
\item singular pronoun \\
In some religions , \textbf{Him} is used to refer to God.
 \textit{
	\begin{itemize}
	\item God will help you if you turn to Him in humility and trust.
	\end{itemize}
}
\end{enumerate}

\section*{daylight}
{\large \color{blue}  }
\subsection*{Explain}
\begin{enumerate}
\item uncountable noun \\
\textbf{Daylight} is the natural light that there is during the day, before it gets  dark .
 \textit{
	\begin{itemize}
	\item It was still daylight but all the cars had their headlights on.
	\item Lack of daylight can make people feel depressed.
	\end{itemize}
}
\item uncountable noun \\
\textbf{Daylight} is the time of day when it begins to get light.
 \textit{
	\begin{itemize}
	\item Quinn returned shortly after daylight yesterday morning.
	\end{itemize}
}
\item  \\
 in broad daylight \textit{
	\begin{itemize}
	\end{itemize}
}
\end{enumerate}

\section*{himself}
{\large \color{blue}  }
\subsection*{Explain}
\begin{enumerate}
\item pronoun \\
You use \textbf{himself} to refer to a man, boy , or male animal.
 \textit{
	\begin{itemize}
	\item He smiles, pouring himself a cup of coffee.
	\item When that doesn't work, he gets angry at his father and himself.
	\item William went away muttering to himself.
	\end{itemize}
}
\item pronoun \\
In written English, \textbf{himself} is sometimes used to refer to a person without saying whether that person is a man or a woman. Some people dislike this use and prefer to use 'himself or herself' or 'themselves'.
 \textit{
	\begin{itemize}
	\item The child's natural way of expressing himself is play.
	\item The student is invited to test each item for himself by means of specific techniques.
	\end{itemize}
}
\item reflexive pronoun \\
In some religions , \textbf{Himself} is used to refer to God .
 \textit{
	\begin{itemize}
	\item He thanked God for concealing Himself from the wise and revealing Himself to the
simple.
	\end{itemize}
}
\item pronoun \\
You use \textbf{himself} to emphasize the person or thing that you are referring to. \textbf{Himself} is sometimes used instead of 'him' as the object of a verb or preposition .
 \textit{
	\begin{itemize}
	\item I heard this with my own ears from the President himself.
	\item He himself had joined the others straight from the office.
	\item There's no work and no future for students like himself.
	\end{itemize}
}
\end{enumerate}

\section*{dentist}
{\large \color{blue}  dentists  }
\subsection*{Explain}
\begin{enumerate}
\item countable noun \\
A \textbf{dentist} is a person who is qualified to examine and treat people's teeth .
 \textbf{The dentist} or \textbf{the dentist's} is used to refer to the surgery or clinic where a dentist works.
 \textit{
	\begin{itemize}
	\item Visit your dentist twice a year for a check-up.
	\item It's worse than being at the dentist's.
	\end{itemize}
}
\end{enumerate}

\section*{it}
{\large \color{blue}  }
\subsection*{Explain}
\begin{enumerate}
\item pronoun \\
You use \textbf{it} to refer to an object, animal, or other thing that has already been mentioned .
 \textit{
	\begin{itemize}
	\item He saw the grey Land-Rover down the by-pass. It was more than a hundred yards from
him.
	\item It's a wonderful city, really. I'll show it to you if you want.
	\item My wife has become crippled by arthritis. She is embarrassed to ask the doctor about
it.
	\item I took a lot of convincing that parenthood was a good idea and I didn't think I'd
be much use at it.
	\end{itemize}
}
\item pronoun \\
You use \textbf{it} to refer to a child or baby whose sex you do not know or whose sex is not relevant to what you are saying .
 \textit{
	\begin{itemize}
	\item He promised to support the child after it was born.
	\end{itemize}
}
\item pronoun \\
You use \textbf{it} to refer in a general  way to a situation that you have just described .
 \textit{
	\begin{itemize}
	\item He was through with sports, not because he had to be but because he wanted it that
way.
	\item Antonia will not be jealous, or if she is, she will not show it.
	\end{itemize}
}
\item pronoun \\
You use \textbf{it} before certain nouns , adjectives , and verbs to introduce your feelings or point of view about a situation.
 \textit{
	\begin{itemize}
	\item It was nice to see Steve again.
	\item It's a pity you never got married, Sarah.
	\item It's funny how you remember things.
	\item It's good of him to spare the time to visit at all.
	\item Is it possible he'll phone you?
	\item He found it hard to work with a microphone pointing at him.
	\item I know it's a good idea to use dental floss.
	\item It's up to us to change things we don't like.
	\item It seems that you are letting things get you down.
	\end{itemize}
}
\item pronoun \\
You use \textbf{it} in passive clauses which report a situation or event .
 \textit{
	\begin{itemize}
	\item It has been said that stress causes cancer.
	\item Yesterday it was reported that a number of people had been arrested in the capital.
	\item It was noted that within a year the incidence of illness had increased quite significantly.
	\end{itemize}
}
\item pronoun \\
You use \textbf{it} with some verbs that need a subject or object, although there is no noun that it refers to.
 \textit{
	\begin{itemize}
	\item Of course, as it turned out, three-fourths of the people in the group were psychiatrists.
	\item I like it here.
	\end{itemize}
}
\item pronoun \\
You use \textbf{it} as the subject of 'be', to say what the time, day , or date is.
 \textit{
	\begin{itemize}
	\item It's three o'clock in the morning.
	\item It was a Monday, so she was at home.
	\item It's December 1989, in Las Vegas.
	\end{itemize}
}
\item pronoun \\
You use \textbf{it} as the subject of a link verb to describe the weather , the light, or the temperature .
 \textit{
	\begin{itemize}
	\item It was very wet and windy the day I drove over the hill to Milland.
	\item It's getting dark. Let's go inside.
	\item It was warm in the kitchen.
	\end{itemize}
}
\item pronoun \\
You use \textbf{it} when you are telling someone who you are, or asking them who they are, especially at the beginning of a phone  call . You also use \textbf{it} in statements and questions about the identity of other people.
 \textit{
	\begin{itemize}
	\item 'Who is it?' he called.—'It's your neighbor.'
	\item Hello Freddy, it's only me, Maxine.
	\end{itemize}
}
\item pronoun \\
When you are emphasizing or drawing  attention to something, you can put that thing immediately after \textbf{it} and a form of the verb 'be'.
 \textit{
	\begin{itemize}
	\item It's really these countries that have the worst environmental records.
	\item It was the country's rulers who devised this system.
	\item It was I who found him there.
	\item It's my father they're accusing.
	\end{itemize}
}
\item  \\
 it's not simply/just that \textit{
	\begin{itemize}
	\end{itemize}
}
\item  \\
 think you're it \textit{
	\begin{itemize}
	\end{itemize}
}
\end{enumerate}

\section*{diary}
{\large \color{blue}  diaries  }
\subsection*{Explain}
\begin{enumerate}
\item countable noun \\
A \textbf{diary} is a book which has a separate  space for each day of the year . You use a diary to write down things you plan to do, or to record what happens in your life day by day.
 \textit{
	\begin{itemize}
	\end{itemize}
}
\end{enumerate}

\section*{myself}
{\large \color{blue}  }
\subsection*{Explain}
\begin{enumerate}
\item pronoun \\
A speaker or writer uses \textbf{myself} to refer to himself or herself. \textbf{Myself} is used as the object of a verb or preposition when the subject refers to the same person.
 \textit{
	\begin{itemize}
	\item I asked myself what I would have done in such a situation.
	\item I looked at myself in the mirror.
	\item I felt ashamed of myself.
	\end{itemize}
}
\item pronoun \\
You use \textbf{myself} to emphasize a first person singular subject. In more formal English, \textbf{myself} is sometimes used instead of 'me' as the object of a verb or preposition, for emphasis .
 \textit{
	\begin{itemize}
	\item I myself enjoy cinema, poetry, eating out and long walks.
	\item I'm fond of cake myself.
	\item He was roughly the same age as myself.
	\item ...a complete beginner like myself.
	\end{itemize}
}
\item pronoun \\
If you say something such as 'I did it \textbf{myself} ', you are emphasizing that you did it, rather than anyone else.
 \textit{
	\begin{itemize}
	\item 'Where did you get that embroidery?'—'I made it myself.'
	\end{itemize}
}
\end{enumerate}

\section*{eclipse}
{\large \color{blue}  eclipses  eclipsing  eclipsed  }
\subsection*{Explain}
\begin{enumerate}
\item countable noun \\
An \textbf{eclipse}  \textbf{of} the sun is an occasion when the moon is between the Earth and the sun, so that for a short time you cannot see part or all of the sun. An \textbf{eclipse}  \textbf{of} the moon is an occasion when the Earth is between the sun and the moon, so that for
a short time you cannot see part or all of the moon.
 \textit{
	\begin{itemize}
	\item ...an eclipse of the sun.
	\item ...the solar eclipse on May 21st.
	\item ...the total lunar eclipse on 10 December.
	\end{itemize}
}
\item verb \\
If one thing \textbf{is eclipsed}  \textbf{by} a second thing that is bigger , newer, or more important than it, the first thing is no longer noticed because the second thing gets all the attention .
 \textit{
	\begin{itemize}
	\item The gramophone had been eclipsed by new technology such as the compact disc.
	\item Of course, nothing is going to eclipse winning the Olympic title.
	\end{itemize}
}
\end{enumerate}

\section*{nobody}
{\large \color{blue}  nobodies  }
\subsection*{Explain}
\begin{enumerate}
\item pronoun \\
\textbf{Nobody} means not a single person, or not a single member of a particular group or set.
 \textit{
	\begin{itemize}
	\item They were shut away in a little room where nobody could overhear.
	\item Nobody realizes how bad things are.
	\item Nobody ever spoke to me at press conferences.
	\item Nobody else in the neighbourhood can help.
	\end{itemize}
}
\item countable noun \\
If someone says that a person is a \textbf{nobody} , they are saying in an unkind way that the person is not at all important .
 \textit{
	\begin{itemize}
	\item A man in my position has nothing to fear from a nobody like you.
	\end{itemize}
}
\end{enumerate}

\section*{fruit}
{\large \color{blue}  fruit  fruits  fruits  fruiting  fruited  }
\subsection*{Explain}
\begin{enumerate}
\item variable noun \\
\textbf{Fruit} or a \textbf{fruit} is something which grows on a tree or bush and which contains seeds or a stone covered by a substance that you can eat .
 \textit{
	\begin{itemize}
	\item Fresh fruit and vegetables provide fibre and vitamins.
	\item ...bananas and other tropical fruits.
	\item Try to eat at least one piece of fruit a day.
	\end{itemize}
}
\item verb \\
If a plant \textbf{fruits} , it produces fruit.
 \textit{
	\begin{itemize}
	\item The scientists will study the variety of trees and observe which are fruiting.
	\end{itemize}
}
\item countable noun \\
\textbf{The}  \textbf{fruits} or \textbf{the}  \textbf{fruit}  \textbf{of} someone's work or activity are the good things that result from it.
 \textit{
	\begin{itemize}
	\item The team have really worked hard and are enjoying the fruits of that labour.
	\item The findings are the fruit of more than three years' research.
	\end{itemize}
}
\item  \\
 to bear fruit \textit{
	\begin{itemize}
	\end{itemize}
}
\item  \\
 the first fruits \textit{
	\begin{itemize}
	\end{itemize}
}
\end{enumerate}

\section*{none}
{\large \color{blue}  }
\subsection*{Explain}
\begin{enumerate}
\item quantifier \\
\textbf{None of} something means not even a small amount of it. \textbf{None of} a group of people or things means not even one of them.
 \textbf{None} is also a pronoun .
 \textit{
	\begin{itemize}
	\item She did none of the maintenance on the vehicle itself.
	\item None of us knew how to treat her.
	\item I turned to bookshops and libraries seeking information and found none.
	\item No one could imagine a great woman painter. None had existed yet.
	\item Only two cars produced by Austin-Morris could reach 100 mph and none could pass the
10-second acceleration test.
	\end{itemize}
}
\item  \\
 have none of \textit{
	\begin{itemize}
	\end{itemize}
}
\item  \\
 none but \textit{
	\begin{itemize}
	\end{itemize}
}
\item  \\
 none too \textit{
	\begin{itemize}
	\end{itemize}
}
\item  \\
 none the \textit{
	\begin{itemize}
	\end{itemize}
}
\end{enumerate}

\section*{girl}
{\large \color{blue}  girls  }
\subsection*{Explain}
\begin{enumerate}
\item countable noun \\
A \textbf{girl} is a female child.
 \textit{
	\begin{itemize}
	\item ...an eleven year old girl.
	\item I must have been a horrid little girl.
	\end{itemize}
}
\item countable noun \\
You can  refer to someone's daughter as a \textbf{girl} .
 \textit{
	\begin{itemize}
	\item We had a little girl.
	\end{itemize}
}
\item countable noun \\
Young women are often referred to as \textbf{girls} . This use could cause offence .
 \textit{
	\begin{itemize}
	\item ...a pretty twenty-year-old girl.
	\end{itemize}
}
\item countable noun \\
Some people refer to a man's girlfriend as his \textbf{girl} .
 \textit{
	\begin{itemize}
	\item I've been with my girl for nine years.
	\end{itemize}
}
\end{enumerate}

\section*{oneself}
{\large \color{blue}  }
\subsection*{Explain}
\begin{enumerate}
\item pronoun \\
A speaker or writer uses \textbf{oneself} as the object of a verb or preposition in a clause where 'oneself' meaning 'me' or 'any person in general' refers to the same person as the subject of the verb.
 \textit{
	\begin{itemize}
	\item One must apply oneself to the present and keep one's eyes firmly fixed on one's future
goals.
	\item To work one must have time to oneself.
	\end{itemize}
}
\item pronoun \\
\textbf{Oneself} can be used as the object of a verb or preposition, when 'one' is not present but
is understood to be the subject of the verb.
 \textit{
	\begin{itemize}
	\item The historic feeling of the town makes it a pleasant place to base oneself.
	\item The only guarantee of having a cabin to oneself is by travelling first class.
	\item It's so easy to feel sorry for oneself.
	\item The doll-like figures in these stories are unlike anybody, let alone oneself.
	\end{itemize}
}
\item emphatic reflexive pronoun \\
To do something \textbf{oneself} means to do it without any help from anyone else.
 \textit{
	\begin{itemize}
	\item It is a very rewarding exercise to work this out oneself.
	\item Some things one must do oneself.
	\end{itemize}
}
\item emphatic reflexive pronoun \\
You use \textbf{oneself} to emphasize that something happens to you rather than to people in general.
 \textit{
	\begin{itemize}
	\end{itemize}
}
\end{enumerate}

\section*{halt}
{\large \color{blue}  halts  halting  halted  }
\subsection*{Explain}
\begin{enumerate}
\item verb \\
When a person or a vehicle \textbf{halts} or when something \textbf{halts} them, they stop moving in the direction they were going and stand  still .
 \textit{
	\begin{itemize}
	\item They halted at a short distance from the house.
	\item The engine note changed as the aircraft landed, taxied and halted.
	\item She held her hand out flat, to halt him.
	\end{itemize}
}
\item verb \\
When something such as growth , development , or activity \textbf{halts} or when you \textbf{halt} it, it stops completely.
 \textit{
	\begin{itemize}
	\item Striking workers halted production at the auto plant yesterday.
	\item He criticised the government for failing to halt economic decline.
	\item The flow of assistance to Vietnam's fragile economy from its ideological allies has
virtually halted.
	\end{itemize}
}
\item verb \\
' \textbf{Halt!} ' is a military order to stop walking or marching and stand still.
 \textit{
	\begin{itemize}
	\item The colonel ordered 'Halt!'
	\end{itemize}
}
\item countable noun \\
A \textbf{halt} is a very small station on a country railway line, which often consists only of a
short platform and no building.
 \textit{
	\begin{itemize}
	\end{itemize}
}
\item  \\
 to call a halt \textit{
	\begin{itemize}
	\end{itemize}
}
\item  \\
 to a halt \textit{
	\begin{itemize}
	\end{itemize}
}
\item  \\
 to a halt \textit{
	\begin{itemize}
	\end{itemize}
}
\end{enumerate}

\section*{hose}
{\large \color{blue}  hoses  hosing  hosed  }
\subsection*{Explain}
\begin{enumerate}
\item countable noun \\
A \textbf{hose} is a long, flexible pipe made of rubber or plastic . Water is directed through a hose in order to do things such as put out fires , clean  cars , or water gardens .
 \textit{
	\begin{itemize}
	\item You've left the garden hose on.
	\end{itemize}
}
\item countable noun \\
A \textbf{hose} is a pipe made of rubber or plastic, along which a liquid or gas flows, for example from one part of an engine to another.
 \textit{
	\begin{itemize}
	\item Water in the engine compartment is sucked away by a hose.
	\end{itemize}
}
\item verb \\
If you \textbf{hose} something, you wash or water it using a hose.
 \textit{
	\begin{itemize}
	\item We wash our cars and hose our gardens without even thinking of the water that uses.
	\end{itemize}
}
\item uncountable noun \\
\textbf{Hose} is used to refer to tights, socks, and stockings.
 \textit{
	\begin{itemize}
	\item If you have varicose veins, consider wearing elastic support hose.
	\end{itemize}
}
\item uncountable noun \\
\textbf{Hose} is an old-fashioned men's garment that looks like a pair of very tight  trousers .
 \textit{
	\begin{itemize}
	\end{itemize}
}
\end{enumerate}

\section*{ourselves}
{\large \color{blue}  }
\subsection*{Explain}
\begin{enumerate}
\item pronoun \\
You use \textbf{ourselves} to refer to yourself and one or more other people as a group.
 \textit{
	\begin{itemize}
	\item We sat round the fire to keep ourselves warm.
	\item It was the first time we admitted to ourselves that we were tired.
	\end{itemize}
}
\item pronoun \\
A speaker or writer  sometimes uses \textbf{ourselves} to refer to people in general . \textbf{Ourselves} is used as the object of a verb or preposition when the subject refers to the same people.
 \textit{
	\begin{itemize}
	\item We all know that when we exert ourselves our heart rate increases.
	\end{itemize}
}
\item pronoun \\
You use \textbf{ourselves} to emphasize a first person plural subject. In more formal English, \textbf{ourselves} is sometimes used instead of 'us' as the object of a verb or preposition, for emphasis .
 \textit{
	\begin{itemize}
	\item Others are feeling just the way we ourselves would feel in the same situation.
	\item The people who will suffer won't be people like ourselves.
	\end{itemize}
}
\item pronoun \\
If you say something such as 'We did it \textbf{ourselves} ', you are indicating that something was done by you and a particular group of other people, rather than anyone else.
 \textit{
	\begin{itemize}
	\item We villagers built that ourselves, we had no help from anyone.
	\end{itemize}
}
\end{enumerate}

\section*{kettle}
{\large \color{blue}  kettles  }
\subsection*{Explain}
\begin{enumerate}
\item countable noun \\
A \textbf{kettle} is a covered container that you use for boiling water. It has a handle, and a spout for the water
to come out of.
 A \textbf{kettle}  \textbf{of} water is the amount of water contained in a kettle.
 \textit{
	\begin{itemize}
	\item I'll put the kettle on and make us some tea.
	\item Pour a kettle of boiling water over the onions.
	\end{itemize}
}
\item countable noun \\
A \textbf{kettle} is a metal pot for boiling or cooking things in.
 \textit{
	\begin{itemize}
	\item Put the meat into a small kettle.
	\end{itemize}
}
\item  \\
 a different kettle of fish \textit{
	\begin{itemize}
	\end{itemize}
}
\end{enumerate}

\section*{regard}
{\large \color{blue}  regards  regarding  regarded  }
\subsection*{Explain}
\begin{enumerate}
\item verb \\
If you \textbf{regard} someone or something \textbf{as} being a particular thing or \textbf{as} having a particular quality , you believe that they are that thing or have that quality.
 \textit{
	\begin{itemize}
	\item He was regarded as the most successful Chancellor of modern times.
	\item I regard creativity both as a gift and as a skill.
	\end{itemize}
}
\item verb \\
If you \textbf{regard} something or someone \textbf{with} a feeling such as dislike or respect, you have that feeling about them.
 \textit{
	\begin{itemize}
	\item They regarded him with a mixture of fondness and alarm.
	\item Displays of emotion are regarded with suspicion.
	\item He was a highly regarded scholar.
	\end{itemize}
}
\item verb \\
If you \textbf{regard} someone in a certain way, you look at them in that way.
 \textit{
	\begin{itemize}
	\item She regarded him curiously for a moment.
	\item The clerk regarded him with benevolent amusement.
	\end{itemize}
}
\item uncountable noun \\
If you have \textbf{regard}  \textbf{for} someone or something, you respect them and care about them. If you hold someone \textbf{in}  high  \textbf{regard} , you have a lot of respect for them.
 \textit{
	\begin{itemize}
	\item I have a very high regard for him and what he has achieved.
	\item There were armed people about, people with little regard for human life.
	\item The Party ruled the country without regard for the people's views.
	\end{itemize}
}
\item plural noun \\
\textbf{Regards} are greetings. You use \textbf{regards} in expressions such as \textbf{best regards} and \textbf{with kind regards} as a way of expressing  friendly feelings towards someone, especially in a letter or email.
 \textit{
	\begin{itemize}
	\item Give my regards to your family.
	\item My best regards to Mary.
	\end{itemize}
}
\item  \\
 as regards \textit{
	\begin{itemize}
	\end{itemize}
}
\item  \\
 in/with regard to \textit{
	\begin{itemize}
	\end{itemize}
}
\item  \\
 in this/that regard \textit{
	\begin{itemize}
	\end{itemize}
}
\end{enumerate}

\section*{level}
{\large \color{blue}  levels  levelling  levelled  }
\subsection*{Explain}
\begin{enumerate}
\item countable noun \\
A \textbf{level} is a point on a scale, for example a scale of amount, quality, or difficulty .
 \textit{
	\begin{itemize}
	\item If you don't know your cholesterol level, it's a good idea to have it checked.
	\item Michael's roommate had been pleasant on a superficial level.
	\item We do have the lowest level of inflation for some years.
	\item The exercises are marked according to their level of difficulty.
	\end{itemize}
}
\item singular noun \\
The \textbf{level} of a river, lake, or ocean or the \textbf{level} of liquid in a container is the height of its surface.
 \textit{
	\begin{itemize}
	\item The water level of the Mississippi River is already 6.5 feet below normal.
	\item The gauge relies upon a sensor in the tank to relay the fuel level.
	\end{itemize}
}
\item adjective \\
In cookery , a \textbf{level}  spoonful of a substance such as flour or sugar is an amount that fills the spoon exactly , without going above the top edge .
 \textit{
	\begin{itemize}
	\item Stir in 1 level teaspoon of yeast.
	\end{itemize}
}
\item singular noun \\
If something is at a particular \textbf{level} , it is at that height.
 \textit{
	\begin{itemize}
	\item Liz sank down until the water came up to her chin and the bubbles were at eye level.
	\end{itemize}
}
\item adjective \\
If one thing is \textbf{level}  \textbf{with} another thing, it is at the same height as it.
 \textit{
	\begin{itemize}
	\item He leaned over the counter so his face was almost level with the boy's.
	\item Amy knelt down so that their eyes were level.
	\end{itemize}
}
\item adjective \\
When something is \textbf{level} , it is completely flat with no part higher than any other.
 \textit{
	\begin{itemize}
	\item The floor was level, but the ceiling sloped toward his head.
	\item ...a plateau of fairly level ground.
	\end{itemize}
}
\item adverb \\
If you draw  \textbf{level} with someone or something, you get closer to them until you are by their side.
 \textbf{Level} is also an adjective .
 \textit{
	\begin{itemize}
	\item I drew level with the platform and was about to walk past.
	\item When the car pulled level with him, he spoke into the lowered passenger window.
	\item He waited until they were level with the door before he turned around sharply.
	\end{itemize}
}
\item adverb \\
If you \textbf{draw}  \textbf{level}  \textbf{with} someone, you manage to improve your performance until it is the same as theirs, by scoring the same number of points or goals as them.
 \textbf{Level} is also an adjective.
 \textit{
	\begin{itemize}
	\item They have drawn level with the opposition.
	\item The teams were level at the end of extra time.
	\end{itemize}
}
\item verb \\
In sport, if a player or team  \textbf{levels} the score, they score a goal or win some points so that their team has the same number of points or goals as the opposing
team.
 \textit{
	\begin{itemize}
	\item Iglesias scored twice to level the score.
	\item The Cincinnati Reds have levelled the score in the National League play-off against
the Pittsburgh Pirates.
	\end{itemize}
}
\item graded adjective \\
If you keep your voice  \textbf{level} , you speak in a deliberately calm and unemotional way.
 \textit{
	\begin{itemize}
	\item He forced his voice to remain level.
	\end{itemize}
}
\item verb \\
If someone or something such as a violent  storm  \textbf{levels} a building or area of land, they destroy it completely or make it completely flat.
 \textit{
	\begin{itemize}
	\item Further tremors could level more buildings.
	\item The storm was the most powerful to hit Hawaii this century. It leveled sugar plantations
and destroyed homes.
	\end{itemize}
}
\item verb \\
If an accusation or criticism \textbf{is}  \textbf{levelled}  \textbf{at} someone, they are accused of doing wrong or they are criticized for something they have done.
 \textit{
	\begin{itemize}
	\item Allegations of corruption were levelled at him and his family.
	\item He leveled bitter criticism against the U.S.
	\end{itemize}
}
\item verb \\
If you \textbf{level} an object at someone or something, you lift it and point it in their direction.
 \textit{
	\begin{itemize}
	\item He said thousands of Koreans still levelled guns at one another along the demilitarised
zone between them.
	\end{itemize}
}
\item verb \\
If you \textbf{level}  \textbf{with} someone, you tell them the truth and do not keep anything secret .
 \textit{
	\begin{itemize}
	\item I'll level with you. I'm no great detective. I've no training or anything.
	\item He has leveled with the American people about his role in the affair.
	\end{itemize}
}
\item  \\
 do one's level best \textit{
	\begin{itemize}
	\end{itemize}
}
\item  \\
 on the level \textit{
	\begin{itemize}
	\end{itemize}
}
\end{enumerate}

\section*{seek}
{\large \color{blue}  seeks  seeking  sought  }
\subsection*{Explain}
\begin{enumerate}
\item verb \\
If you \textbf{seek} something such as a job or a place to live , you try to find one.
 \textit{
	\begin{itemize}
	\item They have had to seek work as labourers.
	\item Four people who sought refuge in the Italian embassy have left voluntarily.
	\item Candidates are urgently sought for the post of Conservative Party chairman.
	\end{itemize}
}
\item verb \\
When someone \textbf{seeks} something, they try to obtain it.
 \textit{
	\begin{itemize}
	\item The prosecutors have warned they will seek the death penalty.
	\item Haemophiliacs are seeking compensation for being given contaminated blood.
	\end{itemize}
}
\item verb \\
If you \textbf{seek} someone's help or advice , you contact them in order to ask for it.
 \textit{
	\begin{itemize}
	\item Always seek professional legal advice before entering into any agreement.
	\item On important issues, they seek a second opinion.
	\item The couple have sought help from marriage guidance counsellors.
	\end{itemize}
}
\item verb \\
If you \textbf{seek}  \textbf{to} do something, you try to do it.
 \textit{
	\begin{itemize}
	\item He also denied that he would seek to annex the country.
	\item Moscow is seeking to slow the growth of Russian inflation.
	\end{itemize}
}
\end{enumerate}

\section*{maid}
{\large \color{blue}  maids  }
\subsection*{Explain}
\begin{enumerate}
\item countable noun \\
A \textbf{maid} is a woman who works as a servant in a hotel or private  house .
 \textit{
	\begin{itemize}
	\item A maid brought me breakfast at half past eight.
	\end{itemize}
}
\end{enumerate}

\section*{somebody}
{\large \color{blue}  }
\subsection*{Explain}
\begin{enumerate}
\item pronoun \\
\textbf{Somebody} means the same as someone .
 \textit{
	\begin{itemize}
	\end{itemize}
}
\end{enumerate}

\section*{maiden}
{\large \color{blue}  maidens  }
\subsection*{Explain}
\begin{enumerate}
\item countable noun \\
A \textbf{maiden} is a young girl or woman.
 \textit{
	\begin{itemize}
	\item ...stories of noble princes and their brave deeds on behalf of beautiful maidens.
	\end{itemize}
}
\item adjective \\
The \textbf{maiden}  voyage or flight of a ship or aircraft is the first official  journey that it makes.
 \textit{
	\begin{itemize}
	\item In 1912, the Titanic sank on her maiden voyage.
	\end{itemize}
}
\end{enumerate}

\section*{someone}
{\large \color{blue}  }
\subsection*{Explain}
\begin{enumerate}
\item pronoun \\
You use \textbf{someone} or \textbf{somebody} to refer to a person without saying  exactly who you mean.
 \textit{
	\begin{itemize}
	\item Her father was shot by someone trying to rob his small retail store.
	\item I need someone to help me.
	\item If somebody asks me how my diet is going, I say, 'Fine'.
	\item He noticed a huge crowd gathered outside–someone really famous must be staying there.
	\end{itemize}
}
\item pronoun \\
If you say that a person is \textbf{someone} or \textbf{somebody}  \textbf{in} a particular kind of work or \textbf{in} a particular place, you mean that they are considered to be important in that kind of work or in that place.
 \textit{
	\begin{itemize}
	\item He was somebody in the law division.
	\item 'Before she came around,' she says, 'I was somebody in this town'.
	\end{itemize}
}
\end{enumerate}

\section*{medicine}
{\large \color{blue}  medicines  }
\subsection*{Explain}
\begin{enumerate}
\item uncountable noun \\
\textbf{Medicine} is the treatment of illness and injuries by doctors and nurses .
 \textit{
	\begin{itemize}
	\item He pursued a career in medicine.
	\item I was interested in alternative medicine and becoming an aromatherapist.
	\item Psychiatry is an accepted branch of medicine.
	\end{itemize}
}
\item variable noun \\
\textbf{Medicine} is a substance that you drink or swallow in order to cure an illness.
 \textit{
	\begin{itemize}
	\item People in hospitals are dying because of shortage of medicine.
	\item ...herbal medicines.
	\end{itemize}
}
\end{enumerate}

\section*{midst}
{\large \color{blue}  }
\subsection*{Explain}
\begin{enumerate}
\item  \\
 in the midst of \textit{
	\begin{itemize}
	\end{itemize}
}
\item  \\
 in the midst of \textit{
	\begin{itemize}
	\end{itemize}
}
\item  \\
 in the midst of \textit{
	\begin{itemize}
	\end{itemize}
}
\item  \\
 in sb's midst \textit{
	\begin{itemize}
	\end{itemize}
}
\end{enumerate}

\section*{subtract}
{\large \color{blue}  subtracts  subtracting  subtracted  }
\subsection*{Explain}
\begin{enumerate}
\item verb \\
If you \textbf{subtract} one number  \textbf{from} another, you do a calculation in which you take it away from the other number. For example , if you subtract 3 from 5, you get 2.
 \textit{
	\begin{itemize}
	\item Mandy subtracted the date of birth from the date of death.
	\item We have subtracted $25 per adult to arrive at a basic room rate.
	\end{itemize}
}
\end{enumerate}

\section*{minority}
{\large \color{blue}  minorities  }
\subsection*{Explain}
\begin{enumerate}
\item singular noun \\
If you talk about a \textbf{minority} of people or things in a larger group, you are referring to a number of them that forms less than half of the larger group, usually much less than half.
 \textit{
	\begin{itemize}
	\item Local authority nursery provision covers only a tiny minority of working mothers.
	\item These children are only a small minority.
	\item ...minority shareholders.
	\end{itemize}
}
\item countable noun \\
A \textbf{minority} is a group of people of the same race , culture , or religion who live in a place where most of the people around them are of a different race, culture,
or religion.
 \textit{
	\begin{itemize}
	\item ...the region's ethnic minorities.
	\item Students have called for greater numbers of women and minorities on the faculty.
	\item A final settlement must respect minority rights.
	\end{itemize}
}
\end{enumerate}

\section*{suck}
{\large \color{blue}  sucks  sucking  sucked  }
\subsection*{Explain}
\begin{enumerate}
\item verb \\
If you \textbf{suck} something, you hold it in your mouth and pull at it with the muscles in your cheeks and tongue, for example in order to get liquid out of it.
 \textit{
	\begin{itemize}
	\item They waited in silence and sucked their sweets.
	\item He sucked on his straw.
	\item Doran was clutching the bottle with both hands and sucking intently.
	\end{itemize}
}
\item verb \\
If something \textbf{sucks} a liquid, gas , or object in a particular  direction , it draws it there with a powerful force.
 \textit{
	\begin{itemize}
	\item The pollution-control team is at the scene and is due to start sucking up oil any
time now.
	\item The air is sucked out by a high-powered fan.
	\item They sucked in deep lungfuls of air.
	\item The pilot was almost sucked from the cockpit when a window shattered.
	\end{itemize}
}
\item passive verb \\
If you \textbf{are sucked}  \textbf{into} a bad  situation , you are unable to prevent yourself from becoming  involved in it.
 \textit{
	\begin{itemize}
	\item He warned that if the President tried to enforce control, the country would be sucked
into a power vacuum.
	\item ...the extent to which they have been sucked into the cycle of violence.
	\end{itemize}
}
\item verb \\
If someone says that something \textbf{sucks} , they are indicating that they think it is very bad.
 \textit{
	\begin{itemize}
	\item The system sucks.
	\end{itemize}
}
\end{enumerate}

\section*{noon}
{\large \color{blue}  }
\subsection*{Explain}
\begin{enumerate}
\item uncountable noun \\
\textbf{Noon} is twelve o'clock in the middle of the day.
 \textit{
	\begin{itemize}
	\item The long day of meetings started at noon.
	\item Our branches are open from 9am to 5pm during the week and until 12 noon on Saturdays.
	\end{itemize}
}
\item adjective \\
\textbf{Noon} means happening or appearing in the middle part of the day.
 \textit{
	\begin{itemize}
	\item The noon sun was fierce.
	\item He expected the transfer to go through by today's noon deadline.
	\end{itemize}
}
\end{enumerate}

\section*{them}
{\large \color{blue}  }
\subsection*{Explain}
\begin{enumerate}
\item pronoun \\
You use \textbf{them} to refer to a group of people, animals, or things.
 \textit{
	\begin{itemize}
	\item The Beatles–I never get tired of listening to them.
	\item Kids these days have no one to tell them what's right and wrong.
	\item She let the dogs into the house and fed them.
	\item His dark socks, I could see, had a stripe on them.
	\end{itemize}
}
\item pronoun \\
You use \textbf{them}  instead of 'him or her' to refer to a person without saying whether that person is a man or a woman. Some people think this use is incorrect .
 \textit{
	\begin{itemize}
	\item It takes great courage to face your child and tell them the truth.
	\end{itemize}
}
\item determiner \\
In non-standard spoken English, \textbf{them} is sometimes used instead of 'those'.
 \textit{
	\begin{itemize}
	\item 'Our Billy doesn't eat them ones,' Helen said.
	\end{itemize}
}
\end{enumerate}

\section*{physician}
{\large \color{blue}  physicians  }
\subsection*{Explain}
\begin{enumerate}
\item countable noun \\
In formal American English or old-fashioned British English, a \textbf{physician} is a doctor.
 \textit{
	\begin{itemize}
	\end{itemize}
}
\end{enumerate}

\section*{themselves}
{\large \color{blue}  }
\subsection*{Explain}
\begin{enumerate}
\item pronoun \\
You use \textbf{themselves} to refer to people, animals, or things when the object of a verb or preposition refers to the same people or things as the subject of the verb.
 \textit{
	\begin{itemize}
	\item They all seemed to be enjoying themselves.
	\item The men talked amongst themselves.
	\item All artists have part of themselves that they can never share with anyone else.
	\end{itemize}
}
\item pronoun \\
You use \textbf{themselves} to emphasize the people or things that you are referring to. \textbf{Themselves} is also  sometimes used instead of 'them' as the object of a verb or preposition.
 \textit{
	\begin{itemize}
	\item Many people with serious health issues are themselves unhappy about the idea of community
care.
	\item Cities themselves are changing rapidly.
	\item Care-givers get a chance to socialize with people in the same position as themselves.
	\end{itemize}
}
\item pronoun \\
You use \textbf{themselves} instead of 'himself or herself' to refer back to the person who is the subject of
a sentence without saying whether it is a man or a woman. Some people think this use is incorrect .
 \textit{
	\begin{itemize}
	\item What can a patient with emphysema do to help themselves?
	\item Nobody was prepared to commit themselves.
	\end{itemize}
}
\item pronoun \\
You use \textbf{themselves} instead of 'himself or herself' to emphasize the person you are referring to without
saying whether it is a man or a woman. \textbf{Themselves} is also sometimes used as the object of a verb or preposition. Some people think
this use is incorrect.
 \textit{
	\begin{itemize}
	\item Each student makes only one item themselves.
	\item After all, what more can anyone be than themselves?
	\end{itemize}
}
\end{enumerate}

\section*{pool}
{\large \color{blue}  pools  pooling  pooled  }
\subsection*{Explain}
\begin{enumerate}
\item countable noun \\
A \textbf{pool} is the same as a swimming pool .
 \textit{
	\begin{itemize}
	\item ...a heated indoor pool.
	\item During winter, many people swim and the pool is crowded.
	\end{itemize}
}
\item countable noun \\
A \textbf{pool} is a fairly small area of still water.
 \textit{
	\begin{itemize}
	\item The pool had dried up and was full of bracken and reeds.
	\item ...beautiful gardens filled with pools, fountains and rare birds.
	\end{itemize}
}
\item countable noun \\
A \textbf{pool of} liquid or light is a small area of it on the ground or on a surface.
 \textit{
	\begin{itemize}
	\item She was found lying in a pool of blood.
	\item It was raining quietly and steadily and there were little pools of water on the gravel
drive.
	\item The lamps on the side-tables threw warm pools of light on the polished wood.
	\end{itemize}
}
\item countable noun \\
A \textbf{pool}  \textbf{of} people, money, or things is a quantity or number of them that is available for an organization or group to use.
 \textit{
	\begin{itemize}
	\item The available pool of healthy manpower was not as large as military officials had
expected.
	\item The new proposal would create a reserve pool of cash.
	\end{itemize}
}
\item verb \\
If a group of people or organizations \textbf{pool} their money, knowledge, or equipment, they share it or put it together so that it
can be used for a particular purpose.
 \textit{
	\begin{itemize}
	\item We pooled ideas and information.
	\item Philip and I pooled our savings to start up my business.
	\end{itemize}
}
\item uncountable noun \\
\textbf{Pool} is a game played on a large table covered with a cloth. Players use a long stick called a cue to hit a white ball across the table so that it knocks coloured balls with numbers on them into six holes around the edge of the table.
 \textit{
	\begin{itemize}
	\item We played pool together and were good mates.
	\item He was shooting pool with two other men.
	\end{itemize}
}
\item plural noun \\
If you do \textbf{the pools} , you take part in a gambling competition in which people try to win money by guessing correctly the results of football  matches .
 \textit{
	\begin{itemize}
	\item The odds of winning the pools are about one in 20 million.
	\end{itemize}
}
\end{enumerate}

\section*{this}
{\large \color{blue}  }
\subsection*{Explain}
\begin{enumerate}
\item determiner \\
You use \textbf{this} to refer back to a particular person or thing that has been mentioned or implied .
 \textbf{This} is also a pronoun.
 \textit{
	\begin{itemize}
	\item Food should be left to stand. During this delay the centre carries on cooking.
	\item The U.S. suspended a proposed $574 million aid package. Of this amount, $250 million
is for military purchases.
	\item I don't know how bad the injury is, because I have never had one like this before.
	\end{itemize}
}
\item pronoun \\
You use \textbf{this} to introduce someone or something that you are going to talk about.
 \textbf{This} is also a determiner .
 \textit{
	\begin{itemize}
	\item This is what I will do. I will phone Anna and explain.
	\item This report is from David Cook of our Science Unit: 'Why did the dinosaurs become
extinct?'
	\end{itemize}
}
\item pronoun \\
You use \textbf{this} to refer back to an idea or situation  expressed in a previous  sentence or sentences.
 \textbf{This} is also a determiner.
 \textit{
	\begin{itemize}
	\item You feel that it's uneconomic to insist that people work together in groups. Why
is this?
	\item A job is pretty much nine-to-five. Is this what you feel would make you happy?
	\item There have been demands for action to put an end to this situation.
	\end{itemize}
}
\item determiner \\
In spoken English, people use \textbf{this} to introduce a person or thing into a story .
 \textit{
	\begin{itemize}
	\item I came here by chance and was just watching what was going on, when this girl attacked
me.
	\item So I just walked up the steps into this big, beautiful church.
	\end{itemize}
}
\item pronoun \\
You use \textbf{this} to refer to a person or thing that is near you, especially when you touch them or point to them. When there are two or more people or things near you, \textbf{this} refers to the nearest one.
 \textbf{This} is also a determiner.
 \textit{
	\begin{itemize}
	\item 'If you'd prefer something else I'll gladly have it changed for you.'—'No, this is
great.'
	\item 'Is this what you were looking for?' Bradley produced the handkerchief.
	\item This is my colleague, Mr Arnold Landon.
	\item This church was built in the eleventh century.
	\end{itemize}
}
\item pronoun \\
You use \textbf{this} when you refer to a general situation, activity, or event which is happening or has just happened and which you feel involved in.
 \textit{
	\begin{itemize}
	\item I thought, this is why I've travelled thousands of miles.
	\item Tim, this is awful. I know what you must think, but it's not so.
	\item Is this what you want to do with the rest of your life?
	\end{itemize}
}
\item determiner \\
You use \textbf{this} when you refer to the place you are in now or to the present time.
 \textbf{This} is also a pronoun.
 \textit{
	\begin{itemize}
	\item We've stopped transporting weapons to this country by train.
	\item This place is run like a hotel ought to be run.
	\item I think coffee is probably the best thing at this point.
	\item Nothing seems certain in this crucial period in Pakistan's political life.
	\item This is the worst place I've come across.
	\item This could have been one of the coldest golf tournaments on record.
	\end{itemize}
}
\item determiner \\
You use \textbf{this} to refer to the next  occurrence in the future of a particular day , month , season , or festival .
 \textit{
	\begin{itemize}
	\item ...this Sunday's 7.45 performance.
	\item We're getting married this June.
	\item Jordan's own-label collection of sweatshirts, T-shirts and caps will be available
this Christmas.
	\end{itemize}
}
\item adverb \\
You use \textbf{this} when you are indicating the size or shape of something with your hands .
 \textit{
	\begin{itemize}
	\item They'd said the wound was only about this big you see and he showed me with his fingers.
	\end{itemize}
}
\item adverb \\
You use \textbf{this} when you are going to specify how much you know or how much you can tell someone.
 \textit{
	\begin{itemize}
	\item I am not going to reveal my plan, but I will tell you this much: if it works out,
the next few years will be very interesting.
	\end{itemize}
}
\item convention \\
If you say  \textbf{this is it} , you are agreeing with what someone else has just said.
 \textit{
	\begin{itemize}
	\item 'You know, people conveniently forget the things they say.'—'Well this is it.'
	\end{itemize}
}
\item pronoun \\
You use \textbf{this} in order to say who you are or what organization you are representing , when you are speaking on the phone , radio , or television .
 \textit{
	\begin{itemize}
	\item Hello, this is John Thompson.
	\item 'Hello, is this Raymond Brown?'—'Yeah, who's this?'.
	\item This is NPR, National Public Radio.
	\end{itemize}
}
\item determiner \\
You use \textbf{this} to refer to the medium of communication that you are using at the time of speaking or writing .
 \textit{
	\begin{itemize}
	\item What I'm going to do in this lecture is focus on something very specific.
	\item These are among the important topics that this book will try to address.
	\item Later in this chapter, I recommend several specific steps we need to take.
	\end{itemize}
}
\item  \\
 this and that \textit{
	\begin{itemize}
	\end{itemize}
}
\end{enumerate}

\section*{proportion}
{\large \color{blue}  proportions  }
\subsection*{Explain}
\begin{enumerate}
\item countable noun \\
A \textbf{proportion of} a group or an amount is a part of it.
 \textit{
	\begin{itemize}
	\item A large proportion of the dolphins in that area will eventually die.
	\item A proportion of the rent is met by the city council.
	\end{itemize}
}
\item countable noun \\
The \textbf{proportion of} one kind of person or thing in a group is the number of people or things of that kind compared to the total number of people or things in the group.
 \textit{
	\begin{itemize}
	\item The proportion of women in the profession had risen to 17.3%.
	\item The radio station has to include a substantial proportion of classical music.
	\end{itemize}
}
\item countable noun \\
The \textbf{proportion of} one amount \textbf{to} another is the relationship between the two amounts in terms of how much there is of each thing.
 \textit{
	\begin{itemize}
	\item Women's bodies tend to have a higher proportion of fat to water.
	\end{itemize}
}
\item plural noun \\
If you refer to the \textbf{proportions} of something, you are referring to its size, usually when this is extremely large.
 \textit{
	\begin{itemize}
	\item In the tropics plants grow to huge proportions.
	\item ...a fraud of breathtaking proportions.
	\end{itemize}
}
\item plural noun \\
If you refer to the \textbf{proportions} in a work of art or design, you are referring to the relative sizes of its different
parts.
 \textit{
	\begin{itemize}
	\item You can vary the relative proportions of things in a picture very simply.
	\end{itemize}
}
\item  \\
 in proportion to \textit{
	\begin{itemize}
	\end{itemize}
}
\item  \\
 in proportion to \textit{
	\begin{itemize}
	\end{itemize}
}
\item  \\
 out of all proportion to \textit{
	\begin{itemize}
	\end{itemize}
}
\item  \\
 out of proportion/in proportion \textit{
	\begin{itemize}
	\end{itemize}
}
\item  \\
 sense of proportion \textit{
	\begin{itemize}
	\end{itemize}
}
\end{enumerate}

\section*{us}
{\large \color{blue}  }
\subsection*{Explain}
\begin{enumerate}
\item pronoun \\
A speaker or writer uses \textbf{us} to refer both to himself or herself and to one or more other people. You can use
 \textbf{us} before a noun to make it clear which group of people you are referring to.
 \textit{
	\begin{itemize}
	\item Neither of us forgot about it.
	\item Heather went to the kitchen to get drinks for us.
	\item They don't like us much.
	\item He showed us aspects of the game that we had never seen before.
	\item Another time of great excitement for us boys was when war broke out.
	\end{itemize}
}
\item pronoun \\
\textbf{Us} is sometimes used to refer to people in general.
 \textit{
	\begin{itemize}
	\item All of us will struggle fairly hard to survive if we are in danger.
	\item Each of us will have our own criteria for success.
	\end{itemize}
}
\item pronoun \\
A speaker or writer may use \textbf{us}  instead of 'me' in order to include the audience or reader in what they are saying .
 \textit{
	\begin{itemize}
	\item This brings us to the second question I asked.
	\end{itemize}
}
\item pronoun \\
In non-standard English , \textbf{us} is sometimes used instead of 'me'.
 \textit{
	\begin{itemize}
	\item 'Hang on a bit,' said Eileen. 'I'm not finished yet. Give us a chance.'
	\end{itemize}
}
\end{enumerate}

\section*{ratio}
{\large \color{blue}  ratios  }
\subsection*{Explain}
\begin{enumerate}
\item countable noun \\
A \textbf{ratio} is a relationship between two things when it is expressed in numbers or amounts. For example , if there are ten  boys and thirty  girls in a room , the ratio of boys to girls is 1:3, or one to three.
 \textit{
	\begin{itemize}
	\item In 1978 there were 884 students at a lecturer/student ratio of 1:15.
	\item The bottom chart shows the ratio of personal debt to personal income.
	\item The adult to child ratio is 1 to 6.
	\end{itemize}
}
\end{enumerate}

\section*{we}
{\large \color{blue}  }
\subsection*{Explain}
\begin{enumerate}
\item pronoun \\
A speaker or writer uses \textbf{we} to refer both to himself or herself and to one or more other people as a group. You
 can use \textbf{we} before a noun to make it clear which group of people you are referring to.
 \textit{
	\begin{itemize}
	\item We both swore we'd be friends ever after.
	\item We ordered another bottle of champagne.
	\item Don't you think we should ask this young man some technical questions?
	\item We students outnumbered our teachers.
	\end{itemize}
}
\item pronoun \\
\textbf{We} is sometimes used to refer to people in general.
 \textit{
	\begin{itemize}
	\item We need to take care of our bodies.
	\item ...the withdrawal symptoms that we all experience at the end of a long, close relationship.
	\end{itemize}
}
\item pronoun \\
A speaker or writer may use \textbf{we} instead of 'I' in order to include the audience or reader in what they are saying , especially when discussing how a talk or book is organized .
 \textit{
	\begin{itemize}
	\item We will now consider the raw materials from which the body derives energy.
	\end{itemize}
}
\end{enumerate}

\section*{reservoir}
{\large \color{blue}  reservoirs  }
\subsection*{Explain}
\begin{enumerate}
\item countable noun \\
A \textbf{reservoir} is a lake that is used for storing water before it is supplied to people.
 \textit{
	\begin{itemize}
	\end{itemize}
}
\item countable noun \\
A \textbf{reservoir}  \textbf{of} something is a large quantity of it that is available for use when needed .
 \textit{
	\begin{itemize}
	\item ...the huge oil reservoir beneath the Kuwaiti desert.
	\item ...the body's short-term reservoir of energy.
	\end{itemize}
}
\end{enumerate}

\section*{what}
{\large \color{blue}  }
\subsection*{Explain}
\begin{enumerate}
\item pronoun \\
You use \textbf{what} in questions when you ask for specific information about something that you do not know .
 \textbf{What} is also a determiner .
 \textit{
	\begin{itemize}
	\item What do you want?
	\item What did she tell you, anyway?
	\item 'Has something happened?'—'Indeed it has.'—'What?'
	\item What are the greatest sources of conflict in the Middle East?
	\item Hey! What are you doing?
	\item What time is it?
	\item What crimes are the defendants being charged with?
	\item 'The heater works.'—'What heater?'
	\item What kind of poetry does he like?
	\end{itemize}
}
\item conjunction \\
You use \textbf{what} after certain words, especially  verbs and adjectives , when you are referring to a situation that is unknown or has not been specified.
 \textbf{What} is also a determiner.
 \textit{
	\begin{itemize}
	\item You can imagine what it would be like driving a car into a brick wall at 30 miles
an hour.
	\item I want to know what happened to Norman.
	\item Do you know what those idiots have done?
	\item We had never seen anything like it before and could not see what to do next.
	\item She turned scarlet from embarrassment, once she realized what she had done.
	\item I didn't know what college I wanted to go to.
	\item I didn't know what else to say.
	\item ...an inspection to ascertain to what extent colleges are responding to the needs
of industry.
	\end{itemize}
}
\item conjunction \\
You use \textbf{what} at the beginning of a clause in structures where you are changing the order of the information to give special  emphasis to something.
 \textit{
	\begin{itemize}
	\item What precisely triggered off yesterday's riot is still unclear.
	\item What I wanted, more than anything, was a few days' rest.
	\item What she does possess is the ability to get straight to the core of a problem.
	\end{itemize}
}
\item conjunction \\
You use \textbf{what} in expressions such as \textbf{what is called} and \textbf{what amounts to} when you are giving a description of something.
 \textit{
	\begin{itemize}
	\item She had been in what doctors described as an irreversible vegetative state for five
years.
	\item It's part of the fashion for what could be called 'retrotainment'.
	\end{itemize}
}
\item conjunction \\
You use \textbf{what} to indicate that you are talking about the whole of an amount that is available to you.
 \textbf{What} is also a determiner.
 \textit{
	\begin{itemize}
	\item He drinks what is left in his glass as if it were water.
	\item He moved carefully over what remained of partition walls.
	\item They had to use what money they had.
	\end{itemize}
}
\item convention \\
You say ' \textbf{What?} ' to tell someone who has indicated that they want to speak to you that you have heard them and are inviting them to continue .
 \textit{
	\begin{itemize}
	\item 'Dad?'—'What?'—'Can I have the car tonight?'
	\end{itemize}
}
\item convention \\
You say ' \textbf{What?} ' when you ask someone to repeat the thing that they have just said because you did not hear or understand it properly. 'What?' is more informal and less polite than expressions such as ' Pardon ?' and ' Excuse me?'.
 \textit{
	\begin{itemize}
	\item 'They could paint this place,' she said. 'What?' he asked.
	\end{itemize}
}
\item convention \\
You say ' \textbf{What} ' to express surprise .
 \textit{
	\begin{itemize}
	\item What! You want Saturday off as well?
	\item 'We've got the car that killed Myra Moss.'—'What!'
	\end{itemize}
}
\item predeterminer \\
You use \textbf{what} in exclamations to emphasize an opinion or reaction .
 \textbf{What} is also a determiner.
 \textit{
	\begin{itemize}
	\item What a horrible thing to do.
	\item What a busy day.
	\item What ugly things; throw them away!
	\item What great news, Jakki.
	\end{itemize}
}
\item adverb \\
You use \textbf{what} to indicate that you are making a guess about something such as an amount or value.
 \textit{
	\begin{itemize}
	\item It's, what, eleven years or more since he's seen him.
	\item This piece is, what, about a half an hour long?
	\end{itemize}
}
\item  \\
 guess what/do you know what \textit{
	\begin{itemize}
	\end{itemize}
}
\item  \\
 or what \textit{
	\begin{itemize}
	\end{itemize}
}
\item  \\
 so what, what of it \textit{
	\begin{itemize}
	\end{itemize}
}
\item  \\
 tell you what \textit{
	\begin{itemize}
	\end{itemize}
}
\item  \\
 what about \textit{
	\begin{itemize}
	\end{itemize}
}
\item  \\
 what about/of \textit{
	\begin{itemize}
	\end{itemize}
}
\item  \\
 what about \textit{
	\begin{itemize}
	\end{itemize}
}
\item  \\
 what have you \textit{
	\begin{itemize}
	\end{itemize}
}
\item  \\
 what if \textit{
	\begin{itemize}
	\end{itemize}
}
\item  \\
 what's what \textit{
	\begin{itemize}
	\end{itemize}
}
\item  \\
 what with \textit{
	\begin{itemize}
	\end{itemize}
}
\item  \\
 you what \textit{
	\begin{itemize}
	\end{itemize}
}
\end{enumerate}

\section*{sailor}
{\large \color{blue}  sailors  }
\subsection*{Explain}
\begin{enumerate}
\item countable noun \\
A \textbf{sailor} is someone who works on a ship or sails a boat .
 \textit{
	\begin{itemize}
	\end{itemize}
}
\end{enumerate}

\section*{whatever}
{\large \color{blue}  }
\subsection*{Explain}
\begin{enumerate}
\item conjunction \\
You use \textbf{whatever} to refer to anything or everything of a particular type.
 \textbf{Whatever} is also a determiner .
 \textit{
	\begin{itemize}
	\item Franklin was free to do pretty much whatever he pleased.
	\item When you're older I think you're better equipped mentally to cope with whatever happens.
	\item He's good at whatever he does.
	\item Whatever doubts he might have had about Ingrid were all over now.
	\end{itemize}
}
\item conjunction \\
You use \textbf{whatever} to say that something is the case in all circumstances .
 \textit{
	\begin{itemize}
	\item We shall love you whatever happens, Diana.
	\item People will judge you whatever you do.
	\item She runs about 15 miles a day every day, whatever the weather.
	\end{itemize}
}
\item adverb \\
You use \textbf{whatever} after a noun group in order to emphasize a negative  statement .
 \textit{
	\begin{itemize}
	\item There is no evidence whatever that competition in broadcasting has ever reduced costs.
	\item I have nothing whatever to say.
	\end{itemize}
}
\item pronoun \\
You use \textbf{whatever} to ask in an emphatic way about something which you are very surprised about.
 \textit{
	\begin{itemize}
	\item Whatever can you mean?
	\item Whatever is the matter with you both?
	\end{itemize}
}
\item conjunction \\
You use \textbf{whatever} when you are indicating that you do not know the precise  identity , meaning , or value of the thing just mentioned .
 \textit{
	\begin{itemize}
	\item I thought that my upbringing was 'normal', whatever that is.
	\item 'I love you,' he said.—'Whatever that means,' she said.
	\end{itemize}
}
\item  \\
 or whatever \textit{
	\begin{itemize}
	\end{itemize}
}
\item convention \\
You say ' \textbf{whatever you say} ' to indicate that you accept what someone has said , even though you do not really  believe them or do not think it is a good idea.
 \textit{
	\begin{itemize}
	\item 'We'll go in your car, Billy.'—'Whatever you say.'
	\end{itemize}
}
\item  \\
 whatever sb does \textit{
	\begin{itemize}
	\end{itemize}
}
\end{enumerate}

\section*{steam}
{\large \color{blue}  steams  steaming  steamed  }
\subsection*{Explain}
\begin{enumerate}
\item uncountable noun \\
\textbf{Steam} is the hot mist that forms when water boils. \textbf{Steam}  vehicles and machines are operated using steam as a means of power.
 \textit{
	\begin{itemize}
	\item In an electric power plant the heat converts water into high-pressure steam.
	\item ...the invention of the steam engine.
	\end{itemize}
}
\item verb \\
If something \textbf{steams} , it gives off steam.
 \textit{
	\begin{itemize}
	\item ...restaurants where coffee pots steamed on their burners.
	\item ...a basket of steaming bread rolls.
	\end{itemize}
}
\item verb \\
If you \textbf{steam}  food or if it \textbf{steams} , you cook it in steam rather than in water.
 \textit{
	\begin{itemize}
	\item Steam the carrots until they are just beginning to be tender.
	\item Leave the vegetables to steam over the rice for the 20 minutes cooking time.
	\item ...steamed clams and broiled chicken.
	\end{itemize}
}
\item  \\
 full steam ahead \textit{
	\begin{itemize}
	\end{itemize}
}
\item  \\
 to let off steam \textit{
	\begin{itemize}
	\end{itemize}
}
\item  \\
 pick up steam \textit{
	\begin{itemize}
	\end{itemize}
}
\item  \\
 to run out of steam \textit{
	\begin{itemize}
	\end{itemize}
}
\item  \\
 under one's own steam \textit{
	\begin{itemize}
	\end{itemize}
}
\end{enumerate}

\section*{who}
{\large \color{blue}  }
\subsection*{Explain}
\begin{enumerate}
\item pronoun \\
You use \textbf{who} in questions when you ask about the name or identity of a person or group of people.
 \textit{
	\begin{itemize}
	\item Who's there?
	\item Who is the least popular man around here?
	\item Who do you work for?
	\item Who do you suppose will replace her on the show?
	\item 'You reminded me of somebody.'—'Who?'
	\end{itemize}
}
\item conjunction \\
You use \textbf{who} after certain words, especially  verbs and adjectives , to introduce a clause where you talk about the identity of a person or a group of people.
 \textit{
	\begin{itemize}
	\item Police have not been able to find out who was responsible for the forgeries.
	\item I went over to start up a conversation, asking her who she knew at the party.
	\item You know who these people are.
	\end{itemize}
}
\item pronoun \\
You use \textbf{who} at the beginning of a relative clause when specifying the person or group of people you are talking
about or when giving more information about them.
 \textit{
	\begin{itemize}
	\item There are those who eat out for a special occasion, or treat themselves.
	\item The woman, who needs constant attention, is cared for by relatives.
	\item The hijacker gave himself up to police, who are now questioning him.
	\end{itemize}
}
\end{enumerate}

\section*{sunrise}
{\large \color{blue}  sunrises  }
\subsection*{Explain}
\begin{enumerate}
\item uncountable noun \\
\textbf{Sunrise} is the time in the morning when the sun first appears in the sky .
 \textit{
	\begin{itemize}
	\item The rain began before sunrise.
	\end{itemize}
}
\item countable noun \\
A \textbf{sunrise} is the colours and light that you see in the eastern part of the sky when the sun first appears.
 \textit{
	\begin{itemize}
	\item There was a spectacular sunrise yesterday.
	\end{itemize}
}
\end{enumerate}

\section*{whoever}
{\large \color{blue}  }
\subsection*{Explain}
\begin{enumerate}
\item conjunction \\
You use \textbf{whoever} to refer to someone when their identity is not yet known.
 \textit{
	\begin{itemize}
	\item Whoever did this will sooner or later be caught and will be punished.
	\item Whoever wins the election is going to have a tough job getting the economy back on
its feet.
	\item Ben, I want whoever's responsible to come forward.
	\end{itemize}
}
\item conjunction \\
You use \textbf{whoever} to indicate that the actual identity of the person who does something will not affect a situation .
 \textit{
	\begin{itemize}
	\item You can have whoever you like to visit you.
	\item Everybody who goes into this region, whoever they are, is at risk of being taken
hostage.
	\end{itemize}
}
\item adverb \\
You use \textbf{whoever} in questions as an emphatic way of saying 'who', usually when you are surprised about something.
 \textit{
	\begin{itemize}
	\item Whoever thought up that joke?
	\item Ridiculous! Whoever suggested such a thing?
	\end{itemize}
}
\item  \\
 or whoever \textit{
	\begin{itemize}
	\end{itemize}
}
\end{enumerate}

\section*{sunset}
{\large \color{blue}  sunsets  }
\subsection*{Explain}
\begin{enumerate}
\item uncountable noun \\
\textbf{Sunset} is the time in the evening when the sun disappears out of sight from the sky .
 \textit{
	\begin{itemize}
	\item The dance ends at sunset.
	\end{itemize}
}
\item countable noun \\
A \textbf{sunset} is the colours and light that you see in the western part of the sky when the sun disappears in the evening.
 \textit{
	\begin{itemize}
	\item There was a red sunset over Paris.
	\end{itemize}
}
\end{enumerate}

\section*{whom}
{\large \color{blue}  }
\subsection*{Explain}
\begin{enumerate}
\item pronoun \\
You use \textbf{whom} in questions when you ask about the name or identity of a person or group of people.
 \textit{
	\begin{itemize}
	\item 'I want to send a telegram.'—'Fine, to whom?'
	\item Whom did he expect to answer his phone?
	\item 'You're too sensitive.'—'Too sensitive for whom?'
	\end{itemize}
}
\item conjunction \\
You use \textbf{whom} after certain words, especially  verbs and adjectives , to introduce a clause where you talk about the name or identity of a person or a group of people.
 \textit{
	\begin{itemize}
	\item He asked whom I'd told about his having been away.
	\item He likes to know whom you've met.
	\item I have resigned, and they have a free hand to appoint whom they like in my place.
	\end{itemize}
}
\item pronoun \\
You use \textbf{whom} at the beginning of a relative clause when specifying the person or group of people you are talking about or when
giving more information about them.
 \textit{
	\begin{itemize}
	\item One writer in whom I had taken an interest was Immanuel Velikovsky.
	\item The Homewood residents whom I knew had little money and little free time.
	\item ...generations of women for whom work provided an escape from family life.
	\end{itemize}
}
\end{enumerate}

\section*{sunshine}
{\large \color{blue}  }
\subsection*{Explain}
\begin{enumerate}
\item uncountable noun \\
\textbf{Sunshine} is the light and heat that comes from the sun.
 \textit{
	\begin{itemize}
	\item In the marina yachts sparkle in the sunshine.
	\item She was sitting outside a cafe in bright sunshine.
	\item I awoke next morning to brilliant sunshine streaming into my room.
	\end{itemize}
}
\end{enumerate}

\section*{whose}
{\large \color{blue}  }
\subsection*{Explain}
\begin{enumerate}
\item pronoun \\
You use \textbf{whose} at the beginning of a relative clause where you mention something that belongs to or is associated with the person or thing mentioned in
the previous clause.
 \textit{
	\begin{itemize}
	\item I saw a man shouting at a driver whose car was blocking the street.
	\item ...a speedboat, whose fifteen-strong crew claimed to belong to the Italian navy.
	\item ...tourists whose vacations included an unexpected adventure.
	\end{itemize}
}
\item pronoun \\
You use \textbf{whose} in questions to ask about the person or thing that something belongs to or is associated with.
 \textit{
	\begin{itemize}
	\item Whose was the better performance?
	\item 'Whose is this?'—'It's mine.'
	\item 'It wasn't your fault, John.'—'Whose, then?'
	\item Whose car were they in?
	\item Whose daughter is she?
	\end{itemize}
}
\item determiner \\
You use \textbf{whose} after certain words, especially  verbs and adjectives , to introduce a clause where you talk about the person or thing that something belongs to or is associated with.
 \textbf{Whose} is also a conjunction .
 \textit{
	\begin{itemize}
	\item I'm wondering whose mother she is then.
	\item I can't remember whose idea it was for us to meet again.
	\item I wondered whose the coat was.
	\item That kind of person likes to spend money, it doesn't matter whose it is.
	\end{itemize}
}
\end{enumerate}

\section*{tile}
{\large \color{blue}  tiles  tiling  tiled  }
\subsection*{Explain}
\begin{enumerate}
\item variable noun \\
\textbf{Tiles} are flat, square pieces of baked clay, carpet , cork , or other substance, which are fixed as a covering onto a floor or wall.
 \textit{
	\begin{itemize}
	\item Amy's shoes squeaked on the tiles as she walked down the corridor.
	\item The cabins had linoleum tile floors.
	\item ...a broken piece of tile.
	\end{itemize}
}
\item variable noun \\
\textbf{Tiles} are flat pieces of baked clay which are used for covering roofs.
 \textit{
	\begin{itemize}
	\item ...a fine building, with a neat little porch and ornamental tiles on the roof.
	\end{itemize}
}
\item verb \\
When someone \textbf{tiles} a surface such as a roof or floor, they cover it with tiles.
 \textit{
	\begin{itemize}
	\item He wants to tile the bathroom.
	\item The terracotta tiled floor gives the place a wonderfully homely character.
	\end{itemize}
}
\item  \\
 a night/out on the tiles \textit{
	\begin{itemize}
	\end{itemize}
}
\end{enumerate}

\section*{you}
{\large \color{blue}  yous  }
\subsection*{Explain}
\begin{enumerate}
\item pronoun \\
A speaker or writer uses \textbf{you} to refer to the person or people that they are talking or writing to. It is possible to use \textbf{you} before a noun to make it clear which group of people you are talking to.
 \textit{
	\begin{itemize}
	\item When I saw you across the room I knew I'd met you before.
	\item You two seem very different to me.
	\item I could always talk to you about anything in the world.
	\item What is alternative health care? What can it do for you?
	\item What you kids need is more exercise.
	\end{itemize}
}
\item pronoun \\
In spoken English and informal written English, \textbf{you} is sometimes used to refer to people in general.
 \textit{
	\begin{itemize}
	\item Getting good results gives you confidence.
	\item In those days you did what you were told.
	\end{itemize}
}
\item plural pronoun \\
In some dialects of English, \textbf{yous} is sometimes used instead of 'you' when talking to two or more people.
 \textit{
	\begin{itemize}
	\item 'Yous two are no' gettin' paid,' he said. 'Ye're too lazy!'
	\end{itemize}
}
\end{enumerate}

\section*{trifle}
{\large \color{blue}  trifles  trifling  trifled  }
\subsection*{Explain}
\begin{enumerate}
\item  \\
 a trifle \textit{
	\begin{itemize}
	\end{itemize}
}
\item countable noun \\
A \textbf{trifle} is something that is considered to have little importance , value, or significance.
 \textit{
	\begin{itemize}
	\item He had no money to spare on trifles.
	\item Believe me, it's the least I can do, a mere trifle.
	\end{itemize}
}
\item variable noun \\
\textbf{Trifle} is a cold dessert made of layers of sponge cake, jelly , fruit, and custard, and usually covered with cream.
 \textit{
	\begin{itemize}
	\end{itemize}
}
\end{enumerate}

\section*{your}
{\large \color{blue}  }
\subsection*{Explain}
\begin{enumerate}
\item determiner \\
A speaker or writer uses \textbf{your} to indicate that something belongs or relates to the person or people that they are
 talking or writing to.
 \textit{
	\begin{itemize}
	\item Emma, I trust your opinion a great deal.
	\item I left all of your messages on your desk.
	\item If you are unable to obtain the information you require, consult your telephone directory.
	\end{itemize}
}
\item determiner \\
In spoken English and informal  written English, \textbf{your} is sometimes used to indicate that something belongs to or relates to people in general.
 \textit{
	\begin{itemize}
	\item Pain-killers are very useful in small amounts to bring your temperature down.
	\item I then realized how possible it was to overcome your limitations.
	\end{itemize}
}
\item determiner \\
In spoken English, a speaker sometimes uses \textbf{your} before an adjective such as ' typical ' or ' normal ' to indicate that the thing referred to is a typical example of its type.
 \textit{
	\begin{itemize}
	\item Stan Reilly is not really one of your typical Brighton Boys.
	\item It's just your average wooden door.
	\end{itemize}
}
\end{enumerate}

\section*{water}
{\large \color{blue}  waters  watering  watered  }
\subsection*{Explain}
\begin{enumerate}
\item uncountable noun \\
\textbf{Water} is a clear thin liquid that has no colour or taste when it is pure . It falls from clouds as rain and enters rivers and seas. All animals and people need water in order to live .
 \textit{
	\begin{itemize}
	\item Get me a glass of water.
	\item ...the sound of water hammering on the metal roof.
	\item ...a trio of children playing along the water's edge.
	\end{itemize}
}
\item plural noun \\
You use \textbf{waters} to refer to a large area of sea, especially the area of sea which is near to a country and which is regarded as belonging to
it.
 \textit{
	\begin{itemize}
	\item The ship will remain outside Chinese territorial waters.
	\item ...the open waters of the Arctic Ocean.
	\end{itemize}
}
\item plural noun \\
You sometimes use \textbf{waters} to refer to a situation which is very complex or difficult .
 \textit{
	\begin{itemize}
	\item ...the man brought in to guide him through troubled waters.
	\item The country may be in stormy economic waters.
	\end{itemize}
}
\item verb \\
If you \textbf{water} plants, you pour water over them in order to help them to grow.
 \textit{
	\begin{itemize}
	\item He went out to water the plants.
	\end{itemize}
}
\item verb \\
If your eyes \textbf{water} , tears build up in them because they are hurting or because you are upset .
 \textit{
	\begin{itemize}
	\item His eyes watered from the smoke.
	\end{itemize}
}
\item verb \\
If you say that your mouth \textbf{is watering} , you mean that you can smell or see some nice food and you might mean that your mouth is producing a liquid.
 \textit{
	\begin{itemize}
	\item ...cookies to make your mouth water.
	\end{itemize}
}
\item  \\
 waters break/break sb's waters \textit{
	\begin{itemize}
	\end{itemize}
}
\item  \\
 water under the bridge \textit{
	\begin{itemize}
	\end{itemize}
}
\item  \\
 in deep water \textit{
	\begin{itemize}
	\end{itemize}
}
\item  \\
 hold water \textit{
	\begin{itemize}
	\end{itemize}
}
\item  \\
 in hot water \textit{
	\begin{itemize}
	\end{itemize}
}
\item  \\
 to pour cold water on something \textit{
	\begin{itemize}
	\end{itemize}
}
\item  \\
 test the water \textit{
	\begin{itemize}
	\end{itemize}
}
\end{enumerate}

\section*{yours}
{\large \color{blue}  }
\subsection*{Explain}
\begin{enumerate}
\item pronoun \\
A speaker or writer uses \textbf{yours} to refer to something that belongs or relates to the person or people that they are talking or writing to.
 \textit{
	\begin{itemize}
	\item I'll take my coat upstairs. Shall I take yours, Roberta?
	\item I believe Paul was a friend of yours.
	\item If yours is a high-stress job, it is important that you learn how to cope.
	\end{itemize}
}
\item convention \\
People write \textbf{yours} , \textbf{yours sincerely} , or \textbf{yours faithfully} at the end of a letter before they sign their name.
 \textit{
	\begin{itemize}
	\item With best regards, Yours, George.
	\item Yours faithfully, Michael Moore, London Business School.
	\item Waiting to hear from you, Yours sincerely, William Faulkner.
	\end{itemize}
}
\end{enumerate}

\section*{watt}
{\large \color{blue}  watts  }
\subsection*{Explain}
\begin{enumerate}
\item countable noun \\
A \textbf{watt} is a unit of measurement of electrical power.
 \textit{
	\begin{itemize}
	\item Use a 3 amp fuse for equipment up to 720 watts.
	\item ...a 100-watt lightbulb.
	\end{itemize}
}
\end{enumerate}

\section*{yourself}
{\large \color{blue}  yourselves  }
\subsection*{Explain}
\begin{enumerate}
\item pronoun \\
A speaker or writer uses \textbf{yourself} to refer to the person that they are talking or writing to. \textbf{Yourself} is used when the object of a verb or preposition refers to the same person as the subject of the verb.
 \textit{
	\begin{itemize}
	\item Have the courage to be honest with yourself and about yourself.
	\item Your baby depends on you to look after yourself properly while you are pregnant.
	\item Treat yourselves to a massage to help you relax at the end of the day.
	\end{itemize}
}
\item pronoun \\
You use \textbf{yourself} to emphasize the person that you are referring to.
 \textit{
	\begin{itemize}
	\item They mean to share the business between them, after you yourself are gone, Sir.
	\item I've been wondering if you yourselves have any idea why she came.
	\end{itemize}
}
\item pronoun \\
You use \textbf{yourself}  instead of 'you' for emphasis or in order to be more polite when 'you' is the object of a verb or preposition.
 \textit{
	\begin{itemize}
	\item A wealthy man like yourself is bound to make an enemy or two along the way.
	\item I wouldn't want to cause such important people as yourselves any bother.
	\end{itemize}
}
\end{enumerate}

\section*{arm}
{\large \color{blue}  arms  }
\subsection*{Explain}
\begin{enumerate}
\item countable noun \\
Your \textbf{arms} are the two long parts of your body that are attached to your shoulders and that
have your hands at the end.
 \textit{
	\begin{itemize}
	\item She stretched her arms out.
	\item He had a large parcel under his left arm.
	\end{itemize}
}
\item countable noun \\
The \textbf{arm} of a piece of clothing is the part of it that covers your arm.
 \textit{
	\begin{itemize}
	\end{itemize}
}
\item countable noun \\
The \textbf{arm} of a chair is the part on which you rest your arm when you are sitting down.
 \textit{
	\begin{itemize}
	\end{itemize}
}
\item countable noun \\
An \textbf{arm}  \textbf{of} an object is a long thin part of it that sticks out from the main part.
 \textit{
	\begin{itemize}
	\item ...the lever arm of the machine.
	\item ...the arms of the doctor's spectacles.
	\end{itemize}
}
\item countable noun \\
An \textbf{arm}  \textbf{of} land or water is a long thin area of it that is joined to a broader area.
 \textit{
	\begin{itemize}
	\item At the end of the other arm of Cardigan Bay is Bardsey Island.
	\end{itemize}
}
\item countable noun \\
An \textbf{arm}  \textbf{of} an organization is a section of it that operates in a particular country or that deals with a particular activity.
 \textit{
	\begin{itemize}
	\item Millicom Holdings is the British arm of an American company.
	\item ...the research arm of Congress.
	\end{itemize}
}
\item  \\
 arm in arm \textit{
	\begin{itemize}
	\end{itemize}
}
\item  \\
 an arm and a leg \textit{
	\begin{itemize}
	\end{itemize}
}
\item  \\
 at arm's length \textit{
	\begin{itemize}
	\end{itemize}
}
\item  \\
 keep sb at arm's length \textit{
	\begin{itemize}
	\end{itemize}
}
\item  \\
 as long as your arm \textit{
	\begin{itemize}
	\end{itemize}
}
\item  \\
 with open arms \textit{
	\begin{itemize}
	\end{itemize}
}
\item  \\
 to twist someone's arm \textit{
	\begin{itemize}
	\end{itemize}
}
\end{enumerate}

\section*{begin}
{\large \color{blue}  begins  beginning  began  begun  }
\subsection*{Explain}
\begin{enumerate}
\item verb \\
To \textbf{begin}  \textbf{to} do something means to start doing it.
 \textit{
	\begin{itemize}
	\item He stood up and began to move around the room.
	\item The weight loss began to look more serious.
	\item Snow began falling again.
	\end{itemize}
}
\item verb \\
When something \textbf{begins} or when you \textbf{begin} it, it takes place from a particular time onwards .
 \textit{
	\begin{itemize}
	\item The problems began last November.
	\item He has just begun his fourth year in hiding.
	\item The U.S. is prepared to begin talks immediately.
	\end{itemize}
}
\item verb \\
If you \textbf{begin}  \textbf{with} something, or \textbf{begin}  \textbf{by} doing something, this is the first thing you do.
 \textit{
	\begin{itemize}
	\item Could I begin with a few formalities?
	\item ...a businessman who began by selling golf shirts from the boot of his car.
	\item He began his career as a sound editor.
	\end{itemize}
}
\item verb \\
You use \textbf{begin} to mention the first thing that someone says.
 \textit{
	\begin{itemize}
	\item 'Professor Theron,' he began, 'I'm very pleased to see you'.
	\item He didn't know how to begin.
	\end{itemize}
}
\item verb \\
If one thing \textbf{began as} another, it first existed in the form of the second thing.
 \textit{
	\begin{itemize}
	\item What began as a local festival has blossomed into an international event.
	\end{itemize}
}
\item verb \\
If you say that a thing or place \textbf{begins}  somewhere , you are talking about one of its limits or edges .
 \textit{
	\begin{itemize}
	\item The fate line begins at the wrist.
	\item Rue Guynemer begins at the front of the Fitzgerald site.
	\end{itemize}
}
\item verb \\
If a word \textbf{begins with} a particular letter , that is the first letter of that word.
 \textit{
	\begin{itemize}
	\item The first word begins with an F.
	\end{itemize}
}
\item verb \\
If you say that you cannot \textbf{begin}  \textbf{to}  imagine , understand , or explain something, you are emphasizing that it is almost  impossible to explain, understand, or imagine.
 \textit{
	\begin{itemize}
	\item You can't begin to imagine how much that saddens me.
	\end{itemize}
}
\item  \\
 to begin with \textit{
	\begin{itemize}
	\end{itemize}
}
\item  \\
 to begin with \textit{
	\begin{itemize}
	\end{itemize}
}
\end{enumerate}

\section*{balloon}
{\large \color{blue}  balloons  ballooning  ballooned  }
\subsection*{Explain}
\begin{enumerate}
\item countable noun \\
A \textbf{balloon} is a small, thin , rubber bag that you blow air into so that it becomes larger and rounder or longer. Balloons are used as toys or decorations.
 \textit{
	\begin{itemize}
	\item She popped a balloon with her fork.
	\end{itemize}
}
\item countable noun \\
A \textbf{balloon} is a large, strong bag filled with gas or hot air, which can carry passengers in a container that hangs underneath it.
 \textit{
	\begin{itemize}
	\item They are to attempt to be the first to circle the Earth non-stop by balloon.
	\end{itemize}
}
\item verb \\
When something \textbf{balloons} , it increases rapidly in amount.
 \textit{
	\begin{itemize}
	\item Attendance has ballooned more than tenfold over the past 16 years.
	\item The budget deficit has ballooned to $25 billion.
	\end{itemize}
}
\end{enumerate}

\section*{characterize}
{\large \color{blue}  characterizes  characterizing  characterized  }
\subsection*{Explain}
\begin{enumerate}
\item verb \\
If something \textbf{is characterized}  \textbf{by} a particular feature or quality, that feature or quality is an obvious part of it.
 \textit{
	\begin{itemize}
	\item This election campaign has been characterized by violence.
	\item A bold use of colour characterizes the bedroom.
	\end{itemize}
}
\item verb \\
If you \textbf{characterize} someone or something \textbf{as} a particular thing, you describe them as that thing.
 \textit{
	\begin{itemize}
	\item Both companies have characterized the relationship as friendly.
	\item This play is characterized as a comedy.
	\end{itemize}
}
\end{enumerate}

\section*{beef}
{\large \color{blue}  beefs  beefing  beefed  }
\subsection*{Explain}
\begin{enumerate}
\item uncountable noun \\
\textbf{Beef} is the meat of a cow, bull, or ox.
 \textit{
	\begin{itemize}
	\item ...roast beef.
	\item ...beef stew.
	\item ...exports of beef and powdered milk.
	\end{itemize}
}
\item verb \\
If someone \textbf{beefs}  \textbf{about} something, they keep complaining about it.
 \textbf{Beef} is also a noun .
 \textit{
	\begin{itemize}
	\item Instead of beefing about what Mrs Martin has not done, her critics might take a look
at what she is trying to do.
	\item I really don't have a beef with Wayne.
	\end{itemize}
}
\end{enumerate}

\section*{compensate}
{\large \color{blue}  compensates  compensating  compensated  }
\subsection*{Explain}
\begin{enumerate}
\item verb \\
To \textbf{compensate} someone \textbf{for}  money or things that they have lost  means to pay them money or give them something to replace that money or those things.
 \textit{
	\begin{itemize}
	\item The official promise to compensate people for the price rise clearly hadn't been
worked out properly.
	\item To ease financial difficulties, farmers could be compensated for their loss of subsidies.
	\end{itemize}
}
\item verb \\
If you \textbf{compensate}  \textbf{for} a lack of something or \textbf{for} something you have done wrong , you do something to make the situation  better .
 \textit{
	\begin{itemize}
	\item The company agreed to keep up high levels of output in order to compensate for supplies
lost.
	\item She would then feel guilt for her anger and compensate by doing even more for the
children.
	\end{itemize}
}
\item verb \\
Something that \textbf{compensates for} something else balances it or reduces its effects.
 \textit{
	\begin{itemize}
	\item MPs say it is crucial that a system is found to compensate for inflation.
	\item The pluses more than compensated for the inconveniences involved in making the trip.
	\end{itemize}
}
\item verb \\
If you try to \textbf{compensate}  \textbf{for} something that is wrong or missing in your life , you try to do something that removes or reduces the harmful effects.
 \textit{
	\begin{itemize}
	\item People who feel inferior have to compensate by way of outward achievement.
	\item Nothing could ever compensate for the pain of being separated from her children.
	\end{itemize}
}
\end{enumerate}

\section*{briefcase}
{\large \color{blue}  briefcases  }
\subsection*{Explain}
\begin{enumerate}
\item countable noun \\
A \textbf{briefcase} is a case used for carrying documents in.
 \textit{
	\begin{itemize}
	\end{itemize}
}
\end{enumerate}

\section*{conceive}
{\large \color{blue}  conceives  conceiving  conceived  }
\subsection*{Explain}
\begin{enumerate}
\item verb \\
If you cannot \textbf{conceive}  \textbf{of} something, you cannot imagine it or believe it.
 \textit{
	\begin{itemize}
	\item I just can't even conceive of that quantity of money.
	\item He was immensely ambitious but unable to conceive of winning power for himself.
	\end{itemize}
}
\item verb \\
If you \textbf{conceive} something \textbf{as} a particular thing, you consider it to be that thing.
 \textit{
	\begin{itemize}
	\item The ancients conceived the Earth as afloat in water.
	\item We conceive of the family as being in a constant state of change.
	\item Elvis conceived of himself as a ballad singer.
	\end{itemize}
}
\item verb \\
If you \textbf{conceive} a plan or idea, you think of it and work out how it can be done.
 \textit{
	\begin{itemize}
	\item She had conceived the idea of a series of novels.
	\item He conceived of the first truly portable computer in 1968.
	\end{itemize}
}
\item verb \\
When a woman \textbf{conceives} , she becomes pregnant.
 \textit{
	\begin{itemize}
	\item Women, he says, should give up alcohol before they plan to conceive.
	\item About one in six couples has difficulty conceiving.
	\item A mother who already has non-identical twins is more likely to conceive another set
of twins.
	\end{itemize}
}
\end{enumerate}

\section*{cattle}
{\large \color{blue}  }
\subsection*{Explain}
\begin{enumerate}
\item plural noun \\
\textbf{Cattle} are cows and bulls .
 \textit{
	\begin{itemize}
	\item ...the finest herd of beef cattle for two hundred miles.
	\end{itemize}
}
\end{enumerate}

\section*{consist}
{\large \color{blue}  consists  consisting  consisted  }
\subsection*{Explain}
\begin{enumerate}
\item verb \\
Something that \textbf{consists of} particular things or people is formed from them.
 \textit{
	\begin{itemize}
	\item Breakfast consisted of porridge served with butter.
	\item Her crew consisted of children from Devon and Cornwall.
	\end{itemize}
}
\item verb \\
Something that \textbf{consists in} something else has that thing as its main or only part.
 \textit{
	\begin{itemize}
	\item His work consisted in advising companies on the siting of new factories.
	\item ...Baudelaire's idea that genius consists in the ability to summon up childhood.
	\end{itemize}
}
\end{enumerate}

\section*{climate}
{\large \color{blue}  climates  }
\subsection*{Explain}
\begin{enumerate}
\item variable noun \\
The \textbf{climate} of a place is the general weather conditions that are typical of it.
 \textit{
	\begin{itemize}
	\item ...the hot and humid climate of Cyprus.
	\end{itemize}
}
\item countable noun \\
You can use \textbf{climate} to refer to the general atmosphere or situation  somewhere .
 \textit{
	\begin{itemize}
	\item The economic climate remains uncertain.
	\item ...the existing climate of violence and intimidation.
	\item A major change of political climate is not in prospect.
	\end{itemize}
}
\end{enumerate}

\section*{criticize}
{\large \color{blue}  criticizes  criticizing  criticized  }
\subsection*{Explain}
\begin{enumerate}
\item verb \\
If you \textbf{criticize} someone or something, you express your disapproval of them by saying what you think is wrong with them.
 \textit{
	\begin{itemize}
	\item His mother had rarely criticized him or any of her children.
	\item The minister criticised the police for failing to come up with any leads.
	\item The regime has been harshly criticized for serious human rights violations.
	\end{itemize}
}
\end{enumerate}

\section*{coach}
{\large \color{blue}  coaches  coaching  coached  }
\subsection*{Explain}
\begin{enumerate}
\item countable noun \\
A \textbf{coach} is someone who trains a person or team of people in a particular sport.
 \textit{
	\begin{itemize}
	\item Tony Woodcock has joined German amateur team SC Brueck as coach.
	\end{itemize}
}
\item verb \\
When someone \textbf{coaches} a person or a team, they help them to become better at a particular sport.
 \textit{
	\begin{itemize}
	\item He coached the team to success in La Liga.
	\item I had coached the Alliance team for some time.
	\end{itemize}
}
\item countable noun \\
A \textbf{coach} is a person who is in charge of a sports team.
 \textit{
	\begin{itemize}
	\end{itemize}
}
\item countable noun \\
In baseball , a \textbf{coach} is a member of a team who stands near the first or third base, and gives signals to other members of the team who are on bases and are trying to score .
 \textit{
	\begin{itemize}
	\end{itemize}
}
\item countable noun \\
A \textbf{coach} is someone who gives people special  teaching in a particular subject, especially in order to prepare them for an examination.
 \textit{
	\begin{itemize}
	\item What you need is a drama coach.
	\end{itemize}
}
\item verb \\
If you \textbf{coach} someone, you give them special teaching in a particular subject, especially in order
to prepare them for an examination.
 \textit{
	\begin{itemize}
	\item He gently coached me in French.
	\end{itemize}
}
\item countable noun \\
A \textbf{coach} is a large, comfortable bus that carries passengers on long journeys .
 \textit{
	\begin{itemize}
	\item As we headed back to Calais, the coach was badly delayed by roadworks.
	\item I hate travelling by coach.
	\end{itemize}
}
\item countable noun \\
A \textbf{coach} is one of the separate sections of a train that carries passengers.
 \textit{
	\begin{itemize}
	\item The train was an elaborate affair of sixteen coaches.
	\end{itemize}
}
\item countable noun \\
A \textbf{coach} is an enclosed vehicle with four wheels which is pulled by horses, and in which people used to travel . Coaches are still used for ceremonial events in some countries, such as Britain.
 \textit{
	\begin{itemize}
	\end{itemize}
}
\end{enumerate}

\section*{deviate}
{\large \color{blue}  deviates  deviating  deviated  }
\subsection*{Explain}
\begin{enumerate}
\item verb \\
To \textbf{deviate}  \textbf{from} something means to start doing something different or not planned , especially in a way that causes problems for others.
 \textit{
	\begin{itemize}
	\item They stopped you as soon as you deviated from the script.
	\item He planned his schedule far in advance, and he didn't deviate from it.
	\item He was determined to become a doctor and never deviated from that ambition.
	\end{itemize}
}
\end{enumerate}

\section*{couch}
{\large \color{blue}  couches  couching  couched  }
\subsection*{Explain}
\begin{enumerate}
\item countable noun \\
A \textbf{couch} is a long, comfortable seat for two or three people.
 \textit{
	\begin{itemize}
	\end{itemize}
}
\item countable noun \\
A \textbf{couch} is a narrow bed which patients lie on while they are being examined or treated by a doctor.
 \textit{
	\begin{itemize}
	\end{itemize}
}
\item verb \\
If a statement  \textbf{is couched}  \textbf{in} a particular style of language, it is expressed in that style of language.
 \textit{
	\begin{itemize}
	\item The new centre-right government's radical objectives are often couched in moderate
terms.
	\item This time the proposal was couched as an ultimatum.
	\end{itemize}
}
\end{enumerate}

\section*{differ}
{\large \color{blue}  differs  differing  differed  }
\subsection*{Explain}
\begin{enumerate}
\item verb \\
If two or more things \textbf{differ} , they are unlike each other in some way.
 \textit{
	\begin{itemize}
	\item The story he told police differed from the one he told his mother.
	\item Management styles differ.
	\end{itemize}
}
\item verb \\
If people \textbf{differ} about something, they do not agree with each other about it.
 \textit{
	\begin{itemize}
	\item The two leaders had differed on the issue of sanctions.
	\item That is where we differ.
	\item Since his retirement, Crowe has differed with the President on several issues.
	\end{itemize}
}
\end{enumerate}

\section*{crystal}
{\large \color{blue}  crystals  }
\subsection*{Explain}
\begin{enumerate}
\item countable noun \\
A \textbf{crystal} is a small piece of a substance that has formed naturally into a regular symmetrical shape.
 \textit{
	\begin{itemize}
	\item ...salt crystals.
	\item ...ice crystals.
	\item ...a single crystal of silicon.
	\end{itemize}
}
\item variable noun \\
\textbf{Crystal} is a transparent rock that is used to make jewellery and ornaments.
 \textit{
	\begin{itemize}
	\item ...a strand of crystal beads.
	\end{itemize}
}
\item uncountable noun \\
\textbf{Crystal} is a high-quality glass, usually with patterns cut into its surface.
 \textit{
	\begin{itemize}
	\item Some of the finest drinking glasses are made from lead crystal.
	\item ...crystal glasses.
	\item ...an immense crystal chandelier.
	\end{itemize}
}
\item uncountable noun \\
Glasses and other containers made of crystal are referred to as \textbf{crystal} .
 \textit{
	\begin{itemize}
	\item Get out your best china and crystal.
	\end{itemize}
}
\end{enumerate}

\section*{dispose}
{\large \color{blue}  disposes  disposing  disposed  }
\subsection*{Explain}
\begin{enumerate}
\end{enumerate}

\section*{dairy}
{\large \color{blue}  dairies  }
\subsection*{Explain}
\begin{enumerate}
\item countable noun \\
A \textbf{dairy} is a shop or company that sells milk and food made from milk, such as butter, cream,
and cheese.
 \textit{
	\begin{itemize}
	\end{itemize}
}
\item countable noun \\
On a farm , the \textbf{dairy} is the building where milk is kept or where cream, butter, and cheese are made.
 \textit{
	\begin{itemize}
	\end{itemize}
}
\item adjective \\
\textbf{Dairy} is used to refer to foods such as butter and cheese that are made from milk.
 \textit{
	\begin{itemize}
	\item ...dairy produce.
	\item ...vitamins found in eggs, meat and dairy products.
	\end{itemize}
}
\item adjective \\
\textbf{Dairy} is used to refer to the use of cattle to produce milk rather than meat .
 \textit{
	\begin{itemize}
	\item ...a small vegetable and dairy farm.
	\item ...the feeding of dairy cows.
	\end{itemize}
}
\end{enumerate}

\section*{distinguish}
{\large \color{blue}  distinguishes  distinguishing  distinguished  }
\subsection*{Explain}
\begin{enumerate}
\item verb \\
If you can \textbf{distinguish} one thing \textbf{from} another or \textbf{distinguish}  \textbf{between} two things, you can see or understand how they are different .
 \textit{
	\begin{itemize}
	\item Could he distinguish right from wrong?
	\item Research suggests that babies learn to see by distinguishing between areas of light
and dark.
	\item It is necessary to distinguish the policies of two successive governments.
	\end{itemize}
}
\item verb \\
A feature or quality that \textbf{distinguishes} one thing \textbf{from} another causes the two things to be regarded as different, because only the first thing has the feature or quality.
 \textit{
	\begin{itemize}
	\item There is something about music that distinguishes it from all other art forms.
	\item The bird has no distinguishing features.
	\end{itemize}
}
\item verb \\
If you can \textbf{distinguish} something, you can see, hear , or taste it although it is very difficult to detect .
 \textit{
	\begin{itemize}
	\item There were cries, calls. He could distinguish voices.
	\end{itemize}
}
\item verb \\
If you \textbf{distinguish}  \textbf{yourself} , you do something that makes you famous or important .
 \textit{
	\begin{itemize}
	\item Over the next few years he distinguished himself as a leading constitutional scholar.
	\item They distinguished themselves at the Battle of Assaye.
	\end{itemize}
}
\end{enumerate}

\section*{finger}
{\large \color{blue}  fingers  fingering  fingered  }
\subsection*{Explain}
\begin{enumerate}
\item countable noun \\
Your \textbf{fingers} are the four long thin parts at the end of each hand.
 \textit{
	\begin{itemize}
	\item She suddenly held up a small, bony finger and pointed across the room.
	\item She ran her fingers through her hair.
	\item There was a ring on each of his fingers.
	\end{itemize}
}
\item countable noun \\
The \textbf{fingers} of a glove are the parts that a person's fingers fit into.
 \textit{
	\begin{itemize}
	\end{itemize}
}
\item countable noun \\
A \textbf{finger of} something such as smoke or land is an amount of it that is shaped rather like a finger.
 \textit{
	\begin{itemize}
	\item ...a thin finger of land that separates Pakistan from the former Soviet Union.
	\item Cover the base with a single layer of sponge fingers.
	\end{itemize}
}
\item verb \\
If you \textbf{finger} something, you touch or feel it with your fingers.
 \textit{
	\begin{itemize}
	\item He fingered the few coins in his pocket.
	\item Self-consciously she fingered the emeralds at her throat.
	\end{itemize}
}
\item verb \\
If you \textbf{finger} a person or organization , you tell someone, usually the police , that the person or organization has done something illegal or wrong .
 \textit{
	\begin{itemize}
	\item Police and prosecutors manipulated the eyewitnesses so they would finger Aldo.
	\item Think of all those instances where CCTV could have fingered the suspects in no time.
	\end{itemize}
}
\item countable noun \\
A \textbf{finger} of a strong  alcoholic  drink is an amount of it which, when it is in a glass, is the same size as the width of a person's finger.
 \textit{
	\begin{itemize}
	\item I poured two final fingers of bourbon into my glass.
	\end{itemize}
}
\item  \\
 to get your fingers burned \textit{
	\begin{itemize}
	\end{itemize}
}
\item  \\
 to cross your fingers \textit{
	\begin{itemize}
	\end{itemize}
}
\item  \\
 to lay a finger on someone \textit{
	\begin{itemize}
	\end{itemize}
}
\item  \\
 to lift a finger \textit{
	\begin{itemize}
	\end{itemize}
}
\item  \\
 a finger in every pie \textit{
	\begin{itemize}
	\end{itemize}
}
\item  \\
 point the finger at/point an accusing finger at \textit{
	\begin{itemize}
	\end{itemize}
}
\item  \\
 to point the finger of suspicion \textit{
	\begin{itemize}
	\end{itemize}
}
\item  \\
 pull/get one's finger out \textit{
	\begin{itemize}
	\end{itemize}
}
\item  \\
 put one's finger on sth \textit{
	\begin{itemize}
	\end{itemize}
}
\item  \\
 to slip through your fingers \textit{
	\begin{itemize}
	\end{itemize}
}
\end{enumerate}

\section*{drink}
{\large \color{blue}  drinks  drinking  drank  drunk  }
\subsection*{Explain}
\begin{enumerate}
\item verb \\
When you \textbf{drink} a liquid, you take it into your mouth and swallow it.
 \textit{
	\begin{itemize}
	\item He drank his cup of tea.
	\item I drink water and green tea, but not coffee.
	\item He drank thirstily from the pool under the rock.
	\end{itemize}
}
\item verb \\
To \textbf{drink} means to drink alcohol.
 \textit{
	\begin{itemize}
	\item He was smoking and drinking too much.
	\item Never accept a ride with people who have been drinking.
	\end{itemize}
}
\item countable noun \\
A \textbf{drink} is an amount of a liquid which you drink.
 \textit{
	\begin{itemize}
	\item I'll get you a drink of water.
	\end{itemize}
}
\item countable noun \\
A \textbf{drink} is an alcoholic drink.
 \textit{
	\begin{itemize}
	\item She felt like a drink after a hard day.
	\end{itemize}
}
\item uncountable noun \\
\textbf{Drink} is alcohol, such as beer , wine , or whisky .
 \textit{
	\begin{itemize}
	\item Too much drink is bad for your health.
	\end{itemize}
}
\item  \\
 drink yourself into a stupor \textit{
	\begin{itemize}
	\end{itemize}
}
\item  \\
 drink someone under the table \textit{
	\begin{itemize}
	\end{itemize}
}
\item  \\
 take to drink \textit{
	\begin{itemize}
	\end{itemize}
}
\item  \\
 drink to that \textit{
	\begin{itemize}
	\end{itemize}
}
\end{enumerate}

\section*{fur}
{\large \color{blue}  furs  furring  furred  }
\subsection*{Explain}
\begin{enumerate}
\item variable noun \\
\textbf{Fur} is the thick and usually soft hair that grows on the bodies of many mammals.
 \textit{
	\begin{itemize}
	\item This creature's fur is short, dense and silky.
	\end{itemize}
}
\item variable noun \\
\textbf{Fur} is the fur-covered skin of an animal that is used to make clothing or small carpets .
 \textit{
	\begin{itemize}
	\item She had on a black coat with a fur collar.
	\item ...the trading of furs from Canada.
	\end{itemize}
}
\item countable noun \\
A \textbf{fur} is a coat made from real or artificial fur, or a piece of fur worn round your neck .
 \textit{
	\begin{itemize}
	\item There were women in furs and men in comfortable overcoats.
	\end{itemize}
}
\item variable noun \\
\textbf{Fur} is an artificial fabric that looks like fur and is used, for example, to make clothing, soft toys , and seat covers.
 \textit{
	\begin{itemize}
	\end{itemize}
}
\item  \\
 fur fly \textit{
	\begin{itemize}
	\end{itemize}
}
\end{enumerate}

\section*{emphasize}
{\large \color{blue}  emphasizes  emphasizing  emphasized  }
\subsection*{Explain}
\begin{enumerate}
\item verb \\
To \textbf{emphasize} something means to indicate that it is particularly  important or true , or to draw  special  attention to it.
 \textit{
	\begin{itemize}
	\item It's been emphasized that no major policy changes can be expected.
	\item Discuss pollution with your child, emphasizing how nice a clean street, lawn, or
park looks.
	\end{itemize}
}
\end{enumerate}

\section*{gas}
{\large \color{blue}  gases  gasses  gassing  gassed  }
\subsection*{Explain}
\begin{enumerate}
\item uncountable noun \\
\textbf{Gas} is a substance like air that is neither liquid nor solid and burns  easily . It is used as a fuel for cooking and heating.
 \textit{
	\begin{itemize}
	\item Coal is actually cheaper than gas.
	\item ...a contract to develop oil and gas reserves.
	\end{itemize}
}
\item variable noun \\
A \textbf{gas} is any substance that is neither liquid nor solid, for example oxygen or hydrogen .
 \textit{
	\begin{itemize}
	\item Helium is a very light gas.
	\item ...a huge cloud of gas and dust from the volcanic eruption.
	\end{itemize}
}
\item variable noun \\
\textbf{Gas} is a poisonous gas that can be used as a weapon .
 \textit{
	\begin{itemize}
	\item ...mustard gas.
	\item The problem was that the exhaust gases contain many toxins.
	\end{itemize}
}
\item variable noun \\
\textbf{Gas} is a gas used for medical purposes, for example to make patients  feel less pain or go to sleep during an operation .
 \textit{
	\begin{itemize}
	\item ...an anaesthetic gas used by many dentists.
	\end{itemize}
}
\item uncountable noun \\
\textbf{Gas} is the fuel which is used to drive motor vehicles.
 \textit{
	\begin{itemize}
	\item ...a tank of gas.
	\item ...gas stations.
	\end{itemize}
}
\item verb \\
To \textbf{gas} a person or animal means to kill them by making them breathe poisonous gas.
 \textit{
	\begin{itemize}
	\item Hundreds of thousands of rabbits are to be gassed because they are destroying the
environment.
	\end{itemize}
}
\item singular noun \\
You can describe a situation or event as \textbf{a gas} when it is very lively , amusing , and enjoyable.
 \textit{
	\begin{itemize}
	\item It was really a gas to find someone I could talk with.
	\end{itemize}
}
\item  \\
 step on the gas \textit{
	\begin{itemize}
	\end{itemize}
}
\end{enumerate}

\section*{engage}
{\large \color{blue}  engages  engaging  engaged  }
\subsection*{Explain}
\begin{enumerate}
\item verb \\
If you \textbf{engage in} an activity, you do it or are actively involved with it.
 \textit{
	\begin{itemize}
	\item It is important for children to have time to engage in family activities.
	\item You can engage in croquet on the south lawn.
	\end{itemize}
}
\item verb \\
If something \textbf{engages} you or your attention or interest , it keeps you interested in it and thinking about it.
 \textit{
	\begin{itemize}
	\item They never learned skills to engage the attention of the others.
	\end{itemize}
}
\item verb \\
If you \textbf{engage} someone \textbf{in} conversation, you have a conversation with them.
 \textit{
	\begin{itemize}
	\item They tried to engage him in conversation.
	\item We want to engage recognized leaders in discussion.
	\end{itemize}
}
\item verb \\
If you \textbf{engage with} something or \textbf{with} a group of people, you get involved with that thing or group and feel that you are connected with it or have real  contact with it.
 \textit{
	\begin{itemize}
	\item She found it hard to engage with office life.
	\item I will keep blogging because it offers me a way to engage with readers.
	\end{itemize}
}
\item verb \\
If you \textbf{engage} someone to do a particular job , you appoint them to do it.
 \textit{
	\begin{itemize}
	\item We engaged the services of a recognised engineer.
	\item He had been able to engage some staff.
	\end{itemize}
}
\item verb \\
When a part of a machine or other mechanism \textbf{engages} or when you \textbf{engage} it, it moves into a position where it fits into something else.
 \textit{
	\begin{itemize}
	\item Press the lever until you hear the catch engage.
	\item ...a lesson in how to engage the four-wheel drive.
	\end{itemize}
}
\item verb \\
When a military force \textbf{engages} the enemy, it attacks them and starts a battle .
 \textit{
	\begin{itemize}
	\item It could engage the enemy beyond the range of hostile torpedoes.
	\end{itemize}
}
\end{enumerate}

\section*{glove}
{\large \color{blue}  gloves  }
\subsection*{Explain}
\begin{enumerate}
\item countable noun \\
\textbf{Gloves} are pieces of clothing which cover your hands and wrists and have individual sections for each finger. You wear gloves to keep your hands warm or dry or to protect them.
 \textit{
	\begin{itemize}
	\item He stuck his gloves in his pocket.
	\item ...a pair of white cotton gloves.
	\end{itemize}
}
\item  \\
 to fit like a glove \textit{
	\begin{itemize}
	\end{itemize}
}
\end{enumerate}

\section*{enquire}
{\large \color{blue}  }
\subsection*{Explain}
\begin{enumerate}
\end{enumerate}

\section*{hair}
{\large \color{blue}  hairs  }
\subsection*{Explain}
\begin{enumerate}
\item variable noun \\
Your \textbf{hair} is the fine  threads that grow in a mass on your head.
 \textit{
	\begin{itemize}
	\item I wash my hair every night.
	\item He has black hair.
	\item ...a girl with long blonde hair.
	\item I get some grey hairs but I pull them out.
	\end{itemize}
}
\item variable noun \\
\textbf{Hair} is the short , fine threads that grow on different parts of your body.
 \textit{
	\begin{itemize}
	\item The majority of men have hair on their chest.
	\item It tickled the hairs on the back of my neck.
	\end{itemize}
}
\item variable noun \\
\textbf{Hair} is the threads that cover the body of an animal such as a dog , or make up a horse's mane and tail .
 \textit{
	\begin{itemize}
	\item I am allergic to cat hair.
	\item ...dog hairs on the carpet.
	\end{itemize}
}
\item countable noun \\
\textbf{Hairs} are short, very fine threads that grow on some insects and plants.
 \textit{
	\begin{itemize}
	\item The stinging nettle has a square stem and little hairs.
	\end{itemize}
}
\item  \\
 to let your hair down \textit{
	\begin{itemize}
	\end{itemize}
}
\item  \\
 to make your hair stand on end \textit{
	\begin{itemize}
	\end{itemize}
}
\item  \\
 not a hair out of place \textit{
	\begin{itemize}
	\end{itemize}
}
\item  \\
 not turn a hair \textit{
	\begin{itemize}
	\end{itemize}
}
\item  \\
 split hairs \textit{
	\begin{itemize}
	\end{itemize}
}
\end{enumerate}

\section*{enroll}
{\large \color{blue}  }
\subsection*{Explain}
\begin{enumerate}
\item verb transitive \\
1.  2.  3.  4.  5.  \textit{
	\begin{itemize}
	\end{itemize}
}
\item verb intransitive \\
6.  \textit{
	\begin{itemize}
	\end{itemize}
}
\end{enumerate}

\section*{find}
{\large \color{blue}  finds  finding  found  }
\subsection*{Explain}
\begin{enumerate}
\item verb \\
If you \textbf{find} someone or something, you see them or learn where they are.
 \textit{
	\begin{itemize}
	\item The police also found a pistol.
	\item They looked at the map but couldn't find a trace of anywhere called Darrowby.
	\item I wonder if you could find me a deck of cards?
	\end{itemize}
}
\item verb \\
If you \textbf{find} something that you need or want , you succeed in achieving or obtaining it.
 \textit{
	\begin{itemize}
	\item Many people here cannot find work.
	\item So far they have not found a way to fight the virus.
	\item He has to apply for a permit and we have to find him a job.
	\item Does this mean that they haven't found a place for him?
	\end{itemize}
}
\item passive verb \\
If something \textbf{is found} in a particular place or thing, it exists in that place.
 \textit{
	\begin{itemize}
	\item Two thousand of France's 4,200 species of flowering plants are found in the park.
	\item Fibre is found in cereal foods, beans, fruit and vegetables.
	\end{itemize}
}
\item verb \\
If you \textbf{find} someone or something in a particular situation , they are in that situation when you see them or come into contact with them.
 \textit{
	\begin{itemize}
	\item They found her walking alone and depressed on the beach.
	\item She returned to her east London home to find her back door forced open.
	\item Thrushes are a protected species so you will not find them on any menu.
	\end{itemize}
}
\item verb \\
If you \textbf{find}  \textbf{yourself} doing something, you are doing it without deciding or intending to do it.
 \textit{
	\begin{itemize}
	\item It's not the first time that you've found yourself in this situation.
	\item I found myself having more fun than I had had in years.
	\item It all seemed so far away from here that he found himself quite unable to take it
in.
	\end{itemize}
}
\item verb \\
If a time or event \textbf{finds} you in a particular situation, you are in that situation at the time mentioned or when the event occurs.
 \textit{
	\begin{itemize}
	\item Daybreak found us on a cold, clammy ship.
	\item His lunch did not take long to arrive and found him poring over his notepad,
	\end{itemize}
}
\item verb \\
If you \textbf{find} that something is the case , you become aware of it or realize that it is the case.
 \textit{
	\begin{itemize}
	\item The two biologists found, to their surprise, that both groups of birds survived equally
well.
	\item At my age I would find it hard to get another job.
	\item We find her evidence to be based on a degree of oversensitivity.
	\item I've never found my diet a problem.
	\end{itemize}
}
\item verb \\
When a court or jury decides that a person on trial is guilty or innocent , you say that the person \textbf{has been found} guilty or not guilty.
 \textit{
	\begin{itemize}
	\item She was found guilty of manslaughter and put on probation for two years.
	\item When they found us guilty, I just went blank.
	\end{itemize}
}
\item verb \\
You can use \textbf{find} to express your reaction to someone or something.
 \textit{
	\begin{itemize}
	\item I find most of the young men of my own age so boring.
	\item We're sure you'll find it exciting!
	\item I find it ludicrous that nothing has been done to protect passengers from fire.
	\item But you'd find him a good worker if you showed him what to do.
	\end{itemize}
}
\item verb \\
If you \textbf{find} a feeling such as pleasure or comfort  \textbf{in} a particular thing or activity, you experience the feeling mentioned as a result of this thing or activity.
 \textit{
	\begin{itemize}
	\item How could anyone find pleasure in hunting and killing this beautiful creature?
	\item I was too tired and frightened to find comfort in that familiar promise.
	\end{itemize}
}
\item verb \\
If you \textbf{find} the time or money \textbf{to} do something, you succeed in making or obtaining enough time or money to do it.
 \textit{
	\begin{itemize}
	\item I was just finding more time to write music.
	\item My sister helped me find the money for a private operation.
	\end{itemize}
}
\item countable noun \\
If you describe someone or something that has been discovered as a \textbf{find} , you mean that they are valuable, interesting, good, or useful .
 \textit{
	\begin{itemize}
	\item Another of his lucky finds was a pair of candle-holders.
	\item His discovery was hailed as the botanical find of the century.
	\end{itemize}
}
\item  \\
 find one's way \textit{
	\begin{itemize}
	\end{itemize}
}
\item  \\
 finds its/their way \textit{
	\begin{itemize}
	\end{itemize}
}
\end{enumerate}

\section*{handbook}
{\large \color{blue}  handbooks  }
\subsection*{Explain}
\begin{enumerate}
\item countable noun \\
A \textbf{handbook} is a book that gives you advice and instructions about a particular subject, tool , or machine .
 \textit{
	\begin{itemize}
	\item ...handbooks on grammar.
	\end{itemize}
}
\end{enumerate}

\section*{fulfill}
{\large \color{blue}  fulˈfilled  fulˈfilling  fulˈfilled  fulˈfilling  }
\subsection*{Explain}
\begin{enumerate}
\item verb transitive \\
1.  2.  3.  \textit{
	\begin{itemize}
	\end{itemize}
}
\end{enumerate}

\section*{handkerchief}
{\large \color{blue}  handkerchiefs  }
\subsection*{Explain}
\begin{enumerate}
\item countable noun \\
A \textbf{handkerchief} is a small square piece of fabric which you use for blowing your nose.
 \textit{
	\begin{itemize}
	\end{itemize}
}
\end{enumerate}

\section*{generalize}
{\large \color{blue}  generalizes  generalizing  generalized  }
\subsection*{Explain}
\begin{enumerate}
\item verb \\
If you \textbf{generalize} , you say something that seems to be true in most situations or for most people, but that may not be completely true in all cases .
 \textit{
	\begin{itemize}
	\item 'In my day, children were a lot better behaved'.—'It's not true, you're generalizing'.
	\item It's hard to generalize about Cole Porter because he wrote so many great songs that
were so varied.
	\end{itemize}
}
\item verb \\
If you \textbf{generalize} something such as an idea , you apply it more widely than its original  context , as if it was true in many other situations.
 \textit{
	\begin{itemize}
	\item A child first labels the household pet cat as a 'cat' and then generalises this label
to other animals that look like it.
	\end{itemize}
}
\end{enumerate}

\section*{inlet}
{\large \color{blue}  inlets  }
\subsection*{Explain}
\begin{enumerate}
\item countable noun \\
An \textbf{inlet} is a narrow strip of water which goes from a sea or lake into the land .
 \textit{
	\begin{itemize}
	\item ...a sheltered inlet.
	\end{itemize}
}
\item countable noun \\
An \textbf{inlet} is a part of a machine through which a flow of liquid enters.
 \textit{
	\begin{itemize}
	\item ...a blocked water inlet.
	\end{itemize}
}
\end{enumerate}

\section*{glance}
{\large \color{blue}  glances  glancing  glanced  }
\subsection*{Explain}
\begin{enumerate}
\item verb \\
If you \textbf{glance}  \textbf{at} something or someone, you look at them very quickly and then look away again immediately .
 \textit{
	\begin{itemize}
	\item He glanced at his watch.
	\item I glanced back.
	\end{itemize}
}
\item verb \\
If you \textbf{glance through} or \textbf{at} a newspaper , report , or book , you spend a short time looking at it without reading it very carefully.
 \textit{
	\begin{itemize}
	\item I picked up the phone book and glanced through it.
	\item I never even glanced at the political page of a daily paper.
	\end{itemize}
}
\item countable noun \\
A \textbf{glance} is a quick look at someone or something.
 \textit{
	\begin{itemize}
	\item Trevor and I exchanged a glance.
	\item ...stealing a quick glance at her watch.
	\end{itemize}
}
\item  \\
 at a glance \textit{
	\begin{itemize}
	\end{itemize}
}
\item  \\
 at first glance \textit{
	\begin{itemize}
	\end{itemize}
}
\item  \\
 to steal a glance \textit{
	\begin{itemize}
	\end{itemize}
}
\end{enumerate}

\section*{jeans}
{\large \color{blue}  }
\subsection*{Explain}
\begin{enumerate}
\item plural noun \\
\textbf{Jeans} are casual trousers that are usually made of strong blue  cotton  cloth called denim.
 \textit{
	\begin{itemize}
	\end{itemize}
}
\end{enumerate}

\section*{hinder}
{\large \color{blue}  hinders  hindering  hindered  }
\subsection*{Explain}
\begin{enumerate}
\item verb \\
If something \textbf{hinders} you, it makes it more difficult for you to do something or make progress .
 \textit{
	\begin{itemize}
	\item Does the fact that your players are part-timers help or hinder you?
	\item Further investigation was hindered by the loss of all documentation on the case.
	\end{itemize}
}
\item verb \\
If something \textbf{hinders} your movement, it makes it difficult for you to move forward or move around.
 \textit{
	\begin{itemize}
	\item A thigh injury increasingly hindered her mobility.
	\item Landslides and bad weather are continuing to hinder the arrival of relief supplies
to the area.
	\end{itemize}
}
\end{enumerate}

\section*{industrialize}
{\large \color{blue}  industrializes  industrializing  industrialized  }
\subsection*{Explain}
\begin{enumerate}
\item verb \\
When a country \textbf{industrializes} or \textbf{is industrialized} , it develops a lot of industries.
 \textit{
	\begin{itemize}
	\item Energy consumption rises as countries industrialise.
	\item Stalin's methods had industrialized the Russian economy.
	\item Britain was the first nation to be industrialised.
	\end{itemize}
}
\end{enumerate}

\section*{lunch}
{\large \color{blue}  lunches  lunching  lunched  }
\subsection*{Explain}
\begin{enumerate}
\item variable noun \\
\textbf{Lunch} is the meal that you have in the middle of the day.
 \textit{
	\begin{itemize}
	\item Shall we meet somewhere for lunch?
	\item He did not enjoy business lunches.
	\item If anyone wants me, I'm at lunch with a client.
	\end{itemize}
}
\item verb \\
When you \textbf{lunch} , you have lunch, especially at a restaurant .
 \textit{
	\begin{itemize}
	\item Only the extremely rich could afford to lunch at the Mirabelle.
	\item Having not yet lunched, we went to the refreshment bar for ham sandwiches.
	\end{itemize}
}
\item  \\
 a free lunch \textit{
	\begin{itemize}
	\end{itemize}
}
\end{enumerate}

\section*{integrate}
{\large \color{blue}  integrates  integrating  integrated  }
\subsection*{Explain}
\begin{enumerate}
\item verb \\
If someone \textbf{integrates} into a social group, or \textbf{is integrated} into it, they behave in such a way that they become part of the group or are accepted into it.
 \textit{
	\begin{itemize}
	\item He didn't integrate successfully into the Italian way of life.
	\item Integrating the kids with the community, finding them a role, is essential.
	\item The way Swedes integrate immigrants is, she feels, 100% more advanced.
	\item If they want to integrate, that's fine with me.
	\end{itemize}
}
\item verb \\
When races \textbf{integrate} or when schools and organizations  \textbf{are integrated} , people who are black or belong to ethnic minorities can join  white people in their schools and organizations.
 \textit{
	\begin{itemize}
	\item Schools came to us because they wanted to integrate.
	\item Encouraging teacher transfer would not, by itself, integrate the teaching corps.
	\end{itemize}
}
\item verb \\
If you \textbf{integrate} one thing \textbf{with} another, or one thing \textbf{integrates}  \textbf{with} another, the two things become closely linked or form part of a whole idea or system. You can also  say that two things \textbf{integrate} .
 \textit{
	\begin{itemize}
	\item ...the problem of integrating the spoken passages with the musical numbers.
	\item Ann wanted the conservatory to integrate with the kitchen.
	\item Little attempt was made to integrate the parts into a coherent whole.
	\item Talks will now begin about integrating the activities of both companies.
	\end{itemize}
}
\end{enumerate}

\section*{manuscript}
{\large \color{blue}  manuscripts  }
\subsection*{Explain}
\begin{enumerate}
\item countable noun \\
A \textbf{manuscript} is a handwritten or typed document, especially a writer's first version of a book before it is published .
 \textit{
	\begin{itemize}
	\item He had seen a manuscript of the book.
	\item ...discovering an original manuscript of the song in Paris.
	\item I am grateful to him for letting me read his early chapters in manuscript.
	\end{itemize}
}
\item countable noun \\
A \textbf{manuscript} is an old document that was written by hand before printing was invented .
 \textit{
	\begin{itemize}
	\item ...early printed books and rare manuscripts.
	\end{itemize}
}
\end{enumerate}

\section*{interfere}
{\large \color{blue}  interferes  interfering  interfered  }
\subsection*{Explain}
\begin{enumerate}
\item verb \\
If you say that someone \textbf{interferes}  \textbf{in} a situation, you mean they get involved in it although it does not concern them and their involvement is not wanted .
 \textit{
	\begin{itemize}
	\item I wish everyone would stop interfering and just leave me alone.
	\item The U.N. cannot interfere in the internal affairs of any country.
	\end{itemize}
}
\item verb \\
Something that \textbf{interferes with} a situation, activity, or process has a damaging effect on it.
 \textit{
	\begin{itemize}
	\item Personal and family stresses will inevitably interfere with work at times.
	\item Alexander wasn't going to let a lack of space interfere with his plans.
	\item Smoking and drinking interfere with your body's ability to process oxygen.
	\end{itemize}
}
\end{enumerate}

\section*{mercury}
{\large \color{blue}  }
\subsection*{Explain}
\begin{enumerate}
\item uncountable noun \\
\textbf{Mercury} is a silver-coloured liquid metal that is used especially in thermometers and barometers.
 \textit{
	\begin{itemize}
	\end{itemize}
}
\end{enumerate}

\section*{intervene}
{\large \color{blue}  intervenes  intervening  intervened  }
\subsection*{Explain}
\begin{enumerate}
\item verb \\
If you \textbf{intervene}  \textbf{in} a situation , you become involved in it and try to change it.
 \textit{
	\begin{itemize}
	\item The situation calmed down when police intervened.
	\item The Government is doing nothing to intervene in the crisis.
	\end{itemize}
}
\item verb \\
If you \textbf{intervene} , you interrupt a conversation in order to add something to it.
 \textit{
	\begin{itemize}
	\item Hattie intervened and told me to stop it.
	\item 'I've told you he's not here,' Irena intervened.
	\end{itemize}
}
\item verb \\
If an event \textbf{intervenes} , it happens  suddenly in a way that stops , delays , or prevents something from happening .
 \textit{
	\begin{itemize}
	\item The South African mailboat arrived on Friday mornings unless bad weather intervened.
	\item I pray that death may not intervene to prevent our meeting with my darling children.
	\end{itemize}
}
\end{enumerate}

\section*{milk}
{\large \color{blue}  milks  milking  milked  }
\subsection*{Explain}
\begin{enumerate}
\item uncountable noun \\
\textbf{Milk} is the white liquid produced by cows, goats, and some other animals, which people drink and use to make butter, cheese, and yoghurt.
 \textit{
	\begin{itemize}
	\item He popped out to buy a pint of milk.
	\item ...basic foods such as meat, bread and milk.
	\item ...empty milk bottles.
	\end{itemize}
}
\item verb \\
If someone \textbf{milks} a cow or goat, they get milk from it, using either their hands or a machine .
 \textit{
	\begin{itemize}
	\item Farm-workers milked cows by hand.
	\end{itemize}
}
\item uncountable noun \\
\textbf{Milk} is the white liquid produced by women to feed their babies .
 \textit{
	\begin{itemize}
	\item Milk from the mother's breast is a perfect food for the human baby.
	\end{itemize}
}
\item variable noun \\
Liquid products for cleaning your skin or making it softer are sometimes  referred to as \textbf{milks} .
 \textit{
	\begin{itemize}
	\item ...sales of cleansing milks, creams and gels.
	\end{itemize}
}
\item verb \\
If you say that someone \textbf{milks} something, you mean that they get as much benefit or profit as they can from it, without caring about the effects this has on other people.
 \textit{
	\begin{itemize}
	\item A few people tried to milk the insurance companies.
	\item The callous couple milked money from a hospital charity to fund a lavish lifestyle.
	\end{itemize}
}
\item  \\
 milk and water \textit{
	\begin{itemize}
	\end{itemize}
}
\end{enumerate}

\section*{minimize}
{\large \color{blue}  minimizes  minimizing  minimized  }
\subsection*{Explain}
\begin{enumerate}
\item verb \\
If you \textbf{minimize} a risk , problem , or unpleasant  situation , you reduce it to the lowest possible level , or prevent it increasing beyond that level.
 \textit{
	\begin{itemize}
	\item Concerned people want to minimize the risk of developing cancer.
	\item Many of these problems can be minimised by sensible planning.
	\end{itemize}
}
\item verb \\
If you \textbf{minimize} something, you make it seem smaller or less significant than it really is.
 \textit{
	\begin{itemize}
	\item Some have minimized the importance of ideological factors.
	\item At his trial, he tried to minimize his behavior.
	\end{itemize}
}
\item verb \\
If you \textbf{minimize} a window on a computer screen , you make it very small, because you do not want to use it.
 \textit{
	\begin{itemize}
	\item Click the square icon again to minimize the window.
	\end{itemize}
}
\end{enumerate}

\section*{mobilize}
{\large \color{blue}  mobilizes  mobilizing  mobilized  }
\subsection*{Explain}
\begin{enumerate}
\item verb \\
If you \textbf{mobilize}  support or \textbf{mobilize} people to do something, you succeed in encouraging people to take  action , especially  political action. If people \textbf{mobilize} , they prepare to take action.
 \textit{
	\begin{itemize}
	\item The best hope is that we will mobilize international support and get down to action.
	\item The purpose of the journey is to mobilise public opinion on the controversial issue.
	\item Faced with crisis, people mobilized.
	\end{itemize}
}
\item verb \\
If you \textbf{mobilize} resources, you start to use them or make them available for use.
 \textit{
	\begin{itemize}
	\item If you could mobilize the resources, you could get it done.
	\end{itemize}
}
\item verb \\
If a country  \textbf{mobilizes} , or \textbf{mobilizes} its armed forces , or if its armed forces \textbf{mobilize} , they are given orders to prepare for a conflict .
 \textit{
	\begin{itemize}
	\item Sudan even threatened to mobilize in response to the ultimatums.
	\item India is now in a better position to mobilise its forces.
	\item It means that their whole army will mobilize.
	\end{itemize}
}
\end{enumerate}

\section*{palm}
{\large \color{blue}  palms  palming  palmed  }
\subsection*{Explain}
\begin{enumerate}
\item variable noun \\
A \textbf{palm} or a \textbf{palm tree} is a tree that grows in hot countries. It has long leaves growing at the top , and no branches.
 \textit{
	\begin{itemize}
	\end{itemize}
}
\item countable noun \\
The \textbf{palm}  \textbf{of} your hand is the inside part.
 \textit{
	\begin{itemize}
	\item Dornberg slapped the table with the palm of his hand.
	\item He wiped his sweaty palm.
	\end{itemize}
}
\item  \\
 in the palm of one's hand \textit{
	\begin{itemize}
	\end{itemize}
}
\end{enumerate}

\section*{multiply}
{\large \color{blue}  multiplies  multiplying  multiplied  }
\subsection*{Explain}
\begin{enumerate}
\item verb \\
When something \textbf{multiplies} or when you \textbf{multiply} it, it increases greatly in number or amount .
 \textit{
	\begin{itemize}
	\item Such disputes multiplied in the eighteenth and nineteenth centuries.
	\item The scale of change multiplies the difficulties and the risks involved.
	\end{itemize}
}
\item verb \\
When animals and insects  \textbf{multiply} , they increase in number by giving  birth to large numbers of young .
 \textit{
	\begin{itemize}
	\item These creatures can multiply quickly.
	\end{itemize}
}
\item verb \\
If you \textbf{multiply} one number by another, you add the first number to itself as many times as is indicated by the second number. For example , 2 multiplied by 3 is equal to 6.
 \textit{
	\begin{itemize}
	\item What do you get if you multiply six by nine?
	\item ...the remarkable ability to multiply huge numbers correctly without pen or paper.
	\end{itemize}
}
\end{enumerate}

\section*{pistol}
{\large \color{blue}  pistols  }
\subsection*{Explain}
\begin{enumerate}
\item countable noun \\
A \textbf{pistol} is a small gun .
 \textit{
	\begin{itemize}
	\end{itemize}
}
\end{enumerate}

\section*{originate}
{\large \color{blue}  originates  originating  originated  }
\subsection*{Explain}
\begin{enumerate}
\item verb \\
When something \textbf{originates} or when someone \textbf{originates} it, it begins to happen or exist .
 \textit{
	\begin{itemize}
	\item The alfalfa plant originated in North Africa.
	\item All carbohydrates originate from plants.
	\item I suppose no one has any idea who originated the story?
	\end{itemize}
}
\end{enumerate}

\section*{scent}
{\large \color{blue}  scents  scenting  scented  }
\subsection*{Explain}
\begin{enumerate}
\item countable noun \\
The \textbf{scent} of something is the pleasant smell that it has.
 \textit{
	\begin{itemize}
	\item Flowers are chosen for their scent as well as their look.
	\end{itemize}
}
\item verb \\
If something \textbf{scents} a place or thing, it makes it smell pleasant.
 \textit{
	\begin{itemize}
	\item Jasmine flowers scent the air.
	\item Scent your drawers and wardrobe with your favourite aromas.
	\end{itemize}
}
\item variable noun \\
\textbf{Scent} is a liquid which women put on their necks and wrists to make themselves smell nice .
 \textit{
	\begin{itemize}
	\item She dabbed herself with scent.
	\end{itemize}
}
\item variable noun \\
The \textbf{scent} of a person or animal is the smell that they leave and that other people sometimes  follow when looking for them.
 \textit{
	\begin{itemize}
	\item A police dog picked up the murderer's scent.
	\item Many kinds of insect find their mates by scent.
	\end{itemize}
}
\item verb \\
When an animal \textbf{scents} something, it becomes aware of it by smelling it.
 \textit{
	\begin{itemize}
	\item ...dogs which scent the hidden birds.
	\end{itemize}
}
\item verb \\
If you \textbf{scent} a situation , you feel that it is going to happen .
 \textit{
	\begin{itemize}
	\item Republicans from Pennsylvania and New York are scenting victory.
	\end{itemize}
}
\end{enumerate}

\section*{participate}
{\large \color{blue}  participates  participating  participated  }
\subsection*{Explain}
\begin{enumerate}
\item verb \\
If you \textbf{participate}  \textbf{in} an activity, you take part in it.
 \textit{
	\begin{itemize}
	\item They expected him to participate in the ceremony.
	\item Over half the population of this country participate in sport.
	\item ...special contracts at lower rates for participating corporations.
	\end{itemize}
}
\end{enumerate}

\section*{smell}
{\large \color{blue}  smells  smelling  smelled  smelt  }
\subsection*{Explain}
\begin{enumerate}
\item countable noun \\
The \textbf{smell} of something is a quality it has which you become  aware of when you breathe in through your nose .
 \textit{
	\begin{itemize}
	\item ...the smell of freshly baked bread.
	\item ...horrible smells.
	\item What is your favourite smell?
	\end{itemize}
}
\item uncountable noun \\
Your sense of \textbf{smell} is the ability that your nose has to detect things.
 \textit{
	\begin{itemize}
	\item ...people who lose their sense of smell.
	\end{itemize}
}
\item link verb \\
If something \textbf{smells} in a particular  way , it has a quality which you become aware of through your nose.
 \textit{
	\begin{itemize}
	\item The room smelled of lemons.
	\item It smells delicious.
	\item ...a crumbly black substance that smells like fresh soil.
	\end{itemize}
}
\item verb \\
If you say that something \textbf{smells} , you mean that it smells unpleasant.
 \textit{
	\begin{itemize}
	\item Ma threw that out. She said it smelled.
	\item Do my feet smell?
	\end{itemize}
}
\item verb \\
If you \textbf{smell} something, you become aware of it when you breathe in through your nose.
 \textit{
	\begin{itemize}
	\item As soon as we opened the front door we could smell the gas.
	\end{itemize}
}
\item verb \\
If you \textbf{smell} something, you put your nose near it and breathe in, so that you can  discover its smell.
 \textit{
	\begin{itemize}
	\item I took a fresh rose out of the vase on our table, and smelled it.
	\end{itemize}
}
\item verb \\
If you \textbf{smell} something, you feel that it is likely to happen or be true .
 \textit{
	\begin{itemize}
	\item He knew virtually nothing about music but he could smell a hit.
	\end{itemize}
}
\end{enumerate}

\section*{persist}
{\large \color{blue}  persists  persisting  persisted  }
\subsection*{Explain}
\begin{enumerate}
\item verb \\
If something undesirable  \textbf{persists} , it continues to exist.
 \textit{
	\begin{itemize}
	\item Contact your doctor if the cough persists.
	\item These problems persisted for much of the decade.
	\end{itemize}
}
\item verb \\
If you \textbf{persist}  \textbf{in} doing something, you continue to do it, even though it is difficult or other people are against it.
 \textit{
	\begin{itemize}
	\item Why does Britain persist in running down its defence forces?
	\item He urged the United States to persist with its efforts to bring about peace.
	\item 'You haven't answered me,' she persisted.
	\item When I set my mind to something, I persist.
	\end{itemize}
}
\end{enumerate}

\section*{sofa}
{\large \color{blue}  sofas  }
\subsection*{Explain}
\begin{enumerate}
\item countable noun \\
A \textbf{sofa} is a long, comfortable seat with a back and usually with arms, which two or three people can sit on.
 \textit{
	\begin{itemize}
	\end{itemize}
}
\end{enumerate}

\section*{plunge}
{\large \color{blue}  plunges  plunging  plunged  }
\subsection*{Explain}
\begin{enumerate}
\item verb \\
If something or someone \textbf{plunges} in a particular direction, especially into water, they fall , rush, or throw themselves in that direction.
 \textbf{Plunge} is also a noun .
 \textit{
	\begin{itemize}
	\item At least 50 people died when a bus plunged into a river.
	\item He ran down the steps to the pool terrace and plunged in.
	\item ...a plunge into cold water.
	\end{itemize}
}
\item verb \\
If you \textbf{plunge} an object \textbf{into} something, you push it quickly or violently into it.
 \textit{
	\begin{itemize}
	\item A soldier plunged a bayonet into his body.
	\item She plunged her face into a bowl of cold water.
	\item I plunged in my knife and fork.
	\end{itemize}
}
\item verb \\
If a person or thing \textbf{is plunged}  \textbf{into} a particular state or situation , or if they \textbf{plunge}  \textbf{into} it, they are suddenly in that state or situation.
 \textbf{Plunge} is also a noun.
 \textit{
	\begin{itemize}
	\item The government's political and economic reforms threaten to plunge the country into
chaos.
	\item 8,000 homes were plunged into darkness as electricity cables crashed down.
	\item Eddy finds himself plunged into a world of brutal violence.
	\item The economy is plunging into recession.
	\item That peace often looked like a brief truce before the next plunge into war.
	\end{itemize}
}
\item verb \\
If you \textbf{plunge into} an activity or \textbf{are plunged into} it, you suddenly get very involved in it.
 \textbf{Plunge} is also a noun.
 \textit{
	\begin{itemize}
	\item The two men plunged into discussion.
	\item The prince should be plunged into work.
	\item Take the opportunity to plunge yourself into your career.
	\item His sudden plunge into the field of international diplomacy is a major surprise.
	\end{itemize}
}
\item verb \\
If an amount or rate  \textbf{plunges} , it decreases quickly and suddenly.
 \textbf{Plunge} is also a noun.
 \textit{
	\begin{itemize}
	\item His weight began to plunge.
	\item The Pound plunged to a new low on the foreign exchange markets yesterday.
	\item Shares have plunged from £17 to £7.55.
	\item The bank's profits plunged by 87 per cent.
	\item Its net profits plunged 73% last year.
	\item Japan's banks are in trouble because of bad loans and the stock market plunge.
	\end{itemize}
}
\item  \\
 to take the plunge \textit{
	\begin{itemize}
	\end{itemize}
}
\end{enumerate}

\section*{steak}
{\large \color{blue}  steaks  }
\subsection*{Explain}
\begin{enumerate}
\item variable noun \\
A \textbf{steak} is a large flat  piece of beef without much fat on it. You cook it by grilling or frying it.
 \textit{
	\begin{itemize}
	\end{itemize}
}
\item uncountable noun \\
\textbf{Steak} is beef that is used for making stews. It is often cut into cubes to be sold .
 \textit{
	\begin{itemize}
	\item ...steak and kidney pie.
	\end{itemize}
}
\item countable noun \\
A fish \textbf{steak} is a large piece of fish that contains few bones .
 \textit{
	\begin{itemize}
	\item ...fresh salmon steaks.
	\end{itemize}
}
\end{enumerate}

\section*{preside}
{\large \color{blue}  presides  presiding  presided  }
\subsection*{Explain}
\begin{enumerate}
\item verb \\
If you \textbf{preside over} a meeting or an event, you are in charge.
 \textit{
	\begin{itemize}
	\item The PM presided over a meeting of his inner Cabinet.
	\item He presided at the trial of the Maguire Seven.
	\item The presiding officer ruled that the motion was out of order.
	\end{itemize}
}
\end{enumerate}

\section*{stocking}
{\large \color{blue}  stockings  }
\subsection*{Explain}
\begin{enumerate}
\item countable noun \\
\textbf{Stockings} are items of women's clothing which fit closely over their feet and legs. Stockings are usually made of nylon or silk and are held in place by suspenders .
 \textit{
	\begin{itemize}
	\item ...a pair of nylon stockings.
	\end{itemize}
}
\item countable noun \\
A \textbf{stocking} is the same as a Christmas stocking .
 \textit{
	\begin{itemize}
	\end{itemize}
}
\end{enumerate}

\section*{prevent}
{\large \color{blue}  prevents  preventing  prevented  }
\subsection*{Explain}
\begin{enumerate}
\item verb \\
To \textbf{prevent} something means to ensure that it does not happen .
 \textit{
	\begin{itemize}
	\item These methods prevent pregnancy.
	\item Further treatment will prevent cancer from developing.
	\item We recognized the possibility and took steps to prevent it happening.
	\end{itemize}
}
\item verb \\
To \textbf{prevent} someone \textbf{from} doing something means to make it impossible for them to do it.
 \textit{
	\begin{itemize}
	\item He said this would prevent companies from creating new jobs.
	\item Its nationals may be prevented from leaving the country.
	\item The police have been trying to prevent them carrying weapons.
	\end{itemize}
}
\end{enumerate}

\section*{stripe}
{\large \color{blue}  stripes  }
\subsection*{Explain}
\begin{enumerate}
\item countable noun \\
A \textbf{stripe} is a long line which is a different colour from the areas next to it.
 \textit{
	\begin{itemize}
	\item She wore a bright green jogging suit with a white stripe down the sides.
	\item The walls in the front bedroom are painted with broad, pale blue and white stripes.
	\end{itemize}
}
\item countable noun \\
In the armed forces or the police , \textbf{stripes} are V-shaped bands of material sewn onto a uniform to indicate the rank of corporal or sergeant . In the United  States , \textbf{stripes} can also  show the length of time that a person has served in an organization.
 \textit{
	\begin{itemize}
	\item ...a soldier with a corporal's stripes on his arms.
	\item He'd lost his stripes for slovenliness and cheek.
	\end{itemize}
}
\end{enumerate}

\section*{protect}
{\large \color{blue}  protects  protecting  protected  }
\subsection*{Explain}
\begin{enumerate}
\item verb \\
To \textbf{protect} someone or something means to prevent them from being harmed or damaged .
 \textit{
	\begin{itemize}
	\item So, what can women do to protect themselves from heart disease?
	\item A long thin wool coat and a purple headscarf protected her against the wind.
	\item The government is committed to protecting the interests of tenants.
	\end{itemize}
}
\item verb \\
If an insurance  policy  \textbf{protects} you \textbf{against} an event such as death , injury , fire , or theft , the insurance company  will give you or your family money if that event happens .
 \textit{
	\begin{itemize}
	\item Many manufacturers have policies to protect themselves against blackmailers.
	\end{itemize}
}
\end{enumerate}

\section*{sweater}
{\large \color{blue}  sweaters  }
\subsection*{Explain}
\begin{enumerate}
\item countable noun \\
A \textbf{sweater} is a warm knitted piece of clothing which covers the upper part of your body and your arms .
 \textit{
	\begin{itemize}
	\end{itemize}
}
\end{enumerate}

\section*{restrain}
{\large \color{blue}  restrains  restraining  restrained  }
\subsection*{Explain}
\begin{enumerate}
\item verb \\
If you \textbf{restrain} someone, you stop them from doing what they intended or wanted to do, usually by using your physical  strength .
 \textit{
	\begin{itemize}
	\item Wally gripped my arm, partly to restrain me and partly to reassure me.
	\item One onlooker had to be restrained by police.
	\end{itemize}
}
\item verb \\
If you \textbf{restrain} an emotion or you \textbf{restrain}  \textbf{yourself from} doing something, you prevent yourself from showing that emotion or doing what you wanted or intended to do.
 \textit{
	\begin{itemize}
	\item She was unable to restrain her desperate anger.
	\item Unable to restrain herself, she rose and went to the phone.
	\item She wanted to ask, 'Aren't you angry with him?' But she restrained herself from doing
so.
	\end{itemize}
}
\item verb \\
To \textbf{restrain} something that is growing or increasing  means to prevent it from getting too large.
 \textit{
	\begin{itemize}
	\item The radical 500-day plan was very clear on how it intended to try to restrain inflation.
	\item In the 1970s, the government tried to restrain corruption.
	\end{itemize}
}
\end{enumerate}

\section*{stare}
{\large \color{blue}  stares  staring  stared  }
\subsection*{Explain}
\begin{enumerate}
\item verb \\
If you \textbf{stare}  \textbf{at} someone or something, you look at them for a long time.
 \textbf{Stare} is also a noun .
 \textit{
	\begin{itemize}
	\item Tamara stared at him in disbelief, shaking her head.
	\item Ben continued to stare out the window.
	\item Mahoney tried not to stare.
	\item Hlasek gave him a long, cold stare.
	\end{itemize}
}
\item  \\
 stare sb in the face \textit{
	\begin{itemize}
	\end{itemize}
}
\end{enumerate}

\section*{torch}
{\large \color{blue}  torches  torching  torched  }
\subsection*{Explain}
\begin{enumerate}
\item countable noun \\
A \textbf{torch} is a small electric light which is powered by batteries and which you can carry in your hand .
 \textit{
	\begin{itemize}
	\end{itemize}
}
\item countable noun \\
A \textbf{torch} is a long stick with burning material at one end, used to provide light or to set things on fire.
 \textit{
	\begin{itemize}
	\item They lit a torch and set fire to the chapel's thatch.
	\item ...a torch-lit march for peace.
	\end{itemize}
}
\item countable noun \\
A \textbf{torch} is a device that produces a hot flame and is used for tasks such as cutting or joining pieces of metal.
 \textit{
	\begin{itemize}
	\item The gang worked for up to ten hours with acetylene torches to open the vault.
	\end{itemize}
}
\item verb \\
If someone \textbf{torches} a building or vehicle , they set fire to it deliberately.
 \textit{
	\begin{itemize}
	\item The rioters torched the local library.
	\item Cars and trucks have been torched, bottles and bricks thrown.
	\end{itemize}
}
\item  \\
 carry a torch for \textit{
	\begin{itemize}
	\end{itemize}
}
\item  \\
 carry the torch \textit{
	\begin{itemize}
	\end{itemize}
}
\end{enumerate}

\section*{suffer}
{\large \color{blue}  suffers  suffering  suffered  }
\subsection*{Explain}
\begin{enumerate}
\item verb \\
If you \textbf{suffer} pain, you feel it in your body or in your mind .
 \textit{
	\begin{itemize}
	\item Within a few days she had become seriously ill, suffering great pain and discomfort.
	\item Can you assure me that my father is not suffering?
	\end{itemize}
}
\item verb \\
If you \textbf{suffer from} an illness or from some other bad condition, you are badly  affected by it.
 \textit{
	\begin{itemize}
	\item He was eventually diagnosed as suffering from terminal cancer.
	\item I realized he was suffering from shock.
	\end{itemize}
}
\item verb \\
If you \textbf{suffer} something bad, you are in a situation in which something painful , harmful , or very unpleasant  happens to you.
 \textit{
	\begin{itemize}
	\item The peace process has suffered a serious blow now.
	\item Romania suffered another setback in its efforts to obtain financial support for its
reforms.
	\end{itemize}
}
\item verb \\
If you \textbf{suffer} , you are badly affected by an event or situation.
 \textit{
	\begin{itemize}
	\item There are few who have not suffered.
	\item It is obvious that poor people will suffer most from this change of heart.
	\end{itemize}
}
\item verb \\
If something \textbf{suffers} , it does not succeed because it has not been given enough attention or is in a bad situation.
 \textit{
	\begin{itemize}
	\item I'm not surprised that your studies are suffering.
	\item Without a major boost in tourism, the economy will suffer even further.
	\end{itemize}
}
\item  \\
 doesn't suffer fools gladly \textit{
	\begin{itemize}
	\end{itemize}
}
\end{enumerate}

\section*{towel}
{\large \color{blue}  towels  towelling  towelled  }
\subsection*{Explain}
\begin{enumerate}
\item countable noun \\
A \textbf{towel} is a piece of thick  soft cloth that you use to dry yourself.
 \textit{
	\begin{itemize}
	\item ...a bath towel.
	\end{itemize}
}
\item verb \\
If you \textbf{towel} something or \textbf{towel} it dry, you dry it with a towel.
 \textit{
	\begin{itemize}
	\item James came out of his bedroom, toweling his wet hair.
	\item I towelled myself dry.
	\item He stepped out of the shower and began towelling himself down.
	\end{itemize}
}
\item  \\
 to throw in the towel \textit{
	\begin{itemize}
	\end{itemize}
}
\end{enumerate}

\section*{swear}
{\large \color{blue}  swears  swearing  swore  sworn  }
\subsection*{Explain}
\begin{enumerate}
\item verb \\
If someone \textbf{swears} , they use language that is considered to be rude or offensive , usually because they are angry .
 \textit{
	\begin{itemize}
	\item It's wrong to swear and shout.
	\item They swore at them and ran off.
	\end{itemize}
}
\item verb \\
If you \textbf{swear}  \textbf{to} do something, you promise in a serious  way that you will do it.
 \textit{
	\begin{itemize}
	\item Alan swore that he would do everything in his power to help us.
	\item We have sworn to fight cruelty wherever we find it.
	\item The police are the only civil servants who have to swear allegiance to the Crown.
	\item I have sworn an oath to defend her.
	\end{itemize}
}
\item verb \\
If you say that you \textbf{swear} that something is true or that you can \textbf{swear} to it, you are saying very firmly that it is true.
 \textit{
	\begin{itemize}
	\item I swear I've told you all I know.
	\item I swear on all I hold dear that I had nothing to do with this.
	\item Behind them was a confusion of noise, perhaps even a shot, but he couldn't swear
to it.
	\end{itemize}
}
\item verb \\
If someone \textbf{is sworn}  \textbf{to}  secrecy or \textbf{is sworn}  \textbf{to}  silence , they promise another person that they will not reveal a secret .
 \textit{
	\begin{itemize}
	\item She was bursting to announce the news but was sworn to secrecy.
	\end{itemize}
}
\end{enumerate}

\section*{trolley}
{\large \color{blue}  trolleys  }
\subsection*{Explain}
\begin{enumerate}
\item countable noun \\
A \textbf{trolley} is an object with wheels that you use to transport heavy things such as shopping
or luggage.
 \textit{
	\begin{itemize}
	\item A porter relieved her of the three large cases she had been pushing on a trolley.
	\item ...supermarket trolleys.
	\end{itemize}
}
\item countable noun \\
A \textbf{trolley} is a small table on wheels which is used for serving drinks or food.
 \textit{
	\begin{itemize}
	\item The waiter had brought the sweet trolley.
	\end{itemize}
}
\item countable noun \\
A \textbf{trolley} is a bed on wheels for moving patients in hospital.
 \textit{
	\begin{itemize}
	\item She was left on a hospital trolley for 14 hours without even a glass of water.
	\end{itemize}
}
\item countable noun \\
A \textbf{trolley} or \textbf{trolley car} is an electric vehicle for carrying people which travels on rails in the streets of a town .
 \textit{
	\begin{itemize}
	\item He took a northbound trolley on State Street.
	\end{itemize}
}
\item  \\
 off one's trolley \textit{
	\begin{itemize}
	\end{itemize}
}
\end{enumerate}

\section*{wait}
{\large \color{blue}  waits  waiting  waited  }
\subsection*{Explain}
\begin{enumerate}
\item verb \\
When you \textbf{wait}  \textbf{for} something or someone, you spend some time doing very little, because you cannot act until that thing happens or that person arrives .
 \textit{
	\begin{itemize}
	\item I walk to a street corner and wait for the school bus.
	\item Stop waiting for things to happen. Make them happen.
	\item I waited to see how she responded.
	\item Angus got out of the car to wait.
	\item We will have to wait a week or so before we know whether the operation is a success.
	\item He told waiting journalists that he did not expect a referendum to be held for several
months.
	\end{itemize}
}
\item countable noun \\
A \textbf{wait} is a period of time in which you do very little, before something happens or before
you can do something.
 \textit{
	\begin{itemize}
	\item ...the four-hour wait for the organizers to declare the result.
	\end{itemize}
}
\item verb \\
If something \textbf{is waiting}  \textbf{for} you, it is ready for you to use, have, or do.
 \textit{
	\begin{itemize}
	\item There'll be a car waiting for you.
	\item When we came home we had a meal waiting for us.
	\item Ships with unfurled sails wait to take them aboard.
	\item Three-hundred railway wagons were waiting to be unloaded.
	\item He had a taxi waiting to take him to the train.
	\item The President had his plane waiting, 20 minutes' drive away.
	\end{itemize}
}
\item verb \\
If you say that something can \textbf{wait} , you mean that it is not important or urgent and so you will  deal with it or do it later .
 \textit{
	\begin{itemize}
	\item I want to talk to you, but it can wait.
	\item Any changes will have to wait until sponsors can be found.
	\end{itemize}
}
\item verb \\
You can use \textbf{wait} when you are trying to make someone feel  excited , or to encourage or threaten them.
 \textit{
	\begin{itemize}
	\item If you think this all sounds very exciting, just wait until you read the book.
	\item As soon as you get some food inside you, you'll feel more cheerful. Just you wait.
	\end{itemize}
}
\item verb \\
\textbf{Wait} is used in expressions such as \textbf{wait a minute} , \textbf{wait a second} , and \textbf{wait a moment} to interrupt someone when they are speaking , for example because you object to what they are saying or because you want them to repeat something.
 \textit{
	\begin{itemize}
	\item 'Wait a minute!' he broke in. 'This is not giving her a fair hearing!'
	\end{itemize}
}
\item verb \\
If an employee  \textbf{waits}  \textbf{on} you, for example in a restaurant or hotel , they take orders from you and bring you what you want.
 \textit{
	\begin{itemize}
	\item There were plenty of servants to wait on her.
	\item Each student is expected to wait at table for one week each semester.
	\end{itemize}
}
\item  \\
 can't wait/can hardly wait \textit{
	\begin{itemize}
	\end{itemize}
}
\item  \\
 wait for it \textit{
	\begin{itemize}
	\end{itemize}
}
\item  \\
 wait for it \textit{
	\begin{itemize}
	\end{itemize}
}
\item  \\
 wait and see \textit{
	\begin{itemize}
	\end{itemize}
}
\item  \\
 what are you waiting for \textit{
	\begin{itemize}
	\end{itemize}
}
\end{enumerate}

\section*{assignment}
{\large \color{blue}  assignments  }
\subsection*{Explain}
\begin{enumerate}
\item countable noun \\
An \textbf{assignment} is a task or piece of work that you are given to do, especially as part of your job or studies.
 \textit{
	\begin{itemize}
	\item The assessment for the course involves written assignments and practical tests.
	\end{itemize}
}
\item uncountable noun \\
You can refer to someone being given a particular task or job as their \textbf{assignment}  \textbf{to} the task or job.
 \textit{
	\begin{itemize}
	\item The Australian division was scheduled for assignment to Greece.
	\end{itemize}
}
\end{enumerate}

\section*{apologise}
{\large \color{blue}  }
\subsection*{Explain}
\begin{enumerate}
\end{enumerate}

\section*{bench}
{\large \color{blue}  benches  }
\subsection*{Explain}
\begin{enumerate}
\item countable noun \\
A \textbf{bench} is a long seat of wood or metal that two or more people can sit on.
 \textit{
	\begin{itemize}
	\item He sat down on a park bench.
	\end{itemize}
}
\item countable noun \\
A \textbf{bench} is a long, narrow table in a factory or laboratory .
 \textit{
	\begin{itemize}
	\item ...the laboratory bench.
	\end{itemize}
}
\item plural noun \\
In parliament , different groups sit on different \textbf{benches} . For example , the government sits on the government \textbf{benches} .
 \textit{
	\begin{itemize}
	\item ...the opposition benches.
	\end{itemize}
}
\item singular noun \\
In a court of law, \textbf{the bench} is the judge or magistrates .
 \textit{
	\begin{itemize}
	\item The chairman of the bench adjourned the case until October 27.
	\end{itemize}
}
\item singular noun \\
If someone serves on \textbf{the bench} , they work as a judge or magistrate.
 \textit{
	\begin{itemize}
	\item Allgood served on the bench for more than 50 years.
	\end{itemize}
}
\item singular noun \\
If a member of a sports  team is on \textbf{the bench} for a particular match , he or she waits at the side of the field until the manager  tells them to play.
 \textit{
	\begin{itemize}
	\item The coach didn't like my way of playing football.I stayed on the bench for a lot
of the games.
	\end{itemize}
}
\end{enumerate}

\section*{approve}
{\large \color{blue}  approves  approving  approved  }
\subsection*{Explain}
\begin{enumerate}
\item verb \\
If you \textbf{approve of} an action, event, or suggestion , you like it or are pleased about it.
 \textit{
	\begin{itemize}
	\item Not everyone approves of the festival.
	\item I approved of the proposal.
	\end{itemize}
}
\item verb \\
If you \textbf{approve of} someone or something, you like and admire them.
 \textit{
	\begin{itemize}
	\item You've never approved of Henry, have you?
	\item I didn't approve of his manner.
	\end{itemize}
}
\item verb \\
If someone in a position of authority  \textbf{approves} a plan or idea , they formally agree to it and say that it can happen .
 \textit{
	\begin{itemize}
	\item The Russian Parliament has approved a program of radical economic reforms.
	\item MPs approved the Bill by a majority of 97.
	\end{itemize}
}
\item verb \\
If a product or person \textbf{is approved} by an official  organization , they are declared to be of a good enough standard to be used or employed .
 \textit{
	\begin{itemize}
	\item We have three suppliers in all who are approved by the Organic Farm Food Association.
	\end{itemize}
}
\end{enumerate}

\section*{boundary}
{\large \color{blue}  boundaries  }
\subsection*{Explain}
\begin{enumerate}
\item countable noun \\
The \textbf{boundary}  \textbf{of} an area of land is an imaginary line that separates it from other areas.
 \textit{
	\begin{itemize}
	\item ...the Bow Brook which forms the western boundary of the wood.
	\item Drug traffickers operate across national boundaries.
	\end{itemize}
}
\item countable noun \\
The \textbf{boundaries}  \textbf{of} something such as a subject or activity are the limits that people think that it has.
 \textit{
	\begin{itemize}
	\item The boundaries between history and storytelling are always being blurred and muddled.
	\item ...extending the boundaries of press freedom.
	\end{itemize}
}
\end{enumerate}

\section*{attribute}
{\large \color{blue}  attributes  attributing  attributed  }
\subsection*{Explain}
\begin{enumerate}
\item verb \\
If you \textbf{attribute} something \textbf{to} an event or situation , you think that it was caused by that event or situation.
 \textit{
	\begin{itemize}
	\item The striker attributes the team's success to a positive ethos at the club.
	\end{itemize}
}
\item verb \\
If you \textbf{attribute} a particular quality or feature \textbf{to} someone or something, you think that they have got it.
 \textit{
	\begin{itemize}
	\item People were beginning to attribute superhuman qualities to him.
	\end{itemize}
}
\item verb \\
If a piece of writing , a work of art , or a remark  \textbf{is attributed}  \textbf{to} someone, people say that they wrote it, created it, or said it.
 \textit{
	\begin{itemize}
	\item This, and the remaining frescoes, are not attributed to Giotto.
	\item ...a Madonna and Child attributed to Pietro Lorenzetti.
	\end{itemize}
}
\item countable noun \\
An \textbf{attribute} is a quality or feature that someone or something has.
 \textit{
	\begin{itemize}
	\item Cruelty is a normal attribute of human behaviour.
	\item He has every attribute you could want and could play for any team.
	\end{itemize}
}
\end{enumerate}

\section*{belong}
{\large \color{blue}  belongs  belonging  belonged  }
\subsection*{Explain}
\begin{enumerate}
\item verb \\
If something \textbf{belongs to} you, you own it.
 \textit{
	\begin{itemize}
	\item The house had belonged to her family for three or four generations.
	\end{itemize}
}
\item verb \\
You say that something \textbf{belongs to} a particular person when you are guessing , discovering , or explaining that it was produced by or is part of that person.
 \textit{
	\begin{itemize}
	\item The handwriting belongs to a male.
	\item They established that the body belonged to a 15-year-old girl.
	\end{itemize}
}
\item verb \\
If someone \textbf{belongs to} a particular group, they are a member of that group.
 \textit{
	\begin{itemize}
	\item I used to belong to a youth club.
	\end{itemize}
}
\item verb \\
If something or someone \textbf{belongs in} or \textbf{to} a particular category , type, or group, they are of that category, type, or group.
 \textit{
	\begin{itemize}
	\item The judges could not decide which category it belonged in.
	\item I realized that he and I belonged to different worlds.
	\end{itemize}
}
\item verb \\
If something \textbf{belongs to} a particular time, it comes from that time.
 \textit{
	\begin{itemize}
	\item The pictures belong to an era when there was a preoccupation with high society.
	\end{itemize}
}
\item verb \\
If you say that something \textbf{belongs to} someone, you mean that person has the right to it.
 \textit{
	\begin{itemize}
	\item ...but the last word belonged to Rosanne.
	\end{itemize}
}
\item verb \\
If you say that a time \textbf{belongs to} a particular system or way of doing something, you mean that that time is or will be characterized by it.
 \textit{
	\begin{itemize}
	\item The future belongs to democracy.
	\end{itemize}
}
\item verb \\
If a baby or child  \textbf{belongs to} a particular adult , that adult is his or her parent or the person who is looking after him or her.
 \textit{
	\begin{itemize}
	\item He deduced that the two children belonged to the couple.
	\end{itemize}
}
\item verb \\
When lovers say that they \textbf{belong}  \textbf{together} , they are expressing their closeness or commitment to each other.
 \textit{
	\begin{itemize}
	\item I really think that we belong together.
	\item He belongs with me.
	\end{itemize}
}
\item verb \\
If a person or thing \textbf{belongs} in a particular place or situation , that is where they should be.
 \textit{
	\begin{itemize}
	\item You don't belong here.
	\item This piece really belongs in the concert hall.
	\item I'm so glad to see you back where you belong.
	\item They need to feel they belong.
	\end{itemize}
}
\end{enumerate}

\section*{bulletin}
{\large \color{blue}  bulletins  }
\subsection*{Explain}
\begin{enumerate}
\item countable noun \\
A \textbf{bulletin} is a short news report on the radio or television .
 \textit{
	\begin{itemize}
	\item ...the early morning news bulletin.
	\end{itemize}
}
\item countable noun \\
A \textbf{bulletin} is a short official announcement made publicly to inform people about an important matter.
 \textit{
	\begin{itemize}
	\item At 3.30 p.m. a bulletin was released announcing that the president was out of immediate
danger.
	\end{itemize}
}
\item countable noun \\
A \textbf{bulletin} is a regular  newspaper or leaflet that is produced by an organization or group such as a school or church .
 \textit{
	\begin{itemize}
	\end{itemize}
}
\end{enumerate}

\section*{boast}
{\large \color{blue}  boasts  boasting  boasted  }
\subsection*{Explain}
\begin{enumerate}
\item verb \\
If someone \textbf{boasts} about something that they have done or that they own, they talk about it very proudly, in a way that other people may find  irritating or offensive .
 \textbf{Boast} is also a noun .
 \textit{
	\begin{itemize}
	\item Witnesses said Furci boasted that he took part in killing them.
	\item Carol boasted about her costume.
	\item He's boasted of being involved in the arms theft.
	\item We remember our mother's stern instructions not to boast.
	\item It is the charity's proud boast that it has never yet turned anyone away.
	\item He was asked about earlier boasts of a quick victory.
	\end{itemize}
}
\item verb \\
If someone or something can \textbf{boast} a particular achievement or possession, they have achieved or possess that thing.
 \textit{
	\begin{itemize}
	\item The houses will boast the latest energy-saving technology.
	\item Frommen says his country boasts a healthy economy.
	\end{itemize}
}
\end{enumerate}

\section*{bus}
{\large \color{blue}  buses  busses  bussing  bussed  }
\subsection*{Explain}
\begin{enumerate}
\item countable noun \\
A \textbf{bus} is a large motor vehicle which carries passengers from one place to another. Buses
 drive along particular routes, and you have to pay to travel in them.
 \textit{
	\begin{itemize}
	\item He missed his last bus home.
	\item They had to travel everywhere by bus.
	\end{itemize}
}
\item verb \\
When someone is \textbf{bussed} to a particular place or when they \textbf{bus} there, they travel there on a bus.
 \textit{
	\begin{itemize}
	\item Students from around the country are being bussed in for the protest.
	\item To get our Colombian visas we bussed back to Medellin.
	\item Essential services were provided by Serbian workers bussed in from outside the province.
	\end{itemize}
}
\item verb \\
In some parts of the United  States , when children are \textbf{bused} to school, they are transported by bus to a school in a different area so that children of different races can be educated together.
 \textit{
	\begin{itemize}
	\item Many schools were in danger of closing because the children were bused out to other
areas.
	\end{itemize}
}
\end{enumerate}

\section*{cling}
{\large \color{blue}  clings  clinging  clung  }
\subsection*{Explain}
\begin{enumerate}
\item verb \\
If you \textbf{cling}  \textbf{to} someone or something, you hold onto them tightly.
 \textit{
	\begin{itemize}
	\item Another man was rescued as he clung to the riverbank.
	\item She had to cling onto the door handle until the pain passed.
	\item They hugged each other, clinging together under the lights.
	\end{itemize}
}
\item verb \\
If someone \textbf{clings}  \textbf{to} a position or a possession they have, they do everything they can to keep it even though this may be very difficult .
 \textit{
	\begin{itemize}
	\item He appears determined to cling to power.
	\item Another minister clung on with a majority of only 18.
	\item Japan's productivity has overtaken America in some industries, but elsewhere the
United States has clung on to its lead.
	\end{itemize}
}
\item verb \\
Clothes that \textbf{cling to} you stay  pressed against your body when you move.
 \textit{
	\begin{itemize}
	\item His sodden trousers were clinging to his shins.
	\end{itemize}
}
\item verb \\
Something that \textbf{is clinging to} something else is stuck on it or just attached to it.
 \textit{
	\begin{itemize}
	\item Her glass had bits of orange clinging to the rim.
	\end{itemize}
}
\item verb \\
If someone \textbf{clings to} a person they are fond of, they do not allow that person to be free or independent .
 \textit{
	\begin{itemize}
	\item I was terrified he would leave me, so I was clinging to him.
	\end{itemize}
}
\item verb \\
If you \textbf{cling to} an idea or way of behaving , you continue to believe in its value or importance , even though it may no longer be valid or useful .
 \textit{
	\begin{itemize}
	\item They know scholars reject their legend, but they still cling to their belief.
	\item They're clinging to the past.
	\end{itemize}
}
\end{enumerate}

\section*{cent}
{\large \color{blue}  cents  }
\subsection*{Explain}
\begin{enumerate}
\item countable noun \\
A \textbf{cent} is a small unit of money worth one-hundredth of some currencies, for example the
 dollar and the euro .
 \textit{
	\begin{itemize}
	\item A cup of rice which cost thirty cents a few weeks ago is now being sold for up to
one dollar.
	\item We haven't got a cent.
	\end{itemize}
}
\end{enumerate}

\section*{comply}
{\large \color{blue}  complies  complying  complied  }
\subsection*{Explain}
\begin{enumerate}
\item verb \\
If someone or something \textbf{complies}  \textbf{with} an order or set of rules, they are in accordance with what is required or expected .
 \textit{
	\begin{itemize}
	\item The commander said that the army would comply with the ceasefire.
	\item Some beaches had failed to comply with European directives on bathing water.
	\item There are calls for his resignation, but there is no sign yet that he will comply.
	\end{itemize}
}
\end{enumerate}

\section*{chemist}
{\large \color{blue}  chemists  }
\subsection*{Explain}
\begin{enumerate}
\item countable noun \\
A \textbf{chemist} or a \textbf{chemist's} is a shop where drugs and medicines are sold or given out, and where you can buy cosmetics and some household goods.
 \textit{
	\begin{itemize}
	\item There are many creams available from the chemist which should clear the infection.
	\item She went into a chemist's and bought some aspirin.
	\end{itemize}
}
\item countable noun \\
A \textbf{chemist} is someone who works in a chemist's shop and is qualified to prepare and sell medicines.
 \textit{
	\begin{itemize}
	\end{itemize}
}
\item countable noun \\
A \textbf{chemist} is a person who does research  connected with chemistry or who studies chemistry.
 \textit{
	\begin{itemize}
	\item She worked as a research chemist.
	\end{itemize}
}
\end{enumerate}

\section*{concentrate}
{\large \color{blue}  concentrates  concentrating  concentrated  }
\subsection*{Explain}
\begin{enumerate}
\item verb \\
If you \textbf{concentrate}  \textbf{on} something, or \textbf{concentrate} your mind  \textbf{on} it, you give all your attention to it.
 \textit{
	\begin{itemize}
	\item It was up to him to concentrate on his studies and make something of himself.
	\item Water companies should concentrate on reducing waste instead of building new reservoirs.
	\item At work you need to be able to concentrate.
	\item This helps you to be aware of time and concentrates your mind on the immediate task.
	\end{itemize}
}
\item verb \\
If something \textbf{is concentrated}  \textbf{in} an area, it is all there rather than being spread around.
 \textit{
	\begin{itemize}
	\item Italy's industrial districts are concentrated in its north-central and north-eastern
regions.
	\item Most development has been concentrated in and around cities.
	\end{itemize}
}
\item variable noun \\
\textbf{Concentrate} is a liquid or substance from which water has been removed in order to make it stronger , or to make it easier to store .
 \textit{
	\begin{itemize}
	\item ...orange juice made from concentrate.
	\end{itemize}
}
\item  \\
 concentrate someone's mind \textit{
	\begin{itemize}
	\end{itemize}
}
\end{enumerate}

\section*{chemistry}
{\large \color{blue}  }
\subsection*{Explain}
\begin{enumerate}
\item uncountable noun \\
\textbf{Chemistry} is the scientific  study of the structure of substances and of the way that they react with other substances.
 \textit{
	\begin{itemize}
	\end{itemize}
}
\item uncountable noun \\
The \textbf{chemistry} of an organism or a material is the chemical substances that make it up and the chemical reactions that go on inside it.
 \textit{
	\begin{itemize}
	\item We have literally altered the chemistry of our planet's atmosphere.
	\item Stress has a profound effect on our body chemistry.
	\end{itemize}
}
\item uncountable noun \\
If you say that there is \textbf{chemistry} between two people, you mean that it is obvious they are attracted to each other or like each other very much.
 \textit{
	\begin{itemize}
	\item ...the extraordinary chemistry between the two actors.
	\item Janis and I became friends but we were never close. The chemistry wasn't there.
	\end{itemize}
}
\end{enumerate}

\section*{congratulate}
{\large \color{blue}  congratulates  congratulating  congratulated  }
\subsection*{Explain}
\begin{enumerate}
\item verb \\
If you \textbf{congratulate} someone, you say something to show you are pleased that something nice has happened to them.
 \textit{
	\begin{itemize}
	\item She congratulated him on the birth of his son.
	\item I was delighted with my promotion. Everyone congratulated me.
	\end{itemize}
}
\item verb \\
If you \textbf{congratulate} someone, you praise them for something good that they have done.
 \textit{
	\begin{itemize}
	\item I really must congratulate the organisers for a well-run and enjoyable event.
	\item We specifically wanted to congratulate certain players.
	\end{itemize}
}
\item verb \\
If you \textbf{congratulate}  \textbf{yourself} , you are pleased about something that you have done or that has happened to you.
 \textit{
	\begin{itemize}
	\item Waterstone has every reason to congratulate himself.
	\item Journalists congratulated themselves on the role of the press in the investigations.
	\end{itemize}
}
\end{enumerate}

\section*{citizen}
{\large \color{blue}  citizens  }
\subsection*{Explain}
\begin{enumerate}
\item countable noun \\
Someone who is a \textbf{citizen} of a particular country is legally accepted as belonging to that country.
 \textit{
	\begin{itemize}
	\item ...American citizens.
	\item The life of ordinary citizens began to change.
	\end{itemize}
}
\item countable noun \\
The \textbf{citizens}  \textbf{of} a town or city are the people who live there.
 \textit{
	\begin{itemize}
	\item ...the citizens of Buenos Aires.
	\end{itemize}
}
\item adjective \\
You describe someone as a \textbf{citizen}  journalist or a \textbf{citizen}  scientist , for example , when they are an ordinary person with no special  training who does something that is usually done by professionals . 
 \textit{
	\begin{itemize}
	\item Several reports are coming from citizen journalists in the area.
	\end{itemize}
}
\end{enumerate}

\section*{contribute}
{\large \color{blue}  contributes  contributing  contributed  }
\subsection*{Explain}
\begin{enumerate}
\item verb \\
If you \textbf{contribute}  \textbf{to} something, you say or do things to help to make it successful .
 \textit{
	\begin{itemize}
	\item The three sons also contribute to the family business.
	\item I believe that each of us can contribute to the future of the world.
	\item He believes he has something to contribute to a discussion concerning the uprising.
	\end{itemize}
}
\item verb \\
To \textbf{contribute} money or resources  \textbf{to} something means to give money or resources to help pay for something or to help achieve a particular purpose.
 \textit{
	\begin{itemize}
	\item The U.S. is contributing $4 billion in loans, credits and grants.
	\item They say they would like to contribute more to charity, but money is tight this year.
	\item NATO officials agreed to contribute troops and equipment to such an operation if
the U.N. Security Council asked for it.
	\end{itemize}
}
\item verb \\
If something \textbf{contributes}  \textbf{to} an event or situation , it is one of the causes of it.
 \textit{
	\begin{itemize}
	\item The report says design faults in both the vessels contributed to the tragedy.
	\item Stress, both human and mechanical, may also be a contributing factor.
	\end{itemize}
}
\item verb \\
If you \textbf{contribute}  \textbf{to} a magazine , newspaper , or book, you write things that are published in it.
 \textit{
	\begin{itemize}
	\item I was asked to contribute to a newspaper article making predictions for the new year.
	\item He is a contributing editor for Vanity Fair magazine.
	\end{itemize}
}
\end{enumerate}

\section*{classification}
{\large \color{blue}  classifications  }
\subsection*{Explain}
\begin{enumerate}
\item countable noun \\
A \textbf{classification} is a division or category in a system which divides things into groups or types.
 \textit{
	\begin{itemize}
	\item Its tariffs cater for four basic classifications of customer.
	\end{itemize}
}
\end{enumerate}

\section*{convince}
{\large \color{blue}  convinces  convincing  convinced  }
\subsection*{Explain}
\begin{enumerate}
\item verb \\
If someone or something \textbf{convinces} you \textbf{of} something, they make you believe that it is true or that it exists .
 \textit{
	\begin{itemize}
	\item I soon convinced the jury of my innocence.
	\item It is difficult to convince the public of the need for change.
	\end{itemize}
}
\item verb \\
If someone or something \textbf{convinces} you \textbf{to} do something, they persuade you to do it.
 \textit{
	\begin{itemize}
	\item In January, he convinced her to join him in the Pyrenees.
	\end{itemize}
}
\end{enumerate}

\section*{claw}
{\large \color{blue}  claws  clawing  clawed  }
\subsection*{Explain}
\begin{enumerate}
\item countable noun \\
The \textbf{claws} of a bird or animal are the thin , hard , curved nails at the end of its feet .
 \textit{
	\begin{itemize}
	\item The cat tried to cling to the edge by its claws.
	\end{itemize}
}
\item countable noun \\
The \textbf{claws} of a lobster , crab, or scorpion are the two pointed parts at the end of its legs which are used for holding things.
 \textit{
	\begin{itemize}
	\end{itemize}
}
\item verb \\
If an animal \textbf{claws}  \textbf{at} something, it scratches or damages it with its claws.
 \textit{
	\begin{itemize}
	\item The wolf clawed at the tree and howled the whole night.
	\end{itemize}
}
\item verb \\
To \textbf{claw}  \textbf{at} something means to try very hard to get  hold of it.
 \textit{
	\begin{itemize}
	\item His fingers clawed at Blake's wrist.
	\item He stumbled, clawed wildly at the air and fell backwards into the water.
	\end{itemize}
}
\item verb \\
If you \textbf{claw} your \textbf{way}  somewhere , you move there with great  difficulty , trying desperately to find things to hold on to.
 \textit{
	\begin{itemize}
	\item Some did manage to claw their way up iron ladders to the safety of the upper deck.
	\end{itemize}
}
\item verb \\
If someone \textbf{claws} their \textbf{way} to a successful position, they achieve it with great determination in spite of many difficulties.
 \textit{
	\begin{itemize}
	\item Gino clawed his way out of underworld obscurity to become a millionaire hotelier.
	\end{itemize}
}
\item plural noun \\
If someone gets their \textbf{claws} into another person, they start doing or saying things, especially  unpleasant things, which affect that person.
 \textit{
	\begin{itemize}
	\item She should take her claws out of Tom and let him get on with his life.
	\end{itemize}
}
\end{enumerate}

\section*{cooperate}
{\large \color{blue}  }
\subsection*{Explain}
\begin{enumerate}
\item verb \\
1.  2.  3.  \textit{
	\begin{itemize}
	\end{itemize}
}
\end{enumerate}

\section*{company}
{\large \color{blue}  companies  }
\subsection*{Explain}
\begin{enumerate}
\item countable noun \\
A \textbf{company} is a business organization that makes money by selling goods or services.
 \textit{
	\begin{itemize}
	\item Sheila found some work as a secretary in an insurance company.
	\item ...the Ford Motor Company.
	\end{itemize}
}
\item countable noun \\
A \textbf{company} is a group of opera  singers , dancers, or actors who work together.
 \textit{
	\begin{itemize}
	\item ...the Phoenix Dance Company.
	\end{itemize}
}
\item countable noun \\
A \textbf{company} is a group of soldiers that is usually part of a battalion or regiment , and that is divided into two or more platoons.
 \textit{
	\begin{itemize}
	\item The division will consist of two tank companies and one infantry company.
	\item C Company's sentries were just ahead.
	\end{itemize}
}
\item uncountable noun \\
\textbf{Company} is having another person or other people with you, usually when this is pleasant or stops you feeling  lonely .
 \textit{
	\begin{itemize}
	\item 'I won't stay long.'—'No, please. I need the company'.
	\item Ross enjoyed the company of his colleagues.
	\item She would be grateful for their company on the drive back.
	\item I'm not in the mood for company.
	\end{itemize}
}
\item  \\
 and company \textit{
	\begin{itemize}
	\end{itemize}
}
\item  \\
 be in good company \textit{
	\begin{itemize}
	\end{itemize}
}
\item  \\
 have company \textit{
	\begin{itemize}
	\end{itemize}
}
\item  \\
 in company \textit{
	\begin{itemize}
	\end{itemize}
}
\item  \\
 in company with sb \textit{
	\begin{itemize}
	\end{itemize}
}
\item  \\
 to keep someone company \textit{
	\begin{itemize}
	\end{itemize}
}
\item  \\
 keep company \textit{
	\begin{itemize}
	\end{itemize}
}
\item  \\
 to part company \textit{
	\begin{itemize}
	\end{itemize}
}
\item  \\
 to part company \textit{
	\begin{itemize}
	\end{itemize}
}
\item  \\
 to part company \textit{
	\begin{itemize}
	\end{itemize}
}
\item  \\
 present company excepted \textit{
	\begin{itemize}
	\end{itemize}
}
\end{enumerate}

\section*{cope}
{\large \color{blue}  copes  coping  coped  }
\subsection*{Explain}
\begin{enumerate}
\item verb \\
If you \textbf{cope}  \textbf{with} a problem or task , you deal with it successfully.
 \textit{
	\begin{itemize}
	\item It was amazing how my mother coped with bringing up eight children.
	\item The problems were an annoyance, but we managed to cope.
	\end{itemize}
}
\item verb \\
If you have to \textbf{cope with} an unpleasant situation, you have to accept it or bear it.
 \textit{
	\begin{itemize}
	\item Never before has the industry had to cope with war and recession at the same time.
	\item She has had to cope with losing all her previous status and money.
	\end{itemize}
}
\item verb \\
If a machine or a system can \textbf{cope}  \textbf{with} something, it is large enough or complex enough to deal with it satisfactorily.
 \textit{
	\begin{itemize}
	\item New blades have been designed to cope with the effects of dead insects.
	\item The hospitals do not have enough money to cope with the numbers of patients.
	\item The banks were swamped by compensation claims and were unable to cope .
	\end{itemize}
}
\item countable noun \\
A \textbf{cope} is a long sleeveless piece of clothing worn by some Christian priests on special  occasions .
 \textit{
	\begin{itemize}
	\end{itemize}
}
\end{enumerate}

\section*{cure}
{\large \color{blue}  cures  curing  cured  }
\subsection*{Explain}
\begin{enumerate}
\item verb \\
If doctors or medical treatments \textbf{cure} an illness or injury , they cause it to end or disappear .
 \textit{
	\begin{itemize}
	\item An operation finally cured his shin injury.
	\item Her cancer can only be controlled, not cured.
	\end{itemize}
}
\item verb \\
If doctors or medical treatments \textbf{cure} a person, they make the person well again after an illness or injury.
 \textit{
	\begin{itemize}
	\item MDT is an effective treatment and could cure all the leprosy sufferers worldwide.
	\item Almost overnight I was cured.
	\item Now doctors believe they have cured him of the disease.
	\end{itemize}
}
\item countable noun \\
A \textbf{cure}  \textbf{for} an illness is a medicine or other treatment that cures the illness.
 \textit{
	\begin{itemize}
	\item There is still no cure for a cold.
	\item Atkinson has been told rest is the only cure for his ankle injury.
	\end{itemize}
}
\item verb \\
If someone or something \textbf{cures} a problem, they bring it to an end.
 \textit{
	\begin{itemize}
	\item Private firms are willing to make large-scale investments to help cure Russia's economic
troubles.
	\item We need to cure our environmental problems.
	\end{itemize}
}
\item countable noun \\
A \textbf{cure}  \textbf{for} a problem is something that will bring it to an end.
 \textit{
	\begin{itemize}
	\item Punishment can never be an effective cure for acute social problems.
	\item The magic cure for inflation does not exist.
	\end{itemize}
}
\item verb \\
If an action or event \textbf{cures} someone \textbf{of} a habit or an attitude , it makes them stop having it.
 \textit{
	\begin{itemize}
	\item The experience was a detestable ordeal, and it cured him of any ambitions to direct
again.
	\item He went to a clinic to cure his drinking and overeating.
	\end{itemize}
}
\item verb \\
When food, tobacco , or animal skin  \textbf{is cured} , it is dried , smoked, or salted so that it will last for a long time.
 \textit{
	\begin{itemize}
	\item Legs of pork were cured and smoked over the fire.
	\item ...sliced cured ham.
	\end{itemize}
}
\end{enumerate}

\section*{division}
{\large \color{blue}  divisions  }
\subsection*{Explain}
\begin{enumerate}
\item uncountable noun \\
The \textbf{division}  \textbf{of} a large unit \textbf{into} two or more distinct parts is the act of separating it into these parts.
 \textit{
	\begin{itemize}
	\item ...Czechoslovakia's division into the Czech Republic and Slovakia.
	\end{itemize}
}
\item uncountable noun \\
The \textbf{division}  \textbf{of} something among people or things is its separation into parts which are distributed among the people or things.
 \textit{
	\begin{itemize}
	\item The current division of labor between workers and management will alter.
	\end{itemize}
}
\item uncountable noun \\
\textbf{Division} is the arithmetical process of dividing one number into another number.
 \textit{
	\begin{itemize}
	\item I taught my daughter how to do division at the age of six.
	\end{itemize}
}
\item variable noun \\
A \textbf{division} is a significant  distinction or argument between two groups, which causes the two groups to be considered as very different and separate.
 \textit{
	\begin{itemize}
	\item The division between the prosperous west and the impoverished east remains.
	\end{itemize}
}
\item countable noun \\
In a large organization, a \textbf{division} is a group of departments whose work is done in the same place or is connected with similar tasks .
 \textit{
	\begin{itemize}
	\item ...the bank's Latin American division.
	\item ...the sales division.
	\end{itemize}
}
\item countable noun \\
A \textbf{division} is a group of military units which fight as a single unit.
 \textit{
	\begin{itemize}
	\item Several armoured divisions are being moved from Germany.
	\end{itemize}
}
\item countable noun \\
In the British Parliament, a \textbf{division} is a vote where the Members of Parliament go into separate rooms in order to record their vote.
 \textit{
	\begin{itemize}
	\end{itemize}
}
\item countable noun \\
In some sports, such as football , baseball , and basketball , a \textbf{division} is one of the groups of teams which make up a league . The teams in each division are considered to be approximately the same standard , and they all play against each other during the season .
 \textit{
	\begin{itemize}
	\item Villa had just been relegated from the First Division.
	\item ...the Scottish Premier Division leaders, Dundee United.
	\end{itemize}
}
\end{enumerate}

\section*{depend}
{\large \color{blue}  depends  depending  depended  }
\subsection*{Explain}
\begin{enumerate}
\item verb \\
If you say that one thing \textbf{depends}  \textbf{on} another, you mean that the first thing will be affected or determined by the second .
 \textit{
	\begin{itemize}
	\item The cooking time needed depends on the size of the potato.
	\item What happened later would depend on his talk with De Solina.
	\item How much it costs depends upon how much you buy.
	\end{itemize}
}
\item verb \\
If you \textbf{depend}  \textbf{on} someone or something, you need them in order to be able to survive physically, financially, or emotionally.
 \textit{
	\begin{itemize}
	\item They may hate what he does but their survival depends on him.
	\item He depended on his writing for his income.
	\item They had grown to depend on each other.
	\item Choosing the right account depends on working out your likely average balance.
	\end{itemize}
}
\item verb \\
If you can \textbf{depend}  \textbf{on} a person, organization , or law , you know that they will support you or help you when you need them.
 \textit{
	\begin{itemize}
	\item 'You can depend on me,' Cross assured him.
	\end{itemize}
}
\item verb \\
You use \textbf{depend} in expressions such as \textbf{it depends} to indicate that you cannot give a clear  answer to a question because the answer will be affected or determined by other factors .
 \textit{
	\begin{itemize}
	\item 'But how long can you stay in the house?'—'I don't know. It depends.'.
	\item It all depends on your definition of punk, doesn't it?
	\end{itemize}
}
\item  \\
 depending on sth \textit{
	\begin{itemize}
	\end{itemize}
}
\end{enumerate}

\section*{document}
{\large \color{blue}  documents  documenting  documented  }
\subsection*{Explain}
\begin{enumerate}
\item countable noun \\
A \textbf{document} is one or more official pieces of paper with writing on them.
 \textit{
	\begin{itemize}
	\item The foreign ministers of the two countries signed the documents today.
	\item ...a policy document for the Labour Party conference.
	\item The police officer wanted to see all our documents.
	\end{itemize}
}
\item countable noun \\
A \textbf{document} is a piece of text or graphics, for example a letter , that is stored as a file on a computer and that you can access in order to read it or change it.
 \textit{
	\begin{itemize}
	\item When you are finished typing, remember to save your document.
	\end{itemize}
}
\item verb \\
If you \textbf{document} something, you make a detailed record of it in writing or on film or tape .
 \textit{
	\begin{itemize}
	\item He wrote a book documenting his prison experiences.
	\item The effects of smoking have been well documented.
	\end{itemize}
}
\end{enumerate}

\section*{devote}
{\large \color{blue}  devotes  devoting  devoted  }
\subsection*{Explain}
\begin{enumerate}
\item verb \\
If you \textbf{devote} yourself, your time, or your energy  \textbf{to} something, you spend all or most of your time or energy on it.
 \textit{
	\begin{itemize}
	\item He decided to devote the rest of his life to scientific investigation.
	\item Considerable resources have been devoted to proving him a liar.
	\item She gladly gave up her part-time job to devote herself entirely to her art.
	\end{itemize}
}
\item verb \\
If you \textbf{devote} a particular proportion of a piece of writing or a speech  \textbf{to} a particular subject, you deal with the subject in that amount of space or time.
 \textit{
	\begin{itemize}
	\item He devoted a major section of his massive report to an analysis of U.S. aircraft
design.
	\item In her 900-page memoirs, only four pages are devoted to the arts.
	\end{itemize}
}
\end{enumerate}

\section*{father}
{\large \color{blue}  fathers  fathering  fathered  }
\subsection*{Explain}
\begin{enumerate}
\item countable noun \\
Your \textbf{father} is your male parent. You can also  call someone your \textbf{father} if he brings you up as if he was this man. You can call your father 'Father'.
 \textit{
	\begin{itemize}
	\item His father was a painter.
	\item He would be a good father to my children.
	\item ...Mr Stoneman, a father of five.
	\end{itemize}
}
\item verb \\
When a man \textbf{fathers} a child, he makes a woman pregnant and their child is born.
 \textit{
	\begin{itemize}
	\item She claims Mark fathered her child.
	\item He fathered four children by different women.
	\end{itemize}
}
\item countable noun \\
The man who invented or started something is sometimes  referred to as the \textbf{father of} that thing.
 \textit{
	\begin{itemize}
	\item ...Max Dupain, regarded as the father of modern photography.
	\end{itemize}
}
\item countable noun \\
In some Christian churches, priests are addressed or referred to as \textbf{Father} .
 \textit{
	\begin{itemize}
	\item I would like your advice on a matter of conscience, Father.
	\item ...Father William.
	\end{itemize}
}
\item proper noun \\
Christians often refer to God as \textbf{our Father} or address him as \textbf{Father} .
 \textit{
	\begin{itemize}
	\item ...Our Father in Heaven.
	\end{itemize}
}
\end{enumerate}

\section*{equip}
{\large \color{blue}  equips  equipping  equipped  }
\subsection*{Explain}
\begin{enumerate}
\item verb \\
If you \textbf{equip} a person or thing \textbf{with} something, you give them the tools or equipment that are needed .
 \textit{
	\begin{itemize}
	\item They equipped their vehicles with gadgets to deal with every possible contingency.
	\item Owners of restaurants would have to equip them to admit people with disabilities.
	\item The country did not possess the modern guns to equip the reserve army properly.
	\end{itemize}
}
\item verb \\
If something \textbf{equips} you for a particular task or experience , it gives you the skills and attitudes you need for it, especially by educating you in a particular way.
 \textit{
	\begin{itemize}
	\item Relative poverty, however, did not prevent Martin from equipping himself with an
excellent education.
	\item A basic two-hour first aid course would equip you to deal with any of these incidents.
	\end{itemize}
}
\end{enumerate}

\section*{feedback}
{\large \color{blue}  }
\subsection*{Explain}
\begin{enumerate}
\item uncountable noun \\
If you get  \textbf{feedback}  \textbf{on} your work or progress , someone tells you how well or badly you are doing, and how you could improve . If you get good feedback you have worked or performed well.
 \textit{
	\begin{itemize}
	\item Continue to ask for feedback on your work.
	\item I was getting great feedback from my boss.
	\end{itemize}
}
\item uncountable noun \\
\textbf{Feedback} is the unpleasant high-pitched sound produced by a piece of electrical  equipment when part of the signal that comes out goes back into it.
 \textit{
	\begin{itemize}
	\end{itemize}
}
\end{enumerate}

\section*{expose}
{\large \color{blue}  exposés  }
\subsection*{Explain}
\begin{enumerate}
\item countable noun \\
An \textbf{exposé} is a film or piece of writing which reveals the truth about a situation or person, especially something involving shocking  facts .
 \textit{
	\begin{itemize}
	\item The movie is an exposé of prison conditions in the South.
	\end{itemize}
}
\end{enumerate}

\section*{fossil}
{\large \color{blue}  fossils  }
\subsection*{Explain}
\begin{enumerate}
\item countable noun \\
A \textbf{fossil} is the hard remains of a prehistoric animal or plant that are found inside a rock.
 \textit{
	\begin{itemize}
	\end{itemize}
}
\end{enumerate}

\section*{feed}
{\large \color{blue}  feeds  feeding  fed  }
\subsection*{Explain}
\begin{enumerate}
\item verb \\
If you \textbf{feed} a person or animal, you give them food to eat and sometimes  actually put it in their mouths .
 \textbf{Feed} is also a noun .
 \textit{
	\begin{itemize}
	\item We brought along pieces of old bread and fed the birds.
	\item She fed him a cookie.
	\item In that part of the world you can feed cattle on almost any green vegetable or fruit.
	\item He spooned the ice cream into a cup and fed it to her.
	\item She's had a good feed.
	\end{itemize}
}
\item verb \\
To \textbf{feed} a family or a community means to supply food for them.
 \textit{
	\begin{itemize}
	\item Feeding a hungry family can be expensive .
	\item We have the technology to feed the population of the planet.
	\end{itemize}
}
\item verb \\
When an animal \textbf{feeds} , it eats or drinks something.
 \textit{
	\begin{itemize}
	\item After a few days the caterpillars stopped feeding.
	\item Slugs feed on decaying plant and animal material.
	\end{itemize}
}
\item verb \\
When a baby \textbf{feeds} , or when you \textbf{feed} it, it drinks breast  milk or milk from a bottle .
 \textit{
	\begin{itemize}
	\item When a baby is thirsty, it feeds more often.
	\item I knew absolutely nothing about handling or feeding a baby.
	\end{itemize}
}
\item variable noun \\
Animal \textbf{feed} is food given to animals, especially  farm animals.
 \textit{
	\begin{itemize}
	\item The grain just rotted and all they could use it for was animal feed.
	\item ...poultry feed.
	\end{itemize}
}
\item verb \\
To \textbf{feed} something to a place, means to supply it to that place in a steady flow.
 \textit{
	\begin{itemize}
	\item ...blood vessels that feed blood to the brain.
	\item ...gas fed through pipelines.
	\end{itemize}
}
\item verb \\
If you \textbf{feed} something \textbf{into} a container or piece of equipment , you put it into it.
 \textit{
	\begin{itemize}
	\item He took the compact disc from her, then fed it into the player.
	\item She was feeding documents into a paper shredder.
	\end{itemize}
}
\item verb \\
If someone \textbf{feeds} you false or secret information, they deliberately tell it to you.
 \textit{
	\begin{itemize}
	\item He was surrounded by people who fed him ghastly lies.
	\item At least one British officer was feeding him with classified information.
	\end{itemize}
}
\item verb \\
If you \textbf{feed} someone's dislike or desire for something, you make it stronger.
 \textit{
	\begin{itemize}
	\item The divorce was painfully public, feeding her dislike of the press.
	\end{itemize}
}
\item verb \\
If you \textbf{feed} a plant, you add substances to it to make it grow well .
 \textit{
	\begin{itemize}
	\item Feed plants to encourage steady growth.
	\end{itemize}
}
\item verb \\
If one thing \textbf{feeds}  \textbf{on} another, it becomes stronger as a result of the other thing's existence.
 \textit{
	\begin{itemize}
	\item The drinking and the guilt fed on each other.
	\end{itemize}
}
\item verb \\
To \textbf{feed} information \textbf{into} a computer means to gradually put it into it.
 \textit{
	\begin{itemize}
	\item An automatic weather station feeds information on wind direction to the computer.
	\end{itemize}
}
\item countable noun \\
A \textbf{feed} is a system that tells a user when an item is available to read , for example on Twitter .
 \textit{
	\begin{itemize}
	\item I saw the news on my Twitter feed.
	\end{itemize}
}
\end{enumerate}

\section*{gallery}
{\large \color{blue}  galleries  }
\subsection*{Explain}
\begin{enumerate}
\item countable noun \\
A \textbf{gallery} is a place that has permanent  exhibitions of works of art in it.
 \textit{
	\begin{itemize}
	\item ...an art gallery.
	\item ...the National Gallery.
	\end{itemize}
}
\item countable noun \\
A \textbf{gallery} is a privately owned building or room where people can look at and buy works of art.
 \textit{
	\begin{itemize}
	\item The painting is in the gallery upstairs.
	\end{itemize}
}
\item countable noun \\
A \textbf{gallery} is an area high above the ground at the back or at the sides of a large room or hall.
 \textit{
	\begin{itemize}
	\item A crowd already filled the gallery.
	\end{itemize}
}
\item countable noun \\
\textbf{The gallery} in a theatre or concert hall is an area high above the ground that usually contains the cheapest seats.
 \textit{
	\begin{itemize}
	\item They had been forced to find cheap tickets in the gallery.
	\end{itemize}
}
\end{enumerate}

\section*{fill}
{\large \color{blue}  fills  filling  filled  }
\subsection*{Explain}
\begin{enumerate}
\item verb \\
If you \textbf{fill} a container or area, or if it \textbf{fills} , an amount of something enters it that is enough to make it full.
 \textbf{Fill up} means the same as fill .
 \textit{
	\begin{itemize}
	\item Fill a saucepan with water and bring to a slow boil.
	\item She made sandwiches, filled a flask and put sugar in.
	\item The victims' lungs fill quickly with fluid.
	\item The boy's eyes filled with tears.
	\item While the bath was filling, he padded about in his underpants.
	\item Pass me your cup, Amy, and I'll fill it up for you.
	\item Warehouses at the frontier between the two countries fill up with sacks of rice and
flour.
	\end{itemize}
}
\item verb \\
If something \textbf{fills} a space, it is so big , or there are such large quantities of it, that there is very little room left.
 \textbf{Fill up} means the same as fill .
 \textit{
	\begin{itemize}
	\item He cast his eyes at the rows of cabinets that filled the enormous work area.
	\item The text fills 231 pages.
	\item ...the complicated machines that fill up today's laboratories.
	\end{itemize}
}
\item verb \\
If you \textbf{fill} a crack or hole , you put a substance into it in order to make the surface smooth again.
 \textbf{Fill in} means the same as fill .
 \textit{
	\begin{itemize}
	\item Fill small holes with wood filler in a matching colour.
	\item The gravedigger filled the grave.
	\item If any cracks have appeared in the tart case, fill these in with raw pastry.
	\end{itemize}
}
\item verb \\
If a sound, smell , or light \textbf{fills} a space, or the air, it is very strong or noticeable .
 \textit{
	\begin{itemize}
	\item In the parking lot of the school, the siren filled the air.
	\item All the light bars were turned on which filled the room with these rotating beams
of light.
	\item The barn was filled with the sour-sweet smell of fresh dung.
	\end{itemize}
}
\item verb \\
If something \textbf{fills} you \textbf{with} an emotion , or if an emotion \textbf{fills} you, you experience this emotion strongly.
 \textit{
	\begin{itemize}
	\item I admired my father, and his work filled me with awe and curiosity.
	\item He looked at me without speaking, and for the first time I could see the pride that
filled him.
	\item He stared at his favourite child, dismayed, filled with fear.
	\end{itemize}
}
\item verb \\
If you \textbf{fill} a period of time with a particular activity, you spend the time in this way.
 \textbf{Fill up} means the same as fill .
 \textit{
	\begin{itemize}
	\item If she wants a routine to fill her day, let her do community work.
	\item On Thursday night she went to her yoga class, glad to have something to fill up the
evening.
	\end{itemize}
}
\item verb \\
If something \textbf{fills} a need or a gap, it puts an end to this need or gap by existing or being active .
 \textit{
	\begin{itemize}
	\item I could take this skill set and turn it into something that fills a need.
	\item She brought him a sense of fun, of gaiety that filled a gap in his life.
	\end{itemize}
}
\item verb \\
If something \textbf{fills} a role , position, or function, they have that role or position, or perform that function,
often successfully.
 \textit{
	\begin{itemize}
	\item The company develops internal candidates to fill management roles.
	\end{itemize}
}
\item verb \\
If a company or organization \textbf{fills} a job  vacancy , they choose someone to do the job. If someone \textbf{fills} a job vacancy, they accept a job that they have been offered .
 \textit{
	\begin{itemize}
	\item The unemployed may not have the skills to fill the vacancies on offer.
	\item A vacancy has arisen which I intend to fill.
	\end{itemize}
}
\item verb \\
If you \textbf{fill}  \textbf{yourself with} food, you eat so much that you do not feel  hungry .
 \textit{
	\begin{itemize}
	\item They joked and drank coffee and filled themselves with chocolate cake.
	\end{itemize}
}
\item verb \\
A play, film, or performer that \textbf{fills} a theatre , concert  hall , or cinema  attracts a very large audience .
 \textit{
	\begin{itemize}
	\item Children are enthralled by his stories; he has been known to fill theatre halls in
Australia.
	\end{itemize}
}
\item verb \\
When a dentist  \textbf{fills} someone's tooth , he or she puts a filling in it.
 \textit{
	\begin{itemize}
	\item ...children having teeth filled due to decay.
	\end{itemize}
}
\item verb \\
If you \textbf{fill} an order or a prescription, you provide the things that are asked for.
 \textit{
	\begin{itemize}
	\item A pharmacist can fill any prescription if, in his or her judgment, the prescription
is valid.
	\end{itemize}
}
\item  \\
 have had one's fill of sth \textit{
	\begin{itemize}
	\end{itemize}
}
\end{enumerate}

\section*{gown}
{\large \color{blue}  gowns  }
\subsection*{Explain}
\begin{enumerate}
\item countable noun \\
A \textbf{gown} is a dress, usually a long dress, which women wear on formal occasions .
 \textit{
	\begin{itemize}
	\item The new ball gown was a great success.
	\item ...wedding gowns.
	\end{itemize}
}
\item countable noun \\
A \textbf{gown} is a loose black garment worn on formal occasions by people such as lawyers and academics.
 \textit{
	\begin{itemize}
	\item ...an old headmaster in a flowing black gown.
	\end{itemize}
}
\end{enumerate}

\section*{impress}
{\large \color{blue}  impresses  impressing  impressed  }
\subsection*{Explain}
\begin{enumerate}
\item verb \\
If something \textbf{impresses} you, you feel  great  admiration for it.
 \textit{
	\begin{itemize}
	\item What impressed him most was their speed.
	\item ...a group of students who were trying to impress their girlfriends.
	\item Cannon's film impresses on many levels.
	\end{itemize}
}
\item verb \\
If you \textbf{impress} something \textbf{on} someone, you make them understand its importance or degree .
 \textit{
	\begin{itemize}
	\item I had always impressed upon the children that if they worked hard they would succeed
in life.
	\item I've impressed upon them the need for more professionalism.
	\item I impressed on him what a huge honour he was being offered.
	\end{itemize}
}
\item verb \\
If something \textbf{impresses}  \textbf{itself on} your mind , you notice and remember it.
 \textit{
	\begin{itemize}
	\item But this change has not yet impressed itself on the minds of the British public.
	\end{itemize}
}
\item verb \\
If someone or something \textbf{impresses} you \textbf{as} a particular thing, usually a good one, they gives you the impression of being that thing.
 \textit{
	\begin{itemize}
	\item Billy Sullivan had impressed me as a fine man.
	\end{itemize}
}
\end{enumerate}

\section*{institution}
{\large \color{blue}  institutions  }
\subsection*{Explain}
\begin{enumerate}
\item countable noun \\
An \textbf{institution} is a large important organization such as a university , church, or bank.
 \textit{
	\begin{itemize}
	\item ...the Institution of Civil Engineers.
	\item Class size varies from one type of institution to another.
	\item The Hong Kong Bank is Hong Kong's largest financial institution.
	\end{itemize}
}
\item countable noun \\
An \textbf{institution} is a building where certain people are looked after, for example people who are mentally ill or children who have no parents .
 \textit{
	\begin{itemize}
	\item Larry has been in an institution since he was four.
	\item He visited various penal institutions in the United Kingdom in the late 1930s.
	\end{itemize}
}
\item countable noun \\
An \textbf{institution} is a custom or system that is considered an important or typical feature of a particular society or group, usually because it has existed for a long time.
 \textit{
	\begin{itemize}
	\item I believe in the institution of marriage.
	\item ...the institution of the family.
	\end{itemize}
}
\item uncountable noun \\
The \textbf{institution} of a new system is the act of starting it or bringing it in.
 \textit{
	\begin{itemize}
	\item There was never an official institution of censorship in Albania.
	\item ...the institution of the forty-hour week.
	\end{itemize}
}
\end{enumerate}

\section*{inform}
{\large \color{blue}  informs  informing  informed  }
\subsection*{Explain}
\begin{enumerate}
\item verb \\
If you \textbf{inform} someone \textbf{of} something, you tell them about it.
 \textit{
	\begin{itemize}
	\item They would inform him of any progress they had made.
	\item My daughter informed me that she was pregnant.
	\item 'I just added a little soy sauce,' he informs us.
	\end{itemize}
}
\item verb \\
If someone \textbf{informs on} a person, they give information about the person to the police or another authority , which causes the person to be suspected or proved  guilty of doing something bad .
 \textit{
	\begin{itemize}
	\item Somebody must have informed on us.
	\item Thousands of American citizens have informed on these organized crime syndicates.
	\end{itemize}
}
\item verb \\
If a situation or activity \textbf{is informed} by an idea or a quality, that idea or quality is very noticeable in it.
 \textit{
	\begin{itemize}
	\item All great songs are informed by a certain sadness and tension.
	\item The concept of the Rose continued to inform the poet's work.
	\end{itemize}
}
\end{enumerate}

\section*{irony}
{\large \color{blue}  ironies  }
\subsection*{Explain}
\begin{enumerate}
\item uncountable noun \\
\textbf{Irony} is a subtle form of humour which involves saying things that you do not mean.
 \textit{
	\begin{itemize}
	\item They find only irony in the narrator's concern.
	\item Sinclair examined the closed, clever face for any hint of irony, but found none.
	\end{itemize}
}
\item variable noun \\
If you talk about the \textbf{irony} of a situation, you mean that it is odd or amusing because it involves a contrast .
 \textit{
	\begin{itemize}
	\item The irony is that many officials in Washington agree in private that their policy
is inconsistent.
	\item The irony of the situation is not lost on Fellaini.
	\end{itemize}
}
\end{enumerate}

\section*{organize}
{\large \color{blue}  organizes  organizing  organized  }
\subsection*{Explain}
\begin{enumerate}
\item verb \\
If you \textbf{organize} an event or activity , you make sure that the necessary  arrangements are made.
 \textit{
	\begin{itemize}
	\item In the end, we all decided to organize a concert for Easter.
	\item ...a two-day meeting organised by the United Nations.
	\item The initial mobilisation was well organised.
	\end{itemize}
}
\item verb \\
If you \textbf{organize} something that someone wants or needs , you make sure that it is provided.
 \textit{
	\begin{itemize}
	\item I will organize transport.
	\item We asked them to organize coffee and sandwiches.
	\end{itemize}
}
\item verb \\
If you \textbf{organize} a set of things, you arrange them in an ordered way or give them a structure.
 \textit{
	\begin{itemize}
	\item He began to organize his materials.
	\item She took a hasty cup of coffee and tried to organize her scattered thoughts.
	\item ...the way in which the Army is organised.
	\end{itemize}
}
\item verb \\
If you \textbf{organize} yourself, you plan your work and activities in an ordered, efficient way.
 \textit{
	\begin{itemize}
	\item ...changing the way you organize yourself.
	\item Go right ahead, I'm sure you don't need me to organize you.
	\item Get organised and get going.
	\end{itemize}
}
\item verb \\
If someone \textbf{organizes} workers or if workers \textbf{organize} , they form a group or society such as a trade union in order to have more power .
 \textit{
	\begin{itemize}
	\item ...helping to organize women working abroad.
	\item It's the first time farmers have decided to organize.
	\item ...organised labour.
	\end{itemize}
}
\end{enumerate}

\section*{justice}
{\large \color{blue}  justices  }
\subsection*{Explain}
\begin{enumerate}
\item uncountable noun \\
\textbf{Justice} is fairness in the way that people are treated.
 \textit{
	\begin{itemize}
	\item He has a good overall sense of justice and fairness.
	\item He only wants freedom, justice and equality.
	\item There is no justice in this world!
	\end{itemize}
}
\item uncountable noun \\
The \textbf{justice}  \textbf{of} a cause, claim , or argument is its quality of being reasonable , fair , or right.
 \textit{
	\begin{itemize}
	\item We are a minority and must win people round to the justice of our cause.
	\end{itemize}
}
\item uncountable noun \\
\textbf{Justice} is the legal system that a country uses in order to deal with people who break the law.
 \textit{
	\begin{itemize}
	\item Many in Toronto's black community feel that the justice system does not treat them
fairly.
	\item A lawyer is part of the machinery of justice.
	\end{itemize}
}
\item countable noun \\
A \textbf{justice} is a judge.
 \textit{
	\begin{itemize}
	\item Thomas will be sworn in today as a justice on the Supreme Court.
	\end{itemize}
}
\item title noun \\
\textbf{Justice} is used before the names of judges .
 \textit{
	\begin{itemize}
	\item A preliminary hearing was due to start today before Mr Justice Hutchison, but was
adjourned.
	\end{itemize}
}
\item  \\
 bring someone to justice \textit{
	\begin{itemize}
	\end{itemize}
}
\item  \\
 do justice \textit{
	\begin{itemize}
	\end{itemize}
}
\item  \\
 do justice \textit{
	\begin{itemize}
	\end{itemize}
}
\item  \\
 do oneself justice \textit{
	\begin{itemize}
	\end{itemize}
}
\item  \\
 rough justice \textit{
	\begin{itemize}
	\end{itemize}
}
\item  \\
 rough justice \textit{
	\begin{itemize}
	\end{itemize}
}
\end{enumerate}

\section*{paralyze}
{\large \color{blue}  }
\subsection*{Explain}
\begin{enumerate}
\item verb transitive \\
1.  2.  \textit{
	\begin{itemize}
	\end{itemize}
}
\end{enumerate}

\section*{kindness}
{\large \color{blue}  kindnesses  }
\subsection*{Explain}
\begin{enumerate}
\item uncountable noun \\
\textbf{Kindness} is the quality of being gentle , caring , and helpful.
 \textit{
	\begin{itemize}
	\item We have been treated with such kindness by everybody.
	\end{itemize}
}
\item countable noun \\
A \textbf{kindness} is a helpful or considerate act.
 \textit{
	\begin{itemize}
	\end{itemize}
}
\end{enumerate}

\section*{prefer}
{\large \color{blue}  prefers  preferring  preferred  }
\subsection*{Explain}
\begin{enumerate}
\item verb \\
If you \textbf{prefer} someone or something, you like that person or thing better than another, and so you
are more likely to choose them if there is a choice .
 \textit{
	\begin{itemize}
	\item Does he prefer a particular sort of music?
	\item I became a teacher because I preferred books and people to politics.
	\item I prefer to go on self-catering holidays.
	\item I would prefer him to be with us next season.
	\item Bob prefers making original pieces rather than reproductions.
	\item The woodwork's green now. I preferred it blue.
	\item Her own preferred methods of exercise are hiking and long cycle rides.
	\end{itemize}
}
\end{enumerate}

\section*{mercy}
{\large \color{blue}  mercies  }
\subsection*{Explain}
\begin{enumerate}
\item uncountable noun \\
If someone in authority shows \textbf{mercy} , they choose not to harm someone they have power over, or they forgive someone they have the right to punish .
 \textit{
	\begin{itemize}
	\item Neither side took prisoners or showed any mercy.
	\item They cried for mercy but their pleas were met with abuse and laughter.
	\item May God have mercy on your soul.
	\end{itemize}
}
\item adjective \\
\textbf{Mercy} is used to describe a special  journey to help someone in great need , such as people who are sick or made homeless by war.
 \textit{
	\begin{itemize}
	\item She vanished nine months ago while on a mercy mission to West Africa.
	\item It's the first so-called mercy flight for a fortnight as the Americans have been
waiting for enough people to fill a 747 jet.
	\end{itemize}
}
\item countable noun \\
If you refer to an event or situation as \textbf{a}  \textbf{mercy} , you mean that it makes you feel  happy or relieved, usually because it stops something unpleasant  happening .
 \textit{
	\begin{itemize}
	\item It really was a mercy that he'd died so rapidly at the end.
	\item The two cars finished up in a run-off area, clear of the circuit, and that was a
mercy.
	\end{itemize}
}
\item  \\
 at the mercy of someone \textit{
	\begin{itemize}
	\end{itemize}
}
\item  \\
 grateful/thankful for small mercies \textit{
	\begin{itemize}
	\end{itemize}
}
\item  \\
 throw yourself on someone's mercy \textit{
	\begin{itemize}
	\end{itemize}
}
\end{enumerate}

\section*{prevail}
{\large \color{blue}  prevails  prevailing  prevailed  }
\subsection*{Explain}
\begin{enumerate}
\item verb \\
If a proposal , principle , or opinion  \textbf{prevails} , it gains influence or is accepted , often after a struggle or argument .
 \textit{
	\begin{itemize}
	\item We hope that common sense would prevail.
	\item Rick still believes that justice will prevail.
	\item Political and personal ambitions are starting to prevail over economic interests.
	\end{itemize}
}
\item verb \\
If a situation , attitude , or custom  \textbf{prevails} in a particular place at a particular time, it is normal or most common in that place at that time.
 \textit{
	\begin{itemize}
	\item A similar situation prevails in America.
	\item ...the confusion which had prevailed at the time of the revolution.
	\item How people bury their dead says much about the prevailing attitudes toward death.
	\end{itemize}
}
\item verb \\
If one side in a battle , contest , or dispute  \textbf{prevails} , it wins .
 \textit{
	\begin{itemize}
	\item He appears to have the votes he needs to prevail.
	\item I do hope he will prevail over the rebels.
	\end{itemize}
}
\item verb \\
If you \textbf{prevail upon} someone \textbf{to} do something, you succeed in persuading them to do it.
 \textit{
	\begin{itemize}
	\item We must, each of us, prevail upon our congressman to act.
	\item Do you think she could be prevailed upon to do those things?
	\end{itemize}
}
\end{enumerate}

\section*{minute}
{\large \color{blue}  minutes  minuting  minuted  }
\subsection*{Explain}
\begin{enumerate}
\item countable noun \\
A \textbf{minute} is one of the sixty parts that an hour is divided into. People often say ' \textbf{a minute} ' or ' \textbf{minutes} ' when they mean a short length of time.
 \textit{
	\begin{itemize}
	\item The pizza will then take about twenty minutes to cook.
	\item Bye Mum, see you in a minute.
	\item Half a minute later she came in the front door.
	\item Within minutes we realized our mistake.
	\end{itemize}
}
\item plural noun \\
The \textbf{minutes} of a meeting are the written records of the things that are discussed or decided at it.
 \textit{
	\begin{itemize}
	\item He'd been reading the minutes of the last meeting.
	\end{itemize}
}
\item verb \\
When someone \textbf{minutes} something that is discussed or decided at a meeting, they make a written record of
it.
 \textit{
	\begin{itemize}
	\item You don't need to minute that.
	\end{itemize}
}
\item  \\
 wait a minute \textit{
	\begin{itemize}
	\end{itemize}
}
\item  \\
 (at) any minute (now) \textit{
	\begin{itemize}
	\end{itemize}
}
\item  \\
 for a/one minute \textit{
	\begin{itemize}
	\end{itemize}
}
\item  \\
 last minute \textit{
	\begin{itemize}
	\end{itemize}
}
\item  \\
 the next minute \textit{
	\begin{itemize}
	\end{itemize}
}
\item  \\
 the minute \textit{
	\begin{itemize}
	\end{itemize}
}
\item  \\
 this minute \textit{
	\begin{itemize}
	\end{itemize}
}
\end{enumerate}

\section*{realise}
{\large \color{blue}  }
\subsection*{Explain}
\begin{enumerate}
\end{enumerate}

\section*{molecule}
{\large \color{blue}  molecules  }
\subsection*{Explain}
\begin{enumerate}
\item countable noun \\
A \textbf{molecule} is the smallest amount of a chemical substance which can exist by itself.
 \textit{
	\begin{itemize}
	\item ...the hydrogen bonds between water molecules.
	\end{itemize}
}
\end{enumerate}

\section*{recognize}
{\large \color{blue}  recognizes  recognizing  recognized  }
\subsection*{Explain}
\begin{enumerate}
\item verb \\
If you \textbf{recognize} someone or something, you know who that person is or what that thing is.
 \textit{
	\begin{itemize}
	\item The receptionist recognized him at once.
	\item He did not think she could recognize his car in the snow.
	\item A man I easily recognized as Luke's father sat with a newspaper on his lap.
	\end{itemize}
}
\item verb \\
If someone says that they \textbf{recognize} something, they acknowledge that it exists or that it is true .
 \textit{
	\begin{itemize}
	\item I recognize my own shortcomings.
	\item Well, of course I recognize that evil exists.
	\end{itemize}
}
\item verb \\
If people or organizations \textbf{recognize} something as valid , they officially accept it or approve of it.
 \textit{
	\begin{itemize}
	\item Most doctors appear to recognize homeopathy as a legitimate form of medicine.
	\item Eisenhower recognized the Castro government at once.
	\item ...a nationally recognized expert on psychology.
	\end{itemize}
}
\item verb \\
When people \textbf{recognize} the work that someone has done, they show their appreciation of it, often by giving
that person an award of some kind .
 \textit{
	\begin{itemize}
	\item The RAF recognized him as an outstandingly able engineer.
	\item He had the insight to recognize their talents.
	\item Nichols was recognized by the Hall of Fame in 1949.
	\end{itemize}
}
\end{enumerate}

\section*{moment}
{\large \color{blue}  moments  }
\subsection*{Explain}
\begin{enumerate}
\item countable noun \\
You can refer to a very short period of time, for example a few seconds , as a \textbf{moment} or \textbf{moments} .
 \textit{
	\begin{itemize}
	\item In a moment he was gone.
	\item She stared at him a moment, then turned away.
	\item Stop for one moment and think about it!
	\item In moments, I was asleep once more.
	\end{itemize}
}
\item countable noun \\
A particular \textbf{moment} is the point in time at which something happens .
 \textit{
	\begin{itemize}
	\item At this moment a car stopped at the house.
	\item I'll never forget the moment when I first saw it.
	\item ...a decision that may have been made in a moment of panic.
	\end{itemize}
}
\item  \\
 (at) any moment (now) \textit{
	\begin{itemize}
	\end{itemize}
}
\item  \\
 at the/this moment/present moment \textit{
	\begin{itemize}
	\end{itemize}
}
\item  \\
 for a/one moment \textit{
	\begin{itemize}
	\end{itemize}
}
\item  \\
 for the moment \textit{
	\begin{itemize}
	\end{itemize}
}
\item  \\
 have one's moments \textit{
	\begin{itemize}
	\end{itemize}
}
\item  \\
 be having a moment \textit{
	\begin{itemize}
	\end{itemize}
}
\item  \\
 the last moment \textit{
	\begin{itemize}
	\end{itemize}
}
\item  \\
 the next moment \textit{
	\begin{itemize}
	\end{itemize}
}
\item  \\
 of the moment \textit{
	\begin{itemize}
	\end{itemize}
}
\item  \\
 the moment \textit{
	\begin{itemize}
	\end{itemize}
}
\end{enumerate}

\section*{rely}
{\large \color{blue}  relies  relying  relied  }
\subsection*{Explain}
\begin{enumerate}
\item verb \\
If you \textbf{rely}  \textbf{on} someone or something, you need them and depend on them in order to live or work properly.
 \textit{
	\begin{itemize}
	\item They relied heavily on the advice of their professional advisers.
	\item The Association relies on member subscriptions for most of its income.
	\end{itemize}
}
\item verb \\
If you can \textbf{rely}  \textbf{on} someone to work well or to behave as you want them to, you can trust them to do this.
 \textit{
	\begin{itemize}
	\item I know I can rely on you to sort it out.
	\item The Red Cross are relying on us.
	\end{itemize}
}
\end{enumerate}

\section*{offset}
{\large \color{blue}  offsets  offsetting  }
\subsection*{Explain}
\begin{enumerate}
\item verb \\
If one thing \textbf{is offset} by another, the effect of the first thing is reduced by the second, so that any advantage or disadvantage is cancelled out.
 \textit{
	\begin{itemize}
	\item The increase in pay costs was more than offset by higher productivity.
	\item The move is designed to help offset the shortfall in world oil supplies.
	\end{itemize}
}
\end{enumerate}

\section*{remind}
{\large \color{blue}  reminds  reminding  reminded  }
\subsection*{Explain}
\begin{enumerate}
\item verb \\
If someone \textbf{reminds} you \textbf{of} a fact or event that you already  know about, they say something which makes you think about it.
 \textit{
	\begin{itemize}
	\item So she simply welcomed him and reminded him of the last time they had met.
	\item I had to remind myself that being confident is not the same as being perfect!
	\end{itemize}
}
\item verb \\
You use \textbf{remind} in expressions such as \textbf{Let me remind you that} and \textbf{May I remind you that} to introduce a piece of information that you want to emphasize . It may be something that the hearer already knows about or a new piece of information. Sometimes these expressions can  sound  unfriendly .
 \textit{
	\begin{itemize}
	\item 'Let me remind you,' said Marianne, 'that Manchester is also my home town.'
	\item May I remind you that the care of your health is a religious duty.
	\item Need I remind you who the enemy is?
	\end{itemize}
}
\item verb \\
If someone \textbf{reminds} you \textbf{to} do a particular thing, they say something which makes you remember to do it.
 \textit{
	\begin{itemize}
	\item Can you remind me to buy a bottle of Martini?
	\item The note was to remind him about something he had to explain to one of his students.
	\end{itemize}
}
\item verb \\
If you say that someone or something \textbf{reminds} you \textbf{of} another person or thing, you mean that they are similar to the other person or thing and that they make you think about them.
 \textit{
	\begin{itemize}
	\item She reminds me of the woman who used to work for you.
	\item This reminds me of Christmas parties.
	\end{itemize}
}
\end{enumerate}

\section*{open}
{\large \color{blue}  opens  opening  opened  }
\subsection*{Explain}
\begin{enumerate}
\item verb \\
If you \textbf{open} something such as a door, window, or lid , or if it \textbf{opens} , its position is changed so that it no longer covers a hole or gap.
 \textbf{Open} is also an adjective.
 \textit{
	\begin{itemize}
	\item He opened the window and looked out.
	\item The church doors would open and the crowd would surge out.
	\item ...an open window.
	\item A door had been forced open.
	\end{itemize}
}
\item verb \\
If you \textbf{open} something such as a bottle, box, parcel , or envelope , you move, remove, or cut part of it so you can take out what is inside.
 \textbf{Open} is also an adjective.
 \textbf{Open up} means the same as open .
 \textit{
	\begin{itemize}
	\item The Inspector opened the suitcase.
	\item The capsules are fiddly to open.
	\item ...an open bottle of milk.
	\item I tore the letter open.
	\item He opened up a cage and lifted out a 6ft python.
	\end{itemize}
}
\item verb \\
If you \textbf{open} something such as a book, an umbrella , or your hand, or if it \textbf{opens} , the different parts of it move away from each other so that the inside of it can
be seen.
 \textbf{Open} is also an adjective.
 \textbf{Open out} means the same as open .
 \textit{
	\begin{itemize}
	\item He opened the heavy Bible.
	\item She opens her umbrella, and walks up River Street.
	\item The flower opens to reveal a Queen Bee.
	\item The officer's mouth opened, showing white, even teeth.
	\item Without warning, Bardo smacked his fist into his open hand.
	\item His mouth was a little open, as if he'd started to scream.
	\item Keith took a map from the dashboard and opened it out on his knees.
	\item ...oval tables which open out to become circular.
	\end{itemize}
}
\item verb \\
If you \textbf{open} a computer file, you give the computer an instruction to display it on the screen.
 \textit{
	\begin{itemize}
	\item Double click on the icon to open the file.
	\end{itemize}
}
\item verb \\
When you \textbf{open} your eyes or your eyes \textbf{open} , you move your eyelids upwards, for example when you wake up, so that you can see.
 \textbf{Open} is also an adjective.
 \textit{
	\begin{itemize}
	\item When I opened my eyes I saw a man with an axe standing at the end of my bed.
	\item His eyes were opening wide.
	\item As soon as he saw that her eyes were open, he sat up.
	\end{itemize}
}
\item verb \\
If you \textbf{open} your arms, you stretch them wide apart in front of you, usually in order to put them
round someone.
 \textit{
	\begin{itemize}
	\item She opened her arms and gave me a big hug.
	\end{itemize}
}
\item graded adjective \\
If you stand or sit in an \textbf{open} way, the front of your body is fully exposed and you are not bending forward or at
an angle to someone.
 \textit{
	\begin{itemize}
	\item Good listeners even sit in an open way: relaxed, arms loose.
	\item I play normal bunker shots with an open stance.
	\end{itemize}
}
\item adjective \\
If you describe a person or their character as \textbf{open} , you mean they are honest and do not want or try to hide anything or to deceive anyone.
 \textit{
	\begin{itemize}
	\item He had always been open with her and she always felt she would know if he lied.
	\item She has an open, trusting nature.
	\end{itemize}
}
\item adjective \\
If you describe a situation, attitude, or way of behaving as \textbf{open} , you mean it is not kept hidden or secret.
 \textit{
	\begin{itemize}
	\item The action is an open violation of the Vienna Convention.
	\item Hearing the case in open court is only one part of the judicial process.
	\end{itemize}
}
\item adjective \\
If you are \textbf{open to}  suggestions or ideas, you are ready and willing to consider or accept them.
 \textit{
	\begin{itemize}
	\item They are open to suggestions on how working conditions might be improved.
	\end{itemize}
}
\item adjective \\
If you say that a system, person, or idea is \textbf{open to} something such as abuse or criticism , you mean they might receive abuse or criticism because of their qualities, effects, or actions.
 \textit{
	\begin{itemize}
	\item The system, though well-meaning, is open to abuse.
	\item They left themselves wide open to accusations of double standards and hypocrisy.
	\end{itemize}
}
\item adjective \\
If you say that a fact or question is \textbf{open}  \textbf{to}  debate , interpretation , or discussion, you mean that people are uncertain whether it is true, what it means, or what the answer is.
 \textit{
	\begin{itemize}
	\item The truth of the facts produced may be open to doubt.
	\item It is an open question how long that commitment can last.
	\end{itemize}
}
\item verb \\
If people \textbf{open} something such as a blocked road or a border, or if it \textbf{opens} , people can then pass along it or through it.
 \textbf{Open} is also an adjective.
 \textbf{Open up} means the same as open .
 \textit{
	\begin{itemize}
	\item The rebels have opened the road from Monrovia to the Ivory Coast.
	\item The solid rank of police officers lining the courtroom opened to let them pass.
	\item We were part of an entire regiment that had nothing else to do but to keep that highway
open.
	\item Can we get the fencing removed and open up the road again?
	\item When the Berlin Wall came down it wasn't just the roads that opened up but the waterways
too.
	\end{itemize}
}
\item verb \\
If a place \textbf{opens}  \textbf{into} another, larger place, you can move from one directly into the other.
 \textbf{Open out} means the same as open .
 \textit{
	\begin{itemize}
	\item The corridor opened into a low smoky room.
	\item ...narrow streets opening out into charming squares.
	\end{itemize}
}
\item adjective \\
An \textbf{open} area is a large area that does not have many buildings or trees in it.
 \textit{
	\begin{itemize}
	\item Officers will also continue their search of nearby open ground.
	\end{itemize}
}
\item adjective \\
An \textbf{open} structure or object is not covered or enclosed.
 \textit{
	\begin{itemize}
	\item Don't leave a child alone in a room with an open fire.
	\item ...open sandwiches.
	\end{itemize}
}
\item adjective \\
An \textbf{open} wound is one from which a liquid such as blood is coming.
 \textit{
	\begin{itemize}
	\end{itemize}
}
\item verb \\
If you \textbf{open} your shirt or coat, you undo the buttons or pull down the zip .
 \textbf{Open} is also an adjective.
 \textit{
	\begin{itemize}
	\item I opened my coat and let him see the belt.
	\item The top can be worn buttoned up or open over a T-shirt.
	\item ...dressing informally in open shirt and cowboy boots.
	\end{itemize}
}
\item verb \\
When a shop, office, or public building \textbf{opens} or \textbf{is opened} , its doors are unlocked and the public can go in.
 \textbf{Open} is also an adjective.
 \textit{
	\begin{itemize}
	\item Banks closed on Friday afternoon and did not open again until Monday morning.
	\item ...a gang of three who'd apparently been lying in wait for him to open the shop.
	\item ...opening and closing times.
	\item His shop is open Monday through Friday, 9am to 6pm.
	\end{itemize}
}
\item verb \\
When a public building, factory, or company \textbf{opens} or when someone \textbf{opens} it, it starts operating for the first time.
 \textbf{Open} is also an adjective.
 \textit{
	\begin{itemize}
	\item The original station opened in 1754.
	\item The complex opens to the public tomorrow.
	\item They are planning to open a factory in Eastern Europe.
	\item The Savoy Theatre was opened in 1881 by Richard D'Oyly Carte.
	\item ...any operating subsidy required to keep the pits open.
	\end{itemize}
}
\item verb \\
If something such as a meeting or series of talks \textbf{opens} , or if someone \textbf{opens} it, it begins.
 \textit{
	\begin{itemize}
	\item ...an emergency session of the Russian Parliament due to open later this morning.
	\item They are now ready to open negotiations.
	\end{itemize}
}
\item verb \\
If an event such as a meeting or discussion \textbf{opens}  \textbf{with} a particular activity or if a particular activity \textbf{opens} an event, that activity is the first thing that happens or is dealt with. You can
also say that someone such as a speaker or singer \textbf{opens}  \textbf{by} doing a particular thing.
 \textit{
	\begin{itemize}
	\item The service opened with a hymn.
	\item She opened with an impressive version of 'I Still Haven't Found What I'm Looking
For'.
	\item I opened by saying, 'Honey, you look sensational.'.
	\item Pollard opened the conversation with some small talk.
	\end{itemize}
}
\item verb \\
On the stock exchange, the price at which currencies, shares, or commodities \textbf{open} is their value at the start of that day's trading.
 \textit{
	\begin{itemize}
	\item The stock had opened at $71.
	\item The dollar opened almost unchanged.
	\end{itemize}
}
\item verb \\
When a film, play, or other public event \textbf{opens} , it begins to be shown, be performed, or take place for a limited period of time.
 \textit{
	\begin{itemize}
	\item A photographic exhibition opens at the Royal College of Art on Wednesday.
	\item This show, too, was virtually sold out before it opened.
	\end{itemize}
}
\item verb \\
If you \textbf{open} an account with a bank or a commercial organization, you begin to use their services.
 \textit{
	\begin{itemize}
	\item He tried to open an account at the branch of his bank nearest to his workplace.
	\end{itemize}
}
\item adjective \\
If an opportunity or choice  \textbf{is open to} you, you are able to do a particular thing if you choose to.
 \textit{
	\begin{itemize}
	\item There are a wide range of career opportunities open to young people.
	\end{itemize}
}
\item verb \\
To \textbf{open} opportunities or possibilities means the same as to \textbf{open} them \textbf{up} .
 \textit{
	\begin{itemize}
	\item The chief of naval operations wants to open opportunities for women in the Navy.
	\item A series of fortunate opportunities opened to him.
	\end{itemize}
}
\item adjective \\
You can use \textbf{open} to describe something that anyone is allowed to take part in or accept.
 \textit{
	\begin{itemize}
	\item A recent open meeting of College members revealed widespread dissatisfaction.
	\item A portfolio approach would keep entry into the managerial profession open and flexible.
	\item ...an open invitation.
	\end{itemize}
}
\item adjective \\
If something such as an offer or job is \textbf{open} , it is available for someone to accept or apply for.
 \textit{
	\begin{itemize}
	\item The offer will remain open until further notice.
	\end{itemize}
}
\item  \\
 in the open \textit{
	\begin{itemize}
	\end{itemize}
}
\item  \\
 in the open/out in the open \textit{
	\begin{itemize}
	\end{itemize}
}
\item  \\
 wide open \textit{
	\begin{itemize}
	\end{itemize}
}
\item  \\
 wide open \textit{
	\begin{itemize}
	\end{itemize}
}
\end{enumerate}

\section*{rid}
{\large \color{blue}  rids  ridding  }
\subsection*{Explain}
\begin{enumerate}
\item  \\
 get rid of sth \textit{
	\begin{itemize}
	\end{itemize}
}
\item  \\
 get rid of sb \textit{
	\begin{itemize}
	\end{itemize}
}
\item verb \\
If you \textbf{rid} a place or person \textbf{of} something undesirable or unwanted , you succeed in removing it completely from that place or person.
 \textit{
	\begin{itemize}
	\item The proposals are an attempt to rid the country of political corruption.
	\item The new vaccine may rid the world of one of its most terrifying diseases.
	\end{itemize}
}
\item verb \\
If you \textbf{rid}  \textbf{yourself}  \textbf{of} something you do not want , you take action so that you no longer have it or are no longer affected by it.
 \textit{
	\begin{itemize}
	\item Why couldn't he ever rid himself of those thoughts, those worries?
	\item ...the country's efforts to rid itself of poverty and hunger.
	\end{itemize}
}
\item adjective \\
If you \textbf{are rid of} someone or something that you did not want or that caused problems for you, they are no longer with you or causing problems for you.
 \textit{
	\begin{itemize}
	\item The family had sought a way to be rid of her and the problems she had caused them.
	\end{itemize}
}
\item  \\
 be well rid of sb \textit{
	\begin{itemize}
	\end{itemize}
}
\end{enumerate}

\section*{parent}
{\large \color{blue}  parents  }
\subsection*{Explain}
\begin{enumerate}
\item countable noun \\
Your \textbf{parents} are your mother and father.
 \textit{
	\begin{itemize}
	\item Children need their parents.
	\item This is where a lot of parents go wrong.
	\item When you become a parent, the things you once cared about seem to have less value.
	\end{itemize}
}
\item adjective \\
An organization's \textbf{parent} organization is the organization that created it and usually still controls it.
 \textit{
	\begin{itemize}
	\item Each unit including the parent company has its own, local management.
	\item ...the zoo's parent body, the Zoological Society of London.
	\end{itemize}
}
\item adjective \\
The \textbf{parent} animal, plant, or organism of a particular animal, plant or organism is the one that
it comes from or is produced by.
 \textit{
	\begin{itemize}
	\item Parent birds began to hunt for food for their young.
	\end{itemize}
}
\end{enumerate}

\section*{rob}
{\large \color{blue}  robs  robbing  robbed  }
\subsection*{Explain}
\begin{enumerate}
\item verb \\
If someone \textbf{is robbed} , they have money or property  stolen from them.
 \textit{
	\begin{itemize}
	\item Mrs Yacoub was robbed of her £3,000 designer watch at her West London home.
	\item Police said Stefanovski had robbed a man just hours earlier.
	\end{itemize}
}
\item verb \\
If someone \textbf{is robbed}  \textbf{of} something that they deserve , have, or need , it is taken away from them.
 \textit{
	\begin{itemize}
	\item She was robbed of a carefree childhood.
	\item I can't forgive Lewis for robbing me of a gold medal.
	\end{itemize}
}
\end{enumerate}

\section*{park}
{\large \color{blue}  parks  parking  parked  }
\subsection*{Explain}
\begin{enumerate}
\item countable noun \\
A \textbf{park} is a public area of land with grass and trees, usually in a town, where people go
in order to relax and enjoy themselves.
 \textit{
	\begin{itemize}
	\item ...Regent's Park.
	\item ...a brisk walk with the dog around the park.
	\item They stopped and sat on a park bench.
	\end{itemize}
}
\item verb \\
When you \textbf{park} a vehicle or \textbf{park} somewhere, you drive the vehicle into a position where it can stay for a period of time, and leave it there.
 \textit{
	\begin{itemize}
	\item Greenfield turned into the next side street and parked.
	\item He found a place to park the car.
	\item Ben parked across the street.
	\item ...rows of parked cars.
	\end{itemize}
}
\item countable noun \\
You can refer to a place where a particular activity is carried out as a \textbf{park} .
 \textit{
	\begin{itemize}
	\item ...a science and technology park.
	\item ...a business park.
	\end{itemize}
}
\item variable noun \\
A private area of grass and trees around a large country house is referred to as a
 \textbf{park} .
 \textit{
	\begin{itemize}
	\item ...a 19th-century manor house in six acres of park and woodland.
	\end{itemize}
}
\item singular noun \\
Some people refer to a football or rugby field as \textbf{the park} .
 \textit{
	\begin{itemize}
	\item Chris was also the best player on the park.
	\end{itemize}
}
\item countable noun \\
A \textbf{park} is a park or stadium where baseball is played. \textbf{Park} is also used in the names of some parks.
 \textit{
	\begin{itemize}
	\item ...a spot where professional baseball has been played in one park or another since
1896.
	\item The network was broadcasting the World Series from Candlestick Park when the quake
struck.
	\end{itemize}
}
\end{enumerate}

\section*{send}
{\large \color{blue}  sends  sending  sent  }
\subsection*{Explain}
\begin{enumerate}
\item verb \\
When you \textbf{send} someone something, you arrange for it to be taken and delivered to them, for example by post .
 \textit{
	\begin{itemize}
	\item Myra Cunningham sent me a note thanking me for dinner.
	\item I sent a copy to the minister for transport.
	\item He sent a basket of exotic fruit and a card.
	\item Sir Denis took one look and sent it back.
	\item A huge shipment of grain had been sent from Argentina to Peru.
	\end{itemize}
}
\item verb \\
If you \textbf{send} someone somewhere , you tell them to go there.
 \textit{
	\begin{itemize}
	\item Inspector Banbury came up to see her, but she sent him away.
	\item He had been sent here to keep an eye on Benedict.
	\item ...the government's decision to send troops to the region.
	\item I suggested that he rest, and sent him for an X-ray.
	\item Reinforcements were being sent from the neighbouring region..
	\end{itemize}
}
\item verb \\
If you \textbf{send} someone \textbf{to} an institution such as a school or a prison , you arrange for them to stay there for a period of time.
 \textit{
	\begin{itemize}
	\item It's his parents' choice to send him to a boarding school, rather than a convenient
day school.
	\item You're saying they are sending too many people to prison?
	\end{itemize}
}
\item verb \\
To \textbf{send} a signal  means to cause it to go to a place by means of radio waves or electricity .
 \textit{
	\begin{itemize}
	\item The transmitters will send a signal automatically to a local base station.
	\item ...in 1989, after a 12-year journey to Neptune, the space probe Voyager sent back
pictures of Triton, its moon.
	\end{itemize}
}
\item verb \\
If something \textbf{sends} things or people in a particular  direction , it causes them to move in that direction.
 \textit{
	\begin{itemize}
	\item The explosion sent shrapnel flying all over the place.
	\item He let David go with a thrust of his wrist that sent the lad reeling.
	\item The slight back and forth motion sent a pounding surge of pain into his skull.
	\end{itemize}
}
\item verb \\
To \textbf{send} someone or something \textbf{into} a particular state means to cause them to go into or be in that state.
 \textit{
	\begin{itemize}
	\item My attempt to fix it sent Lawrence into fits of laughter.
	\item ...before civil war and famine sent the country plunging into anarchy.
	\item An obsessive search for our inner selves, far from saving the world, could send us
all mad.
	\end{itemize}
}
\end{enumerate}

\section*{paw}
{\large \color{blue}  paws  pawing  pawed  }
\subsection*{Explain}
\begin{enumerate}
\item countable noun \\
The \textbf{paws} of an animal such as a cat , dog , or bear are its feet, which have claws for gripping things and soft  pads for walking on.
 \textit{
	\begin{itemize}
	\item The kitten was black with white front paws and a white splotch on her chest.
	\item He removes a thorn from a lion's paw.
	\end{itemize}
}
\item countable noun \\
You can describe someone's hand as their \textbf{paw} , especially if it is very large or if they are very clumsy.
 \textit{
	\begin{itemize}
	\item He shook Keaton's hand with his big paw.
	\end{itemize}
}
\item verb \\
If an animal \textbf{paws} something, it draws its foot over it or down it.
 \textit{
	\begin{itemize}
	\item Madigan's horse pawed the ground.
	\item The dogs continued to paw and claw frantically at the chain mesh.
	\end{itemize}
}
\item verb \\
If one person \textbf{paws} another, they touch or stroke them in a way that the other person finds  offensive .
 \textit{
	\begin{itemize}
	\item Stop pawing me, Giles!
	\item He pawed at my jacket with his free hand.
	\end{itemize}
}
\end{enumerate}

\section*{share}
{\large \color{blue}  shares  sharing  shared  }
\subsection*{Explain}
\begin{enumerate}
\item countable noun \\
A company's \textbf{shares} are the many equal parts into which its ownership is divided. Shares can be bought by people as an investment .
 \textit{
	\begin{itemize}
	\item This is why the British Airways' chairman has been so keen to buy shares in the rival
airline. .
	\item For some months the share price remained fairly static.
	\end{itemize}
}
\item verb \\
If you \textbf{share} something \textbf{with} another person, you both have it, use it, or occupy it. You can also  say that two people \textbf{share} something.
 \textit{
	\begin{itemize}
	\item ...the small income he had shared with his brother from his father's estate.
	\item Two Americans will share this year's Nobel Prize for Medicine.
	\item Scarce water resources are shared between states who cannot trust each other.
	\item Most hostel tenants would prefer single to shared rooms.
	\end{itemize}
}
\item verb \\
If you \textbf{share} a task , duty , or responsibility  \textbf{with} someone, you each carry out or accept part of it. You can also say that two people \textbf{share} something.
 \textit{
	\begin{itemize}
	\item You can find out whether they are prepared to share the cost of the flowers with
you.
	\item The republics have worked out a plan for sharing control of nuclear weapons.
	\end{itemize}
}
\item verb \\
If you \textbf{share} an experience  \textbf{with} someone, you have the same experience, often because you are with them at the time.
You can also say that two people \textbf{share} something.
 \textit{
	\begin{itemize}
	\item Yes, I want to share my life with you.
	\item I felt we both shared the same sense of loss, felt the same pain.
	\end{itemize}
}
\item verb \\
If you \textbf{share} someone's opinion , you agree with them.
 \textit{
	\begin{itemize}
	\item We share his view that business can be a positive force for change.
	\item Prosperity and economic success remain popular and broadly shared goals.
	\end{itemize}
}
\item verb \\
If one person or thing \textbf{shares} a quality or characteristic  \textbf{with} another, they have the same quality or characteristic. You can also say that two
people or things \textbf{share} something.
 \textit{
	\begin{itemize}
	\item ...newspapers which share similar characteristics with certain British newspapers.
	\item ...two groups who share a common language.
	\end{itemize}
}
\item verb \\
If you \textbf{share} something that you have \textbf{with} someone, you give some of it to them or let them use it.
 \textit{
	\begin{itemize}
	\item The village tribe is friendly and they share their water supply with you.
	\item Scientists now have to compete for funding, and do not share information among themselves.
	\item Toddlers are notoriously antisocial when it comes to sharing toys.
	\end{itemize}
}
\item verb \\
If you \textbf{share} something personal such as a thought or a piece of news  \textbf{with} someone, you tell them about it.
 \textit{
	\begin{itemize}
	\item It can be beneficial to share your feelings with someone you trust.
	\item The renowned film critic shares his thoughts on the week's new movies.
	\end{itemize}
}
\item verb \\
If you \textbf{share} something such as a message , photograph , or link , you post it on social  media so that other people can see it.
 \textit{
	\begin{itemize}
	\item With your phone you can take great pictures and share them instantly.
	\end{itemize}
}
\item countable noun \\
If something is divided or distributed among a number of different people or things, each of them has, or is responsible for, a \textbf{share}  \textbf{of} it.
 \textit{
	\begin{itemize}
	\item Sara also pays a share of the gas, electricity and phone bills.
	\item He is counting on winning seats and perhaps a share in the new government .
	\end{itemize}
}
\item countable noun \\
If you have or do your \textbf{share}  \textbf{of} something, you have or do an amount that seems  reasonable to you, or to other people.
 \textit{
	\begin{itemize}
	\item Women must receive their fair share of training for well-paid jobs.
	\item I have had more than my full share of adventures.
	\end{itemize}
}
\end{enumerate}

\section*{preposition}
{\large \color{blue}  prepositions  }
\subsection*{Explain}
\begin{enumerate}
\item countable noun \\
A \textbf{preposition} is a word such as 'by', 'for', 'into', or 'with' which usually has a noun group as
its object .
 \textit{
	\begin{itemize}
	\item Tick all the sentences that contain a preposition.
	\end{itemize}
}
\end{enumerate}

\section*{sleep}
{\large \color{blue}  sleeps  sleeping  slept  }
\subsection*{Explain}
\begin{enumerate}
\item uncountable noun \\
\textbf{Sleep} is the natural state of rest in which your eyes are closed , your body is inactive, and your mind does not think .
 \textit{
	\begin{itemize}
	\item They were exhausted from lack of sleep.
	\item Try and get some sleep.
	\item Be quiet and go to sleep.
	\item Often he would have bad dreams and cry out in his sleep.
	\end{itemize}
}
\item verb \\
When you \textbf{sleep} , you rest with your eyes closed and your mind and body inactive.
 \textit{
	\begin{itemize}
	\item During the car journey, the baby slept.
	\item I've not been able to sleep for the last few nights.
	\item ...a pool surrounded by sleeping sunbathers.
	\end{itemize}
}
\item countable noun \\
A \textbf{sleep} is a period of sleeping.
 \textit{
	\begin{itemize}
	\item I think he may be ready for a sleep soon.
	\end{itemize}
}
\item verb \\
If a building or room  \textbf{sleeps} a particular number of people, it has beds for that number of people.
 \textit{
	\begin{itemize}
	\item The villa sleeps 10 and costs £530 per person for two weeks.
	\end{itemize}
}
\item  \\
 get to sleep \textit{
	\begin{itemize}
	\end{itemize}
}
\item  \\
 lose sleep \textit{
	\begin{itemize}
	\end{itemize}
}
\item  \\
 sleep on it \textit{
	\begin{itemize}
	\end{itemize}
}
\item  \\
 put sth to sleep \textit{
	\begin{itemize}
	\end{itemize}
}
\end{enumerate}

\section*{princess}
{\large \color{blue}  princesses  }
\subsection*{Explain}
\begin{enumerate}
\item title noun \\
A \textbf{princess} is a female member of a royal family, usually the daughter of a king or queen or the wife of a prince.
 \textit{
	\begin{itemize}
	\item Princess Anne topped the guest list.
	\item ...Caroline Lindon, Princess of Monaco.
	\end{itemize}
}
\end{enumerate}

\section*{specialize}
{\large \color{blue}  specializes  specializing  specialized  }
\subsection*{Explain}
\begin{enumerate}
\item verb \\
If you \textbf{specialize in} a thing, you know a lot about it and concentrate a great  deal of your time and energy on it, especially in your work or when you are studying or training. You also use \textbf{specialize} to talk about a restaurant which concentrates on a particular type of food.
 \textit{
	\begin{itemize}
	\item ...a University professor who specializes in the history of the Russian empire.
	\item ...a Portuguese restaurant which specializes in seafood.
	\end{itemize}
}
\end{enumerate}

\section*{publicity}
{\large \color{blue}  }
\subsection*{Explain}
\begin{enumerate}
\item uncountable noun \\
\textbf{Publicity} is information or actions that are intended to attract the public's attention to someone or something.
 \textit{
	\begin{itemize}
	\item Much advance publicity was given to the talks.
	\item ...government publicity campaigns.
	\item It was all a publicity stunt.
	\end{itemize}
}
\item uncountable noun \\
When the news media and the public show a lot of interest in something, you can say that it is receiving  \textbf{publicity} .
 \textit{
	\begin{itemize}
	\item The case has generated enormous publicity in Brazil.
	\item ...the renewed publicity over the Casey affair.
	\end{itemize}
}
\end{enumerate}

\section*{submit}
{\large \color{blue}  submits  submitting  submitted  }
\subsection*{Explain}
\begin{enumerate}
\item verb \\
If you \textbf{submit}  \textbf{to} something, you unwillingly allow something to be done to you, or you do what someone wants , for example because you are not powerful enough to resist .
 \textit{
	\begin{itemize}
	\item In desperation, Mrs. Jones submitted to an operation on her right knee to relieve
the pain.
	\item If I submitted to their demands, they would not press the allegations.
	\end{itemize}
}
\item verb \\
If you \textbf{submit} a proposal , report , or request  \textbf{to} someone, you formally send it to them so that they can consider it or decide about it.
 \textit{
	\begin{itemize}
	\item They submitted their reports to the Chancellor yesterday.
	\item Head teachers yesterday submitted a claim for a 9 per cent pay rise.
	\end{itemize}
}
\end{enumerate}

\section*{robe}
{\large \color{blue}  robes  }
\subsection*{Explain}
\begin{enumerate}
\item countable noun \\
A \textbf{robe} is a loose piece of clothing which covers all of your body and reaches the ground . You can describe someone as wearing a \textbf{robe} or as wearing \textbf{robes} .
 \textit{
	\begin{itemize}
	\item The Pope knelt in his white robes before the simple altar.
	\item ...a fur-lined robe of green silk.
	\end{itemize}
}
\item countable noun \\
A \textbf{robe} is a piece of clothing, usually made of towelling , which people wear in the house , especially when they have just got up or had a bath .
 \textit{
	\begin{itemize}
	\item Ryle put on a robe and went down to the kitchen.
	\end{itemize}
}
\end{enumerate}

\section*{summarize}
{\large \color{blue}  summarizes  summarizing  summarized  }
\subsection*{Explain}
\begin{enumerate}
\item verb \\
If you \textbf{summarize} something, you give a summary of it.
 \textit{
	\begin{itemize}
	\item Table 3.1 summarizes the information given above.
	\item Basically, the article can be summarized in three sentences.
	\item To summarise, this is a clever approach to a common problem.
	\end{itemize}
}
\end{enumerate}

\section*{servant}
{\large \color{blue}  servants  }
\subsection*{Explain}
\begin{enumerate}
\item countable noun \\
A \textbf{servant} is someone who is employed to work at another person's home , for example as a cleaner or a gardener .
 \textit{
	\begin{itemize}
	\end{itemize}
}
\item countable noun \\
You can use \textbf{servant} to refer to someone or something that provides a service for people or can be used by them.
 \textit{
	\begin{itemize}
	\item Like any other public servants, police must respond to public demand.
	\item The question is whether technology is going to be our servant or our master.
	\end{itemize}
}
\end{enumerate}

\section*{sympathize}
{\large \color{blue}  sympathizes  sympathizing  sympathized  }
\subsection*{Explain}
\begin{enumerate}
\item verb \\
If you \textbf{sympathize} with someone who is in a bad  situation , you show that you are sorry for them.
 \textit{
	\begin{itemize}
	\item I must tell you how much I sympathize with you for your loss, Professor.
	\item He would sympathize but he wouldn't understand.
	\end{itemize}
}
\item verb \\
If you \textbf{sympathize}  \textbf{with} someone's feelings , you understand them and are not critical of them.
 \textit{
	\begin{itemize}
	\item He sympathised with her over her frustration at not being chosen for the team.
	\item He liked Max, and sympathized with his ambitions.
	\end{itemize}
}
\item verb \\
If you \textbf{sympathize}  \textbf{with} a proposal or action, you approve of it and are willing to support it.
 \textit{
	\begin{itemize}
	\item Most of the people living there sympathized with the guerrillas.
	\end{itemize}
}
\end{enumerate}

\section*{article}
{\large \color{blue}  articles  }
\subsection*{Explain}
\begin{enumerate}
\item countable noun \\
An \textbf{article} is a piece of writing that is published in a newspaper or magazine.
 \textit{
	\begin{itemize}
	\item ...a newspaper article.
	\item ...a travel article.
	\item According to an article in The Economist the drug could have side effects.
	\item ...Canning's article about the Buxton Festival.
	\end{itemize}
}
\item countable noun \\
You can refer to objects as \textbf{articles} of some kind.
 \textit{
	\begin{itemize}
	\item ...articles of clothing.
	\item He had stripped the house of all articles of value.
	\item ...household articles.
	\end{itemize}
}
\item  \\
 the genuine article \textit{
	\begin{itemize}
	\end{itemize}
}
\item countable noun \\
An \textbf{article}  \textbf{of} a formal  agreement or document is a section of it which deals with a particular point.
 \textit{
	\begin{itemize}
	\item The country appears to be violating several articles of the convention.
	\item ...Article 50 of the U.N. charter.
	\end{itemize}
}
\item plural noun \\
Someone who is in \textbf{articles} is being trained as a lawyer or accountant by a firm with whom they have a written agreement.
 \textit{
	\begin{itemize}
	\item 44 per cent of those admitted to articles were women.
	\end{itemize}
}
\item countable noun \\
In grammar , an \textbf{article} is a kind of determiner. In English, 'a' and 'an' are called  \textbf{the indefinite article} , and 'the' is called \textbf{the definite article} .
 \textit{
	\begin{itemize}
	\end{itemize}
}
\end{enumerate}

\section*{bend}
{\large \color{blue}  bends  bending  bent  }
\subsection*{Explain}
\begin{enumerate}
\item verb \\
When you \textbf{bend} , you move the top part of your body downwards and forwards . Plants and trees also  \textbf{bend} .
 \textit{
	\begin{itemize}
	\item I bent over and kissed her cheek.
	\item Turn the pot if the plants show signs of bending towards the light.
	\item She bent and picked up a plastic bucket.
	\item She was bent over the sink washing the dishes.
	\end{itemize}
}
\item verb \\
When you \textbf{bend} your head, you move your head forwards and downwards.
 \textit{
	\begin{itemize}
	\item Rick appeared, bending his head a little to clear the top of the door.
	\end{itemize}
}
\item verb \\
When you \textbf{bend} a part of your body such as your arm or leg, or when it \textbf{bends} , you change its position so that it is no longer straight .
 \textit{
	\begin{itemize}
	\item These cruel devices are designed to stop prisoners bending their legs.
	\item As you walk faster, you will find the arms bend naturally and more quickly.
	\end{itemize}
}
\item verb \\
If you \textbf{bend} something that is flat or straight, you use force to make it curved or to put an
 angle in it.
 \textit{
	\begin{itemize}
	\item Bend the bar into a horseshoe.
	\item She'd cut a jagged hole in the tin, bending a knife in the process.
	\end{itemize}
}
\item verb \\
When a road, beam of light, or other long thin thing \textbf{bends} , or when something \textbf{bends} it, it changes direction to form a curve or angle.
 \textit{
	\begin{itemize}
	\item The road bent slightly to the right.
	\item Glass bends light of different colours by different amounts.
	\end{itemize}
}
\item countable noun \\
A \textbf{bend} in a road, pipe , or other long thin object is a curve or angle in it.
 \textit{
	\begin{itemize}
	\item The crash occurred on a sharp bend.
	\item ...an historic town nestling in a bend of the river.
	\end{itemize}
}
\item verb \\
If someone \textbf{bends}  \textbf{to} your wishes , they believe or do something different, usually when they do not want to.
 \textit{
	\begin{itemize}
	\item Congress has to bend to his will.
	\item Do you think she's likely to bend on her attitude to Europe?
	\end{itemize}
}
\item verb \\
If you \textbf{bend} rules or laws, you interpret them in a way that allows you to do something they would not normally allow you to do.
 \textit{
	\begin{itemize}
	\item A minority of officers were prepared to bend the rules.
	\end{itemize}
}
\item verb \\
If you \textbf{bend} the truth or \textbf{bend} the facts , you say something that is not exactly  true .
 \textit{
	\begin{itemize}
	\item Sometimes we bend the truth a little in order to spare them the pain of the real
facts.
	\end{itemize}
}
\item  \\
 to bend over backwards \textit{
	\begin{itemize}
	\end{itemize}
}
\item  \\
 drive sb round the bend \textit{
	\begin{itemize}
	\end{itemize}
}
\item  \\
 round the bend \textit{
	\begin{itemize}
	\end{itemize}
}
\end{enumerate}

\section*{blaze}
{\large \color{blue}  blazes  blazing  blazed  }
\subsection*{Explain}
\begin{enumerate}
\item verb \\
When a fire \textbf{blazes} , it burns strongly and brightly.
 \textit{
	\begin{itemize}
	\item Three people died as wreckage blazed, and rescuers fought to release trapped drivers.
	\item The log fire was blazing merrily.
	\item ...a blazing fire.
	\end{itemize}
}
\item countable noun \\
A \textbf{blaze} is a large fire which is difficult to control and which destroys a lot of things.
 \textit{
	\begin{itemize}
	\item Two firefighters were hurt in a blaze which swept through a tower block last night.
	\end{itemize}
}
\item verb \\
If something \textbf{blazes}  \textbf{with} light or colour, it is extremely bright.
 \textbf{Blaze} is also a noun .
 \textit{
	\begin{itemize}
	\item The gardens blazed with colour.
	\item I wanted the front garden to be a blaze of colour.
	\end{itemize}
}
\item verb \\
If someone's eyes \textbf{are blazing}  \textbf{with} an emotion , or if an emotion \textbf{is blazing} in their eyes, their eyes look very bright because they are feeling that emotion so strongly.
 \textit{
	\begin{itemize}
	\item He got to his feet and his dark eyes were blazing with anger.
	\item Eva stood up and indignation blazed in her eyes.
	\item His eyes blazed intently into mine.
	\item Miss Turner turned blazing eyes on the victim.
	\end{itemize}
}
\item singular noun \\
\textbf{A blaze of}  publicity or attention is a great amount of it.
 \textit{
	\begin{itemize}
	\item He was arrested in a blaze of publicity.
	\item ...the sporting career that began in a blaze of glory.
	\end{itemize}
}
\item verb \\
If guns  \textbf{blaze} , or \textbf{blaze}  \textbf{away} , they fire continuously, making a lot of noise .
 \textit{
	\begin{itemize}
	\item Guns were blazing, flares going up and the sky was lit up all around.
	\item She took the gun and blazed away with calm and deadly accuracy.
	\end{itemize}
}
\item  \\
 to blaze a trail \textit{
	\begin{itemize}
	\end{itemize}
}
\end{enumerate}

\section*{blunder}
{\large \color{blue}  blunders  blundering  blundered  }
\subsection*{Explain}
\begin{enumerate}
\item countable noun \\
A \textbf{blunder} is a stupid or careless mistake.
 \textit{
	\begin{itemize}
	\item I think he made a tactical blunder by announcing it so far ahead of time.
	\end{itemize}
}
\item verb \\
If you \textbf{blunder} , you make a stupid or careless mistake.
 \textit{
	\begin{itemize}
	\item No doubt I had blundered again.
	\item You're a blundering fool.
	\end{itemize}
}
\item verb \\
If you \textbf{blunder}  \textbf{into} a dangerous or difficult  situation , you get involved in it by mistake.
 \textit{
	\begin{itemize}
	\item People wanted to know how they had blundered into war, and how to avoid it in future.
	\end{itemize}
}
\item verb \\
If you \textbf{blunder}  somewhere , you move there in a clumsy and careless way.
 \textit{
	\begin{itemize}
	\item He had blundered into the table, upsetting the flowers.
	\end{itemize}
}
\end{enumerate}

\section*{cloak}
{\large \color{blue}  cloaks  cloaking  cloaked  }
\subsection*{Explain}
\begin{enumerate}
\item countable noun \\
A \textbf{cloak} is a long, loose , sleeveless piece of clothing which people used to wear over their other clothes when they went out.
 \textit{
	\begin{itemize}
	\end{itemize}
}
\item singular noun \\
A \textbf{cloak of} something such as mist or snow completely covers and hides something.
 \textit{
	\begin{itemize}
	\item Today most of England will be under a cloak of thick mist.
	\end{itemize}
}
\item singular noun \\
If you refer to something as a \textbf{cloak} , you mean that it is intended to hide the truth about something.
 \textit{
	\begin{itemize}
	\item Preparations for the wedding were made under a cloak of secrecy.
	\end{itemize}
}
\item verb \\
To \textbf{cloak} something means to cover it or hide it.
 \textit{
	\begin{itemize}
	\item ...the decision to cloak major tourist attractions in unsightly hoardings.
	\item A fire could have been deliberately started to cloak small coordinated troop movements.
	\item The beautiful sweeping coastline was cloaked in mist.
	\end{itemize}
}
\end{enumerate}

\section*{boil}
{\large \color{blue}  boils  boiling  boiled  }
\subsection*{Explain}
\begin{enumerate}
\item verb \\
When a hot liquid \textbf{boils} or when you \textbf{boil} it, bubbles appear in it and it starts to change into steam or vapour.
 \textit{
	\begin{itemize}
	\item I stood in the kitchen, waiting for the water to boil.
	\item Boil the water in the saucepan and add the sage.
	\item ...a saucepan of boiling water.
	\end{itemize}
}
\item verb \\
When you \textbf{boil} a kettle or pan , or put it on to \textbf{boil} , you heat the water inside it until it boils.
 \textit{
	\begin{itemize}
	\item He had nothing to do but boil the kettle and make the tea.
	\item Marianne put the kettle on to boil.
	\end{itemize}
}
\item verb \\
When a kettle or pan \textbf{is boiling} , the water inside it has reached boiling point.
 \textit{
	\begin{itemize}
	\item Is the kettle boiling?
	\end{itemize}
}
\item verb \\
When you \textbf{boil}  food , or when it \textbf{boils} , it is cooked in boiling water.
 \textit{
	\begin{itemize}
	\item Boil the chick peas, add garlic and lemon juice.
	\item I'd peel potatoes and put them on to boil.
	\item ...boiled eggs and toast.
	\end{itemize}
}
\item verb \\
If you \textbf{are boiling}  \textbf{with}  anger , you are very angry.
 \textit{
	\begin{itemize}
	\item I used to be all sweetness and light on the outside, but inside I would be boiling
with rage.
	\end{itemize}
}
\item countable noun \\
A \textbf{boil} is a red, painful swelling on your skin, which contains a thick  yellow liquid called  pus .
 \textit{
	\begin{itemize}
	\end{itemize}
}
\item  \\
 bring to the boil/come to the boil \textit{
	\begin{itemize}
	\end{itemize}
}
\end{enumerate}

\section*{cock}
{\large \color{blue}  cocks  cocking  cocked  }
\subsection*{Explain}
\begin{enumerate}
\item countable noun \\
A \textbf{cock} is an adult male chicken .
 \textit{
	\begin{itemize}
	\item The cock was announcing the start of a new day.
	\end{itemize}
}
\item countable noun \\
You refer to a male bird, especially a male game bird, as a \textbf{cock} when you want to distinguish it from a female bird.
 \textit{
	\begin{itemize}
	\item ...a cock pheasant.
	\end{itemize}
}
\item countable noun \\
A man's \textbf{cock} is his penis .
 \textit{
	\begin{itemize}
	\end{itemize}
}
\item verb \\
If you \textbf{cock} a part of your body in a particular direction , you lift it or point it in that direction.
 \textit{
	\begin{itemize}
	\item He paused and cocked his head as if listening.
	\item The Brigadier thought about this for a moment, head cocked to one side.
	\end{itemize}
}
\item verb \\
If someone \textbf{cocks} their ear , they try very hard to hear something from a particular direction.
 \textit{
	\begin{itemize}
	\item He suddenly cocked an ear and listened.
	\item All ears were cocked for the footsteps on the stairs.
	\end{itemize}
}
\item verb \\
When someone \textbf{cocks} a gun , they set a small device in the gun so that it is ready to fire.
 \textit{
	\begin{itemize}
	\item His hands were too weak to cock his revolver.
	\end{itemize}
}
\end{enumerate}

\section*{cheer}
{\large \color{blue}  cheers  cheering  cheered  }
\subsection*{Explain}
\begin{enumerate}
\item verb \\
When people \textbf{cheer} , they shout loudly to show their approval or to encourage someone who is doing something such as taking part in a game .
 \textbf{Cheer} is also a noun .
 \textit{
	\begin{itemize}
	\item We all cheered as they drove up the street.
	\item ...2,000 Villa fans who cheered him into his goal.
	\item ...the Irish Americans who came to the park to cheer for their boys.
	\item Cheering crowds lined the route.
	\item The colonel was rewarded with a resounding cheer from the men.
	\end{itemize}
}
\item verb \\
If you \textbf{are cheered}  \textbf{by} something, it makes you happier or less worried .
 \textit{
	\begin{itemize}
	\item Stephen noticed that the people around him looked cheered by his presence.
	\item The weather was perfect but it did nothing to cheer him.
	\end{itemize}
}
\item uncountable noun \\
\textbf{Cheer} is a feeling of cheerfulness.
 \textit{
	\begin{itemize}
	\item They were impressed by his steadfast good cheer.
	\item A late goal brought some cheer to the home crowd.
	\end{itemize}
}
\item convention \\
People sometimes  say ' \textbf{Cheers} ' to each other just before they drink an alcoholic drink.
 \textit{
	\begin{itemize}
	\end{itemize}
}
\item convention \\
Some people say ' \textbf{Cheers} ' as a way of saying ' thank you' or 'goodbye'.
 \textit{
	\begin{itemize}
	\end{itemize}
}
\end{enumerate}

\section*{computer}
{\large \color{blue}  computers  }
\subsection*{Explain}
\begin{enumerate}
\item countable noun \\
A \textbf{computer} is an electronic machine that can store and deal with large amounts of information .
 \textit{
	\begin{itemize}
	\item The data are then fed into a computer.
	\item The company installed a $650,000 computer system.
	\item It's done on a computer?
	\item The car was designed by computer.
	\end{itemize}
}
\end{enumerate}

\section*{condense}
{\large \color{blue}  condenses  condensing  condensed  }
\subsection*{Explain}
\begin{enumerate}
\item verb \\
If you \textbf{condense} something, especially a piece of writing or speech , you make it shorter, usually by including only the most important parts.
 \textit{
	\begin{itemize}
	\item We have learnt how to condense serious messages into short, self-contained sentences.
	\item The English translation may have been condensed into a single more readable book.
	\end{itemize}
}
\item verb \\
When a gas or vapour  \textbf{condenses} , or \textbf{is condensed} , it changes into a liquid.
 \textit{
	\begin{itemize}
	\item Water vapour condenses to form clouds.
	\item The compressed gas is cooled and condenses into a liquid.
	\item As the air rises it becomes colder and moisture condenses out of it.
	\end{itemize}
}
\end{enumerate}

\section*{crow}
{\large \color{blue}  crows  crowing  crowed  }
\subsection*{Explain}
\begin{enumerate}
\item countable noun \\
A \textbf{crow} is a large black bird which makes a loud , harsh  noise .
 \textit{
	\begin{itemize}
	\end{itemize}
}
\item verb \\
When a cock \textbf{crows} , it makes a loud sound, often early in the morning .
 \textit{
	\begin{itemize}
	\item The cock crows and the dawn chorus begins.
	\end{itemize}
}
\item verb \\
If you say that someone \textbf{is crowing}  \textbf{about} something they have achieved or are pleased about, you disapprove of them because they keep  telling people proudly about it.
 \textit{
	\begin{itemize}
	\item Edwards is already crowing about his assured victory.
	\item We've seen them all crowing that the movement is dead.
	\end{itemize}
}
\item verb \\
If someone \textbf{crows} , they make happy sounds or say something happily.
 \textit{
	\begin{itemize}
	\item She was crowing with delight.
	\item 'I'm not sure I've ever driven a better lap,' crowed a delighted Irvine.
	\end{itemize}
}
\item  \\
 as the crow flies \textit{
	\begin{itemize}
	\end{itemize}
}
\end{enumerate}

\section*{curl}
{\large \color{blue}  curls  curling  curled  }
\subsection*{Explain}
\begin{enumerate}
\item countable noun \\
If you have \textbf{curls} , your hair is in the form of tight curves and spirals.
 \textit{
	\begin{itemize}
	\item ...the little girl with blonde curls.
	\item A curl of black hair fell loosely across his forehead.
	\end{itemize}
}
\item uncountable noun \\
If your hair has \textbf{curl} , it is full of curls.
 \textit{
	\begin{itemize}
	\item Dry curly hair naturally for maximum curl and shine.
	\end{itemize}
}
\item verb \\
If your hair \textbf{curls} or if you \textbf{curl} it, it is full of curls.
 \textit{
	\begin{itemize}
	\item She has hair that refuses to curl.
	\item Maria had curled her hair for the event.
	\item Afro hair is short and tightly curled.
	\end{itemize}
}
\item countable noun \\
A \textbf{curl}  \textbf{of} something is a piece or quantity of it that is curved or spiral in shape.
 \textit{
	\begin{itemize}
	\item A thin curl of smoke rose from a rusty stove.
	\item ...curls of lemon peel.
	\end{itemize}
}
\item verb \\
If your toes , fingers , or other parts of your body \textbf{curl} , or if you \textbf{curl} them, they form a curved or round shape.
 \textit{
	\begin{itemize}
	\item His fingers curled gently round her wrist.
	\item Raise one foot, curl the toes and point the foot downwards.
	\item She sat with her legs curled under her.
	\end{itemize}
}
\item verb \\
If something \textbf{curls}  somewhere , or if you \textbf{curl} it there, it moves there in a spiral or curve.
 \textit{
	\begin{itemize}
	\item Smoke was curling up the chimney.
	\item He curled the ball into the net.
	\end{itemize}
}
\item verb \\
If a person or animal \textbf{curls into} a ball , they move into a position in which their body makes a rounded shape.
 \textbf{Curl up}  means the same as curl .
 \textit{
	\begin{itemize}
	\item He wanted to curl into a tiny ball.
	\item The kitten was curled on a cushion on the sofa.
	\item In colder weather, your cat will curl up into a tight, heat-conserving ball.
	\item She curled up next to him.
	\item He was asleep there, curled up in the fetal position.
	\end{itemize}
}
\item verb \\
When a leaf , a piece of paper , or another flat  object  \textbf{curls} , its edges  bend towards the centre .
 \textbf{Curl up} means the same as curl .
 \textit{
	\begin{itemize}
	\item The rose leaves have curled because of an attack by grubs.
	\item The corners of the lino were curling up.
	\end{itemize}
}
\item ergative verb \\
If you \textbf{curl} your lip , you raise your upper lip slightly at one side , as a way of showing  anger or contempt .
 \textit{
	\begin{itemize}
	\item He curled his upper lip in a show of scepticism.
	\item Her lip curled with scorn.
	\end{itemize}
}
\end{enumerate}

\section*{culture}
{\large \color{blue}  cultures  culturing  cultured  }
\subsection*{Explain}
\begin{enumerate}
\item uncountable noun \\
\textbf{Culture} consists of activities such as the arts and philosophy , which are considered to be important for the development of civilization and of people's minds .
 \textit{
	\begin{itemize}
	\item There is just not enough fun and frivolity in culture today.
	\item ...aspects of popular culture.
	\item ...France's Minister of Culture and Education.
	\end{itemize}
}
\item countable noun \\
A \textbf{culture} is a particular society or civilization, especially considered in relation to its beliefs, way of life, or art.
 \textit{
	\begin{itemize}
	\item ...people from different cultures.
	\item We live in a culture that is competitive.
	\end{itemize}
}
\item countable noun \\
The \textbf{culture} of a particular organization or group consists of the habits of the people in it and the way they generally  behave .
 \textit{
	\begin{itemize}
	\item The benefits system creates a culture of dependency.
	\item Banks need to change their culture to improve efficiency and service.
	\end{itemize}
}
\item countable noun \\
In science , a \textbf{culture} is a group of bacteria or cells which are grown, usually in a laboratory as part of an experiment .
 \textit{
	\begin{itemize}
	\item ...a culture of human cells.
	\item ...a number of tissue culture experiments.
	\end{itemize}
}
\item verb \\
In science, to \textbf{culture} a group of bacteria or cells means to grow them, usually in a laboratory as part
of an experiment.
 \textit{
	\begin{itemize}
	\item To confirm the diagnosis, the hospital laboratory must culture a colony of bacteria.
	\item ...cultured human blood cells.
	\end{itemize}
}
\end{enumerate}

\section*{defy}
{\large \color{blue}  defies  defying  defied  }
\subsection*{Explain}
\begin{enumerate}
\item verb \\
If you \textbf{defy} someone or something that is trying to make you behave in a particular way, you refuse to obey them and behave in that way.
 \textit{
	\begin{itemize}
	\item This was the first (and last) time that I dared to defy my mother.
	\item Nearly eleven-thousand people have been arrested for defying the ban on street trading.
	\end{itemize}
}
\item verb \\
If you \textbf{defy} someone \textbf{to} do something, you challenge them to do it when you think that they will be unable to do it or too frightened to do it.
 \textit{
	\begin{itemize}
	\item I defy you to read this book and not feel motivated to change.
	\item He looked at me as if he was defying me to argue.
	\end{itemize}
}
\item verb \\
If something \textbf{defies}  description or understanding , it is so strange , extreme , or surprising that it is almost impossible to understand or explain .
 \textit{
	\begin{itemize}
	\item The skill of the craftsman who made it defies description.
	\item It's a devastating and barbaric act that defies all comprehension.
	\end{itemize}
}
\item  \\
 defy one's age / the years \textit{
	\begin{itemize}
	\end{itemize}
}
\end{enumerate}

\section*{dialect}
{\large \color{blue}  dialects  }
\subsection*{Explain}
\begin{enumerate}
\item variable noun \\
A \textbf{dialect} is a form of a language that is spoken in a particular area.
 \textit{
	\begin{itemize}
	\item In the fifties, many Italians spoke only local dialect.
	\item They began to speak rapidly in dialect.
	\end{itemize}
}
\end{enumerate}

\section*{deteriorate}
{\large \color{blue}  deteriorates  deteriorating  deteriorated  }
\subsection*{Explain}
\begin{enumerate}
\item verb \\
If something \textbf{deteriorates} , it becomes worse in some way.
 \textit{
	\begin{itemize}
	\item There are fears that the situation might deteriorate into full-scale war.
	\item The weather conditions are deteriorating.
	\item Grant's health steadily deteriorated.
	\end{itemize}
}
\end{enumerate}

\section*{direction}
{\large \color{blue}  directions  }
\subsection*{Explain}
\begin{enumerate}
\item variable noun \\
A \textbf{direction} is the general line that someone or something is moving or pointing in.
 \textit{
	\begin{itemize}
	\item St Andrews was ten miles in the opposite direction.
	\item He drove off in the direction of Larry's shop.
	\item Civilians were fleeing in all directions as soldiers yelled at them to get off the
streets.
	\item The instruments will register every change of direction or height.
	\end{itemize}
}
\item variable noun \\
A \textbf{direction} is the general way in which something develops or progresses .
 \textit{
	\begin{itemize}
	\item They threatened to walk out if the party did not change direction.
	\item I've never done any sustained writing, but that might be one of my next directions.
	\end{itemize}
}
\item plural noun \\
\textbf{Directions} are instructions that tell you what to do, how to do something, or how to get  somewhere .
 \textit{
	\begin{itemize}
	\item I should know by now not to throw away the directions until we've finished cooking.
	\item He proceeded to give Dan directions to the computer room.
	\end{itemize}
}
\item uncountable noun \\
The \textbf{direction} of a film, play, or television  programme is the work that the director does while it is being made.
 \textit{
	\begin{itemize}
	\item His failures underline the difference between theatre and film direction.
	\end{itemize}
}
\end{enumerate}

\section*{disperse}
{\large \color{blue}  disperses  dispersing  dispersed  }
\subsection*{Explain}
\begin{enumerate}
\item verb \\
When something \textbf{disperses} or when you \textbf{disperse} it, it spreads over a wide area.
 \textit{
	\begin{itemize}
	\item The oil appeared to be dispersing.
	\item The intense currents disperse the sewage.
	\item Because the town sits in a valley, air pollution is not easily dispersed.
	\end{itemize}
}
\item verb \\
When a group of people \textbf{disperses} or when someone \textbf{disperses} them, the group splits up and the people leave in different directions .
 \textit{
	\begin{itemize}
	\item Police fired shots and used teargas to disperse the demonstrators.
	\item The crowd dispersed peacefully after prayers.
	\end{itemize}
}
\end{enumerate}

\section*{drawback}
{\large \color{blue}  drawbacks  }
\subsection*{Explain}
\begin{enumerate}
\item countable noun \\
A \textbf{drawback} is an aspect of something or someone that makes them less acceptable than they would otherwise be.
 \textit{
	\begin{itemize}
	\item He felt the apartment's only drawback was that it was too small.
	\end{itemize}
}
\end{enumerate}

\section*{embark}
{\large \color{blue}  embarks  embarking  embarked  }
\subsection*{Explain}
\begin{enumerate}
\item verb \\
If you \textbf{embark}  \textbf{on} something new, difficult , or exciting , you start doing it.
 \textit{
	\begin{itemize}
	\item He's embarking on a new career as a writer.
	\item The government embarked on a programme of radical economic reform.
	\end{itemize}
}
\item verb \\
When someone \textbf{embarks}  \textbf{on} a ship, they go on board before the start of a journey .
 \textit{
	\begin{itemize}
	\item They travelled to Portsmouth, where they embarked on the battle cruiser HMS Renown.
	\item Bob ordered brigade HQ to embark.
	\end{itemize}
}
\end{enumerate}

\section*{essay}
{\large \color{blue}  essays  essaying  essayed  }
\subsection*{Explain}
\begin{enumerate}
\item countable noun \\
An \textbf{essay} is a short piece of writing on one particular subject written by a student .
 \textit{
	\begin{itemize}
	\item We asked Jason to write an essay about his hometown and about his place in it.
	\end{itemize}
}
\item countable noun \\
An \textbf{essay} is a short piece of writing on one particular subject that is written by a writer for publication .
 \textit{
	\begin{itemize}
	\item ...Thomas Malthus's essay on population.
	\end{itemize}
}
\item verb \\
If you \textbf{essay} something, you try to do it.
 \textbf{Essay} is also a noun .
 \textit{
	\begin{itemize}
	\item Sinclair essayed a smile but it could hardly have been rated as a success.
	\item His first essay in running a company was a notoriously tough undertaking.
	\end{itemize}
}
\end{enumerate}

\section*{emigrate}
{\large \color{blue}  emigrates  emigrating  emigrated  }
\subsection*{Explain}
\begin{enumerate}
\item verb \\
If you \textbf{emigrate} , you leave your own country to live in another country.
 \textit{
	\begin{itemize}
	\item He emigrated to Belgium.
	\item They planned to emigrate.
	\end{itemize}
}
\end{enumerate}

\section*{fire}
{\large \color{blue}  fires  firing  fired  }
\subsection*{Explain}
\begin{enumerate}
\item uncountable noun \\
\textbf{Fire} is the hot, bright flames produced by things that are burning.
 \textit{
	\begin{itemize}
	\item They saw a big flash and a huge ball of fire reaching hundreds of feet into the sky.
	\item Many students were trapped by smoke and fire on an upper floor.
	\end{itemize}
}
\item variable noun \\
\textbf{Fire} or a \textbf{fire} is an occurrence of uncontrolled burning which destroys buildings, forests, or other things.
 \textit{
	\begin{itemize}
	\item 87 people died in the fire.
	\item A forest fire is sweeping across portions of north Maine this evening.
	\item Much of historic Rennes was destroyed by fire in 1720.
	\end{itemize}
}
\item countable noun \\
A \textbf{fire} is a burning pile of wood, coal, or other fuel that you make, for example to use for heat, light, or
cooking.
 \textit{
	\begin{itemize}
	\item There was a fire in the grate.
	\item After the killing, he calmly lit a fire to destroy evidence.
	\end{itemize}
}
\item countable noun \\
A \textbf{fire} is a device that uses electricity or gas to give out heat and warm a room.
 \textit{
	\begin{itemize}
	\item The gas fire was still alight.
	\item She switched on one bar of the electric fire.
	\end{itemize}
}
\item verb \\
When a pot or clay object \textbf{is fired} , it is heated at a high temperature in a special oven , as part of the process of making it.
 \textit{
	\begin{itemize}
	\item After the pot is dipped in this mixture, it is fired.
	\item I have watched the potters fire and paint their bowls and vases.
	\end{itemize}
}
\item verb \\
When the engine of a motor vehicle \textbf{fires} , an electrical spark is produced which causes the fuel to burn and the engine to work.
 \textit{
	\begin{itemize}
	\item The engine fired and we moved off.
	\end{itemize}
}
\item verb \\
If a machine \textbf{is fired with} a particular fuel, it operates by means of that fuel.
 \textit{
	\begin{itemize}
	\item The engines were fired with coal and needed water to keep the steam up.
	\end{itemize}
}
\item verb \\
If you \textbf{fire} someone \textbf{with} enthusiasm, you make them feel very enthusiastic . If you \textbf{fire} someone's imagination, you make them feel interested and excited .
 \textit{
	\begin{itemize}
	\item ...the potential to fire the imagination of an entire generation.
	\item It was Allen who fired this rivalry with real passion.
	\item Both his grandfathers were fired with an enthusiasm for public speaking.
	\item By Monday, the President had returned, apparently fired with new determination.
	\end{itemize}
}
\item uncountable noun \\
You can use \textbf{fire} to refer in an approving way to someone's energy and enthusiasm.
 \textit{
	\begin{itemize}
	\item I went to hear him speak and was very impressed. He seemed so full of fire.
	\item His punishing schedule seemed to dim his fire at times.
	\end{itemize}
}
\item  \\
 to catch fire \textit{
	\begin{itemize}
	\end{itemize}
}
\item  \\
 to catch fire \textit{
	\begin{itemize}
	\end{itemize}
}
\item  \\
 to fight fire with fire \textit{
	\begin{itemize}
	\end{itemize}
}
\item  \\
 fire in your belly \textit{
	\begin{itemize}
	\end{itemize}
}
\item  \\
 on fire \textit{
	\begin{itemize}
	\end{itemize}
}
\item  \\
 on fire \textit{
	\begin{itemize}
	\end{itemize}
}
\item  \\
 to play with fire \textit{
	\begin{itemize}
	\end{itemize}
}
\item  \\
 to set fire to something \textit{
	\begin{itemize}
	\end{itemize}
}
\end{enumerate}

\section*{erupt}
{\large \color{blue}  erupts  erupting  erupted  }
\subsection*{Explain}
\begin{enumerate}
\item verb \\
When a volcano  \textbf{erupts} , it throws out a lot of hot , melted rock called lava, as well as ash and steam.
 \textit{
	\begin{itemize}
	\item The volcano erupted, devastating a large area.
	\item Scientists say the volcano could erupt again soon.
	\end{itemize}
}
\item verb \\
If violence or fighting  \textbf{erupts} , it suddenly begins or gets  worse in an unexpected , violent way.
 \textit{
	\begin{itemize}
	\item Heavy fighting erupted there today after a two-day cease-fire.
	\item Violence erupted as the boys were driven away in two police vans.
	\end{itemize}
}
\item verb \\
When people in a place suddenly become angry or violent, you can say that they \textbf{erupt} or that the place \textbf{erupts} .
 \textit{
	\begin{itemize}
	\item In Los Angeles, the neighborhood known as Watts erupted into riots.
	\item This region which had been relatively calm erupted in violence again this spring.
	\end{itemize}
}
\item verb \\
You say that someone \textbf{erupts} when they suddenly have a change in mood , usually becoming  quite  noisy .
 \textit{
	\begin{itemize}
	\item Then, without warning, she erupts into laughter.
	\item The crowd erupted in fury and wouldn't let the competition continue.
	\end{itemize}
}
\item verb \\
If your skin \textbf{erupts} , sores or spots suddenly appear there.
 \textit{
	\begin{itemize}
	\item At the end of the second week, my skin erupted in pimples.
	\end{itemize}
}
\end{enumerate}

\section*{flame}
{\large \color{blue}  flames  flaming  flamed  }
\subsection*{Explain}
\begin{enumerate}
\item variable noun \\
A \textbf{flame} is a hot bright stream of burning gas that comes from something that is burning.
 \textit{
	\begin{itemize}
	\item The heat from the flames was so intense that roads melted.
	\item ...a huge ball of flame.
	\end{itemize}
}
\item countable noun \\
A \textbf{flame} is an email message which severely criticizes or attacks someone.
 \textbf{Flame} is also a verb .
 \textit{
	\begin{itemize}
	\item The best way to respond to a flame is to ignore it.
	\item Ever been flamed?
	\end{itemize}
}
\item verb \\
If someone's face  \textbf{flames} , it suddenly  looks red, usually because they are angry.
 \textit{
	\begin{itemize}
	\item Her cheeks flamed an angry red.
	\item Christopher's listening face flamed at the contempt.
	\end{itemize}
}
\item countable noun \\
You can refer to a feeling of passion or anger as a \textbf{flame}  \textbf{of} passion or a \textbf{flame}  \textbf{of} anger.
 \textit{
	\begin{itemize}
	\item ...that burning flame of love.
	\end{itemize}
}
\item  \\
 to burst into flames \textit{
	\begin{itemize}
	\end{itemize}
}
\item  \\
 to fan the flames \textit{
	\begin{itemize}
	\end{itemize}
}
\item  \\
 go up in flames \textit{
	\begin{itemize}
	\end{itemize}
}
\item  \\
 in flames \textit{
	\begin{itemize}
	\end{itemize}
}
\end{enumerate}

\section*{escalate}
{\large \color{blue}  escalates  escalating  escalated  }
\subsection*{Explain}
\begin{enumerate}
\item verb \\
If a bad situation \textbf{escalates} or if someone or something \textbf{escalates} it, it becomes greater in size, seriousness, or intensity.
 \textit{
	\begin{itemize}
	\item Both unions and management fear the dispute could escalate.
	\item The protests escalated into five days of rioting.
	\item Defeat could cause one side or other to escalate the conflict.
	\end{itemize}
}
\end{enumerate}

\section*{ham}
{\large \color{blue}  hams  hamming  hammed  }
\subsection*{Explain}
\begin{enumerate}
\item variable noun \\
\textbf{Ham} is meat from the top of the back leg of a pig, specially treated so that it can be kept for a long period of time.
 \textit{
	\begin{itemize}
	\item ...a huge baked ham.
	\item ...ham sandwiches.
	\item ...a dozen slices of ham.
	\end{itemize}
}
\item countable noun \\
A \textbf{ham} is a person whose hobby consists of using special radio equipment to talk to other people with the same hobby, often people who are in other countries.
 \textit{
	\begin{itemize}
	\item I became a ham radio operator at the age of eleven.
	\end{itemize}
}
\item countable noun \\
A \textbf{ham} actor is someone who acts badly , exaggerating every emotion and gesture.
 \textit{
	\begin{itemize}
	\end{itemize}
}
\item  \\
 ham sth up \textit{
	\begin{itemize}
	\end{itemize}
}
\end{enumerate}

\section*{evolve}
{\large \color{blue}  evolves  evolving  evolved  }
\subsection*{Explain}
\begin{enumerate}
\item verb \\
When animals or plants \textbf{evolve} , they gradually change and develop into different forms.
 \textit{
	\begin{itemize}
	\item The bright plumage of many male birds has evolved to attract females.
	\item Maize evolved from a wild grass in Mexico.
	\item ...when amphibians evolved into reptiles.
	\end{itemize}
}
\item verb \\
If something \textbf{evolves} or you \textbf{evolve} it, it gradually develops over a period of time into something different and usually
more advanced .
 \textit{
	\begin{itemize}
	\item ...a tiny airline which eventually evolved into Pakistan International Airlines.
	\item Popular music evolved from folk songs.
	\item As medical knowledge evolves, beliefs change.
	\item This was when he evolved the working method from which he has never departed.
	\end{itemize}
}
\end{enumerate}

\section*{highway}
{\large \color{blue}  highways  }
\subsection*{Explain}
\begin{enumerate}
\item countable noun \\
A \textbf{highway} is a main road, especially one that connects towns or cities.
 \textit{
	\begin{itemize}
	\item I crossed the highway, dodging the traffic.
	\end{itemize}
}
\end{enumerate}

\section*{fix}
{\large \color{blue}  fixes  fixing  fixed  }
\subsection*{Explain}
\begin{enumerate}
\item verb \\
If something \textbf{is fixed}  somewhere , it is attached there firmly or securely.
 \textit{
	\begin{itemize}
	\item It is fixed on the wall.
	\item Most blinds can be fixed directly to the top of the window-frame.
	\item He fixed a bayonet to the end of his rifle.
	\end{itemize}
}
\item verb \\
If you \textbf{fix} something, for example a date, price, or policy, you decide and say  exactly what it will be.
 \textit{
	\begin{itemize}
	\item He's going to fix a time when I can see him.
	\item The date of the election was fixed.
	\item The prices of milk and cereals are fixed annually.
	\end{itemize}
}
\item verb \\
If you \textbf{fix} something for someone, you arrange for it to happen or you organize it for them.
 \textit{
	\begin{itemize}
	\item I've fixed it for you to see Bonnie Lachlan.
	\item It's fixed. He's going to meet us at the airport.
	\item They thought that their relatives would be able to fix the visas.
	\item He vanished after you fixed him with a job.
	\item We fixed for the team to visit our headquarters.
	\item They'd fixed yesterday that Mike'd be in late today.
	\end{itemize}
}
\item verb \\
If you \textbf{fix} something which is damaged or which does not work properly, you repair it.
 \textit{
	\begin{itemize}
	\item He cannot fix the electricity.
	\item If something is broken, we get it fixed.
	\end{itemize}
}
\item verb \\
If you \textbf{fix} a problem or a bad situation, you deal with it and make it satisfactory .
 \textit{
	\begin{itemize}
	\item It's not too late to fix the problem, although time is clearly getting short.
	\item Fixing a 40-year-old wrong does not mean, however, that history can be undone.
	\end{itemize}
}
\item countable noun \\
You can refer to a solution to a problem as a \textbf{fix} .
 \textit{
	\begin{itemize}
	\item Many of those changes could just be a temporary fix.
	\end{itemize}
}
\item verb \\
If you \textbf{fix} your eyes \textbf{on} someone or something or if your eyes \textbf{fix on} them, you look at them with complete attention.
 \textit{
	\begin{itemize}
	\item She fixes her steel-blue eyes on an unsuspecting local official.
	\item Her soft brown eyes fixed on Kelly.
	\item The child kept her eyes fixed on the wall behind him.
	\end{itemize}
}
\item verb \\
If you \textbf{fix} someone \textbf{with} a particular kind of expression, you look at them in that way.
 \textit{
	\begin{itemize}
	\item He took her hand and fixed her with a look of deep concern.
	\item He fixed me with a lopsided grin.
	\end{itemize}
}
\item verb \\
If you \textbf{fix} your attention \textbf{on} someone or something, you think about them with complete attention.
 \textit{
	\begin{itemize}
	\item Fix your attention on the practicalities of financing your schemes.
	\item Attention is fixed on the stock market.
	\item She kept her mind fixed on the practical problems which faced her.
	\end{itemize}
}
\item verb \\
If someone or something \textbf{is fixed in} your mind, you remember them well, for example because they are very important, interesting, or unusual .
 \textit{
	\begin{itemize}
	\item Leonard was now fixed in his mind.
	\item Amy watched the child's intent face eagerly, trying to fix it in her mind.
	\end{itemize}
}
\item verb \\
If someone \textbf{fixes} a gun, camera, or radar \textbf{on} something, they point it at that thing.
 \textit{
	\begin{itemize}
	\item The crew fixed its radar on the enemy ship.
	\item The bore of the gun remained fixed on me.
	\end{itemize}
}
\item verb \\
If you \textbf{fix} the position of something, you find out exactly where it is, usually by using radar or electronic equipment.
 \textbf{Fix} is also a noun.
 \textit{
	\begin{itemize}
	\item He had not been able to fix his position.
	\item The satellite fixes positions by making repeated observations of each star.
	\item ...accurate position fixing.
	\item The army hasn't been able to get a fix on the transmitter.
	\end{itemize}
}
\item singular noun \\
If you get \textbf{a fix on} someone or something, you have a clear idea or understanding of them.
 \textit{
	\begin{itemize}
	\item It's been hard to get a steady fix on what's going on.
	\end{itemize}
}
\item verb \\
If you \textbf{fix} some food or a drink for someone, you make it or prepare it for them.
 \textit{
	\begin{itemize}
	\item Sarah fixed some food for us.
	\item Let me fix you a drink.
	\item Scotty stayed behind to fix lunch.
	\end{itemize}
}
\item verb \\
If you \textbf{fix} your hair, clothes, or make-up , you arrange or adjust them so you look neat and tidy , showing you have taken care with your appearance.
 \textit{
	\begin{itemize}
	\item 'I've got to fix my hair,' I said and retreated to my bedroom.
	\item She called a cab, fixed her face, and scrawled a hasty note to Brian.
	\end{itemize}
}
\item verb \\
If you have your teeth \textbf{fixed} , you have treatment from a dentist to make your teeth even, straight, and white.
 \textit{
	\begin{itemize}
	\item The PR man suggested that I might benefit from getting my teeth fixed.
	\item I wonder if Tom ever had his teeth fixed anywhere else?
	\end{itemize}
}
\item verb \\
If someone \textbf{fixes} a race, election , contest, or other event, they make unfair or illegal arrangements or use deception to affect the result.
 \textbf{Fix} is also a noun.
 \textit{
	\begin{itemize}
	\item They offered opposing players bribes to fix a decisive league match.
	\item We didn't 'fix' anything. It'll be seen as it happens.
	\item The debate seems, in retrospect, to have been fixed from the beginning.
	\item ...this week's report of match-fixing.
	\item It's all a fix, a deal they've made.
	\end{itemize}
}
\item verb \\
If you accuse someone of \textbf{fixing} prices, you accuse them of making unfair arrangements to charge a particular price
for something, rather than allowing market forces to decide it.
 \textit{
	\begin{itemize}
	\item ...a suspected cartel that had fixed the price of steel for the construction market.
	\item The company is currently in dispute with the government over price fixing.
	\end{itemize}
}
\item countable noun \\
An injection of an addictive drug such as heroin can be referred to as a \textbf{fix} .
 \textit{
	\begin{itemize}
	\end{itemize}
}
\item countable noun \\
You can use \textbf{fix} to refer to an amount of something which a person gets or wants and which helps them physically or psychologically to survive .
 \textit{
	\begin{itemize}
	\item It turned the country into an 'aid junkie', heavily dependent on its annual fix of
dollars.
	\item I need my fix of sugar, sweets, and chocolate.
	\item The trouble with her is she needs her daily fix of publicity.
	\item ...a quick energy fix.
	\end{itemize}
}
\item singular noun \\
If you are \textbf{in}  \textbf{a fix} , you are in a difficult situation, especially one that you have caused for yourself.
 \textit{
	\begin{itemize}
	\item He was in a fix.
	\item The government has really got itself into a fix.
	\item This will put us in a very difficult economic fix.
	\end{itemize}
}
\item verb \\
To \textbf{fix} something such as a dye or photographic image means to treat it with chemicals so that it does not lose its colour or disappear .
 \textit{
	\begin{itemize}
	\item Paints consist of pigments bound by a medium which fixes the colour.
	\end{itemize}
}
\item verb \\
If you say that you will \textbf{fix} someone, you mean that you will stop their activities permanently.
 \textit{
	\begin{itemize}
	\item That'll fix him.
	\end{itemize}
}
\item verb \\
If you say that you \textbf{are fixing}  \textbf{to} do something, you mean that you are planning or intending to do it.
 \textit{
	\begin{itemize}
	\item I'm fixing to go to graduate school.
	\item He would know when I was fixing to leave. He'd wait by the front door.
	\end{itemize}
}
\end{enumerate}

\section*{june}
{\large \color{blue}  Junes  }
\subsection*{Explain}
\begin{enumerate}
\item variable noun \\
\textbf{June} is the sixth month of the year in the Western  calendar .
 \textit{
	\begin{itemize}
	\item He spent two and a half weeks with us in June 1986.
	\item I am moving out on 5 June.
	\item Last June I decided to take a trip to Marbella.
	\end{itemize}
}
\end{enumerate}

\section*{fling}
{\large \color{blue}  flings  flinging  flung  }
\subsection*{Explain}
\begin{enumerate}
\item verb \\
If you \textbf{fling} something somewhere , you throw it there using a lot of force.
 \textit{
	\begin{itemize}
	\item The woman flung the cup at him.
	\item He once seized my knitting, flinging it across the room.
	\end{itemize}
}
\item verb \\
If you \textbf{fling}  \textbf{yourself} somewhere, you move or jump there suddenly and with a lot of force.
 \textit{
	\begin{itemize}
	\item He flung himself to the floor.
	\end{itemize}
}
\item verb \\
If you \textbf{fling} a part of your body in a particular  direction , especially your arms or head , you move it there suddenly.
 \textit{
	\begin{itemize}
	\item She flung her arms around my neck and kissed me.
	\end{itemize}
}
\item verb \\
If you \textbf{fling} someone to the ground , you push them very roughly so that they fall over.
 \textit{
	\begin{itemize}
	\item The youth got him by the front of his shirt and flung him to the ground.
	\end{itemize}
}
\item verb \\
If you \textbf{fling} something into a particular place or position , you put it there in a quick or angry  way .
 \textit{
	\begin{itemize}
	\item Peter flung his shoes into the corner.
	\item He flung it down on the desk.
	\end{itemize}
}
\item verb \\
If you \textbf{fling}  \textbf{yourself into} a particular activity , you do it with a lot of enthusiasm and energy .
 \textit{
	\begin{itemize}
	\item She flung herself into her career.
	\item I flung myself into poetry.
	\end{itemize}
}
\item verb \\
\textbf{Fling}  can be used instead of 'throw' in many expressions that usually contain 'throw'.
 \textit{
	\begin{itemize}
	\end{itemize}
}
\item countable noun \\
If two people have \textbf{a}  \textbf{fling} , they have a brief  sexual  relationship .
 \textit{
	\begin{itemize}
	\item She claims she had a brief fling with him 30 years ago.
	\end{itemize}
}
\item singular noun \\
\textbf{A}  \textbf{fling} is a short period of enjoyment , especially the last one that you will  get an opportunity to have.
 \textit{
	\begin{itemize}
	\item ...that last fling before you finally give up and take up a job.
	\end{itemize}
}
\end{enumerate}

\section*{kite}
{\large \color{blue}  kites  }
\subsection*{Explain}
\begin{enumerate}
\item countable noun \\
A \textbf{kite} is an object, usually used as a toy , which is flown in the air . It consists of a light frame covered with paper or cloth and has a long string attached which you hold while the kite is flying.
 \textit{
	\begin{itemize}
	\end{itemize}
}
\item countable noun \\
A \textbf{kite} is a bird of prey which hunts and kills small animals for food.
 \textit{
	\begin{itemize}
	\end{itemize}
}
\item  \\
 fly a kite \textit{
	\begin{itemize}
	\end{itemize}
}
\item  \\
 as high as a kite \textit{
	\begin{itemize}
	\end{itemize}
}
\end{enumerate}

\section*{fluctuate}
{\large \color{blue}  fluctuates  fluctuating  fluctuated  }
\subsection*{Explain}
\begin{enumerate}
\item verb \\
If something \textbf{fluctuates} , it changes a lot in an irregular  way .
 \textit{
	\begin{itemize}
	\item Body temperature can fluctuate if you are ill.
	\item ...the fluctuating price of oil.
	\end{itemize}
}
\end{enumerate}

\section*{literature}
{\large \color{blue}  literatures  }
\subsection*{Explain}
\begin{enumerate}
\item variable noun \\
Novels, plays, and poetry are referred to as \textbf{literature} , especially when they are considered to be good or important .
 \textit{
	\begin{itemize}
	\item ...classic works of literature.
	\item ...a Professor of English Literature.
	\item It may not be great literature but it certainly had me riveted!
	\item The book explores the connection between American ethnic and regional literatures.
	\end{itemize}
}
\item uncountable noun \\
\textbf{The}  \textbf{literature} on a particular subject of study is all the books and articles that have been published about it.
 \textit{
	\begin{itemize}
	\item The literature on immigration policy is extremely critical of the state.
	\item This work is documented in the scientific literature.
	\end{itemize}
}
\item uncountable noun \\
\textbf{Literature} is written information produced by people who want to sell you something or give you advice .
 \textit{
	\begin{itemize}
	\item I am sending you literature from two other companies that provide a similar service.
	\item Some companies have toned down the claims on their promotional literature.
	\end{itemize}
}
\end{enumerate}

\section*{harden}
{\large \color{blue}  hardens  hardening  hardened  }
\subsection*{Explain}
\begin{enumerate}
\item verb \\
When something \textbf{hardens} or when you \textbf{harden} it, it becomes stiff or firm .
 \textit{
	\begin{itemize}
	\item Mould the mixture into shape while hot, before it hardens.
	\item Give the cardboard two or three coats of varnish to harden it.
	\end{itemize}
}
\item verb \\
When an attitude or opinion  \textbf{hardens} or \textbf{is hardened} , it becomes harsher, stronger, or fixed .
 \textit{
	\begin{itemize}
	\item Their action can only serve to harden the attitude of landowners.
	\item The bitter split which has developed within Solidarity is likely to harden further
into separation.
	\end{itemize}
}
\item verb \\
When prices and economies  \textbf{harden} , they become much more stable than they were.
 \textit{
	\begin{itemize}
	\item Property prices are just beginning to harden again.
	\end{itemize}
}
\item verb \\
When events  \textbf{harden} people or when people \textbf{harden} , they become less easily  affected emotionally and less sympathetic and gentle than they were before.
 \textit{
	\begin{itemize}
	\item Her years of drunken bickering hardened my heart.
	\item She was hardened by the rigours of the Siberian steppes.
	\item All of a sudden my heart hardened against her.
	\end{itemize}
}
\item verb \\
If you say that someone's face or eyes  \textbf{harden} , you mean that they suddenly  look  serious or angry .
 \textit{
	\begin{itemize}
	\item His smile died and the look in his face hardened.
	\end{itemize}
}
\end{enumerate}

\section*{manner}
{\large \color{blue}  manners  }
\subsection*{Explain}
\begin{enumerate}
\item singular noun \\
The \textbf{manner} in which you do something is the way that you do it.
 \textit{
	\begin{itemize}
	\item She smiled again in a friendly manner.
	\item I'm a professional and I have to conduct myself in a professional manner.
	\item The manner in which young children are spoken to varies depending on who is present.
	\end{itemize}
}
\item singular noun \\
If something is done \textbf{in the}  \textbf{manner}  \textbf{of} something else, it is done in the style of that thing.
 \textit{
	\begin{itemize}
	\item It's a satire somewhat in the manner of Dickens.
	\end{itemize}
}
\item singular noun \\
Someone's \textbf{manner} is the way in which they behave and talk when they are with other people, for example whether they are polite , confident , or bad-tempered .
 \textit{
	\begin{itemize}
	\item His manner was self-assured and brusque.
	\item Her manner offstage, like her manner on, is somewhat surly.
	\end{itemize}
}
\item plural noun \\
If someone has \textbf{good manners} , they are polite and observe social customs . If someone has \textbf{bad manners} , they are impolite and do not observe these customs.
 \textit{
	\begin{itemize}
	\item He dressed well and had impeccable manners.
	\item The manners of many doctors were appalling.
	\item They taught him his manners.
	\end{itemize}
}
\item  \\
 all manner of \textit{
	\begin{itemize}
	\end{itemize}
}
\item  \\
 in a manner of speaking \textit{
	\begin{itemize}
	\end{itemize}
}
\item  \\
 what manner of \textit{
	\begin{itemize}
	\end{itemize}
}
\end{enumerate}

\section*{heave}
{\large \color{blue}  heaves  heaving  heaved  }
\subsection*{Explain}
\begin{enumerate}
\item verb \\
If you \textbf{heave} something heavy or difficult to move somewhere , you push , pull , or lift it using a lot of effort.
 \textbf{Heave} is also a noun .
 \textit{
	\begin{itemize}
	\item It took five strong men to heave the statue up a ramp and lower it into place.
	\item He heaved Barney to his feet.
	\item He heaved himself up off his stool.
	\item It took only one heave to hurl him into the river.
	\end{itemize}
}
\item verb \\
If something \textbf{heaves} , it moves up and down with large regular  movements .
 \textit{
	\begin{itemize}
	\item His chest heaved, and he took a deep breath.
	\item ...the grey, heaving seas.
	\end{itemize}
}
\item verb \\
If you \textbf{heave} , or if your stomach  \textbf{heaves} , you vomit or feel  sick .
 \textit{
	\begin{itemize}
	\item He gasped and heaved and vomited again.
	\item My stomach heaved and I felt sick.
	\end{itemize}
}
\item verb \\
If you \textbf{heave} a \textbf{sigh} , you give a big sigh.
 \textit{
	\begin{itemize}
	\item Mr Collier heaved a sigh and got to his feet.
	\end{itemize}
}
\item verb \\
If a place \textbf{is heaving} or if it \textbf{is heaving with} people, it is full of people.
 \textit{
	\begin{itemize}
	\item The Happy Bunny club was heaving.
	\item Father Auberon's Academy Club positively heaved with dashing young men.
	\end{itemize}
}
\item  \\
 heave into view/heave into sight \textit{
	\begin{itemize}
	\end{itemize}
}
\end{enumerate}

\section*{match}
{\large \color{blue}  matches  matching  matched  }
\subsection*{Explain}
\begin{enumerate}
\item countable noun \\
A \textbf{match} is an organized game of football , tennis , cricket , or some other sport.
 \textit{
	\begin{itemize}
	\item He was watching a football match.
	\item France won the match 28-19.
	\end{itemize}
}
\item countable noun \\
A \textbf{match} is a small wooden  stick with a substance on one end that produces a flame when you rub it along the rough side of a matchbox .
 \textit{
	\begin{itemize}
	\item ...a packet of cigarettes and a box of matches.
	\end{itemize}
}
\item verb \\
If something of a particular colour or design \textbf{matches} another thing, they have the same colour or design, or have a pleasing appearance when they are used together.
 \textbf{Match up} means the same as match .
 \textit{
	\begin{itemize}
	\item Her nails were painted bright red to match her dress.
	\item All the chairs matched.
	\item You don't have to match your lipstick exactly to your outfit.
	\item Mix and match your tableware and textiles from the new Design House collection.
	\item The pillow cover can match up with the sheets.
	\item False eyelashes come in various shades, so it's easy to match them up with your own.
	\end{itemize}
}
\item verb \\
If something such as an amount or a quality \textbf{matches}  \textbf{with} another amount or quality, they are both the same or equal. If you \textbf{match} two things, you make them the same or equal.
 \textit{
	\begin{itemize}
	\item Their strengths in memory and spatial skills matched.
	\item Our value system does not match with their value system.
	\item ...efforts to match demand with supply by building new schools.
	\end{itemize}
}
\item verb \\
If one thing \textbf{matches} another, they are connected or suit each other in some way.
 \textbf{Match up} means the same as match .
 \textit{
	\begin{itemize}
	\item The students are asked to match the books with the authors.
	\item We will try to match you to employers with the vacancies you are looking for.
	\item It can take time and effort to match buyers and sellers.
	\item The sale would only go ahead if the name and number matched.
	\item Pictures of road signs are matched with their Highway Code meanings.
	\item The consultant seeks to match up jobless professionals with small companies in need
of expertise.
	\item With this app, friends of singles match them up with other users.
	\item My sister and I never really matched up.
	\item I'm sure that yellow lead matched up to that yellow socket.
	\end{itemize}
}
\item singular noun \\
If a combination of things or people is a good \textbf{match} , they have a pleasing effect when placed or used together.
 \textit{
	\begin{itemize}
	\item Helen's choice of lipstick was a good match for her skin-tone.
	\item Moira was a perfect match for him.
	\end{itemize}
}
\item verb \\
If you \textbf{match} something, you are as good as it or equal to it, for example in speed , size, or quality.
 \textit{
	\begin{itemize}
	\item They played some fine attacking football, but I think we matched them in every respect.
	\item His record has never been matched.
	\end{itemize}
}
\item verb \\
In a sport or other contest , if you \textbf{match} one person or team against another, in sports or other contests, you make them compete
with each other to see which one is better .
 \textit{
	\begin{itemize}
	\item The finals begin today, matching Chelsea against Manchester United.
	\item Lewis is matched against the heavyweight champion.
	\end{itemize}
}
\item  \\
 to meet your match \textit{
	\begin{itemize}
	\end{itemize}
}
\item  \\
 no match for \textit{
	\begin{itemize}
	\end{itemize}
}
\end{enumerate}

\section*{incline}
{\large \color{blue}  inclines  inclining  inclined  }
\subsection*{Explain}
\begin{enumerate}
\item verb \\
If you \textbf{incline}  \textbf{to}  think or act in a particular way , or if something \textbf{inclines} you \textbf{to} it, you are likely to think or act in that way.
 \textit{
	\begin{itemize}
	\item I incline to the view that he is right.
	\item ...the factors which incline us towards particular beliefs.
	\item Many end up as team leaders, which inclines them to co-operate with the bosses.
	\item Those who fail incline to blame the world for their failure.
	\end{itemize}
}
\item verb \\
If you \textbf{incline} your head, you bend your neck so that your head is leaning  forward .
 \textit{
	\begin{itemize}
	\item Jack inclined his head very slightly.
	\end{itemize}
}
\item countable noun \\
An \textbf{incline} is land that slopes at an angle .
 \textit{
	\begin{itemize}
	\item He came to a halt at the edge of a steep incline.
	\end{itemize}
}
\end{enumerate}

\section*{means}
{\large \color{blue}  }
\subsection*{Explain}
\begin{enumerate}
\item countable noun \\
A \textbf{means} of doing something is a method, instrument, or process which can be used to do it.
 \textbf{Means} is both the singular and the plural form for this use.
 \textit{
	\begin{itemize}
	\item The move is a means to fight crime.
	\item The army had perfected the use of terror as a means of controlling the population.
	\item Business managers are focused on increasing their personal wealth by any available
means.
	\end{itemize}
}
\item plural noun \\
You can refer to the money that someone has as their \textbf{means} .
 \textit{
	\begin{itemize}
	\item ...a person of means.
	\item He did not have the means to compensate her.
	\end{itemize}
}
\item  \\
 to live beyond your means \textit{
	\begin{itemize}
	\end{itemize}
}
\item  \\
 by means of \textit{
	\begin{itemize}
	\end{itemize}
}
\item  \\
 by all means \textit{
	\begin{itemize}
	\end{itemize}
}
\item  \\
 not by any means \textit{
	\begin{itemize}
	\end{itemize}
}
\item  \\
 a means to an end \textit{
	\begin{itemize}
	\end{itemize}
}
\end{enumerate}

\section*{march}
{\large \color{blue}  marches  marching  marched  }
\subsection*{Explain}
\begin{enumerate}
\item verb \\
When soldiers  \textbf{march}  somewhere , or when a commanding  officer  \textbf{marches} them somewhere, they walk there with very regular steps, as a group.
 \textbf{March} is also a noun .
 \textit{
	\begin{itemize}
	\item A Scottish battalion was marching down the street.
	\item Captain Ramirez called them to attention and marched them off to the main camp.
	\item We marched fifteen miles to Yadkin River.
	\item The ice was not thick enough to bear the weight of marching men.
	\item After a short march, the column entered the village.
	\end{itemize}
}
\item verb \\
When a large group of people \textbf{march} for a cause, they walk somewhere together in order to express their ideas or to protest about something.
 \textbf{March} is also a noun.
 \textit{
	\begin{itemize}
	\item The demonstrators then marched through the capital chanting slogans and demanding
free elections.
	\item Hundreds of activists marked the holy day by marching for peace and disarmament.
	\item Organisers expect up to 300,000 protesters to join the march.
	\end{itemize}
}
\item verb \\
If you say that someone \textbf{marches} somewhere, you mean that they walk there quickly and in a determined way, for example because they are angry .
 \textit{
	\begin{itemize}
	\item He marched into the kitchen without knocking.
	\end{itemize}
}
\item verb \\
If you \textbf{march} someone somewhere, you force them to walk there with you, for example by holding their arm tightly.
 \textit{
	\begin{itemize}
	\item Nearly 700 prisoners were marched away.
	\item I marched him across the room, down the hall and out onto the doorstep.
	\end{itemize}
}
\item singular noun \\
\textbf{The}  \textbf{march}  \textbf{of} something is its steady  development or progress .
 \textit{
	\begin{itemize}
	\item It is easy to feel trampled by the relentless march of technology.
	\item Society's march toward ever-increasing materialism was continuing.
	\end{itemize}
}
\item countable noun \\
A \textbf{march} is a piece of music with a regular rhythm that you can march to.
 \textit{
	\begin{itemize}
	\item A military band played Russian marches and folk tunes.
	\end{itemize}
}
\item  \\
 your marching orders \textit{
	\begin{itemize}
	\end{itemize}
}
\item  \\
 to steal a march on someone \textit{
	\begin{itemize}
	\end{itemize}
}
\end{enumerate}

\section*{method}
{\large \color{blue}  methods  }
\subsection*{Explain}
\begin{enumerate}
\item countable noun \\
A \textbf{method} is a particular way of doing something.
 \textit{
	\begin{itemize}
	\item The pill is the most efficient method of birth control.
	\item ...new teaching methods.
	\item The usual method of getting through the Amsterdam traffic is to cycle to your local
railway station and take the train.
	\end{itemize}
}
\end{enumerate}

\section*{merge}
{\large \color{blue}  merges  merging  merged  }
\subsection*{Explain}
\begin{enumerate}
\item verb \\
If one thing \textbf{merges}  \textbf{with} another, or \textbf{is merged}  \textbf{with} another, they combine or come together to make one whole thing. You can also  say that two things \textbf{merge} , or \textbf{are merged} .
 \textit{
	\begin{itemize}
	\item My life merged with his.
	\item The company had merged with its rival the previous December.
	\item The rivers merge just north of a vital irrigation system.
	\item The two countries merged into one.
	\item He sees sense in merging the two agencies while both are new.
	\item Then he showed me how to merge the graphic with text on the same screen.
	\end{itemize}
}
\item verb \\
If one sound, colour, or object \textbf{merges} into another, the first changes so gradually into the second that you do not notice the change.
 \textit{
	\begin{itemize}
	\item Like a chameleon, he could merge unobtrusively into the background.
	\item His features merged with the darkness.
	\item Night and day begin to merge.
	\end{itemize}
}
\end{enumerate}

\section*{mob}
{\large \color{blue}  mobs  mobbing  mobbed  }
\subsection*{Explain}
\begin{enumerate}
\item countable noun \\
A \textbf{mob} is a large, disorganized , and often violent crowd of people.
 \textit{
	\begin{itemize}
	\item Bottles and cans were hurled on the terraces by the mob.
	\item The inspectors watched a growing mob of demonstrators gathering.
	\end{itemize}
}
\item singular noun \\
People sometimes use \textbf{the mob} to refer in a disapproving way to the majority of people in a country or place, especially when these people are behaving in a violent or uncontrolled way.
 \textit{
	\begin{itemize}
	\item If they continue like this, there is a danger of the mob taking over.
	\item They have been exercising what amounts to mob rule.
	\end{itemize}
}
\item singular noun \\
You can refer to the people involved in organized  crime as \textbf{the}  \textbf{Mob} .
 \textit{
	\begin{itemize}
	\item ...casinos that the Mob had operated.
	\item It was a Mob killing.
	\end{itemize}
}
\item verb \\
If you say that someone \textbf{is being mobbed}  \textbf{by} a crowd of people, you mean that the people are trying to talk to them or get near them in an enthusiastic or threatening way.
 \textit{
	\begin{itemize}
	\item Her car was mobbed by the media.
	\item They found themselves being mobbed in the street for autographs.
	\end{itemize}
}
\end{enumerate}

\section*{overflow}
{\large \color{blue}  overflows  overflowing  overflowed  }
\subsection*{Explain}
\begin{enumerate}
\item verb \\
If a liquid or a river \textbf{overflows} , it flows over the edges of the container or place it is in.
 \textit{
	\begin{itemize}
	\item Pour in some of the syrup, but not all of it, as it will probably overflow.
	\item Rivers and streams have overflowed their banks in countless places.
	\end{itemize}
}
\item verb \\
If a place or container \textbf{is overflowing}  \textbf{with} people or things, it is too full of them.
 \textit{
	\begin{itemize}
	\item The great hall was overflowing with people.
	\item Jails and temporary detention camps are overflowing.
	\item He emptied a few overflowing ashtrays.
	\end{itemize}
}
\item verb \\
If someone \textbf{is overflowing}  \textbf{with} a feeling or if the feeling \textbf{overflows} , the person is experiencing it very strongly and shows this in their behaviour .
 \textit{
	\begin{itemize}
	\item Kenneth overflowed with friendliness and hospitality.
	\item Ridley's anger finally overflowed.
	\end{itemize}
}
\item countable noun \\
The \textbf{overflow} is the extra people or things that something cannot contain or deal with because it is not large enough.
 \textit{
	\begin{itemize}
	\item Tents have been set up next to hospitals to handle the overflow.
	\item The loch's overflow cascades into the waterfalls of a Japanese water garden.
	\end{itemize}
}
\item countable noun \\
An \textbf{overflow} is a hole or pipe through which liquid can flow out of a container when it gets too full.
 \textit{
	\begin{itemize}
	\item ...the overflow pipe.
	\end{itemize}
}
\item  \\
 to overflowing \textit{
	\begin{itemize}
	\end{itemize}
}
\end{enumerate}

\section*{mode}
{\large \color{blue}  modes  }
\subsection*{Explain}
\begin{enumerate}
\item countable noun \\
A \textbf{mode} of life or behaviour is a particular way of living or behaving .
 \textit{
	\begin{itemize}
	\item ...the capitalist mode of production.
	\item He switched automatically into interview mode.
	\end{itemize}
}
\item countable noun \\
A \textbf{mode} is a particular style in art, literature , or dress .
 \textit{
	\begin{itemize}
	\item ...a slightly more elegant and formal mode of dress.
	\item Levi is best known for work in a very different mode from what is to be found here.
	\end{itemize}
}
\item countable noun \\
On some cameras or electronic devices, the different \textbf{modes}  available are the different programs or settings that you can choose when you use them.
 \textit{
	\begin{itemize}
	\item ...when the camera is in manual mode.
	\end{itemize}
}
\end{enumerate}

\section*{qualify}
{\large \color{blue}  qualifies  qualifying  qualified  }
\subsection*{Explain}
\begin{enumerate}
\item verb \\
When someone \textbf{qualifies} , they pass the examinations that they need to be able to work in a particular profession .
 \textit{
	\begin{itemize}
	\item But when I'd qualified and started teaching it was a different story.
	\item I qualified as a doctor from London University over 30 years ago.
	\end{itemize}
}
\item verb \\
If you \textbf{qualify} for something or if something \textbf{qualifies} you for it, you have the right to do it or have it.
 \textit{
	\begin{itemize}
	\item To qualify for maternity leave you must have worked for the same employer for two
years.
	\item The basic course does not qualify you to practise as a therapist.
	\item ...skills that qualify foreigners for work visas.
	\item ...highly trained staff who are well qualified to give unbiased, practical advice.
	\end{itemize}
}
\item verb \\
To \textbf{qualify}  \textbf{as} something or to \textbf{be qualified}  \textbf{as} something means to have all the features that are needed to be that thing.
 \textit{
	\begin{itemize}
	\item 13 percent of American households qualify as poor, says Mr. Mishel.
	\item These people seem to think that reading a few books on old age qualifies them as
experts.
	\end{itemize}
}
\item verb \\
If you \textbf{qualify} in a competition, you are successful in one part of it and go on to the next stage.
 \textit{
	\begin{itemize}
	\item Nottingham Forest qualified for the final by beating Tranmere on Tuesday.
	\item Cameroon have also qualified after beating Sierra Leone.
	\item ...a World Cup qualifying match.
	\end{itemize}
}
\item verb \\
If you \textbf{qualify} a statement , you make it less strong or less general by adding a detail or explanation to it.
 \textit{
	\begin{itemize}
	\item I would qualify that by putting it into context.
	\end{itemize}
}
\end{enumerate}

\section*{month}
{\large \color{blue}  months  }
\subsection*{Explain}
\begin{enumerate}
\item countable noun \\
A \textbf{month} is one of the twelve periods of time that a year is divided into, for example  January or February .
 \textit{
	\begin{itemize}
	\item The trial is due to begin next month.
	\item ...an exhibition which opens this month at London's Design Museum.
	\item I send him fifteen dollars a month.
	\end{itemize}
}
\item countable noun \\
A \textbf{month} is a period of about four weeks.
 \textit{
	\begin{itemize}
	\item She was here for a month.
	\item Over the next several months I met most of her family.
	\item ...a month's unlimited train travel.
	\end{itemize}
}
\end{enumerate}

\section*{recur}
{\large \color{blue}  recurs  recurring  recurred  }
\subsection*{Explain}
\begin{enumerate}
\item verb \\
If something \textbf{recurs} , it happens more than once.
 \textit{
	\begin{itemize}
	\item ...a theme that was to recur frequently in his work.
	\item ...a recurring nightmare she has had since childhood.
	\end{itemize}
}
\end{enumerate}

\section*{orientation}
{\large \color{blue}  orientations  }
\subsection*{Explain}
\begin{enumerate}
\item variable noun \\
If you talk about the \textbf{orientation} of an organization or country , you are talking about the kinds of aims and interests it has.
 \textit{
	\begin{itemize}
	\item ...a marketing orientation.
	\item To a society which has lost its orientation he has much to offer.
	\item The movement is liberal and social democratic in orientation.
	\end{itemize}
}
\item variable noun \\
Someone's \textbf{orientation} is their basic beliefs or preferences.
 \textit{
	\begin{itemize}
	\item ...legislation that would have made discrimination on the basis of sexual orientation
illegal.
	\end{itemize}
}
\item uncountable noun \\
\textbf{Orientation} is basic information or training that is given to people starting a new job , school , or course.
 \textit{
	\begin{itemize}
	\item They give their new employees a day or two of perfunctory orientation.
	\item ...a one-day orientation session.
	\end{itemize}
}
\item countable noun \\
The \textbf{orientation} of a structure or object is the direction it faces .
 \textit{
	\begin{itemize}
	\item Farnese had the orientation of the church changed so that the front would face a
square.
	\end{itemize}
}
\end{enumerate}

\section*{rejoice}
{\large \color{blue}  rejoices  rejoicing  rejoiced  }
\subsection*{Explain}
\begin{enumerate}
\item verb \\
If you \textbf{rejoice} , you are very pleased about something and you show it in your behaviour .
 \textit{
	\begin{itemize}
	\item Garbo plays the Queen, rejoicing in the love she has found with Antonio.
	\item A Foreign Ministry statement said that the French government rejoiced at the happy
outcome to events.
	\item Party activists rejoiced that the presidential campaign had finally started.
	\end{itemize}
}
\item  \\
 rejoice in the name of \textit{
	\begin{itemize}
	\end{itemize}
}
\end{enumerate}

\section*{relax}
{\large \color{blue}  relaxes  relaxing  relaxed  }
\subsection*{Explain}
\begin{enumerate}
\item verb \\
If you \textbf{relax} or if something \textbf{relaxes} you, you feel more calm and less worried or tense.
 \textit{
	\begin{itemize}
	\item I ought to relax and stop worrying about it.
	\item For the first time since his arrival he relaxed slightly.
	\item Do something that you know relaxes you.
	\end{itemize}
}
\item verb \\
When a part of your body \textbf{relaxes} , or when you \textbf{relax} it, it becomes less stiff or firm .
 \textit{
	\begin{itemize}
	\item Massage is used to relax muscles, relieve stress and improve the circulation.
	\item His face relaxes into a contented smile.
	\end{itemize}
}
\item verb \\
If you \textbf{relax} your grip or hold on something, you hold it less tightly than before.
 \textit{
	\begin{itemize}
	\item He gradually relaxed his grip on the arms of the chair.
	\end{itemize}
}
\item verb \\
If you \textbf{relax} a rule or your control over something, or if it \textbf{relaxes} , it becomes less firm or strong .
 \textit{
	\begin{itemize}
	\item Rules governing student conduct relaxed somewhat in recent years.
	\item How much can the President relax his grip over the nation?
	\item Some analysts believe that the government soon will begin relaxing economic controls.
	\end{itemize}
}
\end{enumerate}

\section*{project}
{\large \color{blue}  projects  projecting  projected  }
\subsection*{Explain}
\begin{enumerate}
\item countable noun \\
A \textbf{project} is a task that requires a lot of time and effort.
 \textit{
	\begin{itemize}
	\item Money will also go into local development projects in Vietnam.
	\item ...an international science project.
	\item Besides film and record projects, I have continued to work in the theater.
	\end{itemize}
}
\item countable noun \\
A \textbf{project} is a detailed study of a subject by a pupil or student.
 \textit{
	\begin{itemize}
	\item Students complete projects for a personal tutor, working at home at their own pace.
	\end{itemize}
}
\item verb \\
If something \textbf{is projected} , it is planned or expected .
 \textit{
	\begin{itemize}
	\item This sector is projected to double in size over the next 12 months.
	\item The government had been projecting a 5% consumer price increase for the entire year.
	\item ...a projected deficit of $1.5 million.
	\end{itemize}
}
\item verb \\
If you \textbf{project} someone or something in a particular way, you try to make people see them in that way. If you \textbf{project} a particular feeling or quality, you show it in your behaviour.
 \textit{
	\begin{itemize}
	\item Bradley projects a natural warmth and sincerity.
	\item He just hasn't been able to project himself as the strong leader.
	\item His first job will be to project Glasgow as a friendly city.
	\item The initial image projected was of a caring, effective president.
	\end{itemize}
}
\item verb \\
If you \textbf{project} feelings or ideas \textbf{on to} other people, you imagine that they have the same ideas or feelings as you.
 \textit{
	\begin{itemize}
	\item He projects his own thoughts and ideas onto her.
	\end{itemize}
}
\item verb \\
If you \textbf{project} a film or picture  \textbf{onto} a screen or wall, you make it appear there.
 \textit{
	\begin{itemize}
	\item The team tried projecting the maps with two different projectors onto the same screen.
	\end{itemize}
}
\item verb \\
If something \textbf{projects} , it sticks out above or beyond a surface or edge.
 \textit{
	\begin{itemize}
	\item ...the remains of a war-time defence which projected out from the shore.
	\item ...a piece of projecting metal.
	\end{itemize}
}
\end{enumerate}

\section*{renew}
{\large \color{blue}  renews  renewing  renewed  }
\subsection*{Explain}
\begin{enumerate}
\item verb \\
If you \textbf{renew} an activity, you begin it again.
 \textit{
	\begin{itemize}
	\item He renewed his attack on government policy towards Europe.
	\item He believes the peace talks will be renewed soon.
	\item There was renewed fighting yesterday.
	\end{itemize}
}
\item verb \\
If you \textbf{renew} a relationship  \textbf{with} someone, you start it again after you have not seen them or have not been friendly with them for some time.
 \textit{
	\begin{itemize}
	\item When the two men met again after the war they renewed their friendship.
	\item In December 1989 Syria renewed diplomatic relations with Egypt.
	\end{itemize}
}
\item verb \\
When you \textbf{renew} something such as a licence or a contract, you extend the period of time for which
it is valid.
 \textit{
	\begin{itemize}
	\item Larry's landlord threatened not to renew his lease.
	\item The job was for a fixed term and the contract is not being renewed.
	\end{itemize}
}
\item verb \\
You can say that something \textbf{is renewed} when it grows again or is replaced after it has been destroyed or lost .
 \textit{
	\begin{itemize}
	\item Cells are being constantly renewed.
	\item ...a renewed interest in public transport systems.
	\end{itemize}
}
\end{enumerate}

\section*{recognition}
{\large \color{blue}  }
\subsection*{Explain}
\begin{enumerate}
\item uncountable noun \\
\textbf{Recognition} is the act of recognizing someone or identifying something when you see it.
 \textit{
	\begin{itemize}
	\item George said, 'Ida, how are you?' She frowned for a moment and then recognition dawned.
'George Black. Well, I never.'.
	\item He searched for a sign of recognition on her face, but there was none.
	\end{itemize}
}
\item uncountable noun \\
\textbf{Recognition}  \textbf{of} something is an understanding and acceptance of it.
 \textit{
	\begin{itemize}
	\item The CBI welcomed the Chancellor's recognition of the recession and hoped for a reduction
in interest rates.
	\end{itemize}
}
\item uncountable noun \\
When a government gives diplomatic  \textbf{recognition} to another country, they officially  accept that its status is valid .
 \textit{
	\begin{itemize}
	\item Some 110 states extended diplomatic recognition to Kosovo after it split from Serbia.
	\item His government did not receive full recognition by Britain until July.
	\end{itemize}
}
\item uncountable noun \\
When a person receives  \textbf{recognition} for the things that they have done, people acknowledge the value or skill of their work.
 \textit{
	\begin{itemize}
	\item At last, her father's work has received popular recognition.
	\item He is an outstanding goalscorer who doesn't get the recognition he deserves.
	\end{itemize}
}
\item  \\
 beyond recognition/out of all recognition \textit{
	\begin{itemize}
	\end{itemize}
}
\item  \\
 in recognition of \textit{
	\begin{itemize}
	\end{itemize}
}
\end{enumerate}

\section*{rotate}
{\large \color{blue}  rotates  rotating  rotated  }
\subsection*{Explain}
\begin{enumerate}
\item verb \\
When something \textbf{rotates} or when you \textbf{rotate} it, it turns with a circular  movement .
 \textit{
	\begin{itemize}
	\item The Earth rotates round the sun.
	\item Take each foot in both your hands and rotate it to loosen and relax the ankle.
	\end{itemize}
}
\item verb \\
If people or things \textbf{rotate} , or if someone \textbf{rotates} them, they take it in turns to do a particular job or serve a particular purpose .
 \textit{
	\begin{itemize}
	\item The members of the club can rotate and one person can do all the preparation for
the evening.
	\item They will swap posts in a year's time to rotate the leadership.
	\end{itemize}
}
\end{enumerate}

\section*{rocket}
{\large \color{blue}  rockets  rocketing  rocketed  }
\subsection*{Explain}
\begin{enumerate}
\item countable noun \\
A \textbf{rocket} is a space vehicle that is shaped like a long tube .
 \textit{
	\begin{itemize}
	\end{itemize}
}
\item countable noun \\
A \textbf{rocket} is a missile containing explosive that is powered by gas .
 \textit{
	\begin{itemize}
	\item There has been a renewed rocket attack on the capital.
	\end{itemize}
}
\item countable noun \\
A \textbf{rocket} is a firework that quickly goes high into the air and then explodes .
 \textit{
	\begin{itemize}
	\end{itemize}
}
\item verb \\
If things such as prices or social  problems  \textbf{rocket} , they increase very quickly and suddenly .
 \textit{
	\begin{itemize}
	\item Fresh food is so scarce that prices have rocketed.
	\item The nation has experienced four years of rocketing crime.
	\end{itemize}
}
\item verb \\
If something such as a vehicle \textbf{rockets}  somewhere , it moves there very quickly.
 \textit{
	\begin{itemize}
	\item A train rocketed by, shaking the walls of the row houses.
	\item Dublin has rocketed up the charts to become one of Europe's most popular tourist
destinations for city breaks.
	\end{itemize}
}
\end{enumerate}

\section*{rush}
{\large \color{blue}  rushes  rushing  rushed  }
\subsection*{Explain}
\begin{enumerate}
\item verb \\
If you \textbf{rush}  somewhere , you go there quickly.
 \textit{
	\begin{itemize}
	\item A schoolgirl rushed into a burning flat to save a man's life.
	\item Someone inside the building rushed out.
	\item I've got to rush. Got a meeting in a few minutes.
	\item Shop staff rushed to get help.
	\end{itemize}
}
\item verb \\
If people \textbf{rush}  \textbf{to} do something, they do it as soon as they can, because they are very eager to do it.
 \textit{
	\begin{itemize}
	\item Russian banks rushed to buy as many dollars as they could.
	\item Before you rush to book a table, bear in mind that lunch for two would cost £ 150.
	\end{itemize}
}
\item singular noun \\
A \textbf{rush} is a situation in which you need to go somewhere or do something very quickly.
 \textit{
	\begin{itemize}
	\item The men left in a rush.
	\item It was all rather a rush.
	\item Then there was the mad rush not to be late for school.
	\end{itemize}
}
\item singular noun \\
If there is a \textbf{rush}  \textbf{for} something, many people suddenly try to get it or do it.
 \textit{
	\begin{itemize}
	\item Record stores are expecting a huge rush for the single.
	\item ...the rush for contracts.
	\end{itemize}
}
\item singular noun \\
\textbf{The}  \textbf{rush} is a period of time when many people go somewhere or do something.
 \textit{
	\begin{itemize}
	\item The shop's opening coincided with the Christmas rush.
	\item Apply before the rush starts.
	\item ...the annual rush to the beaches.
	\end{itemize}
}
\item verb \\
If you \textbf{rush} something, you do it in a hurry, often too quickly and without much care .
 \textit{
	\begin{itemize}
	\item You can't rush a search.
	\item Chew your food well and do not rush meals.
	\item Instead of rushing at life, I wanted something more meaningful.
	\end{itemize}
}
\item verb \\
If you \textbf{rush} someone or something to a place, you take them there quickly.
 \textit{
	\begin{itemize}
	\item We got an ambulance and rushed her to hospital.
	\item Federal agents rushed him into a car.
	\item We'll rush it round today if possible.
	\end{itemize}
}
\item verb \\
If you \textbf{rush}  \textbf{into} something or \textbf{are rushed}  \textbf{into} it, you do it without thinking about it for long enough.
 \textit{
	\begin{itemize}
	\item He will not rush into any decisions.
	\item They had rushed in without adequate appreciation of the task.
	\item Ministers won't be rushed into a response.
	\item Don't rush him or he'll become confused.
	\end{itemize}
}
\item verb \\
If you \textbf{rush} something or someone, you move quickly and forcefully at them, often in order to
attack them.
 \textit{
	\begin{itemize}
	\item They rushed the entrance and forced their way in.
	\item Tom came rushing at him from another direction.
	\end{itemize}
}
\item verb \\
If air or liquid \textbf{rushes} somewhere, it flows there suddenly and quickly.
 \textbf{Rush} is also a noun .
 \textit{
	\begin{itemize}
	\item Water rushes out of huge tunnels.
	\item The air was rushing past us all the time.
	\item ...the sound of rushing water.
	\item A rush of air on my face woke me.
	\item ...the perpetual rush of the mill stream.
	\end{itemize}
}
\item countable noun \\
If you experience a \textbf{rush}  \textbf{of} a feeling, you suddenly experience it very strongly.
 \textit{
	\begin{itemize}
	\item A rush of pure affection swept over him.
	\item He felt a sudden rush of panic at the thought.
	\end{itemize}
}
\item plural noun \\
\textbf{Rushes} are plants with long thin stems that grow near water.
 \textit{
	\begin{itemize}
	\end{itemize}
}
\item plural noun \\
In film-making , \textbf{the}  \textbf{rushes} of a film are the parts of it that have been filmed but have not yet been edited .
 \textit{
	\begin{itemize}
	\end{itemize}
}
\item  \\
 be rushed off your feet \textit{
	\begin{itemize}
	\end{itemize}
}
\end{enumerate}

\section*{scenery}
{\large \color{blue}  }
\subsection*{Explain}
\begin{enumerate}
\item uncountable noun \\
The \textbf{scenery} in a country area is the land, water, or plants that you can see around you.
 \textit{
	\begin{itemize}
	\item ...the island's spectacular scenery.
	\item Sometimes they just drive slowly down the lane enjoying the scenery.
	\end{itemize}
}
\item uncountable noun \\
In a theatre, the \textbf{scenery} consists of the structures and painted backgrounds that show where the action in the play takes place.
 \textit{
	\begin{itemize}
	\end{itemize}
}
\item  \\
 a change of scenery \textit{
	\begin{itemize}
	\end{itemize}
}
\end{enumerate}

\section*{sink}
{\large \color{blue}  sinks  sinking  sank  sunk  }
\subsection*{Explain}
\begin{enumerate}
\item countable noun \\
A \textbf{sink} is a large fixed container in a kitchen, with taps to supply water. It is mainly used for washing dishes.
 \textit{
	\begin{itemize}
	\item The sink was full of dirty dishes.
	\item ...the kitchen sink.
	\end{itemize}
}
\item countable noun \\
A \textbf{sink} is the same as a washbasin or , basin .
 \textit{
	\begin{itemize}
	\item The bathroom is furnished with 2 toilets, 2 showers, and 2 sinks.
	\end{itemize}
}
\item verb \\
If a boat \textbf{sinks} or if someone or something \textbf{sinks} it, it disappears below the surface of a mass of water.
 \textit{
	\begin{itemize}
	\item In a naval battle your aim is to sink the enemy's ship.
	\item The boat was beginning to sink fast.
	\item The lifeboat crashed against the side of the sinking ship.
	\end{itemize}
}
\item verb \\
If something \textbf{sinks} , it disappears below the surface of a mass of water.
 \textit{
	\begin{itemize}
	\item A fresh egg will sink and an old egg will float.
	\end{itemize}
}
\item verb \\
If something \textbf{sinks} , it moves slowly downwards.
 \textit{
	\begin{itemize}
	\item Far off to the west the sun was sinking.
	\item When they came to build the southern spire the foundations began to sink.
	\end{itemize}
}
\item verb \\
If you \textbf{sink} , you move into a lower position, for example by sitting down in a chair or kneeling .
 \textit{
	\begin{itemize}
	\item Kate laughed, and sank down again to her seat.
	\item She sank into an armchair and crossed her legs.
	\item 'Don't you understand?' I moaned, sinking dramatically to my knees.
	\end{itemize}
}
\item verb \\
If something \textbf{sinks}  \textbf{to} a lower level or standard, it falls to that level or standard.
 \textit{
	\begin{itemize}
	\item Share prices would have sunk–hurting small and big investors.
	\item Pay increases have sunk to around seven per cent.
	\item The pound sank by nearly one per cent against the US dollar.
	\end{itemize}
}
\item adjective \\
People use \textbf{sink} school or \textbf{sink} estate to refer to a school or housing estate that is in a very poor area with few resources .
 \textit{
	\begin{itemize}
	\item ...unemployed teenagers from sink estates.
	\item He has transformed the local sink schools into beacons of hope for parents and children.
	\end{itemize}
}
\item verb \\
If your voice \textbf{sinks} , it becomes quieter .
 \textit{
	\begin{itemize}
	\item Her voice sank, and he moved closer to catch what she was saying.
	\item Her voice had sunk to a whisper.
	\end{itemize}
}
\item verb \\
To \textbf{sink}  \textbf{into} an unpleasant or undesirable  mood , situation, or state means to pass gradually into it.
 \textit{
	\begin{itemize}
	\item She'd sometimes sink into depression.
	\item That night he sank into a deep coma.
	\item Bulgaria's economy has sunk into chaos.
	\end{itemize}
}
\item verb \\
If your heart or your spirits \textbf{sink} , you become depressed or lose hope .
 \textit{
	\begin{itemize}
	\item My heart sank because I thought he was going to dump me for another girl.
	\item Her spirits sank lower and lower.
	\end{itemize}
}
\item verb \\
If something sharp \textbf{sinks} or \textbf{is sunk}  \textbf{into} something solid, it goes deeply into it.
 \textit{
	\begin{itemize}
	\item He sinks the needle into my arm.
	\item I sank my teeth into a peppermint cream.
	\item The spade sank into a clump of overgrown bushes.
	\end{itemize}
}
\item verb \\
If someone \textbf{sinks} a well, mine, or other large hole, they make a deep hole in the ground, usually by
digging or drilling.
 \textit{
	\begin{itemize}
	\item ...the site where Stephenson sank his first mineshaft.
	\item If they carry on sinking boreholes then the land is likely to subside.
	\item ...a one-thousand foot deep hole sunk into the rock.
	\end{itemize}
}
\item verb \\
If you \textbf{sink} money \textbf{into} a business or project, you spend money on it in the hope of making more money.
 \textit{
	\begin{itemize}
	\item He has already sunk $25million into the project.
	\end{itemize}
}
\item verb \\
If someone \textbf{sinks} a number of alcoholic drinks, they drink them quickly.
 \textit{
	\begin{itemize}
	\item She sank two glasses of white wine.
	\end{itemize}
}
\item verb \\
In golf , snooker , and some other games, if you \textbf{sink} a ball or a putt , you successfully hit the ball into a hole.
 \textit{
	\begin{itemize}
	\item He sank two crucial putts in the last three holes.
	\end{itemize}
}
\item  \\
 sink or swim \textit{
	\begin{itemize}
	\end{itemize}
}
\end{enumerate}

\section*{scheme}
{\large \color{blue}  schemes  scheming  schemed  }
\subsection*{Explain}
\begin{enumerate}
\item countable noun \\
A \textbf{scheme} is a plan or arrangement involving many people which is made by a government or other
organization.
 \textit{
	\begin{itemize}
	\item ...schemes to help combat unemployment.
	\item ...a private pension scheme.
	\item The company was pouring around $30 million into the scheme.
	\end{itemize}
}
\item countable noun \\
A \textbf{scheme} is someone's plan for achieving something.
 \textit{
	\begin{itemize}
	\item ...a quick money-making scheme to get us through the summer.
	\item They would first have to work out some scheme for getting the treasure out.
	\item Tourists can be vulnerable to scams and con schemes of all kinds.
	\end{itemize}
}
\item verb \\
If you say that people \textbf{are scheming} , you mean that they are making secret plans in order to gain something for themselves.
 \textit{
	\begin{itemize}
	\item Everyone's always scheming and plotting.
	\item The bride's family were scheming to prevent a wedding.
	\item They claimed that their opponents were scheming against them.
	\item You're a scheming little devil, aren't you?
	\end{itemize}
}
\item  \\
 scheme of things \textit{
	\begin{itemize}
	\end{itemize}
}
\item  \\
 the scheme of things the grand scheme of things \textit{
	\begin{itemize}
	\end{itemize}
}
\end{enumerate}

\section*{slam}
{\large \color{blue}  slams  slamming  slammed  }
\subsection*{Explain}
\begin{enumerate}
\item verb \\
If you \textbf{slam} a door or window or if it \textbf{slams} , it shuts noisily and with great force.
 \textit{
	\begin{itemize}
	\item She slammed the door and locked it behind her.
	\item I was relieved to hear the front door slam.
	\item He slammed the gate shut behind him.
	\end{itemize}
}
\item verb \\
If you \textbf{slam} something \textbf{down} , you put it there quickly and with great force.
 \textit{
	\begin{itemize}
	\item She listened in a mixture of shock and anger before slamming the phone down.
	\end{itemize}
}
\item verb \\
To \textbf{slam} someone or something means to criticize them very severely.
 \textit{
	\begin{itemize}
	\item The famed film-maker slammed the claims as 'an outrageous lie'.
	\item Britain has been slammed by the United Nations for having one of the worst race relations
records in the world.
	\end{itemize}
}
\item verb \\
If one thing \textbf{slams} into or against another, it crashes into it with great force.
 \textit{
	\begin{itemize}
	\item The plane slammed into the building after losing an engine shortly after take-off.
	\item He slammed me against the ground.
	\end{itemize}
}
\end{enumerate}

\section*{spark}
{\large \color{blue}  sparks  sparking  sparked  }
\subsection*{Explain}
\begin{enumerate}
\item countable noun \\
A \textbf{spark} is a tiny  bright piece of burning material that flies up from something that is burning.
 \textit{
	\begin{itemize}
	\item The fire gradually got bigger and bigger. Sparks flew off in all directions.
	\end{itemize}
}
\item countable noun \\
A \textbf{spark} is a flash of light caused by electricity . It often makes a loud sound.
 \textit{
	\begin{itemize}
	\item He passed an electric spark through a mixture of gases.
	\end{itemize}
}
\item verb \\
If something \textbf{sparks} , sparks of fire or light come from it.
 \textit{
	\begin{itemize}
	\item The wires were sparking above me.
	\item I stared into the flames of the fire as it sparked to life.
	\end{itemize}
}
\item verb \\
If a burning object or electricity \textbf{sparks} a fire, it causes a fire.
 \textit{
	\begin{itemize}
	\item A dropped cigarette may have sparked the fire.
	\end{itemize}
}
\item countable noun \\
A \textbf{spark of} a quality or feeling , especially a desirable one, is a small but noticeable amount of it.
 \textit{
	\begin{itemize}
	\item His music lacked that vital spark of imagination.
	\item Even Oliver felt a tiny spark of excitement.
	\end{itemize}
}
\item verb \\
If one thing \textbf{sparks} another, the first thing causes the second thing to start  happening .
 \textbf{Spark off} means the same as spark .
 \textit{
	\begin{itemize}
	\item What was it that sparked your interest in motoring?
	\item The proposals are expected to spark heated debate.
	\item ...a row sparked by a comment about his sister.
	\item That incident sparked it off.
	\item His book sparked off a revolution in the way we think about animals.
	\item ...a political crisis sparked off by religious violence.
	\end{itemize}
}
\item  \\
 sparks fly \textit{
	\begin{itemize}
	\end{itemize}
}
\end{enumerate}

\section*{starve}
{\large \color{blue}  starves  starving  starved  }
\subsection*{Explain}
\begin{enumerate}
\item verb \\
If people \textbf{starve} , they suffer greatly from lack of food which sometimes  leads to their death .
 \textit{
	\begin{itemize}
	\item A number of the prisoners we saw are starving.
	\item In the 1930s, millions of Ukrainians starved to death or were deported.
	\item Getting food to starving people does nothing to stop the war.
	\end{itemize}
}
\item verb \\
To \textbf{starve} someone means not to give them any food.
 \textit{
	\begin{itemize}
	\item They harassed and starved the Native people.
	\item Judy decided I was starving myself.
	\end{itemize}
}
\item verb \\
If a person or thing \textbf{is starved}  \textbf{of} something that they need , they are suffering because they are not getting enough of it.
 \textit{
	\begin{itemize}
	\item The electricity industry is not the only one to have been starved of investment.
	\item ...companies in danger of being starved of capital by their banks.
	\item ...an audience hungry for American films and long starved of choice.
	\end{itemize}
}
\end{enumerate}

\section*{stationery}
{\large \color{blue}  }
\subsection*{Explain}
\begin{enumerate}
\item uncountable noun \\
\textbf{Stationery} is paper, envelopes, and other materials or equipment used for writing.
 \textit{
	\begin{itemize}
	\end{itemize}
}
\end{enumerate}

\section*{supply}
{\large \color{blue}  supplies  supplying  supplied  }
\subsection*{Explain}
\begin{enumerate}
\item verb \\
If you \textbf{supply} someone with something that they want or need, you give them a quantity of it.
 \textit{
	\begin{itemize}
	\item ...an agreement not to produce or supply chemical weapons.
	\item Tourist offices will supply you with a free basic street map.
	\item ...the blood vessels supplying oxygen to the brain.
	\end{itemize}
}
\item plural noun \\
You can use \textbf{supplies} to refer to food, equipment, and other essential things that people need, especially when these are provided in large quantities.
 \textit{
	\begin{itemize}
	\item What happens when food and gasoline supplies run low?
	\item The country's only supplies are those it can import by lorry from Vietnam.
	\end{itemize}
}
\item variable noun \\
A \textbf{supply}  \textbf{of} something is an amount of it which someone has or which is available for them to
use.
 \textit{
	\begin{itemize}
	\item The brain requires a constant supply of oxygen.
	\item Most urban water supplies in the country now contain fluoride.
	\end{itemize}
}
\item uncountable noun \\
\textbf{Supply} is the quantity of goods and services that can be made available for people to buy .
 \textit{
	\begin{itemize}
	\item Prices change according to supply and demand.
	\end{itemize}
}
\item verb \\
If you \textbf{supply} a missing word or piece of information , for example in a puzzle , you say or write it because you know it.
 \textit{
	\begin{itemize}
	\item Supply the missing word(s) and you could win a T-shirt.
	\end{itemize}
}
\item  \\
 in short supply \textit{
	\begin{itemize}
	\end{itemize}
}
\end{enumerate}

\section*{tilt}
{\large \color{blue}  tilts  tilting  tilted  }
\subsection*{Explain}
\begin{enumerate}
\item verb \\
If you \textbf{tilt} an object or if it \textbf{tilts} , it moves into a sloping position with one end or side higher than the other.
 \textit{
	\begin{itemize}
	\item She tilted the mirror and began to comb her hair.
	\item Leonard tilted his chair back on two legs and stretched his long body.
	\item The boat instantly tilted, filled and sank.
	\end{itemize}
}
\item verb \\
If you \textbf{tilt} part of your body, usually your head, you move it slightly  upwards or to one side.
 \textbf{Tilt} is also a noun .
 \textit{
	\begin{itemize}
	\item Mari tilted her head back so that she could look at him.
	\item The nurse tilted his head to the side and inspected the wound.
	\item She tilted her face to kiss me quickly on the chin.
	\item He opened the rear door for me with an apologetic tilt of his head.
	\end{itemize}
}
\item countable noun \\
The \textbf{tilt} of something is the fact that it tilts or slopes, or the angle at which it tilts or slopes.
 \textit{
	\begin{itemize}
	\item ...calculations based on our understanding of the tilt of the Earth's axis.
	\item ...the abrupt tilt of the hill.
	\item The 3-metre-square slabs are on a tilt.
	\end{itemize}
}
\item verb \\
If a person or thing \textbf{tilts}  \textbf{towards} a particular opinion or if something \textbf{tilts} them \textbf{towards} it, they change slightly so that they become more in agreement with that opinion or position.
 \textit{
	\begin{itemize}
	\item When the political climate tilted towards fundamentalism, he was threatened.
	\item He continued to urge the Conservative Party to tilt rightwards.
	\item The paper has done much to tilt American public opinion in favour of intervention.
	\end{itemize}
}
\item singular noun \\
If there is a \textbf{tilt towards} a particular opinion or position, that opinion or position is favoured or begins to be favoured.
 \textit{
	\begin{itemize}
	\item The chairman criticised the plan for its tilt towards higher taxes rather than lower
spending.
	\end{itemize}
}
\item countable noun \\
A \textbf{tilt at} something is an attempt to win or obtain it.
 \textit{
	\begin{itemize}
	\item His first tilt at Parliament came in the same year but he failed to win the seat.
	\item He was determined to use his remaining year with Manchester United for one last tilt
at the League title.
	\end{itemize}
}
\item  \\
 at full tilt/full tilt \textit{
	\begin{itemize}
	\end{itemize}
}
\end{enumerate}

\section*{turkey}
{\large \color{blue}  turkeys  }
\subsection*{Explain}
\begin{enumerate}
\item countable noun \\
A \textbf{turkey} is a large bird that is kept on a farm for its meat .
 \textbf{Turkey} is the flesh of this bird eaten as food.
 \textit{
	\begin{itemize}
	\item It's a proper Christmas dinner, with turkey and bread sauce.
	\end{itemize}
}
\end{enumerate}

\section*{tire}
{\large \color{blue}  tires  tiring  tired  }
\subsection*{Explain}
\begin{enumerate}
\item verb \\
If something \textbf{tires} you or if you \textbf{tire} , you feel that you have used a lot of energy and you want to rest or sleep .
 \textit{
	\begin{itemize}
	\item If driving tires you, take the train.
	\item He tired easily, though he was unable to sleep well at night.
	\end{itemize}
}
\item verb \\
If you \textbf{tire of} something, you no longer wish to do it, because you have become bored of it or unhappy with it.
 \textit{
	\begin{itemize}
	\item He felt he would never tire of international cricket.
	\item Sooner or later he may tire of constantly putting himself last.
	\end{itemize}
}
\item countable noun \\
A \textbf{tire} is the same as a tyre .
 \textit{
	\begin{itemize}
	\end{itemize}
}
\end{enumerate}

\section*{volcano}
{\large \color{blue}  volcanoes  }
\subsection*{Explain}
\begin{enumerate}
\item countable noun \\
A \textbf{volcano} is a mountain from which hot melted rock, gas, steam , and ash from inside the Earth  sometimes  burst .
 \textit{
	\begin{itemize}
	\item The volcano erupted last year killing about 600 people.
	\item Etna is Europe's most active volcano.
	\end{itemize}
}
\end{enumerate}

\section*{vibrate}
{\large \color{blue}  vibrates  vibrating  vibrated  }
\subsection*{Explain}
\begin{enumerate}
\item verb \\
If something \textbf{vibrates} or if you \textbf{vibrate} it, it shakes with repeated small, quick  movements .
 \textit{
	\begin{itemize}
	\item The ground shook and the cliffs seemed to vibrate.
	\item The noise vibrated the table.
	\end{itemize}
}
\end{enumerate}

\section*{warehouse}
{\large \color{blue}  warehouses  }
\subsection*{Explain}
\begin{enumerate}
\item countable noun \\
A \textbf{warehouse} is a large building where raw materials or manufactured goods are stored until they are exported to other countries or distributed to shops to be sold .
 \textit{
	\begin{itemize}
	\end{itemize}
}
\end{enumerate}

\section*{wink}
{\large \color{blue}  winks  winking  winked  }
\subsection*{Explain}
\begin{enumerate}
\item verb \\
When you \textbf{wink}  \textbf{at} someone, you look towards them and close one eye very briefly, usually as a signal that something is
a joke or a secret .
 \textbf{Wink} is also a noun .
 \textit{
	\begin{itemize}
	\item Brian winked at his bride-to-be.
	\item He smiled, winked and nodded, giving his seal of approval.
	\item I gave her a wink.
	\end{itemize}
}
\item verb \\
If a lamp  \textbf{winks} , it shines or reflects light in short flashes.
 \textit{
	\begin{itemize}
	\item From the hotel window, they could see lights winking on the bay.
	\end{itemize}
}
\item  \\
 not sleep a wink/not get a wink of sleep \textit{
	\begin{itemize}
	\end{itemize}
}
\end{enumerate}

\section*{wind}
{\large \color{blue}  winds  winding  winded  }
\subsection*{Explain}
\begin{enumerate}
\item variable noun \\
A \textbf{wind} is a current of air that is moving across the earth's surface.
 \textit{
	\begin{itemize}
	\item There was a strong wind blowing.
	\item Then the wind dropped and the surface of the sea was still.
	\item The leaves rustled in the wind.
	\item During the night a gust of wind had blown the pot over.
	\end{itemize}
}
\item countable noun \\
Journalists often refer to a trend or factor that influences events as a \textbf{wind}  \textbf{of} a particular kind.
 \textit{
	\begin{itemize}
	\item The winds of change are blowing across the country.
	\item The world's entire aerospace industry is feeling the chill winds of recession.
	\end{itemize}
}
\item verb \\
If you \textbf{are winded} by something such as a blow, the air is suddenly  knocked out of your lungs so that you have difficulty breathing for a short time.
 \textit{
	\begin{itemize}
	\item He was winded and shaken.
	\item The cow stamped on his side, winding him.
	\end{itemize}
}
\item uncountable noun \\
\textbf{Wind} is the air that you sometimes swallow with food or drink, or gas that is produced in your intestines , which causes an uncomfortable feeling.
 \textit{
	\begin{itemize}
	\end{itemize}
}
\item verb \\
If you \textbf{wind} a baby, you hit its back gently in order to help it to release air from its stomach .
 \textit{
	\begin{itemize}
	\item If he cries when you put him down after a feed, try winding him.
	\end{itemize}
}
\item adjective \\
The \textbf{wind} section of an orchestra or band is the group of people who produce musical sounds
by blowing into their instruments.
 \textit{
	\begin{itemize}
	\end{itemize}
}
\item  \\
 to break wind \textit{
	\begin{itemize}
	\end{itemize}
}
\item  \\
 to get wind of sth \textit{
	\begin{itemize}
	\end{itemize}
}
\item  \\
 in the wind \textit{
	\begin{itemize}
	\end{itemize}
}
\item  \\
 to put the wind up sb \textit{
	\begin{itemize}
	\end{itemize}
}
\item  \\
 to sail close to the wind \textit{
	\begin{itemize}
	\end{itemize}
}
\item  \\
 to take the wind out of someone's sails \textit{
	\begin{itemize}
	\end{itemize}
}
\item  \\
 which way/how the wind is blowing \textit{
	\begin{itemize}
	\end{itemize}
}
\end{enumerate}

\section*{zip}
{\large \color{blue}  zips  zipping  zipped  }
\subsection*{Explain}
\begin{enumerate}
\item countable noun \\
A \textbf{zip} or \textbf{zip fastener} is a device used to open and close parts of clothes and bags. It consists of two rows of metal or plastic teeth which
 separate or fasten together as you pull a small tag along them.
 \textit{
	\begin{itemize}
	\item He pulled the zip of his leather jacket down slightly.
	\end{itemize}
}
\item verb \\
When you \textbf{zip} something, you fasten it using a zip.
 \textit{
	\begin{itemize}
	\item She zipped her jeans.
	\item I slowly zipped and locked the heavy black nylon bags.
	\end{itemize}
}
\item verb \\
To \textbf{zip} a computer file means to compress it so that it needs less space for storage on disk and can be transmitted more quickly.
 \textbf{Zip up} means the same as zip .
 \textit{
	\begin{itemize}
	\item These files have been zipped up to take up less disk space.
	\end{itemize}
}
\item verb \\
If you say that something or someone \textbf{zips}  somewhere , you mean that they move there very quickly.
 \textit{
	\begin{itemize}
	\item My craft zipped across the bay.
	\item Max zips back and forth across the living room.
	\end{itemize}
}
\item uncountable noun \\
If you say that someone or something has \textbf{zip} , you mean that they show a lot of energy and enthusiasm .
 \textit{
	\begin{itemize}
	\item Conductors are introducing audience participation to inject zip into their concerts.
	\item He gives the choreography his usual class and zip.
	\end{itemize}
}
\end{enumerate}

\section*{advice}
{\large \color{blue}  }
\subsection*{Explain}
\begin{enumerate}
\item uncountable noun \\
If you give someone \textbf{advice} , you tell them what you think they should do in a particular situation .
 \textit{
	\begin{itemize}
	\item Don't be afraid to ask for advice about ordering the meal.
	\item Your community officer can give you advice on how to prevent crime in your area.
	\item Take my advice and stay away from him!
	\item Most foreign nationals have now left the country on the advice of their governments.
	\end{itemize}
}
\item  \\
 take advice \textit{
	\begin{itemize}
	\end{itemize}
}
\end{enumerate}

\section*{breed}
{\large \color{blue}  breeds  breeding  bred  }
\subsection*{Explain}
\begin{enumerate}
\item countable noun \\
A \textbf{breed} of a pet animal or farm animal is a particular type of it. For example , terriers are a breed of dog .
 \textit{
	\begin{itemize}
	\item ...rare breeds of cattle.
	\item Certain breeds are more dangerous than others.
	\end{itemize}
}
\item verb \\
If you \textbf{breed} animals or plants, you keep them for the purpose of producing more animals or plants with particular qualities, in a controlled way.
 \textit{
	\begin{itemize}
	\item He lived alone, breeding horses and dogs.
	\item He used to breed dogs for the police.
	\item These dogs are bred to fight.
	\end{itemize}
}
\item  \\
 See also  cross-breed \textit{
	\begin{itemize}
	\end{itemize}
}
\item verb \\
When animals \textbf{breed} , they have babies .
 \textit{
	\begin{itemize}
	\item Frogs will usually breed in any convenient pond.
	\item The area now attracts over 60 species of breeding birds.
	\end{itemize}
}
\item verb \\
If you say that something \textbf{breeds}  bad  feeling or bad behaviour , you mean that it causes bad feeling or bad behaviour to develop.
 \textit{
	\begin{itemize}
	\item If they are unemployed it's bound to breed resentment.
	\item Violence breeds violence.
	\end{itemize}
}
\item countable noun \\
You can refer to someone or something as one of a particular \textbf{breed}  \textbf{of} person or thing when you want to talk about what they are like.
 \textit{
	\begin{itemize}
	\item Sue is one of the new breed of British women squash players who are making a real
impact.
	\item The new breed of walking holidays puts the emphasis on enjoyment, not endurance.
	\end{itemize}
}
\item  \\
 to be born and bred \textit{
	\begin{itemize}
	\end{itemize}
}
\end{enumerate}

\section*{broadcast}
{\large \color{blue}  broadcasts  broadcasting  }
\subsection*{Explain}
\begin{enumerate}
\item countable noun \\
A \textbf{broadcast} is a programme, performance , or speech on the radio or on television.
 \textit{
	\begin{itemize}
	\item In a broadcast on state radio the government announced that it was willing to resume
peace talks.
	\end{itemize}
}
\item verb \\
To \textbf{broadcast} a programme means to send it out by radio waves , so that it can be heard on the radio or seen on television.
 \textit{
	\begin{itemize}
	\item The concert will be broadcast live on television and radio.
	\item CNN also broadcasts in Europe.
	\end{itemize}
}
\end{enumerate}

\section*{book}
{\large \color{blue}  books  booking  booked  }
\subsection*{Explain}
\begin{enumerate}
\item countable noun \\
A \textbf{book} is a number of pieces of paper, usually with words printed on them, which are fastened
together and fixed inside a cover of stronger paper or cardboard . Books contain information, stories, or poetry , for example.
 \textit{
	\begin{itemize}
	\item His eighth book came out earlier this year and was an instant best-seller.
	\item 'Robinson Crusoe' is one of the most famous books in the world.
	\item ...the author of a book on politics.
	\item ...a book about witches.
	\item ...reference books.
	\end{itemize}
}
\item countable noun \\
A \textbf{book}  \textbf{of} something such as stamps, matches, or tickets is a small number of them fastened
together between thin cardboard covers.
 \textit{
	\begin{itemize}
	\item Can I have a book of first class stamps please?
	\end{itemize}
}
\item verb \\
When you \textbf{book} something such as a hotel room or a ticket, you arrange to have it or use it at a particular time.
 \textit{
	\begin{itemize}
	\item British officials have booked hotel rooms for the women and children.
	\item Laurie revealed she had booked herself a flight home last night.
	\item ...three-star restaurants that are normally booked for months in advance.
	\end{itemize}
}
\item plural noun \\
A company's or organization's \textbf{books} are its records of money that has been spent and earned or of the names of people who belong to it.
 \textit{
	\begin{itemize}
	\item For the most part he left the books to his managers and accountants.
	\item Around 12 per cent of the people on our books are in the computing industry.
	\end{itemize}
}
\item verb \\
When a referee \textbf{books} a football player who has seriously broken the rules of the game, he or she officially writes down the player's name.
 \textit{
	\begin{itemize}
	\item The referee booked him in the first half for a tussle with the goalie.
	\end{itemize}
}
\item verb \\
When a police officer \textbf{books} someone, he or she officially records their name and the offence that they may be
charged with.
 \textit{
	\begin{itemize}
	\item They took him to the station and booked him for assault with a deadly weapon.
	\end{itemize}
}
\item countable noun \\
In a very long written work such as the Bible, a \textbf{book} is one of the sections into which it is divided.
 \textit{
	\begin{itemize}
	\end{itemize}
}
\item  \\
 to be in someone's bad books \textit{
	\begin{itemize}
	\end{itemize}
}
\item  \\
 bring sb to book \textit{
	\begin{itemize}
	\end{itemize}
}
\item  \\
 a closed book \textit{
	\begin{itemize}
	\end{itemize}
}
\item  \\
 fully booked/booked solid \textit{
	\begin{itemize}
	\end{itemize}
}
\item  \\
 in my book \textit{
	\begin{itemize}
	\end{itemize}
}
\item  \\
 to throw the book at someone \textit{
	\begin{itemize}
	\end{itemize}
}
\end{enumerate}

\section*{carve}
{\large \color{blue}  carves  carving  carved  }
\subsection*{Explain}
\begin{enumerate}
\item verb \\
If you \textbf{carve} an object , you make it by cutting it out of a substance such as wood or stone . If you \textbf{carve} something such as wood or stone into an object, you make the object by cutting it
out.
 \textit{
	\begin{itemize}
	\item One of the prisoners has carved a beautiful wooden chess set.
	\item He carves his figures from white pine.
	\item I picked up a piece of wood and started carving.
	\item ...carved stone figures.
	\end{itemize}
}
\item verb \\
If you \textbf{carve}  writing or a design  \textbf{on} an object, you cut it into the surface of the object.
 \textit{
	\begin{itemize}
	\item He carved his name on his desk.
	\item The ornately carved doors were made in the seventeenth century.
	\end{itemize}
}
\item verb \\
If you \textbf{carve} a piece of cooked meat, you cut slices from it so that you can eat it.
 \textit{
	\begin{itemize}
	\item Andrew began to carve the chicken.
	\item Carve the meat into slices.
	\end{itemize}
}
\item verb \\
If you \textbf{carve} a career or a niche  \textbf{for} yourself, you succeed in getting the career or the position that you want by your own efforts .
 \textbf{Carve out}  means the same as carve .
 \textit{
	\begin{itemize}
	\item She has carved a niche for herself as a comic actor.
	\item He is hoping to carve out a much greater role for himself.
	\item Wood has not had much luck in carving out a career.
	\end{itemize}
}
\item verb \\
If a road  \textbf{is carved} through a place, it is built so that it goes through that place.
 \textit{
	\begin{itemize}
	\item Two three-lane roads will be carved through countryside.
	\end{itemize}
}
\end{enumerate}

\section*{century}
{\large \color{blue}  centuries  }
\subsection*{Explain}
\begin{enumerate}
\item countable noun \\
A \textbf{century} is a period of a hundred years that is used when stating a date. For example , the 19th century was the period from 1801 to 1900.
 \textit{
	\begin{itemize}
	\item ...celebrated figures of the late eighteenth century.
	\item ...a 17th-century merchant's house.
	\end{itemize}
}
\item countable noun \\
A \textbf{century} is any period of a hundred years.
 \textit{
	\begin{itemize}
	\item The drought there is the worst in a century.
	\end{itemize}
}
\item countable noun \\
In cricket , a \textbf{century} is a score of one hundred runs or more by one batsman .
 \textit{
	\begin{itemize}
	\end{itemize}
}
\end{enumerate}

\section*{circulate}
{\large \color{blue}  circulates  circulating  circulated  }
\subsection*{Explain}
\begin{enumerate}
\item verb \\
If a piece of writing  \textbf{circulates} or \textbf{is circulated} , copies of it are passed round among a group of people.
 \textit{
	\begin{itemize}
	\item The document was previously circulated in New York at the United Nations.
	\item Public employees, teachers and liberals are circulating a petition for his recall.
	\item This year anonymous leaflets have been circulating in Beijing.
	\end{itemize}
}
\item verb \\
If something such as a rumour  \textbf{circulates} or \textbf{is circulated} , the people in a place tell it to each other.
 \textit{
	\begin{itemize}
	\item Rumours were already beginning to circulate that the project might have to be abandoned.
	\item I deeply resented those sort of rumours being circulated at a time of deeply personal
grief.
	\end{itemize}
}
\item verb \\
When something \textbf{circulates} , it moves easily and freely within a closed place or system.
 \textit{
	\begin{itemize}
	\item ...a virus which circulates via the bloodstream and causes ill health in a variety
of organs.
	\item Cooking odours can circulate throughout the entire house.
	\end{itemize}
}
\item verb \\
If you \textbf{circulate} at a party , you move among the guests and talk to different people.
 \textit{
	\begin{itemize}
	\item Let me get you something to drink, then I must circulate.
	\end{itemize}
}
\end{enumerate}

\section*{chocolate}
{\large \color{blue}  chocolates  }
\subsection*{Explain}
\begin{enumerate}
\item variable noun \\
\textbf{Chocolate} is a sweet  hard food made from cocoa  beans . It is usually brown in colour and is eaten as a sweet.
 \textit{
	\begin{itemize}
	\item ...a bar of chocolate.
	\item Do you want some chocolate?
	\item ...rich chocolate cake.
	\end{itemize}
}
\item uncountable noun \\
\textbf{Chocolate} or \textbf{hot chocolate} is a drink made from a powder containing chocolate. It is usually made with hot milk .
 A cup of chocolate can be referred to as a \textbf{chocolate} or a \textbf{hot chocolate} .
 \textit{
	\begin{itemize}
	\item ...a small cafeteria where the visitors can buy tea, coffee and chocolate.
	\item I sipped the hot chocolate she had made.
	\item I'll have a hot chocolate please.
	\end{itemize}
}
\item countable noun \\
\textbf{Chocolates} are small sweets or nuts  covered with a layer of chocolate. They are usually sold in a box .
 \textit{
	\begin{itemize}
	\item ...a box of chocolates.
	\item Here, have a chocolate.
	\end{itemize}
}
\item colour \\
\textbf{Chocolate} is used to describe things that are dark brown in colour.
 \textit{
	\begin{itemize}
	\item The curtains and the coverlet of the bed were chocolate velvet.
	\item She placed the chocolate-coloured coat beside the case.
	\end{itemize}
}
\end{enumerate}

\section*{clothe}
{\large \color{blue}  clothes  clothing  clothed  }
\subsection*{Explain}
\begin{enumerate}
\item verb \\
To \textbf{clothe} someone means to provide them with clothes to wear.
 \textit{
	\begin{itemize}
	\item She was on her own with two kids to feed and clothe.
	\end{itemize}
}
\end{enumerate}

\section*{coincidence}
{\large \color{blue}  coincidences  }
\subsection*{Explain}
\begin{enumerate}
\item variable noun \\
A \textbf{coincidence} is when two or more similar or related events occur at the same time by chance and
without any planning .
 \textit{
	\begin{itemize}
	\item Mr. Berry said the timing was a coincidence and that his decision was unrelated to
Mr. Roman's departure.
	\item ...a string of amazing coincidences.
	\item The premises of Chabert and Sons were situated by the river and, by coincidence,
not too far away from where Eric Talbot had met his death.
	\end{itemize}
}
\end{enumerate}

\section*{compute}
{\large \color{blue}  computes  computing  computed  }
\subsection*{Explain}
\begin{enumerate}
\item verb \\
To \textbf{compute} a quantity or number means to calculate it.
 \textit{
	\begin{itemize}
	\item I tried to compute the cash value of the ponies and horse boxes.
	\end{itemize}
}
\end{enumerate}

\section*{darling}
{\large \color{blue}  darlings  }
\subsection*{Explain}
\begin{enumerate}
\item countable noun \\
You call someone \textbf{darling} if you love them or like them very much.
 \textit{
	\begin{itemize}
	\item Thank you, darling.
	\item Oh darling, I love you.
	\end{itemize}
}
\item vocative noun \\
In some parts of Britain, people call other people \textbf{darling} as a sign of friendliness.
 \textit{
	\begin{itemize}
	\end{itemize}
}
\item adjective \\
Some people use \textbf{darling} to describe someone or something that they love or like very much.
 \textit{
	\begin{itemize}
	\item To have a darling baby boy was the greatest gift I could imagine.
	\item What a darling film–everyone adored it.
	\end{itemize}
}
\item countable noun \\
If you describe someone as a \textbf{darling} , you are fond of them and think that they are nice .
 \textit{
	\begin{itemize}
	\item He's such a darling.
	\end{itemize}
}
\item countable noun \\
The \textbf{darling}  \textbf{of} a group of people is someone who is especially liked by that group.
 \textit{
	\begin{itemize}
	\item Rajneesh was the darling of a prosperous family.
	\end{itemize}
}
\end{enumerate}

\section*{consent}
{\large \color{blue}  consents  consenting  consented  }
\subsection*{Explain}
\begin{enumerate}
\item uncountable noun \\
If you give your \textbf{consent}  \textbf{to} something, you give someone permission to do it.
 \textit{
	\begin{itemize}
	\item Pollard finally gave his consent to the search.
	\item Can my child be medically examined without my consent?
	\end{itemize}
}
\item verb \\
If you \textbf{consent}  \textbf{to} something, you agree to do it or to allow it to be done.
 \textit{
	\begin{itemize}
	\item He finally consented to go.
	\item The patient must consent to the surgery.
	\item I was a little surprised when she consented.
	\end{itemize}
}
\item  \\
 by common/mutual consent \textit{
	\begin{itemize}
	\end{itemize}
}
\item  \\
 by general/common consent \textit{
	\begin{itemize}
	\end{itemize}
}
\end{enumerate}

\section*{defeat}
{\large \color{blue}  defeats  defeating  defeated  }
\subsection*{Explain}
\begin{enumerate}
\item verb \\
If you \textbf{defeat} someone, you win a victory over them in a battle , game, or contest.
 \textit{
	\begin{itemize}
	\item His guerrillas defeated the colonial army in 1954.
	\item The NHL Stanley Cup was won by the Montreal Canadians, who defeated the Boston Bruins
four games to one.
	\end{itemize}
}
\item verb \\
If a proposal or motion in a debate  \textbf{is defeated} , more people vote against it than for it.
 \textit{
	\begin{itemize}
	\item The proposal was defeated by just one vote.
	\end{itemize}
}
\item verb \\
If a task or a problem  \textbf{defeats} you, it is so difficult that you cannot do it or solve it.
 \textit{
	\begin{itemize}
	\item There were times when the challenges of writing such a huge novel almost defeated
her.
	\end{itemize}
}
\item verb \\
To \textbf{defeat} an action or plan means to cause it to fail .
 \textit{
	\begin{itemize}
	\item The navy played a limited but significant role in defeating the rebellion.
	\item He swore to defeat Odin's plan.
	\end{itemize}
}
\item variable noun \\
\textbf{Defeat} is the experience of being beaten in a battle, game, or contest, or of failing to achieve what you wanted to.
 \textit{
	\begin{itemize}
	\item The most important thing is not to admit defeat until you really have to.
	\item This vote is seen as a defeat for the liberal elite.
	\item A 2-1 defeat by Sweden left them bottom of Group One.
	\end{itemize}
}
\end{enumerate}

\section*{contact}
{\large \color{blue}  contacts  contacting  contacted  }
\subsection*{Explain}
\begin{enumerate}
\item uncountable noun \\
\textbf{Contact} involves meeting or communicating with someone, especially regularly.
 \textit{
	\begin{itemize}
	\item Opposition leaders are denying any contact with the rebels.
	\item He forbade contacts between directors and executives outside his presence.
	\end{itemize}
}
\item  \\
 in contact (with sb) \textit{
	\begin{itemize}
	\end{itemize}
}
\item verb \\
If you \textbf{contact} someone, you telephone them, write to them, or go to see them in order to tell or ask them something.
 \textit{
	\begin{itemize}
	\item Contact the Tourist Information Bureau for further details.
	\item When she first contacted me, Frances was upset.
	\end{itemize}
}
\item adjective \\
Your \textbf{contact}  details or number are information such as a telephone number where you can be contacted.
 \textit{
	\begin{itemize}
	\item You must leave your full name and contact details when you phone.
	\end{itemize}
}
\item uncountable noun \\
If you come \textbf{into contact with} someone or something, you meet that person or thing in the course of your work or other activities.
 \textit{
	\begin{itemize}
	\item Doctors I came into contact with voiced their concern.
	\item The college has brought me into contact with western ideas.
	\end{itemize}
}
\item  \\
 make contact (with sb) \textit{
	\begin{itemize}
	\end{itemize}
}
\item  \\
 to lose contact \textit{
	\begin{itemize}
	\end{itemize}
}
\item uncountable noun \\
When people or things are in \textbf{contact} , they are touching each other.
 \textit{
	\begin{itemize}
	\item They compared how these organisms behaved when left in contact with different materials.
	\item The cry occurs when air is brought into contact with the baby's larynx.
	\item There was no physical contact, nor did I want any.
	\item This shows where the foot and shoe are in contact.
	\end{itemize}
}
\item uncountable noun \\
Radio  \textbf{contact} is communication by means of radio.
 \textit{
	\begin{itemize}
	\item He failed to make radio contact.
	\item The plane lost contact with the control tower shortly after take-off.
	\end{itemize}
}
\item countable noun \\
A \textbf{contact} is someone you know in an organization or profession who helps you or gives you information.
 \textit{
	\begin{itemize}
	\item Their contact in the United States Embassy was called Phil.
	\end{itemize}
}
\end{enumerate}

\section*{desk}
{\large \color{blue}  desks  }
\subsection*{Explain}
\begin{enumerate}
\item countable noun \\
A \textbf{desk} is a table, often with drawers, which you sit at to write or work.
 \textit{
	\begin{itemize}
	\end{itemize}
}
\item singular noun \\
The place in a hotel, hospital , airport , or other building where you check in or obtain  information is referred to as a particular \textbf{desk} .
 \textit{
	\begin{itemize}
	\item I spoke to the girl on the reception desk.
	\item A map and a bird-watchers' field checklist are available at the front desk.
	\item ...the main information desk.
	\end{itemize}
}
\item singular noun \\
A particular department of a broadcasting  company , or of a newspaper or magazine company, can be referred to as a particular \textbf{desk} .
 \textit{
	\begin{itemize}
	\item Let our news desk know as quickly as possible on 414 3926.
	\item Over now to Simon Ingram at the sports desk.
	\end{itemize}
}
\end{enumerate}

\section*{convict}
{\large \color{blue}  convicts  convicting  convicted  }
\subsection*{Explain}
\begin{enumerate}
\item verb \\
If someone \textbf{is convicted}  \textbf{of} a crime , they are found guilty of that crime in a law court .
 \textit{
	\begin{itemize}
	\item In 1977 he was convicted of murder and sentenced to life imprisonment.
	\item There was insufficient evidence to convict him.
	\item ...a convicted drug dealer.
	\end{itemize}
}
\item countable noun \\
A \textbf{convict} is someone who is in prison.
 \textit{
	\begin{itemize}
	\end{itemize}
}
\end{enumerate}

\section*{dessert}
{\large \color{blue}  desserts  }
\subsection*{Explain}
\begin{enumerate}
\item variable noun \\
\textbf{Dessert} is something sweet, such as fruit or a pudding , that you eat at the end of a meal.
 \textit{
	\begin{itemize}
	\item She had homemade ice cream for dessert.
	\item I am partial to desserts that combine fresh fruit with fine pastry.
	\end{itemize}
}
\end{enumerate}

\section*{corrode}
{\large \color{blue}  corrodes  corroding  corroded  }
\subsection*{Explain}
\begin{enumerate}
\item verb \\
If metal or stone  \textbf{corrodes} , or \textbf{is corroded} , it is gradually destroyed by a chemical or by rust.
 \textit{
	\begin{itemize}
	\item He has devised a process for making gold wires which neither corrode nor oxidise.
	\item Engineers found the structure had been corroded by moisture.
	\item Acid rain destroys trees and corrodes buildings.
	\end{itemize}
}
\item verb \\
To \textbf{corrode} something means to gradually make it worse or weaker .
 \textit{
	\begin{itemize}
	\item Suffering was easier to bear than the bitterness he felt corroding his spirit.
	\item The overwhelming guilt ultimately corrodes his sanity.
	\end{itemize}
}
\end{enumerate}

\section*{dimension}
{\large \color{blue}  dimensions  }
\subsection*{Explain}
\begin{enumerate}
\item countable noun \\
A particular \textbf{dimension} of something is a particular aspect of it.
 \textit{
	\begin{itemize}
	\item There is a political dimension to the accusations.
	\item This adds a new dimension to our work.
	\end{itemize}
}
\item plural noun \\
If you talk about the \textbf{dimensions} of a situation or problem , you are talking about its extent and size.
 \textit{
	\begin{itemize}
	\item He considers the dimensions of the problem.
	\item The dimensions of the market collapse were certainly not anticipated.
	\end{itemize}
}
\item countable noun \\
A \textbf{dimension} is a measurement such as length, width, or height. If you talk about the \textbf{dimensions} of an object or place, you are referring to its size and proportions .
 \textit{
	\begin{itemize}
	\item Drilling will continue on the site to assess the dimensions of the new oilfield.
	\item I don't think it would spoil the dimensions of the room.
	\item ...the grandiose dimensions of the building.
	\end{itemize}
}
\item countable noun \\
In mathematics and science , \textbf{dimension} is used in describing  spatial  concepts such as points, lines , and solids .
 \textit{
	\begin{itemize}
	\end{itemize}
}
\end{enumerate}

\section*{counsel}
{\large \color{blue}  counsels  counselling  counselled  }
\subsection*{Explain}
\begin{enumerate}
\item uncountable noun \\
\textbf{Counsel} is advice.
 \textit{
	\begin{itemize}
	\item He had always been able to count on her wise counsel.
	\item His parishioners sought his counsel and loved him.
	\end{itemize}
}
\item verb \\
If you \textbf{counsel} someone \textbf{to} take a course of action , or if you \textbf{counsel} a course of action, you advise that course of action.
 \textit{
	\begin{itemize}
	\item My advisers counselled me to do nothing.
	\item The prime minister was right to counsel caution about military intervention.
	\end{itemize}
}
\item verb \\
If you \textbf{counsel} people, you give them advice about their problems .
 \textit{
	\begin{itemize}
	\item ...a psychologist who counsels people with eating disorders.
	\item Crawford counsels her on all aspects of her career.
	\end{itemize}
}
\item countable noun \\
Someone's \textbf{counsel} is the lawyer who gives them advice on a legal case and speaks on their behalf in court.
 \textit{
	\begin{itemize}
	\item Singleton's counsel said after the trial that he would appeal.
	\item The defence counsel warned that the judge should stop the trial.
	\end{itemize}
}
\item  \\
 keep one's own counsel \textit{
	\begin{itemize}
	\end{itemize}
}
\end{enumerate}

\section*{dinner}
{\large \color{blue}  dinners  }
\subsection*{Explain}
\begin{enumerate}
\item variable noun \\
\textbf{Dinner} is the main meal of the day, usually served in the early part of the evening.
 \textit{
	\begin{itemize}
	\item She invited us to her house for dinner.
	\item Would you like to stay and have dinner?
	\item Enjoy your dinner.
	\item ...four-course dinners.
	\end{itemize}
}
\item variable noun \\
Any meal you eat in the middle of the day can be referred to as \textbf{dinner} .
 \textit{
	\begin{itemize}
	\end{itemize}
}
\item countable noun \\
A \textbf{dinner} is a formal social  event at which a meal is served. It is held in the evening.
 \textit{
	\begin{itemize}
	\item ...a series of official lunches and dinners.
	\item The Professional Cricketers' Association held its annual dinner in London.
	\end{itemize}
}
\end{enumerate}

\section*{dissolve}
{\large \color{blue}  dissolves  dissolving  dissolved  }
\subsection*{Explain}
\begin{enumerate}
\item verb \\
If a substance  \textbf{dissolves} in liquid or if you \textbf{dissolve} it, it becomes mixed with the liquid and disappears .
 \textit{
	\begin{itemize}
	\item Heat gently until the sugar dissolves.
	\item Dissolve the salt in a little boiled water.
	\end{itemize}
}
\item verb \\
When an organization or institution  \textbf{is dissolved} , it is officially ended or broken up.
 \textit{
	\begin{itemize}
	\item The committee has been dissolved.
	\item The King agreed to dissolve the present commission.
	\end{itemize}
}
\item verb \\
When a parliament \textbf{is dissolved} , it is formally ended, so that elections for a new parliament can be held .
 \textit{
	\begin{itemize}
	\item The present assembly will be dissolved on April 30th.
	\item Kaifu threatened to dissolve the Parliament and call an election.
	\end{itemize}
}
\item verb \\
When a marriage or business  arrangement  \textbf{is dissolved} , it is officially ended.
 \textit{
	\begin{itemize}
	\item The marriage was dissolved in 1976.
	\end{itemize}
}
\item verb \\
If something such as a problem or feeling  \textbf{dissolves} or \textbf{is dissolved} , it becomes weaker and disappears.
 \textit{
	\begin{itemize}
	\item His new-found optimism dissolved.
	\item Lenny still could not dissolve the nagging lump of tension in his chest.
	\end{itemize}
}
\end{enumerate}

\section*{elaborate}
{\large \color{blue}  elaborates  elaborating  elaborated  }
\subsection*{Explain}
\begin{enumerate}
\item adjective \\
You use \textbf{elaborate} to describe something that is very complex because it has a lot of different parts.
 \textit{
	\begin{itemize}
	\item ...an elaborate research project.
	\item ...an elaborate ceremony that lasts for eight days.
	\end{itemize}
}
\item adjective \\
\textbf{Elaborate} plans, systems, and procedures are complicated because they have been planned in very great detail, sometimes too much detail.
 \textit{
	\begin{itemize}
	\item ...elaborate efforts at the highest level to conceal the problem.
	\item ...an elaborate management training scheme for graduates.
	\end{itemize}
}
\item adjective \\
\textbf{Elaborate}  clothing or material is made with a lot of detailed artistic  designs .
 \textit{
	\begin{itemize}
	\item He is known for his elaborate costumes.
	\end{itemize}
}
\item verb \\
If you \textbf{elaborate} a plan or theory , you develop it by making it more complicated and more effective .
 \textit{
	\begin{itemize}
	\item His task was to elaborate policies to make a market economy compatible with a clean
environment.
	\item ...the plan elaborated by the five permanent members of the U.N. Security Council.
	\end{itemize}
}
\item verb \\
If you \textbf{elaborate}  \textbf{on} something that has been said , you say more about it, or give more details.
 \textit{
	\begin{itemize}
	\item A spokesman declined to elaborate on a statement released late yesterday.
	\item Would you care to elaborate?
	\end{itemize}
}
\end{enumerate}

\section*{fantasy}
{\large \color{blue}  fantasies  }
\subsection*{Explain}
\begin{enumerate}
\item countable noun \\
A \textbf{fantasy} is a pleasant  situation or event that you think about and that you want to happen , especially one that is unlikely to happen.
 \textit{
	\begin{itemize}
	\item ...fantasies of romance and true love.
	\end{itemize}
}
\item variable noun \\
You can refer to a story or situation that someone creates from their imagination and that is not based on reality as \textbf{fantasy} .
 \textit{
	\begin{itemize}
	\item The film is more of an ironic fantasy than a horror story.
	\end{itemize}
}
\item uncountable noun \\
\textbf{Fantasy} is the activity of imagining things.
 \textit{
	\begin{itemize}
	\item ...a world of imagination, passion, fantasy, reflection.
	\end{itemize}
}
\item adjective \\
\textbf{Fantasy}  football , baseball , or another sport is a game in which players choose an imaginary team and score points based on the actual performances of the members of their team in real games.
 \textit{
	\begin{itemize}
	\item Haskins said he has been playing fantasy baseball for the past five years.
	\end{itemize}
}
\end{enumerate}

\section*{endeavor}
{\large \color{blue}  }
\subsection*{Explain}
\begin{enumerate}
\item verb intransitive \\
1.  \textit{
	\begin{itemize}
	\item to endeavor to finish first
	\end{itemize}
}
\item verb transitive \\
2.  \textit{
	\begin{itemize}
	\end{itemize}
}
\item noun \\
3.  \textit{
	\begin{itemize}
	\end{itemize}
}
\end{enumerate}

\section*{feast}
{\large \color{blue}  feasts  feasting  feasted  }
\subsection*{Explain}
\begin{enumerate}
\item countable noun \\
A \textbf{feast} is a large and special meal.
 \textit{
	\begin{itemize}
	\item Lunch was a feast of meat and vegetables, cheese, yoghurt and fruit, with unlimited
wine.
	\item The fruit was often served at wedding feasts.
	\item On the following day a feast was given in King John's honour.
	\end{itemize}
}
\item verb \\
If you \textbf{feast}  \textbf{on} a particular food, you eat a large amount of it with great enjoyment .
 \textit{
	\begin{itemize}
	\item They feasted well into the afternoon on mutton and corn stew.
	\end{itemize}
}
\item verb \\
If you \textbf{feast} , you take part in a feast.
 \textit{
	\begin{itemize}
	\item Only a few feet away, their captors feasted in the castle's banqueting hall.
	\end{itemize}
}
\item singular noun \\
You can refer to a large number of good, interesting , or enjoyable things as \textbf{a}  \textbf{feast}  \textbf{of} things.
 \textit{
	\begin{itemize}
	\item This new series promises a feast of special effects and set designs.
	\item Chicago provides a feast for the ears of any music lover.
	\end{itemize}
}
\item countable noun \\
A \textbf{feast} is a day or time of the year when a special religious celebration takes place.
 \textit{
	\begin{itemize}
	\item The Jewish feast of Passover began last night.
	\item It was Candlemas, a Catholic feast day celebrating the purification of the Virgin
Mary.
	\end{itemize}
}
\item  \\
 to feast your eyes \textit{
	\begin{itemize}
	\end{itemize}
}
\end{enumerate}

\section*{exchange}
{\large \color{blue}  exchanges  exchanging  exchanged  }
\subsection*{Explain}
\begin{enumerate}
\item verb \\
If two or more people \textbf{exchange} things of a particular kind, they give them to each other at the same time.
 \textbf{Exchange} is also a noun .
 \textit{
	\begin{itemize}
	\item We exchanged addresses and Christmas cards.
	\item The two men exchanged glances.
	\item He exchanged a quick smile with her then entered the lift.
	\item He ruled out any exchange of prisoners with the militants.
	\item ...a frank exchange of views.
	\end{itemize}
}
\item verb \\
If you \textbf{exchange} something, you replace it with a different thing, especially something that is better or more satisfactory .
 \textit{
	\begin{itemize}
	\item ...the chance to sell back or exchange goods.
	\item If the car you have leased is clearly unsatisfactory, you can always exchange it
for another.
	\end{itemize}
}
\item countable noun \\
An \textbf{exchange} is a brief  conversation , usually an angry one.
 \textit{
	\begin{itemize}
	\item There've been some bitter exchanges between the two groups.
	\end{itemize}
}
\item countable noun \\
An \textbf{exchange}  \textbf{of} fire, for example, is an incident in which people use guns or missiles against each other.
 \textit{
	\begin{itemize}
	\item There was an exchange of fire during which the gunman was wounded.
	\item This could intensify the risk of a nuclear exchange.
	\end{itemize}
}
\item countable noun \\
An \textbf{exchange} is an arrangement in which people from two different countries visit each other's country, to strengthen  links between them.
 \textit{
	\begin{itemize}
	\item ...a series of sporting and cultural exchanges with Seoul.
	\item ...educational exchanges for young people.
	\item I'm going to go on an exchange visit to Paris.
	\end{itemize}
}
\item noun, in names \\
\textbf{Exchange} is used in the names of some places where people used to trade and do business with
each other.
 \textit{
	\begin{itemize}
	\item ...the Royal Exchange.
	\end{itemize}
}
\item countable noun \\
\textbf{The}  \textbf{exchange} is the same as the telephone exchange .
 \textit{
	\begin{itemize}
	\end{itemize}
}
\item  \\
 in exchange \textit{
	\begin{itemize}
	\end{itemize}
}
\end{enumerate}

\section*{feat}
{\large \color{blue}  feats  }
\subsection*{Explain}
\begin{enumerate}
\item countable noun \\
If you refer to an action, or the result of an action, as a \textbf{feat} , you admire it because it is an impressive and difficult achievement.
 \textit{
	\begin{itemize}
	\item A racing car is an extraordinary feat of engineering.
	\end{itemize}
}
\end{enumerate}

\section*{expand}
{\large \color{blue}  expands  expanding  expanded  }
\subsection*{Explain}
\begin{enumerate}
\item verb \\
If something \textbf{expands} or \textbf{is expanded} , it becomes larger.
 \textit{
	\begin{itemize}
	\item Engineers noticed that the pipes were not expanding as expected.
	\item The money supply expanded by 14.6 per cent in the year to September.
	\item We have to expand the size of the image.
	\item ...a rapidly expanding universe.
	\item ...strips of expanded polystyrene.
	\end{itemize}
}
\item verb \\
If something such as a business, organization, or service \textbf{expands} , or if you \textbf{expand} it, it becomes bigger and includes more people, goods, or activities.
 \textit{
	\begin{itemize}
	\item The popular ceramics industry expanded towards the middle of the 19th century.
	\item The interest rate's coming down. I'll be able to expand or stay in business.
	\item I owned a bookshop and desired to expand the business.
	\item Health officials are proposing to expand their services by organising counselling.
	\end{itemize}
}
\end{enumerate}

\section*{festival}
{\large \color{blue}  festivals  }
\subsection*{Explain}
\begin{enumerate}
\item countable noun \\
A \textbf{festival} is an organized series of events such as musical  concerts or drama  productions .
 \textit{
	\begin{itemize}
	\item Numerous Umbrian towns hold their own summer festivals of music, theatre, and dance.
	\item I had just returned from the Cannes Film Festival.
	\end{itemize}
}
\item countable noun \\
A \textbf{festival} is a day or time of the year when people have a holiday from work and celebrate some special event, often a religious event.
 \textit{
	\begin{itemize}
	\item ...the Hindu festival of Diwali.
	\end{itemize}
}
\end{enumerate}

\section*{explode}
{\large \color{blue}  explodes  exploding  exploded  }
\subsection*{Explain}
\begin{enumerate}
\item verb \\
If an object such as a bomb  \textbf{explodes} or if someone or something \textbf{explodes} it, it bursts loudly and with great force, often causing damage or injury .
 \textit{
	\begin{itemize}
	\item They were clearing up when the second bomb exploded.
	\item A school bus was hit by gunfire which exploded the fuel tank.
	\end{itemize}
}
\item verb \\
If someone \textbf{explodes} , they express  strong  feelings suddenly and violently.
 \textit{
	\begin{itemize}
	\item Do you fear that you'll burst into tears or explode with anger in front of her?
	\item 'What happened!' I exploded.
	\item George caught the look and decided that Bess had better leave before she exploded.
	\end{itemize}
}
\item verb \\
If something \textbf{explodes} , it increases suddenly and rapidly in number or intensity .
 \textit{
	\begin{itemize}
	\item The population explodes to 40,000 during the tourist season.
	\item Investment by Japanese firms has exploded.
	\end{itemize}
}
\item verb \\
If someone \textbf{explodes} a theory or myth , they prove that it is wrong or impossible .
 \textit{
	\begin{itemize}
	\item Electricity privatisation has exploded the myth of cheap nuclear power.
	\item Such rumours have only recently been exploded.
	\end{itemize}
}
\item verb \\
If something \textbf{explodes} , it makes a sudden very loud noise.
 \textit{
	\begin{itemize}
	\item She heard laughter explode, then die.
	\end{itemize}
}
\end{enumerate}

\section*{function}
{\large \color{blue}  functions  functioning  functioned  }
\subsection*{Explain}
\begin{enumerate}
\item countable noun \\
The \textbf{function} of something or someone is the useful thing that they do or are intended to do.
 \textit{
	\begin{itemize}
	\item The main function of the merchant banks is to raise capital for industry.
	\end{itemize}
}
\item verb \\
If a machine or system \textbf{is functioning} , it is working or operating.
 \textit{
	\begin{itemize}
	\item The authorities say the prison is now functioning normally.
	\item Conservation programs cannot function without local support.
	\end{itemize}
}
\item verb \\
If someone or something \textbf{functions}  \textbf{as} a particular thing, they do the work or fulfil the purpose of that thing.
 \textit{
	\begin{itemize}
	\item On weekdays, one third of the room functions as workspace.
	\end{itemize}
}
\item countable noun \\
A \textbf{function} is a series of operations that a computer performs, for example when a single  key or a combination of keys is pressed .
 \textit{
	\begin{itemize}
	\end{itemize}
}
\item countable noun \\
If you say that one thing is \textbf{a}  \textbf{function}  \textbf{of} another, you mean that its amount or nature  depends on the other thing.
 \textit{
	\begin{itemize}
	\item Investment is a function of the interest rate.
	\end{itemize}
}
\item countable noun \\
A \textbf{function} is a large formal dinner or party .
 \textit{
	\begin{itemize}
	\end{itemize}
}
\end{enumerate}

\section*{fight}
{\large \color{blue}  fights  fighting  fought  }
\subsection*{Explain}
\begin{enumerate}
\item verb \\
If you \textbf{fight} something unpleasant , you try in a determined way to prevent it or stop it happening .
 \textbf{Fight} is also a noun .
 \textit{
	\begin{itemize}
	\item More units to fight forest fires are planned.
	\item I've spent a lifetime fighting against racism and prejudice.
	\item ...the fight against drug addiction.
	\end{itemize}
}
\item verb \\
If you \textbf{fight} for something, you try in a determined way to get it or achieve it.
 \textbf{Fight} is also a noun.
 \textit{
	\begin{itemize}
	\item Our members are willing to fight for a decent pay rise.
	\item Lee had to fight hard for his place on the expedition.
	\item I told him how we had fought to hold on to the company.
	\item The team has fought its way to the cup final.
	\item I too am committing myself to continue the fight for justice.
	\end{itemize}
}
\item verb \\
If an army or group \textbf{fights} a battle with another army or group, they oppose each other with weapons . You can also say that two armies or groups \textbf{fight} a battle.
 \textit{
	\begin{itemize}
	\item The two men fought a battle over land and water rights.
	\item In the latest incident police fought a gun battle with a gang.
	\item Clans had fought each other for centuries over ownership of pastures.
	\end{itemize}
}
\item verb \\
If a person or army \textbf{fights} in a battle or a war , they take part in it.
 \textit{
	\begin{itemize}
	\item He fought in the war and was taken prisoner by the enemy.
	\item If I were a young man I would sooner go to prison than fight for this country.
	\item My father did leave his university to fight the Germans.
	\item Last month rebels fought their way into the capital.
	\end{itemize}
}
\item verb \\
If one person \textbf{fights} with another, or \textbf{fights} them, the two people hit or kick each other because they want to hurt each other. You can also say that two people \textbf{fight} .
 \textbf{Fight} is also a noun.
 \textit{
	\begin{itemize}
	\item As a child she fought with her younger sister.
	\item I did fight him, I punched him but it was like hitting a wall.
	\item He wrenched the crutch from Jacob, who didn't fight him for it.
	\item I refuse to act that way when my kids fight.
	\item You get a lot of unruly drunks fighting each other.
	\item He had had a fight with Smith and bloodied his nose.
	\end{itemize}
}
\item verb \\
If one person \textbf{fights} with another, or \textbf{fights} them, they have an angry  disagreement or quarrel. You can also say that two people \textbf{fight} .
 \textbf{Fight} is also a noun.
 \textit{
	\begin{itemize}
	\item She was always arguing with him and fighting with him.
	\item Gwendolen started fighting her teachers.
	\item Mostly, they fight about paying bills.
	\item We think maybe he took off because he had a big fight with his dad the night before.
	\end{itemize}
}
\item verb \\
If you \textbf{fight} your way to a place, you move towards it with great  difficulty , for example because there are a lot of people or obstacles in your way.
 \textit{
	\begin{itemize}
	\item I fought my way into a carriage just before the doors closed.
	\item Peter fought his way through a blizzard to save one of the chickens.
	\end{itemize}
}
\item countable noun \\
A \textbf{fight} is a boxing match.
 \textit{
	\begin{itemize}
	\item This was Hyer's last fight, for no one else challenged him.
	\item The referee stopped the fight.
	\end{itemize}
}
\item verb \\
To \textbf{fight}  means to take part in a boxing match.
 \textit{
	\begin{itemize}
	\item In a few hours' time one of the world's most famous boxers will be fighting in Britain
for the first time.
	\item I'd like to fight him because he's undefeated and I want to be the first man to beat
him.
	\item I'd like to fight him for the title.
	\end{itemize}
}
\item verb \\
If you \textbf{fight} an election , you are a candidate in the election and try to win it.
 \textit{
	\begin{itemize}
	\item The former party treasurer helped raise almost £40 million to fight the election
campaign.
	\end{itemize}
}
\item countable noun \\
You can use \textbf{fight} to refer to a contest such as an election or a sports match.
 \textit{
	\begin{itemize}
	\item ...the fight for power between the two parties.
	\end{itemize}
}
\item verb \\
If you \textbf{fight} a case or a court action, you make a legal case against someone in a very determined way, or you put forward a defence when a legal case is made against you.
 \textit{
	\begin{itemize}
	\item Watkins fought his case in various courts for 10 years.
	\item The newspaper is fighting a damages action brought by the actress.
	\end{itemize}
}
\item uncountable noun \\
\textbf{Fight} is the desire or ability to keep fighting.
 \textit{
	\begin{itemize}
	\item I thought that we had a lot of fight in us.
	\end{itemize}
}
\item verb \\
If you \textbf{fight} an emotion or desire, you try very hard not to feel it, show it, or act on it, but do not always  succeed .
 \textit{
	\begin{itemize}
	\item I desperately fought the urge to giggle.
	\item He fought with the urge to smoke one of the cigars he'd given up a while ago.
	\item He fought to be patient with her.
	\end{itemize}
}
\item  \\
 to fight for breath \textit{
	\begin{itemize}
	\end{itemize}
}
\item  \\
 a fighting chance \textit{
	\begin{itemize}
	\end{itemize}
}
\item  \\
 fighting fit \textit{
	\begin{itemize}
	\end{itemize}
}
\item  \\
 fight for one's life \textit{
	\begin{itemize}
	\end{itemize}
}
\end{enumerate}

\section*{guidance}
{\large \color{blue}  }
\subsection*{Explain}
\begin{enumerate}
\item uncountable noun \\
\textbf{Guidance} is help and advice.
 \textit{
	\begin{itemize}
	\item ...an opportunity for young people to improve their performance under the guidance
of professional coaches.
	\item The nation looks to them for guidance.
	\end{itemize}
}
\end{enumerate}

\section*{grab}
{\large \color{blue}  grabs  grabbing  grabbed  }
\subsection*{Explain}
\begin{enumerate}
\item verb \\
If you \textbf{grab} something, you take it or pick it up suddenly and roughly .
 \textit{
	\begin{itemize}
	\item I managed to grab her hand.
	\item I grabbed him by the neck.
	\end{itemize}
}
\item verb \\
If you \textbf{grab at} something, you try to grab it.
 \textbf{Grab} is also a noun .
 \textit{
	\begin{itemize}
	\item He was clumsily trying to grab at Alfred's arms.
	\item I made a grab for the knife.
	\item Mr Penrose made a grab at his collar.
	\end{itemize}
}
\item verb \\
If you \textbf{grab} someone who is walking  past , you succeed in getting their attention.
 \textit{
	\begin{itemize}
	\item Grab that waiter, Mary Ann.
	\end{itemize}
}
\item verb \\
If you \textbf{grab} someone's attention, you do something in order to make them notice you.
 \textit{
	\begin{itemize}
	\item I jumped on the wall to grab the attention of the crowd.
	\end{itemize}
}
\item verb \\
If you \textbf{grab} something such as food, drink , or sleep , you manage to get some quickly.
 \textit{
	\begin{itemize}
	\item At night the kids grabbed a pizza from Frankie's.
	\end{itemize}
}
\item verb \\
If you \textbf{grab} something such as a chance or opportunity , or \textbf{grab at} it, you take advantage of it eagerly.
 \textit{
	\begin{itemize}
	\item She grabbed the chance of a job interview.
	\item He grabbed at the opportunity to buy his castle.
	\end{itemize}
}
\item countable noun \\
A \textbf{grab for} something such as power or fame is an attempt to gain it.
 \textit{
	\begin{itemize}
	\item ...a grab for personal power.
	\end{itemize}
}
\item  \\
 up for grabs \textit{
	\begin{itemize}
	\end{itemize}
}
\end{enumerate}

\section*{heart}
{\large \color{blue}  hearts  }
\subsection*{Explain}
\begin{enumerate}
\item countable noun \\
Your \textbf{heart} is the organ in your chest that pumps the blood around your body. People also use \textbf{heart} to refer to the area of their chest that is closest to their heart.
 \textit{
	\begin{itemize}
	\item The bullet had passed less than an inch from Andrea's heart.
	\item The only sound inside was the beating of his heart.
	\item He gave a sudden cry of pain and put his hand to his heart.
	\end{itemize}
}
\item countable noun \\
You can refer to someone's \textbf{heart} when you are talking about their deep  feelings and beliefs .
 \textit{
	\begin{itemize}
	\item Alik's words filled her heart with pride.
	\item I just couldn't bring myself to admit what I knew in my heart to be true.
	\end{itemize}
}
\item variable noun \\
You use \textbf{heart} when you are talking about someone's character and attitude towards other people, especially when they are kind and generous .
 \textit{
	\begin{itemize}
	\item She loved his brilliance and his generous heart.
	\item She's got a good heart but she's calculating.
	\end{itemize}
}
\item singular noun \\
If you refer to things \textbf{of}  \textbf{the heart} , you mean love and relationships .
 \textit{
	\begin{itemize}
	\item This is an excellent time for affairs of the heart.
	\end{itemize}
}
\item singular noun \\
\textbf{The}  \textbf{heart of} something is the most central and important part of it.
 \textit{
	\begin{itemize}
	\item The heart of the problem is supply and demand.
	\item Money lies at the heart of the debate over airline safety.
	\end{itemize}
}
\item singular noun \\
\textbf{The}  \textbf{heart}  \textbf{of} a place is its centre.
 \textit{
	\begin{itemize}
	\item ...a busy dentists' practice in the heart of London's West End.
	\end{itemize}
}
\item countable noun \\
The \textbf{heart} of a lettuce , cabbage, or other vegetable is its centre leaves.
 \textit{
	\begin{itemize}
	\end{itemize}
}
\item countable noun \\
A \textbf{heart} is a shape that is used as a symbol of love: ♥.
 \textit{
	\begin{itemize}
	\item ...heart-shaped chocolates.
	\end{itemize}
}
\item uncountable noun \\
\textbf{Hearts} is one of the four suits in a pack of playing cards. Each card in the suit is marked with one or more red symbols in
the shape of a heart.
 A \textbf{heart} is a playing card of this suit.
 \textit{
	\begin{itemize}
	\item West had to decide whether to play a heart.
	\end{itemize}
}
\item  \\
 with all one's heart \textit{
	\begin{itemize}
	\end{itemize}
}
\item  \\
 at heart \textit{
	\begin{itemize}
	\end{itemize}
}
\item  \\
 at heart \textit{
	\begin{itemize}
	\end{itemize}
}
\item  \\
 to break someone's heart \textit{
	\begin{itemize}
	\end{itemize}
}
\item  \\
 break sb's heart \textit{
	\begin{itemize}
	\end{itemize}
}
\item  \\
 broken heart \textit{
	\begin{itemize}
	\end{itemize}
}
\item  \\
 by heart \textit{
	\begin{itemize}
	\end{itemize}
}
\item  \\
 a change of heart \textit{
	\begin{itemize}
	\end{itemize}
}
\item  \\
 close to one's heart/near to one's heart \textit{
	\begin{itemize}
	\end{itemize}
}
\item  \\
 to your heart's content \textit{
	\begin{itemize}
	\end{itemize}
}
\item  \\
 cross my heart \textit{
	\begin{itemize}
	\end{itemize}
}
\item  \\
 from the heart/from the bottom of one's heart \textit{
	\begin{itemize}
	\end{itemize}
}
\item  \\
 give sb heart \textit{
	\begin{itemize}
	\end{itemize}
}
\item  \\
 not have the heart \textit{
	\begin{itemize}
	\end{itemize}
}
\item  \\
 in your heart of hearts \textit{
	\begin{itemize}
	\end{itemize}
}
\item  \\
 your heart is not in sth \textit{
	\begin{itemize}
	\end{itemize}
}
\item  \\
 to lose heart \textit{
	\begin{itemize}
	\end{itemize}
}
\item  \\
 lose your heart \textit{
	\begin{itemize}
	\end{itemize}
}
\item  \\
 heart is in your mouth \textit{
	\begin{itemize}
	\end{itemize}
}
\item  \\
 to open your heart \textit{
	\begin{itemize}
	\end{itemize}
}
\item  \\
 heart in the right place \textit{
	\begin{itemize}
	\end{itemize}
}
\item  \\
 to set your heart on something \textit{
	\begin{itemize}
	\end{itemize}
}
\item  \\
 wear one's heart on one's sleeve \textit{
	\begin{itemize}
	\end{itemize}
}
\item  \\
 heart and soul \textit{
	\begin{itemize}
	\end{itemize}
}
\item  \\
 take heart \textit{
	\begin{itemize}
	\end{itemize}
}
\item  \\
 take sth to heart \textit{
	\begin{itemize}
	\end{itemize}
}
\end{enumerate}

\section*{happen}
{\large \color{blue}  happens  happening  happened  }
\subsection*{Explain}
\begin{enumerate}
\item verb \\
Something that \textbf{happens} occurs or is done without being planned .
 \textit{
	\begin{itemize}
	\item We cannot say for sure what will happen.
	\item The accident happened close to Martha's Vineyard.
	\end{itemize}
}
\item verb \\
If something \textbf{happens} , it occurs as a result of a situation or course of action .
 \textit{
	\begin{itemize}
	\item She wondered what would happen if her parents found her.
	\item He trotted to the truck and switched on the ignition. Nothing happened.
	\end{itemize}
}
\item verb \\
When something, especially something unpleasant , \textbf{happens to} you, it takes place and affects you.
 \textit{
	\begin{itemize}
	\item If we had been spotted at that point, I don't know what would have happened to us.
	\item It's the best thing that ever happened to me.
	\end{itemize}
}
\item verb \\
If you \textbf{happen}  \textbf{to} do something, you do it by chance. If \textbf{it}  \textbf{happens}  \textbf{that} something is the case, it occurs by chance.
 \textit{
	\begin{itemize}
	\item We happened to discover we had a friend in common.
	\item I looked in the nearest paper, which happened to be the Daily Mail.
	\item If it happens that I'm wanted somewhere, my mother will take the call and let me
know.
	\end{itemize}
}
\item  \\
 as it happens \textit{
	\begin{itemize}
	\end{itemize}
}
\end{enumerate}

\section*{illusion}
{\large \color{blue}  illusions  }
\subsection*{Explain}
\begin{enumerate}
\item variable noun \\
An \textbf{illusion} is a false idea or belief.
 \textit{
	\begin{itemize}
	\item Do not have any illusions that an industrial tribunal will right all employment wrongs.
	\item No one really has any illusions about winning the war.
	\end{itemize}
}
\item countable noun \\
An \textbf{illusion} is something that appears to exist or be a particular thing but does not actually exist or is in reality something else.
 \textit{
	\begin{itemize}
	\item Floor-to-ceiling windows can give the illusion of extra height.
	\item This eerie calm is an illusion.
	\end{itemize}
}
\end{enumerate}

\section*{hunt}
{\large \color{blue}  hunts  hunting  hunted  }
\subsection*{Explain}
\begin{enumerate}
\item verb \\
If you \textbf{hunt}  \textbf{for} something or someone, you try to find them by searching carefully or thoroughly.
 \textbf{Hunt} is also a noun .
 \textit{
	\begin{itemize}
	\item A forensic team was hunting for clues.
	\item Some new arrivals lose hope even before they start hunting for a job.
	\item Chryssa hunted for Patra, and found her busy at a corner of the site.
	\item The couple had helped in the hunt for the toddlers.
	\end{itemize}
}
\item verb \\
If you \textbf{hunt} a criminal or an enemy , you search for them in order to catch or harm them.
 \textbf{Hunt} is also a noun.
 \textit{
	\begin{itemize}
	\item Detectives have been hunting him for seven months.
	\item Her irate husband was hunting him with a gun.
	\item Despite a nationwide hunt for the kidnap gang, not a trace of them was found.
	\end{itemize}
}
\item verb \\
When people or animals \textbf{hunt} , they chase and kill wild animals for food or as a sport.
 \textbf{Hunt} is also a noun.
 \textit{
	\begin{itemize}
	\item As a child I learned to hunt and fish.
	\item A leopard hunts alone, and an injured leopard cannot hunt.
	\item He got up at four and set out on foot to hunt black grouse.
	\item He set off for a nineteen-day moose hunt in Nova Scotia.
	\end{itemize}
}
\item verb \\
In Britain , when people \textbf{hunt} , they ride horses over fields with dogs  called hounds and try to catch and kill foxes , as a sport.
 \textbf{Hunt} is also a noun.
 \textit{
	\begin{itemize}
	\item She liked to hunt as often as she could.
	\item The hunt was held on land owned by the Duke of Marlborough.
	\end{itemize}
}
\item countable noun \\
In Britain, a \textbf{hunt} is a group of people who meet regularly to hunt foxes.
 \textit{
	\begin{itemize}
	\end{itemize}
}
\item  \\
 in the hunt \textit{
	\begin{itemize}
	\end{itemize}
}
\end{enumerate}

\section*{integrity}
{\large \color{blue}  }
\subsection*{Explain}
\begin{enumerate}
\item uncountable noun \\
If you have \textbf{integrity} , you are honest and firm in your moral principles.
 \textit{
	\begin{itemize}
	\item I have always regarded him as a man of integrity.
	\item He was praised for his fairness and high integrity.
	\end{itemize}
}
\item uncountable noun \\
The \textbf{integrity} of something such as a group of people or a text is its state of being a united  whole .
 \textit{
	\begin{itemize}
	\item Separatist movements are a threat to the integrity of the nation.
	\end{itemize}
}
\end{enumerate}

\section*{melt}
{\large \color{blue}  melts  melting  melted  }
\subsection*{Explain}
\begin{enumerate}
\item verb \\
When a solid substance  \textbf{melts} or when you \textbf{melt} it, it changes to a liquid, usually because it has been heated.
 \textit{
	\begin{itemize}
	\item The snow had melted, but the lake was still frozen solid.
	\item Meanwhile, melt the white chocolate in a bowl suspended over simmering water.
	\item Add the melted butter, molasses, salt, and flour.
	\end{itemize}
}
\item verb \\
If something such as your feelings  \textbf{melt} , they suddenly disappear and you no longer feel them.
 \textbf{Melt away}  means the same as melt .
 \textit{
	\begin{itemize}
	\item His anxiety about the outcome melted, to return later but not yet.
	\item He would have struggled but his strength had melted.
	\item When he heard these words, Shinran felt his inner doubts melt away.
	\end{itemize}
}
\item verb \\
If a person or thing \textbf{melts into} something such as darkness or a crowd of people, they become difficult to see , for example because they are moving away from you or are the same colour as the background .
 \textit{
	\begin{itemize}
	\item The youths dispersed and melted into the darkness.
	\item The squadron's armour is draped in sand-coloured nets that melt into the landscape.
	\end{itemize}
}
\item verb \\
If someone or something \textbf{melts} your heart , or if your heart \textbf{melts} , you start to feel love or sympathy towards them.
 \textit{
	\begin{itemize}
	\item When his lips break into a smile, it is enough to melt any woman's heart.
	\item When a bride walks down the aisle to a stirring tune, even the iciest of hearts melt.
	\end{itemize}
}
\item countable noun \\
A \textbf{melt} is a piece of bread which has meat or fish on it, and melted cheese on top .
 \textit{
	\begin{itemize}
	\item ...a tuna melt.
	\end{itemize}
}
\end{enumerate}

\section*{mood}
{\large \color{blue}  moods  }
\subsection*{Explain}
\begin{enumerate}
\item countable noun \\
Your \textbf{mood} is the way you are feeling at a particular time. If you are \textbf{in} a good  \textbf{mood} , you feel  cheerful . If you are \textbf{in} a bad  \textbf{mood} , you feel angry and impatient .
 \textit{
	\begin{itemize}
	\item He is clearly in a good mood today.
	\item When he came back, he was in a foul mood.
	\item Lily was in one of her aggressive moods.
	\item His moods swing alarmingly.
	\end{itemize}
}
\item countable noun \\
If someone is \textbf{in a}  \textbf{mood} , the way they are behaving  shows that they are feeling angry and impatient.
 \textit{
	\begin{itemize}
	\item She was obviously in a mood.
	\end{itemize}
}
\item singular noun \\
\textbf{The}  \textbf{mood} of a group of people is the way that they think and feel about an idea , event , or question at a particular time.
 \textit{
	\begin{itemize}
	\item The government seemed to be in tune with the popular mood.
	\item They largely misread the mood of the electorate.
	\end{itemize}
}
\item countable noun \\
The \textbf{mood} of a place is the general  impression that you get of it.
 \textit{
	\begin{itemize}
	\item First, set the mood with music.
	\item I wanted different moods in each room.
	\end{itemize}
}
\item variable noun \\
In grammar , the \textbf{mood} of a clause is the way in which the verb forms are used to show whether the clause is, for example , a statement , a question, or an instruction .
 \textit{
	\begin{itemize}
	\end{itemize}
}
\end{enumerate}

\section*{mingle}
{\large \color{blue}  mingles  mingling  mingled  }
\subsection*{Explain}
\begin{enumerate}
\item verb \\
If things such as sounds, smells , or feelings  \textbf{mingle} , they become mixed together but are usually still recognizable.
 \textit{
	\begin{itemize}
	\item Now the cheers and applause mingled in a single sustained roar.
	\item Foreboding mingled with his excitement.
	\end{itemize}
}
\item verb \\
At a party, if you \textbf{mingle}  \textbf{with} the other people there, you move around and talk to them.
 \textit{
	\begin{itemize}
	\item Go out of your way to mingle with others at the wedding.
	\item Guests ate and mingled.
	\item Alison mingled for a while and then went to where Douglas stood with John.
	\end{itemize}
}
\end{enumerate}

\section*{office}
{\large \color{blue}  offices  }
\subsection*{Explain}
\begin{enumerate}
\item countable noun \\
An \textbf{office} is a room or a part of a building where people work sitting at desks .
 \textit{
	\begin{itemize}
	\item He had an office just big enough for his desk and chair.
	\item At about 4.30 p.m. Audrey arrived at the office.
	\item Phone their head office for more details.
	\item ...an office block.
	\end{itemize}
}
\item countable noun \\
An \textbf{office} is a department of an organization, especially the government, where people deal with a particular  kind of administrative work.
 \textit{
	\begin{itemize}
	\item Thousands have registered with unemployment offices.
	\item ...Downing Street's press office.
	\item ...the Congressional Budget Office.
	\end{itemize}
}
\item countable noun \\
An \textbf{office} is a small building or room where people can go for information, tickets, or a service of some kind.
 \textit{
	\begin{itemize}
	\item The tourist office operates a useful room-finding service.
	\item ...the airline ticket offices.
	\end{itemize}
}
\item countable noun \\
A doctor's or dentist's \textbf{office} is a place where a doctor or dentist  sees their patients .
 \textit{
	\begin{itemize}
	\end{itemize}
}
\item uncountable noun \\
If someone holds  \textbf{office} in a government, they have an important  job or position of authority.
 \textit{
	\begin{itemize}
	\item The events to mark the President's ten years in office went ahead as planned.
	\item They are fed up with the politicians and want to vote them out of office.
	\item The president shall hold office for five years.
	\item The Vietnam War dashed President Johnson's hopes of a second term of office.
	\item He ran for office.
	\end{itemize}
}
\item  \\
 good offices \textit{
	\begin{itemize}
	\end{itemize}
}
\end{enumerate}

\section*{murmur}
{\large \color{blue}  murmurs  murmuring  murmured  }
\subsection*{Explain}
\begin{enumerate}
\item verb \\
If you \textbf{murmur} something, you say it very quietly , so that not many people can hear what you are saying .
 \textit{
	\begin{itemize}
	\item He turned and murmured something to the professor.
	\item She murmured a few words of support.
	\item 'How lovely,' she murmured.
	\item Murmuring softly that they must go somewhere to talk, he led her from the garden.
	\end{itemize}
}
\item countable noun \\
A \textbf{murmur} is something that is said which can hardly be heard.
 \textit{
	\begin{itemize}
	\item They spoke in low murmurs.
	\item She gave a little murmur.
	\end{itemize}
}
\item singular noun \\
A \textbf{murmur} is a continuous low sound, like the noise of a river or of voices far  away .
 \textit{
	\begin{itemize}
	\item The piano music mixes with the murmur of conversation.
	\item I could hear the murmur of the sea.
	\item The clamor of traffic has receded to a distant murmur.
	\end{itemize}
}
\item countable noun \\
A \textbf{murmur}  \textbf{of} a particular  emotion is a quiet  expression of it.
 \textit{
	\begin{itemize}
	\item The promise of some basic working rights draws murmurs of approval.
	\item Already there are murmurs of discontent.
	\end{itemize}
}
\item countable noun \\
A \textbf{murmur} is an abnormal sound which is made by the heart and which shows that there is probably something wrong with it.
 \textit{
	\begin{itemize}
	\item The doctor said James had now developed a heart murmur.
	\end{itemize}
}
\item  \\
 without a murmur \textit{
	\begin{itemize}
	\end{itemize}
}
\end{enumerate}

\section*{permission}
{\large \color{blue}  permissions  }
\subsection*{Explain}
\begin{enumerate}
\item uncountable noun \\
If someone who has authority over you gives you \textbf{permission}  \textbf{to} do something, they say that they will  allow you to do it.
 \textit{
	\begin{itemize}
	\item He asked permission to leave the room.
	\item Finally his mother relented and gave permission for her youngest son to marry.
	\item Police said permission for the march had not been granted.
	\item They cannot leave the country without permission.
	\end{itemize}
}
\item countable noun \\
A \textbf{permission} is a formal , written statement from an official group or place allowing you to do something.
 \textit{
	\begin{itemize}
	\item ...oil exploration permissions.
	\end{itemize}
}
\end{enumerate}

\section*{pause}
{\large \color{blue}  pauses  pausing  paused  }
\subsection*{Explain}
\begin{enumerate}
\item verb \\
If you \textbf{pause} while you are doing something, you stop for a short period and then continue .
 \textit{
	\begin{itemize}
	\item 'It's rather embarrassing,' he began, and paused.
	\item He had to pause to clear his throat.
	\item He worked steadily, and fast, pausing only to toss away clumps of grass roots.
	\item On leaving, she paused for a moment at the door.
	\item He talked for two hours without pausing for breath.
	\end{itemize}
}
\item countable noun \\
A \textbf{pause} is a short period when you stop doing something before continuing.
 \textit{
	\begin{itemize}
	\item After a pause Alex said sharply: 'I'm sorry if I've upset you'.
	\end{itemize}
}
\item  \\
 give sb pause for thought \textit{
	\begin{itemize}
	\end{itemize}
}
\end{enumerate}

\section*{procession}
{\large \color{blue}  processions  }
\subsection*{Explain}
\begin{enumerate}
\item countable noun \\
A \textbf{procession} is a group of people who are walking , riding , or driving in a line as part of a public  event .
 \textit{
	\begin{itemize}
	\item ...a funeral procession.
	\item ...religious processions.
	\end{itemize}
}
\end{enumerate}

\section*{pity}
{\large \color{blue}  pities  pitying  pitied  }
\subsection*{Explain}
\begin{enumerate}
\item uncountable noun \\
If you feel \textbf{pity}  \textbf{for} someone, you feel very sorry for them.
 \textit{
	\begin{itemize}
	\item He felt a sudden tender pity for her.
	\item She knew that she was an object of pity among her friends.
	\end{itemize}
}
\item verb \\
If you \textbf{pity} someone, you feel very sorry for them.
 \textit{
	\begin{itemize}
	\item I don't know whether to hate or pity him.
	\end{itemize}
}
\item singular noun \\
If you say that it is \textbf{a pity} that something is the case , you mean that you feel disappointment or regret about it.
 \textit{
	\begin{itemize}
	\item It is a great pity that all pupils in the city cannot have the same chances.
	\item Pity you haven't got your car, isn't it.
	\item It seemed a pity to let it all go to waste.
	\end{itemize}
}
\item uncountable noun \\
If someone shows  \textbf{pity} , they do not harm or punish someone they have power over.
 \textit{
	\begin{itemize}
	\item One should avoid showing too much pity.
	\item She saw no pity in their faces.
	\end{itemize}
}
\item  \\
 more's the pity \textit{
	\begin{itemize}
	\end{itemize}
}
\item  \\
 for pity's sake \textit{
	\begin{itemize}
	\end{itemize}
}
\item  \\
 take pity on someone \textit{
	\begin{itemize}
	\end{itemize}
}
\item  \\
 the pity is that/the pity of it is that \textit{
	\begin{itemize}
	\end{itemize}
}
\end{enumerate}

\section*{psychology}
{\large \color{blue}  }
\subsection*{Explain}
\begin{enumerate}
\item uncountable noun \\
\textbf{Psychology} is the scientific study of the human mind and the reasons for people's behaviour.
 \textit{
	\begin{itemize}
	\item ...Professor of Psychology at Bedford College.
	\end{itemize}
}
\item uncountable noun \\
The \textbf{psychology}  \textbf{of} a person is the kind of mind that they have, which makes them think or behave in the way that they do.
 \textit{
	\begin{itemize}
	\item ...a fascination with the psychology of murderers.
	\end{itemize}
}
\end{enumerate}

\section*{refresh}
{\large \color{blue}  refreshes  refreshing  refreshed  }
\subsection*{Explain}
\begin{enumerate}
\item verb \\
If something \textbf{refreshes} you when you have become hot , tired , or thirsty , it makes you feel  cooler or more energetic .
 \textit{
	\begin{itemize}
	\item The lotion cools and refreshes the skin.
	\item They had stopped by a spring to refresh themselves.
	\end{itemize}
}
\item verb \\
If you \textbf{refresh} something old or dull , you make it as good as it was when it was new.
 \textit{
	\begin{itemize}
	\item Many view these meetings as an occasion to share ideas and refresh friendship.
	\item Lettie appeared, her make-up refreshed.
	\end{itemize}
}
\item verb \\
If someone \textbf{refreshes} your memory, they tell you something that you had forgotten .
 \textit{
	\begin{itemize}
	\item Allow me to refresh your memory.
	\item He walked on the opposite side of the street to refresh his memory of the building.
	\end{itemize}
}
\item verb \\
If you \textbf{refresh} a web page, you click a button in order to get the most recent version of the page.
 \textit{
	\begin{itemize}
	\item Press the 'reload' button on your web browser to refresh the site.
	\end{itemize}
}
\end{enumerate}

\section*{scandal}
{\large \color{blue}  scandals  }
\subsection*{Explain}
\begin{enumerate}
\item countable noun \\
A \textbf{scandal} is a situation or event that is thought to be shocking and immoral and that everyone knows about.
 \textit{
	\begin{itemize}
	\item ...a financial scandal.
	\end{itemize}
}
\item uncountable noun \\
\textbf{Scandal} is talk about the shocking and immoral aspects of someone's behaviour or something that has happened .
 \textit{
	\begin{itemize}
	\item He loved gossip and scandal.
	\end{itemize}
}
\item singular noun \\
If you say that something is a \textbf{scandal} , you are angry about it and think that the people responsible for it should be ashamed .
 \textit{
	\begin{itemize}
	\item It is a scandal that a person can be stopped for no reason by the police.
	\end{itemize}
}
\end{enumerate}

\section*{reply}
{\large \color{blue}  replies  replying  replied  }
\subsection*{Explain}
\begin{enumerate}
\item verb \\
When you \textbf{reply}  \textbf{to} something that someone has said or written to you, you say or write an answer to them.
 \textit{
	\begin{itemize}
	\item 'That's a nice dress,' said Michael. 'Thanks,' she replied solemnly.
	\item He replied that this was absolutely impossible.
	\item Grace was too terrified to reply.
	\item I've not replied to Lee's letter yet.
	\item To their surprise, hundreds replied to the advertisement.
	\end{itemize}
}
\item countable noun \\
A \textbf{reply} is something that you say or write when you answer someone or answer a letter or advertisement .
 \textit{
	\begin{itemize}
	\item I called out a challenge, but there was no reply.
	\item David has had 12 replies to his ad.
	\item They went ahead without waiting for a reply from the Germans.
	\item He said in reply that the question was unfair.
	\end{itemize}
}
\item verb \\
In sports , if you \textbf{reply}  \textbf{to} a goal or a number of points scored by your opponents , you also score a goal or a number of points.
 \textit{
	\begin{itemize}
	\item The Gloucester centre replied with two penalties.
	\item Deryck Fox gave Featherstone the lead early in the first half with a penalty, but
Saints were quick to reply.
	\end{itemize}
}
\item verb \\
If you \textbf{reply}  \textbf{to} something such as an attack  \textbf{with}  violence or \textbf{with} another action, you do something in response.
 \textit{
	\begin{itemize}
	\item Farmers threw eggs and empty bottles at police, who replied with tear gas.
	\item The National Salvation Front has already replied to this series of opposition moves
with its own demonstrations.
	\end{itemize}
}
\end{enumerate}

\section*{square}
{\large \color{blue}  squares  squaring  squared  }
\subsection*{Explain}
\begin{enumerate}
\item countable noun \\
A \textbf{square} is a shape with four sides that are all the same length and four corners that are all right angles.
 \textit{
	\begin{itemize}
	\item Serve the cake warm or at room temperature, cut in squares.
	\item There was a calendar on the wall, with large squares around the dates.
	\item Most of the rugs are simple cotton squares.
	\end{itemize}
}
\item countable noun \\
In a town or city, a \textbf{square} is a flat open place, often in the shape of a square.
 \textit{
	\begin{itemize}
	\item The house is located in one of Pimlico's prettiest garden squares.
	\item The town centre is thick with churches and cafe-lined squares.
	\item ...St Mark's Square.
	\end{itemize}
}
\item adjective \\
Something that is \textbf{square} has a shape the same as a square or similar to a square.
 \textit{
	\begin{itemize}
	\item Round tables seat more people in the same space as a square table.
	\item His finger nails were square and cut neatly across.
	\end{itemize}
}
\item adjective \\
\textbf{Square} is used before units of length when referring to the area of something. For example, if something is three metres long and two metres wide , its area is six square metres.
 \textit{
	\begin{itemize}
	\item Canary Wharf was set to provide 10 million square feet of office space.
	\item The Philippines has just 6,000 square kilometres of forest left.
	\end{itemize}
}
\item adjective \\
\textbf{Square} is used after units of length when you are giving the length of each side of something
that is square in shape.
 \textit{
	\begin{itemize}
	\item ...a linen cushion cover, 45 cm square.
	\item ...two pieces of wood 4 inches square.
	\end{itemize}
}
\item verb \\
To \textbf{square} a number means to multiply it by itself. For example, \textbf{3 squared} is 3 x 3, or 9. \textbf{3 squared} is usually written as 3².
 \textit{
	\begin{itemize}
	\item Take the time in seconds, square it, and multiply by 5.12.
	\item A squared plus B squared equals C squared.
	\end{itemize}
}
\item countable noun \\
The \textbf{square}  \textbf{of} a number is the number produced when you multiply that number by itself. For example,
the square of 3 is 9.
 \textit{
	\begin{itemize}
	\item ...the square of the speed of light, an exceedingly large number.
	\end{itemize}
}
\item verb \\
If you \textbf{square} two different ideas or actions \textbf{with} each other or if they \textbf{square with} each other, they fit or match each other.
 \textit{
	\begin{itemize}
	\item That explanation squares with the facts, doesn't it.
	\item He set out to square his dreams with reality.
	\end{itemize}
}
\item verb \\
If you \textbf{square} something \textbf{with} someone, you ask their permission or check with them that what you are doing is acceptable to them.
 \textit{
	\begin{itemize}
	\item I squared it with Dan, who said it was all right.
	\item She should have squared things with Jay before she went into this business with Walker.
	\end{itemize}
}
\item  \\
 square the circle \textit{
	\begin{itemize}
	\end{itemize}
}
\item  \\
 back to square one \textit{
	\begin{itemize}
	\end{itemize}
}
\item  \\
 a square peg in a round hole \textit{
	\begin{itemize}
	\end{itemize}
}
\end{enumerate}

\section*{research}
{\large \color{blue}  researches  researching  researched  }
\subsection*{Explain}
\begin{enumerate}
\item uncountable noun \\
\textbf{Research} is work that involves studying something and trying to discover facts about it.
 \textit{
	\begin{itemize}
	\item 65 percent of the 1987 budget went for nuclear weapons research and production.
	\item ...cancer research.
	\item ...his researches into which kinds of flowers bees get their best honey from.
	\end{itemize}
}
\item verb \\
If you \textbf{research} something, you try to discover facts about it.
 \textit{
	\begin{itemize}
	\item She spent two years in South Florida researching and filming her documentary.
	\item So far we haven't been able to find anything, but we're still researching.
	\end{itemize}
}
\end{enumerate}

\section*{stroke}
{\large \color{blue}  strokes  stroking  stroked  }
\subsection*{Explain}
\begin{enumerate}
\item verb \\
If you \textbf{stroke} someone or something, you move your hand slowly and gently over them.
 \textit{
	\begin{itemize}
	\item Carla, curled up on the sofa, was stroking her cat.
	\item She walked forward and embraced him and stroked his tousled white hair.
	\end{itemize}
}
\item countable noun \\
If someone has a \textbf{stroke} , a blood vessel in their brain bursts or becomes blocked , which may kill them or make them unable to move one side of their body.
 \textit{
	\begin{itemize}
	\item He had a minor stroke in 1987, which left him partly paralysed.
	\end{itemize}
}
\item countable noun \\
The \textbf{strokes} of a pen or brush are the movements or marks that you make with it when you are writing or
 painting .
 \textit{
	\begin{itemize}
	\item Fill in gaps by using short, upward strokes of the pencil.
	\end{itemize}
}
\item countable noun \\
When you are swimming or rowing, your \textbf{strokes} are the repeated movements that you make with your arms or the oars.
 \textit{
	\begin{itemize}
	\item I turned and swam a few strokes further out to sea.
	\item The boatmen accompany the stroke of their oars with the sound of their voices.
	\end{itemize}
}
\item countable noun \\
A swimming \textbf{stroke} is a particular style or method of swimming.
 \textit{
	\begin{itemize}
	\item She spent hours practising the breast stroke.
	\end{itemize}
}
\item countable noun \\
The \textbf{strokes} of a clock are the sounds that indicate each hour.
 \textit{
	\begin{itemize}
	\item On the stroke of 12, fireworks suddenly exploded into the night.
	\end{itemize}
}
\item countable noun \\
In sports such as tennis , baseball , cricket , and golf , a \textbf{stroke} is the action of hitting the ball.
 \textit{
	\begin{itemize}
	\item Compton was sending the ball here, there, and everywhere with each stroke.
	\end{itemize}
}
\item singular noun \\
\textbf{A stroke of}  luck or good fortune is something lucky that happens .
 \textit{
	\begin{itemize}
	\item It didn't rain, which turned out to be a stroke of luck.
	\end{itemize}
}
\item singular noun \\
\textbf{A stroke of}  genius or inspiration is a very good idea that someone suddenly has.
 \textit{
	\begin{itemize}
	\item At the time, his appointment seemed a stroke of genius.
	\end{itemize}
}
\item  \\
 at a stroke \textit{
	\begin{itemize}
	\end{itemize}
}
\item  \\
 do a stroke \textit{
	\begin{itemize}
	\end{itemize}
}
\end{enumerate}

\section*{revolt}
{\large \color{blue}  revolts  revolting  revolted  }
\subsection*{Explain}
\begin{enumerate}
\item variable noun \\
A \textbf{revolt} is an illegal and often violent  attempt by a group of people to change their country's political system.
 \textit{
	\begin{itemize}
	\item It was undeniably a revolt by ordinary people against their leaders.
	\item The newly-occupied Italian colony of Libya rose in revolt in 1914.
	\end{itemize}
}
\item verb \\
When people \textbf{revolt} , they make an illegal and often violent attempt to change their country's political
system.
 \textit{
	\begin{itemize}
	\item In 1375 the townspeople revolted.
	\item Zanzibar's fortunes declined after the islanders revolted against the sultanate in
1964.
	\end{itemize}
}
\item variable noun \\
A \textbf{revolt} by a person or group against someone or something is a refusal to accept the authority of that person or thing.
 \textit{
	\begin{itemize}
	\item The prime minister is facing a revolt by party activists over his refusal to hold
a referendum.
	\item Soon the entire armed forces were in open revolt.
	\end{itemize}
}
\item verb \\
When people \textbf{revolt}  \textbf{against} someone or something, they reject the authority of that person or reject that thing.
 \textit{
	\begin{itemize}
	\item Three senior cabinet members revolted and resigned in protest on Friday night.
	\item Caroline revolted against her ballet training at sixteen.
	\end{itemize}
}
\end{enumerate}

\section*{temptation}
{\large \color{blue}  temptations  }
\subsection*{Explain}
\begin{enumerate}
\item variable noun \\
If you feel you want to do something or have something, even though you know you really should avoid it, you can refer to this feeling as \textbf{temptation} . You can also refer to the thing you want to do or have as a \textbf{temptation} .
 \textit{
	\begin{itemize}
	\item Will they be able to resist the temptation to buy?
	\item ...the many temptations to which you will be exposed.
	\end{itemize}
}
\end{enumerate}

\section*{revolve}
{\large \color{blue}  revolves  revolving  revolved  }
\subsection*{Explain}
\begin{enumerate}
\item verb \\
If you say that one thing \textbf{revolves}  \textbf{around} another thing, you mean that the second thing is the main  feature or focus of the first thing.
 \textit{
	\begin{itemize}
	\item Since childhood, her life has revolved around tennis.
	\item This plot revolves around a youngster who is shown various stages of his life.
	\end{itemize}
}
\item verb \\
If a discussion or conversation  \textbf{revolves}  \textbf{around} a particular topic , it is mainly about that topic.
 \textit{
	\begin{itemize}
	\item The debate revolves around specific accounting techniques.
	\item The conversation revolved around the terrible condition of the road.
	\end{itemize}
}
\item verb \\
If one object  \textbf{revolves}  \textbf{around} another object, the first object turns in a circle around the second object.
 \textit{
	\begin{itemize}
	\item The satellite revolves around the Earth once every hundred minutes.
	\end{itemize}
}
\item verb \\
When something \textbf{revolves} or when you \textbf{revolve} it, it moves or turns in a circle around a central point or line.
 \textit{
	\begin{itemize}
	\item Overhead, the fan revolved slowly.
	\item Monica picked up her pen and revolved it between her teeth.
	\end{itemize}
}
\end{enumerate}

\section*{text}
{\large \color{blue}  texts  texting  texted  }
\subsection*{Explain}
\begin{enumerate}
\item singular noun \\
\textbf{The text} of a book is the main part of it, rather than the introduction , pictures , or notes.
 \textit{
	\begin{itemize}
	\item The text is precise and informative.
	\end{itemize}
}
\item uncountable noun \\
\textbf{Text} is any written material.
 \textit{
	\begin{itemize}
	\item The machine can recognise hand-written characters and turn them into printed text.
	\item The manuscript consisted of 500 pages of typed text.
	\end{itemize}
}
\item countable noun \\
The \textbf{text} of a speech, broadcast , or recording is the written version of it.
 \textit{
	\begin{itemize}
	\item A text of his speech had been circulated to all of the bishops.
	\end{itemize}
}
\item countable noun \\
A \textbf{text} is a book or other piece of writing, especially one connected with science or learning .
 \textit{
	\begin{itemize}
	\item Her text is believed to be the oldest surviving manuscript by a female physician.
	\end{itemize}
}
\item countable noun \\
A \textbf{text} is a written or spoken passage, especially one that is used in a school or university
for discussion or in an examination .
 \textit{
	\begin{itemize}
	\item I'll read the text aloud first.
	\item His early plays are set texts in universities.
	\end{itemize}
}
\item countable noun \\
A \textbf{text} is the same as a text message .
 \textit{
	\begin{itemize}
	\item He sent a text to his boss insisting that he had done nothing wrong.
	\end{itemize}
}
\item verb \\
If you \textbf{text} someone, you send them a text message on a mobile phone.
 \textit{
	\begin{itemize}
	\item Mary texted me when she got home.
	\end{itemize}
}
\end{enumerate}

\section*{rot}
{\large \color{blue}  rots  rotting  rotted  }
\subsection*{Explain}
\begin{enumerate}
\item verb \\
When food, wood , or another substance \textbf{rots} , or when something \textbf{rots} it, it becomes softer and is gradually destroyed .
 \textit{
	\begin{itemize}
	\item If we don't unload it soon, the grain will start rotting in the silos.
	\item Sugary canned drinks rot your teeth.
	\item ...the smell of rotting fish.
	\end{itemize}
}
\item uncountable noun \\
If there is \textbf{rot} in something, especially something that is made of wood, parts of it have decayed and fallen apart .
 \textit{
	\begin{itemize}
	\item Investigations had revealed extensive rot in the beams under the ground floor.
	\item Neither the timber frame nor metal chassis were protected against rot.
	\end{itemize}
}
\item singular noun \\
You can use \textbf{the rot} to refer to the way something gradually gets  worse . For example , if you are talking about the time when \textbf{the rot}  \textbf{set in} , you are talking about the time when a situation  began to get steadily worse and worse.
 \textit{
	\begin{itemize}
	\item In many schools, the rot is beginning to set in. Standards are falling all the time.
	\item The country's leaders are unwilling to take unpopular measures to stop the rot.
	\end{itemize}
}
\item verb \\
If you say that someone is being left to \textbf{rot} in a particular place, especially in a prison , you mean that they are being left there and their physical and mental condition is being allowed to get worse and worse.
 \textit{
	\begin{itemize}
	\item Most governments simply leave the long-term jobless to rot on the dole.
	\end{itemize}
}
\item uncountable noun \\
If you say that what someone is saying is \textbf{rot} , you mean that they are saying things that are untrue or stupid .
 \textit{
	\begin{itemize}
	\item What a load of pompous, pseudo-intellectual rot.
	\item You do talk rot!
	\end{itemize}
}
\end{enumerate}

\section*{title}
{\large \color{blue}  titles  titling  titled  }
\subsection*{Explain}
\begin{enumerate}
\item countable noun \\
The \textbf{title} of a book, play , film, or piece of music is its name.
 \textit{
	\begin{itemize}
	\item 'Patience and Sarah' was first published in 1969 under the title 'A Place for Us'.
	\end{itemize}
}
\item verb \\
When a writer , composer , or artist  \textbf{titles} a work, they give it a title.
 \textit{
	\begin{itemize}
	\item Pirandello titled his play 'Six Characters in Search of an Author'.
	\item The painting is titled 'The Dream'.
	\item Their story is the subject of a new book titled 'The Golden Thirteen'.
	\end{itemize}
}
\item countable noun \\
Publishers and booksellers often refer to books or magazines as \textbf{titles} .
 \textit{
	\begin{itemize}
	\item It has become the biggest publisher of new poetry in Britain, with 50 new titles
a year.
	\end{itemize}
}
\item countable noun \\
A person's \textbf{title} is a word such as ' Sir ', ' Lord ', or ' Lady ' that is used in front of their name, or a phrase that is used instead of their name, and indicates that they have a high rank in society .
 \textit{
	\begin{itemize}
	\item She relinquished everything but her title as Queen of Scots.
	\item The title of Prince of Wales was often bestowed on the monarch's eldest son.
	\item He had no title and was not the heir to a great estate.
	\end{itemize}
}
\item countable noun \\
Someone's \textbf{title} is a word such as 'Mr', 'Mrs', or ' Doctor ', that is used before their own name in order to show their status or profession .
 \textit{
	\begin{itemize}
	\item She has been awarded the title Professor.
	\end{itemize}
}
\item countable noun \\
Someone's \textbf{title} is a name that describes their job or status in an organization .
 \textit{
	\begin{itemize}
	\item He was given the title of deputy prime minister.
	\item 'Could you tell me your official job title?'—'It's Operations Manager.'
	\end{itemize}
}
\item countable noun \\
If a person or team  wins a particular  \textbf{title} , they win a sports  competition that is held regularly. Usually a person keeps a title until someone else defeats them.
 \textit{
	\begin{itemize}
	\item He became Jamaica's first gold medallist when he won the 400m title in 1948.
	\item He retained his title as world chess champion.
	\end{itemize}
}
\item uncountable noun \\
\textbf{Title} is the legal ownership of something, especially  land or property.
 \textit{
	\begin{itemize}
	\item He never had title to the property.
	\end{itemize}
}
\end{enumerate}

\section*{search}
{\large \color{blue}  searches  searching  searched  }
\subsection*{Explain}
\begin{enumerate}
\item verb \\
If you \textbf{search}  \textbf{for} something or someone, you look carefully for them.
 \textit{
	\begin{itemize}
	\item The Turkish security forces have started searching for the missing men.
	\item They searched for a spot where they could sit on the floor.
	\item Nonetheless there are signs that both sides may be searching for a compromise.
	\end{itemize}
}
\item verb \\
If you \textbf{search} a place, you look carefully for something or someone there.
 \textit{
	\begin{itemize}
	\item Armed troops searched the hospital yesterday.
	\item She searched her desk for the necessary information.
	\item Relief workers are still searching through collapsed buildings looking for victims.
	\end{itemize}
}
\item countable noun \\
A \textbf{search} is an attempt to find something or someone by looking for them carefully.
 \textit{
	\begin{itemize}
	\item There was no chance of him being found alive and the search was abandoned.
	\item Egypt has said there is no time to lose in the search for a Middle East settlement.
	\end{itemize}
}
\item verb \\
If a police  officer or someone else in authority  \textbf{searches} you, they look carefully to see whether you have something hidden on you.
 \textit{
	\begin{itemize}
	\item The man took her suitcase from her and then searched her.
	\item His first task was to search them for weapons.
	\end{itemize}
}
\item verb \\
If you \textbf{search}  \textbf{for} information on a computer , you give the computer an instruction to find that information.
 \textbf{Search} is also a noun .
 \textit{
	\begin{itemize}
	\item You can use a directory service to search for people on the internet.
	\item He was doing a computer search of local news articles.
	\end{itemize}
}
\item  \\
 in search of \textit{
	\begin{itemize}
	\end{itemize}
}
\item  \\
 search me \textit{
	\begin{itemize}
	\end{itemize}
}
\end{enumerate}

\section*{transaction}
{\large \color{blue}  transactions  }
\subsection*{Explain}
\begin{enumerate}
\item countable noun \\
A \textbf{transaction} is a piece of business, for example an act of buying or selling something.
 \textit{
	\begin{itemize}
	\end{itemize}
}
\end{enumerate}

\section*{slide}
{\large \color{blue}  slides  sliding  slid  }
\subsection*{Explain}
\begin{enumerate}
\item verb \\
When something \textbf{slides}  somewhere or when you \textbf{slide} it there, it moves there smoothly over or against something.
 \textit{
	\begin{itemize}
	\item She slid the door open.
	\item I slid the wallet into his pocket.
	\item Tears were sliding down his cheeks.
	\end{itemize}
}
\item verb \\
If you \textbf{slide} somewhere, you move there smoothly and quietly .
 \textit{
	\begin{itemize}
	\item He slid into the driver's seat.
	\item 'Nice meeting you, Zoe,' I said and slid off.
	\end{itemize}
}
\item verb \\
To \textbf{slide into} a particular mood , attitude , or situation means to gradually start to have that mood, attitude, or situation often without intending to.
 \textit{
	\begin{itemize}
	\item She had slid into a depression.
	\item He needs them to stop the country sliding into chaos.
	\end{itemize}
}
\item verb \\
If currencies or prices \textbf{slide} , they gradually become worse or lower in value.
 \textbf{Slide} is also a noun .
 \textit{
	\begin{itemize}
	\item The U.S. dollar continued to slide.
	\item The upset sent share prices sliding to their lowest level for almost 18 months.
	\item Shares slid 11p to 293p after brokers downgraded their profit estimates.
	\item Its share slid from 24.24 per cent to 22.17 per cent.
	\item ...the dangerous slide in oil prices.
	\end{itemize}
}
\item countable noun \\
A \textbf{slide} is a small piece of photographic film which you project onto a screen so that you can see the picture .
 \textit{
	\begin{itemize}
	\item ...a slide show.
	\end{itemize}
}
\item countable noun \\
A \textbf{slide} is a single page of an electronic  presentation which you usually view on a computer or on a larger screen.
 \textit{
	\begin{itemize}
	\item He shows slides revealing that most people think the Government is doing poorly.
	\end{itemize}
}
\item countable noun \\
A \textbf{slide} is a piece of glass on which you put something that you want to examine through a microscope .
 \textit{
	\begin{itemize}
	\end{itemize}
}
\item countable noun \\
A \textbf{slide} is a piece of playground equipment that has a steep slope for children to go down for fun .
 \textit{
	\begin{itemize}
	\end{itemize}
}
\item  \\
 let sth slide \textit{
	\begin{itemize}
	\end{itemize}
}
\end{enumerate}

\section*{typewriter}
{\large \color{blue}  typewriters  }
\subsection*{Explain}
\begin{enumerate}
\item countable noun \\
A \textbf{typewriter} is a machine that was commonly used in the past and which has keys that are pressed in order to print letters , numbers , or other characters onto paper .
 \textit{
	\begin{itemize}
	\end{itemize}
}
\end{enumerate}

\section*{smash}
{\large \color{blue}  smashes  smashing  smashed  }
\subsection*{Explain}
\begin{enumerate}
\item verb \\
If you \textbf{smash} something or if it \textbf{smashes} , it breaks into many pieces, for example when it is hit or dropped .
 \textit{
	\begin{itemize}
	\item Someone smashed a bottle.
	\item A crowd of youths started smashing windows.
	\item Two or three glasses fell off and smashed into pieces.
	\end{itemize}
}
\item verb \\
If you \textbf{smash} through a wall , gate , or door , you get through it by hitting and breaking it.
 \textit{
	\begin{itemize}
	\item The demonstrators used trucks to smash through the embassy gates.
	\item Soldiers smashed their way into his office.
	\end{itemize}
}
\item verb \\
If something \textbf{smashes} or \textbf{is smashed} against something solid , it moves very fast and with great  force against it.
 \textit{
	\begin{itemize}
	\item The bottle smashed against a wall.
	\item He smashed his fist into Anthony's face.
	\end{itemize}
}
\item verb \\
To \textbf{smash} a political group or system means to deliberately destroy it.
 \textit{
	\begin{itemize}
	\item Their attempts to clean up politics and smash the power of party machines failed.
	\end{itemize}
}
\item verb \\
If you \textbf{smash} something, you do it extremely  well or successfully.
 \textit{
	\begin{itemize}
	\item We're really going to smash this task.
	\item I didn't know she could sing like that. She smashed it.
	\end{itemize}
}
\item countable noun \\
A \textbf{smash} is the same as a smash hit .
 \textit{
	\begin{itemize}
	\item It is the public who decide if a film is a smash or a flop.
	\end{itemize}
}
\item countable noun \\
You can  refer to a car crash as a \textbf{smash} .
 \textit{
	\begin{itemize}
	\item He was near to death after a car smash.
	\end{itemize}
}
\end{enumerate}

\section*{typist}
{\large \color{blue}  typists  }
\subsection*{Explain}
\begin{enumerate}
\item countable noun \\
A \textbf{typist} is someone who works in an office typing letters and other documents .
 \textit{
	\begin{itemize}
	\end{itemize}
}
\end{enumerate}

\section*{tumble}
{\large \color{blue}  tumbles  tumbling  tumbled  }
\subsection*{Explain}
\begin{enumerate}
\item verb \\
If someone or something \textbf{tumbles}  somewhere , they fall there with a rolling or bouncing movement.
 \textbf{Tumble} is also a noun .
 \textit{
	\begin{itemize}
	\item A small boy tumbled off a third floor fire escape.
	\item The dog had tumbled down the cliff.
	\item He fell to the ground, and the gun tumbled out of his hand.
	\item He injured his ribs in a tumble from his horse.
	\end{itemize}
}
\item verb \\
If prices or levels of something \textbf{are tumbling} , they are decreasing rapidly.
 \textbf{Tumble} is also a noun.
 \textit{
	\begin{itemize}
	\item House prices have tumbled by almost 30 per cent in real terms since mid-1989.
	\item Share prices continued to tumble today on the stock market.
	\item ...tumbling inflation.
	\item Oil prices took a tumble yesterday.
	\end{itemize}
}
\item verb \\
If water \textbf{tumbles} , it flows quickly over an uneven surface.
 \textit{
	\begin{itemize}
	\item Waterfalls crash and tumble over rocks.
	\item ...the aromatic pines and tumbling streams of the Zonba Plateau.
	\end{itemize}
}
\item verb \\
If you say that someone \textbf{tumbles}  \textbf{into} a situation or place, you mean that they get into it without being fully in control of themselves or knowing what they are doing.
 \textit{
	\begin{itemize}
	\item The whole region seemed to be tumbling into crisis.
	\item They tumble into bed at eight o'clock, too tired to take their clothes off.
	\end{itemize}
}
\end{enumerate}

\section*{warrant}
{\large \color{blue}  warrants  warranting  warranted  }
\subsection*{Explain}
\begin{enumerate}
\item verb \\
If something \textbf{warrants} a particular action, it makes the action seem  necessary or appropriate for the circumstances .
 \textit{
	\begin{itemize}
	\item The allegations are serious enough to warrant an investigation.
	\item No matter was too small to warrant his attention.
	\end{itemize}
}
\item countable noun \\
A \textbf{warrant} is a legal document that allows someone to do something, especially one that is signed by a judge or magistrate and gives the police  permission to arrest someone or search their house.
 \textit{
	\begin{itemize}
	\item Police confirmed that they had issued a warrant for his arrest.
	\item ...a search warrant.
	\item Equipment is allocated by warrant.
	\end{itemize}
}
\item uncountable noun \\
If you say that there is \textbf{no}  \textbf{warrant}  \textbf{for} something, you mean that there is no good reason to justify it.
 \textit{
	\begin{itemize}
	\item There is some warrant for holding back on full-scale aid.
	\end{itemize}
}
\item verb \\
If you \textbf{warrant}  \textbf{that} something is true or will happen , you say officially that it is true, or guarantee that it will happen.
 \textit{
	\begin{itemize}
	\item All entrants must warrant that their entry is entirely their own work.
	\item The contract warrants that an experienced person is on board all the time.
	\end{itemize}
}
\end{enumerate}

\section*{turn}
{\large \color{blue}  turns  turning  turned  }
\subsection*{Explain}
\begin{enumerate}
\item verb \\
When you \textbf{turn} or when you \textbf{turn} part of your body, you move your body or part of your body so that it is facing in a different or opposite direction.
 \textbf{Turn around} or \textbf{turn round} means the same as turn .
 \textit{
	\begin{itemize}
	\item He turned abruptly and walked away.
	\item He turned to his publicist and jokingly asked, 'What's next?'.
	\item He sighed, turning away and surveying the sea.
	\item He turned his head left and right.
	\item He waited for the woman to turn her face back to the road.
	\item I felt a tapping on my shoulder and I turned around.
	\item Turn your upper body round so that your shoulders are facing to the side.
	\end{itemize}
}
\item verb \\
When you \textbf{turn} something, you move it so that it is facing in a different or opposite direction,
or is in a very different position.
 \textit{
	\begin{itemize}
	\item They turned their telescopes towards other nearby galaxies.
	\item Turn the cake the right way up on to a wire rack.
	\item I turned my jacket inside out.
	\item She had turned the bedside chair to face the door.
	\item The lid, turned upside down, served as a coffee table.
	\end{itemize}
}
\item verb \\
When something such as a wheel \textbf{turns} , or when you \textbf{turn} it, it continually moves around in a particular direction.
 \textit{
	\begin{itemize}
	\item As the wheel turned, the potter shaped the clay.
	\item The engine turned a propeller.
	\end{itemize}
}
\item verb \\
When you \textbf{turn} something such as a key, knob , or switch , or when it \textbf{turns} , you hold it and twist your hand, in order to open something or make it start working.
 \textit{
	\begin{itemize}
	\item Turn a special key, press the brake pedal, and your car's brakes lock.
	\item Turn the heat to very low and cook for 20 minutes.
	\item I tried the doorknob and it turned.
	\end{itemize}
}
\item verb \\
When you \textbf{turn} in a particular direction or \textbf{turn} a corner, you change the direction in which you are moving or travelling.
 \textbf{Turn} is also a noun.
 \textit{
	\begin{itemize}
	\item He turned into the narrow terraced street where he lived.
	\item Now turn right to follow West Ferry Road.
	\item The man with the umbrella turned the corner again.
	\item You can't do a right-hand turn here.
	\end{itemize}
}
\item verb \\
The point where a road, path, or river \textbf{turns} , is the point where it has a bend or curve in it.
 \textbf{Turn} is also a noun.
 \textit{
	\begin{itemize}
	\item ...the corner where Tenterfield Road turned into the main road.
	\item ...a sharp turn in the road.
	\end{itemize}
}
\item verb \\
When the tide  \textbf{turns} , it starts coming in or going out.
 \textit{
	\begin{itemize}
	\item There was not much time before the tide turned.
	\end{itemize}
}
\item verb \\
When someone \textbf{turns} a cartwheel or a somersault , they do a cartwheel or somersault.
 \textit{
	\begin{itemize}
	\item They were still doing wild acrobatics in the yard, turning somersaults and cartwheels.
	\end{itemize}
}
\item verb \\
When you \textbf{turn} a page of a book or magazine, you move it so that is flat against the previous page,
and you can read the next page.
 \textit{
	\begin{itemize}
	\item He turned the pages of a file in front of him.
	\end{itemize}
}
\item verb \\
If you \textbf{turn} a weapon or an aggressive feeling \textbf{on} someone, you point it at them or direct it at them.
 \textit{
	\begin{itemize}
	\item He tried to turn the gun on me.
	\item The crowd than turned their anger on the Prime Minister.
	\end{itemize}
}
\item verb \\
If you \textbf{turn to} a particular page in a book or magazine, you open it at that page.
 \textit{
	\begin{itemize}
	\item To order, turn to page 236.
	\end{itemize}
}
\item verb \\
If you \textbf{turn} your attention or thoughts \textbf{to} a particular subject or if you \textbf{turn to} it, you start thinking about it or discussing it.
 \textit{
	\begin{itemize}
	\item We turned our attention to the practical matters relating to forming a company.
	\item We turn now to the British news.
	\end{itemize}
}
\item verb \\
If you \textbf{turn to} someone, you ask for their help or advice.
 \textit{
	\begin{itemize}
	\item For assistance, they turned to one of the city's most innovative museums.
	\item There was no one to turn to, no one to tell.
	\end{itemize}
}
\item verb \\
If you \textbf{turn}  \textbf{to} a particular activity, job, or way of doing something, you start doing or using it.
 \textit{
	\begin{itemize}
	\item These communities are now turning to recycling in large numbers.
	\item She quickly turned to the practical task of finding the nurse.
	\item Universities are turning from academic to commercial sponsorship.
	\end{itemize}
}
\item verb \\
To \textbf{turn} or \textbf{be turned}  \textbf{into} something means to become that thing.
 \textit{
	\begin{itemize}
	\item A prince turns into a frog in this cartoon fairytale.
	\item Their grief turned to hysteria when the funeral procession arrived at the cemetery.
	\item ...an ambitious programme to turn the country into a functioning democracy.
	\item He soon turned his dreams to reality.
	\item ...an MP turned diplomat.
	\end{itemize}
}
\item link verb \\
You can use \textbf{turn} before an adjective to indicate that something or someone changes by acquiring the
quality described by the adjective.
 \textit{
	\begin{itemize}
	\item If the bailiff thinks that things could turn nasty, he will enlist the help of the
police.
	\item She announced that she was going to turn professional.
	\end{itemize}
}
\item link verb \\
If something \textbf{turns} a particular colour or if something \textbf{turns} it a particular colour, it becomes that colour.
 \textit{
	\begin{itemize}
	\item The sea would turn pale pink and the sky blood red.
	\item Her contact lenses turned her eyes green.
	\end{itemize}
}
\item link verb \\
You can use \textbf{turn} to indicate that there is a change to a particular kind of weather. For example,
if it \textbf{turns} cold, the weather starts being cold.
 \textit{
	\begin{itemize}
	\item If it turns cold, cover plants.
	\item The weather had turned warm and thundery overnight.
	\end{itemize}
}
\item countable noun \\
If a situation or trend takes a particular kind of \textbf{turn} , it changes so that it starts developing in a different or opposite way.
 \textit{
	\begin{itemize}
	\item The scandal took a new turn over the weekend.
	\item ...the latest turn in the fighting.
	\item Retailers have given up waiting for a turn in the housing market.
	\end{itemize}
}
\item ergative verb \\
In sports, if a game \textbf{turns} , or \textbf{is turned} , something significant happens which changes the way the game is developing.
 \textit{
	\begin{itemize}
	\item The game turned in the 56th minute.
	\item With two direct hits and another sharp throw, Richards had turned the game.
	\end{itemize}
}
\item verb \\
If a business \textbf{turns} a profit, it earns more money than it spends.
 \textit{
	\begin{itemize}
	\item The firm will be able to service debt and still turn a modest profit.
	\item He says the fares are just too low to turn profits.
	\end{itemize}
}
\item verb \\
When someone \textbf{turns} a particular age, they pass that age. When it \textbf{turns} a particular time, it passes that time.
 \textit{
	\begin{itemize}
	\item It was his ambition to accumulate a million dollars before he turned thirty.
	\item It had just turned twelve o'clock.
	\end{itemize}
}
\item singular noun \\
\textbf{Turn} is used in expressions such as \textbf{the turn of the century} and \textbf{the turn of the year} to refer to a period of time when one century or year is ending and the next one is beginning.
 \textit{
	\begin{itemize}
	\item They fled to South America around the turn of the century.
	\end{itemize}
}
\item verb \\
When someone \textbf{turns} a wooden or metal object that they are making, they shape it using a special tool.
 \textit{
	\begin{itemize}
	\item ...the joys of making a living from turning wood.
	\item ...finely-turned metal.
	\end{itemize}
}
\item countable noun \\
If it is your \textbf{turn}  \textbf{to} do something, you now have the duty, chance, or right to do it, when other people
have done it before you or will do it after you.
 \textit{
	\begin{itemize}
	\item Tonight it's my turn to cook.
	\item Let each child have a turn at fishing.
	\item Students are expected to take their turn leading the study group.
	\end{itemize}
}
\item countable noun \\
If you say that someone is having a \textbf{turn} , you mean they feel suddenly very unwell for a short period of time.
 \textit{
	\begin{itemize}
	\item He is having one of his turns.
	\item He gets funny turns, you know. It's his age.
	\end{itemize}
}
\item  \\
 by turns \textit{
	\begin{itemize}
	\end{itemize}
}
\item  \\
 turn of events \textit{
	\begin{itemize}
	\end{itemize}
}
\item  \\
 at every turn \textit{
	\begin{itemize}
	\end{itemize}
}
\item  \\
 a good turn \textit{
	\begin{itemize}
	\end{itemize}
}
\item  \\
 turn sth inside out \textit{
	\begin{itemize}
	\end{itemize}
}
\item  \\
 turn sth inside out/turn sth upside down \textit{
	\begin{itemize}
	\end{itemize}
}
\item  \\
 in turn \textit{
	\begin{itemize}
	\end{itemize}
}
\item  \\
 in turn \textit{
	\begin{itemize}
	\end{itemize}
}
\item  \\
 turn of mind \textit{
	\begin{itemize}
	\end{itemize}
}
\item  \\
 speak out of turn/talk out of turn \textit{
	\begin{itemize}
	\end{itemize}
}
\item  \\
 turn of speed \textit{
	\begin{itemize}
	\end{itemize}
}
\item  \\
 take turns/take it in turns \textit{
	\begin{itemize}
	\end{itemize}
}
\item  \\
 take a turn for the worse/take a turn for the better \textit{
	\begin{itemize}
	\end{itemize}
}
\end{enumerate}

\section*{world}
{\large \color{blue}  worlds  }
\subsection*{Explain}
\begin{enumerate}
\item singular noun \\
\textbf{The world} is the planet that we live on.
 \textit{
	\begin{itemize}
	\item It's a beautiful part of the world.
	\item More than anything, I'd like to drive around the world.
	\item The satellite enables us to calculate their precise location anywhere in the world.
	\end{itemize}
}
\item singular noun \\
\textbf{The}  \textbf{world}  refers to all the people who live on this planet, and our societies , institutions , and ways of life.
 \textit{
	\begin{itemize}
	\item The world was, and remains, shocked.
	\item He wants to show the world that anyone can learn to be an ambassador.
	\item ...his personal contribution to world history.
	\item ...inflationary pressures in the world economy.
	\end{itemize}
}
\item adjective \\
You can use \textbf{world} to describe someone or something that is one of the most important or significant of its kind on earth.
 \textit{
	\begin{itemize}
	\item He was regarded as a leading world statesman.
	\item China has once again emerged as a world power.
	\item He was one of Newcastle's most distinguished medical men, a world authority on heart-diseases.
	\end{itemize}
}
\item singular noun \\
You can use \textbf{world} in expressions such as \textbf{the Arab world} , \textbf{the western world} , and \textbf{the ancient world} to refer to a particular group of countries or a particular period in history.
 \textit{
	\begin{itemize}
	\item Athens had strong ties to the Arab world.
	\item ...the developing world.
	\item Dogs were also associated with healing in the ancient world.
	\end{itemize}
}
\item countable noun \\
Someone's \textbf{world} is the life they lead, the people they have contact with, and the things they experience.
 \textit{
	\begin{itemize}
	\item His world seemed so different from mine.
	\item I lost my job and it was like my world collapsed.
	\item I tried to understand the adult world and could not.
	\end{itemize}
}
\item singular noun \\
You can use \textbf{world} to refer to a particular field of activity, and the people involved in it.
 \textit{
	\begin{itemize}
	\item The publishing world had certainly never seen an event quite like this.
	\item ...the latest news from the world of finance.
	\end{itemize}
}
\item countable noun \\
You can use \textbf{world} to refer to a place or way of life by describing its strongest  features .
 \textit{
	\begin{itemize}
	\item ...a golf course set in a hidden world of parkland, forest and lakes.
	\item The patient must re-enter a world full of problems and stresses.
	\end{itemize}
}
\item singular noun \\
You can use \textbf{world} in expressions such as \textbf{this world} , \textbf{the next world} , and \textbf{the world to come} to refer to the state of being alive or a state of existence after death .
 \textit{
	\begin{itemize}
	\item Good fortune will follow you, both in this world and the next.
	\end{itemize}
}
\item singular noun \\
You can use \textbf{world} to refer to a particular group of living things, for example  \textbf{the animal world} , \textbf{the plant world} , and \textbf{the insect world} .
 \textit{
	\begin{itemize}
	\end{itemize}
}
\item countable noun \\
A \textbf{world} is a planet.
 \textit{
	\begin{itemize}
	\item He looked like something from another world.
	\item Man was drawing closer to the stars, opening new worlds.
	\end{itemize}
}
\item  \\
 worlds apart \textit{
	\begin{itemize}
	\end{itemize}
}
\item  \\
 the best of both worlds \textit{
	\begin{itemize}
	\end{itemize}
}
\item  \\
 bring a child into the world \textit{
	\begin{itemize}
	\end{itemize}
}
\item  \\
 a world of difference \textit{
	\begin{itemize}
	\end{itemize}
}
\item  \\
 for the world \textit{
	\begin{itemize}
	\end{itemize}
}
\item  \\
 do sb/sth a world of good/do sb/sth the world of good \textit{
	\begin{itemize}
	\end{itemize}
}
\item  \\
 in the world \textit{
	\begin{itemize}
	\end{itemize}
}
\item  \\
 what in the world/who in the world/where in the world \textit{
	\begin{itemize}
	\end{itemize}
}
\item  \\
 in an ideal/a perfect world \textit{
	\begin{itemize}
	\end{itemize}
}
\item  \\
 the world owes them a living \textit{
	\begin{itemize}
	\end{itemize}
}
\item  \\
 man of the world \textit{
	\begin{itemize}
	\end{itemize}
}
\item  \\
 out of this world \textit{
	\begin{itemize}
	\end{itemize}
}
\item  \\
 the outside world \textit{
	\begin{itemize}
	\end{itemize}
}
\item  \\
 the world over \textit{
	\begin{itemize}
	\end{itemize}
}
\item  \\
 in a world of one's own \textit{
	\begin{itemize}
	\end{itemize}
}
\item  \\
 to think the world of someone \textit{
	\begin{itemize}
	\end{itemize}
}
\item  \\
 go/come up/down in the world \textit{
	\begin{itemize}
	\end{itemize}
}
\end{enumerate}

\section*{whirl}
{\large \color{blue}  whirls  whirling  whirled  }
\subsection*{Explain}
\begin{enumerate}
\item verb \\
If something or someone \textbf{whirls} around or if you \textbf{whirl} them around, they move around or turn around very quickly.
 \textbf{Whirl} is also a noun .
 \textit{
	\begin{itemize}
	\item Not receiving an answer, she whirled round.
	\item He was whirling Anne around the floor.
	\item The smoke began to whirl and grew into a monstrous column.
	\item ...the barely audible whirl of wheels.
	\end{itemize}
}
\item countable noun \\
You can refer to a lot of intense  activity as a \textbf{whirl} of activity.
 \textit{
	\begin{itemize}
	\item In half an hour's whirl of activity, she does it all.
	\item Your life is such a social whirl.
	\end{itemize}
}
\item  \\
 in a whirl \textit{
	\begin{itemize}
	\end{itemize}
}
\item  \\
 give it a whirl \textit{
	\begin{itemize}
	\end{itemize}
}
\end{enumerate}

\section*{aim}
{\large \color{blue}  aims  aiming  aimed  }
\subsection*{Explain}
\begin{enumerate}
\item verb \\
If you \textbf{aim}  \textbf{for} something or \textbf{aim}  \textbf{to} do something, you plan or hope to achieve it.
 \textit{
	\begin{itemize}
	\item He is aiming for the 100 metres world record.
	\item Businesses will have to aim at long-term growth.
	\item ...an appeal which aims to raise funds for children with special needs.
	\end{itemize}
}
\item countable noun \\
The \textbf{aim} of something that you do is the purpose for which you do it or the result that it
is intended to achieve.
 \textit{
	\begin{itemize}
	\item The aim of the festival is to increase awareness of Hindu culture and traditions.
	\item ...a research programme that has largely failed to achieve its principal aims.
	\end{itemize}
}
\item passive verb \\
If an action or plan \textbf{is aimed}  \textbf{at} achieving something, it is intended or planned to achieve it.
 \textit{
	\begin{itemize}
	\item The new measures are aimed at tightening existing sanctions.
	\item ...talks aimed at ending the war.
	\end{itemize}
}
\item verb \\
If you \textbf{aim}  \textbf{to} do something, you decide or want to do it.
 \textit{
	\begin{itemize}
	\item Are you aiming to visit the gardens?
	\item I didn't aim to get caught.
	\end{itemize}
}
\item verb \\
If your actions or remarks  \textbf{are aimed}  \textbf{at} a particular person or group, you intend that the person or group should notice them and be influenced by them.
 \textit{
	\begin{itemize}
	\item The stark message was aimed at the heads of some of Britain's biggest banks.
	\item Advertising aimed at children should be curbed.
	\end{itemize}
}
\item verb \\
If you \textbf{aim} a weapon or object \textbf{at} something or someone, you point it towards them before firing or throwing it.
 \textit{
	\begin{itemize}
	\item He was aiming the rifle at Wade.
	\item ...a missile aimed at the arms factory.
	\item I didn't know I was supposed to aim at the same spot all the time.
	\end{itemize}
}
\item singular noun \\
Your \textbf{aim} is your skill or action in pointing a weapon or other object at its target.
 \textit{
	\begin{itemize}
	\item He stood with the gun in his right hand and his left hand steadying his aim.
	\end{itemize}
}
\item verb \\
If you \textbf{aim} a kick or punch at someone, you try to kick or punch them.
 \textit{
	\begin{itemize}
	\item They aimed kicks at his shins.
	\end{itemize}
}
\item  \\
 take aim \textit{
	\begin{itemize}
	\end{itemize}
}
\item  \\
 take aim at \textit{
	\begin{itemize}
	\end{itemize}
}
\end{enumerate}

\section*{attack}
{\large \color{blue}  attacks  attacking  attacked  }
\subsection*{Explain}
\begin{enumerate}
\item verb \\
To \textbf{attack} a person or place means to try to hurt or damage them using physical violence .
 \textbf{Attack} is also a noun .
 \textit{
	\begin{itemize}
	\item Fifty civilians were killed when government planes attacked the town.
	\item The robbers brutally attacked the security guard.
	\item While Haig and Foch argued, the Germans attacked.
	\item The infantry would use hit and run tactics to slow attacking forces.
	\item ...a campaign of air attacks on strategic targets.
	\item Refugees had come under attack from federal troops.
	\end{itemize}
}
\item verb \\
If you \textbf{attack} a person, belief , idea , or act, you criticize them strongly.
 \textbf{Attack} is also a noun.
 \textit{
	\begin{itemize}
	\item He publicly attacked the people who've been calling for secret ballot nominations.
	\item A newspaper ran an editorial attacking him for being a showman.
	\item The role of the state as a prime mover in planning social change has been under attack.
	\item The committee yesterday launched a scathing attack on British business for failing
to invest.
	\end{itemize}
}
\item verb \\
If something such as a disease, a chemical, or an insect \textbf{attacks} something, it harms or spoils it.
 \textbf{Attack} is also a noun.
 \textit{
	\begin{itemize}
	\item The virus seems to have attacked his throat.
	\item Several key crops failed when they were attacked by pests.
	\item This greatly reduces attacks from pests and diseases.
	\end{itemize}
}
\item verb \\
If you \textbf{attack} a job or a problem, you start to deal with it in an energetic way.
 \textit{
	\begin{itemize}
	\item ...an attempt to attack the budget problem.
	\end{itemize}
}
\item verb \\
In games such as football , when one team  \textbf{attacks} the opponent's goal , they try to score a goal.
 \textbf{Attack} is also a noun.
 \textit{
	\begin{itemize}
	\item Now the U.S. is controlling the ball and attacking the opponent's goal.
	\item The goal was just reward for Villa's decision to attack constantly in the second
half.
	\item Lee was at the hub of some incisive attacks in the second half.
	\end{itemize}
}
\item countable noun \\
An \textbf{attack}  \textbf{of} an illness is a short period in which you suffer  badly from it.
 \textit{
	\begin{itemize}
	\item It had brought on an attack of asthma.
	\end{itemize}
}
\end{enumerate}

\section*{average}
{\large \color{blue}  averages  averaging  averaged  }
\subsection*{Explain}
\begin{enumerate}
\item countable noun \\
An \textbf{average} is the result that you get when you add two or more numbers together and divide the total by the number of numbers you added
together.
 \textbf{Average} is also an adjective .
 \textit{
	\begin{itemize}
	\item Take the average of those ratios and multiply by a hundred.
	\item The average price of goods rose by just 2.2%.
	\item The average age of the women interviewed was only 21.5.
	\end{itemize}
}
\item singular noun \\
You use \textbf{average} to refer to a number or size that varies but is always approximately the same.
 \textit{
	\begin{itemize}
	\item It takes an average of ten weeks for a house sale to be completed.
	\end{itemize}
}
\item adjective \\
An \textbf{average} person or thing is typical or normal.
 \textit{
	\begin{itemize}
	\item The average adult man burns 1,500 to 2,000 calories per day.
	\item Packaging is about a third of what is found in an average British dustbin.
	\end{itemize}
}
\item singular noun \\
An amount or quality that is \textbf{the average} is the normal amount or quality for a particular group of things or people.
 \textbf{Average} is also an adjective.
 \textit{
	\begin{itemize}
	\item Most areas suffered more rain than usual, with Northern Ireland getting double the
average for the month.
	\item £2.20 for a coffee is average.
	\item ...a woman of average height.
	\end{itemize}
}
\item adjective \\
Something that is \textbf{average} is neither very good nor very bad , usually when you had hoped it would be better .
 \textit{
	\begin{itemize}
	\item I was only average academically.
	\end{itemize}
}
\item verb \\
To \textbf{average} a particular amount means to do, get, or produce that amount as an average over a
period of time.
 \textit{
	\begin{itemize}
	\item We averaged 42 miles per hour.
	\item ...pay increases averaging 9.75%.
	\end{itemize}
}
\item  \\
 on average/on an average \textit{
	\begin{itemize}
	\end{itemize}
}
\item  \\
 on average \textit{
	\begin{itemize}
	\end{itemize}
}
\end{enumerate}

\section*{bite}
{\large \color{blue}  bites  biting  bit  bitten  }
\subsection*{Explain}
\begin{enumerate}
\item verb \\
If you \textbf{bite} something, you use your teeth to cut into it, for example in order to eat it or break
it. If an animal or person \textbf{bites} you, they use their teeth to hurt or injure you.
 \textit{
	\begin{itemize}
	\item Both sisters bit their nails as children.
	\item He bit into his sandwich.
	\item He had bitten the cigarette in two.
	\item Every year in this country more than 50,000 children are bitten by dogs.
	\item Llamas won't bite or kick.
	\end{itemize}
}
\item countable noun \\
A \textbf{bite} of something, especially food, is the action of biting it.
 A \textbf{bite} is also the amount of food you take into your mouth when you bite it.
 \textit{
	\begin{itemize}
	\item He took another bite of apple.
	\item You cannot eat a bun in one bite.
	\item Look forward to eating the food and enjoy every bite.
	\end{itemize}
}
\item singular noun \\
If you have \textbf{a bite} to eat, you have a small meal or a snack.
 \textit{
	\begin{itemize}
	\item It was time to go home for a little rest and a bite to eat.
	\end{itemize}
}
\item verb \\
If a snake or a small insect \textbf{bites} you, it makes a mark or hole in your skin, and often causes the surrounding area
of your skin to become painful or itchy.
 \textit{
	\begin{itemize}
	\item When an infected mosquito bites a human, spores are injected into the blood.
	\item We were all badly bitten by mosquitoes.
	\end{itemize}
}
\item countable noun \\
A \textbf{bite} is an injury or a mark on your body where an animal, snake, or small insect has bitten you.
 \textit{
	\begin{itemize}
	\item Any dog bite, no matter how small, needs immediate medical attention.
	\end{itemize}
}
\item verb \\
When an action or policy  begins to \textbf{bite} , it begins to have a serious or harmful effect.
 \textit{
	\begin{itemize}
	\item As the sanctions begin to bite there will be more political difficulties ahead.
	\item The recession started biting deeply into British industry.
	\end{itemize}
}
\item verb \\
If an object \textbf{bites} into a surface, it presses hard against it or cuts into it.
 \textit{
	\begin{itemize}
	\item There may even be some wire or nylon biting into the flesh.
	\item The car's tires bit loudly on the rutted snow in the street.
	\end{itemize}
}
\item uncountable noun \\
If you say that a food or drink has \textbf{bite} , you like it because it has a strong or sharp  taste .
 \textit{
	\begin{itemize}
	\item ...the addition of tartaric acid to give the wine some bite.
	\end{itemize}
}
\item singular noun \\
If the air or the wind has \textbf{a bite} , it feels very cold .
 \textit{
	\begin{itemize}
	\item There was a bite in the air, a smell perhaps of snow.
	\end{itemize}
}
\item uncountable noun \\
If something such as a performance or a piece of writing has \textbf{bite} , it is exciting or effective .
 \textit{
	\begin{itemize}
	\item The teams have that extra bite when they are playing against their neighbours.
	\item The novel seems to lack bite and tension–even passion.
	\end{itemize}
}
\item verb \\
If a fish \textbf{bites} when you are fishing, it takes the hook or bait at the end of your fishing line in its mouth.
 \textbf{Bite} is also a noun .
 \textit{
	\begin{itemize}
	\item After half an hour, the fish stopped biting and we moved on.
	\item If I don't get a bite in a few minutes I lift the rod and twitch the bait.
	\end{itemize}
}
\item countable noun \\
A \textbf{bite}  \textbf{of} something is a small part or amount of it.
 \textit{
	\begin{itemize}
	\item ...bites of conversation.
	\end{itemize}
}
\item  \\
 to bite the hand that feeds you \textit{
	\begin{itemize}
	\end{itemize}
}
\item  \\
 bite someone's head off \textit{
	\begin{itemize}
	\end{itemize}
}
\item  \\
 bite one's lip \textit{
	\begin{itemize}
	\end{itemize}
}
\item  \\
 take a bite out of \textit{
	\begin{itemize}
	\end{itemize}
}
\end{enumerate}

\section*{brake}
{\large \color{blue}  brakes  braking  braked  }
\subsection*{Explain}
\begin{enumerate}
\item countable noun \\
\textbf{Brakes} are devices in a vehicle that make it go slower or stop .
 \textit{
	\begin{itemize}
	\item The brakes began locking.
	\item A seagull swooped down in front of her car, causing her to slam on the brakes.
	\end{itemize}
}
\item verb \\
When a vehicle or its driver  \textbf{brakes} , or when a driver \textbf{brakes} a vehicle, the driver makes it slow down or stop by using the brakes.
 \textit{
	\begin{itemize}
	\item He heard tires squeal as the car braked to avoid a collision.
	\item She braked sharply to avoid another car.
	\item The system automatically brakes the car if there is an imminent risk of a collision.
	\item She braked to a halt and switched off.
	\end{itemize}
}
\item countable noun \\
You can use \textbf{brake} in a number of expressions to indicate that something has slowed down or stopped.
 \textit{
	\begin{itemize}
	\item Illness had put a brake on his progress.
	\item You can take the financial brakes off in June.
	\end{itemize}
}
\end{enumerate}

\section*{card}
{\large \color{blue}  cards  }
\subsection*{Explain}
\begin{enumerate}
\item countable noun \\
A \textbf{card} is a piece of stiff paper or thin cardboard on which something is written or printed.
 \textit{
	\begin{itemize}
	\item Check the numbers below against the numbers on your card.
	\end{itemize}
}
\item countable noun \\
A \textbf{card} is a piece of cardboard or plastic , or a small document, which shows information about you and which you carry with
you, for example to prove your identity .
 \textit{
	\begin{itemize}
	\item They check my bag and press card.
	\item ...her membership card.
	\item The authorities have begun to issue ration cards.
	\end{itemize}
}
\item countable noun \\
A \textbf{card} is a rectangular piece of plastic, issued by a bank, company, or shop, which you
can use to buy things or obtain money.
 \textit{
	\begin{itemize}
	\item He paid the whole bill with an American Express card.
	\item Holiday-makers should beware of using plastic cards in foreign cash dispensers.
	\end{itemize}
}
\item countable noun \\
A \textbf{card} is a folded piece of stiff paper with a picture and sometimes a message printed on it, which
you send to someone on a special occasion .
 \textit{
	\begin{itemize}
	\item She sends me a card on my birthday.
	\item ...millions of get-well cards.
	\end{itemize}
}
\item countable noun \\
A \textbf{card} is the same as a postcard .
 \textit{
	\begin{itemize}
	\item Send your details on a card to the following address.
	\end{itemize}
}
\item countable noun \\
A \textbf{card} is a piece of thin cardboard carried by someone such as a business person in order
to give to other people. A card shows the name, address , telephone number, and other details of the person who carries it.
 \textit{
	\begin{itemize}
	\item Here's my card. You may need me.
	\end{itemize}
}
\item countable noun \\
\textbf{Cards} are thin pieces of cardboard with numbers or pictures printed on them which are used
to play various games.
 \textit{
	\begin{itemize}
	\item ...a pack of cards.
	\item Kurt picked up his hand and fanned out the cards one by one.
	\end{itemize}
}
\item uncountable noun \\
If you are playing \textbf{cards} , you are playing a game using cards.
 \textit{
	\begin{itemize}
	\item A group of officers was sitting round a table in the sun playing cards.
	\end{itemize}
}
\item countable noun \\
You can use \textbf{card} to refer to something that gives you an advantage in a particular situation. If you play a
particular \textbf{card} , you use that advantage.
 \textit{
	\begin{itemize}
	\item It was his strongest card in their relationship–that she wanted him more than he
wanted her.
	\item This permitted Western manufacturers to play their strong cards: capital and technology.
	\end{itemize}
}
\item uncountable noun \\
\textbf{Card} is strong, stiff paper or thin cardboard.
 \textit{
	\begin{itemize}
	\item She put the pieces of card in her pocket.
	\end{itemize}
}
\item countable noun \\
In computing , a \textbf{card} is a circuit board that can be put into a computer to provide additional  memory or functions.
 \textit{
	\begin{itemize}
	\end{itemize}
}
\item countable noun \\
You can use \textbf{card} to refer to a series of races or matches at a particular sporting event.
 \textit{
	\begin{itemize}
	\item Paradise Boy and the Galloping General have clear chances in the opening two events
on the card.
	\end{itemize}
}
\item verb \\
If you \textbf{are carded} , someone in authority asks you to show a document to prove that you are old enough to do something, for example,
to buy or drink alcohol .
 \textit{
	\begin{itemize}
	\item For the first time in many years, I got carded.
	\end{itemize}
}
\item  \\
 on the cards \textit{
	\begin{itemize}
	\end{itemize}
}
\item  \\
 to play your cards right \textit{
	\begin{itemize}
	\end{itemize}
}
\item  \\
 to put your cards on the table \textit{
	\begin{itemize}
	\end{itemize}
}
\end{enumerate}

\section*{chase}
{\large \color{blue}  chases  chasing  chased  }
\subsection*{Explain}
\begin{enumerate}
\item verb \\
If you \textbf{chase} someone, or \textbf{chase}  \textbf{after} them, you run after them or follow them quickly in order to catch or reach them.
 \textbf{Chase} is also a noun .
 \textit{
	\begin{itemize}
	\item She chased the thief for 100 yards.
	\item He said nothing to waiting journalists, who chased after him as he left.
	\item He was reluctant to give up the chase.
	\item Police said he was arrested without a struggle after a car chase through the streets
of Biarritz.
	\end{itemize}
}
\item verb \\
If you \textbf{are chasing} something you want , such as work or money, you are trying  hard to get it.
 \textbf{Chase} is also a noun.
 \textit{
	\begin{itemize}
	\item In Wales, 14 people are chasing every job.
	\item There are too many schools chasing too few pupils.
	\item ...publishers and booksellers chasing after profits from high-volume sales.
	\item They took an invincible lead in the chase for the championship.
	\end{itemize}
}
\item verb \\
If someone \textbf{chases} someone that they are attracted to, or \textbf{chases}  \textbf{after} them, they try hard to persuade them to have a sexual relationship with them.
 \textbf{Chase} is also a noun.
 \textit{
	\begin{itemize}
	\item I didn't go around flirting or chasing women.
	\item 'I was always chasing after unsuitable men,' she says.
	\item The chase is always much more exciting than the conquest anyway.
	\end{itemize}
}
\item verb \\
If someone \textbf{chases} you from a place, they force you to leave by using threats or violence .
 \textit{
	\begin{itemize}
	\item Many farmers will then chase you off their land quite aggressively.
	\item Angry demonstrators chased him away.
	\end{itemize}
}
\item  \\
 cut to the chase \textit{
	\begin{itemize}
	\end{itemize}
}
\item verb \\
To \textbf{chase} someone \textbf{from} a job or a position or \textbf{from} power means to force them to leave it.
 \textit{
	\begin{itemize}
	\item The army will not allow its commander-in-chief to be chased from power.
	\end{itemize}
}
\item verb \\
If you \textbf{chase}  somewhere , you run or rush there.
 \textit{
	\begin{itemize}
	\item They chased down the stairs into the narrow, dirty street.
	\item ...chasing about late at night in search of life's necessities.
	\end{itemize}
}
\item singular noun \\
\textbf{The chase} is the activity of hunting animals.
 \textit{
	\begin{itemize}
	\item ...bear robes, mountain lion hides, and other trophies of the chase.
	\end{itemize}
}
\item noun, in names \\
\textbf{Chase} is often used in the name of horse races in which the horses have to jump over obstacles such as fences or bushes .
 \textit{
	\begin{itemize}
	\item ...the Champion Hunter Chase.
	\end{itemize}
}
\item  \\
 give chase \textit{
	\begin{itemize}
	\end{itemize}
}
\item  \\
 the thrill of the chase \textit{
	\begin{itemize}
	\end{itemize}
}
\end{enumerate}

\section*{civilian}
{\large \color{blue}  civilians  }
\subsection*{Explain}
\begin{enumerate}
\item countable noun \\
In a military situation , a \textbf{civilian} is anyone who is not a member of the armed forces.
 \textit{
	\begin{itemize}
	\item The safety of civilians caught up in the fighting must be guaranteed.
	\end{itemize}
}
\item adjective \\
In a military situation, \textbf{civilian} is used to describe people or things that are not military.
 \textit{
	\begin{itemize}
	\item ...the country's civilian population.
	\item ...civilian casualties.
	\item ...a soldier in civilian clothes.
	\end{itemize}
}
\end{enumerate}

\section*{chat}
{\large \color{blue}  chats  chatting  chatted  }
\subsection*{Explain}
\begin{enumerate}
\item verb \\
When people \textbf{chat} , they talk to each other in an informal and friendly way.
 \textbf{Chat} is also a noun .
 \textit{
	\begin{itemize}
	\item The women were chatting.
	\item I was chatting to him the other day.
	\item He's chatting with his dad.
	\item We chatted about old times.
	\item I had a chat with John.
	\end{itemize}
}
\item verb \\
When people \textbf{chat} , they exchange short  written messages on the internet or on their phones .
 \textbf{Chat} is also a noun.
 \textit{
	\begin{itemize}
	\item The software allows students to collaborate on documents and to chat via instant
messaging.
	\item Kirstie was chatting with friends on a website.
	\item The author took part in a live web chat.
	\end{itemize}
}
\end{enumerate}

\section*{cocaine}
{\large \color{blue}  }
\subsection*{Explain}
\begin{enumerate}
\item uncountable noun \\
\textbf{Cocaine} is a powerful drug which some people take for pleasure , but which they can become addicted to.
 \textit{
	\begin{itemize}
	\end{itemize}
}
\end{enumerate}

\section*{close}
{\large \color{blue}  closes  closing  closed  }
\subsection*{Explain}
\begin{enumerate}
\item verb \\
When you \textbf{close} something such as a door or lid or when it \textbf{closes} , it moves so that a hole, gap , or opening is covered.
 \textit{
	\begin{itemize}
	\item If you are cold, close the window.
	\item Zacharias heard the door close.
	\item Keep the curtains closed.
	\end{itemize}
}
\item verb \\
When you \textbf{close} something such as an open book or umbrella , you move the different parts of it together.
 \textit{
	\begin{itemize}
	\item Slowly he closed the book.
	\end{itemize}
}
\item verb \\
If you \textbf{close} something such as a computer file or window, you give the computer an instruction to remove it from the screen.
 \textit{
	\begin{itemize}
	\item To close your document, press CTRL+W on your keyboard.
	\end{itemize}
}
\item verb \\
When you \textbf{close} your eyes or your eyes \textbf{close} , your eyelids move downwards , so that you can no longer see.
 \textit{
	\begin{itemize}
	\item Bess closed her eyes and fell asleep.
	\item When we sneeze, our eyes close.
	\end{itemize}
}
\item verb \\
When a place \textbf{closes} or \textbf{is closed} , work or activity stops there for a short period.
 \textit{
	\begin{itemize}
	\item Shops close only on Christmas Day and New Year's Day.
	\item It was Saturday; they could close the office early.
	\item Government troops closed the airport.
	\item The restaurant was closed for the night.
	\end{itemize}
}
\item verb \\
If a place such as a factory , shop, or school \textbf{closes} , or if it \textbf{is closed} , all work or activity stops there permanently.
 \textbf{Close down} means the same as close1 .
 \textit{
	\begin{itemize}
	\item Many enterprises will be forced to close.
	\item If they do close the local college I'll have to go to Worcester.
	\item Minford closed down the business and went into politics.
	\item Many of the smaller stores have closed down.
	\end{itemize}
}
\item verb \\
To \textbf{close} a road or border means to block it in order to prevent people from using it.
 \textit{
	\begin{itemize}
	\item The police had to close the road to traffic.
	\end{itemize}
}
\item verb \\
To \textbf{close} a conversation , event, or matter means to bring it to an end or to complete it.
 \textit{
	\begin{itemize}
	\item DNA tests could close the case.
	\item He needs another $30,000 to close the deal.
	\item The Prime Minister is said to now consider the matter closed.
	\item ...the closing ceremony of the National Political Conference.
	\end{itemize}
}
\item verb \\
If you \textbf{close} a bank account, you take all your money out of it and inform the bank that you will no longer be using the account.
 \textit{
	\begin{itemize}
	\item He had closed his account with the bank five years earlier.
	\end{itemize}
}
\item verb \\
On the stock market or the currency markets, if a share price or a currency \textbf{closes} at a particular value, that is its value at the end of the day's business.
 \textit{
	\begin{itemize}
	\item Dawson shares closed at 219p, up 5p.
	\item The U.S. dollar closed higher in Tokyo today.
	\end{itemize}
}
\item singular noun \\
\textbf{The}  \textbf{close}  \textbf{of} a period of time or an activity is the end of it. To bring or draw something \textbf{to a close} means to end it.
 \textit{
	\begin{itemize}
	\item By the close of business, they knew the campaign was a success.
	\item Brian's retirement brings to a close a glorious chapter in British football history.
	\item As the year draws to a close, the story is changing.
	\end{itemize}
}
\end{enumerate}

\section*{destination}
{\large \color{blue}  destinations  }
\subsection*{Explain}
\begin{enumerate}
\item countable noun \\
The \textbf{destination} of someone or something is the place to which they are going or being sent .
 \textit{
	\begin{itemize}
	\item Spain is still our most popular holiday destination.
	\item Only half of the emergency supplies have reached their destination.
	\end{itemize}
}
\end{enumerate}

\section*{combat}
{\large \color{blue}  combats  combating  combatting  combated  combatted  }
\subsection*{Explain}
\begin{enumerate}
\item uncountable noun \\
\textbf{Combat} is fighting that takes place in a war .
 \textit{
	\begin{itemize}
	\item Over 16 million men had died in combat.
	\item Yesterday saw hand-to-hand combat in the city.
	\item ...combat aircraft.
	\end{itemize}
}
\item countable noun \\
A \textbf{combat} is a battle , or a fight between two people.
 \textit{
	\begin{itemize}
	\item It was the end of a long combat.
	\end{itemize}
}
\item verb \\
If people in authority  \textbf{combat} something, they try to stop it happening .
 \textit{
	\begin{itemize}
	\item Congress has criticised new government measures to combat crime.
	\end{itemize}
}
\end{enumerate}

\section*{dragon}
{\large \color{blue}  dragons  }
\subsection*{Explain}
\begin{enumerate}
\item countable noun \\
In stories and legends , a \textbf{dragon} is an animal like a big lizard. It has wings and claws, and breathes out fire.
 \textit{
	\begin{itemize}
	\end{itemize}
}
\item countable noun \\
If someone calls a woman, especially an older woman, a \textbf{dragon} , they mean that she is fierce and unpleasant .
 \textit{
	\begin{itemize}
	\end{itemize}
}
\end{enumerate}

\section*{cough}
{\large \color{blue}  coughs  coughing  coughed  }
\subsection*{Explain}
\begin{enumerate}
\item verb \\
When you \textbf{cough} , you force air out of your throat with a sudden , harsh  noise . You often cough when you are ill , or when you are nervous or want to attract someone's attention .
 \textbf{Cough} is also a noun .
 \textit{
	\begin{itemize}
	\item Graham began to cough violently.
	\item He coughed. 'Excuse me, Mrs Allsworthy, could I have a word?'
	\item Coughs and sneezes spread infections much faster in a warm atmosphere.
	\item They were interrupted by an apologetic cough.
	\end{itemize}
}
\item countable noun \\
A \textbf{cough} is an illness in which you cough often and your chest or throat hurts .
 \textit{
	\begin{itemize}
	\item ...if you have a persistent cough for over a month.
	\end{itemize}
}
\item verb \\
If you \textbf{cough}  blood or mucus , it comes up out of your throat or mouth when you cough.
 \textbf{Cough up}  means the same as cough .
 \textit{
	\begin{itemize}
	\item I started coughing blood so they transferred me to a hospital.
	\item Keats became feverish, continually coughing up blood.
	\end{itemize}
}
\item verb \\
If an engine or other machine  \textbf{coughs} , it makes a sudden, harsh noise.
 \textit{
	\begin{itemize}
	\item Then suddenly, the engine coughed, spluttered and died.
	\end{itemize}
}
\end{enumerate}

\section*{east}
{\large \color{blue}  }
\subsection*{Explain}
\begin{enumerate}
\item uncountable noun \\
\textbf{The}  \textbf{east} is the direction which you look towards in the morning in order to see the sun  rise .
 \textit{
	\begin{itemize}
	\item ...the vast swamps which lie to the east of the River Nile.
	\item The principal range runs east to west.
	\end{itemize}
}
\item singular noun \\
\textbf{The}  \textbf{east}  \textbf{of} a place, country, or region is the part which is in the east.
 \textit{
	\begin{itemize}
	\item ...a village in the east of the country.
	\item They are said to control large parts of the east and south of the country.
	\end{itemize}
}
\item adverb \\
If you go  \textbf{east} , you travel towards the east.
 \textit{
	\begin{itemize}
	\item To drive, go east on Route 9.
	\end{itemize}
}
\item adverb \\
Something that is \textbf{east}  \textbf{of} a place is positioned to the east of it.
 \textit{
	\begin{itemize}
	\item ...just east of the center of town.
	\end{itemize}
}
\item adjective \\
The \textbf{east}  edge , corner , or part of a place or country is the part which is towards the east.
 \textit{
	\begin{itemize}
	\item ...a low line of hills running along the east coast.
	\end{itemize}
}
\item adjective \\
\textbf{East} is used in the names of some countries, states , and regions in the east of a larger area
 \textit{
	\begin{itemize}
	\item He had been on safari in East Africa with his son.
	\end{itemize}
}
\item adjective \\
An \textbf{east} wind is a wind that blows from the east.
 \textit{
	\begin{itemize}
	\end{itemize}
}
\item singular noun \\
\textbf{The East} is used to refer to the southern and eastern part of Asia, including India , China , and Japan .
 \textit{
	\begin{itemize}
	\item Every so often, a new martial art arrives from the East.
	\end{itemize}
}
\end{enumerate}

\section*{crawl}
{\large \color{blue}  crawls  crawling  crawled  }
\subsection*{Explain}
\begin{enumerate}
\item verb \\
When you \textbf{crawl} , you move forward on your hands and knees.
 \textit{
	\begin{itemize}
	\item Don't worry if your baby seems a little reluctant to crawl or walk.
	\item I began to crawl on my hands and knees towards the door.
	\item As he tried to crawl away, he was hit in the shoulder.
	\end{itemize}
}
\item verb \\
When an insect \textbf{crawls}  somewhere , it moves there quite slowly.
 \textit{
	\begin{itemize}
	\item I watched the moth crawl up the outside of the lampshade.
	\end{itemize}
}
\item verb \\
If someone or something \textbf{crawls} somewhere, they move or progress slowly or with great  difficulty .
 \textbf{Crawl} is also a noun .
 \textit{
	\begin{itemize}
	\item I crawled out of bed at nine-thirty.
	\item They had not foreseen the higher inflation in France when most of Western Europe
was crawling out of recession.
	\item Hairpin turns force the car to crawl at 10 miles an hour in some places.
	\item The traffic on the approach road slowed to a crawl.
	\end{itemize}
}
\item verb \\
If you say that a place \textbf{is crawling with} people or animals, you are emphasizing that it is full of them.
 \textit{
	\begin{itemize}
	\item This place is crawling with police.
	\item ...rock-hard earth littered with rubbish and crawling with vermin.
	\end{itemize}
}
\item singular noun \\
\textbf{The crawl} is a kind of swimming stroke which you do lying on your front , swinging one arm over your head, and then the other arm.
 \textit{
	\begin{itemize}
	\end{itemize}
}
\item  \\
 to make your skin crawl \textit{
	\begin{itemize}
	\end{itemize}
}
\end{enumerate}

\section*{economy}
{\large \color{blue}  economies  }
\subsection*{Explain}
\begin{enumerate}
\item countable noun \\
An \textbf{economy} is the system according to which the money, industry, and trade of a country or region
are organized .
 \textit{
	\begin{itemize}
	\item Zimbabwe boasts Africa's most industrialised economy.
	\end{itemize}
}
\item countable noun \\
A country's \textbf{economy} is the wealth that it gets from business and industry.
 \textit{
	\begin{itemize}
	\item The Japanese economy grew at an annual rate of more than 10 per cent.
	\end{itemize}
}
\item uncountable noun \\
\textbf{Economy} is the use of the minimum amount of money, time, or other resources needed to achieve
something, so that nothing is wasted.
 \textit{
	\begin{itemize}
	\item ...improvements in the fuel economy of cars.
	\item There was mostly silence. I have never known such economy with words.
	\end{itemize}
}
\item countable noun \\
If you make \textbf{economies} , you try to save money by not spending money on unnecessary things.
 \textit{
	\begin{itemize}
	\item They will make economies by hiring fewer part-time workers.
	\end{itemize}
}
\item adjective \\
\textbf{Economy} services such as travel are cheap and have no luxuries or extras .
 \textit{
	\begin{itemize}
	\end{itemize}
}
\item adjective \\
\textbf{Economy} is used to describe large packs of goods which are cheaper than normal-sized packs.
 \textit{
	\begin{itemize}
	\item ...an economy pack containing 150 assorted screws.
	\end{itemize}
}
\item  \\
 a false economy \textit{
	\begin{itemize}
	\end{itemize}
}
\end{enumerate}

\section*{credit}
{\large \color{blue}  credits  crediting  credited  }
\subsection*{Explain}
\begin{enumerate}
\item uncountable noun \\
If you are allowed \textbf{credit} , you are allowed to pay for goods or services several weeks or months after you have received them.
 \textit{
	\begin{itemize}
	\item The group can't get credit to buy farming machinery.
	\item You can ask a dealer for a discount whether you pay cash or buy on credit.
	\end{itemize}
}
\item uncountable noun \\
If someone or their bank account is \textbf{in credit} , their bank account has money in it.
 \textit{
	\begin{itemize}
	\item The idea that I could be charged when I'm in credit makes me very angry.
	\item I made sure the account stayed in credit.
	\item Interest is payable on credit balances.
	\end{itemize}
}
\item verb \\
When a sum of money \textbf{is credited}  \textbf{to} an account, the bank adds that sum of money to the total in the account.
 \textit{
	\begin{itemize}
	\item She noticed that only $80,000 had been credited to her account.
	\item The bank decided to change the way it credited payments to accounts.
	\item Interest is calculated daily and credited once a year, on 1 April.
	\end{itemize}
}
\item countable noun \\
A \textbf{credit} is a sum of money which is added to an account.
 \textit{
	\begin{itemize}
	\item The statement of total debits and credits is known as a balance.
	\end{itemize}
}
\item countable noun \\
A \textbf{credit} is an amount of money that is given to someone.
 \textit{
	\begin{itemize}
	\item The senator outlined his own tax cut, giving families $350 in tax credits per child.
	\item Banks provide credit to customers in the form of loans and overdrafts.
	\end{itemize}
}
\item uncountable noun \\
If you get  \textbf{the}  \textbf{credit}  \textbf{for} something good, people praise you because you are responsible for it, or are thought to be responsible for it.
 \textit{
	\begin{itemize}
	\item We don't mind who gets the credit so long as we don't get the blame.
	\item It would be wrong for us to take all the credit.
	\item Some of the credit for her relaxed manner must go to Andy.
	\end{itemize}
}
\item verb \\
If people \textbf{credit} someone \textbf{with} an achievement or if it \textbf{is credited to} them, people say or believe that they were responsible for it.
 \textit{
	\begin{itemize}
	\item The staff are crediting him with having saved Hythe's life.
	\item The mayor is credited with helping make Los Angeles the financial capital of the
West Coast.
	\item There are 630 words whose first-time use is credited to Milton by the Oxford English
Dictionary.
	\end{itemize}
}
\item verb \\
If you \textbf{credit} someone \textbf{with} a quality, you believe or say that they have it.
 \textit{
	\begin{itemize}
	\item I wonder why you can't credit him with the same generosity of spirit.
	\item They are crediting science with power it doesn't possess.
	\end{itemize}
}
\item singular noun \\
If you say that someone is \textbf{a credit to} someone or something, you mean that their qualities or achievements will make people
have a good opinion of the person or thing mentioned .
 \textit{
	\begin{itemize}
	\item He is one of the greatest British players of recent times and is a credit to his
profession.
	\end{itemize}
}
\item verb \\
If you cannot \textbf{credit} something, you cannot believe that it is true.
 \textit{
	\begin{itemize}
	\item Roosevelt either did not learn of the scandal or refused to credit what he heard.
	\item It seems hard to credit that such things went on among senior Directors.
	\end{itemize}
}
\item countable noun \\
The list of people who helped to make a film, a CD, or a television programme is called \textbf{the}  \textbf{credits} .
 \textit{
	\begin{itemize}
	\item The star of the film wants his name removed from the credits.
	\item ...a moviegoer who remains in his seat until the credits are over.
	\end{itemize}
}
\item countable noun \\
A \textbf{credit} is a successfully-completed part of a higher education course. At some universities
and colleges you need a certain number of credits to be awarded a degree.
 \textit{
	\begin{itemize}
	\end{itemize}
}
\item  \\
 to do sb credit \textit{
	\begin{itemize}
	\end{itemize}
}
\item  \\
 credit where credit's due \textit{
	\begin{itemize}
	\end{itemize}
}
\item  \\
 to give someone credit for sth \textit{
	\begin{itemize}
	\end{itemize}
}
\item  \\
 on the credit side \textit{
	\begin{itemize}
	\end{itemize}
}
\item  \\
 to sb's credit \textit{
	\begin{itemize}
	\end{itemize}
}
\item  \\
 with/have sth to your credit \textit{
	\begin{itemize}
	\end{itemize}
}
\end{enumerate}

\section*{essence}
{\large \color{blue}  essences  }
\subsection*{Explain}
\begin{enumerate}
\item uncountable noun \\
The \textbf{essence}  \textbf{of} something is its basic and most important characteristic which gives it its individual identity.
 \textit{
	\begin{itemize}
	\item The essence of consultation is to listen to, and take account of, the views of those
consulted.
	\item ...the essence of life.
	\item Others claim that Ireland's very essence is expressed through the language.
	\end{itemize}
}
\item variable noun \\
\textbf{Essence} is a very concentrated liquid that is used for flavouring food or for its smell .
 \textit{
	\begin{itemize}
	\item ...a few drops of vanilla essence.
	\item ...exotic bath essences.
	\end{itemize}
}
\end{enumerate}

\section*{dash}
{\large \color{blue}  dashes  dashing  dashed  }
\subsection*{Explain}
\begin{enumerate}
\item verb \\
If you \textbf{dash}  somewhere , you run or go there quickly and suddenly .
 \textbf{Dash} is also a noun.
 \textit{
	\begin{itemize}
	\item Suddenly she dashed down to the cellar.
	\item She dashed in from the garden.
	\item ...a 160-mile dash to hospital.
	\end{itemize}
}
\item verb \\
If you say that you have to \textbf{dash} , you mean that you are in a hurry and have to leave immediately .
 \textit{
	\begin{itemize}
	\item Oh, Tim! I'm sorry but I have to dash.
	\item See you tomorrow night. Must dash now.
	\end{itemize}
}
\item countable noun \\
A \textbf{dash}  \textbf{of} something is a small quantity of it which you add when you are preparing food or
 mixing a drink.
 \textit{
	\begin{itemize}
	\item Add a dash of balsamic vinegar.
	\end{itemize}
}
\item countable noun \\
A \textbf{dash}  \textbf{of} a quality is a small amount of it that is found in something and often makes it more
interesting or distinctive .
 \textit{
	\begin{itemize}
	\item ...a story with a dash of mystery thrown in.
	\item ...A fake fur collar or cuff adds a dash of glamour to even the simplest style.
	\end{itemize}
}
\item verb \\
If you \textbf{dash} something \textbf{against} a wall or other surface, you throw or push it violently, often so hard that it breaks.
 \textit{
	\begin{itemize}
	\item She seized the doll and dashed it against the stone wall with tremendous force.
	\end{itemize}
}
\item verb \\
If an event or person \textbf{dashes} someone's hopes or expectations , it destroys them by making it impossible that the thing that is hoped for or expected will ever  happen .
 \textit{
	\begin{itemize}
	\item The announcement dashed hopes of an early end to the crisis.
	\item They had their championship hopes dashed by a 3–1 defeat.
	\end{itemize}
}
\item singular noun \\
If you do something in \textbf{a}  \textbf{dash} , you do it very quickly, perhaps with very bad results.
 \textit{
	\begin{itemize}
	\item With three laps to go he was edged out in a dash to the line, finishing fourth.
	\item ...the dash to buy shares in internet companies.
	\end{itemize}
}
\item countable noun \\
A \textbf{dash} is a short fast race.
 \textit{
	\begin{itemize}
	\end{itemize}
}
\item countable noun \\
A \textbf{dash} is a straight , horizontal line used in writing, for example to separate two main clauses whose meanings are closely connected.
 \textit{
	\begin{itemize}
	\end{itemize}
}
\item exclamation \\
You can say \textbf{dash} or \textbf{dash it} or \textbf{dash it all} when you are rather annoyed about something.
 \textit{
	\begin{itemize}
	\item Dash it all. It's just not playing the game, is it?
	\end{itemize}
}
\item countable noun \\
The \textbf{dash} of a car is its dashboard .
 \textit{
	\begin{itemize}
	\end{itemize}
}
\item uncountable noun \\
\textbf{Dash} is a mixture of stylishness, enthusiasm , and courage .
 \textit{
	\begin{itemize}
	\item The Prince was driving with great fire and dash.
	\end{itemize}
}
\item  \\
 cut a dash \textit{
	\begin{itemize}
	\end{itemize}
}
\item  \\
 make a dash \textit{
	\begin{itemize}
	\end{itemize}
}
\end{enumerate}

\section*{goal}
{\large \color{blue}  goals  }
\subsection*{Explain}
\begin{enumerate}
\item countable noun \\
In games such as football , netball or hockey, the \textbf{goal} is the space into which the players try to get the ball in order to score a point for their team .
 \textit{
	\begin{itemize}
	\item The keeper was back in goal after breaking a knuckle.
	\end{itemize}
}
\item countable noun \\
In games such as football or hockey, a \textbf{goal} is when a player gets the ball into the goal, or the point that is scored by doing
this.
 \textit{
	\begin{itemize}
	\item They scored five goals in the first half of the match.
	\item The scorer of the winning goal.
	\end{itemize}
}
\item countable noun \\
Something that is your \textbf{goal} is something that you hope to achieve , especially when much time and effort  will be needed .
 \textit{
	\begin{itemize}
	\item It's a matter of setting your own goals and following them.
	\item The goal is to raise as much money as possible.
	\end{itemize}
}
\end{enumerate}

\section*{debate}
{\large \color{blue}  debates  debating  debated  }
\subsection*{Explain}
\begin{enumerate}
\item variable noun \\
A \textbf{debate} is a discussion about a subject on which people have different  views .
 \textit{
	\begin{itemize}
	\item An intense debate is going on within the Israeli government.
	\item There has been a lot of debate among scholars about this.
	\end{itemize}
}
\item countable noun \\
A \textbf{debate} is a formal discussion, for example in a parliament or institution , in which people express different opinions about a particular subject and then vote on it.
 \textit{
	\begin{itemize}
	\item There are expected to be some heated debates in parliament over the next few days.
	\end{itemize}
}
\item verb \\
If people \textbf{debate} a topic , they discuss it fairly formally, putting forward different views. You can  also  say that one person \textbf{debates} a topic \textbf{with} another person.
 \textit{
	\begin{itemize}
	\item The United Nations Security Council will debate the issue today.
	\item The causes of anorexia are much debated.
	\item Scholars have debated whether or not Yagenta became a convert.
	\item He likes to debate issues with his friends.
	\end{itemize}
}
\item verb \\
If you \textbf{debate} whether to do something or what to do, you think or talk about possible  courses of action before deciding  exactly what you are going to do.
 \textit{
	\begin{itemize}
	\item Taggart debated whether to have yet another coffee.
	\item At the moment we are debating what furniture to buy for the house.
	\item I debated going back inside, but decided against it.
	\end{itemize}
}
\item  \\
 open to debate \textit{
	\begin{itemize}
	\end{itemize}
}
\end{enumerate}

\section*{hobby}
{\large \color{blue}  hobbies  }
\subsection*{Explain}
\begin{enumerate}
\item countable noun \\
A \textbf{hobby} is an activity that you enjoy doing in your spare time.
 \textit{
	\begin{itemize}
	\item My hobbies are letter writing, football, music, photography, and tennis.
	\end{itemize}
}
\end{enumerate}

\section*{discharge}
{\large \color{blue}  discharges  discharging  discharged  }
\subsection*{Explain}
\begin{enumerate}
\item verb \\
When someone \textbf{is discharged}  \textbf{from}  hospital , prison , or one of the armed services, they are officially allowed to leave, or told that they must leave.
 \textbf{Discharge} is also a noun .
 \textit{
	\begin{itemize}
	\item He has a broken nose but may be discharged today.
	\item You are being discharged on medical grounds.
	\item Five days later Henry discharged himself from hospital.
	\item He was given a conditional discharge and ordered to pay compensation.
	\end{itemize}
}
\item verb \\
If someone \textbf{discharges} their duties or responsibilities, they do everything that needs to be done in order to complete them.
 \textit{
	\begin{itemize}
	\item ...the quiet competence with which he discharged his many college duties.
	\end{itemize}
}
\item verb \\
If someone \textbf{discharges} a debt, they pay it.
 \textit{
	\begin{itemize}
	\item The goods will be sold for a fraction of their value in order to discharge the debt.
	\end{itemize}
}
\item verb \\
If something \textbf{is discharged} from inside a place, it comes out.
 \textit{
	\begin{itemize}
	\item The resulting salty water will be discharged at sea.
	\item The bird had trouble breathing and was discharging blood from the nostrils.
	\end{itemize}
}
\item variable noun \\
When there is a \textbf{discharge} of a substance, the substance comes out from inside somewhere .
 \textit{
	\begin{itemize}
	\item They develop a fever and a watery discharge from their eyes.
	\item All discharges and disposals of radioactive waste from Springfields were within relevant
limits.
	\end{itemize}
}
\item verb \\
If someone \textbf{discharges} a gun, they fire it.
 \textit{
	\begin{itemize}
	\item Lewis was tried for unlawfully and dangerously discharging a weapon.
	\end{itemize}
}
\end{enumerate}

\section*{instinct}
{\large \color{blue}  instincts  }
\subsection*{Explain}
\begin{enumerate}
\item variable noun \\
\textbf{Instinct} is the natural tendency that a person or animal has to behave or react in a particular way.
 \textit{
	\begin{itemize}
	\item I didn't have as strong a maternal instinct as some other mothers.
	\item The basis for training relies on the dog's natural instinct to hunt and retrieve.
	\item He always knew what time it was, as if by instinct.
	\end{itemize}
}
\item countable noun \\
If you have an \textbf{instinct}  \textbf{for} something, you are naturally good at it or able to do it.
 \textit{
	\begin{itemize}
	\item Farmers are increasingly losing touch with their instinct for managing the land.
	\item Irene is so incredibly musical and has a natural instinct to perform.
	\end{itemize}
}
\item variable noun \\
If it is your \textbf{instinct}  \textbf{to} do something, you feel that it is right to do it.
 \textit{
	\begin{itemize}
	\item I should've gone with my first instinct, which was not to do the interview.
	\end{itemize}
}
\item variable noun \\
\textbf{Instinct} is a feeling that you have that something is the case , rather than an opinion or idea based on facts .
 \textit{
	\begin{itemize}
	\item There is scientific evidence to support our instinct that being surrounded by plants
is good for health.
	\item He seems so honest and genuine and my every instinct says he's not.
	\end{itemize}
}
\end{enumerate}

\section*{display}
{\large \color{blue}  displays  displaying  displayed  }
\subsection*{Explain}
\begin{enumerate}
\item verb \\
If you \textbf{display} something that you want people to see , you put it in a particular place, so that people can see it easily .
 \textbf{Display} is also a noun .
 \textit{
	\begin{itemize}
	\item Among the war veterans proudly displaying their medals was Aubrey Rose.
	\item The cabinets display seventeenth-century blue-and-white porcelain.
	\item Most of the other artists whose work is on display were his pupils or colleagues.
	\end{itemize}
}
\item verb \\
If you \textbf{display} something, you show it to people.
 \textit{
	\begin{itemize}
	\item He displayed his scars to the twelve members of the jury.
	\item The chart can then display the links connecting these groups.
	\end{itemize}
}
\item verb \\
If you \textbf{display} a characteristic, quality, or emotion , you behave in a way which shows that you have it.
 \textbf{Display} is also a noun.
 \textit{
	\begin{itemize}
	\item It was unlike Gordon to display his feelings.
	\item He has displayed remarkable courage in his efforts to reform the party.
	\item Normally, such an outward display of affection is reserved for his mother.
	\end{itemize}
}
\item verb \\
When a computer \textbf{displays} information, it shows it on a screen.
 \textit{
	\begin{itemize}
	\item They started out by looking at the computer screens which display the images.
	\item Using the option to display only text speeds things up a lot.
	\end{itemize}
}
\item countable noun \\
A \textbf{display} is an arrangement of things that have been put in a particular place, so that people
can see them easily.
 \textit{
	\begin{itemize}
	\item ...a display of your work.
	\item She was leaning against a display case of prints of Paris.
	\end{itemize}
}
\item countable noun \\
A \textbf{display} is a public performance or other event which is intended to entertain people.
 \textit{
	\begin{itemize}
	\item ...the firework display.
	\item ...gymnastic displays.
	\item ...the Royal Air Force Red Arrows display team.
	\end{itemize}
}
\item countable noun \\
The \textbf{display} on a computer screen is the information that is shown there. The screen itself can
also be referred to as the \textbf{display} .
 \textit{
	\begin{itemize}
	\item A hard copy of the screen display can also be obtained from a printer.
	\item ...obscure error messages appearing on the display.
	\end{itemize}
}
\end{enumerate}

\section*{left}
{\large \color{blue}  }
\subsection*{Explain}
\begin{enumerate}
\item  \\
\textbf{Left} is the past  tense and past participle of leave .
 \textit{
	\begin{itemize}
	\end{itemize}
}
\item adjective \\
If there is a certain amount of something \textbf{left} , or if you have a certain amount of it \textbf{left} , it remains when the rest has gone or been used.
 \textit{
	\begin{itemize}
	\item Is there any gin left?
	\item He's got plenty of money left.
	\item They still have six games left to play.
	\end{itemize}
}
\end{enumerate}

\section*{echo}
{\large \color{blue}  echoes  echoing  echoed  }
\subsection*{Explain}
\begin{enumerate}
\item countable noun \\
An \textbf{echo} is a sound which is caused by a noise being reflected off a surface such as a wall .
 \textit{
	\begin{itemize}
	\item He listened and heard nothing but the echoes of his own voice in the cave.
	\end{itemize}
}
\item verb \\
If a sound \textbf{echoes} , it is reflected off a surface and can be heard again after the original sound has stopped.
 \textit{
	\begin{itemize}
	\item His feet echoed on the bare board floor.
	\item The bang came suddenly, echoing across the buildings, shattering glass.
	\end{itemize}
}
\item verb \\
In a place that \textbf{echoes} , a sound is reflected off a surface, and is repeated after the original sound has
stopped.
 \textit{
	\begin{itemize}
	\item The room echoed.
	\item The corridor echoed with the barking of a dozen dogs.
	\item ...the bare stone floors and the echoing hall.
	\end{itemize}
}
\item verb \\
If you \textbf{echo} someone's words, you repeat them or express agreement with their attitude or opinion.
 \textit{
	\begin{itemize}
	\item Many phrases in the last two chapters echo earlier passages.
	\item Their views often echo each other.
	\item 'That was a truly delicious piece of lamb,' he said. 'Yes, wasn't it?' echoed Penelope.
	\end{itemize}
}
\item countable noun \\
An \textbf{echo} is an expression of an attitude, opinion, or statement which has already been expressed.
 \textit{
	\begin{itemize}
	\item I hear an echo of the thinking that got us into this mess in the first place.
	\item Political attacks work only if they find an echo with voters.
	\end{itemize}
}
\item countable noun \\
A detail or feature which reminds you of something else can be referred to as an \textbf{echo} .
 \textit{
	\begin{itemize}
	\item The accident has echoes of past disasters.
	\end{itemize}
}
\item verb \\
If one thing \textbf{echoes} another, the first is a copy of a particular detail or feature of the other.
 \textit{
	\begin{itemize}
	\item Pinks and beiges were chosen to echo the colours of the ceiling.
	\end{itemize}
}
\item verb \\
If something \textbf{echoes} , it continues to be discussed and remains important or influential in a particular situation or among a particular group of people.
 \textit{
	\begin{itemize}
	\item The old fable continues to echo down the centuries.
	\end{itemize}
}
\end{enumerate}

\section*{likelihood}
{\large \color{blue}  }
\subsection*{Explain}
\begin{enumerate}
\item uncountable noun \\
The \textbf{likelihood}  \textbf{of} something happening is how likely it is to happen .
 \textit{
	\begin{itemize}
	\item The likelihood of infection is minimal.
	\item There didn't seem much likelihood of it happening.
	\item There is every likelihood that sanctions will work.
	\end{itemize}
}
\item singular noun \\
If something is a \textbf{likelihood} , it is likely to happen.
 \textit{
	\begin{itemize}
	\item The likelihood is that your child will not develop diabetes.
	\item That, as we all know, is not only a possibility but a likelihood.
	\end{itemize}
}
\item  \\
 in all likelihood \textit{
	\begin{itemize}
	\end{itemize}
}
\end{enumerate}

\section*{estimate}
{\large \color{blue}  estimates  estimating  estimated  }
\subsection*{Explain}
\begin{enumerate}
\item verb \\
If you \textbf{estimate} a quantity or value, you make an approximate judgment or calculation of it.
 \textit{
	\begin{itemize}
	\item Try to estimate how many steps it will take to get to a close object.
	\item I estimate that the total cost for treatment will be $12,500.
	\item He estimated the speed of the winds from the degree of damage.
	\item Some analysts estimate its current popularity at around ten per cent.
	\item His personal riches were estimated at £368 million.
	\end{itemize}
}
\item countable noun \\
An \textbf{estimate} is an approximate calculation of a quantity or value.
 \textit{
	\begin{itemize}
	\item ...the official estimate of the election result.
	\item This figure is five times the original estimate.
	\item A recent estimate was that factories were undermanned by about 30 per cent.
	\end{itemize}
}
\item countable noun \\
An \textbf{estimate} is a judgment about a person or situation which you make based on the available  evidence .
 \textit{
	\begin{itemize}
	\item I hadn't been far wrong in my estimate of his grandson's capabilities.
	\end{itemize}
}
\item countable noun \\
An \textbf{estimate} from someone who you employ to do a job for you, such as a builder or a plumber , is a written statement of how much the job is likely to cost.
 \textit{
	\begin{itemize}
	\item Estimates for curtain-making can be prepared by computer on the spot.
	\end{itemize}
}
\end{enumerate}

\section*{lorry}
{\large \color{blue}  lorries  }
\subsection*{Explain}
\begin{enumerate}
\item countable noun \\
A \textbf{lorry} is a large vehicle that is used to transport goods by road .
 \textit{
	\begin{itemize}
	\item ...a seven-ton lorry.
	\end{itemize}
}
\item  \\
 off the back of a lorry \textit{
	\begin{itemize}
	\end{itemize}
}
\end{enumerate}

\section*{extract}
{\large \color{blue}  extracts  extracting  extracted  }
\subsection*{Explain}
\begin{enumerate}
\item verb \\
To \textbf{extract} a substance means to obtain it from something else, for example by using industrial or chemical processes.
 \textit{
	\begin{itemize}
	\item ...the traditional method of pick and shovel to extract coal.
	\item Citric acid can be extracted from the juice of oranges, lemons, limes or grapefruit.
	\item ...looking at the differences in the extracted DNA.
	\end{itemize}
}
\item verb \\
If you \textbf{extract} something \textbf{from} a place, you take it out or pull it out.
 \textit{
	\begin{itemize}
	\item He extracted a small notebook from his hip pocket.
	\item Patterson went straight to the liquor cabinet and extracted a bottle of Scotch.
	\item She reached into the wardrobe and extracted another tracksuit.
	\end{itemize}
}
\item verb \\
When a dentist  \textbf{extracts} a tooth , they remove it from the patient's mouth .
 \textit{
	\begin{itemize}
	\item A dentist may decide to extract the tooth to prevent recurrent trouble.
	\item She is to go and have a tooth extracted at 3 o'clock today.
	\end{itemize}
}
\item verb \\
If you say that someone \textbf{extracts} something, you disapprove of them because they take it for themselves to gain an advantage .
 \textit{
	\begin{itemize}
	\item The capitalist system extracts huge profits from arms production at the tax-payers'
expense.
	\item He sought to extract the maximum political advantage from the cut in interest rates.
	\item His development policies have extracted cash from the city centre.
	\end{itemize}
}
\item verb \\
If you \textbf{extract} information or a response  \textbf{from} someone, you get it from them with difficulty , because they are unwilling to say or do what you want .
 \textit{
	\begin{itemize}
	\item He made the mistake of trying to extract further information from our director.
	\item He used her cash card, and the PIN number he had extracted from her, to take £500
from cashpoints.
	\end{itemize}
}
\item verb \\
If you \textbf{extract} a particular  piece of information, you obtain it from a larger amount or source of information.
 \textit{
	\begin{itemize}
	\item I've simply extracted a few figures.
	\item Trade figures can be extracted from export-and-import documentation at ports.
	\item ...files of data extracted from the departmental archives.
	\end{itemize}
}
\item passive verb \\
If part of a book or text  \textbf{is extracted}  \textbf{from} a particular book, it is printed or published .
 \textit{
	\begin{itemize}
	\item This material has been extracted from 'Collins Good Wood Handbook'.
	\end{itemize}
}
\item countable noun \\
An \textbf{extract}  \textbf{from} a book or piece of writing is a small part of it that is printed or published separately.
 \textit{
	\begin{itemize}
	\item Read this extract from an information booklet about the work of an airline cabin
crew.
	\end{itemize}
}
\item variable noun \\
An \textbf{extract} is a substance that has been obtained from something else, for example by means of
a chemical or industrial process.
 \textit{
	\begin{itemize}
	\item Blend in the lemon extract, lemon peel and walnuts.
	\item ...fragrances taken from plant extracts.
	\end{itemize}
}
\end{enumerate}

\section*{north}
{\large \color{blue}  }
\subsection*{Explain}
\begin{enumerate}
\item uncountable noun \\
The \textbf{north} is the direction which is on your left when you are looking towards the direction where the sun  rises .
 \textit{
	\begin{itemize}
	\item In the north the ground becomes very cold as the winter snow and ice covers the ground.
	\item Birds usually migrate from north to south.
	\end{itemize}
}
\item singular noun \\
\textbf{The}  \textbf{north} of a place, country, or region is the part which is in the north.
 \textit{
	\begin{itemize}
	\item The scheme mostly benefits people in the North and Midlands.
	\item ...a tiny house in a village in the north of France.
	\end{itemize}
}
\item adverb \\
If you go \textbf{north} , you travel towards the north.
 \textit{
	\begin{itemize}
	\item Anita drove north up Pacific Highway.
	\end{itemize}
}
\item adverb \\
Something that is \textbf{north}  \textbf{of} a place is positioned to the north of it.
 \textit{
	\begin{itemize}
	\item ...a little village a few miles north of Portsmouth.
	\end{itemize}
}
\item adjective \\
The \textbf{north} edge, corner , or part of a place or country is the part which is towards the north.
 \textit{
	\begin{itemize}
	\item ...the north side of the mountain.
	\item They were coming in to land on the north coast of Crete.
	\end{itemize}
}
\item adjective \\
' \textbf{North} ' is used in the names of some countries, states, and regions in the north of a larger
area.
 \textit{
	\begin{itemize}
	\item There were demonstrations this weekend in cities throughout North America, Asia and
Europe.
	\end{itemize}
}
\item adjective \\
A \textbf{north} wind is a wind that blows from the north.
 \textit{
	\begin{itemize}
	\item ...a bitterly cold north wind.
	\end{itemize}
}
\item singular noun \\
\textbf{The North} is used to refer to the richer , more developed countries of the world.
 \textit{
	\begin{itemize}
	\item Developing countries are critical of the North's environmental attitudes.
	\end{itemize}
}
\end{enumerate}

\section*{flare}
{\large \color{blue}  flares  flaring  flared  }
\subsection*{Explain}
\begin{enumerate}
\item countable noun \\
A \textbf{flare} is a small device that produces a bright flame. Flares are used as signals, for example on ships.
 \textit{
	\begin{itemize}
	\item ...a ship which had fired a distress flare.
	\end{itemize}
}
\item verb \\
If a fire \textbf{flares} , the flames suddenly become larger.
 \textbf{Flare up} means the same as flare .
 \textit{
	\begin{itemize}
	\item Camp fires flared like beacons in the dark.
	\item Don't spill too much fat on the barbecue as it could flare up.
	\end{itemize}
}
\item verb \\
If something such as trouble , violence , or conflict  \textbf{flares} , it starts or becomes more violent .
 \textbf{Flare up} means the same as flare .
 \textit{
	\begin{itemize}
	\item Even as the President appealed for calm, trouble flared in several American cities.
	\item Dozens of people were injured as fighting flared up.
	\end{itemize}
}
\item verb \\
If people's tempers \textbf{flare} , they get  angry .
 \textit{
	\begin{itemize}
	\item Tempers flared and harsh words were exchanged.
	\end{itemize}
}
\item verb \\
If someone's nostrils  \textbf{flare} or if they \textbf{flare} them, their nostrils become wider, often because the person is angry or upset .
 \textit{
	\begin{itemize}
	\item I turned to Jacky, my nostrils flaring in disgust.
	\item He stuck out his tongue and flared his nostrils.
	\end{itemize}
}
\item verb \\
If something such as a dress  \textbf{flares} , it spreads outwards at one end to form a wide shape.
 \textit{
	\begin{itemize}
	\item ...a simple black dress, cut to flare from the hips.
	\end{itemize}
}
\item plural noun \\
\textbf{Flares} are trousers that are very wide at the bottom .
 \textit{
	\begin{itemize}
	\end{itemize}
}
\end{enumerate}

\section*{northeast}
{\large \color{blue}  }
\subsection*{Explain}
\begin{enumerate}
\item noun \\
1.  2.  \textit{
	\begin{itemize}
	\end{itemize}
}
\item adjective \\
3.  4.  5.  \textit{
	\begin{itemize}
	\item northeast Lincolnshire
	\end{itemize}
}
\item adverb \\
6.  \textit{
	\begin{itemize}
	\end{itemize}
}
\end{enumerate}

\section*{foam}
{\large \color{blue}  foams  foaming  foamed  }
\subsection*{Explain}
\begin{enumerate}
\item uncountable noun \\
\textbf{Foam} consists of a mass of small bubbles that are formed when air and a liquid are mixed  together .
 \textit{
	\begin{itemize}
	\item The water curved round the rocks in great bursts of foam.
	\end{itemize}
}
\item variable noun \\
\textbf{Foam} is used to refer to various kinds of manufactured  products which have a soft , light texture  like a thick liquid.
 \textit{
	\begin{itemize}
	\item ...shaving foam.
	\end{itemize}
}
\item variable noun \\
\textbf{Foam} or \textbf{foam rubber} is soft rubber full of small holes which is used, for example , to make mattresses and cushions .
 \textit{
	\begin{itemize}
	\item ...modern three-piece suites filled with foam rubber.
	\item We had given him a large foam mattress to sleep on.
	\end{itemize}
}
\item verb \\
If a liquid \textbf{foams} , it is full of small bubbles and keeps moving slightly .
 \textit{
	\begin{itemize}
	\item I let the water run into it and we watched as it foamed and bubbled.
	\item ...ravines with foaming rivers rushing through them.
	\end{itemize}
}
\item  \\
 foam at the mouth \textit{
	\begin{itemize}
	\end{itemize}
}
\end{enumerate}

\section*{objective}
{\large \color{blue}  objectives  }
\subsection*{Explain}
\begin{enumerate}
\item countable noun \\
Your \textbf{objective} is what you are trying to achieve .
 \textit{
	\begin{itemize}
	\item Our main objective was the recovery of the child safe and well.
	\item His objective was to play golf and win.
	\end{itemize}
}
\item adjective \\
\textbf{Objective} information is based on facts .
 \textit{
	\begin{itemize}
	\item He had no objective evidence that anything extraordinary was happening.
	\end{itemize}
}
\item adjective \\
If someone is \textbf{objective,} they base their opinions on facts rather than on their personal feelings.
 \textit{
	\begin{itemize}
	\item I believe that a journalist should be completely objective.
	\item I would really like to have your objective opinion on this.
	\end{itemize}
}
\end{enumerate}

\section*{gallop}
{\large \color{blue}  gallops  galloping  galloped  }
\subsection*{Explain}
\begin{enumerate}
\item verb \\
When a horse \textbf{gallops} , it runs very fast so that all four legs are off the ground at the same time. If
you \textbf{gallop} a horse, you make it gallop.
 \textit{
	\begin{itemize}
	\item The horses galloped away.
	\item Staff officers galloped fine horses down the road.
	\end{itemize}
}
\item verb \\
If you \textbf{gallop} , you ride a horse that is galloping.
 \textit{
	\begin{itemize}
	\item Major Winston galloped into the distance.
	\end{itemize}
}
\item singular noun \\
A \textbf{gallop} is a ride on a horse that is galloping.
 \textit{
	\begin{itemize}
	\item I was forced to attempt a gallop.
	\end{itemize}
}
\item verb \\
If something such as a process \textbf{gallops} , it develops very quickly and is often difficult to control .
 \textit{
	\begin{itemize}
	\item In spite of the recession, profits have galloped ahead.
	\item ...galloping inflation.
	\end{itemize}
}
\item verb \\
If you \textbf{gallop} , you run somewhere very quickly.
 \textit{
	\begin{itemize}
	\item They are galloping around the garden playing football.
	\end{itemize}
}
\item  \\
 at a gallop \textit{
	\begin{itemize}
	\end{itemize}
}
\end{enumerate}

\section*{occupation}
{\large \color{blue}  occupations  }
\subsection*{Explain}
\begin{enumerate}
\item countable noun \\
Your \textbf{occupation} is your job or profession.
 \textit{
	\begin{itemize}
	\item I suppose I was looking for an occupation which was going to be an adventure.
	\item Occupation: administrative assistant.
	\end{itemize}
}
\item countable noun \\
An \textbf{occupation} is something that you spend time doing, either for pleasure or because it needs to be done .
 \textit{
	\begin{itemize}
	\item Parachuting is a dangerous occupation.
	\end{itemize}
}
\item uncountable noun \\
The \textbf{occupation} of a country happens when it is entered and controlled by a foreign army .
 \textit{
	\begin{itemize}
	\item Prii had become fluent in German during the Wehrmacht's occupation of Estonia in
1942.
	\item ...the deportation of Jews from Paris during the German occupation.
	\end{itemize}
}
\item uncountable noun \\
The \textbf{occupation} of a building is the act or fact of someone living or working in it.
 \textit{
	\begin{itemize}
	\item ...people who sell their home and buy another one for their own occupation.
	\end{itemize}
}
\end{enumerate}

\section*{grasp}
{\large \color{blue}  grasps  grasping  grasped  }
\subsection*{Explain}
\begin{enumerate}
\item verb \\
If you \textbf{grasp} something, you take it in your hand and hold it very firmly.
 \textit{
	\begin{itemize}
	\item He grasped both my hands.
	\item She was trying to grasp at something.
	\end{itemize}
}
\item singular noun \\
A \textbf{grasp} is a very firm hold or grip.
 \textit{
	\begin{itemize}
	\item His hand was taken in a warm, firm grasp.
	\end{itemize}
}
\item singular noun \\
If you say that something is \textbf{in} someone's \textbf{grasp} , you disapprove of the fact that they possess or control it. If something slips  \textbf{from} your \textbf{grasp} , you lose it or lose control of it.
 \textit{
	\begin{itemize}
	\item The people in your grasp are not guests, they are hostages.
	\item She allowed victory to slip from her grasp.
	\item ...the task of liberating a number of states from the grasp of tyrants.
	\end{itemize}
}
\item verb \\
If you \textbf{grasp} something that is complicated or difficult to understand, you understand it.
 \textit{
	\begin{itemize}
	\item The Government has not yet grasped the seriousness of the crisis.
	\item He instantly grasped that Stephen was talking about his wife.
	\end{itemize}
}
\item singular noun \\
A \textbf{grasp}  \textbf{of} something is an understanding of it.
 \textit{
	\begin{itemize}
	\item They have a good grasp of foreign languages.
	\end{itemize}
}
\item  \\
 within someone's grasp \textit{
	\begin{itemize}
	\end{itemize}
}
\end{enumerate}

\section*{orient}
{\large \color{blue}  orients  orienting  oriented  }
\subsection*{Explain}
\begin{enumerate}
\item verb \\
When you \textbf{orient}  \textbf{yourself to} a new situation or course of action, you learn about it and prepare to deal with it.
 \textit{
	\begin{itemize}
	\item You will need the time to orient yourself to your new way of eating.
	\item ...orienting students to new ways of thinking about their participation in classroom
learning.
	\item Anxiety comes from not being able to orient yourself in your own existence.
	\end{itemize}
}
\item verb \\
When you \textbf{orient}  \textbf{yourself} , you find out exactly where you are and which direction you are facing in.
 \textit{
	\begin{itemize}
	\item She lay still for a few seconds, trying to orient herself.
	\end{itemize}
}
\end{enumerate}

\section*{groan}
{\large \color{blue}  groans  groaning  groaned  }
\subsection*{Explain}
\begin{enumerate}
\item verb \\
If you \textbf{groan} , you make a long, low sound because you are in pain, or because you are upset or unhappy about something.
 \textbf{Groan} is also a noun .
 \textit{
	\begin{itemize}
	\item Slowly, he opened his eyes. As he did so, he began to groan with pain.
	\item They glanced at the man on the floor, who began to groan.
	\item She was making small groaning noises.
	\item She heard him let out a pitiful, muffled groan.
	\item As his ball flew wide, there was a collective groan from the stands.
	\end{itemize}
}
\item verb \\
If you \textbf{groan} something, you say it in a low, unhappy voice .
 \textit{
	\begin{itemize}
	\item 'My leg–I think it's broken,' Eric groaned.
	\end{itemize}
}
\item verb \\
If you \textbf{groan}  \textbf{about} something, you complain about it.
 \textbf{Groan} is also a noun.
 \textit{
	\begin{itemize}
	\item His parents were beginning to groan about the price of college tuition.
	\item Listen sympathetically to your child's moans and groans about what she can't do.
	\end{itemize}
}
\item verb \\
If wood or something made of wood \textbf{groans} , it makes a loud sound when it moves .
 \textit{
	\begin{itemize}
	\item The timbers groan and creak and the floorboards shift.
	\end{itemize}
}
\item verb \\
If you say that something such as a table  \textbf{groans}  \textbf{under} the weight of food , you are emphasizing that there is a lot of food on it.
 \textit{
	\begin{itemize}
	\item The bar counter groans under the weight of huge plates of the freshest fish.
	\item ...a table groaning with food.
	\end{itemize}
}
\item verb \\
If you say that someone or something \textbf{is groaning under} the weight of something, you think there is too much of that thing.
 \textit{
	\begin{itemize}
	\item Consumers were groaning under the weight of high interest rates.
	\item Bookshelves groan under the burden of books on threats to the environment.
	\end{itemize}
}
\end{enumerate}

\section*{guard}
{\large \color{blue}  guards  guarding  guarded  }
\subsection*{Explain}
\begin{enumerate}
\item verb \\
If you \textbf{guard} a place, person, or object, you stand near them in order to watch and protect them.
 \textit{
	\begin{itemize}
	\item Gunmen guarded homes near the cemetery with shotguns.
	\item ...the heavily guarded courtroom.
	\end{itemize}
}
\item verb \\
If you \textbf{guard} someone, you watch them and keep them in a particular place to stop them from escaping.
 \textit{
	\begin{itemize}
	\item Marines with rifles guarded them.
	\item He is being guarded by a platoon of police.
	\end{itemize}
}
\item countable noun \\
A \textbf{guard} is someone such as a soldier, police officer, or prison officer who is guarding a particular place or person.
 \textit{
	\begin{itemize}
	\item The prisoners overpowered their guards and locked them in a cell.
	\end{itemize}
}
\item singular noun \\
A \textbf{guard} is a specially organized group of people, such as soldiers or police officers, who protect or watch someone
or something.
 \textit{
	\begin{itemize}
	\item We have a security guard around the whole area.
	\item A heavily armed guard of police have sealed off the city centre.
	\end{itemize}
}
\item countable noun \\
On a train, a \textbf{guard} is a person whose job is to travel on the train in order to help  passengers , check  tickets , and make sure that the train travels safely and on time.
 \textit{
	\begin{itemize}
	\end{itemize}
}
\item verb \\
If you \textbf{guard} some information or advantage that you have, you try to protect it or keep it for yourself.
 \textit{
	\begin{itemize}
	\item He closely guarded her identity.
	\item ...a threat to the country's jealously guarded unity.
	\end{itemize}
}
\item countable noun \\
A \textbf{guard} is a protective device which covers a part of someone's body or a dangerous part of a piece of equipment.
 \textit{
	\begin{itemize}
	\item ...the chin guard of my helmet.
	\item A blade guard is fitted to protect the operator.
	\end{itemize}
}
\item countable noun \\
Some regiments in the British Army , or the soldiers in them, are referred to as \textbf{Guards} .
 \textit{
	\begin{itemize}
	\item ...the Grenadier Guards.
	\end{itemize}
}
\item  \\
 catch someone off guard \textit{
	\begin{itemize}
	\end{itemize}
}
\item  \\
 lower/drop your guard, let your guard down \textit{
	\begin{itemize}
	\end{itemize}
}
\item  \\
 mount guard \textit{
	\begin{itemize}
	\end{itemize}
}
\item  \\
 on your guard \textit{
	\begin{itemize}
	\end{itemize}
}
\item  \\
 on guard \textit{
	\begin{itemize}
	\end{itemize}
}
\item  \\
 stand guard \textit{
	\begin{itemize}
	\end{itemize}
}
\item  \\
 under guard \textit{
	\begin{itemize}
	\end{itemize}
}
\end{enumerate}

\section*{paperback}
{\large \color{blue}  paperbacks  }
\subsection*{Explain}
\begin{enumerate}
\item countable noun \\
A \textbf{paperback} is a book with a thin  cardboard or paper cover. Compare  hardback .
 \textit{
	\begin{itemize}
	\item She said she would buy the book when it comes out in paperback.
	\end{itemize}
}
\end{enumerate}

\section*{import}
{\large \color{blue}  imports  importing  imported  }
\subsection*{Explain}
\begin{enumerate}
\item verb \\
To \textbf{import}  products or raw materials means to buy them from another country for use in your own country.
 \textbf{Import} is also a noun .
 \textit{
	\begin{itemize}
	\item Britain last year spent nearly £5000 million more on importing food than selling
abroad.
	\item ...oil, soy and other products it plans to import from Brazil.
	\item ...imported goods from Mexico and India.
	\item There were new restrictions on the import of sweet chestnut trees.
	\item On July 3rd the government slashed import duties on cars.
	\end{itemize}
}
\item countable noun \\
\textbf{Imports} are products or raw materials bought from another country for use in your own country.
 \textit{
	\begin{itemize}
	\item ...farmers protesting about cheap imports.
	\end{itemize}
}
\item uncountable noun \\
The \textbf{import} of something is its importance.
 \textit{
	\begin{itemize}
	\item Who leads Canada is also of some import to the rest of the world.
	\item Such arguments are of little import.
	\end{itemize}
}
\item verb \\
If you \textbf{import}  files or information into one type of software from another type, you open them in a format that can be used in the new software.
 \textit{
	\begin{itemize}
	\item This device will enable you to import these files onto your computer.
	\end{itemize}
}
\item singular noun \\
The \textbf{import} of something is its meaning, especially when the meaning is not clearly  expressed .
 \textit{
	\begin{itemize}
	\item I have already spoken about the import of his speech.
	\end{itemize}
}
\end{enumerate}

\section*{petroleum}
{\large \color{blue}  }
\subsection*{Explain}
\begin{enumerate}
\item uncountable noun \\
\textbf{Petroleum} is oil which is found under the surface of the Earth or under the sea bed . Petrol and paraffin are obtained from petroleum.
 \textit{
	\begin{itemize}
	\end{itemize}
}
\end{enumerate}

\section*{neglect}
{\large \color{blue}  neglects  neglecting  neglected  }
\subsection*{Explain}
\begin{enumerate}
\item verb \\
If you \textbf{neglect} someone or something, you fail to look after them properly.
 \textbf{Neglect} is also a noun .
 \textit{
	\begin{itemize}
	\item The woman denied that she had neglected her child.
	\item Feed plants and they grow, neglect them and they suffer.
	\item ...an ancient and neglected church.
	\item The town's old quayside is collapsing after years of neglect.
	\item Niwano's business began to suffer from neglect.
	\end{itemize}
}
\item verb \\
If you \textbf{neglect} someone or something, you fail to give them the amount of attention that they deserve .
 \textit{
	\begin{itemize}
	\item He'd given too much to his career, worked long hours, neglected her.
	\item If you are not careful, children tend to neglect their homework.
	\end{itemize}
}
\item verb \\
If you \textbf{neglect}  \textbf{to} do something that you ought to do or \textbf{neglect} your duty , you fail to do it.
 \textit{
	\begin{itemize}
	\item We often neglect to make proper use of our bodies.
	\item They never neglect their duties.
	\end{itemize}
}
\end{enumerate}

\section*{platform}
{\large \color{blue}  platforms  }
\subsection*{Explain}
\begin{enumerate}
\item countable noun \\
A \textbf{platform} is a flat , raised structure, usually made of wood, which people stand on when they make speeches or give a performance .
 \textit{
	\begin{itemize}
	\item Nick finished what he was saying and jumped down from the platform.
	\end{itemize}
}
\item countable noun \\
A \textbf{platform} is a flat raised structure or area, usually one which something can stand on or land
on.
 \textit{
	\begin{itemize}
	\item Some of these flood shelters are on raised platforms, which have allowed helicopters
to land amid the continuing floods.
	\item They found a spot on a rocky platform where they could pitch their tents.
	\end{itemize}
}
\item countable noun \\
A \textbf{platform} is a structure built for people to work and live on when drilling for oil or gas at sea, or when extracting it.
 \textit{
	\begin{itemize}
	\end{itemize}
}
\item countable noun \\
A \textbf{platform} in a railway station is the area beside the rails where you wait for or get off a train.
 \textit{
	\begin{itemize}
	\item The train was about to leave and I was not even on the platform.
	\end{itemize}
}
\item countable noun \\
The \textbf{platform} of a political party is what they say they will do if they are elected .
 \textit{
	\begin{itemize}
	\item ...a platform of political and economic reforms.
	\item The Socialist Party won a landslide victory on a nationalist platform.
	\end{itemize}
}
\item countable noun \\
If someone has a \textbf{platform} , they have an opportunity to tell people what they think or want .
 \textit{
	\begin{itemize}
	\item The demonstration provided a platform for a broad cross-section of speakers.
	\end{itemize}
}
\item singular noun \\
In a bus , \textbf{the}  \textbf{platform} is the area of floor at the front or back where you get on and off.
 \textit{
	\begin{itemize}
	\item I stood on the crowded back platform of the seven o'clock bus.
	\end{itemize}
}
\end{enumerate}

\section*{pat}
{\large \color{blue}  pats  patting  patted  }
\subsection*{Explain}
\begin{enumerate}
\item verb \\
If you \textbf{pat} something or someone, you tap them lightly, usually with your hand held flat.
 \textbf{Pat} is also a noun .
 \textit{
	\begin{itemize}
	\item 'Don't you worry about any of this,' she said patting me on the knee.
	\item The landlady patted her hair nervously.
	\item Wash the lettuce and pat it dry.
	\item He gave her an encouraging pat on the shoulder.
	\end{itemize}
}
\item countable noun \\
A \textbf{pat}  \textbf{of}  butter or something else that is soft is a small lump of it.
 \textit{
	\begin{itemize}
	\end{itemize}
}
\item adjective \\
If you say that an answer or explanation is \textbf{pat} , you disapprove of it because it is too simple and sounds as if it has been prepared in advance .
 \textit{
	\begin{itemize}
	\item There's no pat answer to that.
	\item Despite the film's merits I felt it was too pat.
	\end{itemize}
}
\item  \\
 a pat on the back/pat sb on the back \textit{
	\begin{itemize}
	\end{itemize}
}
\item  \\
 have sth down pat/have sth off pat \textit{
	\begin{itemize}
	\end{itemize}
}
\item  \\
 stand pat \textit{
	\begin{itemize}
	\end{itemize}
}
\end{enumerate}

\section*{possibility}
{\large \color{blue}  possibilities  }
\subsection*{Explain}
\begin{enumerate}
\item countable noun \\
If you say there is a \textbf{possibility}  \textbf{that} something is the case or \textbf{that} something will  happen , you mean that it might be the case or it might happen.
 \textit{
	\begin{itemize}
	\item We were not in the least worried about the possibility that sweets could rot the
teeth.
	\item Tax on food has become a very real possibility.
	\end{itemize}
}
\item countable noun \\
A \textbf{possibility} is one of several different things that could be done.
 \textit{
	\begin{itemize}
	\item One possibility would be to compensate us with other property.
	\item There were several possibilities open to each manufacturer.
	\end{itemize}
}
\item  \\
 not beyond the realms/bounds of possibility \textit{
	\begin{itemize}
	\end{itemize}
}
\end{enumerate}

\section*{progress}
{\large \color{blue}  progresses  progressing  progressed  }
\subsection*{Explain}
\begin{enumerate}
\item uncountable noun \\
\textbf{Progress} is the process of gradually improving or getting nearer to achieving or completing something.
 \textit{
	\begin{itemize}
	\item The medical community continues to make progress in the fight against cancer.
	\item The two sides made little if any progress towards agreement.
	\end{itemize}
}
\item singular noun \\
\textbf{The}  \textbf{progress}  \textbf{of} a situation or action is the way in which it develops.
 \textit{
	\begin{itemize}
	\item The CEO is reported to have been delighted with the progress of the first day's talks.
	\item Ellen would keep me abreast of the progress by phone.
	\end{itemize}
}
\item verb \\
To \textbf{progress} means to move over a period of time to a stronger, more advanced, or more desirable state.
 \textit{
	\begin{itemize}
	\item He will visit once a fortnight to see how his new staff are progressing.
	\item Were you surprised that his disease progressed so quickly?
	\item He started with sketching and then progressed to painting.
	\end{itemize}
}
\item verb \\
If events \textbf{progress} , they continue to happen gradually over a period of time.
 \textit{
	\begin{itemize}
	\item As the evening progressed, sadness turned to rage.
	\item Life was hard, and it became harder as the war progressed.
	\end{itemize}
}
\item verb \\
If you \textbf{progress} something, you cause it to develop.
 \textit{
	\begin{itemize}
	\item Very little was done to progress the case in the first 10 or so months after K was
charged.
	\end{itemize}
}
\item  \\
 in progress \textit{
	\begin{itemize}
	\end{itemize}
}
\end{enumerate}

\section*{probability}
{\large \color{blue}  probabilities  }
\subsection*{Explain}
\begin{enumerate}
\item variable noun \\
The \textbf{probability}  \textbf{of} something happening is how likely it is to happen , sometimes  expressed as a fraction or a percentage .
 \textit{
	\begin{itemize}
	\item Without a transfusion, the victim's probability of dying was 100%.
	\item The probabilities of crime or victimization are higher with some situations than
with others.
	\item You cannot prove conclusively that Sellafield caused cancer. You can only work on
the basis of probability.
	\end{itemize}
}
\item variable noun \\
You say that there is a \textbf{probability} that something will happen when it is likely to happen.
 \textit{
	\begin{itemize}
	\item There's an excellent probability that unless action is quickly taken, pipes will
freeze.
	\item If you've owned property for several years, the probability is that values have increased.
	\item Formal talks are still said to be a possibility, not a probability.
	\item His story-telling can push the bounds of probability a bit far at times.
	\end{itemize}
}
\item  \\
 in all probability \textit{
	\begin{itemize}
	\end{itemize}
}
\end{enumerate}

\section*{protest}
{\large \color{blue}  protests  protesting  protested  }
\subsection*{Explain}
\begin{enumerate}
\item verb \\
If you \textbf{protest}  \textbf{against} something or \textbf{about} something, you say or show publicly that you object to it. In American English, you usually say that you \textbf{protest} it.
 \textit{
	\begin{itemize}
	\item Groups of women took to the streets to protest against the arrests.
	\item The students were protesting at overcrowding in the university hostels.
	\item They were protesting soaring prices.
	\item He picked up the cat before Rosa could protest.
	\end{itemize}
}
\item variable noun \\
A \textbf{protest} is the act of saying or showing publicly that you object to something.
 \textit{
	\begin{itemize}
	\item The opposition now seems too weak to stage any serious protests against the government.
	\item The unions called a two-hour strike in protest at the railway authority's announcement.
	\item ...a protest march.
	\end{itemize}
}
\item verb \\
If you \textbf{protest}  \textbf{that} something is the case, you insist that it is the case, when other people think that it may not be.
 \textit{
	\begin{itemize}
	\item When we tried to protest that Mo was beaten up they didn't believe us.
	\item 'I never said any of that to her,' he protested.
	\item He has always protested his innocence.
	\end{itemize}
}
\item countable noun \\
A \textbf{protest}  \textbf{that} something is true is a strong declaration that it is true.
 \textit{
	\begin{itemize}
	\item That was how she usually dealt with their protests that she was spoiling her grandchildren.
	\end{itemize}
}
\end{enumerate}

\section*{purpose}
{\large \color{blue}  purposes  }
\subsection*{Explain}
\begin{enumerate}
\item countable noun \\
The \textbf{purpose} of something is the reason for which it is made or done.
 \textit{
	\begin{itemize}
	\item The purpose of the occasion was to raise money for medical supplies.
	\item Various insurance schemes already exist for this purpose.
	\item ...the use of nuclear energy for military purposes.
	\item He was asked about casualties, but said it would serve no purpose to count bodies.
	\item Most of them are destroyed because they've served their purpose.
	\end{itemize}
}
\item countable noun \\
Your \textbf{purpose} is the thing that you want to achieve .
 \textit{
	\begin{itemize}
	\item They might well be prepared to do you harm in order to achieve their purpose.
	\item His purpose was to make a profit by improving the company's performance.
	\end{itemize}
}
\item uncountable noun \\
\textbf{Purpose} is the feeling of having a definite  aim and of being determined to achieve it.
 \textit{
	\begin{itemize}
	\item The teachers are enthusiastic and have a sense of purpose.
	\end{itemize}
}
\item  \\
 for all practical purposes/to all intents and purposes \textit{
	\begin{itemize}
	\end{itemize}
}
\item  \\
 on purpose \textit{
	\begin{itemize}
	\end{itemize}
}
\end{enumerate}

\section*{recycle}
{\large \color{blue}  recycles  recycling  recycled  }
\subsection*{Explain}
\begin{enumerate}
\item verb \\
If you \textbf{recycle} things that have already been used, such as bottles or sheets of paper, you process them so that they can be used again.
 \textit{
	\begin{itemize}
	\item The objective would be to recycle 98 per cent of domestic waste.
	\item All glass bottles which can't be refilled can be recycled.
	\item It is printed on recycled paper.
	\end{itemize}
}
\end{enumerate}

\section*{quartz}
{\large \color{blue}  }
\subsection*{Explain}
\begin{enumerate}
\item uncountable noun \\
\textbf{Quartz} is a mineral in the form of a hard, shiny crystal. It is used in making electronic  equipment and very accurate  watches and clocks .
 \textit{
	\begin{itemize}
	\item ...a quartz crystal.
	\end{itemize}
}
\end{enumerate}

\section*{reform}
{\large \color{blue}  reforms  reforming  reformed  }
\subsection*{Explain}
\begin{enumerate}
\item variable noun \\
\textbf{Reform} consists of changes and improvements to a law, social system, or institution. A \textbf{reform} is an instance of such a change or improvement.
 \textit{
	\begin{itemize}
	\item The party embarked on a programme of economic reform.
	\item He has urged reform of the welfare system.
	\item The Socialists introduced fairly radical reforms.
	\end{itemize}
}
\item verb \\
If someone \textbf{reforms} something such as a law, social system, or institution, they change or improve it.
 \textit{
	\begin{itemize}
	\item ...his plans to reform the country's economy.
	\item A reformed party would have to win the approval of the people.
	\end{itemize}
}
\item verb \\
When someone \textbf{reforms} or when something \textbf{reforms} them, they stop doing things that society does not approve of, such as breaking the law or drinking too much alcohol .
 \textit{
	\begin{itemize}
	\item When his court case was coming up, James promised to reform.
	\item We will try to reform him within the community.
	\end{itemize}
}
\end{enumerate}

\section*{refund}
{\large \color{blue}  refunds  refunding  refunded  }
\subsection*{Explain}
\begin{enumerate}
\item countable noun \\
A \textbf{refund} is a sum of money which is returned to you, for example because you have paid too much or because you have returned goods to a shop .
 \textit{
	\begin{itemize}
	\end{itemize}
}
\item verb \\
If someone \textbf{refunds} your money, they return it to you, for example because you have paid too much or
because you have returned goods to a shop.
 \textit{
	\begin{itemize}
	\item We guarantee to refund your money if you're not delighted with your purchase.
	\item Take the goods back to your retailer who will refund you the purchase price.
	\end{itemize}
}
\end{enumerate}

\section*{reproach}
{\large \color{blue}  reproaches  reproaching  reproached  }
\subsection*{Explain}
\begin{enumerate}
\item verb \\
If you \textbf{reproach} someone, you say or show that you are disappointed , upset , or angry because they have done something wrong .
 \textit{
	\begin{itemize}
	\item She is quick to reproach anyone who doesn't live up to her own high standards.
	\item She had not even reproached him for breaking his promise.
	\end{itemize}
}
\item variable noun \\
If you look at or speak to someone with \textbf{reproach} , you show or say that you are disappointed, upset, or angry because they have done
something wrong.
 \textit{
	\begin{itemize}
	\item He looked at her with reproach.
	\item Public servants and political figures must be beyond reproach.
	\end{itemize}
}
\item verb \\
If you \textbf{reproach}  \textbf{yourself} , you think with regret about something you have done wrong.
 \textit{
	\begin{itemize}
	\item You've no reason to reproach yourself, no reason to feel shame.
	\item We begin to reproach ourselves for not having been more careful.
	\end{itemize}
}
\item singular noun \\
If you consider someone's actions or behaviour to be \textbf{a}  \textbf{reproach}  \textbf{to} a group of people, you consider them to be harmful or insulting to that group.
 \textit{
	\begin{itemize}
	\item The shootings and bombings were 'a scandal and reproach to all of us'.
	\end{itemize}
}
\end{enumerate}

\section*{repertoire}
{\large \color{blue}  repertoires  }
\subsection*{Explain}
\begin{enumerate}
\item countable noun \\
A performer's \textbf{repertoire} is all the plays or pieces of music that he or she has learned and can perform.
 \textit{
	\begin{itemize}
	\item Meredith D'Ambrosio has thousands of songs in her repertoire.
	\end{itemize}
}
\item singular noun \\
The \textbf{repertoire} of a person or thing is all the things of a particular kind that the person or thing
is capable of doing.
 \textit{
	\begin{itemize}
	\item ...Mike's impressive repertoire of funny stories.
	\item This has been one of the most successful desserts in my repertoire.
	\end{itemize}
}
\item singular noun \\
You can refer to all the plays or music of a particular kind as, for example , \textbf{the}  classical  \textbf{repertoire} or \textbf{the}  jazz  \textbf{repertoire} .
 \textit{
	\begin{itemize}
	\item It is no coincidence that the works in the 'standard repertoire' tend to have names.
	\end{itemize}
}
\end{enumerate}

\section*{retrospect}
{\large \color{blue}  }
\subsection*{Explain}
\begin{enumerate}
\item  \\
 in retrospect \textit{
	\begin{itemize}
	\end{itemize}
}
\end{enumerate}

\section*{rhythm}
{\large \color{blue}  rhythms  }
\subsection*{Explain}
\begin{enumerate}
\item variable noun \\
A \textbf{rhythm} is a regular series of sounds or movements.
 \textit{
	\begin{itemize}
	\item His music of that period fused the rhythms of Jazz with classical forms.
	\item He had no sense of rhythm whatsoever.
	\item She could hear the constant rhythm of his breathing.
	\end{itemize}
}
\item countable noun \\
A \textbf{rhythm} is a regular pattern of changes, for example changes in your body, in the seasons , or in the tides .
 \textit{
	\begin{itemize}
	\item Begin to listen to your own body rhythms.
	\item ...the seasonal rhythm of the agricultural year.
	\end{itemize}
}
\end{enumerate}

\section*{return}
{\large \color{blue}  returns  returning  returned  }
\subsection*{Explain}
\begin{enumerate}
\item verb \\
When you \textbf{return}  \textbf{to} a place, you go back there after you have been away.
 \textit{
	\begin{itemize}
	\item The Prime Minister will return to London tonight.
	\item Our correspondent has just returned from the camps on the border.
	\item So far more than 350,000 people have returned home.
	\end{itemize}
}
\item singular noun \\
Your \textbf{return} is your arrival back at a place where you had been before.
 \textit{
	\begin{itemize}
	\item Ryle explained the reason for his sudden return to London.
	\end{itemize}
}
\item verb \\
If you \textbf{return} something that you have borrowed or taken, you give it back or put it back.
 \textbf{Return} is also a noun .
 \textit{
	\begin{itemize}
	\item I enjoyed the book and said so when I returned it.
	\item The car was not returned on time, then was reported stolen.
	\item The main demand of the Indians is for the return of one-and-a-half-million acres
of forest to their communities.
	\end{itemize}
}
\item verb \\
If you \textbf{return} something somewhere , you put it back where it was.
 \textit{
	\begin{itemize}
	\item He returned the notebook to his jacket.
	\end{itemize}
}
\item verb \\
If you \textbf{return} someone's action, you do the same thing to them as they have just done to you. If
you \textbf{return} someone's feeling, you feel the same way towards them as they feel towards you.
 \textit{
	\begin{itemize}
	\item Back at the station the Chief Inspector returned the call.
	\item She will be disappointed if her feelings are not returned.
	\end{itemize}
}
\item verb \\
If a feeling or situation \textbf{returns} , it comes back or happens again after a period when it was not present.
 \textbf{Return} is also a noun.
 \textit{
	\begin{itemize}
	\item Calm is returning to the country.
	\item The pain returned in waves.
	\item It was like the return of his youth.
	\end{itemize}
}
\item verb \\
If you \textbf{return}  \textbf{to} a state that you were in before, you start being in that state again.
 \textbf{Return} is also a noun.
 \textit{
	\begin{itemize}
	\item Life has improved and returned to normal.
	\item He made an uneventful return to normal health.
	\item The opposition now fears a return to martial rule.
	\end{itemize}
}
\item verb \\
If you \textbf{return}  \textbf{to} a subject that you have mentioned before, you begin talking about it again.
 \textit{
	\begin{itemize}
	\item The power of the Church is one theme all these writers return to.
	\end{itemize}
}
\item verb \\
If you \textbf{return}  \textbf{to} an activity that you were doing before, you start doing it again.
 \textbf{Return} is also a noun.
 \textit{
	\begin{itemize}
	\item He is 52, young enough to return to politics if he wishes to do so.
	\item He has not ruled out the shock possibility of a return to football.
	\end{itemize}
}
\item verb \\
When a judge or jury \textbf{returns} a verdict, they announce whether they think the person on trial is guilty or not.
 \textit{
	\begin{itemize}
	\item They returned a verdict of not guilty.
	\end{itemize}
}
\item adjective \\
A \textbf{return} ticket is a ticket for a journey from one place to another and then back again.
 \textbf{Return} is also a noun.
 \textit{
	\begin{itemize}
	\item He bought a return ticket and boarded the next train for home.
	\item That's enough Airmiles for two returns to Paris, Amsterdam or Brussels.
	\end{itemize}
}
\item adjective \\
The \textbf{return}  trip or journey is the part of a journey that takes you back to where you started from.
 \textit{
	\begin{itemize}
	\item Buy an extra ticket for the return trip.
	\end{itemize}
}
\item countable noun \\
The \textbf{return}  \textbf{on} an investment is the profit that you get from it.
 \textit{
	\begin{itemize}
	\item Profits have picked up this year but the return on capital remains tiny.
	\item Higher returns and higher risk usually go hand in hand.
	\end{itemize}
}
\item countable noun \\
A tax \textbf{return} is an official form that you fill in with details about your income and personal
situation, so that the income tax you owe can be calculated .
 \textit{
	\begin{itemize}
	\item He was convicted of filing false income tax returns.
	\item Anyone with complications in their tax affairs is asked to fill in a return.
	\end{itemize}
}
\item plural noun \\
\textbf{Returns} are the results of votes after an election.
 \textit{
	\begin{itemize}
	\item Early returns show the opposition party may have won.
	\end{itemize}
}
\item  \\
 many happy returns \textit{
	\begin{itemize}
	\end{itemize}
}
\item  \\
 in return \textit{
	\begin{itemize}
	\end{itemize}
}
\item  \\
 the point of no return \textit{
	\begin{itemize}
	\end{itemize}
}
\end{enumerate}

\section*{show}
{\large \color{blue}  shows  showing  showed  shown  }
\subsection*{Explain}
\begin{enumerate}
\item verb \\
If something \textbf{shows}  \textbf{that} a state of affairs exists, it gives information that proves it or makes it clear to people.
 \textit{
	\begin{itemize}
	\item Research shows that a high-fibre diet may protect you from bowel cancer.
	\item He was arrested at his home in Southampton after a breath test showed he had drunk
more than twice the legal limit for driving.
	\item These figures show an increase of over one million in unemployment.
	\item New airline technology was shown to be improving fuel consumption.
	\item You'll be given regular blood tests to show whether you have been infected.
	\end{itemize}
}
\item verb \\
If a picture, chart , film, or piece of writing \textbf{shows} something, it represents it or gives information about it.
 \textit{
	\begin{itemize}
	\item Figure 4.1 shows the respiratory system.
	\item ...a coin showing Cleopatra.
	\item The cushions, shown left, measure 20 x 12 inches and cost $39.95.
	\item Much of the film shows the painter simply going about his task.
	\item Our photograph shows how the plants will turn out.
	\end{itemize}
}
\item verb \\
If you \textbf{show} someone something, you give it to them, take them to it, or point to it, so that
they can see it or know what you are referring to.
 \textit{
	\begin{itemize}
	\item Cut out this article and show it to your bank manager.
	\item He showed me the flat he shares with Esther.
	\item I showed them where the gun was.
	\item Show me which one you like and I'll buy it for you.
	\end{itemize}
}
\item verb \\
If you \textbf{show} someone to a room or seat, you lead them there.
 \textit{
	\begin{itemize}
	\item Let me show you to my study.
	\item Milton was shown into the office.
	\item John will show you upstairs, Mr Penry.
	\item I'll show you the way.
	\end{itemize}
}
\item verb \\
If you \textbf{show} someone how to do something, you do it yourself so that they can watch you and learn how to do it.
 \textit{
	\begin{itemize}
	\item Claire showed us how to make a chocolate roulade.
	\item There are seasoned professionals who can teach you and show you what to do.
	\item Mother asked me to show you how the phones work.
	\item Dr. Reichert has shown us a new way to look at those behavior problems.
	\end{itemize}
}
\item verb \\
If something \textbf{shows} or if you \textbf{show} it, it is visible or noticeable.
 \textit{
	\begin{itemize}
	\item He showed his teeth in a humourless grin.
	\item His beard was just beginning to show signs of grey.
	\item Faint glimmers of daylight were showing through the treetops.
	\item I'd driven both ways down this road but my tracks didn't show.
	\end{itemize}
}
\item verb \\
If you \textbf{show} a particular attitude , quality, or feeling, or if it \textbf{shows} , you behave in a way that makes this attitude, quality, or feeling clear to other people.
 \textit{
	\begin{itemize}
	\item Elsie has had enough time to show her gratitude.
	\item She showed no interest in her children.
	\item Ferguson was unhappy and it showed.
	\item You show me respect.
	\item Mr Clarke has shown himself to be resolutely opposed to compromise.
	\item The baby was tugging at his coat to show that he wanted to be picked up.
	\end{itemize}
}
\item verb \\
If something \textbf{shows} a quality or characteristic or if that quality or characteristic \textbf{shows}  \textbf{itself} , it can be noticed or observed .
 \textit{
	\begin{itemize}
	\item The story shows a strong narrative gift and a vivid eye for detail.
	\item The peace talks showed signs of progress yesterday.
	\item Her popularity clearly shows no sign of waning.
	\item How else did his hostility to women show itself?
	\end{itemize}
}
\item countable noun \\
\textbf{A}  \textbf{show}  \textbf{of} a feeling or quality is an attempt by someone to make it clear that they have that
feeling or quality.
 \textit{
	\begin{itemize}
	\item Miners gathered in the city centre in a show of support for the government.
	\item A crowd of more than 10,000 has gathered in a show of strength.
	\item She said goodbye to Hilda with a convincing show of affection.
	\item Mr Morris was determined to put on a show of family unity.
	\end{itemize}
}
\item uncountable noun \\
If you say that something is \textbf{for}  \textbf{show} , you mean that it has no real purpose and is done just to give a good impression .
 \textit{
	\begin{itemize}
	\item The change in government is more for show than for real.
	\end{itemize}
}
\item verb \\
If a company \textbf{shows} a profit or a loss, its accounts indicate that it has made a profit or a loss.
 \textit{
	\begin{itemize}
	\item It is the only one of the three companies expected to show a profit for the quarter.
	\item Lonrho's mining and minerals businesses showed some improvement.
	\end{itemize}
}
\item verb \\
If a person you are expecting to meet does not \textbf{show} , they do not arrive at the place where you expect to meet them.
 \textbf{Show up} means the same as show .
 \textit{
	\begin{itemize}
	\item There was always a chance he wouldn't show.
	\item We waited until five o'clock, but he did not show up.
	\item He always shows up in a fancy car.
	\item If I don't show up for class this morning, I'll be kicked out.
	\end{itemize}
}
\item countable noun \\
A television or radio \textbf{show} is a programme on television or radio.
 \textit{
	\begin{itemize}
	\item I had my own TV show.
	\item This is the show in which the presenter visits the houses of the famous.
	\item ...a popular talk show on a Cuban radio station.
	\item A daily one-hour news show can cost $250,000 to produce.
	\end{itemize}
}
\item countable noun \\
A \textbf{show} in a theatre is an entertainment or concert , especially one that includes different items such as music, dancing, and comedy .
 \textit{
	\begin{itemize}
	\item How about going shopping and seeing a show in London?
	\item He has earned a reputation as the man who can close a show with a bad review.
	\item The band are playing a handful of shows at smaller venues.
	\end{itemize}
}
\item verb \\
If someone \textbf{shows} a film or television programme, it is broadcast or appears on television or in the cinema.
 \textit{
	\begin{itemize}
	\item The BBC World Service Television news showed the same film clip.
	\item The drama will be shown on American TV next year.
	\item At its peak, the film showed in 93 theaters nationwide in its third weekend.
	\end{itemize}
}
\item countable noun \\
A \textbf{show} is a public exhibition of things, such as works of art, fashionable clothes, or things that have been entered in a competition.
 \textit{
	\begin{itemize}
	\item The venue for the show is the city's exhibition centre.
	\item Gucci will be holding fashion shows to present their autumn collection.
	\item Two complementary exhibitions are on show at the Africa Centre.
	\item Today his picture goes on show at the National Portrait Gallery.
	\end{itemize}
}
\item verb \\
To \textbf{show} things such as works of art means to put them in an exhibition where they can be
seen by the public.
 \textit{
	\begin{itemize}
	\item 50 dealers will show oils, watercolours, drawings and prints from 1900 to 1992.
	\item ...one of the city's better-known galleries, where he showed and sold his work.
	\end{itemize}
}
\item verb \\
In a horse race, if a horse \textbf{shows} , it finishes first, second, or third.
 \textit{
	\begin{itemize}
	\end{itemize}
}
\item adjective \\
A \textbf{show} home, house, or flat is one of a group of new homes. The building company decorates
it and puts furniture in it, and people who want to buy one of the homes come and look round it.
 \textit{
	\begin{itemize}
	\end{itemize}
}
\item  \\
 show of hands \textit{
	\begin{itemize}
	\end{itemize}
}
\item  \\
 have something to show for sth \textit{
	\begin{itemize}
	\end{itemize}
}
\item  \\
 I'll show you \textit{
	\begin{itemize}
	\end{itemize}
}
\item  \\
 it just goes to show \textit{
	\begin{itemize}
	\end{itemize}
}
\item  \\
 run the show \textit{
	\begin{itemize}
	\end{itemize}
}
\item  \\
 to steal the show \textit{
	\begin{itemize}
	\end{itemize}
}
\end{enumerate}

\section*{scorn}
{\large \color{blue}  scorns  scorning  scorned  }
\subsection*{Explain}
\begin{enumerate}
\item uncountable noun \\
If you treat someone or something \textbf{with}  \textbf{scorn} , you show contempt for them.
 \textit{
	\begin{itemize}
	\item Researchers greeted the proposal with scorn.
	\item He reserves particular scorn for the senators who tried to prevent his confirmation.
	\item He became the object of ridicule and scorn.
	\end{itemize}
}
\item verb \\
If you \textbf{scorn} someone or something, you feel or show contempt for them.
 \textit{
	\begin{itemize}
	\item Several leading officers have quite openly scorned the peace talks.
	\item People scorn me as a single parent.
	\end{itemize}
}
\item verb \\
If you \textbf{scorn} something, you refuse to have it or accept it because you think it is not good enough or suitable for you.
 \textit{
	\begin{itemize}
	\item ...people who scorned traditional methods.
	\end{itemize}
}
\item  \\
 to pour scorn on something \textit{
	\begin{itemize}
	\end{itemize}
}
\end{enumerate}

\section*{scream}
{\large \color{blue}  screams  screaming  screamed  }
\subsection*{Explain}
\begin{enumerate}
\item verb \\
When someone \textbf{screams} , they make a very loud , high-pitched cry, for example because they are in pain or are very frightened .
 \textbf{Scream} is also a noun .
 \textit{
	\begin{itemize}
	\item People were screaming; some of the houses nearest the bridge were on fire.
	\item If I hear one more joke about my hair, I shall scream.
	\item He staggered around the playground, screaming in agony.
	\item To play in front of 40,000 screaming fans was a great experience.
	\item Hilda let out a scream.
	\item ...screams of terror.
	\end{itemize}
}
\item verb \\
If you \textbf{scream} something, you shout it in a loud, high-pitched voice .
 \textit{
	\begin{itemize}
	\item 'Brigid!' she screamed. 'Get up!'
	\item I was screaming at them to get out of my house.
	\item They started screaming abuse at us.
	\end{itemize}
}
\item verb \\
When something makes a loud, high-pitched noise , you can  say that it \textbf{screams} .
 \textbf{Scream} is also a noun.
 \textit{
	\begin{itemize}
	\item She slammed the car into gear, the tyres screaming as her foot jammed against the
accelerator.
	\item As he talked, an airforce jet screamed over the town.
	\item There was a scream of brakes from the carriageway outside.
	\end{itemize}
}
\item singular noun \\
If you say that someone is \textbf{a scream} , you think they are very funny .
 \textit{
	\begin{itemize}
	\end{itemize}
}
\end{enumerate}

\section*{tempo}
{\large \color{blue}  tempos  tempi  }
\subsection*{Explain}
\begin{enumerate}
\item singular noun \\
The \textbf{tempo} of an event is the speed at which it happens .
 \textit{
	\begin{itemize}
	\item ...owing to the slow tempo of change in an overwhelmingly rural country.
	\item Both teams played with a lot of quality, pace and tempo.
	\end{itemize}
}
\item variable noun \\
The \textbf{tempo} of a piece of music is the speed at which it is played.
 \textit{
	\begin{itemize}
	\item In a new recording, the Boston Philharmonic tried the original tempo.
	\item Elgar supplied his works with precise indications of tempo.
	\end{itemize}
}
\end{enumerate}

\section*{shiver}
{\large \color{blue}  shivers  shivering  shivered  }
\subsection*{Explain}
\begin{enumerate}
\item verb \\
When you \textbf{shiver} , your body shakes slightly because you are cold or frightened .
 \textbf{Shiver} is also a noun .
 \textit{
	\begin{itemize}
	\item He shivered in the cold.
	\item I was sitting on the floor shivering with fear.
	\item The emptiness here sent shivers down my spine.
	\item Alice gave a shiver of delight.
	\end{itemize}
}
\end{enumerate}

\section*{truck}
{\large \color{blue}  trucks  trucking  trucked  }
\subsection*{Explain}
\begin{enumerate}
\item countable noun \\
A \textbf{truck} is a large vehicle that is used to transport goods by road .
 \textit{
	\begin{itemize}
	\item Now and then they heard the roar of a heavy truck.
	\end{itemize}
}
\item countable noun \\
A \textbf{truck} is an open vehicle used for carrying goods on a railway.
 \textit{
	\begin{itemize}
	\item They were loaded on the railway trucks to go to Liverpool.
	\end{itemize}
}
\item verb \\
When something or someone \textbf{is trucked}  somewhere , they are driven there in a lorry .
 \textit{
	\begin{itemize}
	\item The liquor was sold legally and trucked out of the state.
	\end{itemize}
}
\item  \\
 have no/little truck with \textit{
	\begin{itemize}
	\end{itemize}
}
\end{enumerate}

\section*{spread}
{\large \color{blue}  spreads  spreading  spread  }
\subsection*{Explain}
\begin{enumerate}
\item verb \\
If you \textbf{spread} something somewhere , you open it out or arrange it over a place or surface, so that all of it can be
seen or used easily.
 \textbf{Spread out} means the same as spread .
 \textit{
	\begin{itemize}
	\item She spread a towel on the sand and lay on it.
	\item His coat was spread over the bed.
	\item He extracted several glossy prints and spread them out on a low coffee table.
	\item In his room, Tom was spreading out a map of Scandinavia on the bed.
	\end{itemize}
}
\item verb \\
If you \textbf{spread} your arms, hands , fingers , or legs, you stretch them out until they are far apart.
 \textbf{Spread out} means the same as spread .
 \textit{
	\begin{itemize}
	\item Sitting on the floor, spread your legs as far as they will go without overstretching.
	\item He stepped back and spread his hands wide. 'You are most welcome to our home.'
	\item David spread out his hands as if showing that he had no explanation.
	\item You need a bed that's large enough to let you spread yourself out.
	\end{itemize}
}
\item verb \\
If you \textbf{spread} a substance on a surface or \textbf{spread} the surface \textbf{with} the substance, you put a thin layer of the substance over the surface.
 \textit{
	\begin{itemize}
	\item Spread the mixture in the cake tin and bake for 30 minutes.
	\item A thick layer of wax was spread over the surface.
	\item Spread the bread with the cheese.
	\end{itemize}
}
\item variable noun \\
\textbf{Spread} is a soft food which is put on bread.
 \textit{
	\begin{itemize}
	\item ...a wholemeal salad roll with low fat spread.
	\end{itemize}
}
\item verb \\
If something \textbf{spreads} or \textbf{is spread} by people, it gradually reaches or affects a larger and larger area or more and more
people.
 \textbf{Spread} is also a noun.
 \textit{
	\begin{itemize}
	\item The industrial revolution which started a couple of hundred years ago in Europe is
now spreading across the world.
	\item ...the sense of fear spreading in residential neighborhoods.
	\item He was fed-up with the lies being spread about him.
	\item The greatest hope for reform is the gradual spread of information.
	\item Thanks to the spread of modern technology, trained workers are now more vital than
ever.
	\end{itemize}
}
\item verb \\
If something such as a liquid, gas, or smoke  \textbf{spreads} or \textbf{is spread} , it moves outwards in all directions so that it covers a larger area.
 \textbf{Spread} is also a noun.
 \textit{
	\begin{itemize}
	\item Fire spread rapidly after a chemical truck exploded.
	\item A dark red stain was spreading across his shirt.
	\item In Northern California, a wildfire has spread a haze of smoke over 200 miles.
	\item The situation was complicated by the spread of a serious forest fire.
	\end{itemize}
}
\item verb \\
If you \textbf{spread} something \textbf{over} a period of time, it takes place regularly or continuously over that period, rather
than happening at one time.
 \textit{
	\begin{itemize}
	\item You can eat all your calorie allowance in one go, or spread it over the day.
	\item The course is spread over a five week period.
	\end{itemize}
}
\item verb \\
If you \textbf{spread} something such as wealth or work, you distribute it evenly or equally.
 \textbf{Spread} is also a noun.
 \textit{
	\begin{itemize}
	\item ...policies that spread the state's wealth more evenly.
	\item The loss of jobs has been far more evenly spread across the regions than it was during
the early 1980s.
	\item There are easier ways to encourage the even spread of wealth.
	\end{itemize}
}
\item singular noun \\
A \textbf{spread}  \textbf{of} ideas, interests, or other things is a wide variety of them.
 \textit{
	\begin{itemize}
	\item ...primary schools with a typical spread of ability.
	\item We have an enormous spread of industries around the country.
	\end{itemize}
}
\item countable noun \\
A \textbf{spread} is a large meal, especially one that has been prepared for a special occasion .
 \textit{
	\begin{itemize}
	\end{itemize}
}
\item countable noun \\
A \textbf{spread} is two pages of a book, magazine , or newspaper that are opposite each other when you open it at a particular place.
 \textit{
	\begin{itemize}
	\item There was a double-page spread of a dinner for 46 people.
	\end{itemize}
}
\item singular noun \\
\textbf{Spread} is used to refer to the difference between the price that a seller  wants someone to pay for a particular stock or share and the price that the buyer is willing to pay.
 \textit{
	\begin{itemize}
	\item Market makers earn their livings from the spread between buying and selling prices.
	\end{itemize}
}
\end{enumerate}

\section*{weekday}
{\large \color{blue}  weekdays  }
\subsection*{Explain}
\begin{enumerate}
\item countable noun \\
A \textbf{weekday} is any of the days of the week except Saturday and Sunday.
 \textit{
	\begin{itemize}
	\item If you want to avoid the crowds, it's best to come on a weekday.
	\item Visitor Centre and shop open 9 a.m.–5 p.m. weekdays; 10 a.m.–5 p.m. weekends.
	\end{itemize}
}
\end{enumerate}

\section*{stab}
{\large \color{blue}  stabs  stabbing  stabbed  }
\subsection*{Explain}
\begin{enumerate}
\item verb \\
If someone \textbf{stabs} you, they push a knife or sharp object into your body.
 \textit{
	\begin{itemize}
	\item Somebody stabbed him in the stomach.
	\item Dean tried to stab him with a screwdriver.
	\item Stephen was stabbed to death in an unprovoked attack nearly five months ago.
	\end{itemize}
}
\item verb \\
If you \textbf{stab} something or \textbf{stab at} it, you push at it with your finger or with something pointed that you are holding .
 \textit{
	\begin{itemize}
	\item Bess stabbed a slice of cucumber.
	\item Goldstone flipped through the pages and stabbed his thumb at the paragraph he was
looking for.
	\item He stabbed at the omelette with his fork.
	\end{itemize}
}
\item singular noun \\
If you have \textbf{a stab at} something, you try to do it.
 \textit{
	\begin{itemize}
	\item Several tennis stars have had a stab at acting.
	\end{itemize}
}
\item singular noun \\
You can  refer to a sudden, usually unpleasant feeling as \textbf{a stab of} that feeling.
 \textit{
	\begin{itemize}
	\item ...a stab of pain just above his eye.
	\item She felt a stab of pity for him.
	\end{itemize}
}
\item  \\
 stab someone in the back \textit{
	\begin{itemize}
	\end{itemize}
}
\end{enumerate}

\section*{witness}
{\large \color{blue}  witnesses  witnessing  witnessed  }
\subsection*{Explain}
\begin{enumerate}
\item countable noun \\
A \textbf{witness}  \textbf{to} an event such as an accident or crime is a person who saw it.
 \textit{
	\begin{itemize}
	\item Witnesses to the crash say they saw an explosion just before the disaster.
	\item No witnesses have come forward.
	\end{itemize}
}
\item verb \\
If you \textbf{witness} something, you see it happen .
 \textit{
	\begin{itemize}
	\item Anyone who witnessed the attack should call the police.
	\item It was the quickest swimming lesson I'd ever witnessed.
	\end{itemize}
}
\item countable noun \\
A \textbf{witness} is someone who appears in a court of law to say what they know about a crime or other event.
 \textit{
	\begin{itemize}
	\item In the next three or four days, eleven witnesses will be called to testify.
	\end{itemize}
}
\item countable noun \\
A \textbf{witness} is someone who writes their name on a document that you have signed , to confirm that it really is your signature.
 \textit{
	\begin{itemize}
	\end{itemize}
}
\item verb \\
If someone \textbf{witnesses} your signature on a document, they write their name after it, to confirm that it
really is your signature.
 \textit{
	\begin{itemize}
	\item Ask a friend to witness your signature.
	\end{itemize}
}
\item verb \\
If you say that a place, period of time, or person \textbf{witnessed} a particular event or change , you mean that it happened in that place, during that period of time, or while that person
was alive .
 \textit{
	\begin{itemize}
	\item India has witnessed many political changes in recent years.
	\item The year 1886 witnessed the first extended translation into English of the writings
of Eliphas Levi.
	\item At present, we are witnessing another building boom.
	\end{itemize}
}
\item verb \\
You use \textbf{witness} to introduce an example of what you have just been talking about.
 \textit{
	\begin{itemize}
	\item Americans are generous: witness the increase in charitable giving, even during the
recession.
	\end{itemize}
}
\item  \\
 be witness to \textit{
	\begin{itemize}
	\end{itemize}
}
\item  \\
 to bear witness to \textit{
	\begin{itemize}
	\end{itemize}
}
\end{enumerate}

\section*{sting}
{\large \color{blue}  stings  stinging  stung  }
\subsection*{Explain}
\begin{enumerate}
\item verb \\
If a plant, animal, or insect \textbf{stings} you, a sharp part of it, usually covered with poison, is pushed into your skin so that you feel a sharp pain.
 \textit{
	\begin{itemize}
	\item The nettles stung their legs.
	\item I jumped as if I had been stung by a scorpion.
	\item This type of bee rarely stings.
	\end{itemize}
}
\item countable noun \\
The \textbf{sting} of an insect or animal is the part that stings you.
 \textit{
	\begin{itemize}
	\item Remove the bee sting with tweezers.
	\end{itemize}
}
\item countable noun \\
If you feel a \textbf{sting} , you feel a sharp pain in your skin or other part of your body.
 \textit{
	\begin{itemize}
	\item This won't hurt–you will just feel a little sting.
	\end{itemize}
}
\item verb \\
If a part of your body \textbf{stings} , or if a substance  \textbf{stings} it, you feel a sharp pain there.
 \textit{
	\begin{itemize}
	\item His cheeks were stinging from the icy wind.
	\item Never put any essential oils near the eyes. They are very strong and could sting.
	\item Sprays can sting sensitive skin.
	\end{itemize}
}
\item verb \\
If someone's remarks  \textbf{sting} you, they make you feel hurt and annoyed .
 \textit{
	\begin{itemize}
	\item He's a sensitive lad and some of the criticism has stung him.
	\item She burst into tears, stung by the harshness of his words.
	\end{itemize}
}
\item countable noun \\
A \textbf{sting} is a clever  secret  plan  carried out by the police in order to catch  criminals .
 \textit{
	\begin{itemize}
	\item The police ran a sting operation to crack down on illegal guns.
	\item ...a sting set by the FBI.
	\end{itemize}
}
\item  \\
 a sting in the tail \textit{
	\begin{itemize}
	\end{itemize}
}
\item  \\
 take the sting out of sth \textit{
	\begin{itemize}
	\end{itemize}
}
\end{enumerate}

\section*{appliance}
{\large \color{blue}  appliances  }
\subsection*{Explain}
\begin{enumerate}
\item countable noun \\
An \textbf{appliance} is a device or machine in your home that you use to do a job such as cleaning or cooking . Appliances are often electrical.
 \textit{
	\begin{itemize}
	\item ...the vacuum cleaner, washing machine and other household appliances.
	\item They are one of America's biggest domestic appliance manufacturers.
	\end{itemize}
}
\item singular noun \\
The \textbf{appliance} of a skill or of knowledge is its use for a particular purpose.
 \textit{
	\begin{itemize}
	\item These advances were the result of the intellectual appliance of science.
	\end{itemize}
}
\end{enumerate}

\section*{blush}
{\large \color{blue}  blushes  blushing  blushed  }
\subsection*{Explain}
\begin{enumerate}
\item verb \\
When you \textbf{blush} , your face becomes redder than usual because you are ashamed or embarrassed .
 \textbf{Blush} is also a noun .
 \textit{
	\begin{itemize}
	\item 'Hello, Maria,' he said, and she blushed again.
	\item I blushed scarlet at my stupidity.
	\item 'The most important thing is to be honest,' she says, without the trace of a blush.
	\end{itemize}
}
\item  \\
 to spare someone's blushes \textit{
	\begin{itemize}
	\end{itemize}
}
\end{enumerate}

\section*{battery}
{\large \color{blue}  batteries  }
\subsection*{Explain}
\begin{enumerate}
\item countable noun \\
\textbf{Batteries} are small devices that provide the power for electrical  items such as torches and children's toys .
 \textit{
	\begin{itemize}
	\item The shavers come complete with batteries.
	\item ...a battery-operated radio.
	\item ...rechargeable batteries.
	\end{itemize}
}
\item countable noun \\
A car \textbf{battery} is a rectangular  box containing acid that is found in a car engine. It provides the electricity  needed to start the car.
 \textit{
	\begin{itemize}
	\item ...a car with a flat battery.
	\end{itemize}
}
\item countable noun \\
A \textbf{battery}  \textbf{of}  equipment such as guns, lights, or computers is a large set of it kept together in one place.
 \textit{
	\begin{itemize}
	\item They stopped beside a battery of abandoned guns.
	\item ...batteries of spotlights set up on rooftops.
	\end{itemize}
}
\item countable noun \\
A \textbf{battery}  \textbf{of} people or things is a very large number of them.
 \textit{
	\begin{itemize}
	\item ...a battery of journalists and television cameras.
	\end{itemize}
}
\item countable noun \\
A \textbf{battery}  \textbf{of} tests is a set of tests that is used to assess a number of different aspects of something, such as your health .
 \textit{
	\begin{itemize}
	\item We give a battery of tests to each patient.
	\end{itemize}
}
\item adjective \\
\textbf{Battery}  farming is a system of breeding chickens and hens in which large numbers of them are kept in small cages, and used for their meat and eggs .
 \textit{
	\begin{itemize}
	\item ...battery hens being raised in dark, cramped conditions.
	\end{itemize}
}
\end{enumerate}

\section*{capture}
{\large \color{blue}  captures  capturing  captured  }
\subsection*{Explain}
\begin{enumerate}
\item verb \\
If you \textbf{capture} someone or something, you catch them, especially in a war.
 \textbf{Capture} is also a noun .
 \textit{
	\begin{itemize}
	\item The guerrillas shot down one aeroplane and captured the pilot.
	\item The whole town celebrated when two tanks were captured.
	\item King Arthur himself captures the beast and cuts off its head.
	\item The United States captured Puerto Rico from the Spaniards in 1898.
	\item ...the murders of fifteen thousand captured Polish soldiers.
	\item ...the final battles which led to the army's capture of the town.
	\item The shooting happened while the man was trying to evade capture by the security forces.
	\end{itemize}
}
\item verb \\
If something or someone \textbf{captures} a particular quality, feeling , or atmosphere , they represent or express it successfully.
 \textit{
	\begin{itemize}
	\item The mood was captured by a cartoon in the New York Post.
	\end{itemize}
}
\item verb \\
If something \textbf{captures} your attention or imagination , you begin to be interested or excited by it. If someone or something \textbf{captures} your heart , you begin to love them or like them very much.
 \textit{
	\begin{itemize}
	\item ...the great names of the Tory party who usually capture the historian's attention.
	\item ...the issue that has captured the imagination of nearly the whole nation.
	\item ...one man's undying love for the woman who captured his heart.
	\end{itemize}
}
\item verb \\
If an event  \textbf{is captured} in a photograph or on film, it is photographed or filmed.
 \textit{
	\begin{itemize}
	\item The incident was captured on video.
	\item The images were captured by TV crews filming outside the base.
	\item ...photographers who captured the traumatic scene.
	\end{itemize}
}
\item verb \\
If you \textbf{capture} something that you are trying to obtain in competition with other people, you succeed in obtaining it.
 \textit{
	\begin{itemize}
	\item The company aims to capture more sales at a time of significant challenges in the
supermarket sector.
	\item The Socialist candidate has captured eighty-five per cent of the vote in the three-way
presidential race.
	\end{itemize}
}
\end{enumerate}

\section*{being}
{\large \color{blue}  beings  }
\subsection*{Explain}
\begin{enumerate}
\item  \\
\textbf{Being} is the present participle of be1 .
 \textit{
	\begin{itemize}
	\end{itemize}
}
\item link verb \\
\textbf{Being} is used in non-finite clauses where you are giving the reason for something.
 \textit{
	\begin{itemize}
	\item It being a Sunday, the old men had the day off.
	\item Little boys, being what they are, might decide to play on it.
	\item Of course, being young, I did not worry.
	\end{itemize}
}
\item countable noun \\
You can refer to any real or imaginary  creature as a \textbf{being} .
 \textit{
	\begin{itemize}
	\item Every living being is subject to decay
	\item ...beings from outer space.
	\end{itemize}
}
\item uncountable noun \\
\textbf{Being} is existence. Something that is \textbf{in being} or comes  \textbf{into being} exists or starts to exist.
 \textit{
	\begin{itemize}
	\item Abraham Maslow described psychology as 'the science of being.'
	\item The Kingdom of Italy formally came into being on 17 March 1861.
	\item ...the complex process by which the novel is brought into being.
	\end{itemize}
}
\item  \\
 being as \textit{
	\begin{itemize}
	\end{itemize}
}
\end{enumerate}

\section*{cease}
{\large \color{blue}  ceases  ceasing  ceased  }
\subsection*{Explain}
\begin{enumerate}
\item verb \\
If something \textbf{ceases} , it stops happening or existing .
 \textit{
	\begin{itemize}
	\item At one o'clock the rain had ceased.
	\end{itemize}
}
\item verb \\
If you \textbf{cease}  \textbf{to} do something, you stop doing it.
 \textit{
	\begin{itemize}
	\item He never ceases to amaze me.
	\item The secrecy about the President's condition had ceased to matter.
	\item A small number of firms have ceased trading.
	\end{itemize}
}
\item verb \\
If you \textbf{cease} something, you stop it happening or working .
 \textit{
	\begin{itemize}
	\item The Daily Herald ceased publication, to be replaced by The Sun.
	\end{itemize}
}
\end{enumerate}

\section*{biology}
{\large \color{blue}  }
\subsection*{Explain}
\begin{enumerate}
\item uncountable noun \\
\textbf{Biology} is the science which is concerned with the study of living things.
 \textit{
	\begin{itemize}
	\end{itemize}
}
\item uncountable noun \\
The \textbf{biology} of a living thing is the way in which its body or cells behave .
 \textit{
	\begin{itemize}
	\item The biology of these diseases is terribly complicated.
	\item ...human biology.
	\end{itemize}
}
\end{enumerate}

\section*{clash}
{\large \color{blue}  clashes  clashing  clashed  }
\subsection*{Explain}
\begin{enumerate}
\item verb \\
When people \textbf{clash} , they fight , argue , or disagree with each other.
 \textbf{Clash} is also a noun .
 \textit{
	\begin{itemize}
	\item A group of 400 demonstrators clashed with police.
	\item Behind the scenes, Parsons clashed with almost everyone on the show.
	\item The working groups have also clashed over genetically modified crops.
	\item There have been a number of clashes between police in riot gear and demonstrators.
	\end{itemize}
}
\item verb \\
Beliefs , ideas , or qualities that \textbf{clash}  \textbf{with} each other are very different from each other and therefore are opposed .
 \textbf{Clash} is also a noun.
 \textit{
	\begin{itemize}
	\item Don't make any policy decisions which clash with official company thinking.
	\item Here, morality and good sentiments clash headlong.
	\item Inside government, there was a clash of views.
	\end{itemize}
}
\item verb \\
If one event \textbf{clashes with} another, the two events happen at the same time so that you cannot attend both of them.
 \textit{
	\begin{itemize}
	\item I couldn't go on the trip as it clashed with my final exams.
	\end{itemize}
}
\item verb \\
If one colour or style  \textbf{clashes}  \textbf{with} another, the colours or styles look ugly together. You can also say that two colours or styles \textbf{clash} .
 \textit{
	\begin{itemize}
	\item The red door clashed with the soft, natural tones of the stone walls.
	\item So what if the colours clashed?
	\end{itemize}
}
\item reciprocal verb \\
Sports  journalists  sometimes say that two individuals or teams who compete against each other \textbf{clash} , especially when a lot of feeling is involved .
 \textbf{Clash} is also a noun.
 \textit{
	\begin{itemize}
	\item He will clash with his rival in the final.
	\item The two sides will clash there only if Chelsea beat Sunderland in their quarter-final
replay.
	\item Australia's rugby union team for the return clash with New Zealand is weakened by
injury.
	\end{itemize}
}
\item verb \\
When metal  objects  \textbf{clash} , they make a lot of noise by being hit together.
 \textbf{Clash} is also a noun.
 \textit{
	\begin{itemize}
	\item The golden bangles on her arms clashed and jingled.
	\item ...a noise like the clash of cymbals.
	\end{itemize}
}
\end{enumerate}

\section*{birthday}
{\large \color{blue}  birthdays  }
\subsection*{Explain}
\begin{enumerate}
\item countable noun \\
Your \textbf{birthday} is the anniversary of the date on which you were born.
 \textit{
	\begin{itemize}
	\end{itemize}
}
\end{enumerate}

\section*{collapse}
{\large \color{blue}  collapses  collapsing  collapsed  }
\subsection*{Explain}
\begin{enumerate}
\item verb \\
If a building or other structure \textbf{collapses} , it falls down very suddenly.
 \textbf{Collapse} is also a noun .
 \textit{
	\begin{itemize}
	\item A section of the Bay Bridge had collapsed.
	\item The roof collapsed in a roar of rock and rubble.
	\item Most of the deaths were caused by landslides and collapsing buildings.
	\item Governor Deukmejian called for an inquiry into the freeway's collapse.
	\end{itemize}
}
\item verb \\
If something, for example a system or institution , \textbf{collapses} , it fails or comes to an end completely and suddenly.
 \textbf{Collapse} is also a noun.
 \textit{
	\begin{itemize}
	\item His business empire collapsed under a massive burden of debt.
	\item This system has collapsed in most countries where it ruled.
	\item The rural people have been impoverished by a collapsing economy.
	\item The coup's collapse has speeded up the drive to independence.
	\item Their economy is teetering on the brink of collapse.
	\end{itemize}
}
\item verb \\
If you \textbf{collapse} , you suddenly faint or fall down because you are very ill or weak .
 \textbf{Collapse} is also a noun.
 \textit{
	\begin{itemize}
	\item He collapsed following a vigorous exercise session at his home.
	\item It's commonplace to see people collapsing from hunger in the streets.
	\item A few days after his collapse he was sitting up in bed.
	\end{itemize}
}
\item verb \\
If you \textbf{collapse} onto something, you sit or lie down suddenly because you are very tired .
 \textit{
	\begin{itemize}
	\item She arrived home exhausted and barely capable of showering before collapsing on her
bed.
	\end{itemize}
}
\item verb \\
If something with air inside  \textbf{collapses} , it falls inwards and becomes smaller or flatter .
 \textit{
	\begin{itemize}
	\item He plunged 300ft to the ground when his parachute collapsed.
	\item He was rushed to hospital last week after suffering a collapsed lung.
	\end{itemize}
}
\end{enumerate}

\section*{booth}
{\large \color{blue}  booths  }
\subsection*{Explain}
\begin{enumerate}
\item countable noun \\
A \textbf{booth} is a small area separated from a larger public area by screens or thin  walls where, for example , people can make a phone  call or vote in private .
 \textit{
	\begin{itemize}
	\item I called her from a public phone booth near the entrance to the bar.
	\item In Darlington, queues formed at some polling booths.
	\end{itemize}
}
\item countable noun \\
A \textbf{booth} in a restaurant or café consists of a table with long fixed  seats on two or sometimes  three  sides of it.
 \textit{
	\begin{itemize}
	\item They sat in a corner booth, away from other diners.
	\end{itemize}
}
\item countable noun \\
A \textbf{booth} is a small tent or stall, usually at a fair, in which you can buy goods or watch some form of entertainment .
 \textit{
	\begin{itemize}
	\end{itemize}
}
\item countable noun \\
A \textbf{booth} is a stall at an exhibition , for example with a display of goods for sale or with information  leaflets .
 \textit{
	\begin{itemize}
	\end{itemize}
}
\end{enumerate}

\section*{crash}
{\large \color{blue}  crashes  crashing  crashed  }
\subsection*{Explain}
\begin{enumerate}
\item countable noun \\
A \textbf{crash} is an accident in which a moving vehicle hits something and is damaged or destroyed .
 \textit{
	\begin{itemize}
	\item His elder son was killed in a car crash a few years ago.
	\item ...a plane crash.
	\end{itemize}
}
\item verb \\
If a moving vehicle \textbf{crashes} or if the driver  \textbf{crashes} it, it hits something and is damaged or destroyed.
 \textit{
	\begin{itemize}
	\item The plane crashed mysteriously near the island of Ustica.
	\item ...when his car crashed into the rear of a van.
	\item He crashed his bike into a parked car and broke his arm.
	\item Her body was found near a crashed car.
	\end{itemize}
}
\item verb \\
If something \textbf{crashes}  somewhere , it moves and hits something else violently, making a loud noise.
 \textit{
	\begin{itemize}
	\item The door swung inwards to crash against a chest of drawers behind it.
	\item My words were lost as the walls above us crashed down, filling the cellar with brick
dust.
	\item I heard them coming, crashing through the undergrowth, before I saw them.
	\end{itemize}
}
\item countable noun \\
A \textbf{crash} is a sudden, loud noise.
 \textit{
	\begin{itemize}
	\item Two people in the flat recalled hearing a loud crash about 1.30 a.m.
	\end{itemize}
}
\item verb \\
If a business or financial system \textbf{crashes} , it fails suddenly, often with serious  effects .
 \textbf{Crash} is also a noun .
 \textit{
	\begin{itemize}
	\item When the market crashed, they assumed the deal would be cancelled.
	\item He predicted correctly that there was going to be a stock market crash.
	\end{itemize}
}
\item verb \\
If a computer or a computer program \textbf{crashes} , it fails suddenly.
 \textit{
	\begin{itemize}
	\item ...after the computer crashed for the second time in 10 days.
	\end{itemize}
}
\end{enumerate}

\section*{brother}
{\large \color{blue}  brothers  }
\subsection*{Explain}
\begin{enumerate}
\item countable noun \\
Your \textbf{brother} is a boy or a man who has the same parents as you.
 \textit{
	\begin{itemize}
	\item Oh, so you're Peter's younger brother.
	\item Have you got any brothers and sisters?
	\end{itemize}
}
\item countable noun \\
You can describe a man as your \textbf{brother} if he belongs to the same race , religion , country , profession, or trade union as you, or if he has similar  ideas to you.
 \textit{
	\begin{itemize}
	\item He told reporters he'd come to be with his Latvian brothers.
	\item ...the Cardinal and his brother bishops.
	\end{itemize}
}
\item title noun \\
\textbf{Brother} is a title  given to a man who belongs to a religious community such as a monastery .
 \textit{
	\begin{itemize}
	\item ...Brother Otto.
	\item ...the Christian Brothers community which owns the castle.
	\end{itemize}
}
\item countable noun \\
\textbf{Brothers} is used in the names of some companies and shops .
 \textit{
	\begin{itemize}
	\item ...the film company Warner Brothers.
	\end{itemize}
}
\end{enumerate}

\section*{damage}
{\large \color{blue}  damages  damaging  damaged  }
\subsection*{Explain}
\begin{enumerate}
\item verb \\
To \textbf{damage} an object means to break it, spoil it physically, or stop it from working properly.
 \textit{
	\begin{itemize}
	\item He maliciously damaged a car with a baseball bat.
	\item The sun can damage your skin.
	\end{itemize}
}
\item verb \\
To \textbf{damage} something means to cause it to become less good, pleasant , or successful .
 \textit{
	\begin{itemize}
	\item Jackson doesn't want to damage his reputation as a political personality.
	\item He warned that the action was damaging the economy.
	\end{itemize}
}
\item uncountable noun \\
\textbf{Damage} is physical harm that is caused to an object.
 \textit{
	\begin{itemize}
	\item The blast caused extensive damage to the house.
	\item Many professional boxers end their careers with brain damage.
	\end{itemize}
}
\item uncountable noun \\
\textbf{Damage} consists of the unpleasant  effects that something has on a person, situation , or type of activity .
 \textit{
	\begin{itemize}
	\item Incidents of this type cause irreparable damage to relations with the community.
	\item Adhering to the new rules meant inflicting serious damage on motor racing.
	\end{itemize}
}
\item plural noun \\
If a court of law  awards  \textbf{damages} to someone, it orders money to be paid to them by a person who has damaged their reputation or property , or who has injured them.
 \textit{
	\begin{itemize}
	\item He was vindicated in court and damages were awarded.
	\end{itemize}
}
\item  \\
 the damage is done \textit{
	\begin{itemize}
	\end{itemize}
}
\end{enumerate}

\section*{cable}
{\large \color{blue}  cables  cabling  cabled  }
\subsection*{Explain}
\begin{enumerate}
\item variable noun \\
A \textbf{cable} is a thick wire, or a group of wires inside a rubber or plastic  covering , which is used to carry electricity or electronic  signals .
 \textit{
	\begin{itemize}
	\item ...overhead power cables.
	\item ...strings of coloured lights with weatherproof cable.
	\end{itemize}
}
\item variable noun \\
A \textbf{cable} is a kind of very strong, thick rope, made of wires twisted together.
 \textit{
	\begin{itemize}
	\item ...the heavy anchor cable.
	\item Steel cable will be used to replace worn ropes.
	\end{itemize}
}
\item uncountable noun \\
\textbf{Cable} is used to refer to television systems in which the signals are sent along underground wires rather than by radio waves .
 \textit{
	\begin{itemize}
	\item They ran commercials on cable systems across the country.
	\item The channel is only available on cable.
	\end{itemize}
}
\item countable noun \\
A \textbf{cable} is the same as a telegram .
 \textit{
	\begin{itemize}
	\item She sent a cable to her mother.
	\end{itemize}
}
\item verb \\
If you \textbf{cable} someone, you send them a message in the form of a telegram.
 \textit{
	\begin{itemize}
	\item 'Don't do it again,' Franklin cabled her when he got her letter.
	\item She had to decide whether or not to cable the news to Louis.
	\item ...a new formula which is being cabled back to capitals for approval.
	\end{itemize}
}
\item verb \\
If a country, a city, or someone's home  \textbf{is cabled} , cables and other equipment are put in place so that the people there can receive cable television.
 \textit{
	\begin{itemize}
	\item In France, 27 major cities are soon to be cabled.
	\item In the U.K., 254,000 homes are cabled.
	\end{itemize}
}
\end{enumerate}

\section*{decay}
{\large \color{blue}  decays  decaying  decayed  }
\subsection*{Explain}
\begin{enumerate}
\item verb \\
When something such as a dead body, a dead plant, or a tooth  \textbf{decays} , it is gradually destroyed by a natural process.
 \textbf{Decay} is also a noun .
 \textit{
	\begin{itemize}
	\item The bodies buried in the fine ash slowly decayed.
	\item The ground was scattered with decaying leaves.
	\item When not removed, plaque causes tooth decay and gum disease.
	\end{itemize}
}
\item verb \\
If something such as a society , system, or institution  \textbf{decays} , it gradually becomes weaker or its condition gets  worse .
 \textbf{Decay} is also a noun.
 \textit{
	\begin{itemize}
	\item Popular cinema seems to have decayed.
	\item Congress has tried dozens of approaches to revitalize decaying urban and rural areas.
	\item There are problems of urban decay and gang violence.
	\end{itemize}
}
\end{enumerate}

\section*{cinema}
{\large \color{blue}  cinemas  }
\subsection*{Explain}
\begin{enumerate}
\item countable noun \\
A \textbf{cinema} is a place where people go to watch films for entertainment .
 \textit{
	\begin{itemize}
	\item The country has relatively few cinemas.
	\end{itemize}
}
\item singular noun \\
You can talk about \textbf{the cinema} when you are talking about seeing a film in a cinema.
 \textit{
	\begin{itemize}
	\item I can't remember the last time we went to the cinema.
	\item They decided to spend an evening at the cinema.
	\end{itemize}
}
\item uncountable noun \\
\textbf{Cinema} is the business and art of making films.
 \textit{
	\begin{itemize}
	\item Contemporary African cinema has much to offer in its vitality and freshness.
	\item ...in the early days of cinema.
	\end{itemize}
}
\end{enumerate}

\section*{delay}
{\large \color{blue}  delays  delaying  delayed  }
\subsection*{Explain}
\begin{enumerate}
\item verb \\
If you \textbf{delay} doing something, you do not do it immediately or at the planned or expected time, but you leave it until later.
 \textit{
	\begin{itemize}
	\item For sentimental reasons I wanted to delay my departure until June.
	\item They had delayed having children, for the usual reason, to establish their careers.
	\item So don't delay, write in now for your chance of a free gift.
	\end{itemize}
}
\item verb \\
To \textbf{delay} someone or something means to make them late or to slow them down.
 \textit{
	\begin{itemize}
	\item Can you delay him in some way?
	\item Various set-backs and problems delayed production.
	\item The passengers were delayed for an hour.
	\end{itemize}
}
\item verb \\
If you \textbf{delay} , you deliberately take longer than necessary to do something.
 \textit{
	\begin{itemize}
	\item If he delayed any longer, the sun would be up.
	\end{itemize}
}
\item variable noun \\
If there is a \textbf{delay} , something does not happen until later than planned or expected.
 \textit{
	\begin{itemize}
	\item They claimed that such a delay wouldn't hurt anyone.
	\item Although the tests have caused some delay, flights should be back to normal this
morning.
	\end{itemize}
}
\item uncountable noun \\
\textbf{Delay} is a failure to do something immediately or in the required or usual time.
 \textit{
	\begin{itemize}
	\item There is no time left for delay.
	\item We'll send you a quote without delay.
	\end{itemize}
}
\end{enumerate}

\section*{current}
{\large \color{blue}  currents  }
\subsection*{Explain}
\begin{enumerate}
\item countable noun \\
A \textbf{current} is a steady and continuous flowing movement of some of the water in a river, lake, or sea.
 \textit{
	\begin{itemize}
	\item Under normal conditions, the ocean currents of the tropical Pacific travel from east
to west.
	\item The couple were swept away by the strong current.
	\end{itemize}
}
\item countable noun \\
A \textbf{current} is a steady flowing movement of air.
 \textit{
	\begin{itemize}
	\item I felt a current of cool air blowing in my face.
	\end{itemize}
}
\item countable noun \\
An electric \textbf{current} is a flow of electricity through a wire or circuit .
 \textit{
	\begin{itemize}
	\item A powerful electric current is passed through a piece of graphite.
	\end{itemize}
}
\item countable noun \\
A particular \textbf{current} is a particular feeling, idea, or quality that exists within a group of people.
 \textit{
	\begin{itemize}
	\item Each party represents a distinct current of thought.
	\item ...the current of excitement that passed invisibly through the village when something
new was happening.
	\end{itemize}
}
\item adjective \\
\textbf{Current} means happening , being used, or being done at the present time.
 \textit{
	\begin{itemize}
	\item The current situation is very different to that in 1990.
	\item He plans to repeal a number of current policies.
	\item When asked for your views about your current job, on no account must you be negative.
	\end{itemize}
}
\item adjective \\
Ideas and customs that are \textbf{current} are generally accepted and used by most people.
 \textit{
	\begin{itemize}
	\item Current thinking suggests that toxins play only a small part in the build up of cellulite.
	\item This custom was still current in the late 1960s.
	\end{itemize}
}
\end{enumerate}

\section*{desire}
{\large \color{blue}  desires  desiring  desired  }
\subsection*{Explain}
\begin{enumerate}
\item countable noun \\
A \textbf{desire} is a strong wish to do or have something.
 \textit{
	\begin{itemize}
	\item I had a strong desire to help and care for people.
	\item They seem to have lost their desire for life.
	\end{itemize}
}
\item verb \\
If you \textbf{desire} something, you want it.
 \textit{
	\begin{itemize}
	\item She had remarried and desired a child with her new husband.
	\item But Fred was bored and desired to go home.
	\item He desired me to inform her that he had made his peace with God.
	\end{itemize}
}
\item uncountable noun \\
\textbf{Desire} for someone is a strong feeling of wanting to have sex with them.
 \textit{
	\begin{itemize}
	\item Teenage sex, for instance, may come not out of genuine desire but from a need to
get love.
	\end{itemize}
}
\item verb \\
If you \textbf{desire} someone, you want to have sex with them.
 \textit{
	\begin{itemize}
	\item It never occurred to him that she might not desire him.
	\end{itemize}
}
\item  \\
 if desired \textit{
	\begin{itemize}
	\end{itemize}
}
\item  \\
 one's heart's desire \textit{
	\begin{itemize}
	\end{itemize}
}
\item  \\
 to leave a lot to be desired \textit{
	\begin{itemize}
	\end{itemize}
}
\end{enumerate}

\section*{deck}
{\large \color{blue}  decks  decking  decked  }
\subsection*{Explain}
\begin{enumerate}
\item countable noun \\
A \textbf{deck} on a vehicle such as a bus or ship is a lower or upper area of it.
 \textit{
	\begin{itemize}
	\item ...sitting on the top deck of the number 13 bus.
	\item ...a luxury liner with five passenger decks.
	\end{itemize}
}
\item countable noun \\
The \textbf{deck} of a ship is the top part of it that forms a floor in the open air which you can walk on.
 \textit{
	\begin{itemize}
	\item She stood on the deck and waved.
	\end{itemize}
}
\item countable noun \\
A tape  \textbf{deck} or record \textbf{deck} is a piece of equipment on which you play tapes or records.
 \textit{
	\begin{itemize}
	\item ...the tape deck in my car.
	\item I stuck a tape in the deck.
	\end{itemize}
}
\item countable noun \\
A \textbf{deck} of cards is a complete set of playing cards.
 \textit{
	\begin{itemize}
	\item Matt picked up the cards and shuffled the deck.
	\end{itemize}
}
\item countable noun \\
A \textbf{deck} is a flat wooden area next to a house, where people can sit and relax or eat .
 \textit{
	\begin{itemize}
	\item A natural timber deck leads into the main room of the home.
	\end{itemize}
}
\item verb \\
If something \textbf{is decked}  \textbf{with}  pretty things, it is decorated with them.
 \textit{
	\begin{itemize}
	\item Villagers decked the streets with bunting.
	\item The house was decked with flowers.
	\end{itemize}
}
\item verb \\
If someone \textbf{decks} you, they hit you so that you fall over.
 \textit{
	\begin{itemize}
	\item He decked him with a single blow.
	\end{itemize}
}
\item  \\
 below decks \textit{
	\begin{itemize}
	\end{itemize}
}
\item  \\
 to clear the decks \textit{
	\begin{itemize}
	\end{itemize}
}
\item  \\
 hit the deck \textit{
	\begin{itemize}
	\end{itemize}
}
\end{enumerate}

\section*{dip}
{\large \color{blue}  dips  dipping  dipped  }
\subsection*{Explain}
\begin{enumerate}
\item verb \\
If you \textbf{dip} something \textbf{in} a liquid, you put it into the liquid for a short time, so that only part of it is
covered, and take it out again.
 \textbf{Dip} is also a noun.
 \textit{
	\begin{itemize}
	\item They dip the food into the sauce.
	\item Quickly dip the base in and out of cold water.
	\item One dip into the bottle should do an entire nail.
	\end{itemize}
}
\item verb \\
If you \textbf{dip} your hand \textbf{into} a container or \textbf{dip}  \textbf{into} the container, you put your hand into it in order to take something out of it.
 \textit{
	\begin{itemize}
	\item She dipped a hand into the jar of sweets and pulled one out.
	\item Watch your fingers as you dip into the pot.
	\item Ask the children to guess what's in each container by dipping their hands in.
	\end{itemize}
}
\item verb \\
If something \textbf{dips} , it makes a downward movement, usually quite quickly.
 \textbf{Dip} is also a noun.
 \textit{
	\begin{itemize}
	\item Blake jumped in expertly; the boat dipped slightly under his weight.
	\item The sun dipped below the horizon.
	\item I noticed little things, a dip of the head, a twitch in the shoulder.
	\end{itemize}
}
\item verb \\
If an area of land, a road, or a path \textbf{dips} , it goes down quite suddenly to a lower level.
 \textbf{Dip} is also a noun.
 \textit{
	\begin{itemize}
	\item The road dipped and rose again.
	\item ...a path which suddenly dips down into a tunnel.
	\item Where the road makes a dip, turn right.
	\end{itemize}
}
\item verb \\
When farmers  \textbf{dip} sheep or other farm animals, they put them into a container of liquid with chemicals
in it, in order to kill harmful insects which live on the animals' bodies.
 \textit{
	\begin{itemize}
	\item Their father was helping to dip the sheep.
	\end{itemize}
}
\item uncountable noun \\
\textbf{Dip} is a liquid with chemicals in it which animals or objects can be dipped in to disinfect or clean them.
 \textit{
	\begin{itemize}
	\item ...sheep dip.
	\end{itemize}
}
\item verb \\
If the amount or level of something \textbf{dips} , it becomes smaller or lower, usually only for a short period of time.
 \textbf{Dip} is also a noun.
 \textit{
	\begin{itemize}
	\item Unemployment dipped to 6.9 per cent last month.
	\item The president became more cautious as his popularity dipped.
	\item ...the current dip in farm spending.
	\end{itemize}
}
\item variable noun \\
A \textbf{dip} is a thick creamy sauce . You dip pieces of raw vegetable or biscuits into the sauce and then eat them.
 \textit{
	\begin{itemize}
	\item Maybe we could just buy some dips.
	\item ...prawns with avocado dip.
	\end{itemize}
}
\item countable noun \\
If you have or take a \textbf{dip} , you go for a quick swim in the sea, a river, or a swimming pool .
 \textit{
	\begin{itemize}
	\item She flicked through a romantic paperback between occasional dips in the pool.
	\end{itemize}
}
\item verb \\
If you are driving a car and \textbf{dip} the headlights, you operate a switch that makes them shine downwards, so that they do not shine directly into the eyes of other drivers .
 \textit{
	\begin{itemize}
	\item He dipped his headlights as they came up behind a slow-moving van.
	\item This picture shows the view from a car using normal dipped lights.
	\end{itemize}
}
\item verb \\
If you \textbf{dip into} a book, you have a brief look at it without reading or studying it seriously .
 \textit{
	\begin{itemize}
	\item ...a chance to dip into a wide selection of books on Buddhism.
	\end{itemize}
}
\item verb \\
If you \textbf{dip into} a sum of money that you had intended to save , you use some of it to buy something or pay for something.
 \textit{
	\begin{itemize}
	\item Just when she was ready to dip into her savings, Greg hastened to her rescue.
	\end{itemize}
}
\end{enumerate}

\section*{ecology}
{\large \color{blue}  ecologies  }
\subsection*{Explain}
\begin{enumerate}
\item uncountable noun \\
\textbf{Ecology} is the study of the relationships between plants, animals, people, and their environment,
and the balances between these relationships.
 \textit{
	\begin{itemize}
	\item ...a senior lecturer in ecology.
	\end{itemize}
}
\item variable noun \\
When you talk about the \textbf{ecology} of a place, you are referring to the pattern and balance of relationships between plants, animals, people, and the environment
in that place.
 \textit{
	\begin{itemize}
	\item ...the ecology of the rocky Negev desert in Israel.
	\item ...the extinction of the marshes' unique ecology.
	\item Global ecological efforts can easily be at odds with local ecologies.
	\end{itemize}
}
\end{enumerate}

\section*{dive}
{\large \color{blue}  dives  diving  dived  }
\subsection*{Explain}
\begin{enumerate}
\item verb \\
If you \textbf{dive}  \textbf{into} some water, you jump in head-first with your arms held straight above your head.
 \textbf{Dive} is also a noun .
 \textit{
	\begin{itemize}
	\item He tried to escape by diving into a river.
	\item She was standing by a pool, about to dive in.
	\item Joanne had just learnt to dive.
	\item Pat had earlier made a dive of 80 feet from the Chasm Bridge.
	\end{itemize}
}
\item verb \\
If you \textbf{dive} , you go under the surface of the sea or a lake, using special  breathing  equipment .
 \textbf{Dive} is also a noun.
 \textit{
	\begin{itemize}
	\item Bezanik is diving to collect marine organisms.
	\item This sighting occurred during my dive to a sunken wreck off Sardinia.
	\end{itemize}
}
\item verb \\
When birds and animals \textbf{dive} , they go quickly downwards , head-first, through the air or through water.
 \textit{
	\begin{itemize}
	\item ...a pelican which had just dived for a fish.
	\item The shark dived down and swam under the boat.
	\end{itemize}
}
\item verb \\
If an aeroplane  \textbf{dives} , it flies or drops down quickly and suddenly .
 \textbf{Dive} is also a noun.
 \textit{
	\begin{itemize}
	\item He was killed when his monoplane stalled and dived into the ground.
	\item Witnesses said the plane failed to pull out of a dive and smashed down in a field.
	\end{itemize}
}
\item verb \\
If you \textbf{dive} in a particular direction or into a particular place, you jump or move there quickly.
 \textbf{Dive} is also a noun.
 \textit{
	\begin{itemize}
	\item They dived into a taxi.
	\item The cashier dived for cover when a gunman opened fire.
	\item He would dive under one obstacle, round another, and lightly step over a third.
	\item He made a sudden dive for Uncle Jim's legs to try to trip him up.
	\end{itemize}
}
\item verb \\
If you \textbf{dive}  \textbf{into} a bag or container , you put your hands into it quickly in order to get something out.
 \textit{
	\begin{itemize}
	\item She dived into her bag and brought out a folded piece of paper.
	\end{itemize}
}
\item verb \\
If shares, profits , or figures  \textbf{dive} , their value falls suddenly and by a large amount.
 \textbf{Dive} is also a noun.
 \textit{
	\begin{itemize}
	\item If we cut interest rates, the pound would dive.
	\item Profits have dived from £7.7m to £7.1m.
	\item The shares dived 22p to 338p.
	\item Stock prices took a dive.
	\end{itemize}
}
\item countable noun \\
If you describe a bar or club as a \textbf{dive} , you mean it is dirty and dark, and not very respectable .
 \textit{
	\begin{itemize}
	\item We've played in all the little pubs and dives around Liverpool.
	\end{itemize}
}
\end{enumerate}

\section*{electrician}
{\large \color{blue}  electricians  }
\subsection*{Explain}
\begin{enumerate}
\item countable noun \\
An \textbf{electrician} is a person whose job is to install and repair electrical equipment .
 \textit{
	\begin{itemize}
	\end{itemize}
}
\end{enumerate}

\section*{divorce}
{\large \color{blue}  divorcés  }
\subsection*{Explain}
\begin{enumerate}
\item countable noun \\
A \textbf{divorcé} is a man who is divorced.
 \textit{
	\begin{itemize}
	\end{itemize}
}
\end{enumerate}

\section*{electricity}
{\large \color{blue}  }
\subsection*{Explain}
\begin{enumerate}
\item uncountable noun \\
\textbf{Electricity} is a form of energy that can be carried by wires and is used for heating and lighting, and to provide power for machines .
 \textit{
	\begin{itemize}
	\item We moved into a cabin with electricity but no running water.
	\item The electricity had been cut off.
	\end{itemize}
}
\end{enumerate}

\section*{doze}
{\large \color{blue}  dozes  dozing  dozed  }
\subsection*{Explain}
\begin{enumerate}
\item verb \\
When you \textbf{doze} , you sleep lightly or for a short period, especially during the daytime .
 \textbf{Doze} is also a noun .
 \textit{
	\begin{itemize}
	\item For a while she dozed fitfully.
	\item After lunch I had a doze.
	\end{itemize}
}
\end{enumerate}

\section*{electron}
{\large \color{blue}  electrons  }
\subsection*{Explain}
\begin{enumerate}
\item countable noun \\
An \textbf{electron} is a tiny particle of matter that is smaller than an atom and has a negative electrical charge.
 \textit{
	\begin{itemize}
	\end{itemize}
}
\end{enumerate}

\section*{dread}
{\large \color{blue}  dreads  dreading  dreaded  }
\subsection*{Explain}
\begin{enumerate}
\item verb \\
If you \textbf{dread} something which may  happen , you feel very anxious and unhappy about it because you think it will be unpleasant or upsetting .
 \textit{
	\begin{itemize}
	\item I'm dreading Christmas this year.
	\item I dreaded coming back, to be honest.
	\item I suffer badly from cold sores and dread them appearing on my wedding day.
	\item I'd been dreading that the birth would take a long time.
	\end{itemize}
}
\item uncountable noun \\
\textbf{Dread} is a feeling of great anxiety and fear about something that may happen.
 \textit{
	\begin{itemize}
	\item She thought with dread of the cold winters to come.
	\end{itemize}
}
\item adjective \\
\textbf{Dread}  means  terrible and greatly feared.
 \textit{
	\begin{itemize}
	\item ...a more effective national policy to combat this dread disease.
	\end{itemize}
}
\item adjective \\
You can use \textbf{dread} to describe something that you, or a particular group of people, find  annoying or undesirable .
 \textit{
	\begin{itemize}
	\item ...the dread phrase 'politically correct'.
	\end{itemize}
}
\item  \\
 dread to think \textit{
	\begin{itemize}
	\end{itemize}
}
\end{enumerate}

\section*{elevator}
{\large \color{blue}  elevators  }
\subsection*{Explain}
\begin{enumerate}
\item countable noun \\
An \textbf{elevator} is a device that carries people up and down inside buildings.
 \textit{
	\begin{itemize}
	\end{itemize}
}
\end{enumerate}

\section*{drift}
{\large \color{blue}  drifts  drifting  drifted  }
\subsection*{Explain}
\begin{enumerate}
\item verb \\
When something \textbf{drifts}  somewhere , it is carried there by the movement of wind or water.
 \textit{
	\begin{itemize}
	\item We proceeded to drift on up the river.
	\item The climbing balloon drifted silently over the countryside.
	\item The waves became rougher as they drifted.
	\end{itemize}
}
\item verb \\
If someone or something \textbf{drifts}  \textbf{into} a situation, they get into that situation in a way that is not planned or controlled.
 \textit{
	\begin{itemize}
	\item We need to offer young people drifting into crime an alternative set of values.
	\item She and her husband drifted apart and, eventually, they divorced.
	\item There is a general sense that the country and economy alike are drifting.
	\end{itemize}
}
\item verb \\
If you say that someone \textbf{drifts} around, you mean that they travel from place to place without a plan or settled way of life.
 \textit{
	\begin{itemize}
	\item You've been drifting from job to job without any real commitment.
	\end{itemize}
}
\item countable noun \\
A \textbf{drift} is a movement away from somewhere or something, or a movement towards somewhere or
something different.
 \textit{
	\begin{itemize}
	\item ...the drift towards the cities.
	\end{itemize}
}
\item verb \\
To \textbf{drift} somewhere means to move there slowly or gradually.
 \textit{
	\begin{itemize}
	\item As rural factories shed labour, people drift towards the cities.
	\end{itemize}
}
\item verb \\
If sounds \textbf{drift} somewhere, they can be heard but they are not very loud .
 \textit{
	\begin{itemize}
	\item Voices drifted to him across the parking lot.
	\end{itemize}
}
\item verb \\
If snow \textbf{drifts} , it builds up into piles as a result of the movement of the wind.
 \textit{
	\begin{itemize}
	\item The snow, except where it drifted, was only calf-deep.
	\item The storm caused severe drifting.
	\item ...the white and drifted snow.
	\end{itemize}
}
\item countable noun \\
A \textbf{drift} is a mass of snow that has built up into a pile as a result of the movement of wind.
 \textit{
	\begin{itemize}
	\item ...a nine-foot snow drift.
	\end{itemize}
}
\item countable noun \\
A \textbf{drift}  \textbf{of} something is an amount of it that has been created by the movement of wind or water.
 \textit{
	\begin{itemize}
	\item There was a drift of smoke above the trees.
	\end{itemize}
}
\item singular noun \\
\textbf{The}  \textbf{drift of} an argument or speech is the general point that is being made in it.
 \textit{
	\begin{itemize}
	\item Grace was beginning to get his drift.
	\item Anybody who's listening will get the drift of what he was saying.
	\item I follow the drift of her conversation.
	\end{itemize}
}
\end{enumerate}

\section*{failure}
{\large \color{blue}  failures  }
\subsection*{Explain}
\begin{enumerate}
\item uncountable noun \\
\textbf{Failure} is a lack of success in doing or achieving something, especially in relation to a particular activity .
 \textit{
	\begin{itemize}
	\item This policy is doomed to failure.
	\item Three attempts on the British 200-metre record also ended in failure.
	\item ...feelings of failure.
	\end{itemize}
}
\item countable noun \\
If something is \textbf{a}  \textbf{failure} , it is not a success.
 \textit{
	\begin{itemize}
	\item The marriage was a failure and they both wanted to be free of it.
	\item His six-year transition programme has by no means been a complete failure.
	\end{itemize}
}
\item countable noun \\
If you say that someone is \textbf{a}  \textbf{failure} , you mean that they have not succeeded in a particular activity, or that they are unsuccessful at everything they do.
 \textit{
	\begin{itemize}
	\item Elgar received many honors and much acclaim and yet he often considered himself a
failure.
	\item I just felt I had been a failure in my personal life.
	\end{itemize}
}
\item uncountable noun \\
Your \textbf{failure}  \textbf{to} do a particular thing is the fact that you do not do it, even though you were expected to do it.
 \textit{
	\begin{itemize}
	\item She accused the Foreign Office of disgraceful failure to support British citizens
arrested overseas.
	\item ...their failure to get the product mix right.
	\end{itemize}
}
\item variable noun \\
If there is a \textbf{failure} of something, for example a machine or part of the body, it goes  wrong and stops  working or developing properly.
 \textit{
	\begin{itemize}
	\item There were also several accidents mainly caused by engine failures on take-off.
	\item He was being treated for kidney failure.
	\item Researchers found an almost total crop failure and a severe shortage of drinking
water.
	\end{itemize}
}
\item variable noun \\
If there is a \textbf{failure} of a business or bank , it is no longer able to continue  operating .
 \textit{
	\begin{itemize}
	\item Business failures rose 16% last month.
	\end{itemize}
}
\item variable noun \\
If you say that someone has a \textbf{failure of} a particular quality or ability , you mean that they do not have enough of it.
 \textit{
	\begin{itemize}
	\item There is, too, a simple failure of imagination.
	\item He remained on his knees for a long time afterwards, ashamed by his failure of nerve.
	\end{itemize}
}
\end{enumerate}

\section*{envy}
{\large \color{blue}  envies  envying  envied  }
\subsection*{Explain}
\begin{enumerate}
\item uncountable noun \\
\textbf{Envy} is the feeling you have when you wish you could have the same thing or quality that someone else has.
 \textit{
	\begin{itemize}
	\item Gradually he began to acknowledge his feelings of envy towards his mother.
	\item They gazed in a mixture of envy and admiration at the beauty of the statue.
	\end{itemize}
}
\item verb \\
If you \textbf{envy} someone, you wish that you had the same things or qualities that they have.
 \textit{
	\begin{itemize}
	\item I don't envy the young ones who've become TV superstars and know no other world.
	\item I have a rich brother and a lot of people envy the fact.
	\item He envied Caroline her peace.
	\item 'You must've seen the world by now,' said Frannie, 'I envy you that.'
	\end{itemize}
}
\item singular noun \\
If a thing or quality is \textbf{the envy of} someone, they wish very much that they could have or achieve it.
 \textit{
	\begin{itemize}
	\item ...an economic expansion that was the envy of many other states.
	\end{itemize}
}
\end{enumerate}

\section*{field}
{\large \color{blue}  fields  fielding  fielded  }
\subsection*{Explain}
\begin{enumerate}
\item countable noun \\
A \textbf{field} is an area of grass, for example in a park or on a farm. A \textbf{field} is also an area of land on which a crop is grown.
 \textit{
	\begin{itemize}
	\item ...a field of wheat.
	\item They went for walks together in the fields.
	\end{itemize}
}
\item countable noun \\
A sports \textbf{field} is an area of grass where sports are played.
 \textit{
	\begin{itemize}
	\item ...a football field.
	\item He was the fastest thing I ever saw on a baseball field.
	\item The player was helped from the field with ankle injuries.
	\end{itemize}
}
\item countable noun \\
A \textbf{field} is an area of land or sea bed under which large amounts of a particular mineral have
been found.
 \textit{
	\begin{itemize}
	\item ...an extensive natural gas field in Alaska.
	\end{itemize}
}
\item countable noun \\
A magnetic, gravitational , or electric \textbf{field} is the area in which that particular force is strong enough to have an effect.
 \textit{
	\begin{itemize}
	\item Electromagnetic fields from electric power lines might increase the risk of cancer.
	\end{itemize}
}
\item countable noun \\
A particular \textbf{field} is a particular subject of study or type of activity.
 \textit{
	\begin{itemize}
	\item Exciting artistic breakthroughs have recently occurred in the fields of painting,
sculpture and architecture.
	\item She is said to be one of the leading experts in her field.
	\end{itemize}
}
\item countable noun \\
A \textbf{field} is an area of a computer's memory or a program where data can be entered, edited , or stored.
 \textit{
	\begin{itemize}
	\item Put your postcode into the website search field to find local support services.
	\end{itemize}
}
\item countable noun \\
You can refer to the area where fighting or other military action in a war takes place
as \textbf{the}  \textbf{field} or \textbf{the}  \textbf{field}  \textbf{of battle} .
 \textit{
	\begin{itemize}
	\item We never defeated them on the field of battle.
	\item ...the need for politicians to leave day-to-day decisions to commanders in the field.
	\end{itemize}
}
\item countable noun \\
Your \textbf{field} of vision or your visual  \textbf{field} is the area that you can see without turning your head.
 \textit{
	\begin{itemize}
	\item Our field of vision is surprisingly wide.
	\end{itemize}
}
\item countable noun \\
\textbf{The}  \textbf{field} is a way of referring to all the competitors taking part in a particular race or
sports contest.
 \textit{
	\begin{itemize}
	\item Going into the fourth lap, the two most broadly experienced riders led the field.
	\item The field were so close that they would have caught us if I hadn't begun the sprint.
	\item ...one of the strongest fields ever assembled for the Women's Bowling Association
championship.
	\end{itemize}
}
\item adjective \\
You use \textbf{field} to describe work or study that is done in a real, natural environment rather than
in a theoretical way or in controlled conditions.
 \textit{
	\begin{itemize}
	\item I also conducted a field study among the boys about their attitude to relationships.
	\item Our teachers took us on field trips to observe plants and animals, firsthand.
	\item The man offering help is a field worker.
	\end{itemize}
}
\item verb \\
In a game of cricket , baseball , or rounders , the team that \textbf{is fielding} is trying to catch the ball, while the other team is trying to hit it.
 \textit{
	\begin{itemize}
	\item When we are fielding, the umpires keep looking at the ball.
	\end{itemize}
}
\item verb \\
If you say that someone \textbf{fields} a question, you mean that they answer it or deal with it, usually successfully.
 \textit{
	\begin{itemize}
	\item He was later shown on television, fielding questions.
	\end{itemize}
}
\item verb \\
If a sports team \textbf{fields} a particular number or type of players, the players are chosen to play for the team on a particular occasion .
 \textit{
	\begin{itemize}
	\item England intend fielding their strongest team in next month's World Youth Championship.
	\end{itemize}
}
\item verb \\
If a candidate in an election is representing a political party, you can say that the party \textbf{is fielding} that candidate.
 \textit{
	\begin{itemize}
	\item The new party aims to field candidates in elections scheduled for next year.
	\end{itemize}
}
\item  \\
 have a field day \textit{
	\begin{itemize}
	\end{itemize}
}
\item  \\
 in the field \textit{
	\begin{itemize}
	\end{itemize}
}
\item  \\
 lead the field \textit{
	\begin{itemize}
	\end{itemize}
}
\item  \\
 to play the field \textit{
	\begin{itemize}
	\end{itemize}
}
\end{enumerate}

\section*{export}
{\large \color{blue}  exports  exporting  exported  }
\subsection*{Explain}
\begin{enumerate}
\item verb \\
To \textbf{export}  products or raw  materials  means to sell them to another country.
 \textbf{Export} is also a noun .
 \textit{
	\begin{itemize}
	\item The nation also exports rice.
	\item They expect the antibiotic products to be exported to Southeast Asia and Africa.
	\item To earn foreign exchange we must export.
	\item ...the production and export of cheap casual wear.
	\item A lot of our land is used to grow crops for export.
	\item ...illegal arms exports.
	\end{itemize}
}
\item countable noun \\
\textbf{Exports} are goods which are sold to another country and sent there.
 \textit{
	\begin{itemize}
	\item He did this to promote American exports.
	\item Ghana's main export is cocoa.
	\end{itemize}
}
\item verb \\
To \textbf{export} something means to introduce it into another country or make it happen there.
 \textit{
	\begin{itemize}
	\item It has exported inflation at times.
	\item ...hecklers who said the deal would export jobs to Mexico.
	\end{itemize}
}
\item verb \\
In computing , if you \textbf{export}  files or information from one type of software into another type, you change their format so that they can be used in the new software.
 \textit{
	\begin{itemize}
	\item Its maps can be exported in the GPX file format.
	\end{itemize}
}
\end{enumerate}

\section*{film}
{\large \color{blue}  films  filming  filmed  }
\subsection*{Explain}
\begin{enumerate}
\item countable noun \\
A \textbf{film} consists of moving pictures that have been recorded so that they can be shown at the cinema or on television.
A film tells a story , or shows a real  situation .
 \textit{
	\begin{itemize}
	\item Everything about the film was good. Good acting, good story, good fun.
	\item ...a government health film about the dangers of smoking.
	\end{itemize}
}
\item verb \\
If you \textbf{film} something, you use a camera to take moving pictures which can be shown on a screen
or on television.
 \textit{
	\begin{itemize}
	\item He had filmed her life story.
	\item Considering the restrictions under which she filmed, I think she did a commendable
job.
	\end{itemize}
}
\item uncountable noun \\
\textbf{Film} of something is moving pictures of a real event that are shown on television or on
a screen.
 \textit{
	\begin{itemize}
	\item They have seen news film of families queueing in Russia to buy a loaf of bread.
	\end{itemize}
}
\item variable noun \\
A \textbf{film} is the narrow  roll of plastic that is used in some cameras to take photographs.
 \textit{
	\begin{itemize}
	\item The photographers had already shot a dozen rolls of film.
	\end{itemize}
}
\item uncountable noun \\
The making of cinema films, considered as a form of art or a business, can be referred to as \textbf{film} or \textbf{films} .
 \textit{
	\begin{itemize}
	\item Film is a business with limited opportunities for actresses.
	\item She wanted to set up her own company to invest in films.
	\end{itemize}
}
\item countable noun \\
A \textbf{film}  \textbf{of}  powder , liquid, or oil is a very thin layer of it.
 \textit{
	\begin{itemize}
	\item The sea is coated with a film of raw sewage.
	\end{itemize}
}
\item uncountable noun \\
Plastic \textbf{film} is a very thin sheet of plastic used to wrap and cover things.
 \textit{
	\begin{itemize}
	\item Cover with plastic film and refrigerate for 24 hours.
	\end{itemize}
}
\end{enumerate}

\section*{forecast}
{\large \color{blue}  forecasts  forecasting  forecasted  }
\subsection*{Explain}
\begin{enumerate}
\item countable noun \\
A \textbf{forecast} is a statement of what is expected to happen in the future, especially in relation to a particular event or situation .
 \textit{
	\begin{itemize}
	\item ...a forecast of a 2.25 per cent growth in the economy.
	\item He delivered his election forecast.
	\item The weather forecast is better for today.
	\end{itemize}
}
\item verb \\
If you \textbf{forecast} future events, you say what you think is going to happen in the future.
 \textit{
	\begin{itemize}
	\item They forecast a humiliating defeat for the Prime Minister.
	\item He forecasts that average salary increases will remain around 4 per cent.
	\end{itemize}
}
\end{enumerate}

\section*{growth}
{\large \color{blue}  growths  }
\subsection*{Explain}
\begin{enumerate}
\item uncountable noun \\
The \textbf{growth}  \textbf{of} something such as an industry , organization , or idea is its development in size, wealth , or importance .
 \textit{
	\begin{itemize}
	\item ...the growth of nationalism.
	\item ...Japan's enormous economic growth.
	\item ...high growth rates.
	\end{itemize}
}
\item uncountable noun \\
A \textbf{growth} in something is an increase in it.
 \textit{
	\begin{itemize}
	\item A steady growth in the popularity of two smaller parties may upset the polls.
	\item The area has seen a rapid population growth.
	\item The market has shown annual growth of 20 per cent for several years.
	\end{itemize}
}
\item adjective \\
A \textbf{growth} industry, area, or market is one which is increasing in size or activity .
 \textit{
	\begin{itemize}
	\item Computers and electronics are growth industries and need skilled technicians.
	\item Real estate lending has become the biggest growth area for American banks.
	\end{itemize}
}
\item uncountable noun \\
Someone's \textbf{growth} is the development and progress of their character .
 \textit{
	\begin{itemize}
	\item ...the child's emotional and intellectual growth.
	\item Different teachers make different contributions to a student's growth.
	\end{itemize}
}
\item uncountable noun \\
\textbf{Growth} in a person, animal, or plant is the process of increasing in physical size and development.
 \textit{
	\begin{itemize}
	\item ...hormones which control fertility and body growth.
	\item Cells divide and renew as part of the human growth process.
	\end{itemize}
}
\item variable noun \\
You can use \textbf{growth} to refer to plants which have recently developed or which developed at the same time.
 \textit{
	\begin{itemize}
	\item This helps to ripen new growth and makes it flower profusely.
	\item Pinch out the tips of the young growths to make for compact, bushy plants.
	\end{itemize}
}
\item countable noun \\
A \textbf{growth} is a lump that grows inside or on a person, animal, or plant, and that is caused by a disease .
 \textit{
	\begin{itemize}
	\item This type of surgery could even be used to extract cancerous growths.
	\end{itemize}
}
\end{enumerate}

\section*{gaze}
{\large \color{blue}  gazes  gazing  gazed  }
\subsection*{Explain}
\begin{enumerate}
\item verb \\
If you \textbf{gaze}  \textbf{at} someone or something, you look steadily at them for a long time, for example because you find them attractive or interesting , or because you are thinking about something else.
 \textit{
	\begin{itemize}
	\item She stood gazing at herself in the mirror.
	\item Sitting in his wicker chair, he gazed reflectively at the fire.
	\end{itemize}
}
\item countable noun \\
You can talk about someone's \textbf{gaze} as a way of describing how they are looking at something, especially when they are looking steadily at it.
 \textit{
	\begin{itemize}
	\item The Monsignor turned his gaze from the flames to meet the Colonel's.
	\item She felt increasingly uncomfortable under the woman's steady gaze.
	\item The interior was shielded from the curious gaze of passersby.
	\end{itemize}
}
\item  \\
 public gaze \textit{
	\begin{itemize}
	\end{itemize}
}
\end{enumerate}

\section*{leaf}
{\large \color{blue}  leaves  leafs  leafing  leafed  }
\subsection*{Explain}
\begin{enumerate}
\item countable noun \\
The \textbf{leaves} of a tree or plant are the parts that are flat, thin, and usually green. Many trees
and plants lose their leaves in the winter and grow new leaves in the spring.
 \textit{
	\begin{itemize}
	\item In the garden, the leaves of the horse chestnut had already fallen.
	\item The Japanese maple that stands across the drive had just come into leaf.
	\end{itemize}
}
\item countable noun \\
A \textbf{leaf} is one of the pieces of paper of which a book is made.
 \textit{
	\begin{itemize}
	\item He flattened the wrappers and put them between the leaves of his book.
	\end{itemize}
}
\item  \\
 to take a leaf from someone's book \textit{
	\begin{itemize}
	\end{itemize}
}
\item  \\
 to turn over a new leaf \textit{
	\begin{itemize}
	\end{itemize}
}
\item  \\
 to shake like a leaf \textit{
	\begin{itemize}
	\end{itemize}
}
\end{enumerate}

\section*{grip}
{\large \color{blue}  grips  gripping  gripped  }
\subsection*{Explain}
\begin{enumerate}
\item verb \\
If you \textbf{grip} something, you take hold of it with your hand and continue to hold it firmly.
 \textit{
	\begin{itemize}
	\item She gripped the rope.
	\end{itemize}
}
\item countable noun \\
A \textbf{grip} is a firm , strong hold on something.
 \textit{
	\begin{itemize}
	\item His strong hand eased the bag from her grip.
	\end{itemize}
}
\item singular noun \\
Someone's \textbf{grip}  \textbf{on} something is the power and control they have over it.
 \textit{
	\begin{itemize}
	\item The president maintains an iron grip on his country.
	\item The Labour leader last night tightened his grip on Labour MPs with new powers to
root out trouble-makers.
	\end{itemize}
}
\item verb \\
If something \textbf{grips} you, it affects you very strongly.
 \textit{
	\begin{itemize}
	\item Pain gripped him.
	\item The entire community has been gripped by fear.
	\end{itemize}
}
\item verb \\
If you \textbf{are gripped}  \textbf{by} something such as a story or a series of events, your attention is concentrated on it and held by it.
 \textit{
	\begin{itemize}
	\item The nation is gripped by the dramatic story.
	\end{itemize}
}
\item uncountable noun \\
If things such as shoes or car  tyres have \textbf{grip} , they do not slip .
 \textit{
	\begin{itemize}
	\item ...a new way of reinforcing rubber which gives car tyres better grip.
	\end{itemize}
}
\item countable noun \\
A \textbf{grip} is a bag that is smaller than a suitcase , and that you use when you are travelling.
 \textit{
	\begin{itemize}
	\end{itemize}
}
\item  \\
 get/come to grips with \textit{
	\begin{itemize}
	\end{itemize}
}
\item  \\
 get a grip \textit{
	\begin{itemize}
	\end{itemize}
}
\item  \\
 in the grip of sth \textit{
	\begin{itemize}
	\end{itemize}
}
\item  \\
 to lose your grip \textit{
	\begin{itemize}
	\end{itemize}
}
\item  \\
 a grip on reality \textit{
	\begin{itemize}
	\end{itemize}
}
\end{enumerate}

\section*{life}
{\large \color{blue}  lives  }
\subsection*{Explain}
\begin{enumerate}
\item uncountable noun \\
\textbf{Life} is the quality which people, animals, and plants have when they are not dead, and
which objects and substances do not have.
 \textit{
	\begin{itemize}
	\item ...a baby's first minutes of life.
	\item Amnesty International opposes the death penalty as a violation of the right to life.
	\item ...the earth's supply of life-giving oxygen.
	\end{itemize}
}
\item uncountable noun \\
You can use \textbf{life} to refer to things or groups of things which are alive .
 \textit{
	\begin{itemize}
	\item Is there life on Mars?
	\item The book includes some useful facts about animal and plant life.
	\end{itemize}
}
\item countable noun \\
If you refer to someone's \textbf{life} , you mean their state of being alive, especially when there is a risk or danger of them dying.
 \textit{
	\begin{itemize}
	\item Your life is in danger.
	\item A nurse began to try to save his life.
	\item The intense fighting is reported to have claimed many lives.
	\end{itemize}
}
\item countable noun \\
Someone's \textbf{life} is the period of time during which they are alive.
 \textit{
	\begin{itemize}
	\item He spent the last fourteen years of his life in retirement.
	\item For the first time in his life he regretted that he had no faith.
	\end{itemize}
}
\item countable noun \\
You can use \textbf{life} to refer to a period of someone's life when they are in a particular situation or
job.
 \textit{
	\begin{itemize}
	\item Interior designers spend their working lives keeping up to date with the latest trends.
	\item That was the beginning of my life in the television business.
	\end{itemize}
}
\item countable noun \\
You can use \textbf{life} to refer to particular activities which people regularly do during their lives.
 \textit{
	\begin{itemize}
	\item My personal life has had to take second place to my career.
	\item Most diabetics have a normal sex life.
	\end{itemize}
}
\item uncountable noun \\
You can use \textbf{life} to refer to the events and experiences that happen to people while they are alive.
 \textit{
	\begin{itemize}
	\item Life won't be dull!
	\item It's the people with insecurities who make life difficult.
	\item ...the sort of life we can only fantasise about living.
	\end{itemize}
}
\item uncountable noun \\
If you know a lot about \textbf{life} , you have gained many varied experiences, for example by travelling a lot and meeting different kinds of people.
 \textit{
	\begin{itemize}
	\item I was 19 and too young to know much about life.
	\item I needed some time off from education to experience life.
	\end{itemize}
}
\item uncountable noun \\
You can use \textbf{life} to refer to the things that people do and experience that are characteristic of a
particular place, group, or activity.
 \textit{
	\begin{itemize}
	\item How did you adjust to college life?
	\item ...the culture and life of north Africa.
	\item He abhors the wheeling-and-dealing associated with conventional political life.
	\end{itemize}
}
\item uncountable noun \\
A person, place, book, or film that is full of \textbf{life} gives an impression of excitement , energy, or cheerfulness.
 \textit{
	\begin{itemize}
	\item The town itself was full of life and character.
	\item The rejection of the Jewish theme meant the rejection of everything that gave the
script passion and life.
	\item He's sucked the life out of her.
	\end{itemize}
}
\item countable noun \\
A \textbf{life} of a person is a book or film which tells the story of their life.
 \textit{
	\begin{itemize}
	\item A life of John Paul Jones had long interested him.
	\end{itemize}
}
\item uncountable noun \\
If someone is sentenced to \textbf{life} , they are sentenced to stay in prison for the rest of their life or for a very long time.
 \textit{
	\begin{itemize}
	\item He could get life in prison, if convicted.
	\end{itemize}
}
\item countable noun \\
The \textbf{life} of something such as a machine, organization, or project is the period of time that
it lasts for.
 \textit{
	\begin{itemize}
	\item The repairs did not increase the value or the life of the equipment.
	\end{itemize}
}
\item uncountable noun \\
In art, \textbf{life} refers to the producing of drawings, paintings, or sculptures that represent actual people, objects, or places, rather than images from the artist's imagination .
 \textit{
	\begin{itemize}
	\item ...learning to draw from life.
	\item She had once posed for Life classes when she was an art student.
	\end{itemize}
}
\item  \\
 be one's life \textit{
	\begin{itemize}
	\end{itemize}
}
\item  \\
 bring sth to life/come to life \textit{
	\begin{itemize}
	\end{itemize}
}
\item  \\
 come to life \textit{
	\begin{itemize}
	\end{itemize}
}
\item  \\
 life after death \textit{
	\begin{itemize}
	\end{itemize}
}
\item  \\
 fight for one's life \textit{
	\begin{itemize}
	\end{itemize}
}
\item  \\
 for life \textit{
	\begin{itemize}
	\end{itemize}
}
\item  \\
 for the life of me \textit{
	\begin{itemize}
	\end{itemize}
}
\item  \\
 for one's life/for dear life \textit{
	\begin{itemize}
	\end{itemize}
}
\item  \\
 live life to the full \textit{
	\begin{itemize}
	\end{itemize}
}
\item  \\
 get a life \textit{
	\begin{itemize}
	\end{itemize}
}
\item  \\
 life goes on \textit{
	\begin{itemize}
	\end{itemize}
}
\item  \\
 have a life \textit{
	\begin{itemize}
	\end{itemize}
}
\item  \\
 in sb's life \textit{
	\begin{itemize}
	\end{itemize}
}
\item  \\
 in all my life \textit{
	\begin{itemize}
	\end{itemize}
}
\item  \\
 the fright of your life \textit{
	\begin{itemize}
	\end{itemize}
}
\item  \\
 larger than life \textit{
	\begin{itemize}
	\end{itemize}
}
\item  \\
 to lay down your life \textit{
	\begin{itemize}
	\end{itemize}
}
\item  \\
 to risk life and limb \textit{
	\begin{itemize}
	\end{itemize}
}
\item  \\
 new life \textit{
	\begin{itemize}
	\end{itemize}
}
\item  \\
 not on your life \textit{
	\begin{itemize}
	\end{itemize}
}
\item  \\
 live one's own life \textit{
	\begin{itemize}
	\end{itemize}
}
\item  \\
 rule one's life \textit{
	\begin{itemize}
	\end{itemize}
}
\item  \\
 to save one's life \textit{
	\begin{itemize}
	\end{itemize}
}
\item  \\
 the life and soul of the party \textit{
	\begin{itemize}
	\end{itemize}
}
\item  \\
 start life \textit{
	\begin{itemize}
	\end{itemize}
}
\item  \\
 take sb's life \textit{
	\begin{itemize}
	\end{itemize}
}
\item  \\
 that's life \textit{
	\begin{itemize}
	\end{itemize}
}
\item  \\
 come to life/spring to life/roar into life \textit{
	\begin{itemize}
	\end{itemize}
}
\item  \\
 what a life \textit{
	\begin{itemize}
	\end{itemize}
}
\item  \\
 life isn't worth living/makes life worth living \textit{
	\begin{itemize}
	\end{itemize}
}
\end{enumerate}

\section*{guess}
{\large \color{blue}  guesses  guessing  guessed  }
\subsection*{Explain}
\begin{enumerate}
\item verb \\
If you \textbf{guess} something, you give an answer or provide an opinion which may not be true because you do not have definite  knowledge about the matter  concerned .
 \textit{
	\begin{itemize}
	\item The suit was faultless: Wood guessed that he was a very successful publisher or a
banker.
	\item You can only guess at what mental suffering they endure.
	\item Paula reached for her camera, guessed distance and exposure, and shot two frames.
	\item Guess what I did for the whole of the first week.
	\item If she guessed wrong, it meant twice as many meetings the following week.
	\end{itemize}
}
\item verb \\
If you \textbf{guess}  \textbf{that} something is the case , you correctly form the opinion that it is the case, although you do not have definite
knowledge about it.
 \textit{
	\begin{itemize}
	\item By now you will have guessed that I'm back in Ireland.
	\item As you've probably guessed, the problem was electrical.
	\item He should have guessed what would happen.
	\item Someone might have guessed our secret and passed it on.
	\end{itemize}
}
\item countable noun \\
A \textbf{guess} is an attempt to give an answer or provide an opinion which may not be true because you do not
have definite knowledge about the matter concerned.
 \textit{
	\begin{itemize}
	\item My guess is that the chance that these vaccines will work is zero.
	\item He'd taken her pulse and made a guess at her blood pressure.
	\item Well, we can hazard a guess at the answer.
	\end{itemize}
}
\item  \\
 anyone's guess/anybody's guess \textit{
	\begin{itemize}
	\end{itemize}
}
\item  \\
 at a guess \textit{
	\begin{itemize}
	\end{itemize}
}
\item  \\
 I guess \textit{
	\begin{itemize}
	\end{itemize}
}
\item  \\
 keep someone guessing \textit{
	\begin{itemize}
	\end{itemize}
}
\item  \\
 guess what \textit{
	\begin{itemize}
	\end{itemize}
}
\end{enumerate}

\section*{movie}
{\large \color{blue}  movies  }
\subsection*{Explain}
\begin{enumerate}
\item countable noun \\
A \textbf{movie} is a film.
 \textit{
	\begin{itemize}
	\item In the first movie Tony Curtis ever made he played a grocery clerk.
	\item ...a horror movie.
	\end{itemize}
}
\item plural noun \\
You can talk about \textbf{the movies} when you are talking about seeing a movie in a movie theater.
 \textit{
	\begin{itemize}
	\item He took her to the movies.
	\end{itemize}
}
\end{enumerate}

\section*{hug}
{\large \color{blue}  hugs  hugging  hugged  }
\subsection*{Explain}
\begin{enumerate}
\item verb \\
When you \textbf{hug} someone, you put your arms around them and hold them tightly, for example because you like them or are pleased to see them. You can also  say that two people \textbf{hug} each other or that they \textbf{hug} .
 \textbf{Hug} is also a noun .
 \textit{
	\begin{itemize}
	\item She had hugged him exuberantly and invited him to dinner the next day.
	\item They hugged each other like a couple of lost children.
	\item We hugged and kissed.
	\item Syvil leapt out of the back seat, and gave him a hug.
	\end{itemize}
}
\item verb \\
If you \textbf{hug} something, you hold it close to your body with your arms tightly round it.
 \textit{
	\begin{itemize}
	\item Shaerl trudged toward them, hugging a large box.
	\item She hugged her legs tight to her chest.
	\item She stood hugging her quilted jacket round her.
	\end{itemize}
}
\item verb \\
Something that \textbf{hugs} the ground or a stretch of land or water stays very close to it.
 \textit{
	\begin{itemize}
	\item The road hugs the coast for hundreds of miles.
	\item Our pilot reduced height until we hugged the ground.
	\end{itemize}
}
\end{enumerate}

\section*{organism}
{\large \color{blue}  organisms  }
\subsection*{Explain}
\begin{enumerate}
\item countable noun \\
An \textbf{organism} is an animal or plant, especially one that is so small that you cannot see it without using a microscope .
 \textit{
	\begin{itemize}
	\item Not all chemicals normally present in living organisms are harmless.
	\item ...the insect-borne organisms that cause sleeping sickness.
	\end{itemize}
}
\end{enumerate}

\section*{increase}
{\large \color{blue}  increases  increasing  increased  }
\subsection*{Explain}
\begin{enumerate}
\item verb \\
If something \textbf{increases} or you \textbf{increase} it, it becomes greater in number, level , or amount.
 \textit{
	\begin{itemize}
	\item The population continues to increase.
	\item Japan's industrial output increased by 2%.
	\item The company has increased the price of its cars.
	\item The increased investment will help stabilise the economy.
	\item We are experiencing an increasing number of problems.
	\end{itemize}
}
\item countable noun \\
If there is an \textbf{increase}  \textbf{in} the number, level, or amount of something, it becomes greater.
 \textit{
	\begin{itemize}
	\item ...a sharp increase in productivity.
	\item He called for an increase of 1p on income tax.
	\item ...an increase of violence along the border.
	\end{itemize}
}
\item  \\
 on the increase \textit{
	\begin{itemize}
	\end{itemize}
}
\end{enumerate}

\section*{prisoner}
{\large \color{blue}  prisoners  }
\subsection*{Explain}
\begin{enumerate}
\item countable noun \\
A \textbf{prisoner} is a person who is kept in a prison as a punishment for a crime that they have committed .
 \textit{
	\begin{itemize}
	\item The committee is concerned about the large number of prisoners sharing cells.
	\end{itemize}
}
\item countable noun \\
A \textbf{prisoner} is a person who has been captured by an enemy , for example in war.
 \textit{
	\begin{itemize}
	\item ...wartime hostages and concentration-camp prisoners.
	\item He was held prisoner in Vietnam from 1966 to 1973.
	\item He was taken prisoner in North Africa in 1942.
	\end{itemize}
}
\item countable noun \\
If you say that you are a \textbf{prisoner of} a situation , you mean that you are trapped by it.
 \textit{
	\begin{itemize}
	\item We are all prisoners of our childhood and feel an obligation to it.
	\item She was a prisoner of her own ego.
	\end{itemize}
}
\end{enumerate}

\section*{interview}
{\large \color{blue}  interviews  interviewing  interviewed  }
\subsection*{Explain}
\begin{enumerate}
\item variable noun \\
An \textbf{interview} is a formal meeting at which someone is asked questions in order to find out if they are suitable for a job or a course of study .
 \textit{
	\begin{itemize}
	\item When I went for my first interview for this job I arrived extremely early.
	\item The interview went well.
	\item Not everyone who writes in can be invited for interview.
	\end{itemize}
}
\item verb \\
If you \textbf{are interviewed} for a particular job or course of study, someone asks you questions about yourself to find out if
you are suitable for it.
 \textit{
	\begin{itemize}
	\item When Wardell was interviewed, he was impressive, and on that basis, he was hired.
	\item He was among the three candidates interviewed for the job.
	\end{itemize}
}
\item countable noun \\
An \textbf{interview} is a conversation in which a journalist  puts questions to someone such as a famous person or politician .
 \textit{
	\begin{itemize}
	\item Allan gave an interview to the Chicago Tribune newspaper last month.
	\item There'll be an interview with Mr Brown after the news.
	\end{itemize}
}
\item verb \\
When a journalist \textbf{interviews} someone such as a famous person, they ask them a series of questions.
 \textit{
	\begin{itemize}
	\item I seized the chance to interview Chris Hani about this issue.
	\end{itemize}
}
\item verb \\
When the police  \textbf{interview} someone, they ask them questions about a crime that has been committed .
 \textit{
	\begin{itemize}
	\item The police interviewed the driver, but had no evidence to go on.
	\end{itemize}
}
\end{enumerate}

\section*{production}
{\large \color{blue}  productions  }
\subsection*{Explain}
\begin{enumerate}
\item uncountable noun \\
\textbf{Production} is the process of manufacturing or growing something in large quantities .
 \textit{
	\begin{itemize}
	\item That model won't go into production until next year.
	\item ...tax incentives to encourage domestic production of oil.
	\end{itemize}
}
\item uncountable noun \\
\textbf{Production} is the amount of goods manufactured or grown by a company or country.
 \textit{
	\begin{itemize}
	\item We needed to increase the volume of production.
	\item It expected to maintain production of cars at the same level as last year.
	\end{itemize}
}
\item uncountable noun \\
The \textbf{production of} something is its creation as the result of a natural process.
 \textit{
	\begin{itemize}
	\item These proteins stimulate the production of blood cells.
	\end{itemize}
}
\item uncountable noun \\
\textbf{Production} is the process of organizing and preparing a play, film, programme , or CD, in order to present it to the public.
 \textit{
	\begin{itemize}
	\item During the film's production, the director wanted to shoot a riot scene.
	\item She is head of the production company.
	\end{itemize}
}
\item countable noun \\
A \textbf{production} is a play, opera, or other show that is performed in a theatre .
 \textit{
	\begin{itemize}
	\item For this production she has learnt the role in Spanish.
	\item ...a critically acclaimed production of Othello.
	\end{itemize}
}
\item  \\
 on production of something/on the production of something \textit{
	\begin{itemize}
	\end{itemize}
}
\end{enumerate}

\section*{jump}
{\large \color{blue}  jumps  jumping  jumped  }
\subsection*{Explain}
\begin{enumerate}
\item verb \\
If you \textbf{jump} , you bend your knees , push against the ground with your feet, and move quickly upwards into the air.
 \textbf{Jump} is also a noun.
 \textit{
	\begin{itemize}
	\item I jumped over the fence.
	\item They came into the front hall, jumping up and down to knock the snow off their boots.
	\item I'd jumped seventeen feet six in the long jump, which was a school record.
	\item Whoever heard of a basketball player who doesn't need to jump?
	\item She was taking tiny jumps in her excitement.
	\end{itemize}
}
\item verb \\
If you \textbf{jump} from something above the ground, you deliberately push yourself into the air so that
you drop towards the ground.
 \textit{
	\begin{itemize}
	\item He jumped out of a third-floor window.
	\item She has jumped from an aeroplane four times.
	\item I jumped the last six feet down to the deck.
	\end{itemize}
}
\item verb \\
If you \textbf{jump} something such as a fence , you move quickly up and through the air over or across it.
 \textit{
	\begin{itemize}
	\item He jumped the first fence beautifully.
	\end{itemize}
}
\item verb \\
If you \textbf{jump}  somewhere , you move there quickly and suddenly.
 \textit{
	\begin{itemize}
	\item Adam jumped from his seat at the girl's cry.
	\item She jumped to her feet and ran downstairs.
	\item 'I'll do it, Eleanor,' Angus said, jumping up.
	\end{itemize}
}
\item verb \\
If something \textbf{makes} you \textbf{jump} , it makes you make a sudden movement because you are frightened or surprised.
 \textit{
	\begin{itemize}
	\item The phone shrilled, making her jump.
	\end{itemize}
}
\item verb \\
If an amount or level \textbf{jumps} , it suddenly increases or rises by a large amount in a short time.
 \textbf{Jump} is also a noun.
 \textit{
	\begin{itemize}
	\item Sales jumped from $94 million to over $101 million.
	\item The number of crimes jumped by ten per cent last year.
	\item Shares in Euro Disney jumped 17p.
	\item ...a big jump in energy conservation.
	\end{itemize}
}
\item verb \\
If someone \textbf{jumps} a queue , they move to the front of it and are served or dealt with before it is their turn.
 \textit{
	\begin{itemize}
	\item The prince refused to jump the queue for treatment at the local hospital.
	\end{itemize}
}
\item verb \\
If someone \textbf{jumps}  \textbf{on} you or \textbf{jumps} you, they attack you suddenly.
 \textit{
	\begin{itemize}
	\item A week later, the same guys jumped on me on our own front lawn.
	\item Two guys jumped me with clubs in the carpark.
	\end{itemize}
}
\item verb \\
If you \textbf{jump at} an offer or opportunity , you accept it quickly and eagerly.
 \textit{
	\begin{itemize}
	\item Members of the public would jump at the chance to become part owners of the corporation.
	\end{itemize}
}
\item verb \\
If someone \textbf{jumps on} you, they quickly criticize you if you do something that they do not approve of.
 \textit{
	\begin{itemize}
	\item A lot of people jumped on me about that, you know.
	\end{itemize}
}
\item verb \\
If someone \textbf{jumps} you, they attack you suddenly or unexpectedly.
 \textit{
	\begin{itemize}
	\item Half a dozen sailors jumped him.
	\end{itemize}
}
\item  \\
 get a jump on/get the jump on \textit{
	\begin{itemize}
	\end{itemize}
}
\item  \\
 jump up and down \textit{
	\begin{itemize}
	\end{itemize}
}
\end{enumerate}

\section*{productivity}
{\large \color{blue}  }
\subsection*{Explain}
\begin{enumerate}
\item uncountable noun \\
\textbf{Productivity} is the rate at which goods are produced.
 \textit{
	\begin{itemize}
	\item The third-quarter results reflect continued improvements in productivity.
	\item His method of obtaining a high level of productivity is demanding.
	\end{itemize}
}
\end{enumerate}

\section*{lag}
{\large \color{blue}  lags  lagging  lagged  }
\subsection*{Explain}
\begin{enumerate}
\item verb \\
If one thing or person \textbf{lags}  \textbf{behind} another thing or person, their progress is slower than that of the other.
 \textit{
	\begin{itemize}
	\item Britain still lags behind most of Europe in its provisions for women who want time
off to have babies.
	\item The restructuring of the pattern of consumption in Britain also lagged behind.
	\item He now lags 10 points behind the champion.
	\item They are lagging a point behind their rivals.
	\item Hague was lagging badly in the polls.
	\end{itemize}
}
\item countable noun \\
A time \textbf{lag} or a \textbf{lag} of a particular length of time is a period of time between one event and another
related event.
 \textit{
	\begin{itemize}
	\item There's a time lag between infection with HIV and developing AIDS.
	\item Price rises have matched rises in the money supply with a lag of two or three months.
	\end{itemize}
}
\item verb \\
If you \textbf{lag} the inside of a roof , a pipe, or a water tank , you cover it with a special material in order to prevent heat escaping from it or to prevent it from freezing .
 \textit{
	\begin{itemize}
	\item If you have to take the floorboards up, take the opportunity to lag any pipes at
the same time.
	\item Water tanks should be well lagged and the roof well insulated.
	\end{itemize}
}
\end{enumerate}

\section*{quarter}
{\large \color{blue}  quarters  quartering  quartered  }
\subsection*{Explain}
\begin{enumerate}
\item fraction \\
A \textbf{quarter} is one of four equal parts of something.
 \textbf{Quarter} is also a predeterminer.
 \textbf{Quarter} is also an adjective .
 \textit{
	\begin{itemize}
	\item A quarter of the residents are over 55 years old.
	\item I've got to go in a quarter of an hour.
	\item Prices have fallen by a quarter since January.
	\item Cut the peppers into quarters.
	\item The largest asteroid is Ceres, which is about a quarter the size of the moon.
	\item ...the past quarter century.
	\item He closed his door and started the quarter-mile walk down the hill.
	\end{itemize}
}
\item countable noun \\
A \textbf{quarter} is a fixed period of three months. Companies often divide their financial year into four quarters.
 \textit{
	\begin{itemize}
	\item The group said results for the third quarter are due on October 29.
	\end{itemize}
}
\item uncountable noun \\
When you are telling the time, you use \textbf{quarter} to talk about the fifteen  minutes before or after an hour . For example, 8.15 is \textbf{quarter past}  eight , and 8.45 is \textbf{quarter to}  nine . In American English, you can also say that 8.15 is a \textbf{quarter after} eight and 8.45 is a \textbf{quarter of} nine.
 \textit{
	\begin{itemize}
	\item It was quarter to six.
	\item See you about quarter to nine.
	\item I got a call at a quarter of seven one night.
	\item Nobody else turned up till a quarter past ten.
	\item The time was recorded at a quarter after five.
	\end{itemize}
}
\item verb \\
If you \textbf{quarter} something such as a fruit or a vegetable, you cut it into four roughly equal parts.
 \textit{
	\begin{itemize}
	\item Chop the mushrooms and quarter the tomatoes.
	\end{itemize}
}
\item verb \\
If the number or size of something \textbf{is quartered} , it is reduced to about a quarter of its previous number or size.
 \textit{
	\begin{itemize}
	\item The doses I suggested for adults could be halved or quartered.
	\end{itemize}
}
\item countable noun \\
A \textbf{quarter} is an American or Canadian coin that is worth 25 cents.
 \textit{
	\begin{itemize}
	\item I dropped a quarter into the slot of the pay phone.
	\end{itemize}
}
\item countable noun \\
A particular \textbf{quarter} of a town is a part of the town where a particular group of people traditionally
live or work.
 \textit{
	\begin{itemize}
	\item Look for hotels in the French Quarter.
	\end{itemize}
}
\item countable noun \\
To refer to a person or group you may not want to name, you can talk about the reactions or actions from a particular \textbf{quarter} .
 \textit{
	\begin{itemize}
	\item Help came from an unexpected quarter.
	\item The pharmaceuticals industry is regarded with scepticism in some quarters.
	\end{itemize}
}
\item plural noun \\
The rooms provided for soldiers, sailors , or servants to live in are called their \textbf{quarters} .
 \textit{
	\begin{itemize}
	\item Mckinnon went down from deck to the officers' quarters.
	\end{itemize}
}
\item verb \\
If people \textbf{are quartered}  somewhere , they are provided with accommodation for a short time, usually while they are working away from home.
 \textit{
	\begin{itemize}
	\item Our soldiers are quartered in Peredelkino.
	\end{itemize}
}
\item  \\
 at close quarters \textit{
	\begin{itemize}
	\end{itemize}
}
\item  \\
 no quarter \textit{
	\begin{itemize}
	\end{itemize}
}
\end{enumerate}

\section*{mention}
{\large \color{blue}  mentions  mentioning  mentioned  }
\subsection*{Explain}
\begin{enumerate}
\item verb \\
If you \textbf{mention} something, you say something about it, usually briefly.
 \textit{
	\begin{itemize}
	\item She did not mention her mother's absence.
	\item I may not have mentioned it to her.
	\item I had mentioned that I didn't really like contemporary music.
	\item She shouldn't have mentioned how heavy the dress was.
	\item I felt as though I should mention it as an option.
	\end{itemize}
}
\item variable noun \\
A \textbf{mention} is a reference to something or someone.
 \textit{
	\begin{itemize}
	\item The statement made no mention of government casualties.
	\item At the community centre, mention of funds produces pained looks.
	\end{itemize}
}
\item verb \\
If someone \textbf{is mentioned} in writing , a reference is made to them by name, often to criticize or praise something that they have done.
 \textit{
	\begin{itemize}
	\item I was absolutely outraged that I could be even mentioned in an article of this kind.
	\item As for your father, he won't be mentioned in my will.
	\item ...Brigadier Ferguson was mentioned in the report as being directly responsible.
	\end{itemize}
}
\item verb \\
If someone \textbf{is mentioned}  \textbf{as} a candidate for something such as a job , it is suggested that they might become a candidate.
 \textit{
	\begin{itemize}
	\item His appointment is a complete surprise–he has never been mentioned as a front runner.
	\item Her name has been mentioned as a favoured leadership candidate.
	\end{itemize}
}
\item variable noun \\
A special or honourable  \textbf{mention} is formal praise that is given for an achievement that is very good, although not usually the best of its kind .
 \textit{
	\begin{itemize}
	\item Many people have helped me but I would like to pick out a few for special mention.
	\end{itemize}
}
\item  \\
 don't mention it \textit{
	\begin{itemize}
	\end{itemize}
}
\item  \\
 not to mention \textit{
	\begin{itemize}
	\end{itemize}
}
\end{enumerate}

\section*{rap}
{\large \color{blue}  raps  rapping  rapped  }
\subsection*{Explain}
\begin{enumerate}
\item uncountable noun \\
\textbf{Rap} is a type of music in which the words are not sung but are spoken in a rapid, rhythmic way.
 \textit{
	\begin{itemize}
	\item For some people, rap–the music of the hip-hop generation–is just so much noise.
	\item ...a rap group.
	\end{itemize}
}
\item verb \\
Someone who \textbf{raps} performs rap music.
 \textit{
	\begin{itemize}
	\item ...the unexpected pleasure of hearing the band not only rap but even sing.
	\end{itemize}
}
\item countable noun \\
A \textbf{rap} is a piece of music performed in rap style, or the words that are used in it.
 \textit{
	\begin{itemize}
	\item Every member contributes to the rap, singing either solo or as part of a rap chorus.
	\end{itemize}
}
\item verb \\
If you \textbf{rap}  \textbf{on} something or \textbf{rap} it, you hit it with a series of quick blows.
 \textbf{Rap} is also a noun .
 \textit{
	\begin{itemize}
	\item Mary Ann turned and rapped on Simon's door.
	\item ...rapping the glass with the knuckles of his right hand.
	\item A guard raps his stick on a metal hand rail.
	\item There was a sharp rap on the door.
	\end{itemize}
}
\item countable noun \\
A \textbf{rap} is a statement in a court of law that someone has committed a particular crime , or the punishment for committing it.
 \textit{
	\begin{itemize}
	\item You'll be facing a rap for aiding and abetting an escaped convict.
	\end{itemize}
}
\item countable noun \\
A \textbf{rap} is an act of criticizing or blaming someone.
 \textit{
	\begin{itemize}
	\item FA chiefs could still face a rap and a possible fine.
	\item Timeshare companies also come in for a rap as they continue to flout the rules.
	\end{itemize}
}
\item verb \\
If you \textbf{rap} someone \textbf{for} something, you criticize or blame them for it.
 \textit{
	\begin{itemize}
	\item Water industry chiefs were rapped yesterday for failing their customers.
	\item The minister rapped banks over their treatment of small businesses.
	\end{itemize}
}
\item singular noun \\
\textbf{The}  \textbf{rap} about someone or something is their reputation , often a bad reputation which they do not deserve .
 \textit{
	\begin{itemize}
	\item The rap against Conn was that he was far too reckless.
	\item The rap on this guy is that he doesn't really care.
	\item He said statisticians gave them a bad rap by 'lying with figures'.
	\end{itemize}
}
\item verb \\
If you \textbf{rap}  \textbf{with} someone \textbf{about} something, you talk about it in a relaxed or informal way.
 \textit{
	\begin{itemize}
	\item Today we are going to rap about relationships.
	\end{itemize}
}
\item  \\
 rap sb's knuckles, rap sb on/over the knuckles \textit{
	\begin{itemize}
	\end{itemize}
}
\item  \\
 a rap on the knuckles \textit{
	\begin{itemize}
	\end{itemize}
}
\item  \\
 take the rap \textit{
	\begin{itemize}
	\end{itemize}
}
\item  \\
 beat the rap \textit{
	\begin{itemize}
	\end{itemize}
}
\end{enumerate}

\section*{murder}
{\large \color{blue}  murders  murdering  murdered  }
\subsection*{Explain}
\begin{enumerate}
\item variable noun \\
\textbf{Murder} is the deliberate and illegal killing of a person.
 \textit{
	\begin{itemize}
	\item The three accused, aged between 19 and 20, are charged with attempted murder.
	\item She refused to testify, unless the murder charge against her was dropped.
	\item ...brutal murders.
	\end{itemize}
}
\item verb \\
To \textbf{murder} someone means to commit the crime of killing them deliberately.
 \textit{
	\begin{itemize}
	\item ...a thriller about two men who murder a third to see if they can get away with it.
	\item ...the body of a murdered religious and political leader.
	\end{itemize}
}
\item  \\
 get away with murder \textit{
	\begin{itemize}
	\end{itemize}
}
\item  \\
 blue murder \textit{
	\begin{itemize}
	\end{itemize}
}
\end{enumerate}

\section*{sigh}
{\large \color{blue}  sighs  sighing  sighed  }
\subsection*{Explain}
\begin{enumerate}
\item verb \\
When you \textbf{sigh} , you let out a deep breath, as a way of expressing feelings such as disappointment , tiredness, or pleasure .
 \textbf{Sigh} is also a noun .
 \textit{
	\begin{itemize}
	\item Michael sighed wearily.
	\item Roberta sighed with relief.
	\item Dad sighed and stood up.
	\item She kicked off her shoes with a sigh.
	\item Prue heaved a weary sigh.
	\end{itemize}
}
\item verb \\
If you \textbf{sigh} something, you say it with a sigh.
 \textit{
	\begin{itemize}
	\item 'Oh, sorry. I forgot.'—'Everyone forgets,' the girl sighed.
	\end{itemize}
}
\item verb \\
If the wind  \textbf{sighs} through a place, it moves through the place with a sound like a sigh.
 \textit{
	\begin{itemize}
	\item The wind sighed through the valley.
	\end{itemize}
}
\item  \\
 sigh of relief \textit{
	\begin{itemize}
	\end{itemize}
}
\end{enumerate}

\section*{overthrow}
{\large \color{blue}  overthrows  overthrowing  overthrew  overthrown  }
\subsection*{Explain}
\begin{enumerate}
\item verb \\
When a government or leader  \textbf{is overthrown} , they are removed from power by force.
 \textbf{Overthrow} is also a noun .
 \textit{
	\begin{itemize}
	\item That government was overthrown in a military coup three years ago.
	\item ...an attempt to overthrow the president.
	\item They were charged with plotting the overthrow of the state.
	\end{itemize}
}
\end{enumerate}

\section*{target}
{\large \color{blue}  targets  targeting  targetting  targeted  targetted  }
\subsection*{Explain}
\begin{enumerate}
\item countable noun \\
A \textbf{target} is something at which someone is aiming a weapon or other object.
 \textit{
	\begin{itemize}
	\item The village lies beside a main road, making it an easy target for bandits.
	\item The missiles missed their target.
	\item He missed the target only once yesterday.
	\item We threw knives at targets.
	\end{itemize}
}
\item countable noun \\
A \textbf{target} is a result that you are trying to achieve .
 \textit{
	\begin{itemize}
	\item He's won back his place too late to achieve his target of 20 goals this season.
	\item ...school leavers who failed to reach their target grades.
	\end{itemize}
}
\item verb \\
To \textbf{target} a particular person or thing means to decide to attack or criticize them.
 \textbf{Target} is also a noun .
 \textit{
	\begin{itemize}
	\item They targeted her as vulnerable in her bid for reelection.
	\item He targets the economy as the root cause of the deteriorating law and order situation.
	\item In the past, they have been the target of racist abuse.
	\item The professor has been a frequent target for animal rights extremists.
	\end{itemize}
}
\item verb \\
If you \textbf{target} a particular group of people, you try to appeal to those people or affect them.
 \textbf{Target} is also a noun.
 \textit{
	\begin{itemize}
	\item The campaign will target insurance companies.
	\item The company has targeted adults as its primary customers.
	\item Students are a prime target group for marketing strategies.
	\end{itemize}
}
\item  \\
 on target \textit{
	\begin{itemize}
	\end{itemize}
}
\end{enumerate}

\section*{regret}
{\large \color{blue}  regrets  regretting  regretted  }
\subsection*{Explain}
\begin{enumerate}
\item verb \\
If you \textbf{regret} something that you have done, you wish that you had not done it.
 \textit{
	\begin{itemize}
	\item I simply gave in to him, and I've regretted it ever since.
	\item Ellis seemed to be regretting that he had asked the question.
	\item Five years later she regrets having given up her home.
	\end{itemize}
}
\item variable noun \\
\textbf{Regret} is a feeling of sadness or disappointment , which is caused by something that has happened or something that you have done or not done.
 \textit{
	\begin{itemize}
	\item My great regret in life is that I didn't bring home the America's Cup.
	\item Lillee said he had no regrets about retiring.
	\end{itemize}
}
\item verb \\
You can  say that you \textbf{regret} something as a polite way of saying that you are sorry about it. You use expressions such as \textbf{I regret to say} or \textbf{I regret to inform you} to show that you are sorry about something.
 \textit{
	\begin{itemize}
	\item 'I very much regret the injuries he sustained,' he said.
	\item I regret that the United States has added its voice to such protests.
	\item Her lack of co-operation is nothing new, I regret to say.
	\item I regret to inform you he died as a consequence of his injuries.
	\end{itemize}
}
\item uncountable noun \\
If someone expresses  \textbf{regret} about something, they say that they are sorry about it.
 \textit{
	\begin{itemize}
	\item He expressed great regret and said that surgeons would attempt to reverse the operation.
	\item She has accepted his resignation with regret.
	\end{itemize}
}
\end{enumerate}

\section*{telegram}
{\large \color{blue}  telegrams  }
\subsection*{Explain}
\begin{enumerate}
\item countable noun \\
A \textbf{telegram} is a message that is sent electronically and then printed and delivered to someone's home or office. In the past , telegrams were sent by telegraph.
 \textit{
	\begin{itemize}
	\item Scores of congratulatory telegrams and letters greeted Franklin on his return.
	\item She received a briefing by telegram.
	\end{itemize}
}
\end{enumerate}

\section*{request}
{\large \color{blue}  requests  requesting  requested  }
\subsection*{Explain}
\begin{enumerate}
\item verb \\
If you \textbf{request} something, you ask for it politely or formally.
 \textit{
	\begin{itemize}
	\item Mr Dennis said he had requested access to a telephone.
	\item She had requested that the door to her room be left open.
	\end{itemize}
}
\item verb \\
If you \textbf{request} someone \textbf{to} do something, you politely or formally ask them to do it.
 \textit{
	\begin{itemize}
	\item They requested him to leave.
	\item Students are requested to park at the rear of the Department.
	\end{itemize}
}
\item countable noun \\
If you make a \textbf{request} , you politely or formally ask someone to do something.
 \textit{
	\begin{itemize}
	\item France had agreed to his request for political asylum.
	\item Vietnam made an official request that the meeting be postponed.
	\end{itemize}
}
\item countable noun \\
A \textbf{request} is a song or piece of music which someone has asked a performer or disc  jockey to play.
 \textit{
	\begin{itemize}
	\item If you have any requests, I'd be happy to play them for you.
	\end{itemize}
}
\item  \\
 at sb's request/at the request of sb \textit{
	\begin{itemize}
	\end{itemize}
}
\item  \\
 on request \textit{
	\begin{itemize}
	\end{itemize}
}
\end{enumerate}

\section*{telegraph}
{\large \color{blue}  telegraphs  telegraphing  telegraphed  }
\subsection*{Explain}
\begin{enumerate}
\item uncountable noun \\
\textbf{Telegraph} is a system of sending messages over long distances, either by means of electricity or by radio signals. Telegraph was used more often before the invention of telephones and computers.
 \textit{
	\begin{itemize}
	\end{itemize}
}
\item verb \\
In the past , to \textbf{telegraph} someone meant to send them a message by telegraph.
 \textit{
	\begin{itemize}
	\item Churchill telegraphed an urgent message to Wavell.
	\item 'Please,' he telegraphed, 'just leave it alone.'.
	\item He telegraphed to me asking me to do something.
	\end{itemize}
}
\item verb \\
If someone \textbf{telegraphs} something that they are planning or intending to do, they make it obvious , either deliberately or accidentally, that they are going to do it.
 \textit{
	\begin{itemize}
	\item He explicitly telegraphed his voting intentions at the next meeting.
	\end{itemize}
}
\end{enumerate}

\section*{rescue}
{\large \color{blue}  rescues  rescuing  rescued  }
\subsection*{Explain}
\begin{enumerate}
\item verb \\
If you \textbf{rescue} someone, you get them out of a dangerous or unpleasant  situation .
 \textit{
	\begin{itemize}
	\item Helicopters rescued nearly 20 people from the roof of the burning building.
	\item He had rescued her from a horrible life.
	\end{itemize}
}
\item uncountable noun \\
\textbf{Rescue} is help which gets someone out of a dangerous or unpleasant situation.
 \textit{
	\begin{itemize}
	\item Lights clipped onto life jackets improve the chances of rescue.
	\item A big rescue operation has been launched for a trawler missing at sea.
	\end{itemize}
}
\item countable noun \\
A \textbf{rescue} is an attempt to save someone from a dangerous or unpleasant situation.
 \textit{
	\begin{itemize}
	\item A major air-sea rescue is under way.
	\item Five children were pulled from the smoke-filled house in heroic rescues by fire crews.
	\end{itemize}
}
\item  \\
 go to sb's rescue/come to sb's rescue \textit{
	\begin{itemize}
	\end{itemize}
}
\end{enumerate}

\section*{telephone}
{\large \color{blue}  telephones  telephoning  telephoned  }
\subsection*{Explain}
\begin{enumerate}
\item uncountable noun \\
The \textbf{telephone} is the electrical system of communication that you use to talk directly to someone else in a different place. You use the telephone by dialling a number on a piece of equipment and speaking into it.
 \textit{
	\begin{itemize}
	\item They usually exchanged messages by telephone.
	\item I dread to think what our telephone bill is going to be.
	\item She was wanted on the telephone.
	\end{itemize}
}
\item countable noun \\
A \textbf{telephone} is the piece of equipment that you use when you talk to someone by telephone.
 \textit{
	\begin{itemize}
	\item He got up and answered the telephone.
	\item The telephone in Rizzoli's room rang.
	\end{itemize}
}
\item verb \\
If you \textbf{telephone} someone, you dial their telephone number and speak to them by telephone.
 \textit{
	\begin{itemize}
	\item I felt so badly I had to telephone Owen to say I was sorry.
	\item They usually telephone first to see if she is at home.
	\end{itemize}
}
\item  \\
 on the telephone \textit{
	\begin{itemize}
	\end{itemize}
}
\item  \\
 on the telephone \textit{
	\begin{itemize}
	\end{itemize}
}
\end{enumerate}

\section*{ride}
{\large \color{blue}  rides  riding  rode  ridden  }
\subsection*{Explain}
\begin{enumerate}
\item verb \\
When you \textbf{ride} a horse, you sit on it and control its movements.
 \textit{
	\begin{itemize}
	\item I saw a girl riding a horse.
	\item Can you ride?
	\item He was riding on his horse looking for the castle.
	\item They still ride around on horses.
	\end{itemize}
}
\item verb \\
When you \textbf{ride} a bicycle or a motorcycle , you sit on it, control it, and travel along on it.
 \textit{
	\begin{itemize}
	\item Riding a bike is great exercise.
	\item Two men riding on motorcycles opened fire on him.
	\item He rode to work on a bicycle.
	\end{itemize}
}
\item verb \\
When you \textbf{ride}  \textbf{in} a vehicle such as a car, you travel in it.
 \textit{
	\begin{itemize}
	\item He prefers travelling on the Tube to riding in a limousine.
	\item I was riding on the back of a friend's bicycle.
	\item I remember the village full of American servicemen riding around in jeeps.
	\item I rode to Lily's in a cab.
	\end{itemize}
}
\item countable noun \\
A \textbf{ride} is a journey on a horse or bicycle, or in a vehicle.
 \textit{
	\begin{itemize}
	\item She took some friends for a ride in the family car.
	\item Would you like to go for a ride?
	\item She lives just a short bus ride from school.
	\end{itemize}
}
\item countable noun \\
In a fairground, a \textbf{ride} is a large machine that people ride on for fun .
 \textit{
	\begin{itemize}
	\end{itemize}
}
\item verb \\
If you say that one thing \textbf{is riding on} another, you mean that the first thing depends on the second thing.
 \textit{
	\begin{itemize}
	\item Billions of pounds are riding on the outcome of the election.
	\item Everything rides on the judgment of these few men.
	\end{itemize}
}
\item  \\
 be riding high \textit{
	\begin{itemize}
	\end{itemize}
}
\item  \\
 a rough ride \textit{
	\begin{itemize}
	\end{itemize}
}
\item  \\
 to be taken for a ride \textit{
	\begin{itemize}
	\end{itemize}
}
\end{enumerate}

\section*{sob}
{\large \color{blue}  sobs  sobbing  sobbed  }
\subsection*{Explain}
\begin{enumerate}
\item verb \\
When someone \textbf{sobs} , they cry in a noisy  way , breathing in short breaths.
 \textit{
	\begin{itemize}
	\item She began to sob again, burying her face in the pillow.
	\item Her sister broke down, sobbing into her handkerchief.
	\end{itemize}
}
\item verb \\
If you \textbf{sob} something, you say it while you are crying.
 \textit{
	\begin{itemize}
	\item 'Everything's my fault,' she sobbed.
	\end{itemize}
}
\item countable noun \\
A \textbf{sob} is one of the noises that you make when you are crying.
 \textit{
	\begin{itemize}
	\end{itemize}
}
\end{enumerate}

\section*{voltage}
{\large \color{blue}  voltages  }
\subsection*{Explain}
\begin{enumerate}
\item variable noun \\
The \textbf{voltage} of an electrical  current is its force measured in volts.
 \textit{
	\begin{itemize}
	\item The systems are getting smaller and using lower voltages.
	\item ...high-voltage power lines.
	\end{itemize}
}
\end{enumerate}

\section*{survey}
{\large \color{blue}  surveys  surveying  surveyed  }
\subsection*{Explain}
\begin{enumerate}
\item countable noun \\
If you carry out a \textbf{survey} , you try to find out detailed information about a lot of different people or things, usually by asking people a series of questions .
 \textit{
	\begin{itemize}
	\item The council conducted a survey of the uses to which farm buildings are put.
	\item According to the survey, overall world trade has also slackened.
	\end{itemize}
}
\item verb \\
If you \textbf{survey} a number of people, companies , or organizations , you try to find out information about their opinions or behaviour , usually by asking them a series of questions.
 \textit{
	\begin{itemize}
	\item Business Development Advisers surveyed 211 companies for the report.
	\item Only 18 percent of those surveyed opposed the idea.
	\end{itemize}
}
\item verb \\
If you \textbf{survey} something, you look at or consider the whole of it carefully.
 \textit{
	\begin{itemize}
	\item He pushed himself to his feet and surveyed the room.
	\item He surveys American politics with a conservative world view.
	\end{itemize}
}
\item singular noun \\
If you give something a brief  \textbf{survey} or a quick  \textbf{survey} , you look at or consider all of it quickly, but not in detail.
 \textit{
	\begin{itemize}
	\item ...a brief survey of some important books on astrology.
	\item He sniffed the perfume she wore, then gave her a quick survey.
	\end{itemize}
}
\item countable noun \\
If someone carries out a \textbf{survey} of an area of land, they examine it and measure it, usually in order to make a map
of it.
 \textit{
	\begin{itemize}
	\item ...the organizer of the geological survey of India.
	\item The scientists conducted two aerial surveys followed by two ground surveys.
	\end{itemize}
}
\item verb \\
If someone \textbf{surveys} an area of land, they examine it and measure it, usually in order to make a map of
it.
 \textit{
	\begin{itemize}
	\item The council commissioned geological experts to survey the cliffs.
	\end{itemize}
}
\item countable noun \\
A \textbf{survey} is a careful  examination of the condition and structure of a house , usually carried out in order to give information to a person who wants to buy it.
 \textit{
	\begin{itemize}
	\item ...a structural survey undertaken by a qualified surveyor.
	\end{itemize}
}
\item verb \\
If someone \textbf{surveys} a house, they examine it carefully and report on its structure, usually in order
to give advice to a person who is thinking of buying it.
 \textit{
	\begin{itemize}
	\item ...the people who surveyed the house for the mortgage.
	\end{itemize}
}
\end{enumerate}

\section*{wreck}
{\large \color{blue}  wrecks  wrecking  wrecked  }
\subsection*{Explain}
\begin{enumerate}
\item verb \\
To \textbf{wreck} something means to completely destroy or ruin it.
 \textit{
	\begin{itemize}
	\item He wrecked the garden.
	\item A coalition could have defeated the government and wrecked the treaty.
	\item His life has been wrecked by the tragedy.
	\item ...missed promotions, lost jobs, wrecked marriages.
	\end{itemize}
}
\item verb \\
If a ship \textbf{is wrecked} , it is damaged so much that it sinks or can no longer sail .
 \textit{
	\begin{itemize}
	\item The ship was wrecked by an explosion.
	\item ...a wrecked cargo ship.
	\end{itemize}
}
\item countable noun \\
A \textbf{wreck} is something such as a ship, car , plane , or building which has been destroyed, usually in an accident .
 \textit{
	\begin{itemize}
	\item ...the wreck of a sailing ship.
	\item The car was a total wreck.
	\item We thought of buying the house as a wreck, doing it up, then selling it.
	\end{itemize}
}
\item countable noun \\
A \textbf{wreck} is an accident in which a moving  vehicle  hits something and is damaged or destroyed.
 \textit{
	\begin{itemize}
	\item He was killed in a car wreck.
	\item ...the little girl that survived that plane wreck.
	\item What would he tell his parents if he had a wreck?
	\end{itemize}
}
\item countable noun \\
If you say that someone is a \textbf{wreck} , you mean that they are very exhausted or unhealthy .
 \textit{
	\begin{itemize}
	\item You look a wreck.
	\item It was embarrassing and sad to see this man reduced to a mumbling wreck.
	\end{itemize}
}
\end{enumerate}

\section*{wipe}
{\large \color{blue}  wipes  wiping  wiped  }
\subsection*{Explain}
\begin{enumerate}
\item verb \\
If you \textbf{wipe} something, you rub its surface to remove dirt or liquid from it.
 \textbf{Wipe} is also a noun .
 \textit{
	\begin{itemize}
	\item I'll just wipe the table.
	\item When he had finished washing he began to wipe the basin clean.
	\item Lainey wiped her hands on the towel.
	\item She gave the table a quick wipe and disappeared behind the counter.
	\end{itemize}
}
\item verb \\
If you \textbf{wipe} dirt or liquid from something, you remove it, for example by using a cloth or your hand.
 \textit{
	\begin{itemize}
	\item Gleb wiped the sweat from his face.
	\item He shook his head and wiped his tears with a tissue.
	\end{itemize}
}
\item countable noun \\
A \textbf{wipe} is a small moist cloth for cleaning things and is designed to be used only once.
 \textit{
	\begin{itemize}
	\item ...antiseptic wipes.
	\end{itemize}
}
\item  \\
 to wipe the smile off someone's face \textit{
	\begin{itemize}
	\end{itemize}
}
\end{enumerate}

\section*{agency}
{\large \color{blue}  agencies  }
\subsection*{Explain}
\begin{enumerate}
\item countable noun \\
An \textbf{agency} is a business which provides a service on behalf of other businesses.
 \textit{
	\begin{itemize}
	\item We had to hire maids through an agency.
	\item ...a successful advertising agency.
	\end{itemize}
}
\item countable noun \\
An \textbf{agency} is a government organization responsible for a certain area of administration .
 \textit{
	\begin{itemize}
	\item ...the government agency which monitors health and safety at work in Britain.
	\item He is retiring as head of the Central Intelligence Agency (CIA).
	\end{itemize}
}
\end{enumerate}

\section*{allocate}
{\large \color{blue}  allocates  allocating  allocated  }
\subsection*{Explain}
\begin{enumerate}
\item verb \\
If one item or share of something \textbf{is allocated}  \textbf{to} a particular person or \textbf{for} a particular purpose, it is given to that person or used for that purpose.
 \textit{
	\begin{itemize}
	\item Tickets are limited and will be allocated to those who apply first.
	\item This year's budget allocated £15m to cycle safety in the capital.
	\item Our plan is to allocate one member of staff to handle appointments.
	\end{itemize}
}
\end{enumerate}

\section*{agent}
{\large \color{blue}  agents  }
\subsection*{Explain}
\begin{enumerate}
\item countable noun \\
An \textbf{agent} is a person who looks after someone else's business affairs or does business on their behalf.
 \textit{
	\begin{itemize}
	\item You are buying direct, rather than through an agent.
	\item ...a written declaration by someone, authorizing another person to act as his agent.
	\end{itemize}
}
\item countable noun \\
An \textbf{agent} in the arts  world is a person who gets work for an actor or musician , or who sells the work of a writer to publishers .
 \textit{
	\begin{itemize}
	\end{itemize}
}
\item countable noun \\
An \textbf{agent} is a person who works for a country's secret  service .
 \textit{
	\begin{itemize}
	\item All these years he's been an agent for the East.
	\end{itemize}
}
\item countable noun \\
If you refer to someone or something as the \textbf{agent}  \textbf{of} a particular effect, you mean that they cause this effect.
 \textit{
	\begin{itemize}
	\item Children can be agents of change in their communities.
	\end{itemize}
}
\item countable noun \\
A chemical that has a particular effect or is used for a particular purpose can be referred to as a particular kind of \textbf{agent} .
 \textit{
	\begin{itemize}
	\item ...the bleaching agent in white flour.
	\item ...fibrinogen, a blood clotting agent.
	\end{itemize}
}
\end{enumerate}

\section*{bow}
{\large \color{blue}  bows  bowing  bowed  }
\subsection*{Explain}
\begin{enumerate}
\item verb \\
When you \textbf{bow}  \textbf{to} someone, you briefly bend your body towards them as a formal way of greeting them
or showing respect.
 \textbf{Bow} is also a noun .
 \textit{
	\begin{itemize}
	\item They bowed low to Louis and hastened out of his way.
	\item He bowed slightly before taking her bag.
	\item I gave a theatrical bow and waved.
	\end{itemize}
}
\item verb \\
If you \textbf{bow} your head, you bend it downwards so that you are looking towards the ground, for example because you want to show respect or because you are thinking deeply about something.
 \textit{
	\begin{itemize}
	\item He bowed his head and whispered a prayer of thanksgiving.
	\item She stood still, head bowed, hands clasped in front of her.
	\end{itemize}
}
\item verb \\
If you \textbf{bow to} pressure or to someone's wishes , you agree to do what they want you to do.
 \textit{
	\begin{itemize}
	\item Some shops are bowing to consumer pressure and stocking organically grown vegetables.
	\item Parliament has bowed to the demand for a referendum next year.
	\end{itemize}
}
\item passive verb \\
If you \textbf{are bowed} by something, you are made unhappy and anxious by it, and lose  hope .
 To \textbf{be bowed down} means the same as to \textbf{be bowed} .
 \textit{
	\begin{itemize}
	\item ...their determination not to be bowed in the face of the allied attacks.
	\item I am bowed down by my sins.
	\end{itemize}
}
\item  \\
 bow to the inevitable \textit{
	\begin{itemize}
	\end{itemize}
}
\item  \\
 take a bow \textit{
	\begin{itemize}
	\end{itemize}
}
\end{enumerate}

\end{document}